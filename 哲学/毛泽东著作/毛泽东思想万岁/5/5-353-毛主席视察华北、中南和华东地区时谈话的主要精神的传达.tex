\section[毛主席视察华北、中南和华东地区时谈话的主要精神的传达]{毛主席视察华北、中南和华东地区时谈话的主要精神的传达}
\subsection{一、视察沿途讲话主要精神}

一、形势:形势大好,全国已解决了七个省(指革委会),基本上解决了八个省(指河南,湖南已建立筹备小组),共十五个省。争取今年再解决十个省,南方五个,北方五个,共二十五个。(实际上二十四个省。因黑龙江解决两次,中央是支持黑龙江革命委员会,革命委员会不能垮,坚决支持。)春节前,全国要基本解决,纳入轨道。文化大革命七、八、九三个月大进一步。形势和任务就是如此。

二、上下级关系:为什么搞的那么苦?打、罚跪、挂牌、戴高帽不好,这样把团结——批评——团结的原则破坏了。对干部要扩大教育面,缩小打击面,对干部不要一棍子打死,最顽固的也要给一碗饭吃。

北京开武装干部会,不只武的要去,文的也要去,党政群干部也要参加,左派也要去,红卫兵也要去。红卫兵权力很大,又很凶,也要训练。对干部不能不教而诛,教也不能诛。

要扩大教育面,主席在上海说了几次,并在许多省都讲了,中央文革也认真讨论了的。

三、大联合。那条语录:“在工人阶级内部,没有根本的利害冲突。在无产阶级专政下的工人阶级内部,更没有理由一定要分裂成为势不两立的两大派组织。”是七月十八日,主席在武汉第一次讲的。这句话没有被接受,武汉问题如果用这个思想去解决就更好。主席反复谈这个问题,七月谈过,八月也谈过,(我)在上海把这条语录给工人讲了,很灵,上海大联合高潮就是主席这几句话搞起来的,主席关于大联合指示,全国一宣传,效果很好。

×××插话,主席是这样讲的:“一个工厂,都是工人阶级,没有根本利害冲突,为什么要分裂为势不两立的两大派,我就想不过,这是有人操纵,无非是:一种走资派操纵,第二种地富反坏右搞鬼,第三种是小集团思潮影响。”

春桥接着说:反复宣传主席这个思想,主要是工人,对学生,机关干部也有效。工人阶级真正成为文化革命的主力军,让工人来左右局势,上海最有发言权的是工人,不是学生说了算数,上海一乱主席就问我:“乱得起来吗?”我说:“不要紧,工总司不动就不会乱。要确立工人阶级领导地位,不要被学生牵着鼻子跑。”

四、三结合。主席讲:“上海军、干、群三方面关系比较好。”主要是三方面经常开门整风,这次以军队为主,召集革命群众团体,下次以革命群众为主,如此轮流,这个制度比较好。工总司每天坚持半天学习毛著,半天工作,天塌下来也不管。军队负责召集。干部向题是一个重要条件,上海部、局长一级干部已经解放了百分之五十至六十。

×××插话;主席说:干部绝大多数是好的,要教育干部,解放干部,使用干部,干部中除了走资派为什么斗那么凶,斗那么厉害呀!有些干部坐汽车了,房子住好了,官做大了,工资高了,这都还可以,但是不要有架子,不讲民主,脱离群众。所以,一有机会就起来攻你。

当我们问到“5.16”时,春桥同志说:“5.16”是个反革命组织,有三条,一反中央,二反解放军,三反革委会。姚文元同志的文章后一段,讲革委会一段,是主席亲自加的。

\subsection{二、视察上海时的指示}


毛主席到上海视察。在上海,见过毛主席的人说,毛主席红光满面,步履尤健,身体非常健康。

我们经历了“一月风暴”和“九月高潮”,这是毛主席亲自提出的。

一月革命的经验,主席总结为:无产阶级革命派联合起来,夺党内走资本主义道路当权派的权。

九月份毛主席亲临上海,对工人阶级、红卫兵的大联合作了重要指示,使上海的大联合达到了高潮,出现了空前未有的大好形势,使我们取得了巨大的成就。

\kaoyouerziju{ (徐景贤同志九月二十五日在上海市革委会扩大会上的讲话传达)}

在讨论训练干部的问题时,主席讲:现在红卫兵都当权了,不训练也很难办,可以开训练班,这个事情跟红卫兵讲一讲。他们还很年轻,容易犯错误,他们犯错误像列宁说的,上帝还是会原谅的,我们犯错误就不行了。训练可以人少一些,深谈一下,一次不行两次,可以多谈几次,要用自己犯错误的经验,告诉他们犯了错误怎么办。对红卫兵要进行教育,加强学习。

主席对红卫兵很关心。你们可以采取这个办法:几个负责人在一起多谈谈,不要急于一次谈成,不要动不动就说你没诚意,帽子很多。主席跟我们讲:如果在七大,不用整风方法把大多数团结起来,革命就会受损失。告诉那样争名争地位的人,在七大选举,有的人争中央委员,有的人反对王明当委员,但主席提名叫他当。主席说:不当中央委员不见得是坏人,当中央委员的不见得都是好人。王明就是坏人,只有这样,才能团结大多数。八大时很人多不同意王明当中央委员,主席仍提名。把那些反对你但实践证明是犯了错误的人一道参加工作也没有什么坏处。主席讲:革委会也可以叫几个保守派进来嘛。

文化大革命是触及灵魂的思想斗争,把人关起来不是办法。有些人应该让他们活动,他们总是要走到自己的反面的。斗争会要允许斗争对象讲话,允许人家答辩,不要不让讲话。主席总是讲:让人家讲话嘛。有的造反派讲,我们是按《湖南农民运动考察报告》上所说的那样做的。主席跟我讲过:那是对付地主阶级,现在对于部,同对付地主阶级不一样,对干部应允许答辩,允许他们讲话,包括陈丕显,曹荻秋。主席在上海看过电视斗争陈、曹大会,没戴高帽子,没挂牌子,比较文明,主席很满意。主席还说了,为什么不允许他们讲完呢?人家还没说几句就喊“打倒”,……允许人家把话讲完,然后再反驳,真理是在我们手里嘛!

你们提口号要留有余地。动不动就发勒令,“否则就采取革命行动”,例如对付刘、邓。如果他不出中南海怎么办?你们自己毫无余地。过去我们曾对美国发出一次通牒,没有通过主席,后来主席就批评说:发通牒干什么?不照办怎么办?

最难的是教育革命。主席在去年七月份就讲学校斗批改要靠同学自己搞,最近主席要我们小学、中学、大学都拿出一些典型经验来。

{\raggedleft (张春桥同志九月二十九日召开上海高校负责人会议上传达)

有的头头私心杂念重,根本不顾国家利益,争核心、争名称、名位、名次,不是按照路线比较正确。上海交大反到底兵团就是这样争,主席问我和姚文元:这个大学不是一月风暴不错吗?现在怎么弄成这个样子呢?劝他们不要争了,核心有啥好争的?核心总是在斗争中形成的。

\kaoyouerziju{ (九月二十八日中央首长接见东北三省代表团时,张春桥同志的讲话传达)}

主席在上海看了怎样砸上柴“联司”的电视,上面两次出现“揪军内一小撮”的字样,主席就指示,提“军内一小撮”是不对的,要去掉它。

主席在上海视察时,除了对工人大联合作了重要指示外,对学生也作了重要指示。就是在谈到交大“反到底”时,要他们“无条件实现革命的大联合”。[“按:人民日报、红旗杂志、解放军报国庆社论改引如下:革命的红卫兵和革命的学生组织要实现革命的大联合。只要两派都是革命的群众组织,就要在革命的原则下实现革命的大联合。”]

\kaoyouerziju{ (张春桥同志九月二十五日对上海市革委会的电话传达)}

毛主席在关键时刻指示:

把“团结——批评——团结”改为“团结——批评和自我批评——团结。”

\kaoyouerziju{ (徐景贤同志九月十九日在党校传达)}

\subsection{三、视察浙江时的指示}


毛主席到杭州,只休息了一小时,就召集省军管会负责同志了解浙江的无产阶级文化大革命情况。被接见的有二十军政委南萍,空五军政委陈励耘等,参加接见的还有杨成武、张春桥。吴法宪等同志。

毛主席对浙江文化大革命情况很了解,接谈了两小时,主席作了很多指示。主席详细地询问了舟山文化大革命情况,对军队问题问得很详细。

南萍同志汇报说:“二十军进入杭州前,军党委作过决议,向空五军学习。”

毛主席说: “空五军支左很好嘛!”

陈励耘向志汇报说:“空五军党委也做出决议,向二十军学习。”

主席说:“大家互相学习。”

主席对南、陈两负责人说:“你们要很好地向群众学习,不要架子,放下官架子,什么事情都是群众创造的。”并对吴法宪同志说:“飞机不是战士打下来的吗?”

南萍汇报说:“我们还有罚跪、戴高帽等现象。”

主席说:“我向来反对这样搞,对待干部不能像对地主一样。我们有个好的传统:团结——批评——团结,对干部要一分为二。”

主席对南萍说:“金华转得很好嘛!”  (据悉毛主席曾于九月十六日视察了金华。)

毛主席对浙江的文化大革命情况很满意,并指示:“这样发展下去,浙江在春节前后可以看出一个清楚的眉目来了。”

在整个接见中,毛主席精神非常好,神采奕奕,接见开始时,毛主席问二十军政委叫什么名字,二十军政委说“姓南”,毛主席就讲了一个唐朝的故事,说明姓南的是怎样来的,毛主席抽香烟,南萍给毛主席点烟,毛主席风趣地说:“自力更生,我自己来。”

\kaoyouerziju{  (根据二十军首长回忆大意)}

\subsection{四、接见江西省革筹小组成员时的指示}

九月十七日清晨,我们最最敬爱的伟大领袖毛主席来到了南昌。当天上午,毛主席接见了省革命委员会筹备小组程世清、杨栋梁、黄先、刘瑞森、郭光洲、陈昌奉等同志,作了极其重要极为英明的最高指示。

随主席一起参加接见的,有×××,张春桥同志,汪东兴同志,和×××等。

当程世清等同志走到我们伟大领袖毛主席的面前的时候,我们心中最红最红的红太阳、全世界人民的伟大领袖毛主席满面笑容站了起来,他老人家伸出伟大、温暖的手亲切地和程世清等同志等一一握手,并问了每个同志的情况。

我们伟大的领袖毛主席,身穿一身普通的布的便服,穿一双我们解放军战士穿的布鞋。这里我们报告同志们一个最最振奋人心的好消息,我们伟大的领袖毛主席他老人家满面红光,神采奕奕。身体非常健康,非常健康!脸上看不见皱纹,白头发很少。我们当时又想多看看他老人家,又想多听听他老人家的亲切教导,又想多记些他老人家的最新指示,又想多汇报些江西的情况。我们感到浑身充满无限的力量。这是全国人民最大的幸福,是全世界人民最大的幸福。

我们伟大领袖毛主席他老人家记忆力非常好,洞悉一切,高瞻远瞩,是全世界最伟大的天才。对我们教育最大,感受最深。主席对江西省文化大革命情况非常熟悉,非常关心,在谈话中,当谈到抚州问题时,毛主席亲自指出了抚州九个县的名字:“临川、金溪、资溪、南城、南丰、黎川,宜黄、崇仁、乐安。”在谈话中,主席曾三次谈到大联筹、大中红司办的《火线战报》。对江西过去在文化大革命中发生的重大事件,比如“6.29”事件、赣州事件,抚州叛乱事件,以及当前的情况,都很清楚。主席对当前江西文化大革命是满意的。我们向主席汇报当中,主席和蔼可亲,平易近人,谈笑风生,主席向我们问了很多问题,都是江西省文化大革命的重大问题,给我们作了许多最高最新的指示。

下边分五个问题向全省无产阶级革命派、红卫兵革命小将和广大人民群众传达。

(一)毛主席对形势问题的指示

当我们汇报到主席的思想、政策已经在江西广大人民的心中深深扎根,对江西形势起了决定的作用时,毛主席教导我们:“五月底,我写了几句话,给林彪同志、总理,说江西军区同群众为什么这样对立,值得研究,我没有下结论。我指的是江西、湖南、湖北、河南。”

毛主席还说:“六、七、八月间最紧张,紧张的时候,我就看出问题揭开了,事情好解决了,不紧张怎么解决呀!”

当我们汇报到抚州问题时,主席听了汇报后教导我们:“抚州的问题值得研究一下,为什么他们会这样大胆?他们总要开会研究形势,认为江西、全国和世界形势对他们有利,才这样干。他们对形势估计不正确,我看是。抚州问题实际是叛乱,是典型之一。说中国没有内战,我看这就是内战不是外战,是武斗,不是文斗。在赣州、吉安、宜春等地,还搞农管,一个生产队抽一个人,一个大队抽十几个人,采取强迫的办法,记工分,一天六毛钱。现在农村包围城市,我看不行。”

当我们汇报《文汇报》写了一些好文章,影响很大时,张春侨同志插话说:“人家还反我们右倾,说我们变右了。”

接着,主席教导我们:“那有那么多复辟呀,他们已经垮了,不能再复辟了。有一种说法,索性垮就垮了。其实,天下是不会乱的,天也不会塌下来。”

(二)毛主席对干部问题的指示

汇报中间,毛主席指示:“干部垮掉这样多,是好事还是坏事,你们研究了这个问题没有?总要给他们时间,来认识和改正错误,要批评打倒一切的思想。”

毛主席教导我们说:“有些人犯了错误,要给他改正错误的机会。”

当我们汇报到造反派里有的同志有报复思想时,主席教导我们:“不能不教而诛,诛就是杀,诛就是杀人。不能不教而处罚人,过去就是吃了这个亏嘛!“

“我看还应该从教育人手,坏人总是少数。“

“我对现在的右派不那样看死,有坏人是少数,多数是认识问题。有的把认识问题说成是立场问题,一提到立场问题就上了纲,一辈子不得翻身。难道立场问题就不能变吗?对大多数人来说立场是能变的,对极少数坏人是不能变的。总而言之,打击面要缩小,教育面要扩大,教育要包括左、中、右。“

主席询问了过去省委的一些人的情况,然后指示我们:“我还是倾向于多保一些人,能挽救的还是挽救,只要我们争取了多数,极少数人顽固下去也可以嘛,我们给他饭吃算了。“当主席问了过去省委几个犯了严重错误的干部时说:“如果能改,能多争取几个人也好嘛!“

当我们汇报到正在集训军队一些干部时,主席教导我们:“开训练班中央应该开,主要是各省开,不仅军队开,地方党政文教也要集训。训比不训好,时间顶多二个月,久了不行。过去黄埔五个月入伍期,四个月训练。林彪同志只住了五个月的黄埔嘛。有些军事学校,学的时间越长,学得越糊涂。“

当我们汇报到有些同志还要到外边去抓什么人时,主席教导我们:“要保护,不要使人下不了台,要使人有机会搞正错误。“

主席还教导我们:“江西还站出来一批干部嘛。你们是省一级的,省级要多站出一些人,还有市的,能站出多少于部?“

(三)毛主席对造反派教育问题的指示

当我们汇报到大联筹准备召开政治工作会议时,主席教导我们:“这个好,造反派也要训练。他们坐不下来,心野了。造反派人很多,一批不行可以训练二批三批。“

主席教导我们:“左派不教育变成极左。“

当我们汇报到有些人反右倾时,主席教导我们:“是教育左派的问题,不是右倾的问题。比如,过去有多少山头呀,江西有中央苏区,湘赣苏区,湘鄂赣苏区,闽赣苏区,还有鄂豫皖苏区,通南巴,陕北,抗战的时候根据地就更多了,我们用一个纲领团结起来。“

(四)毛主席对参加保守组织群众政策问题的指示

当我们汇报到因为前一阶段造反派受迫害、压抑,现在有些同志有报复思想时,主席教导说:“要很好说服,不打击报复,下跪、高帽子、挂牌子、还有什么喷气式啰,这不好。“

毛主席还教导我们说:“杀人总不好,人家杀你不好,你杀他也不好。“

当我们汇报到联络总站一个负责人自己回来,写了检讨,现在还在造反派那里检讨时,主席指示:“把他收回来好了,不要搞得太苦了。“

(五)毛主席对军队问题的指示

当我们汇报到人武部、军分区的情况时,主席很关心地听了汇报后,教导我们说:“人武部总是好人多,军分区有很多人是受蒙蔽的。“

“到处抓赵永夫、谭震林,哪有那么多赵永夫、谭震林?“

主席还教导我们说:“你们先把武装部干部训练一下。我看训练的办法好,内蒙古一个独立营八百多人,是支保的,反对中央对内蒙间题的决定,他们到了北京,气可大了,大闹,都不听总理的话,打破家具,会开不下去,向中央提出五条要求。以后到北京新城高碑店训练了四十天,都转了,回去还支左不支右,独立营、独立师训一下就转过来了。“

主席还特别指出:“现在有人挑拨战士反对官长,说你们每月只有六块钱,当官的钱多,还坐汽车。农民是愿意当解放军的,解放军很光荣,他们每月还有六块钱,家里还有优待,农民是愿意当兵的,我看是挑拨不起来的。“

对江西军区搞四大,主席教导我们:“江西军区搞四大,不要搞得太苦啦,战士一起来火就很大,浙江现在每天都斗,一斗就是戴高帽、挂黑牌、下跪、搞喷气式,人家受不了,也不雅致嘛。“

   当我们汇报到6011部队四排的事迹时,主席听了很满意,主席说“李文忠排的事迹,我看了火线战报,有他们三个人的照片,他们三个人都很年青。”

\kaoyouerziju{          (原稿由江西省无产阶级革命造反派大联合筹委会翻印)}

\subsection{五、接见湖南省革筹小组成员时的指示}

我们最最敬爱的伟大领袖毛主席九月十八日来到长沙,接见了湖南省革命委员会筹备小组黎源、华国锋、章伯森三同志,对湖南省的无产阶级文化大革命作了极其重要的指示。

毛主席号召我们迅速实现革命的大联合。毛主席说:工人阶级内部,没有根本的利害冲突,为什么要分两大派呢?两派要互相少讲别人的缺点,多作自我批评,才有利于革命的大联合。

毛主席教导我们要正确对待受蒙蔽的群众。毛主席说:两派都是工人,一派造反,一派保守。保守是上头有人蒙蔽了他们,对受蒙蔽的群众,不能压。

在汇报到湘潭情况时,毛主席说:这样多的产业工人,不会一辈子保皇。要正确对待,至于他们的头头,靠下面自己起来造反。

毛主席教导我们要正确对待干部。毛主席说:干部大多数是好的,我们要团结大多数,包括犯错误向群众作了检讨的,除了极少数坏人,打击面太宽了不好。对干部除投敌、叛变、自首者外,过去十几年,几十年总做过一些好事嘛!要扩大教育面,缩小打击面。进行批判斗争时,要用文斗,不要搞武斗,不要侮辱。

毛主席教导我们造反派要加强学习。毛主席说:对造反派也要进行教育,现在正是犯错误的时候。年青人,不要性急,现在紧张、严肃有余,团结、活泼不足,缺乏民主作风,不平等待人。打人、骂人、拍桌子,把我们的传统搞乱了,把我们坚持“团结——批评——团结”的公式搞乱了。

(以上根据黎源、华国锋、章伯森同志传达)

开始汇报湘潭问题时,毛主席说:这么多工人不能一辈子保皇,要正确对待,黑头头由他们受蒙蔽的群众自己起来造反。

谈到常德、安江问题时,汇报这些地方农民进城的问题。主席说:打打也好,受受教育嘛。许多农民进城不容易,现在是十五个工分,有的是抽签去打,有的是两元钱,有的是一百元打一次仗。

谈到内战问题时。有人反映,有的组织现在争人数,认为人多就要以它为核心。主席说:那也不一定。

谈到抢枪时。主席说:抢枪把你们吓得不得了。名义上是抢的,有的是发的,政干校就是把枪发给造反派了。主席还说:抢枪不要怕,民兵的枪就有××万条。

谈到解放军下工厂,造反派武装保护解放军时,主席说:这也是个经验。

谈到“高司”问题时。主席问到姓万的(指万达),群众不同意万达到革筹小组,主席说:等一段。

谈到军队有一段不能拿枪时,主席说:解放军也五不了。(此时原有四不了,现在是五不了)  (主席算了一下,湖南军队两个月没有拿枪)主席说:开大会军队可以武装参加。

谈到火车不通时,主席说:不通的反面就是通,社会上和内部有压力。

主席问湘潭王治国时,华国锋汇报了王被斗的情况,主席说:为什么斗得那么厉害呢?他身体不好嘛,军区错了影响到军分区,也影响到县人民武装部。

谈到公检法问题时,主席说:好像没有公检法就不得了,它一垮台我就高兴。

谈到干部问题时,主席说:清理干部要搞群众运动,这样多的军分区、人武部,不是文化大革命是搞不动的。主席又说,“……最后要团结大多数,包括犯了错误向群众作了检讨的,除极少数的坏人外,打击面太宽了不好。”

谈到对犯过错误的干部进行训练时,主席说:只要犯过错误的?军队的、党的机关都要训,要吸收红卫兵参加。

谈到红旗军问题时,主席说:再研究一下嘛,看一下,恢复了再说,世界上的事不要怕。  (张春桥同志插话,红旗军问题,林彪同志去年十一月批过文件,说服他们和别的组织联合,或参加到别的组织。)

谈到湘潭成立临时革筹小组,主席说:十月份都要搞起革筹小组来。 (张春桥同志插话说。江西也成立了临时小组,驻军召开了全省会议)主席说:开会也是训练。

谈到造反派团结时,主席说,两派要少讲别人的缺点,过去军队和地方的关系,军队老是先做自我批评,这样就好了嘛。 (张春桥同志插话说:主席语录中有这么一段话) 主席又说:“两派是工人,一派造反,一派保守,我想总是上头有人。对保守的不能压,越压越反抗,蒋介石压我们,我们就有希望了。一压就压出三十万共产党,三十万红军,后来是自己犯错误才有二万五千里长征。”

谈到长沙的碉堡工事时,主席说:不放心就保存一个时期嘛! (×× ×插话,水泥的不要拆,也是一个备战)。

陪同主席接见的有;张春桥、  ×××、汪东兴等同志。

\subsection{六、视察武汉时的指示}

我们最敬爱的伟大领袖、我们心中最红最红的红太阳毛主席在汉视察了两天三夜。主席在湖南视察了一天,坐火车来,也是坐火车走。毛主席巡视了武汉市容,两次都是白天,我们出于对伟大领袖毛主席的无限敬仰、无限关心,劝他老人家下半夜巡视,主席说不行,要白天看市容,而且是坐吉普车,主席巡视了市容和街上的大字报,感到很满意。回住处跟我们谈到深夜二点。春桥同志说:主席该休息了。我们在他老人家讲话时不插话,仔细地听,那里有这么好的机会啊!在讲话中,主席经常看语录,讲语录,边讲边笑,一直很高兴,精神焕发。主席讲了很多党的历史,从党的历史联想到文化大革命中的问题,讲中国历史,外国历史,耐心教育我们。

  毛主席在汉接见了武汉部队负责人曾思玉、刘丰和武汉警备区负责人方铭、张纯青等四同志,二十三日刘丰同志送主席到郑州返汉,可惜街上大批判专栏被涂坏了,我们很难过。

根据我们回忆,主席讲话精神如下。

(一)武汉文化大革命形势空前大好,湖北的问题基本解决了。武汉出了一个“七.二〇”事件,好比脓包穿了头,坏事变成了好事,问题就是不破不立,乱透了就会好的,党内走资派搞的一些鬼,看得很清楚。他们被彻底打倒了,他们无权了。主席说:湖北的问题基本解决了,形势好得很,乱透了就好,好解决问题。主席问我们:湖北第二批建立革委会行不行?形势大好啊,这个机会错过了,那以后就赶不上。还问我们:今冬明春的工作安排了没有?明年元月、五月,八月准备搞些什么?  (我们想 “三结合”基本具备了条件,七大造反派组织是现成的,军队是现成的,地方干部如何?我们心里还没把握。) 主席问:“你们看美帝敢不敢打?”主席很乐观地说: “我看不行,它不敢打,它连一个越南也对付不了。南越最初只有九条枪,后来越南把美国搞得骑虎难下。美帝没有什么了不起”。我们的文化大革命取得了伟大的决定性的胜利,今年又风调雨顺,主席很高兴。主席对武汉寄有很大希望,我们想,还要作艰苦工作,才能由“基本解决”走到“完全解决”,现在离建立革委会还有一段路程。  (张春桥同志插话:你们千万不能打乱毛主席的战略部署)。

毛主席他老人家高瞻远瞩,想的很远,我们湖北一定紧紧跟上毛主席战略部署。同志们考虑一下,过了国庆节先把工代会搞起来,再结合一两个革命领导干部,十月份建立革筹小组,尽快实现革命大联合,争取在元旦开个会,腊月十五再开个会,春节放三天假,大家好好过个年。

(二)要解放一大批干部。

主席又从历史上讲起,从中央苏区讲到延安整风。主席说:对犯错误的干部,要允许人家改正错误,要给人家改正错误的机会。打击面要缩小,教育面要扩大。古田会议前,教条主义者还整我,陈总对军队问题解决不了,以后陈总又把我接到军队来。在古田会议上,讲了十条,从正面批评了那些“主义”。以后,又排斥我,说“山上没有马列主义”。朱德、李德(一个洋人)不听意见,打了败仗,后来才长征。在遵义会议上,洋人投了我一票,我才达到四比三。所以,核心不能自封。即使王明这样对待我,我还是主张选他当中央委员。

主席讲,八一南昌起义是首先向国民党开的第一枪,所以是革命行动,不能取消“八一”。贺龙是个土匪,南昌失败,实际不会打仗,贺山头指靠苏联,但八一起义还是革命行动。前几次上天安门都是我批的,你们打你们的,我批我的。我这个人曾经五次被人家排斥。后来又要请我回来,南昌失败后,干部不多了,但都成了骨干。什么叫长征?长征是打出来的,是逼出来的,长征之后,质量就高了,就是现在这些干部,哪个在战场上没受伤?在民主革命中还是立了不少功,虽然在文化革命中犯了错误,只要改正就行了。老干部到打起仗来,我的命令一下,他们上战场还是很勇敢的。不能怀疑一切,打倒一切;怀疑一切,打倒一切,不好,这对革命事业,对党不利。

主席讲:听说对“百万雄师”、独立师揪得太厉害了,那样搞会逼上梁山的。独立师一万多人,上街的有二千多人,连人家的家属也倒霉,这样不好,搞得太凶就会脱离群众。(主席很注意看小报,每个小报都看了。)主席对曾思玉同志说:“听说你来时,不是有人要打倒你?”曾答:“打倒没有关系”。刘丰说:今天(九月二十八日)有人写信给我,要为“百万雄师”翻案,说不翻案就没有好下场。署名“狂人”。主席说;坏人总是少数,好人总是大多数,广大群众是好的。独立师广大干部和群众都是好的,是受蒙蔽的。你们要多请示。刘、邓就不喜欢请示。我们这样大的国家,一个个自己决定,决定错了怎么办?

主席还指示:湖北、安徽的干部还没有站出来,有的不能当省长了,当副省长也可以。红卫兵小将也可当干部。

(三)要文斗,不要武斗。

主席说,体罚这个做法,我反感。是不是把《湖南农民运动考察报告》里打倒土豪劣绅的那一套办法都搞来了?那是对敌人的,现在时代背景不同了。文化大革命要触及灵魂。凡是爱整人的人,整来整去,最后都要整到自已头上来,“喷气式”是王光美搞的,王光美是资本家的姑娘。

现在全国到处搞武斗,这翻不了天,让那些人跳出来好嘛。

(四)团给一一批评一一团结。

现在有些人破坏了这个优良传统,不喜欢这个公式。解放军是团结、紧张、严肃、活泼八个大字,现在是紧张、严肃有余,团结、活泼不足。  (曾思玉插话。有些人叫什么勤务员,有的是大司令,比我们还大,官一大了,有时就不民主。) 主席说:“你这太急躁了”。

又对刘丰说:“他要是急躁起来,你这个政委就给他泼点冷水。”

主席设;张国焘两次南下,徐向前在这个问题上是好的。许世友是个和尚,但是员战将,他也要打倒?看一个人,要全面的看,历史的看,又要比较的看。要有几个反面教员,没有反面教员,叫独裁。要团结一切可以团结的人们。

(五)革命大联合

毛主席说:“工人阶级分为两大派,我就想不通”。联合不起来,主席早就发觉了。七月十八日主席就在汉指示:“在工人阶级内部,没有根本的利害冲突。在无产阶级专政下的工人阶级内都,更没有理由一定要分裂成为势不两立的两大派组织”。

我们一定要替毛主席争气。武汉点名的七大组织,都在他老人家那里挂了号的,联合的关键在你们,主席对你们各个小报都看了,好坏都有评价,有些小报派性特别强,纸倒特别好。主席指示我们:“多给他们(注:指造反派)讲些历史。过去苏区里面打内战,无非就是一个马克思主义多一点、少一点的问题。”主席非常关心武汉的无产阶级文化大革命,临走时还问我们:“两个月,联不联合得起来?”

曾、刘首长说:无产阶级革命派要很好学习中央首长对北京“天派”、“地派”的讲话。中央对革命小将的批评,武汉造反派要引以为戒。  “三新”不要骄傲、二司骄傲不得,工总也不要以为人数多,老是骄傲就会犯错误,这一段犯了点小错误,是支流,最大也不过是汉水,不是长江。 “五.一六”兵团,我们这里也有,武汉有黑手,中央已经打了招呼。安徽两派联合起来后,黑手就自己揪出来了。七月十六日,主席本来想渡江的,死了那么多人,结果没渡成。有黑手,要警惕。主席讲:“现在是革命小将有可能犯错误的时候了”。我们要防止坏人用极“左”的口号、山头主义搞得我们犯错误。我们的革命小将不要去管工人的事,坐下来好好学习,写文章,多起促进作用。林总从苏联回来时,对毛主席讲,中国一定比苏联强,因为中国有很多优秀儿女。你们革命组织的勤务员一定要做优秀儿女,不要掉队。

毛主席还指示:“干部不要从外地调,可以精简。”接着问张政委:“你的警备区有多少人?”张说:八十多人。毛主席说:“有一十到二十就够了”。

主席参观市容时,正值“三司革联”开红代会,街上很多宣传车,主席对司机说:“你告诉曾司令员,这不好,我很反感”。我们不要搞经济主义。上海张春侨、姚文元同志一直抓工人工作,那里没发生抢枪,大批判专栏办得最好,我们要学习,办好大字报专栏,坚决抵制小报新闻,扫除一切非宣传品,取消小报买卖.曾、刘首长还表示,我们支左,决不支派。又问:“干部怎么结合?张体学这个人行不行?他过去做实际工作多。”

毛主席的整个谈话长达两个半小时,他老人家红光满面,神采奕奕,总是乐哈哈的,一点也不觉得累。毛主席身体这样健康。是全中国人民、全世界人民的最大幸福。毛主席为我们撑腰,我们一定为毛主席争气。

(曾、刘首长传达)

毛主席在汉指示:

目前的主要任务就是大批判,大斗争,大联合,尽快实现三结合。希望你们成为对党内走资派斗争的模范;希望你们成为大联合的模范;希望你们成为反对小团体主义、经济主义、自私自利的模范。中国历次革命,以我经历看来,真正有希望的是想问题的人,不是出风头的人,现在大吵大闹的人,一定会成为历史上昙花一现的人物。

\subsection{七、视察河南时的指示}

九月二十二日,我和王新同志,纪登奎同志怀着万分激动的心情,见到了我们最最敬爱的伟大领袖,我们心中最红最红的红太阳毛主席。

毛主席红光满面,神采奕奕,身体非常健康,非常健康。他老人家非常关心咱们河南无产阶级文化大革命的情况。我们向最伟大的领袖毛主席汇报情况时,他老人家精神焕发,笑容满面,频频点头。

这次随同主席来河南的,有×××,张春桥等同志。

主席一见纪登奎就笑着说:“你是纪登奎?老朋友了。”

我说:“他被关了四个月,挨了四个月的斗争。”

主席笑着对纪说:“你说一点好处都没有吗?”

纪答:“大有好处。”

主席接着说:“那是文敏生、赵文甫、何运洪他们搞的,我上次路过郑州,看到了一张大标语叫‘大局已定,二七必胜’。河南形势很好。”

当我们汇报各地站出来一些革命干部时,主席说:“这是何运洪干的好事。何运洪那么厉害呀!”

当汇报到要武装干部到北京集训时主席说:“集训也要去好人。”

当汇报到开封情况时,化肥厂主张打,主席说:“不讲俘虏政策不好,要给八二四做工作,我赞成你们的意见。”

当汇报到少数人不讲政策,随便开枪,有时打死人时,主席说:“群众起来议论,不赞成他们,群众起来都反对就收场了。“抓军内一小撮的口号搞不久,现在已经不那么响了。他们不拥军,一拥军就没有对象了。”

王新说:“主要是从理论上把他们的错误思想批倒。”

主席说:“对!”

\subsection{八、在回京列车上,以“斗私、批修”四个字祝贺铁路工人大联合}


这次我们跟随毛主席出去视察,坐了一段火车。车上有三派(都是铁路系统的),都说自己是造反派,人家是保守派。起初我和××同志去做工作,他们说我们“合稀泥”。后来毛主席把三派的代表找来,听他辩论,他们辩论了两个小时。

主席说:“我看你们没有根本的利害冲突,应该联合嘛!”

三派还有些不想大联合,都指责对方,不做自我批评。

主席说:“不要尽看别人的缺点错误,你就讲你自己的,人家的问题他自己会讲的。人家的缺点应该人家讲,你讲什么?”

这样就没吵起来。等到他们从主席车厢走出来,还是说对方是“老保”。我就和××同志一直和三派做工作,和他们一起学习《毛主席语录》135页那一段话,我们说,主席和一个省谈话也不过两个小时,和你们却谈了两个小时,你们还不联合,回去怎么交账? 结果三派均照主席指示检查了自己,彼此越听越感动,他们联合起来。当然,也不那么简单,有一派还提条件,可尽是一些鸡毛蒜皮的事,离北京只剩三个小时,经过做工作,也通了。

这样,三派达成了联合的协议,到毛主席那里去报喜。

主席说:“祝贺你们,送你们四个字:‘斗私,批修'”。“斗私,批修”就是这么来的。

的确,如果私字不去掉,无法达成大联合的协议,达成了,也会分裂。就是把黑手斩断了,不把私心杂念去掉,特别是不去掉造反派领导人头脑里的私心杂念,就不能实现革命大联合、三结合。即使勉强联合也是不巩固的。

斗私、批修是相互联系的。

姚文元同志九月二十九日晚接见华东四个铁路分局革命派时传达说:最近铁路工人群众组织,实现了革命大联合,我们向毛主席汇报,革命大联合后搞什么?

毛主席讲:斗私,批修。
