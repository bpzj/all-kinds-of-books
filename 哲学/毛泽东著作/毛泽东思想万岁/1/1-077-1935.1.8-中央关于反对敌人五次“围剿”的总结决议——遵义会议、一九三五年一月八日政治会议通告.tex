\section[中央关于反对敌人五次“围剿”的总结决议——遵义会议、一九三五年一月八日政治会议通告]{中央关于反对敌人五次“围剿”的总结决议——遵义会议、一九三五年一月八日政治会议通告}
\datesubtitle{(一九三五年一月八日)}


听了××同志关于五次“围剿”总结的报告及×××同志的付报告之后,政治局扩大会议认为××同志的报告基本上是不正确的。

(一)党中央关于敌人五次“围剿”的决议中,曾经清楚地指出五次围剿是帝国主义与国民党的反动对于苏维埃革命运动的更加残酷的进攻,但同时指出了在这剧烈的阶级决战中,帝国主义、国民党内部的弱点与革命形势的新的紧张化,这造成了国内阶级力量的对比有新的有利于我们的变动。得出了“在五次围剿中间我们有着比以前更加充分的取得决战胜利的一切条件”的反动结论(一九三四年七月廿日中央决议)。而××同志在他的报告中过分估计了客观的困难,把五次“围剿”不能在中央苏区粉碎,原因归之于帝国主义、国民党反动力量的强大,同时对于目前的革命形势却又估计不足.这必然会得出客观上五次围剿根本不能粉碎的机会主义的结论。

(二)党中央根据于自己的正确估计,定出了反对敌人五次围剿的具体任务。一年半反对围剿的困苦战争,证明党中央的政治路线无疑是正确的。特别中央苏区的党在中央直接领导之下,在动员广大工农群众积极参加革命战争方面,得到了空前的成绩,扩大红军运动成为群众的热情,动员工农群众的分子武装上前钱,达到了十万人以上,使红军大大的扩大了。模范亦少数开始成为红军的现成后备军,赤少队的群众武装组织有了极大的发展。党在“一切为了前线上的胜利”的口号下解决了前方红军财政上的粮食上的与其他一切物质上的需要。苏区内部阶级斗争的深入,苏维埃的经济建设以及苏维埃政府与群众的关系彻底的改善,更大大的发展了广大群众参加革命战斗的热情。一切这些造成了彻底粉碎五次围剿的有利条件,而××同志在他的报告中,对于这次顺利的条件,显然是估计不足的.这种估计不足,也必然得出在主观上我们没法子粉碎围剿的结论。

(三)应该指出在我们的工作中还有许多严重的缺点:党对于扩大白区工农群众的反帝反国民党与日常斗争的领导依然没有显着的进步,游击战争的发展与瓦解白军士兵工作依然薄弱,吾苏区红军在统一战略意志之下的相互呼应与配合还是不够,有些弱点无疑也要影响到反对五次围剿的行动,成为五次围剿不能粉碎的重要原因。但决不应该以为这些弱点的存在,乃是不能粉碎五次围剿的主要原因,而××同志在报告中与结论中都夸张这些工作的弱点,对军事领导上战略战术基本上是错误的估计,与又不认识,与又不承认,这使我们没有法子了解我们红军主力不能不离开中央苏区与我们不能在中央苏区粉碎围剿的主要原因究竟在哪里,这就掩盖了我们在军事领导上战略战术上的错误路线所产生的恶果,红军的英雄善战,模范的后方工作,广大群众的拥护,如果我们不能在军事领导上运用正确的战略战术,则战争决战的胜利,是不可能的,五次围剿不能在中央苏区粉碎的主要原因正在这里。

(四)国民党蒋介石以及他的帝国主义军事顾问等,经过四次围剿失败之后,知道用“长驱直入”的战略战术同我们在苏区内作战是极端不利的。因此五次围剿中,采取了持久战和堡垒主义的战略战术,企图逐渐消耗我们的有生力量与物质资源,紧缩我们的苏区,最后寻求我主力作战,以达到消灭我们的目的。

在这种情况下,我们的战略路线应该是决战防御(攻势防御),集中优势兵力,选择敌人的弱点,在这运动战中,有把握地消灭敌人之一部或大部,以各个击破敌人,彻底粉碎敌人的围剿。然而在反对五次围剿战争中,都以单纯防御路线(或专守防御)代替了决战防御,以阵地战堡垒战代替了运动战,并以所谓“短促突击”的战术原则来支持这种单纯的防御战略路线。这就使敌人持久战与堡垒主义的战略战术,达到了他们的目的,使我们的主力红军受到一部分的损失,并离开了中央苏区根据地。应该指出,这一路线同我们红军取得胜利的战略战术的基本原则,是完全相反的。

(五)目前在中国国内战争的阶段上,在我们还没有大的城市工人的白暴动军士兵的哗变的配合,在我们红军数量上还是非常不够,在我们苏区还只是中国的一小部分,在我们还没有飞机大炮等特种兵器,在我们还处于内线作战的环境,当着敌人向我们进攻上举行“围剿”时,我作战战线,虽然是决战防御,即使我们的防御不是单纯的防御,而是为了寻求作战的防御,是为了转入反攻的防御,单纯防御可以阻挡削弱敌人力量,可以在某一时期内保持土地,但最终的粉碎敌人的“围剿”以保卫苏区是不可能的,最后胜利的前途是没有的,只有从防御转入反攻(战役的与战略的)以至进攻,取得决战的胜利。大量消灭敌人的有生力量,我们才能粉碎敌人,保卫苏区,发展苏区革命这动。

在这一战略战线之下,当我们还没有发现和造成敌人的弱点时,我们对于进攻的敌人不应该与之进行无胜利把握的决战。我们应该以次要力量(如游击队群众武装,独立营团,部分主力红军等),在各方面迷惑或引诱敌人,在这方面,主要的运动钳制敌人,而主力则退至适当距离或转移到敌人侧翼后方,隐蔽集结,以寻找有利时机,攻击敌人,在内线作战下,当敌人以绝对的优势兵力向我们前进时,红军的退却与隐蔽,足以疲劳敌人,使敌人骄羚懈怠发生过失与暴露弱点,这就是创造了转入反攻取得决战胜利的条件,要最审填地分析与判断敌情,以便适时恰当的布署战斗。不要与由于敌人向我们挑衅与佯攻,而不必要地调动我们的力量与投入战斗,使我们疲于奔命,失去了在一定方向取得决战胜利的机会,为了求得胜利,当敌人按照某计划前进时,我们在突击向用不着去阻止它,应该待到进至适当距离,然后包围消灭之(即诱敌深入)。为了求得胜利,即使暂时放弃一部分苏区的土地,甚至主力暂时离开苏区根据地,都是在所不辞的,因为我们知道,只要我们能够消灭敌人,粉碎敌人的“围剿”而避免一切被动的与不利的结果。

然而在五次战争中,对于这些原则都统统是违反的。共产国际去年二月来电说得很对,“我们觉得似乎在目前这时期中,军事指挥所采用的计划和步骤,差不多可以说常常是敌人迫近而产生的。敌人和我们挑拨,使我们常常必要的改组我们的力量。因而我们的力量由于继续不断地变动,就不能积极地加入决战。我们觉得应该在那些我们已经获得了某些胜利的地方击败敌人,不要企图在全部战线上同时击败敌人。”单纯防御路线的领导者,对于共产国际的这种指示是无法了解的,所以不但去年二月以前是如此,直至主力红军退出苏区仍是如此。甘心情愿把自己处于被动地位的单纯防御路线,并不是,也不能企图在全部战线同时击败故人,而且企图在全部战线同时阻止敌人。×××同志过去提出过的“全面出击”的口号,在五次战争中则变为全线防御,而在战略上则一者都是错误的。“不放弃苏区寸土”的口号,在政治上是正确的,而机械地运用到军事上尤其在战略上,则是完全错误,而适足成为单纯防御路线的掩盖物。

(六)为了求得决战的胜利,在决战方面,集中优势兵力是绝对必要的。在目前敌我力量对比上,敌人的兵力是绝对占优势,他们常常拿多于我们数倍至数十倍的兵力向我们进攻。然而这对于我们不是可怕的。由于敌人是处于外线,战略上采取包围与分进合击的方针,这就造成了我{门各个击破敌人的机会,使我们在战略的内线作战下,能够收到战线的外线作战(局部的外线)的利益,即是以我军的一部钳制敌人一路或数路而集中最大力量包围敌人一路而消灭之,用这种办法去各个击破敌人,粉碎敌人的“围剿”。在战略的内线作战情况下,只有集中优势兵力寻求战线的外线作战取得胜利,才能使红军经常握住主动权,敌人则迫使他陷入被动地位,而最后打破他的整个计划。

但是过去单纯防御路线的领导者,为了防御各方面敌人的前进,差不多经常分散(主要是一三军团的分散)兵力。这种分散兵力的结果,就使我们经常处于被动地位,就使我们的兵力处处薄弱,而便于敌人对我们各个击破。五次战争中,许多次的战役(如三河口战役,团村战役,建宁役战,温防战役等等)都由于我们主力不集中而未能得到伟大的胜利。单纯防御路线的领导者给红军的中心任务,是阻止敌人的前进与企图以“短促突击”消灭部分的敌人,而不是争取主动权,不是争取决战的胜利。其结果就是红军消灭敌人的数量数少,而苏区也终于受敌人的蹂躏。

(七)在运动战中消灭敌人,是我们工农红军的特长。共产国际在敌人五次“围剿”开始时(前年十月来电)即向我们指出:“我们的行动不应该釆取阵战方式,而应当在敌人的两翼采取运动战。”去年三月来电又重复的说:“很明显的,根据过去的经验,我们的队伍在运动战中已经获得了许多伟大的胜利,但不能在强攻敌人堡垒地带的作战中,获得胜利。”国际这些指示,是完全正确的。在五次“围剿”敌人堡垒主义下,我们虽没有象在一二三四次战争中当敌人“长驱直入”时釆取大规模运动战的机会,然而运动战的可能依然存在,事实上已经多次的证明了(沟口、团村、将军殿、建宁、湖坊、温坊各役,特别是十九路军改变时)。然而五次战争中,由于对堡垒主义的恐惧所产生的单纯防御路线占华夫同志的“短促突击”理论,却使我们以运动战接受到阵地战,而这种阵地战的方式仅对于敌人有利,而对于现时工农红军是极端不利的。

强攻敌人的堡垒,在目前技术条件下,是应该拒绝的。只有这堡垒不坚固或孤立的情况下,为了打击敌人增援部队,或为了调动敌人的情况下,才容许攻击敌人的堡垒。五次战争中轻易强攻堡垒,其没有任何效果是不足为奇的,因为这是把战争当儿戏。

对于五次战争中运动的可能估计不足,因而把敌人五次“围剿”绝然的分开,因而绝然否认过去运动战的经验,绝然否定诱敌进来给以消灭的战法,并且不得不在实际上拒绝共产国际的正确指示,在这单纯防御与短促突击的领导者是自然的道。

(八)由于对敌人堡垒主义的估计过高与对运动的估计不足,便产生了胜利只能起始于战术上的理论,以为只有战术上的胜利,才能转变为战线上的胜利,然而由战线的胜利才能起战线上有利于我们的变化(华夫同志的文章及××,××两同志给林彪,×××两同志的信),以为“在堡垒主义之下,只能有许多小的胜利,而不能有痛快林离的胜利”(见××同志政治局发言及××同志红军报的文章),认为只有分兵抵御与短促突击才能对付堡垒主义。所有这些革命战争中机会主义战略战术的理论与实际,在五次战争中是完全破产了。

我们不能否认堡垒主义造成了粉碎敌人五次“围剿”的新的困难。然而他们最近却以左的空谈轻视堡垒主义,(见××同志红军报文章),不否认而且应准备红军的技术条件(飞机大炮)特别是堡垒主义内部工农士兵暴动,以战胜将来敌人更坚固的堡垒,但就在现时条件下,堡垒主义也是能够粉碎的。堡垒主义疲劳了敌人的精力并分散了兵力,养成了敌人对于堡垒的依赖性,使他们脱离了堡垒即失去其胜利的信心,同时敌人无法不脱离堡垒向我们前进,又无法在全国范围内遍筑足以限制红军活动的堡垒。一切这些造成了使我们能够克服堡垒主义的顺利条件。因此我们红军粉碎堡垒主义的方法,依然是依靠于运动战,依靠在堡垒线前左右发展游击战争以配合红军的行动,以及依靠深入的军士兵运动。所谓运动战粉碎堡垒主义,在堡垒线内即是待敌人前进时大量消灭敌人的部队,在堡垒线外,即是在红军转到广大堡垒主义地带活动时,迫使敌人不得不离开堡垒来和我们做运动战,只要我们灵活的、艺术的,出奇制胜的运用运动战的战略战术原则,我们就一定能够粉碎敌人的堡垒主义。而且只有正确的战略方针,才能正确的领导战线(争),并正确地运用战术,以粉碎堡垒主义与粉碎“围剿”单纯防卸与“短促突击”,胜利主要不依靠于战略战役的正确指导,而仅仅依靠于战术,事实上只是对于堡垒主义的投降,至底不能粉碎堡垒主义。

(以下略)

(九)在持久与速决战问题上(略)

