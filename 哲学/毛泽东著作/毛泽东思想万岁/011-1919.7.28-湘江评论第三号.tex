\section{湘江评论第三号}
\datesubtitle{(一九一九年七月二十八日)}

\subsection{世界杂评}

畏德如虎的法兰西法国于德国畏惧他如虎狼。德国这么大败,法国尚畏惧得很。割萨尔煤矿,划来因左岸独立,毁希里哥伦炮台,力波兰独立以蹙其东陲,助捷克独立以阻其南出。日耳曼奥地利也并归德国,则不惜破坏民族自决主义,多方以妨之。殖民地,陆海天空军备,则多方以消之。商船亦须交出大部,以阻其海外贸易之恢复。这样也算够了,还不止此。又向英美两国,请求保卫。前日电传,威尔逊于离法以前,签订一约。系证将来法国一受攻击,美国当起而援助。劳合乔治亦以英国名义,签定一同一性质的条约。此意何等深刻!何等惨淡!籍非法国有不可告之大缺点,何至有这样的畏惧。法兰西民族素负豪气,何至竟象妇人孺子,斤斤乞人保护。我觉得这不是法兰西的好现象!

和约的内容斯末资将军说:“我之签订和约,非因和约乃满意的文件。为结束战争起见,不得不签订他。”又说:“新生活,人类大主义的胜利,人民趋向于新国际制度和优善世界,所抱如此希望之践行,象这样的约言,均没有截上和约。于今只有国民心腔里抒发义侠和人道的新真意,乃能解决和会里政治家困难而止的问题。”又说:“我很以和约里取消黩武主义,仅限于敌国为憾。”斯末资是英国一个武人,是手签和约的一个人,他于签约后所发议论是这样,我们就可想见那和约的内容。

日德密约巴黎的路透电说:“近今外间又有日德密约的谣传。密约是什么东西?还有什么妄人想发现于今后的国际间么?日德密约更是什么东西?日俄密约,为列宁政府所宣布了,不但没成,反丢了脸一大抉。日英法密约成了,我们的山东就要危险。什么日德密约,前年也谣传了多次。据说一千九百一十七年,德国允许日本自由处置荷兰的殖民地,爪哇苏门答腊在内,为英政府听至,告诉了荷兰,阴谋方止。我们应知道日本和德国,是屡次寻求未遂的狗男女,他们虽未遂,那寻奸的念头,是永远不×打断的。日本的强权政府军阀浪人不割除,德国的爱贝尔政府不革命,娼夫和淫妇,还未拆开,危险正多呢。

政治家斯末资云:“惟人民的新真意,乃能解决和会里政治家困难而止的问题。”人民的真意,和政治家的见解,何以这么不相同?政治家何以这么畏难?人民何以这么不畏难?这里面果有一层解释呢?我自来疑惑所谓“政治家”,怕英不是一种好东西?我如今获得了证据。巴黎和约签订后,路易乔治回到英国,在下院演说道:“我们英国,得到许多成功是我们伟大国民团结兴奋的动力。我们于今欢欣鼓舞,但不要存着祸患业已过去的妄念。已使我们获胜的精神,仍要保持,以应付将来事件。”我们不要消费精力于彼此相争。这就是政治家的大本领,这就是政治家的大魔力。不要浪费精力于彼此相争。就是说道,你们人民不要拿着生活痛苦,国民真意,种种无聊问题,和我们政府为难。那些问题都小,都不关痛痒。将来寻着事端,我们还要和别国打仗。爱国,兴奋,团结,对外,是最重要没有的。我正式告诉路易乔治这一类的政治家,你们所说的一大篇,我们都清白是“鬼话”,是“胡说”。我们已经醒了。我们不是从前了。你们且收着,不要再来罢。

\subsection{湘江杂评}

不信科学便死两星期前,长沙城里的大雷,电触死了数人,岳麓山的老树下一个屋子里面,也被雷触死了数人。城里街渠污秽,雷气独多,应建高塔。设避雷针数处。老树电多,不宜在他的下面第屋,这点科学常识,谁也应该晓得,长沙城里的警察,长沙城里三十余万的住民,没一人有闲工夫注意他。有些还说是“五百蛮雷,上天降罚”。死了还不知死因。可怜!

死鼠鼠是瘟疫发生的一个原因,长沙城里到处看见死鼠,张眼望警察,警察却站在死鼠的旁边,早几年的长沙城,都没有这个样子,警察先生们,还是请你们注意点。
