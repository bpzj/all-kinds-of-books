\section[关于部队中民主生活的指示(一九四八年三月)]{关于部队中民主生活的指示}
\datesubtitle{(一九四八年三月)}


用民主讨论方式,发动士兵群众,在作战前、作战中、作战后讨论如何攻克阵敌,歼灭敌人,完成战斗任务,特别是在作战中放手发动连队支部、班排小组,反复讨论如何攻克敌阵,收效极大。陕北攻克蟠龙敌整一师一六七旅阵地的经验,也是如此。当时有一个团打了几天,上面认为无法打了,下令撤退,但连队认为可打,不肯撤。连队战士分组讨论,找出了办法继续打,结果获得胜利。陕北将此种情形叫做军事民主,而将诉苦运动,三查三整叫做政治民主与经济民主,这些军队中的民主生活,有益无害,一切部队均应实行。

