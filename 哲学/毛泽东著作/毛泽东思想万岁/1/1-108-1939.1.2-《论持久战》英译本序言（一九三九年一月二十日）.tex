\section[《论持久战》英译本序言(一九三九年一月二十日)]{《论持久战》英译本序言}
\datesubtitle{(一九三九年一月二十日)}


上海的朋友在将我的《论持久战》翻成英文本,我听了当然是高兴的,因为伟大的中国抗战,不但是中国的事,东方的事,也是世界的事。民主国家如英、美、法,有广大民众,包括各个阶层的一切前进人们,都是同情中国抗战,反对日本帝国主义侵略中国的;除了一部分顽固党反对中国抗战。关于顽固党,有的是顽固成性一向同情日本军阀的,有的则是不明白中国抗战的必然规律,经过艰苦路程日本必败中国必胜这个必然规律。因而由悲观而失望而不愿意援助中国,这类人我想也不会有的。倘能因我的书给于这类人以明白事情真相的机会当然是我的希望。至于大多数同情中国抗战的人们,也许至今还有若干人同样不明白中国抗战的真相,虽同情抗战也存在着苦闷,这类同情的苦闷,尤其是我们应该为之解释的。我的这本小书是一九三八年五月间作的,因为它是论整个中日战争过程的东西,所以它的时间性是长的。至于书中论点是否正确,有过去的全部抗战经验为之证实,今后经验也将为之证实,抗战在武汉、广州失守后正在向着一个新的阶段一一有利于中国不利于日本的新的阶段的发展,这个阶段就是敌我相持的阶段,敌因被迫结束其战略进攻转入战略保守,我因坚决抗战与力量增加而结束自己的战略退却(主力军,不是游击队),转入战略相持,这种局面快要到来了。新阶段中,我之全部任务在于准备反攻,这种准备时间也许是长的,但我们有全部勇气与精力来进行这种准备。一定要把也必然能把日本帝国主义赶出中国去,在伟大抗战中,根本的依靠中国自力更生,中国的力量也正在发动,不但将成为不可战胜的力量,且将压倒敌人而驱逐之!这是没有疑义的,但同时需要外援的配合,我们的敌人是世界性的敌人,中国的抗战是世界性的抗战,孤立战争的观点历史已指明其不正确了,在英美诸民主国尚存在孤立观点,不知道中国如果战败,英美等国将不能安枕,这种错误观点十分不合时宜,援助中国就是援助他们自己,才是当前的具体真理。因此我希望此书能在英语各国间唤起若干的同情,为了中国利益,也为了世界利益,中国在困难之中进行战争,但世界各大国间的战争火焰已日益迫近,任何国家要置身于事外是不可能的,我们同意罗斯福总统保卫民主的宣言,但坚决反对张伯伦对于西方法西斯国家的退让政策,张伯伦对于日本至今还保持着怯懦心理。我希望英美民众积极起来,督责其政府采取反对侵略战争的新的政策,为了中国也为了英美自身。

<p align="right">(原载一九三九年二月十五日《八路军军政杂志》二期)</p>

