\section[接见非洲外宾时的谈话(一九六○年五月七日)]{接见非洲外宾时的谈话(一九六○年五月七日)}


欢迎朋友们。我们是朋友。我们和你们是站在一条战线上,共同反对帝国主义、殖民主义。帝国主义国家大多数不承认我们。他们实际上统治中国一百多年。使中国变成一种很贫穷的状况,变成一穷二白。穷就是贫困,白就是文盲多。这种状况现在开始有了改变。中国过去名义上是独立国,实际上是帝国主义的半殖民地。中国有六亿多人口,可是蒋介石只有十万吨钢,比利时那么小的国家,就有三百万吨钢,所以我们很穷。我们经过几十年的斗争,才得到解放。光武装斗争我们就打了二十二年,整个中国大陆在一九四九年解放了,只剩下台湾,还被帝国主义霸占着。美帝国主义在东方建立了很多军事基地,譬如台湾、南朝鲜、菲律宾、南越、泰国、巴基斯坦,对我们的威胁很大。

西方人说我们不行,说我们中国人不行,说我们是有色人种,有色人种就是不行的。是成不了事的,是不讲卫生的,很脏的,不高尚的。我们这个人种似乎和你们非洲人差不多,似乎西方人也说你们不行,又不帮助你们发展工业,就是发展一点工业,也是属于帝国主义所有的。所以我们的地位相同。你们现在很好,团结起来了。整个非洲团结起来、觉悟起来了,或者正在一步步地觉悟中间。你们非洲有两亿多人口,你们团结起来,觉悟起来,组织起来,帝国主义是怕你们的。帝国主义散布恐怖情绪,他们杀人,或者经过他们的走狗杀人。在中国是经过蒋介石杀我们。你们国家也可能有这样的人,承帝国主义的意旨办事。这样的人很少,顶多十个人中有一个,或者不到一个。所以你们可能团结的人,十个人中有九个,或者更多。实际上帝国主义是不可怕的。帝国主义每天都宣传他们的力量大,来恐吓我们。从前我们中国人也曾有一个时期怕美国人,怕帝国主义,怕他们的走狗蒋介石。因为他们杀人,或者用各种别的方法,譬如,把人抓起来,关在班房里头,总而言之,要使我们怕美国人,消灭我们的斗志。我们中国人也是一步一步觉悟起来的。后来就慢慢不怕了,跟他们的走狗打了。我们开始是手无寸铁的,又不会打仗。我们从他们手里学了这个办法。你可以压迫我们,我可不可以压迫你呢?十个人中间有一个压迫我们九个人,我们九个人可不可以团结起来把那一个人赶走呢?得出结论,我们说,可以。一个人压迫九个人,我们九个人不团结起来,把他赶走,这是没有道理的。结果我们闹了几十年革命,还不是胜利?!我们的敌人有强大的外国援助,是美帝国主义援助蒋介石。他们的武器很强大,他们有兵工厂,有外国人送他的武器,有军舰,有重炮,有坦克,有空军。我们什么都没有,也没有炮,也没有飞机,也没有坦克。我们有的是步枪、轻炮。这些东西是那里来的?不是兵工厂造的,而是抢的,是战争中得来的。美帝国主义经过蒋介石把枪送给我们,我们就有了枪炮了。后来我们又有了坦克、重炮了,我们就可以打大仗了。一九四九年我们就占领了大陆。他们的空军每天在我们头上轰炸,也没有把我们吓倒。后来变成他们害怕我们了,不是我们怕他们,而是他们怕我们了。后来蒋介石也怕我们,美国人也有点怕我们,因为我们把百分之九十以上的人都团结起来了。还是人要紧;武器是第二位的,是次要的。只要把人团结起来,手里掌握着武器,殖民主义者就怕我们。当然不是只有打仗这一种办法,还有别的办法。你们在座的有些国家不是用打仗夺取政权的,像几内亚。阿尔及利亚现在还在打仗,阿尔及利亚的战争帮助了几内亚,几内亚的朋友们也是这样看,是不是真的?(几内亚的外宾说:是的。)因为法国有五十万军队被阿尔及利亚人吸引在他们的国家中,它就没有好多兵了。帝国主义占的地方太宽了,中国俗话说,十个指头按着十个跳蚤,一个跳蚤也捉不到。(全场活跃)因为它管的太宽了。美国现在在世界上占的地方太多了。你看,在日本、台湾、南朝鲜、菲律宾、南越、还有拉丁美洲、非洲,有好多国家都有美国的军事基地。它要控制欧洲,还要控制土耳其、伊朗、巴基斯坦。这几天情形有一些变化。南朝鲜人民没有出路了,起来反对美国的走狗李承晚。李承晚是美国的走狗,是一个老走狗。(笑声)南朝鲜人一起来,一骂一轰,几十万群众一示威,李承晚就垮台了。李承晚有七十五个师,而南朝鲜人民群众一支枪也没有,可是他们一起来,李承晚就倒了。当然,现在问题还没有解决,美国人还在南朝鲜,选择了新的走狗许政。他们的斗争还会发展下去。还有土耳其的群众也起来反对美国的走狗。所以我们这几天举行群众大会支援南朝鲜人民,又举行群众大会支援土耳其人民。还有日本人民正在起来。今天是七号,后天日本将有广大的群众运动,听说有几十万或几百万人起来反对岸信介政府和美国订立的军事同盟条约。我们也要举行群众大会支持日本人民群众。你们可能有人说,南朝鲜、日本、土耳其离美国很远,因此他们不怕美国,敢于起来反对他的走狗。但是请你看一看古巴。古巴在什么地方?离美国很近,飞机航行距离只要半小时。古巴人民也是手无寸铁的,古巴的统治者巴蒂斯塔在几年中杀死古巴人两万人之多。你们也可能说,中国是一个大国,人多。古巴可不是个大国,只有六百万人。离美国那么近,人口只有六百万人,巴蒂斯塔在六百万中间杀死过两万人。但是一九五七年七月二十六日,古巴的民族英雄××××率领八十二人,从墨西哥坐了一只小船,到古巴登陆。开头第一仗打了败仗,八十二人只剩下八个人,其中有××××和他的弟弟××·××××。他们只好隐藏起来。然后又进来,一下子打了两年半,抢了许多枪炮,还抢了坦克,巴蒂斯塔只好跑了。你看,古巴人民也是手无寸铁,而巴蒂斯塔是武装到牙齿的政权,美国那么大的国家支持他,离得那么近,但是人民团结起来就把巴蒂斯塔赶跑了。你们有没有人到古巴去过?如果没有人去过,我们建议你们到古巴走一趟。研究古巴的经验很有必要,这么个小国敢于在美国身旁搞革命,所以古巴的革命有世界意义。整个拉丁美洲人民都欢迎古巴的政权。

非洲的反殖民主义的斗争更有世界意义。不是一个国家,而是很多国家都有革命。不只是在几百万人中间,而是在几千万或者更多的人口中进行革命的民族解放斗争。我们完全同情你们,我们完全支持你们。我们认为你们的斗争支持了我们,你们帮助了我们。我们认为古巴的斗争帮助了我们,整个拉丁美洲的斗争帮助了我们。我们认为南朝鲜、土耳其、南越、日本这些国家的斗争帮助了我们,整个亚洲人民都帮助了我们,当然社会主义国家首先是帮助我们的,××是帮助我们的。在社会主义国家之外,亚洲、非洲、拉丁美洲人民的广大的反殖民主义的斗争帮助了我们。这就分散了敌人的力量,使我们身上的压力减轻了。因此,我们有义务要支持你们,因为你们帮助了我们。我们是互相支持,互相帮助。同时我们也支持大国会议。四国首脑会议将在法国召开,这也是一种办法。按照我们中国的说法,这也叫两条腿走路。大国会议是跟他们在桌子上谈,这是一条腿;亚洲、非洲、拉丁美洲人民反殖民主义、反帝国主义的斗争,又是一条腿。两条腿走路就好走,可以站起来,少了一条腿就不好,就不好走路了。我们相信你们也赞成不要打世界大战,世界大战我们是反对的。但是我们同时赞成受帝国主义压迫的各国人民有权利起来反对他们的压迫。要不打世界大战,就要各国人民起来,反对压迫者。我们可以举一个具体例子。例如阿尔及利亚,牵制了法国五十万军队。假如打世界大战,法国参加的力量就很少了,它只有那么多军队。南朝鲜人民起来,就牵制着美国驻南朝鲜的军队。日本人民如果也起来,又可以牵制美国一部分军队。土耳其人民起来,又牵制着美国一部分军队。有些人说,要世界和平,就不要反对帝国主义,免得帝国主义不高兴,各国都不要搞反对帝国主义的斗争。我看还是两条腿走路。各人民起来对压迫者进行反抗,这是一条腿,我们说这是一条重要的腿,也许是第一条腿。跟他们在一起在桌面上开大国会议,讲什么裁军、解决德国问题这些事情,也是一条腿。两条腿走路,世界大战就难于打了。如果只一条腿,帝国主义不打世界大战就没有保证。

这是我们中国人讲的一些意见,也许你们不一定赞成。我们是交换意见的性质。请你们讲一讲好不好?你们的情形和意见我很愿意听。

〔索马利兰的朋友讲到争取独立的情形时〕谢谢。祝贺你们所有的爱国者,祝贺你们的胜利。你们一定会胜利的,哪有不胜利的道理?

〔马达加斯加的朋友讲到争取独立的情况时〕总有一天,你们会完全独立的。


(喀麦隆的青年代表:我们赞成主席所说的两条腿走路的办法。但是实行起来我们有些怀疑。帝国主义、殖民主义者从来不实践自己的诺言,他们举行各种会议都不过是一些阴谋。我看首脑会议这一条腿,是一条有病的腿;还是另一条腿比较健全,更为重要,那就是反殖民主义那一条腿。)

讲的对,我属于你一派。帝国主义是会搞欺骗的。帝国主义也有两条腿,有欺骗的一条腿。对于帝国主义的欺骗,我们和你们一样是怀疑的,因为他老欺骗。但为什么要支持大国会议呢?是借此看一看,看一看就可以暴露它,暴露它那一条腿有病。(全场活跃)

〔阿尔及利亚工会代表在发言中说要和平共处必须排除殖民主义时〕对,有殖民主义怎么共处啊?

〔阿尔及利亚学生代表在发言中表示了对法帝国主义进行长期斗争的决心,并说:法帝国主义在我们的周围建立了强大的壁垒,要想窒息我们,我们只有要求朋友帮助,我们不需要人力,而是需要物资帮助,这是阿尔及利亚战争的特征,希望毛主席对此发表一些意见。〕

再讲几句。我赞成这位朋友所讲的这种思想,像阿尔及利亚这样的国家以及大体上和它相同的国家,要准备进行长期的斗争。有这样的精神准备比较有利。困难是有的,有时是很大的困难。我说过,中国的斗争光是武装斗争就是二十二年,你们的斗争才进行了六年。我们进行了二十二年,还犯了几次错误,犯过右倾机会主义的错误两次,犯过“左”倾机会主义错误三次,曾经使我们的力量遭到很大损失,军事力量原来有三十万人,因为犯错误,还剩下不到三万人,不到十分之一。重要的是这个时候不要动摇。三万人比三十万人哪个更强大?因为受到了教训,不到三万人的队伍,要比三十万人更强大。后来得到机会,又发展到一百万人,这是一九四五年日本投降的时候。一九四六年,美国和蒋介石向我们进攻,美国不是亲自出面,而是用帮助的办法支持蒋介石,曾经使我们丧失很多地方,丧失许多城市。它全面向我们进攻,我们采取后退的策略,消灭敌人的有生力量。一年十个战役,地方丧失很多,但是敌人的军力被我们消灭了一百多个师,这时我们开始反攻。到一九四九年,我们变为优势,蒋介石变为劣势,大部分军队被我们消灭,我们占领了沈阳、北京、天津、济南、郑州等许多城市。他们的地方被我们占领,主力被我们消灭。这时,他们要求讲和,派代表到北京。我们就用两条腿走路的办法。我们知道他们讲和是骗人的,但是如果我们不讲,老百姓不相信,似乎蒋介石爱好和平,似乎我们爱好战争了。好吧!就讲和吧!派代表吧!这时他们派来了一个代表团,我们讲了三个星期,我们说,你们要缴枪,把政权交给我们。他们代表签了字,派人到国民党政府所在地南京去,请求批准。蒋介石说不行,不能缴枪,不能移交政权,这就撕破了他的和平面目。他们今天拒绝签字,明天我们就渡过了长江,这一条腿就伸过去了。(全场活跃)敌人经常欺骗我们,我们要看得清楚。有时需要接受谈判,在谈判中揭露它。两条腿就是这么走的。不是投降敌人,而是要敌人投降。譬如现在世界人民要裁军,我们赞成,看你裁不裁,你裁,那很好,不裁就证明你是欺骗。要揭露敌人,要用各种方法揭露敌人。和平谈判实际上就是一种揭露的方法,我们是这样看的。我们不相信艾森豪威尔爱和平。帝国主义哪里爱和平?他们爱好的是殖民主义。

我们高兴地看到非洲朋友有这么多人破除了迷信。迷信的第一条就是怕帝国主义。你们破除了这一条,不怕帝国主义了。但是我相信,你们非洲二亿人口中还有些人怕帝国主义,对帝国主义还是有迷信的,或者说是有幻想的,因此你们还要向他们作工作。有十年、八年,慢慢地人就多了,两亿人口中可以有一亿人或者一亿多的人,完全破除迷信,站起来,不怕帝国主义,胜利就有把握了。人常常是有很多迷信的。迷信帝国主义是迷信的一种,再有一种就是不相信自己的力量,觉得自己力量很小,觉得我们不行,他们西方世界很行。他们白种人可以干的事,我们黄种人、黑种人、棕色种人都可以干,而且还可以比他们干得好些。因为他们人数很少,只有几亿。而且白种人并不是坏人,只有十分之一的坏人,十分之九是好人,或者暂时受人欺骗,不觉悟,总有一天他们会觉悟起来的。这就是无产阶级还有其他同情无产阶级的人——劳动者,农民。真正怕原子战争的,白种人也有,有些资本家也怕,他们几个国家相互之间闹矛盾,所以有机可乘。他们并不那么团结,美国人和英国人并不那么团结,并美国人和西德人也不是那么团结的,阿登纳同英国人也不对头。所以全世界劳动者,受帝国主义压迫的爱国人民,同盟军是很多的。

我们得出一条经验,在战略上不怕敌人。帝国主义已经削弱了,十个指头已经砍掉了一个、二个、三个了。在苏联,沙皇没有了,变成了列宁主义的苏联了。中国也脱离了帝国主义的统治。除了这两国以外,还有十个社会主义国家,这是帝国主义的指头,也都砍掉了。剩下的是亚洲、非洲、拉丁美洲,有些国家已经独立,有些国家正在争取独立。可以说帝国主义剩下的这几个指头也受了伤了。譬如古巴就在美国旁边,把美国的走狗赶跑了,阿尔及利亚有很大一块解放区;还有几内亚也独立了,非洲还有其他几个独立的国家。看起来,很大的风暴,正在非洲掀起来,同样的风暴正在拉丁美洲酝酿。有人说,亚洲最近几年民族独立运动比较低落,可是一九五八年七年十四日伊拉克革命,一九五七年苏彝士运河战争,帝国主义没有胜利,埃及得到了胜利。最近几个星期又有南朝鲜、土耳其人民起来。看起来日本人民也有希望,所以现在帝国主义睡不着觉。朋友们讲到有些国家有困难,有忧愁。我们认为有高兴的一面,又有忧愁的一面。我看帝国主义只有忧愁的一面,高兴的方面看不见,你说美国人能睡得着觉?我不相信。还有英国、法国,还有什么比利时,阿登纳,还有日本政府。中国有一句俗语:他们是十五个吊桶打水,七上八下。所以我们在战略上完全有理由轻视他们。帝国主义制度是要灭亡的,全世界人民是要站起来的,这是在战略上讲。在战术上讲呢?我们要谨慎,每一个步骤都要好好研究,要重视它,要认真办事。合起来就是,战略上藐视敌人,战术上重视敌人,这样才能敢想、敢说、敢做。大家要看一看中国的经验,我们很欢迎。有些经验也许可以做你们的参考,包括革命的经验和建设的经验。可是我要提醒朋友们:中国有中国的历史条件,你们有你们的历史条件,中国的经验,仅仅只能作你们的参考。

祝贺我们的团结。由于团结我们一定会胜利。祝贺我们的胜利,让我们团结起来取得胜利。


