\section[蒋介石李宗仁优劣论(一九四九年二月二十一日)]{蒋介石李宗仁优劣论}
\datesubtitle{(一九四九年二月二十一日)}


桂系首领战争罪犯李宗仁、白崇禧的言论行动,究竟是否和蒋介石、顾祝同辈有区别的问题,引起了人们谈论的兴趣。

人们说,从一九四九年一月一日起蒋介石谈和平,从同年同月二十二日起李宗仁谈和平,两个人都谈和平,这是没有区别的。蒋介石没有下过如像言论自由,停止特务活动等项命令,李宗仁下了这些命令,这是有区别的。但是李宗仁的命令全是空头支票,不但一样都没有实行,而且人民被压迫的更厉害了。南京方面连和平促进会也封闭了,上海方面屠杀了罢工工人,白崇禧则活像顾祝同。顾祝同命令刘峙炸毁了津浦路蚌埠淮河大铁桥。白崇禧也正在命令张轸准备炸毁平汉路长台关淮河大铁桥及武胜关的隧道工程,积两年半之经验,黄河南北的人民深知桂系军队的野蛮,较之蒋系军队有过之无不及。

人们骂蒋介石为美帝主义的走狗,蒋介石听惯了,从来不申辩。人们骂李宗仁为美帝国主义的走狗,李宗仁没有听得惯,急急忙忙起来申辩。例如李宗仁在一月二十七日经过中央社发表的“致电毛泽东”里面说:“贵方广播屡谓,政府此次倡导和平为政府与某国勾结之阴谋,此种观点系基于某种成见而来。“这里,李宗仁不但替一月二十二日以后的李宗仁政府求洗刷,而且替一月二十二日以前的蒋介石政府求洗刷,人们知道“倡导和平”这件事,蒋介石在前,李宗仁在后。

蒋介石昨天是凶神恶煞。李宗仁白崇禧及其桂系,昨天是凶神恶煞,今天则有些像笑面虎了。

蒋介石撒起谎来,大都是空空洞洞的,例如“还政于民”“我历来要和平”之类,不让人家在他的话里捉住什么具体的事物。李宗仁在这件事上显得蹩脚,容易给人抓住小辫子。例如,在他那个“致电毛泽东”里面说:“现政府方面,已从言论与行动上表明和平之诚意。所有以往全国各方人民所要求者,如释放政治犯,开放言论,保障人民自由等,在逐步施实。“事实俱在,何得谓虚?”人们说;“事实毫无,何得谓实?”李宗仁说:“事实俱在,何得谓虚?”李宗仁就是具有这样一种傻劲的人物。

但是李宗仁也有胜过蒋介石的地方。在应否惩办战争罪犯这个问题上,蒋介石及其死党从来不说可以惩办的话。他们或者不说话,例如在一月二十一日蒋介石的引退文告里对于中共的八条一字不提,或者表示反对态度,例如雷震、朱家驿、潘公展等人所发表的言论,根本反对将战犯当作问题来讨论。孙科也近似这些人,他说和谈条件必须“公平合理”,意思就是反对惩办战犯。李宗仁不是这样,他是又赞成,又反对,这就是李宗仁别致的地方。

李宗仁在其一月二十二日的声明里说:

“中共方面所提八项条件,政府即愿开始商谈。”这即是说,李宗仁的政府愿意即刻开始商谈中共方面所提的惩办战争罪犯一项条件以及其他七项条件,他首先给你一点甜的东西吃。过了六天,李宗仁的腔调变了,而且变得很厉害。一九四九年一月二十七日,国民党反动卖国政府的代总统,在其“致电毛泽东”里面说:“贵方所提八项条件,政府方面已承认可以作为基础进行和谈,各项问题均可在谈判中商讨决定。在双方商谈尚未开始以前,即要求对方必须先执行某项条件,则何得为之为和谈?以往恩怨是非倘加十分重视,则仇仇相报,宁有已时。哀吾同胞,恐无瞧类,先生与弟将同为民族千古之罪人矣。”哎哟哟,李宗仁来得厉害,这一枪非同小可。但是李宗仁的枪法,仍然不过是小诸葛桂系教程里的东西。中国自有孙子兵法足以破之。夫“在双方尚未开始商谈以前,即要求对方必须先执行某项条件”者,是因为南京国民党反动卖国政府自其兵败如山倒以后,即如丧考妣地要求谈判。中共说,好,待我们准备好了你们即来谈。战犯们说,不行,非立刻开谈不可。中共说,你们闲得发慌,给你们一件工作作罢,你们去逮捕一批(自然不是全部)战犯。故事的过程就是这样。后来,中共又将逮捕改为监视,算是作了一个极大的让步,战犯们才安静下来,不再吵闹了。这是后话,不提。且说一月二十七日,李宗仁又说,恩怨是非不要过分重视,即是说不要分清战争责任,不要惩办战争罪犯,那怕黄河以南直至长江,黄河以北至松花江,发生了“惨绝人寰的浩劫”(谨按,此语见之于李代总统一九四九年一月二十二日的文告),那也算不了什么。如果你们一定要惩办战犯,则战犯们的拥护者会要报复的。这种报复,可能达到可怕的程度,即全国同胞中没有一个能吃东西了,都死完了。如此,你毛泽东和我李宗仁两个,将要被我们民族(谨按,既然都死完了,为什么还有民族,待考)判决为一千年那么长久时期内的犯罪者。还好,只有一千年受罪,一干零一年又是一条好汉,这算是李代总统的恩典。

人们请看,李宗仁就是这样反复无常的,又赞成商谈惩办战争犯,又不赞成实行惩办战争犯,他的脚踏在两条船上,这就是他和蒋介石不同的地方。

