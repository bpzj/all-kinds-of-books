\section[基础战术基础战术]{基础战术基础战术}


第一章 绪论

一、民众怎样负担战争行动

赤手空拳的民众,比不荷枪实弹的武装,怎么能够冲锋陷阵,杀敌致果去负担战争的实际工作呢?这是一个很普遍的很合理的疑问。但是我们要知道军队所用的兵器效力,和军队所做的工作的目的在什么地方,那就可明了我们民众在赤手空拳中还有制敌兵器和工作!

军队的兵器的最大效用不过是杀敌,最终的目的不过是削弱或消灭敌人的战斗力,那末,我们日常生活中,那一件东西不可杀敌,那一种行动不可以减少或消灭战斗力呢?譬如,刀,木棍、斧头、锄头、板凳和石头,件件都可以杀人。如剪电线、拆桥梁、造谣言、散毒药、绝接济,处处均可使敌不方便或减少敌人的战斗力,就是我们不肯决心去利用。如果我们实在要去杀敌灭敌,到处都是我们的兵器,到处都有我们的工作,来从事军民联合行动。

二特别注意事件

此外还须特别注意的,就是现在的国际战争,它的残酷使我们意外,时间也很持久,我们不要受着未曾见过的残酷战争痛苦,就马上受着威胁屈服,我们也不要受长期战争的影响,就马上忍耐不住而厌倦了,我们应该以最刚强的精神,最热烈的爱国情绪,和持久的意志,去和敌人作顽强的斗争。我们要知道,战争的状况和时间虽然是残酷和无情的,还不象战争失败了,国家灭亡,使整个民族陷在大劫不复的地位的时候,那是更残酷而无止期的。故而,无论战争如何残酷,最后五分钟的奋斗,我们要绝对的坚持忍耐下来,尤其是我们目下的敌人,是利在战局的速决,而我们失利战局的持久的策略。

三、不要怕敌

我们看见敌人的时候:就不要以为敌人手上拿着武器就如老鼠见猫一样怕得不得了,不敢接近他,混入其中去作破坏的工作。我们是人,敌人也是人,均是人,那么怕什么呢?怕他有兵器吗?兵器我们可用方法夺取过来的,不过是怕给敌人打死。可是受敌人压迫到这步田地,难道还有谁怕死?死还不怕,那还有什么敌人可怕呢?所以我们看见敌人的时候,无论他是多少,要当他是可以充饥的面包,马上就可以将他吞下去。以上各件,是我们在实行扩大的时候,亦应该注意的。

四、游击战术的定义

在陆军主力,不便和敌人作大规模的会战时,故派出别动队或游击队,釆取避实攻虚,飘忽无常,随机制敌。按战术正规的应对敌人的方法,就叫游击战术。

第二章战术

在我们国防设备未完全,武器不及敌人精备的时候,我们要对敌决战,应该守着下列的几个原则。

一、行军警戒

行军的时候,在前卫之前或后卫之后,侧卫之侧的四五里的距离,均须派遣便衣队,身怀短枪,以充侦察,防敌人意外的攻击,或多余的冲突。

二、驻军警戒

我们驻札地方如果有敌人接近之虞的时候,每日对于敌方,必派遣一队作远出二三里的侦探,或作联络地方团体,鼓动民众抗战的宣传工作。若发现敌人,该队一面抵抗,一面报告,使我们能准备警戒撤退,不至打不必要的战。

三、不可攻坚

若敌人固守,或据险抗拒。我们非有特别把握,切不可去攻他,因为攻他很费时日,我们必有倍敌的死伤。且在游击战中,我们炮兵不健全,若轻易攻坚,则一个时期急切难下,四方敌人很容易从四面包拢来。关于这点,军民务要镇静,不可轻易凭一时气愤乱干。

四、不打硬仗

不有十分胜利的把握的仗不可打,因为杀人一千,自杀八百,那是不合算的,尤其是在我们的游击战中,人员弹药补充困难。如果打不胜仗,自己失了许多人员弹药,在我们算是失败。

五、敌情不明不可战

我们驻在某处忽然发现敌人,不知敌人多少,或从何来,则绝不可战,应决然撤退至十里外。如十分接近,只好派掩护队。因为敌来攻,一定兵力占优势或有计划的。我们千万不要上他的大当,如敌大我退,自然得便宜,如敌小我退,不过有点疲劳,再去打也不迟。

六、要有民众组织和联络

现代战争,非军队可单独胜任之事,尤其是在游击战斗中,必须民众的力量,才能有利的把握。因为有了民众的帮助,则凡关于运输,救护,即不幸失败、也有方法逃脱或收容,因此,民众没有组织和联络的地方,不可轻易作战。不要侦探,扰乱等,有很大的便利,同时可陷敌于孤立的地方,则于我之便利特多。即不幸失败,也有方法逃脱或收容,因此,民众没有组织和联络的地方,不可轻易作战。

七、用民众实行封锁的袭击

在敌人由我们包围封锁的时候,我们就要鼓动民众,四方断绝敌人的交通,使敌人不知道我军已在其附近,然后利用暗夜或拂晓将其击溃。

八、单纯军队的袭击

探知敌人,距我于数里之外,警戒必定疏忽,此时我则轻装速进,于拂晓前,出敌不意,把他歼灭。

九、用民众扰敌

以军队主力决定之。在战斗时,以一部分军队,分为两数组,至少以一排为单位,领导地方民团,警察,义勇军,或其他农工民众,打许多军旗,占领四面山头,或村镇,利用铜锣、梭标、土炮、刀矛、喇叭等,满山遍野,叫喊起来,去扰乱敌人耳目,或日夜断续的在四处鸣枪,使他们恐慌,精神疲倦,然后我军以全力出其不意,从侧面击溃它。

十、打圈子脱离敌人

大敌当前,我们无力迎击,则利用打圈子办法,向没有敌人的地方跑,利用山脚使敌人赶不上。同时沿途利用民众,使之担任前后的侦探工作,不致受敌人前后夹攻。

十一、脱离危境

后有追兵,前有阻挡,或追兵过于强大的时候,为要脱离此危境计,可以一部分部队在距离四五里的地方,引敌人走入大道,主力则取间道脱离敌人,或绕敌后而袭击之,或用民团,警察,在另一路上走去,遗些物品,与足迹路标,贴标语,引敌穷追,我主力则从小道冲击,首尾攻击,四面包围而歼灭之。

十二、声东击西

军队要攻击某处,不直接去,绕行他处,到了中途忽然打转,以迅雷不及掩耳的手段击溃他。

十三、埋伏掩击

敌人急急追来,我择地埋伏,候敌到达,可一网打尽。

十四、中途狙击

探悉敌人要从某方前进,中途山峦复杂,道路狭小的地方,派一部兵或射击精良者,藏匿于道傍的山上或树林中,候敌人重要部分经过时,从山上滚石,并猛烈扫射,或狙击其乘马的指挥官。

十五、坚壁清野

侦知敌人将临,而我们的力量又不与战的时候,即须实行坚壁清野的计划,把民众的粮食、锅灶、柴米、器具等悉行藏匿,以绝敌人给养;并将乡区的民众除留老人,妇女及儿童,以备侦探消息外,其所有壮丁悉数率其逃跑使敌人无人与之挑运、带路、探消息。同时更派些人击敌后方交通路,截断敌人的接济,捕捉传令兵等,剪断或破坏通信设备。

十六、对于优势敌人的应付

1、敌进我退:敌人比我们力弱,是不敢攻击前进的,如他向我们前进时,我们可断定敌人定挟优势兵力而来。主动的有计划的和有准备的,则便宜避其锋,预先退却,如中途遭遇,在敌情不明或预知敌军的优势的时候,丝毫不犹豫的掩护退却。

至退却的地区,不宜由大道行远走,而受敌人穷追,必须盘旋于附近,施打圈。敌出我前,便绕其后;敌在山上,我在山下;敌在中间,便退两侧;敌在左岸,便退右岸;敌在右为我便退左。

但退却时可在又路上,我们不走的那条路,故意遗弃物品,或派出一小部分兵马,践踏些足迹,或与些标语或符号,而将走的那条路,用标记截断,使敌其追击方向不明。


在这个时候,当地民众和警察,民团,或义勇军等武装团体,最好由各方面路退去,以混敌人耳目。留一部分在原地,把制服或武器等藏于地下,以小贩之类去造谣,或假献殷勤去探敌人数,及企图或敌人宿营的地点询问,并警戒的情形。如敌人询问我军退却的方向和兵力,就指东说西,指南说北,以小而大,以大而小,胡乱造谣。等军队要未攻打的时候,就把人安置好,武器拿出来从中取利,把敌人打得落花流水,无路可归。

2、敌退我追:敌军退便要乘势前进,因此际敌人军事必有变动。才施行判断没有决心和我们决战。并乘势掩击其后。敌方的掩护部队一定没有决战先头部队。在敌方整个计划上,很难回转。如在山岭崎岖,道路狭小,河流险阻,桥梁很多的地方,即折回亦须很久的时间,即或折回,后卫已被消灭自己解除.

当时民众团体就要设法把敌人退路的桥梁或联络电线破坏,最好等他们的大部分退去,趁着后卫和我方军队对持中.把退路塞住。使他们欲归不得,欲救不能。

但此时民众所负的重大责任,还是要先侦察敌人退却方向。和有无埋伏,诈退,从两侧包围我们的企图,马上报告。使我军得放胆去追,或设法逃避。

3、敌住我扰:敌人新到我们领土内,地势不熟,言语不通,侦探派出无用,等于走入绝境。此时我们再加扰乱,到处枪声,一定使彼睡卧不宁,使彼精神上和肉体上,受很大的影响。哪怕怎么强悍的军队,也要动摇,也要疲劳的。等到他们精神动摇,肉体倦极之际,我军联合起来,一致突入,定可把他们全歼。

十七、对于劣势敌人的处置

为民族生存而战,为达成游击任务,为消灭敌人的战斗力,扶起民众勇敢的精神。对于劣势之敌,当然要联合当地人民群起包围,一举歼灭之。

十八、唤起民众

民众每多贪小利而忘大,往往受敌重利而心为虎作伥。故当敌人到以前,对于当地民众,而尽力鼓吹,唤起他们的抗战情绪,使他们有绝对奋斗到底的精神,不贪利,不妥协、不投降,切实服从指挥,同我军合作抗敌。同时并组织抗战会,和其他职业团体,以便指挥,必要时举行清乡,肃清汉奸为敌人利用。

第三章战争的目的

游击战争最终的目的,固然在解除敌人武装,消灭敌人的战斗力,收回被占的土地,救出被蹂躏的同胞!然而有时因客观状况及种种关系,不能达到此目的时,则完好的土地,完全的给敌人来支配,那是不应该的,因此,我们就要想法子去破坏他的政治,经济和交通,使他们虽然占领我们的土地也没有用,自己自动的退回去!

在游击战争中,要守着“得地不是喜,失地不是悲”的原则。失城失地是不要紧的,最要紧是想法将敌人彻底消灭。因敌人如实力未损,纵然得到城市也保不住。反之,我们力量不足的时候,虽把城市失去,还有恢复的希望,切不可固守城市,而把实力去做牺牲。

第四章组织

一、组织时机

1、在宽阔的地带从事战斗,人烟稀少,文化落后,尤以交通不便,各种通信设备不完全的地区为宜。

2、狭窄的山陵地带,不容许大部队行动的起伏地及隘路附近。

3、敌人后方的人民同情于我军时。

4、敌人的武器精良,实力雄厚,不得不避免真面目的战斗时。

5、敌深入我境内,企图处处施行扰乱牵制敌人的行动时。

6、丛林杂草,深可没人的地区,尤以恰逢青纱帐起的夏暑之季,为最有利。

二、组织的方式

游击队动作.是将以下的三个方式而形成。

1、从本军派出骑兵大部队,并配备骑炮兵,或骑兵随伴排多配属轻自火器,尽可能的迅速潜入敌人的后方,切断后方一切交通联络,从事毁灭其附近粮草弹药存积所。并从该后方分出小的支队和分队,破坏或消灭敌军一切有军事意义的场所,待至任务完成后,迅速的从另一方面以战斗开始道路归还本军。

2、派出骑兵,或步兵特别支队。其兵力应由一排或几连,尽可能的深入敌人后方,以飘忽的行踪,转战东西,如不得已时或敌某时到达时,亦可秘密地巢居该处若干时间,依情况之需要确定,是用全体人员,或是一部。有时以在敌人后方不能再继续驻扎,或是所授任务已经完成后,或因敌人已察觉我之行迹及任务,而加以确实的防卫时。

3、在敌人后方,选当地强悍少壮居民,组成若干分队,受我所派素有经验及训练的人员指导,或是受预先在当地所培植的有经验的人员指导。这些分队秘密的动作,是由此一地区转至彼一地区,变换他们的服装番号与外行,用尽各种方法,掩盖自己的行踪。

4,或由部队征求志愿者,配以精灵的轻武器,以对此素有经验与研究的官长率领,为特种游击队。

5,游击队以素质而别,由挑选自愿者为特种游击队。由部队某一部做普遍组织者,为“基干游击队”。由地方组织者,为地方游击队。基干游击队与地方游击队联合行动时,得由基于游击队队长指挥之。

6,队员之选择,基干游击队员,要由健康、坚定、忍耐、勇敢、敏活的士兵充任之,而这些兵,要自己愿意加入这支队或分队的,因游击时独立行动,其任务是否按照命令做到,往往无法证实,常常脱离上官监视的行动,故队员的选择与训练以“忠实负责”为中心。

?、游击队的支队长和分队长之选择及委任尤须审慎考虑,因为他们的智力,以及忠勇的才干,和军事知识,尤其是游击队动作的战术知识,与相机应变的活泼脑筋,而踏实果敢等,为完成企图与任务必具的条件。

三、兵力

游击队兵力的强弱,以任务需要而定,常由五人十人至数百人,其最大限度不可超过一团。盖如兵力过大.则兵力行动呆滞,粮秣供给困难,难于伪装,难于隐蔽。因此一切企图尚未达到,即被发觉或识破,弹药补充均成问题。又往往受困于不良道路,是以一切企图不独归于泡影,且常陷于往返奔劳之苦。

小游击队主要的优点,就是玲珑敏活,不要多的时间和力量,就可得食物,也易于找到休息场所,既不要多的粮秣和宿营地,更不受恶劣环境所阻止,弹药补充亦易,如不成功,可倏然退回。

四、兵种

游击队之兵种,以骑兵工兵及大运动性兵种为宜。骑兵为担任扰乱敌人的翼侧,或于边击时压迫敌人的后尾,并扰乱其侧方及后方,骑兵均仍为一,且骑兵无论何时,均为游击队唯一的通讯工具和侦探工具。工兵是破坏敌人的后方交通(如铁路电话电线桥梁等),故骑工兵为游击队不可缺少部分,至于大运动性之部队,益使敌有神出鬼没行卧不安之感。

五、武器游击队之武器,除步骑枪各种轻机枪手榴弹外,并须多配以手枪及冲锋枪,机关枪。如地形许可,并附以重机枪及迫击炮或小炮。

六、官兵及行李

轻便敏活为游击队之物质,故一切行李马匹,装具弹药等,均应为极力简单。以求轻便。其战斗员非战斗员,均以适合游击战为适当的编组,其他非必须的人员力求减少。

1、游击之官兵,每班不得超过八名,每排不得超过二十六名,每连不得超过百名。

2、如自动火器较多时,更可酌减人数,游击队亦可以五六名更换行动,收扰乱侦探之收效最大。

3、各主管官之勤务兵,至多不得超过一名,其余则按任务之繁简或二人三人合用一名,更注意不得滥增传令兵代替勤务兵之用,或同一任务派遣不必要之多数传达,致减少本身战斗力量。故当派遣传令兵时,其能否达到任务与否,必宜审慎考虑。

4、每伙食单位,以不带伙食担子为佳,分带干粮尽时,乘机借居,家具做家粮补充,如必须带伙食担子,每单位不得超过二担。

5、文件担子,除必要外,不得多带,普通团二担,营一担,连一担即可,每担重量,不得超过四十公斤。

6、官兵,其私人被着及包袱等,均应自行携带,不必另外长工挑担,但事前须严厉规定之。

七、携带物品

游击队最好有如下各物:

一、破坏铁路、电话、电线,以及军械库等的器具,和炸药。

二、临时之救急药视季候而定,但绷带等必须长期备用。

三、指北针,和游击队动作地域内的地图。

四、尤须携带轻无线电机,以便随时报告敌情,及偷取敌人情报。

五、携相当数量之金钱以备不时之需要及釆办粮食之用。

八、纪律

游击队之军纪良否,影响我们军的信誉,及能否得民众同情与拥护,严格的纪律,方能保障其一切独立行动之彻底胜利。因此,对于违反军纪而危害民众利益,与不坚决执行上级命令的人员,必须毫不客气的严重惩罚之。游击队对纪律之运用,不只是单纯之运用,加强官兵的政治教育,提高官兵的政治觉悟,在无形中减少许多违反纪律之行为,使官兵洞悉民众心理,适时善用与百姓打成一片。

九、政治组织

1、游击队的各个队各分队,皆为设政治指导员一名,部队应设政训处,以便管理官兵,进行政治工作,及司理一切政训人员之人事问题。

2、游击队每一伙食单位均应设司务员一名,以防反动分子之潜入与活动,和劝导对共产主义认识不清,意志动摇之士兵。

3、为防止士兵之逃跑,游击队中应成立反逃跑委员会,及十人团,十人团与反逃跑委员会是防止逃亡的消极办法,兹将其组织及工作,概要略举如下:

(1)为防止逃亡起见,部队应成立反逃跑委员会,每个伙食单位应成立十人团。

(2)反逃跑委员会之人数,为七人至九人,会长一名,余为会员,由刻苦耐劳,思想坚定之下级干部,及十人团团长组成之。十人团之人数共十人,团长一名,余为团员,由忠实可靠之士兵组成之。

(3)十人团在工作时,则服从反逃跑委员会。在军务上,则服从其主管官,及反逃跑委员会。在工作上,则服从政训处。在军务上,则服从队长。十人团,或反逃跑委员会在工作上,亦得受其主管官之指导。

(4)十人团工作之进行,注意一般官兵之行动与言论,特别是对“营混子”之流。

①秘密监视不稳定分子,×团员亲身或其朋友。

②每于极其困难之后,或我军稍挫之余,于宿营时,宜特别注意士兵所生之不良心理,及危及士气的谈话。

(5)反逃跑委员会之工作,主要的是在考查各十人团工作情形,适时监督之,指导之。并可召集各十人团团长会议,或团员大会,讨论一切工作进行。

4、沙漠缝式的军人生活,每天学科术所与之疲劳,而使人发生厌烦及反感。为使军中娱乐,调剂枯燥生活,游击队中应成立俱乐部与娱乐室(其组织细则及工作,详后)

十、特种军事组织

1、为补救弹药之不足,及射击技术不精良起见,每连有三名至九名之特等射手,专为狙击远距离或特种目标(敌军官,机枪射手及炮手,传令兵等)之用。

2、游击队的各支队及分队长,均应选定目力极强之传令兵,作观察员。通令支队长有两名,分队长有一名,专为补救战场观察之不及。

3、游击队的各支队各分队,应有看护兵两名,专为临时医治官兵病症并进行卫生教育。

4、为确实明了敌情,不失时机的得以应付起见,游击队中应普遍的成立侦探队,通常队有一排,支队的一班,分队一组即可。地方侦探队之建立,亦由侦探队到处组织之,或事先预伏侦探人员。

第五章任务

游击队动泎的主要目的,在尽量于敌以精神的打击,和于敌后身引起混乱与骚动,吸引敌人之正面兵力,到两翼或后身来,停止或迟缓敌之作战,致分散彼之兵力,牧各个击破之良果,以神出鬼没的行动,使敌于进退维谷之丧颓状态。

一、破坏敌人行动地域内之铁路及公路,并路上重要之建筑物,电话尤为重要。

二、损坏敌人主要的及次要的兵站.

三、破坏敌人粮食和军械的仓库。

四、袭击敌军后方的辎重队,传递步哨,以及骑兵侦探等,并掠夺敌人从后方运到各方的生活品及弹药。

五、袭击敌人独立支队,及其未曾确实占领的住民地。

六、主要发动民众,组织民众,扶助民众之自卫。

七、破坏敌后方飞机场及空军仓车。

第六章动作

一、行动

1、首在秘密和审慎的准备,神速和突然的袭击,如暴风急雨浓雾密布之时,或敌军疲劳困惫之际,或深夜黑暗之天候,皆为游击队进袭之绝好机会。

2、游击队的动作,应该是一种进攻的战斗,无论游击队兵力之强弱,都可出其不意,乘其无备施行袭击。但如有不利之征候或无胜把握时,宜急速退回,以免受无益的损失。知游击队预定的袭击失利而敌人转为攻势吋,则游击队亦宜急速退却。惟在无法避免敌人之追击时,才对敌人抗战,然后逐步退回。

二、战术的运用

1、游击队之利害,并不全关乎自己兵力之大小,而在于使用自己突然的袭击与埋伏,以声东击西,忽此忽彼的出现,真假立旗帜,虚张声势,散布自己力量的谣言等,以震恐敌人之军心,造成无限的恐怖。此外注意活学活用敌进我退、敌退我进、敌驻我扰、真伪装挺进等法。

2、游击队之发动民众,坚壁清野,诱敌深入,使敌人消息断绝,给养困难,兵力疲劳,地势不利,然后出击,实为致敌于死地良策。

3、游击队之袭击埋伏坚壁清野等法,皆不能奏效时,则釆用攻其旗必救之方法,诱敌于运动中,施行不意之袭击。如一部攻击某处,而以大部兵力,潜伏敌援军必经之路旁,施行截击。

4、游击队应极力避免阵地战,及一切真面目之战斗。地方游击队未受正式军事训练,不可以之对敌进行正当的和持久的战斗。因此地方游击队在最初成立时,须使之随基干游击队,或特种游击队行动,经过相当时期,方可独立行动。

5、或攻敌最感痛苦之处,以吸收其大部兵力来援,然后我分主力至他处,消灭所余的孤弱守军,或截击未至的部队于中途。

三、地区的利害

1、游击队在开阔地上动作时,因少良好之遮蔽容身,与我稍有不利。因游击队之活动,特以隐蔽地,山地及复杂的地形内有利。

2、游击队须详知其活动区域内的地形,经常的考虑如何能从敌军料想不到的地方,取秘密而隐蔽的路径、溪谷、森林、羊肠小道等,以接近敌军乘其自信安全,毫不提防之际,用迅雷不及掩耳之手段予以突然之袭击,并且忽而隐蔽的遁去无踪,至敌退却攻战守行止坐卧均感不安的程度。

3、较大之村庄、市镇,与粮食财物较多的地方,皆为敌人追扰之目标,游击队应经常侦察敌踪,预先埋伏于半途施以截击。

4、游击队在其所担负的区域内,用尽一切方法,使敌之小部队,无法窜入,敌之主力无法隐蔽,必要时,亦可用暴露游击队区域以外之敌人兵力与配置,或其企图等。

四、季候

游击队应考虑季候(适于动作的冬天夏天或秋天),考虑敌我力量,特别应考虑战斗兵器,并应详知敌军部队的编制,某季候与我有利否,亦视地域而定。

五、秘密的行动

游击队动作之物质,均在乘敌不意,因此必须采用一切方法,保守军事秘密,其详如下:

1、队长对部属说明其任务与动作计划时,应在刚出发前,或行进中,必要时得将整个企图,分期逐渐说明之,其他人员不在必要时,不使知之。

2、游击队之命令下达法,最好是由队长与部属当面说明,一切笔记命令,务宜少用,以防泄漏秘密。

3、对向导与居民,不应与之讨论自己的一切行动和计划,至对我同情的居民也是如此,特别是开往某处袭击时,更要禁止讨论。

4、预先派出忠实可靠之侦探,前往宿营地点,或埋伏于敌后要道,以封锁彼之消息。

5、行军时,暗号与路标,由后卫彻底负责除去之,且釆用曲线式的行进,使敌莫测我之行进方向。

6、部队番号皆用规定之代字,严禁直呼部队之本名。

7、一切文件,除必要外,阅后立即焚毁。

8、为掩饰游击队之真实企图起见,除采用上述方法外,有时亦可利用当地居民散播关于游击队行动的假消息,以欺骗敌人。

六、运动的部署及准备

为使运动迅速起见,除将一切组织,力求简单外,不论何时,并应有完善之行动准备(路线之调查及侦探、病号之安置、向导之预备,最好用当地同情游击队的农民,或可靠的人)以应带齐三日的干粮为最佳。如是则欲行则行,欲止则止,不须临时部署。

七、胜利条件

1、游击队胜利的条件,是在游击队之官兵,有绝对的勇敢与坚定性,并须具有协同的精神与充分的机构,和忠实于自己的任务。此外还须有健康的身体,能够耐无限之劳苦,善于使用自己的武器等。

2、游击队不应于困难时期,沮丧志气,纵遇危险环境亦不停止其活动。对于战争之必胜心,与事业成功之信念,尤其对于民族敌人的仇恨,更是以唤起勇往迈进之志向。


八、联合行动

如小的游击队,因自己人少,不能进行其所受任务时,可以同时联合几个游击队,来达成此种任务。夜间是游击队活动的最好机会。

第七章袭击

一、袭击前的任务审慎考虑

当游击队集合完毕时,侦探通信联络等都已设施圆满,预备开始袭击某个住居地时,游击队长首先应查明下列各点:

1、该住居地卫戍兵力之大小,配置如何?武装如何?战斗力如何?其派出警戒又如如何?

2、附近有无其他敌人?如有,距离若干?能否迅速援助此卫戍队?设想将如何来援?由何方来?

3、游击队方面和敌人方面,有何种道路可通?所拟袭击地方,有何种隐蔽接近路径?怎么走近袭击的地方?以上三点,不独为袭击所要知道,即事后退回,也不可不知。

4、袭击时间之决定,最好能在夜间举行,因为有黑暗的掩护,即使袭击失败也可以使敌人恐慌。但我们必须熟悉地形,明了敌人的配备,或者有很好的向导时才行,否则这种袭击,宁可选在拂晓以前。设若对兵站袭击,则其时间最好在深夜,因为兵站上的人马和物品,他们重新上路,是从很早的清晨开始。

5、当地居民能否助战?如何使居民不能因此起而自扰,在谨慎考虑袭击的计划中,但要避免各种繁琐的计划。

二、出发前的注意

1、游击队在出发前,应完成行军准备(同第三节之六)。并应携带担架,以为运送伤兵之用。

2、袭击敌人的方法,在事前不只是队长及各支队长,和各独立的分队的一切人员,都要根本了解。这种传达方式,应由指挥者与部下当面说明为好,一切笔记命令,务须减少,以免发生遗误。

3、各级军官在出发前,均应指定代理人,一面为表示自身之坚决牺牲,并为预防其伤亡时,部队动作,不致因彼而影响全局。

三、运动中的注意

1、游击队之运动须力求隐蔽,勿使敌人发觉,因此行进时应离开大路及大村落,而选择偏僻地方。甚至在完全没有道路的地方,而沿羊肠小径行进。但要避去泥泞湿路,以免过于疲困。

2、行进时不应老在一条路上行进,因为敌人容易就此寻找出其踪迹。又为秘密其行动起见,通常以夜间运动为宜,甚至长途行进,也应在夜间.

3、行进时,为遮蔽自己起见,派出的警戒,应减少到最小限度,通常只派出几个路上侦探就够了,并须有很好的向导。

4。设若不能自信绝无敌人间谍来窥时,最好是用小的分队向不同的方向分进。而在秘密的指定地点集合。

5、游击队运动时,经常应有与敌遭遇之准备,因此游击队之军官,通常率其次级军官,在道路上侦探后或老兵后或部队前方行进(部队则交副者率领之),情况易明了,部署极敏捷,见可进则进,知难后退,只须二三长官,一度会商,便可决定。既免命令往返,迟又误时机,更少后方遥制,不合情况之弊。

6、游击队之士兵,除路上侦探外,皆不应装子弹,以防途中走火为敌察觉。

四、途中遇敌时的处置

1、游击队在未到达目的地以前,绝不应引起无目的战斗,虽于途中遭遇敌军,亦须设法绕过之。甚至与预先指定的有所偏差,亦必如此做去。但在无法避免战斗时,应从埋伏中,与迅速准备中,以出其不意的奇袭以歼灭之。惟是切行之际,须注意敌人停止或前进情况,须从各方面进行侦察。若敌军不曾准备战斗,或有力而乏机警时,即应立即对之冲锋,否则仍然继续隐蔽,静待时机。

2、路上遇敌之哨所和侦探等,要不被彼等瞧见,静肃的迂回过去,如遇审查有机可乘时,则用迅速手段,不开枪而捕获之。

五、袭击时兵力部署

游击队当袭击时,其兵力之部署.大约如下:

1、以大兵力,向敌之要害处猛烈攻击,施于迅速坚决之突击。另外一部绕胁其侧后方,积极活动,以混乱其判断,使其莫测我主力之所在。

2、全力向敌之一点袭击,但他方面仍假立旗帜,以少数散兵,虚张声势,以混乱敌之耳目,分散敌之兵力。

3、如事前能判定敌人之退却路线,则须在可能范围内,派出一部分兵力去堵截。如敌有大炮及辎重配置于村落外时,则指定特别支队夺取之。

4、如游击队兵力大时,应分成几个纵队,从二三方或多方施行袭击,以求截断敌之退路,但须顾虑周到,以免引起自己队伍的混乱,将自己错认为敌。因此在事前必须预定信号或记号。

5、在袭击敌军之际,如有某方敌军增援之处时,事前应配置一小部队于其增援路上以阻止其增援,或将此种危害,告于本队。

6、袭击时对敌攻击重点之选择,及兵力之区分(通常主攻方面,用全兵力三分之二,助攻方面,只用三分之一)。务须使敌之兵力不及开展与增加,而为我各个击破。

7、游击队的各支队,应在袭击的最近距离内,分遣兵力,联络前进,最好在冲锋地点的距离上。因此可以避免迷失道路,及过早分散兵力之弊,也可以保障被袭击的危险。如各个独立纵队,或小游击队之间相隔愈大,地形愈生疏,则希望于同时突击愈困难。

六、突然强击的成功

大部在于趁敌不备,而于其惊惶失措中,猝然冲锋。为使袭击能出敌不意而得到成功,则一,须迅速秘密,不可过早暴露自己企图。第二,要敌之警戒不甚机警时。第三,要虚张声势,分数路同时袭击,使敌举措纷乱,自相惊扰,不能尽全力与我顽抗时。第四,在实行袭击时需依照预定的钟点,不喧哗,不射击,不喊“杀”声,要使每一士兵皆须了解袭击时武器的使用,就是刺刀和手榴弹,不可闻枪声而还射。只有在敌人有机可乘时才能前呼后应,选用正而的、侧面的、迂回的、直接的突击。

七、袭击成功后的处置

1、一到袭击任务完成之后,游击队即应迅速遁去,遁去之先,最好是先向某个虚假的方向过几里,然后再转到自己真实的方向,以使敌人无从知我踪迹,无法追踪。

2、游击队不宜押带俘虏,及大批的胜利品,以阻碍进行,最好是将俘虏缴械,然后遣散之,或格杀之,胜利品则由地方政府或居民输送之。

3、战斗间,每连以官兵三名,专负收集遗弃枪弹之责,战斗胜利后,须尽全力收拾战场,且可号召附近民众收集之,不使有×锱铢之遗。

八、袭击失利后的处置

若袭击失利时,则迅速退至预先指定的集合场,通常集合场,是在前夜的宿营的地方,如兵力充足时,则于指定的退却路上,留置一部预备队以备收容俘虏及伤兵。

第八章侦察

一、侦察时的注意

1、一切情报之搜集,必须不失时机的报告上官,或通报友军。

2、搜集情报时务须详尽,一切马虎敷衍之报告,应严厉的制止。

3、侦察之范围,不仅限于敌情,且应注意地形,一切有害于我而特别有利于敌的地形、隘路、渡河点,以及迂回此渡河点与隘路之可能等,都应该知道。

4、与游击队有关的一切事物,应力求详细周密,直至彻底明了而后已。

5、注意人民对敌我双方情绪和关系,是否积极援助我方?其积极性义如何表现?

二、侦察的手段

游击队派出勇敢而聪明的人员及侦探,向各方进行侦察外,尤其与当地民众,亲密的达成一片,并于紧要地方,利用可靠居民,或同情于游击队的人民(如利用封建关系,找亲戚或找被敌杀害之家属,或仇敌之人民均可)给以较优之工资、秘密的布置侦探网,及递步哨,使我消息炅通。

三、对敌人兵力及其技术兵器的侦察

1、在何处发现敌人的步兵,与骑兵、炮兵、或其他部队若干?敌人的装甲汽车、装甲火车、坦克、飞机等数目若干?并在何处?

2、敌人在其正面与后方地域或城市,以及其他地方,如何筑壕?并以何种部队布防?

3、敌之兵站及其仓库在何处?

4、敌军情绪如何?是否愿意战争?其对人民及其长官关系如何?

5、敌军武器、被服、粮秣,以及其他预备品的供给如何?

四、对地形的侦察

1、首应注意此地地域内之重要道路,及其方向、宽度、土质状况等,是否适合于诸兵种之通行。

2、有无森林?如有,须注意森林属于何种?及所占面积。有无沼泽?在何处;占面积若干,能否通行?如能通行,更须注意其能通行何兵种。

第九章埋伏

一、埋伏的种类

从隐蔽配置中,突然向正在进行之敌人实行袭击,谓之埋伏。游击队唯一惯技,就是埋伏,埋伏极易得到良好的效果,而且通常皆属顺利,其动作分:

1、诱伏:即我军所谓伸开两手,诱敌深入也。

其法先以主力,埋伏于道路两旁,或一侧之荫蔽处,另以小部兵力,向敌攻击。佯作败退之势,引敌深入,然后主力由一翼或两翼突出袭击之。

2、待伏:与诱伏略同,但不须以部队佯败,致敌深入,乃于高处设置观察所,以观察敌军之行动,与其主力达到地点,然后突出袭击之。

二、埋伏的地点与对象

对敌之单独兵、通信兵,整个运动部队、辎重队、运输队,火车等,皆可设置埋伏,其详如下:

1、对敌之骑兵与步兵,则应埋伏于其不能使用武器,及不易发挥威力的地方。

2、对敌之辎重队和运输队,则应埋伏于森林中或村缘。

3、埋伏路旁袭击敌之小部队,或整个运动部队,汽车运输纵队乃最有价值。但首先须明了其企图,及其前进方向,与通过时间,并详虑及侦察易于奏效的埋伏地点,同时对自己的退却道路,亦应预先选好。

4、对于铁路列车,实行埋伏时,游击队可分成三部。第一部于铁路附近,占领阵地,以防列车抵抗。第二部占领列车两旁,向车厢发射。第三部是担任冲到车上收查,卸货缴械等。

三、埋伏地形

1、埋伏的地方应有良好的隐蔽,避免敌之觉察,并且便于观察敌方。

2、能发扬射击之最大效力。

3、须能从埋伏地方,迅速地一跃扑进敌前,因此,埋伏地方与敌军之间,须设有丛林之阻,沼泽之陷,以及其他隘路断绝地等。

四、埋伏的距离

突施袭击的游击队,如因兵力充足,且欲扑近敌身时,则应埋伏于道路近旁,反之,如敌之兵力甚大,仅图向彼扰乱时,则停留于道路之远处。

五、埋伏的要诀

1、伏兵最宜静肃,无论日间夜间,绝对禁止高声谈话,或梭巡线上.

2、久处于埋伏中,足以暴露企图,增大危险,因过久人员紧张状态,渐趋于减弱,慎重态度,不宜于保持,故而被敌察觉。最宜注意者,如已被敌发觉时,则立即动作,或迅速遁去。

第十章对敌征发队实行袭击

一、袭击时机

1、对敌人征发队冲锋,得在下列时机实行之。

2、在进村庄实行之。

3、待其进入村庄,分向各户征发时行之。

4、待其将完成征发事务,满载而归时从埋伏中袭击之。上列诸种,究以何者为宜,须视当时之情况由游击队负责者,妥为估量,适时处置之。

二、村落中的袭击

在村落中袭击敌人的征发队,最为有利。因此时敌之征发队之大部分,分散各处,不易猝然集合,但此种袭击须偷过敌之警戒哨所,或者毫无声息的捕获其哨兵,然后行之。

三、村落外的袭击

袭击者力量薄弱时,必须等待征发终结,并候征发纵列进到便于袭击的地方,如通过森林、桥梁、隘路等处,然后实行袭击。

四、袭击的目的

游击队击溃敌人掩护征发的部队时,只能算执行了自己任务的一部,必须尽毁或俘虏其车辆。因此,游击队首先要将敌人掩护队卷入战争,而以主力袭击其辎重并夺获之。

五、袭击时应注意的事项

在袭击敌人征发队时,易于得到地方居民之援助,故在可能范围内,须分一部财物于民众,以提高其勇气.

第十一章袭击敌人运输队

一、袭击敌人运输队

可获得自己需要的武器与粮食,以及供给品等,这是游击队中最有利的一件事。

二、突然袭击

可使敌人惊慌失措,顿成纷乱之象。因运输之夫役,多系压逼而来的胆小农民,加以掩护队之兵力有限,并常是延伸于很长的距离上,如翻倒车轮之一,可使后方的输送车锄全部停止。

三、袭击法

1、游击队须不要忘记自己的任务,并不是要战胜敌人,而是要夺获敌人的车辆,所以为要同敌人掩护队战斗起见,只须拨出一部分人员即可,而余的人员,则使之掠夺,驱逐和毁坏敌之运输品,故每次袭击,都要力图迅速径向运输队开火,使其停止进行,以扩大其扰敌及恐慌。


2、为要停止整个的运输纵队,只须射击其先头部分已足,因在群情惶惑中,前面停止了车辆,彼此互相倾轧,颠复道旁,遂演成极度混乱的现象。如大批辎重,因先头部队遭到射击,而后方车辆,为企图折转,躲避时,则游击队必须拨出少数射手,猛烈狙击其纵列后尾,使其不敢回转。

3、如游击队劣势,而敌掩护队又极机警时,则游击队以继续不断地警报,疲劳敌人,当运输队通过森林谷雪地及其某种隘路,敌之辎重纵列不易由道路上回转过来吋,即迅速的实行袭击。

在村庄中,袭击辎重队不能时常有利,因掩护队和辎重队易于利用家屋,及其他掩蔽物,实行顽强抵抗。

4、如掩护队已被击溃,辎重队的抵抗也被压服,而增援队未能赶到时,则游击队便着手毁灭车辆及载物,或将自己不能携带和不能利用的物件尽量毁坏之。

第十二章游击队的通信联络及后方交通的破坏

一、联络的对象

为求互相援助,和随时得到敌情起见,游击队对于地方之居民,应尽量的保持最密切而最确实的通信联络。

二、联络工具

维系此种通信联络,除在可能范围内,得利用电话外,应使用一切现有的工具,如徒步兵,乘马兵,脚踏车兵,事前部署的秘密的通信兵,及递步哨,以及信号与约定记号等。

三、报告送达法

1、重要的通信联络,及有时间性的重要报告,最好用乘马来联系。如不能时,则须派出忠实的长于路行的人员,或预先部署的秘密通信送达之亦可。有时并得派出数人,以保障此报告能确实送达(此法必限于最重要之报告)。

2。如不甚重要的普通报告,通常是用徒步兵与脚踏车兵转达之。有时还可雇用熟悉道路而忠实的当地居民递送之。

四、联络信号

为便于指挥各游击队或分队,在日间或夜间,在山地或森林中动作起见,游击队长应预先规定若干基本信号,和记号(如夜间火光,日间浓烟,色巾、旗帜,旗语、口笛,等)。


五、对后方交通路的处置

敌后之交通路,当破坏与否?宜详加考虑,如认为自己军队,将来不须利用或不能利用时,可破坏之。

六、破坏交通路的注意

如欲破坏交通路,须熟知地形,才能神出鬼没的敏捷出现,迅速遁去,不发枪声解除敌人警戒,更不使敌人惊逸。

七、开始破坏交通路的准备

当开始破坏交通路时,须先派出支队到发现敌人的地方,监视敌人巡查,及其小支队,使其不能迅速的秘密的接近本队工作的地方。如开始工作后被敌发现,则开始射击,以阻敌接近。

八.各种地物的破坏法

1、破坏铁路,当在最不易修复的地方,如铁路转弯处,或在铁路掩蔽处,警戒松懈处,能遮蔽工作处,以及能作成大的破坏处。在破坏铁路轨时,须拆却之,或挖空其下面低洼地方,则作成沟渠,隧道则填塞之。

2、枕木、木桥、电信杆、电话杆则用火烧之,铁丝则拆走或抛入水中。

3、在车站上的信号转辙器、十字杆、铁路车辆等,则损坏之,最好以炸药炸毁之。

4、破坏石敷路和公路与桥梁及其他建筑物等,均依该建筑物之性质而破坏之。

第十三章经常隐藏地与住址时的警戒

一、队伍的整顿

不仅是休息与整顿队伍,且用以保存弹药粮食,和受伤人员与患病人的安身休养地方。故此地,通常在战斗上也是支撑点,一遇敌人追击时,便向此退却,秘密的隐蔽起来,以便待机而动或重新开始对抗敌人。

二、地点的选择

1、能持久休息的隐蔽地,便是树林深处,沼泽附近的茅舍,土屋,山中的岩洞,独立农庄,隐僻的小村落。因同情故,小游击队寻得经常隐藏地并不困难。

2、游击队对于选定的几个隐藏地方,务必严守秘密,就是最密切的人如与本队没有关系,也不可使之知道。如原有隐藏地面被敌人发现,则通常不待敌人来袭击,便须迅速迁移。

3、有时此种隐藏地,又作为军械库,火药库和粮食库,且以之安置伤病员者,但各种仓库,通常是选定独立的秘密地方于隐藏地附近,因隐藏地,一经人们时常出入其间,极易被敌发现。

4、人民中拥护游击队的人愈多,而游击队又能与人民维持通信联络,则游击队愈易寻得隐藏地,有时为避免敌人追击,求得良好的隐藏起见,不得已在相当时间内分散本队,使各人设法隐藏于各居民家,在如此场合,居民也是游击队员唯一的救星。

三,粮食问题

1,如在居民仇视游击队的地方,只有强行征发,但须派队中可靠人员以免发生掠夺。

2、游击队在不怕暴露本身时,可派出特别小队,征发粮食或收集捐助之食品,或要地方供给。

四、变换驻地

游击队保障自己安全的最好方法,必须动作灵活。如必要时,可每夜变换自己驻地(如白天在甲村,黄昏移至乙村)。

五、住地的占据

游击队宿营时,其兵力分配,全以动作的性质来决定,但不应该占据自己力量不能扼守之大村落。若不得已而位置于此种地方时,只须占领几个独立配置而便于防御的集团家屋。最好是在适当的村落中,能了望各方,特别是敌之来路,但切不可将队员散住于各屋。以贪个人之便,致为敌所乘。为不使敌人知我驻宿地点,最好是深夜才进入村中,并监视村之四周,无论何人均不许外出。

六、警戒的程度

为使队员不致过劳,而得到真实的休息起见,无须派出大的警戒,只要对于各接近地和道路(敌人必经或与我有关的道路)设置军团哨与潜伏哨,对于各主要方面,二里至四里处,派出侦察即可。

七、准备

游击队在驻止时,无论官兵,随时均须作战斗准备,特别是黄昏后,每官兵,翻赃收拾身旁的武器和装具,使有秩序,以便于黑夜中如有警报,能迅速出动应战。

八、有敌袭击的顾虑时

]、如游击队本身非常机警,侦察闭之组织非常严密,地方又同情我方,常将一切行动,报告前来时,则敌人之突然袭击,是非常困难的事,但在任何情况下,还须慎重。

2、为防止敌人,借仇视的居民,来袭击起见,必须采取特别预防的特别方法。例如,以恐吓手段,警戒居民,和拘捕押人,但部队尤须××与准备。

3、设有警报时,可集合全体于有抵抗准备的屋内,此屋内,应派适当之枪前哨和了望兵,屋之出入口,应以活动的障碍物封闭起来,并规定防御信号。武装和装具,应准备妥当,靠着身边。

4、当情况十分紧急之际,支队之一部,须担任阵地和住所的警,对于邻近敌方敌人情形随时报告游击队。

5、用人工的障碍物,来堵塞出路,对于第一线与预备队的交通,以及与地方必要的交通与通信联络,都应设立。

6、村落内的街道,如有必要时,可完全封闭,或留有通口,各支队兵力在容许时,须有预备阵地。

九、敌袭时的处置

1、发现敌人向我运动时,如侦知敌兵力不多,则一举而击灭之,如敌优势于我数倍时,则迅速退却。但当退却时,须以虚假方向,显示敌军,而掩饰自己真实的退路。

2、如敌袭而我不及避免时,则充分利用村落防御战的优点,与之顽抗,然后乘机退却。

3、已失了的村落,应以反攻或逆袭的手段,迅速夺回,以便救出被俘虏的同志,以及据守一点依然誓死顽抗的同志,如动作迅速,均可达到此种目的,因敌人在胜利之后,常处于极端混乱,且疏于警戒的状态中。

4、实行这种逆袭或反攻的最好时机,是紧随着在敌人袭击胜利之后,冲锋所受的牺牲,比较逃跑,或比敌人袭击时停在恶劣阵地上所受的牺牲少。

第十四章训练

一、训练范围

不仅限于军事技术,而政治及识字运动,及卫生训练等尤须注意,因此游击队之操作时间,必须平均分配识字训练,可使之随时随地练习。

为使游击队之一切训练收得圆满效果起见,必须提高官兵自动的学习精神.故除政治方面,提高其政治觉悟外,还应增进军中娱乐,调和枯燥生活,扶助人民自卫,使民众武力,与我打成一片。

二、各科的训练

各种的训练步骤,虽难以等齐,惟普遍方式,均由浅入深,先松后严,由简易而入繁难。先局部而后普及,每种必须以实验证明其理论,以坚定其信念。

三、人民自卫力的培植

1、游击队最迫切和最主要的任务,就是在被敌人占据的地方内,不断的往来游击,捕杀一切汉奸和反动分子,保卫民众。同时调查敌方具体罪恶,用尽各种手段揭破其欺骗与阴谋,宣扬我之善战。努力联合民众。扶助民众力量。这种动作即在敌人领土内,亦可施行。更当用尽方法和力量,吸引人民来照本身的行动,鼓动其积极的对敌战斗,并指导其战斗动作。

2、扶助人民自卫力量,应有持久性的,而非动于一时,务使人民知道游击队不论何时,都是为民众而奋战和牺牲的,就在危急紧要关头,也决不危害民众拋弃民众,居民被我引发起来的战争,如遭第一次失败,其斗争情绪必有相应消沉,则我应于其颓唐之际设法鼓舞之。使彼等斗争情绪重新提高起来。

3、游击队系人民中最觉悟而先进的分子,故先联合不满意敌人的民众,受所派人的指导,更须帮助人民拟定计划,取得武器,并与受压迫毗邻的村落,及其他城市中的群众团体,取得联系和助力,但在进行此种工作时,都要保守极端的秘密。

四、卫生的训练

1、为巩固其本身战斗力起见,每一伙食单位应设一名或二名的看护兵,以便适时医治官兵之病症,并讲解卫生常识,以及协助主官督促队中一切卫生事宜。

2、医药补充,游击队中最感困难,故须按季候筹备些救急与必需的药品,以防于万一,对受伤或重伤的队员,不得已时只有托付稍能懂得医术的同伴,或同情于我的居民。

五、军事教育

1、课目:军事教育以敌军为对象,造成超过敌人各项特殊技能。其特殊注意事项,即解散、集合、射击、行军、爬山、工事构筑、夜间战斗,山地战、隘路战、侦探警戒、搜索、联络等各动作。

2、方法:军事教育之进行,宜特别注意教育与进解诸方式,并略举如下。

(1)理论之讲解,可釆用启发式及问答式,一切讲演式与注入式教育尽量减少之.

(2)讲解动作时,应注意现地讲话,使士兵易于明了。

(3)多作实际动作少讲空洞理论,因之讲堂时间及次数,应力求减少,实地演习之次数应加多。

(4)讲堂上之一切讲解,以能与野外演习的对照为佳。

(5)一切动作,实施前须精密准备,一切敷衍塞责的马虎现象,务必扫除之。

(6)一切工作,设法官兵鼓动进行比赛,以提高其工作自动性与积极性,致促进工作速率。

(7)增加应有教练,减少制式教练,并纠正制式教练足以维持军纪之错误观念。

(8)教练计划,须适合环境地域及时间,切忌千篇一律的教育计划,抓紧一切机会,厉行机会教育,兹略举如下:

①利用行军时,进行方位判定,地形识别,距离测量,侦探动作,目标指示法与地形物之利用。

②宿营时则利用警戒配备,演习前哨以下之各种警戒动作,与工事构筑等。

⑧利用作战机会,在出发前,或战斗开始前,根据受领之任务,讲解埋伏袭击或主攻助攻等动作。

④利用待机动作之时机,就实地讲解御敌冲锋,与射击等各种战斗动作。

⑤利用战后讲评(这种讲评事前须有详细之调查),指出战斗经过中一般动作之缺点,与个别指挥之当否,给全体官兵以实际的教训。

⑧利用早晚点名时间,进行各种讲话。

⑦利用游戏时间,进行含有军事意义的游戏及歌舞新戏等,使官兵无形中增高其悔改及效法心。

⑧利用每次赏罚,在官兵中深切的宣传,以提高士兵之向上心及羞恶心,逐渐养成良好之军风纪。

六、政治教育

为使游击队一切独立行动,取得彻底胜利起见,除军事方面,加紧训练外,最主要须使官兵有极高的“政治觉悟”和“忠实”于自己事业。政治教育,就是达到这种目的的唯一办法,其内容详后。

七、识字运动

为提高官兵之文化程度,易于接受一切训练起见,每一伙食单位就必须进行识字教育,其法如下:

1、甲班,凡识字在五十字左右者编入之。

2、乙班,凡识字在二十字左右者编入之。

3、丙班,凡一字不识者编入之。

4、各班识字之指导者,即以队中文化程度较高者,充任之。

5、驻止时,每日须有一小时之识字时间,行军时,则利用进行中或休息时间进行教育,此项工作,只求经常有恒,不求太速,普通每日能识两字者即算上等。

第十五章政治工作

一、游击队政治工作的目的

是在坚固并提高队员的战斗力,因游击队的战斗力,不只是靠军事技术来决定,最主要是政治觉悟,政治影响,发动广大民众,瓦解敌军,使广大民众受我指挥,游击队中无论政治、军事,以及其他一切设施,都是向着这唯一目的迈进。

二、政治工作的主要内容

实施复兴民族的政治教育启发其民族意识,与爱国爱民爱群之观念,使游击队中,每一官兵皆能明了其所负之民族责任,为国家奋斗的必要。

并应注意拥护领袖,精诚团结,彻底执行上级命令,维持铁的军纪,使之万众一心,共赴国难的决心与志气。

游击队除了巩固本身战斗力外,更须向群众宣传侵略者及敌人的阴谋。

三、小组讨论

此事系团结意志,坚定信念,宣扬主义的良好工作。

1、集中各同志意见,以免隔阂,以收集思广益之效。

2、借以训练干部增加工作能力,更可熟悉开会制式及发言法,借以迅速的介及问题,更可检查过去,改造将来。

3、考察联络,吸收同志。

4、便于训练,充分说明了各同志的环境能力和知识。

5、依性能可分为,讨论会,检阅会,批评会。

四、方法

会前准备,即通知组员,确定讨论中心,同时报告上级。

1、人数,以三五人为最佳。

2、不拘仪式,随时随地均可讨论。

3、时限,不宜过长,以一小时为最大限。

4、以每周一次为适宜,其程序,主席报告,出席人报告讨论,结论由组长担任之,并记录主席某人,讨论题目、出席人数与地点。

5、问题,不超过两个以上之问题,先从本身问题讨论起。

6、发言方法,组员方面,发言宜扼要简单明了,做有系统不重复并以诚意和蔼的态度,道答之,切忌含有讥讽意味,并应注意他人发言,同时作根据题目的判断。

7、组长方面,报告简单,不再重复介绍,抓住时机,引起组员发言。

8、结论以归纳法,批评整个言论,有不同意结论可再发言。

五、政治工作的进行

不只靠几个政训员,最好能吸收培养觉悟分子,或对此有兴趣的官兵,参加工作,训练全体队员使均能从事政治工作。

六、政治工作分类

大概可分为平时,战时,与战后三种,至于宣传鼓动方面如下:

1、平时政治工作:

加紧政治教育,提高政治觉悟,统一思想言论及行动,维持铁的军纪,与民众打成一片,其法略举如下:

(1)确实做到不扰民,不害民(如买卖公平,讲话和平,借物归还,损坏赔偿)。

(2)随时随地帮助民众,解除民众困难(助民众收割、耕种,或遣我军医为民众防疫治病,或慰问被难民众,设法救济之),确保军民联合,甘苦与共之精神。

(3)常与民众谈话,借以得知我军之纪律,与民众之感情,更详知民间饥苦。

(4)常开军民联欢大会,借以调和军民间一切隔阂,增进军民感情。

3、调和上下级官兵间一切隔阂,其法略举如下:

(1)政训人员,除与士兵同甘苦外,并常同士兵谈话,详察其切肤之痛苦,随时报告上官,设法改善之。

(2)对上下级之一切意见,须站在纯理智的立场,以诚恳的态度说服之,解释之,务使上下融洽确如一人,益增其团结力。

(3)对违犯纪律之士兵,应釆取教育方法说服之,一切体罚与责骂,务须减少。

(4)常开官兵同乐大会,以提高官兵之感情。

3、提高官兵之敌忾同仇心,增进誓死杀敌之决心。敌忾同仇心之增进,是巩固士气唯一要素,所以游击队特应注意敌人一切残暴行为,及屠杀我军民的事实,在军中做普遍的宣传,以提高官兵之誓死之勇气,坚定其有敌无我,有我无敌之杀敌决心。

4、巩固对敌战争必胜信心,略举如下:

(1)时常的拿着本身过去之光荣战史,对其官兵进行宣传与鼓励。

(2)列举敌军之缺点(及其×间所遭遇之困难或崩溃等情形,以证敌终必失败之事实)。

(3)列举我方之优点(如军民拥护,消息灵通,地形熟悉等)及现在的胜利情形,以证明我之必然胜利。

(4)就敌方惯用之伎俩,晓示我军今后应注意之处,以防畏敌或轻敌之心理之产生。

(5)受打击后,一时陷于困难于疲劳,而妄自菲薄,张大敌人力量,失了胜利心。

七、战时政治工作

出发前之政治工作:

1、先由最高统帅者,召集干部会议,说明“目前政治形势,如何有利于我们,及胜利条件,与作战意义,及为达到此目的而采取的方法和注意之处,但不涉及军事秘密”。

2、政讯处要根据干部会议,立即召集各级政训人员开会,说明宣传之中心与方法,然后具体的分配工作。

3、支队和各分队立即召集官兵大会,除报告“目前政治形势,和我们必胜的把握”外并提出竞赛条约。如“轻伤不下火线,重伤不喊痛”,又如“看谁枪械缴得多”,“或捉得俘虏多”?同时分配各个政训员,分×的工作(如监督,领导去宣传)。

4、对附近居民,则派员召集开会谈话,鼓励他们参战,或加入担架队,运输队,对民众团体,则指导开会,作战与动员配备之方法。

5、战斗开始后,担任主攻的或特别重要的作战部队,则指派重要的政训人员,鼓舞之。次要的作战部队,则派次要的政训人员前往鼓动之。

6、预先派出宣传组及歌舞团(皆为活泼可爱的儿童,着一律的好看服装),在军队行进路旁,作短促的讲话,或歌唱,或跳舞,或喊口号,尽量的鼓舞官兵之作战勇气。

八、战斗开始后的政治工作

1、战斗开始后,应注重向对方士兵讲话,喊口号。以涣散对方士气,即破坏工作之一种也。

2、当战场入于休战状态中或阵地对峙中,则设法与对方士兵开联欢会,借以送食品给对方士兵,以博其欢心,然后再进行宣传工作,事前应有准备。

3、战斗开始后,对外固须宣传对内尤须鼓励。其法不一,以适合当时环境为主,例如:

①攻击前进时遇敌袭击时,则进行下列解释。“同志们,飞机是不能解决战斗的,我们应抓紧这个机会,赶快前进,迅速接近地上敌人,同敌人拚刺刀去”。

⑨火战开始时,则进行以下鼓励。

“同志们,不要乱射击,不瞄准不发射,我们要做到一颗子弹打死一个敌人。”

⑧接近敌方将突击时,其鼓励法。

“同志们,解决敌人时机到了,我们要不顾一切牺牲,鼓起勇气,战胜敌人而歼灭之,拿我们的胜利鼓动全军,赶快冲上去吧!杀!”

④第一次冲锋顿挫时,作第二次突击时,仍须进行以下的鼓励。“同志们,我们是无敌的铁军,我们是每战必胜的劲旅,我们定要消灭这个敌人。保持我们的光荣。”

⑤战斗中如官长伤亡时,则进行以下的鼓励。

“同志们,我们的官长某某已经光荣的牺牲了,我们要踏着他们的血迹,完成他们未了的任务,消灭当面的敌人,替他报仇去!”

⑥如敌方面呈动摇时,则进行以下的鼓励。

“同志们,敌人动摇了,赶快冲上去,活捉敌人师长!”

⑦追击时,则进行以下之鼓励。

“同志们,敌人退却了,快快追过去吧!冲破敌人收容队,解决主力,消灭全部,看谁缴枪械多,看谁捉得俘虏多,打胜仗不追击,是最可惜的。”

4、防御时。

①受命后即进行下列宣传:

“同志们,敌人来了,这是消灭敌人的最好机会,巧妙的利用地物,沉着射击,杀伤敌人愈多,我们主攻方面就愈容易奏效。”

⑧敌人冲锋时,则进行下列鼓励。

“同志们,敌人要冲锋了,上好刺刀,准备好手榴弹,拿出我们的威风,保持我们已往的光荣,消灭敌人于我阵地前。”

⑧被敌包围时,则进行下列鼓励。

“同志们,我们是百战百胜的部队,我们是英勇无敌的铁军。要为民族国家血战到底,流尽最后一滴血,至死不致枪,杀身不投降,缴枪是自杀,投降是无上耻辱,突破一点,冲出去!”

④反攻时,或攻势转移时。

“同志们!我们反攻了,缴敌人枪去,活捉敌人官长,看谁是最勇敢的!”

⑤退却时之宣传。

“同志们!秘密我们的行动,迷惑敌人的判断,放开两手,诱敌深入,不落伍,不掉队,不动摇,不恐慌,不怕牺牲,坚决到底执行上级命令,最后胜利是我们的。”

⑧掩护退却时,其鼓励的方法,与上略同。

九、战后政治工作

l、战斗结束后,为防止轻敌或畏敌心理之产生,应注意下列各点:

(1)正确指出胜败的原因,即不可因小胜而矜骄,更不可因小挫失却胜利信心。


(2)确定今后应取之对策,或应注意之处。

2、收集胜利材料及英勇作战的部队或个别官兵姓名,然后制就宣传大纲,歌舞,新旧剧等。

3、印发大批捷报、标语,张贴各处,同时组织流动宣传队,分向规定区域宣传并号召民众开捷报大会。

4、开捷报大会应注意下列各点:

(1)报告胜利意义,与当前的任务,或为达此任务,而采取具体办法。

(2)报告勇敢作战的部队,及个别官兵之姓名,至于伤亡的官兵可釆其最有价值者报告之。

(3)表演新编之剧曲与歌舞。

参加开会的部队,当报告(1)(2)两项或表演戏剧时,应高呼口号以和之,并应提出抚恤阵亡官兵家属的办法。

并将被俘虏的官兵,或胜利品,集合于众目之下,则更能增加我军民之斗争情绪与勇气。

5、各机关在祝捷大会,并应发起慰劳运动,其应注意如下:

(1)物质的慰劳,慰劳品不在多少,而有意义,不在精美,而在适用,如草鞋、面巾、猪羊、鸡鸭等,皆可做慰劳品。

(2)名誉的慰劳,当慰劳品缺乏时,应以名誉的慰劳代替物资,如制锦旗或编成歌曲以纪念之,或传令嘉奖亦可。

(3)如小胜后,则不需做扩大的慰劳,仅由战地附近设法慰劳之即可。

6、宣扬协同动作与独断专行,及坚决执行命令之模范,例如:

(1)于战斗方酣之际,如因协同动作、独断专行坚决执行命令,而取得胜利者,应尽量宣扬赞美之,其不幸而失利者,亦应嘉奖之。

(2)图保存实力观望不前,或执行任务不力,致使友军失利,而受处罚者,亦应在部队中作广大宣扬,教育全体官兵或其他部队,使之知所警惕。

十、俱乐部及娱乐室的工作

在调剂军中枯燥生活,也是防止逃跑的一种办法,兹将其组织工作及概要,略述如下。

1、组织条例:

(1)为增进军中娱乐,调剂枯燥生活,提高工作兴趣,鼓舞学习精神起见,游击队中,每一伙食单位,应组织娱乐室,内分军事组,游击组,体育组。及每一伙食单位之官兵,均应各就其性之所近,任选一组,亦可参加二组或三组。

(2)如娱乐室应设主任一名,由连指导员或有活动能力的排长兼任之。其所属各组各设组长一名,由组员会决定之,六月为一期,亦可连任。

(3)组长会议及各组会议,均应每星期举行一次,股员会,每月一次,由主任组长分别召集之。

(4)娱乐室在工作进行上,应服从上级俱乐部之指导,并受主管官之监督与指导,但在军务上,则绝对服从主管官。

(5)为指导并统一各娱乐室之工作进行,大游击队,应成立娱乐部,内设主任一名,干事一名,专司其事。

(6)俱乐部应隶于政训处,因为娱乐工作,系政训之一种,但在无政训处时,方直属于主管官。

(7)俱乐部之工作进行,在指导与推动各娱乐室之工作,因此,每星期召集各娱乐室主任开会一次,每月召开官兵或军民联欢大会一次。

(8)无论俱乐部娱乐室,其一切工作之进行,皆以不妨碍军事行政,军事教育,军事行动为原则。

2、工作概要:

军事组在诱发官兵自动学习精神,讨论军事问题,或相互的纠正动作,以补训练射击时之缺憾不及,其内容如下:

(1)刺枪组(与以预想之敌国为目标,而演习之。)

(2)掷弹组(以木刻制手榴弹,向靶投掷。)

(3)军事游击组。

(4)架上瞄准。

(5)瞄准检查。

(6)射击姿势。

(7)击发检查。

(8)利用地形物。

(9)野外各种目标之射击法。

3、体育组。此组工作,强健官兵之体格,更可补助军事教育不及,其内容如下:

(1)球术(篮球、足球、排球、网球、垒球等)。

(2)田径赛(跳高、跳远、赛跑、障碍超越)。

(3)拳术及大刀。

4、游艺组。此组工作,在增进军中娱乐,调剂枯燥生活,以提高工作情趣与学习精神之增进。

(1)笑话组,不论何时,均可进行,但应注意下列各点。

①说笑话时,应注意通俗,如笑林广集之类,但不可过于猥亵。

②讲故事时要讲古人之丰功伟业,及嘉言懿行,以收鼓励之效。

⑧报告新闻时,应注意我方胜利情形,及敌方之残暴行为。

(2)戏剧组利用晚间及日间休息时,表演各种有政治意义的,及有滑稽性质的新剧、戏曲、双簧、花鼓,以扬士气。

(3)歌舞组,根据部队所处之环境之任务,编制各种不同歌谱,引起官兵学习唱歌之兴趣,或举行化装跳舞,做种种滑稽形象,引起观者捧腹。

(4)音乐组内分胡琴、口琴、乐琴,及其他,歌可与戏剧舞同奏之。

(5)各种工作之推行方式,应根据时间及环境,活泼运用之,切忌呆板形式。

(6)各组工作之进行,因有赖于严格的检查及督促,更应采取竞赛方式,使全体官兵,自动努力。

(7)各种歌曲及新旧剧等。

(8)各种各系应有专门人才,以利工作之进行。

(9)参加表演之官兵得酌免勤务。

