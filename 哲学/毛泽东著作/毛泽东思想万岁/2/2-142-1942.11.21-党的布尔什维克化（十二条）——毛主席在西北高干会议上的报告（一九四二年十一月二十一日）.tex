\section[党的布尔什维克化(十二条)——毛主席在西北高干会议上的报告(一九四二年十一月二十一日)]{党的布尔什维克化(十二条)——毛主席在西北高干会议上的报告}
\datesubtitle{(一九四二年十一月二十一日)}


同志们:

这好高干会议开得很好,已经解决了一个重大问题。开了十一天会议,这个问题解决得很恰当,很多同志都讲话了,据我听见的,大多数同志都是好的,讲得很好,只有个别同志差一点。批评错误路线,所有的这些批评,绝大多数的同志,我看是很好,只有个别同志差一点。这中间有××同志的结论,××同志的发言,我认为是完全正确的,我完全同意。这次会议在批评朱理治、郭洪涛他们两个人历史路线上的错误,在这次斗争中间教育了我们大家,也使得中央更加明了了情况,懂得了历史。在这次会议中间,暴露了这样的事实:就是创造边区的以及后来参加边区工作的同志,这中间绝大多数都是好的同志,而且进步很快。在各个根据地比较起来,究竟那个根据地的党更强一些?那个根据地的干部比较起来更好一些?当然各个根据地的党,都是执行党中央路线的党,各根据地的干部,绝大多数都是好同志。但是比较起来,边区的党、边区的干部有更好的地方,就是因为这个边区是曾经经过内战。经过土地革命。西北同边区是经过了这三个时期的大革命时期,土地革命时期,抗日时期。而别的根据地的党,拿每一个根据地的全党来说,那么他们的党龄,大多数同志的党龄要短一些;干部中间有一部分是新来的,外面来的,是新提拔的。而我们这个根据地则不同。这个根据地是经过三个时期。大革命时期,虽然那时还没有根据地,可是有工作;因为有工作,所以后来能创造这个根据地,比方那时学校有支部,有的城市如绥德,延安等地都有支部,乡下也许有个别知识分子在工作。就由于他们,所以后来建立了根据地。他们和群众农民结合起来了,有兵营,有国民革命军,这是好的。后来国民革命军有一小部分就变成了红军。这时我们开始有了武装部队,这是地方和军民结合起来了,因此就创造了根据地。拿同志们的表现来说,在这次会议中:表现出进步很快。在这一点上,我们后来许多参加工作的同志,外来的同志,应该向本地向志学习,特别向本地同志学习。我想实实在在是这样的情形。因为本地同志,特别是其中大部分同志是经过了三个时期,他们亲身经历了这样长期的三个时期的斗争。因此他们观察问题、解决问题他们有经验,比较更好一些,比较那些只做部分工作而没有这样的独立创造出一个根据地,没有经过长久的历史,没有这样丰富的经验,自己应该懂得,我们是差一点。这些同志应该向本地同志学习。事实证明他们是比我们强,甚至某些职位高的同志,也要向现在职位低的同志学习。事实上在许多问题上,这些同志比我们强一些,更正规一些,这些同志更原则些,这些同志此我们更马列主义一些,更布尔什维克化一些。所以那些轻视本地干部,有这种观念的,当然口里不说,他口里绝不会说:“我轻视你们,看不起你们”,但是心里有着看起人家所谓“土包子”。人家没有走那样远的路;而自己走了若干路就成了洋包子。比方拿那两个根据地来说,这个是根据地,那个也是根据地,他们在这个根据地没有走一段路,是土包子;洋包子是从那个根据地到这个根据地来工作,所不同的就是多走了一段路和没有走那一段路。像从那个房子走到这个房子,你就洋起来了,你多走了一段路,所差就是这一段路,如果说两个根据地所差在什么地方呢?就是从那个根据地到这个根据地,就是一段路,一个走了一段路,一个没走一段路。这一段路算不算经验?算经验。因为多走一段路,两万五千里或几千里。从上海、天津来的,但是有一个缺点,一个什么缺点呢?就是这个根据地现在还在,你们那个上海天津,现在呢?现在垮光了,全国白区十分之十,全国白区的百分之九十九都光了,全国苏区百分之九十都垮了,只剩下这一个,你那样高明,又是你那里垮了。是不是每个同志都要负责?都不高明?所有万里长征或走几万里来的都要负责呢?那不是这样说,但是从那里走来的同志应该懂得,自己此本地同志所差别的,就是多走了一段路。从那里走到这里。你那里失败的责任谁负责?上海天津垮了谁负责?大多数同志要负责,尤其是领导同志要负这个责任,但我们在那里工作也是土包子,土包子与土包子,这里头,没有理由产生轻视人家,因为你那里领导不好,再加上客观的原因,敌人的压力很大,主观领导的错误,所以白区弄得差不多完了,各个根据地都失败了。所以我想许多新同志以及老干部,到边区工作的同志,应该诚心地向本地干部学习,这样大家绝不会嘴里讲,轻视他们土包子。口里没有那个可能讲,讲出来那不像样子。在心里没有一个,就有十个这样,但有若干是这样想的,这是毫无根据的。在这次会议中清楚地表现这一点。有些外来的同志,甚至职位高一些的,比起本地同志来要差一些,他们不及本地干部。应该使他们懂得,本地干部更原则一些,他们的工作更有经验一些,工作更有办法一些。关于这一点,我想在这个会议中,应讲清楚的。因为这个会里有两部分同志,要很好地结合起来。本地同志很进步,进步很快,掌握政策很好,那么就没有缺点了吗?不要向外来同志学习?他们就不能当我们的先生?那可不然,无论任何同志与同志间大家相互都是先生,就是中央和地方,和边区同志,相互间都是先生。中央不向你们学习,中央从那里做起?中央不拿你们作先生,事情是办不了的。比方这次解决历史问题,没有这十一天的会议,没有你们贡献的材料,暴露了这样的事实,要中央同志完全了解,是不可能的。因为他们没有参加,你们不讲,他怎么能知道呢?所以中央同志要以你们为先生,要以他领导的同志为先生,要向他们学习。同样他们也应该这样向中央同志学习,也要向外来的干部学习,那怕他就就是一个新来的,新加入党不久的,因他总有多少长处。比方一个新的知识分子,一我们可以请他当教员上文化课,这不是先生吗?这些本地同志应了解,自己还有缺点,我们要提高文化,提高理论,××同志讲:“要提高我们的理论,补足我们的缺点”。这样相互学习,相互尊重,不是相互轻视,相互排斥,这样来解决问题就好了,问题就能好好解决。我想这次会议,不但解决历史问题,而且解决将来团结问题。将来团结问题,这里面有一部分本地同志,另外还有一部分外来同志,这两部分同志必须很好地合作。很好合作必须除掉轻视的观念,尤其不应轻视本地的同志,我们应向他们学习。关于这个问题,是这次会议已经过去的一个段落,我讲是这样一点意见。


今天我想讲斯大林同志这十二条,讲十二条与刚才讲的全党路线搞清楚,全党团结起来,在这个基础上讲这十二条的。

斯大林同志说:“要实现布尔什维克化,至少必须具备若干基本条件,没有这些条件,各国共产党的布尔什维克化根本是不可能的。”

现在我们同志互相敬礼时,说布尔什维克的敬礼――致“布礼”的敬礼。就是好像大家都说,你也是布尔什维克,我也是布尔什维克,他写信给我,致布尔什维克的敬礼,我写信给你,也是致布尔什维克的敬礼,但我们是不是空的?这个布尔什维克敬礼是不是空谈呢?我说要看是否空谈,就要按照斯大林同志所说的具备了条件没有。具备了必要的条件就不是空谈;要实现布尔什维克化,如果没有具备这些条件,或者具备不完全,那么我们就不是叫布化,只是向着布化。“化”是不容易的。斯大林讲十二条,布尔什维克有二三个条件,还有些条件不合,那么你还能算布化吗?不能够。是向着布化,但是没有“化”。

现在拿我们全党来说,是怎样的情况呢?拿我们全党来说,我想应当是这样看,我们还是向着布化。要讲布化我党实在有多少条件?具备了多少条件呢?一般地说,我们是布尔什维克党。从它的路线,从它的工作说起来,从这个党的经验、觉悟程度、同群众的关系说起来,是一个布尔什维克党。但是要讲到化,什么叫“化”呢?就是彻底完全得很,完全的布尔什维克。那么在我们如果拿这个标准来说,完全布化,我看我们还差,还是许多条件不具备,或者不完全具备。我们来看这些条件,像斯大林所讲的。

第一条:“必须使党不把自己看成国会选举机构的附属品,像社会民主党在实际上把自己看成的那样;必须使党不把自己看成工会的免费附加品,像某些无政府工团主义分子有时翻来覆去所说的那样,而必须使党把自己看成无产阶级的阶级联合的最高形式,即负有领导无产阶级组织的其他一切形式(从工会到国会党团)的使命的形式。”

这一条对我们党怎么样呢?我们来检查一下。这一条在我们党许多地方、许多部门没有实行,或者没有能够完全实行。这个里头,现在我们是不是国会党团的问题?是不是工会主义,无政府主义?斯大林举的例子是这样的例子,但是我们不是国会党团的问题,不是工团主义、无政府主义问题。因为我们的国会党团――我们重庆参政会有党团,边区参政会有党团,这个党团没有像斯大林所讲的脱离党把党看成附属品,而拿那个东西作为主体一一以国会党团为主体,实际就是中央。在欧洲社会民主党的国会党团就是党中央,那一种现象我们没有。我们许多同志说:这一条是不是可以要?我们许多同志读了这一条说:这一条和我们不适合,我们没有这一条。那么我们就讲到别的一些问题,就是我们看“党是一切阶级的最高组织形式”,这个“最高”是领导一切形式的组织。一般组织形式,在这里所谓职工会、国会党团,这个问题是有别的例子。现在的事实来说,我们有政府、我们有军队、我们有民众团体,党领导什么东西呢?现在拿我们各个根据地来说,党就是领导政府,领导军队,领导民众团体。除了这三个工作以外,我们还领导什么东西呢?我们不领导别的什么东西,因为除了这三个东西以外没有别的东西了。除了这三种以外,我们还有党务工作,整理党的组织。那么整理党务工作的目的在什么地方呢?为了整理党务工作而整理党务工作是不是可以呢?我们说可以的。为什么呢?完全不为别的,就是为着要把军队搞好、政府搞好、民众团体搞好,所以我们要搞党务工作。党务工作里头包括干部工作,为什么要把干部搞好呢?因为这些干部,不是军队干部就是政府干部,很大多数干部,除了少数党务工作干部以外,十分之九是军队的干部,政府工作的,民众团体工作的干部,民众团体工作里头,有工会、农会、青年、妇女、合作社,文化团体,文化团体中间又有各种团体,有文学会、哲学会、戏剧会一一戏剧里面又有新戏旧戏,有木刻等等,这一套都是民众团体,这一套可多得很,你这党不管军队、政府、民众团体的事,你管什么?你这个党的干部,到底在什么地方工作?所以党务工作的目的,完全不是为别的,搞这么许多人,给你小米吃,吃得你发涨(笑声),就要作工作。做什么工作呢?就是党务工作,做这工作的目的是什么?就是要搞军队,搞政府、搞群众团体,除此以外,没有别的。把这些搞好,才好跟敌人作斗争。

但是现在似乎有这样一种观念,实实在在存在着这一观念,你这个党部就管你的党务工作,我的政府你少管点,我的军队你少管点,我的民众团体你少管点。这种情形是想象中的呢?还是事实上就有呢?少管到不管,不管到反对。反对的是什么人呢?是张国焘,他就是反对我们,他不要我们管,他自己搞一个中央,“陈桥兵变,黄袍加身”,闹独立性,自己有军队,反对中央管他,而他要推翻这个中央。

另外还有一个,就是抗战中间还有这样的同志,中央的命令他不执行,就是皖南事变以前有这样的事,其它的军队像这样严重的事是没有的,还没有到这样的程度,像张国焘是最标准的,突出的。第二个是项英,是比较次一点,是没有公开暴露这些事情来。再有的是服从一部分,不服从一部分,这些在军队里有没有?军队里面也有过,合我脾气的就服从,不合我脾气的我不服从,口里没有公开讲不服从,但心里却这样想,做起来是这样做。在边区党政军民关系上,就有这样的现象,党的意志,决议案不能执行。比如党政军民关系上,军民合作不能够执行吗?相当长期不能执行。这个不执行是不是原则问题,是原则性的问题;不能够执行,没有法子贯彻执行,是一个重要的原则性问题。政府的工作,我们实际的政策,政府执行的政策,应该是很合符党的政策的,这方面我们有没有缺点?有缺点的,步调不一,政府内部步调不一,政府与党,政府里头工作的党员与政府党领导步调不一。

第二,群众团体,群众团体有没有这种事呢?也有过这样的事情,现在这个问题解决了。过去群众团体之间也发生过这样的事情,如青年工作,所谓闹独立性,闹什么东西,我不太清楚,反正跟党的政策不相合,随便搞一套。文化团体中的党员对党的政策。对党的领导这个关系在过去也是有过许多毛病的。比如《解放日报》第四版,那时有一个时期是一个独立国,什么人也不能干涉,这一版和别的版甚至变成这样的事:如像大英帝国各个殖民地和英国差不多。在某一点上甚至有过之。实际上这一版是闹独立性。报馆的意志不能在这一版实行。现在我们无论那一版,那一篇文章里那一个字假使要干涉也可以。一共有四万字,三万九千九百九十九个都是服从的,只有一个字不服从有什么关系呢?不提出时不是原则,不提出时不关大局,如果他一个字不准报馆编辑干涉,如此下去四万字都要闹独立性。讲出来党一个字也要管,好像说你这个党挖苦得很,一个字都要管,一个字不要管怎么样?允许他订一个条约,说:你可以闹独立性。那么第二个字,第三个字都来了,他可以闹独立性,为什么我又不可以闹独立性呢?不可以的。党是管一切的,他如果要管一切都是可以的。什么叫党?现在我们,在“九一”决定中已经搞清楚了。好像政府工作就不算党,西北局不算党,木刻、戏剧就不算党,只是群众团体,那么这个党还很小,只有这底下几个房子才叫党,这样的观念是错误的观念。我们所谓党是一切党员,党是党员组成的,……(原稿不清)。旧戏里头有党,新戏里头有党,工会,农会、青年、妇女中都有党。比方除了军队、政府、民众团体,工人团体,农民团体,文化团体以外,还有什么呢?一个党也没有了,就完了。假使这样尊重党的领导,可是我们这里不能叫党,我叫军队,我叫民众团体,那么实际上就是说不要这个党,因此就没有党了。西北局不过是几间房子住在那里,有几个人在那里。

所以关于这一条,我们现在的例子,现在要执行斯大林这一条,就是领导一元化,要承认中央的“九一”决定,要承认这一次大会整个关系的决议案。承认党是一切无产阶级组织的最高形式。一切无产阶级党以外的组织,通过他的党员,一定要归党领导。这种闹独立性差不多自有红军起就有了,在根据地中特别发生这些事情,在根据地以外这个问题少些。

根据这种情况看来,我们说比较布化,当我们写这个字时,我们就想想我们“化”的程度,我们布化的程度还是差一点,拿全党来说,我们是布尔什维克党,拿部分的干部,甚至相当一些同志来说,布尔什维克化还没有,还没有布尔什维克化。

第二条:第二讲论,“必须使得党,特别是他的领导者,完全地精通与革命实践不可分离地联系着的马克思主义的革命理论。”

第二就是说要精通马克思主义,而马克思主义是不可脱离实际的,跟实际是联系的,完全不可分割的。关于这一条,在我们党来说,我们党是有过历史的。我们的历史怎样呢?第一个时期,很高的热情接受马克思主义,在马克思主义指导下,我们干了一次革命,叫做第一次大革命。从有党到一九二七年,前一段是准备大革命,后一段执行大革命。这是在什么原则之下?这是在什么思想之下?是在马克思主义的原则之下,是在马克思主义的指导之下。这一条是不是事实?是事实。因为中国的党是在一九二一年成立的,是一九一七年十月革命胜利,一九二一年中国共产党成立的,那时不但是马克思主义,而且是列宁主义在中国有很大的传播。我们中国马克思主义广泛的传播是在“五四”运动之后。在“五四”运动以前,中国的工人阶级,中国的知识分子,对于世界有所谓马克思主义,还有所谓发展了的马克思主义一一列宁主义是不大知道的,差不多一般说是不知道的。因为中国落后得很,发生了这样大的事变,世界上马克思主义创造了这样久,无产阶级运动有了这样长的时间,现在列宁主义在俄国有一个胜利,有过几次革命,在一九一七年十月革命以前,中国人是不大知道的,有某一些人,有某一些翻译印刷,但是一般的是没有的。但在这一个时期,十月革命本身对中国是有很大宣传,十月革命本身影响了中国许多知识分子倾向于社会主义,相信社会主义。十月革命以后,有很多东西流传到中国来,加之第一次世界大战,世界历史有很大的事变,在这情况下中国产生了“五四”运动,产生了一九一九年的“五四”运动。在这时候克思主义的宣传就更多起来了。就有一些知识分子,中国自己产生了一部分知识分子,以李大钊为首的,就产生一部分知识分子研究马克思主义,开始自觉地研究,所以能够在一九二一年成立中国共产党。不然,一九二一年可不可能建立中国共产党呢?这个可能是没有的。因为在一九二一年以前,一九一七年十月革命对于中国是一个很大的宣传,经过几年的酝酿,又产生了“五四”运动。没有十月革命,“五四”运动的产生都是难的。有十月革命,又有了“五四”运动,又有马克思列宁主义的宣传,所以能在一九二一年开始建设了中国共产党,在一九二一一一九二七年这六年、七年当中,我们党是个幼年的党,幼年的党干了一件大事,就是大革命。像这样一种历史,在外国都不很多,只有六、七年的工夫,干出这样一个大革命来。国共合作,在这中间马列主义起了很大作用,大革命这个运动的本身是在共产国际指导之下的,是在中国共产党指导之下的,中国共产党是在共产国际指导之,那时候共产国际、中共中央以及全党的活动,都是以马列主义作基础,同时按中国党的实际情况提出纲领进行活动。有些人似乎觉得中国党开始没有理论的,特别是靠后一个时期,有那么一些同志感觉只有自己是有理论的。我们就讲五四以前,就讲一九一七至一九二一年这四年中,马列主义在中国已经有相当普遍的宣传。那时虽然没有中国共产党,但是有了共产主义小组。共产党本身就是在马列主义指导下产生的。不然怎样设想呢?怎么忽然产生了一个共产党呢?这个共产党不是马列主义是什么主义呢?是基督教主义呢?还是孔教主义呢?(笑声)后来有些同志觉得从前那一套不是什么马列主义,只有自己才是马列主义,这个我看是不对的。另外还有些同志也觉得我们从前那一套不是马列主义,而只有新翻来的才是马列主义。那有这个事,不是这样说的,事实就不是这样,没有共产党以前,就有了马克思主义了。如果没有马克思主义怎么会有共产党呢?事实上那时候的报纸、刊物、书籍都在,那时李大钊他们就是做这个工作的。一九二一年到一九二四年间李大钊就宣传共产主义唯物史观,那不是马列主义是什么主义?又不是基督教主义,也不是孔教主义,是马列主义,实实在在有书为证。所以从党的建立到第一次大革命是在马列主义指导之下的。以后还有一个内战阶段,一个是抗日阶段,就是两个东西的斗争,一个是马列主义,一个是反马列主义;一个是真马列主义,一个是假马列主义。这个会开了十一天,××同志也这样讲,一边的口号是“为马列主义而斗争”,朱理治的文章就是这样写的,有两万多字;这一边便叫做“右倾机会主义”,也算一个主义,不过是个“机会主义”。(笑声)历史现在证明,这个现在不要书上找证明,书上也有证明而且有人证明,究竟哪一个正确呢?就是叫做“右倾机会主义”的那个正确,被杀的那些人正确。杀了二百四十个,杀的是什么人呢?杀的是马列主义者,杀了二百四十个马列主义者,坚决的马列主义者,很好的马列主义者被杀掉了。再有一批因为中央来了,没有杀得及,大呼一声“刀下留人”(笑声)。这个情形只有一个苏区吗?不只一个苏区有,特别严重的有三个苏区;鄂豫皖、湘鄂西和这里。中央苏区,赣东北,四川苏区也有,不过没这里严重。这是一个错误的路线。一闹,这个错误的路线就损害整个的革命,对中国革命有损害,这中间包括了一部分共产党员,马列主义者,把共产党员与马列主义者屠杀了,这是一个结果。此外,就是搞光十分十、十分之九的苏区工作和全国十分之十的白区工作就是这样搞掉的。第一阶段中间以马列主义为基础,开始,中国共产党领导了大革命,到了后期分裂了,陈独秀也自称是马列主义者,但是那时候的政治路线是错误的,那时候的马列主义,他后期的马列主义是什么马列主义呢?是假马列主义。结果怎样呢?结果被正确的马列主义克服了。

“八七”会议把他清除了。清算了大革命的失败,反对了右倾机会主义,反对了陈独秀右倾机会主义。这种陈独秀右倾机会主义合不合符马列主义?不合乎。是真的马列主义还是假的马列主义?是假的马列主义。什么人把假的马列主义陈独秀这部分人清除出去和他作了斗争?党里面正确的马列主义,后来也向立三路线作过斗争,向右倾机会主义作过斗争。李立三也是假借马列主义之名,所办的事是违反马克思主义的原则的。后头就是“九八”到遵义会议这一时期。所以我们党第一次大革命开始有向陈独秀机会主义的斗争,后来整个内战时期有向李立三的路线的斗争,有向朱理治、郭洪涛“左”倾机会主义的斗争,这个时期整个的斗争,有向张国焘的斗争。张国焘是“左”的机会主义。整个内战时期,就有这样三个斗争。这三个斗争是什么对什么?是马列主义对非马列主义的斗争,对不合符马列主义原则的斗争,向这样的几个路线作斗争,向立三路线、苏维埃后期的路线、张国焘路线,向这三个路线作斗争。两个作了结论,一个还没作结论。我现在作了一个西北的结论,是很好的。关于全党的结论,是不是这样呢?是这样一个性质的,不过那是全党的,你们是西北的。这样一个结论要不要作呢?要的,“七六”就要作这个结论的。遵义会议以后,我们办的事是一种什么路线呢?现在同志们都讲了,现在的路线是比较正确的,比较合乎实际的。这是不是事实呢?我想这个是事实。遵义会议以后,党的政治路线,党的领导思想,组织路线是比较正确的,比较好的。因为鉴于大革命那个时期的胜利,那个时期的失败,又鉴于陈独秀机会主义,又鉴于内战时期我们有很大胜利,可是我们也有失败,我们犯了立三路线,我们中间那个路线和张国焘的路线,有过这一连串的历史。这两个阶段;第一次国共合作,第二次国共分裂。这样的经验就能够使得我们有可能在这样的基础上搞得比较正确一点。全党的觉悟在这三个阶段里也比较过去要提高一些。但是有没有缺点呢?我想还有很大的缺点。在开幕那天我曾经说过,我们党内,现在有一种自由主义。像中央所提出来的一样,我们反对主观主义、宗派主义、党八股,这些不是统治的路线,但是在全党还有残余。这种残余合不合乎马列主义原则?不合的,对于这种不合马列主义的残余应该采取什么态度呢?可以有两种态度:一种态度是自由主义的态度,隐瞒包庇,不注意;一种是向它作斗争,肃清这种残余。为什么要提出整顿三风?从去年七月一日起,中央发了关于增强党性的决定,调查研究的决定,今年二月中央又提出了整顿三风。现在差不多一年工夫了。在全党进行整顿三风的工作,进行了这样大的一个学习,进行了工作检查。这是一种什么呢?就是对不正确的残余,对这个歪风的残余,不应该采取自由主义的态度,应该釆取批评的态度,纠正的态度。在遵义会议后(一九三五年一月)自由主义这种态度就产生了。遵义会议以前是一种“左”倾的错误,对马列主义的态度来说,拿政治斗争,党内关系来说,是一种“左”倾的错误;遵义会议以后路线是正确的,但是党内是不是还存在毛病呢?毛病是有的,选就产生了一种自由主义的坏倾向。在这时期的主要偏向,我们党内的表现是什么呢?就是自由主义的一种坏现象,而不是一种过火的斗争。遵义会议以后,这几年来,我们党内的主要坏现象是过火斗争呢?还是自由主义呢?是自由主义。过火斗争有没有?还是有的。在某些地方斗争还是大的,但不是主要的坏倾向。遵义会议以来,党内主要的坏倾向是自由主义的倾向。在过去犯这种“左”倾错误的同志,在遵义会议以后,就容易犯这种自由主义的倾向。你们这次十一天的会议,朱理治、郭洪涛他们的态度是什么态度呢?过去是过火的“左”倾斗争,“左”的错误。抗战以来他们的态度是什么态度呢?是“左”倾为主还是自由主义?朱理治在银行工作,就闹独立性。郭洪涛散布谣言,挑拨离间,这是左还是自由主义呢?是自由主义,过去“左”的转成了右的态度。

现在我们讲到第二条,我们把中国党分为三个时期,北伐时期,内战时期、抗日时期。这三个时期从开始就是在马列主义之下建设的,马列主义在中国的发展分这样三个时期:第一个时期是以马列主义为基础,指导中国革命;后来产生陈独秀的机会主义,来了一次斗争;第二个时期,以马列主义为基础,指导中国的内战一一国内战争,但是偏向又产生了,李立三路线,中间这样一个路线,张国焘路线,不过这三个路线都被我党正确的路线克服了。哪个路线是正确的,哪个路线是不正确的?那几个不正确路线已把它克服了,现在是正确的路线,而拿这个正确的路线克服了不正确的路线,所以这个路线是比较正确的,是正确的路线。但是还有一个东西在这个时候产生了,如果说过去是“左”,在这个时期,党内到处表现了右的倾向。教条主义是什么呢?就是拿了马列主义实行对马列主义采取自由主义态度,以马列主义作招牌。马列主义自以为懂了,你就要去做,但是不做。我们中国的教条主义有两种:一种是咬文嚼字,做也做,做的时候他就做错了,他乱搬运公式。还有一种不做事,只读书。读了好多“箭”不放出去;一种是无的放矢;一种是有箭不放,当作宝贝古董。在延安近来发生的哪一种呢?是后一种。他们读了许多马列主义的书,连边币跌价都不能解释,这种读书是脱离现状的。这种偏向要不要改革呢?要改革。能不能釆取自由主义态度?不应该,特别是在宣传教育机关中,这个现象不能容忍了,再不能够发展了,所以要整顿三风。边区党员是三千,二千,是二千九百呢?总之有一部分党员,名为党员,实际上于党不利,是反党的,是党棍,我们党内包括一部分反革命奸细,托派反动分子,他们以党员招牌进行活动。吴奚如就是这样一个人,吴奚如是个文化人,是参加高级学习组的人,皖南事变的时候,国民党把他抓住了,以后又把它放出来,叫他到这里来闹乱子,现在向新四军打电报,证明他是被俘过,他也承认怎样当特务,怎样订条件,怎样放他的。王实味最近也发现了,怎样发现的呢?他是以共产党员的资格,在这里讲话,他们组织了五个人的反党集团。中央研究院政治研究室,一共有一百几十个人,他们五个人就组织了一个反党的集团,这些人就是王实味、成全、王里、潘苏、宗真。什么人知道呢?只知道他们是共产党员,那晓得他们反党呢?读他们的文章还说很好。整顿三风时,他们就先要来整顿三风(台下哄动),我们现在有许多党员在这个时期麻木了,不自觉了,许多党员马列主义的作风看不见,容忍这样的人。这样的人现在暴露了,这样的人他终久要说话做事,在他的说话做事的表现中间,不像一个共产党员的样子,但是我们有些同志不懂得,以为这样也算个共产党员,而不把共产党员和这些入加以区别,许多人没有嗅觉,没有警觉性,这都是麻木现象。这次会议我希望同志们要注意这个问题,我们党要来一个清理。这算什么呢?是不是自由主义呢?他们是自曲主义态度。对于主观主义,宗派主义,党八股残余不说话,不斗争是自由主义的态度。在郭洪涛的破坏党;朱理治闹独立性,以前不讲,这次会上讲了,是马克思主义,没有自由主义。过去有些同志不报告,听了他的造谣不反映,这次会议反映了很多关于过去采取自由主义的态度。我们在这里谈了一点马列主义在中国发展的历史。斯大林告诉我们,把它列作第二条,要我们注意理论。我与党校的同志们商量了一下,我们党校准备要读几十本书。中央的理论学习计划在整风以后,还要继续执行。我们党里有相当多的干部,每人要读三、四十本马、恩、列、斯的书,特别是在座的同志们,有经验的同志们,不久我们要读起来,以前我们读了一些,没有这样多。从前党内有这样的两千多人,高级的同志每人读三、四十本马,恩、列、斯的书,如果读通了,那恐怕我们就要大大的提高了。我们在明年就要开始做这个工作。三、四十本读得了读不了呢?读得了。在党校有一年半。就很好了,没有一年半也差不多。有许多书一个星期就可以读完一本,像《共产党宣言》那样薄的小本,不作别的事,一个星期读完一本,一个月读四本,十个月就四十本,读了四十本就差不多了,眼睛就打开了。在座的同志们,工作的同志们,假如有计划,三年的工夫可读四十本。从《共产党宣言》到《季米特洛夫文选》止。可以选三,四十本,有些可以读二十几本,有些甚至读几本,主要的读几本,最高的要读完三、十四本,在座的同志有三年左右的时间,在党校的同志有十个月左右的时期。我们这样实行,这样计划,我们有这样丰富的经验,有这样长的三个历史,再要能读一、二十本到三、四十本马,恩,列、斯的书,这样就把我们的党大大武装起来了。我们的面貌就要有进步,我们的布尔什维克化就能化得更好。现在我们很弱,马列主义的战士很少,理论方面很弱。因为很弱,所以××同志也讲到,为什么朱理治那一套在这里能出卖呢,居然还有人信他的。有些人不信他,又不能很好提出意见说服他。为什么许多人信他的呢?为什么许多人不能拿很好的武器和他作斗争呢?这就是因为我们很弱,在理论上很弱。现在我们要增强理论。增强理论办得到办不到呢?关于这个问题,有许多同志觉得自己对于很多马、恩、列、斯的书很难清楚,很难读懂,自己没有信心。关于这个问题,这几个月的整风学习是解决了的。我看见许多以前认为自己对马列主义没有可能学到的这些人,现在有转变了。以前他们认为读马列主义大概就是些专门读马列主义的人才能精通,这件事是他们时事,至于自己虽然很想读,但是没有可能读通。最近在二十二个文件学习中间,引起了他们的信心,建立了他们学习马列主义可以学到的信心,一点点地,慢慢地学是可以学到的,大大地引起他们读书的兴趣。因此在这次高干会议以后,我们应该有一个学习运动,看书能看进去的同志,能看懂的同志,我们要有这样的目的,一个人选择几十本书,认真地天天读,一这一遍地读,在职的三年为限,学校的十个月为期,一定要实行斯大林讲的第二条。实行这第二条,使我们更加布尔什维克化了。如果不然,那么布尔什维克化是不能那样讲的,不能讲我们就是布尔什维克化了。

第三条,必须使他在制定口号和指示的时候,不是根据背熟了的公式和历史的比拟,而是根据对革命运动所处的国内外的具体条件的仔细分析的结果,同时还必须考虑到各国的革命经验。

我们作决议案,提出口号,向全党,向下面以及向各地作出指示的时候,根据什么呢?斯大林同志讲不应该根据一些东西,而应该根据另一些东西,即不应该根据背熟了的公式、历史的比拟,而应该根据什么呢?根据革命运动中的具体条件,这一条拿我们前一年的历史来说很清楚,斯大林这一条我们很容易理解,首先应该根据这样的东西,有两样东西都可以根据:一样就是公式,历史的比拟;另外一样就是根据革命运动的历史条件。过去我们所犯的错误,所发生的毛病就是根据了第一样那个东西,如果实行他这一条就没有这种错误与毛病。过去我们在历史上犯的几次错误,恰好就是没有依照斯大林讲的这一条。如你们这次会议中,像朱理治、郭洪涛他们那些“指示”,他们根据一些什么东西呢?比如讲“优势劣势问题”,国民党加日本他们说敌人处于劣势,而我们则变成优势,那时全国红军有多少呢?在提出这种优势劣势的时候,全国红军不到十万,后来也只有十几万人,具体的条件很清楚:一个是多数,多数多到那个程度,少数少到那个程度,在这种情况下,在口号中间还有一个“打通国际路线”的口号。××讲过,把这三千人一个一个地摆起来,也不能打通国际路线(笑声)。你只三千人怎么能打通国际路线呢?还有“以洛川为中心,向三边发展”的口号,这也差之不多。(笑声)但三千人从洛川到鹿县的六十里恐怕也摆不够,三千人可以摆多长呢?一个人也只能有这样宽,再有一个胖子也不过这样宽(笑声)。况且还有小鬼。要从具体条件出发。这是根据具体条件吗?不要在稍山里边建立工会,工人根本没有,可是一定要建立工会。还有一个口号是要“打中心城市”,还有一个口号叫做“决战”,朱理治的文章里就有决战这个口号,提出决战那个口号时,还要打中心城市,担绝作会门工作,拒绝做土匪工作,他们发出的就是这样的指示。恰好另外一些同志和他相反,如×××、××他们制定的口号,他们制定的指示,他们要作会门工作、土匪工作,把会门工作,会门斗争的形式,变成对我们有利的形式,把土匪工作,土匪的斗争形式变成有利于我们的斗争形式,如和军阀联络一下,朱理治说这就是和军阀勾结,我们不是和军阀勾结,而是利用他们。那时全国积极要求抗日,我们要打击在野的党派,在朝的打不倒就要打倒在野的,人家已经在野了,你还要打倒他(笑声)?要在稍山里建立工会,世界上那里能找出这样的文件,这样的指示(笑声)?为什么这样?这也因为外国有这样的事情,他们引用历史上的比拟,所以也要写一下。我们也要搞一个集体农场,因为过去苏联有过。要打中心城市,或许是因为在北伐的时候打过武汉,外国一一现苏联也打过彼得格勒,于是我们也可以打中心城市吧?这是不是周密分析的结果呢?这种口号,这种指示是不是对中国革命的具体条件加以周密分析的结果呢?关于那个时候对国际国内的革命具体条件,并没有作过分析,或者作过分析,但这个分析完全是不合实际情况的。因此造成许多笑话,造成大错。另外斯大林还讲到,同时还必须考虑到各国的革命经验。他这里把国际国内的具体条件作周密的分析放在第一位,而要不要考虑国际经验呢?他讲必须要考虑,不考虑国际经验不行。国际无产阶级革命运动有广大的经验,苏联是革命已经胜利了的国家,这样好的经验你不要,完全靠自己搞,完全要自己来,对不对?不对的。但应当把这一部分放在后面,把那一部分放在前面。以前有一种译本是把后一半放在前面的,现在有正确的译本,首先要照顾到国内国外的当前的具体情况,并加周密的分析。我们制定的各种口号、指示,就应该是依据这些具体条件加以周密分析的结果,应是根据这种周密分析的结果来制定口号、指示,同时必须考虑到外国的经验,但对外国的经验要恰当的估计,不是硬搬。

从前在这一点上的缺点就是硬搬。要知道外国经验是在外国当时当地,所谓它的时间、地点、条件的东西。我们中国应当考虑到人家这种经验,而且必须要考虑,但是必须要估计到两种时间、地点、条件的不同。外国反对自由主义,外国反对德波林,我们也反对自由主义;我们也应该反对中国的“德波林”。但是,是不是一样呢?不消说,自然不能一模一样。苏联要肃清主观主义、宗派主义、党八股,我们现在也要肃清主观主义、宗派主义、党八股;我们要不要釆取那个经验呢?当然要采取的,但是一模一样的搬取就不行。在外国,在苏联有过清党运动,我们现在也要把党来一个洗刷。要不要采取外国的苏联的经验?要采取外国的经验,但主要应从我们国内的具体情况加以周密的研究分折。比如王实味确确实实是个托派,吴奚如确确卖实是个特务,边区里也确实有一部分党员是党棍,是坏人。我们对这些人,对我们党员,要一个一个具体来看,要按照具体条件来处理,同样也要考虑到外国的经验,正如上而所讲讲,制定口号这些东西都必须注意。比如我们刚才讲过的历史上有许多缺点、许多口号制定的不确当,糊里糊涂制定口号,像朱理治,郭洪涛制定的就是这样。现在我们看,今天有些什么口号,有抗日统一战线,抗日的口号制定了没有?制定了,是怎样制定的?是从当前具体情况出发制定的。像国共合作,各阶级合作。又比如,从前我们有一个八小时工作的口号,现在我们还坚持这个口号,但八小时工作制在中国现有条件下,是一个宣传口号,不是一个行动口号,而不是像过去某些人所认识的那样,在农村里马上实行八小时工作制,朱理治,郭洪涛就要在稍山里实行八小时工作制。我们现在还是十小时工作制,……八小时工作制只当作一个目标。将来全国工业发展了,到那个时候一定要实行八小时工作制;苏联现在工业发展的结果,实行了七小时工作制,比八小时还减了一个小时,所以,八小时工作制度只是一个目标。另外,过去还提出了不分富农、中农的田的口号,今天我们提出了减租减息,缴租缴息的口号以及抗日统一战线的口号。我们所提出的口号是不是根据国内外的具体情况加以很周密的分析呢?是不是根据这些条件周密分析的结果而制定的呢?是根据周密的分析,周密的研究。如减租减息,缴租缴息的口号在今天是适用的,在抗日统一战线时期是有效的。又比如三三制的口号,是不是根据当前国内具体条件周密分析的结果呢?是的,我们周密分析各种经验,认为要实行三三制,所以我们开参议会,我们和各党派人士合作。我们今天实行三三制是糊里糊涂制定的呢?还是周密分析以后制定的呢?是周密分析以后制定的。十小时工作制度,减租减息,交租交息,三三制,抗日民族统一战线(这是总口号)都是这样制定的。今天整顿三风、精兵简政这样的口号提出来了,如整顿三风这个口号,是糊里糊涂提出来的呢,还是确有需要呢?回答说,我们现在确有需要。精兵简政是糊里糊涂提出来的呢?还是确有需要?项英同志很早以前即提出了精兵主义,这在当时是不恰当的,只有现在提出这个问题才恰当。这是在现在这样具体条件下我们加以具体分析的结果而提出的口号,各地方要实行,就要讨论,我们边区一定要实行精兵简政。有些同志在去年参加过参议会,在参议会通过精兵简政以后,并没有加以周密具体的分析,因此不能够进行彻底。精兵简政过去不只进行过一次,为什么会进行得不彻底呢?就是因为没加以周密分析,没有把具体条件加以周密分析。比如我们边区有多少入,要穿多少衣服?这中间是有矛盾的,老百姓少,公家人多,所以我们今天一方面要发展生产,一方面要精兵简政。究竟在今天能够简多少?政府系统,党的系统,民众团体系统,执行这个精兵简政政策的时候,也应该有一个精密的分析;像现在西北局,边区政府、联防司令部他们所作的那些工作,是先加以具体的周密的分析,然后再加以具体的规定,这个工作作风是对的,、不是那样糊里糊涂的说现在要精兵简政了,就简。以前简过好几次都说没有简彻底,那就带了一点糊里糊涂,几次精兵简政都没有实行好,没有彻底。糊里糊涂,没有加以周密的分析。所以斯大林讲的这一条要不要注意呢?我看斯大林讲的这一条很重要。我们要更加布尔什维克化,“就要注意这一条。如果没有这一条行不行呢?如果没有这一条就不行,就要以洛川为中心向三边发展,就要打通国际路线,在稍山里建立工会,搞集体农场,打大城市,再搞肃反把马列主义者肃掉,这些经过具体分析没有?没有,就是糊里糊涂乱杀一顿。斯大林讲这一条,就是讲我们要具体分析,所以这一条就是一个方法论,是一个思想方法,是二个看问题的方法。我们拿前一年的历史反省一下,过去凡是对具体条件加以周密分析研究,同时及顾外国经验,那工作就做得好,反之,不注意对具体条件加以周密分析研究,工作就做不好。结果制定的口号,政策就不合实际。不合实际的口号、指示就一定行不通,做不好,所以这是一个方法论,这是第三条。

第四条,“必须使党在群众的革命斗争的烈火中捡查这些口号和指示的正确性”。

这里有个第四条,后面有个第十二条,都是讲检查,但这两条的意思是不同的,这条里的“检查”指什么?主要讲检查口号、指示,实行原则的路线;后面是讲检查工作作风。这个路线的实行,口号的执行,具体的工作。第三条说,要根据对革命运动所处的国内外的具体条件的仔细分析来制定口号和指示。这是一个方法论。但是单有这一部分够不够呢?不够的,因为你说是根据具体条件的仔细分析所制定的口号、指示,方法论还没有完。还有一部分,就是说所制定的指示要在实践中间得到证明。理论从实践中抽出来又在客现实际中得到证明。第三条讲的这个从实际调查中抽出来就是讲所谓调查研究。口号,指示,决定、决议案,就像你们现在所做的讨论的东西,从实际反映,成为口号、指示,这是方法论的,第一条。还有一条,你们说正确么?那么那时他们也讲根据实际呢?朱理治也讲的,他也是根据国际形势有两个世界的对立,国内形势有两个政权的对立等……。要在实践中考验,结果是白区十分之十,苏区十分之九搞掉了,在革命烈火中检查这个口号,检查结果怎样呢?烈火中那个可谓烈矣,国共两党十年内战可谓烈矣,检查的结果,那时的口号、指示正确不正确呢?不正确的。像打中心城市,稍山里建立工会、打通国际路线,肃反这些东西,无论那个都是不正确的;而另外一些东西是正确的,××、×××倒是正确的,岂不是革命烈火斗争检查正确不正确呢?朱理治、郭洪涛不信我们试试看,使得使不得,不然就转弯。现在我们的指示,我们刚才讲了:抗日民族统一战线,三三制,减租减息,交租交息,十小时工作制,整顿三风,精兵简政……等等。像我们这些大大小小的政策,这些口号,这些指示,究竟跟实际合不合?我们要来检查。比如我们过去党内的教育制度,学校的教育制度,我们检查这个制度合不合教学的方法?我们检查结果是教条主义的教育方法。那么怎样?就要改。在革命斗争的火焰中,马列学院,党校也算火焰,教育了一两年检查一下,检查里头有教条、里头有三风。过去我们边区有两次作了精兵简政的决定,现在实行了几次,在今年春上,检查的结果,还不彻底。今年我们这个高干会,决定了许多,这次高干会以什么为准呢?开这个会是不是正确?我们承认是正确伪。如果不承认正确,那我们就不要作决议,既然作了决议、就要承认是正确的,是根据具体条件周密分析,这样作出来的。但是是否正确?不在于嘴上讲的,究竟怎样?最后的证明要在你去做,作对了就是对。但在作的过程中,不对的还要修改,你这决定的许多东西中是不是每条完全对?如果每条完全对,那是你周密的分析很恰当;如果还有一二条不对,那就是周密的分析还差。一切中央的指示、口号、决定都是这样,不能糊里糊涂地讲中央一切都对。这样讲的人是盲目的,是一个没有觉悟的人,我们说现在中央的都对,这是因为中央过去有这样多的经验,有这样大的胜利,、我们犯过许多错误,现在是比较慎重的解决问题,是来一个周密的分析。但是什么时候才能彻底正确?只有在实行以后,证明这个东西以后,才能说正确,不然是不能说的。中央不能说自己提出来的口号,指示一经提出,不要证明,就是正确的,不能这样说的。如果可以这样说,斯大林的第四条提出来做什么呢?岂不是多余了吗?理论是从客现实践中抽出来的,又从客现实践得到证明,这是第四条。我在这里讲话,我决不能说我的话一讲出,那就完全正确,这要在实行以后证明他是正确的,才是正确。今后任何领导者,任何同志要有这样的态度。客观实践,实践是考验真理的标准尺度。用什么尺度来衡量是真理呢?就是用实践这个尺,可以说不在他的决议案、宣言、而在他的行动。朱理治、郭洪涛他们的宣言,我看也差不多,他隐瞒党,单把他所说的所讲的一点和过去此,有进步,但我们相信不相信呢?在他作出以后,过去他隐瞒党,以后不隐瞒党了,过去胡乱搞,今后不乱搞了,过去犯错误,以后不犯错误了,过去破坏党,以后不破坏党了……那时我们就信了。马、恩、列、斯告诉我们说:宣言,决议案这些东西,不是检查一个党,一个干部,一个同志的主要标准,检查一个党,一个干部,一个同志的主要标准是在他实行的结果。所以第四条是非常重要的,是非常主要的。一个人做事是有计划的,他要做决议,口号,没有第四条是不行的。单有第三条,没有第四条是不行的。

第五条:

“必须按照新的革命的方式来改组党的全部工作,特别是当社会民主党的流毒在党内还没有清除的时候。这样党的每个步骤和每人行动就能自然而然地使群众革命化,使工人阶级的广大群众受到革命的训练和教育。”

这一条讲什么东西呢?这一条就是讲工作作风,他这里的原意是不要社会民主党的作风,要有布尔什维克的作风;不要有改良的作风,要有革命的作风。因为社会民主党那时就没有革命的作风,而是一种改良的作风,所以应当把党的作风改造过来,放在新的革命的基础上,新的革命的步调上,新的革命的精神上。俄国布尔什维克是从第二国际分裂出来的。第二国际是改良主义统治的,布尔什维克党就是从那种改良主义中分裂出来的。第三国际是从第二国际分裂出来的。因为它不革命。条件变化了,今天是无产阶级革命的时代。社会民主党呢?它是改良的作风,它没有办法领导无产阶级去革命。所以要有一个新的党,列宁就建立了这样一个新的党。各国按照布尔什维克作风建立了共产党,如法国、俄国,英国、美国和我们中国。这样的作风,他这指示、口号、政策、作风实行以后,自然而然地使群众一天天革命化,使工人农民就不再那样想改良,不再妄想第二国际想的那一套,使他们一天天革命化,教育广大的群众,他讲的本意就是这个。这对于我们怎么样呢?我们有没有社会民主党的传统?我们没有。×××同志的文章讲得很好:我们中国没有社会民主党的传统。我们中国党有两个缺点:一个是“左,犯过许多大的错误。一个是右的,我们今天讲这一条,应该讲什么呢?就应讲反对自由主义。现在我们党内自由主义相当浓厚,同时我们当然还要反对过“左的,比如征粮工作,不调查乱派一顿,命令主义。这样实行能不能让群众自然而然地革命化呢?这是已经革命的地方,还可以使他们更革命,还需要他们更革命。比如,当兵打日本,要出公粮,就是革命工作,命令主义能不能使边区农民自然而然地出公粮?当兵呢?不能的。同时我们有自由主义,就是刚才讲的。党内有很多的坏蛋,有王实味,有吴奚如,我们中央研究院,过去的马列学院,这具体的例子,是不是一种新的作风?那里头缺乏新的革命的作风。我们鲁艺,在报纸上看到,在××同志领导下,现在转向新的革命作风。你说过去就不革命吗?不对的,过去也是革命的。那里有很好的新的革命作风,使得群众自然而然革命化,中央研究院的研究员、鲁艺的学生,这都是群众。怎么样使他们自然而然革命化?我想具体的步骤就是整顿三风的办法。在这个月中间,你们看中央研究院,在范××同志的领导下,整顿三风是在新的革命步调下实行的。这些步骤,使中央研究院的群众自然而然地革命化,使得鲁艺的群众,在××同志的领导下自然而然地革命化。现在每一个机关,比如“解放日报”社,五湖四海,集合了许多入,要特别注意自由主义。我们边区这地方有党校,有边区师范,有财政系统,有银行,有各种机关,从各方面来的入都有,大家不熟悉,不了解,带来了各种不同的倾向,我们就实行这种步骤,叫做整顿三风,实行这一种此过去更进一步时新的革命作风,使这些人自然而然革命化。比如什么笔记啊,漫谈会啊,大座谈会啊,小座谈会啊,传观笔记啊,开展批评啊,实行这样的步骤,使得这些人自然而然地革命化。全边区要实行这个步骤?你们回去也要实行这一条。我们过去的作风,有许多不健全与不正确的地方,那是改良呢?还是革命化呢?我们没有社会民主党的传统,我们拿每一个口号,每一个指示、每一个动作,使干部,使群众有新的革命作风,这样使学生自然而然地进步,使机关工作人员工作效率自然而然地提高。过去,不安心工作的问题是没有解决的,是存在着实际问题,很多的入,天天想调动工作,是不是事实呢?是的。有多少呢?不只一两个,有好几百,好几千个在工作岗位上不安心,因此我们就来一个适当的解决,就是整顿三风,自我批评,这样一来,他们就安心了。他们调动工作也愿意了,从前调动工作哭哭啼啼,讲价钱,现在觉悟了,已革命化了,已进步了。斯大林讲“改良与革命”对我们有没有用呢?有用的。但是对于社会民主党问题,我们是没有的。我们有的问题,命令主义,自由主义。这些东西我们不要,我们就是要新的革命作风。我想这样来解决,这样来实行这一条,这样来培养和教育工人阶级的广大群众,这样来培养和教育党内的广大干部,党内的广大群众,这样来培养教育广大青年,然后在这个基础上,来培养和教育边区这一百四十万人民,这样来培养和教育我们三、四十个工厂里的工人和工作人员,这样来培养和教育各个机关里的工作人员,这样来培养和教育我们军队里的干部和士兵。是不是应该这样做?是不是应该这样解释?我想是应该的。这第五条对我们是有用的。

第六条:“必须使党在自己的工作中善于把最高的原则性(不能和关门主义混为一谈!)和与群众最广泛的联系及接触(不能和尾巴主义混为一谈,)结合起来。不然,党不但不可能教导群众,而且也不可能向群众学习;不但不可能引导群众并把他们提高到党的水平,而且也不能倾听群众的呼声和预料到他们的迫切需要。”

这里讲这一条要把原则性,要把革命的原则性同联系群众结合起来,不然,党就不可能教育群众,而且不可能向群众学习。你要讲要教育他们,要向群众学习,要倾听他们的呼声,我们对群众的关系是:一方面要教育群众;一方面要向群众学习。如果不是这样。想把革命的原则同联系群众结合起来是不可能的。不联系群众就不能教育群众,就不能向群众学习。你只讲党的最高原则是不行的,党员的作用是指导群众,联系群众,教育群众。不联系群众,你有什么办法教育群众?有什么办法向群众学习呢?你不联系嘛,教育个屁。教育不了,学习不了的。你不联系就脱离,你与群众脱离了关系,不仅不可能引导群众和把群众提高到党的水平,而且也不可能倾听群众的呼声,以及推知群众的迫切需要。党是无产阶级的先锋队,与群众脱离了关系你怎么提高他呢?要把群众提高到党的水平,先锋队的水平,将来,几十年,几百年以后,逐渐地提高到了党的水平,那时共产党就不要了,全世界的阶级都废除了。阶级废除了以后,群众的文化发展了,教育发展了,群众跟党差不多了,那时候党就不要了。我们现在在革命的过程中,就要这样逐渐地教育群众,把群众提高到党的水平。如果不联系不接触群众,那怎样去提高他们呢?而且也不可能倾听到群众的呼声,你怎样知道人家的迫切需要呢?人家的迫切需要是要打游击,他们迫切需要的是农民的事,而你在这里要在稍山里建立工会;人家需要的是分土地。你需要搞集体农庄;人家需要十小时工作制,你要搞八小时工作。

这第六条是讲群众工作问题。斯大林告诉我们对于群众工作要“把群众的日常生活上的需求同基本要求联系起来”,基本要求就是最高原则。但是用什么方法去搞社会主义、集体农庄、八小时工作制呢?这就要依据具体的条件,尤其是现在中国的环境,要求低一些,比如破除迷信是最高原则,我们是唯物主义者。但是话一说出来,就到处破除迷信,到处打庙,就发生了问题。比如在×县有一个庙打了,老百姓就不高兴,我们是唯物主义者,我们是先锋队,我们是相信共产主义的,我们主张破除迷信,知道没有什么神与佛,我们是不要庙的你们为什么还要呢?不知道他们现在还要是有原因的,是因为生活条件、经济条件的限制。比如,外国人开洋船,我们中国人是开木船,开木船就要信龙王菩萨,开洋船就不必信了。木船不信龙王菩萨翻了不得了,洋船翻得很少,很大的风浪也不怕。乡下信观音菩萨,所谓送子观音。破除迷信是最高的原则,如果今天破除了迷信,同志,他们没有中央医院,人家生娃娃生不出来怎么办?生死了怎么办?(笑声)我们有中央医院可进,中央医院有金大夫,有外国医生,如果生不出,他可以开刀破肚,如果生不出可以挖出来。那就是灵得很,这就不要迷信了。迷信是有原则的,是受生产条件、经济条件限制的。比如我们是反对求菩萨吃药的,但有人要求神,因为这个有两个好处:第一吃了不会死人;第二比较便宜。要是请医生,还要给他吃猪肉,这就要好几元钱,还要送“色对”。医生的药有时还要吃死人的,其实如果到处有西医,如果中医进步的话,那么求菩萨也会减少。我们这破除迷信是个最高的原则,但我们现在应该迁就他们。比如婚姻自由是我们的一个最高原则,现在“解放日报”写了一篇文章,相当地鼓励了这点,而“统统是乱七八糟”(此处未听清楚)。我们说不要破除迷信吧,这就是忘记了最高原则。中国讲民主,但在重庆、西安去讲是不行的,我们现在忍耐一下。所以要把最高的原则性同群众当前的日常要求联系起来。不要忘记了最高的原则性:破除迷信、婚姻自由、民主集中制、社会主义、集体农场、打大城市……等等最高原则不要忘记,这个原则是我们建立党的最高目标,忘记了就不算共产党。可是还有一条,一定要按照群众的要求。今天能够办到的,可能做到的,就这样来做,这样才算与群众密切联系,才算与群众接触了。澎湃同志在海陆丰,他在那里自己也去敬菩萨。澎湃同志是农民运动的大王,他是一个大学生,又是一个留学生,他是一个地主,又是全国农民运动的大王,是个共产党员,是中央委员。他怎样做呢?他自己去拜观音菩萨。老百姓是二月十九日拜观音菩萨,你也去,他们就说你好。如果不去,他们就说你不大好。为什么不相信菩萨呢?你也去,他们就说你好。如果不去,他们就说你不大好,为什么不相信菩萨呢?我看你这个人不大正派吧,菩萨不可信,我说我们就来信一下吧,你说我相信菩萨,人家看到你是大学生,又是留学生,看到你相信菩萨,人家说,你说个好同志,群众见了你就请你坐下来,就请你吃茶,这就是联系了群众,还应该穿群众的衣服,要同群众打成一片,如果戴这样高的拿破伦的帽子,穿一双皮鞋,手里拿着司的克,这样就不行,一定要穿农民的衣服,要同群众打成一片,要迁就他们的落后,要接近他们,同他们联系,倾听他们的呼声,这样就可以教育他们、也就可以向他们学习,也就可以懂得他们的迫切要求。我们懂得了以后就要慢慢来,把土豪打了,田地一分,菩萨就少了。在中央苏区就只有老太婆还相信菩萨,特别是有儿子在军队里当兵的老太婆,他们还相信菩萨,他们求菩萨保佑她的儿子打胜仗,如果她有几个儿子在军队中,或者死了儿子的,她就更相信菩萨了。而在一般的群众中就不相信,特别是青年人不相信菩萨,因为他们是身强力壮的,靠自己,所以他们不相信菩萨。老太婆相信菩萨,她们是有各种原因的。这是一条,就是群众工作的原则,要把这两个原则结合起来,一个是最高的原则,一个是联系群众,最高原则不要变成了关门主义,郭洪涛、朱理治他们搞的是关门主义;联系群众不要变成了尾巴主义。敬观音菩萨敬一次还可以,如果天天敬,这样就把共产党混同于老百姓一样了,就变成了老百姓的尾巴。人家天天敬观音菩萨,如果你也天天敬观音菩萨,这样的相信观音菩萨,你就放弃了最高原则性,相信菩萨要能把握住最高原则性。苏联还有教堂,但是现在全国教堂已经没有用处了。现在只有几个老太婆还相信,在高加索有几个地方的老头子,老太婆,他们还相信,他们相信也就让他们相信,但是不要跟他们做尾巴。苏联天天反对信宗教,信宗教有信的自由,反宗教有反的自由;信宗教的宣传宗教,反对宗教的也宣传反对宗教。信宗教的,他们天天喊上帝万岁,如果取消了反对宗教,专门宣传宗教,也天天喊上帝万岁,那就变成了尾巴主义。这一条是群众工作的原则。应该不是关门主义,又不是尾巴主义,要作两条路线的斗争。

第七条:“必须使党在自己的工作中善于把不可调和的革命性(不能和革命的冒险主义混为一谈)和最大限度的灵活性及机动性(不能和迁就行为混为一谈,)结合起来。”这就是要把革命性与机动性,灵活性这两个东西相配合,如果不然,党怎么样呢?那么党就不可能掌握各种斗争形式和组织形式,不可能把无产阶级的日常利益和无产阶级革命的根本利益联系起来,也不可能在自己的工作中把合法斗争和非法斗争配合起来。

有些在前面已经讲过了,这一条特别提出,第六条着重于群众工作,第七条讲统一战线原则,讲统一战线的战略战术,讲斗争的战略战术。斗争的形式,组织的形式,这就是战略战术问题。这一条可以成为一大本书,斯大林每一条都可写成一大本书,第二条是理论,第三、第四条是唯物主义,每一条都可成为一本很大的书,第六条讲群众工作,第七条讲统一战线,斗争形式,因为我们不但有部分基本群众,而且有一部分别的阶级,别的集团。我们革命要按照可能性,可能与他们联合就同他们联合,这里有一个最大限度的灵活性与机动性,比如我们的三三制,这个三三制是最大的灵活性与机动性。我们要把不调和的革命性同、灵活性配合起来,我们曾经打倒过地主,现在要联合地主,实行三三制。过去曾经没收过地主的土地,现在是减租减息与交租交息。这个办法叫什么呢?叫灵活性、机动性,这是不是丧失了革命性呢?托派这样讲我们,也说我们投降了资产阶级,投降了国民党。过去我们同国民党作战十年,今天日本人打来了,又同他联合,这叫做灵活性、机动性。而托派说我们反革命,投降了国民党。我们说相当的改善工人生活,他们说我们投降了资本家。所以“左”倾的人就拿这些话来骂共产党。说共产党是右倾。列宁作了一本书,叫做《共产主义运动中的“左派”幼稚病》,那本书上讲的就是这个第七条,告诉党员要把革命性同灵活性、机动性配合起来。那时候欧洲英国、法国、德国、荷兰、意大利的共产党中有些人提出反对打迂回,想在一个早上革命成功。莫斯科的革命成功了,我们还做什么国会工作,他们反对做国会工作,也反对利用国会做讲台,他们反对同社会民主党左派联合。列宁在一九二○年四月作了这本书,过去不大注意它,郭洪涛、朱理治过去就没有看过这本书,也不忠实这本书,看,也是眼花了没有看进去,而这上面许多东西都讲了,第三条上面的东西大概那上面也讲了,那本书上主要讲了统一战线问题,讲反对“左”倾机会主义,如果不讲灵活性,只讲革命性,那么有没有可能来掌握各种斗争形式,各种组织形式呢?没有可能的。

要只讲打仗这一种斗争形式,这种斗争形式是主要的斗争形式,武装斗争形式。八路军、新四军要不要灵活性呢?我们去开参议会。去开国民参议会,这个要不要呢?我们还是要的。我们还是要到重庆去开国民参议会。就拿日本占领区、沦陷区来说,八路军在华北打,新四军在华中打,天津,北平、上海、冀中现在是日本占领了,我们岂不是束手无策吗?人家占领了,我们打不出,我们有没有办法?我们有种办法,叫做合法斗争。伪军是合法的,人家承认的,我们可以利用伪军,在伪军中做点工作,比如像怠工,这不是很大的违法,如今天推托出发,修路推迟两个钟头到工也不算很大违法,这就是掩护我们的工作人员过路,晋西北老百姓在沦陷区用怠工的方式不去修沟,有几个区的老百姓就不去修沟,这是合法的,是在合法形式下进行的。八路军、共产党提出这样的口号,但沦陷区老百姓要提出这样的口号,头就会不见了,那一个区域的老百姓,李四张三跑出来说:“日本人,我们打倒你。”那他的头就保不住了。要搞合法斗争,你老爷叫我修路,好吧,回去后就可以少修一点,可以不修一点。现在日本士兵开了代表大会,有很多条要求,在“解放日报”上登了,在各个地方都作了。那些要求是什么呢?要日本人发手巾,要日本人允许炒菜馆子存在。这是什么斗争形式呢?这是合法斗争。因为炒菜馆子是日本人承认的,因为法律上有,后来不承认了,发手巾法律上也有,现在不承认了,可是没有手巾就不能洗脸。这是合法斗争,这个如果实行了,手巾发了,炒菜馆子也能存在了,也就满足士兵群众之意了,那么这个合法斗争就胜利了。没有手巾就发动斗争,这就慢慢把觉悟程度提高了,变成为群众运动,这样的一次二次三次就慢慢地把,群众的革命性,积极性,觉悟程度提高了,这就是合法斗争与非法斗争的配合。列宁在《共产主义运动中的“左”派幼推病》中讲:“布尔什维克三次革命的经验比世界上任何一国都是最丰富的。”的确,是最丰富的。布尔什维克在俄国十五年中(一九○三一一一九一七年)经过三次革命,有合法斗争和非法斗争,流血斗争与不流血斗争。这一方面,中国的经验也许还要丰富些。因为我们合法斗争与非法斗争是同时存在的。布尔什维克在一九○五年以前没有苏维埃,只可以作些合作斗争。我们不然,我们过去十年内战是非法,可是在白区我们可以作些合法斗争是他们许可的,如搞合作社,黄色工会。那时××同志搞黄色一工会、合作社,讲他是机会主义,这种说法是错误的。在国民党区内可以作两种斗争:一种是利用合作社、黄色工会,学校等等作合法斗争;另外还可以搞非法斗争,白区也有暴动示威,只要条件可能,虽然法律不允许,我们还可以做。在日本人区域也是一样,在日本人区域作非法斗争,有时还不适宜,有时还是要的。现在法国、德国立刻准备推翻希特勒的斗争是必要的。我们演过一个新木马计的戏,他所写的斗争是什么性质呢?是合法斗争与非法斗争两种,像神秘的不能使人知道的那一部分是非法的,像可以同人家合作的,如要求加工资。章程上有的他不做,我们就要求,这是合法的斗争。新木马计描写了这两种斗争,比如《水游传》上的祝家庄,两次打不进去,第三次打进去了,因为做了新木马计,有一批人假装合作打宋江,帮助我们打宋江的那就欢迎的很,相信他们,这就是合法的,但暗中准备非法斗争,等宋江打到了面前,内部就起来暴动。革命没有内部变化是不行的。中国的三打祝家庄,外国的新木马计都是这样。单单有一种斗争形式,只有合法斗争,没有非法斗争也不行,如社会民主党。单有非法斗争,没有合法斗争是“左”倾。列宁曾批评过“左派”幼稚病,要把这两个东西配合起来,要有灵活性。现在联合整个世界反对法西斯,这个阵线包括罗斯福、丘吉尔、还有我们中国的国民党,包括范围这样广大,苏德协定并没有放弃和英美的关系,苏联的大使还在英美,而苏德协定破裂的时候,英美苏协定便建立起来。英美苏合作不成功,因为张伯伦反对,要搞慕尼黑。张伯伦要联合希特勒打苏联,我们让他们打张伯伦,我们要和德国合作,因为他要联合德国打我们,德苏协定,德国和我们订条约。张伯伦到了胜利就变心了。实现公开化就是去年(一九四一年)六月二十二日的苏德战争。英美苏等国的联合,灵活得很,如果没有这样的策略,不采取这样的斗争形式和组织形式就不行。有哪些组织形式呢?英美协定、工会、青年大会,美国也有个青年大会,援军青年大会,有中国代表,有美国代表,这是组织形式。要灵活地运用各种组织形式来达到革命的目的,只有一种死板的固定的组织形式与斗争形式是不好的,斯大林专门一条讲统一战线的组织形式与斗争形式的问题,要有最高的革命性,同时要有灵活性。从前是打倒国民党,西安事变的时候我们变化了,我们联合国民党。人家进行反共高潮和我们斗争,我们只好斗争,等到形势能够转变的时候,我们立刻放弃斗争。阎锡山搞新军的时候,我们不能不援助新军。当那时候我们能够搞好,我们便和他搞好。斗争的手段是为了达到团结,我们要有理有利有节。今天只讲这一点,下次再讲(鼓掌)。

今天我把这个问题讲完。关于第六条、第七条还有些同志不大清楚,为什么原则性不能和关门主义相混淆?为什么与群众最广泛的联系及接触不能和尾巴主义相混淆?底下又说,为什么不可调合的革命性不能和冒险主义相混淆,最大限度的灵活性及机动性不能和迁就行为相混淆?关于这一点,这个原则性,他是说与群众最广泛的联系及接触的最高原则性,这是讲我们做群众工作,我们对基本群众,比如讲群众里面有先进分子和落后分子,要把落后的提到先进地位,要把群众提高到党的水平,提到先进的党的水平,这就是最高的原则性和最高的觉悟性,但是不要离开群众,这里头不要有关门主义,所以讲不要与关门主义相混淆。如果讲原则性,比如拿党内来讲,经常有许多同志所谓提高到原则的高度。比如在中央苏区,有一个学校里头,就是关于吃辣椒的问题也开展斗争,把吃辣椒提高到原则高度,说不准吃辣椒,如果吃就没有原则,提高到原则的很高程度,这个东西是不行的,这那里是原则性呢?这种东西实际上不是原则性,而是关门主义,是乱用原则。这叫做原则?这个不叫原则,这叫做关门主义。这样只能搞一个不吃辣椒的党,那么吃辣椒的人都不能进来。所以有许多关门主义假借着原则性,他们的口号是原则性,实际上不是原则性而是关门主义。我们要善于区别,不要把原则性同关门主义相混淆。原则性是广大群众他们所适合的,广大群众今天不适合,将来也是适合的。比如前天讲破除迷信,这一点要破除,但今天广大群众信迷信立即提高到先锋队的不信迷信,是不行的。我们怎样把广大群众的信迷信提高到先锋队的不信迷信?这就要和广大群众接触,逐渐地慢慢地一点一点地工作。在这个陕北,你来破除迷信,你来提倡婚姻自由都要注意这一条。所谓原则性,我们要善于区别。有些人假借原则这个名词而实际上是关门主义,那是实际上不要原则性。至于联系广大群众时又不要忘记了这个原则性,最高的原则性不要忘记了。如果忘记了,比如迷信的破除,今天老百姓是信迷信的,因此我们就不做破除迷信的宣传工作,一切信迷信的可进工会、农会。你讲在我们党内也有少数人信迷信,也有的,我们陕甘宁边区三万个党员,你说一个也不迷信吗?信迷信的还是有的,还有一部分,还有一部分不赞成婚姻自由的。那么现在我们不能拿信不信迷信,拿赞成不赞成婚姻自由来作为检查党员的标准,要拿另外几条,作为标准。我们跟群众最广泛的联系,但不要做群众的尾巴,不然就做了群众的尾巴。这叫做最广泛的联系但不能做群众的尾巴。最广泛的联系是一件事,做尾巴又是另一件事,因此我们讲最广泛的联系不能与做尾巴相混淆。我们讲最高原则并不是脱离群众,但我们要向着最高原则性走去。我们要经常记得:最高原则绝不是关门主义,关门主义是一种东西,原则性是另一种东西。这是讲原则性和落后性,怎么使群众的落后性提高?但不是尾巴主义?我们要经常记得,最高原则性绝不是关门主义。

至于第七条“不可调和的革命性和最大限度的灵活性及机动性&quot;,这是讲革命与妥协的关系。共产主义运动中的“左派”幼稚病,那本书上已经讲了,因为西欧的共产党,他们提出这样的口号:“不作任何妥协。”列宁批评这个口号是要不得的,是错误的。但是当西欧的共产党提出那样的口号的时候,他们自命为“很革命”,自命为“不可调和的革命性”,不作任何妥协是革命性。但是他们是不是革命性呢?不是革命性,这是冒险主义,他们不作任何妥协,可是群众的觉悟是这样,他们不利用国会,不利用合法斗争,不看群众的觉悟程度,他们不作国会斗争,不利用其他的合法斗争,就是进攻,这就叫做冒险主义。这个时候要迂回,要妥协,列宁讲到俄国布尔什维克党的历史的时候,他说:在党的历史中曾经作过许多妥协,这一妥协要有最大限制的灵活性和机动性,但这不是迁就行为,第二国际的行为才是迁就行为。共产党布尔什维克要和别的阶级、别的成分进行妥协,进行合作,甚至利用敌人。例如俄国的布尔什维克利用了沙皇的国会,这个国会是敌人的,不是同盟者,但是因为法律上有这样规定:共产党员可以进去当议员,共产党可以以若干票当议员。这时候全国没有大革命,像这样的当然可以利用,如果不利用,这不是革命,实际上是冒险主义,是冒险的进攻。所以需要灵活性,这个灵活性是最大限制的灵活性,比方讲这样的例子,苏联跟德国法西斯妥协过,就是在去年六月以前的德国,曾经订过德苏协定。我们这个殖民地半殖民族国家的党,在十年前就和资产阶级妥协过,在全世界上恐怕中国第一次和资产阶级进行妥协,共同进行反帝反军阀斗争。甚至现在全世界共产党,除了法西斯国家的资产阶级外统统同他们进行妥协的。在中国是同国民党、同地主,在我们边区是同地主。所以那样思想是不正确的。但是无产阶级对于别的阶级只有小资产阶级能够领导它,至于资产阶级那是不能的。现在我们不但同资产阶级合作,而且同地主合作,各个根据地的地主阶级、资产阶级在共产党领导下共同抗日。英国资产阶级的保守党,美国资产阶级的民主党,中国的资产阶级、地主阶级,现在世界上的大多数的资产阶级和地主阶级,无产阶级,无产阶级政党,无产阶级祖国苏联都同他们进行妥协,这个妥协不同于迁就行为。这个妥协吗迁就行为吗?这不叫迁就行为,这是把不可调和的革命性和最大限度的灵活性及机动性结合起来了。我们和他们进行妥协,是不是忘了革命,就一直妥协下去了?资产阶级、地主阶级他们是这样想,这样希望,但我们不是这样,我们是妥协不忘革命,革命不忘妥协(笑声)。现在需要妥协,因为妥协有很大利益,而且要同全世界大资产阶级、地主阶级妥协起来。比如波斯同苏联订条约,波斯的皇帝向苏联妥协了,但这不叫迁就行为,完全丧失立场。革命完全不管,比如张国焘就是这样,张国焘跑出去以后到了汉口,在《大公报》》上落了宣言,发表他的意见(这个宣言现在还可以找到),他讲的妥协就是迁就行为,就是完全不要原则性,完全不要革命性,革命与妥协的关系他只要妥协不要革命;现在我们有些同志,过去那些冒险主义,一一九一八以后的冒险主义,在这里开会,你们清算了朱埋治、郭洪涛两位冒险主义,他们是“左”顿机会主义,是清算什么东西?他们是不是革命性?他们不是革命性是冒险主义。

这种冒险主义绝不能与革命性相混淆。革命中间的冒险主义,他们的这种冒险主义,他们自称为革命性,自称为是革命理论,自称为是革命的政策,所以有些人也觉得他也有革命理论,也有革命的政策,这是不懂得什么叫革命的理论,革命的政策的人,就糊里糊涂地无疑问地相信他这种宣传,以为这是革命的理论,革命的政策。但是这不是你搞错了,这是冒险主义。我们讲不可调和的革命性不是和冒险主义相混淆,不可调和的革命性是一个东西,冒险主义又是一个东西,这是两个东西。斯大林讲了,在统一战线中右倾的东西可能增长,所以我们要防止那种借口最高限度的灵活性,实际上变成了迁就行为。典型的是张国焘,口头上是灵活,实际上是迁就。斯大林讲了要我们注意,这是这六条,第七条的补充。

第八条:“必须使党不掩饰自己的错误,使它不怕批评,使它善于用自己的错误来提高和教育自己的干部。”

第十条:“必须使党经常地改善自己的组织的社会成分,清除那些腐化的机会主义分子,以便达到最大限度的团结一致。”

这两条在联共(布)党史结束语中也分作两条:关于反对机会主义,关于我们自己队伍中间有错误应该批评,这两条分开。在这里第十条的范围和《联共(布)党史》结束语所讲的那个似乎多少有些出入,有些不同,但是大体上意思差不多。现在我们在这里就把《联共(布)党史》结束语第四条来看看。《联共(布)党史》结束语一共有六条;第一条讲党,要有一个革命的党,同这个布尔什维克化的十二条的第一条一样,基本点是一样的;第二条要精通马列主义理论,要有革命的理论,跟布尔什维克化的十二条第二条一样;第三条讲工人阶级的统一;第四条讲共产党的统一;第五条讲自我批评;第六条讲群众工作。关于工人阶级统一,第三条就是讲工人阶级统一,工人阶级队伍中间应该是统一的,那些欺骗工人阶级的政党,实际上是反革命政党,应该和他们斗争。要有革命的党,不是社会民主党,而是共产党,列宁式的党,我们有马克思主义、列宁主义的武器,这样的党,有马克思主义、列宁主义的武器的共产党。工人阶级要清洗自己的队伍中那些反革命分子,要与反革命分子作斗争,使工人阶级统一,还有就是使党统一起来。

第四条就是讲党的统一,我们就是讲党的一元化。

其次,党史教导我们说:工人阶级政党如果不同自己队伍中的机会主义者作不调和的革命斗争,如果不粉碎自己队伍中的投降主义者,就不能有自己队伍的统一和纪律,就不能实现其为无产阶级革命的组织者和领导者的使命,就不能实现其为新社会即社会主义建设的使命。

“我们党内生活发展的历史,乃是在反对党的机会主义集团――‘经济主义者’,孟什维克、托洛茨基分子,同布哈林分子反民族主义倾向者作斗争并把他们粉碎的历史”。而斯大林分析了这种东西,说应该粉碎,并且说:“也许有人以为布尔什维克为了与党内机会主义者作斗争而耗费的时间未免太多,以为我们把这些机会主义分子的意义未免看得太高。但这种想法是完全不正确的,决不能容忍自己队伍中间有机会主义存在,正如不能容忍健全身体上有毒疮生长一样。党是工人阶级的领导部队,工人阶级的先头堡垒,工人阶级的战斗司令部。在工人阶级的领导司令部中,决不容许有缺乏信心者、机会主义者、投降主义者和叛徒之足。在自己的司令部中,在自己的堡垒中留有投降主义者和叛徒而不同资产阶级作殊死的斗争,就会陷于腹背受击的地位。不难了解,这样的斗争只会受到失败的结局。堡垒是最容易从内部攻破的。为要达到胜利,首先就必须把工人阶级政党中间,工人团级的领导司全部中间,工人阶级先头堡垒中间所有的投降主义者、逃兵、工贼和叛徒清洗出去。”(《联共党史》四三三页第八行至第十五行。)

这里讲了几种人,讲了机会主义,投降主义,叛徒。比方讲机会主义,如果他不是叛徒,那就是机会主义。但是党除了对机会主义以外还有没有别的?还有叛徒,比如王实味、吴奚如。前天讲了吴奚如表面是共产党的,实际上给国民党作事,这叫机会主义嘛!王实味是个托派,他在这里组织五人民党集团。这叫什么?这类叫叛徒。机会主义就是在政治上,比方讲,这一次你们开了十一天会,清算了过去的历史,朱理治、郭洪涛过去那一套搞在陕甘宁边区来损害党,损害革命。这样的前方堡垒一一陕甘宁边区共产党内有这样的机会主义者,那么在进行斗争的时候是什么呢?前面是敌人,后面也是敌人,堡垒中也有敌人,参谋部内也有敌人,结果捉了许多人,把××同志,×××同志也捉起来了,杀了二百四十个干部,几乎把领导人都杀了,把大批共产党员都杀了,××,×××也几乎杀了,你说危险不危险,堡垒是最容易从内部夺取的。这叫三打祝家庄。一打祝家庄打不进去,二打祝家庄打不进去,三打祝家庄打进去了,为什么打了进去呢?小说作者写得非常好,写得完全符合事实,不但正面进攻打进去,而且那堡垒是从内部夺取的。新木马计告诉我们,要我们从内部夺取。我们对敌人的态度如此,敌人对我们的态度也是如此。如果我们自己,我们堡垒里面,我们党里,特别是我们领导机关,这个危险在什么地方呢?危险性就是在最高领导机关,像中央、中央局、中央代表团、大代表团、小代表团等这样的权力机关里头有机会主义。朱理治究竟是不是叛徒?现在要查,查出来是叛徒的话,就是以机会义的帽子进行叛徒的实质。我们同意一个同志的分析,同意张秀山同志的分析,张秀山同志讲得对,他说:他的前途有三个:第一个前途,反对党,退出共产党,走到反革命方面去;第二个前途,继续两面派;第三个前途改正错误。这三种都有可能。这种分析我认为是很恰当的,这三种是讲现在,将来总有一条,要就是反党闹事;要就是继续两面派,隐瞒党,欺骗党;要就是改正错误。至于这三种可能那一种可能性大?现在难以分析,我看我们现在的政策,还是允许他们在党内,使他们进步,这是我们的。但还有他们自己,要他们自己觉悟。如果他们自己觉悟了,可能走第三条路一改正错误,不然就是第一条或第二条。

《联共(布)党史》结束语第五条讲什么呢?党史又教导我们说“如果党竟因为胜利而骄傲起来,如果党竟已看不到工作的缺点,如果党竟害怕承认自己的错误,害怕及时来公开诚恳改正这些错误,那它就不能实现其为工人阶级领导者的使命。”

这里讲到应该看到缺点,应该看到自己工作中的缺点,不要骄傲,应该看到自己工作中的错误,不要害怕,应该承认错误,不要怕及时的公开的承认这些错误,要公开的承认,并且要诚恳的纠正,要公开的讲,不要关在小屋子里几个人讲一讲。既然认为是错误,就应该公开的纠正错误,公开的承认,诚恳的纠正这些错误。如果不是这样那么党就不能成领导者。

“如果党不害怕批评与自我批评,如果党不去掩盖自己工作中的错误和缺点,如果党能根据党工作中错误的实例来教育和训练干部,如果善于及时改正自己的错误,那它就会是不可能战胜的。”

这里《联共(布)党史》告诉我们要在犯错误中间来教育自己,教育干部。如果隐瞒错误又怎样呢?这里讲了,第一,讲应该怎样;第二,讲如果我们纠正了这些错误,就会很好,这样党就是不能被战胜的,如果总隐瞒不纠正又怎么样呢?“如果党竟隐瞒自己的错误,抹杀迫切困难的问题,用百事大吉的粉饰词令来掩盖自己的错误,不能容忍批评和自我批评,浸透自满情绪,一味自高自大,高枕而卧那它就会必遭灭亡。”

我们现在有没有这样的人呢?这不是路线问题,所谓机会主义路线问题,也不是什么叛徒、奸细的问题,而是我们工作中个别错误的问题。(这样的)。这次这个会,过去十一天是清算过去历史上的问题,以后若干天还有讨论,这个性质是什么呢?这里我们又反对闹独立性反对自由主义的错误,我们要讨论闹独立性的问题,自由主义错误的问题,我们要精兵简政,我们要整顿三风。现在我们整顿三风的基本方向是什么呢?就是说要纠正自由主义的错误,纠正主观主义,宗派主义,党八股这些错误。犯了这种错误的人,犯了独立性、犯了自由主义错误的人,他们应该釆取什么态度呢?应该采取隐瞒错误的态度吗?不应该的。对党中央、西北局、高干会、应该隐瞒这些事情吗?高干会是党的,西北局是党的一个组织,高干会是整个边区党的高级人员的会议,应该隐瞒自己的错误吗?如果隐瞒这个错误,如果隐瞒这个迫切的困难,精兵简政是不是迫切的困难?整顿三风是不是迫切的困难?如果百事大吉的虚夸(粉饰词令)来隐瞒自己的缺点。我们边区党的系统,政府系统,军事系统,民众团体中有妇女、工人、农民、文化人等,应该不应该用百事大吉的虚夸(粉饰词令)来隐瞒自己的缺点?不应该!应该承认错误,应该运用批评与自我批评。人家对我们批评,我们自己也应该作批评。如果党开始骄傲起来,不看见自己工作中的缺点;如果党害怕承认自己的缺点,害怕及时的公开的承认、纠正这些缺点,那么党就可能免于灭亡。所以这里讲的,党史教导我们说,我们应该承认错误,应该看到错误,应该及时的公开的和诚恳的纠正错误。

第二、如果是这样的公开纠正错误,承认错误,那么党是不可战胜的。党史再告诉我们,我们党如果不是这样,害怕批评和自我批评,百事大吉,隐瞒错误,抹杀迫切困难的问题,一味自高自大。《联共(布)党史》讲到自高自大,在我们延安也有自高自大,不好!老子天下第一,不好!这里没有讲老子天下第一,你加上一句也差不多,这里讲自高自大要不得,把枕头枕得高高的睡觉,那就要灭亡。

我们讲,如果不纠正错误,主观主义、宗派主义、党八股就要发展,就要亡党亡国亡头,你这个头也要亡掉,二百四十个干部的头亡掉了,××,×××的头几乎亡掉了,“刀下留情”恐怕党就不免灭亡。《联共(布)党史》在这里首先引了列宁下一段话:“政党对于本身错误所持的态度,就是表明这个党是否郑重,是否在真正执行自己对本阶级和劳动群众所负义务最重要最可靠的尺度之一。”对于错误所抱的态度,是考察这个党是不是真正的党,是不是认真办事的党,是不是在真正执行它的义务。这个党对于无产阶级,对于劳动人民的义务,这就是讲要公开地承认错误,揭露产生错误的原因,分析产生错误的环境,仔细讨论改正错误的方法一一这是郑重的党的标志,这就是执行自己的义务,这就是教育和训练自己的阶级,以至于群众。这种错误是不是包括路线错误?我想也可以包括路线的错误。在一些时,以及不久以前的时候,党的领导机关犯路线错误,因此我们应该公开承认路线的错误。因路线的错误而一直走下去,像孟什维克路线,托洛茨基路线的错误,后面没有作反革命的事,还是革命的时候,那是路线错误,但是不包括成为一种派别的继续的斗争,同党对立起来,以至于走到反革命,这是属于那一条呢?这是属于第四条。我想这个也可以包括犯大的错误,也可以包括犯小的错误,个别的错误。至于现在我们整顿三风这个思想斗争是属于那一条呢?是属于第五条的,《联共(布)党史》结束语告诉我们,要我们有这个区别,要把这第四条和第五条加以区别。成为党内一种派别,小组织派别,继续他一贯的路线错误,这是属于第四条的。这些应该给以无情打击。在第三条里头引了列宁的话,他说:“在社会革命时代,只有极端革命的马克思主义政党,只有同其它一切政党进行无情的斗争,才能实现无产阶级的统一。”对于第四条,要清除出去,要像割我们身上毒疮一样割掉;必须把工人阶级政党中间所有的机会主义者,投降主义者,逃兵、工贼和叛徒清除出去。至于第五条,就是一个教育态度,就是要用教育的方法,治病救人的方法。对于第三条、第四条,党外反革命的派别,用治病救人的方法吗?不能用的。因为小组,小派别,一贯的反党,那不能用治病救人的方法,对王实味、吴奚如也用治病救人的方法吗?这个不是的。他们可不可以觉悟呢?当然他们也可能觉悟。我们陕甘宁边区施政纲领上写:对于反革命分子或者特务分子,我们给他回头,给他们活路。但是这些人能不能当党员了?不能当党员。过去我们把这个东西,(现在已需要讲一下,)就是我们党内长期没有这个区别,没有第四条和第五条的区别,把第四条和第五条混合起来。现在我们根据《联共(布)党史》结束语,要加以区别。假如真正是一个共产党员,他们想做好,现在我们能釆取第五条,“惩前毖后,治病救人”,反之,没有希望了的,暗中釆取两面派,继续第二条,第三条,继续破坏党的行为,像第四条中所列举的,像俄国的孟什维克、托洛茨基分子,布哈林分子等等这样一些,没有问题?这些人后期当了敌人的侦探,而他们前期因为一贯错误路线,这些人应该用第四条的态度。把他们清除出去,给他们以无情的打击。甚至还可以救药的,还有希望的,用第五条的态度。我想以后应该这样的分别第四条和第五条。我们的党,现在正需要这样的分别,过去是没有这个分别的、长期在我们党内,对于这两条是分不清的,差不多犯了小的错误,个别的错误,也给以无情的打击,跟反革命没有区别,共产党员犯了错误,跟反革命没有区别。

第八条就是《联共(布)党史》结束语的第五条,必须使党不害怕批评自我批评,不去掩盖自己工作中的错误和缺点,能根据自己工作中错误的实例来教育和训练干部,善于及时改正自己的错误。

这里我们解释一下宽大政策,党内的宽大政策和党外的宽大政策不同,这两种宽大政策在党内有很大的影响。遵义会议以来,我们党内有宽大政策,对外也有宽大政策,但是这个宽大政策不同的,和一些同志所想的那样的宽大政策是不相同的。譬如像发展党员的决定,一九三八年三月中央有个决定,要发展党员,那个上面说要大批发展共产党员,同时说不要使一个坏分子混进来,这叫做开门政策,大开门,但他同时又关了一扇门,对坏分子是关了门的。你说要宽大政策,是的,我们要大批吸收共产党员,革命的积极分子可以进来,但对于坏分子要关门,对于不是坏分子是革命的要开门。我们的施政纲领是对外的,对外要釆取宽大政策。敌人分子,俘虏,特务分子,不管他是什么人,只要他回心向善,不坚决作坏事,愿意改正错误的,应该采取宽大政策,这是不是正确的?是正确的。对愿意改正错误的是宽大政策,但是没讲对任何反革命统统宽大,这样日本人也可以来开会,日本飞机可以降下来在这里添了油再飞上去(笑声),这是不行的。中央的宽大政策一一对党内对党外的宽大政策,是不是像一些同志所想的那样?这不行。我们的宽大政策在实行中有了毛病,有了自由主义,把宽大政策变为自由主义,在各方面生长了自由主义,我们的党务工作有自由主义,政权工作有自由主义,军队工作有自由主义,我们的财政经济工作有自由主义,锄奸工作有自由主义,我们的宣传工作也有自由主义,以致使我们一些部门,有相当多的人,眼睛看不清,像吴奚知这样的人也不去看一看,像王实味这样的人也不去看一看。吴奚如、王实味的文章发表在我们的党报上,我们的党报会发表不正确的文章。我们党报上的文章是不是统统都是革命的人写的?现在证明有两个反革命的人:一个叫吴奚如,一个叫王实味,在延安反革命,以共产党员的招牌在共产党的党报上发表他们的文章。其余是不是还有第三个吴奚如,第四个王实味呢?这是应该审查的。我们各部门工作中有没有这样的自由主义态度?我们所谓宽大政策,麻痹了、误解了,解释不正确,而把自己搞得昏头昏脑,很多问题不加解释,没有斗争,对于干部保持一团和气,有了斗争是不是原则性地展开批评自我批评呢?很多地方没有。说这是执行中央的干部政策,像这样的执行,不是正确的执行中央的干部政策。有些部门,有些干部没有严肃的态度,强调一方面,缺乏一方面。团结教育,强调了这方面,缺乏另一方面,批评,对错误的斗争。经常教育、斗争性这方面是缺乏的。如果缺乏这方面,这样的政策叫不叫干部政策?至少不能讲是完全正确的,这样的干部政策是错误的,是有错误的干部政策。干部之间的关系,领导者与被领导者之间的相互关系,有一个原则:第一个是团结,第二个是有了错误要斗争,要有正确的批评与自我批评。这就是布尔什维克化的第八条,也是《联共(布)党史》结束语的第五条。

我们是不是完全执行了这第八条?我们是不是完全实现了《联共(布)党史》结束语教导的第五条?我们的干部政策完全布尔什维克化了吗?我们党内关系完全是布尔什维克化了吗?拿斯大林讲的十二条的第八条和党史的第五条来看,有许多部门,过去如中央研究院,鲁艺,从前的马列学院,那样的自由主义作风,以自由主义的态度对付马列主义,对付党。我昨天讲过这两个学校改造得很好。从前那种作风一延安许多机关学校的作风,那叫正确的关系吗?那叫正确的制度和干部政策吗?不能的。在那些严重的地方,完全是不正确的;在那些不十分严重的地方,是不十分正确的。我们有些同志不善于发现错误,不愿意公开承认自己的错误,不承认这个错误的严重性。所以列宁告诉我们要“公开承认错误,揭露错误的原因,分析产生错误的环境,仔细讨论改正错误的方法”,我们办了没有?没有办。我们就有错误,我们就是采取了自由主义的态度。

最近中央关于对党外的宽大政策,有了一个解释,再过几天请博古同志印些单张,在会场上发一发,大家讨论下这个问题。现在有相当多的反革命分子、奸细藏在我们党内,藏在党的环节中,我们不懂得,因此忽视了,好像不大要紧。抗战以来大批人进党,其中混进一些坏人来,他们藏在共产党里,我们不知道,对于这样一些人也釆取宽大政策吗?我们现在整顿三风,精兵简政,是两种斗争,一种是无产阶级思想对小资产阶级思想的斗争,以无产阶级思想反对小资产阶级思想,对大多数党员是这样性质的斗争;但还有第二种,是革命对反革命的斗争,这就是对于吴奚如、王实味这样的人,现在我们开始在几个机关里审查,发现这样很严重的问题。因为在座的同志都是高级领导同志,在各个机关里,领导部门里,要非常仔细地注意这个问题,要分别这样的两种斗争:一个是第四条,一个是第五条。第四条是一贯反党的错误路线,后来一直走到反党的那样派别,那样的思想,要无情的打击,清刷除去;第五条就是讲犯错误,共产党员犯错误,有大的,有小的,要自我批评,揭露犯错误的原因,分析产生错误的环境,应该区别这两种。当前整顿三风的时期,也正是有这两种:一种是对我们大多数同志,拿中央研究院来讲,一百二十多个人,反革命的只有几个人,那么他们一百一十多个着重自我批评,着重自我教育,是无产阶级思想对小资产阶级思想作斗争;对五人反党集团是革命对反革命的斗争。我们现在有两种斗争,这两种斗争是互相联系的,没有自我批评和互相批评,反对就查不出来,最近两个月,大家学习,整风,发现了这两种斗争,在无产阶级思想和小资产阶级思想作斗争的过程中,发现了反革命,展开了革命与反革命的斗争。一个是少数,一个是多数,少数是领导人,多数是群众,多数人要和少数人配合,只有这样才能作好。我们整顿三风也只有少数的领导和多数的群众相配合,我们无产阶级思想与小资产阶级思想的斗争,革命和反革命的斗争,这两个斗争要少数和多数的配合,一定要有少数人的领导,但只有少数人是不行的。

怎样使我们党完全统一,达到高度的一元性,使我们的心成为一条心,团结得像一个人一样。你说团结得像一个人一样,怎样像法?人有男的女的,不一样,老头子和十七,八岁的娃娃不一样,高的矮的不一样,我们是讲我们的心一样,不是讲别的一样。我们要一条心,但我们党内有一种人只有半条心(笑声),完全是无产阶级思想的人是一条心,还有小资产阶级思想的人要用无产阶级思想克服它,有小资产阶级思想的人是半条心,他要革命,但有许多东西是妨碍革命的。再有一种人是两条心,就是王实味,吴奚如,朱理治到底是半条心还是两条心?这是路线的不同,郭洪涛和朱理治他们是两条心,××,×××他们是一条心,这两种人是两条路线的不同。那么吴奚如,王实味他们也是两条心,他们不但是两条路线的不同,是革命与反革命的两条心,在革命队伍中有路线的不同,这是两条心;革命与反革命也是两条心,路线不同实际上是反革命,但他们自己还是一个共产党员犯错误,假如朱理治不是反革命故意混进党内,那么是共产党员犯了错误。现在我们要把这两条心分别清楚,路线不对就是要把路线的错误克服。是反革命,革命要把反革命清除出去,要把半条心用教育的方法,自我批评的方法教育过来。陕甘宁边区没有半条心?闹独立性是一条心还是半条心?闹的轻的就是半条心,闹的重的就是两条心,张国焘他闹的重了,就是两条心,在边区大体上说还是半条心。现在党,政、军、民、学、有闹独立性的不是一条心,这样是不行的,我们要步伐整齐。

在这里我们还要讲一下关于批评错误的提法。这一个问题有的同志深怕抹杀了他的成绩。有没有这样的同志呢?有。这样的同志不只一二个,有相当的一部分,深怕抹杀了他的成绩。抹杀成绩,这样是不对的,因为本来有成绩,为什么要抹杀呢P但是问题要看怎样的提法。王实味他有一个提法,我们又有一个提法。王实味他是不讲成绩的,抹杀成绩,只暴露黑暗,他是反革命的,他要达到他的目的。我们自我批评,现在这里有两个文件可以作证明,我在党校二月一号整顿三风报告中,关于我们全党成绩的问题,这个报告一共有一万多字,二月八号反对党八股的,报告也有一万多言,两个文章合起来有两万多字,但是讲成绩的只有一百多字。那么,是不是我把成绩抹杀了呢?我没有抹杀,这里有这样一段:“我们党还有什么问题呢?党的总路线是正确的,是没有问题的,我们党的工作也是有成绩的。我们党有几十万党员,他们在和人民一道,在领导人民,向民族敌人作艰苦卓绝的斗争。这种英勇牺牲的精神,这种为人民服务的成绩,这是大家看见的,是不能怀疑的。”我在这里讲的只有一百多个字,一个也不多,一个也不少,一百另一个少一个(笑声),那就对不起(此段是按照《整顿党的作风》一文排印,字数此原搞略有减少)。关于我们全党的成绩,我们中国共产党这二十一年全党的成绩,我只讲了这一百个字,全文有多少字呢?全文共有二万多字。但是讲成绩我只讲到这一百个宇。那么是不是可以讲我抹杀成绩了呢?我不能这样讲,我这里并没有抹杀。我们的成绩只有一百个字,其它都是讲缺点和我们的错误。至于这个错误放在什么位置呢?就是说,“主观主义,宗派主义、党八股,现在已不是占统治地位的作风了。”假如还要多讲的话,还可以讲一点,扯它一长篇,要这样作是可以作得到的事。但是我们现在是作的什么呢?我们作的是自我批评,不是吹成绩,我们只有这样一句话:“主观主义、宗派主义、党八股,现在已不是占统治地位的作风了。”谁敢驳我们说,现在主观主义,宗派主义,党八股完全是占统治地位呢?我讲不是占统治地位,这个是上了书的(笑声)。关于理论的问题,我们首先要问“我们党的理论水平究竟是高还是低呢?确实,我们的理论水平是比较过高了些。”再下面讲的就是我们的理论还是不够。你说我统统都讲我们的理论水平不高吗?和过去一样吗?我没有这样讲,我讲的是比较过去高了些。如果再要多讲的话,也还可以扯它一长篇,但是我们现在是作什么呢?我们是要把缺点指出,所以有这样一句话就差不多了(笑声)。

关于知识分子,我们批评知识分子的缺点,骄傲自大,脱离实际。关于知识分子的长处,“我们尊重知识分子是完全应该的,没有革命的知识分子,革命就不会胜利。”唉呀!那么我们革命知识分子就了不起!再下面都是讲知识分子的缺点。关于中央“九一”决定、“关于统一抗日根据地党的领导及调整各组织关系的决定。”怎样估计我们党的成绩呢?我们的统一,我们的团结,估计了没有呢?我们估计了,就是“抗战以来各抗日根据地的党的统一领导,一般是统一的,团结的,党、政、军、民、(民众团体),各组织间的关系基本上是团结的,因而支持了几年来艰苦斗争的局面,配合了全国的抗战。这有什么证据呢?这是有证据的。假如没有统一、团结,乱七八糟的话,那么抗战怎么能打了这么久呢?所以关于自我批评的提法,据我所知道的,还有相当一部分同志不懂得这个问题,有些同志在会议上怕抹杀他的成绩,在会议上非要摆成绩不可,那么我们可以雇上一个人,吃饱了饭,一天到晚地说:成绩,成绩,成绩!成绩!……(笑声)这样的讲上一个钟头。如果还不嫌少的话,我们边可以再讲,这是可以办到的。但是同志们,我们在这里开的是什么会?如果是驳王实味,那就不同了,因为他是否认成绩,我们就摆一大堆成绩给他看,他否认成绩,我们就是要成绩。我们有一百个字和这七十个字是讲好的,我们不否认成绩。我们的目的就是讲:我们已经有了这些成绩,但是我们的成绩还是不够的,我们现在还有毛病,闹独立性,还有自由主义,这有其他一些具体的毛病,因此,有那三两句话来讲成绩就差不多了。再下面就是分析缺点和错误。列宁曾经说过:要我们“公开承认错误,揭露错误的原因,分析产生错误的环境,产生错误的条件,仔细讨论改正错误的办法。”但是我们有些同志不懂得这个东西,他们不愿意公开地承认他们的错误,不愿意分析错误的原因,不愿意分析产生错误的环境、条件,他们过去不愿意这样,也不会这样搞的。

现在讲第九条:第九条讲什么呢?“必须使党善于把先进战士中的优秀分子选拔到基本的领导核心中去”。

这一条斯大林已作为布尔什维克一个条件,要有这个条件才能是布尔什维克。没有这条就不是布尔什维克。这条讲什么事情呢?对我们现在有没有意思呢?这条很值得我们注意,而且过去我们对这条很有收获。但在许多地方还有许多同志,他们完全不懂得这条,这一条的重要性他们完全不懂。这里形成领导核心的问题,有中央领导核心,有地方领导核心。中央的领导核心是经过大会选举出来的。斯大林告诉我们,在我们中央领导机关建立的时候要注意地方领导核心、机关、学校、团体的领导核心要不要呢?也要的。没有领导核心,事情办不好。现在我们特别是在一些地方的机关团体中不注意,完全没有注意,或者注意的很不够。这就是讲如果没有领导核心,真正要办好那里的工作,建立根据地,可能不可能呢?不可能。要办好一个学校可能不可能呢,不可能。要办好一个团体可能不可能呢?不可能。比如说文化团体,没有一个领导核心,要把文化工作搞好,可能不可能呢?不可能。青年工作亦是一样。边区应该有一个领导核心。过去在边区工作的同志,相当多的人不懂这一条。似乎不要领导核心可以,似乎我自己就是一个领导核心。(笑)

他觉得,我是一个领导核心,他亦觉得我是一个领导核心,我也觉得我是一个领导核心,这样就是三个领导核心了。(大笑)因为各人都以自己为核心,那么就是有了好多核心了,但领导核心只能有一个。一个桃子剖开来有几个核心吗?不,只有一个核心。领导的一元化,还是两元化,三元化?什么叫闹独立性?独立性实际上就是多元论,个人自以为是领导核心(笑),你要附属于我,我是核心(笑),你们都是我的附属品我便舒服了,这不是实事求是。“九一”决定便讲这个问题。“九一”决定这里有两种领导核心。边区已经开始实行了,以西北局为领导核心,所有党、政、军,民、学的领导同志都要参加西北局,但是要集中在西北局,不能各人搞一样,在一个床上睡觉、作的梦各不相同,在关中、三边、直属县,陇东、绥德,要建立五个领导核心。别的根据地也是一样,建立两种领导核心。关中党、政、军、民、学没有一个集中领导机关,怎样进行斗争呢?怎样搞经济,怎样搞财政,怎样搞干部教育,怎样和特务作斗争呢?不可以的。为什么呢?只有统一才好进行斗争,才便于和敌人斗争。在机关学校中,还有好多同志,不运用这一条。现在中央研究院很好了,过去那种状态好不好?不好!过去那种状态,一百二十个人,像一百二十块砖头,在这里没有分别,没有形成领导核心。那时,王实味他们形成了一个领导核心,即五人反党集团。我们共产党没有形成这样的领导核心。王实味利用整风,墙报一出来,那时便是王实味的天下了,许多人被拉到他们那里去了。他们有讨论,他们有汇报,他们有研究。你文章不会作,我王实味写一篇,用你的名字。他们用多种策略、汇报制度、研究政策、研究如何斗争,你写的文章我怎样驳都研究一下。他们有领导核心,但我们没有,所以在前一时期,我们打了败仗。尔后,我们便集合部队,批评我们的缺点,好多已经跑到王实味那边去了的人,由于我们的部队开会检讨了一番,又一个一个的回来了,反过来打王实味了(笑)。中央研究院最近大半年是很可以研究的,是一个很大的教育。现在形成了领导核心,是怎样形成的?是不是随便搞三四五六块砖,便成为领导核心呢?在一百二十块砖中,随便找几块出来,便是机械的凑合,不是有机的配合,这样是不行的。是要从斗争中产生出群众的领导分子,积极的活动的最先进的分子。现在中央研究院不同了,不仅一百二十人觉悟了,特别有几个人,他们会鉴别王实味、李实味,他们有办法,他们是在斗争中形成的干部,只有在斗争中才能形成干部。但是现在有的机关、学校,没有注意这一条。平均主义,大家都是干部,不是在同志中,在干部中,依照干部的实际来分析。天天讲分析,自己不作分析。假如分析,总是大体上可以分为积极的、中间的、落后的三种,只要上了一百个人,便有这样的情况。这次分析边区三万党员,有多少积极的,多少中间的不是很坏的,多少很坏的?我们作了任务。讲党内政策,对于积极分子要团结,在积极分子里,要由一少部分人组成为领导核心。另外还有许多积极分子,仍可吸收他们中有威信有办法的。通过他们去教育很多人,教育不很积极又不很坏的,教育他们联系群众,你也联系一个群众,他也联系一个群众(笑),这就联系了广大群众,于是乎,大家进步,对少数落后分子进行教育。领导核心自己要教育自己,要进步。很多人进步了,少数落后分子也落后不起来,也不大像样子了。你们联系群众我看就不大像样子。反革命王实味,现在住在窑洞里怕得很,过去王实味谁也去看他,你的“野百合花&quot;写得好呀!(笑)现在是臭狗屎了。我们的鲁艺、延大、党校、解放日报社、军事学院,那一个部门,那一个机关,那一个地方,都要依照这个办法。这是斯大林同志讲的第九条,搞个领导核心。

这个领导核心,究竟有什么条件呢?斯大林讲两个条件,就是“这些优秀分子十分忠诚,足以成为革命无产阶级的意向的真正表达者;他们有丰富的经验,足以成为能运用列宁主义的策略和战略的真正的无产阶级革命领袖。”人家讲是十分忠诚,不是九分,十分忠诚,十分有经验。这里讲的主要是全国全党的领导核心。每一个学校,每一个机关要十分忠诚。人家说要十分忠诚,你只有九分忠诚,那这个领导核心就建立不起来,就不合斯大林的条件?那也不是的。那也要按照你的情形,要把一百二十个人中间,最积极,最原则的分子团结起来。

十分忠诚,十分有经验,什么叫作忠诚?斯大林同志讲:所谓十分忠诚足以成为无产阶级的代表。所谓十分有经验,是足以成为领导阶级斗争的领袖。简单的讲就是这样。原文是这样:“这些优秀分子是十分忠实的,足以成为革命无产阶级之意向底真正的表达者,并且他们是十分有经验的,足以成为无产阶级革命底真正领袖,善于运用列宁主义底策略和战略的领袖。”他的忠诚足以成为无产阶级的代表,他的经验足以成为领导阶级斗争的领袖。领导阶级斗争就是(运用)马列主义,所以他底下讲;要善于运用马列主义的策略和战略。就是阶级的代表,不是说个人的代表,小集团的代表;来一个宗派主义好不好?来一个派别代表,个人代表、个人野心家好不好?那可不行。斯大林说,这可不行,而是要足以成为无产阶级的代表,阶级意向的真正的表达者。朱理治、郭洪涛过去所办的事,是不是真正的“代表&quot;?“真正的”三个字不要,“革命无产阶级之意向”那几个字也不要,他是朱理治、郭洪涛的代表,是一部分人的小集团的野心家,或者是一个人的代表,他个人的代表,代表他自己。闹独立性,这个独立性到底代表谁呢?代表无产阶级吧!但无产阶级已经讲好啦!无产阶级的总代表是斯大林(笑声),他在第一条中就讲好了,我们是不要独立性的。而他们呢?又要闹独立性,你们就不能代表,没有资格,资格取消了。这样,领导核心的人,第一条是忠诚,第二条是有经验;忠诚足以成为阶级的代表,不是小集团、个人野心家,闹独立性。在今天,他们足以对付王实味,足以教育同志。比方在一个学校里,善于拿无产阶级的思想去克服小资产阶级的思想,又善于拿革命去克服反革命。在各县的同志又善于作别的一些事,要善于财政经济,善于征收救国公粮,还善于一些别的事,要作这样的领导核心。

第十条:

“必须使得党经常地改善自己组织的社会成分,清除那些腐化党的机会主义分子,以便达到最高限度的一致性。”

这一条,上面已经讲过了。它指示要经常改善自己的社会成分。怎样叫作改善?比方说我们边区有三万党员,要洗刷一部分,现在听说有三千个党棍、极坏的分子。其他的要教育,那么现在在边区三万党员中,清洗那些腐化党的机会主义分子。党棍是不是叫腐化党的机会主义分子?叫做。在延安也清洗了一部份。比方讲,如果把王实味这一套包括在内,要洗刷的还有与王实味不同的,没有组织五人民党集团的,就是那些很坏的,完全不够党员资格的,也要清洗他们。要经常清洗,要经常吸收。现在边区就有许多在党外的比在党内的还好,党员比他们差。要把那些好的吸收进来,要把那些差的清洗出去,这样叫做经常改善自己组织的社会成分。我想这一条在边区党可以作,在延安完全可以实行。

第十一条:

“必须使得党建立起铁的无产阶级的纪律。”要制定这样一种纪律。这样一条,不但在别的条件上区别于社会民主党,而且在这一条上也区别于社会民主党,因为社会民主党他们不要这种纪律,不要这种铁的纪律。消极的自由主义,现在在我们党内存在着,在边区存在着,在延安存在着严重的自由主义,因为这种自由主义是破坏党的纪律的。自由主义的发展,这种自由主义发展了,就没有纪律,闹独立性,小广播,讲价钱,调工作不动,不服从决议案,讲了不做,见了坏分子不批评,见了不好的思想不作斗争。这种自由主义发展了,那无产阶级铁的纪律就不能创立起来了。斯大林是要我们创立铁的纪律,而这种自由主义的发展能不能使这种铁的纪律创立起来呢?是不能的。底下讲到,这样的纪律,建立这样铁的纪律,要有什么条件,斯大林严正说:“这种纪律是基于思想的一致性,运动之目的之明确性。”要整顿三风,要开高干会,是基于思想的统一性,运动目的之明确性。运动目标是要明确,我们边区的党还有不明确的,“要打出去。”但现在我们运动的目标是为了团结抗战,一部分人想打出去,就跟这个运动的目标相反。整顿三风也算是一个目标,整顿三风的目标有两个,一个是以无产阶级的思想去克服小资产阶级的思想,一个是要克服暗藏的反革命,运动目标要搞明确,如果不搞明确,如在整顿三风开始时有些同志不明确,认为这个运动的目标是整一部分人,而我是不在内,是整某一部人,不是整所有的人。后来把这个目标讲清楚了,整顿三风是整顿全党,拿边区来说,拿延安来说,我们党主要是干部,所以我们干部要参加这个整顿三风的运动。不认识字的要听讲,认识字的要学文件;这时整顿三风的目标明确了,因而现在有了大的发展,在延安有中央研究院、鲁艺、延大、自然科学院以及各部门,各机关,中央各部门,军委各系统,那些部门都统统要进行整顿三风,同时学习检查工作,每人有份,整顿每个人,那时目标就明确了,这样就可以建立起纪律来。有些人躲风,我有病了!哎呀!肚子痛!要门诊部给我看病,出去门诊,却不看病,而到山上去打扑克(笑声),这样就躲了风,没有整了,这样有没有纪律?这样就没有纪律。现在有不少人躲风,现在整风一来,他的歪风就躲了,于是就找门诊部,哎呀!肚子痛!各种办法都来了,这时要实行纪律。运动目标明确了,少数人的纪律就好执行。如果运动目标大家都不明确,那就没有法子整风。

精兵简政也算是一个运动,这个运动的目标,要明确它是实行在困难时期,我们的政府、军事机关工作人员太多,要精简。我们的干部有很多缺点,如官僚主义等。有这样情形,所以我们完了几项,要精兵简政,官僚主义要打倒,精简节约,统一效能,反对官僚主义,要简政,要精兵简政,人要减少,因此一定要统一思想,不要闹独立性,要使每个机关,每个工作人员能发生很大效能,不要马马虎虎,要节约,现在是困难,要反对官僚主义。这样一来,把精兵简政这个目标就明确了,这个运动的目标就明确了。全国是抗日民族统一战线,现在唯一的目的,一切的势力,就是为了打倒日本帝国主义。在全世界一切的目的,一切的势力就是为了打倒法西斯阵线。这问题,在前年参议会上讲过,这样的目标我们经常提出。参议会为了什么?是什么目标?没有别的,专门为了打倒日本,除此以外没有什么任何别的目标。现在我们当前的运动,这样明确的目标,可是虽然这样讲,又发了“七七”宣言,我们还有一部分同志,他们还不懂得,所以我们还要研究,这次会议还要讲的,要使运动的目标明确,实际行动统一,思想统一,运动的目标统一明确,实际行动要统一,不是你搞你的,我搞我的。什么地方看出思想统一与远动目标明确呢?要看实际行动,实际行动的统一,就证明你的思想统一。

运动的目标要看清楚,还有党内广大群众。对党的任务要自觉。这个担子我不愿担,勉勉强强的担负起,如果这样的党员多了,纪律就建立不起来。许多党员对于党所提出的任务,他还不懂得,完全是被动的,你要我担我就不担,没有自觉性。要使党内广大群众懂得党所提出的任务。党的任务是抗日的任务,团结的任务,我们反对自由主义的任务,要精兵简政,要整顿三风,大的小的这些任务,每个任务要党员去作的时候,要使党员清楚,要他有自觉性,要使它真正了解。现在我们边区有三万多党员,在做工作的时候,有很多党员,对他们的任务还不明确。比如讲征收救国公粮,过去曾经发生过许多毛病。由于党员对于这个任务不自觉,党员对:于征收救国公粮的这个任务不自觉。有那么一些党员,在征收救国公粮时发生了毛病,就不能执行纪律。

斯大林讲,纪律的基础是基于什么?斯大林说:“这种纪律是基于思想的一致性、运动的目的之明确性,实际行动的统一性及广大党员群众对党的任务之自觉态度而成长起来的。”

只有在这样的基础上成长起来的纪律。纪律是逐渐成长起来的,不是从天上掉下来的,也不是马克思发了一个命令,要我们搞纪律我们就搞纪律,不是这样的。因此这样纪律就搞不起来,是不是这样呢?不是的。这样的教条是不是可以建立?也不可以!马、恩、列、斯指导我们一个方向,纪律是建立在这样的基础上,那就好办了,纪律是创造出来的,不是现成的东西,我们有工作做,纪律就有了,思想的统一性,运动目标的明确性,实际行动的统一性,广大党员群众对党的任务要有自觉的态度。现在我们自觉的态度与作风很多党员是不够的,那么我们铁的纪律怎样?不够!那么布尔什维克化,化的怎样?化得不够。不要以为我们党一化就化得很好,如果化得很好,还有什么事做?没有化得好。我来讲这十二条,我向同志们说明。就是要同志们懂得,我们现在的毛病很多,不要自以为了不起,我们的党是布尔什维克的党,党要布尔什维克化,就要化得彻底,现在还没有那样地好。拿纪律一条来讲,好容易一个纪律,思想统一,运动目标明确,实际行动统一,党内广大党员群众对于任务要有自觉性,做到了这些,就有铁的纪律,做不到这些,就是像钢,某些地方像豆腐,某些地方像水一样,离铁就很远,那么铁的纪律就没有。

第十二条:“必须使得党有系统地检查自己的决定和指示之执行”,“不然,这些决定和指示就有变成空文的危险,这只能破坏广大的无产阶级群众对于党的信任。”

要检查决议案的执行,看怎样行的,一个是决,一个是行。决而不行就要检查,决而行,决而不行,决而行了一半,那有什么办法知道?就是检查,如果不然,那么这些决定和指示,就有变成空文的危险,这只能破坏广大的无产阶级的群众对觉的信任。

搞了一个决议案又不执行,搞了一个土地政策,我们就要执行,我们决定了要行,要在陕甘宁边区行,陕甘宁边区要按照具体情形,还要决一下,中央决一下,他们还要决一下,所以现在边区还要搞一个土地政策、减租减息交租交息决定,这很多决定,还是一个决,还是不行,什么时候行?就要在乡下行,现在有多少村多少乡?要在乡下村子里办到减租减息,交租交息,那叫做行。怎么晓得行,要检查!这是斯大林告诉我们的话。

所以这十二条,斯大林讲:“没有这些和类似这些的条件,布尔什维克化便是空想。”

如果没有这些和类似这些的条件,那么布尔什维克化是什么东西?没有这些条件,布尔什维克化,化得了化不了?化不了!那就困难,斯大林把检查工作放在最后一条,就是告诉我们要经常检查工作。

整顿三风以后,我们要有一次大的检查,这次大会也算一次检查,首先是作了历史的检查,再还要讨论许多的问题,你们下去实行去,做了将来还要检查。

这是十二条,比较明显扼要!这一共大概一千多字,在所有的文件里头。恐怕是最短的一个文件,有一千四、五百字。只有一千五百字,就把我们整个党一切重要的原则问题都提到了。这十二条,是全世界共产主义运动一百年的经验的总结,从一八四三年到明年就是一百年。全世界共产主义运动,无产阶级革命运动,一百年了。现在我们整顿三风,这些文件,不但总结了中国二十一年的经验。而且总结了全世界一百年的经验。这个文件是斯大林在一九二五年写的。一九三五年,一九四二年,十七、八年了,差不多二十年了,但是我们今天看,对于我们中国党是完全适合的。将来,有党存在的一天,都是适用的。斯大林写的文章又不是马克思写的文章,怎么叫一百年?没有一百年的经验,斯大林是写不出来?马克思就写不了这样具体。在马克思那时候,不能写得这样具体?能不能写出这十二条呢?很难?没有这些经验!《联共党史》结束语第三、四、五条,没有后来的经验,就很难写,就是第二条,他写得那样具体,那样明确,不但可以而且一定要以新的结论,这种思想,马克思早已有了。马克思他自己讲他的东西不是教条而是行动的指南。只有总结这一百年的经验,才能写出。那些个结论原则是被新的结论原则代替了,列宁的那一条去代替了马克思的那一条,以前能不能写?没有经验怎么能写出!这是告诉了我们以后怎样做,写得这样具体了!

关于第一条:要有革命的党,要有领导一切的革命的党。这一条相当于《联共党史》结束语的第一条。我们有“九一”决定,我们有党的统一领导,还有一个“增强党性的决定”,在这个决定中讲;“不要闹独立性”。这一次高干会,又整顿各组织间关系,叫做整顿关系,这是我们对于第一条这样的实行。

关于第二条:我们要精通马克思主义,精通要联系实际,不脱离实际的那种马克思主义。这就是相当于《联共党史》结束语的第二条。第二条就讲得很详细,在这里他只讲一两句话。季米特洛夫论干部教育政策,在这里,讲到教育政策时,告诉我们要做些什么东西。他说有两种教育方法;一种是教条式的,一种是真正马克思主义的。六中全会我们也讲过这个东西,六中全会讨论马克思主义中国化,有“中央关于干部学习的决定”,有“在职干部教育的决定”,这两个决定,都提到怎样学习马列主义,至于其他的就不讲了。这是如何精通马列主义的问题。

第三条:就是我们的口号、指示,要根据具体的条件分析,再照顾国际的经验。这我们有一个“调查研究的决定”,“中央关于调查研究的决定”。反对根据公式,那种公式主义,根据历史的类此来制订口号,政策。而要根据客观实际,具体的调查分析,来制订口号、政策。

关于第三条、第四条是讲方法的问题,制订口号,指示的方法论如何指示,如何检查。第三条制订,第四条就是证明。证明这个口号指示正确不正确?要在实际中间,在群众斗争的烈火中才能证明。这是第三条、第四条。

第五条:关于要有新的革命作风,使群众自然而然地一步一步地革命化。我们有“宣传指南”,就不要搞教条式的,采取革命的步骤使得群众自然而然地革命化。而不要那一种跟“宣传指南”相反的那一种宣传,那一种态度。我们说反对党八股,我们还有一个“四三”决定,在“四三”决定上有许多步骤,“四三”决定上讲的,在各机关、学校已经实行了,在中央研究院实行的有很大成绩;在开始,有的同志站的是反革命立场,而现在这些同志逐渐转过来了,站在我们这方面了;后来使反革命分子统统孤立,使过去不觉悟的分子也没受到损失;这一次搞好了,有很好的经验,在前两个星期,在《解放日报》上登了一篇文章,关于中央研究院总结六、七月的学习,那个东西很可以看,那个东西不是党八股,那是按照实际的过程写出来的。

第六条:讲群众工作,讲群众工作中的原则性同群众的落后怎样适应。群众落后,我们要讲原则,就是这个原则性跟广大群众不要脱离,即不要关门主义,讲原则不是关门主义,讲接近(联系)群众不是尾巴主义。这个中央以前有一个“关于群众工作的决定”,也就是《联共党史》结束语的第六条,即是最后一条,专门讲党与群众的关系。而在我们这里,边区,比如像在绥德、陇东那些区,怎样使我们又能坚持原则性,高度的原则性,但是又不要变成关门主义。要跟群众密切的接近,又不是跟着落后群众的尾巴走,而且把他提高一步。怎样实行?我们有经验。

第七条:不要冒险主义,又不要迁就行为。这是讲斗争形式和组织形式,是讲统一战线中的问题。这个有《共产主义运动中的“左派”幼稚病》一书,有季米特洛夫在七次大会上的报告,有中央关于国共合作的各种文件,从“八一”宣言起,在反摩擦中,在反“反共”高潮中,反摩擦斗争中,我们有许多斗争经验。我们的三三制,我们的土地关系,对于土地关系的处理,对于劳资关系的处理,我们的锄奸政策等等各种政策;我们的施政纲领,这些东西,都是跟斯大林讲的这一条,是一类的性质。不要冒险主义,但也不要迁就行为。冒险主义不等于高度的革命性,不可调和的革命性。而最大限度的灵活性,也绝不是尾巴主义。而我们研究一下我们的工作,自己的历史。关于国共合作的时期,关于国共合作以前,关于国共合作的中期,抗战的初期,关于最近二三年,关于我们现在实行的三三制,我们的土地政策,劳动政策,锄奸政策审查一下,在有些同志他所想象的宽大政策,在这个会中间,小组讨论中间,有许多同志不满意,我们应该杀的不杀,应该捉的不捉,土匪闹到那样程度,还要宽大政策,斯大林的十二条中是没有的。土匪天天打,打到瓦窑堡、延安附近,还要讲宽大政策?!他那种宽大政策是从那里来的?!人家打得你要死,群众在到处叫,你还在说施政纲领的宽大政策?!施政施纲的宽大政策那么好!这样一来,日本的飞机也来加油,你还要宽大,你说可以吗?就讲这样的政策?就是检查这样的问题。

第八条:要讲自我批评,要改正错误。就有列宁、斯大林论自我批评。二十个文件中有的,这里引列宁论自我批评,有《联共党史》结束语的第五条,有“四三”决定,“四三”决定中告诉我们怎么样做自我批评,有这样一些问题,有《反对自由主义》有《反对党内几种不良倾向》。这样一些文件,都是关于第八条的。我们所办的,我们所实行的这些文件,

过去的不讲,讲现在的,我们的整顿三风是属于这种性质的。我们的所谓正确的干部政策究竟是什么东西?这要搞清楚,胡里胡涂的团结,缺乏原则性的团结,团结得了团结不了呢?团结不了的。这样党内的宽大政策是不是马列主义的宽大政策?缺乏批评性的,缺乏斗争性的,这样的干部政策是不正确的。

第九条:必须使党善于把先进战士中的优秀分子选拔到基本的领导核心中去。季米特洛夫论干部教育政策中有四个干部标准,这四个标准也可以说是干部的四个条件(无限忠诚、联系群众、有独立工作能力、遵守纪律)。第一条要忠诚。刚才讲的斯大林论布尔什维克化第九条中讲的二条;一个是忠诚,一个是经验。这两条在季米特洛夫的四条标准中有,斯大林时是第一条,季米特洛夫的也是第一条,这个忠诚要经过考验,你自己讲那不行,要在法庭上、战斗中、艰苦工作中证明了的。第二条要跟群众联系,密切地联系,要有证据,你说你跟群众联系了,我说我跟群众联系了,那个不行。要群众自己觉得。你是他的领袖,要群众觉得你行。他不觉得你是他的领袖,那就不行,那不是季米特洛夫讲的要使群众自己觉得你是他的领袖。现在有那么一部分人,跟群众没有联系,而自称为领袖,站在领袖的职位上,自称为群众的领袖,而实际上,群众并不承认他是领袖。郭洪涛,朱理治,那时候在陕北,自以为也是领袖,而群众不承认他们是领袖。……群众、老百姓不证明、不了解你,他决不承认你是他的领袖的。所谓联系群众,什么标准呢?就是群众赞成你。所以有些同志出去到一个地方,我就和他讲,开头不要夸夸其谈大讲一顿,你不跟群众、干部联系,就没有办法知道群众与干部的情形。你讲的话一定会错,不是二定会错,(笑)这是一定的,你没有联系,没有工作,干部就不认识你,你怎么工作呢了!如果在这里不夸夸其谈,只有跟干部连系,眼睛多看一些,耳朵多听一些,然后再讲,那么,夸夸其谈一定会少了,人家就信了。不然人家戴上耳套,戴上眼镜来听,你讲的不大切合实际,人家相信不相信呢?有原则一条――不相信你,你讲得不像,谁吃了小米饭来听你那个“不像”。因此,这样的同志,站在领导地位就要很好的注意。

季米特洛夫讲的干部四个标准的第三条,是讲独立工作能力,斯大林的两个标准讲了。他讲的两个标准;一条是忠诚,是足以成为无产阶级的代表。一条就是要有经验,足以领导无产阶级的斗争。季米特洛夫讲的第三个标准就是讲的这个,要我们鉴别干部的时候,能够独立处理问题。第四条就是连系群众。

所以这个第九条,要形成领导核心,我们党要有领导核心,不要是平均主义,中央、地方、西北局以及各个警备区、我们的学校、机关,我们现在把这些发展起来,扩大起来,都要做领导的骨干,领导的核心,这些骨干分子有什么条件?斯大林讲有两个条件,因为他在这样短的文章内,不能把所有的条件列举出来,我们照季米特洛夫讲的有四个条件:领导骨干要十分忠诚,经过考验;同群众连系,使群众承认你同他们有连系;要有独立工作能力;要守纪律。所以关于第九条,我们可以参照季米特洛夫给干部的四个标准。这里,平均主义是不对的,对干部,党员平均主义的一种看法,没有分析,没有积极分子同坏分子的区别,这也是不对的。自由主义的态度不对,搞领导骨干就要进行教育,对不好的倾向,就要进行斗争。

第十条,就是讲“必须使觉经常地改善自己的组织的社会成分,清除那些腐化的机会主义分子,以便达到最大限度的团结一致。”这就是《联共党史》结束语的第四条,这个结束语的第四条与第十条相关,虽然内容不同,但大体相当《联共党史》结束语的第四条。……巩固党必须要清洗坏分子,经常地改善党的组织,那么党的组织应该是什么人呢?应该有个标准,怎么样做个共产党员?共产党员应该如何……(不清)中央有巩固党的决定,《联共党史》第四条讲了这个问题。对坏人要怎么样同他斗争,要把他清洗出去。

第十一条,就是讲纪律,要建立纪律,革命的党没有纪律不成。斯大林讲思想的一致性,运动目标的明确性,实际工作的统一性,党员的自觉性。这样就生长起来了纪律。列宁、斯大林论党的纪律与党的民主,我们选有这样的文章,是列在二十二个文件之内。

第十二条,最后一条是讲检查。斯大林论领导与检查,他那样的文章,在马克思那时候讲是没有可能的。在马克思那个时候没有可能讲民主与检查那样的文章。比如,什么是检查?在这次边区高干会议以及你们回去进行精兵简政、整党、整政、整军、整财、整民、整关系,是有系统的,而且是有威信的人,而不是普通的人。……现在延安各机关也要检查,延安有的机关现在已经检查了,整顿三风有部分的检查,跟着有大的检查,要那一个部门的主要领导者去检查,在“四三”决定上面讲了这个问题。

所以,关于这个布尔什维克化的十二条,我想很值得我们好好地研究一下,因为它的字很少,只有一千五百字。这是我们全党的圣经,这就是经,是圣经,而不是教条,是可以变化的,现在我们能不能使用呢?我想每一条都能使用。可是有些情况,像社会民主党,我们中国没有社会民主党。我们可以利用这一条,使用这一条。如果我们这样作下去,那么布尔什维克化怎么样呢?可不可以说我们会此现在更加布尔什维克化呢?要向那个方向去化他一步呢?我们在今年这一年,整顿三风、精兵简政,在边区高干会是要化一部分,而是相当大的一部分。这个“化”,我们是要自觉的,我们要很清楚,我来讲这个十二条,斯大林也讲了,而且是在多年前讲的,在一九二五年就是这样讲的。而这十二条,对我们恰好是适用的,我们要自觉地去做这工作。

我们对马列主义要学习,前天我所讲的,我们要学习,干部要有计划地读书,读三、四十本马列主义的书。特别是有实际工作经验的同志,不读则已,一读就读通了,就可以读进去(笑)。马列主义呀!我们是土包子,我们怎“能让得通呢?我们有什么资格读马列主义呢?你们是讥笑我吧!”而恰恰是土包子,就容易读得通,而有些人倒不容易读通。像外面来的,没有工作经验的干部,他们读了,也只是书面上的。而土包子就容易读通,一读就容易读懂。这次整顿三风的文件,就是一部经书,二十二个文件,再加上三个,一共二十五个文件,全部只有十二万九千字,不到十三万字,我们读八个月,这样读下去。在这里,分两种人,一种是有实际工作;经验的同志,一种是知识分子。有工作经验的同志,他们就容易读懂,没有工作经验的同志,就要钻牛角尖。比如;为什么叫“四三”决定?为什么不叫“四二”决定?(笑)又,什么是精神实质?又是精神又是实质,你讲我给听听。他们钻的不得了,搞不清楚。这些同志,要他们先读,然后再让他们到工作中去,然后过了一二年,再叫他读,那就有味道了。所以读马列主义,精通这本书是完全可以的,特别是有工作经验的同志,一读就有效力。朱理治、郭洪涛过去没有工作经验,他们过去大概读过几本书,他到这里就大摇大摆地为列宁主义而斗争(笑),写了文章也是一、二万字。他的列宁主义是从什么地方来的?是哪一个国家的?他的列宁主义一定不是苏联的。哪个列宁主义这次会证明了的,他的列宁是哪国的列宁呢?是他的列宁。这就等于没有读书,如真读几本书,也就没有那个。我们有工作经验的同志,读了,是不是像他们那样呢?一定不会的。我们读几本马、恩、列、斯的书,就够了。我们不要装腔作势,借以吓人,不会这样的。我们读三、四十本马、恩、列、斯的书,我们的工作,是会做得更加好。因为马、恩,列、斯都是很好的人,他们不是像朱理治那样的人,你们没有看见过,我也没有看见过,认识是在照片上看见过。那个是不相同的,样子也不相同,那个真是革命的圣人,他们是精通得很,他们的态度也是好得很,他们和我们一样,同我们很接近,不是和我们离得很远,他们很好,有实际工作经验的人。读了马、恩、列、斯的著作,就愈要同群众接近,如果不接近群众,那就是与群众离得很远,那就是愈读得多,就越是没有读进去。列宁就不是这样的。他们根本搞错了,那是反对了列宁,反对了马、恩、列、斯。

所以我们要有信心,我们党一定会搞得好,要布尔什维克化,一定会化的,我们已经有了进步,整顿三风,高干会,精兵简政,……这些同志们作下去,要有信心搞下去,一定会搞好,一定会更进一步的布尔什维克化。要我们全党,我们实际工作的同志能够这样去做,慢慢地来,一年搞一本,搞三、四十年就可以搞三、四十本。有些人是走马看花,一走就过去了。还有一种是下马看花,要仔仔细细地看(笑)。今天是要下马看花,要慢慢地看,我们读他三,四十本书,一年读一本要三,四十年,今年已经六、七十岁了,还能读几本书?我们要读三、四十本书,有的一星期可以读它一、二本,甚至几本,比如像党校,我们规定党校,计划八个月读三、四十本书,我们有一个大概,把三、四十本书下马看一下,将来做工作,一、二年以后再来翻一翻。如果说早把字典所有的字完全读了,是不可能,如果说把字典所有的字都读了,这样的人,在全天下都难找。连马克思在内,也是不可能的。我们要今天翻一下,明天翻一下,慢慢地来。我们的工农干部、有工作经验的干部,完全可以这样,达到这个目的。这样一来。我们可以更进一步。请同志们考虑一下,我讲的话完了。(全场热烈鼓掌)

