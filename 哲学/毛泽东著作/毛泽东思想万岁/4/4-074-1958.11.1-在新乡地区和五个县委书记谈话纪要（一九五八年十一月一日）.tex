\section[在新乡地区和五个县委书记谈话纪要(一九五八年十一月一日)]{在新乡地区和五个县委书记谈话纪要}
\datesubtitle{(一九五八年十一月一日)}


你们这里没有钢铁?(×××:有很多铁矿。)路东有什么铁?(×××:路东没有,都在路西。)有多少参加搞钢铁?(×××:120万人)包括炼钢铁炼炭、运输?(××:包括。)区有多少人口?(820万人。)八个里头一个还多。天气冷起来你们怎么办?你们搞起来多少铁?是铁渣?是铁水?(×××:全年任务××万吨,已炼铁××万吨。)不算铁吧,同志。(×××:有10%的好铁可以上调的。其余我们可以自己炼钢。)可以炼钢是不是商城又炒又打的办法?只炒不打可以吧?(×××:要打。)一个炉子要多少人?(×××:六、七人。)一半炒一半打?(×××:大致是。)你们炼多少钢?(××:全年任务××,已炼钢××。)那你们可以不要干事情,睡大觉。下雨下雪怎么办?你们打算没有?(×××:我们正在开会研究,打算十一月分下山一批。)(×××:釆取精兵的方法。)什么时候再上山?明年开春。你们准备下山多少?(××:六十万。)还有六十万住在什么地方?下雨下雪,要叫他们生活好,工作好。山上的人都是青壮年吧?(×××:也有很少年纪大一些的。)有妇女没有?(×××:有,妇女顶大事。)顶大事。(××:有这样个道理,妇女搞农业生产,没有男人熟练,不如男人。搞钢铁男女都不会,大家一块学习。(×××:有的妇女比男的学得还快。)啊!有这个道理。120万里头有多少妇女,(×××:40%,四、五十万人。)

七里营的棉花收多少?(×××:全年平均每亩皮棉200多斤。)我去看的那一块呢?

(×××:还要多些。)你们种多少麦子?去年下种多少?(×××:已经完成麦播计划,去年每亩地下种十斤左右,今年都在三十斤以上。)多一半还多。(×××:还有下种几百斤,一千斤的。)太多了,挤死出不来。(×××:分层种,像楼梯一样。)啊,麦子在楼梯上站着。(主席笑,大家都笑。)你们的地耕得深吗?(×××:一尺二寸左右。)没有七、八寸深的?

(×××:有。)五、六寸深的?(×××:很少。)要注意浇水、追肥、锄草、管理,要注意一下。

群众生活怎么样?食堂办的怎么样?(×××:以连为单位办食堂,一个食堂几百人。)要轮流吃饭。(×××:原来有打回家吃的,现在都集中吃。)打回家吃饭就冷了。吃多少粮食?(×××:可放开肚子吃,不限。)这是一件大事,吃饭还有菜,都有菜吃吗?(×××:有。)什么菜?(×××:红薯、萝卜。)(×××:七里营群众普遍可以吃上豆腐。×××:还有豆芽。)冬天吃什么菜?(×××:萝卜。)每人每月吃多少盐?能不能吃一斤?(×××:差不多。一个同志说:现在市称改为十两一斤了。)油呢?一天一个人五钱?你们明年要有计划种油料,你们有棉花,可以吃棉子油,每人每月十五两。(×××:可以。)不仅是饭,还要是菜,菜里还要有盐有油。猪肉一个礼拜吃一次行不行?(×××:行。)鸡蛋呢?(×××:猪肉、鸡蛋国家都收购,不能吃的太多。)

公社的工资普遍发没有?发的什么?(×××:人民币。)人民币从那里得来的?(×××:政府收购物资。有几个人民公社不能发工资?(××:都能发。)靠不住。不出经济作物的地方,它只产一点粮食,哪有钱发工资?(×××:受灾地方困难些,人民公社成立后,有灾区,有丰收区,可以互相调济。)奖励工资究竟好不好?每个月都要评,争吵,你多我少。(×××:我们是津贴。)你们是津贴怎么分配?(×××:大人小孩平均每人每年七十元。)津贴多少?(××:供给部分70%,津贴部分30%。)

钢铁基地上有医生没有?(×××:有。)这比打日本好,比打蒋介石好,打仗要死人,这也可能死一个两个的。

你们还有模范食堂?(×××:有。)你们把食堂排排队,分一、二、三类,号召二、三类食堂向一类看齐。(安阳县委书记:安阳挖出一个赵匡胤炼铁的炉子。)你有什么根据(看县志和听老年人说的。)安阳出曹操、袁世凯(坟在那里),没听说赵匡胤在那里炼铁。赵匡胤是洛阳生的人,他的父母是个小官吏,是五代梁、唐、晋、汉、周时的人,你说的可能是宋朝时候的炼铁炉。(大家都笑了。)

管理人员是件大事,一是管小人的,一是管吃饭的,过去都不愿干这种事,不愿当保姆,这个事情不管好怎么办?优越性就是比父母管的好,管的一样就没有优越性,管的比父母差一些,他还要拿回去,没优越性就不行,出个大字报,看你怎么办?大灶不比小灶好,怎么能行呢?现在大家一股劲,将来大家一算细账,说不好。(×××:能办好。)你们都有信心吗?(大家说:有信心。有个县委书记说,现在食堂办的比单吃的好。)有垮台的没有?(×××:没有,许多妇女决心大,把小锅都砸了。)这个革命可革的厉害。幸福院幸福不幸福?(×××:幸福。)有这样的事,自己不愿去,儿子媳妇硬叫他去,我看他不幸福。(×××:我们的幸福院,现在只收五保户。)他们做活不做?(×××:做。)是单独搞还是跟群众一起搞?(×××:有单独纺纱的。)他们吃饭怎样?(×××:小孩……吃中灶,青壮年吃大灶。)这个好,客人吃饭要不要饭票?(×××:不要,实行供给制。有一个老人来了客,队上给他做四个菜,还有酒,他高兴极了。)伙食是个大事,你们解决得很对。

睡觉问题解决的如何?睡觉要当任务,每人每天能不能睡八小时?七小时?六小时总可以。(×××:睡不到六小时。)要强迫命令,一定要睡六小时。紧张时也一定要睡六小时,睡五小时就没有完成任务。在工地上疲劳的时候,叫他们睡半小时再起来干,睡觉是一大任务。(×××:现在炼钢铁,秋收秋种,过去这一紧张阶段就可以了。)你们研究出几条睡觉的宪法,规定小人、青壮年、老人的睡觉时间,睡不好将来要受损失。(×××:七里营妇女摘棉花,上午两个小时可以摘六十斤,下午四个小时只摘六十斤,疲劳了劳动效率就不高。)不管你如何紧张,一定要睡六小时,少半个小时也不行。要强迫命令。这个强迫命令老百姓欢迎,主要是干部,小人要睡八个小时到十个小时。你们不是有营、连长吗?(×××:有。)叫营长、连长下个命令,躺下来睡,叫农民睡个午觉,你们研究一下,这是大事。一个吃饭,一个睡觉,一个管好小孩子。

种地用深耕细作的方法,达到少种多收的目的。亩产搞他一万斤,先搞两千斤,加一番再搞四、五千斤,再翻一番就是一万斤。地耕一尺二寸深,分层施肥,省水、省肥、省人力(×××:我们再搞卫星田。)搞大面积卫星田(×××:全省八十万亩小麦,卫星田一千六百万亩。)占百分之二十。二、三年后,公社把耕地面积缩小。深耕三、四尺,亩产一万斤,一个深耕细作,一个机械化。过去浅耕粗作,广种薄收,改为深耕细作,可以少种多收。我出题目你们研究一下,不要一下弄的没饭吃,可能这是一条出路,加上机械化不要搞得累的要死,好,咱们就谈到这里,谢谢大家。


