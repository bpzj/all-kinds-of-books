\section[在中央工作会议上的讲话(一九六四年十二月二十八日)]{在中央工作会议上的讲话}
\datesubtitle{(一九六四年十二月二十八日)}


没有多的讲。这个文件(指二十三条)行不行?

第一条,性质问题,这样规定可不可以?

有三种提法,前两种较好?还是第三种较好?\marginpar{\footnotesize 199}

常委谈过,又跟几位地方同志谈过,认为还是第三种提法较好。

因为,运动的名称即叫社会主义教育运动,不是四清教育运动,也不是什么矛盾交错的教育运动。

一九六二年北戴河会议,十中全会,搞了一个公报,就是讲要搞社会主义,不能搞资本主义。

六二年上半年,刮了“单干风”,还有“三自一包”、“三和一少”,刮得可凶啦!“单干风”,邓子恢是一位,另外还有几位。有些同志听进去了;还有的听了,不答腔,不回答问题。

搞社会主义,搞了许多年,而有些同志听了不答腔,不能回答问题。

四、五月间,没有一个地方同志讲形势好,只有军队同志说形势奸,我直接听到的是许世友、×××,间接听到的是杨得志、韩先楚。当时是五月,就是说形势不好,有那么一股空气。

到六月,我到济南,有几位同志说形势很好。时间隔了一个月,为什么变了?五月末割麦,六月割了。

为什么在北戴河我要讲形势?因为那时有人说,“不包产到户,要十年八年方能恢复”。搞社会主义还是搞资本主义?这是一种阶级斗争。所以,提出“有没有阶级、阶级矛盾和阶级斗争存在。”

因此,政治局常委觉得,大家讨论了也觉得,第三种提法较妥,较为适当,概括了问题的性质。

重点是整党内走资本主义道路的当权派。陈毅同志说,他也是当权派,只要你不走资本主义道路的,还当你的外交部长。

第十六条,工作态度,就是要讲点民主。

天天说要民主,就是不民主。叫别人讲民主,自己就不民主。

军队本来早就有三大民主。堡垒打不开,找士兵、战士、班长开会,大家议,怎么打?办法就出来的。这就是军事民主。

政治民主——三大纪律。

经济民主——伙食,要由战士管伙食。现在还管不管?不能单叫司务长管。连里有两个人。一个文书上士,一个司务长。文书上士叫师爷,搞抄写的,就是搞表报的,可了不起了!因为他认识几个字。

好话坏话都要听。好话,爱听,不成问题。问题是坏话。七千人大会上我讲过“老虎屁股摸不得,老子偏要摸”,后来认为那句话不那么文明,搞成另一种形式了。老子者,就是劳动人民,下级干部。我们这些人就是不大好摸的,你想揭他的疮疤,疮疤可不容易揭!

正确的话,错话都要听。正确的要听,错了也得听下去。人家批评你批评的错了,有什么问题呢?!你本是正确的,人家批评错了,责任在批评者,你听着,有什么问题呢?!你不听,那不好。正确的,批评得对的,要听。人家批评错了,那更好听了。还有一个,特别是那些反对你的话,要耐心听。做到这个,比较困难。

要让人家把话说完,这也是有点困难。他讲那么长,水分很多,米很少,是稀饭。我就受过很多次这样的灾难。有人讲了两个钟头,还没出题目,我问他要我帮他什么忙,他才出题目。×××,在延安,有一次找我谈话,讲了两点钟不着边际,我问他,你来找我谈,\marginpar{\footnotesize 200}要我帮你什么忙!他才出题目。另外,有的同志,他就是训话,我出题他不答题,我只好听训。这样的人,还不止一个。世界上有这种人,专门训人的,对我这种人,他就要训,长篇大论地训。

宣传和鼓动的区别。宣传者,许多概念连接起来。鼓动者,一个概念一个口号。比如罢工,提出一个口号,简单得很,就叫鼓动。写文章,做报告,长篇大论,叫宣传。贴标语,叫鼓动(专为一种事动员)。

×××发明这个道理,他说讲了两次,一次是五十分,一次是零分,人家不愿意听了嘛!我历来提倡,听讲话,不要鼓掌。不爱听的,可以打瞌睡。你讲的没有味道,他还不如打瞌睡,保养身体,还是保养身体免受这场灾难为好。还有一个办法,看小说。我从前在学校听课时就是这样。这就把教员整倒了。(讲了念书时的故事)这也许是我的毛病,也许是因为先生讲的不引人注意,我就看小说,后来发明了打瞌睡。不要说我这个人没有发明,我也有发明。(大家笑)用这种办法,我整了那些不是交谈式,而是训话式的人;整了那些老师,他只爱讲,不能让学生提问题,不是提问题启发学生。上课有了讲义,就不必讲了,让学生去看,再出点题目,让学生讨论。这次政府工作报告,我就主张不再念的,后来说有些人不识字,还是念好,我让了步,还是念了,也鼓掌了,这种会,鼓掌,我也赞成。

在同志们,不要使人怕。对敌人,要使他怕。在同志中,使人怕,那可不行!使人家怕,总是你有鬼,不然为什么使人怕你呢?凡是使人怕的,大概道理少一点。

过去军队里,班长带兵有三个办法:打人、骂人、关禁闭,别无办法。他不搞民主。后来我们说不许打人、骂人,禁闭现在也取消了。逃兵,逃了算了,何必捉!捉回枪毙,岂有此理!人家为什么逃?无非是在你这里过不下去了。跑了算了。如果你要把他捉回来,那就向他承认错误,请他吃顿饭,有猪肉吃,并且对他说,你还想逃,你就逃之,不想逃,就蹲下来。不能用打人、骂人、禁闭、捉逃兵的办法。逃兵让他逃,他的积极性已经不高,留他有何用!逃到外国,有什么了不起?!中国那么多人。无非是出去骂我们一顿。骂我们的人多得很。赫鲁晓夫、肯尼廸并不是中国人,他也骂。音乐家付聪,逃到英国去了,我说是好事。这种人留在国内又有何用?

我只讲这两点:一是性质问题,二是工作态度。
