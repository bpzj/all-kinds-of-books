\section[致蔡元培信(一九三六年九月二十六日)]{致蔡元培信}
\datesubtitle{(一九三六年九月二十六日)}


孑民先生左右:

五四运动时期,北大课堂,旧京集会,湘城讲座,敬聆先生之崇论宏议,不期忽忽二十年矣。今者何日?民族,国家存亡绝续之日,老者如先生一辈,中年者如泽东一辈,少者则今日之学生,不论贫富,不分工农商学,不别信仰尊简,将群入于异族侵略者之手,河山将非复我之河山,人民将非复我之人民,城郭将非复我之城郭,所谓亡国灭种者,旷古旷世,无与伦比,先生将何以处此也?共产党创议抗日统一战线,国人皆曰可行,知先生亦必曰可行,独于当权在世之衮衮诸公,或则曰不可行,或日当缓行,盗入门而不拒,虎噬人而不斗,率通国而入于麻木不仁,窒息待死之绝境,先生其何以处此也?孙中山先生联俄联共与工农政策,行之于一九二五年至一九二七年之第一次大革命而大效,国共两党合作此时期,亦即国民党最革命之时期,孙先生革命政策之毁弃,内战因之而连绵不绝,外患乃溃围决堤滔滔不可收拾矣。八月二十五日共产党致国民党书,虽旧策之重提,实救亡之至计,先生将何以处此也?读新文字意见书赫然列名于首位者先生也。二十年后忽见我敬爱之孑民先生发表了崭然不同于一般新旧顽固之簇新议论,先生当知见之而欢跃者绝不止我一人,绝不止共产党,必为无数量人也。从同志从朋友称述先生同情抗日救国事业,闻之而欢跃者绝不止我一人,绝不止共产党,必为全民族之诚实儿女,又毫无疑义也。然而百尺竿头更进一步,持此大义起而率先,以光复会、同盟会之民族伟人,北京大学,中央研究学院之学术领袖,当民族危亡之顷,作狂澜逆挽之谋,不但坐言,而且起行,不但同情,而且倡导,痛责南京当局,立即停止内战,放弃其对外退让,对内苛求之错误政策,撤废其爱国有罪,卖国有赏之亡国方针,发动全国海陆空军,实行真正之抗日作战,恢复孙中山先生革命的三民主义与三大政策精神,拯救四万万五千万同胞于水深火热之境,召集各党、各派、各界、各军之抗日救国代表大会,召集人民选举之全国国会,建立统一对外之国防政府,建立真正之民主共和国,致国家于富强隆盛之域,置民族于自由解放之林。若然,则先生者,必将照耀万古,留芳千代,买丝争绣,遍于通国之人,置邮而传,沸于全民之口矣。先生其将不舍救千里外曾聆敬益之人,稍稍减杀其欢跃之情而更增之,增之以至于无已乎!

宋庆龄先生、何香凝先生……(共具名宋、何等53人,略一一编者注。)

以及一切之党国故人,学术师友,社会朋旧,统此致讯。

寇深祸急,率尔进言,风雨同舟,愿闻

明教,敬颂

道安,不具

<p align="right">一九三六年九月二十二日</p>

