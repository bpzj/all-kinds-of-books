\section[在十五省市委书记会议上的讲话(一九五五年)]{在十五省市委书记会议上的讲话}
\datesubtitle{(一九五五年)}


民主人士下乡,提出粮食和镇反问题。

必须依靠贫农,团结中农,合作社必须互利才能自愿,没有互利就不能自愿。牲口不归公,假公道,投资平摊,贫农给。

停、缩、发。该停者停,该发者发,该缩者缩。

民主人士下乡考察,口径对一对,十几个省问题较多,他们提出来,我们抓住。

成绩、缺点,他们只讲成绩,不讲缺点。其实乱子不少,大体还好。看一下,好的、坏的、中等的,加以分析,清一清,多少没有乱子的。

反动势力猖獗,要讲一下。民主人士眼里没有反革命,借他们压一下有好处。

他们当民主人士也不好做,上不着天,下不着地。

合作社也乱子不少,大体还好。一日停,二日缩,三日发。缩,全缩半缩,少缩多缩,按实际情况。不干不行。

后解放区,一般就是发,基本是发。华北、东北也要有发。

该发即发,该停即停,该缩即缩。

自愿互利,互利换得自愿。取得中农入社,即有利于贫农。半,半有利,不能解释为中农吃亏。从马列都没有说过这种话。贫农给贷款,腰杆硬起来。

章程要完全不使中农吃亏。

反革命五年抓一百五十万,每年三十万,少杀,关改多。我主多抓,民主人士主大赦,不便反对。宪法有,可以讨论。赦好?抓好?我已大赦过,党团三反都赦了,战犯也赦了不少,也有赦了又作反革命。

手上还有一批,上校以上七百多,讲放不放,要张治中立状子,放出去能钓鱼。

大捉特捉是重点,大赦特赦不是重点,是人民的意见。七月开会时,大家都认为有大捉必要。

粮食问题,党内外有潮流,曰:大事不好。我说:大事还好,有点乱子。五年计划,要搞粮食、搞供销。

农民中缺粮户要粮,余粮户也要粮。从生产上解决。过剩粮不可买,空下来。

主要矛盾:个体与国家,社会主义的矛盾。

粮食工作五利:一利缺粮户,二利市民,三利灾民,四利城工,五利打台湾。

粮食工作无损自给户,稍损余粮户。

现在供应三亿人。

河南材料,缺粮户百分之一点五,该叫。

统购问题,有人说定产高了,是偏低,是偏高?偏高多少?高有无百分之二十?弄清楚才能答复党内外的人。会议上插话(广西省汇报时):

一万七千个社缩一、二百,小事。发展五千个社,少吧!

(湖南省汇报时):

怪不得湖南不叫。(指粮食工作)

发展四千个,这么少,不到一翻呀!

(江西省汇报时):

发展八千个,百分之三十四的乡。再发展一万四千个。

(湖北省汇报时):

周转粮不能没有,不放入计划,好呀!不买过头粮,不计周转粮。

(浙江省汇报时):

现在五万五千个,已退七千个,再退三千个,尚余四万五千,空白乡还要发展,其余各省可否翻一翻。

(江苏省汇报时):

落后乡百分之五有反革命,百分之十坏人,蜕化分子,这是反革命同盟军,县区乡三级干部要调查。不能让中农作主,中农可以教育,他要按党的政策办事也可以。

三万二千个不发展?!要比一比浙江。

(安徽省汇报时):

不发达的省,要利用发达省的经验。江苏赶上去,湘鄂两广要放手发展。合作社多数要大发展,百分之九十较好,百分之十退。

(山东省汇报时):

死人真省粮?湖南说死几十人,只死一人。现象、本质分开,一股台风天黑了,不得了。一查,是假的,合作社百分之九十是好的,是经多次风的。再讲几点:

一、证明百分之三至五,机动幅度再研究,不决定。八百八十亿,十一亿重点减。湖南、安徽、山东要加一点。交粮食会议,为粮食需要者设。

二、层层负责有好处,但会形成无数独立王国。二十多万个要统一、分级。统一计划下各省负责。

三、周转粮、细粮、粗粮照旧。不要在粮食上舞弊,供应太广。××缺是假缺。

四、合作社自愿互利,如果自愿,要拍扳。承认了的,照着办,保证百分之九十可靠。

五、习、法、公核有争执,党委统一,合法又敏觉,做好。饶、潘、杨案子,包庇三个反革命。用大叛徒胡京和。可以传达到十四级以上。

胡风案子另发通知,那个字号不清楚。


