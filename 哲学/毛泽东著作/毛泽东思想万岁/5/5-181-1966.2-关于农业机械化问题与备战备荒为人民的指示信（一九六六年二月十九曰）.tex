\section[关于农业机械化问题与备战备荒为人民的指示信(一)(一九六六年二月十九曰)]{关于农业机械化问题与备战备荒为人民的指示信}
\subsection{(一)(一九六六年二月十九)}


此件看了,觉得很好。请送××同志,请他酌定,是否可以发给各省、市、区党委研究。农业机械化的问题,各省、市、区应当在自力更生的基础上,做出一个五年、七年、十年的计划,从少数试点,逐步扩大,用二十五年时间,基本上实现农业机械化。至于二十五年以后,那是无止境的,那时提法也不同了,大概是:在过去二十五年的基础上再作一个二十五年的计划吧,目前是抓紧从今年起的十五年。已经过去十年了,这十年我们抓的不大好。

\kaoyouerziju{(给王××的信)}

\subsection{(二)(一九六六年三月十二曰)}

三月十一日信收到了。小计委派人去湖北,同湖北省委共同研究农业机械化五年、七年、十年的方案。并参观那里自力更生办机械化的试点,这个意见很好。建议各中央局,各省、市、区党委也各派人去湖北共同研究。有七天至十天的时间即可以了。回去后,各作一个五、七、十年计划的初步草案,酝酿几个月,然后在大约今年八、九月间召开的工作会议上才有可议。若事前无准备,那时议也怕议不好的。此事以各省、市、区自力更生为主,中央只能在原材料等方面,对原材料等等方面不足的地区有所帮助,但要由地方出钱购买。也要中央确有材料储备可以出售的条件,不能一哄而起,大家伸手。否则推迟时间,几年后再说。为此原材料(钢铁),工作母机,农业机械,应国家管理。地方制造,超过国家计划甚远者(例如超一倍以上者)或超过额内,准予留下三成至五成,让地方购买使用。此制不立,地方积极性调动不起来。为了农业机械化,多产农、林、牧、副、渔等品种,要为地方争一部分机械制造权。所谓一部分机械制造权,就是大超额分成权,小超额不在内。一切统一于中央,卡得死死的,不是好办法。又此事应与备战、备荒为人民联系起来,否则地方有条件也不会热心去做。第一是备战。人民和军队总得先有饭吃,先有衣穿才能打仗,否则虽有枪炮无所用之。第二是备荒。遇到荒年,地方无粮、棉、油等储备,依赖外省接济,总不是长远之计,一遇战争,困难更大。局部地区的灾荒,无论那一个省,常常是不可避免的,几个省合起来看,就更不可避免。第三是国家积累不可太多,要为一部分人至今口粮还不够吃,衣被甚少着想,再则要为全体人民分散储备以为备战备荒之用着想,三则更加要为地方积累资金用之于扩大再生产着想。所以农业机械化再同这几方面联系起来,才能动员群众,为较快地但是稳步地实现这种计划而奋斗。苏联的农业政策,历来就有错误,竭泽而渔,脱离群众,以至造成现在的困境,主要是长期陷在单纯再生产坑内,一遇荒年,连单纯再生产也保不住。我们也有过几年竭泽而渔(高征购)和很多地区荒年保不住单纯再生产的经验,总应引以为戒吧。现在虽只提出备战备荒为人民(这是最好地同时为国家的办法,\marginpar{\footnotesize 253}还是“百姓足,君孰与不足”的老话)的口号,究竟能否持久地认真地实行,我看还是一个问题。要待将来才能看得出是否能够解决。苏联的农业不是基本上机械化了吗?是何原因至今陷于困境呢?此事很值得想一想。

以上几点意见是否可行,请予酌定。又小计委何人去湖北,拟以余秋里,林××二同志为宜。如果让各中央局,各省市区党委也派人去的话,各省拟以管农业书记一人,计委一人去为宜,总共大约七十人左右去那里,开一个七天至十天的现场会,是否可行,并请斟酌。

\kaoyouerziju{(给刘××的信)}

