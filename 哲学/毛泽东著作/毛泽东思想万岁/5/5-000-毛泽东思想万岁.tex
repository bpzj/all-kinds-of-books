\documentclass[b5paper,oneside,12pt]{ctexbook}
\usepackage[hmargin=0.25in,vmargin=0.5in]{geometry} 
\usepackage[]{hyperref}

\pagestyle{plain} %整书页眉页脚设置
\ctexset{chapter/numbering=false}
\ctexset{
    section={numbering=false, afterskip = 0ex},
    subsection={format=\large\heiti\centering,numbering=false,beforeskip=1ex,afterskip = 1.75ex}
}
\newcommand\datesubtitle[1]{{\centering\large #1\par\vspace{1ex}}}  %自定义日期副标题格式,为了保险,最好使用两层大括号

% 下面是修改了脚注样式
% 一些LATEX内部命令含有@字符,如\@addtoreset,如果需要在文档中使用这些内部命令,就需要借助于另两个命令\makeatletter和\makeatother.
\makeatletter
% 无上标的 \@makefnmark
\def\nosuper@makefnmark{\hbox{\normalfont\@thefnmark\space}}
% 补丁
\usepackage{etoolbox}
\patchcmd\@makefntext{\@makefnmark}{\nosuper@makefnmark}{}{}
% 给脚注编号前后添加 〔〕
\renewcommand\thefootnote{{\hspace{-0.55em}〔\arabic{footnote}〕\hspace{-0.68em}}} 

% 引用样式
\newenvironment{yinyong}{%
    \begin{list}{}{\parsep\parskip
        \setlength\topsep{0pt}
        \setlength\itemindent{2em}%
        \setlength\parindent{2em}
        \setlength\listparindent{2em}
        \setlength{\leftmargin}{2em}
        \setlength{\rightmargin}{2em}
        \kaishu
    }
    \item[]
}{
  \end{list}
}

\title{毛泽东思想万岁5}
\author{毛泽东}
\date{}

\begin{document}

\frontmatter
\maketitle
\tableofcontents

\mainmatter
% \chapter{}
\section[在中央工作会议上的讲话(一九六一年一月十三日)]{在中央工作会议上的讲话}\datesubtitle{(一九六一年一月十三日)\footnote{据《毛泽东年谱》,1月13日,在中南海怀仁堂主持中央工作会议全体会议。}}

这次工作会议,据我看比过去几次都要好,大家头脑比过去清醒了些,冷热结合得好了一些。过去总是冷得不够,热得多了些,这次比过去有了进步,对问题有了分析,对情况比较摸底了。当然,有许多情况还是不摸底。中央和省市都有这种情况,比如对一、二、三类的县、社、队比较摸底,一类是好的,执行政策,不刮共产风。二类也比较好,三类是落后的,不好的,有的领导权被地、富、反、坏分子篡夺了,实际上是打着共产党的招牌,干国民党、地主阶级的事情,是国民党、地主阶级的复辟。全国县、社、队有百分之三十是好的,百分之五十是一般的,百分之二十是坏的。在一个具体地方,坏的有超过百分之二十的,有不到百分之二十的。但是究竟情况怎样,也不是完全清楚,也不完全准确,只能说大体上是这样。不要以为一、二类社、队都是好的,其中同样也有坏人,三类队中也有好人。××同志批了河南灵宝县的一个报告,指出了一、二类社里也有问题,群众发动以后,谁是好人,谁是坏人,群众是摸底的,公社是摸底的,就是我们不太摸底。总的看好的和较好的占百分之八十,还是好的多,群众知道好坏,就是领导不摸底。我们要有决心,这些地方没有强有力的领导,如果不派大批干部深入发动群众,找出贫农和下中农中的积极分子,采取两头压的办法,是不能解决问题的。

灵宝县一、二类社尚有许多问题,也还有坏人,何况三类社?现在我们虽然还不完全摸底,但已向这个方向进了一步,今后好好地进行调查研究,就可以更摸底。譬如粮食产量究竟有多少?现在比较摸底了,口粮搞低标准,瓜菜代,粮食过秤入库,比较摸了底。但也有地方不摸底,河北省还有百分之××的县、社、队不摸底。口粮标准有的不按省里规定吃,吃多了。

至于城市工业问题,比较接近实际。今年钢只定××××万吨,煤、木材、矿石、运输还得搞那么多。煤的指标要增加,不但冬季烧煤不够,而且发电用煤也不够。今年着重在搞质量、规格、品种。钢的产量已居世界第×位,数量不算少,目前是质量不够,所以今年不着重发展吨数。

省委书记、常委,包括地委第一书记,他们究竟摸不摸底?他们不摸底就成问题了。应该说现在比过去进一步,也在动了。要用试点方法去了解情况,调查问题。调查不需要很多,全国有通海口一个就行了,但现在也只有这么一个报告。三类社、队的问题,有信阳地区的整顿经验的报告,那么整三类社、队的问题就够了。还有河北保定的一个材料很有说服力,这个报告说什么时候刮共产风,如何纠正,如何整顿组织,如何改进领导,以及怎样实现大生产。现在河南出了好事,出了信阳文件,纪登奎的报告。希望大家回去后,把别的事放开,带一两个助手,调查一两个社、队,在城市也要彻底调查一两个工厂、城市人民公社。

省委第一书记只有那么一个人,怎么能又搞农村又搞城市呢?因此要有个助手,分头去调查,使自己心里有底。心中没底是不能行动的。过去打仗,心中有底,靠什么?解放战争初期,中央直接指挥的经验少,但有两个办法;\marginpar{1}一靠陕北打胡宗南的经验,到四七年四、五月间,就靠各地区前方的报告,这是阳的,还靠阴的,即各方面的情报,所以情况很清楚。现在这些情报没有了,死官僚又封锁了消息,中央就得不到更多的消息。

我们下去搞调查研究,检查工作,要用眼睛去看,用耳朵去听,用手去摸,用嘴去讲,要开座谈会。看粮食是否增了产?够不够吃?要察颜观色,看看是否面有菜色,骨瘦如柴。这是眼睛可以看得出来的。保定的办法是请老农、干部开座谈会,与总支书、支书谈,群众也发言议论,这些意见是有钱买不到的东西。

这些年来,这种调查研究工作不大作了。我们的同志不作调查研究工作,没有基础,没有底,凭感想和估计办事。劝同志们要大兴调查研究之风,一切从实际出发,没有把握就不要乱发言,不要下决心。作调查研究也并不那么困难,人不要那样多,时间也不要那么长,在农村有一两个社队,在城市有一两个工厂,一两个学校,一两个商店,合起来有七、八个,十来个,也就行了。也不必都自己亲自去搞,自己搞一两个,其他就组织班子去搞,亲自加以领导。保定的报告是农村工作部搞的,是个大功劳;通海口是省委抽人下去的,灵宝县的报告是纪登奎同志下去搞的,信阳的报告是搞造后的地委下去搞的。

调查研究这件事极为重要,要教会许多人。所有省委书记、常委、各部门负责人、地委、县委、公社党委,都要进行调查研究,不做,情况就不清楚。公社有多少部门,第一书记不一定知道,一个公社,有三十多个队,公社党委只要摸透好、中、坏三个队就行。做工作要有三条:一是情况明,二是决心大,三是方法对。这里情况明是第一条,这是一切的基础。情况不明,一切都无从谈起,这就要搞调查研究。资产阶级是讲调查研究的。美国发言人总是说胡志明的军队进入老挝,但究竟进去什么兵,什么官,什么兵种,他们不说。资产阶级比我们老实,不知道就不讲。我们有时没有底,哇里哇啦一套。但是资产阶级也有冒失鬼,资本主义国家有个杂志说:从五一年到六〇年,就把苏联和一切社会主义国家都消灭掉。

这次会议,情况逐渐明朗,决心逐步大。但是决心还是参差不齐的。有的同志讲刮共产风要破产还债,听起来不好听,但实际上是要破产还债。县、区、社两级通通破掉就好了,破掉以后再来真正的白手起家。……我们是马克思列宁主义者,不能剥夺劳动者,只能剥夺剥夺者,这条是马克思列宁主义的基本原则。……资产阶级、地主阶级剥夺劳动人民,马克思列宁主义者不能剥夺劳动人民。资产阶级、地主阶级的方法比我们还高明,他们是逐步使劳动者破产欠债,我们是一下子平掉,用这种办法建立社有经济、国营经济。我们的国营经济赚钱太多,到农村中去收购,常常压级压价,剥夺农民,交换非常不等价,这就使工人阶级脱离他们的同盟者。这个道理,同志们也懂得,话也好讲,但实行起来决心不大,不那么容易。是不是所有的省委书记都有那么大的决心破产还债,还得看看。这也是不平衡的,各省也会是参差不齐的。可能有的省决心大,彻底一些,把群众团结在自己的周围。有些省决心不大,作的差一些。一省之内,几十个县也会是不平衡的,因为领导人的情况不同。一类县、社、队有百分之三十共产风刮了一下,停的早,五九年郑州会议后就停下来了,他们懂得不能剥夺农民,不能黑手起家,决心大,退赔的彻底,以后就不再刮了。有些搞变得不彻底,一次再一次刮共产风。去年春季,中央情况不明,以为共产风不很严重,所以搞得不彻底。其实去年春季就应该开这样的会,纠正共产风,可是没有开。我们对情况不够明,问题不集中,决心不大,方法也不大那样对头,不是像现在信阳、通海口、保定、灵宝的方法。所以这件事是个大事情,这是一场大斗争,要在实践与斗争中认识问题,解决问题。农忙过后还要再搞,一、二类社、队也还不少,还要抓紧搞,下决心搞彻底。\marginpar{2}总而言之,过去抗战时期、解放战争时期,调查研究比较认真,实事求是,从实际出发,情况明了,决心就大,方法就对头,解决问题的措施也较有利。只有正确的方针政策,但情况不明,决心不大,方法不对,还是等于零。郑州会议讲不能一平二调,方针是对的,说不算账、不退赔,这点不对。上海会议十八条讲了要退赔,紧接着我批了浙江、麻城的经验报告。五九年三、四月,我批了两万多字的东西。现在看来,光打笔墨官司,不那么顶用。他封锁你,你情况不明,有什么办法?那时省委、地委的同志也不那么认识共产风的危害性。有的同志讲郑州会议是压服,不是说服,思想还有距离,所以决心不大,搞的不够彻底。

工业开始摸了一些底,还要继续摸底。要缩短工业战线,重工业战线,特别是基本建设战线。要延长农业战线,轻工业要发展。重工业除煤炭、矿山、木材、运输之外,不搞新的基本建设,过去搞了的,有些还要搞,但有些也不搞,癞了头就让它癞头去吧。

长远计划现在搞不出来,我们要再搞十年,从六〇年到六九年,这是个革命。中国的封建主义搞了那么多年,民主革命也搞了那么多年,没有民主革命的胜利,就没有社会主义。搞社会主义建设不能那么急,十分急搞不成,要波浪式前进。陈伯达同志提出,社会主义建设是否也有个周期率,若干年发展较快,有几年较低,如同行军一样,有大休息、中休息、小休息,要劳逸结合,两个战役间要休整。这次工作会议也有劳逸,决议文件也不多,譬如郑州会议就只搞了那么一个决议嘛!还是看情况明不明,决心大不大,方法对不对头。

现在看一个材料说:西德钢产量去年是三千四百万吨,英国二千四百万吨,西德六〇年比五九年增加百分之十五,法国是一千七百万吨,日本是二千二百万吨。但他们的生产率是长期积累的,搞了那么多年,才那么多,我们才几年,就××××万吨。今、明、后年,搞几年慢腾腾,搞扎实一些,然后再上去。指标不要那么高,把质量搞上去,让帝国主义说我们大跃进垮台了,这样对我们比较有利,不要务虚名而得实祸。要提高质量、规格、品种,提高管理水平,提高劳动生产率。现在我们劳动生产率很低。五七年我们职工有二千四百多万人,现在有五千多万人,还要下放。不然,五六个人围着一台机器,一个人做,几个人看,这不行。解决这个问题也是要情况明,决心大,方法对。

陈伯达同志有个材料,美国一个农民劳动力养活三十个人,英国二十六个人,苏联六个人,我们只有三个半人。有人说我们也可以养四个人,那就看你怎样养了,如果一天只吃几两米,那不行。

国际形势我看也是很好的。原来我们讲要硬着头皮顶,准备顶它十年。从前年西藏闹事到现在,不过二十多个月,现在反华的空气大为稀薄了,但空气还是有,有时还有寒流。莫斯科会议以后,空气还好一些。

今年搞一个实事求是年。实事求是是汉朝的班固在汉书上说的,一直流传到现在。我党有实事求是的传统,但最近几年来不大了解情况,大概是官做大了,摸不了底了。今年要摸它一个工厂、一个学校、一个商店、一个连队、一个城市人民公社,不搞典型就不好工作。这次会议以后,我就下去搞调查研究工作。总而言之,现在摸到这个方向,大家都要进行。不要只讲人家的坏话,有的地方工作有错误,人家搞了,就要欢迎人家。\marginpar{3}


\section[在八届九中全会上的讲话(一九六一年一月十八日)]{在八届九中全会上的讲话}
\datesubtitle{(一九六一年一月十八日)\footnote{据《毛泽东年谱》,1月18日 在中南海怀仁堂主持中共八届九中全会全体会议}}

这次会议,因为经过二十天的工作会议的准备,开得比较顺利。今天想讲一讲工作会议上讲过的调查研究问题,别的问题也讲一下。

我们在反帝反封建的民主革命时期,提倡调查研究,那时全党调查研究工作作风比较好,解放后十一年来就较差了。什么原因?要进行分析。在民主革命时期,犯过几次路线错误,在解放后又出过高岗路线。右的不搞调查研究,“左”的也不搞调查研究。那时,中国是什么情况,应采取什么战略方针和策略方针才适合中国实际情况,长时期没有得到解决。自从我们党一九二一年成立起,到一九三五年遵义会议,十四年间,有正确的时候,也有错误的时候。大革命遭到了损失,第二次国内革命战争也遭到了损失,长征损失也很大,在遵义会议后到了延安,我们党经过了整风,七大时,……王明路线基本上克服了。抗战八年我们积蓄了力量,因此在一九四九年,我们取得了全国革命的胜利,夺取了政权。解放战争时期,我们和蒋介石作战,情况比较清楚,比较注意搞调查研究,对革命一套比较熟悉,那时情况也比较单纯。胜利后有了全国政权,几亿人口情况比较复杂了。我们过去有过几次错误,陈独秀机会主义错误,立三路线的错误,王明路线的错误等等,有了几个比较,几个反复,容易教育全党。近几年来,我们也进行了一些调查研究,但比较少,情况不甚了解。譬如农村中,地主阶级复辟问题,不是我们有意识给他挂上这笔账,而是事实是这样,他们打着共产党的旗子,实际上搞地主阶级复辟。在出了乱子以后,我们才逐步认识在农村中的阶级斗争是地主阶级复辟。凡是三类社、队,大体都是与反革命有关系,这里边也有死官僚。死官僚实际上是帮助了反革命,帮助了敌人,是地、富、反、坏、蜕化变质分子的同盟军。因为死官僚不关心人民生活,不管主观愿望如何,实际上帮助了敌人,是反革命的同盟军。还有一部分是糊涂人,不懂得什么叫三级所有,队为基础,不懂得共产风刮不得。反革命、坏分子、蜕化变质分子就利用死官僚、糊涂人把坏事做尽。一九五九年,有一个省,本来只有××××亿斤粮食,硬说有××××亿斤,估得高,报得高。出现了四高,就是高指标、高估产、高征购、高用粮,直到去年北戴河工作会议,才把情况摸清楚。现在事情又走到了反面,是搞低标准,瓜菜代。经过调查研究,从不实际走到比较合乎实际。

农、轻、重,工农业并举,两条腿走路,我讲了五年,庐山会议也讲了,但去年没有实行。看来今年可能实行,我只说可能实行,因为现在还没有兑现。一九六一年国民经济计划已经反映了这一点,注意了农、轻、重,就可能变成现实。

对地主的复辟,我们也缺少调查研究。我们进城了,对城市反革命分子比较注意,比较有底。一九五六年匈牙利事件以后,我们让他们分散的大鸣大放,出了几万个小匈牙利。这样把情况弄清楚了,就进行了反右斗争。整出了×××万个右派,搞得比较好,底摸清了,决心就转大了。农村那年也整了一下,没有料到地主阶级复辟问题。当然,抽象的讲是料到的。\marginpar{4}过去我们总是提出国内矛盾是资产阶级和无产阶级的矛盾,是社会主义与资本主义两条道路的矛盾,基本矛盾是阶级矛盾。是资本主义的天下,还是社会主义的天下,是地主的天下,还是人民的天下?

没有调查研究,情况不明,决心就不大。一九五九年就反对刮共产风,由于情况不明,决心就不大,中间又加上了一个庐山会议,反右倾机会主义。本来庐山会议要纠正“左”倾错误,总结工作,可是被右倾机会主义进攻打断了,反右是非反不可的。会后,共产风又刮起来了,急于过渡,搞了几个大办。大办社有经济、大办水利、大办养猪、大办县社企业、大办土铁路。同时要这么些个大办,如养猪什么也不给,这就刮起共产风来了。当然大办水利、大办工业取得了很多的成绩,不可抹煞。还有大办文化、大办教育、大办卫生等等,不考虑能不能做。共产风问题,反革命复辟问题,死官僚问题,糊涂人问题,干部情况问题,县、社、队分为一、二、三类,各占百分之××、百分之××、百分之××问题,这些问题以前我们就没有搞清楚,有的摸了,我们也没有讲清楚,或讲清楚了也不灵。郑州会议反共产风,只灵了六个月,庐山会议后冬天又刮起共产风。庐山会议前,“左”的情况还没有搞清楚,党内又从右边刮来一股风。彭德怀等人与国际修正主义分子、国内右派相呼应,打乱了我们纠“左”步骤。

去年一年国际情况比较清楚,对国内问题也应该聚精会神调查研究,工人阶级要团结农民大多数,首先是贫农、下中农和较好的中农,依靠他们对付地主反革命。三类社、队要成立贫下中农委员会,在党的领导下,主持整风整社,并临时代行社、队管理委员会的职权。我们党内也有代表地主,资产阶级,小资产阶级的人,应该纯洁党的组织,经过整风、整顿组织,使党纯洁起来,使绝大多数党员都代表贫农、下中农的利益,同时也不损害富裕中农的利益,坚持不剥夺农民利益的马克思列宁主义原则。刮共产风是非常错误的,是剥夺农民的,是反马克思列宁主义的,必须坚决退赔。经验证明,只要退赔,群众就满意了,情况就改善了。

这次工业计划比较切合实际,缩短了基本建设战线,延长了农业、轻工业战线。与农业有关的基本建设还要搞,有的重工业,像煤、木材、矿石、铁路还要搞。上下一本找账,不搞两本账。不要层层加码。总之,要实事求是,使一切从实际出发。粮食要过秤入库,不搞四高,搞低标准,瓜菜代,坚决退赔,整顿五风,不准不赔,不准不退。

城市也要整风,正在搞试点,还要一、二个月才能搞出来,也要搞十二条。

今年计划看来比去年高不了多少。有人建议钢仍然搞×××××××万吨到××××万吨,也增加不了多少;这个提法有道理。第二个五年计划钢的指标,早己超额完成,还剩两年,就是要搞质量、规格、品种,在质量上好好跃进一下,数量上不准备多搞。帝国主义者、修正主义者,会说大跃进垮台了,他们要讲就让他们讲,他们讲坏话也好,讲我们好反而不好。实际上我们现在就是要搞质量、规格、品种,搞企业管理制度、技术措施,提高劳动生产率,降低成本,成龙配套,要搞调整、充实、提高,就是要在这方面努力。英国、日本的钢,暂时还比我们的多,再有×年,我们总会赶上他们,并且还会超过他们。能否超过西德,还要看一看。讲打仗,斗地主,我们有一套经验,搞建设还比较缺乏经验,我与斯诺谈话就谈到这一点。凡是规律总要经过几次反复才能找到,我们只希望不要像民主革命花二十八年才成功。其实二十八年也不算很长,许多国家的党同我们同年产生,现在也还没有成功。搞建设是不是可以二十年取得验验,我们搞了十一年,看再有九年行不行。曾想缩短很多,看来不行。凡是没有被认识的东西,你就没法改造它。\marginpar{5}

工业还是要鼓干劲,不然几次会议一开,劲就没有了。泄了二、三个月的气,然后再开一次鼓干劲的会,反右倾。大家回去以后,要实实在在的干,不要老算账。搞计划要好好调查研究,搞清情况,鼓足干劲,力争上游,多快好省,坚持总路线。有人说现在不用多快了,这不对,搞粮食就要多快嘛,搞工业讲质量、成龙配套等等,也是要搞多快嘛!

团结问题。中央委员会的团结是全党团结的核心。庐山会议有少数人闹不团结,我们希望和他们团结,不管他们的错误有多大,只要他们能改。他们讲你们也有错误,不错,错误人人皆有,但错误大小轻重不同,性质不同,数量质量不同。不要一犯错误就抬不起头来。有的同志工作职位降低了,降低了也好。一年来有进步,不管真假,总是值得欢迎。地方工作的同志有的也犯了错误,欢迎他们改正。

河南、甘肃、山东三省问题比较严重,情况不明,决心不大,方法不对。现在情况明了就好了。有些地方政权也夺回来了,面貌已经开始一新。甘肃也开始好转,其他各省也总要烂掉若干县、社、队,大体是百分之××左右,严重的超过百分之××,好的不到百分之××。不光是因为粮食问题。林彪同志讲,军队有××个单位,烂掉了××个,占百分之×,这并不是因为粮食问题。这种情况在城市、工厂、学校一定会有。对××类干部,要按政策清洗出去,死官僚要改造,变成活官僚,长久活不起来的也要清洗。这些人是少数,合起来也不过百分之几,百分之九十以上是好人,其中也有糊涂人。我们也糊涂过,不然在民主革命时期,大革命为什么失败,南方根据地丧失,白区力量丧失,要长征,是因为不了解情况。

现在搞社会主义建设,是个新问题,我们缺少经验,要开训练班,把县、社、队干部轮训一遍,使他们懂得政策。如果一个省只有一个县委书记能讲清政策,不训练干部怎么行?每个县都要有一个县委书记真正能懂政策,弄清政策,那就好了。现在中央下放了八千多个干部帮助农村整风整社。大多数农村干部是好的,可靠的。如果大多数是国民党,我们还能在这里安心开会吗?所有一切可团结的人要团结。就是对反革命分子也不能都杀,不杀不足以平民愤的才杀,有的要关起来,管起来。杀人要谨慎,切不可重复过去所犯过的错误,如过去搞根据地时杀人多了一些。延安时规定一条,干部一个不杀。现在还关了一个潘汉年,绝对不杀。杀了就要比,这个杀了,那个杀不杀?总是不开杀戒。但是不是说社会上一个不杀?有些不杀不足以平民愤的人,民愤很大的人,不能不杀几个。至于中央委员犯了错误就不牵涉杀不杀的问题,还是留在中央委员会工作。要与各兄弟党团结,要和苏联的党团结,要和八十一个共产党和工人党团结,我们要采取团结的方针。

过去我们吃了亏,就是不注意调查研究,只讲普遍真理。六一年要成为调查研究年,在实践中调查研究,专门进行调查研究。

\section[《反对本本主义》说明(一九六一年三月十一日)]{《反对本本主义》说明\\{\large(原名:“关于调查研究”)}}
\datesubtitle{(一九六一年三月十一日)\footnote{据《毛泽东年谱》,3月10日—13日 在广州小岛招待所主持召开三南会议,主要讨论人民公社体制和工作条例问题。同时,刘少奇、周恩来、陈云、邓小平于三月十一日至十三日在北京主持召开有中共中央西北局、东北局、华北局及所属各省、市、自治区负责人参加的工作会议(即三北会议),讨论的问题与三南会议相同。后来,北京方面向毛泽东建议,两边合起来开会,得到毛泽东同意。十四日,参加三北会议的同志到达广州。3月11日 同胡乔木、田家英谈《调查工作》一文修改问题。同日 为印发《调查工作》一文给三南会议写如下批语:“这是一篇老文章,是为了反对当时红军中的教条主义思想而写的。那时没有用‘教条主义’这个名称,我们叫它做‘本本主义’。写作时间大约在一九三〇年春季,已经三十年不见了。一九六一年一月,忽然从中央革命博物馆里找到,而中央革命博物馆是从福建龙岩地委找到的。看来还有些用处,印若干份供同志们参考。”并批注:“送林彪同志阅,一九三〇年的,从闽西找出来的。阅后退毛。”毛泽东在印发这篇文章时,对正文作了一些文字修改,将标题改为《关于调查工作》。}}

这是一篇老文章,是为了反对当时红军中的教条主义思想而写的。那时没有用“教条主义”这个名称,我们叫它做“本本主义”。\marginpar{6}写作时间大约在一九三〇年春季,已经三十年不见了。一九六一年一月,忽然从中央革命博物馆里找到。而中央革命博物馆是从福建龙岩地找到的。看来还有些用处。印若干分供同志们参考。

\kaitiqianming{毛泽东}
\kaoyouriqi{一九六一年三月十一日}


\section[《中央关于认真进行调查研究工作问题给各中央局、省、市、自治区党委一封信》(摘录)(一九六一年三月二十七日)]{《中央关于认真进行调查研究工作问题给各中央局、省、市、自治区党委一封信》(摘录)}
\datesubtitle{(一九六一年三月二十七日)}

毛主席提倡哲学要走出课堂,走出书斋。毛主席讲:真理在谁手里,我们就跟谁走,挑大粪的人有真理,我们就跟挑大粪的人走。

\section[在广州会议上的讲话(节录)(一九六一年三月)]{在广州会议上的讲话(节录)}
\datesubtitle{(一九六一年三月)}

主要是两个平均主义的问题。一个是生产大队(即原来的生产队)内部,各生产队(即原来的生产小队)与生产队之间的平均主义,一个是生产队内部,社员与社员之间的平均主义。这两个问题很大,不彻底解决,不可能真正地全部地调动群众的积极性。

人民公社是生产大队的联合组织。

公社、生产大队不能瞎指挥,县、地、省、中央也不能瞎指挥。

不能用领导工业的办法来领导农业,也不能用领导农业的方法来领导工业。

\datesubtitle{(三月二十三日)}

几年来出的问题,大体上都是因为胸中无数,情况不明,政策就不对,决心就不大,方法也就不对头。最近几年吃情况的亏很大,付出的代价很大。

要作系统的由历史到现状的调查研究,省、地、县、社的第一书记,都要亲自动手。不做好调查工作,一切工作都无法做好。第一书记亲身调查很重要,足以影响全局。今后我们必须摆脱一部分事务工作,交别人去做。报告也要看,但是不要满足于看报告。最重要的是亲自作典型调查,走马观花只能是辅助的方法。

只有原理原则,没有具体政策不能解决问题。没有调查研究,就不能产生正确的具体政策。

调查研究的态度,不可以先入为主,不可以自以为是,不可以老爷式的,决不可以当钦差大臣。而要讨论式的,同志式的,商(量)的。

不要怕听不同的意见,原来的判断和决定,经过实际检验,是不对的,也不要怕推翻。



\section[在接见亚洲、非洲外宾时的谈话(一九六一年四月二十八日)]{在接见亚洲、非洲外宾时的谈话}
\datesubtitle{(一九六一年四月二十八日)}

毛主席对非洲、阿拉伯各国人民反对帝国主义的斗争表示深切的同情和支持。毛主席指出当前的国际形势对非洲、亚洲、拉丁美洲人民的反帝斗争是非常有利的。毛主席说,对帝国主义进行斗争当中,采取正确的路线,依靠工人、农民,团结广大的革命知识分子,小资产阶级和反对帝国主义的民族资产阶级以及一切爱国反帝力量,紧紧地联系群众,就有可能取得胜利。毛主席指出,革命政党和力量,在开始时都是处于少数地位的,但最有前途的就是他们。

毛主席严厉地谴责美帝国主义对古巴的侵略,并指出,美国帝国主义迫不及待地进攻古巴,再一次在全世界面前揭露了它的真面目,说明了肯尼迪政府只能比艾森豪威尔政府更坏些,而不是更好些。美国帝国主义利用联合国作为工具侵略刚果和杀害芦蒙巴的罪行,将非洲人民对美国帝国主义的认识进一步提高了。

毛主席表示,最近万隆亚非团结会议上表达的亚非人民同拉丁美洲人民加强团结的愿望,是好的。对世界人民反对帝国主义斗争的共同事业是有益的。

毛主席说,中国人民把亚洲、非洲、拉丁美洲人民的反帝国主义斗争的胜利看作是自己的胜利,并对他们的一切反帝国主义;反殖民主义的斗争给以热烈的同情和支持。

\kaoyouriqi{(《人民日报》一九六一年四月二十九日)}



\section{在接见古巴文化代表团时的谈话(一九六一年四月十九日)}


中国和古巴是两个友好的国家,我们相互帮助,相互支持。我们的目标是一个。反对帝国主义。美国帝国主义是帝国主义中最大的一个,它不但压迫我们,也压迫你们,它压迫全世界人民。在一些不是帝国主义的国家中,都有一些帝国主义的走狗。我们不但要反对帝国主义,也要反对帝国主义的走狗。

毛主席最后在送别古巴朋友的时候,祝古巴人民在反对美国帝国主义侵略的斗争中取得胜利,致以亲切的问候。并祝古巴人民庄反对美国帝国主义侵略的斗争中取得胜利。



\section[调查成灾一例(一九六一年五月三十日)]{调查成灾一例\\{\large——对“关于‘调查研究’的调查”的批示}}
\datesubtitle{(一九六一年五月三十日)}

如果还是如同去长辛店铁道机车车辆制造厂做调查的那些人们实行官僚主义的老爷式的使人厌恶得透顶的那种调查方法,党委有权教育他们。死官僚不听话的,党委有权把他们轰走。\marginpar{8}\footnote{5月28日 阅田家英报送的戚本禹五月十二日写的材料《关于“调查研究”的调查》和田家英报送这个材料的信。田家英的信中说:秘书室工作人员戚本禹,去年六月下放到长辛店机车车辆工厂劳动。最近他寄了一份材料给我,反映一些机关、学校人员到工厂作调查的情况。这个材料提出了一些在大兴调查研究之风中间值得注意的问题。戚本禹的材料说,他们利用业余时间摸了一下各级领导机关到长辛店机车车辆厂做调查研究工作的情况,认为在二十几个调查组的工作里,比较普遍地存在着“十多十少”的问题。毛泽东为戚本禹的材料拟了一个题目《调查成灾的一例》。批示:“此件印发工作会议各同志。同时印发中央及国家机关各部门各党组。派调查组下去,无论城乡,无论人多人少,都应先有训练,讲明政策、态度和方法,不使调查达不到目的,引起基层同志反感,使调查这样一件好事,反而成了灾难。”30日,毛泽东对这个材料再次批示:“此件,请中央及国家机关各部门各党组,各中央局,各省、市、区党委,一直发到县、社两级党委,城市工厂、矿山、交通运输基层党委,财贸基层党委,文教基层党委,军队团级党委,予以讨论,引起他们注意,帮助下去调查的人们,增强十少,避免十多。如果还是如同下去长辛店铁道机车车辆制造工厂做调查的那些人们,实行官僚主义的老爷式的使人厌恶得透顶的那种调查法,党委有权教育他们。死官僚不听话的,党委有权把他们轰走。同时,请将这个文件,作为训练调查组的教材之一。”}\footnote{见链接《毛泽东年谱(1949—1976)》http://dangshi.people.com.cn/n/2013/0527/c85037-21626561-7.html}

\section{在北京会议上的讲话(一九六一年六月十二日)}


人民公社问题,在一九五八年的北戴河会议以后,开了两次郑州会议。第一次会议解决集体所有制和全民所有制的界线问题,社会主义和共产主义的界线问题。第二次会议解决公社内部三级所有制的界线问题。这两次会议的基本方向是正确的。但是,会议开得很仓促,参加会议的同志没有真正在思想上解决问题,对于社会主义建设的客观规律,开始懂得了一些,还是懂得不多。在一九五九年三月的上海会议上,通过了关于人民公社的十八个问题的纪要。后来,我给小队以上的干部写了一封“党内通讯”,对农业方面的六个问题提了意见。在这一段时间内,普遍地对人民公社进行了整顿,使工作中的缺点和错误逐步地得到纠正。不过,由于各级干部还不真正懂得什么是社会主义,什么是按劳分配,什么是等价交换,他们对党中央关于人民公社的许多意见和规定,还没有认识清楚,他们的思想问题还没有得到解决。一九五九年夏季庐山会议上,右倾机会主义分子向党进攻,我们举行反击,获得胜利。反右以后。工作中出现了一些假象。有些地方有些同志以为从此再不要根据两次郑州会议的精神,继续克服工作中存在的缺点和错误了。一九六〇年春,我看出“共产风”又来了,批转了广东省委关于当前人民公社工作中几个问题的指示。在广州,召集中南各省的同志开了三小时的会,接着在杭州又召集华东、西南各省的同志开了三、四天会,后来又在天津召集东北、华北、西北各省同志开了会。这些会,都因为时间短,谈的问题很多,没有把反“一平二调”、反“共产风”的问题作为中心突出来,结果没有解决问题。几个大办一来,糟糕,那不是“共产风”又来了吗,一九六〇年北戴河会议,用百分之七、八十的时间谈国际问题,只是在会议快结束的时候,谈了一下粮食问题,没有接触到人民公社内部的平均主义问题。同年十月,中央发了关于人民公社十二条的指示,从此开始认真纠正“一平二调”的错误,但是仍然坚持供给制、公共食堂、粮食到堂的作法。而且,在执行中,只对三类县、社、队进行了比较认真的整顿,对于一、二类县、社、队的“五风”基本上没有触动,放过去了。一九六一年一月九中全会以后,经过农村调查,在广州开会,强调提出人民公社内部存在着必须解决的两个平均主义问题,起草了农村人民公社六十条。这次会议,启发了思想,解放了思想,然而还不彻底,继续保留了三七开(即供给部分三成,按劳部分七成的分配办法)、公共食堂、粮食到堂的尾巴。经过会后的试点和调查,到这次会议,大家的思想彻底解放了,上面所说几个问题的尾巴最后解决丁,大家对社会主义建设规律的认识,也比过去清楚得多了。由此可见,对客观世界的认识是逐步深入的,任何人不能例外。

<p align="center">×××</p>

我们党从一九二一年成立,经过了陈独秀右倾机会主义和三次“左”倾机会主义的严重挫折,经过了万里长征,经过延安整风,通过了“关于若干历史问题的决议”,到一九四五年的七次代表大会,共用了二十四年的时间,本形成了思想上真正的统一,并且在政治、军事、经济、文化和党的建设等方面形成了一整套实现民主革命总路线的具体政策,保证了抗日战争和解放战争的胜利。建设社会主义社会,我们过去谁也没有干过,必须在实践中才能逐步学会。我们已经搞了十一年,有了社会主义建设总路线,积累了很多经验。只有总路线还不够,还必须有一整套具体政策。现在要好好地总结经验,逐步地把各方面的具体政策制定出来。我们已经有可能这样做,并且已经制定了人民公社六十条。

最近林彪同志下连队做调查研究,一次是在广州,一次是在杨村,了解了很多情况,发现了我们部队建设中一些重要的问题,提出了几个很好的部队建设的措施。要搞具体政策,没有调查研究是不行的。

形成一整套的具体政策,看来还需要一段时间,也许还需要十多年。这是一种设想。如果大家都觉悟了,也可能缩短一些。

<p align="center">×××</p>

现在的重要问题是要重新教育干部。干部教育好了,我们的事业就大有希望。不教育好干部,我们就毫无出路。我们要利用人民公社六十条等文件,作为教材,用延安整风的方法,去教育干部。这次参加会议的同志,思想通了,就要去教育地、县的干部,他们的思想通了,再由他们去教育社、队的干部,使大家具正懂得什么叫社会主义,什么叫按劳分配、等价交换。教育干部的事情,今年一定要做出一点成绩来,并且一定要长期地做下去。搞民主革命,我们长期地教育干部,搞社会主义也必须如此。



\section[给江西共产主义劳动大学的一封信(一九六一年七月三十日)]{给江西共产主义劳动大学的一封信}
\datesubtitle{(一九六一年七月三十日)}

{\noindent 同志们:}

你们的事业我是完全赞成的。半工半读,勤工俭学,不要国家一文钱,小学、中学、大学都有,分散在全省各个山头,少数在平地。这样的学校确是很好的。在校的青年居多,也有一部分中年干部。我希望不但在江西有这样的学校,各省也应有这样的学校。各省应派有能力有见识的负责同志到江西来考察,吸收经验,回去试办。初时学生宜少,逐渐增多,至江西这样有五万人之多。再则党、政、民(工、青、妇)机关,也要办学校,半工半学。不过同江西这类的半工半学不同。江西的工,是农业、林业、牧业这一类的工,学是农、林、牧这一类的学。而党、政、民机关的工,则是党、政、民机关的工,学是文化科学、时事、马列主义理论这样一些学,所以两者是不同的。中央机关已办的两个学校,一个是中央警卫团的,办了六、七年了,战士、干部们从初识文字进小学,然后进中学,然后进大学,一九六零年他们已进大学部门了。他们很高兴,写了一封信给我,这封信,可以印给你们看一看。另一个是去年(一九六零年)办起的。是中南海的各种机关办的,同样是半工半读。工是机关的工,无非是机要人员、生活服务人员、招待人员、医务人员、保卫人员及其他人员。警卫团是军队,他们也有警卫职务,即是站岗守卫,这是他们的工。他们还有严格的军事训练。这些,与文职机关的学校是不同的。

一九六一年八月一日,江西共产主义劳动大学三周年纪念,主持者要我写几个字。这是一件大事,因此为他们写了如上的一些话。

{\raggedleft 毛泽东\\一九六一年七月三十日\par}



\section[对《各地贯彻执行六十条的情况和问题》的批语(一九六一年九月六日)]{对《各地贯彻执行六十条的情况和问题》的批语}
\datesubtitle{(一九六一年九月六日)}

此件很好,印发各同志。并带同去,印发县、市、区党委一级的委员同志们,开一次省委扩大会,有地委同志参加,对此件第二部分所提出的十个问题,作一次认真的解决。时间越早越好,以便在秋收、秋耕、秋种和秋收分配时间政策实行兑现,争取明年丰收。冬春两季六个月整风整社,训练干部,也在这一次省委扩大会上作出布置,主动权就更大了。生产、征购、生活安排,同时并举,就更加主动了。

\section[给政治局常委及有关同志的信]{给政治局常委及有关同志的信(一九六一年九月二十九日)}
\datesubtitle{(一九六一年九月二十九日)}

我们对农业方面的严重平均主义的问题,至今还没有完全解决,还留下一个问题。农民说,六十条就是缺了这一条。这一条是什么呢?就是生产权在小队,分配权却在大队,即所谓“三包一奖”的问题。这个问题不解决,农、林、牧、副、渔的大发展即仍然受束缚,群众的生产积极性仍然要受影响。如果要使一九六二年的农业比较一九六一年有一个较大的增长,我们就应在今年十二月工作会议上解决这个问题。我的意见是:“三级所有,队为基础”。即基本核算单位是队而不是大队。所谓大队“统一领导”要规定界限,河北同志规定了九条。如果不作这种规定,队的九种有许多是空的,还是被大队抓去了。此问题,我在今年三月广州会议上,曾印发山东一个暴露这个严重矛盾的材料,又印了广东一个什么公社包死任务的材料,并在这个材料上面批了几句话:可否在全国各地推行。结果,没有通过。待你们看了湖北、河北这两批材料,并且我们一起讨论过了之后,我建议。把这些材料,并附中央一信发下去,请各中央局、省、市、区党委、地委及若干县委亲身下去,并派有力的调查研究组下去,作两星期调查工作,同县、社、大队、队、社员代表开几次座谈会。看究竟那样办好。由大队实行“三包一奖”好,还是队为基础(河北人叫做分配大包干)好?要调动群众对集体生产的积极性,要在明年一年及以后几年,大量增产粮、棉、油、麻、丝、茶、糖、菜,烟、果、药、杂以及猪、马、牛、羊、鸡、鸭、鹅等类产品,我以为非走此路不可。在这个问题上,我们过去过了六年之久的糊涂日子(一九五六年高级社成立时起),第七年应该醒过来了吧!



\section{在接见日本朋友时的谈话(一九六一年十月七日)}


日本除了亲美的垄断资本家和军国主义军阀之外,广大人民都是我们的真正朋友。你们也会感到中国人民是你们的真正朋友。朋友有真有假,但通过实践可以看清谁是真朋友,谁是假朋友。

中国有句古话,物以类聚,人以群分。日本的岸信介和池田勇人是美帝国主义和蒋介石集团的好朋友。日本人民同中国人民是好朋友。

是美帝国主义迫使我们中日两国人民团结起来。我们两国人民都遭受美帝国主义的压迫,我们有着共同的遭遇,就团结起来了。我们要扩大团结的范围,把全亚洲、非洲、拉丁美洲以及全世界除了帝国主义和各国反动派以外的百分之九十以上的人民团结在一起。

尽管斗争道路是曲折的,但是日本人民的前途是光明的。中国革命经过无数次的曲折,胜利、失败、再胜利、再失败,最后的胜利属于人民。日本人民是有希望的。

<p align="right">(原载《新华月报》1961年第十一期)</p>



\section{关于科学研究十四条的指示(一九六二年一月)}


……科学研究工作十四条,这些条例草案已经在实行或者试行,以后要修改,有些还可能大改,“应当好好地总结经验,制定一整套的方针政策和办法,使他们在正确的轨道上前进”。,“在总路线指导下,制定一整套的方针、政策和办法,必须通过从群众中来的方法,通过做系统的、周密的调查研究的方法,对群众中的成功经验和失败经验,作历史的考察,才能找出客观事物所固有的而不是人们主观臆造的规律,才能制定适合情况的各种条例。这事很重要,请同志们注意到这点”。



\section[给郭沫若回信中的几句话(一九六二年一月)]{给郭沫若回信中的几句话}
\datesubtitle{(一九六二年一月)}

一九六一年,郭沫若看了《孙悟空三打白骨精》后,写了一首诗\footnote{郭沫若诗:《七律·孙悟空三打白骨精》人妖颠倒是非淆,对敌慈悲对友刁。咒念金箍闻万遍,精逃白骨累三遭。千刀当剐唐僧肉,一拔何亏大圣毛。教育及时堪赞赏,猪犹智慧胜愚曹。},毛主席十一月十七日写了一首七律《和郭沫若同志》\footnote{一从大地起风雷,便有精生白骨堆。僧是愚氓犹可训,妖为鬼蜮必成灾。金猴奋起千钧棒,玉宇澄清万里埃。今日欢呼孙大圣,只缘妖雾又重来。},站得更高,看得更远。以后一九六二年一月六日郭沫若又作了一首和主席的诗,诗曰:

赖有晴空霹雳雷,不教白骨聚成堆。九天四海澄迷雾,八十一番弭大灾。僧受折磨知悔恨,猪期振奋报涓埃。金睛火眼无容赦,哪怕妖精亿度来。

该和诗由康生同志转交主席,主席回信郭沫若说“和诗好,不是‘千刀当剐唐僧肉’。对中间派采取了统一战线政策,这就好了。”

\section[在扩大的中央工作会议上的讲话(一九六二年一月三十日)]{在扩大的中央工作会议上的讲话}
\datesubtitle{(一九六二年一月三十日)}


各中央局、各省、市、自治区党委、中央各部委、国家机关和人民团体各党委、党组、总政治部:

毛泽东同志一九六二年一月三十曰《在扩大的中央工作会议上的讲话》是一个十分重兵的马克思列宁主义的文件。中央决定将这个本件发给你们,供党内县团级以上干部学习。毛泽东同志在这个讲话中,着重讲了民主集中制的问题。这个问题是我们党的生活中一个根本性的问题。在我们党掌握了全国政权之后,这个问题尤其重要。毛泽东同志最近指出:“看来问题很大,真要实现民主集中制,是要经过认真的教育、试点和推广,并且经过长期反复进行,才能实现的,否则在大多数同志们当中,始终不过是一句空话。

望各地区、各部门根据毛泽东同志的指示,认真地学习这个文件,发扬批评和自我批评的精神,教育广大干部,特别是领导干部,认真贯彻实行民主集中制和纠正违反民主集中制的各种不良倾向。

《发至县团党委,不登党刊》

<p align="right">中央

一九六六年二月十二日</p>

同志们:

我现在讲几点意见。(热烈鼓掌)一共讲六点,中心是讲一个民主集中制的问题,同时也讲到一些其他问题。

第一点、这次会议的开会方法

这次扩大的中央工作会议,到会的有七千多人。在这次会议开始的时候,×××同志和别的几位同志,准备了4个报告稿子。这个稿子,还没有经过中央政治局讨论,我就向他们建议。不要先开中央政治局会议讨论了,立即发给参加大会的同志们,请大家评论,提意见。同志们,你们有各方面的人,各地方的人,有各省委、地委、县委的人,有企业党委的人,有中央各部门的人。你们当中的多数人是比较接近下层的,你们应当比我们中央常委、中央政治局和书记处的同志更加了解情况和问题。还有,你们站在各种不同的岗位,可以从各种角度提出问题,因此要请你们提意见。报告稿子发给你们了,果然议论纷纷,除了中央提出的基本方针以外,还提出许多意见。后来又由××同志主持,组织了二十一个人的起草委员会,这里有各中央局的负责同志参加,经过八天讨论,写出了书面报告的第二稿。应当说,报告第二稿是中央集中了七千多人议论的结果。如果没有你们的意见,这个第二稿不能写成。在第二稿里面,第一部分和第二部分有很大修改,这是你们的功劳。听说大家对第二稿的评价不坏,认为它是比较好的。如果不是采用这种方法。而是采用通常那种开会方法,就是先来一篇报告,然后进行讨论,大家举手赞成,那就不可能做到这样好。

这是一个开会的方法问题。先把报告草稿发下去,请到会的人提意见,加以修改,然后再作报告。报告的时候不是照着本子念,而是讲一些补充意见,作一些解释。这样,就更能充分地发扬民主,集中各方面的智慧,对各种不同的看法有所比较,会也开得活泼一些。我们这次会议是要总结十二年的工作经验,特别是要总结最近四年来的工作经验,问题很多,意见也会很多,宜于采用这种办法。是不是所有的会议郡可以采用这种方法呢?那也不是,采用这种方法,要有充裕的时间。我们的人民代表大会的会议,有时也可以采用这种方法,省委、地委、县委的同志们,你们以后召集会议,如果有条件的话,也可以采用这种方法。当然你们的工作忙,一般地不能用很长的时间开会,但是在有条件的时候,不妨试一试看。

这个方法是一个什么方法呢?是一个民主集中制的方法,是一个群众路线的方法。先民主,后集中,从群众中来,到群众中去,领导同群众相结合的方法。这是我讲的第一点。

第二点、民主集中制问题

看起来,我们有些同志,对于马克思、列宁所说的民主集中制,还不理解。有些同志已经是老革命了,“三八式”的或者别的什么式的,总之,已经工作了几十年的共产党员了,但是他们还不懂得这个问题。他们怕群众,怕群众讲话,怕群众批评。哪有马克思列宁主义者怕群众的道理呢?有了错误,自己不讲,又怕群众讲。越怕越有鬼。我看不应当怕。有什么可怕的呢?我们的态度是:坚持真理,随时修正错误。我们的工作中是和非的问题,正确和错误的问题,这是属于人民内部矛盾的问题。解决人民内部矛盾,不能用咒骂,也不能用拳头,更不能用刀枪,只能用讨论的方法,说理的方法,批评和自我批评的方法,一句话,只能用民主的方法,让群众讲话的方法。

不论党内党外,都要有充分的民主生活,就是说,都要认真实行民主集中制。要真正把问题敞开,让群众讲话,那怕是骂自己的话,也要让人家讲,骂的结果,无非是自己倒台,不能做这项工作了,降到下级机关去做工作,或者调到别的地方去做工作,那又有什么不可以呢?一个人为什么只能上升不能下降呢?为什么只能做这个地方的工作而不能调到别个地方去呢?我认为这种下降和调动不论正确与否,都是有益处的,可以锻炼革命意志,可以调查和研究许多新鲜情况,增加有益的知识。我自己就有这一方面的经验,得到很大益处。不信你们不妨试一试看。司马迁说过:“文王拘而演周易,仲尼厄而作春秋。届原放逐,乃赋离骚。左丘失明,厥有国语。孙子膑足,兵法修列。不韦迁蜀,世传吕览。韩非囚秦,说难孤愤。诗三百篇,大抵圣贤发愤之所为作也”,这几句话当中,所谓文王演周易,孔子作春秋,究竟有无其事,近人已有怀疑,我们可以不去理它,让专家们去研究吧,但是司马迁是相信有其事的。文王拘、仲尼厄的确有其事,司马迁讲的这些事情,除左丘的一例之外,都是指当时领导对他们作了错误处理的。我们过去也错误地处理了一些干部,对这些人不论是全部处理错的,或者是部分处理错的,都应当按照具休情况,加以甄别和平反。但是,一般地说,这种错误处理,让他们下降,或者调动工作,对他们的革命意志总是一种锻炼,而且可以从人民群众中吸取许多新知识。我在这里申明,我不是提倡对于部对同志,对任何人,可以不分青红皂白,作出错误处理,像古代人拘文王,厄孔子,放逐屈原,去掉孙膑的膝盖骨那样,我不是提倡这样做,而是反对这样做的。我是说,人类的各个历史阶段,总是有这样处理错误的事实。在阶级社会,这样的事实很多。在社会主义社会,也在所难免。不论在正确路线的领导时期,还是在错误路线领导时期,都在所难免。不过有一个区别,在正确路线领导时期,一经发现有错误处理的,就能甄别,平反,向他们赔礼道歉,使他们心情舒畅,重新抬起头来,而在错误路线的领导时期,则不可能这样做,只能由代表正确路线的人们,在适当的时机,通过民主集中制的方法,起来纠正错误。至于自己犯了错误.经过同志们的批评和上级的鉴定,作出正确处理,因而下降或调动工作的人,这种下降或调动,对于他们改正错误,获得新的知识,会有益处,那就不待说了。

现在有些同志,很怕同志开展讨论,怕他们提出同领导机关,领导者意见不同的意见。一讨论问题就压制群众的积极性,不许人家讲话,这种态度非常恶劣。民主集中制是上了我们的党章的。上了我们宪法的,他们就是不实行。同志们,我们是干革命的,如果真正犯了错误,这种错误是不利于党的事业,不利于人民的事业的,就应当征求人民群众和同志们的意见,并且自己做检讨,这种检讨,有的时候,要有若干次,一次不行,大家不满意,再来第二次,还有不满意,再来第三次,一直到大家没有意见了,才不再做检讨。有的省委就是这样做的。有一些省比较主动,让大家讲话,早的,一九五九年就开始做自我批评,晚的,也在一九六一年开始做自我批评。还有一些省委是被迫做检讨的,像河南、甘肃、青海。另外一些省,有人反映,好像现在才刚刚开始作自我批评。不管是主动的,被动的,早做检讨,晚做检讨,只要正视错误,肯承认错误,肯改正错误,肯让群众批评,只要采取了这种态度,都应当欢迎。

批评和自我批评是一种方法,是解决人民内部矛盾的方法,而且是唯一的方法。除此以外,没有别的方法。但是如果没有充分的民主生活,没有真正实行民主集中制,就不可能实行批评和自我批评这种方法。

我们现在不是有许多困难吗?不依靠群众,不发动群众和干部的积极性,就不可能克服困难。但是,如果不向群众和干部说明情况,不向群众和干部交心,不让他们说出自己的意见,他们还对你感到害怕,不敢讲话,就不可能发动他们的积极性。我在一九五七年这样说过:要造成“又有集中,又有民主,又有纪律,又有自由,又有统一意志,又有个人心情舒畅,生动活泼那样一种政治局面。”党内党外都应当有这样的政治局面。没有这样的政治局面,群众的积极性是不可能发动起来的。克服困难,没有民主不行。当然没有集中更不行,但是没有民主就没有集中。

没有民主,不可能有正确的集中,因为大家意见分歧,没有统一的认识,集中制就建立不起来。什么吗集中?首先是要集中正确的意见。在集中正确意见的基础上,做到统一认识,统一政策,统一计划,统一指挥,统一行动,叫做集中统一。如果大家对问题不了解,有意见还没有发表,有气还没出,你这个集中统一怎么建立得起来呢?没有民主,就不可能正确地总结经验。没有民主,意见不是从群众中来,就不可能制定出好的路线方针、政策和办法。我们的领导机关,就制定路线、方针、政策和办法这一方面说来,只是个加工工厂。大家知道,工厂没有原料就不可能进行加工。没有数量上充分的、质量上适当的原料,就不可能制造出好的成品来。如果没有民主,不了解下情,情况不明,不充分搜集各方面的意见,不使上下通气,只由上级领导机关凭着片面的或者不真实的材料决定问题,那就难免不是主观主义的,也就不可能达到统一认识,统一行动,不可能实现真正的集中。我们这次会议的主要议题,不是要反对分散主义,加强集中统一吗?如果离开充分发扬民主,这种集中,这种统一是真的还是假的,是实的还是空的?是正确的还是错误的?当然只能是假的、空的、错误的。

我们的集中制,是建立在民主基础上的集中制。无产阶级的集中,是在广泛民主基础上的集中。各级党委是执行集中领导的机关,但是,党委的领导,是集体的领导,不是第一书记个人独断。在党委会内部只应当实行民主集中制。第一书记同其他书记和委员之间的关系是少数服从多数。拿中央常委或者政治局来说,常常有这样的事情,我讲的话,不管是对的还是不对的,只要大家不赞成,我就得服从他们的意见,因为他们是多数。听说现在有一些省委、地委、县委,有这样的情况。一切事情,第一书记一个人说了就算数。这是很错误的。哪有一个人说了就算数的道理呢?我这是指大事,不是指有了决议后的日常工作。只要是大事,就得集体讨论,认真地听取不同的意见,认真地对于复杂的情况和不同的意见加以分析。要想到事情的几种可能性,估计情况的几个方面,好的和坏的,顺利的和困难的,可能办到的和不可能办到的。尽可能地慎重一些,周一些。如果不是这样,就是一人称霸。这样的第一书记,应当叫做霸王,不是民主集中制的“班长”。以前有个项羽,叫做西楚霸王,他就不爱听别人的不同意见。他那里有个范增,给他出过主意,可是项羽不听范增的话。另外一个人叫刘邦,就是汉高祖,他比较能够采纳不同意见。有个知识分子叫郦食其,去见刘邦。初一报,说是读书人,孔夫子一派的。回答说:现在军事时期,不见儒生。这个郦食其就发了火,他向管门房的人说:你给我滚进去报告,老子是高阳酒徒,不是儒生。管门房的人进去照样报告了一遍。好,请。请了进去,刘邦正在洗脚,连忙起来欢迎。郦食其因为刘邦不见儒生的事,心中还有火,批评了刘邦一顿。他说,你究竟要不要取天下,你为什么轻视长者!这时候,郦食其已经六十多岁了,刘邦比他年轻,所以他自称长者。刘邦一听,向他道歉,立即采纳了郦食其夺取陈留县的意见。此事见《史记》郦食其和朱建传。刘邦是在封建时代被历史家称为“豁达大度”“从谏如流”的英雄人物。刘邦同项羽打了好几年仗,结果刘邦胜了,项羽败了,不是偶然的。我们现在有一些第一书记,连封建时代的刘邦都不如,倒有点像项羽。这些同志不改,最后要垮台的。不是有一出戏叫《霸王别姬》吗?这些同志如果不改,难免有一天要“别姬”就是了。(笑声)我为什么讲得这样厉害呢?是想讲的挖苦一点,对于一些同志戳得痛一点,让这些同志好好地想一想,最好有两天睡不着觉。如果他们睡得着觉,我就不高兴,因为他们还没有被戳痛。

我们有些同志,听不得相反的意见,批评不得。这是很不对的。在我们这次会议中间,有一个省,会本来是开得生动活泼的,省委书记到那里一坐,鸦雀无声,大家不讲话了。这位省委书记同志,你坐到那里去干什么呢?为什么不坐到自己房子里想一想问题,让人家去纷纷议论呢?本来养成了这样一股风气,当着你的面不敢讲话,那末,你就应当回避一下。有了错误,一定要做自我批评,要让人家讲话,让人批评。去年六月十二号,在中央北京工作会议的最后一天,我讲了自己的缺点和错误。我说,请同志们传达到各省、各地方去。事后知道,许多地方没有传达。似乎我的错误就可以隐瞒,而且应当隐瞒。同志们,不能隐瞒。凡是中央犯的错误,直接的归我负责,间接的我也有份。因为我是中央主席,我不是要别人推卸责任,其他一些同志也有责任,但是第一个负责的应当是我。我们的省委书记、地委书记、县委书记直到区委书记、企业党委书记、公社党委书记,既然做了第…书记,对于工作中的缺点错误,就要担起责任。不负责任,怕负责任,不许人讲话,老虎屁股摸不得,凡是采取这种态度的人,十个就有十个要失败。人家总是要讲的,你老虎屁股真是摸不得吗?偏要摸。

在我们国家,如果不充分发扬人民民主和党内民主,不充分实行无产阶级的民主制,就不能有真正的无产阶级的集中制。没有高度的民主,不可能有高度的集中,而没有高度的集中,就不可能建立社会主义经济。我们的国家,如果不建立社会主义经济,那会是一种什么状态呢?就会变成南斯拉夫那样的国家,变成实际上是资产阶级的国家,无产阶级专政就会转化成资产阶级专政,而且会是反动的、法西斯式的专政。这是一个十分值得警惕的问题,希望同志们好好想一想。

没有民主集中制,无产阶级专政不可能巩固。在人民内部实行民主,对人民的敌人实行专政,这两个方面是分不开的,把这两个方面结合起来,就是无产阶级专政,或者叫人民民主专政。我们的口号是:“无产阶级领导的,以工农联盟为基础的人民民主专政。”无产阶级怎样实行领导呢?经过共产党来领导。共产党是无产阶级的先进部队。无产阶级团结一切赞成、拥护和参加社会主义革命和社会主义建设的阶级和阶层。对反动阶级,或者说,对反动阶级的残余实行专政。在我们国内,人剥削人的制度已经消灭,地主阶级和资产阶级的经济基础已经消灭,现在反动阶级已经没有过去那末厉害了,比如说,已经没有一九四九年人民共和过刚建立的时候那么厉害了,也没有一九五七年资产阶级右派猖狂进攻的时候那末厉害了,所以我们说是反动阶级的残余。但是对于这个残余,千万不可轻视,必须继续同他们做斗争,已经被推翻的阶级,还企图复辟。在社会主义社会,还会产生新的资产阶级分子。整个社会主义阶段,存在着阶级和阶级斗争,这种阶级斗争是长期的、复杂的、有时甚至是很激烈的。我们的专政工具不能削弱,还应当加强。我们的公安系统是掌握在正确的同志的手里的。也可能有个别地方的公安部门,是掌握在坏人手里。还有一些作公安工作的同志,不依靠群众,不依靠党,在肃反工作中不是执行在党委领导下通过群众肃反的路线,只依靠秘密工作,只依靠所谓专业工作。专业工作是需要的,对于反革命分子,侦察、审讯是完全必要的,但是,主要是实行党委领导下的群众路线。特别是对于整个反动阶级的专政,必须依靠群众,依靠党。对于反动阶级实行专政,这并不是说把一切反动阶级分子统统消灭掉,而是要改造他们,用适当的方法改造他们,使他们成为新人。没有广泛的人民民主,无产阶级专政不能巩固,政权会不稳。没有民主,没有把群众发动起来,没有群众的监督,就不可能对反动分子和坏分子实行有效的专政,也不可能对他们实行有效的改造,他们就会继续捣乱,还有复辟的可能,这个问题应当警惕,也希望同志们好好想一想。

第三点、我们应当联合那一些阶级?压迫那一些阶级?这是一个根本立场的问题。

工人阶级应当联合农民阶级,城市小资产阶级,爱国的民族资产阶级,首先要联合的是农民阶级。知识分子,例如科学家、工程技术人员、教授、作家、艺术家、演员、医务工作者、新闻工作者,他们不是一个阶级,他们或者附属于资产阶级,或者附属于无产阶级。对于知识分子,是不是只有革命的我们才去团结呢?不是的。只要他们爱国,我们就要团结他们,并且要让他们好好工作。工人、农民、城市小资产阶级分子、爱国的知识分子、爱国的资本家和其他爱国的人士,这些人占全入口的百分之九十五以上。这些人,在我们人民民主专政下面,都属于人民的范围。在人民的内部,要实行民主。人民民主专政要压迫的是地主、富农、反革命分子、坏分子和反共的右派分子。反革命分子、坏分子和反共的右派分子,他们代表的阶级是地主阶级和反动的资产阶级。这些阶级和坏人,大约占全人口的百分之四、五。这些人是我们要强迫改造的。他们是人民民主专政的对象。

我们站在哪一边?站在占全人口百分之九十五以上的人民群众一边?还是站在占全人口百分之四、五的地、富、反、坏、右一边呢?必须站在人民群众这一边,绝不能站到人民敌人那一边去。这是一个马克思列宁主义者的根本立场问题。

在国内是如此,在国际范围内也是如此,各国的人民,占人口总数的百分之九十以上的人民大众,总是要革命的,总是会拥护马克思列宁主义的。他们不会拥护修正主义,有些人暂时拥护,将来终究会抛弃它。他们总会逐步地觉醒起来,总会反对帝国主义和各国反动派,总会反对修正主义。一个真正的马克思列宁主义者,必须坚定地站在占世界人口百分之九十以上的人民大众这一边。

第四点、关于认识客现世界的问题

人对客观世界的认识,由必然王国到自由王国的飞跃,要有一个过程,例如对于在中国如何进行民主革命的问题,从一九二一年党的建立直到一九四五年党的第七次代表大会,一共二十四年,我们全党的认识才完全统一起来。中间经过一次全党范围的整风,从一九四二年春天到一九四五年夏天,有三年半的时间。那是一次细致的整风。采用的方法是民主的方法:就是说,不管什么人犯了错误,只要认识了,改正了,就好了,而且大家帮助他认识,帮助他改正,叫做“惩前毖后,治病救人”,“从团结的愿望出发,经过批评或者斗争,分清是非,在新的基础上达到新的团结”。“团结——批评——团结”,这个公式就是在那个时候产生的。那次整风帮助全党同志,统一了认识。对于当时的民主革命应当怎么办,党的总路线和各项具体政策应当怎样定,这些问题,都是在那个时期,特别是在整风之后,才得到完全解决。

从党的建立到抗日时期,中间有北伐战争和十年土地革命战争。我们经过了两次胜利,两次失败。北伐战争胜利了,但是到一九二七年,革命遭到了失败。土地革命战争曾经取得了很大的胜利,红军发展到三十万人,后来又遭到挫折,经过长征,这三十万人缩小到两万多人,到陕北以后补充了一点,还是不到三万人,就是说,不到三十万人的十分之一。究竟是那三十万人的军队强些,还是这不到三万人的军队强些?我们受了那样大的挫折,吃过那样大的苦头,就得到锻炼,有了经验,纠正了错误路线,恢复了正确路线,所以这不到三万人的军队,比起过去那个三十万人的军队来,要更强些。×××同志在报告里说,最近四年,我们的路线是正确的,成绩是主要的,我们在实际工作中犯过一些错误,吃了苦头,有了经验了,因此我们更强了,而不是更弱了。情况正是这样。在民主革命时期,经过胜利,失败,再胜利,再失败,两次比较,我们才认识了中国这个客观世界。在抗日战争前夜和抗日战争时期,我写了一些论文,例如《中国革命战争的战略问题》,《论持久战》,《新民主主义论》,《共产党人发刊词》,替中央起草过一些关于政策、策略的文件,都是革命经验的总结。那些论文和文件,只有在那个时候才能产生,在以前不可能,因为没有经过大风大浪,没有两次胜利和两次失败的比较,还没有充分的经验,还不能充分认识中国革命的规律。

中国这个客观世界,整个地说来,是由中国认识的,不是在共产国际管中国问题的同志们认识的。共产国际的这些同志就不了解或者说不很了解中国社会、中国民族。对于中国这个客观世界,我们自己在很长时间内都认识不清楚,何况外国同志呢?

在抗日时期,我们才制定了合乎情况的党的总路线和一整套具体政策。这时候,我们已经干了二十来年的革命,过去那么多年的革命工作,是带着很大的盲目性的。如果有人说,有那一位同志,比如前中央的任何同志,比如说我自己,对于中国革命的规律,在一开始的时候就完全认识了,那是吹牛,我们切记不要相信,没有那同事。过去,特别是开始时期,我们只是一股劲儿要革命,至于怎么革法,革些什么,那些先革,那些后革,那些要到下一阶段才革,在一个相当长的时间内,都没有弄清楚,或者说没有完全弄清楚。我讲我们中国共产党人在民主革命时期艰难地但是成功地认识中国革命规律这一段历史情况的目的,是想引导同志们理解这样一件事:对建设社会主义的规律的认识,必须有一个过程。必须从实践出发,从没有经验到有经验,从有较少的经验,到有较多的经验,从建设社会主义这个未被认识的必然王国,到逐步地克服盲目性,认识客观规律,从而获得自由,在认识上出现一个飞跃,到达自由王国。

对于社会主义建设我们还缺乏经验。我和好几个国家的兄弟党的代表团谈过这个问题,我说:对建设社会主义经济,我们没有经验。这个问题我也向一些资本主义国家的新闻记者谈过,其中有一个美国人叫斯诺,他老要来中国,一九六零年让他来了。我同他谈过一次话,我说:“你知道,对于政治、军事,对于阶级底级斗争,我们有一套经验,有一套方针、政策和办法,至于社会主义建设,过去没有干过,还没有经验。你会说,不是已经干十一年了吗?是干了十一年了,可是还缺乏知识,还缺乏经验,就算开始有了一点,也还不多。”斯诺要我讲讲中国建设的长期计划。我说:“不晓得。”他说“你讲话太谨慎。”我说:“不是什么谨慎不谨慎,我就是不晓得事呀!就是没有经验呀。”同志们,也真是不晓得,我们确实还缺乏经验,确实还没有这样一个长期计划。一九六○年,那正是我们碰了许多钉子的时候。一九六一年,我同蒙哥马利谈话,也说到上面那些意见。他说:“再过五十年,你们就了不起了。”他的意思是说,过了五十年,我们就会壮大起来,而且会“侵略”人家,五十年内还不会,他的这种看法,一九六零年他来中国的时候就对我说过。我说:“我们是马克思列宁主义者,我们的国家是社会主义国家,不是资本主义国家,因此,一百年,一万年,我们也不会侵略别人。至于建设强大的社会主义经济,在中国,五十年不行,会要一百年,或者更多的时间。在你们国家,资本主义的发展,经过了好几百年。十六世纪不算,那还是中世纪。从十七世纪到现在,已经有三百六十多年。在我国,要建设起强大的社会主义经济,我估计要花一百多年。”十七世纪是什么时代呢?那是中国的明朝末年和清朝初年。再过一世纪,到十八世纪的上半期,就是清朝乾隆时代,《红楼梦》的作者曹雪芹就生活在那个时代,就是产生贾宝玉这种不满意封建制度的小说人物的时代。乾隆时代,中国已经有了一些资本主义生产关系的萌芽,但是还是封建社会。这就是出现大观园里那一群小说人物的社会背景。在那个时候以前,在十七世纪,欧洲的一些国家已经在发展资本主义了,经过三百多年,资本主义的生产力有了现在这样子。社会主义和资本主义比较,有许多优越性,我们国家经济的发展,会比资本主义国家快得多。可是,中国的人口多,底子薄,经济落后,要使生产力很大地发展起来,要赶上和超过世界上最先进的资本主义国家,没有一百多年的时间,我看是不行的。也许只要几十年,例如有些人所设想的五十年,就能做到。果然这样,谢天谢地,岂不甚好。但是我劝同志们宁肯把困难想得多一点,因而把时间设想得长一点。三百几十年建设了强大的资本主义经济,在我国,五十年内外到一百年内外,建设起强大的社会主义经济,那又有什么不好呢?从现在起,五十年内外到一百年内外,是世界上社会制度彻底变化的伟大时代,是一个翻天覆地的时代,是过去任何一个历史时代都不能比拟的。处在这样一个时代,我们必须准备进行同过去时代的斗争形式有着许多不同特点的伟大斗争。为了这个事业,我们必须把马克思列宁主义的普遍真理同中国社会主义建设的具体实际,并同今后世界革命的具体实际,尽可能好一些地结合起来,从实践中一步一步地认识斗争的客观规律。要准备着由盲目性遭到许多的失败和挫折,从而取得经验,取得最后胜利。由这点出发,把时间设想得长一点,是有许多好处的,设想得短了反而有害。

在社会主义建设上,我们还有很大的盲目性。社会主义经济,对于我们来说,还有许多未被认识的必然王国。拿我来说,经济建设工作中的许多问题还不懂得。工业、商业,我就不大懂。对于农业我懂得一点。但是也是比较地懂得,还是懂得不多。要较多地懂得农业,还要懂得土壤学、植物学、作物栽培学、农业化学、农业机械等等,还要懂得农业内部各个专业部门,例如粮、棉、油、麻、丝、茶、糖、菜、烟、果、药、杂等等,还有畜牧业,还有林业。我是相信苏联威廉氏土壤学的,在威廉氏的土壤学著作里,主张农、林、牧三结合。我认为必须要有这种三结合,否则对于农业不利。所有这些农业生产问题,我劝同志们,在工作之暇,认真研究一下,我也还想研究一点。但是到现在为止,这些方面,我的知识很少。我注意的较多的是制度方面的问题,生产关系方面的问题,至于生产方面,我们知识很少。社会主义建设,从我们全党来说,知识都非常不够。我们应当在今后一段时间内,积累经验,努力学习,在实践中间逐步地加深对它的认识,弄清楚它的规律,一定要下一番苦功,要切切实实地去调查它,研究它。要下去蹲点,到生产大队、生产队,到工厂,到商店,去蹲点。调查研究,我们从前做得比较好,可是进城以后,不认真做了,一九六一年我们又重新提倡,现在情况已经有所改变。但是,在领导干部中间,特别是在高级领导干部中间,有一些地方部门和企业,至今还没有形成风气。有一些省委书记,到现在还没有下去蹲过点,如果省委书记不去,怎么能叫地委书记、县委书记下去蹲点呢,这个现象不好,必须改变过来。

从中华人民共和国成立到现在已经十二年了。这十二年分前八年和后四年,一九五零年到一九五七年底是前八年。一九五八年到现在,是后四年,我们这次会议已经初步总结了过去工作的经验,主要是后四年的经验。这个总结.反映在×××同志的报告里面。我们已经制定或者正在制定,或者将要制定各个方面的具体政策。已经制定了的,例如农村工作六十条,工业企业七十条,高等教育六十条,科学研究工作十四条,这些条例草案已经在实行或者试行,以后还要修改,有些还可能大改。正在制定的,例如商业工作条例。将要制定的,例如中小学教育条例。我们的党政机关和群众团体的工作,也应当制定一些条例。军队已经制定了一些条例。总之,工、农、商、学、兵、政、党这七个方面的工作,都应当好好地总结经验,制定一整套的方针、政策和方法,使它们在正确的轨道上前进。

有了总路线还不够,还必须在总路线指导下,在工、农、商、学、兵、政、党各个方面,有一整套适合情况的具体的方针、政策和办法,才有可能说服群众和干部,并且把这些当作教材去教育他们,使他们有一个统一的认识和统一的行动,然后才有可能取得革命事业和建设事业的胜利,否则是不可能的。对于这一点,我们在抗日时期就有了深刻的认识。在那时候,我们这样做了,就使得干部和群众对于民主革命时期的一整套具体的方针、政策和办法,有了统一的认识,因而有了统一的行动,使当时的民主革命事业取得了胜利,这是大家知道的。在社会主义革命和社会主义建设的时期,头几年内,我们的革命任务,在农村是完成对封建主义的土地制度的改革和接着实现农业合作化,在城市是实现对资本主义商业的社会主义改造。在经济建设方面,那时候的任务是恢复经济和实现第一个五年计划。不论在革命方面和建设方面,那时候都有一条适合客观情况的,有充分说服力的总路线,以及在总路线指导下的一整套方针,政策和办法,因此教育了干部和群众,统一了他们的认识,工作也就比较做得好。这也是大家知道的。但是,那时候有这样一种情况,因为我们没有经验,在经济建设方面,我们只是照抄苏联,特别是在重工业方面,几乎一切都抄苏联。自己的创造性很少。这在当时是完全必要的,同时又是一个缺点,缺乏创造性,缺乏独立自主的能力。这当然不应当是长久之计。从一九五八年起,我们就确立了自力更生为主,争取外援为辅的方针。在一九五八年党的八大二次会议上,通过了“鼓足干劲,力争上游,多快好省地建设社会主义”的总路线,在那一年又办起了人民公社,提出了大跃进的口号。在提出社会主义建设总路线的一个相当时期内,我们还没有来得及,也没有可能规定一整套适合情况的具体的方针、政策和办法,因为经验还不足。在这种情况下,干部和群众,还得不到一整套的教材,得不到系统的政策教育,也就不可能真正有统一的认识和统一的行动。要经过一段时间,碰到一些钉子,有了正、反两方面的经验,才有这样的可能,现在好了,有了这些东西了,或者正在制定这些东西。这样,我们就可以更加妥善地进行社会主义革命和社会主义建设。在总路线指导之下,制定一整套具体的方针、政策和办法,必须通过从群众中来的方法,通过系统的、周密的调查研究的方法,对工作中的成功经验和失败经验,作历史的考察,才能找出客现事物所固有的而不是人们主观臆造的规律,才能制定适合情况的各种条例。这种事很重要,请同志们注意到这点。

工、农、商、学、兵、政、党,这七个方面,党是领导一切的。党要领导工业、农业、商业、文化教育、军队和政府。我们的党,一般说来是很好的,我们党员的成分,主要的是工人和贫苦农民,我们的绝大多数干部都是好的,他们都在辛辛苦苦地工作。但是,也要看到,我们党内还存在一些问题,不要想象我们党的情况什么都好,我们现在有一千七百多万党员,这里面差不多有百分之八十的人是建国以后入党的,五十年代入党的。建国以前入党的只占百分之二十。在这百分之二十的人里面,一九三零年以前入党的,二十年代入党的,据前八年计算,有八百多人,这两年死了一些,恐怕只有七百多人了。不论在老的和新的党员里面,特别是在新党员里面,都有一些品质不纯和作风不纯的人。他们是个人主义者,官僚主义者,主观主义者,甚至是变了质的分子。还有些人挂着共产党员的招牌,但是并不代表工人阶级,而是代表资产阶级。党内并不纯洁,这一点必须看到,否则我们是要吃亏的。

上面是我讲的第四点。就是讲。我们对于客观世界的认识要有一个过程。先是不认识,或者不完全认识,经过反复的实践,在实践里面得到成绩,有了胜利,又翻过筋斗,碰了钉子,有了成功和失败的比较,然后才有可能发展成为完全的认识或者比较完全的认识。在那个时候,我们就比较主动了,比较自由了,就变成比较聪明一些的人了。自由是对必然的辩证规律。所谓必然,就是客观存在的规律性。在没有认识它以前,我们的行为总是不自觉的,带有盲目性的。这时候,我们是一些蠢人。最近几年,我们不是干过许多蠢事吗!第五点,关于国际共产主义运动这个问题,我只简单地讲几句。

不论在中国,在世界各国,总而言之,百分之九十以上的人终究是会拥护马克思列宁主义的。在世界上,现在还有许多人,在社会民主党的欺骗之下,在修正主义的欺骗之下,在帝国主义的欺骗之下,在各国反动派的欺骗之下,他们还不觉悟。但是他们总会逐步地觉悟过来,总会拥护马克思列宁主义。马克思列宁主义这个真理,是不可抗拒的,人民群众是要革命的。世界革命总是要胜利的。不准革命,像鲁迅所写的赵太爷,钱太爷,假洋鬼子不准阿Q革命那样,总是要失败的。

苏联是第一个社会主义国家,苏联共产党是列宁创造的党。虽然苏联的党和国家的领导现在被修正主义篡夺了,但是,我们劝同志们坚决相信,苏联广大的人民,广大的党员和干部是好的,是革命的,修正主义的统治是不会长久的。无论什么时候,现在,将来,我们这一辈子,我们的子孙,都要向苏联学习,学习苏联的经验。不学习苏联要犯错误。人们会问:苏联被修正主义统治了,还要学吗?我们学习的是苏联的好人好事,苏联党的好经验.至于苏联的坏人坏事,苏联的修正主义者,我们应当看作反面教员,从他们那里吸取教训。

我们永远要坚持无产阶级的国际主义团结的原则,我们始终主张社会主义和世界共产主义运动一定要在马克思列宁主义的基础上巩固地团结起来。

国际修正主义者在不断地骂我们。我们的态度是:由他骂去。在必要的时候,给予适当的回答。我们这个党是被人家骂惯了的。从前骂的不说,现在呢,在国外,帝国主义者骂我们,反动的民族主义者骂我们,修正主义者骂我们。在国内蒋介石骂我们,地、富、反、坏、右骂我们。历来就是这么骂的。……我们是不是孤立的呢?我就不感觉孤立。我们在座的有七千多人,七千多人还孤立吗?世界各国人民群众已经或者将要同我们站到一起,我们会是孤立的吗?

最后一点,第六点,要团结全党和全体人民。

这个问题我只简单地讲几句。

要把党内、党外的先进分子,积极分子团结起来,把中间分子团结起来,去带动落后分子,这样就可以使全党、全民团结起来。只有依靠这些团结,我们才能够做好工作,克服困难,把中国建设好。要团结全党、全民,这并不是说我们没有倾向性。有些人说共产党是“全民的党”,我们不这样看。我们的党是无产阶级政党,是无产阶级的先进部队,是用马克思列宁主义武装起来的战斗部队。我们是站在占总人口百分之九十五以上的人民大众一边,绝不站在占总人口百分之四、五的地、富、反、坏、右那一边。在国际范围也是这样,我们是同一切马克思列宁主义者,一切革命人民、全体人民讲团结的,绝不同反共反人民的帝国主义者和各国反动派讲什么团结。只要有可能,我们也要同这些人建立外交关系,争取在五项原则的基础上和平共处。但是这些事,跟我们和各国人民的团结是不同范畴的两同事。

要使全党全民团结起来,就必须发扬民主,让人讲话。在党内是这样,在党外也是这样。省委的同志、地委的同志、县委的同志,你们回去,一定要让人讲话。在座的同志们要这样做,不在座的同志们也要这样做。一切党的领导人员都要发扬民主,让人讲话。界限是什么呢?一个是遵守党的纪律,少数服从多数,全党服从中央。另一个是,不准组织秘密集团。我们不怕公开反对派,只怕秘密的反对派,这种人当面不讲真话,当面讲的尽是些假的,骗人的话,真正的目的不讲出来。只要不是违犯纪律的。只要不是搞秘密集团活动的,我们都允许他讲话,而且讲错了也不要处罚,讲错了话可以批评,但要用道理说服人家。说而不服怎么办:让他保留意见。只要服从决议,服从多数人决定的东西,少数人可以保留不同意见。在党内、党外,允许少数人保留意见,是有好处的,错误的意见,让他暂时保留,将来他会改的。许多时候,少数人的意见倒是正确的。历史上常常有这样的事实,起初,真理不是在多数人手里,而是在少数人手里。马克思、恩格斯手里有真理,可是他们在开始的时候是少数。列宁在很长一个时期内也是少数。我们党内也有这样的经验,在陈独秀统治的时候,在“左”倾路线统治的时候,真理都不在领导机关的多数人手里,而是在少数人手里。历史上的自然科学家,例如:哥白尼、伽利略、达尔文,他们的学说曾经在一个长时间内不被多数人承认,反而被看作错误的东西,当时,他们是少数。我们党在1921年成立的时候,只有几十个党员,也是少数人。可是这几十个人代表了真理,代表了中国的命运。

有一个捕人、杀人的问题,我还想讲一下。在现在的时候,在革命胜利还只有十几年的时候。在被打倒了的反动阶级还没有被改造好,有些人并且企图阴谋复辟的时候,人总会要捕一点,杀一点的,否则不能平民愤,不能巩固人民的专政。但是,不要轻于捕人,尤其不要轻于杀人。有一些坏人,钻到我们队伍里面的坏分子,蜕化变质分子,这些人,骑在人民头上拉屎拉尿,穷凶极恶,严重地违法乱纪,这是些小蒋介石。对于这种人得有个处理,罪大恶极的,也要捕一些,还要杀几个。因为对这样的人,完全不捕、不杀,不足以平民愤。这就是所谓的“不可不捕,不可不杀。”但是绝不可多捕、多杀。凡是可捕可不捕的,可杀可不杀的,都要坚决不捕,不杀。有个潘汉年,此人当过上海市副市长,过去秘密投降了国民党,是个CC派人物,现在关在班房里头,我们没有杀他。像潘汉年这样的人,只要杀一个,杀戒一开,类似的人都得杀。还有个王实味,是个暗藏的国民党探子。在延安的时候,他写过一篇文章,题名《野百合花》,攻击革命,诬蔑共产党。后来把他抓起来,杀掉了。那是保安机关在行军中间,自己杀的,不是中央的决定。对于这件事,我们总是提出批评,认为不应当杀。他当特务,写文章骂我们,又死不肯改,就把他放在那里吧,让他劳动去吧,杀了不好。人要少捕,少杀。动不动就捕人、杀人,会弄得人人自危,不敢讲话。在这种风气下面,就不会有多少民主。

还不要给人乱戴帽子。我们有些同志惯于拿帽子压人,一张口就是帽子满天飞,吓得人不敢讲话。当然,帽子总是有的,×××同志的报告里面不是就有许多帽子吗?“分散主义”不是帽子吗?但是不要动不动就给人戴在头上,弄得张三分散主义,李四分散主义,什么人都是分散主义。帽子最好由人家自己戴,而且要戴得合适,最好不要由别人去戴。他自己戴了几回,大家不同意他戴了,那就取消了。这样,就会有很好的民主空气。我们提倡不抓辫子,不戴帽子,不打棍子,目的就是要使人心里不怕,敢于讲意见。

对于犯了错误的人,对于那些不让别人讲话的人,要采取善意帮助的态度。不要有这样的空气,似乎犯不得错误,一犯错误,从此不得翻身。一个人犯了错误,只要他真心愿意搞正,只要他确实有了自我批评,我们就要表示欢迎。头一、二次自我批评,我们不要要求过高,检查得还不彻底,不彻底也可以,让他再想一想,善意地帮助他。人是要有人帮助的。应当帮助那些犯错误的同志认识错误。如果人家诚恳地作了自我批评,愿意改正错误,我们就要宽恕他,对他采取宽大的政策。只要他的工作成绩还是主要的,能力也还行,就还可以让他在那里继续工作。

我在这个讲话里批评了一些现象,批评了一些同志,但是没有指名道姓,没有指出张三、李四来。你们自己心里有数。(笑声)我们这几年工作中的缺点、错误,第一笔账,首先是中央负责,中央又是我首先负责;第二笔账,是省委、市委、自治区党委的;第三笔账,是地委一级的;第四笔账,是县委一级的,第五笔账,就算到企业党委,公社党委的了。总之,各有各的账。

同志们,你们回去,一定要把民主集中制健全起来。县委的同志,要领导公社把民主集中制健全起来。首先要建立和加强集体领导,不要再实行长期固定的“分片包干”的领导方法了,那个方法,党委书记和委员们各搞各的,不能真正的集体讨论,不能有真正的集体领导,要发扬民主,要启发人家批评,要听人家的批评。自己要经得起批评。应当争取主动,首先作自我批评。有什么就检讨什么,一个钟头,顶多两个钟头,倾箱倒筐而出,无非是那么多。如果人家认为不够,请他提出来,如果说得对,我就接受。让人讲话,是采取主动好,还是被动好?当然是主动好。已经处在被动地位了怎么办?过去不民主,现在陷入被动,那也不要紧,就请大家批评吧。白天出气,晚上不看戏,白天晚上都请你们批评。(掌声)这个时候,我坐下来,冷静地想一想,两三天晚上睡不着党,想好了,想通了,然后诚诚恳恳地作一篇检查。这不就好了吗?总之,让人讲话,天不会塌下来,自己也不会垮台。不让人家讲话呢?那就难免有一天要垮台。

我今天的讲话就讲这一些。中心是讲了一个实行民主集中制的问题,在党内、党外发扬民主的问题。我向同志们建议,仔细考虑一下这个问题。有些同志还没有民主集中制的思想,现在要开始建立这个思想,开始认识这个问题。我们充分地发扬了民主,就能把党内党外广大群众的积极性调动起来,就能使占总人口百分之九十五以上的人民大众团结起来。做到了这些,我们的工作就会越做越好,我们遇到的困难就会较快地得到克服,我们事业的发展就会顺利得多。(热烈鼓掌)



\section[接见几内亚政府经济代表团和妇女代表团的谈话(一九六二年五月三日)]{接见几内亚政府经济代表团和妇女代表团的谈话}
\datesubtitle{(一九六二年五月三日)}


主席:你们是来自友好国家、友好政府的代表团,欢迎你们。所有非洲的朋友,都受到中国人民的欢迎。我们与所有非洲国家人民的关系都是好的,不管是独立或没有独立正在斗争中的人民。非洲正出现一个很大的争取民族独立,反对帝国主义、反对殖民主义的革命运动。非洲有多少人口?二亿吧?二亿人民要翻身,不管已经站起来或者将要站起来。还有拉丁美洲也是二亿人口,亚洲的十几亿人口和全世界的革命人民。我们不是孤立的,到处都有我们的朋友,你们也不是孤立的。你们来中国可以感到中国人民是十分欢迎你们的。来了几天了?

凯塔(几政府经济代表团团长)。我们是四月十九日末的。

主席。她们呢?(指妇女代表团)

凯塔:她们是四月二十五日来的。

主席;听说你们明天要走了。

凯塔。她们不走。

主席:欢迎。他(指柯庆施同志)是上海的主人,柯庆施是中共中央政治局委员,同你们是民主党的中央政治局委员一样,对吗?

柯庆施:你们为何不多住几天?

凯塔。我们的日程排得很紧,国内工作很多,五月十五日以前要完成改组党的各级机构,如有时间,我们很愿意在中国访问一个月。

主席:你们的觉是很好的党,是个联系群众的党,有纪律的党,是一个有以反对帝国主义、反对殖民主义和建立民族经济作为纲领的党,一个独立自主国家的领导的党。我们感到同你们是很接近的。我们两国、两党互相帮助,互相支持,你们不捣我们的鬼,我们也不捣你们的鬼。如果我们有人在你们那里做坏事,你们就对我们讲。例如看不起你们,自高自大,表现大国沙文主义态度,有没有这种人?

凯塔:没有。

主席:如有这种人,我们要处分他们。

凯塔:有些国家的技术人员有这种情况,中国专家没有这种情况,他们都工作得很好。

主席:是不是有比你们几内亚专家薪水高、特殊化的情况?(对叶××说)恐怕有,要检查,待遇要一样,最好低一些。(叶××:周总理正在要方×同志检查。)

凯塔:这个问题是值得研究的,但直到现在中国专家并没有过分的要求,有些国家的专家要比几专家高二、三倍,相反,中国的专家没有过分的要求。

主席:驻几内亚大使是谁?是柯华吗?(旁人答是的)

凯塔:只有中国专家和越南专家待遇一样。

主席:是否有人损害你们的民族利益?搞颠覆活动?

凯塔:有,但不是中国人。对搞颠覆活动的人,我们也不是听任他们去搞的,发生这种情况,我们要迅速采取措施加以回击,我们不愿意做人家的尾巴。过去发生的事件你们是知道的,我们对这些事件的态度你们也是知道的。

主席:你们做得对。凡有人在你们那里称王称霸,不服从你们的法律,搞颠覆活动,应把他们赶掉。我们希望你们站住脚,不仅在政治上,而且要在经济上站住脚,不要被人颠覆掉了。你们站住脚我们高兴。你们倒台我们不高兴。因为你们是一个革命的党,是一个革命的政府,在非洲有很大的影响。经过你们,可以在非洲许多国家做工作,使他们得到解放。你们也有这个责任,不要自己独立就不管别人了。我们也一样,不能因为自己独立了就不管别人了。所谓管别人是指友好的支持、帮忙。你们知道我们现在还有些困难,帮忙不大。再过五年、十年我们的情况可能好一些,那时的帮助可能多一些。我们的国家,有一个很大的缺点,人太多,这么多人要吃饭,要穿衣,所以现在还有不少困难,但这些困难不是不可克服的,而是能够克服的,正在采取措施克服。我国的经济、文化与你们差不多,差不多是在没有什么遗产的情况下搞起来的。你们是法国的殖民地,我们是几个国家的殖民地。你们与法国建立了外交关系吗?

凯塔:关于跟法国建交的问题,还有一些悬而未决的问题至今没有解决。我们独立后与法国双方互派过代表,进行过谈判,想解决这些问题,但这些问题至今还未解决,我们希望能够解决。

主席:你们与阿尔及利亚的关系好吗?

凯塔:好的。

主席:与马里呢?

凯塔:非常好。

主席:与索马里呢?

凯塔:差一些,还没有外交关系,往来较少。

主席:与加纳呢?

凯塔:好的。

主席:与摩洛哥和突尼斯呢?

凯塔:跟摩洛哥和突尼斯也好,但有些不同。非洲有两个不同的集团,即卡萨布兰卡集团和蒙罗维亚集团。

主席:蒙罗维亚集闭?

凯塔:卡萨布兰卡集团是阿尔及利亚、加纳、几内亚、马里、利比里亚和摩洛哥六国集团。蒙罗维亚集团是过去非洲马尔加什联盟的国家。卡集团较进步,蒙集团不大进步,与殖民主义联系较多。

主席:蒙集团是否属于法属共同体?

凯塔:是的。我们与卡萨布兰卡集团关系好…些。跟蒙罗维亚集团的有些国家,如塞内加尔、利比里亚、象牙海岸等国,边境相连,遭遇和问题都差不多。我们认为非洲分为这样两个集团并不符合非洲人民的利益,所以杜尔总统向所有非洲国家采取外交措施,创议召开非洲国家首脑会议,五月份要开会,要协调相互的立场,取得一些共同点,取得合作。正如主席所说,进步力量应该支持邻国人民,非洲许多国家与殖民主义势力有联系,与欧洲共同市场有联系,而不是与兄弟的邻国有联系。我们预备开会讨论非洲共同市场问题,以便发展非洲自己的经济,摆脱非洲殖民主义势力的控制。非洲所有国家首脑会议五月份在埃塞俄比亚首都亚的斯亚贝巴召开。如果几内亚创议的非洲首脑会议有结果,可以使非洲国家的关系进一步密切。

主席:非洲国家要联合,另外还要一个更大的联合,即亚、非、拉美三大洲的联合。

凯塔:我们也意识到这种大联合的重要性,因此首先非洲国家自己要联合,以便在大联合中起积极作用。我们不少非洲国家正在受痛苦,还在受殖民主义的痛苦,特别是经济上受新殖民主义的痛苦。要实现大联合,以反对帝国主义和殖民主义。

主席:几内亚有多少人口?

凯塔:几内亚国家很小,有四百万人口。

主席:土地面积多少?

凯塔:二十五万平方公里,平均每平方公里十二――十三人。

主席:很大的土地,有很大的发展前途。有森林吗?

凯塔:很多,特别是矿产的前途很大,我国有丰富的铁矿、铝矿、铬矿。是非洲矿产最丰富的国家,还有丰富的水力资源,可以利用来发电,以便在当地自己提炼矿砂。

主席:听说你们在建造一座大水坝?

凯塔:在法国殖民统治时期,法国人已有此计划。法国想在孔库雷河上建造一座大水坝,每年可发六十亿度的电,用此电力提炼铝。法国组织了国际公司,并与国际银行建立了关系,以便取得资金。一九五八年几内亚独立了,法国认为不安全,就放弃了这个计划。独立后,几内亚政府想搞,苏联原则上同意与东欧几个社会主义国家一起援助几建设水坝,但现在苏、几政治关系复杂化了,恐怕不准备搞了。

主席:还没有搞吗?

凯塔:没有搞,杜尔总统访华时经过莫斯科,苏原则同意援助,但至今没有动静。

主席:听说有个货币问题,解决了没有?

凯塔:我们在一九六零年建立了几内亚法郎,目的是退出法郎区,建立独立的货币区。这种

凯塔:法郎不能兑换外币,以避免殖民主义者掌握大量几内亚货币兴风作浪。但有些邻国以几法郎投机,他们从几带出大量货币,换美元和英镑,或以低于官价出售几法郎,压低币值,或贩运货物到几内亚来换取几币搞投机。所以几政府在今年四月决定取消旧币换新币,在外国的旧币一律作废。

主席:你们自己能印钞票吗?

凯塔:不能。

主席:在哪里印呢?

凯塔:起初在捷克,最近一次在英国印。现在正设法自己弄到印钞票的机器,以便保证不断地印自己的钞票。

主席:几内亚妇女有选举权吗?

卡玛拉(几妇女代表团团长):有的,在党内、政府内都有。

主席。党里有妇女领导人吗?

卡玛拉:党的街道委员会,村委员会和省委员会都有妇女领导人,恩廸阿依、贡代二位都得到了独立勋章。

主席。你们的革命是群众性的,党也是群众性的。我们的党中央员会女的太少了,女的有,但比例是男的多,女的少,地方党委也是如此。你们走在我们前面去了。

凯塔:我们的比例也少,十七个政治局委员中只有二个女的,政府中只有一个部长是女的,就是卡玛拉夫人。我们那里也仅仅是开始,正如周恩来总理所说,妇女受到双重压迫,不仅有帝国主义、殖民主义和封建主义的压迫,而且男女不平等,女的上学机会不多,革命胜利后男的还有封建思想,女的积极斗争,现在有女的市长、村长等。

主席;慢慢来。

凯塔:她(指卡玛拉夫人)想一下子解决问题。

卡玛拉;他想阻拦。(全场笑)

主席:我们与你们的情况差不多,比较接近,所以我们同你们谈得来,没有感到我欺侮你,你欺侮我,没有什么优越感,都是有色人种。有人想欺侮我们,认为我们生来就不行,认为我们没有办法,命运注定了,一万年该受帝国主义的压迫,不会管理国家,不会搞工业,不能解决吃饭问题,科学文化也不行。他们不想一想,这种状况是他们造成的,经济、文化水平低是他们造成的。管理国家过去是他们代替我们管理的。本国人讲管理是可以的,但要学,学多少年,慢慢来,可是你们不是慢慢来,而是一下子就取得政权,我们也是,夺取了政权再学嘛!不会管理慢慢就会管理了,有错误就改嘛!难道只有我们有错误,西方国家没有错误?他们的错误比我们更大,他们犯了反革命的错误。我们根本上没有错误,我们是革命。没有工业可以逐步搞工业,没有现代化的农业可以逐步搞现代化的农业,科学文化水平也能一年一年提高,例如地质人员,你们现在开始有了吗?

凯塔:以前的地质人员都是别国的,现在有许多几留学生在别的国家培养。

主席:我们国民党、蒋介石遗留下来的地质人员只有二百人,现在十三年来有了十几、二十万人。(问柯庆施同志:各省都有吗?柯说有。)能搞起来的。难道只有西方国家能搞,我们就不能搞起来吗?

凯塔:杜尔总统也认为争取独立首先要自己相信自己,管理国家也是一样。只有打铁,才能成为铁匠,只有学了才会,管理国家慢慢能学会。如一九五八年独立时,几学生只有三万二千人(指在学校的),现在有了十二万。

主席:增加了三倍多。你们过去可能没有大学。

凯塔:没有,很少有机会上大学,上大学要到巴黎去。当时殖民主义者对培养当地的干部没有兴趣,只有二百个大学生,现在有一万五千个大学生,各省都有。

主席:比例不小,四百万人口中有一万五千人。

凯塔:现在科纳克里正在搞技术高等学校,培养地质、农业等技术人材。一方面继续向外国派,另一方面尽量在国内培养高等学校的学生,使能适合本国的条件。

主席:今晚你们有何活动?柯庆施:晚上我们要举行宴会欢迎他们。

主席:他(指柯)是主人了,我们这就告一段落,好吗?

凯塔;我们没有别的话。在北京已与许多负责同志讲过,现再一次向主席表示:几内亚对于中国的友谊和合作寄以极大的希望,对中国为几所作的一切表示感谢。这次谈判印象很深,中国政府领导人很谅解我们,谈判进行得顺利,得到了积极的成果。再一次表示两国关系是巩固的,代表几政府和人民向主席表示感谢。

主席:我们感谢你们,这是互相支持,我们很抱歉,不能完全满足你们的要求。

凯塔:你们已尽了你们的能力。非洲有句话:援助的方式比援助的东西更重要。

主席:我们的关系是平等的,友好、坦率、诚恳、不讲假话,讲老实话。以后继续往来。你们来过中国吗?

凯塔:他们都是第一次,我一九六零年陪杜尔总统一起来过,到过北京、武汉、广州、上海等地。主席在北京接见过我们。到上海时柯市长举行了宴会,我们还参观过上海汽轮机厂,宋庆龄副主席在上海接见了我们。

主席:你们回去后替我问候塞古.杜尔总统,问候你们中央的各位领导人,祝他们好。

凯塔:在我们离开这儿前,再一次祝主席身体健康,祝主席在建设社会主义的事业中取得新的更大成就。



\section[在北戴河中央工作会议上的讲话(一九六二年八月六日)]{在北戴河中央工作会议上的讲话}
\datesubtitle{(一九六二年八月六日)}


先开工作会议,为中央全会准备文件。从明天起,开始讨论,刘××建议成立核心小组,还有许多小组,解决六个大组不能畅所欲言的问题。核心小组有常委、书记处,再加大区第一书记,中央各口负责同志,共二十三人:毛、刘××、周、朱×、邓××、彭×、富春、先念、谭××、伯达、陆××、富治、谷牧、罗××、陈毅、杨××,加上各大区第一书记。

一、社会主义国家,究竟存在不存在阶级?在外国有人讲,没有阶级了,因此党是全民的党,不是阶级的工具,无产阶级的党了。无产阶级专政不存在了,全民专政没有对象了,只有对外矛盾了。像我们这样的国家是否也适应?这个问题是否谈一下。我同几个大区的同志都谈了话,了解到有的人听说,国内还有阶级存在,大吃一惊。资产阶级右派从来不承认有阶级存在,认为没有阶级了,不要改造,不承认阶级斗争,说阶级斗争是马克思捏造出来的。资产阶级不承认阶级斗争,孙中山就不讲阶级,说只有大贫小贫之分。有没有阶级,这是个基本问题。

二、形势问题,也要谈一下,国际问题要找几个人准备一下,究竟是什么情况?帝国主义、修正主义、反动的民族主义、广大人民群众各阶层、民族资产阶级、农民、城市小资产阶级……

国内形势谈一谈。究竟这两年如何?有什么经验?过去几年有许多工作没搞好,有许多还是搞好了,如工业建设、农业建设、水利等等。有人说,农村去年比前年好,今年比去年好,这个说法对不对?工业上半年不那样好,有主客观原因,下半年怎样,还要看一看。有些同志过去曾经认为是一片光明,现在是一片黑暗,没有光明了。是不是一片黑暗?两种看法那种对?如果都不对,是不是应有第三种看法?不是一片黑暗,基本光明,有黑暗,问题不少,确实很大。回到一九五九年庐山会议的三句话。“成绩很大,问题不少,前途光明”。两年调整,彻底调整、巩固、充实、提高的方针做得不那么好。以农业为基础,讲了三年,一九五九年至一九六二年,四个年头,实际上没有实行。中央的东西,有些没有下去,有些成了废品。所谓没有实行,就是没有认真做,个别做了,或者做得很不好。形势问题,我倾向于不那么悲观,不是一片黑暗。现在一片光明的看法没有了,不存在。有些人思想混乱,没有前途,丧失信心,不对。

三、矛盾问题。有些什么矛盾?一类是敌我矛盾,一类是人民内部矛盾。人民内部矛盾有两类。有一种矛盾,对资产阶级的矛盾,实质上敌对的,是社会主义与资本主义的矛盾,我们当作人民内部矛盾处理,如果承认国内阶级还存在,就应该承认社会主义与资本主义的矛盾是存在的。阶级的残余是长期的,矛盾也是长期存在的,不是几十年,我想是几百年,究竟那一年进入社会主义,进入了社会主义是不是就没有矛盾了?没有阶级,就没有马克思主义了,就成了无矛盾论,无冲突论了。现在有一部分农民闹单干,究竟有百分之几十,有说百分之二十,安徽更多。就全国来说,这时期比较突出。究竞走社会主义还是走资本主义道路?农村合作化要不要?“包产到户”还是集体化?已经“包产到户”的,不要强迫纠正,要做工作。为什么要搞这么多文件?为了巩固集体经济。现在就有闹单干之风,越到上层越大,有阶级就有阶层,地、富残余还存在着,闹单干的是富裕阶层、中农阶层、地富残余,资产阶级争夺小资产阶级闹单干,如果无产阶级不注意领导,不做工作,就无法巩固集体经济,就可能搞资本主义。有些人也是要闹单干的。

再有,生产和分配的矛盾,积累与消费的矛盾。积累过多,消费就少了。

再有,集中与分散的矛盾,七千人大会之后,我看没有解决,还要继续做工作,民主与集中的矛盾,要用民主的方法达到集中的目的。要让人家讲话,不民主,集中不起来,还要做工作。

社会主义与资本主义,本质上是敌对矛盾,我们当作人民内部矛盾来处理。积累过多,民主与集中还要做工作。阶级存在不存在?国内形势如何?矛盾,一个是敌我矛盾,一个是人民内部矛盾。敌我矛盾有个肃反问题,还有反革命存在,要看到,看不到不好,看得太严重也不合乎事实。


\section[在北戴河中央工作会议中心小组会上的讲话(一九六二年八月九日)]{在北戴河中央工作会议中心小组会上的讲话}
\datesubtitle{(一九六二年八月九日)}


今天单讲共产党垮得了垮不了的问题。共产党垮了谁来?反正两大党,我们垮了,国民党来。国民党干了二十三年,垮了台,我们还有几年。

农民本来已经发动起来,但是还有资产阶级、右派分子、地主、富农复辟的问题。还有南斯拉夫的方向。(有人插话:国民党在台湾搞了一个政纲,土地收为农民所有,但又保护地主)各地方、各部专搞那些具体问题,而对最普遍、最大量的方向问题不去搞。

单干势必引起两极分化,两年也不要,一年就要分化。

(李××同志揭露邓子恢的问题)派干部下去,而思想不“定一”,不讨论就走,这种办法不好。为什么不请邓子恢来?他不来,我们对台戏唱不成。建议中心小组再加五个人:邓子恢、王××、康生、吴××、胡×。

资本主义思想,几十年、几百年都存在,不说几千年,讲那么长吓人。社会主义才几十年,就搞得干干净净?历代都是如此。苏联到现在几十年,还有修正主义,为国际资本主义服务,实际是反革命。

《农村社会主义高潮》一书,有一段按语讲资产阶级消灭了,只有资本主义思想残余的影响,讲错了,要更正。

有困难,对集体经济是个考验,这是一种大考验,将来还要经受更重大的考验,苏联经过两次大战的大考验,走了许多曲折的道路,现在还出修正主义。我们的困难比苏联的困难更多。

全世界合作化,我们搞得最好。因为从全国说,土改比较彻底,但也有和平土改的地方。政权中混进了不少坏分子与马步芳分子。改变了生产资料所有制,不等于解决了意识的反映。社会主义改造消灭了剥削阶级的所有制,不等于政治上、思想上的斗争没有了。思想意识方面的影响是长期的。高级合作化、一九五六年社会主义改造,完成了消灭资产阶级的所有制,一九五七年提出思想政治革命,补充了不足。资产阶级是可以新生的,苏联就是这个情况。

苏联从一九二一年到一九二八年单干了近十年,没有出路,斯大林才提出搞集体化。一九三五年才取消各种购物券,他们的购物券并不比我们少。

找几个同志把苏联由困难到发展的过程,整理一个资料。这事由康生同志负责,搞一个经济资料。

动摇分子坚决闹单干,以后看形势不行又要求回来。最好不批准,让他们单干二、三年再说,他们早回来,对集体经济不会起积极作用。

要有分析,不要讲一片光明,也不能讲一片黑暗,一九六○年以来,不讲一片光明了,只讲一片黑暗,或者大部分黑暗。思想混乱,于是提出任务:单干,全部或者大部单干。据说只有这样才能增产粮食,否则农业就没有办法。包产百分之四十到户,单干、集体两下竞赛,这实质上叫大部分单干。任务提得很明确,两极分化,贪污盗窃,投机倒把,讨小老婆,放高利贷,一边富裕,而军、烈、工、干四属,五保(户)这边就要贫困。

赫鲁晓夫还不敢公开解散集体农场。

(康生同志插话:现在的价格,低出高进,不利于集体经济。)

内务部一个司长,到凤城宣传安徽包产到户的经验。中央派下去的人常出毛病,要注意。中央下去的干部,要对下面有所帮助,不能瞎出主意,不能随便提出个人意见。政策只能中央制定,所有东西都应由中央批准,再特殊也不能自立政策。

思想上有了分歧,领导要有个态度,否则错误东西泛滥。反正有三个主义:封建主义、资本主义、社会主义。资本主义有买办阶级,美国资本主义农场,平均每个场有十六户,我们一个生产队二十多户。包产到户,大户还要分家,父母无人管饭,为天下中农谋福利。

河北胡开明,有这么一个人,“开明”,但就是个“胡”开明,是个副省长。听了批评“一片黑暗”的论调的传达,感到压力,你压了我那么久,从一九六○年以来,讲两年多了,我也可以压你一下么。

有没有阶级斗争?广州有人说,听火车轰隆轰隆的声音,往南去的像是“走向光明”,“走向光明”,往北开的像是“没有希望”,“没有希望”。

有人发国难财,发国家困难之财,贪污盗窃。党内有这么一部分人,并不是共产主义,而是资本主义、封建主义。

每一个省都有那么一种地方,所谓后解放区,实际上是民主革命不彻底。

党员成分,有大量小资产阶级,有一部分富裕农民及其子弟,有一批知识分子。还有一批未改造过的坏人,实际上不是共产党。名为共产党,实为国民党。对这部分人的民主革命还不彻底,明显的贪污、腐化,这部分人好办。知识分子、地富子弟,有马克思主义化了的,有根本未化的,有的程度不好的。这些人对社会主义革命没有精神准备,我们没有来得及对他们进行教育。资产阶级知识分子,全部把帽子摘掉?资产阶级知识分子,阳过来,阴过去,阴魂未散,要作分析。

民主革命二十八年,在人民中宣传反帝、反封建,宣传力量比较集中,妇孺皆知,深入人心。社会主义才十年,去年提出对干部重新进行教育,是个重要任务。“六大”只说资产阶级不好,但是对资产阶级加了具体分析,反对的是官僚、买办资产阶级,对别的资产阶级就反得不多,三反五反搞了一下。没收国民党、大资本家、帝国主义的财产,这些拿到我们手上,就是社会主义性质,拿到别人手上是资本主义性质。一九五三、一九五四年搞合作社,开始搞社会主义。互助组、合作化、初级社、高级社,一直发展下来。真正社会主义革命是从一九五三年开始的。以后经过多次运动,社会主义建设与社会主义改造在全国展开。一九五八年已有些精神不对,中间有些工作有错娱,最主要的是高征购,瞎指挥,共产风,几个大办,安徽“三改”,引黄灌溉(本来是好的,不晓得盐碱化)。因此四个矛盾再加上一个矛盾,正确与错误的矛盾。高指标,高征购,这是认识上的错误,不是什么两条道路的问题。好人犯错误同走资本主义道路的完全不同,与混进来的及封建主义等更不相同。如基本建设多招了二千万人,没看准,农民就没有饭吃,就要浮肿,现在又减人。

有些同志一有风吹草动,就发生动摇,那是对社会主义革命没有精神准备,和没有马克思主义。没有思想准备,没有马列主义,一有风就顶不住。对这些人应让他们讲话,让他们讲出来,讲比不讲好,言者无罪,但我们要心中有数,行动要少数服从多数,要有领导。××同志的报告中说:“要正确处理单干,纪律处分,开除党籍……”。我看带头的可处分,绝大多数是教育问题,不是纪律处分,但不排除对带头搞分裂的纪律处分。

大家都分析一下原因。

这是无产阶级和富裕农民之间的矛盾。地主、富农不好讲话,富裕农民就不然,他们敢出来讲话。上层影响要估计到。有的地委、省委书记(如曾希圣),就要代表富裕农民。

要花几年功夫,对干部进行教育,把干部轮训搞好,办高级党校,中级党校,不然搞一辈子革命,却搞了资本主义,搞了修正主义,怎么行?

我们这政权包了很多人下来,也包了大批国民党下来,都是包下来的。

罗隆基说,我们现在采取的办法,都是治标的办法。治本的办法是不搞阶级斗争。我们要搞一万年的阶级斗争,不然,我们岂不变成国民党、修正主义分子了。

和平过渡,就是稳不过渡,永远不过渡。

我在大会上只出了个题目,还没有讲完,有的只露了一点意思,过两天可能顺成章。

三年解放战争,猛烈土地改革。土改后,对两种资本主义的改造很顺利。有的地区的民主革命还是不彻底,比如潘汉年、饶漱石,长期没发现。

修正主义的国内根源是资本主义残余,国外是屈从帝国主义的压迫,莫斯科宣言上这两句话是我加的。

一九五七年国际上有一点小风波,风乍起吹皱一池春水。六月刮起十二级台风,他们准备接管政府,我们来个反攻,所有学校的阵地都拿过来了。反右后,五八年算半年,五九年、六○年大跃进。六一年开始搞十二条,六○年搞工业七十条,农业六十条。

过去分田是农民跟地主打架,现在是农民跟农民打架,强劳动力压弱劳动力。

有这样一种农民,两方面都要争夺,地富要争夺,我们要争夺。

小资产阶级、农民有两重性,碰到困难就动摇,所以我们要争夺无产阶级领导权,就是共产党领导。农村的事,依靠贫农,还要争取中农,我们是按劳分配,但要照顾四属、五保。

二千万人呼之则来,挥之则去。不是共产党当权,哪个能办到。五八年十一月第一次郑州会议,提出的商业政策,没执行,按劳分配的政策,也不执行,不是促进农业,集体经济的发展,反而起了不利的影响。商业部应该改个名字,叫“破坏部”,同志们听了不高兴,我故意讲得厉害一点,以便引起注意。商业政策、办法,要从根本上研究。这几年兔、羊、鹅有发展,这是因为这几样东西不征购。打击集体,有利单干,这次无论如何得解决这个问题。

中央有事情总是同各省、市和各部商量,可是有些部就是不同中央商量,中央有些部作得好,像军事、外交,有些部门像计委、经委,还有财贸办、农业办等口子,问题总是不能解决。中央大权独揽,情况不清楚,怎样独揽?人吃了饭要革命,不一定要在一个部门闹革命,为什么不可以到别的部门或下面去革命呢?我是湖南人,在上海、广州、江西七、八年,陕北十三年。不一定在一个地区干,永远如此。中央、地方部门之间,干部交流,再给试一年,看能否解决,陈伯达同志说不能再给了。

财经各部委,从不做报告,事前不请示,事后不报告,独立王国,四时八节,强迫签字,上不联系中央,下不联系群众。

谢天谢地,最近组织部来了一个报告。

外国的事我们都晓得,甚至肯尼廸要干什么也晓得,但是北京各个部,谁晓得他们在干些什么?几个主要经济部门的情况,我就不知道。不知道,怎么出主意?据说各省也有这个问题。



\section{对中共中央组织部的批评(一九六二年八月十二日)}


中共中央组织部从来不向中央作报告,以至中央同志对组织部同志的活动一无所知,全部封锁,成了一个独立王国。



\section[在八届十中全会上的讲话(一九六二年九月二十四日上午怀仁堂)]{在八届十中全会上的讲话}
\datesubtitle{(一九六二年九月二十四日上午怀仁堂)}


现在是十点,开会。

这次中央全会解决了几个重大问题:一是农业问题;二是商业问题,这是两个重要问题,还有工业问题,计划问题,这是第二位的问题,第三个是党内团结问题。有几位同志讲话,农业问题由陈伯达同志说明,商业问题由李先念同志说明,工业计划问题由李富春、薄××说明。另外,还有监察委员会扩大名额问题,干部上下左右交流问题。

会议不是今天开始的,这个会开了两个多月了,在北戴河开了一个月,到北京差不多也是一个月。实际问题在八、九两月,各个小组(在座的人都参加了)经过小组,实际上是大组,都讨论清楚了,现在开大会不需要很多时间了,大约三天就够了,二十七号不够就开到二十八号,至迟二十八号要结束。

我在北戴河提出三个问题:阶级、形势、矛盾。阶级问题,提出这个问题,因为阶级问题没有解决。国内不要讲了。国际形势,有帝国主义、民族主义、修正主义存在,那是资产阶级国家,是没有解决阶级问题的,所以我们有反帝任务,有支持民族解放运动的任务,就是要支持亚、非、拉三大洲广大的人民群众,包括工人、农民、革命的民族资产阶级和革命的知识分子。我们要团结这么多的人,但不包括反动的民族资产阶级,如尼赫鲁,也不包括反动的资产阶级知识分子,如日共叛徒春日庄次郎,主张结构改革论,有七、八个人。

那末,社会主义国家有没有阶级存在?有没有阶级斗争?现在可以肯定,社会主义国家有阶级存在,阶级斗争肯定是存在的。列宁曾经说,革命胜利后,本国被推翻的阶级,因为国际上有资产阶级存在,国内还有资产阶级残余,小资产阶级的存在,不断产生资产阶级,因此,被推翻了的阶级还是长期存在的,甚至要复辟的。欧洲资产阶级革命,如英国、法国等都曾几次反复。社会主义国家也可能出现这种反复,如南斯拉夫就变质了,是修正主义了,由工人、农民的国家变成一个反动的民族主义分子统治的国家。我们这个国家就要好好掌握,好好认识,好好研究这个问题。要承认阶级长期存在,承认阶级与阶级斗争,反动阶级可能复辟。要提高警惕,要好好教育青年人,教育于部,教育群众,教育中层和基层干部,老干部也要研究,教育。不然,我们这样的国家还会走向反面。走向反面也没有什么要紧,还要来个否定的否定,以后又会走向反面。\marginpar{\footnotesize 34}如果我们的儿子一代搞修正主义,走向反面,虽然名为社会主义,实际是资本主义,我们的孙子肯定会起来暴动的,推翻他们的老子,因为群众不满意。所以我们从现在起就必须年年讲,月月讲,天天讲,开大会讲,开党代会讲,开全会讲,开一次会就讲,使我们对这个问题有一条比较清醒的马克思列宁主义的路线。

国内形势:过去几年不大好,现在已经开始好转。一九五九年、一九六〇年,因为办错了一些事情,主要由于认识问题,多数人没有经验。主要是高征购,没有那么多粮食,硬说有。瞎指挥,农业、工业都有瞎指挥。还有几个大办的错误。一九六〇年下半年就开始纠正。说起来就早了,一九五八年十月第一次郑州会议开始了,然后十一月、十二月武昌会议,一九五九年二、三月第二次郑州会议,然后四月上海会议,就注意纠正。这中间,一九六〇年有一段时间对这个问题讲的不够,因为修正主义来了,压我们,注意反对赫鲁晓夫了。从一九五八年下半年开始,他就想封锁中国海岸,要在我们国家搞共同舰队,控制沿海,要封锁我们。赫来我国就是为了这个问题。然后是一九五九年九月中印边界问题,赫支持尼攻击我们,塔斯社发表声明。以后赫来,十月在我国国庆十周年宴会上,在我们讲坛上攻击我们。然后一九六〇年布加勒斯特会议围剿我们。然后两党会议,二十六国起草委员会,八十一国莫斯科会议,还有一个华沙会议,都是马列主义与修正主义的争论,一九六〇年一年,与赫打仗。你看,社会主义国家,马列主义中出现这样的问题,其实根子很远,事情很早就发生了,就是不许中国革命。那是一九四五年,斯大林就是阻止中国革命,说不能打内战,要与蒋介石合作,否则中华民族就要灭亡。当时我们没有执行,革命胜利了。革命胜利后,又怀疑中国是南斯拉夫,我就变成铁托。以后到莫斯科,签订中苏同盟互助条约,也是经过一场斗争的,他不愿签,经过两个月的谈判最后签了。斯大林相信我们是从什么时候起呢?是从抗美援朝起。一九五〇年冬季,相信我们不是铁托,不是南斯拉夫了。但是,现在我们又变成“左倾冒险主义”、“民族主义”、“教条主义”、“宗派主义”者了。而南斯拉夫倒变成“马列主义”者了。现在南斯拉夫可行啊,吃得开了,听说变成了“社会主义”,所以社会主义阵营内部也是复杂的,其实也是简单的。道理就是一条,就是阶级斗争问题。无产阶级与资产阶级的斗争问题,马列主义与反马列主义的斗争问题,马列主义与修正主义之间的斗争的问题。

至于形势,无论国际、国内都是好的。开国初期,包括我在内,还有刘××,曾经有这个看法,认为亚洲的党和工会、非洲党,恐怕受摧残。后来证明,这个看法是不正确的,不是我们所想的。第二次世界大战后,蓬蓬勃勃的民族解放斗争,无论亚洲、非洲、拉丁美洲都是一年比一年地发展的。出现了古巴革命,出现了阿尔及利亚独立,出现了印尼亚洲运动会、几万人示威,打烂印度领事馆,印度孤立,西伊里安荷兰交出来了,出现了越南南部的武装斗争,那是很好的武装斗争,出现了苏伊士运河事件,埃及独立,阿联偏右,出现了伊拉克,两个都是中间偏右的,但它反帝。阿尔及利亚不到一千万人口,法国八十万军队,打了七、八年之久,结果阿尔及利亚胜利了。所以,国际形势很好。陈毅同志作了一个很好的报告。

所谓矛盾,是我们同帝国主义的矛盾,全世界人民同帝国主义的矛盾,是主要的。各国人民反对反动资产阶级,各国人民反对反动的民族主义,各国人民同修正主义的矛盾,帝国主义国家之间的矛盾,民族主义国家与帝国主义国家之间的矛盾,帝国主义国家内部的矛盾,社会主义与帝国主义之间的矛盾。中国的右倾机会主义,看来改个名字好,叫做中国的修正主义。从北戴河到北京的两个月会议,是两种性质的问题,一种是工作问题,一种是阶级斗争的问题,就是马克思主义与修正主义的斗争。\marginpar{\footnotesize 35}工作问题也是与资产阶级思想斗争的问题。工作问题有几个文件,有工业的、农业的、商业的等,有几个同志讲话。

关于党如何对待国内、党内的修正主义问题,资产阶级问题,我看还是照我们原来的方针不变。不论犯了什么错误的同志,还是一九四二年到一九四五年整风时的那个路线,只要认真改变,都表示欢迎,就要团结他,要团结,治病救人,惩前毖后,团结——批评——团结。但是,是非要搞清楚,不能吞吞吐吐,敲一下吐一点,不能采取这样的态度。为什么和尚念经要敲木鱼?《西游记》里讲,取回的经被黑鱼精吃了,敲一下吐一个字,就是这么来的。不要采取这种态度,和黑鱼精一样,要好好想想。犯了错误的同志,只要认识错误,回到马克思主义的立场方面来,我们就与你团结。在座的几位同志,我欢迎,不要犯了错误见不得人。我们允许犯错误,你已经犯了嘛!也允许改正错误。不要不允许犯错误,不许改正错误。有许多同志改的好,改好了就好嘛!李××的发言就是现身说法。李××的错误改了,我们信任嘛!一看二帮,我们坚决这样做。还有很多同志,我也犯过错误,去年我就讲了,你们也要允许我犯错误,允许我改正错误,改了,你们也欢迎。去年我讲,对人是要分析的,人是不能不犯错误的。所谓圣人,说圣人没有缺点是形而上学的观点,而不是马克思主义、辩证唯物主义的观点。任何事物都是可以分析的,我劝同志们,无论是里通外国的也好,搞什么秘密反党小集团的也好,只要把那一套统统倒出来,真正实事求是讲出来,我们就欢迎,还给工作做,决不采取不理他们的态度,更不采取杀头的办法。杀戒不可开,许多反革命都没有杀,潘汉年是一个反革命嘛,胡风、饶漱石也是反革命嘛,我们都没有杀嘛。宣统皇帝是不是反革命?还有王耀武、康泽、杜聿明、杨广等战犯,也有一大批没杀。多少人改正了错误就赦免他们嘛,我们也没有杀。右派改了的摘了帽子嘛。近日平反之风不对,真正错了才平反,搞对了不能平反。真错了的平反,全错全平反,部分错了部分平反,没有错的不平反,不能一律都平反。

工作问题,还请同志们注意,阶级斗争不要影响了我们的工作。一九五九年第一次庐山会议本来是搞工作的,后来出了彭德怀,说:“你操了我四十天娘,我操你二十天娘不行?”这一操,就被扰乱了,工作受到影响。二十天还不够,我们把工作丢了。这次可不能,这次传达要注意,各地、各部们要把工作放在第一位,工作与阶级斗争要平行,阶级斗争不要放在很突出的地位(抄者按:此指对敌斗争)。现在已经组成两个专案审查委员会,把问题搞清楚,潘汉年是一个反革命嘛!胡风、饶漱石也是反革命嘛,我们也没有杀嘛。不要因阶级斗争干扰我们的工作,等下次或再下次全会再未搞清楚。把问题说清楚,要说服人。阶级斗争要搞,但要有专门人搞这个工作。公安部门是专搞阶级斗争的,它的主要任务是对付敌人的破坏。有人搞破坏工作,我们开杀戒,只是对那些破坏工厂,破坏桥梁,在广州边界搞爆破案,杀人放火的人。保卫工作要保卫我们的事业,保卫工厂、企业、公社、生产队、学校、政府、军队、党、群众团体,还有文化机关,包括报馆、刊物、新闻社。保卫上层建筑。

现在不是写小说盛行吗?利用写小说搞反党活动,是一大发明。凡是要想推翻一个政权,先要制造舆论,要搞意识形态,搞上层建筑,革命如此,反革命也如此。我们的意识形态是搞革命的,马克思的学说,列宁的学说,马列主义普遍真理和中国革命具体实践相结合。结合的好,问题就解决的好些,结合的不好就失败受挫折。讲社会主义建设时,也是普遍真理与建设相结合,现在是结合好了还是没有结合好?我们正在解决这个问题。军事建设也是如此。如前几年的军事路线与这几年的军事路线就不同。叶剑英同志搞了部著作,很尖锐,大关节是不糊涂的,我一向批评你不尖锐,这次可尖锐了。\marginpar{\footnotesize 36}我送你两句话:“诸葛一生唯谨慎,吕端大事不糊涂。”

请邓××宣布那几个人不参加全会。政治局常委决定五人不参加。

(×××宣布:政治局常委决定五个同志不参加全会:彭、习、张、黄、周,是被审查的主要分子,在审查期间,没有资格参加会议。)

因为他们的罪恶实在太大了,没有审查清楚以前,没有资格参加这次会议,也不参加重要会议,也不要他们上天安门。主要分子与非主要分子要有分析,是有区别的。非主要分子今天参加了会议。非主要分子彻底改正错误,给他们工作。主要分子如果彻底改正错误,也给工作。特别寄希望于非主要分子觉悟,当然也希望主要分子觉悟。

从现在起以后要年年讲阶级斗争,月月讲,开大会讲,党代会要讲,开一次会要讲一次,以使我们有清醒的马列主义的头脑。

一九五九年八届九中全会胜利粉碎了彭反党集团向党的进攻。十中全会又一次揭露彭反党活动一高饶反党分子成员习仲勋。


官僚主义小则误国误民,大则害国害民。

第一、不调查,不研究,脱离实际,脱离群众的官僚主义。

第二、主观瞎指挥的官僚主义。

第三、忙忙碌碌,不抓政治,迷失方向的官僚主义。

官僚主义发展严重了,一种革命意志衰退、腐化、堕落。一种是互相勾结,敌我不分,官僚主义是修正主义的温床。

治官僚主义的办法:接触群众,接触实际。

三自:固步自封,骄傲自满,夜郎自大。

一高:高官厚禄。

一爱:爱形而上学,爱好资产阶级思想方法。

爱好形而上学,缺乏两分法,这表现在爱讲成绩,不讲缺点,爱听表扬,不爱听批评,老虎屁股摸不得。

遇事不做全面分析,扶得东来西又倒。


\section[在八届十中全会上对工业支援农业的指示(一九六二年七月二十四日)]{在八届十中全会上对工业支援农业的指示}
\datesubtitle{(一九六二年七月二十四日)}


〔在谈到坚持集体化方向,工业支援农业时说〕:搞了四年,以农业为基础的方针不大自觉,不大愿意。巩固集体经济有两方面:一是政策,二是支援农业。从长远来说是农业的技术改造,总的是如何搞社会主义建设的问题。

科学的研究,没有抓农业。科学院党组书记说:科学院是搞尖端的。

要抓农业技术改造。

民主与集中,集中与分散,分散主义首先是中央(指综合部门)。农业机械化要搞个文件,二十五年左右实现机械化,同时实现工业化。有的同志在困难面前躺下。是消极好,还是鼓足干劲好?是单干,还是集体好,要提起注意。

对困难发生消极情绪是不是思想问题?

计委、经委、工交各部要加强支援农业。要抓紧改,这是机会,还来得及。

计委搞两个文件:支援农业的报告,三个部分:方向,面向农业,支援农业的方针。



\section[关于电台的指示(一九六二年十月一日)]{关于电台的指示}
\datesubtitle{(一九六二年十月一日)}


中近东许多国家发生政变。搞政变的人开始就要夺取电台,向全国和全世界说话,原政府的声音人们就听不到了。我们的电台怎么样?是否掌握在可靠的人手里?要从部队调一个强的干部去。


\section[批评新华社(一九六二年)]{批评新华社}
\datesubtitle{(一九六二年)}


《内部参考》登那么多包产到户的材料是错误的,今后不要再登。办内部参考要有方向。


\section[听取中印边境自卫反击战汇报时的指示(一九六三年二月)]{听取中印边境自卫反击战汇报时的指示}
\datesubtitle{(一九六三年二月)}


看来我们的军队还是要政治工作,抓四个第一,抓三大民主,加强薄弱环节,搞好党的建设。



\section[同苏修大使契尔沃年科的谈话(一九六三年二月二十三日)]{同苏修大使契尔沃年科的谈话}
\datesubtitle{(一九六三年二月二十三日)}


(苏修大使契尔沃年科求见主席,主席称不适。因再三要求,主席着睡衣接见。)
\begin{duihua}
\item[\textbf{契:}] 听说你们要发表文章,不必要的,感到很沉重。

\item[\textbf{主席:}] 不必要,你们为什么发表那么多文章?没有什么沉重的,不过互相论战,不过是唇枪舌箭而已。

\item[\textbf{契:}] 请你到莫斯科去谈一谈,可以吗?

\item[\textbf{主席:}] 我已经老了,不中用了。老而不死,破套鞋去不成了。
\end{duihua}


\section[接见阿尔巴尼亚劳动青年联盟代表团、新闻工作者代表团、工会代表团和档案工作者代表团的谈话(一九六三年五月四日)]{接见阿尔巴尼亚劳动青年联盟代表团、新闻工作者代表团、工会代表团和档案工作者代表团的谈话}
\datesubtitle{(一九六三年五月四日)}


主席:你们到中国有多久了?是一起来的吗?

各代表团团长:新闻代表团四月二十一日到中国,劳动青年联盟代表团和工会代表团都是四月二十六日,档案代表团四月三日。

主席:你们的党是很好的党,你们的国家是很好的国家。我们两党、两国的关系也很好。我们两党、两国,修正主义都攻击我们。对修正主义的攻击要分析:第一是坏,攻击我们当然不好,第二也好;修正主义骂我们,这对我们有好处,修正主义不骂我们就不好了。修正主义不援助我们,经济上不援助我们,撤走专家,这也要分析,第一是不好,撤走专家,不帮助了,当然不好;但也是好事,让我们自己干,自力更生,对吗?要自力更生,要不怕困难。

你们到过××吗?有机会最好到那个国家去看看,不一定就是你们去。……他们建立了重工业,他们自力更生。我们的农业还没有过关,他们过关了,他们的农业比我们好。我们的朋友不少,帝国主义、反动派、修正主义要孤立我们,但孤立不了。

我们是多数,他们没有多少人,多数人民并不赞成帝国主义、反动派和修正主义,各个修正主义领导国家的人民,并不见得很欢迎修正主义。这方面的情况你们知道一些。人民是好的,包括修正主义领导国家的人民,干部中也不都是坏的,不是一块铁板,并不都是修正主义者,是修正主义领导集团坏。

你们在中国还有多久?

各代表团团长:工会代表团还有一个月,新闻代表团和青年联盟代表团还有二十天,档案代表团六月七日走。

主席:你们可以到处走走。前几年我们的情况不好,最近一、二年比较好一些,现在政治、经济情况都比较更好些了。但还有困难,困难不少。社会上和党内还有些问题。困难可以克服,正在克服中,问题在解决。

社会主义国家经常会生长资本主义因素。有些共产党员挂了党员的招牌,实际上是资产阶级分子,这不是多数,但有一部分是如此,搞投机倒把、贪污盗窃、铺张浪费。浪费问题很大,这个问题如能适当解决,就可以搞几十亿美元。美国人不会借款给我们,社会主义国家也不借给我们,我们有个办法,就是向官僚主义者借款。(外宾笑了)

马马基(新闻工作者代表团团长):美国借款给他们——所谓建设社会主义的南斯拉夫。

主席:你们有一个“好”邻居(指南斯拉夫),他们锻炼你们,他们反对你们,南斯拉夫和其他资本主义国家包围着你们。

马马基:但是我们的朋友很多,我们的人民很坚强,因为和中国人民在一起。

主席:即使中国是资本主义国家,你们也不会灭亡。

卢鲍尼亚(劳动青年联盟代表团团长):但我们感到幸运,中国给了我们国际主义的援助,鼓舞了我们。

主席:但是援助很少,有些科学技术关键问题,我们自己也没有解决,要从资本主义国家购买技术,所以不能完全满足你们的要求,我们过意不去。再过五年到十年,我们会好一些。

马马基:修正主义企图破坏我们的五年计划,而你们是帮助我们。

主席:你们原定的五年计划有没有修改?

吉贝罗(工会代表团团长):有的,一般说项目有些增加,大型项目有增加。

马马基:五年计划有些改动,但总的说增加的比减少的多。去年冬季的气候对农业的影响很大,雨下得多了,涝了。

主席:春季生产怎样了?

马马基:直到三月还在下雨。

主席:你们雨下多了,我们有些地方少了。

马马基:如果有可能,我们可以把我们的雨送一些给你们。(全场笑)

卢鲍尼亚:沈阳很久没有下雨,我们劳动青年联盟代表团一到就下雨了。

主席:很好。你们可以到郊外去看看。这里(指上海)春季生产还可以,有麦、油菜、蚕豆和豌豆。

卢鲍尼亚:我们从北京坐火车到上海,看到一路上庄稼长得很好,尤其是从山东到上海这一段,很深刻的印象是农民没有荒废一寸土地。

主席:这是因为我们土地不够,平均每人只有五分之一公顷,长江以南每人只有十五分之一公顷。我们是寸土必争,否则没有饭吃。这里一年要种两次,冬季种麦,夏季种稻。南方有的地方不种麦,种两季稻子,早稻和晚稻。东北只有一季麦子,只有一百二十天无霜期。这些地方(指上海附近)比较好。上海是北纬三十一度,你们在北纬多少度?

卢鲍尼亚:我们和北京一样,纪诺卡斯特城(霍查同志的故乡)和北京在一个北纬度。

主席:北京是北纬四十度。请你们问候霍查同志和谢胡、卡博、巴卢库、阿里雅和凯莱齐等同志。这些同志我都认识。

卢鲍尼亚:一定转达您的问候。我们出国时,同志们也一再要我们向毛主席致最衷心的问候,祝您身体健康。

主席:谢谢。

卢鲍尼亚:您身体健康,这不仅有利于中国人民,而且有利于国际共产主义运动。

主席:帝国主义、反动派、修正主义骂我们,我们要和全世界百分之九十以上的人民一起,反对他们。团结百分之九十以上的人民,难道还孤立吗?印度还是反动派统治,但印度人民对我们很好。

卢鲍尼亚:我们有些同志过去来过中国,这次再来,看到中国发生了巨大的变化,感到无限鼓舞。我一九五七年来过中国,这次再来,北京不认识了。十大工程是世界上绝妙的杰作,只有像中国人民这样有才能的人才建造得出来。也看到上海的工业展览会,使我们十分惊奇,生产出很精密的仪器,还有重工业,这使我们十分鼓舞,欢欣愉快。我们在以霍查同志为首的阿尔巴尼亚劳动党的领导下,正在进行反对帝国主义和反对修正主义的斗争,你们党以自己的英雄气概鼓舞了我们。我们劳动党和霍查同志也一再指示我们要向你们学习。我们劳动青年联盟代表团和中国青年的会面,使我们受到了革命的鼓舞。

主席:不是一国支持一国,而是互相支持。不是一国帮助一国,而是互相帮助。有了你们站起来反对修正主义,全世界人民都高兴,不仅中国人民。

卢鲍尼亚:这是我们的国际主义义务。

主席:都是国际主义义务,你们是国际主义义务,我们也是国际主义义务。国际主义应当如此,应当坚持马列主义,应当互相支持。

你们今天还有什么活动吗?

阿外宾:主席的接见就是最重要的节目。

主席:今天谈得不多,以后还会有机会的。

卢鲍尼亚:主席接见我们,不仅是给我们,而且是给我们阿尔巴尼亚人民、劳动党和青年的荣誉,是我们一生中最难忘的大事。

主席:我要谢谢你们来看我。



\section[对四个文件的批示(一九六三年五月八日)]{对四个文件的批示}
\datesubtitle{(一九六三年五月八日)}


这几个文件很好,看到了问题,抓起了工作,正确地解决了大量的人民内部的矛盾和敌我之间的矛盾,政策和方针都是正确的,因而大大地推动了农业生产。可以作为各省、地、县、社进行社会主义教育工作的光辉榜样。应当组织干部学习这些文件。中央、各中央局、各省、市、区党委,都需要收集这种又有原则、又有名有姓、有事情、有阶段、有过程、有结论的文件。请你们注意这件大事,认真调查研究,是为至要。



\section[对东北和河南两件报告的批示(一九六三年五月八日)]{对东北和河南两件报告的批示}
\datesubtitle{(一九六三年五月八日)}


宋××同志报告一分,河南省委报告一分,都可以供各地同志参考。河南报告说明,他们在中央二月会议以前是没有根据十中全会指示的精神,认真地进行社会主义教育工作的,或者是没有抓住问题的要点,没有采用适当的方法。二月会议以后,他们抓起了这个工作,并且抓住了问题的要点,采取了适当的方法。第一步,只用了二十几天的时间就训练了十五万多干部。第二步,还要训练一百五十万干部和贫下中农积极分子。然后才是全面铺开,作为第三步。在采取头两个步骤时,并经过试点。这种分步骤的进行工作并经过试点的方法,是正确的。报告所说的其他各项政策也是对的。总之,必须团结绝大多数(百分之九十五以上)的干部和群众,适当地解决人民内部矛盾,即解决程度不同的不正常的干群关系问题,组成有领导的广大干群队伍,以便一致对敌。对坏人坏事,也要有分析。轻重不同,处理的方法也不同。必须以教育为主,以惩办为辅。真正要惩办的,是群众和领导都认为非惩办不可的极少数人。宋××同志所讲的用讲村史、家史、社史的方法教育青年群众这件事,是普遍可行的。社会主义教育是一件大事,请你们检查一下自己在这方面的认识和工作,检查一下是不是抓住了要点和采取的方法是否适当,查一查是否还有很多的地、县、社没有抓住这方面的工作。如果有的话(看来一定是有的),应当在农忙间隙,在不误生产的条件下,抓住进行。上半年作不完,可以在下半年作。同年作不完,可以在明年作。特别要注意分步骤的方法,试点的方法和团结大多数、孤立极少数的政策。


\section[在关于四清运动中央工作会议上的讲话(一九六三年五月)]{在关于四清运动中央工作会议上的讲话}
\datesubtitle{(一九六三年五月)}


先看二十个材料引起大家讨论,先看三天。各中央局、省开会也是如此,不要传达中央文件有一个框框。不要性急,横竖准备搞它一年、两年,两年搞不完搞三年。这样一个大的运动需要时间,不要性急。

这个革命运动是土改以来第一次最大的斗争。这样全面,这样广,这样深远是几年没有的。三反五反搞城市,一九五七年反右是思想战线上,反高饶是党内的。全面的、党内党外的这样的阶级斗争是十几年来没有的。这次从党内到党外,而上到下,从干部到群众,这样理解有好处。这是土改以来第一次大斗争。开始要训练县以上的干部队伍,再训练大队以上的干部,还有训练生产队干部和贫下中农积极分子。

没有蚂蚁的地区不要硬去找蚂蚁。譬如一类社队过去进行了阶级斗争,进行了社会主义教育,你一定去找地富活动,没有一例外也不好。

人民内部矛盾是大量的,差不多都有,大的小的。河南材料说到一个支部很好,另一材料说的一个公社干部经过洗手洗澡真正一尘不染的只有两个人,不能说这个支部不好,还是百分之九十五以上。现在看来我们干部真正一尘不染是有,但不能说太多。铺张浪费、多吃多占,一点没沾上的少,大多数沾了,洗手洗澡交待就好了吗?这次“四清”“五反”大家都出点汗,洗温水澡,轻松愉快,才能轻装上阵,一致对敌。为什么轻松愉快?一致对敌,我们身上不干净没有力量,搞干净就能团结一致对敌。有的干部多吃多占,有的和地富女儿勾搭,不洗就不能对敌。有些人对敌斗争有劲,对人民内部矛盾就不大积极,有顾虑。

解决人民内部矛盾,多吃多占,上了当的,只要自己说出来,又退了赃,不算贪污分子。将来机关、工厂、企业也可以这样办。当场宣布不算贪污分子,可不公布姓名。东北局有几个贪污一百元、二百元,自己讲了,开大会宣布不算贪污。至于贪污大的案处理了,大约不过上万元,如自己交待了又退了,处理可以减轻。既要有严肃性,又要有政策性,“四清”“五反”一定要有,不反不行。一定要交待清楚,不退赃物赃款不行。但要退得合情合理,多吃多占的退的时候,不要退得太挖苦了。使干部生活过不下去也不好,有的已吃了用了,教育他向群众检讨一下退出若干,参加劳动,这样群众不会叫一次退出,分期分批退,不至使生活不好过。还可以采取自报公议。这个政策很复杂,看起来自己好。

这次运动中间要换一批。劳动好的人,看来是少数。处分的,也是少数。议论干部受处分的可能不到百分之一,不要太多了,要多做教育工作。加强运动的领导,有的时候须要靠各地区县社队广大干部,上面去的人不要包办代替,要把广大干部发动起来,要依靠广大干部去搞。用这种方法——自我教育方法,发动广大干部方法,来的力量大。

一个坚决把运动搞起来,一个怕搞乱了。

(××同志讲:我懂主席心情,第一要搞,第二要搞好。经常向主席反映情况,得到主席指示,不要搞乱了。)

三个伟大革命斗争,不搞好不行,定要搞好。

注意总结经验,回去中央局开十天会,搞一个月工作,到七月中央局再开一次会,总结一下经验,摸一下情况,到七月底八月中央开会,除这个外还要搞工业。

要有强的领导才能发动运动,分期分批,一批批搞不算落后。这次运动将要大大提高各地的自觉性,中央局、省、市、县的人下去一起来运动。

四清历来不清,阶级斗争粗糙,这次运动,要提高自觉性,要忠心诚恳地帮助社队工作搞好,帮助干部洗温水澡,帮助四清搞好,除了个别不行的,烂了,蜕化变质的帮不上去,或太坏了,要派工作队代替他们搞,除此要诚心诚意地帮助他们搞。

你们对干部怎么样?我不清楚。现在看起来对干部要说服教育,特别是用实际证据来说服教育。照理说有,拿出实际证据来说,也有阶级斗争实际证据,昔阳县实际征据,浙江参加劳动,四个好文件是实际证据。检查一下我们是否照理说多,证据说比较少。

你们有机会去一个区搞十几天(我们说:没有),你们下去干部是否很紧张,熟了就不很紧张,多尊重人家,不要指手划脚,“三不”,对干部我们要团结他们,要洗手洗澡,要抓一下。

这次运动会出现杀人灭迹。

发动群众搞四清是厉害的事情,河北经验,有些公安机关搞不清,发动群众四清搞出来了。有人讲阶级斗争靠公安部门搞,人民内部靠监委搞,当然要靠,但除此还要充分的发动群众,依靠群众。

这个运动抓起纲领就好办了,分期分批搞,搞第二、第三批不算不名誉,还是名誉。

(众议:有地方走过场,雨过地皮湿)走了过场再搞嘛,就是不要伤人,不是敌人当敌人,不是……。

(大家议:乱子一点不出不行,主席同意我们的看法。十九日大厦跳楼死人,黑龙江有个地富杀死三十八人,去年枪毙了反革命十三人。)上海死一人,在厕所吊死,留字“过路君子”说好为他申寃,根本没斗他就死了。

要坚持说服教育,分期分批试点,划清界线,团结百分之九十五以上的群众和干部,有强的领导,只要有几条搞好就可少出乱子。

不打无准备之仗,材料没准备,兵没练好,不要搞。这一仗是全国性的革命运动,要像解放战争时来打仗,辽沈战役、锦州、淮海、过长江战役。不要打百团大战,不要像皖南事变那样打法。

第二是解放战争几个战役取得了全国性的胜利,这次打仗打好了是全国革命的胜利,对世界革命贡献更大了。


\section[关于《山西省昔阳县干部参加劳动已形成社会风尚》一文的批语(一九六三年五月)]{关于《山西省昔阳县干部参加劳动已形成社会风尚》一文的批语}
\datesubtitle{(一九六三年五月)}


《山西省昔阳县干部参加劳动已形成社会风尚》一文和省委的批语都很好,一并发给你们参考。干部参加劳动,是党的优良传统之一,是党在社会主义建设时期的一项极为重要的政策。认真贯彻执行这项政策,对于农村工作来说,其重要性是很明显的。农业合作化以来的无数事例证明:凡是办得好的社、队,无例外的都具备有社、队的领导干部经常和社员在一起积极参加劳动的特点。反之,凡是办得不好的社、队,往往具有一个相反的特点,即这些社、队的领导干部不愿意和社员在一起积极参加劳动,因而脱离群众,不能抵抗剥削阶“思想的侵袭,生活特殊化,贪污、多占群众的劳动果实,有的甚至逐步蜕化变质,堕落成为富裕农民和资本主义分子利益的代言人,修正主义的社会基础。

人民公社工作条例(修正草案)对于人民公社各级干部参加劳动问题,已经作出明确规定,可是直到现在,不少地方还没有认真贯彻执行。有的县委和公社党委对这一规定的重大意义认识不足,甚至认为大队和生产队干部的补贴工分不得超过生产队工分总数百分之二的规定根本行不通。应该请他们好好读一读昔阳县的经验。昔阳县的经验证明了:这项政策能否得到正确执行的根本关键,恰恰在于县委和公社党委是否有决心,是否以身作则。这个县的县、社两级干部一九六二年在生产队作的劳动日,县级每人平均六十二个,公社级每人平均八十二个。他们到那里下乡工作,就在那里参加劳动,并且一直坚持不懈,经过几年的努力,才逐步形成风气。应该说,昔阳县的同志们能够这样做,所有各县也可以这样做的。

中央要求各省、市、自治区党委、地委,要认真帮助县委弄通道理,结合整风、整社工作,把人民公社工作条例关于干部参加劳动和补贴工分的规定,抓紧加以解决,以利于人民公社的巩固和健全发展。



\section[对《浙江省七个关于干部参加劳动的好材料》的批示(一九六三年五月九日)]{对《浙江省七个关于干部参加劳动的好材料》的批示}
\datesubtitle{(一九六三年五月九日)}


浙江省这七个材料,都是很好的。文字也不难看,建议发到各中央局、各省、地、县、社,给干部们阅读。可以从中选两三件向识字不多的干部宣读和讲解,以便引起他们的注意,逐步加深广大干部特别是县、社、大队、生产队四级干部对于参加生产劳动的伟大革命意义的认识,减少许多思想落后的干部的抵抗和阻力。中央曾在今年三月二十三日发出山西省昔阳县全县四级干部无例外地参加生产劳动的模范事例,并作了批语,对于这个重大问题,有些同志是注意了,例如浙江,在全省党代表大会上着重讨论了并且作了具体安排,其他地方,则反映尚少。建议各级领导同志利用适当机会,对于干部参加劳动这个极端重大的问题在今年内进行几次讨论,并普遍宣读山西昔阳县那个文件。各省、市、自治区,一定有自己的好范例,应当选出一些(不要太多)让干部学习。我们希望争取在三年内能使全国全体农村支部书记认真参加生产劳动,而在第一年,能争取有三分之一的支部书记参加劳动,那就是一个大胜利。城市工厂支部书记也应当是生产能手。阶级斗争、生产斗争和科学实验,是建设社会主义强大国家的三项伟大革命运动,是使共产党人免除官僚主义,避免修正主义和教条主义,永远立于不败之地的确实保证,是使无产阶级能够和广人劳动群众联合起来,实行民主专政的可靠保证。不然的话,让地、富、反、坏、牛鬼蛇神一齐跑了出来,而我们的干部则不闻不问,有许多人甚至敌我不分,互相勾结,被敌人腐蚀侵袭,分化瓦解,拉出去,打进来,许多工人、农民和知识分子也被敌人软硬兼施,照此办理,那就不要很多时间,少则几年、十几年,多则几十年,就不可避免地要出现全国性的反革命复辟,马列主义的党就一定会变成修正主义的党,变成法西斯党,整个中国就要改变颜色了。请同志们想一想,这是一种多么危险的情景啊!

解决这个问题是不是很困难呢?并不很困难。只要看到问题的严重性.经过调查研究收集了可靠的材料,明了了情况,下定了决心,政策和方法又都是正确的,又有政治上强有力的几个同志作为核心领导,那末,就一个公社的范围来说,有几个星期就够了,就一个县来说,有几个月也就够了,就一个省来说,分期分批,搞好搞透,大约需要一年、二年,或者更多一点时间。因为这一次社会主义教育运动是一次伟大的革命运动,不但包括阶级斗争问题,而且包括干部参加劳动的问题,而且包括用严格的科学态度,经过实验,学会在企业和事业中解决一批问题这样的工作。看起来很困难,实际上只要认真对待,并不难解决。这一场斗争是重新教育人的斗争,是重新组织革命的阶级队伍,向着正在对我们猖狂进攻的资本主义势力和封建势力作尖锐的针锋相对的斗争,把他们的反革命气焰压下去,把这些势力中间的绝大多数人改造成为新人的伟大的运动,又是干部和群众一道参加生产劳动和科学实验,使我们的党进一步成为更加光荣、更加伟大、更加正确的党,使我们的干部成为既懂政治、又懂业务、又红又专,不是浮在上面、做官当老爷、脱离群众,而是同群众打成一片,受群众拥护的真正的好干部。这一次教育运动完成以后,全国将会出现一种欣欣向荣的气象。差不多占地球四分之一的人类出现了这样的气象,我们的国际主义的贡献也就会更大了。



\section[关于农村社会主义教育等问题的指示(一九六三年五月)]{关于农村社会主义教育等问题的指示}
\datesubtitle{(一九六三年五月)}


\subsection{一、关于社会主义社会的阶级斗争}

在社会主义社会中,有没有阶级?有没有阶级斗争?外国有一种说法:他们国内没有阶级了,他们的党是全民的党了;无产阶级专政也没有对象了,他们的国家是全民的国家了。我们国内也有类似的说法。资产阶级每天在斗争无产阶级,就是不承认有阶级存在,就是不承认有阶级斗争,说阶级斗争是马克思捏造的。不光是外国的修正主义者和国内的资产阶级不承认有阶级和阶级斗争存在,我们有许多干部、党员、对于敌情的严重性也是认识不足的,甚至熟视无睹的。

以上这些看法对不对?完全不对。已被推翻的反动统治阶级是不甘心于死亡的,他们总是企图复辟的。同时,资产阶级分子会新生,反革命分子也会新生。而在这些阶级敌人的后面,还站着帝国主义,现代修正主义和反动的民族主义。因此,党的八届十中全会公报指出:“在无产阶级革命和无产阶级专政的整个历史时期内(这个时期需要几十年,甚至更多的时间),存在着社会主义和资本主义这两条道路的斗争。”

〔翻印者注:公报上原文为:“在无产阶级革命和无产阶级专政的整个历史时期,在由资本主义过渡到共产主义的整个历史时期(这个时期需要几十年,甚至更多的时间)存在着无产阶级和资产阶级之间的阶级斗争,存在着社会主义和资本主义这两条道路的斗争。”〕

社会上的阶级斗争,一定要反映到我们党内来。我们这样大的国家,又存在阶级,在党内不反映资产阶级思想、封建阶级思想、富裕农民思想那才是怪事!阶级斗争所以会反映到党内来,还有一个重要根源。从党内成份来看,我们党内主要是工人、贫雇农、下中农,主要成份是好的。但是党内有大量的小资产阶级,其中有的是城乡上层小资产阶级分子,也有一批是知识分子,还有相当数量的地主、富农的子女。这些人,有的马列主义化了,有的化了一点,没有全部马列主义化,有的完全没有化,组织上入了党,思想上没有入党。\marginpar{\footnotesize 46}这些人对社会主义革命没有思想准备。另外,这几年还钻进一些坏人,他们贪污腐化,严重违法乱纪。民主革命不彻底,坏人钻进来,这个问题要注意。但是比较好处理。主要问题是没有改造好的小资产阶级分子,知识分子和地主、富农子女,对这些人需要做更多的工作。因此,对党员、干部要进行教育,再教育,这是一个重要任务。

\subsection{二、关于社会主义教育运动问题}

这次社会主义教育运动,是一次伟大的革命运动。“革命尚未成功”这是孙中山的话。我们现在是:社会主义革命正在进行,有些地方民主革命尚未成功。社会主义革命没有完成,就要继续进行社会主义革命;民主革命没有成功,就要进行民主革命的补课,还有一些地方,地主根本没有打倒,那些地方是重新革命的问题。

资产阶级右派和中农分子把希望寄托在自留地、自由市场、自负盈亏和包产到户,这“三自一包”上面。我们搞社会主义革命,在城市搞“五反”,在农村搞“四清”,就是挖资产阶级的社会基础,挖资本主义的根子,挖修正主义的根子。

党的八届十中全会以后,有些地方比较认真执行了中央关于社会主义教育的指示,做得很好,不仅制止了“单干风”,而且把农村中阶级斗争的盖子揭开了,把各种矛盾揭开了,把各种破坏社会主义的牛鬼蛇神揭露出来了。可见,阶级斗争一抓就灵,也有些地方,虽然进行了社会主义教育,但是没有抓住要点,没有找到正确的方法。今后,还需要抓住要点,采取正确的方法和步骤,进一步开展社会主义教育运动。

这次社会主义教育的要点是什么?要点就是阶级和阶级斗争,干部洗手洗澡,依靠贫、下中衣,“四清”,干部参加集体劳动这样一套。凡是社会主义教育一般化,不触及洗手洗澡,不触及贪污盗窃的地方,就不能抓住主要问题。

方法是什么?方法就是说服教育,洗手洗澡,轻装上阵,团结对敌,就是要团结百分之九十五以上的群众和干部,同阶级敌人作斗争,对百分之九十五以上的人,不抓辫子,不打棍子,不戴帽子,还要加上不追不逼,不打不骂。有错误的人,只要彻底坦白悔改,就算在百分之九十五以内。手脚不干净,要批评,要洗手洗澡,还要继续做工作。伤人不要过多,但少数人是要伤的。要奖励一些好人,处理少数坏人,组织处理一定要经过批准手续。有的地方采取讲社史、村史、家史的办法,对青年进行阶级教育。这个办法很值得推行。贫农受剥削、受压迫的家史可以讲,贫农富裕起来的家史也可以讲。地主、富农的家史也可以作为反面教材讲给贫、下中衣听,讲他们是怎样剥削压迫人的。

步骤是什么?就是:经过试点,分期、分批、分地区地进行。一个县之内,也要分期、分批进行。要注意到各地区的不同情况,允许有先有后,允许参差不齐,开始训练县级干部,再训练公社和大队的干部,然后训练生产队干部和贫、下中农的积极分子。试点很重要,各地都要搞试点,经过试点把情况弄清楚。有的同志,开始对阶级斗争、社会主义教育不大相信?后来他去试点以后,就相信了,相信了就抓起来了。

有人对社会主义教育有顾虑,无非是两条:一是怕耽误生产,一是怕“伤人”太多,阶级斗争和社会主义教育一定会有利于增加生产。“伤人”不能过多,但少数人是要伤的。解决人民内部矛盾问题也要有点“紧张”,精神轻松愉快,这是就其结果说的。不是说在社会主义教育过程中没有一点紧张。\marginpar{\footnotesize 47}只有搞了社会主义教育,“五反”、“四清”,组织贫、下中农阶级队伍,才能做到心情舒畅。贫、下中农起来,几股黑风不打倒,干部不洗手洗澡,能够心情舒畅吗?干部心情不舒畅,就要搞我们这一套。贫、下中农心情不舒畅(此处遗漏了一段)才能出现真正的心情舒畅的局面。

当然也不要急躁,不要蛮干。过去社会主义教育搞得不深的地方,要从搞得不深的实际情况出发,要跳起来,一哄而起。不打无把握之仗,要准备好了再打。没有试点,情况不明,或者兵没有练好,干部和贫、下中农没有训练好,就不要急急忙忙的、大规模地开展运动。对干部要着重说服教育,口头说不服的,就用事实材料去说服。要派强有力的干部去领导运动,不会打仗的人,不要他当指挥官。没有蚂蚁的地方不要硬找蚂蚁。例如,过去有些一类社、队过去注意了阶级斗争,注意了社会主义教育,就不一定完全采取现在这一套办法来搞。但是人民内部矛盾是普遍存在着的,在一类社、队,也要解决人民内部矛盾。

总之,这次运动是一次大考验,干部好不好,行不行,都要在这次运动中受到考验。只要我们分期、分批、分地区去搞,经过试点,认真对待,再加上保护百分之九十五以上的干部和群众,大毛病是可以避免的。

\subsection{三、关于四清问题}

在农村中,不搞“五反”,只搞“四清”。“四清”就是清理账目,清理工分,清理仓库,清理财物。其中又主要是清理账目、清理工分两项。首先应当发动群众把1962年以来的账目、仓库、财物、工分,同时把国家投资、银行贷款和商业部门赊销所添置的资产,全面彻底地清查一次。

这是一项同社会主义教育运动相结合的大规模的群众运动,主要解决人民内部矛盾,但对于贪污、盗窃、投机倒把、蜕化变质分子来说,也是一场严重的阶级斗争,农村中的“四清”运动,同城市中正在进行的“五反”运动一样,都是打击和粉碎资本主义势力猖狂进攻的社会主义革命斗争。

我们在农村中十年没有搞阶级斗争了。1952年搞“三反”、“五反”,是在城市。在农村,1957年搞了一次全民整风,但不是用现在这个方法搞的。合作化以来,农村中的现金、工分、财物、仓库就没有清理过,没有向社员全面公布账目。“四清”能搞出很多贪污盗窃、牛鬼蛇神。公安机关搞不出来,“四清”能搞出来。

进行“四清”方法要对,要采取扎根串连,依靠贫、下中农,这一套办法,放手发动群众,有些干部不听领导的话,他们却不能不听群众的话,把群众发动起来,事情就好办了。要把百分之九十五以上的人团结教育过来,打击极少数严重贪污盗窃分子。有些人坦白了,退赔了,就可以不戴贪污分子的帽子,要使多数洗温水澡,轻装上阵,团结对敌。“四清”中一切问题的处理都要发动群众充分讨论。赃款、赃物,不退不行,但要退得合情合理。不退群众不允许,太挖苦了,有些干部过不去,群众过些时候也会同情他们的。对贪污盗窃分子,一般不采用群众大会斗争的方式,可以一方面采用背靠背的方式(在必要的时候,也可以在较小范围的群众会上),让群众充分揭发和批判;一方面组织专门小组清理账目,进行调查研究,然后根据确凿的证据,核实定案。要严防敷敷衍衍“走过场”,也要防止逼、供、信,严禁打人、骂人和任何变相的体罚。已经搞过的要复查。凡是搞得不好,真不彻底的,必须重搞,一些必要的制度还没有建立的,必须建立起来。\marginpar{\footnotesize 48}

\subsection{四、关于组织贫、下中农革命的阶级队伍}

不论在革命中,或者在社会主义建设中,都必须解决依靠谁的问题。(翻印者注:上文与下文联系起来看,这里一定遗漏了一段,请读者注意),因为那时还有唯物论和唯心论,还有先进和落后。没有阶级差别了,总还有左、中、右。

在农村中依靠谁?

不是依靠全民,而是依靠贫农、下中农。贫农、下中农大约占农村人口百分之五十到七十左右。他们是农村中的多数。是农村中的无产阶级和半无产阶级。我们要同地、富、反、坏作斗争,就要团结大多数,首先是依靠贫农、下中农。要多数,还是少数,要无产阶级专政,还是牛鬼蛇神专政?真正的马列主义者就是依靠多数,修正主义者名曰依靠全民,实际上是依靠少数。一些农村干部有一种说法,说:“地主听话,中农好办,贫农糊涂。”其实,地主是要你听他的话,县社干部都不注意听贫农的话,那么贫农一辈子都翻不了身。

依靠贫农、下中农,树立贫、下中农在农村的优势,是进行社会主义革命和社会主义建设的重要问题也是巩固无产阶级专政的重要问题。在整个社会主义历史阶段,一直到进入共产主义以前,我们在农村都要依靠贫、下中农。

要依靠贫、下中农,就必须建立贫、下中农的阶级组织。组织起来,就有了中心。贫、下中农组织起来就可以更好地团结中农。有些地方,贫农一经组织起来,中农就打听消息,表示要走社会主义道路,还不是大家都组织起来了。

建立贫下中农阶级组织,要在斗争中发动群众去建立,不要形式主义地建立。搞形式有什么用?形式上是社会主义,实际上不是。南斯拉夫还不是挂着社会主义招牌。

开始组织,不一定数量很多。比如一个生产队有二十户,先组织两三户,以后四、五户,有十一、二户组织起来了,就很起作用。要像滚雪球一样逐步搞起来,不要一哄而起,一步一步搞,扎扎实实搞。

有些地方提出贫、下中农委员会的主任,由党支部副书记兼任。这不好,要由群众选,可以选支部书记兼任,也可以不选支部书记兼任,不能硬性规定,保谁当选。

在农村中,无产阶级和资产阶级都在争夺农民中的富裕阶层,这个阶层本身就是产生资产阶级的东西,也容易接受资产阶级的影响。但是,对这个阶层,要作具体分析。例如,要看生活上升,还要看政治表现,只要不是剥削者,思想上又赞成社会主义,积极劳动,还是要把他们团结起来,共同建设社会主义。

\subsection{五、关于干部参加集体生产劳动问题}

干部参加集体生产劳动的问题,对于社会主义制度来说,是带根本性的一件大事,干部不参加集体生产劳动,势必脱离广大的劳动群众,势必出修正主义。

我们的党是无产阶级的党,是劳动群众的先进的党。我们党的基层组织必须掌握在劳动积极的先进分子手里。农村中的党支部书记不但在政治上应当是最先进的分子,而且必须力争成为生产能手,成为劳动模范。县社以上干部也要认真参加集体劳动。干部不劳动了就会慢慢变质,甚至变成国民党,修正主义就有基础了。\marginpar{\footnotesize 49}浙江省就有一位大队支部书记应四官说:“不参加劳动,工作就像浮萍一样浮在水面,摸不到底。”参加劳动,就可以解决这个问题,至少可以减少贪污、多占问题,可以了解农、林、牧、副、渔这一套,支部书记参加了,大队长、队长、会计就要参加,整党整团就好办了。这样,修正主义就少了。

现在有些群众对干部作官很有意见。群众说:“大队干部了不起,一吃二用三送礼。”大队干部就这样大?现在给他一个“四清”,一个劳动。你不干,就当老百姓。这是一个很大的斗争,没有一大批积极分子起来是搞不好的。

山西昔阳县干部参加劳动的情况是个好榜样。昔阳县的干部既然能够长期坚持,其他县的干部也应当是能够办到的。我们争取在三年之内,分期分批,使农村支部书记认真参加劳动。比如一百个支部,第一年先有三分之一参加劳动,第二年又有三分之一,就有百分之六十的人。这样声势就大了,其他的人就要参加进去了。

有些劳动模范现在不参加劳动了,不参加劳动,还作什么模范?不能参加劳动,理由无非是怕耽误工作,会议太多。公社以上的领导机关要减少开会。一个县,每年只开一、两次三级干部会议就够了。必须开的会要事先作好充分准备,有些会还可以到下边去开。

\subsection{六、关于运用马克思主义科学方法进行调查研究的问题}

调查研究有两种方法。一种是大胆的主观的假设,小心的主观主义的求证。这是个很坏的方法。一种是马克思主义的科学方法,河北省保定地委关于“四清”的调查就是这种方法。保定地委开始并不是去搞“四清”,是去搞分配问题的。群众不同意,提出搞“四清”。保定地委听了群众的意见,改变了计划,搞了“四清”。这才是真正的调查研究。

调查研究的范围,一个是生产斗争,一个是阶级斗争,一个是科学实验。不对这方面进行调查研究,哪有马克思主义?浙江省清田县搞试验田,带点科学试验性质。他们试验到山里去了。那里人民开头不赞成冬水田,经过试验,冬水田第二年收成好,贫农看到以后就接受了。所以要调查,要试验。礼会主义教育为什么有人不相信?就是没有试点,没有认真调查研究。比如走路,像平常那样走,什么也看不见,弯下腰来细看,就可以看到地上的蚂蚁很多,就能看到很多东西。否则,不仅新鲜的萌芽的东西看不见,就是大量普遍存在的现象也看不见。例如阶级斗争和干部不参加劳动,是大量存在的现象,有些人都看不见。

当然,这些事情也是要逐步认识的,要从现象到本质。比如干部不参加劳动,势必产生修正主义,有许多同志就看不清。干部不参加劳动,了解和反映的情况就不会真实。比如打仗不亲自参加战斗,还不是纸上谈兵,怎么能懂得打仗呢?单是进军事学校也不行。

为了造成调查研究的风气,做好我们的工作,各级党委在日常工作中讲哲学,对干部进行马列主义认识论的教育。唯物论、唯心论、世界观、辩证法,都是讲的认识论。物质可以变为精神,精神可以变为物质。这些道理,应当让干部懂得、群众懂得。让哲学从哲学家的课堂上和书本里解放出来,变为群众的尖锐武器。

人的思想是从哪里来的,我们有些同志是不知道的,对于精神可以变为物质,有些同志就更糊涂了,但是不识字的农民是懂得推理的。比如农民认为地主是人,是剥削压迫他们的人。人、地主是两个概念,农民把这两个概念联结起来,进行判断推理,得出结论说:地主是剥削人的人。农民的这种认识,是从生活中来的,不一定识字才懂得,所以要破除迷信(当然不要破除了科学),不要把哲学看得那么神秘,那么困难。\marginpar{\footnotesize 50}哲学是可以学到的。雷锋那样年青的同志就懂得一点哲学。

总之,这一次社会主义教育运动是一次伟大的革命运动,不但包括阶级斗争问题,而且包括了干部参加劳动的问题,而且包括严格的科学态度,经过试验,学会在企业和农业中解决一批问题这样的工作。看起来很困难,实际上只要认真对待,并不难解决。这一场斗争是重新教育人的斗争,是重新组织革命的阶级队伍,向着正对我们猖狂进攻的资本主义势力和封建势力作尖锐的针锋相对的斗争,把他们的反革命气焰压下去,把这些势力中间的节大多数人改造成为新人的伟大运动,又是干部和群众一道参加生产劳动和科学实验,使我们的干部成为既懂政治、又懂业务,又红又专,而不是浮在上面,做官当老爷,脱离群众,而是同群众打成一片,受群众拥护的真正好干部。这一次教育运动完成以后,全国将会出现一种欣欣向荣的气象。差不多占地球四分之一的人类出现了这样的气象,我们的国际主义的贡献也就会更大了。


\section[在杭州会议上的谈话(一九六三年五月)]{在杭州会议上的谈话}
\datesubtitle{(一九六三年五月)}


\subsection{一、形势问题}

生产的形势,一年比一年好。阶级斗争形势是严重的,尖锐的。(列举农村阶级斗争情况)为什么农村出现这样严重的情况?有三个原因,一个是阶级原因,一个是历史原因,一个是认识原因。

阶级原因:主要是社会主义社会还是有阶级的社会,存在着阶级和阶级斗争。正确理解和处理阶级矛盾和阶级斗争,正确处理敌我矛盾和人民内部矛盾,是领导和团结全党,领导和团结全体人民群众,顺利地进行社会主义革命和社会主义建设的保证。

历史原因:一方面是有的地区民主革命任务尚未完成,有的地区社会主义革命未完成。封建地主没有打倒的地方,是重新革命的问题。另一方面是工作历史方而的原因。土改以后,我们就没有再搞阶级斗争。“三反”、“五反”,一九五七年的反右斗争,都搞了一下,但不是这样的做法。苏联在一九三二年以后,一九三七年、一九三八年又搞了二次肃反,此后十六年当中不搞阶级斗争,他们的集体化依靠谁?不搞阶级斗争,无产阶级专政就没有可靠的社会基础。

华北局机关五反搞得好。说是“清水衙门”,但是一清就清出好多专案来。

认识原因:阶级斗争是客观的存在,没有认识到,怎样领导阶级斗争?

\subsection{二、认识问题}

十中全会后,跑了十一省,只有××、××滔滔不绝地讲社会主义教育,其他人都不讲。二月会议后,情况有了变化。河南五个月没有抓阶级斗争,二月会议以后,抓得很好。有变化,但并不是都通了,有个地委书记,二月会议以后,就不通,下去试点以后才通。\marginpar{\footnotesize 51}

我看了湖南第二个材料,现在才懂得一点,即搞规划、生产经营中间,也有两条道路的斗争。

我问了许多人,思想是从哪里来的?都回答不出来。物质变精神,精神变物质,是生活中常见的现象,不识字的农民也懂得这一点。比如说,你问农民,他知道张三是地主,是压迫我们的,有了“张三”、“地主”这两个概念,就可以推理:地主是剥削人的人。农民的认识是从生活中来的,不识字也可以懂哲学。成吉思汗就不识字。

一言可以兴邦,一言可以丧邦。这就是精神变物质。马克思就是一言,要无产阶级革命和无产阶级专政,这不是一言可以兴邦吗?赫鲁晓夫也是一言,就是不要阶级斗争,不要革命,这不是一言可以丧邦吗?

哲学要在实际工作中讲,要在开会中讲。要告诉你身边的同志,哲学并不难。军事学也不难,我们人民解放军的元帅、将军中间,只有林彪、刘伯承等有数的几个人是从军事学校中出来的。翻了军事书,看了欧洲战史,和中国情况对不上。不是黄埔军校的洋包子打败了土包子,是土包子打败了洋包子。林彪同志是黄埔军校的半年的入伍生,……派出来当连长,根本不能打仗,听班长的。班长说怎么打就怎么打。军事是从实践中学的,所以不要把马克思主义看得那么神秘,不要把哲学看得那么神秘。我看过雪峰一部分日记,此人就懂得一点哲学。

大学生学习五年就学好哲学?我不相信。许多哲学家都不是大学学习的。中国的哲学家中,王充、范缜、付玄,柳宗元、王船山、李贽、戴东原、魏源……都不是专门搞哲学的。黑格尔也不是专门搞哲学的,他的学问很广。康德是一个天文学家,他的天体论到现在还有价值。马、恩、列、斯也都不是专门搞哲学的。

山沟里出哲学。醴陵那样好的报告,不出在湘潭,不出在常德,而出在醴陵。在困难中,在斗争中才能够出哲学。逆境出哲学,顺境能够出哲学吗?三国的黄盖兄,醴陵人;程颐、程灏的老师周廉溪,是宋代的大理学家,朱熹和他是一个系统的,也是醴陵人,是醴陵专区的道县人。张载是陕西人,那是另一派。唐代的大书法家怀素,也是这里的。柳宗元从三十岁到四十岁,整整十年都住在醴陵,当时叫做永州。他的山水文章,和韩愈辩论的文章,都是在那里写的。

所以要破除迷信,不过要注意,不要像前几年那样,把不应该破的也破了。

事物有现象有本质,要透过现象看本质。现象和本质是对立的统一。本质是看不到的,要透过现象去抓到本质。比如干部不参加劳动,势必会产生修正主义。又比如平常我们走路看不到蚂蚁,大踏步就更看不到了,要蹲下来,才能看到蚂蚁,就能看到很多东西。否则不仅新鲜的萌芽的东西看不到,就是大量普遍存在的东西也看不见。比如阶级斗争和干部不参加劳动是大量存在的,有些人却看不见。要用科学的方法,进行调查研究。有的人是主观主义地大胆地假设,主观主义地小心地求证。河北各地委下去调查研究,只有保定地委是科学的,其他都是主观主义的。保定地委开始并不是去搞“四清”。而是去搞分配的,群众不同意,提出搞四清。保定地委听了群众的意见,改变了计划。搞了“四清”,这才是真正的调查研究。

讲哲学不要超过一小时,讲半小时以内,讲多了就糊涂了。

我在莫斯科会议上讲了哲学,莫斯科宣言写上了,在国内反倒没有人讲。\marginpar{\footnotesize 52}

\subsection{三、要点}

运动的要点是什么?是十个问题,其中一部分是认识问题,是要高级领导干部、领导干部解决的。还有一些问题是普遍工作要解决的,普遍工作中的要点有以下五点:

1.阶级、阶级斗争。用什么方法进行阶级斗争?一定要用阶级观点去分析问题,最先写四大家族的是曹雪芹。《红楼梦》写的贾、史、王、薛大家族,他们是奴隶主,三十二人。写奴隶女,鸳鸯、晴雯、小红等,都是很好的,受害的是这些人。林黛玉不是属于四人家族的。

2.社会主义教育。社会主义教育的方法有两大条:

第一条是把中央的精神和干部、群众见面,讲解清楚,结合当地的具体情况、具体工作、具体事实,让群众揭盖子。

第二条,要让老一辈重新回忆受压迫、受剥削的历史,激发阶级感情,让青年一代知道革命斗争果实来之不易,让他们续一续无产阶级的家谱。

3.依靠贫下中农。依靠谁的问题,一万年也有,到将来总还有唯心主义和唯物主义、先进和落后、左中右的矛盾。在今天依靠谁?总得有一个阶级。依靠全民?说依靠全民,实际上是依靠少数人。有人说“地富听话,中农调皮,贫农糊涂。”地富怎么不听话?又送东西,又送女人,可是他是要你听他的话。

什么叫心情舒畅?贫农、下中衣受到压抑,不能抬头,心情怎么能舒畅?贫农、下中农不舒畅,干部怎么能够舒畅?

资产阶级说他们后继无人,怎么说后继无人?黑格尔的后继人是马克思,资产阶级的后继人是无产阶级。资产阶级抓“三自一包”,想卷土重来,我们就要在这方面打击他,打掉他的基础,不让他拉后继人。

\subsection{四、四清}

什么叫贪污?五十元?一百元?二百元?只要坦白了,退了赃,就不算贪污。

赃要退,也要合情合理,退到手脚干净,又要退到让干部能够生活。这样做究竟退多少?是不是采取自报公议的办法。

惩办要控制在百分之一。

今年不要开杀戒,明年再说。罪大恶极的也先放慢一些,现行反革命按规定办理。群众要求非杀不可的,是有道理的,你领导可以等一等嘛!

5.干部参加集体生产劳动。只有参加劳动,才能解决贪污、多占问题,也可以了解生产情况,而不是浮在上面。干部不参加劳动,势必脱离劳动群众,势必出修正主义。

昔阳干部劳动很好。昔阳在山上,很穷。很穷就革命。

要把农村党的基层组织建立在先进劳动者、劳动积极分子手里。

(有人说,有些劳动模范不参加劳动。)

劳动模范不参加劳动,还算什么模范?取消好了。有的因为会多,接待访问太忙,这个问题要解决,你们可以到田间去访问嘛!\marginpar{\footnotesize 53}

县干部也要参加劳动,基层干部不参加劳动,不就跟国民党保甲长一样吗?你们是做大官的,也有做小官的。小官权也很大。过去一个团长,给不少办公费。现在我们基层干部,一个参加劳动,一个“四清”,不愿意干就回家当老百姓去。

干部参加劳动了,贪污盗窃、投机倒把的就少了。贪污盗窃、投机倒把,什么时候都有,一万年都有,不然辩证法就不灵了,就没有对立统一了。

贪污揭发得越多,我越高兴。你们抓过虱子没有?身上本来很多,抓得越多越高兴。

\subsection{五、方法}

要采取积极态度。

1.要注意训练和教育干部;

2.不要着急,今年搞不完明年,明年搞不完后年。土改不是搞了三、四年吗?有的人不信,不要去责备他,你一围攻,他一着急,就乱来。要慢慢地说服,着什么急?我们革命胜利比苏联还不是晚三十多年?

3.要试点,要踏踏实实地搞深搞透,要防止敷敷衍衍地走过场,一定要搞试点。

4.要区别不同情况,少数民族地区,边境地区不要一起搞。(讲了西汉陈平的故事)(对李××)你四川那么大一个省,一下子能够搞得了哇?

5.精简。要精简一些干部下去搞劳动锻炼,搞阶级斗争锻炼。我身边原有二、三十人,现在只剩下十几个人。我对江××说,江苏四千多万人口,省直机关工作人员五千,可以精简一千五百人或两千人。这是一个老问题,长期没有解决。

6.要抓住重点。“不唱天来不唱地,只唱一本《香山记》”。《香山记》是讲庄王的女儿(即观世音菩萨)的故事,七个字一点,开头两句就是这个。天和地可以隔开,天和地都不唱,单唱《香山记》,就抓阶级斗争。


\section[批示×××在农村蹲点至少五个月(一九六三年六月三日)]{批示×××在农村蹲点至少五个月}
\datesubtitle{(一九六三年六月三日)}
{\noindent\kaishu\centering (写在中央工作会议简报李雪峰同志发言上面)\par}

此件送×××一阅,阅后退我。

你应当下决心在今冬明春这段期间,在北京农村地区或天津郊区蹲点,至少五个月。家里工作可以间接或抽时间回来处理。从新华社和人民日报抽出一批人相当地干部合组一个工作队,包一个最坏的人民公社,一直把工作做完,以后并成为你们经常联系的一个点。还要在一个冬春,参加城市五反。千万不要放弃参加这次伟大革命的机会。

\kaitiqianming{毛泽东}
\kaoyouriqi{六月三日}
\marginpar{\footnotesize 54}


\section[接见古巴文化、工会、青年等代表团的谈话(一九六三年七月二十六日)]{接见古巴文化、工会、青年等代表团的谈话}
\datesubtitle{(一九六三年七月二十六日)}


主席:文化代表团到什么地方去看了一下没有?

奥斯明费尔南德斯(以下简称奥):到上海、广州、武汉去参观过。

主席:一共有多少时间?

奥:刚好二十六天。

主席:去年代表团是什么时候到的?

佛利斯·库左拉(以下简称佛):上个礼拜天到的。

张××:(以下简称张)。周总理昨天接见时,希望青年代表团到延安去看看。

佛:昨天中央的同志对我们讲,可以到延安去参观。

主席:那个地方很落后。到落后的地方去看看,也有好处。文化代表团来签订文化协定,签好了没有?

奥;前天中午(二十四日)已签了字,我们非常满意。

主席:非常满意吗?

奥:是。非常满意。

主席:协定包含些什么内容?

奥:协定的主要内容,是关于两国互派代表团访问的事情,古巴方面将邀请中国杂技艺术团去访问,同时我们也要派艺术团来中国。另外,双方要互派语言留学生,交换影片、书刊以及其他文化资料等。

主席:我们的影片,恐怕不那么高明。

奥:电影是教育人民群众的有力工具,我们相信,中国电影能够起到这个作用。

主席:影片的作用是不小,但我们的好片子太少。

奥:中国的电影事业,像其他经济建设事业一样,正在飞快地向前发展。

主席:我们的胶片还不能够完全自造。你们能够制造胶片吗?

奥:古巴不出胶片。

主席:是进口吗?

奥:进口。古巴的电影,主要是些记录片和短片。

主席:慢慢发展下去,就会搞长片。我们的杂技,是历史上遗留下来的玩意。

奥:到哈瓦那去的,是由武汉的杂技团组成的,我们亲自到武汉去看过他们的节目。

主席:杂技团去过古巴没有?

张:没有去过。以前,去过歌舞,京剧团。

主席:杂技团过去不是去过吗?

张:那是到拉丁美洲巴西等国访问。当时,古巴革命还未成功,我们的艺术团进不去。

主席:\marginpar{\footnotesize 55}古巴革命进展很快,几年之内,就战胜了敌人。我们的革命时间很长,搞了二十几年。有本国的敌人,也有外国的敌人。外国的敌人是日本,我们和他打了八年;国内的敌人是蒋介石,和他打了十四年。日本没有打进来以前,和蒋介石打了十年,日本投降后,又打了四年。蒋介石后面有美帝国主义支持。后来,又在朝鲜和美国人打了差不多三年。有很多人怕美国人。在我们国家,也有很多人怕美国人。打了三年之后,怕美国的人少了。现在,美国人又在越南南部打,越南南方人民只有很落后的武器,而美国有很多新式武器,包括飞机、大炮、直升飞机,细菌战,化学战,但是,打了五、六年,南方人民和军队在发展,根据地在不断扩大,他们不怕美国人了。这些经验很值得讲一讲。

在某种程度上,美国兵还不如日本兵。我们和日本打了八年。美国的士兵还是有战斗力的,他们武器多、武器好,但是,他们不喜欢打仗。美国帝国主义为了霸占全球,用战争威胁全世界人民,作各种战斗准备。一种是核战争,这种战争有可能打,也有可能不打;另一种是常规武器的战争。他们的方针是打常规武器的战争。在越南打的就是这种常规战争。过去,我们和朝鲜人民打的,也是这种战争。那时,他们不是没有原子弹,而是不敢打;现在原子弹更多了,可是在越南南部也不敢打。美国现在的三军参谋长泰勒,此人就是在朝鲜和我们打过的。他写了一本书大家有机会最好看看。书名是:《音调不定的号角》。在这本书里,他批评杜鲁门和艾森豪威尔过去是不重视常规武器的。战争,叫喊打原子战争,但又不打,这就叫做音调不定。泰勒在艾森豪威尔执政时期,当过陆军参谋长,后来因为不能实现自己的主张,辞职不干了。肯尼廸当选为总统后,他又起来了,当了陆、海、空三军的参谋长。

据您们看,全世界人民现在还那么怕美帝国主义吗?

奥:从我们来看,现在世界各国的民族解放运动蓬勃发展,在拉丁美洲,如委内瑞拉,秘鲁等国,人民的革命战争日益高涨他们国家虽然很小,进行常规武器的战争力量不够,但还是在那里进行斗争,他们不怕美帝国主义,敢于反对在美国支持下的独裁反动统治。

主席:他们和你们一样。如果只是怕,把手缩回来,让人家抓住关在监牢里,或者杀掉,那革命的火焰就熄灭了。我们参加朝鲜战争的时候,和蒋介石打仗的时候,在我们的队伍中,也有很多人怕美国人,但是,怕又有什么办法呢?难道能解除武装吗?那个时候,大多数人无所谓怕不怕,因为敌人已经打来了,怕又有什么用?打的结果,我们胜利了,可见不必那么怕,怕美国人是多余的。除美国之外,还有一些帝国主义国家,如法国,他们在越南北部打,结果胡志明同志打胜了。在阿尔及利亚,阿民族解放军也打胜了。他们打了七年,经历了许多艰难,法国军队多到八十万之多,而阿民族解放军只有三、四万人。他们很多人都同我讲过,包括现在的议长,过去的总理阿巴斯。你们知道阿巴斯吗?

奥:知道他的名字。

主席:他们对我们谈过很多,谈过他们的困难,阿牺牲了九分之一的人口,就是说,九百万人口中死了一百万。另外,法国军队领导机关还出版我写的小册子,企图打败阿民族解放军。我告诉他们,我写的小册子,是人民战争的小册子,反人民战争的那一方面,不可能利用,他们想利用,实际上是不可能的。\marginpar{\footnotesize 56}我们在国内战争时期,蒋介石也利用我写的小册子,想把我们打败,结果还是不行。美国人也想利用我们的办法,他们有很多人研究中国的游击战、运动战的战略战术,但是,在朝鲜战争中间,没有得到什么好处。

在朝鲜战争中,除美国外,还有十五个国家参了战。包括美国,法国,加拿大,澳大利亚,新西兰,土耳其,哥伦比亚等。当然主力是美国人。美国去了一个师,土耳其去了一个旅,哥伦比亚去了一个营,他们是凑合起来的。他们到朝鲜来,是打着联合国的旗子。打了二年半(快三年),谈判就整整谈了二年,当时是一面打,一面谈。谈判的地方就是双方交界的一个小地方——板门店。

奥:我们在朝鲜时去过那个地方。

张:文化代表团访问中国之前曾到过朝鲜。

主席:朝鲜很可以去看一看。要研究那个时候怎么一面打,一面谈;怎么用很少的人和武器,战胜了拥有强大武器的敌人。

奥:我们在朝鲜呆了十天。

主席:在朝鲜看过防御工事没有?

奥:我们在朝鲜访问时,去了开城的板门店,参观了军事博物馆,展览馆中有一个一千一百一十二高地模型,山头被敌人打去了两公尺。

主席:山里面有我们的工事,有地道,我们用各种办法对付敌人,山地有山地的办法,平原有平原的办法。不管敌人武器多么好,多么强,因为他们是反对革命,不利于人民的,不可能得到胜利。他们的道路只有一条,就是失败。最后,所有的帝国主义和各国反动派,都要灭亡,我们在座的人就是证据。美国支持的反动派巴蒂斯塔,不是也失败了吗?你们在和反动派斗争中,有外国援助没有?

奥:只有七支步怆。

主席:是啊!没有任何外国援助,就是步枪,而且很少,也胜利了,我们中国的战争也证明了这一点。你们在强大的敌人面前,因为敢于和它打,终于打胜了。

恩来同志劝青年代表团到延安看看,是有理由的。延安虽然很落后,但在当时是我们领导机关的所在地。蒋介石在南京、上海,南京是一百万人口的城市;上海是七百万人口的城市,那里住着外国人,很容易得到外国人的援助,延安城和附近只有七千人。看来,不是大城市打胜小城市,而是小城市打胜大城市;不是大城市打胜乡村,而是乡村的人民包围城市,最后夺取城市;不是有大炮、飞机、坦克的敌人打胜我们,而是只有步枪、轻炮、手榴弹的人民军队打胜了敌人。起初,我们没有大炮,打到第二年,就有了大炮,打到第三年,大炮就更多了。是哪里来的呢?是美国的,是蒋介石运输给我们的。后来,我们也有了坦克,也是蒋介石送的。当时就是没有飞机,现在空军是解放以后建设起来的。没有空军也可以打胜有空军的,飞机的作用很小,打不死几个人,破坏不了多少工厂和房屋。你们有人到过伦敦没有?

奥:我们代表团中,有一个同志到过。

主席:伦敦在第二次世界大战中,处于很危险的地位。英国军队从敦克尔克撤退后,军队毫无组织,海防线没有防御了,空军不能防卫伦敦了,海军也不能保卫运输线,陆军、空军损失很大。当时,希特勒的空军猛烈轰炸伦敦,可是并没有破坏了多少,据我们这里看过的同志讲,后方陆海空军很快就恢复了,\marginpar{\footnotesize 57}所以说,空军的作用不大。在战争当中决定胜负的还是步兵、陆军。你们在战争中,敌人有无空军?

奥:他们猛烈轰炸解放区。

主席:作用大不大?

奥:唯一的效果,就是敌人费了很多钱。

主席:我们在解放战争中,国民党也曾大编队轰炸延安,有一次,打死了一条猪,一个人也没有打伤打死,另外一次,在乡下打死了两个老百姓。今天谈的太多了吧!

奥:我们非常渴望听到毛主席的谈话。

主席:我今天是谈了一点历史,以后有机会再谈。


\section[八连颂(一九六三年八月一日)]{八连颂}
\datesubtitle{(一九六三年八月一日)}


好八连,天下传。

为什么,意志坚。

为人民,几十年。

拒腐蚀,永不沾。

因此叫,好八连。

解放军,要学习,

全军民,要自立。

不怕压,不怕迫,

不怕刀,不怕戟,

不怕鬼,不怕魅,

不怕帝,不怕贼。

奇儿女,如松柏。

上参天,傲霜雪。

纪律好,如坚壁。

军事好,如霹雳。

政治好,称第一。

思想好,能分析。

分析好,大有益。

益在哪?团结力。

军民团结如一人,试看天下谁能敌。

\marginpar{\footnotesize 58}

\section[呼吁世界人民联合起来反对美国帝国主义的种族歧视、支持美国黑人反对种族歧视的斗争的声明(一九六三年八月八日)]{呼吁世界人民联合起来反对美国\\帝国主义的种族歧视、支持美国\\黑人反对种族歧视的斗争的声明}
\datesubtitle{(一九六三年八月八日)}


现在在古巴避难的一位美国黑人领袖,美国全国有色人种协进会北卡罗来纳州门罗分会前任主席,罗伯特·威廉先生,今年曾经两次要求我发表声明,支援美国黑人反对种族歧视的斗争。我愿意借这个机会,代表中国人民,对美国黑人反对种族歧视、争取自由和平权利的斗争,表示坚决的支持。

美国黑人共一千九百余万人,约占美国总人口的百分之十一。他们在社会中处于被奴役、被压迫和被歧视的地位。绝大部分黑人被剥夺了选举权。他们一般只能从事最笨重和最受轻视的劳动。他们的平均工资只及白人的三分之一到二分之一。他们的失业比率最高。他们在许多州不能同白人同校读书,同桌吃饭,同乘公共汽车,或者火车旅行。美国各级政府、三K党和其他种族主义者经常任意逮捕、拷打和残杀黑人。在美国南部的十一个州,集居着美国黑人的百分之五十左右。在那里,美国黑人所受到的歧视和迫害,是特别骇人听闻的。

美国黑人正在觉醒,他们反抗日益强烈。近几年来,美国黑人反对种族歧视、争取自由和平等权利的群众性斗争,有日益发展的趋势。

一九五七年,阿肯色州小石城的黑人,为了反对当地公立学校不准黑人入学,展开了剧烈的斗争。当局使用了武装力量来对付他们,造成了震动世界的小石城事件。

一九六〇年,二十多个州的黑人举行了“静坐”示威,抗议当地餐馆、商店和其他公共场所实行种族隔离。

一九六一年,黑人为了反对在乘车方面实行种族隔离,举行了“自由乘客运动”,这个运动迅速地遍及好几个州。

一九六二年,密西西比州的黑人为了争取进入大学的平等权利而进行的斗争,遭到当局镇压,造成流血惨案。

今年,美国黑人的斗争是四月初从亚拉巴马州伯明翰市开始的。赤手空拳,手无寸铁的黑人群众,只是由于举行集会和游行,反对种族歧视,竟然遭到大规模的逮捕和最野蛮的镇压。六月十二日,密西西比州的黑人领袖梅加·埃费斯甚至惨遭杀害。被激怒了的黑人群众,不畏强暴,更加英勇地进行斗争,并且迅速地得到美国各地广大黑人和各附属人民的支持。目前,一个全国性的、声势浩大、波澜壮阔的斗争,正在美国的几乎每一个州和每一个城市展开,而且还在继续高涨。美国黑人团体已经决定在九月二十八日举行二十五万人的回华盛顿的“自由进军”。

美国黑人斗争的迅速发展是美国国内阶级斗争和民族斗争日益尖锐化的表现,引起了美国统治集团日益严重的不安。肯尼廸政府采取了阴险的两面手法。它一方面继续纵容和参与对黑人的歧视和迫害,甚至派遣军队进行镇压;另一方面,又装着一付主张“维护人权”、“保障黑人公民权利”的面孔,呼吁黑人“忍耐”,在国会里提出一套所谓“民权计划”,

企图麻痹黑人的斗志,欺骗国内群众。但是,肯尼廸政府的这种手法,已经被越来越多的黑人所识破。美国帝国主义对黑人的法西斯暴行,揭穿了美国的所谓民主和自由的本质,暴露了美国政府在国内的反动政策和在国外的侵略政策之间的内在联系。

我呼吁,全世界白色、黑色、黄色、棕色等各色人种中的工人、农民、革命的知识分子、开明的资产阶级分子和其他开明人士联合起来,反对美国帝国主义的种族歧视,支持美国黑人反对种族歧视的斗争。民族斗争,说到底,是一个阶级斗争问题。在美国压迫黑人的,是白色人种中的反动统治集团。他们绝不能代表白色人种中占绝大多数的工人、农民、革命的知识分子和其他开明人士。目前,压迫、侵略和威胁全世界绝大多数民族和人民的,是以美国为首的一小撮帝国主义者和支持他们的各同反动派。他们是少数,我们是多数,全世界三十亿人口中,他们最多也不到百分之十。我深信,在全世界百分之九十以上的人民的支持下,美国黑人的正义斗争是一定要胜利的。万恶的殖民主义、帝国主义制度是随着奴役和贩卖黑人而兴盛起来的,它也必将随着黑色人种的彻底解放而告终。

{\raggedleft (《人民日报》一九六三年八月九日)\par}



\section[反对美国——吴庭艳集团侵略和屠杀越南南方人民的声明(一九六三年八月二十九日)]{反对美国——吴庭艳集团侵略\\和屠杀越南南方人民的声明}
\datesubtitle{(一九六三年八月二十九日)}


最近,南越吴庭艳反动集团加紧对越南南方的佛教徒、大、中学校的学生、知识分子和广大人民进行血腥镇压,中国人民对此表示极大愤慨,并且强烈谴责吴庭艳集团的这一滔天罪行。胡志明主席已经发表声明,对于美吴集团的罪恶行为,表示强烈抗议。我们中国人民,热烈支持胡主席的声明。

美帝国主义及其走狗吴庭艳,采取了变越南南方为美国殖民地的政策、发动反革命战争的政策和加强法西斯独裁统治的政策。这就迫使越南南方各阶层人民广泛地团结起来,同美国——吴庭艳集团进行坚决的斗争。

与越南南方全体人民为敌的美国——吴庭艳集团,现在发现他们自己处在越南南方全体人民的包围之中。不论美帝国主义使用什么样的灭绝人性的武器,不论吴庭艳集团使用如何残暴的镇压手段,吴庭艳政权终将不能逃脱众叛亲离、土崩瓦解的结局,美帝国主义终将从越南南方滚出去。

吴庭艳是美帝国主义的一条忠实的走狗。但是,如果一条走狗已经丧失了它的作用,甚至成为美帝国主义推行侵略政策的累赘,美帝国主义是不惜换用另一条走狗的。南朝鲜李承晚的下场,就是一个先例。死心塌地让美帝国主义牵着鼻子走的奴才,到头来只能为美帝国主义殉葬。

美帝国主义破坏了第一次日内瓦会议的协议,阻挠越南的统一,对越南南方公开地进行武装侵略,打了多年的所谓特种战争。美帝国主义又破坏了第二次日内瓦会议的协议,对老挝进行了露骨的干涉,企图在老挝重新挑起内战。除了存心欺骗的人们或者十分天真的人们以外,谁也不会相信,一纸条约会使美帝国主义放下屠刀,立地成佛,或者变得稍为规矩些。\marginpar{\footnotesize 60}

被压迫人民和被压迫民族,决不能把自己的解放寄托在帝国主义及其走狗的“明智”上面。而只有通过加强团结、坚持斗争,才能取得胜利。越南南方人民就是这样做的。

越南南方人民反对美国一一吴庭艳集团的爱国正义斗争,不论在政治上或者军事上,都取得了重大的胜利。我们中国人民是坚决支持越南南方人民的正义斗争的。

我深信,越南南方人民一定能够通过斗争实现解放越南南方的目标,并且为祖国的和平统一作出贡献。

我希望,全世界工人阶级、革命人民和进步人士,都站在越南南方人民一边,响应胡志明主席的号召,支援英勇的越南南方人民的正义斗争,反对美、吴反革命集团的侵略和压迫,使越南南方人民免于被屠杀,并且获得彻底的解放。

\kaoyouriqi{(此文原载《人民日报》一九六三年八月三十日)}


\section[电唁杜波依斯博士逝世(一九六三年八月二十九日)]{电唁杜波依斯博士逝世}
\datesubtitle{(一九六三年八月二十九日)}


杜波依斯夫人:

我沉痛地获悉杜波依斯博士逝世的消息,谨向你表示深切的哀悼。

杜波依斯博士是我们时代的一伟人。他为黑人和全人类的解放进行英勇的斗争的事迹、在学术上的卓越成就,和他对中国人民的真挚友谊,将永远留在中国人民的记忆里。

{\raggedleft 毛泽东\\一九六三年八月二十九日\\(《人民日报》1963.6.30.)\par}


\section[在中央工作会议上对文学艺术的指示(一九六三年九月)]{在中央工作会议上对文学艺术的指示}
\datesubtitle{(一九六三年九月)}


戏剧要推陈出新,不应推陈出陈,光唱帝王将相,才子佳人和他们的丫头保镖之类。



\section[祝贺霍查同志五十五岁生日的电报(一九六三年十月十五日)]{祝贺霍查同志五十五岁生日的电报}
\datesubtitle{(一九六三年十月十五日)}

地拉那
阿尔巴尼亚劳动党中央委员会第一书记
亲爱的思维尔·霍查同志:

在你五十五岁生日的时候,我代表中国共产党和中同人民,并且以我个人的名义,向你,阿尔巴尼亚劳动党的创始者和领导者,阿尔巴尼亚人民敬爱的领袖、中国人民亲密的朋友,致以衷心的、兄弟般的祝贺。

你把自己的全部精力献给阿尔巴尼亚人民反对法西斯的解放斗争和社会主义革命、社会主义建设的事业。以你为首的久经考验的阿尔巴尼亚劳动党正确地领导着英雄的阿尔巴尼亚人民,高举反对帝国主义的旗帜,为反对现代修正主义、反对现代教条主义,捍卫马克思主义和维护国际共产主义运动的团结,作出了卓越的贡献。中国共产党和中国人民,对你和以你为首的阿尔巴尼亚劳动党,对阿尔巴尼亚人民,表示崇高的敬意。

祝阿尔巴尼亚人民在社会主义建设事业之中取得更加辉煌的成就。祝中间两党和两国人民的兄弟友谊万古长青。祝你,亲爱的霍查同志,健康,长寿!

{\raggedleft 中国共产党中央委员会主席 毛泽东\\一九六三年十月十五日\par}

\section[接见阿尔巴尼亚总检察长等的谈话(一九六三年十一月十五日)]{接见阿尔巴尼亚总检察长等的谈话}
\datesubtitle{(一九六三年十一月十五日)}


主席:很欢迎同志们。你们来了几天了?

阿拉尼特·切拉(以下简称切拉):十天了。

主席:走北路来的,还是走南路来的?

切拉:最近是从朝鲜来的,我们住朝鲜住了一个月,在朝鲜是休假。到朝鲜是从北路走的。

主席:他们让你们过?

切拉:让我们经过了,但对我们冷遇。

主席:冷遇啊!请抽烟。(外宾说,不会抽),朝鲜的同志们很好,他们的工作做得很好。

切拉:我们也是这样认为的。

主席:这几个月在全世界,反对修正主义斗争更加发展了。你们坚决地站稳了立场,并且取得了胜利。你们的国家是被他们包围的。\marginpar{\footnotesize 62}你们对全世界真正的马克思列宁主义者是一个很大的鼓舞。

回去时,问候你们的领导同志们好,问候霍查同志、谢胡同志,还有其他同志。

切拉:一定转达。

主席:请喝点茶。除了问候霍查同志、谢胡同志,还有卡博同志、阿利雅、巴卢库等其他同志,也替我转达问候他们。

切拉:一定转达。

主席:你们今年的收成怎么样?

切拉:我们今年的收成情况是:去年冬天雨下得多了,造成今年夏收不好;但今年春耕春种搞的好,所以今年秋收是好的。可以说今年的年成是个好的年成。

索弗克利·巴巴华西里(以下简称巴巴华西里):今年的气候对我们的秋耕秋种是有利的。

主席:很好。今年我们有点灾,一般说来是增产的。如果没有南边的旱灾和北边的水灾,那今年是个大丰收。去年比前年增产一千万吨。今年有好多社会主义国家的农业不好。

切拉:我们来的时候,经过布达佩斯。听说那里的人民意见很大,有抱怨情绪。他们今年的收成不好,政府向美国买粮食。我们去的时候,还没有告诉人民,现在也许告诉了。

主席:你们是否最近就要回国?

切拉:现在预定二十六日离开中国。

主席:今天是十五日,还要到外边去2

张××:还要到上海、杭州、广州、昆明,从那里离开中国回去。

主席:好,到那些地方去看看。

切拉:我们将会看到很多东西。你们的同志对我们的帮助很大。

主席:交换意见嘛。

切拉:是帮助了我们。

主席:互相交换经验。你们阿尔巴尼亚同志到中同来,中国同志都是很欢迎灼。

切拉:我们具体地看到了。虽然在阿尔巴尼亚早已知道你们会欢迎我们的,到这里我们亲眼看到了。

主席:你们是第一次来中国?

切拉:是第一次。

主席:你们两位都是做政法工作的吗?

切拉:不,我是司法工作者,他(指巴巴华西里)是在党中央当视察员。

主席:没有去东北看看?

切拉:时间有限。我们在朝鲜停了一个月,余下的时间就不多了。现在想利用这些时间到中国南方看看。要到中国各个地方都走一趟,这是一件难事。

主席:我刚才从南方回来。南方的秋收还没有完全结束,现在大概差不多了,广东可能还没有收完。你们这次到不到广东去?

黄火星:要去广州。

主席:对付反革命分子,对付贪污浪费分子,单是用行政的办法,法律的办法是不行的,要依靠群众的力量。检察院、法院和公安部门,同党的工作,同群众的工作配合起来,这样比较好一些。比如讲,铺张浪费、贪污分子,一般说靠行政是整不好的,他们就是怕群众。叫做上下夹攻,他们就无路可走了。\marginpar{\footnotesize 63}

要隔几年就整顿一次。即使不是一年一次,几年就要整一次。比如,一个机关,几十人、几百人的机关,过几年就会发生一些问题。我们国家仍然存在着相当严重的阶级斗争。我们过去十年没有抓这个问题了。从去年起,我们准备用几年的时间,把阶级斗争的问题和其他有关的问题抓一下,不然,就很不好搞。有旧的资产阶级残余存在,又产生新的资产阶级分子,就是做投机生意的,贪污的等等。这些人就是修正主义的社会基础,如果现在不整,再过十几年,中国会出修正主义。当然,他们的人数比较是少数,大概百分之几的样子。

我们主要不靠捉人、杀人,主要靠批评教育。但不是说一个也不捉,一个也不杀。

对罪大恶极的,罪恶很大,人民群众要求把他们捉起来,就非捉起来不可;有破坏行为,如杀人放火,破坏工厂、破坏桥梁等少数分子。就是那些普通的破坏分子,他们反对社会主义,比如讲,放谣言啊等等,不是严重的破坏分子都不捉,依靠群众来监督他们,在劳动中去改造。看来,这个方法可能是一个比较好的方法。我们的经验供你们参考,各国的情况不同,你们根据你们的实际情况办事,相信你们会做得更好。

司法工作是不容易做的。检察院,法院和公安部门都是专政的工具。

切拉:毛泽东同志,我们同你们的同志淡了些问题,他们还把我们带到北京监狱去看了,他们向我们介绍了很多情况。我们感到你们教育人的工作做得很出色。

主席:不是每个地方都做得好的。

切拉:也可能你们的工作还有缺点,但基础是正确的。

主席;就是用教育的方法改造人。

巴巴华西里:对于你们用的这种方法,感到受益不少。

主席:第一条,我们要相信群众;第二条,就是这些反革命分子是劳动力。如果把他们捉起来,杀掉,他们的家庭和生产队就丧失了这些劳动力。第二条,对于他们的子女不好做工作,他们的子女要恨我们。所以,用教育的方法来改造,就可以避免了。我们相信依靠群众是可以把他们教育改造好的,他们又是一些劳动力,可以参加社会生产。这样又可以做好他们的子女和家属的工作,使他们不恨我们。

但是并不是每一个地方的工作都做得好。有那么一些同志性急,喜欢用简单的方法解决问题。动不动就把人抓起来或者要求把他杀掉。我们这些同志是把矛盾上交,从下面交到上面来。把矛盾上交的方法并不是一个奸的方法,上面不好处理,还不如放在群众中间,一面教育,一面让他们劳动好。至于有少数分子,你们不是看了北京监狱吗?那是要抓起来的,但也是采取教育的方法进行改造。

他们看了哪个监狱?

黄火星:北京市的监狱。

主席:那些人有工作做吗?

黄火星:那里有塑料厂、鞋厂、袜厂等等。

主席:他们学了技术,放出去以后好劳动。

切拉:我们认为这样做是很对的。我们国家的劳改营里也有些劳动,但没有你们开展得这样广泛。我们监狱的工作是薄弱的,虽然,在我们监狱中关的人很少,是那些非常危险的分子。尽管这样,对这种人还是采取教育的方法。

主席:对!我们第一要相信人是可以改造过来的,在一定的条件下,\marginpar{\footnotesize 64}在无产阶级专政的条件下,一般说是可以把人改造过来的。只有个别人改造不过来。那也不要紧,刑期满了放回去,有破坏活动就再捉回来。有的放出去一次,他照样破坏;放二次,他再破坏;放三次,他再要破坏。是有这样的人,那我们只好把他长期养下去,把他关在监狱的工厂工作。或者把他们家属也搬来,有些刑满了不愿意回去的就把家属也接来。

张××:刑满了可以把家属搬来,安置就业。

主席:对,就是安置就业。有些人是自己不愿意回去的,因为回到当地名誉不好,他在这里已经有很多熟人了,这样就可以把他的家属也搬来,等于迁居了。这样的也不少。

黄火星:北京市那个监狱,也有就业的,有四百多人。

主席:我们把一个皇帝也改造得差不多了。

切拉:我们听说过,他叫溥仪。

主席:我在这里见过他。他现在有五十几岁了,他现在有职业了,听说还重新结了婚。

切拉:听说他还写了本书,叫《我的前半生》。

主席:现在这本书还没有公开发行。我们觉得他这木书写得不怎么好。他把自己说得太坏了,好像一切责任都是他的。其实,应当说这是一种社会制度下的一种情况。在那样的旧的社会制度下产生这样一个皇帝,那是合乎情理的。不过对这个人,我们也还要看。

谈到这里好不好。现在让我们照相吧!


\section[给林彪、荣臻等同志的信(一九六三年十一月十六日)]{给林彪、荣臻等同志的信}
\datesubtitle{(一九六三年十一月十六日)}


林彪、荣臻、××同志:

国家工业各个部门现在有人提议从上至下(即从部到厂矿),都学解放军,都设政治部、政治处和政治指导员,实行四个第一和三八作风。我并建议从解放军调几批好的干部去工业部门那里去作政治工作(分几年完成,一年调一批人),如同石油部那样。据×××同志说:现在已有水利电力部、冶金工业、化学工业部正在学习石油部学解放军的办法在做。我已收到冶金部学解放军的详细报告,他们主张从上到下设政治部、处和指导员。看来不这样做是不行的,是不能振起整个工业部门(还有商业部,还有农业部门)成百万成千万的干部和工人的革命精神的。要这样做,政治干部的来源,我想有四个办法解决:一是从解放军中调出一部分强的而又可能调出的政治干部和懂政治的军事干部送到工、商、农部门中去(先着重工业部门);二是由工业及其他部门派得力同志到解放军的军师团去学习几个月;三是由他们派人到现在莫文骅管的政治学院去当学生,按期毕业,回去工作;四是他们自己抓起来做,将解放军一套思想政治工作条例办法,拿去略加改变(必须适合各个部门的情况),做为自己的东西去实行,现在已有四个部这样做厂。看来这第四项办法是主要的,因为解放军不可能调出很多的干部。但解放军要给他们以帮助,是肯定的,请你们考虑一下是否可行,然后我和中央常委同志问你们淡一下(有个别管工业的同志参加。林有病可不出席),把方针确定下来。这个问题我考虑了几年了,现在因为工业部门主动提出学解放军,并有石油部的伟大成绩可以说服人,这就到了普遍实行的时候了。\marginpar{\footnotesize 65}解放军的思想政治工作和军事工作,经林彪同志提出四个第一、三八作风之后,比较过去有了一个很大的发展,更具体化又更理论化了,因而更便于工业部门采用和学习了。

\kaoyouerziju{一九六三午十一月十六日}

\section[接见古巴诗人、作家和艺术家联合会文学部主任比达·罗德里格斯夫妇的谈话(一九六三年十一月二十六日)]{接见古巴诗人、作家和艺术家联合会文学部主任比达·罗德里格斯夫妇的谈话}
\datesubtitle{(一九六三年十一月二十六日)}

\begin{duihua}
    
\item[\textbf{主席:}] 欢迎古巴同志,欢迎古巴诗人。

\item[\textbf{比达·岁德里格斯(以下简称比达):}] 形式上我是一个诗人,实际上我是一个革命者。

\item[\textbf{皮塔·桑托夫大使(以下简称大使):}] 主席身体好吗?

\item[\textbf{主席:}] 夏天有些感冒。到南方走了一个月,身体好一些。在北京要看很多文件,到外面去可以爬山,可以接近群众。

\item[\textbf{大使:}] 主席脸色比去年还好。

\item[\textbf{主席:}] 去年何时见过?

\item[\textbf{大使:}] 您接见古巴军事代表团时见过。

\item[\textbf{主席:}] 大使身体好吗?你们二位(指比达夫妇)怎样?

\item[\textbf{比达:}] 我们身体很好。我们非常幸福,在中国我们变得更年轻了。

\item[\textbf{主席:}] 你们到中国多久了?

\item[\textbf{比达:}] 两个月了。到南方访问了一个月。我们在上海时,听说主席也在上海。

\item[\textbf{主席:}] 那时正是南方秋收的季节。

\item[\textbf{比达:}] 在上海,我以无比激动的心情参观了鲁迅故居。鲁迅是我们长期以来钦佩的文豪。

\item[\textbf{主席:}] 鲁迅是中同革命文豪。他前半生是民主主义左派,后半生转为马列主义者。

\item[\textbf{比达:}] 不久以前,古巴全国出版社出版了《鲁迅选集》十万册。这在古巴是个大数字,在中国是微不足道。古巴一本书,一般出三万册就很多了。这说明,古巴人民对鲁迅是多么崇敬。

\item[\textbf{主席:}] 鲁迅对帝国主义、封建主义的斗争很明确。他是从那个社会出来的,他知道那个社会的情况,也知道如何去斗争。旧知识分子说他具有二心,是叛徒,所以他写了《二心集》,又说他运气不好,正交华盖运,他就出了一本集子叫《华盖集》,还说他是堕落的文人,他采用了“落文”为笔名。鲁迅对那些人的批判毫不放松。被他批判的人,有一部分转到革命队伍里来,另一部分跟美帝国主义走了。

\item[\textbf{比达:}] 对敌人不能给以喘息的机会。

\item[\textbf{大使:}] 这是真理,可用在生活各个方面。

\item[\textbf{比达:}] 我们很荣幸地访问了您的故乡韶山。

\item[\textbf{主席:}] 那是个小地方穷地方,山多地少,可以去看看。

\item[\textbf{比达:}] 韶山对我们来说,不是值得去看看,而是应该去看看。我们怀着十分激动的心情去瞻仰了一个产生革命根源的地方。\marginpar{\footnotesize 66}

\item[\textbf{主席:}] 过去韶山穷人很多,常侵犯大地主,被大地主称为土匪。我们共产党自成立之日起直到今天,蒋介石还称我们是“共匪”。帝国主义说我们是“好战分子”、“侵略者”。我看这些名字倒不错。他们说我们,第一侵略中国,因为我们反对美国侵略中国;第二侵略朝鲜,因为我们在朝鲜跟美国人打仗;第三侵略西藏;还说我们侵略越南、老挝。大概也说你们在侵略古巴吧!

\item[\textbf{比达:}] 他们是如此说的。

\item[\textbf{大使:}] 的确,鲁迅著作,毛主席著作在“侵略”古巴。

\item[\textbf{主席:}] 哈哈!美国说你们要“侵略”拉丁美洲。我看值得“侵略”一下。

\item[\textbf{比达:}] 美国害怕古巴星星之火燃烧起拉丁美洲的大火。

\item[\textbf{主席:}] 那个哈瓦那大会影响很大,尤其是第二个。古巴革命有两个任务,第一个任务是使古巴能存在下去;第二个任务是帮助拉丁美洲革命取得胜利,让美国的“后院”烧起火来。每个国家都有革命党,有些所谓革命党不革命了。但总有革命的人,如在委内瑞拉、秘鲁、哥伦比亚、乌拉圭、智利、阿根廷、厄瓜多尔、墨西哥、危地马拉、尼加拉瓜等都有革命者。你们国家旁边的两个国家海地、多米尼加也有革命者。很多共产党跟资产阶级跑。这不要紧,总有革命者起来。古巴即如此。《七月二十六日运动》开始并不是马列主义政党吧?

\item[\textbf{大使:}] 不是。

\item[\textbf{主席:}] 有马列主义者参加,如格瓦拉同志。你(指比达)多大年记?

\item[\textbf{比达:}] 五十四岁。

\item[\textbf{主席:}] 你和格瓦拉的年纪差不多吧?

\item[\textbf{比达:}] 大一些。

\item[\textbf{主席:}] 比罗加呢?

\item[\textbf{比达:}] 小一些。

\item[\textbf{大使:}] 格瓦拉四十五岁,他是古巴革命领导成员中最老的一个。

比达;古巴革命几乎是青年人搞起来的。

\item[\textbf{大使:}] 有点像中国革命,开始时领导同志都很年青。

\item[\textbf{主席:}] 一般说来,年青人比较进步,但并非都比马克思、恩格斯、列宁进步。有许多人到后来不革命了。生活把他们拋到后面去了。他们失去了革命的敏感,害怕革命。其称号为革命党,一谈革命就害怕,这算什么革命党。他们不愿接近人民,接近最贫苦的下层人民,即工人与贫农。我们的革命胜利,我们的政权巩固下来,就是依靠工人和贫农。这两部分人占人口的绝大多数。只有这两部分人团结起来,富裕中农就靠拢了。知识分子也有左、中、右,首先团结左派,中间派就跟着靠拢。右派只要反帝、爱国也可以团结,有暂时的作用,没有他们有时也不行。如大学教授、中小学教员,都是旧社会留下来的,不是共产党员或很少是共产党员。十四年来,一部分经过改造,加入了共产党,一部分还保留自己的老观点。文学艺术工作者也是如此。改造他们要花很长的时间,有小部分人基本上不可能改造。不要紧,他们是少数,让他们带着右派观点去见上帝吧!我不清楚你们国家的情况,可能也有这几类人。

\item[\textbf{比达:}] 完全一样。我们也有同样的斗争,同样的改造过程。我们的革命青年能看到革命前景,在革命胜利前就参加革命,也有的在胜利后参加革命。另有一部分人在观望。\marginpar{\footnotesize 67}

\item[\textbf{主席:}] 让他们看看。北京也有一些人在观望;革命到底谁胜谁负?他们要看看。反修斗争究竟如何?他们也要看看。对国内社会主义建设,他们也要看看,每个大风浪,他们总要动摇。

\item[\textbf{比达:}] 他们的“耐心”太大。

\item[\textbf{主席:}] 他们只好看看,我们也只好看看。他们是少数,我们不怕他们,不欺他们,不杀他们。你(指大使)在北京住了几年?

\item[\textbf{大使:}] 三年。

\item[\textbf{主席:}] 你可以看到,我们很少逮捕人,很少杀人,而用群众监督的办法监督坏人劳动。依靠百分之七十、八十、九十的大多数人民群众去监督百分之一、二、三的人劳动。一般说来,坏人大多数在一定条件下能改造成为好人。

\item[\textbf{张××(以下简称张):}] 比达同志见过溥仪。

\item[\textbf{比达:}] 我正要告诉主席,见过溥仪。

\item[\textbf{主席:}] 我也见过他一次,请他吃过饭。他可高兴啦!

\item[\textbf{张:}] 溥仪今年五十七岁了。

\item[\textbf{比达:}] 他给我的印象是确实改造了。他和我长谈他过去的错误,很真诚。

\item[\textbf{主席:}] 他很不满意他过去不自由的生活。当皇帝是很不自由的。

\item[\textbf{大使:}] 向主席提一个问题;反对帝国主义,保卫马列主义原则的斗争今后的前景如何?

\item[\textbf{主席:}] 现在已看得很清楚,帝国主义斗我们不赢,帝国主义是快要灭亡的阶级。美国出了这样的乱子,即一个总统白天被人打死了。你们国家下了半旗没有?

\item[\textbf{大使:}] 没有。

\item[\textbf{主席:}] 奏哀乐没有?

\item[\textbf{比达:}] 没有。

\item[\textbf{主席:}] 我们没有下半旗,没有奏哀乐,也没有打电报吊唁。

\item[\textbf{大使:}] 我们得到国内的指示:如果中国同志下半旗,我们也下半旗。

\item[\textbf{主席:}] 共产党不赞成暗杀的。这不是一个人问题,是制度问题。你们国家改变了制度,不是换了巴蒂斯塔个人,巴蒂斯塔没有死吧?

\item[\textbf{大使:}] 没有。

\item[\textbf{主席:}] 但他对古巴不起作用了。帝国主义内部矛盾是得不到解决的。首先是工人阶级和资产阶级的矛盾无法解决,除非革命。其次为这一垄断集团与另一垄断集团的矛盾。还有所谓盟国问题,美国与欧洲、北美(加拿大)、日本、澳大利亚、新西兰都有矛盾,国与尼赫鲁、铁托也有矛盾。你(指大使)到过印度吗?

大使;在印度呆过几天,深刻的印象是,印度与中国有鲜明的对比。印度走资本主义道路,中国走社会主义道路。

\item[\textbf{主席:}] 印度人民,百分之九十不赞成尼赫鲁的统治。印度半数以上的人很穷苦,印度的状况比缅甸还差。你(指大使)到过缅甸吗?

\item[\textbf{大使:}] 没有。

\item[\textbf{主席:}] 总之,世界在变,变得对革命有利,对反革命不利。为什么有些修正主义分子对肯尼迪之死如同丧亡了父母一样悲痛呢?这反映了世界不稳,一些依靠资产阶级“明智”派的人,“明智”派一倒就吓慌了。\marginpar{\footnotesize 68}这些所谓的“明智”派就是在你们猪湾(即吉隆滩)登陆的指挥者,也是去年十月加勒比海事件的主持人。吉隆滩事件,艾森豪威尔未做过。只有你们有警觉。你们领导者提出一手拿砍刀(割甘蔗,生产用的),一手拿武器的口号很好,不能放弃这口号。美国奈何不了你们。依我看来,是美国怕你们,不是你们怕美国。是大的怕小的,不是小的怕大的。当然,小的也有点怕的,说一点不怕也不真实。美国飞机每天在上空飞,那么多海军陆战队,又有关塔那摩基地,如何不令人担心。但大局说来,是大的怕小的。在南越,美国与南越统治者的力最不小,可是胜利没有希望,美国与南越统治者怕南越人民的游击战。你(指大使)想不想去南越看看?

\item[\textbf{大使:}] 我去过越南民主共和国。

\item[\textbf{主席:}] 到越南了解越南南方人民如何搞游击战,使美国人如此惊慌。你(指比达)就要回去了吗?

\item[\textbf{比达:}] 明天走。

\item[\textbf{主席:}] 可惜没有抽点时间到越南去看一看。

\item[\textbf{比达:}] 这也需要我再来一次。

\item[\textbf{主席:}] 下次来,到越南北方调查南越游击战的情况。再去看看朝鲜人民共和国。看看他们如何搞自力更生的。他们经过大破坏,一九五〇年、一九五一年、一九五二年、一九五三年,美国把朝鲜炸得稀烂。但是,现在不仅工业,而且农业都解决了问题,真该去看看这两个国家。中国经验当然应该研究。朝鲜过去是日本的殖民地。战后只有七百万人口。被美国搞去了二百多万(战争死亡的不算)。从一九五三年下半年至今,十年来已经完全恢复而且发展了。人口也增加很快,由七百万增加到一千二百万。

\item[\textbf{大使:}] 我们占了毛主席很多时间。主席谈到了美国与社会主义国家的矛盾,美国与民族解放运动的矛盾,美国与盟国的矛盾,美国统治集团内部的矛盾等。美同发生肯尼迪事件,主要因素是内部矛盾。这是否说明我们对帝国主义内部矛盾应予以更大的注意。帝国主义一贯玩弄两手,一手和平,一手战争。有人说肯尼迪之死是主张战争的人于的,当然我不是说肯尼迪是明智派。

\item[\textbf{主席:}] 也可能,但现在搞不清楚。美国不暴露谁杀死肯尼迪。可以设想肯尼迪事件对共产党有利一些,民主党受到一次大打击。共和党有好几派,究竟那一派干的,不知道。现在总统约翰逊能否当选还是个问题。他的威望不及肯尼迪。这个人要钱,名声不好,是个大贪污分子。我国搞五反,他是五反对象。(众笑)

\item[\textbf{张:}] 主席该休息了吧!

\item[\textbf{主席:}] 没有关系,多谈一会儿。总的看来,世界在变,变得对反革命不利,对革命有利。几十年的历史证明了这一点。古巴的变化,你们自己看到了。古巴就在美国门口几十海里,起了变化,谁能说古巴没有起变化呢?!中国变化了,这也是事实,我是知道的,你们也看到了。北京,帝国主义和蒋介石就来不了嘛!非洲也起了很大变化,亚洲也起了很大变化如印尼、柬埔寨等,小小的柬埔寨竟敢拒绝“美援”,为何有此大胆?只因为;第一,美国要他的命,第二,国内人民的觉悟;第三,法国、中国帮助他们。柬埔寨人口跟古巴差不多,有六百万人口。现在发生了怪现象,美国每年“援助” 柬埔寨三千万美元,过去共“援助”了三亿几千万美元,柬埔寨感到美“援”是危险的东西。它在收买干部,腐化干部,并搞颠覆活动。这就可以解释为何柬埔寨拒绝美“援”。好吧,就谈到这里。\marginpar{\footnotesize 69}
\end{duihua}


\section[对柯庆施同志有关上海曲艺革命化改革总结报告的批示(一九六三年十二月十二日)]{对柯庆施同志有关上海曲艺革命化改革总结报告的批示}
\datesubtitle{(一九六三年十二月十二日)}


此件可以一看。各种艺术形式――戏剧、曲艺、音乐、美术、舞蹈、电影、诗和文学等等,问题不少,人数很多,社会主义改造在许多部门中,至今收效甚微。许多部门至今还是“死人”统治着。不能低估电影、新诗、民歌、美术、小说的成绩,但其中的问题也不少。至于戏剧等部门,问题就更大了。社会经济基础已经改变了,为这个基础服务的上层建筑之一的艺术部门,至今还是大问题。这需要从调查研究着手,认真地抓起来。

许多共产党人热心提倡封建主义和资本主义的艺术,却不热心提倡社会主义的艺术,岂非咄咄怪事。

{\raggedleft 毛泽东\\一九六三年十二月十二日\par}


\section[关于加强相互学习,克服故步自封骄傲自满的指示(一九六三年十二月十三日)]{关于加强相互学习,克服故步自封骄傲自满的指示}
\datesubtitle{(一九六三年十二月十三日)}


现将湖南省委李瑞山、华国锋两同志,一九六三年十一月六日写的一个参观农业生产情况的报告,以及附在上面的湖南省委一九六三年十二月七日写的一个指示,发给你们研究。中央认为,这种虚心学习外省、外市、外区优良经验的态度和办法,是很好的,是发展我国经济、政治、思想文化、军事、党务的重要方法之一。故步自封、骄傲自满,对于自己所管区域的工作,不采取马克思列宁主义的辩证分析方法(一分为二,既有成绩,也还有错误),只研究成绩一方面,不研究缺点错误一方面。只爱听赞扬的话,不爱听批评的话,对于外省、外市、外区,别的单位的工作很少有兴趣组织得力的高级干部去虚心地加以考察,便于和本省、本市、本区、本单位的情况结合起来加以改进,永远陷于本地区、本单位这个狭隘世界,不能打开自己的眼界,不知道还有别的新天地,这叫做夜郎自大。对于外国人、外地人以及中央派下去的人只让看好的,不让看坏的,只向他们谈成绩,不向他们谈缺点及错误,要淡也谈得不深刻,敷衍了事。中央多次对同志们提出这个问题、认为一个共产党员必须具备对于成绩与缺点、真理与错误这两分法的马克思主义的辩证思想。事物(经济、政治、思想、文化、军事、党务等等)总是做为过程而向前发展的,而任何一个过程,都是由矛盾着的两个侧面互相联系而又互相斗争而得到发展的。这应当是马克思列宁主义的常识。但是中央和各地同志中,有许多人却很少认真用这种观点去思索工作。他们的头脑长期存在着形而上学的思想方法而不能解脱。所谓形而上学,就是否认事物的对立统一,对立斗争(两分法),矛盾着的事物在一定条件下互相转化,走向他的反面,这样一个真理。就是人们故步自封、骄傲自满,只见成绩,不见缺点,只愿听好话,不愿听批评话。自己不愿批评(对自己的两分法),更怕别人批评。中央有几十个部,明明有几个工作成绩,工作作风较好的部,例如石油部,别的部却熟视无睹,永远不到那里考察研究请教一番。一个部所管企业、事业,明明有许多厂矿、企业事业、科学研究所及其人员工作得较好,上面都不知道,因而不能提倡人们向那些单位学习。同志们,中央在这里所说的犯有形而上学错误的同志,是一部分同志,但是,应当指出有大量的好同志却被那些高官厚禄、养尊处优、骄傲自满、故步自封、爱好资产阶级形而上学的同志,亦即官僚主义者所不知道,现在必须加以改革。凡不虚心对本地、本单位、本人作分析,对别单位、对别人作分析,拒绝马克思主义辩证分析方法的同志,要进行同志式的劝告和批评,以便把不良情况改变过来。把向别部、别省、别市、别区、别单位的好经验、好作风、好方法学过来,这样一种方法定为制度,这个问题是个大问题。请你们加以讨论,以后还要在中央工作会议及中央全会上加以讨论。湖南省委在过去一个时期内,不做调查研究,主观的下达许多指示,往往灌的东西多,由下面反映上来的真实情况少,因而脱离群众,产生很大困难。从一九六一年起,他们开始改变了,以致情况大好起来。但是他们认为还远不如广东和上海,所以派遣大批省、地、县三级干部,还有省和市的干部,组成了两个考察团分别到上海和广东去学习。这一点请你们注意研究,是否可以这样办。中央认为不但可以而且应当这样办。如有不同意见,请你们提出。



\section[谦虚——戒骄(一九六三年十二月十三日)]{谦虚——戒骄}
\datesubtitle{(一九六三年十二月十三日)}


固步自封,骄傲自满,对自己所管的区域的工作,不采取马克思主义的辩证分析方法(一分为二,既有成绩,也有缺点错误),只研究成绩一方面,不研究缺点错误一方面;只爱听赞扬的话,不爱听批评的话,对于外省、外市、外地区、别的单位的工作,很少有兴趣组织得力的高级、中级干部去虚心地学习,认真地加以考察,以便和本省、本市、本地区、本单位的情况结合起来,加以改进,永远限于本地区、本单位这个狭隘的世界,不能打开自己的眼界,不知道还有别的新天地,这叫做夜郎自大。对外国人、外地人或中央派下去的人,只让看好的,不让看坏的。只向他们谈成绩,不向他们谈缺点及错误。要谈也谈得不深,敷衍几句了事。中央屡次对同志们提出这个问题,作为一个共产党人,必须具备对于成绩与缺点、真理与错误这两分法的马克思主义的辩证思想。事物(经济、政治、思想、文化、军事、党务等等)总是作为一个过程而向前发展的。而任何一个过程,都是因矛盾着的两个侧面互相联系,又互相斗争而得到发展的。这应当是马克思主义者的普通常识。但是中央和各地同志中有很多人都很少认真地运用这个观点去思索,去工作,他们的头脑长期存在形而上学的思想方法而不能解脱。所谓形而上学,就是否认事物的对立统一,对抗斗争(两分法),矛盾着、对立着的事物,在一定条件下互相转化,走向他们的反面这样一个真理,就使人固步自封,骄傲自满,只见成绩,不见缺点,只愿听好话,不愿听批评的话,自己不愿批评(对自己的两分法),更怕别人批评。

“满招损,谦受益”这句话,站在无产阶级立场上,从人民的利益出发考虑,是一个真理。

(一)骄傲自满可以在各种情况下,以各种不同的形式产生和滋长。但是一般说来,通常在胜利的情况下,就更容易产生和滋长骄傲自满情绪。这是因为当处在困难的时候,一般是容易看到自己的弱点,也是比较谨慎的,而且客观上的困难摆在面前,不虚心谨慎也不行。可是每当胜利的时候,由于有人感谢,有人赞扬,甚至过去的敌人也会掉过脸来奉承一番,阿谀一番,因而就容易为胜利的环境冲昏头脑,而全身轻飘飘起来,真以为“天下从此定矣。”我们党深深地懂得,越是在胜利的时候,骄傲自满的细菌就越是容易袭击党。

(二)产生骄傲自满情绪,一类是在胜利的情况下产生的,那就是胜利冲昏头脑,自以为了不起;另一类是在无特殊胜利,亦无特殊失败的平常情况下产生的,他们经常以比上不足,比下有余来安慰自己,并原谅自己的不进步,他们还善于用“没有功劳,也有苦劳”,“二十年媳妇熬成婆”等等来自我陶醉;再一类是在落后的情况下产生的,即虽然已经落后了,也还是骄傲。他们认为“我们工作虽然没有做好,比过去总是好多了”,“某某同志或某某单位还不如我们呢!”他们每每炫耀自己的历史,三句话不到就是“想当年……”讲起来眉飞色舞。

(三)只要我们稍微忽视一下群众的力量,我们就会骄傲起来,只要我们眼界狭隘一些,只看到局部而看不到整体,我们就会骄傲起来;只要我们稍微把成绩估计得高了一些,把缺点估计得低一些,我们就会骄傲起来;只要我们的主观认识落后于客观事物的发展,我们就会骄傲起来。

(四)骄傲自满的情绪,从本质上说,乃是从个人主义的立场上引伸出来的,同时它又培养和滋长了个人主义,因而骄傲自满的本质就是个人主义。

(五)就阶级根源来分析,骄傲自满基本上是剥削阶级思想,其次则是小生产者的思想。

(六)小生产者就其本身是劳动者一面而言,他们是具有很多优点的。他们勤劳朴实,刻苦谨慎和实事求是。但是就其本身是小私有者一面而言,则他们是个人主义的,更重要的是由于他们的劳动条件和劳动方式是落后的生产工具,分散经营,眼界不广,见闻不多。因此,他们往往看不到集体的力量,而只是看到个人的力量。另一方面,他们也容易满足,当他们取得一点微小的成绩以后,就产生“这个不错了”,“这也到顶了”,“该享享福了”以及“比上不足,比下有余”等等这一类思想。

(七)骄傲自满是在资产阶级唯心世界观的基础上派生出来的,它使人在看待周围客观事物时经常违背事物发展的规律,把人们引向失败的道路。唯物主义的历史观证明:社会发展的历史,不是个别英雄人物的历史,而是劳动人民群众的历史。可是骄傲自满的人,总是夸大个人的作用,居功自傲,而忽视、低估群众的力量。

(八)因此,骄傲自满在本质上是反马克思列宁主义的,是反对党的辩证唯物主义和历史唯物主义的世界观的。

(九)骄傲自满的人往往不能忘情于自己的许多优点。他们一方面把自己的许多缺点掩盖起来,而另一方面又企图把别人的许多优点抹煞掉。他们经常拿别人的缺点和自己的优点相比,从而私自窃喜,看到人的优点则又觉得“没有什么了不起”,“算不上个啥”。

(十)事实上,把自己估计得越高,所得的结果就越坏。俄国大文豪列夫·托尔斯泰就幽默地说过:“一个人就好像是一个分数,他的实际才能好比分子,而他对自己的估计好比分母,分母越大,则分数的值就越小。”

(十一)谦虚,它是每一个革命工作者都应有的美德。因为谦虚对人民的事业有利,而骄傲自满却会把人民的事业引向失败。所以,谦虚也是对人民事业负责的一种表现。

(十二)一个人要真正称得上一个名副其实的革命工作者,必须做到下列两点:首先,他们必须尊重群众的创造,肯听群众的意见,把自己看成是群众的一员,毫无自私自利之心,毫不夸大自己的作用,实心实意地为群众工作。这种精神就是鲁迅先生所说的,“俯首甘为孺子牛”的精神,也就是谦虚的美德。

(十三)其次是他必须有不屈不挠、永远向前的精神,时时刻刻保持清醒的头脑,对新鲜事物具有敏锐的感觉,缜密的思考能力。因此他们必须始终保持谦虚的态度,胜不骄,败不馁,不贪天下之功,也不满足已有的成绩,这种精神就是实事求是的精神,也是高贵的谦虚美德。

(十四)一个人如果能够认真的从工作中、生活中和其他实践斗争中去学习,经常总结自己的思想和行动,无情而坚定地和骄傲自满情绪作斗争,并且毫不保留地加以彻底的克服,那他是完全可以锻炼成为一个具有谦虚美德的人。


(十五)真正具有谦虚的高尚品德的人,他必须是满腔热情地无条件地为党、为人民、为集休的事业而忠诚不渝地积极工作的人。他之所以积极工作,不是为了炫耀自己,也不是为了获得某种奖励和荣誉,不夹杂任何自私自利的欲望和要求在内,而是全心全意地为着给人民带来愉快与利益。因此,他总是埋头苦干地作着对党对人民的革命事业有利的工作,从不抛头露面,从不计较自己的地位、自己的声望、自己的待遇,他不仅不在别人面前夸耀自己的功勋和成就,而且在自己内心中也不让这些功勋和成就占地位,他全付精力所考虑的是更好地为人民工作。

(十六)真正的集体主义者,为什么必须要求自己具有谦虚的美德呢?

第一、因为他懂得,他的一切知识和成就的获得,虽然他自己也尽了一定的力量,但更重要的是由于群众的努力,没有群众的帮助和支持,他就不可能获得知识,也就不可能获得工作上的成就。作为一个集体主义者,他就认为不应该抹杀群众的功绩,不应该掠他人之美,贪别人之功。因而他觉得自高自大是可耻的。

第二、因为他懂得,他所学习到的一些知识,所作的一些工作,在整个知识宝库里和整个革命的工作当中,仅仅是“沧海中之一粟”,是非常渺小的,因为革命的知识和革命的工作,又是在不断发展的。他既然是一个集体主义者,他便要用他宝贵的生命去最大限度的获得对人民有用的知识,最大限度的对革命事业贡献自己的力量。因此,他就觉得不应该固步自封,替自己关起进步的大门。

第三、因为他懂得,整个革命事业像一架大机器,是由大小各式轮盘、螺丝、钢架和其他机件紧密结合而构成的,谁也少不了谁,他既然是一个集体主义者,他便觉得应该尊重每一个人的工作,尊重每一个人的成就,就像尊重自己的工作和成就一样。为了把革命工作做得更好,他就必须使自己的工作和别人的工作紧密配合,他感到他离不开集体,他热爱自己的伙伴。因此,他便必然会用谦虚的态度来待人接物,而不会对任何人狂妄自大。

第四、因为他懂得,一个人的眼界往往是窄小的,能够看到的范围总是有限的,而革命知识和革命工作的范围却是极为广阔的,并且内容又是非常丰富的、非常复杂的。因此,他便进一步懂得了任何人总难免会有若干缺点,会犯若干错误,这些缺点和错误又常常不是自己全部及时觉察得到的。他既然是一个集体主义者,为了要把革命工作搞好,为了对人民负责,他就得要求自己看得更深更广,要求能及时地发觉自己的缺点和错误,以便迅速改正。因此他便要虚心恭谨地向别人学习和请教,他便要诚恳地欢迎别人对他的批评。

由此可知,真正的从集体利益出发的人,是必须具备谦虚的精神的。而谦虚实质上就是高度的革命热情,强烈的群众观点,旺盛的进取精神和科学的实事求是的态度的集中反映。

(十七)克服骄傲自满和培养谦虚品质的另一个根本的方法,就是要努力提高自己的共产主义觉悟,这就必须加强马克思列宁主义理论的学习。为什么?

(十八)因为马克思列宁主义理论,可以帮助我们科学地认识世界,认识个人与群众,个人与集体,个人与组织,个人与党的相互关系。正确地认识人民群众和个人在革命斗争中的作用。马克思列宁主义告诉我们:劳动人民是社会财富的创造者和革命斗争的基本力量。我们要在中国建立社会主义和共产主义社会,只有依靠工人阶级及其先锋队领导下的亿万劳动人民的无穷无尽的创造力量。至于个人,在革命事业中只不过是一个小小的螺丝钉,马克思列宁主义告诉我们:任何一个成就都是集体力量的结晶,个人是离不开集体的,个人想做一点事业,如果没有党的领导,没有组织和人民群众的支持,就将会寸步难行,一事无成。如果我们真正深刻地理解了人民群众和个人在历史上的作用及其相互关系,我们便会自觉地谦虚起来。

因为马克思列宁主义的理论可以提高我们对前途和方向的认识,开阔我们的眼界,使我们的思想从狭隘范围里解放出来。当人们的眼界只能看到脚下,看不到高山和大洋的时候,他就会像“井底之蛙”那样自负不凡的。但当他抬起头来,看到宇宙之大,事物之变化无穷,人类事业之雄伟浩壮,人才之多和知识之无极限,他便会谦虚起釆。我们所从事的事业,是天翻地覆的大事业,我们不仅要看到我们自己的眼前的工作和幸福,而且要看到整个的、长远的、全面的工作和幸福。马克思列宁主义帮助我们朝前看,而不是朝后看;帮助我们全面地、客观地看问题,而不是片面地、主观地看问题,因而就能帮助我们克服那种因小小成就、小小胜利而自满自足的小生产者的思想,而促使我们孜孜不倦、力求进步的渴望,同时又可以帮助我们克服唯心的主观主义的思想方法。

(十九)谦虚和自卑不是同义词,谦虚并不等于小视自己。因为谦虚本身是实事求是的态度,是正视客观现实的进取精神的表现。而自卑却是一种非实事求是的、缺乏自信力的、对困难采取畏缩的态度的表现。

自卑和自夸,自高自大,同样都是错误的,都是以主观主义为其思想垦础的,是对自己的两种极端的主观主义的错误的估计。那些自高自大的人,离开了客观实际,把自己估计得过高,夸大了自己的实际能力和作用,因而他总是自命不凡,自以为了不起,他就不再前进了,他也不能及时地吸收什么新鲜的事物了,他于是就不可避免的要犯错误。那些自卑的人,虽然从表面上看和自高自大的人相反,但同样也离开了客现实际,把自己估计得过低,忘记了自己还可以努力提高自己,还可以从工作中锻炼自己,过分地降低了自己在革命事业中所已经起的和可能起的作用。于是,便从而丧失了前进的勇气和自信。松懈了斗争意志。

总之,无论是自高自大或自卑,同样都是错误的估计了自己在革命事业中的作用,都是非实事求是的非科学的态度,因而都是错误的,都会使革命事业遭受损失。

所以,我们不仅要坚决反对骄傲自满,自高自大一类的习性,而且要严格地把谦虚与自卑的界限划分开来,免得由一个极端又倾向于另一个极端。



\section[给林彪同志的信(一九六三年十二月十四日)]{给林彪同志的信}
\datesubtitle{(一九六三年十二月十四日)}


林彪同志:

你的来信早收到了。身体有起色,甚为高兴。开春以后,宜到户外散步。你对两个文件的看法是正确的。国内外形势均已向好,均已走上正确的轨道。可以预计,更大的发展,是会到来的。关于农村社会主义运动的两个文件,十一月中旬就发出去了,本月上旬各省已有反映,在一些地方的生产大队全体人员及五类分子,(有的多到七百多人听讲)开会时向他们宣读,分组讨论,效果很好,军队如能照此办理,那也一定会好的。由团营两级理解力强的军政干部向连队一切人员分几次宣读、讲解、讨论,由群众提出意见,讲解员解答疑难问题,是会成为一个大规模社会主义政策教育运动的。(军师二级也可派一部分强的干部下去,杂在团营干部中,向连队宣读、讲解,作为军官当兵的一种形式。至于高级首长,例如××、××、杨×、廖××、许世友、黄永胜、刘亚楼等等同志,也应该择一、二连队去作一、二次讲解。讲解要联系环境,先要对准备去讲解的连队情况作一些大略的调查。)不知已按你的意见作了布置没有?据我从北京几个军事基层单位的少数同志接触,他们尚不知道此事,没有看过文件,也没有听过宣读。此事其实不难,只要由总政下一通知,叫各军区各兵种印发文件,每一个支部一本,传下去。由团营合组宣讲队伍,分头下达到连队,照本宣讲,以排或班为单位,进行讨论,自由发言,容许讲不同的意见,甚至反对意见,就可以在一个短时期内(例如几个星期)出现一个高潮,提高政策水平。(因为不能耽误操课任务,宣读文件只能夹在正常操课中间去作,所以需要几个星期,如果暂停操课,那就一、二个星期够了。)一次宣讲之后,过几个月再作一次宣讲,使人们得到更深理解。军队一动起来,还可以抽出一些干部帮助地方,向工厂、农村作宣讲工作。这样可以使军民联合起来,人民了解和拥护军队,军队了解和帮助人民,更是一大好事。是否可以如此做,请你们和罗、肖诸同志商酌处理。
祝好!

{\raggedleft 毛泽东\\一九六三年十二月十四日\par}

曹操有一首题为《神龟寿》的诗,讲养生之道的,很好。希你找来一读,可以增强信心。又及。

附:曹操《神龟寿》

神龟虽寿,犹有竟时。腾蛇乘雾,终为土灰。老骥伏枥,志在千里;烈士暮年,壮心不已。盈缩之期,不但在天;养怡(一作恬)之福,可得永年。〔幸甚至哉,歌以咏志。〕



\section[论反对官僚主义(一九六三年)]{论反对官僚主义}
\datesubtitle{(一九六三年)}


我们这些机关高高在上,官僚主义是容易犯的弊病,有了官僚主义,必然对上闹分裂主义。比如“跃进号”抓了才清楚的。下边也闹地方主义,根子都是官僚主义。前年下放权利那么多,文件是我起草的,造成了分散主义。有人说要反对,顶不住,问题还是官僚主义。

官僚主义是一个剥削阶级遗留下来的东西。党外部长和我们一道,也希望借重一下你们的归劝。

官僚主义,思想上表现和个人主义、分散主义、本位主义、自由主义、命令主义、事务主义、组织上的宗派主义、自由主义相结合的,因此官僚主义也必然联系到这些主义。总之,要集中的反对剥削阶级的思想作风。三月一日,中央五反指示中说:“官僚主义在抬头”。我看带有普遍性。

我尝归纳官僚主义二十种表现:

一、高高在上,孤陋寡闻。不了解下情,不作调查研究,因而脱离实际,脱离领导。不作政治思想工作,不抓具体政策,上脱离领导,下脱离实际,一旦发号施令,必然祸国殃民。这是脱离领导、脱离群众的官僚主义。

二、狂妄自大,骄傲自满,空谈政策,不抓业务,主观片面,不听人言,蛮横专断,强迫命令,不顾实际,盲目指挥。这是强迫命令式的官僚主义。

三、从早到晚,忙忙碌碌,一年到头,辛辛苦苦,但是作事不调查,对人不考察,发言无准备,工作无计划。这是无头无脑,迷失方向的官僚主义。

四、官气熏天,唯我独尊,不可亲近,望而生畏,对干部颐指气使,作风粗暴,动辄骂人。这是官老爷式的官僚主义。

五、不学无术,耻于下问,浮夸谎报,弄虚作暇,欺上瞒下,文过饰非,功则归己,过则归人。这是不老实的官僚主义。

六、不学政治,不钻业务,遇事推委,怕负责任,办事拖拉,长期不决,工作上讨价还价,政治上麻木不仁。这是不负责任的官僚主义。

七、遇事敷衍,得过且过,与人为争,老于事故,上捧下拉,两面俱圆,八面玲珑。这是作官混饭吃的官僚主义。

八、政治学不成,业务钻不进,人云亦云,语气无味,尸位素餐。领导无方,滥竿充数。这是满预无能的官僚主义。

九、糊糊涂涂,混混沌沌,人云亦云,得过且过,饱食终日,无所用心。这是糊涂无用的官僚主义。

十、送文件不看就批,批错了不承认,文件听别人读,别人读他睡着了,心中无数,不和人商量事情推来推去不负责,对下不懂装懂,指手划脚,对同级貌合神离,同床异梦。这是懒汉误事的官僚主义。

十一、机构庞大,人事庞杂,层次重迭,浪费资产,人多事乱,遇事团团转,不务正业,人多事少,工作效率低。这是机关式的官僚主义。

十二、指示多不看,报告多不批,会议多不传,报表多不用,往来多不谈。这叫“五多”的官僚主义。

十三、图享受,好伸手,走“后门”,怕艰苦,一人得道鸡犬升天,一人作官全家享福,内外不一请客送礼。这是特殊化的官僚主义。

十四、官越作越大,脾气越来越坏,房子越来越大,陈设越来越好,生活要求越高,供应越多,分配东西越多,价钱越低。这是摆官架子的官僚主义。

十五、自私自利,假公济私,以私作公,监守自盗,知法犯法,多吃多占,不退不还。这是自私自利的官僚主义。

十六、争名夺利,向党伸手,对待遇斤斤计较,对工作挑肥拣瘦,对同志拉拉扯扯,对群众漠不关心。这是争权夺利的官僚主义。

十七、多头领导,互不团结,政出多门,工作散乱,上下隔离,互相排挤,既不集中,又不民主。这是闹不团结的官僚主义。

十八、目无组织,任用私人,结党营私,互相包庇,个人利益、派别利益高于一切,损大公肥小私。这是闹宗派的官僚主义。

十九、革命意志衰颓,政治生活蜕化,靠老资格吃饭,摆官架子,好逸恶劳,游山玩水,既不用脑,又不动手,不关心国家和人民利益。这是蜕化的官僚主义。

二十、助长歪风邪气,纵容坏人坏事,打击报复,压制民主,欺压群众,包庇坏人,敌我不分,作奸犯科。这是助长歪风邪气的官僚主义。

总之,使干部脱离实际,脱离群众,漠视群众利益,使党的路线政策受损失。不作为普通劳动者,不同群众同甘共苦,政治上空谈,不老实,不负责任,不能、无用,埋头于事务主义,搞特殊化,自私自利,闹不团结,搞宗派,最后发展蜕化变质。

官僚主义的思想根源、阶级社会根源、历史根源、思想根源,是剥削阶级的思想作风,既有资产阶级的个人主义、实用主义,也有封建的家长制。(红楼梦四大家族,农奴主四十人,官僚占三分之二的人。)

阶级社会根源;新的资产阶级,老的资产阶级,还有城乡封建势力。在国际上有资本主义包围,而且帝国主义、修正主义联合起来了。

历史根源:我们的革命打碎了旧的国家机器,建立了新的国家机器,但旧的统治势力,传统影响,旧人员包下来,政策是对的,但带来了副作用,一九五一年“三反”重点是反贪污,一九五七年重点是反右,去年主要是批判了分散主义,所以历年来没有把官僚主义当成重点来搞。现在滋生官僚主义的土壤是肥沃的,也是修正主义、教条主义的土壤。



\section[在聂荣臻同志汇报时的谈话(一九六三年十二月)]{在聂荣臻同志汇报时的谈话}
\datesubtitle{(一九六三年十二月)}


〔谈到松辽平原的经验时〕石油部是第一个运用解放军的一套办法,工业部门都要学习解放军,设立政治工作部门。用政治工作来保证建设任务的完成。石油部是学习解放军的经验,像连队的政治工作一样,不脱离业务。

石油部比较单纯,一机部复杂(指产品),要调些人到工业部门作政治工作,解放军是出人材的学校。

〔汇报科学技术十年规划任务时〕要打这一仗,科学技术是生产力,过去打上层建筑,也是为了发展生产力,不打这一仗,生产力无法提高。要以革命的精神来搞科学技术工作。

〔汇报基础理论时〕不搞理论是不行的,要搞一批理论队伍,也包括社会科学。

〔汇报留学生工作时〕只派留学生,国内却固步自封,不向好的单位学习。

〔汇报向国外进口书刊时〕多少外汇?(答:××万美金。)不多嘛。社会科学的买吗?(答:也买。)影印外文杂志,广告不要弄掉。

〔汇报十年基建投资××亿时〕十年××亿,每年×亿,不多嘛!

〔汇报到受激光发射时〕要有些人专门搞这事,长远来搞。从数量上看,人家比我们多,我们搞不过人家。但是从历史上看,攻防两手,防我们要考虑。比如城墙,筑起来是为了防御。

〔汇报治理黄淮海问题时〕这个研究工作,要几万人来搞。

〔汇报医疗卫生问题时〕感冒药要认真解决。

〔汇报探索性工作时〕允许公开犯错误,但是发现错误要批评,又要鼓励,允许人家公开改正错误。

〔谈到朝鲜的战备工作,搞了××万里山洞作地下工厂时〕我们也要作蠢事情。

〔最后谈到三大革命时,问科学实验的含义是什么〕我讲的科学实验,主要是讲自然科学。社会科学、哲学、政治经济学、军事科学是不能搞科学实验的。商品,价值法则不能搞科学实验。战争不能搞科学实验。辩证法不能搞科学实验。理论法则是概括出来的。军事演习不能搞实验室。社会科学的一部分在一定意义上也可说科学实验。

\section[同×××谈人民日报要学习解放军(一九六四年一月八日)]{同×××谈人民日报要学习解放军}
\datesubtitle{(一九六四年一月八日)}


人民日报要学习解放军。

新闻工作要学习解放军。

人民日报要看一下林总对解放军报的意见。

学习解放军,学习石油部,主要学习他们怎样抓思想政治工作。

文汇报、解放日报在这方面,很可以看看。他们把政治思想工作的纲抓起来了。

人民日报要注意发表学术性文章,发表历史、哲学和其他学术文章。

抓哲学,要抓活哲学。我写文章不大引用马克思、列宁怎么说。报纸老引我的话,引来引去,我就不舒服。应该学会用自己的话来写文章。列宁就很少引人家的话,而用自己的话写文章。当然不是说不要引人家的话,是说不要处处都引。



\section[慰问恩克鲁玛总统的信(一九六四年一月九日)]{慰问恩克鲁玛总统的信}
\datesubtitle{(一九六四年一月九日)}


阿克拉加纳共和国总统,人民大会党主席兼总书记克瓦米·恩克鲁玛阁下:

首先,我对于加纳人民的敌人又一次用卑鄙无耻的手段来暗害阁下的罪恶行为表示极大的愤慨,同时对您平安脱险感到无限的高兴。请你接受我个人和中国人民的最亲切的慰问。

帝国主义和反动派对非洲各国的人民领袖和著名政治家一次又一次地进行暗害阴谋活动表明:他们是不甘心自己在非洲的失败的,是决不会自动退出历史舞台的。无论过去、现在和将来,帝国主义和反动派总是要千方百计地阻挠和破坏非洲各国人民的独立和进步的事业。事实已经证明,而且还将继续证明:帝国主义和反动派的疯狂挣扎只会使非洲各国人民更加提高警惕,更加坚定地为反对帝国主义和新老殖民主义、为维护民族独立和争取自己国家的繁荣进步而奋斗。

中国人民将永远支持加纳人民和非洲各国人民的正义斗争。祝加纳共和国在阁下的领导下,在各方面取得新的成就。祝非洲各国人民在反对帝国主义和新老殖民主义的基础上,加强团结,胜利前进。

再一次向您表示最良好的祝愿!

<p align="right">毛泽东

一九六四年一月九日</p>



\section[就巴拿马人民反对美帝国主义的爱国斗争对《人民日报》记者发表的谈话(一九六四年一月十二日)]{就巴拿马人民反对美帝国主义的爱国斗争对《人民日报》记者发表的谈话}
\datesubtitle{(一九六四年一月十二日)}


目前巴拿马人民正在英勇地进行的反对美国侵略、维护国家主权的斗争,是伟大的爱国斗争。中国人民坚决站在巴拿马人民的一边,完全支持他们反对美国侵略者,要求收回巴拿马运河区主权的正义行动。

美帝国主义是全世界人民最凶恶的敌人。美帝国主义不仅对巴拿马人民犯了严重的侵略罪行,精心一意地策划扼杀社会主义的古巴,而且一直在掠夺和压迫拉丁美洲各国人民,镇压这些国家的民族民主革命斗争。

在亚洲,美帝国主义霸占着中国的台湾,把朝鲜南部和越南南部变作它的殖民地,对日本实行控制和半军事占领,破坏老挝的和平、中立和独立,阴谋颠覆柬埔寨王国政府,对亚洲其他国家进行干涉和侵略。它最近又决定把美国舰队派到印度洋,威胁东南亚各国的安全。

在非洲,美帝国主义加紧推行新殖民主义政策,力图取代老殖民主义者的地位,掠夺和奴役非洲各国人民,破坏和扑灭民族解放运动。

美帝国主义的侵略政策和战争政策,也严重地威胁着苏联、中国和其他社会主义国家。它还力图对社会主义国家推行“和平演变”政策,实行资本主义复辟,瓦解社会主义阵营。

美帝国主义甚至对它在西欧、北美和大洋洲的盟国,也实行“弱肉强食”的政策,力图把它们踩在自己的脚下。

美帝国主义称霸全世界的侵略计划,从杜鲁门、艾森豪威尔、肯尼廸到约翰逊,是一脉相承的。

社会主义阵营各国人民要联合起来,亚洲、非洲、拉丁美洲各国人民要联合起来,全世界各大洲的人民要联合起来,所有爱好和平的国家要联合起来,所有受到美国侵略、控制、干涉和欺负的国家要联合起来,结成最广泛的统一战线,反对美帝国主义的侵略政策和战争政策,保卫世界和平。

美帝国主义到处横行霸道,把它自己放在同全世界人民为敌的地位,使它自己越来越陷于孤立。美帝国主义手里的原子弹、氢弹,是吓不倒一切不愿意做奴隶的人们的。全世界人民反对美国侵略者的怒潮是不可阻挡的。全世界人民反对美帝国主义及其走狗的斗争一定会取得更加伟大的胜利。

{\raggedleft (《人民日报》一九六四年一月十三日)\par}



\section[谈报纸革命化问题(一九六四年一月)]{谈报纸革命化问题}
\datesubtitle{(一九六四年一月)}


毛主席认为,办好报纸的根本问题,是报社人员的革命化问题。革命化就是肃清一切封建思想、资产阶级思想影响的问题。有些错误思想是容易看得出来,有些就不容易看出来,比如文汇报的《胆与识》一文错误容易看出来。(按:此文表扬“年青一代”敢于批评将军。)不革命化的另一表现是头脑中缺乏辩证法,往往把问题说死了。

革命化是立场、观点问题。报纸究竟要宣传那些东西,究竟要系统宣传那些东西,能不能选择得好,都是革命化问题,解放日报的干部那么多,为什么写不出好东西?要革命化一定得下去,参加实际斗争的锻炼,要使人革命化,同时要使机务革命化。

报纸一定要抓思想。主席最近谈到解放日报的好处,说:“好在比较注意抓思想,比较抓思想工作。”一张报纸从头到尾都要思想化。你们(按:解放日报)过去和现在,这一点作得不够,常常把一些好的东西,当另件处理,有时候又把一些一般的东西当作大东西处理了。

依靠什么人办报?要依靠社会主义积极分子办报,好的东西应该让群众自己写。

你们要善于抓住活的东西,把他提到理论高度,宣传活的哲学。报纸要有各种议论。

国际消息。主席说:“外宾接待消息,地方报纸可以少登一些,这样,可以让出地方来宣传活人活事活哲学。”

要总结报纸工作的经验。

一九六四年主席对柯庆施同志讲:

革命化的三个意思:(1)要反对封建主义、资本主义思想;(2)要下去实践,同工农结合;(3)要学习唯物辩证法。



\section[就最近日本人民反对美帝国主义的爱国正义斗争发表谈话(一九六四年一月二十七日)]{就最近日本人民反对美帝国主义的爱国正义斗争发表谈话}
\datesubtitle{(一九六四年一月二十七日)}


日本人民在一月二十六日举行的反美大示威,是一次伟大的爱国运动。我谨代表中国人民,向英勇的日本人民,致以崇高的敬意。

最近,日本全国掀起了大规模的群众运动,反对美国F——105D型核飞机和核潜艇进驻日本,要求撤除一切美国军事基地和撤走美国武装部队,要求归还日本的领土冲绳,要求废除日美“安全条约”等等。所有这些,都反映了日本全体人民的意志和愿望。中国人民衷心地支持日本人民的正义斗争。

日本在第二次世界大战以后,在政治上、经济上、军事上一直遭受美帝国主义压迫。美帝国主义不仅压迫日本的工人、农民、学生、知识分子、城市小资产者、宗教界人士、中小企业家,而且还控制日本的许多大企业家,干预日本的对外政策,把日本当作附庸国。美帝国主义是日本民族的最凶恶的敌人。

日本民族是一个伟大的民族。它是绝不会让美帝国主义长期骑在自己头上的。这些年来,日本各阶层人民反对美帝国主义侵略、压迫和控制的爱国统一战线不断地扩大。这是日本人民反美爱国斗争胜利的最可靠的保证。中国人民深信,日本人民一定能够把美帝国主义者从自己的国土上驱逐出去,日本人民要求独立、民主、和平、中立的愿望,一定能够实现。

中日两国人民要联合起来,亚洲各国人民要联合起来,全世界一切被压迫人民和被压迫民族要联合起来,一切爱好和平的国家要联合起来,一切受美帝国主义侵略、控制、干涉和欺负的国家和人士要联合起来,结成反对美帝国主义的广泛的统一战线,挫败美帝国主义的侵略计划和战争计划,保卫世界和平。

美帝国主义从日本滚出去,从西太平洋滚出去,从亚洲滚出去,从非洲和拉丁美洲滚出去,从欧洲和大洋洲滚出去,从一切受它侵略、干涉和欺负的国家和地方滚出去!

<p align="right">(《人民日报》一九六四年一月二十八日)</p>



\section[接见阿尔及利亚民族解放阵钱代表和法律工作者代表团的谈话(一九六四年一月二十八日)]{接见阿尔及利亚民族解放阵钱代表和法律工作者代表团的谈话}
\datesubtitle{(一九六四年一月二十八日)}


主席:代表团有几个人?

本·阿卜杜拉:十二个人。

主席:来了多少时间?哪一天到北京来的?

阿卜杜拉:一月十七日到北京的,十三日就离开阿尔及利亚了。

主席:准备到什么地方去参观?

阿卜杜拉:还要到上海去两天,杭州两天,广州两天,然后回国。

主席:到广州从南路回去吗?

阿卜杜拉:是的。

主席:你们是从南路来的,还是从北路来的?

阿卜杜拉:从南路来的,经过开罗、仰光和昆明。很希望在中国多呆几天,因为国内有工作,已经延长了几天,现在不得不回去了。主席:你们胜利了,我们很高兴。你们的胜利是个典型,大胜利,是少数战胜了多数,法国几十万军队被打败了。为什么少数能变成多数呢?原因就是你们有群众,人民群众能够战胜帝国主义。

阿卜杜拉:我们胜利还不久。

主席:阿尔及利亚现在是有一千二百万人口吗?有人也说是一千万。

阿卜杜拉:一千二百万。不过阿尔及利亚从来没有进行过人口调查。过去法殖民统治者把阿尔及利亚人民当狗看待,多一、二百万、少一、二百万,对他们说算不了什么。

主席:过去你们临时政府告诉我,阿尔及利亚有一千万人口,包括一百万法国人。那么本国人只有九百万,战争上又牺牲了一百多万,只有八百万不到一点。人民一定会胜利。人口在打过仗之后也只会增加不会减少,我是根据我们的经验说这话的。

我们党在开始的时候只有五十七个党员,在一九二一年召开第一次党代表大会时只有十二个代表。现在十二个人中只剩下两个人,那十个人或者牺牲了,或者叛变了,可是革命力量发展了,越来越大了。我们的革命一共化了二十几年才胜利,从一九二七年打仗打到一九四九年,整整二十二年。一九二七年大革命失败时,革命力量受到很大损失,由五万党员降到几千党员。那时我们没有经验,蒋介石叛变革命,同我们打了十年国内战争。大革命失败是因为我们党内产生了右倾。然后在战争中党壮大了,军队壮大了,根据地也扩大了,有了三十万党员,三十万军队,根据地的人口也有几千万。这时又产生了“左”倾,他们要打大城市,社会政策不对,只要工人、农民,民族资产阶级、小资产阶级都不要。对民族资产阶级的政策不对,没有把民族资产阶级与买办资产阶级分开,提出的一切政策也就不对,结果又失败了,被迫举行了万里长征。万里长征不是我们愿意干的,这是政策失败了,没办法,从南方跑到北方。这一来三十万军队剩下不到三万,被搞掉百分之九十以上,三十万党员只剩下几万,所有大城市的组织差不多都完了,又是一个大失败。可是这两次失败,一九二七年右倾失败,及以后的“左”倾失败,是好的还是不好的呢?

阿卜杜拉:主席所讲的这点,在我们来中国以前就认识了,失败有消极的一面,但也有另一面,可以从失败中得到经验,所以也可以说失败是成功的基础。

主席:我们就是这么看。没有这两次失败,中国革命不能胜利,不能总结经验,反对右倾机会主义,又反对“左”倾机会主义,使我们能采取正确的政策。又团结又斗争的政策。

第一次失败,是没有看到朋友变成敌人,只讲团结不讲斗争。第二次失败,是只讲斗争不讲团结,把小资产阶级、民族资产阶级全看成敌人。这两次党内关系也不正常。我们就总结经验,所以抗日战争时期,我们的政策就比较正确了。抗日战争经过八年,由两万五千军队发展到一百二十多万,根据地的人口由十几万发展到一亿多。胜利时日本人跑了,美国人又来了。蒋介石有四百多万军队,一切大城市和铁路、矿山资源都在他手里。国民党向全国解放区发动进攻,占领我们很多城镇和乡村,我们把延安都失掉,许多外国朋友也认为我们不行了。延安是个小城市,这个小城市只有几千人口,是山区,我们拿它做根据地,后来这个根据地也失掉了,很多人都认为共产党没有希望了。后来我们釆取正确的退却政策,退却一年的样子,退却过程中消灭了国民党八个旅,一直到一年以后,我们才可以举行反击。解放战争一共用了三年半时间,蒋介石跑到台湾去了。现在蒋介石还在联合国里“代表中国”,我们还被叫“土匪”。(全场大笑)法国人昨天同我们建立外交关系。你们胜利之后,法国人才承认你们,那也好嘛!有些国家至今还不承认我们,意大利、比利时、西德、日本,主要是美国,他们的政策是孤立我们,那时我们与你们一样。你们在没有同法国签定埃维昂协定之前,你们的情况也不那么好,好像很孤立的样子,其实你们并不孤立,有什么孤立的?突尼斯的关系与你们搞的不好,不久前摩洛哥的关系也搞的不好,我看没有什么要紧,同情你们的人很多很多,中国人民同情你们,整个亚洲、非洲、拉丁美洲的绝大部分人民都是同情你们的。你们在欧洲也有朋友,法国人中也有你们的朋友。

法国政府过去不是你们的敌人吗?你们的敌人也是我们的敌人,帝国主义是我们的共同敌人。世界上的事情是会起变化的。当法国人跑了的时候,你们多困难,没有粮食,没有教员,没有医生,没有药品,工厂开工资金不足,现在你们也还在困难阶段,没有工程师,没有技术员;要有地质工作人员,要勘探石油,要勘探各种矿产,但是要有个过程才能建立这样的地质工作队伍。总之,你们是会搞起来的,由没有到有,由少到多,由不会、不懂,到学会,到懂,我是根据我们的经验讲这个话的。比如军队,我们没有,现在有了,你们也没有军队,现在也有了。又如打仗,谁会?我就不会打仗,还不是学会的。军队的事,打仗的事,能由没有到有,由不会到会,为什么经济建设和文化建设我们就不能搞起来?困难可以克服,不会的人可以学会,没有的东西是可以有的,不要那么多迷信,要破除迷信,只要肯干,我是不大信迷信的,过去也有过迷信。很多是敌人教会了我们的,必须团结国内人民,只要依靠人民就有出路。

脱离群众是不行的,是不是这样?过去的一些领导人,你们不要了,我们有一个时候不知道是什么原因,后来才清楚,就是他们脱离了群众。是这样吗,不知对不对?

阿卜杜拉:主席的分析很正确。阿尔及利亚的领导人于一九六二年二月在的黎波里举行了会议,在会议上起草和通过了一个纲领。那时出现了一个多数和少数,多数中不包括过去临时政府的某些领导成员及其他部门担任领导工作的干部。对中国朋友们说,这没有什么秘密,世界报也谈到过。那就是少数人不愿意服从多数,不愿意接受多数的观点,可以明确指出,当时出现的多数反映了阿尔及利亚广大人民的意志。后又经过选举产生了国民会议,由国民会议任命了现政府。

主席:革命中总有一部分人,他们可以反对帝国主义,反对殖民主义,但再进一步他们就不干了。我们也发生过这样的事,当反对帝国主义的时候他们干,反对封建主义的时候他们干,他们自认为共产党员,但实际上是资产阶级民主革命者,一到搞社会主义他们就不干了,他们就反动了,有这么一部分人。哪一个党内都有这样的事情,特别是在革命转变时期,是不可避免的。过去多数是进步的,少数是落后,如果政权掌握在落后分子手里,那就危险了。你们现在正要建立一个党,不久开党的代表大会?

阿卜杜拉:在这个问题上,阿卜杜拉希德·日拉伯兄弟是我们党中央领导成员,他可以谈一谈。

阿卜杜拉希德:阿尔及利亚民族解放阵线政治局决定举行一次党的代表大会。刚才本·阿卜杜拉兄弟谈到一九六二年的黎波里会议,那是在战争中阿尔及利亚内部分歧斗争的表现。当时说的分歧是斗争方法上的分歧,是釆取什么方式进行斗争。在一九六二年的黎波里的阿尔及利亚民族解放阵线代表大会上出现分歧意见,领导方面存在的真正矛盾表现出来了,方法上的分歧不是实质,实质是思想意识上的分歧。在的黎波里会议上的潮流,是进步的潮流,是向前进的潮流。但能否说的黎波里会议之后什么矛盾都解决了呢?不能这么说,由于我们斗争的需要,在经济建设时期又出现了新的矛盾,因为我们的斗争是继续发展的,需要革命的纲领,革命的领导集团,使我们不断前进。

我们这样的思想意识是在我们过去和现在的斗争中不断建立起来的,将来还要向前进步,要建设社会主义,要做出这样的选择,必须是明确的选择,我们正在准备报告,将来在党代表会上提出。主席:什么时候开?阿卜杜拉希德:日期还没有定。在即将召开的党代表大会上,特别要总结过去的经验,武装斗争的经验和独立后在本·贝拉兄弟领导下进行经济建设的经验,准备在大会上总结一九五四年以来的革命经验,这些总结是重要的,将使我们了解过去的经验是什么,这样也能使我们明确在各种各样的思想意识中,今后选择哪一方面。这个大会将为我们奠定不可动摇的社会主义基础。这是独立之后举行的第一次党代表大会,是历史性的会议,是阿尔及利亚人民历史转折点的会议,对于阿尔及利亚的政治、经济建设是很重要的。

主席:你是他们一起来的吗?

阿卜杜拉希德:是一起来的,分开接待。

主席:只一个人吗?

阿卜杜拉希德:一个人。本·阿卜杜拉兄弟领导的法律代表团是来中国学习法律方面的经验的。我是中共中央接待,负责另一方面的工作,学习另一方面的经验的。

主席:不能学习,我们也是在摸索过程中,有很多错误和缺点,要全面分析再接受,不要认为中国什么都是好的,中国也有不好的一面。有进步的一面,有落后的一面,工作中有正确的一面,也有错误的一面,我们这几十年就是这么过来的,经常犯错误,改正错误。不隐瞒错误,你们(指陪见人)不要隐瞒错误,只介绍正确的东西,不介绍走弯路和错误的方面。中国农业很落后,工业现在与先进国家比还差的远,在我们的社会上和党内,干部也有变化,有的变成了贪污分子,实际上是资本家,我们的政策是对他们进行教育,进行社会主义教育运动。可惜时间不够,不能细致地介绍我们的政策。像你们这样的党,这样的国家,我们应该把一切经验毫无保留地介绍给你们。对待反革命分子和犯人的政策,我们也犯过错误。你们看见过我们的监狱没有?

阿卜社拉:我们看过北京监狱。

主席:我们的监狱也有办得好的,也有办得不好的。(指陪见人)北京监狱是那个办得好的吧?办得不好的不让你们看。(向陪见人)要让他们看一个办得坏的。就是应该这样介绍,有好的,也有坏的。人民公社也有办得好的,也有办得不好的,我们现在要做的,就是使办得不好的办得好起来。军队里也可以看看。军队是专政工具的主要工具,你们是搞法律工作的,不看军队不好,你们国家如果没有军队,你们的法律工作还能搞吗?没有军队的保卫,你们就不能生存。还要看看警察和公安部队。(向陪见人)与军委和公安部联系一下,让他们看看军队、警察、公安部队,以及民兵的情况。你们大家都是搞法律工作的,专门在法律条文上作文章是作不出什么来的。光靠监狱解决不了问题,要靠人民群众来监视少数坏人,主要不是靠法院判决和监狱关人,要靠人民群众中多数监视、教育、训练、改造少数坏人。监狱里关很多人不好,主要劳动力坐牢就不能生产了。今天我们讲不完了,没有时间,只能介绍些要点。你们还过几天走?

阿卜杜拉:我们在北京只有二十四小时了。

主席:有些问题可以到上海去解决。可以告诉他们,说是我讲的,要介绍正确的方面,也要介绍错误的方面,要介绍好的方面,也要介绍坏的方面。这样才是好的方法。

阿卜杜拉:从现在起我们就采取这种方法。

主席:(对陪见人)给各地要讲清这个问题。

阿卜杜拉:我也应该表示,在这里介绍的情况没有什么隐瞒,他们向我们介绍了胜利的经验,也讲了失败的原因。

主席:好嘛,应该这么作。我讲我们过去的历史,就讲了我们是怎么犯错误这一点,错误对我们很有益处,教育了我们。从成功的方面学得的经验,也从失败的方面学得经验。我看你们总结你们的历史也会是这样的,总有代表比较正确的一面。回去后你们的本·贝拉总统和其他朋友们,你们真的搞社会主义,我很高兴,那我们不仅是反对帝国主义和封建主义斗争的同志,而且是搞社会主义的同志。搞社会主义要团结大多数人,团结一切反对帝国主义,反对封建主义的人,团结一切干社会主义的人。

阿卜杜拉:我代表阿尔及利亚法律工作者代表团全体成员向主席表示感谢,感谢给予我们的荣誉和骄傲。代表团和阿尔及利亚人民向主席和中国人民表示亲切的祝愿。现任请允许我们把这把刀送给主席,这是阿尔及利亚武装斗争胜利的象征。

主席:很有意义的礼物,这东西是对付敌人的。



\section[几段插话(一九六四年一月)]{几段插话}
\datesubtitle{(一九六四年一月)}


(谈到工业问题,要建立“托拉斯”问题时)

目前这种按行政方法管理经济的方法,不好,要改。比如说,企业里用了那么多的人,干什么!人是要吃饭的,要消耗的,不像孙猴子吃铁砂,拉铁屎。用那么多人,就是不按经济法则办事。

生产出来的物质,必须按合同收购。商业部门说是因为计划变了,不收购;是谁变的计划?国家变的,就由国家收购。总之,不能积压在工厂里头。

(谈到企业管理不好的原因时)

解放军一道命令,可以通到底,行得通。说解放军所以搞得好,是由于共产党领导。那经济也是共产党领导的呀,为什么搞得四分五裂?抢钱(利润分成)、抢物质(物质分成)、发生冲突(闹关系),多年不得解决?商业为什么不能按经济渠道经营管理,为什么只能按行政设置机构?打破省、专、县界嘛!就是要按经济渠道办事。企业跟军队一样,一通到底嘛!党委管思想、管政治、管仲裁(冲突)、管人、实行监察嘛!

(谈到专业化和协作、制造主机和辅机的关系时)

主机和辅机的矛盾,是对立物的统一。资本家办一个工厂,管工厂的资本家总是少数,做工的工人总是多数。少数和多数的矛盾,也是对立物的统一。

(讲到树立标兵,领导上不要埋没英雄时)

我说高官厚禄、固步自封有的是,比如有的部的报告就说他们也有多少模范、典型、标兵,但多年没有发现,这不是被固步自封的官僚主义所压住了?!现在被压住的,没有发现的还很多。要改变这种情况,抓活人活事。

(讲到报纸应该怎样宣传时)

文汇报、光明日报办得比较好,有些议论,也有科学研究、哲学、历史研究等方面的文章。人民日报单纯些。(它担负人家不担负的任务。)报导国际消息,一礼拜几次也就可以了。要写点新鲜事物,活人活事。加上一版,专门报导学习解放军,学习石油部。写“活”哲学,不要写“死”哲学。现在写文章,引语太多,看了就心烦,少引一些可以。我写文章很少引马、恩、列、斯。要写活的哲学。许多老粗懂得哲学,最近解放军发现一个炊事员写的文章说,他从前烧饭每顿二斤四两煤炭,后来经过调查研究,掌握了煤炭的客观规律,每顿只用六两煤炭。

(讲到石油部的经验时)

过去封建皇帝时代,还可以据理力争。我们解放军有一条,真理在谁手里,就服从谁,在伙夫手里就听伙夫的,在班长手里就服从班长的。真理不是谁的官大、官小来决定。

(在和日本人谈到对垄断资本反美的态度时)

(一)日本垄断资本现在有变化,八番钢铁托拉斯,富士钢铁托拉斯,东海渔业托拉斯,反美反得很紧张。过去只纺织业来反,现在垄断资本家也反了。这一件事,对日本共产党来讲,是个新问题。过去讲得太死了,切记不要讲死。我讲了法国的经验,戴高乐掌握了民族独立的旗帜,掌握了反美的旗帜,而法共则去欢迎艾森豪威尔,为肯尼廸流泪,所以戴高乐上台,倒不了。苏加诺也是大资本家,但他掌握了反美和民族独立的旗帜,所以他能维持。而印尼共产党就把这些接过来。希特勒上台后,宣布废除凡尔赛条约、收复失地,抓民族独立的旗帜,德国共产党台尔曼未表态,失败了。日本垄断资产阶级是两个拳头作战,一手反美,一手反共。你们要给他放松一头。日本垄断资本家提出反美,要美国撤走基地,要反控制,你们应该把这些接过来。日本目前的形势是民族矛盾超过了阶级矛盾。就是要跟垄断资本家在反美问题上搞统一战线。我们也给四大家族(中国垄断资本家)搞过统一战线。这样做,可能麻痹了工人。但我们也要想想工人的心理,资本家反对美国控制,工人有事做,工人是同意的。我们跟蒋介石的经验也是。日本侵略中国,民族矛盾上升为主要矛盾。我们同蒋介石是采取两手:又团结,又斗争,斗争也是为了团结,以斗争求团结。反共高潮只三次,一打,他退回去了,我们还是联合。不联合,哪能发展这样快,这样大,从二万五千人发展到一百二十万人?如果日本不投降,这一政策还要继续下去。我们在这中间发展,我们要趁这个空子发展嘛!

(二)为什么苏联出了修正主义?这个问题是带普遍性的,许多人脑子里有这个问题。解答这一问题,还是要用阶级、阶级分析。这是从斯大林时候就包下来的。联共党史写了,宪法也写了,只提工人、农民、知识分子全民一致,不提工人、农民、知识分子以外的不一致,不提还有资本主义分子,还有未改造的知识分子;此外,也不提还会产生新的资产阶级分子,高薪阶层,工人贵族。问题不在于赫鲁晓夫一个人,而在于这个基础,基本问题,即有新的资本主义产生的基地。所以,只说反赫鲁晓夫不行,打倒一个,还有第二、第三、第四个,……。不只苏联出了修正主义,欧洲十几个国家都出了修正主义,代表什么?代表工人贵族。我说工人阶级的广大贫苦阶层出马克思列宁主义,少数工人贵族出修正主义。

(讲到不用武力来解决领土争端问题时)

台湾海峡,两重性。这是国内问题,谁人都不得干涉。我们也是两手,和平解放或者武力解放。不管那一手都是内政问题,谁也不能干涉。我们用武力解决,并未说死。

(讲到苏联现在请南斯拉夫以观察员身份列席经互会议时)

经互会要去,南斯拉夫参加,我也去。现在形势变了,赫鲁晓夫的指挥棒不灵了。南斯拉夫你参加,我也参加。将来要把经互会转到抵抗苏联的控制。大西洋公约国家反对美国控制,以法国为代表嘛!东欧国家也反对苏联控制,并且正在发展。……现在指挥棒不灵了……。要看到这一形势。

(谈到介绍工人中间的模范人物时)

老粗出人物。我们军区司令百分之九十都是老粗,行伍出身。但是,没有几个知识分子也不行。自古以来,能干的皇帝大多是老粗。汉朝刘邦是封建皇帝里边最厉害的一个。刘静劝他不要建都洛阳,要建都长安,他立刻去长安;鸿沟划界,项羽引退,他也想到长安休息,张良说什么条约不条约,要进攻,他立即听了张良的话,向东进。韩信要求封假齐王,刘邦说不行,张良踢了他一脚,他立即改口说:“他妈的,要封就是真齐王,何必假的。”而项羽则有三次错误,鸿门宴不听范增的话,那时他有四十万军队,刘邦只有十万人。鸿沟协定他认真了,建都徐州(那时叫彭城)。南北朝宋、齐、梁、陈,五代梁、唐、晋、汉、周,很有几个老粗。文的也有几个好的,如李世民。我们中央上过大学的也很少,过去上了大学的就算做官了,还革什么命。现在有许多新的好的典型,要提倡。



\section[对《人民日报加强学术文章的报告》的批示(一九六四年二月三日)]{对《人民日报加强学术文章的报告》的批示}
\datesubtitle{(一九六四年二月三日)}


××、××同志:

《人民日报》历来不重视思想理论工作,哲学社会科学文章很少,把这个理论阵地送给《光明日报》、《文汇报》和《新建设》月刊。这种情况必须改变过来才好。现在他们有了改的主意了,请书记处讨论一下,并给他们解决干部问题为盼。



\section[对《中央关于传达石油工业部关于大庆石油会战情况通知》的批示(一九六四年二月五日)]{对《中央关于传达石油工业部关于大庆石油会战情况通知》的批示}
\datesubtitle{(一九六四年二月五日)}


大庆油田的经验虽然有其特殊性,但是具有普遍意义,他们贯彻执行了党的社会主义建设总路线,坚持政治挂帅,坚持群众路线,系统地学习和运用解放军的政治工作经验,把政治思想、革命干劲和科学管理紧密结合起来,把工作做活了。这是一个多、快、好、省的典型。它的一些主要经验,不仅在工业部门中适用,在交通、财贸、文教各部门,在党、政、军、群众团体的各级机关中也都适用或者可做参考。



\section[接见新西兰共产党总书记威尔科克斯夫妇时的谈话(一九六四年二月九日)]{接见新西兰共产党总书记威尔科克斯夫妇时的谈话}
\datesubtitle{(一九六四年二月九日)}


什么都是可以分开的。譬如说,过去认为原子不能再分,后来科学发达了,知道原子可以再分成原子核和电子,还利用电子发电。后来又进了一步,知道原子核还能再分,而且那里面复杂得很,可以分成好多部分。电子能不能再分呢?理论上是可以分的,虽然实际上我们对电子世界还懂得很少,科学家还没有把它分开,但是,列宁在《唯物主义和经验批判主义》中已经说过,电子是可以再分的,我不懂这门科学,但是我相信这个道理。

科学是无限的。无限大的世界和无限小的世界都有无限的工作可做。……

任何社会无论今天和将来,都一分为二,总是由矛盾推动社会发展。在现在,是阶级斗争推动社会前进。我们的社会主义不是已经十四年了吗?但是还是阶级斗争在推动我们的社会前进。过了几十年,或者几百年以后,建成了社会主义,进入了共产主义社会,那时是什么推动社会前进呢?那时是先进集团同落后集团的斗争。一万年以后还是先进和落后,正确和错误。不可能设想铁板一块,都正确,没有一点错误。没有错误,那里有正确?

社会是复杂的。根据马克思主义,根据对立统一的规律,一百万年或一千万年以后,还是一分为二的,还是有正确和错误。社会结构也是分成几百个阶段或几千个阶段前进的。我就不信在一百万年以后,所有的人就都是那么文明、高尚、道德,都是圣人,没有肯定,没有否定。

而且一个阶段代替另一个阶段也是要通过斗争的。……

在中国,还是有保持原状的人,还有人反对我们,还有人表面上不反对我们,但实际上反对社会主义制度;还有人表面上服从社会主义改造,实际上心里不满意。将来也还会有这样的人。改造社会和改造人是永远也做不完的工作。一个社会总是一分为二,有正面,有反面。如果我们这一代什么都改造完了,那么下一代干什么?如果说再过一万年社会改造得十全十美,每个人都成了马克思、恩格斯、列宁那样的人,那么一万年之后的人干什么呢?一万年之后,还是会有量变、质变,还是会有飞跃,还是会有社会革命。我就不相信在进入共产主义后,社会经济将永远是同样的一种经济,人永远是同样的人。现在当然还没有人谈这个问题,但是我就不相信会是那样。

实际上,社会总是复杂的,一个统一体总是可以分的,至少可以一分为二。

现在,美帝国主义代替了英、日、法等帝国主义。在泰国、老挝、南越,它代替了法国;在南朝鲜和台湾,它代替了日本。它到处都要伸手,我们就要反对。有人骂我们用肤色来划分人。如果真是这样,为什么我们不同蒋介石团结起来?为什么你们不支持美国的统治者?为什么×××同志不喜欢英帝国主义?各国都是用阶级来划分的。

……

任何社会,任何事物都是一分为二。不仅是资产阶级同无产阶级一分为二,而且无产阶级也一分为二。有共产党,有社会民主党;有修正主义者好,没有就不好。这不是人为的,是自然的。

事物都是一分为二的,国际共产主义运动也势必一分为二,从来是一分为二的,从马克思的时候起,就是如此。

就是你们和我们,也是一分为二的。你刚才不是讲,你们过去以为社会主义阵营和共产主义运动只应该有团结而不应该有斗争吗?这种思想是唯心主义的,是形而上学的,这是你们思想中错误的一方面。但是你们思想中正确的那个方面占主要地位。因为你们是真正为人民服务的。你们的党的脱产干部很少,你们的大多数干部都是靠自己劳动吃饭的,这样就减少了官僚主义。我们正在做这方面的工作。我们这里的官僚主义可不少咧!应该让威尔科克斯同志看看我们关于城市五反的文件,看看官僚主义危害多大。五反就是反对官僚主义、反对分散主义、反对铺张浪费、反对贪污盗窃、反对投机倒把。这个文件应该翻译成外文,给你们带回去,让你们的中央委员都能看到。

中国的社会是一分为二的,谁也不能说中国是不能分的。不能说只有光明的一面,而没有黑暗的一面。不能说只有正确的一面,而没有错误的一面。不能说只有马列主义的一面,而没有修正主义的一面。不能说只有廉洁的一面,而没有贪污盗窃的一面。否则就不符合事实。

对外国同志介绍情况时,只说好的,不讲坏的,这是不正确的,不是真正马列主义的态度。……

列宁、斯大林的时候是肯定阶段,现在是否定阶段。但是事物的发展会走向否定的否定,修正主义也会走向它的反面,势必如此。广大的苏联人民、党员和干部是反对修正主义的,但需要时间,或者十年,八年,或者更多一点的时间。

我是从修正主义对我们的好处和帮助谈起,联想到我们历史上错误的政策和机会主义路线对我们的好处,以及党外的敌人,帝国主义和蒋介石对我们的封锁,断绝经济关系和大举进攻对我们的好处。这个道理在马克思主义者的队伍中还没有得到充分的发挥,有人总是认为敌人的压迫、杀人、被打入地下、党的组织缩小等等,是坏事。认为帝国主义在全世界猖狂进攻,是坏事。日本过去进攻中国,占领了大半个中国。有些日本资产阶级的代表,现在见到我们就道歉,说:对不起得很,我们过去侵略了你们。我说,不,没有你们的侵略和占领大半个中国,我们不能胜利;你们的侵略激起全中国人民都起来反对你们。就是因为日本占领了大半个中国,所以中国人民都起来了。

所以,日本侵略中国,有两重性:有坏一面,也有好的一面。坏的一面是,杀中国人,破坏村庄,抢人物质。好的一面是,激起了中国人民,强迫中国人民团结起来。否则我们的一百二十万军队建立不起来,一亿人口的解放区建立不起来,也不能解决经济问题,不能解决吃饭、穿衣、住房子的问题。枪炮问题也不能解决。我们的枪炮是他们送来的,后来是美国人送来的。我们军火的主要来源是美国。其次是蒋介石的兵工厂,我们自己的兵工厂很少。到一九四九年,我们的军火工业才开始发展,有了十万工人,开始制枪,小型的炮,步枪和机关枪,制造子弹和炮弹。在开始的时候,我们就是小米加步枪,没有飞机、坦克、大炮,没有外援。但是我们打败了有飞机、坦克、大炮和有大量美援的敌人。



\section[关于出版三十本马、恩、列、斯著作的指示(一九六四年二月十三日)]{关于出版三十本马、恩、列、斯著作的指示}
\datesubtitle{(一九六四年二月十三日)}


×××同志:

一、此件看过,很好,可以立即发下去。

二、三十本书,大字线装,分册(一部大书分十册、八册,小书不分册,中书仍要分册),请你督促迅速办下去。希望今年办成,可以吗?你想一下告我为盼。每部印一万、两万分,好吗?我急于想要看这种大字书。



% \input{5-0}
% \input{5-0}
% \input{5-0}


\section{革命委员会的三条基本经验}\datesubtitle{(一九六八年三月三十日)}

革命委员会的基本经验有三条:一条是有革命干部的代表,一条是有军队的代表,一条是有革命群众的代表,实现了革命的三结合。革命委员会要实行一元化的领导,打破重迭的行政机构,精兵简政,组织起一个革命化的联系群众的领导班子。
{\raggedleft (转摘自《人民日报》、《红旗》杂志、《解放军报》一九六八年三月三十日社论《革命委员会好》\par}
\section{支持美国黑人抗暴斗争的声明}\datesubtitle{(一九六八年四月十六日)}

最近,美国黑人牧师马丁·路德·金突然被美帝国主义者暗杀。马丁·路德·金是一个非暴力主义者,但美帝国主义者并没有因此对他宽容,而是使用反革命的暴力,对他进行血腥的镇压。这一件事,深刻地教训了美国的广大黑人群众\footnote{原文为“黑大群众”,更正为“黑人群众”。},激起了他们抗暴斗争的新风暴,席卷了美国一百几十个城市,是美国历史上前所未有的。它显示了在两千多万美国黑人中,蕴藏着极其强大的革命力量。

这场黑人的斗争风暴发生在美国国内,是美帝国主义当前整个政治危机和经济危机的一个突出表现。它给陷于内外交困的美帝国主义以沉重的打击。

美国黑人的斗争,不仅是被剥削、被压迫的黑人争取自由解放的斗争,而且是整个被剥削、被压迫的美国人民反对垄断资产阶级残暴统治的新号角。它对于全世界人民反对美帝国主义的斗争,对于越南人民反对美帝国主义的斗争,是一个巨大的支援和鼓舞。我代表中国人民,对美国黑人的正义斗争,表示坚决的支持。

美国的种族歧视,是殖民主义、帝国主义制度的土产物。美国广大黑人同美国统治集团之间的矛盾,是阶级矛盾。只有推翻美国垄断资产阶级的反动统治,摧毁殖民主义、帝国主义制度,美国黑人才能够取得彻底解放。美国广大黑人同美国白人中的广大劳动人民,有着共同的利益和共同的斗争目标。因此,美国黑人的斗争正在获得越来越多的美国白色人种中的劳动人民和进步人士的同情和支持。美国黑人斗争必将同美国工人运动相结合,最终结束美国垄断资产阶级的罪恶统治。

我在一九六三年《支持美国黑人反对美帝国主义种族歧视的正义斗争的声明》中说过:“万恶的殖民主义、帝国主义制度是随着奴役和贩卖黑人而兴旺起来的,它也必将随着黑色人种的彻底解放而告终。”我现在仍然坚持这个观点。

当前,世界革命进入了一个伟大的新时代。美国黑人争取解放的斗争,是全世界人民反对美帝国主义的总斗争的一个组成部分,是当代世界革命的一个组成部分。我呼吁:世界各国的工人、农民,革命知识分子和一切愿意反对美帝国主义的人们,行动起来,给予美国黑人的斗争以强大的声援!全世界人民更紧密地团结起来,向着我们的共同敌人美帝国主义其帮凶们发动持久的猛烈的进攻!可以肯定,殖民主义、帝国主义和一切剥削制度的彻底崩溃,世界上一切被压迫人民、被压迫民族的彻底翻身,已经为期不远了。
\section[对派性要进行阶级分析——几段最新指示]{对派性要进行阶级分析\\{\large——几段最新指示}}\datesubtitle{(一九六八年四、五月)}

对派性要进行\footnote[1]{原文为“进在”,据《人民报纸》更正为“进行”}阶级分析。

{\raggedleft《人民日报》、《解放军报》一九六八年四月二十日社论:\\《无产阶级革命派的胜利》\par}

党外有党,党内有派,历来如此。

除了沙漠,凡有人群的地方,都有左、中、右,一万年以后还是这样。

{\raggedleft《红旗》杂志一九六八年四月二十六日评论员文章:\\《对派性要进行阶级分析》\par}

派别是阶级的一翼。

{\raggedleft《人民日报》、《红旗》杂志、《解放军报》五一社论:\\《乘胜前进》\par}

\end{document}