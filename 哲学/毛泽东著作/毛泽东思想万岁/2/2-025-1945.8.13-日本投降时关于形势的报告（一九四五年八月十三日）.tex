\section[日本投降时关于形势的报告(一九四五年八月十三日)]{日本投降时关于形势的报告}
\datesubtitle{(一九四五年八月十三日)}


(一)

现在时局是同志们关心的问题。时局怎样?我党方针怎样?时局发展很快。日本投降了,投降如何。日本未正式下命令投降,现在还是谈判中。日本要求投降,同盟国允许它投降,日本天皇如何处置,同志们都知道。现在消息,日本有回信给同盟国,内容不公开,不知道,大概是在讲价钱,它还有些资本,与德国的被打到柏林城市不同,美国还未打到日本本土。它在中国及南洋还保持着军队,目前最危急的问题还是满洲,它是在红军面前。投降也许还有几天,也许在明天。日本请求投降,盟国方面准许投降,天皇如何处置,日本是否马上下命令投降,还在谈判。日本投降大势已定,决定因素是苏联的力量,日本四面被包围。

抗战打了八年,日本投降大势已定,还在讲价还价,决定力量是苏联从后面打了进来,这个力量是不可抵抗。这种情况,抗战历史阶段是否已过去?当做抗日历史阶段已完结,日本投降大势已定,还有点尾巴叫做过渡,河里到冬天,搭上小桥,现在桥上叫做过渡,整个阶段是过去了。现在因为它还未缴枪,同盟国占领日本、我们收复失地还有一个时间。欧洲也是这样。当德国宣布投降、接受投降签字那天,斯大林说:“欧洲战争过去了,和平发展时期到来了”。

刚才说,日本投降大势已定,在它接受同盟国条件,向它的军队下令投降,双方代表订条约签字,如德国签字后,是和平发展时期,虽然如此,目前一来未签字,二来签字后还要缴枪,还要占领,目前还在过渡状况。今天讲的,就是这样情况。

(二)

中国国内各阶级各党派的关系现在怎样?将来可能怎样?代表两大阶级,资产阶级与无产阶级,统治者与人民,都有一个党,就是国民党与共产党,此外,还有第三者,中间势力党派。

国民党怎样?国民党怎样同志们知道,看它的过去就可知现在,看它的过去、现在,就可以知道将来。这个党,过去与我们打过十年内战,27—37,抗战中它袖手旁观,等候胜利。它这政策做得相当好,它硬不肯打,胜利还是到来了,它讲日苏必战,日美必战,被它讲灵了。中国地主资产阶级代表蒋介石,此人大家都知道,是个狡猾的家伙,他的政策就是袖手旁观,等候胜利。现在算达到他的目的,又借美国一点枪,武装一下,现在委员长要下山了。八年来,我们与他调了一个位置,以前我们在山上,他在水边,抗战期中,我们在敌后,他在山上,现在他要下山了。现在的政治形势就是这样:过去是袖手旁观,等候胜利,现在委员长要下山去,夺取抗战果实。

我们抗战八年,在敌后抗战,不是每枪都是我们打的,他也打了几枪,但主要是我们打的。我们英勇奋斗,艰苦抗战,他袖手旁观,如果没有我们打,有无委员长?他躲在山上,有人给他守卫,我们守卫、站岗、打敌人,才有二万万人的地方未被敌人占领,这个原因在那里?“委员长”消极抗战有几枪抬着,这也是个原因,但是解放区英勇抵抗是主要的。保卫二万万人民也是保卫委员长。这也就给他袖手旁观,赢得胜利的时间。没有我们,他也旁观不成,旁观是我打他吹,这样他就有时间。八年○一日,有了地方,二万万人占的地方。这样条件是我们给的,那末委员长是否感谢我们?不!此人历来是不知感恩的。他是怎样上台的呢?他很大权力是谁给的?是北伐战争,大革命第一次国共合作,人民拥护他,他上了台,是人民给他的,他上了台不感谢人员,反把人民一个巴掌打下去,一脚踢开。这个历史同志们都知道,这次中国人民又保卫了他,他有几支枪,拿枪的是什么人,一不是土豪劣绅,二不是资本家,也是中国人民。国民党一百多万军队,战士归谁?不是土豪,是中国工农。解放区一万万人民,百万军队,几百万民兵,更不用说。现在胜利了,日本投降,他绝不感谢中国人民。相反,他翻翻一力二七年的历史,还想照样干,他现在未下讨伐令,他们过去十年不叫内战叫剿共。不管叫什么,总之是杀人,是不是杀人?我们七大讲过,他要杀人,内战危机很严重。虽全国规模内战还未来,全国人民与党内有许多人在此问题的认识上,不是如我们刚才讲的,认为他一定要杀人,认为那件事还未来,内战还不普遍,不公开不大量,就有许多人以为不一定,其中还有许多人与我党意见不一致,还有许多人怕打内战。过去打了十年、八年,怕打,是很自然的,因此我们反对内战,不赞成内战,要制止内战。我们向全国宣传,口号就是“反对内战,制止内战”。但内战是要来的。因为蒋介石的方针就是要下山夺取胜利果实,反对人民,按他的方针是要打内战,按我们的方针是不要打内战,不要打内战是中国共产党和中国人民,可惜不是蒋介石。不要打是一个方面,一个不要打,一个却要打,如果两方都不打,才打不起来。现在是有一方面不要打,我们这方面的力量还不足以制止,内战就要打起来,内战危险很大。我党已几次不失时机指出:七大前,七大中,七大后,这次日本投降前,我们已相当充分的在人民中指出这点,使中国人民、我党、我军有充分思想准备。这点,很重要,有这点与没有这点有很大不同。一九二七年我党在幼年时期,对蒋介石的突然事变毫无精神准备,以致人民取得的胜利果实,跟着就失掉,而且使人民长期受苦,光明的中国变成黑暗的中国。这次党是无产阶级的先锋队,政治觉悟与政治成熟了,提高了,向全国人民和全党提出了警告,使大家处在有准备的状况,现在准备缴日本人的枪,最后消灭日本是有准备的。委员长要夺抗战果实,果子熟了,他要吃,他要垄断,一个人吃。他要打内战,我们讲清楚了,有了准备,内战如果来了怎么办?内战是他强迫中国人民来接受的,打日本也是一样,是日本帝国主义强迫中国人民接受的,它不打到中国来,中国是和平的。如果他这样强迫我们接受战争,这只手拿着大刀,那只手也拿着大刀,我们被迫也得两手拿着大刀,这是经过调查研究,这个调查研究大家都不注意,当他手里拿东西,就要调查,是刀,刀有两用,一是切菜,二是杀人,刀自己不会伤人,要用手拿。再调查一下中国人民也有手,也有刀,没有刀也可以打一把。中国人民经过长期调查研究,发现这个真理:“土豪劣绅有刀,拿刀要杀人,法西斯有刀,拿刀要杀人。”人民知道了,就可以照样办理。我们有些人就不调查,陈独秀就不知道,拿刀能杀人,这是常识。有人说,中国共产党领导人还能不知?难说。他没有调查研究,故叫机会主义。没有调查研究就没有发言权,一九二七年取消他的发言权。发言权拿在我们手上,谁要拿刀杀人,我们就照样办理,要把刀拿在一切有觉悟的人手里,人家磨刀,我们也磨刀,关中人家调十几个师,附近来了六个师,三个师打进我们边区,宽九百,深二十里,这是七、八月的事。我们也照他的办,把他宽百里深二十里消灭,他在那里修碉堡,有几个连占着,我们也能修碉堡。一一他与外国人一样,亏大本,算小帐,从黑龙江通到重庆,不算,百里,也要算,留五个连修堡垒,,我们把它消灭了,按我们的意旨,坚决、干净、全部消灭。现在他心里呕了气,又在调队伍,是否还打,不得而知。我们的方针是:“针锋相对,寸土必争”。中国人民得到的权利,决不许轻易丧失,必须经过战斗。如果我们打不赢,不怪天,不怪地,只怪我们没打赢,被他夺了土地夺了人。新四军皖南事件,虽然怪蒋介石王八蛋,但为什么有枪有手被人抢去呢?针锋相对,寸土必争,就是不能轻易丧失得到的土地。

去年有个新闻记者问:“你们办事,谁给的权利?”我说是老百姓给的,不是老百姓是谁?蒋介石、外国人、上帝均没有给,老百姓要解放,就把权利委托给能代表他们利益,能忠实的代表他们说话的就是中共,就是你们大家,我们当代表,就要代表得好。陈独秀没代表好,他是“针锋不对,寸士不争”,是差个把字,几个月把大革命的权力丧失干净。这次我们与陈不同,必须针锋相对,寸土必争。寸士要争不是内战时的“不放弃苏区一寸土”。这次我放弃宽百里,深二十里,但七月放弃,八月收回。国民党联络参谋问我们队伍动向,我说你天天在延安还不清楚,“你攻我也攻,你停我也停”,现在是“蒋反我也反,蒋停我也停”,我们是照你的办法的。经过调查研究,仿照办理,先生就是你。

这样,目前我们可以看到一般的说,有个很短的阶段,是对付日本人,因为日本未投降,地方未占,枪未缴到,到了占的地方,枪缴了,日本军俘虏了。这个阶段就是。现在还是过渡阶段(接第一阶段)。这个阶段,我们主要的注意力还是对付敌伪,对付得了与对付不了,靠我们斗争方法。胜利果实归谁,例如一棵果树,树上结了果子,桃子是胜利果实,蒋介石伸出这样长的手要摘桃子,我们也伸出这样长的手摘果实。该谁摘?那要看水是谁挑的、浇的,蒋一挑子也不担,我们解放区天天浇水,现在不浇水的人也要摘桃子,说:“此桃子的所有权是他”。我说他是吹牛皮。我们挑了水该我们摘,他现在下了命令不准摘,说他是地主,我是佃户。我们今天报上驳了他,明天还驳他,“你的权力很小,你没有浇水,不能摘,权力应属我的。”但果子究竟落谁手,又是一回事。同志们不要以为靠得住落在我们手。落得住也有,河北、察、热、晋大部,山东、苏北这些大块乡村,乡村打成一片,七、八十个城市,五、六十个城市一块,大、小、三、四、五、六块,中等城市,小城市,一、二十,三、四十,五、六十,这些靠得住的,我们的力量能得到这些果实,得这批桃子可能性大,得这些果实,在我们历史上是头一次。历史上是一九三一年下半年,一九三八年我们苏区刚起来,有二十一个县城,还没有府城,二十一个县城堆在一起就取得那样大的胜利,最多时人民有二百万,就能奋斗那样久,粉碎那样大的“围剿”,后来打输了,不能怪蒋,要怪我们没打好。如果这次大小城五、六十,三、四、五、六块。我们有三、四、五、六块大于中央苏区的地方,事情确实就是确实。这条可以说是确实。达到这一条,历史未有的,得到一批中小桃子,第二批是得不到的大桃子,沪杭宁。我们还是能占乡村,大城市也可以摘一下,但大致可以看到这些桃子是他的,挑水是我,桃子归蒋,因为现在未实现耕者有其田。

第三批要抢,要用权力去夺,津浦、胶济、平绥中段,正太白晋、德石、郑州以东的陇海,这些地方中间的中小桃子是必争的,都是我们灌溉的,平绥西段,太原南的津浦,沪宁、杭沪角,大江川南,我们送给蒋,不送也要送。大江以北必须力争。现在才只这几天,究竟怎样,还难说。两三个月后可以讲,现在只能讲二字“力争”。

还有问题,蒋有美援,我们有什么人援助?今天为止,还没有,要看将来。现在红军离我们还很远,柏林开会,什么意见,还未公开,发表的是那一套?什么外国来帮我,还未写上,放在什么基点上,应放在自己力量的基础上,叫自力更生,争取外援,内战就是争果实的斗争。这个时期有机会主义,就是孤立者,自愿把果实送蒋,虽人民不自觉。把果实送给人是自然的,历史有些,俄国二月革命就是,因工人大都不够觉悟,相信孟什维克与社会革命党,自愿把果实送给资产阶级。十月革命,工人不要孟什维克,选了布尔什维克,就胜利了。今天,中国解放区人民,选了中国布尔什维克,沦陷区人民欢迎谁,还未决定,要靠我们广泛宣传组织,人民觉悟不是容易,要去掉其落后的东西,去掉黑暗东西。机会主义,要经过政治的扫帚,你们都知道,用扫帚扫房子,扫了三个角,一个未扫,灰尘是不会自动从中国人民脑筋中跑出,落后的东西要我们去扫除。

中国人民有一部分还不觉悟。信蒋不信我,是由于我们的宣传组织工作不够,马克思主义是现实主义。从未说有不扫的灰尘会去掉,会否起大风?不会。我党整风也是这样整到那地方,小资产阶级思想就跑了,那地方未整训,小资产阶级思想就未跑。这样的情形是合乎逻辑的。我们的扫帚就是中国共产党、八路军。人家也有扫帚,扫中共,扫人民,扫关中。胡宗南扫去百里宽,我现不扫回。

边区有个介子河,河南是洛川,河北是富县,河南是人家的,龌龊东西多,河南河北是两个世界。我们有些人过于相信政治影响,是迷信。可看延安政治影响可谓大点,有政治问题的,我们扫一扫他就反省,不扫他不反省。凡是落后黑暗的东西,不以扫帚扫不能去掉,苏联政治影响最大,列宁格勒的芬兰,爱沙尼亚就不受影响。红军到了,反革命才扫除,一九三六年,我们在保安,有个土围子,我们首都在保安,政治影响可谓大矣,南扫北扫不成,后在里边扫,他才说不成。世界上的东西都是这样,桌子不搬不走,红军不进东北,日本不投降,我军不击,他不投降。我们不去打,他不投降,非到了非缴枪不可他才缴枪。党内有这么一种情绪,相信政治影响。朱熹说:“黎明即起,洒扫庭除。”黎明者,天将亮未亮也,有人告诉我,给我任务,说是努力工作,努力奋身,只有这样想,这样作,才有益处。中国地面很广阔,要我们一寸土一寸土的扫。

内战会不会起来,有国际国内的因素,国内因素是我们觉悟的程度。所谓觉悟,是一切欺骗的东西欺骗不了我们,我们清醒的头脑,不受骗。我们应承认蒋是很厉害的,他的政治手段、花样,如何反共反人民他有经验,承认这点有好处的。切不可以为人家不成。因此,我们要有清醒的头脑,包括有正确的方针,不犯错误。

整个说来,抗战阶段过去了。民族战争结束,新的情况与任务是国内斗争,会不会因国内国外因素会推迟内战发生,逻辑可能性是有的,蒋要下去摘果子是一定的,夺果子就是斗争,有的我摘,有的他摘,许多地方我与他抢摘,在抢摘中要爆发战争。一个时期,可能有许多地方性的抢摘。

我们有力量:①解放区一万万人民;②国民党统治区觉悟人民反内战,可能牵制蒋力量;⑧国民党内部一部份人不赞成内战,这是国内因素。有个苏联在北边,解放区有一万万人民,国民党区有许多人民内战,中国有四十五个希腊大,丘吉尔在希腊搞,中国不同,延安有个观察组,组长是包瑞德,他与我谈:“你们要听赫尔利的话,派几个人到国民党去作官。”我与他说:“捆住手的官不好做,要做就放手放足,大摇大摆的作,这就是联合政府。”他说:“不做不好,一美国人骂你,二美国给蒋的钱。”我说:“你美国人吃面包,愿撑蒋的腰,我不干涉。不过有一条,中国是什么人的中国,绝不是蒋介石的。你可撑一百年,一百○一年。我下命令给你们,不准撑蒋。一定要撑,我不撑不成。现在我是小米加步枪,你们是面包加大炮,你们要骂就骂。”这是吓人的,帝国主义有一套,殖民地有些人就怕吓,他不知有人不怕吓。我们过去批评这点是对有许多中国人士不了解说:“你们在老虎上釆。”我们就是无法无天,你们没与我们讲什么条约,蒋也不承认我们。我们参加国民参政会以文化团体资格参加。我们说我们是武化团体,不是文化团体。今年三月一号,蒋说中共交出军队,才有合法地位。我未交,还没有合法地位。世界上六十多国,没一国承认我们,我们就无法无天,我们责任是向人民负责,每个政策,每句话对人民负责,犯错误一定改正,这就是向人民负责。美国不承认我们,帮蒋打我们,对美对蒋,我们不负任何责任。

苏联来了,这是三个皇帝,盘古开天以来未有的事,哪本书也找不到,外国军队援助中国打法西斯,在中国境内,这个变化发生的影响不可估量。一个原子弹想把红军的政治影响扫掉是扫不掉的。我们有人相信原子弹。不对,原子弹是有,不能解决战争,不能使日本投降,有原子弹没有作斗争,原子弹是空的,如果原子弹能解决问题,为何去请苏联?投了一个原子弹,为什么日本不投降,苏联一参战,日本就投降。我们是马列主义者,讲问题,想问题,一定不能离开立场。我们有些人不如贵族,英国蒙巴顿勋爵,他说:“认为原子弹能解决战争是错误的。”我们有人比蒙巴顿不如,蒙还有点教条主义,他们都不以唯物史观看问题。这些影响从那里来,是从资产阶级学校,资产阶级报纸宣传他就相信,学马列主义差几里路。要洗清脑子的资产阶级影响。有二种世界观,方法论(资产阶级世界观,马列主义世界观),有些人常把马列主义的世界观抛到后边,如官僚主义……可举二十条,冲口而出,原子弹问题就是,美不一定直接援助蒋打内战,蒋统区人民拖住蒋。这次内战与一九二七年有所不同?一定有所不同,那时无这样的国内国际条件,那时党是幼年的党,无清醒的头脑,无武装斗争经验,无针锋相对,寸土必争的方针。现在不同了,现在蒋介石的影响还有没有?还有一些,相当一部分人还信他。

切勿以为一切人与我们一样,延安讨论“论联合政府”,蒋介石有无革命性?那小册子已写清楚,就未写“没有革命性”五个字,他的政令是法西斯政令,军令是失败主义军令,还有什么革命性?要说清楚,是否要中央宣布谈话,提出打倒蒋介石,现在还不要,现在还不提,现在还在防御。“蒋反我反,蒋停我停”。他下台不许我缴枪,我再缓和一下,不成,必须答复他,我们下了命令,这就叫做针锋相对,蒋反我反。

现在国际国内形势可能把内战制于局部范围,但会向公开破裂。我们要有精神准备,他讨伐剿共,不管什么名义,他要杀人不管什么时候破裂,明天破裂也要准备。另方面,苏联打起来,我力量与国际因素,暂时我与蒋有局部内战。第一条我准备,第二条早已如此,现在已经有几个地方打起来了。

总而言之,现在到了新的阶段,新的阶段是抗战线束了,国内的民族民主斗争开始了,今天是过渡阶段,我们要缴枪,抢地方,稳定不稳定,国内斗争就加紧。从前是抗日统一战线,一九二七年是民族统一战线,我们总是一个民族,蒋在抗战后要“建国”,建什么国?今后就是什么国的斗争。是建立无产阶级领导的人为大众反帝反封建的新民主主义国家,还是建立土豪劣绅,大地主大资产阶级领导的半殖民地半封建的法西斯国家。这个斗争很复杂,与一九二七年不同。我们要准备,可能恰当的应付这个局面。

