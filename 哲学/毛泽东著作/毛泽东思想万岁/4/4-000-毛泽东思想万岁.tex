\documentclass[b5paper,oneside,12pt]{ctexbook}
\usepackage[hmargin=0.3in,vmargin=0.5in]{geometry} 
\usepackage[]{multirow}
\usepackage[perpage,hang]{footmisc} %脚注

\pagestyle{plain} %整书页眉页脚设置
\setlength{\marginparsep}{2pt}
\setlength{\marginparwidth}{20pt}

\ctexset{chapter/numbering=false}
\ctexset{
    section={numbering=false, afterskip = 0ex},
    subsection={format=\large\heiti\centering,numbering=false,beforeskip=1ex,afterskip = 1.75ex}
}
\newcommand\datesubtitle[1]{{\centering\large #1\par\vspace{1ex}}}  %自定义日期副标题格式,为了保险,最好使用两层大括号

% 靠右对齐,右边距2字
\newcommand{\kaoyouerziju}[1]{{\raggedleft #1 \hspace{2em} \par}}
% 楷体,右边距5字
\newcommand{\kaitiqianming}[1]{{\raggedleft\large\kaishu\ziju{1} #1 \hspace{5em} \par}}

% 一人带职位
\newcommand{\yirendaizhiwei}[2]{
    {\setlength{\tabcolsep}{0em}
    {\raggedleft\begin{tabular} {cc}%
        #1 & \quad{} #2 \hspace{4em} \\ 
        \end{tabular} \\[1pt]}}
}

% 右下定宽
\newcommand{\youxiadingkuan}[1]{
    \begin{list}{}{
        \setlength{\topsep}{0pt}        % 列表与正文的垂直距离
        \setlength{\partopsep}{0pt}     % 
        \setlength{\parsep}{\parskip}   % 一个 item 内有多段,段落间距
        \setlength{\itemsep}{0pt}       % 两个 item 之间,减去 \parsep 的距离
        \setlength{\itemindent}{0pt}%
        \setlength\parindent{0pt}
        \setlength\listparindent{0pt}
        \setlength{\leftmargin}{0.4\linewidth}
        \setlength{\rightmargin}{2em}
    }
    \item[] #1
    \end{list}
}

\usepackage{etoolbox}

% 下面是修改了脚注样式
% 一些LATEX内部命令含有@字符,如\@addtoreset,如需使用这些内部命令,就需要借助于另两个命令\makeatletter和\makeatother.
\makeatletter
% 补丁,改脚注文本前面的序号的字体,去掉其上标样式 
\patchcmd{\@makefntext}
    {\@makefnmark}
    {\hbox{\normalfont\@thefnmark}}
    {}{}

% 给脚注编号前后添加 〔〕
\renewcommand\thefootnote{{〔\arabic{footnote}〕}} 

%% 开启 footmisc 的 hang 选项
\setlength{\footnotemargin}{1.25em}     % 整个脚注文本的左边距,加此边距,来显示脚注序号。
\setlength{\skip\footins}{1\baselineskip} % 脚注线 和 脚注内容 间距 
% \setlength{\footnotesep}{\skip\footins} % 两个脚注文本之间的间距
\renewcommand{\hangfootparskip}{0pt}
\renewcommand{\hangfootparindent}{2em}

% \patchcmd{\@makefntext}
% {\ifFN@hangfoot\bgroup}
% {\ifFN@hangfoot\bgroup\def\@makefnmark{\normalfont\@thefnmark}}
% {}{}

\makeatother

\usepackage{calc} % 可以在命令中计算长度
\usepackage[]{hyperref} % 放在 footmisc 后面

% 引用样式:使用 latex 原始的 list 环境
\newenvironment{yinyong}{%
    \begin{list}{}{\parsep\parskip
        \setlength\topsep{0pt}
        \setlength\itemindent{2em}%
        \setlength\parindent{2em}
        \setlength\listparindent{2em}
        \setlength{\leftmargin}{2em}
        \setlength{\rightmargin}{2em}
        \kaishu
    }
    \item[]
}{
  \end{list}
}

\title{毛泽东思想万岁\\1958.1—1960}
\author{毛泽东}
\date{}

\begin{document}

\frontmatter
\maketitle
\tableofcontents

\mainmatter

\input{4-001-1958.1.3-在杭州会议上的讲话(一)(一九五八年一月三日).tex}
\input{4-002-1958.1.4-在杭州会议上的讲话(二)(一九五八年一月四日).tex}
\input{4-003-1958.1.11-在南宁会议上的讲话(一)(一九五八年一月十一日).tex}
\input{4-004-1958.1.12-在南宁会议上的讲话(二)(一九五八年一月十二日).tex}
\input{4-005-1958.1.12-关于报纸工作给刘建勋、韦国清同志的一封信(一九五八年一月十二日).tex}
\input{4-006-1958.1.19-给《文艺报》编委会的一封信关于1958年第二期《再批判》栏的按语(一九五八年一月十九日).tex}
\input{4-007-1958.1.28-在最高国务会议上的讲话(一九五八年一月二十八日).tex}
\input{4-008-1958.1-在最高国务会议上讲话要点(一九五八年一月二十八、三十日).tex}
\input{4-009-1958.1.31-在中央政治局会议上讨论教育工作时的指示(一九五八年一月三十一日).tex}
\input{4-010-1958.2.8-反浪费反保守是当前整风运动的中心任务(一九五八年二月八日).tex}
\input{4-011-1958.1.31-工作方法六十条(草案)(一九五八年一月三十一日).tex}
\input{4-012-1958.3.9-在成都会议上的讲话(一)(一九五八年三月九日).tex}
\input{4-013-1958.3.1-在成都会议上的讲话(二)(一九五八年三月十日).tex}
\input{4-014-1958.3.2-在成都会议上的讲话(三)(一九五八年三月二十日).tex}
\input{4-015-1958.3.22-在成都会议上的讲话(四)(一九五八年三月二十二日).tex}
\input{4-016-1959.3.25-在成都会议上的讲话(五)(一九五九年三月二十五日).tex}
\input{4-017-1958.3.26-在成都会议上的讲话(六)(一九五八年三月二十六日).tex}
\input{4-018-1958.3-在成都会议上的插话(一九五八年三月).tex}
\input{4-019-1958.3.19-对《中国农村社会主义高潮》一书《按语》的批示(一九五八年三月十九日).tex}
\input{4-020-1958.3.22-对《上海化工学院两个右派分子的大字报》的批语(一九五八年三月二十二日).tex}
\input{4-021-1958.3.28-和《江峡》轮船员的谈话(一九五八年三月二十八日).tex}
\input{4-022-1958.3-视察四川省一个养猪场时的谈话(一九五八年三月).tex}
\input{4-023-1958.4.6-在汉口会议上的讲话(一九五八年四月六日).tex}
\input{4-024-1958.4.6-在汉口会议上的讲话(二).tex}
\input{4-025-1958.4.1-在汉口会议上的插话(一九五八年四月一日至六日).tex}
\input{4-026-1958.4.29-视察抚顺煤矿时的指示(一九五八年四月二十九日).tex}
\input{4-027-1958.4-对当前工作的十七项指示(传达记录)(一九五八年四月).tex}
\input{4-028-1958.5.16-对宋××关于苏联专家问题报告的批示(一九五八年五月十六日).tex}
\input{4-029-1958.5.8-在八大二次会议上的讲话(摘要)(一)(一九五八年五月八日下午四时五十分).tex}
\input{4-030-1958.5.17-在八大二次会议上的讲话(二)(一九五八年五月十七日下午).tex}
\input{4-031-1958.5.2-在八大二次会议上的讲话(三)(一九五八年五月二十日下午).tex}
\input{4-032-1958.5.18-在八人二次会议代表团团长会议上的讲话(一九五八年五月十八日).tex}
\input{4-033-卑贱者最聪明,高贵者最愚蠢(一九五入年五月十八日).tex}
\input{4-034-1958.5.23-在八大二次会议上的讲话(四)(一九五八年五月二十三日下午).tex}
\input{4-035-1958.5-接见阿联军事代表团时的谈话(摘录)(一九五八年五月).tex}
\input{4-036-1958.6-关于原子弹、氢弹的指示(一九五八年六月).tex}
\input{4-037-1958.6.17-关于《第二个五年计划指标》的指示(一九五八年六月十七日).tex}
\input{4-038-1958.6.21-在军委扩大会议上的讲话(一九五八年六月二十一日).tex}
\input{4-039-1958.6-在军委扩大会议小组长座谈会上的插话(根据记录整理为九条)(一九五八年六月廿三日).tex}
\input{4-040-1958.6.28-在军委扩大会议小组长座谈会上的讲话(一九五八年六月二十八日).tex}
\input{4-041-1958.6-接见应举社社长时的谈话(摘录)(一九五八年六月).tex}
\input{4-042-1958.7.8-对新华社、《人民日报》的指示谈肃反斗争宣传等问题(一九五八年七月八日××传达).tex}
\input{4-043-1958.7.28-谴责殖民主义者侵略西亚(一九五八年七月二十八日).tex}
\input{4-044-1958.8.4-视察徐水时的谈话(摘录)(一九五八年八月四日).tex}
\input{4-045-1958.8.9-视察山东时的谈话(摘录)(一九五八年八月九日).tex}
\input{4-046-1958.8.13-视察天津时的谈话(汇集)(一九五八年八月十三日).tex}
\input{4-047-1958.8.15-接见西哈努克时的讲话(摘录)(一九五八年八月十五日).tex}
\input{4-048-1958.8-在《中央关于在农村建立人民公社问题的决议》中所加的一段话(一九五八年八月).tex}
\input{4-049-1958.8.17-在北戴河政治局扩大会议上的讲话(一)(一九五八年八月十七日).tex}
\input{4-050-1958.8.19-在北戴河政治局扩大会议上的讲话(二)(一九五八年八月十九日).tex}
\input{4-051-1958-在北戴河政治局扩大会议上的讲话(三)(一九五八年入月二十一日上午).tex}
\input{4-052-1958.8.21-在北戴河政治局扩大会议上的讲话(四)(一九五八年八月二十一日下午).tex}


\end{document}