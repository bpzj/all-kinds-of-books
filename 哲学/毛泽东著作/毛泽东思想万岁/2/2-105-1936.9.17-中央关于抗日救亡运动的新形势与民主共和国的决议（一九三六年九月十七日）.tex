\section[中央关于抗日救亡运动的新形势与民主共和国的决议(一九三六年九月十七日)]{中央关于抗日救亡运动的新形势与民主共和国的决议}
\datesubtitle{(一九三六年九月十七日)}


(一)日本帝国主义自去年策动华北独立以来,对于中国的侵略没有一刻停止的,不论它侵略的方法与方式怎样随具体情况的变化而变化,然而它并吞全中国的基本政策,是始终不变的。在华北进兵之后,高唱着的“经济提携”,实际上不过是以经济的侵掠方法巩固已得的阵地,并给新的进攻准备基础。平津与察北的增兵,绥东的进攻,成都、北海、海口、上海、丰台等地的挑衅,表示出日寇的新侵略行动又将开始。民族危机,较之过去是更加严重了。这就指出:保卫华北、保卫西北、保卫中国、收复东北失地、驱逐日本帝国主义出中国的民族革命战争的任务,更加严重的放到革命政党与全民族的身上了。

(二)一年来全中国抗日救亡运动的广大发展,抗日统一战线的开始发动,已经给了日寇侵略计划以相当的打击,暂时的阻止了日寇建立华北国与建立“防共统一战线”的实施,这是中国人民一年来抗日救亡运动的胜利。然而由于抗日救亡运动的发展还不够广泛,它的力量还不够雄伟,全国各党各派各界各军的有组织的抗日民族统一战线现时还仅在开始的阶段,最大的政党一一国民党及其领导与影响下的军队还没有尽参加这个战线,国民党的政策还没有基本的转变,因之,民族革命战争还没有能够发动,以致我们不但没有两个停止日寇的继续侵略与保持中国领土主权的再不受到损害,而且日寇抂巩固了它在华北的地位,获得了对于中国领土主权的新的侵占,病尊卑行的大举进攻。这证明:日本帝国主义是能够战胜的,但是需要全中国各党各派各界各军的共同行动与艰苦卓绝的奋斗。

(三)在日本帝国主义不断进攻之下,中国人民的抗日救亡运动现在已经进入了一个新的阶段。这主意的表现在全国工农小资产阶级广大群众抗日救亡运动的继续增长,中国共产党苏维埃红军抗日救国主张的得到全国广大人民的赞助与全国主力红军的集中于西北抗日前进阵地,一部分民族资产阶级的开始转向抗日战线,国民党军队官兵中广大成分抗日情绪的增长,和在这些基础上产生的国民党及其南京政府的分化与动摇。国民党南京政府内外政策的摇摆不定,其言行的自相矛盾;与其各派间关于抗日问题上的争论,冒险的表示出它现在是在动摇的中间。在日寇继续进攻,抗日救亡运动继续发展,国际形势新的变动等条件下,国民党南京政府有缩小、以至结束其动摇地位,而转向参加抗日运动的可能。

(四)为着集中全国力量主抵抗日寇的侵略,驱逐日寇出中国,我们不仅要收集更广泛的民众的力量、和一切真正革命的觉悟的纯洁的分子,而且要争取阶级阶层一切可能的部分到抗日斗争中来,使抗日民族统一战线更加扩大起来,更加增强自己的阵容与力量。推动国民党南京政府及其军队参加抗日战争,是实行全国性大规模的严重的抗日武装斗争之必要条件。但这绝对不应放松对于国民党南京政府一切违反民族利益出错误政策的严厉的批评与斗争。只有这样,才能使国民党南京政府内部的抗日倾向日渐发展,扩大抗日分子的影响,克服其本身的动摇,战胜亲日派。而走上抗日救国的大道。中央必须着重指出:共产党在为实现最广泛的抗日民族统一战线的斗争中不但对于统一战线之公开的或秘密的敌人,应该进行严厉的斗争,而且对于口头上赞成而实际上消极的假抗日分子,以及各种各样的同盟者,应该保持批评的完全自由。同时,中国共产党党费助一切真正的抗日战争之发动,即使这种发动是部分的。但是主要的,应用尽一切方法与力量,最迅速的促进大规模的全国性的真正对日武装抗战。为此目的,共产党应继续坚持“停止内战一致抗日”的口号,反对一切在民族危亡面前自相残杀的内战。

(五)中央认为在目前形势之下,有提出建立民主共和国口号的必要,因为这是团结一切抗日力量采保障中国领土完整和预防中国人民遭受亡国灭种的惨祸的最好方法,而且这也是从广大人民的民主要求产生出来的最适当的统一战线的口号,是较之一部另领土上的苏维埃制度在地域上更普及的民主,较之全中国主要地区上国民党的一党专政大大进步的政治制度,因此便更能保障抗日战争的普遍发动与彻底胜利。同时,民主共和国不但能够使全中国最广大的人民群众参加到政治生活中来,提高他们的觉悟程度与组织力量,而且也给中国无产阶级及其首领共产党为着将来的社会主义的胜利而斗争以自由活动的舞台。因此,中国共产党宣布:积极赞助民主共和国运动。并且宣布:民主共和国在全中国建立,依据普选权的国会实行召集之时,苏维埃区域即将成为它的一个组成部分,苏区人民将选派代表参加国会,并将在苏区内完成同样的民主制度。

(六)中央着重指出:只有继续开展全中国人民的抗日救亡起动,扩大各党各派各军的抗日民族统一战线,加强中国共产党在民族统一战线中的政治领导作用,极大的巩固苏维埃与红军,同一切丧权辱国及削弱民族统一战线力量的言论行动进行坚决的斗争,我们才能推动国民党南京政府走向抗日,才能给民主共和国的实现准备前提。没有艰苦的持久的斗争,没有全中国人民的发动与革命的高涨,民主共和国的实现是不可能的。中国共产党在为民主共和国而斗争的过程中,应该使这个民主共和国从实行本党所提出的抗日救国十大纲领开始,一直到中国资产阶级民主革命的基本任务彻底的完成。

(七)去年十二月中央政治决议上确定的抗日民族统一战线的政治路线,基本上是正确的。大半年来,党在执行这一总路线下,得到了许多成绩,主要的在于苏维埃红军的力量是加强了,党在全国的政治影响是扩大了,向各党各派各军进行统一战线的工作是进步了,这些成绩应当成为今后工作的有利基础,但是许多党的组织极不善于具体的运用统一战线的策略,对于每一个人、每一个派别、每一个社会团体、每一个武装队伍、每一阶级与阶层,常常不善于根据它们不同的情况,不同的需要与要求,在抗日救国的总方针下,同它们接洽,协商、谈判,以求订立各种地方的、局部的、暂时的或长久的、成文的或口头的具体实际的行动纲领,并在为着实现这些纲领的共同斗争中,引导推动与组织他们走向抗日民族统一战线的最高形式一一全国抗日救国代表会议(或国防会议),国防政府,抗日联军,以至民主共和国。我们许多党的组织不善于这样做,而仅仅满足于一般的抗日民族统一战线的号召,满足于少数先进分子的活动,以抽象的刻板的与机械的方法方式,去对付各个具体的问题。同时在统一战线一时不能建立的场合,却又表现出:“共产党员以共产主义精神来教育群众、动员群众与组织群众的独立工作”之放松或不足。这些弱点,是使抗日民族统一战线直到今天还没有成为千千万万抗日大众实际参加的主观上的重要原因。

(八)中央认为必须及时纠正那种以阶级斗争的发动会妨碍民族统一战线的观点。民族革命的胜利,决不是少数上层分子所能完成的。不吸收成千百万工人农民与小资产阶级群众参加到抗日民族统一战线中来,就不能形成抗日救国的雄厚力量,就不能推动与逼迫动摇的游移的上层分子与当权者走向真正的抗日的道路,就不能实现民主共和国。而领导工人农民小资产阶级群众的日常经济政治的斗争,解决他们迫切的生活要求,是组织他们进入抗日民族统一战线的主要关键。但在领导这种日常经济政治的斗争中,共产党要同样的善于运用统一战线的策略,吸收最广大的群众到斗争中来。并根据于他们的觉悟程度与组织力量,把群众的日常斗争提到更高的阶段,使这种斗争成为推动抗日民族统一战线继续扩大,继续发展与继续前进的力量,成为抗日民族统一战线的坚实的基础。

(九)在建立抗日民族统一战线与实现民主共和国的斗争过程中,绝对不应该削弱苏维埃红军的力量。抗日民族统一战线的国防政府与抗日联军,是苏维埃红军在一定纲领上同其他政权及武装力量所成立的政治军事协定,但并不与其他政权及武装力量相混合。可以在国防政府与抗日联军的统一指挥之下,但并不取消苏维埃红军组织上与领导上的独立性。须充分注意号红军的扩大与巩固,那种不经过选择的允许学生及其他军队的旧军官加入红军的意见是错误的,因为这样足以破坏红军的统一和团结。必须充分注意加强苏维埃红军的领导成份,那种允许资产阶级参参加苏区政治管理的意见性是错误的,因为他们可以从内部来破坏苏维埃机关。即在民主共和国建立之后,共产党也决不放弃对于苏区人民与原有武装力量的绝对的领导,相反的,党在坚决领导全中国人民群众的抗日斗争与日常经济政治斗争中,要坚持着扩大与巩固自己的政治的与军事的力量,保障抗日战争与民主共和国之彻底胜利,争取社会主义前途的实现。

(十)扩大与巩固共产党,保障共产党政治上组织上的完全独立性和内部团结一致性,是使抗日的民族统一战线与民主共和国得到彻底的胜利的最基本的条件。因此在苏区内特别是在非苏区内,有系统的征收党员是非常必要的。但必须避免大批入党的办法,而只吸收经过考察的工人农民与革命知识分子入党。在这个意义上,去年中央十二月决议中“一切愿意为着共产党的主张而奋斗的人,不问他的社会出身如何,都可以加入共产党”与“党不怕某些投机分子侵入”的意见是不正确的。党的各级领导机关必须采取实际办法培养党的工作干部,因为不论在苏区与非苏区,在彻底实现党的政治路线上,有独立工作能力的优秀干部,有着决定一切的意义。动员成千成万的党员到一切无组织与有组织的群众中去去争取千百万的群众在自己的周围,正确的建立公开工作与秘密工作的联系,是中国共产党在为着完成抗日民族统一战线的伟大政治任务面前一刻不可放松的任务。在扩大抗日民族统一战线与为民主共和国而斗争的过程中,对于忽视党的政治上与组织上的独立性,忽视巩固苏维埃与红军,放松对于广大下层群众的日常经济政治斗争的领导,对同盟者批评不够等等右的机会主义倾向,必须及时的给以纠正。但在目前说来,“左”的关门主义倾向,依然是彻底实现抗日民族统一战线策略的主要危险。正确的党内思想斗争的发展,将是完成党的一切政治任务的有力武器。

<p align="right">(原载《毛泽东选集》上册,晋冀鲁豫中央局编第二四五页)</p>

