\section[在北戴河政治局扩大会议上的讲话(四)(一九五八年八月二十一日下午)]{在北戴河政治局扩大会议上的讲话(四)}
\datesubtitle{(一九五八年八月二十一日)}


五九年粮食方针问题。劲鼓的比今年大还是和今年差不多?劲还是愈鼓愈好,明年还是要大于今年,现在不要愁丰收有灾,不要怕多就不鼓劲。但要有节奏的生产,现在劳动强度很大,要使农民有适当的休息,一个月休息两天,半个月休息一天,忙的时候休息少些,闲的时候休息多些,离工地较远的可以在工地里集体吃饭、睡觉,这样可以节省来往时间,多得到休息。这个意思要写到文件里去,但不要讲的太多。

粮食多,油还不够。粮、棉、油都要增产,中心是深耕,今年是多数未深耕,密植也不够,太密了不通风也不好。深耕才能密植,蓄水、施肥、除虫(××插话:密植一千万株以上可能有失败的,五百万株没有什么问题,大面积密植要创造经验。)政治经济学和历史唯物论有些问题要重新写。我们解决了一个马克思主义的理论问题。先搞农业,同时搞重工业。赫鲁晓夫与莫洛托夫之争,就是说重工业多了,我们一反苏联之所为,先搞农业,促进工业发展,先搞绿叶,后搞红花。先搞绿叶后搞红花有什么不好?看来有些问题,需要重新解释。经济学和历史唯物论要有新的补充和发展。生产关系中的三个方面,所有制、劳动者的相互关系和分配问题,都未展开,苏联的集体农庄、手工业合作社还是集体所有制,为什么不搞全民所有制?全民所有制不只是中央的,而是全民的。过去所有制是表现为×××、赵尔陆所有,这就是苏联的办法。我们现在管二十八个省、市、自治区也管不了,无非是开开会,一年抓四次,从前管得更少,无非是发发指示,通报一些情况。现在百分之二十中央管,百分之八十地方管,省也要向下分权,直到企业也要有一定的权限和独立性。石景山钢铁厂投资包干,可以从六十万吨钢搞到一百三十万吨钢,第二期就可搞到三百万吨,这是什么原因?这里边有鬼,请大家好好想一下,是群众的积极性来了。×××管的时候,实际上是设计员在专政。这里有一个问题大家要想一想,我看跟民族独立有同样的道理,这联系到人民的问题,印度独立之后比英国统治时积极性高。一独立就有积极性。当然它是表现在阶级斗争上,街道、工厂、民办学校,由集体所有制变为全民所有制也发展了。

“死”活斗争问题。“死”活斗争一万年也有。控制“死”还是不控制“死”呢?没有死不行,统得太死也不行,一点不死也不行,五九年的钢如果是××吨,××必须卡死,××吨是活的,如果××吨,就有××吨是活的,如果超过××吨,还可以分成食油多的多吃,少的少吃,这就活了。死者保证重点,活者重点之外不妨碍重点。大包干就是有死有活,大家都要管。死与活两方面就是统一与分散,兼而有之。包干制就是有死有活的矛盾统一。大权独揽小权分散就是这个道理。中央究竟谁当家?大权独揽揽到何处?只有经济建设委员会是否够了?可否分设工业生产委员会和工业基建委员会,总要寃各有头,包工包干,使大家有奔头。我们说六项纪律,是搞神经战,主要是吓人,不坐班房,大家不犯法就是嘛!

历史唯物论关于上层建筑的问题,是政权问题,已经解决了。人民公社是政、社合一,那里将会逐渐没有政权,人民公社是几个人中加一个坏人,这就专了政,六亿人口中只有一百主十万劳改犯不算多。军队过去说自己落后,会一开,相互关系一改变,就出现了新气象,各地军队都在开会,军队大跃进已经起来了,可以搞各种名堂,军队拿出三分之一的时间搞政治、文化、劳动、影不影响军事训练?不但没有影响,反而搞得更好。公安、法院也正在整风。法律这个东西没有也不行,但我们有我们这一套,还是马青天那一套好,调查研究,就地解决。调解为主。大跃进以来,都搞生产,大鸣大放大字报,就没有时间犯法了。对付盗窃犯不靠群众不行。(刘××插话:到底是法治,还是人治?看法实际靠人。法律只能作办事的参考,南宁会议、成都会议、“八大”二次会议,北戴河会议的决定,大家去办就……。上海梅林公司搞双法,报上一登,全国开展。不能靠法律治多数人,多数人要靠养成习惯。军队靠军法治人,治不了,实际上是一千四百人的大会治了人,民法刑法那样多条谁记得了。宪法是我参加制定的,我也记不得;韩非子是讲法治的,后来儒家是讲人治的,我们每个决议案都是法,开会也是法,治安条例也靠成了习惯才能遵守,成为社会舆论,都自觉了,就可以到共产主义了。我们各种规章制度,大多数,百分之九十是司局搞的,我们基本不靠那些,主要靠决议,开会,一年搞四次,不靠民法刑法来维持秩序。人民代表大会,国务院开会有他们那一套,我们还是靠我们那一套。这是讲上层建筑部分。

意识形态、宇宙现、方法论、报纸、文化教育的作用大得很。资产阶级的自由破坏得越多,无产阶级的自由就越多。苏联对资产阶级的自由没有彻底破坏,因而没有充分建立超无产阶级的自由。我们政治思想上的革命搞得比较彻底,干部参加生产,和群众打成一片,彻底改革规章制度,就是对资产阶级自由的彻底破坏,工人的干劲冲天。政治经济学谈到这些问题几句话就过去了。

分配问题。苏联干部职工工资等级太多,和工农收入相差太悬殊,农民义务交售制,负担百分之四十,限制农业四十年不发展,我们只拿百分之五至百分之八(间接负担除外),我们藏富于民,“粮食足,军食孰能不足”。赫鲁晓夫来了,就是只说国家搞多少粮食。不讲生产多少,我们就是讲生产的。人们知道我们反正是为了他们,积极性高。有人说“大国人多难办事”。看什么办法,只要办法对头,再有十亿人也好办。我们的方法,反正是大鸣大放,自己管理自己。我们是服从真理的,真理在下级的,上级就服从,兵高明军官就服从兵,学生编教材,比教员先生编得好,先生就应该服从学生。编教材要党、学生和教员中的积极分子“三结合”。一门一门的科学来清理,资产阶级霸占的情况必须攻破。科学院中药研究所所长赵承嘏,他会提炼一种治高血压的药(蛇根草),始终不向别人讲,青年科学人员不服气,苦战了几天,也就搞出来了。因此,要抓研究员青年人,使这些教授孤立起来。这种斗争很激烈。因此还要几年。

意识形态的重要性。意识形态是客现实际的反映,关心基础,为基础服务。改革规章制度。开会就是搞意识形态,北戴河会议就是搞意识形态。去年三中全会,今年南宁会议、成都会议、党代表大会,提出了破除迷信的口号,起了很大作用。因此才有大跃进。不正确地反映客观规律危害很大,八股文章、孔夫子的思想传了几千年,达赖喇嘛的屎和土都有人吃,蠢得不得了。张道陵的每人五斗米的教,出五斗米就有饭吃,传到江西的张天师就变坏了。吃粮食是有规律的,大口小口一年三石六斗,放开量叫他吃,薛仁贵一天吃一斗米,总是少数。我们搞公共食堂,也可以打回去吃。吃饭不要钱的办法,可以逐步实行,暂时不定,……职工宿舍要搞搭配,大片宿舍比公馆好。资产阶级法权不能完全废掉,大学教授比学生吃好一点。河南搞八十亿土方,粮食翻一番,河南能办到的,全国都应该办到。

明年建国十周年,宣传是大搞还是小搞?我们是为中国人民作宣传,对全人民是鼓劲,不考虑影响外国的问题,实际上外国会受影响。大搞请不请外国人,请多少?

同去告诉军队同志,军官要当一个月的兵,先从少数人搞起,一个人搞起来了,别人都要搞,一个十月革命,全世界都要革命;一个合作社搞千斤亩,全国都要搞千斤亩。到底是少数服从多数还是多数服从少数?历来都是多数服从少数,因为少数人反映了多数人的意见。你们来开会,还不是邓××发了一个通知,把你们都找来了,这不是多数服从少数吗?达尔文进化论,哥白尼太阳系的理论,都是一个人搞的,别人都服从。马克思、恩格斯是两个人,反映了客观规律,或者反映了多数人的意见。蛋白质的公式还未找到,活性染料一百六十七种,已经找到了公式了,世界第一,沼气是四碳一氢,屁是二氢一硫,石膏是硫化钙。就这样一点来说,那是少数人的意见反映多数人的意见。

河北省徐水县搞军事化、战斗化、纪律化,这三个口号提也可以,不提也可以,组织形式不一定搞团、营、连、排、班,设大队、中队、小队也可以。实际上是一个劳动组织与民主化问题。帝国主义为这件事造谣,但我们不怕它。强迫命令当然不好,但工作中有点强制也需要,这是纪律。大家宋北戴河开会,也是如此。苏联的军事共产主义是余粮征集制,我们有二十二年的军事传统,搞供给制,是军事共产主义。我们是在干部中搞共产主义,不包括老百姓,但老百姓也受影响,恩格斯说,许多东西都是从军队搞起来的,确实如此。我们从城市到农村。和半无产阶级结合,组织党和军队,我们吃大锅饭,没有礼拜,没有薪水,是共产主义性质的供给制。一到城里来,自惭形秽,过去一套吃不开了,要穿呢子衣服,刮胡子,干部知识分子化,薪水制否定了供给制,衣分三色,食分五等,群众路线在城乡也不充分了。解放后到五二年还好,五三年到五六年主要反映中国资产阶级思想,第二是照搬苏联。过去我们不得不请资产阶级当参谋,我们对资产阶级法权观点不自觉。几亿农民,七百万生产工人,二千多万干部和教员,资产阶级的海洋把我们淹到胸口,有的人被淹死了,刘绍棠成了右派,姚文元不错,比流沙河好。

人民公社当决议草案发下去,每一县搞一二个试点,不一定一下铺开,也不一定都搞团、营、连、排、班。要有领导有计划地去进行规划,现在不搞不行了,不搞要犯错误。自留地要增加,耕畜要私养为主,大社要变小社等几件事,是向富裕中农让步。经过这个过程是可以的,不算严重的原则性错误,在当时条件下,还有某些积极意义,现在又否定了。个别的猪,私人可以喂。社以大为好,人民公社的特点是一曰大二曰公,主要是许多社合为一个大公社。《农村社会主义高潮》一书中几个按语,都说办大社好,山区也可以搞大社,多种经营综合发展,开始办小一些也有好处。工资制度青年、妇女都高兴。增加自留地那一套道理都是农村工作部出来的。一九五五年我就提倡办大社。全国搞一万五千到两万五千个社,每社五千到六千户,二、三万人一社,相当大了,便于搞工、农、兵、学、商与农、林、牧、副、渔。这一套我看将来有些大城市要分散,二、三万的居民点,什么都会有,乡村就是小城市,哲学家、科学家多半要出在那里。每个大社都将公路修通,修一条宽一点的洋灰路或柏油路,不种树,可以落飞机,就是飞机场。将来每个省都搞一、二百架飞机,每个乡平均两架,大省自己搞飞机工厂。

各地不一定按徐水的办法去搞。三句话(军事化、战斗化、纪律化)各地都有。五星公社的简章要在《红旗》杂志上发表,大体可用,各地参照执行。


