\section[中国国民党选派学生赴莫斯科孙文大学(一九二五年十二月十三日)]{中国国民党选派学生赴莫斯科孙文大学}
\datesubtitle{(一九二五年十二月十三日)}


孙中山先生的革命精神,不独中国人景仰。全世界被压迫民众无不景仰,厌恶他的只有各国压迫人民的统治阶级,即帝国主义与军阀。苏俄当革命之际,各帝国主义国家勾结台尼金,兰格尔,柯尔恰克等白党反革命派四面侵袭,其危急情形殆仿佛两个月前之广东。此时孙先生曾有一勉励的电报致列宁。据俄代表鲍罗廷在十一月廿二日美洲同盟会欢宴席上所述,谓当此危急存亡之秋,首领列宁,接到孙中山先生一电,嘱其奋斗,列宁等十分感激。故当孙先生逆陈炯明叛变避走上海之际,其时势力颇为薄弱,但苏俄派遣代表越飞到沪,致意孙先生,期以合作,共同打倒帝国主义。是为中俄两国大同盟之起点云云。确实现在全世界被压迫民众的大敌只有一个,就是帝国主义。而要打倒帝国主义。必须全世界革命势力联合一致,方能和他决战而不失败,中国所需于苏俄者在此,苏俄所需于中国者亦在此。京沪等处一班高等知识分子哗然以联俄为导,乃不识现今世界上革命及反革命双方争斗的形势,不识国民党革命策略的意义之故,莫斯科孙文大学之设乃苏俄民众景仰孙先生革命精神,设此大学以容纳中国信仰孙先生主义的革命青年使为深切之研究,以养成国民革命之领导人才。据莫斯科来信,孙先生之三民主义建国方略两巨著已由彼邦学者翻为俄文,孙文大学现在积极筹备中,照目前予测,将来必有非常可喜之成绩。该校董事会主席为著名之越飞博士,曾任俄国驻华代表,董事会会员则为该校校长拉突,普拉夫打报主笔卜哈连,克鲁斯加耶夫人,职工联合会执行部主席汤斯基,及其他名流,中俄两国各设团及个人青著者甚多,经济方面,甚为充足。校长拉突曾对人言,该大学宗旨在培养社会领袖人才,课程主要科为经济历史,近代世界史,苏俄革命之经过及其意义,而中国之国民为革命运动则要列成专科,各科教授大概取研究方式,鼓励学生关于政治经济及各种社会问题独立之研究及创造的著作。各种成绩,并于著名报章杂志发表。至孙文大学学生人数,闻第一次定额五百名,在广东方面招百五十名。现在业由国民党政治委员会考取一百四十七名,从一千○三十人中选出,其方法分笔试口试两种,必两种成绩均优者方能及格。考取各生姓名列左:

(从略)

以上共百四十名,尚有七名姓名待查,至学生姓别籍贯年龄等项政治委员会已制有统计表,录之如下:

(一)姓别:男一百三十九人,女八人。

(二)省别:广东七十一人,湖南廿八人,江西十人,云南七人,四川七人,江西五人,湖北五人,浙江三人,贵州三人,福建二人,江苏一人,山东一人,山西一人,未详三人。

(四)婚嫁:已婚嫁者四十五人,未婚嫁者九十六人,未详六人。

(五)职业:学界二十五人,教育界七人,报界二人,农界一人,其它七人,未详十人。

各在现在准备放洋,分为数批出发,第一批日内即可登船,直放海参威。每人须带路费二百五十元,由国民党津贴一百元,自备一百五十元,连日各方面开会欢送者有新学生社,中国国民党广东省党部及其他团体,国民政府亦邀集学生聚会一次。在此聚会中,汪精卫曾致勋勉之词南并与学生约定下列三事:(一)关于广州政治情形,政府方面,必一周或二周报告留俄各位,希望各位亦必时将其近况报告回来。(二)孙文大学的中国学生须团结起来,始终誓为孙文主义奋斗。(三)关于赴俄各种预备,请各生举出干事数人,专向政府接洽,俾将所得消息随时向各人报告,在此叙会中,俄代表鲍罗廷有一长篇演说,说明孙文大学之目的,谓科学在帝国主义手里,用为压迫弱小民族的工具,若在弱小民族的手里,则可用为解放自己的工具,美国及英德诸帝国主义国家的大学,乃帝国主义的宣传机关。孙文大学的目的,则在使一般学生了解中山先生的主义以完成中国国民革命。兹录其演说全文于左。

(从略)

<p align="right">(抄自“政治周报”第二期14一一15页1925年12月13日广州政治周报社出版)</p>

