\section[在八大二次会议上的讲话(四)(一九五八年五月二十三日下午)]{在八大二次会议上的讲话(四)(一九五八年五月二十三日下午)}
\datesubtitle{(一九五八年五月二十三日)}


我们的大会是有成绩的,开得好,做了认真的工作,制定了我们的总路线。世界上的事情就怕认真,一认真,不管什么困难都可以打开局面。我国在世界上人口最多,国家大,人民群众得到了解放。资产阶级民主革命胜利了,社会主义革命取得基本胜利,建设有很大的发展,这样已经使我们可以看到我们的前途。以前还不清楚不知道什么时候可以摆脱被动状态、落后状态。以前我们在世界上没有地位。使人看不起,杜勒斯把我们看不在眼里。这和我们的情况不相称,其中也有道理,就是因为你虽然人口多,力量还没有表现出来,有一天赶上英国、美国,杜勒斯就得看上眼。确实有这个国家。我们的方针,这个客人暂不请。那时你找上门来。我们只好招待。过去几年,前年还看不清楚,还有人反总路线,多快好省的方针怀疑的人不少.这种情况也是不可避免的,是客现存在。这许多人能多快好省建设社会主义,那时怀疑的人、反对的人不少。有些人看到了,有些人看不到。看到。要经过曲折才能看到,经过一个时期,看到的人就多了。道路总是曲折的。以后还会有曲折。大会制定了多快好省、鼓足干劲、力争上游的总路线,还要在客观实践中证明。过去有些已经证明了。过去三年是马鞍形,两头高,中间低,前年高去年低,今年又高。有了这个变化,这个会就开好了。这次大家反映了人民的情绪、要求、干劲。多快好省建设社会主义,应该说是去年九月三中全会开始反映这一方面,前年十一月二中全会反映得不够。没有能够占上风。

一九五五年冬季。有两件事没有料到,就是国际上反斯大林,发生了波匈事件。世界上出现了反苏反共高潮,影响了全世界,影响了我们党。国内没想到来了个反冒进。没料到这件事。成都会议上就说过。请到会的同志注意,将来还可能发生曲折,请各省委研究。要预料到,前次在大会上讲了,有战争的可能、有分裂的可能,预料到就不要紧了。大家要研究一下,各省对可能有战争有分裂……还要研究。因为料到了就不怕。并非现在有战争,但是有可能,世界上有疯子。在莫斯科会议上就讲过,要防备疯子,宣言说打起来它就得完蛋。世界是我们的。会出乱子,但是不正确的力量总要被批判的,正确的力量总要胜利。但是要预料到,党内也要想一想,那么多省市委、自治区一半以上出了问题,但都没有推翻省市委,都克服下去了,如×××、××、×××、×××、××等等很不少。地委、县委、支部(多多少少)都有过一些问题。这是阶级斗争的正常现象。有些是属于好人犯错误,如对多快好省不了解,有些是坏人混进党来。×××是好人犯错误,……丁玲是暗藏在党内的坏人。早已叛党。

跟什么人走的问题,首先跟什么人?首先是跟人民学习,跟人民走,人民里面这么多干劲,多快好省。许多发明创造,一类社,千斤亩,两千斤亩。工业方面突破定额,发明创造。总之,工业、农业、商业、文教、军事各方面,思想理论各方面,有各种人材。代表人民的。大会讲了这么多经验,要我讲讲不出来,你们讲的比我好,是正确地反映了人民的要求、思想、感情。根据这些正确的反映。制成比较完备的体系,如这次大会决议和报告,过去没有这样。经过这八年,特别是第一个五年计划,一九五七年三中全会的鼓励就给了全党全国人民比较明确的方向,经过全党的努力。最近半年,去冬今春的大跃进,又经过杭州会议、南宁会议、成都会议,给这次大会做了准备。写了总结、决议,又搞了六十条,还没完成,还要改写。大体意思搞出来了,过几个月再改写一下。这就是先跟人民,然后人民跟我们。首先是理论来自实践,然后理论来指导实践,理论与实践统一是马克思主义的原则。这就是理论来自实践.然后又指导实践。开头没什么马克思主义,因为有了阶级斗争的实践,反映到人民脑子里,首先是反映到先觉者马克思、思格斯、列宁、斯大林的脑子里。客观规律反映到主观世界,有了理论性的总结,而他们发展为理论,给我们做模范。如果要政治上不犯错误,就要理论指导实践。但理论又必须从实践中得来的,离开革命实践不可能制造出理论系统来。关着房门不可能制造出理论来。大会的总路线制定不可能是某些人突然想出来的。不曾你地位多高。官多大。多么有名,如果不下去联系人民,或者向人民有联系的干部同志们接触。不与人民中的积极分子接触,只要半年你不与人民联系,什么也不知道,就贫乏了。所以规定每年四个月下去是很必要的。下去联系人民向与人民有联系的干部、人民中的积极分子接触。了解他们想些什么,做些什么,经过什么艰苦,然后总结上来。

“鼓是干劲,力争上游”的口号很好,反映了人民的干劲。“干劲”用“鼓足”二宇比较好,比“鼓起”好。真理有量的问题。因为干劲早鼓起了。问题是足不足。最少有六、七分,最好八、九分,十分才足。所以用“鼓足”二字比较好。干劲各有不同。

“鼓足干劲”这一句是新话,“力争上游”以前也有,不是新话。

“鼓足干劲,力争上游,多快好省”,外国人看来可能不懂,好像不通,没有主词,本来想加一句“调动一切积极因素”当主词,现在想不要也行。六亿人民就是主词。六亿人口中绝大多数人的干劲,除了章伯钧、罗隆基、章乃器、××等等这些人可能干劲不大。

插红旗,辨别风向,你不插人家插。任何一个大山小山,任何一亩田,看到那些地方没有旗帜就去插。看到白旗就拔。灰的也要拔,灰旗不行,要撤下来。黄旗也不行,黄色工会,等于白旗。任何大山上小山上,要经过辩论,插上红旗。

上次讲的是风向,不是方向,风向即东风还是西风。反冒进,一九五六年六月开始的。那时已有十大关系,多快好省。还有促进会。四月中在政治局扩大会议上.有省市委书记参加,那时没有明确的决议就是君子协定,大家赞成,不像这次大会有明确的决议、报告。一九五六年十一月二中全会,也没有明确决议。但有报道,重点是千方百计增产节约。那股风没有能够挡住,这是件坏事,转为好事,使我们有了比较,成都、南宁都谈过。这次大会,同志们有很多好的发言。……铁托是专门泄气,是那一方面的干劲,莫斯科宣言是我们这方面的干劲。南斯拉夫纲领是灭无产阶级志气,长敌人威风。

以后注意辨别风向。大风一来,十二级,屋倒,人倒,这样好辨别。小风不易辨别。宋玉写的《凤赋》,值得看。他说风有两种,一种是贵族之凤,一种是贫民之风(所谓“大王之风”与“庶民之风”)。风有小风、中风、大风。宋玉说:“风生于地,风起于青萍之末。侵谣溪谷。盛怒于土囊之口。”那时最不容易辨别。


