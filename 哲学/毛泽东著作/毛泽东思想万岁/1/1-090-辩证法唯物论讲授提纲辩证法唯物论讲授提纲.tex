\section[辩证法唯物论讲授提纲辩证法唯物论讲授提纲]{辩证法唯物论讲授提纲辩证法唯物论讲授提纲}


第一章唯心论与唯物论

本章讨论下列各问题:

一、哲学中的两军对战;

二、唯心论与唯物论的区别;

三、唯心论发生与发展的根源;

四、唯物论发生与发展的根源。

一、哲学中的两军对战

全部哲学史,都是唯心论和唯物论这两个互相对抗的哲学派别的斗争和发展的历史。一切的哲学思潮和派别都是这两个基本派别的变相。

各种哲学学说,都是隶属于一定社会阶级的人们所创造的,这些人们的意识,又是历史地被一定的社会生活所决定。所有的哲学学说,表现着一定社会阶级的需要,反映着社会生产力发展的水平和人类认识自然的历史阶段。哲学的命运,看哲学满足社会阶级的需要之程度如何而定。

唯心论和唯物论的社会根源,存在于阶级的矛盾的社会结构中,最初唯心论之发生是原始野蛮人类迷妄无知的产物。此后生产力发展,促使科学知识也随之发展,唯心论应该衰退,唯物论理应起而代之。然而从古至今,唯心论不但不曾衰退,反而发展起来,同唯物论竞长争高,互不相下,原因就在于社会有阶级的划分。一方面压迫阶级为着自己的利益,不得不发展与巩固其唯心论学说,一方面被压迫阶级同样为着自己的利益,不得不发展与巩固其唯物论学说。唯心论和唯物论学说都是作为阶级斗争的工具而存在,在阶级没有消灭以前,唯心论和唯物论的对战是不会消灭的。唯心论在自己的历史发展过程中,代表剥削阶级的意识形态,起着反动的作用。唯物论则是革命阶级的宇宙观,它在阶级社会内,从反动哲学的唯心论之不断的战争中生长与发展起来。由此.哲学中唯心论与唯物论的斗争,始终反映着反动阶级与革命阶级在利害上的斗争。哲学中的某一倾向,不管哲学者自身意识到与否,结局总是被他们所属阶级的政治方向所左右的。哲学上的任何倾向,总是直接间接助长着他们所属阶级的根本的政冶利害。在这意义下,哲学中的一定倾向的贯彻,便是他们所属阶级的政策之特殊形态。

马克思主义的哲学――辩证法唯物论的特征,在于要明确地理解一切社会意识(哲学也在内)的阶级性,公然声明它那无产阶级的性质的向有产阶级的唯心论哲学作坚决的斗争,并且把自己的特殊任务,从属于推翻资本主义组织、建立无产阶级专政,与建设社会主义的一般任务之下。在中国目前阶段上,哲学的任务,是从属于推翻帝国主义与半封建制度,彻底实现资产阶级的民主主义,并准备转变到社会主义与共产主义社会去的一般任务之下。哲学的理论与政治实践是应该密切联系着的。

二、唯心论与唯物论的区别

唯心论与唯物论的根本区别在那里呢?在于对哲学的根本问题,即精神与物质的关系问题(意识与存在的关系问题)之相反的回答。唯心论认精神(意识,观念,主体)为世界一切的根源,物质(自然界及社会,客体)不过为其附属物。唯物论认物质离精神而独立存在,精神不过为其附属物。从这个根本问题的相反的回答出发,就生出一切问题上的分歧意见来。

在唯心论看来,世界或者是我们各种知觉的综合,或者是我们的或世界的理性所创造的精神过程。对外面的物质世界,或者完全把它看成虚构的幻想,或者把它看成精神元素之物质的外壳。人类的认识,是主体的自动,是精神的自己产物。

唯物论相反,认宇宙的统一就在他的物质性。精神(意识)是物质的本性之一,是物质发展到一定阶段时才发生的。自然,物质,客观世界,存在于精神之外,离精神而独立。人的认识,是客观外界的反映。

三,唯心论发生与发展的根源

唯心论认物质为精神的产物,颠倒着实在世界的姿态。这种哲学的发生与发展的根源何在?

前面说过,最初唯心论之发生,是原始野蛮人类迷妄无知的产物。但在生产发展之后,促使唯心论形成哲学思潮之首先的条件,乃是肉体劳动与精神劳动的分裂。社会生产力发展的结果,社会发生分工,分工再发展,分出了专门从事精神劳动的人们。但在生产力贫弱时期,两者的分裂还没有达到完全分离的程度。到了阶级出现,私产发生,剥削成为支配阶级存在的基础之时,就起了大变化了。精神劳动成为支配阶级的特权,肉体劳动成为被压迫阶级的命运。支配阶级开始颠倒地去考察自己与被压迫阶级之间的相互关系,不是劳动者给他们以生活资料,反而是他们以生活资料给与劳动者,因此他们鄙视肉体劳动,发生了唯心论的见解。消灭肉体与精神劳动的区别,是消灭唯心论哲学的条件之一。

使唯心论哲学能够发展的社会根源,主要的进在于这种哲学意识地表现剥削阶级的利害。唯心论哲学在一切文化领域的优越,应该拿这个去说明。假如没有剥削阶级的存在,唯心论就会失掉它的社会根据。唯心论哲学之最后消灭,必须在阶级消灭与共产主义社会成立之后。

使唯心论能够发达,深化,并有能力同唯物论斗争,还须在人类的认识过程中寻找其根源。人类在使用概念来思考的时候,存在着溜到唯心论头的可能性。人类思考时,不能不使用概念,这就使我们的认识分裂为二方面,一方面是个别的与特殊性质的事物,一方面是一般性质的概念(例如“延安是城”这个判断)。特殊和一般本来是互相联系不可分裂的,分裂就脱离了客观真理。客观真理是表现于一般与特殊之一致的,没有特殊,一般就不存在,没有一般,也不会有特殊。把一般同特殊脱离开来,即把一般当作客观的实体看待,把特殊只当作一般之存在的形式,这就是一切唯心论者所釆取的方法。一切唯心论者,即是拿意识,精神或观念来代替离开人的意识而独立存在的客观实体的。从这里出发,唯物论便强调着人类意识在社会实践中的能动性,他们不能指出意识受物质限制的这种唯物论的真理,却主张只有意识是能动的,物质不过是不动的集合体。加上被阶级的本性所驱策,唯心论者便用一切方法把意识的能动性夸张起来,片面地发展了它,使这一方面在心智之中无限制的胀大成为支配的东西,掩蔽着别一方面,并使之服从,而把这一人工的胀大的东西确定为一般的宇宙观,以至化为物神或偶像。经济学上的唯心论,过分夸大交换中非本质的一方面,把供求法则提高到资本主义的根本法则。许多人看到科学在社会生活上发生了能动的作用,不知道这种作用受一定的社会生产关系所规定与限制,而作出科学是社会发动力的结论。唯心论历史家把英雄看成历史的创造者,唯心论政治家把政治看成万能的东西,唯心论军事家实行拚命主义的作战,唯心论革命家主张白朗基主义,蒋介石说要复兴民族惟有恢复旧道德,都是过分夸张主观能动性的结果。我们的思维不能一次反映出当作全体看的对象,而是构成一个具有接近于现实的一切种类的无数色调的,生动的,认识之辩证法的过程。唯心论依据于思维的这种特性,夸大其个别方面,不能给过程以正确反映,反把过程弄弯曲了。列宁说:“人类的认识不是直线的,而是曲线的。这一曲线之任何一段,都可以变为一段单独的完整的直线,这段直线就有引你陷入迷阵的可能。直线性和片面性是见树不见林和呆板固执性,主观主义和主观盲目性一一这些就是唯心论的认识论的根源”。哲学的唯心论是将认识的一个片段或一个方面,片面地夸张成为一种脱离物质、脱离自然的神化的绝对体。唯心论就是宗教的教义,这是很对的。

马克思以前的唯物论(机械唯物论)没有强调思维在认识上的能动性,仅给以被动作用,把它当作反映自然的镜子看。机械唯物论对唯心论采取横暴的态度,不注意其认识论的根源。因此不能克服唯心论。只有辩证唯物论,正确地指出思维的能动性,同时又指出思维受物质的限制。指出思维从社会实践中发生,同时又能动的指导实践。只有这种辩证法的“知行合一”论,才能彻底地克服唯心论。

四、唯心论发生与发展的根源

承认离意识而独立存在于外界的物质是唯物论的基础,这一基础是人类从实践中得到的。劳动生产的实践,阶级斗争的实践,科学实验的实践,使人类逐渐从迷信与妄想(唯心论)脱离,逐渐认识世界之本质,而达到于唯物论。

屈服于自然力之前而只能使用简单工具的原始人类,不能说明周围的事变,因而求助于神灵,这就是宗教同唯心论的起源。

然而人类在长期的生产过程中,同周围的自然界接触,作用于自然界,变化着自然界,造衣食住用的东西,使之适合于人类的利益,使人类深信物质的客观地存在着。

人类在社会生活中,人同人之间互相发生关系与影响,在阶级社会中并且实行着阶级斗争。被压迫阶级考虑形势,估计力量,建立计划,在他们的斗争成功时,使他们确信自己的见解并不是幻想的产物,而是客观上存在着的物质世界的反映。被压迫阶级因为釆取错误的计划而失败,又因为改正其计划而成功,使他们懂得只有主观的计划依靠于对客观世界的物质性及规律性的正确的认识,才能达到目的。

科学的历史给人类证明世界的物质性及规律性,使人类觉悟到宗教与唯心论的幻想之无用,直到达于唯物论的结论。

总之,人类的实践史一一向自然斗争史,阶级斗争史,科学史,在长久年月中,为了生活与斗争的必要,考虑物质的现实及其法则,证明了唯物论哲学的正确性,找到了自己斗争的思想工具一一唯物论哲学。社会的生产发展越发进到高度,阶级斗争越发发展,科学认识越发暴露了自然的“秘密”。唯物论哲学就越发发展与巩固,人类便能逐渐从自然与社会的双重压迫之下解放了出来。

资产阶级在为了向封建阶级斗争的必要、及无产阶级还没有威胁他们的时候,也曾经找到了并使用了唯物论作为自己斗争的工具,也曾经确信周围的事物是物质的产物。而不是精神的产物。直至他们自己变成了统治者,无产阶级的斗争又威胁着他们时,才放弃这个“无用”的工具,重新拿起另一个工具一一哲学的唯心论。中国资产阶级的代言人戴季陶,吴稚晖,在一九二七年以前及其以后思想的变化一一从唯物论到唯心论的变化,就是眼前的活证据。

资本主义的掘墓人一一无产阶级,他们本质上是唯物论的。但由于无产阶级是历史上最进步的阶级。就使得无产阶级的唯物论不同于资产阶级的唯物论,是更彻底更深刻的,只有辩证法的性质.没有机械论的性质。无产阶级吸收了人类全历史中一切实践的成果,同时又由于自己的实践,经过他们的代言人与领导者一一马克思、恩格斯之手,造成了辩证法唯物论。不但主张物质离人的意识而独立存在,而且主张物质是变化的,成为整个完整系统的崭新的世界观与方法论,这就是马克思主义的哲学。

第二章辩证法唯物论

这个题目中准备讨论下列问题,

一、无产阶级革命的武器一一辩证法唯物论;

二、过去哲学遗产同辩证法唯物论的关系;

三、在辩证法唯物论中宇宙观和方法论的一致;

四、哲学对象问题;

五、物质论;

六、运动论;

七、时空论;

八、意识论;

九、反映论;

十、真理论;

十一、实践论。

下面简述这些问题的观点.

一、辩证法唯物论是无产阶级革命的武器

这个问题在第一章中已经说过.这里再简单地说一点。

辩证法唯物论,是无产阶级的宇宙观。历史给予无产阶级以消灭阶级的任务,无产阶级就用辩证法唯物论作为他们斗争的精神上的武器,作为他们各种见解之哲学基础。辩证法唯物论这种宇宙观,只有当我们站在无产阶级的立场去认识世界的时候,才能够被我们正确地和完整地把握住;只有从这种立场出发,现实世界才得真正客观地被认识。这是因为一方面只有无产阶级才是最先进与最革命的阶级;又一方面,只有辩证法唯物论才是高度的和严密的科学性,同彻底的和不妥协的革命性密切地结合着的、一种最正确的和最革命的宇宙观和方法论。

中国无产阶级担负了经过资产阶级民主革命到达社会主义与共产主义的历史任务,必须采取辩证法唯物论作为自己精神的武器。如果辩证法唯物论被中国无产阶级、共产党、及一切愿意站在无产阶级立场的人们之广大革命分子所采取的话,那末,他们就得到了一种最正确和最革命的宇宙观和方法论,他们就能够正确地了解革命运动的发展变化,提出革命的任务,团结自己的同盟者的队伍,战胜反动的理论,釆取正确的行动,避免工作的错误,达到解放中国与改造中国的目的。辩证法唯物论对于指导革命运动的干部人员,尤属必修的科目,因为主观主义与机械观这两种错误的理论与工作方法,常常在干部人员中间存在着,因此常常引导干部人员违反马克思主义,在革命运动中走入歧途。要避免与纠正这种缺点,只有自觉地研究与了解辩证法唯物论,把自己的头脑重新武装起来。

二、旧的哲学遗产同辩证法唯物论的关系

现代的唯物论,不是过去各种哲学学说之简单的继承者。它是从反对过去统治哲学的斗争中,从科学解除其唯心论和神秘性的斗争中产生和成长起来的。马克思主义的哲学一一辩证法唯物论,不但继承了唯物论的最高产物一一黑格尔学说的成果,同时还克服了这一学说的唯心论,唯物地改造了他的辩证法。马克思主义又不但是一切过去唯物论发展的继续和完成,同时还是一切过去唯物论的狭隘性之反对者,即机械的直觉的唯物论(主要的是法国唯物论与费尔巴哈唯物论)之反对者。马克思主义的哲学一一辩证法唯物论,继承了过去文化之科学遗产,同时又给此种遗产以革命的改造,形成了一种历史上从来没有过的、最正确最革命的、有最完备的哲理的科学。

中国在一九一九年五四运动以后,随着中国无产阶级自觉地走上政治舞台及科学水平之提高,发生了与发展着马克思主义的哲学运动。然而在它的第一时期,中国的唯物论思潮中,唯物辩证法的了解还很微弱,受资产阶级影响的机械唯物论,和德波林派的主观主义风气占着主要的成分。一九二七年革命失败以后,马克思列宁主义的了解进了一步,唯物辩证法的思想逐渐发展起来。到了最近,由于民族危机与社会危机的严重性,也由于苏联哲学清算运动的影响,便在中国思想界发展了一个广大的唯物辩证法运动。这个运动目前虽还在青年的阶段上,然从其广大的姿态来看,它将随着中国与世界无产阶级同革命人民的革命斗争之发展,以横扫的阵势树立自己的权威,指导中国革命运动勇往迈进,定下中国无产阶级领导中国革命进入胜利之途的基础。

由于中国社会进化的落后,中国今日发展着的辩证法唯物论哲学思潮,不是从继承与改造自己哲学的遗产而来的,而是从马克思列宁主义的学习而来的。然而要使辩证法唯物论思潮在中国深入与发展下去,并确定地指导中国革命向着彻底胜利之途,便必须同各种现存的反动哲学作斗争,在全国思想战线上树立批判的旗帜,并因而清算中国古代的哲学遗产,才能达到目的。

三、辩证法唯物论中宇宙观和方法论的一致性

辩证法唯物论是无产阶级的宇宙观,同时又是无产阶级认识周围世界的方法和革命行动的方法;它是宇宙观和方法论的一致体,唯心论的马克思主义修正派认为辩证法唯物论的全部实质只在于它的“方法”,他们把方法从一般哲学的宇宙观割裂开来,把辩证法从唯物论割裂开来。他们不了解马克思主义的方法论一一辩证法,不是如同黑格尔一样的唯心的辩证法,而是唯物的辩证法,马克思主义的方法论是丝毫也不能离开它的宇宙论的。另一方面,机械唯物论者却又仅把马克思主义哲学看作一般哲学的宇宙观,割去了它的辩证法,而且认为这种宇宙观就是机械的自然科学之各种结论。他们不了解马克思主义的唯物论不是简单的唯物论,而是辩证法的唯物论。对于马克思主义哲学之这两种割裂的看法都是错误的,辩证法唯物论是宇宙观和方法论的一致体。

四、唯物辩证法的对象问题一一唯物辩证法是研究什么的


列宁把(作为马克思主义的哲理科学来看的)唯物辩证法看做关于客观世界的发展法则。及(在辩证法的各范畴中反映这客观世界的)认识的发展法则的学问。他说:论理学不是关于思维的外在形式的学问而是关于一切物质的,自然的,及精神的事物之发展法则的学问,即关于世界的一切具体内容及其认识之发展法则的学问。换言之,论理学是关于世界认识之历史的总计、总合、结论。列宁虽然把作为一般的科学方法论看的唯物辩证法的意义强调起来,然而这是因为辩证法系由世界认识的历史中得出来的结论。因此他说:“辩证法就是认识的历史。”

上述列宁对于当作科学看的唯物辩证法及其对象所给予的定义,他的意思是说:第一,唯物辩证法与其他任何科学同样,有它的研究对象,这个对象便是自然、历史和人类思维之最一般的发展法则。并且研究的时候,唯物辩证法的任务,不是从头脑里想出存在于各现象间的关系,而是要在各现象本身中观察出他们之间的关系来。列宁的这种见解同少数派唯心论者把(事实上离开了具体科学及具体知识的)范畴的研究当做唯物辩证法的对象之间,存在着根本的区别。因为少数派唯心论者企图建立一个认识历史社会科学和自然科学的现实发展中游离了的各范畴的哲学体系,这样他们就事实上放弃唯物辩证法。第二,各个科学分析(数学、力学、化学、物理学、生物学、经济学及其他自然科学、社会科学),是研究物质世界及其认识之发展的各个方面。因此各个科学的法则是狭隘地片面地被各个具体研究领域所限制了的。唯物辩证法则不然,它是一切具体科学中的一切有价值的一般内容,及人类的其他一切科学认识之总计、结论、加工和普遍化。这样,唯物辩证法的概念、判断和法则,是极其广泛的(包含着一切科学的最一般的法则,因此也包含着物质世界的本质的)各种规律性和规定。这是一方面,在这方面,它是宇宙观。另一方面,唯物辩证法是从一切空想、僧侣主义和形而上学解放出来的真正科学认识上理论学和认识论的基础,因此,它同时又是研究具体科学的唯一确实的有客观真实性的方法论。我们说唯物辩证法或辩证法唯物论是宇宙观和方法论的一致体,在这里更加明白了。这样对于否认哲学存在权的马克思主义哲学的歪曲者和庸俗化者的错误,也可以懂得了。

关于哲学对象问题,马克思、恩格思和列宁,都反对使哲学脱离实在的现实,使哲学变为某种独立实质的东西。指出了那根据实在生活和实在关系的分析而生长出来的哲学之必然性,反对单单以理论观念和理论观念的自身做研究的对象,如同形式理论就是少数派唯心论的那种干法。所谓根据实在生活和实在关系的分析生长出来的哲学。就是唯物辩证法这种论发展的学说。马克思、恩格斯和列宁,都解说唯物辩证法为论发展的学说。恩格斯称唯物辩证法为“论自然社会及思维之运动和发展的一般法则”的学说。列宁把唯物辩证法看作“最多方向的、内容最丰富的和最深刻的发展学说”。他们都认为在这种学说以外的其他一切哲学学说所述一切发展原则的公式,概属狭隘的、无内容的“失去了自然和社会之实际发展过程的东西”(列宁)。至于唯物辩证法之所以被称为最多方面的、内容最丰富的和最深刻的发展学说的原故,乃是因为唯物辩证法是最多方面地和最丰富地、最深刻地反映了自然和社会变化过程中的矛盾性和飞跃性,而不是因为别的东西。

在哲学对象问题中还要解决一个问题。就是辩证法、论理学、认识论的一致性的问题。

列宁着重指出辩证法论理学及认识论的同一性,说这是“极其重要的问题”,说“三个名词是多余的,它们只是一个东西”,根本反对那些马克思主义修正派把三者当做完全各别独立的学说去处理的那种干法。

唯物辩证法是唯一科学的认识论,也是唯一科学的论理学。唯物辩证法研究吾人对外界认识的发生及发展,研究由不知到知,由不完全的知到更完全的知的转移,研究自然及社会的发展法则在人类头脑中日益深刻和日益增多的反映,这就是唯物辩证法与认识论的一致。唯物辩证法研究客观世界最一般的发展法则,研究客观世界最发展的姿态在思维中的反映形态,这就是唯物辩证法研究现实事物的各过程及各现象的发生发展消灭及相互转化的法则,同时又研究反映客观世界发展法则的人类思维的形态,这就是唯物辩证法与论理学的一致。

要彻底了解辩证社论理学认识论三者为什么是一个东西,我们看下面唯物辩证法怎样解决关于理论的东西与历史的东西之相互关系这个问题,就可以明白了。

恩格斯说:“对于一切哲学家的思维方法来说,黑格尔思维方法的长处就在于横亘在根底面的极其丰富的历史感,他的形式虽说是抽象的唯心论的,然而他的思想的发展却非常是与世界历史的发展平行着的。并且历史原来就是思想的检证。历史常常在飞跃地错杂地进行着。因为有这两种情形,所以假若常常要依从历史的话,不但要注意许多不重要的材料,而且会不得不使思想行程中断。这时唯一适当的方法,就是论理的方法。然而这一论理的方法权本仍然是历史的方法,不是舍去了那历史的形态与偶然性罢了。”这种“论理发展与历史发展一致”的思想,是被马克思、恩格斯、列宁充分注意了的。“论理学的范畴,是外的定在与活动之无数个别性的简约”。“范畴就是分离的阶段,帮助我们去认识这一个网和网的结节点的”。“人的实践活动,把人类的意识几十亿次反复不息地应用到各种这样的论理学式子里面,这样,这些式子就得到了所谓公理的意义了”。“人类的实践,反复了几十亿次,才当做论理的式子固定在人类意识中。这些式子,都有着成见的永续性,因为是反复几十亿次的结果,才有着公理的性质。”上述列宁的那些话,指明唯物辩证法的论理学的特点,不象形式论理学那样,把它的法则私范畴看成空虚的、脱离内容而独立的,对于内容无关心的形式,也不象黑格尔那样,把它看成脱离物质世界而独立发展的观念要素,而是把它当做反映到和移植到我们头脑里并且经由头脑加工制造过的,物质运动的表现去处理。黑格尔立脚在存在和思维的同一性上,把辩证法,论理学,和认识论的同一性当做唯心论的同一性去处理。反之,马克思主义的哲学里,辩证法、论理学和认识论的同一性,是建立在唯物论基础上的。只有照唯物论解决存在与思维的关系问题,只有站在反映论的立场上,才能使辩证法论理学和认识论的问题得到彻底的解决。

用辩证法唯物论去解决论理的东西和历史的东西的相互关系的最好的模范,首先要算马克思的资本论。资本论中包含了资本主义社会的历史发展,同时又包含了这一社会的理论发展。资本论所分析的,是那把资本主义社会的发生发展及消灭、反映出来的各经济范畴的发展的辩证法。这问题之解决的唯物论性质,在于他以物质的客观历史做基础,在于把概念和范畴当做这一现实历史的反映。资本主义的理论和历史的一致,资本主义的社会的论理学和认识论的一致,模范地表现在资本论里面,我们可以从它懂得一点辩证法论理学和认识论一致的门径。

以上是辩证法唯物论的对象问题

五、物质论

马克思主义继续和发展哲学中的唯物论路线,正确地解决了思维与存在的关系问题,即彻底唯物地指出世界的物质性,物质的客观实在性和物质对于意识的根源性(或意识对于存在的依赖关系)。

承认物反对于意识的根源性,是以世界的物质性及其客观存在为前提的。隶属于唯物论营垒的第一个条件就是承认物质世界离人的意识而独立存在一一人类出现以前它就存在,人类出现以后也是离开人的意识而独立存在的。承认这一点是一切科学研究的根本前提。

拿什么来证明这一点呢?证据是多得很的。人类时刻同外界接触,还须用残酷的手段去对付外界(自然界同社会)的压迫和反抗;还不但应该而且能够克服这些压迫和反抗一一所有这些在人类社会的历史发展中表现出来的人类社会实践的实在情形,就是最好的证据。经过了万里长征的红军,不怀疑经过地区连同长江大河、雪山草地以及和他作战的敌军等等的客观存在,也不怀疑红军自己的客观存在;中国人不怀疑侵略中国的日本帝国主义同中国人自己的客观存在;抗日军政大学的学生也不怀疑这个大学和学生自己的客观存在;这些东西都是客观地离开我们意识而独立存在的物质的东西,这是一切唯物论的基本观点,也就是哲学的物质观。

哲学的物质观同自然科学的物质是不相同的。如果说哲学的物质观在于指出物质的客观存在,所谓物质就是说的离开人的意识而独立存在的整个世界(这个世界作用于人的感官,引起人的感觉,并在感觉中得到反映),那末这种说法是永远不起变化的,是绝对的。自然科学的物质观则在于研究物质的构造,例如从前的原子论,后来的电子论等等,这些说法是随着自然科学的进步而变化的,是相对的。

根据辩证法唯物论的见地去区别哲学的物质观与当然科学的物质观,是彻底贯彻哲学的唯物论方向之必须条件,在向唯物论和机械唯物论作斗争方面,有着重要的意义。

唯心论者根据电子论的发见,宣传物质消灭的谬说,他们不知道关于物质构造之科学知识的进步,正是证明辩证法唯物论的物质论之正确性。因为表现在旧的物质概念中的某些物质属性(重量、硬度、不可入性、惰性等等),经过现代自然科学的发现即电子论的发现,证明这些属性仅存在于某几种物质形态中,而在其他物质形态中则不存在。这种事实,破除了旧唯物论对于物质观念的片面性与狭隘性,而对于承认世界的物质性及其客观存在之辩证法唯物论的物质观,却恰恰证明其正确。原来辩证认唯物论的物质观,正是以多样性去看物质的世界的统一,就是物质多样性的统一。这种物质观,对于物质由一形态转化到另一形态之永久普遍的运动变化这一种事实,丝毫也没有矛盾。“以太”、电子、原子、分子、结晶体、细胞社会现象、思维现象一一这些都是物质发展的种种阶段,是物质发展史中的种种暂时形态。科学研究的深入,各种物质形态的发现(物质多样性的发现),只是丰富了辩证法唯物论的物质观的内容,那里还会有什么矛盾?区别哲学的物质观同自然科学的物质观是必要的,因为二者有广狭之别,然而是不相矛盾的,因为广义的物质包括了狭义的物质。

辩证法唯物论的物质观,不承认世界有所谓非物质的东西(独立的精神的东西)。物质是永久与普遍存在的,不论在时间与空间上都是无限的,如果说世界上有一种“从来如此”与“到处如此”的东西(就其统一性而言),那就是哲学上的所谓客观存在的物质。用彻底的唯物论见地(即唯物辩证法见地)来看意识这种东西,那末,所谓意识不是别的,它是物质运动的一种形态,是人类物质头脑的一种特殊性质,是使意识以外的物质过程反映到意识之中来的那种物质头脑的特殊性质。由此可知,我们区别物质同意识并把二者对立起来是有条件的,就是说,只在认识的见地内有意义。因为意识或思维只是物质(头脑)的属性,所以认识与存在的对立就是认识的物质同被认识的物质的对立,不会多一点。这种主体同客体的对立,离开认识论领域,就毫无意义。假如在认识论以外还把意识同物质对立起来,就无异于背叛唯物论。世界上只有物质同它的各种表现,主体自身也是物质的,所谓世界的物质性(物质是永久与普遍的),物质的客观实在性与物质对于意识的根源性,这是这个意思。一句话,物质是世界的一切,“一切归于司马懿”,我们说“一切归于物质”,这就是世界的统一原理。

以上是辩证法唯物论的物质论。

六、运动论(发展论)

辩证法唯物论的第一个基本原则,在于它的物质论,即承认世界的物质性,物质客观实在性和物质对于意识的根源性。这种世界的统一原理,在前面物质论中已经解决了。

辩证法唯物论的第二个基本原则,在于它的运动论(或发展论),即承认运动是物质存在的形式,是物质内在的属性,是物质多样性的表现,这就是世界的发展原理。世界的发展原理同上述世界的统一原理相结合,就成为辩证法唯物论整个的宇宙观。世界不是别的,就是无限发展的物质世界(或物质世界是无限发展的)。

辩证法唯物论的运动观,对于(一)离开物质而思考运动,(二)离开运动而思考物质,(三)物质运动的简单化,都是不能容许的。辩证法唯物论的运动论,就是同这些唯心的、形而上学的及机械的观点作明确而坚决的斗争建立起来的。

辩证法唯物论的运动论,首先是同哲学的唯心论及宗教的神道主义相对立的。一切哲学的唯心论及宗教的神道主义的本质,在于他们从否认世界的物质统一性出发,设想世界的运动及发展是没有物质的,或在最初是没有物质的,而是精神作用或上帝神力的结果。德国唯心论哲学家黑格尔认为现在的世界是从所谓“世界理念”发展而来的,中国的周易哲学及宋明理学都作出唯心论的宇宙发展观。基督教说上帝创造世界,佛教及中国一切拜物教都把宇宙万物的运动发展归之子神力。所有这些离物质而思考运动的说法都和辩证法唯物论根本不相容。不但唯心论与宗教,就是马克思以前的一切唯物论及现在一切反马克思主义的机械唯物论,当他们说到自然现象时是唯物论的运动论者,但一说到社会现象时就无不离开物质的原因,而归着于精神的原因了。

辩证法唯物论坚决驳斥所有这些错误的运动观,指出他们的历史限制性一一阶级地位的限制与科学发展程度的限制,而把自己的运动观建设在以无产阶级立场及最发达的科学水准为基础的彻底的唯物论上面,辩证法唯物论首先指出运动是物质存在的形式,是物质内在的属性(不是由外力推动的),设想没有物质的运动,同设想没有运动的物质是一样不可思议的事。把唯物的运动观同唯心的及唯神的运动观尖锐地对立着。

离开运动而考察物质,则有形而上学的宇宙不动论或绝对平衡论。他们认为物质是永远不变的,在物质中没有发展这回事,认为绝对的静止是物质的一般状态或原始状态。辩证法唯物论坚决反对这种意见,认为运动是物质存在的最普遍的形式,是物质内在的不可分离的属性。一切的静止与均衡仅有相对的意义,而运动则是绝对的。辩证法唯物论承认一切物质形态均有相对的静止或均衡之可能,并认为这是辨别物质因而亦即辨别生命的最重要条件(恩格斯),但认为静止或均衡只是运动的要素之一,是运动的一种特殊情况。离开运动而考察物质的错误,就在于把这种静止要素或均衡要素夸张起来,把它掩蔽了并代替了全体,把运动的特殊情况一般化,绝对化起来。中国古代形而上学思想家爱说的一句话:“天不变道亦不变’,就是这样的宇宙不动论。他们也承认宇宙及社会现象的变动,但否认其本质的变动,在他们看来,宇宙及社会的本质是永远不变动的。他们之所以如此,主要的原因在于他们的阶级限制性,封建地主阶级如果也承认宇宙及社会的本质是运动与发展的,就无异在理论上宣布他们自己阶级的死刑。一切反动势力,他们的哲学都是不动论。革命的阶级同民众,却眼看到了世界的发展原理,因而主张改造这个社会及世界,他们的哲学是辩证法唯物论。

此外,辩证法唯物论也不承认简单化的运动观,就是说把一切的运动都归结到一种形式上去,即归结到机械的运动,这是旧唯物论宇宙观的特点。旧唯物论(十七八世纪的法国唯物论,十九世纪的德国费尔巴哈唯物论)也承认物质的永久存在和永久运动(承认运动的无限性),但仍然没有跳出形而上学的宇宙观.不去说他们在社会论上的见解依然是唯心论的发展观,就在自然论上,也把物质世界的统一,归结到某种片面的属性,归结到运动的一个形态一一机械的运动。这种运动的原因在外力。象机械一样由外力推之而运动。他们不从本质上也不从内部原因上去说明物质或运动,或关系的一切多样性,而从单纯的外面的发现形式上,从外力原因上去说明它,这样在实际上就失掉了世界的多样性。他们把世界一切的运动,都解作场所的移动与数量的增减。物体某一瞬间在某一场所另一瞬间则在另一场所,这样就叫做运动。如果有变化,也只是数量增减的转化,没有性质的变化,变化是循环的,是反复产生同一结果的。辩证法唯物论与此相反,不把运动看作单纯的场所移动及循环运动,而把它看作无限的质的多样性,看作由一形态向他一形态的转化,世界物质的统一和物质的运动,便是世界物质无限多样性的统一与运动。恩格斯说:“运动的一切高级形态必然同力学的(外的或分子的)运动形态结合着,例如如果没有热和电气的变化,化学的作用就不可能,如果没有力学的(分子的)热量的,电气的,化学的变化等等,有机的生命也不可能,这当然是不能否认的。然而如果只有某些低级运动形态的存在,是决不能包括各种状况中主要形态的体质的”。这话是千真万确地合于事实。即使就单纯机械运动而论,也不能从形而上学的观点去解释它。须知一切运动形态都是辩证法的,虽然它们之间的辩证法内容的深度与多面性有着很大差异。机械运动仍然是辩证法的运动,所谓物体某一瞬间“在”某处,其实是同时“在”某处,同时又不在某处,所谓“在”某处,所谓“不动”,仅是运动的一种特殊情况,它根本上依然是在运动。物体在被限制着的时间内和被限制着的空间内运动着。物体总是不绝地克服这种限制性,跑出这种一定的有限制的时间及空间的界限以外去,成为不绝的运动之流.而且机械运动只是物质的运动形态之一,在实在的现实世界中,没有它的绝对独立的存在,它总是联系于别种运动形态的。热、化学的反映、光、电气,一直到有机现象与社会现象,都是质地上特殊的物质运动形态。十九世纪与二十世纪交界时期的自然科学的划时代的大功劳,就在于发现了运动转化法则,揩出物质的运动总是由一形态转化成为另一形态,这样地转化的新形态是与旧形态本质上不同的。物质所以转化的原因不在外部而在内部,不是由于外部机械力的推动,而是由于内部存在着性质不同的互相矛盾的两种因素相争相斗,推动着物质的运动和发展。由于这个运动转化法则的发现,辩证法唯物论就能够把世界的物质统一原理扩大到自然与社会的历史上去,不但把世界当作永远运动的物质去考察,而且把世界当做由低级形态到高级形态的无限前进运动的物质去考察,即把世界当做发展、当做过程去考察。做一句话来说:“统一的物质世界是一个发展的过程”。这样就把旧唯物论的循环论击破了。辩证法唯物论深刻地多方面地观察了自然及社会的运动形态,认为当作全体看的世界之发展过程是永久的(无始无终的),但同时各个历史地进行的具体的运动形态又是暂时的(有始有终的),就是说,它是在一定的条件下发生,并在一定的条件下消灭的,认为世界的发展过程由低级的运动形态生出高级的运动形态,表示了它的历史性与暂时性,但同时任何一个运动形态都不是绝对最初的,也不是绝对最后的,它是处在永久的长流中(无始无终的长流中)。依据着对立斗争的法则(自己运动的原因)。使每一运动形态总是较之先行形态进到了高一级的阶段,它是向前直进的,但同时就各个运动形态来说(就各个具体的发展过程来说),却也会发生转向的运动或后退运动,前进运动同后退运动相结合,在全体上就成为复杂的螺旋的这动,认为新的这动形态是作为旧的运动形态的对立的(反对物)而发生的,但同时新的运动形态又必然保存着旧的运动形态中的许多要素,新东西是从旧东西里面生长出来的。认为事物的新形态新性质新属性的出现,是由连续性的中断即经过冲突和破坏而飞跃地产生的,但同时事物的连结和相互关系又决不会绝对破坏。最后,辩证法唯物论认为世界无穷尽(无限),不但就其全体来看是这样,同时就其局部来看也是这样的,电子不是同原子分子一样表现着一个复杂而无穷尽的世界么?

物质运动的根本形态又规定根本的自然科学与社会科学各科目。辩证法唯物论把世界的发展当作无机界经过有机界而达到最高物质运动形态(社会)的一个前进运动去考察,这一运动形态的从属关系就成了和它相应的科学(无机界科学,有机界科学,社会科学)的从属关系的基础。恩格斯说:“各种各类的科学是把特定的运动形态或相互关系相互推移的一联的运动形态拿来分析,因此科学的分类就在于要依从着运动的固有顺序去把各个运动分类排列起来,仅在这一点来说,分类才有意义”。

“整个世界包括人类社会在内,是采取质地不相同的各种形式的物质的这动,因此也就不能忘记物质运动的各种具体形式这个问题。所谓“物质一般”与“运动一般’是没有的,世界上只有各种不同形式的具体的物质与运动。“物质”和“运动”这些字眼只是一些简写的名词,在这些名词中,我们依照它们的共同特性是把各种不同的被感觉的事物一概包括在内”(恩格斯)。

以上就是辩证法唯物论的世界运动论或世界发展原理。这个学说是马克思主义哲学的精髓,是无产阶级的宇宙观与方法论,无产阶级及一切革命的人们如果拿着这个彻底科学的武器,他们就能够理解这个世界并改造这个世界。

七、时空论

运动是物质存在的形式,空间和时间也是物质存在的形式,运动的物质存在于空间和时间中,并且物质的运动本身是以空间和时间这两种物质存在的形式为前提的。空间和时间不能与物质相分离。“物质存在于空间”这句话,是物质本身具有伸张性,物质世界是内部存在着伸张性的世界,不是说物质被放在一种非物质的空虚的空间中。空间和时间都不是独立的非物质的东西,也不是我们感觉性的主观形式,它们是客观物质世界存在的形式。它们是客观的,不存在于物质以外,物质也不存在于它们以外。

把空间和时间看作物质存在的形式的这种见解,是彻底的唯物论的见解。这种时空观,同下列几种唯心论的时空观是根本相反的:(一)康德主义的时空观,认时间和空间不是客观的实在,而是人类的直觉形式;(二)黑格尔主义的时空观,认发展着的时间和空间的概念,日益接近于绝对观念;(三)马赫主义的时空观,认时间和空间是“感觉的种类’,“使经验和谐化的工具”。所有这些唯心论观点,都不承认时间和空间的客观实在性,都不承认时间和空间的概念在自身发展中反映着物质存在的形式,这些错误理论,都被辩证法唯物论一个一个地驳了。

辩证法唯物论在时空问题上,不但要同上述那些唯心论观点作斗争,而且要同机械唯物论作斗争。特别显著的是牛顿的机械论,他把空间看做同时间无关系的不动的空架子.物质被安置到这种塞架子里面去。辩证法唯物论反对这种机械论,指出我们的时空观念是在发展的。“世界上除了运动的物质以外便没有别的东西,而运动的物质若不在空间和时间中便无运动之可能。人类关于空间和时间的概念是相对的,但是这些相对的概念积集起来就成为绝对的真理。这些相对的概念不断发展着,循着绝对真理的路线而前进,日益走近于绝对真理。人类关于空间和时间概念的变动性,始终不能推翻二者的客观实在性,这正和关于物质的运动形式及其组织之科学知识的变动性,不能推翻外界的客观实在性,是一样的”(列宁)。

以上是辩证法唯物论的时空论。

八、意识论

辩证唯物论意识是物质的产物,是物质发展之一形式,是一定物质态的形特性。这种唯物主义同历史主义的意识论是和一切唯心论及机械唯物论对于这个问题的观点根本相反的。

依照马克思主义的见解,意识的来源,是由无意识的无机械发展到具有低级意识形态的动物界,再发展到具有高级意识形态的人类。高级意识形态不但同生理发展中的高级神经系统不可分离,而且同社会发展中的劳动生产不可分离。马克思、恩格斯曾着重指出意识对物质生产发展的依赖关系,和意识同人类言语发展的关系。

所谓意识是一定物质形态的特性,这种物质形态就是组织复杂的神经系统,这样的神经系统只能发生于自然界进化的高级阶段上。整个无机界,植物界和低级的动物界,都没有认识在它们内面或外面发生着的那些过程的能力,它们是没有意识的。仅在有高级神经系统的动物体,才具有认识过程的能力,即具有自内反映或领悟这些过程的能力。吾人神经系统中的客观生理过程,是同它之内部取意识形式的主观表现相随而行的。凡就本身论是客观的东西,是某种物质的过程,它对于具有头脑的实体却同时又是主观的心理的行为。

特殊思想实质的精神是没有的,有的只是思想的物质一一脑子。这种思想的物质是有特别质地的物质,这种物质随着人类社会生活中言语的发展而达到高度的发展。这种物质具有思想这一种特殊性质,这是任何别的物质所不具备的。

然而庸俗唯物论者却认思想是脑子分泌出来的物质,这种见解歪曲了我们关于这个问题的观念。须知思想、感情和意志的行为,.不具有重量和伸张性的东西,意识同重量伸张等是同一物质之不同的性质。意识是运动的物质之内部状态,是反映着在这动的物质中所发生的生理过程的特殊性质。这种特殊性,同客观的神经作用过程不可分离,但又不与这过程相同,把这两者混同起来,推翻意识的特殊性,这就是庸俗唯物论的观点。

和这同样,冒牌的马克思主义的机械论,附和心理学中某些资产阶级的左翼学派的见解,实质上也完全推翻了意识。他们把意识解作理化的生理的过程,认为高级实体的行为之研究可以由客观生理学和生物学的研究去执行。他们不了解意识的本质之质的特殊性,看不到意识是人类社会实践的产物。他们把客观和主体之具体历史一致代之以主客的等同,代之以片面的机械的客观世界。这种把意识混同于生理过程的观点,无异取消了思维与存在关系这个哲学中的根本的问题。

孟塞维克的唯心论企图用一种妥协理论去代替马克思主义的意识论,把唯物论同唯心论调和起来。他们拿客观主义同主观主义的“联盟”的原则,去对抗辩证法的原则,而这种原则既非机械的客观主义,也非唯心的主观主义,而是客观和主观之具体历史的一致。

可是还有怪议论,这就是普列汉诺夫关于意识问题的物活论的见解,在他的“石子也是有意识的”一句名言中充分表现着。照他的意见,意识不是发生于物质发展过程中的,而是从最初就存在于一切物质的,石子及低级有机体的意识和人的意识之间,仅仅在于程度上的区别。这种反历史的见解,对于辩证法唯物论认为意识是最后发生的具备着质的特殊性的见解,也是根本相反的。

只有辩证唯物论的意识才是意识问题上的正确的理论。

九、反映抡

做一个彻底的唯物论者,单承认物质对于意识的根源性是不够的,还须承认意识对于物质的可认识性。

关于物质能否被认识的问题,是一个复杂的问题,是一切过去哲学都觉得无力对付的问题,只有辩证法唯物论能够给予正确的解决。在这个问题上,辩证法唯物论的立场即同不可知论相反,又同直率的实在论不同。

休谟同康德的不可知论,把认识的主体从客体隔离开来,认为越出主体的界限是不可能的,“自在之物”和它的形象之间存在着不可跳过的深沟。

马赫主义的直率实在论,则把客体同感觉等同起来,认为真理在感觉中就已经成就了完成的形态。同时,他们不但不了解感觉是外界作用的结果,而且不了解主体在认识过程中的积极作用,即外界作用在主体的感觉机关和思想的脑子中所做的改造工夫(取印象和概念的形式表现出来)。

只有辩证唯物论的反映论,肯定地答复了可认识性问题,成为马克思主义认识论的“灵魂”。根据这一理论,指明我们的印象和概念不但被客观事物所引起,而且还反映客观事物。指明印象和概念,既不象唯心论者所说的那样是主体自动发展的产物.也不是不可知论者所说的那样是客观事物的标准,而是客观事物的反映,照像和样本。

客观的真理是不依靠主体而独立存在的,它虽然反映在我们的感觉和概念中,但不是一下子就取完成的形态,而是一步一步完成的。认为客观真理在感觉中就已经取着完成形态而被我们获得的那种直率实在论的见解,是一种错误的见解。

客观真理在我们感觉和概念中,虽不是一次就取完成的形态,然而不是不能认识的。辩证唯物论的反映论反对不可知论的见解,认为意识是能够在认识过程中反映客观真理的。认识过程是一个复杂的过程,在这个过程中。当未被认识的“自在之物”,反映到我们的感觉印象概念上来时,就变成“为我之物”了。感觉和思维,并不是如同康德所说的那样把我们同外界隔离开来,而是把我们同外界联系起来的,感觉和思维就是客观外界的反映。思想的东西(印象和概念)并非别的,不过是“人类头脑中所转现出来和改造过来的物质的东西”(马克思)。在认识过程中,物质世界是愈走而愈接近地、愈精确地、愈多方面地和愈深动地反映在我们的认识中。向着马赫主义和康德主义作两条战线的斗争,揭破直率实在论和不可知论的错误,是马克思主义认识论的任务。

唯物辩证法的反映论认为我们认识客观世界的能力是无限度的,这和不可知论者认为人的认识能力是有限度的那种见解根本相反。但我们之接近绝对真理,却每一次有其历史上的确定界限。列宁这样说:吾人知识之接近客观的绝对真理,是历史地有限度的.但是这一真理的存在是绝对的,我们不断地向真理接近也是绝对的。图画的外形是历史地有条件的,但这张图画描绘着客观上存在的模型则是绝对的。我们承认人的认识受历史条件的限制,真理是不能一次获得的。但我们不是不可知论者,我们又承认真理能够完成于人类认识的历史运动中。列宁还说:对于自然在人类思想中的反映,不要死板板地或绝对去了解它。认识不是无这动的与无矛盾的,认识是处于永久的这动过程中。即矛盾之发生和解决的永久过程中。认识运动是一个复杂的充满着矛盾与斗争的运动,运就是辩证唯物论的认识之见解。

一切哲学在认识论上的反历史观点,都不把认识当作过程看待,因此都带着狭隘性。感觉主义的经验论之狭隘性,在感觉和概念之间挖开了深沟。理性主义学派的狭隘性,则使概念脱离了感觉。只有把认识当作过程看待的辩证唯物论的认识论(反映论),才彻底除去了这样狭隘性,把认识放在唯物的与辩证的地位。

反映论指出:反映过程不限于感觉印象,也存在于思维中(抽象的概念中),认识是一个由感觉到思维的运动过程。列宁曾说:“反映自然的认识,不是简单的、直接的、整体的反映,而是许多抽象的思考、概念,法则等等之形成过程。”

同时列宁还指出:由感觉到思维的认识过程,是飞跃式地进行的,在这一点上,列宁精确地阐明了认识中的经验元素和理性元素相互关系之辩证唯物论的见解。许多哲学家都不了解认识的运动过程中即从感觉到思维(从印象到概念)的运动过程中所发生的突变。因此,理解这一由矛盾而产生的飞跃式的转变,即理解感觉和思维的一致为辩证的一致,便是理解了列宁反映论的本质之最重要的元素。

十、真理论

真理是客观的,相对的,又是绝对的,这就是唯物辩证法的真理论。

真理首先是客观的。在承认了物质的客观实在性及物质对于意识的根源性之后,就等于承认了真理的客观性。所谓客观真理,就是说,客观存在的物质世界,是我们的知识或概念的内容之唯一来源,再也没有别的来源。只有唯心论者否认物质世界离人的意识而独立存在一一这一唯物论的基本原则,才主张知识或概念是主观自生的,不要任何客观的内容,因而承认主观真理,否认客观真理。然而这是不合事实的。任何一种知识或一个概念,如果它不是反映客观世界的规律性,它就不是科学的知识,不是客观真理,而是主观的自欺欺人的迷信或妄想。人类以改变环境为目的之一切实际行动,不管是生产行动也罢,阶级斗争或民族斗争的行动也罢,其他任何一种行动也罢,都要受着思想(知识)的指挥的,这种思想如果不适合于客观的规律性,即客观规律性没有反映到行动的人的脑子里去,没有构成他的思想或知识的内容,那末这种行动是一定不能达到目的的。革命运动中所谓主观指导犯错误,就是指的这种情形。马克思主义所以成为革命的科学知识,就是因为它正确地反映了客观世界的实际规律,它是客观的真理。一切反马克思主义的思想所以都是错的东西,就是因为他们不根据于正确的客观规律,完全是主观的妄想。有人说,一般公认的就是客观真理(主观唯心论者波格达诺夫就这样说)。照这种意见,那末,宗教和偏见也是客观真理了,因为宗教和偏见虽然实质上是谬见,可是却常常为多数人所公认,有时正确的科学的思想反不及这些谬见的普及。唯物辩证法根本反对这种意见,认为只有正确地反映客观规律性的科学知识,才能被称为真理,一切真理必须是客观的。真理与谬说是绝对对立的,判断一切知识是否为真理,唯一的看它们是否反映客观的规律。如果不合乎客观规律,尽管是一般人都承认的,或革命运动中某些说得天花乱坠的理论,都只能把它当作谬说看待。

唯物辩证法真理论的第一个问题,是主观真理和客观真理的问题,它的答复是否认前者而承认后者。唯物辩证法真理论的第二个问题,是绝对真理和相对真理的问题,它的答复不是片面地承认或否认某一方面,而是同时承认它们,并指出它们正确的相互关系,即指出它们的辩证法。

唯物辩证法在认识客观真理时,就是承认了绝对真理的。因为当我们说知识的内容是客观世界的反映时,这就等于承认了我们知识的对象是那个永久的绝对世界。“关于自然之一切真理的认识,就是永久的无穷的认识,因此它实质上是绝对的”(恩格斯)。然而客观的绝对的真理不是一下子全部成为我们的知识,而是在我们认识之无穷的发展过程中,经过无数相对真理的介绍,而到达于绝对的真理,这无数相对真理之总和,就是绝对真理的表现。人类的思维,就它的本性说,能给我们以绝对真理,绝对真理乃由许多相对真理积集而成,科学发展的每一阶段,增加新的种子到这个绝对真理的总和中去。但是每一科学原理的真理界限却总是相对的。绝对真理仅能表现在无数相对真理之上,如果不经过相对真理的表现,绝对真理就无从认识。唯物辩证法并不否认一切知识之相对性,但这只是指吾人知识接近于客观绝对真理的限度之历史条件性而言,并不是说知识本身只是相对的。一切科学上的发明,都是历史地有限度的和相对的,但是科学知识和谬说不同,它显示着描画着客观的绝对的真理,这就是绝对真理与相对真理相互关系之辩证法的见解。

有两种见解,一种是形而上学的唯物论,另一种是唯心论的相对论,对于绝对真理与绝对真理之相互关系问题都是不正确的。

形而上学的唯物论者,根据于它们“物质世界无变化”的形而上学的基本原则,认为人类思维也是不变化的,即认为在人的意识中这一成不变的客观世界,是一下子整个被摄取了。这就是说,他们承认绝对真理,而这个绝对真理是一次被人获得的。他们把真理看成不动的、死的、不发展的东西。他们的错误不在于他们承认有绝对真理一一承认这一点是正确的,而在于他们不了解真理的历史性,不把真理的获得看成一个认识的过程,不了解所谓绝对真理者,只能在人类认识的发展过程中一步一步地开发出来,而每一步向前的认识,都表现着绝对真理的内容,但对于全部真理说来,它具有相对的意义,并不能一下子获得绝对真理的全部。形而上学的唯物论关于真理的见解,表现了认识论的一个极端。

认识论中关于真理问题的再一个极端,就是唯心论的相对论。他们否认知识之绝对真理,只承认它的相对意义。他们认为一切科学的发明,都不包含绝对真理,因而也不是客观真理,真理只是主明的与相对的。既然这样,那末一切谬说都有存在的权利了,帝国主义侵略弱小民族,统治阶级剥削劳动群众,这些侵略主义与剥削制度也就是真理,因为真理横竖只是主观与相对的,否认客观真理与绝对真理的结果,必然到达这样的结论,并且唯心论的相对论,它们的目的本来就是为着要替统治阶级作辩护的,例如相对论的实用主义(或实验主义)之目的,就在于此。

这样看来,不论是形而上学的唯物论,或是唯心论的相对论,都不能正确解决绝对真理和相对真理的相互关系的问题。只有唯物辩证法,既给思维与存在相互关系问题以正确的解答,并且随之而来又确定了科学知识的客观性,再则,还同时给了绝对与相对真理以正确的理解。这就是唯物辩证法的真理论。

十一、实践论(略)

第三章唯物辩证法

前面简述了“唯心论与唯物论”及“辩证法唯物论”两个问题。关于辩证法问题,仅有概略的提到,现在来系统地讲这个问题。

马克思主义的世界观(或叫宇宙观),是辩证法唯物的唯物论,不是形而上学的唯物论(或叫机械的唯物论),这一点区别,是一个天翻地覆的大问题。世界是一个什么样子的?从古至今有三种主要答案。第一种是唯心论,(不管是形而上学的唯心论,或辩证法的唯心论)说世界是心造的,引申起来又可说是神造的。第二种是机械唯物论,否认世界是心的世界,说世界是物质的世界,但物质是不发展的,不变化的,第三种是马克思主义答案,推翻了前面两种,说世界不是心造的,也不是不发展的物质,而是发展的物质世界,这就是辩证法唯物论。马克思主义这样的看世界,把世界在从来人眼睛中的样子翻转了过来,这不是天翻地覆的大议论吗?这世界是发展的物质世界这种议论,在西洋古代的希腊就有人说过了,不过因为时代限制,还只简单地笼统地说了一说,叫做朴素的唯物论,没有(也不可拢有)科学的基础;然而议论是基本上正确的。黑格尔创造了辩证的唯心论,说世界是发展的,但是心造的,他是唯心发展论,其正确是发展论(即辩证法),其错误是唯心发展论。西洋十七、十八、十九三个世纪法德等国的资产阶级唯物论,则是机械观的唯物论,他们说世界是物质世界,这是对的,就是象机械一样的运动,只有增减或位置的变化,没有性质上的变化,这是很不对的。马克思继承了希腊朴素的辩证唯物论,改造了机械唯物论与辩证唯心论,创造了从古以来没有过的、放在科学基础之上的辩证唯物论,成为全世界无产阶级及一切被压迫人民的革命的武器。

唯物辩证法是马克思主义的科学方法论,是认识的方法,是论理的方法,然而它就是世界观。世界本来是发展的物质世界,这是世界观,拿了这样的世界观转过来去看世界,去研究世界上的问题,去指导革命,去做工作,去从事生产,去指挥作战,去议论人家长短,这就是方法论,此外并没有别的什么单独的方法论。所以在马克思主义者手里,世界观同方法论是一个东西,辩证法,认识论、论理学,也是一个东西。

我们要系统地来讲唯物辩证法,就要讲到唯物辩证法的许多问题,这就是它的许多范畴,许多规律,许多法则(这几个名词是一个意思)。

唯物辩证法究竟有些什么法则呢?这些法则中那些是根本法则,那些是附从于根本法则而又为唯物辩证法学说中不可缺少、不可不解决的方面、侧而或问题呢?所有这些法则为什么不是主观自造的,而是客观世界本来的法则呢?对于这些法则的学习、了解是为了什么呢?

这个完整的革命的唯物辩证法学说,创建于马克思与思格斯,列宁发展了这个学说,到了现在苏联社会主义胜利与世界革命时期,这个学说又走上了新的发展阶段,更加丰富了它的内容,这个学说中包含的范畴首先如下各项:

矛盾统一法则。

质量互变法则。

否定之否定法则。

以上是唯物辩证法的根本法则。除古代希腊的朴素唯物论曾经简单地无系统地指出了这些法则的某些意义,及黑格尔唯心地发展了这些法则外,都是被一切形而上哲学(所谓形而上哲学,就是反发展论的学说)所否定了的,直到马克思、恩格斯,才唯物地改造了黑格尔的这些法则成为马克思主义世界观与方法论之最基本的部分。

唯物辩证法所包含的范畴除了上述根本法则外,同这些根本法则联系着的,还有如下各范畴:

本质与现象。

形式与内容。

原因与结果。

根据与条件。

可能与现实。

偶然与自由。

链与环等等。

这些范畴,有些是从来形而上学及唯心辩证法所着重研究过的,有些是从来哲学片面地研究过的,有些则是马克思主义新提出的。这些范畴,在马克思主义的革命理论家与实践家手里,揭去了从来哲学唯心的及形而上学的外衣,克服其片面性,发现了它们的真实形态,并且随着时代的进步,极大地丰富了它们的内容,成为革命的科学方法论中重要的成份,拿这些范畴同上述根本的范畴合在一起,就形成一个完整的深刻的辩证唯物法的系统。

所有这些法则或范畴,都不是人的思想自己造出来的,而是客观世界本来的法则。一切唯心论都说精神造出物质,那末,在他们看来,哲学的法则、原则、规律或范畴,自然更是心造的了,发挥了辩证法系统的黑格尔,就是这样去看辩证法的。在他看来,辩证法不是从自然和社会的历史中抽取出来的法则,而是纯粹思想上的理论系统,人的思想造出了这一套系统之后,再把它们套到自然和社会上去。马克思、思格斯揭去黑格尔神秘的外衣,丢弃了他的唯心论,把辩证法放在唯物论的地位。思格斯说:“辩证法的法则是从自然和人类历史抽取出来的,但它们并非别的,就是这两个历史发展领域的最普遍的发展法则,就实质论,可以归纳为质量互变、矛盾统一、否定之否定这三个根本法则。”辩证法法则是客观世界的法则,同时也是主观思想里头的法则。因为人的思想里头的法则不是别的,就是客观世界的法则通过实践在人类头脑中的反映。辩证法,认识论、论理学是一个东西,前面已经讲过了。

我们学习辩证法是为了什么呢?不为别的,单单为了要改造这个世界,要改造这个世界上面人与人、人与物的老关系。这个世界上面的人类大多数都过着苦难日子,受着少数人所控制的各种政治经济制度的压迫。在我们中国这个地方生活着的人类,受着惨无人道的双重性制度的压迫一一民族压迫与社会压迫,我们必须改变这些老关系,争取民族解放与社会解放。

要达到改造中国与世界的目的,为什么要学习辩证法呢?因为辩证法是自然与社会的最普遍的发展法则,我们明白它,就得到了一种科学的武器,在改造自然与社会的革命实践中,就有了同这种实践相适应的理论同方法。唯物辩证法本身是一种科学(一种哲理的科学),它是一切科学的出发点,又是方法论。我们的革命实践本身也是一种科学,叫做社会科学或政治科学,如果不懂得辩证法,则我们的事情是办不好的,革命中间的错误无一不违反辩证法,但如果懂得了它,那就能生出绝大的效果。一切做对了的事,考究起来,都是合乎辩证法的,因此一切革命的同志们首先是干部,都应用心地研究辩证法。

有人说,许多人懂得实际的辩证法,而且也是实际的唯物论者,他们虽没有读过辩证法书,可是做起事来是做得对的,实际上合乎唯物辩证法,他们就没有特别研究辩证法的必要了。这种话是不对的。唯物辩证法是一种完备的深刻的科学,实际上具有唯物的与辩证的头脑之革命者,他们虽从实践中学到了许多辩证法,但是没有系统化,没有如同已经成就的唯物辩证法那样的完备性与深刻性,因此还不能洞察运动的远大前途,不能分析复杂的发展过程,不能捉住重要的政治关节,不能处理各方面的革命工作,因此仍有学习辩证法的必要。

又有人说,辩证法是深奥难懂的,一般人没有学会的可能,这话也是不对的。辩证法是自然、社会与思想的法则,任何有了一些社会经验(生产与阶级斗争的经验)的人,他就本来了解了一些辩证法;社会经验更多的人,他本来了解的辩证法就更多些,不过还处在零乱的常识状态,没有完备的深刻的了解。就将这种常识辩证法加以整理与深造,是并不困难的。辩证法之所以使人觉得困难,是因为没有善于讲解的辩证法书,中国许多辩证法书,不是错误就是写得不好,或不大好,使人人望而生畏。所谓善于讲解的书,在于以通俗的言语,讲亲切的经验,这种书将来总是要弄出来的。我这个讲义也是不好的,因为我自己还在开始研究辩证法,还没有可能写出一本好书,也许将来有此可能,我也有这个志愿,但要依研究的情形才能决定。

以下分述辩证法的几个法则。(略)

<p align="right">(一九三七年八月九日)</p>

