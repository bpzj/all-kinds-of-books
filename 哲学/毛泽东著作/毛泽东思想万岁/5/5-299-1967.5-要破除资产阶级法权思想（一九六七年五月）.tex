\section[要破除资产阶级法权思想(一九六七年五月)]{要破除资产阶级法权思想}
\datesubtitle{(一九六七年五月)}


多少派也是两大派,不是反革命就要做工作,慢慢观点就会越来越近的,不会越远的。

搞供给制,共产主义生活是马列主义作风,与资产阶级作风对立。我看还是农村作风,游击习气好。卅二年战争都打胜了,为什么建设共产主义就不行了呢?为什么要搞工资制?这是向资产阶级让步,是借农村作风和游击习气来贬低我们,结果发展了个人主义,讲说服不要压服也忘掉了。是不是由部队带头恢复供给制?

要破除资产阶级的法权思想。例如争地位,争级别,要加班费,智力劳动者工资高,体力劳动者工资少等等,都是资产阶级思想的残余。

各取所值,是法律规定的,这也是资产阶级的东西。将来坐汽车要不要分等级?不一定要有专车。对老年人、体弱的可以照顾一下,其余的就不要分等级了。我们的党是连续打了卅多年仗的党,长期实行供给制,从几十万人增加到几百万人,一直到解放。初期大体过平均主义的生活,工作都很努力,打仗都很勇敢,完全不是靠什么物质刺激,而是靠革命精神的鼓舞。

<p align="right">(根据聂荣臻同志的传达)</p>


