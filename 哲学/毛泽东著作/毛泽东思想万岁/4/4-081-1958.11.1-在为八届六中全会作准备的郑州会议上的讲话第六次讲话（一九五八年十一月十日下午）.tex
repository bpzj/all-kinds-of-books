\section[在为八届六中全会作准备的郑州会议上的讲话第六次讲话(一九五八年十一月十日下午)]{在为八届六中全会作准备的郑州会议上的讲话第六次讲话(一九五八年十一月十日下午)}
\datesubtitle{(一九五八年十一月十日)}


记录改为政治局决议。下面回去就实行。政治局补行法律手续。

报上尽是诗。大跃进有些把人搞得昏昏沉沉。“诗无达话”,是不对的。诗有诂,字句确凿。睡不着,想说一点。试图搬斯大林,继续对一些同志作说服工作。我是自以为正确。对立面如果正确,我服从。

一个是不要划分社会主义和共产主义界限的问题,一个是不要混淆两种所有制,即全民所有制和集体所有制的问题。

商品生产制度,不是为了利润,而是为了发展生产,为了农民,为了工农联盟。对于集体所有制采取资产阶级遗留下来的形式,现在仍然是农民问题。有些同志忽然把农民看得很高,以为农民是第一,工人是第二了。农民甚至比工人阶级还高,是老大哥了。农村有些走在前面。这是现象,不是本质。究竟鞍钢是大哥,徐水是大哥?有人认为中国无产阶级在农村。鞍钢八级工资制未成立人民公社,是落后了。有些同志在徐水跑了两天,就以为徐水是大哥了。好像农民是无产者,工人是小资产阶级,这样看,是不是马克思主义?有些同志读马克思教科书时是马克思主义,一到实际当中遇到实际问题他的马克思主义就要打折扣。这是一股风,这是要向几十万干部进行教育的问题。干部中有几十万,甚至几百万人,至于群众也有些昏昏沉沉,觉得好像快要上天了。于是你们谨慎小心,避开使用还有积极因素的资本主义的范畴——商品生产、商品流通、价值法则等来为社会主义服务。并以第三十六条为例,尽量用不明显的文字,来蒙混过关,以便显得农民进入共产主义了。这是对于马克思主义不彻底、不严肃的态度。这是关系到几亿农民的事。斯大林说不能剥夺农民,理由如下。(一)他的劳动力如同种子一样,是属于集体农庄和公社。这和鞍钢工人不同,他们是为全民生产。集体农庄和公社,不但有种子,还有肥料、产品。产品所有权在农民,不给他东西,不等价作买卖,他是不给你的。是轻率地还是谨慎地对待这个问题呢?斯大林死啃着大机器,看起来好像要啥拿啥。实际上心疼得很。修武县第一书记,不敢宣布全民所有制,一条是怕灾荒,减产了,发不了工资,国家不包,又不能补贴,二条是丰产了,怕国家拿去。这个同志是想事的,不冒失,不像徐水一样,急急忙忙往前闯。我们没有宣布土地国有,而是宣布土地、种子、牲畜、大小农具社有。因此这一段时间只有经过商品生产、商品交换的形式,才能引导农民发展生产,进入全民所有制。

《苏联社会主义经济问题》第一章第四页第二段说,自由是被认识了的必然。客观法则是独立于人们的意识之外的,客观法则和人们的主观认识是相对立的,认识客观法则才能去驾驭它。社会主义政治经济学的法则,要研究它的必然性。成都会议提出的意见,看来行之有效,八大二次会议作了报告,看来还灵,是不是合乎法则?是否就是这些?是否还会栽筋斗?还要继续在实践中得到考验。时间要几年,或要十几年,我曾对××××说过,你还要看十年,我们过去,革命是被人怀疑的,中国革命应不应该,应不应夺取政权,国际上某些人是坚决反对的,但革命这些年证明了路线是正确的,实践证明是对的,但还只是一个阶段的证明。合作社、公私合营,增产都是证明。增是增了,但建设八年,才搞了三千七百亿斤粮食,今年搞多一点,晓得明年如何,×××同志建议,今年十二月,明年一月、二月、三月

这四个月,第一书记还要抓一抓农业。这关系到夏收,请大家考虑。抓钢铁同时抓农业。第一书记召集一个会,把劳动力分一分,宣布一下,强迫命令一下,交给农业书记负责。省、地、县都要负责,搞不好不行,不然无人负责。各说各有理,大家都拖到钢铁方面去了。山西说:工业、农业、思想三胜利,这个口号是好的。搞掉一个就是铁拐李了。缺农业就成了斯大林了,搞农业的要死心塌地的搞农业。决议上再写一下,不要把农业丢掉了。第一书记要心挂两头、三头、四头,学会多面手,一个月搞一天,四个月搞四天,太少了,工业、农业、思想是成都会议提的,这是山西人的创造。如同苦战三年是河南人提的,现在变成全国的口号,搞试验田是湖北的口号,我们这些人头脑是不出任何东西的,无非是把各地的经验集中起来加以推广,作成如成都会议、北戴河会议那样一些产品。这次出了两个产品,一个立即实行,一个是初稿。

我们的措施是否完全符合于客观规律呢?大体符合就可以了。斯大林说。

“苏维埃政权当时必得在所谓‘空地上’创造新的社会主义的经济形式。这个任务无疑是困难而复杂的,是没有先例的。”母胎里只有思想,经济形式是后创的。我们是有先例的,有苏联成功和失败的经验。斯大林这本书极为有用,编了教科书,我未好好看,现在逼着我看,越看越有兴趣了。××、富春知道的多,去取过经。有了先例,我们应该比他们搞得更好一点。如果搞糟了,就证明中国的马克思主义者用处就不多。

该书第六页第二段,第七页第二段,说消灭创造法则是不对的,客观法则和政策法令不能混为一谈。有计划发展的法则是作为无政府状态的对立物而产生的,那里无政府,这里有政府。我们也搞过计划,也有经验,一个风潮煤太多了,又一个风潮,糖多了,一个风潮,钢铁多了,强迫卖给苏联,第二天又毁约,因为又不够了。上月说多,下月说少,心中烦闷不知如何是好。如像苍蝇在玻璃房里盲目地向玻璃乱碰。经过这些曲折,马鞍形的教训五六年小跃进,五七年跃退,经过比较,想出了一条路,叫作总路线。农业四十条也有曲折。一时说灵,一时说不灵,归根结底还是灵。已经基本完成了,总不能说它是错误的,新四十条是不是也灵?我也有怀疑,准备再谈一下。十五年按人口赶上英国是不是可以?苦战三年不发表,以免吓死人。那时吓帝国主义一下,也吓朋友。一千万,三千万,过几年吓惯了就不怕了。

客观法则,是不是总路线的一套,是反应得比较完全,还是没有搞好呢?搞钢就无煤。上海、武汉没有饭吃。对客观世界要逐步认识,不到时候没有展开矛盾,不反映到人们的头脑中来,不能认识。八年不晓得以钢为纲,今年九月才抓住了以钢为纲,抓住了主要矛盾的主要侧面。一元论不是多元论,抓住了主要矛盾就带动了一切。不是煤、铁、机平列,大中小以大为纲,中央、地方以中央为纲。和跳舞一样.男女跳舞女的闹独立性如何行?应该在服从中取得独立性,不服从男的就没有独立性.你不跳舞,文化贫弱,我和你没有共同语言。

“必须研究这个经济法则,必须掌握它,必须学会熟练地应用它,必须制定出能够完全反映这个法则的要求的计划。”斯大林这段话很好。我们还没有充分掌握,学习熟练地应用这个经济法则,不能说过去八年,我们是完全正确地进行计划生产的,是完全反映经济法则的要求的,当然成绩还是有的,是主要的,缺点、错误是第二位的。计划机关是什么?是中央委员会,大区、各省、各级都是计划机关,不光是计委。有可能计划好,但不能与现实混为一谈。要变成现实,就必须研究,必须学会熟练地应用它,制定出完全反映法则的计划。你(富春)要注意哟!小土群一吨铁十吨煤,是不是一个法则呢?洋的只要1:1.7,1:2。这就有法则问题。一吨钢要百分之二的铜铝。不能说八年计划完全反映法则,不能说今年一年都完全反映这个经济的法则要求。

斯大林谈到这里为止.这个问题没有展开,他研究到什么程度,我是怀疑的,为什么不两条腿走路呢?为什么重工业要那么多规章制度呢?问题是我们是否研究了,掌握了,熟练地应用了?至少是不充分。我们的计划如同他们一样,也是没有完全反映法则,因此要研究。

第二章,商品生产。现在,我们有些人大有消灭商品生产之势,有不少人向往共产主义。一提到商品生产就发愁,觉得这是资本主义的东西,没有区别社会主义与资本主义商品的本质差别.没有懂得利用其作用的重要性。这是不承认客观法则的表现,不承认五亿农民的问题。社会主义初期,应当利用商品生产来团结几亿农民。在社会主义建设时期,我以为有了公社,商品生产,商品交换更要发展,有计划地大大发展社会主义的商品生产,例如畜产品、大豆、黄麻、肠衣、香肠、果木、毛皮,现在云南火腿不好吃了。破除迷信又恢复迷信。有人倾向不要商业了,至少有几十万人想不要商业了,这个观点是错误的,这是违背客观法则的。把张德生的核桃拿来吃,一个钱不给,你是否干?不认识五亿农民,无产阶级对农民应采取什么态度,把七里营的棉花无代价地调出来行不行?要马上打破脑袋。产品在旧社会对人是有控制作用的。我们只占有生产资料和社会产品的一小部分。斯大林分析恩格斯的话是对的,我们的全民所有制是很小的一部分。只有把一切生产资料都占有了,社会商品十分丰富了,才能废除商业。我们的经济学家似乎没有懂得这一点。我是用斯大林这个死人来压活人。斯大林对英国革命成功后废除商品仍有保留,我看英国与加拿大合成一个同家才好办。(十页第二段)恐怕至少有一部分能废除,斯大林对这个问题并不武断,他没有作结论。

第十一页三段中说。有一些“可怜的马克思主义者”要剥夺农村中的小生产者,我们也有这种人。有些同志急于要宣布全国所有。不要说剥夺小生产者,只要说废除商业,实行调拨,那就是剥夺,就会使台湾高兴。我们五四年犯过错娱,征购太多,要搞九百三十亿斤粮食.全体农民反对我们,人人说粮食,户户谈统购,这也是“可怜的马克思主义者”,因为不知道农民手里有多少粮食。曾经有过这种经验,犯过这种错误,后来,我们减下来了,决定搞八百三十亿斤粮食。第一个反对的是章乃器,可见资产阶级唯恐我们天下不乱。总之,这个规律我们过去没有摸到。中国农民有劳动所有权,土地、生产资料(种子、工具、水利工程、林木、肥等)所有权,因此有产品所有权.不知道什么道理,我们的哲学家、经济学家显然把这些问题忘记了。我们还有脱离农民的危险。所以,“三三制”,三分之一上缴(包括县内调济),十年之内做到可能是要的。你(谭)要留八分之二,农民要打架的,你也打不赢。中国工人穷惯了,工作起来很努力,又有干部下放,打成一片,干十二小时,叫他们回去都不走。农民炼铁炼钢以后说,工人老大哥可不简单。农民太穷,工资太少,现在拿的多了不好。每人(按全部人口)平均五元是少数。

第九页关于商品生产命运问题。列宁的回答:“夺取政权”,没收工业我们办到了。公社比苏联进了一步。发展工业,加强农庄,我们也在作。不要剥夺农庄。公社办工业比斯大林胆大。会不会搞资本主义?不会,因为有政权,依靠贫农、下中农,有党,有县委,有成千成万的党员,过去想过赚钱的工业要乡政府搞,不要合作社搞,这有点斯大林主义残余。政社合一,放在区委管理之下办工业,到处发展,遍地开花,这样,不是钱少了,而是多了,李先念也想通了。现在公社太穷。管饭之外,发工资很少。有的只发几角钱,吃也是很穷,在水平以下,还是一穷二白。我说这是好事。老根据地就不起劲,

“不想前,不想后,只想高级化前土改后”,那时是黄金时代。“革命到了头。革命革不到头,革命革到自己的头”,这是山西的话。不断革命没搞,停顿了。我说要动中农,“和尚动得,我动不得?”列宁说:在一定时期保持商品生产(通过买卖交换)。这是唯一可按受的。要全力展开贸易。这段话,我们曾大吹大擂,这是社会主义性质的贸易。斯大林说是唯一适当的道路。我看是对的。只能贸易,不能剥夺,五四年.我们还是购买,不是调拨搞多了,农民还反对。

列宁的五条,我们都作了,并且建立了人民公社。以全力发展工业、农业和商业。问题还是一个农民问题,必须谨慎小心,一九五六年的错误的根源是没有看到农民问题。现在又是不懂得农民问题,农民的冲天干劲一来,又容易把农民当工人看,以为农民比工人还高,这是从右到“左”的转化。

第十页第十段“不能把商品生产与资本主义混为一谈”,为什么怕商品,无非是怕资本主义。现在是国家与人民公社作生意,早已排除资本主义,怕商品做什么?不要怕.我看要大大发展。中国是商品生产最不发达的国家,比印度、巴西落后。印度的铁路和纺织比中国发达。我国有没有资本主义剥削工人,没有。为什么怕?不能孤立地看商品生产,斯大林完全正确(第十三页)。商品生产要看它与什么经济相联系。商品与资本主义相联系就出资本主义,和社会主义联系就不是资本主义。就出社会主义。商品生产从古就有,商朝就是作生意的意思。把纣王、秦始皇、曹操看作坏人是完全错误的,纣王伐徐州之夷,打了胜仗。捉了干部,俘虏太多,血流漂杵(旗杆)”。孟子说,尽信书不如无书。说不相信有血流漂杵之事。奴隶时代并没有引导到资本主义。封建时期已经形成了资本主义,这一点斯大林说得不正确,资本主义母胎中已经孕育了无产阶级和马克思主义,社会主义的意识形态(已经产生)。

一九四九年二中全会就说限制资本主义,不是漫天限制。五○年开始让他们扩展六年之久,但又加工订货,到五六年公私合营,实际上空手过来了。“决定性的经济条件”。我们完全有了。试问为什么会引导到资本主义?这句话很重要。已经把鬼吃了,还怕鬼。不要怕,不会引导到资本主义,因为已经没有了资本主义的经济基础。商品生产可以乖乖地为社会主义服务,把五亿农民引导到全民所有制。商品生产是不是所有制的工具?为了五亿农民,充分利用这一工具发展社会主义生产。应当肯定。要把这个问题提到干部中讨论。

劳动、土地、其他生产资料统统是农民的,是人民公社所有的,因此商品也是公社所有的。只顾商品换商品,此外的联系,农庄都不接受,我们不要以为中国农民特别进步。修武县委书记的设想是完全正确的。商品流通的必要性是共产主义者要考虑的。必须在产品充分发展之后,才可能使商品流通趋于消失。同志们,我们才九年就急于不要商品,只有当中央组织有权支配一切产品的时候,才可能使商品经济因无必要而消失。吴××同志也要同陈伯达同志搞在一起,马克思主义太多了,不要急于在四年搞成。不要以为四年之后农民就会和郑州工人一样。游击战争用了二十二年,搞社会主义建设没有耐心如何行?没有耐心不行。我们曾经耐心等待胜利。对台湾也是如此。争取台湾一部分中下级和上级分裂不是没有可能的。杜蒋在一起好,还是争取一部分到我们这边好。我们谨慎小心,蒋也谨慎小心。对美国老是警告,说明我们是受气,许多人对我们警告不了解,我看警告三千六百次。现在美国不搞了,可见很灵。

只要存在两种所有制,商品生产就极其必要,极其有用。你们来驳斯大林?两种所有制如何过渡的问题,斯大林自己没有解决。他很聪明,说要单独讨论需要很多篇幅。

只要把“我国”改成“中国”,就有兴趣了。“只限于个人消费品”,行不通。还有农业工具、手工业工具也是商品,是否会导致资本主义?不会,赫鲁晓夫不是把机器卖给农庄了吗?历来就有商品生产。现在加一种社会主义商品生产。


