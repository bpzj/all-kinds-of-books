\section[在延安党校大礼堂关于国共谈判报告(一九四五年十月十七日)]{在延安党校大礼堂关于国共谈判报告}
\datesubtitle{(一九四五年十月十七日)}


同志们!今天我讲一些关于时局的情况。首先来谈一下,这一次重庆的谈判,谈了几十天,谈的结果如何,如大家所知道的,在报上发表了。现在我们的代表还在继续谈。

这次谈判,是在时局发生了新的情况下进行的。在抗战时期,两党谈了多少次,都是谈不拢的。这一次,一部分谈拢了,还有一部分没有谈拢,将来会求得一个妥协的,但要经过一个时期。

谈判分作两部分,谈拢的和没有谈拢的。这次谈拢的是过去不能解决的,这次解决了的仍然还只是纸上的。大家知道纸上的东西,不等于实际上解决的东西。要使它变为实际的东西,还要经过很大的努力。

但是谈判是有收获的,在谈判中间,我们延安和重庆,都有这样一种意见,说谈判毫无希望,中国没有和平妥协的希望,只有内战,在外国人中也有同样的怀疑。但谈判的结果证明并非毫无希望。还有一种想法,认为一谈就可以谈拢,样样都能依照我们的意见解决,认为谈判完全是乐观的。这种人在延安和大后方都有。这两种想法都不恰当。事实证明,而且将来还要证明。关于和平方针、民主权利这两个问题。现在和平方针是确定了,坚决避免内战,国共两党和平合作建设新中国。但实际上在广东、浙江、江苏、安徽、湖北、河南、山东、山西……一切有解放区的地方都在打仗,或者在准备打仗。国民党开了很多军队来打我们,我们也不能等着挨打。

这次公报上头一条就是和平,那末写在纸上的和事实是不是矛盾?是不是冲突?是个矛盾,所以要把纸上的东西变成实际的还要靠我们努力,还要给进攻我们解放区的军队来一个打击。为什么解放区的问题没有得到解决呢?因为国民党想占领大城市,想大大打击我们一下,最好消灭了我们,既使不能消灭,也要使我们更不利,他们的地位更有利一点。他们这样想,也在这样做。这是他们一方面的想法,和全中国全世界人民的想法要矛盾的。现在和平写在公报上,但事实上和平还没有实现,将来的前途发展怎么样,现在仗打的相当大,比如山西打了一个大仗,阎锡山的十二个师进攻我们,我们是不是针锋相对,寸土必争?我们“对”了“争”了,对也对的好,争也争的好,就是说把它的十二个师全部消灭了。他们调集三万八千人,我们出发了三万一千人,他们三万八千被消灭了三万五千,跑掉了两千,一千人散掉了。这样的事情还会有一个时期,他们拼命的争,这个问题好像不好解决。他们为什么要这样争呢?在我们手里不是很好吗?这是我们的想法,人民的想法,他们不是这样想法,要是他们也同样想法,那就统一了,都是同志了,都可以到党校大礼堂开会了。可见事实上世界的人并不都是同志,他们坚决反对我们,他们想不开,我们占了的地方社他们抢去,我们也想不开。两个想不开合在一块就要打仗。所以,它来争是很自然的,我们反击也是很自然的。既然是两个想不开,为什么又谈判?又发表公报长期合作?世界上的事是很复杂的,是由各方面因素决定的,所以要从各方面来想,不只能从单方面去想,我们的脑筋也锻炼的很复杂了。在重庆有人说蒋介石是靠不住的,是欺骗的,要谈好也不可能,我碰到许多人都这样说过,其中有国民党党员。我说你们说的是有理由的,是有根据的,积十八年之经验,除知有这一回事,就是说你们有理由、有根据,不是乱说的,国共一定谈不好,一定破裂,一定打内战。但这只是说中了一方面,还有一个方面还有许多因素,有三种因素:解放区大后方的人民,国际形势,共产党领导下的解放区的人民力量和军队。大后方的广大人民、新收复区的人民,他们都愿意和平,需要民主。没有到过重庆的人没有这种感觉。这次我到了重庆,才深深感到大后方广大的人民,广大的朋友支持我们,不满意国民党政府。经济恐慌,使许多资本家不满,他们亲自找我谈话,请我吃饭。有些大学教授,其中有中央大学教授,不愿意在那里教书,我劝他们继续教书,他们不耐烦。有些文化界戏剧界电影界也不愿在那里演戏,我劝他们继续工作,他们不愿意,有的还挥下泪来,他们希望我这次去了,天下事就差不多了,什么都解决了,我说我这是来受考试的,我受谁考试?受你们考试,受中国人民的考试,我恐怕这次交不了卷,能交卷就成了。我说中国的事不是由我一个人来办,一方面来办就好了。如果都像解放区一样,那你们愿意到那里教书,就到那里教书,愿到那里做事,就到那里做事,那是真正的自由。但是今天我是来作客的,是被人家请来重庆谈判的。外国人也是这样,美国政府的政策,很多是反动的,他们打日本是作了好事,他们的政策是专门扶助蒋介石反对人民的。我这样说,你们不要以为外国人都是这样子。我看到许多外国人,比如有些新闻记者、空军战士、经济工作者、大使馆的一些人都对我们表示同情,很热心。就是说广大的外国人,不满意中国的反动势力,愿意看到中国人民的力量。在这样的外国中间,有一个外国叫做苏联,苏联和其他国家不同,它和中国发生了法律关系,就是订立了中苏条约。

中苏条约究竟对中国有好处没有?应该说是有很大好处,对中国人民是完全有利的,国民党和美国政府中,开始有人高兴,以为中苏条约孤立了共产党,有利于国民党。但是经过了一个多月,他们的看法改变了,国民党中有一个将军说:中苏条约是给我们打了一针,打的不是补血针,而是疟疾针,使他们浑身忽冷忽热了一阵。说这话的不是我们,而是国民党员。关于这个今天我们不加评论。在今年十月革命节,每个解放区和我们延安要开庆祝会,以庆祝中苏条约为中心,要求坚决彻底执行中苏条约,中苏条约规定了中国不能参加任何反苏集团,签订中苏条约的不是别人而正是蒋介石,他是合法政府,他签字最好,因为一切的反苏言论都是从他这个反动集团发出的,签订了中苏条约后,像一九二七年以后那样再在国内大规模的反共打内战是不可能了,这是大势所趋,在中苏条约下的形势已经改变了,就不是一方面的问题了。他要坚持独裁,消灭共产党,这是他的主观愿望,在客观上是行不通的。到了这时候就只好讲现实主义,现在全中国人民,解放区人民和同盟国都不要打内战,如果中国发生了内战,就会牵动全世界。我们的党也不是一九二七年的党了,今天我们的地位提高了。他们从来不肯承认共产党的平等地位,现在也承认了,这在公报的文字中也可以看出来了。共产党第一次取得平等合法地位。解放区和共产党的工作已影响到全中国全世界了。在这样情况下,人家讲实现主义请我们去,我们也讲现实主义去谈判。我二十八号到重庆,二十八号晚上对王世杰讲,和平团结,“九一八”事变以后后就产生了这种需要,我们要求了,没有实现,到西安事变后,“七七”抗战前才实现,我和他们讲,和蒋介石也讲。抗战八年主要是打日本,但内战也有一些,但内战不是主要的,打日本是主要的。尽管打法不同,有政治纠纷,有磨擦,要说没有内战是欺骗是不合实际的。但打日本是一致的。八年中我们一再提出,只要政府政策有所转变,我们是愿意谈判的。

“七大”我们也讲过了是“洗脸”政策,我们洗人家的脸,人家不洗,还是洗不成。现在人家说洗,那么就洗吧。这次公报上发表的和平方针和若干民主的协议,一方面写在纸上,要实现还要经过努力。但另一方面也不能看做全部是假的,这是由各方面力量决定的,解放区人民和大后方人民的力量和国际形势大势所趋,使得蒋介石也不能不讲现实主义。

这就是我和同志们讲的形势问题,大家所关心的有许多矛盾现象,为什么有些问题解决了,有些问题没有解决?为什么公报上宣布了和平团结,而同时又在打仗,这些矛盾,同志们想不开,蒋介石历年反共,我们为什么又希望和他谈判呢?我们“七大”决定,只要政府政策有所改变,我们是愿意和他谈判的。这是不是对呢?我们刚才讲的就是答复这些问题。

有些同志说为什么要让出八个解放区?让出八个地方非常可惜。让出来好,为什么非常可惜。因为这是人民用血汗创造出来的,艰苦的建设起来的解放区。让出的地方必定要对当地人民很好的解释清楚,要很好的安置,但是让出来为好,为什么?因为现在全中国的所有宣传机关,除解放区以外,都控制在国民党手里,他们就是一个谣言制造厂。这一次谈判,他们造谣说共产党就是要地盘,要军队,不肯让步,我们的方针是保护人民的基本利益,在这样原则之下,容许作一些让步,过去有些比现在还大。应当知道这次让步并不是第一次,我们过去成立过中央苏区,你们知道吗7但是在抗战前夕(一九三七年春夏)为争取国民党抗战,我们自动的取消了苏区。我们实行过土地革命,但为了团结地主,以使他们安心打日本,改为减租减息。这一次让步不另起炉灶成立中央政府,不实行耕者有其田,还是减租减息。地区方面,我们在南方让出了若干地区,在全国人民面前使他们看清楚了共产党的让步,他们的谣言就被击破了。军队方面也是一样的,我们先提出了四十八个师,国民党是二百六十三个师,我们占他六分之一,国民党宣传说,我们就是要枪杆子,我们就减到四十三个师,他们仍为二百六十三个师,他们说他们要缩编为一百二十个师,一百个师,九十个师,我们说照比例减下来,编二十个师,还是六分之一。国民党的军队官多兵少,一个师六千人还不到,照他们这样编法,那么是不是我们要将枪杆子交给他们呢?那也不是的,交给他们,他们岂不是又多了么?人民武装一枪一弹都要保存,不能交出去。这样他们就无话可说了,一切谣言都取消了。剩下的解放区,你承认不承认?关于承认解放区的问题,我们会提出五次之多,将来这个皮还是有得扯。寸土必争是我们的方针,为什么现在寸土都让了?寸土必争那是一个口号,人家有一张床摆在那里,你要站在旁边,人家就不能安心睡觉,人家要回南京,浙江地方都非请我们走不可,无论如何他也要争,我们不让也得让,争也争不到,何必不慷慨一点让出来呢,有些地方还得快点走,广东是走不了啦,我们王震将军就走得很辛苦,打了两脚水泡,到广东刚接上头的时候,日本人就投降了,我们王震又要回来。回来就回来吧,他们走得很辛苦,从延安走到广东回到湖北,国民党军队跟着他们屁股打,有欢迎的也有欢送的。但他们没有整住我们,我们在那里大闹天宫,如入无入之境,还是撤过了长江,苦是苦,还是一个胜利。这次谈判,湖南解放区也作了让步。又如浙东,我们在撤退中顾祝同拦腰打我们,但我们倒还是撤到上海附近了,没有撤到长江以北。这一次算点账没有亏本,寸土必争,没有吃亏,这个地方失了,那个地方得了,失了一寸,得了一尺,还赚九寸。

针锋相对要看形势,人家请我们不去就是“相对”了吗?有时不去是针锋相对,有时候去也是针锋相对。过去不去是对的,这次去也是对的,也是针锋相对。他们连来三次电报请我们,我去了,他们毫无准备,一切提案都由我们提出,击破了国民党说我们不要和平不要团结的谣言。从现在起我们要抓紧协定,要国民党实现,继续要求和平,不达到和平不止,如果他们要打,把他们彻底消灭,就有了和平,如在上党地区,太行太岳中条山中间有一个脚盆,在那个脚盆里有鱼有肉,阎锡山反动派去抢了,我们把他们消灭了,他们就舒服了,消灭了一点舒服一点,消灭的多舒服的多,彻底消灭,彻底舒服。中国的问题是复杂的,我们的脑子也要复杂一点,人家来了我们就打,打是打,和平是打不破的,因为大势所趋不会错的。

第三,我讲一点工作问题,有许多人要往前方去,在座的有些人要留在延安学习、工作,我对这两部分同志要提供一点意见。走的满怀热心争着要去。现在要说去呀,我们就开欢送会,这种欢送会可不是国民党欢送王震一样。这样一种积极性和热情是可贵的,可是我今天要泼一点冷水。我们有些人不是想到那里有许多困难要去解决,而是认为那里一切都是顺便的,比延安舒服。他们的意见好像要到那里去享福。有没有这种想法呢?我看是有的,有的同志要求到东北去养病,不是为工作去。据我看全中国养病还是延安好,重庆我也去过,那里也不好。我劝同志们去那里是为了解决困难,去做工作。什么叫工作呢?就是斗争。那里有困难、有问题要我们去解决,要以解决困难的精神到那里去,越困难的地方越要去,就是好同志。这些工作是艰苦的,艰苦的工作就像挑担子,看我们敢不敢挑它,一百斤和八十斤的两付担子摆在我们面前,要挑那一个?(当然我所说的是力量差不多,十四岁的娃娃不能做比)就有人想担八十斤的,却有些不好意思,就在旁边问了“天气好不好”,但另外一些同志却不问天气如何,只问那个担子重,有人告诉他这个担子重,他就挑起一百斤这个重担子,这是好同志,这种精神我们应该学习。

有些本地干部要离乡背井。我们在精神上要作准备到东北去,到那里去生根开花结果。最近解放日报有一篇文章叫做“在本地工作中生根开花”(见8.28解放日报)这篇文章写得很好,值得广播,有许多南方同志参加八路军在延安生根开花,现在又要到东北去生根开花结果了,大地就是人民,我们少数去那里就是种子,种到人民中去,生根开花要和人民结合起来,并和他们搞好团结,团结的人民越多越好。这就是“七大”的路线,每到一个地方,凡是挑担子有八十斤一百斤的问题发生,选重的挑,吃东西自己少吃一馒头,叫人家多吃一个馒头,这是最好的同志,是最好的共产主义者,值得大家钦佩。

对于留下的同志也说几句,听说现在留人需要说服人,他们会说,人家吃饭,为什么叫我一个人读书?现在人家都到东北,为什么要把我留在延安?当然到前方去是积极的,是好的,是革命的要求,但是延安不能统统走光,假如要做个决议,把中央、西北局、边区政府都取消好不好呢?要付表决,我看你们都不会举手。中央、西北局、边区政府都还要人,不能走光,这怎么办呢?我说去的就去,留的就留。有人说天下大势已定,中央要搬家,我何不先走一步呢?我说天下大势未定,现在决定不搬,将来再走,中央现在不走。什么时候走呢?等到搬家比不搬好的时候再搬,不然就不搬,现在是不搬为好,要搬也容易。“前边乌龟爬烂路,后面乌龟追路爬&quot;,他们几万人都走过了,脚在我们身上,要走就走,不是很容易吗?中央、西北局、边区政府,各个机关都是需要人工作的,我劝留下来的还是安心下。

中国革命是一个长期过程,胜利是逐渐的,一口咬个全中国是办不到的,谈判结果如何要靠我们在斗争中努力如何来决定。半年左右时间之内时局还是动荡不安,要斗争,努力争取基本上有利于全国人民的局势。前途是光明的(不管有多少困难),道路是曲折的。“七大”设想过很多困难,我们宁愿把困难设想更多一点,在党内有些同志是不愿多想困难,困难是事实,有多少承认多少,我们共产党是现实主义者,对困难不采取不承认主义。对汉奸王克敏等,我们是采取不承认主义,抓住他们要杀头,可是困难不能不承认我们承认困难,分析困难,要和困难作斗争。我希望去的同志,留的同志,都要这样想,不要片面的看问题。

总的形势,第二次世界大战后,世界前途是光明的,伦敦外长会议失败了,是不是要打第三次世界大战?不会的,打不起来的,还是会妥协的。因为妥协有好处,过去是为了打法西斯,现在为着和平。为什么社会主义要和资本主义妥协呢?这一点我不必多讲,第三次世界大战如果打起来,那就是反人民反延安、反南斯拉夫、反苏反共的战争,全世界无产阶级,和人民,都是要坚决反对的,使战争打不起来。我在重庆给许多人讲,凡是年龄在四十岁左右的人,都经过两次世界大战,第一次一九一四年八月打起,到一九一八年打完,第二次一九三九年(?)或一九四一年打起,到一九四五年八月打完,中间休息了二十八年。在三十年内打过两次世界大战,以前死的人和没有生下来的人都没有这样福气。

人类五十万年历史,只有这三十年打了两次世界大战。为什么三十年内打了两次世界大战?因为地球缩小的两个人碰头,资产阶级和无产阶级。资产阶级和无产阶级相互碰头,碰一到了就要挤要磨擦,碰的大就打大仗,第一次世界大战后二十几年间世界进步了多少?这一次大战后世界进步更要快。第一次世界大战后产生了苏联,全世界无产阶级产生了几十个共产党,这是从前没有过的。第二次大战后,苏联会成为什么样子?中国会变成什么样子?欧洲会变会什么样子?美国会变成什么样子,美国的无产阶级和人民的阶级觉悟政治觉悟大大的提高了,全世界力量更加团结了。第三次世界大战是可能反掉的,反动势力是可以反对下去的,全世界是进步的光明的,应当常常向人民宣传。但是我们还要告诉人民,告诉同志们,道路是曲折的,世界上没有直路,要准备走曲折的路,不要贪便宜,不要设想一个早上,一切反动派会自己跪在地下,我们说世界是光明的,道路是曲折的,全国人民全世界人民共同努力一定可以克服困难达到最后胜利。

