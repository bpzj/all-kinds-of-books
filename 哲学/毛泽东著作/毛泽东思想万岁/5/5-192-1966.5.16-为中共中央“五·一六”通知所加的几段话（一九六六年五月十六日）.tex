\section[为中共中央“五·一六”通知所加的几段话(一九六六年五月十六日)]{为中共中央“五·一六”通知所加的几段话}
\datesubtitle{(一九六六年五月十六日)}


撤销原来的“文化革命五人小组”及其办事机构,重新设立文化革命小组,隶属于政治局常委之下。

(……这场大斗争的目的是对吴晗〕及其他一大批反党反社会主义的资产阶级代表人物(中央和中央各机关、各省、市、自治区,都有这样一批资产阶级代表人物)的批判。

无产阶级对资产阶级斗争,无产阶级对资产阶级专政,无产阶级在上层建筑其中包括在各个文化领域的专政,无产阶级继续清除资产阶级钻在共产党内打着红旗反红旗的代表人物等等,在这些基本问题上,难道能够允许有什么平等吗?几十年以来的老的社会民主党和十几年以来的现代修正主义,从来就不允许无产阶级同资产阶级有什么平等。他们根本否认几千年的人类历史是阶级斗争史,根本否认无产阶级对资产阶级的阶级斗争,根本否认无产阶级的革命和对资产阶级的专政。相反,他们是资产阶级、帝国主义的忠实走狗,同资产阶级、帝国主义一道,坚持资产阶级压迫、剥削无产阶级的思想体系和资本主义的社会制度,反对马克思列宁主义的思想体系和社会主义的社会制度。他们是一群反共、反人民的反革命分子,他们同我们的斗争是你死我活的斗争,丝毫谈不到什么平等。因此,我们对他们的斗争也只能是一场你死我活的斗争,我们对他们的关系绝对不是什么平等的关系,而是一个阶级压迫另一个阶级的关系,即无产阶级对资产阶级实行独裁或专政的关系,而不能是什么别的关系,例如所谓平等关系、被剥削阶级同剥削的阶级的和平共处关系、仁义道德关系等等。

不破不立。破,就是批判,就是革命。破,就要讲道理,讲道理就是立,破字当头,立也就在其中了。

其实,那些支持资产阶级学阀的党内走资本主义道路的当权派,那些钻进党内保护资产阶级学阀的资产阶级代表人物,才是不读书、不看报、不接触群众、什么学问也没有、专靠“武断和以势压人”、窃取党的名义的大党阀。

……或者虽然已经开始了斗争,但是绝大多数党委对于这场伟大斗争的领导还很不理解,很不认真,很不得力的时候,……

他们对于一切牛鬼蛇神却放手让其出笼,多年来塞满了我们的报纸、广播、刊物、书籍、教科书、讲演、文艺作品、电影、戏剧、曲艺、美术、音乐、舞蹈等等,从不提倡要受无产阶级的领导,从来也不要批准。这一对比,就可以看出,提纲的作者们究竟处在一种什么地位了。

……高举无产阶级文化革命的大旗,彻底揭露那批反党反社会主义的所谓“学术权威”的资产阶级反动立场,彻底批判学术界、教育界、新闻界、文艺界、出版界的资产阶级反动思想,夺取在这些文化领域中的领导权。而要做到这一点,必须同时批判混进党里、政府里、军队里和文化领域的各界里的资产阶级代表人物、清洗这些人,有些则要调动他们的职务。尤其不能信用这些人去做领导文化革命的工作,而过去和现在确有很多人是在做这种工作,这是异常危险的。

混进党里、政府里、军队里和各种文化界的资产阶级代表人物,是一批反革命的修正主义分子,一旦时机成熟,他们就会要夺取政权,由无产阶级专政变为资产阶级专政。这些人物,有些已被我们识破了,有些则还没有被识破,有些正在受到我们信用,被培养为我们的接班人,例如赫鲁晓夫那样的人物,他们现正睡在我们的身旁,各级党委必须充分注意这一点。


