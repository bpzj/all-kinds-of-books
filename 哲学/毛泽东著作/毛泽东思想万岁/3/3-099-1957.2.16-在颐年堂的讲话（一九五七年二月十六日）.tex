\section[在颐年堂的讲话(一九五七年二月十六日)]{在颐年堂的讲话}
\datesubtitle{(一九五七年二月十六日)}


今天谈文学艺术、学术思想等百家争鸣的方针问题,文艺有了缺点,应当如如对待。

王蒙写了一篇小说,赞成他的很起劲,反驳他的也很起劲,但是反驳的态度不怎么适当。

〔王蒙的《组织部新来的年轻人》正在讨论,问题在于批评态度。小说揭发官僚主义,很好,揭发得不深刻,但很好。刘宾雁的小说并没有批评整个的官僚主义。王蒙的小说有片面性,正面的积极力量写得不够,要批评。应该有批评,也应该有保护。正面人物林震写得无力,而反面人物很生动。〕

我们的官僚主义就是没有整垮,应该批评。过去有一个片子(指《荣誉属于谁》)没有演,那不是因为批评了老干部,要保护老干部,而是因为在那个片子里,那官僚主义没有整垮。

《星星》(四川的诗刊)里面有几首诗还是好的,有几首是不好的,要酝酿一下,不要骂。

〔《星星》在叫喊,不要忙,沉住气,酝酿一下,不要仓皇应战,不打无准备之仗,过去的“批评”文章,群众投我们的票少。〕

对那些写了坏作品的作者要帮助,要调查。

对起义将领傅作义,对荣毅仁,我们都是帮助、改造。这样做了,他们就同我们合作。

坏小说,无非里面资产阶级思想,唯心论,只要作者在政治上同我们合作,就和胡风有区别,不能一棍子打死。

王蒙是不会写。他会写反面人物,可是正面人物写不好。写不好,有生活的原因,有观点的原因。

李希凡说王蒙写的地方不对,不是典型环境,说北京在中央。难道不可能出现这样的问题,这是不能说服人的。

中央里面就出过坏人,像张国焘,高、饶、李立三、王明。坏人多了怎么办?照他们的办法就要用油锅煮。

〔李希凡的文章不能说服人,认为中央所在地的区委不会有官僚主义,是不对的。中央都出过高、饶,区委(有官僚主义)更不希奇了,“难免论”吗。(对林震也有些“难免论”。)

对如何处理人民内部的错误的方针,很多同志实际没懂。对这个方针,十个部长大概有九个反对。

〔问题是作者(王蒙)对人民内部矛盾问题没有搞清楚,所以小说有消极影响。我们也没有搞清楚。因为写北京,(地方)太具体,就不是典型环境中的典型性格,这样(批评)不能说服人。〕

王蒙的小说有小资产阶级思想,他的经验也还不够,但他是新生力量,要保护,批评他的文章没有保护之意。

是有官僚主义。我们党的威望大,靠党的威望,官僚主义就横行霸道,违法乱纪,是不是应该揭发?

中国是一个小资产阶级的大王国,小资产阶级一共有五亿五千万之多,……这也是客观实际。

〔中国是小资产阶级大王国,党内也不可能没有小资产阶级。〕

说共产党的缺点不能揭发,这种观点不对。

党的统一战线政策,实际赞成的人很少,真正懂得的人很少,“整”,不能服人。

〔干部中真正搞通统一战线思想的人不多——要帮助人家改造,达到真正的团结。怎样帮助改造人家思想,很不够。办法常是整一通。“整”,很容易,但不是好办法,应不用。“整”,不能团结。〕

现在王明还是有选票,他不还是中华人民共和国的公民,而不是一脚把他踢倒。

“一脚踢倒”是老办法,那很容易。喜欢用这办法的人,最好开枪,开机关枪,那是国民党的办法。

…………

〔小资产阶级大王国是客观现实。工人阶级只有二千四百万,其中产业工人占一半,干部也占一半,这是一头;另一头是地富和资产阶级,大约有三千万;中间五亿五千万都是小资产阶级。一千二百万干部也并不都是无产阶级化,真正化了的大约有二百万。所以,看文艺问题要从这现实出发。〕

现在是大变的时期,有的人不满。农业合作化,富裕中农就不满。

〔大学生中地富和资产阶级子弟占80%,都是非工人阶级,青年问题多,并不奇怪。其中要搞匈牙利事变的极少,多数是拥护共产主义的。〕

但是,不满的人过去还是拥护抗美援朝,他们不搞“匈牙利”。当然,个别想搞匈牙利事变的人也是有的。

我们的同志就是怕,怕匈牙利事件。我看匈牙利事件也没有什么不好。要讲辩证法,要懂得事物的两面性,不这么一闹,就没有真正好的匈牙利。

〔……矛盾总要表面化,闹一闹还是好的。闹是片面性,我们用片面性去反对片面性,不能解决问题。〕

我们的同志有教条主义,用片面性反对片面性,用形而上学反形而上学。

王蒙的小说有片面性,又有反官僚主义的一面,我看他的文章写得相当好,不是很好。

他暴露了我们的缺点,不能用李希凡那样的批评。

〔许多批评文章,立场对,但简单化。〕

李希凡现在在高级机关,当了政协委员,吃光饭,听党的命令,当了婆婆,写的文章就不生动了,使人读不下去。文章的头半截,使人读不懂。

马寒冰是什么人?也是一个什么官长,总是一个什么军长级的干部吧,他在《文汇报》的文章就写得不好,教条主义。

〔陈其通等四人的文章,立场很好,人民日报也可以发表,但是说服不了人,似乎百花齐放以后都错了,这是教条主义,朵朵香花是不可能的。教条主义是没有力量的,它所以滋长,原因之一是共产党当了政。马克思、恩格斯批评杜林,列宁批评卢那卡尔斯基,所以要下功夫,驳倒了他们。斯大林不同,(他当了政),所以他们批评不平等,很容易,像老子骂儿子。“一朝权在手,便把令来行”。批评不要利用当政的权力,需要真理,用马克思主义,下功夫,是能战胜的。〕

〔用教条主义不能批评人家,因无力量,请看一下列宁是如何写《经验批制论》的。斯大林后来就不同了,不是平等讨论问题了,不是搜集大量材料发表意见。有些东西写得好,有些东西只坐在山岗上,拣起石头打人,使人看了后不大舒服。〕

陈其通、马寒冰等四人联名写的文章,后讲清楚原因,没有办法。陈沂,你写了几篇文章?有没有教条主义?出一个选集吧!全面检查一下。

教条主义的文章干巴巴,简单化,不能说服人。

教条主义滋长,是因为当了政。

马、恩驳杜林,很用了一番心思。但是当了政的斯大林就不一样了,他骂人不平等,没有说服力。

应该研究分析人家的文章。

当了政,骂人像骂儿子一样,不好。

当政党同人民的关系,不应当是老爷同人民的关系,不应该骂。

教条主义不是马克思主义。

过去批评胡适,取得很大的成功,开头我们说,不能全抹煞胡适,他对中国的启蒙运动起了作用。康有为、梁启超也不能抹煞。

胡适说,我是他的学生。他当教授,我当小职员,工资不一样,但我不是他的学生。

现在不必恢复胡适的名誉,到二十一世纪再来研究这个问题吧。

〔过去因为是斗争,所以讲缺点,今天也不必平反。今天他是帝国主义走狗。到二十一世纪,历史上还是要讲清楚的。〕

〔对地富、资产阶级要安排、改造,何况对小资产阶级?我们写文章常常笔下不留情。教条主义的本领是戴帽子、骂人、片面性,不是从团结出发,目的也不是团结。不是帮助改造缺点,达到真正的团结。

〔对人民内部的错误,要同对付敌人的错误严格地区别,(错误的杂文也是不区别)。对敌人无情斗争,对人民是从团结出发,经过斗争达到新的团结,否则就容易杀人。〕

应当“刀下留人”。

…………

过去延安整风,我们不是说过“惩前毖后,治病救人”吗?我们有些同志不喜欢治病救人,而是庸医杀人。对小资产阶级应当用适当的方法将他们改造。〔那么多小资产阶级,我们靠他们吃饭,要把他们改造为工人阶级。〕我们有百分之五、六十的同志不了解中央这个方针:批评团结,治病救人。

他们就是怕闹事。

清华有个学生,说要杀几千万人,这太多了。这个学生也不要开除他的学籍。

学生闹事有理由,但也不提倡大罢课。他们反国民党反成了习惯。

不允许工人罢工是不正确的。宪法上没禁止罢工。贴标语是言论自由,开会是集会自由。

对闹事的人不应当通通叫他们写悔过书,也不要写检讨。

因为有问题,还是闹一下好。

学生闹事,不等于造反。

六亿人,一年有一百万人闹事是正常的,六百分之一。

闹事的人根本不能说是反革命,可能其中有个别反革命。

对付官僚主义,最好是罢工、罢课、打扁担,因为老不解决问题吆!

当然,我也不登广告,提倡全国罢工。

这些矛盾是暂时性的矛盾,不是根本性的矛盾。

出了事情,要看两面。(闹事的人也有两面性,警惕我们。有脓疮,出脓就好了。)

说老子是“老革命”不能反对。国民党也是“老革命”,比我们更老。我们不能采取国民党对人民的态度。

马克思主义都是同敌人作斗争发展起来的,同资产阶级思想作斗争,从资产阶级思想内,取合理的部分,发展起来,这样才形成了马克思主义。

现在的危险是以为天下太平了,因此看见王蒙的批评就不高兴。只是打,是锻炼不出文

学艺术来的。

是不是允许香花,不允许毒草存在?〔百花齐放应允许毒草存在,允许风格不同。〕

毒草是毒人的,但是香花是同毒草作斗争才发展起来的。

粮食地里有许多野草,庄稼是同野草作斗争才发展起来的。苏联建设了几十年社会主义,地里一样有野草。有野草不要紧,一翻过来,就是肥料。

是毒草,可以说明,有的人主张写:“此是毒草,不许尝试”。苏联的办法是只要香花,不要毒草。其实许多毒草是借香花之名以生。我们的主张是:毒草与香花齐放,落霞与孤鹜齐飞。

斯大林有唯心论,有唯物论,他有片面性。〔形而上学,有唯心也有唯物,斯大林两者都有,有辩证法,也有形而上学。正因为这样,所以有功有过,功大于过。〕

苏联同志转不过来,喜欢采用高压的办法。

教条主义的方法,就是形而上学的方法。

对王蒙作品赞美、骂,都是片面性。

王蒙有两重性,一是好处,一是缺点。

一点里有两点,一个事物包含有两个不同的侧面。

商品有二重性,王蒙也有二重性。

《文汇报》上姚文元的文章好。〔《教条与原则》很有说服力。〕

不能说以前都是教条主义。国民党曾经一家独鸣,所以打倒他们后,共产党也有一阵一家独鸣,这个一家独鸣是必需的,现在就不同了。

〔东欧出乱子,就是这原因。现在情况变了,必须百家争鸣。

要争鸣就要有准备。通过鸣来教育改造,而不是自由主义。一切不要一脚踢开,对错误要批评,也要承认二重性。〕

可以对大家讲清楚,不要仓促应战,不要仓促写文章。打仗,不是说不打无准备的仗,不打没有把握的仗吗?没有把握的也就是无准备的。现在打的仗也就是无准备的。

有资产阶级、地主、富农的残余,有从那里出来的小资产阶级知识分子,有思想斗争,有对他的教育责任。

我们现在是以少数教育多数。

不要用简单的办法开除。

他们可以当教员,很难找到这样的教员。

马寒冰、李希凡的文章也有两重性,是教条主义,但可以促起注意。他们实际不赞成百花齐放、百家争鸣的方针。

实际,一万年以后还有非马克思主义。

而且,有一天马克思主义自己也要完毕。五百年以后怎样设想?现在五百年当过去多少万年。

将来的世界,阶级斗争完了,马克思主义有些东西就没有用了。

宣布自己是永恒真理,就不是马克思主义。

将来没有阶级斗争,但有新的斗争。那时社会科学领域内总会有新的学说出来。

当然,有些东西是不能推翻的,如同说地球旋转而太阳不旋转,是地球绕看太阳转。而

不是太阳绕着地球转。但是地球有一天还是要烂的。

人类也会被否定。有一天这样的人类不适合了,就一切毁灭,但这样对宇宙进化有好处。

三千年前,人类还用石头,以后进化到用铜铁,一直到用机器。人类历史五十万年,章太炎在“书”就讲过。铜器时代否定了石器时代,人类在地球上就是挖地皮过日子。谁也没有选举人类当地球的主人,许多动物,野兽没有选,细菌也没有选。大概只有植物才不反对我们。我们死了,可以供给植物养料,我们排泄的东西也可以供它们补用。

人类本身要进化,马克思主义者也要进步。

如何建设这样一个大国家,是马克思主义者的一个新问题。如何少走弯路?我们这样一个大国,六亿人口,马克思没有想到,列宁也没有想到,他建设社会主义国家也没有几年。

我们六亿人口,将来有一天上街,在大街上走都得排队。将来在街上会挤杀人,怎样分配报纸,怎样看电影,怎样逛公园,都会成问题。

不能说一切问题都解决了,还有很多问题要发现,要解决。

不要怕歪风。风总有两种,一歪一正。

吉林今年大水灾,就是下多了雨。两股流,一股寒流,一股暖流,不在黑龙江上空相遇,也不在辽宁上空相遇,偏偏在吉林上空相遇,作战,就下大雨。雨下多了,不好。但是如果不下雨,就是旱灾。两股还缺不得一股。

越斗争,越丰富,就会出现新道理。

(××:现在有一种倾向,写爱情多。)

没有爱情,人类就绝种。

(××:杂文有“难免论”。)

“难免论”要多登几篇,这样就引起注意了。

掌握政权也有两重性。许多人就是怕“放”,怕“鸣”。

〔我们应特别(注意)到,掌握政权之后,想用简单的办法把人打倒。百花齐放为什么(有人)怕放?怕饭票子过河。

青年人反对官僚主义不那么怕,因为他们还(没)有当事,官僚主义还未轮到他们头上。〕

最近这种压法,缺乏说理,不大妙,应给人家出路,帮助。

(××:王蒙要求谈。)

谈两点:1.好,反官僚主义;2.文章还有缺点。

王蒙很有希望,新生力量,有文才的人难得。

现在就是碰不得官僚主义,好像骂官僚主义就是骂自己,好,你自己既然承认官僚主义,就得骂。

不能讲北京不能批评,不能讲不能写党的缺点。

(××:说不搞工农兵了。)

不搞了,全国就剩下工农兵加知识分子了。

(××:工农兵搞得好,大家还是欢迎。)

是啊,我们反杜林反了许多年,还不知道杜林是怎样的,可不可以找到杜林的书?

(××:谈到辩证逻辑和形式逻辑的关系。)

〔黑格尔、马赫。新建设周谷城论大逻辑,有点道理。〕

〔学校只听校长意见,不听学生意见不对。〕

①(翻印说明:这一材料是根据三份传达记录稿整理而成,以一份记录稿为主,〔〕内是根据另两份记录稿插入的。)

②这是中央宣传工作会议之前主席召开的一次座谈会,对文、哲、教育界谈文化思想问题,着重谈文学批评问题。


