\section[接见阿尔及利亚民族解放阵钱代表和法律工作者代表团的谈话(一九六四年一月二十八日)]{接见阿尔及利亚民族解放阵钱代表和法律工作者代表团的谈话}
\datesubtitle{(一九六四年一月二十八日)}


主席:代表团有几个人?

本·阿卜杜拉:十二个人。

主席:来了多少时间?哪一天到北京来的?

阿卜杜拉:一月十七日到北京的,十三日就离开阿尔及利亚了。

主席:准备到什么地方去参观?

阿卜杜拉:还要到上海去两天,杭州两天,广州两天,然后回国。

主席:到广州从南路回去吗?

阿卜杜拉:是的。

主席:你们是从南路来的,还是从北路来的?

阿卜杜拉:从南路来的,经过开罗、仰光和昆明。很希望在中国多呆几天,因为国内有工作,已经延长了几天,现在不得不回去了。主席:你们胜利了,我们很高兴。你们的胜利是个典型,大胜利,是少数战胜了多数,法国几十万军队被打败了。为什么少数能变成多数呢?原因就是你们有群众,人民群众能够战胜帝国主义。

阿卜杜拉:我们胜利还不久。

主席:阿尔及利亚现在是有一千二百万人口吗?有人也说是一千万。

阿卜杜拉:一千二百万。不过阿尔及利亚从来没有进行过人口调查。过去法殖民统治者把阿尔及利亚人民当狗看待,多一、二百万、少一、二百万,对他们说算不了什么。

主席:过去你们临时政府告诉我,阿尔及利亚有一千万人口,包括一百万法国人。那么本国人只有九百万,战争上又牺牲了一百多万,只有八百万不到一点。人民一定会胜利。人口在打过仗之后也只会增加不会减少,我是根据我们的经验说这话的。

我们党在开始的时候只有五十七个党员,在一九二一年召开第一次党代表大会时只有十二个代表。现在十二个人中只剩下两个人,那十个人或者牺牲了,或者叛变了,可是革命力量发展了,越来越大了。我们的革命一共化了二十几年才胜利,从一九二七年打仗打到一九四九年,整整二十二年。一九二七年大革命失败时,革命力量受到很大损失,由五万党员降到几千党员。那时我们没有经验,蒋介石叛变革命,同我们打了十年国内战争。大革命失败是因为我们党内产生了右倾。然后在战争中党壮大了,军队壮大了,根据地也扩大了,有了三十万党员,三十万军队,根据地的人口也有几千万。这时又产生了“左”倾,他们要打大城市,社会政策不对,只要工人、农民,民族资产阶级、小资产阶级都不要。对民族资产阶级的政策不对,没有把民族资产阶级与买办资产阶级分开,提出的一切政策也就不对,结果又失败了,被迫举行了万里长征。万里长征不是我们愿意干的,这是政策失败了,没办法,从南方跑到北方。这一来三十万军队剩下不到三万,被搞掉百分之九十以上,三十万党员只剩下几万,所有大城市的组织差不多都完了,又是一个大失败。可是这两次失败,一九二七年右倾失败,及以后的“左”倾失败,是好的还是不好的呢?

阿卜杜拉:主席所讲的这点,在我们来中国以前就认识了,失败有消极的一面,但也有另一面,可以从失败中得到经验,所以也可以说失败是成功的基础。

主席:我们就是这么看。没有这两次失败,中国革命不能胜利,不能总结经验,反对右倾机会主义,又反对“左”倾机会主义,使我们能采取正确的政策。又团结又斗争的政策。

第一次失败,是没有看到朋友变成敌人,只讲团结不讲斗争。第二次失败,是只讲斗争不讲团结,把小资产阶级、民族资产阶级全看成敌人。这两次党内关系也不正常。我们就总结经验,所以抗日战争时期,我们的政策就比较正确了。抗日战争经过八年,由两万五千军队发展到一百二十多万,根据地的人口由十几万发展到一亿多。胜利时日本人跑了,美国人又来了。蒋介石有四百多万军队,一切大城市和铁路、矿山资源都在他手里。国民党向全国解放区发动进攻,占领我们很多城镇和乡村,我们把延安都失掉,许多外国朋友也认为我们不行了。延安是个小城市,这个小城市只有几千人口,是山区,我们拿它做根据地,后来这个根据地也失掉了,很多人都认为共产党没有希望了。后来我们釆取正确的退却政策,退却一年的样子,退却过程中消灭了国民党八个旅,一直到一年以后,我们才可以举行反击。解放战争一共用了三年半时间,蒋介石跑到台湾去了。现在蒋介石还在联合国里“代表中国”,我们还被叫“土匪”。(全场大笑)法国人昨天同我们建立外交关系。你们胜利之后,法国人才承认你们,那也好嘛!有些国家至今还不承认我们,意大利、比利时、西德、日本,主要是美国,他们的政策是孤立我们,那时我们与你们一样。你们在没有同法国签定埃维昂协定之前,你们的情况也不那么好,好像很孤立的样子,其实你们并不孤立,有什么孤立的?突尼斯的关系与你们搞的不好,不久前摩洛哥的关系也搞的不好,我看没有什么要紧,同情你们的人很多很多,中国人民同情你们,整个亚洲、非洲、拉丁美洲的绝大部分人民都是同情你们的。你们在欧洲也有朋友,法国人中也有你们的朋友。

法国政府过去不是你们的敌人吗?你们的敌人也是我们的敌人,帝国主义是我们的共同敌人。世界上的事情是会起变化的。当法国人跑了的时候,你们多困难,没有粮食,没有教员,没有医生,没有药品,工厂开工资金不足,现在你们也还在困难阶段,没有工程师,没有技术员;要有地质工作人员,要勘探石油,要勘探各种矿产,但是要有个过程才能建立这样的地质工作队伍。总之,你们是会搞起来的,由没有到有,由少到多,由不会、不懂,到学会,到懂,我是根据我们的经验讲这个话的。比如军队,我们没有,现在有了,你们也没有军队,现在也有了。又如打仗,谁会?我就不会打仗,还不是学会的。军队的事,打仗的事,能由没有到有,由不会到会,为什么经济建设和文化建设我们就不能搞起来?困难可以克服,不会的人可以学会,没有的东西是可以有的,不要那么多迷信,要破除迷信,只要肯干,我是不大信迷信的,过去也有过迷信。很多是敌人教会了我们的,必须团结国内人民,只要依靠人民就有出路。

脱离群众是不行的,是不是这样?过去的一些领导人,你们不要了,我们有一个时候不知道是什么原因,后来才清楚,就是他们脱离了群众。是这样吗,不知对不对?

阿卜杜拉:主席的分析很正确。阿尔及利亚的领导人于一九六二年二月在的黎波里举行了会议,在会议上起草和通过了一个纲领。那时出现了一个多数和少数,多数中不包括过去临时政府的某些领导成员及其他部门担任领导工作的干部。对中国朋友们说,这没有什么秘密,世界报也谈到过。那就是少数人不愿意服从多数,不愿意接受多数的观点,可以明确指出,当时出现的多数反映了阿尔及利亚广大人民的意志。后又经过选举产生了国民会议,由国民会议任命了现政府。

主席:革命中总有一部分人,他们可以反对帝国主义,反对殖民主义,但再进一步他们就不干了。我们也发生过这样的事,当反对帝国主义的时候他们干,反对封建主义的时候他们干,他们自认为共产党员,但实际上是资产阶级民主革命者,一到搞社会主义他们就不干了,他们就反动了,有这么一部分人。哪一个党内都有这样的事情,特别是在革命转变时期,是不可避免的。过去多数是进步的,少数是落后,如果政权掌握在落后分子手里,那就危险了。你们现在正要建立一个党,不久开党的代表大会?

阿卜杜拉:在这个问题上,阿卜杜拉希德·日拉伯兄弟是我们党中央领导成员,他可以谈一谈。

阿卜杜拉希德:阿尔及利亚民族解放阵线政治局决定举行一次党的代表大会。刚才本·阿卜杜拉兄弟谈到一九六二年的黎波里会议,那是在战争中阿尔及利亚内部分歧斗争的表现。当时说的分歧是斗争方法上的分歧,是釆取什么方式进行斗争。在一九六二年的黎波里的阿尔及利亚民族解放阵线代表大会上出现分歧意见,领导方面存在的真正矛盾表现出来了,方法上的分歧不是实质,实质是思想意识上的分歧。在的黎波里会议上的潮流,是进步的潮流,是向前进的潮流。但能否说的黎波里会议之后什么矛盾都解决了呢?不能这么说,由于我们斗争的需要,在经济建设时期又出现了新的矛盾,因为我们的斗争是继续发展的,需要革命的纲领,革命的领导集团,使我们不断前进。

我们这样的思想意识是在我们过去和现在的斗争中不断建立起来的,将来还要向前进步,要建设社会主义,要做出这样的选择,必须是明确的选择,我们正在准备报告,将来在党代表会上提出。主席:什么时候开?阿卜杜拉希德:日期还没有定。在即将召开的党代表大会上,特别要总结过去的经验,武装斗争的经验和独立后在本·贝拉兄弟领导下进行经济建设的经验,准备在大会上总结一九五四年以来的革命经验,这些总结是重要的,将使我们了解过去的经验是什么,这样也能使我们明确在各种各样的思想意识中,今后选择哪一方面。这个大会将为我们奠定不可动摇的社会主义基础。这是独立之后举行的第一次党代表大会,是历史性的会议,是阿尔及利亚人民历史转折点的会议,对于阿尔及利亚的政治、经济建设是很重要的。

主席:你是他们一起来的吗?

阿卜杜拉希德:是一起来的,分开接待。

主席:只一个人吗?

阿卜杜拉希德:一个人。本·阿卜杜拉兄弟领导的法律代表团是来中国学习法律方面的经验的。我是中共中央接待,负责另一方面的工作,学习另一方面的经验的。

主席:不能学习,我们也是在摸索过程中,有很多错误和缺点,要全面分析再接受,不要认为中国什么都是好的,中国也有不好的一面。有进步的一面,有落后的一面,工作中有正确的一面,也有错误的一面,我们这几十年就是这么过来的,经常犯错误,改正错误。不隐瞒错误,你们(指陪见人)不要隐瞒错误,只介绍正确的东西,不介绍走弯路和错误的方面。中国农业很落后,工业现在与先进国家比还差的远,在我们的社会上和党内,干部也有变化,有的变成了贪污分子,实际上是资本家,我们的政策是对他们进行教育,进行社会主义教育运动。可惜时间不够,不能细致地介绍我们的政策。像你们这样的党,这样的国家,我们应该把一切经验毫无保留地介绍给你们。对待反革命分子和犯人的政策,我们也犯过错误。你们看见过我们的监狱没有?

阿卜社拉:我们看过北京监狱。

主席:我们的监狱也有办得好的,也有办得不好的。(指陪见人)北京监狱是那个办得好的吧?办得不好的不让你们看。(向陪见人)要让他们看一个办得坏的。就是应该这样介绍,有好的,也有坏的。人民公社也有办得好的,也有办得不好的,我们现在要做的,就是使办得不好的办得好起来。军队里也可以看看。军队是专政工具的主要工具,你们是搞法律工作的,不看军队不好,你们国家如果没有军队,你们的法律工作还能搞吗?没有军队的保卫,你们就不能生存。还要看看警察和公安部队。(向陪见人)与军委和公安部联系一下,让他们看看军队、警察、公安部队,以及民兵的情况。你们大家都是搞法律工作的,专门在法律条文上作文章是作不出什么来的。光靠监狱解决不了问题,要靠人民群众来监视少数坏人,主要不是靠法院判决和监狱关人,要靠人民群众中多数监视、教育、训练、改造少数坏人。监狱里关很多人不好,主要劳动力坐牢就不能生产了。今天我们讲不完了,没有时间,只能介绍些要点。你们还过几天走?

阿卜杜拉:我们在北京只有二十四小时了。

主席:有些问题可以到上海去解决。可以告诉他们,说是我讲的,要介绍正确的方面,也要介绍错误的方面,要介绍好的方面,也要介绍坏的方面。这样才是好的方法。

阿卜杜拉:从现在起我们就采取这种方法。

主席:(对陪见人)给各地要讲清这个问题。

阿卜杜拉:我也应该表示,在这里介绍的情况没有什么隐瞒,他们向我们介绍了胜利的经验,也讲了失败的原因。

主席:好嘛,应该这么作。我讲我们过去的历史,就讲了我们是怎么犯错误这一点,错误对我们很有益处,教育了我们。从成功的方面学得的经验,也从失败的方面学得经验。我看你们总结你们的历史也会是这样的,总有代表比较正确的一面。回去后你们的本·贝拉总统和其他朋友们,你们真的搞社会主义,我很高兴,那我们不仅是反对帝国主义和封建主义斗争的同志,而且是搞社会主义的同志。搞社会主义要团结大多数人,团结一切反对帝国主义,反对封建主义的人,团结一切干社会主义的人。

阿卜杜拉:我代表阿尔及利亚法律工作者代表团全体成员向主席表示感谢,感谢给予我们的荣誉和骄傲。代表团和阿尔及利亚人民向主席和中国人民表示亲切的祝愿。现任请允许我们把这把刀送给主席,这是阿尔及利亚武装斗争胜利的象征。

主席:很有意义的礼物,这东西是对付敌人的。


