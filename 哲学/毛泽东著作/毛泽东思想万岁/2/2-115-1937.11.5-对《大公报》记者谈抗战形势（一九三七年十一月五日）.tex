\section[对《大公报》记者谈抗战形势(一九三七年十一月五日)]{对《大公报》记者谈抗战形势}
\datesubtitle{(一九三七年十一月五日)}


从上海太原等地失守后,不容讳言的,我们在抗战的军事上受到了相当的挫折。但在各个战场的挫折中使我们获得了最可宝贵的伟大的教训,这教训是什么?就是我们这次民族抗战,虽然是革命性的,但它的革命性,还不完全。我们参战的地域虽然是全国性,这是自帝国主义者侵略中国以来未有的好现象,但参战的成分,却不是全国性的,最大的缺憾尚未动员全国人民到抗战中来。反对日本帝国主义侵略的战争而不带群众性是必然会遭遇失败的,在近代弱小民族反对帝国主义侵略的历史上,阿比西尼亚抗战的失败,可为我们前车之鉴。拿最现实的例子来说,西班牙政府军所以能够坚守马德里,击退德意撑腰的法西斯们进攻,主要的在于能够发动群众,使大家能拚其血肉,以保卫马德里,守住自由民主的西班牙。西班牙政府军现有的地区,不及我们江西省之半,因为动员了全体人民,便在抗战中发挥无与匹敌的伟大力量,要是我们真正实践了动员全国人民,不难击溃日本帝国主义的进攻!此外,已往各个战场上所釆的战略战术,犯了“专守防御”的错误。军事上战略战术的基本原则,是保护自己,消灭敌人。因此,我们要设法减削敌人优势武器之运用,避实就虚,击中敌人致命的弱点。敌人在每次战斗中,釆用迂回及中央突破战略,我们便不能专门着重在“单纯防御&quot;,死守正面,使敌人恰恰施展其优势武器,而集中击破我正面。必要的阵地和城市,我们当然要守,但主要的还是我们的防御,还要配合上侧击或敌人后方迅雷不及掩耳的攻击,要以独立自主的运动战来歼灭敌人。

眼前最要紧的是改造军队素质,加强军队的政治工作。中国只要精兵三十万,具有最高度的民族意识,与政治自觉性,再配以新式武器,军官与士兵一律富于高度的攻击性,便可以使目前的战局,为之全盘改观。</font></p>

<p align="center"><font face="宋体">×××</p>

虽然太原失守了,但八路军在冀察晋边区及晋西北绥东一带,已经据有华北游击战的基点,正在广泛的游击战,彻底破坏了敌人自大同至太原的公路,以及正太同浦路的交通线。我们一定坚持在华北的游击战中,完成华北抗战的战略基点。不要说敌人占了太原及晋北的几个城市,就是敌人吞了山西全省,我们仍坚持干下去,决不南退。敌人南进,我们得北进,以摧毁敌人的后方。我敢说敌人吞了华北,决不是一服补剂,相反的却是一枚炸弹。</font></p>

<p align="center"><font face="宋体">×××</p>

日寇原欲在中国不战而赏其大欲,但经中国长期抗战的结果,将使这个先天不足后天失调的帝国主义者陷入崩溃的深渊。从这一方面说,中国抗战不但为了自救,且在全世界反侵略阵线中,尽力最伟大的任(义?)务。

<p align="right">(《八路军的战略和战术》。上海生活出版社一九三八年一月版)</p>

