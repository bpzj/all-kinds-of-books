\section[蹲点问题(一九六四)]{蹲点问题(一九六四)}


革命失败后的一段时间,调查没有研究,实在气得要死。红军到一个地方,屁股坐不住,吃了就走,当时林彪同志说要有“灵活战术”,大家反对说这叫新花样,“步兵操典上没讲,步兵操典是人家经过流血写出来的,搞什么新花样”。

蹲点不蹲点,不知道什么叫工业,一不请教工人,二不请教技术人员,这怎么行?不了解情况,为什么不学?应该懂些嘛。搞了几年了,恐怕就是不蹲点,毫不知道。可以到白银厂学习,没有到工厂学习,不向工人学习,不向技术人员请教,总是内心无主。蹲点也要采取比较法,比学赶帮,首先是比较法嘛,马克思主义和修正主义也是比较嘛。


