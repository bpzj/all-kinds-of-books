\section[苏联《政治经济学教科书》阅读笔记(补遗)]{苏联《政治经济学教科书》阅读笔记(补遗)}


一、关于我国工业化问题

苏联第一个五年计划完成以后,大工业总产值占工农业总产值的70%。就宣布实现了工业化。我们很快就可以达到这个标准,但即使这样,我们也还不宣布实现了工业化。我们还有五亿多农民从事农业生产,如果工业产值占70%就宣布实现了工业化,不仅不能确切地反映我国国民经济的实际情况。而且有可能由此产生松劲情绪。

我们“八大”一次会议曾说:要在第二个五年计划建立社会主义工业化的巩固基础,又说:要在十五年或者更多的时间内建成完整工业体系。这两个说法有点矛盾,没有完备的工业体系怎么能说有了社会主义工业化的巩固基础呢?现在看来,我们再有三年,主要工业产品产量可以超过英国,然后再有五年就可以实现建立工业体系的任务。

长时期内,我们这个国家应该叫做工农业国,即使钢有了一亿多吨,也还是这样。如果按人口平均的产量超过英国,那么我们的钢产量最少需要三亿五千万吨。

找一个国家来比赛,这个方法很有意义。我们一直总是提赶英国。第一步按主要产品的产量来赶,下一步按人口平均产量来赶。造船业、汽车制造业,我们还比他们落后得很远,我们要一切争取赶过它。日本这样的小国都有四百万吨的商船了,我们这样大的国家没有远洋的轮船自己运货,说不过去。

一九四九年,我国拥有机床九万多台,一九五九年增加到四十九万多台。日本一九五七年有六十万台。拥有机床多少,这是工业发展水平的重要标志。

我国机械化的水平很低,从上海就可以看出来。根据最近调查材料说,那里的现代企业中机械化劳动、半机械化劳动、手工劳动各占1/3.

苏联工业劳动生产率现在还没有超过美国,我们则差得更远。人口虽然多,但是劳动生产率比不上人家,从一九六○年起,十三年中还要紧张的努力。

二、关于人民的地位和能力

486页说社会主义社会中,人的地位只决定于劳动和个人的能力。未必如此。聪明人往往出在地位低,被人看不起,受过侮辱,而且年青的人中,社会主义社会也不例外。旧社会的规律,被压迫者文化低,但是聪明些,压迫者文化高,但总是愚蠢些。在社会主义社会的高薪阶层也有些危险,他们的文化知识多些,但是同那低薪阶层比较起来,要愚蠢些,我们的干部子弟就不如非干部子弟。

有很多创造发明。都是在小厂里头出来的,大厂设备好,技术新,因此往往架子大,安于现状,.不求进取,他们的创造往往不如小厂多。最近常州有一个纺织厂创造了一个提高织布机效力的技术装置,有助于使纺纱、织布、印染得到能力平衡,这个新技术不在上海、天津创造出来,而在常州这样的小地方创造出来。

一

知识是经过困难得到的。屈原如果继续做官,他的文章就没有了,正是因为丢了官,下放劳动,才有可能接近社会生活,才有可能产生像“离骚”这样好的文学作品。孔子也是因为在许多国家受到了挫折。才转过来搞学问。他团结了一批“失业者”。想到处出卖劳动力。可是人家不要,一直不得志,没有办法了,只好收集民歌(诗经),整理史料(春秋)。

历史上许多先进的东西不出在先进的国家,而出在比较落后的国家。马克思主义就不出在当时资本主义比较发展的国家一一英国。法国,而出在资本主义只有中等发展水平的德国并不是没有理由的。

科学发明也不一定出于文化高明的人,现在许多大学教援并没有发明,而普通的工人反而有发明。当然我们并不是否定工程师和普通工人的差别,不是不要工程师。但是这里确实有个问题,历史常常是文化低的打败文化高的,在我们的国内战争中,我们的各级指挥员,从文化上说比国民党的那些从国内外军事学校出来的军官低,但我们打败了他们。

人这种动物有一种毛病,就是看不起人。有了点成就的人,看不起还没有成就的人,大国、富国看不起小国、穷国,西方国家是历来看不起俄国人的,中国现在还处于被人看不起的地位。人家看不起我们也是有理由的,因为我们还不行,这么大的国家,只有那么一点锭。有这么多的文盲,人家看不起我们,对我们有好处,逼着我们努力,逼着我们进步。

三、关于依靠群众的问题

列宁的这句话:“社会主义是生气勃勃的,创造性的,是人民群众本身的创造”(332页)讲得好。我们的群众路线就是这样的,是不是合乎列宁主义呢?教科书引了这句话以后说:”广大劳动群众日益直接地和积极地参加生产的管理,参加国家机关的工作,参加国家社会生活的一切部门的领导。”(332页)也讲得好。但是讲是讲,做是做,做起来并不容易。

联共中央一九二八年的一个决议中说:“只有党和工农群众最大限度地动员起来,才能解决在技术和经济方法赶上并超过资本主义国家的任务”(377页)这句话也讲得很好。我们现在就是这样做的,斯大林那个时候,没有什么别的指靠,只有靠群众,所以他们要求党和工农群众最大限度的动员起来,后来有了点东西了,就不那么依靠群众了。

列宁说。“真正民主意义上的集中制,……使地方的首创性、主动精神和各种各样达到总目标的道路、方式和方法,都能充分地顺利地发展。(450-451页)说得好。群众能够创造出道路来,俄国的苏维埃是群众创造的。我们的人民公社也是群众创造的。

四、关于苏联和我围发展过程中的一些比较

教科书422页上引用列宁的话:“国家政权如果掌握在工人阶级手中,通过国家资本主义,也可以过渡到共产主义”等等。这些话讲得很好。列宁是个干实事的人,他在十月革命以后,因为看到无产阶级管理经济没有经验,曾经企图通过国家资本主义的方法来训练无产阶级管理经济的能力。那时候的俄国资产阶级对无产阶级力量估计错误,不接受列宁的条件进行怠工破坏,逼着工人不得不没收资产阶级的财产,所以国家资本主义没有能发展。

在内战时期,俄国困难确实很大,农业破坏了,商业联系被打断了,交通运输业不灵了。搞不到原料,不少工厂没收了,也不能开工,因为实在没有办法,只好对农民实行余粮征集制,这在实际上是无代价的取得农民劳动生产品的办法,实行这样的办法,势必对农民翻箱倒缸。这个办法实在不妥,在内战结束以后,才用粮食税制代替余粮征集制。

我们的内战时期比他们长得多,二十二年间我们在根据地中历来实行征收公粮和购买余粮的办法,我们对农民采取了正确的政策,在战争中紧紧地依靠农民。

我们搞了二十二年的根据地政权工作,积累了根据地管理经济的经验,培养了管理经济的干部,同农民建立了联盟。所以在全国解放以后,很快地进行和完成了经济恢复工作.接着我们就提出了过渡时期的总路线,把主要力量放在社会主义革命方面,同时开始了第一个五年计划的建设。在进行社会主义改造中间,我们是联合农民来对付资本家,而列宁曾经在一个时期说过宁愿同资本家打交道,想把资本主义变成国家资本主义.来对付小资产阶级的自发势力。这种不同的政策,是由不同的历史条件决定的。

苏联在新经济政策时期,因为需要富农的粮食,所以对富农釆取限制的政策,有点像我们解放初期对待民族资本家的政策。只有等到集体农庄和国营农场,一共生产了四亿普特的粮食的时候。才对富农下手。提出消灭富农,实现全盘集体化的口号①。而我们呢?却在土地改革中就把富农经济实际上搞掉了。

苏联的合作化“在一开始农业曾经付出了很大的代价”。(397页)这一点使东欧许多国家在实行合作化的问题上增加了许多顾虑.不敢大搞,搞起来也很慢。我们的合作化没有减产,反而大大增产,开始许多人不信,现在相信的人慢慢地多起来了。教科书中光拿苏联的经验吓唬人。影响不好。

斯大林在一九二九年十二月的《论苏联土地政策的几个问题》中说。“富农在一九二七年生产了六亿多普特粮食。其中通过农村外的交换卖出了大约一亿三千万普特。这是一个十分严重而不可忽视的力量。当时我们的集体农庄和国营农场生产了多少呢?大约八千万普持。其中运到市场去的(商品粮食)约为三千五百万普特.”所以斯大林断定,在这种情况下。向富农进行坚决进攻是不可能的。斯大林接着说。“现在我们有充分的物质基础打击富农”。因为一九二九年集体农庄和国营农场生产的粮食不下四亿普特,其中的商品粮食在一亿三千万普特以上。(斯大林全集十二卷一四八页)

五、关于总路线形成和巩固的过程

这两年我们做了个大试验。

全国解放初期,我们还没有管理全国经济的经验,所以第一个五年计划期间.只能照抄苏联的办法。但总觉得不满意,一九五五年基本上完成了三个改造的时候,在年底和一九五六年春,找了三十几个干部谈话,结果提出了十大关系,提出了多快好省。当时看了斯大林一九四六年选举演说。沙皇时代的俄国原生产钢四百多万吨,到了一九四○年发展到生产钢一千八百万吨。如果以一九二一年算起,二十年只增加了一千四百万吨。当时就想。都是社会主义,我们是不是可以搞得多一点,快一点。后来提出两种方法的问题。同时还搞了一个四十条农业发展纲要,此外,当时还没有提出其他具体措施。

一九五六年的跃进后。出来了一个反冒进,资产阶级右派抓住这条辫子.举行猖狂进攻。否定社会主义建设的成就。1957年6月人民代表大会上.周总理作报告给资产阶级右派一个回击。同年9月党的三中全会恢复多快好省.四十条纲要,促进会等口号,11月在莫斯科修改在人民日报上多快好省地社论。这年冬季全国展开了大规模水利建设的群众运动。

一九五八年春先后在南宁、成都开会,把问题扯开了,批判反冒进。确定了以后不准再反冒进。提出社会主义建设总路线。如无南宁会议,搞不出总路线来。五月××同志代表中央向八届二次代表大会作报告,会议正式通过总路线,但是总路线不巩固,接着搞具体措施,主要是在北戴河提出了钢产量翻一翻,大搞钢铁的群众运动。即西方报纸所说的后院钢铁。

同时开展人民公社化,又夹着金门打炮,这些事情,惹翻了一些人,得罪了一些人。工作中也出现了一些毛病,吃饭不要钱,把粮食和副食吃得紧张起来,刮共产风,百分之几的日用品供应不上。一九五九年钢产量北戴河定了三千万吨,武昌会议降为二千万吨,上海会议又降为一千六百五十万吨。一九五九年六月间又降为一千三百万吨,所有这一些被那些不同意我们的人抓住,但是他们在中央反“左”的时候,不提意见,两次郑州会议不提,武昌、北京、上海会议不提,等到“左”已经反掉了,指标已经落实了,在反“左”必出右,庐山会议需要反右的时候,出来反“左”了。

这些说明天下并不太平,总路线的确不巩固。经过两次曲折,经过庐山会议,总路线现在比较巩固了,但是事不过三,恐怕还要准备一次曲折,如果再来一次,总路线也将更加巩固起来。据浙江省委的材料,有些公社最近又出现了一平二调的情况,共产风也还可能再次出现。

一九五六年反冒进的那次曲折,国际上出现波匈事件,全世界反苏。一九五九年这次曲折,是全世界反华。

一九五七年和庐山会议的两次整风反右,把资产阶级思想影响和资产阶级残余势力批判得比较彻底,使群众从它的威胁下面解放出来,同时破除了各种迷信,包括所谓“马钢宪法”之类迷信。

搞社会主义革命,过去不知道怎么革法,以为合作化了,公私合营以后就解决了。资产阶级右派的猖狂进攻,使我们提出了政治战线和思想战线上的社会主义革命。庐山会议实际上也进行了这个革命,而且是很尖锐的革命。如果不在这次会议把右倾机会主义那条路线打下去,是不行的。

六、关于帝国主义各国间的矛盾及其他

我们应当把帝国主义之间的相互斗争,看作是一件大事。列宁是把它看成一件大事的。斯大林也是这样看的,他们所说的革命的间接后备军就是指这个。中国搞革命根据地也吃过这碗饭,我们过去存在着地主买办阶级各派的矛盾,这个矛盾背后,是各帝国主义国家之间的矛盾。因为,它们内部有这样的矛盾,只要我们善于利用这种矛盾,那么直接同我们作战的在一个时期中就只有一部分敌人,而不是全部敌人,而且我们常常能得到休整的时间。

十月革命胜利能够巩固下来,一条重要的原因是帝国主义内部的矛盾多。当时有十四个国家出兵干涉,但每个国家派的兵都不多,而且不齐心,相互勾心斗角。朝鲜战争中,美国和他的同盟国也不齐心,战争也打不大,不但美国下不了决心,而且英国不愿意,法国不愿意。

国际资产阶级现在非常不安宁,任何风吹草动,它们都害怕,警惕性很高,但是章法很乱。

第二次大战后,资本主义社会中的经济危机同马克思的时候不同了,变化了。过去大体上是七、八年或十年来一次,第二次世界大战后到一九五九年十四年中已经来了三次。

现在国际局势比第一次世界大战以后的国际局势紧张得多了,当时资本主义还有一个相对稳定时期,除苏联以外,其他国家的革命都失败了。英国和法国很神气,各国资产阶级对苏联也不那么怕,除了德国的殖民地被剥夺之外,整个帝国主义的殖民体系还没有瓦解。第二次世界大战以后,三个战败了的帝国主义垮台了,英、法也削弱了,衰落了,社会主义革命在十几个国家成功,殖民体系瓦解了,资本主义世界已经再也不能有第一次世界大战后的相对稳定。

七、中国的工业革命为什么能够最迅速

西方资产阶级舆论中,现在也有人承认“中国是工业革命最迅速的国家之一”。(美国的康伦公司关于美国外交政策的报告中就说到这一点)

在世界上已经有了许多国家进行过工业革命,比起以往所有的国家工业革命,中国看来将是最快的一个。

为什么我国的工业革命能够最迅速呢?重要原因之一是我们的社会主义革命进行得比较彻底。

我们彻底地进行对资产阶级的革命,尽力肃清资产阶级的一切影响,破除迷信,力求使人民群众在各个方面得到彻底解放。

八、人口问题

消灭人口过剩,农村人口是个大问题,要解决就要生产大发展,中国有五亿多人口从事农业生产,每年劳动而吃不饱,这是最不合理的现象。美国农业人口只占百分之十三,平均每人有两千斤粮食,我们还没有他们多,农村人口要减少怎么办?不要引入城市,就在农村大办工业,使农民就地成为工人,这样有一个极其重要的政策问题,就是要农村生活不低于城市,或者大体相同,或者略高于城市,每个公社都要有自己的经济中心。有自己的高等学校,培养自己的知识分子,这样才能真正解决农村人口过剩的问题。


