\section[关于大学文科改革的指示(摘录)(一九六五年十月)]{关于大学文科改革的指示(摘录)}
\datesubtitle{(一九六五年十月)}


我看了三篇文章,写的都很好。学哲学就要学这些有实际的哲学,我们老一辈不行了。

资产阶级讲天赋人钱,那里有天赋人权,都是革命来的,都是工人贫下中农斗争来的。希望学哲学的人,都要到工厂、农村跑跑。我看了南京大学一个学生参加四清后写的一篇体会,写得很好,善于通过现象看本质,本质看不见摸不着,只有调查研究才能看到。同志们要多学点东西,学点植物学,土壤学。现在大学教育我很怀疑,上大学,看不见务工务农务商的,学完了不知工人怎样做工,农民怎样种地,还把身体搞坏了。我告诉我的孩子学农务商,学完了到农村,就说我到这里来补课。

大学里要学那么长时间,一个人两岁学会说话,三岁就会打架,五岁可以看父母种田,从小学会很多概念。现在教育太脱离实际了,大学教育要很好改进,不要学那么长时间,特别是文科要改进。不然,学哲学的不懂哲学,学历史不懂历史,学文科写不出文章。大学生要到工厂、农村去,下连队当兵,接触实际。高中毕业后先做几年实际工作,然后再读几年书,过去的大发明家都不是什么大学出身。

哲学是值得研究的问题,恩格斯讲辩证法是二条,斯大林讲四条,我看就一条——对立统一。什么叫综合?我说综合就是把敌人吃掉。什么叫分析?吃螃蟹就叫分析,把肉吃了剩下壳。分析和综合分不开,什么事情都有两重性,有对有错,杨献珍这些人应该下去,帮助他们改造。形式逻辑是一门专门科学,和辩证法合在一起不实际。总之,大学文科非改革不行,不然就要出修正主义。

