\section[关于中共更加布尔什维克化十二条(一九四二年十一月二十三日)]{关于中共更加布尔什维克化十二条}
\datesubtitle{(一九四二年十一月二十三日)}


这次高干会议开得很好,××同志的发言是完全正确的,我完全同意批评朱理治、郭洪涛两人历史路线上的错误。大会说明创造边区及参加边区工作的同志都是好同志,边区的干部更好一些,经历了内战、土地革命、抗日三个时期,外来和本地干部要互相团结,“洋包子”决比土包子多走了一段路,“洋包子”高明,那么你的根据地怎么垮的?所以原则不好,加上原则不好,加上客观原因,敌人压力很大,主观原则的错误,把白区差不多弄完了。今天我想讲斯大林同志十二条。讲这十二条,跟刚才讲的全党路线搞清楚,全党团结起来,在这基础上讲十二条。

斯大林说:“我们要实行布尔什维克化;要有那些条件,至少有若干种种条件,如果没有这样的条件,布尔什维克化是空的。”如果没具备这些条件或不完全,就不能布尔什维克化,可是布尔什维克化“化”字是不容易的。

我们全党情况怎样?是向着布尔什维克化,经过廿一年的斗争,可说是布尔什维克,从他的条件他的工作说起来,这个党的经验觉悟程度同群众的联系说起来,是个布尔什维克党,但还讲不到“化”,化就是彻底的很,就是完全的完整的布尔什维克,如斯大林所说的第一条,必须使得党不把自己看成为国会选举的附属物一一象社会民主党,在实质看成的那样,也不能看成职工会的不要钱的附属品一如果某些无政府主义工团敌对分子有时对于这总反复说明的那样,而应看着是无产阶级的阶级联合之最高形式,负责创造无产阶级底其它一切形式的组织一一从职工会到全国党团一一的使命。

这一条我党许多地方部分没有实行,或完全实行,我们是否国会党团问题?是否职工会无政府主义问题?我们不是国会党团,工联主义、无政府主义,所以我们的国会党团没有象斯大林所讲的脱离党把党团员看成附属品,而拿那个东西为主体,(重庆参政会边参会有党团)而凡以国会党团为主体,实际就是中央,在欧洲国会党团,就是党中央的现象,我们许多同志说这一条是否可要?有些同志说对我们不适用。我讲“党是一个阶级的最高组织形式”,这个最高原则领导一切形式的组织,一股组织形式在这所谓职工会,全国党团这个问题,我们有别的例子,如我们有政府军队,民众团体,党领导什么?就是领导政府、军队、民众团体,不领导别的。除外,还有党务工作,整顿党组织,整理党组织是为整理党务工作,整理党务工作是为把政府、军队、民众团体搞好,也包括干部工作。干部是政府人员,除了党务干部十分之九是军队和政府干部、民众团体干部,民众团体有工作,青、妇、合作社、文化团体(包括文学会、哲学会、戏剧社、木刻会等)把这些搞好,才能和敌人作斗争。‘

但是现在有这样一种概念,你这党部就管你的党务工作,我的政府、军队、民众团体,你少管,少管则不管,不管则反对,这些是什么人?是张国焘,他就是反对我们管,他自己搞个中央“陈桥兵变,黄袍加身”自己有军队,反对中央管他,而他要推翻这个中央。

另外,还有一个项英,中央命令他不执行,就是皖南事变这样的事。其他军队还没有这样的严重。张国焘是最标准最标准的,比较次的是没有公开暴露这些事情,再有就是服从一部分,不服从一部分,服我脾气我服从,否则不服从,口里没有讲,心里这样想,作起来也这样作,边区党政军队关系就是这样的现象,党的意志决议不能执行。如军民合作相当长期没有执行,不执行是原则的问题,步调不一致,政府与党、政府里工作的党员与政府领导不一致,群众团体之间也曾有这样的事,如青年工作闹独立斗争,跟党的政策不符,随便槁一套。文化团体也有这样的事,如解放日报第四版有个时期成为独立王国,什么也不能干涉,报馆的意志在这版不能执行。一版有一个字不服从就是闹独立性,有人说党一个字也要管,挖苦得很。党是管一切的,所以叫党“决定一切”。“九一决定中”已搞清楚。我们所谓党是党员组成的,一切党员就是党的组织,我这里叫军队、民众团体……实际是不要这个党,现在执行斯大林这一条,就是领导一元化,承认中央“九一决定”与这次会议中整顿关系的决议案,知道党是一切无产阶级的最高形式。

第二条讲理论,“必须使党特别是它的创造者完全精通马克思主义理论,这种理论与革命实际工作不可分离的联系着”。这条我党历史上有个过程,第一个时期很高的热情接受马克思主义,在马克思主义指导下,我们干了一个革命叫大革命,从有党到l927年前一段,准备大革命,后一段执行大革命,是马克思主义原则和思想指导下那时(1921)党成立,在十月革命之后,马克思主义在中国有很大的宣传,特别是“五四”后,有广泛的传播,“五四”以前中国工人阶级知识分子对马列主义是不知道的,十月革命本身就是极大的宣传,影响,中国许多知识分子倾向机会主义,加以世界大战,中国产生“五四”运动,以李大钊为首一部分知识分子,开始自觉地研究马克思主义,因此党能在21年建立。没有十月革命,“五四”的酝酿,党的产生是困难的。1921~1927年这六年中,幼年的党干了一件大事,这是在外国史上也是不多见的,就是大革命,国共合作,对马列主义起了很大作用,大革命是在国际指导下,按中国实际情况提出纲领进行活动,有些人觉得中国党开始没有理论,我们党较后一个时期是没有理论的,我们说21年有了共产党本身就是马列主义指导下产生的,不然怎么样设想呢?怎样忽然出来一个共产党呢?1921-1924年之间,大钊就宣传共产主义,大革命,内战、抗日三个阶段都是真假马列主义在斗争,朱理治文章是写“为马列主义而斗争”实际是“右倾机会主义”,什么人证明呢?被抓、被杀的那些证明,杀了240个马列主义者,还有一批,因为中央来了,未来得及杀,叫了声“刀下留人”又放了。这种现象不只是一个边区,特别是鄂豫皖,湘鄂西和这里严重,中央苏区,赣东北,四川也有,但不严重,这是个错误路线,枪杀了革命干部共产党员,把共产党员马列主义者杀了。苏区搞光90%,白区是100%,大革命后期陈独秀是假的马列主义,结果被正确的马列主义克服了,“八七”会议把他排除了,我党大革命时期,有向陈独秀右倾机会主义斗争,有向张国焘左倾机会主义的斗争,这都是马克思主义与非马克思主义的斗争。

遵义会议后,基于前者的失败,路线是比较正确的,有否缺点?开幕那天我说过党内现在有一种自由主义,又象中央所提出来的一样,我们反对主观主义,宗派主义、党八股,这些不是统治路线,但就是残余也是不符合原则的。对之可有两种态度,一是自由主义态度,隐瞒、包庇、不理,一种是向之作斗争。从去年七一一八月起,中央颁布了“增强党性的决定”,“调查研究的决定”,今年又提出了严整顿三风,“就是对不正确路线采取批评态度,遵义会议后,不是过火的斗争,而是自由主义,过去犯左倾错误的,遵义会议后就容易犯自由主义的毛病,朱、郭过去是“左”,抗战以来朱右,则工作闹独立性,郭敬散布谣言,挑拨离间。

党就是在马列主义指导下建立的,党在三个时期中克服了错误路线,但现在又产生右的倾向。教条主义是什么?就是拿马列主义对马列主义釆取自由主义,以马列主义作招牌,懂了就是不作,我们中国教条主义有两种,一种咬文嚼字,他也作,错了就乱搬公式,还有一种是不作事,只读书,读了好多,一箭不放。一种是“无的放失”,一种是“有箭不放”当作宝贝古董,延安发现的是后一种,读了很多东西,对边币和法币不能解释,是脱离现状的,边区党员有名为党员,实际与党不一致,还包括一部分反革命奸细,托派分子,如吴契如就是这样的人,是个文化人,皖南事变时,国民党抓了他,把他放出,叫他到这里来,现在已由新四军打电报来证实他被俘过,他也承认怎样当特务。王实味最近也发现了,他以党的资格在这里讲话,组织了一个五人反党集团,(中央研究院政治研究室)就是王实味,成全、理璠芬、宋铮,他们说话作事不象共产党员,但有同志就不加区别,这是否自由主义?郭洪涛的破坏党、朱理治的闹独立性,以前不讲,这次会上讲了是马列主义。

斯大林要我们注意理论,我们与党校同志商量一下,党校准备读几本书,要读三十一一四十本,如果我们有这样丰富的经验,这样长的斗争历史,能读几十本马恩列斯的书,就把党武装起来,朱理治那一套能卖?有人陪?就是我们理论弱。

第三条必须使党在制定各种口号和指示时,不是根据熟读了的公式和历史类比,而是根据对革命运动的关系条件、国内外具体条件周密分析的结果,同时顾及各国革命的经验。斯大林这一条很易理解,有一种是公式历史类比,另一种是根据革命运动的历史条件,如口号中有一个“打通国际路线”,把区三千人,一个个括起来也不能打通国际路线,从洛川到郦县六十里,恐怕也拉不够一个日本。也不过这么宽,(笑声)还有一种口号要打“中心成市”,拒绝作“会门工作”、“土匪工作,刘志丹曾提出作“会门工作”,“土匪工作”,把全国的斗争形势变成我们有利的斗争形势,要和军阀笼络一下。朱理治就说是和军阀勾结,我们不是和军阀勾结,而是利用,那时全国积极要求抗日救国,而我们打击在野党派,在朝的打不倒,在野的人家去经在野了,你还要打在山里办工会,世界上那能出这样的文章,他们引用外国历史上的类比,外国在苏联打过彼得格勒,于是我们也可以打中心城市吧?斯大林同志讲到要照顾外国的经验,他是把国际国内当前具体条件作具体的分析放在第一位,国际经验要照顾,但要恰当估计,不是硬搬,过去的缺点就是硬搬,在外国有清党起动,我们也要把党清刷,要釆取外国经验,但主要应从国内具体情况加以周密的具体研究,如王实味确实是托派,吴契如确实是特务,边区确有部分人是党棍坏人。

现在我们看今天有何口号?有抗日统一战线的抗日口号,制定设有?制定的条件是国共合作,各阶级合作,过去提出过不分贫下中农的口号,今天提出了减租减息、交租交息及统一战线口号,这些根据当前国内外具体条件的具体分析,今天实行“三三制”、“十小时作制”、“精兵简政”是糊里糊涂提出来的还是确有需要?项英同志很早就提出“精兵主义”这在当时是不恰当的,现在提确是恰当的,是在这样的具体条件下,加以周密分析提出的“精兵简政”,不加调查研究、周密分析就不箭进行彻底。

没有这一条就要以洛川为中心向三边发展,就要打通国际路线,在稍山里建立工会,搞集体农场,打大城市、再搞反马列主义。斯大林这条就是要我们作具体分析,同时又照顾到外国的经验,工作就会作得好,反之就作不好。制定口号凡不符合实际,制定政策指示一定行不通,作不好,所以这是一个方法论。

第四条:必须使得党在群众革命斗争的烈火中检查这些口号、指示的正确性。这条和十二条都是讲检查,这个检查主要检查口号、指示,实行原则路线。后面是讲检查工作作风这条路线的执行,具体的工作。第三条讲要根据具体条件周密分析,制定口号指示这是个方法论,单有这部分是不够的,还要在实际中得到证明,理论是实际中抽出来的。第三条讲调查研究,实际的反映指示和口号。朱理治也讲根据实际,他是根据国际形势有两个世界对立,国内形势有两个对立等,在实践中检查结果,白区100%,边区90%搞掉了,从革命烈火中检查这些口号,国共十年内战可谓烈火矣,检查那时的口号,指示是不正确的,象打中心城市、稍山里建工会、国际路线、肃反、都不正确,刘志丹是正确的,朱理治、郭洪涛不相信,我们试了看,信得反不得,抗日民族统一战线、三三制、精兵简政,究竟跟实践合不合,要检查,如我们过去在党内的教育制度,学校的教育制度,合不合教学方法,结果是教条主义的,是否正确,最近证明,要在你去作,对了就对,过程中不对就要参考,不能讲中央一切都对,这样讲的人是盲目的,是一个没有觉悟的人,说现在中央一切都对,是因为过去有那些样子的经验,那样大的胜利也犯过许多错误,彻底正确只有在实行之后,理论是从容现实践中抽出,从客观实践中得到证明的,这是第四条,你的客观实践的态度,实践是检验真理的标准尺度,用什么采衡量真理?就用实践,所以说不在他的决议签名,而在他的行动,检查一个干部、一个党的重要标准,要看他实行的结果,有第三条而无第四条是不行的。

第五条必须使得党的全部工作一一特别若是党里的社会民主党传统还未消失的话一一改造过来,建筑在新的革命的步调上,使得党的每一步骤和行动自然而然走向群众革命,走向革命的精神上培养和教育工人阶级的广大群众。

这是讲工作作风,原意是不要社会民主党的作风,不是改良而是革命的作风,因为社会民主党,那时就没有革命的作风,应该把党的作风改造过来,放在新的革命基础上,新的革命步调上,新的精神上,布尔什维党是从第二国际分裂出来的,第二国际是改良主义统治的,第三国际是从第二国际分裂出来的因为他不革命,条件变化了,是无产阶级革命的时代,社会民主党是改良的作风,他没有办法领导无产阶级革命,所以要有一个新的党,列宁就是建立一个新的党,各国按布尔什维克的作风,建立共产党。如××说的,我们党内无社会民主党的传统,中国党有两种缺点:一是左,二是右,我们今天讲的这一条,就要反对自由主义,现在党内自由主义相当严重浓厚,同时也要反过左,如征款工作不调乱派,这就不能使群众自然而然的革命化,党内有自由主义,有很多坏蛋,有王实味,吴契如,我们中央研究院(过去的马列主义学院),缺乏革命作风,怎样使他们自然而然的革命化呢?具体办法是整顿三风,过去很多在工作岗位上不安心,因此就来一个适当的解决,就是整顿三风,自我批评,这样他们就安心了!我们没有社会民主党,但有新的问题,自由主义,我们要新的革命作风,培养教育广大群众。

第六条,必须使党在给自己的工作中把最高原则性(不要把这和关门主义混淆)和与群众最大限度联系及接触(不要把这与尾巴主义相混淆)相结合。这里讲要把革命的原则性和联系群众配合起来,否则不仅不能教育群众,而且不能向群众学习。你讲要教育群众,要向群众学习,要倾听他们的呼声。我们对群众的关系是一方面要教育他们,另方面要向群众学习。不这样想要把革命的原则和联系群众结合起来是不可能的,不联系群众就不能教育群众,不能向群众学习,不联系群众怎能提高,怎知人家的迫切需要?人家须要打游击?在稍山里你去建工会,人家需要是分配土地,你搞集体农庄;人家需要十小时工作制,你搞八小时工作制。这第六条是讲群众工作的问题,斯大林告诉我们对于群众工作,要把群众的日常生活的要求,同基本要求联系起来,基本的要求是最高的原则,但是用什么方法,天天搞社会主义集体农庄、八小时工作制,尤其是现在的中国环境要求低,破除迷信是最高原则,但话一说出来就到处破除迷信,到处打庙就发生了问题,他们的要求是因为生活条件、经济条件的限制,比如外国人开洋船,中国人民是开木船,开木船就要信龙王、菩萨,开洋船的就不信,木船不信翻了不得了。破除迷信,婚姻自由、民主集中制、社会主义集体农庄、打大城市等最高原则,是我们党建立最高的目的,忘记了就不算是共产党员,可是还有一条,一定要按照群众的要求,今天能够办到,可能作到的那样来作,这些才算与群众联系,才算与群众接触。彭湃是共产党员,中央委员,他在海陆丰拜过菩萨,这就是群众工作的原则,要把这两个原则合起来,一是最高原则,一是联系群众,最高原则不要变成关门主义,郭洪涛、朱理治他们搞关门主义。联系群众也不要变成尾巴主义,祭菩萨一下还可以,天天祭就成尾巴了,相信菩萨要把握住原则,不要关门也不要尾巴,要作两条路线的斗争。

第七条:必须使得党在自己的工作中,善于把不调合的革命性(不要把这个革命性与冒险主义混淆)和最大限度的灵活性机动性(不要与迁就性相混淆)相配合,否则不能掌握各种斗争与组织形式,就不能把无产阶级的日常利益和革命的根本利益结合起来,并且不能把合法斗争与非法斗争配合起来。第六条着重群众工作,第七条讲统一战线原则,讲抗战的战略战术,斗争形式、组织形式,这就是战略战术的问题,(第二条讲理论,第三条是唯物主义)讲抗战的斗争形式或组织形式,因为我们不但有基本群众,而且有一部分的阶级,别的集团,可能联系就同他们联系,这里有最大限制的灵活性、机动性,要把不调合的革命同灵括性配合起来,我们曾打倒地主,现在要联合地主,过去曾没收地主的土地,现在变减租减息,交租交息,这就叫灵活性,机动性,决不是丧失革命性,托派就讲我们投降了资产阶级国民党,过去和国民党打了十年,今天日本来了,又同他联合,托派说我们投降了资本家,所以“左派”就拿这话骂共产党。列宁作一本书叫《“左”派幼稚病》,那书就是讲第七条,告诉党员要革命性同灵活性配合起来,那时候英、法、德、苏、意的共产党员,他们当中有些人提出打迂回,想在一个早上革命成功。莫斯科的革命成功了,我们还作什么国会工作,他们就反对作国会工作,他们反对利用国会作讲台,反对同社会民主党左翼联合,一九二一年四月列宁作了这本书,郭洪涛、朱理治过去没有看过这本书,也不重视这本书,看也是眼花了没有看进去,那上面很多东西都讲了,第三条那上面也讲了只有革命性,那么有否可能来掌握各种斗争形式、组织形式呢?没有可能,我只讲到这种形式是我们主要的斗争形式,武装革命形式八路军、新四军要不要灵活性呢?我们在开参政会,开国民参政会,要不要呢?还是要,要去重庆开参政会,就拿日占领区、沦陷区来说,八路军在华北,新四军在华东打,天津、上海、华中日本占去了,人家占住了,我们打不了,我们有一种办法,叫合法斗争,我们可用伪军,在伪军中作工作,日本不发粮不让办炒菜馆子,就起来发动斗争,提高群众革命积极性,合法与非法要结合起来,列宁在《“左”派幼稚病》中讲布尔什维克三次革命的经验比世界上任何一个国家都要丰富,有合法与非法,流血与不流血的斗争,布尔什维克一九○五年是非法的,一九○五年后没有苏维埃,只可作一些合法斗争,如合作社、黄色工会,那时××在搞黄色工会、合作社讲他是机会主义是不对的,在白区有条件也搞非法斗争,如暴动示威,只要条件可能,革命没有内部变化是不行的,中国的三打祝家庄,外国的新木马计,都是这样。只有一种斗争形式,有合法无非法不行,如欧洲共产党只有非法斗争,否认合法斗争是“左”倾,列宁批评“左”倾幼稚病,要把两种东西配合起来,要有灵活性联合整个世界反法西斯这个阵线,包括丘吉尔、罗斯福,还有中国的国民党,德苏协定后并没有放弃和英美的关系,苏联的大使馆还在英美,德苏协定破裂时,英美苏协定建立起来不釆取这样的组织形式、斗争形式就不行,有那些组织形式?英美协定,工会青年大会,美国也有个青年大会,援军青年大会,有中国代表、英国代表,这个组织形式要灵活运用,那些组织形式、斗争形式达到革命的目的,只有一种死板的固定的组织形式、斗争形式是不好的,斯大林专门一条是统一战线组织形式斗争的问题,要有革命性,同时要有灵活性。从前打倒国民党,西安事变就变化了,人家进行反共高潮,跟我们斗争,我们只好斗争,斗争形式转变时,我们立刻放弃斗争,阎锡山搞新军时,我们不能不援助,斗争手段是为了达到团结,要有理、有利,有节。

为什么原则性不要与关门主义混淆,最大限度和群众接触为什么不要和尾巴主义相混淆?又说不可调合的革命性不要与冒险主义相混淆?最大限度灵活性、机动性不要和迁就性相混淆。

最大限度的和群众接触,同群众联系的最高原则性,这是讲我们作群众工作,要把群众提高到党的水平,这就是有最高的原则性和最高的觉悟,但不要离开群众,不要关门主义,所以讲不要和关门主义混淆,关于原则性,中央苏区有个学校,对吃辣椒的也开展斗争,提高到原则高度,说吃了就是有原则,那就是乱用原则,是关门主义,因此要善于区别原则性与关门主义,至于联系群众,又不要忘记这个原则性,不能因为今天老百姓迷信,我们就不能作破除迷信的工作进行宣传,边区有三万党员,只有个别的迷信,不赞成婚姻自由,但不能作尾巴,因此不能混淆。讲最高原则,但不是脱离群众(第六条解释)。“不可调和的革命性和最大限度的灵性、机动性”是讲革命和妥协的关系,“左”派幼稚病那本书已讲了,因为西欧的共产党,他们有这样的口号:“不作任何的妥协”列宁批评了,但西欧的共产党自认为很革命,这是冒险主义,不作任何妥协,可是群众的觉悟是这样,他们不利用国会,不利用合法斗争,不看群众的觉悟程度,只讲进攻,这就叫冒险主义,而这是不要迂回。列宁讲俄国的历史说:“在他们的历史中,曾有过许多妥协,这一种妥协是灵活性,要有最大限度的灵活性、机动性,但非迁就。”第三国际就是迁就,党要和别的阶级、成分进行妥协,合作,甚至利用敌人,比如俄共利用沙皇的国会,因为法律上允许共产党员当议员,这时全国没有大的革命可以利用,不利用就不是革命,苏联和德国法西斯妥协过,就在去年六月曾订过苏德协定,我们这个殖民地、半殖民地的党在十五年前就和资产阶级妥协过,全世界恐怕中国第一次和资产阶级妥协,共同进行反帝反军阀的斗争,现在全世界除法西斯国的资产阶级外,统统同他们进行妥协,在中国是同国民党、地主,边区是同地主妥协的,张国焘就是完全丧失自己的立场。革命完全不管,跑到汉口在大公报上签了宣言,他讲的迁就是完全不要原则性,完全不要革命性。

第八条:必须对党不隐藏自己的缺点和错误,不怕批评,而且要善于在自己的错误上改善和教育自己的干部。

第十条:“必须对党经常地改善自己组织的社会成分,清除那些腐化党的机会主义分子以便达到最高限度的一元化。”这两条在联共党史结束语中也分作两条,关于反对机会主义,关于我们队伍中间有错误应该批评,这两条应该分开,这是第一条,与联共党史结束语有点出入,不同,大体意思差不多,联共党史结束语,一共有六条,第一条讲党要有一个革命党,同这十二条的第一条一样,基本上是一样的,第二条要精通马克思主义理论,要有革命的理论,跟这十二条第二条一样,第三条讲工人阶级的统一,工人阶级队伍中间应该是统一的,那些欺骗工人阶级的政策实际上是反革命的政党,应该跟他作斗争,要有革命的党,不是社会民主党,是共产党,是列宁式的党,他们有马克思列宁主义的武器,这样的党为马列主义的党,工人阶级要清洗自己队伍中那些反革命分子,与它作斗争,使工人阶级统一,使党统一。第四点是讲党的统一,我们是讲党的一元化,其次教训,我们说党不同自己队伍中的机会主义作不调和的斗争,不粉碎投降主义者就不能保存自己队伍的统一和纪律,不能实现其为无产阶级组织者的作用,就不能实现为社会主义建设者的作用。我们内部生活发展的历史乃是反对党内机会主义集团,经济主义者,孟什维克,托洛茨基分子、布哈林的民族主义倾向而斗争并粉碎之的历史。有人也许觉得布尔什维克化费大多的精力进行党内机会主义的斗争,也许觉得布尔什维克党过分估计这些机会主义分子的意义,这是完全不对的,在自己的内部容忍机会主义正如容忍肌体内毒疮一样,党是工人阶级的先进队伍,是工人阶级的前方堡垒,战斗的参谋部,不能允许那些缺乏信念者、机会主义、投降主义和叛徒,和他们同资产阶级作悬死斗争,那就陷于背叛受敌,堡垒是容易从内部夺取的,因此要从党内清除这样人,吴契如以党员面目给国民党作事,这叫机会主义吗?王实味是个托派,在这里组织反党集团,这一类叫叛徒,这次开会清算了过去的历史,朱、郭在边区搞形成前边是敌人,后边也有敌人;堡垒参谋部的有敌人,抓了许多人,把刘志丹也抓起来了,差点被杀了。你说危险不危险?我完全同意张×山的分析,他们前途有三:一,前途是反党,退出共产党走到反革命方面去;二,前途继续是两面派;三,改正错误,联共党史结评第五条说:如果被成绩熏醉而骄傲起来,如果党已不愿再看见自己工作中的缺点,害怕承认自己的错误,纠正错误,就不能实现其为工人阶级领导者的作用,这里讲应该看到缺点不骄傲,承认缺点并及时纠正,公开的承认,不要在小屋子里几个人讲,要在犯错误中教育自己,党是不可以战胜的,否则就要自取灭亡,这不是叛徒的问题,而是工作中个别错误的问题。这次要清算过去历史问题,闹独立性,自由主义、要精兵简政”整顿三风、就是要纠正自由主义、主观主义、宗派主义、党八股,犯错误的人应该不隐瞒自己的错误,不应用虚夸来隐瞒缺点,能这样作的党是不可战胜的,列宁说:“政党对自己的错误,所把的态度是重要最可靠的尺度,以考虑这个党是否慎重和他在实质上执行自己对本阶级劳动群众的义务。”对错误所抱的态度,是考虑这个党的性质的问题。

第八条就是党史结束语的第五条,“必须使党不掩盖自己的错误,不怕批评,善于在自己的错误上改进和教育自己的干部。”这里要说:宽大政策在党外不同,一九三八年三月党中央有个决定要发展党员,同时说不要使一个坏分子进来,这叫开门政策,同时又关了一掮门,对坏分子关了门,我们的施政纲领是对外的,对外要采取宽大政策,俘虏、特务,只要不坚持作坏事,愿意改正错误,应采取宽大政策,但不是讲对任何反革命都是宽大政策,日本的飞机可以随便在这里降下来,日本人可以随便来这里开会,这是不行的,我们的宽大政策在实行中有了毛病,有了自由主义,把宽大政策变成了自由主义,以致相当多的人弄不清吴契如、王实味,我们要审查各个部门有否这样的自由主义?很多事项不加解释,没有斗争,一团和气,领导与被领导之间,有个原则,第一叫团结;第二有错误要斗争。要正确的批评和自我批评,这就是第八条,联共党史第五条,我们有些同志不善于发现错误,不愿公开承认错误,对钻进我们党内的奸细、反革命也釆取宽大政策吗?我们现在整顿三风、精兵简政是两种斗争,一种是无产阶级思想与小资产阶级思想的斗争,还有一种是革命对反革命的斗争,对大多数是前者,少数个别是后者。

有些同志受批评,怕抹煞成绩,我在党校二月十日讲了整顿三风,党八股的报告有万多字,但成绩只有百多字,那是否抹煞成绩呢?不能这样提,我们现在是作自我批评,“九一决定”估计了我们的统一战线,估计了我党抗日以来根据地领导一般是统一的,团结的,党政军民各组织关系基本上是团结的,因而支持了八年抗战艰苦斗争的局面,不统一团结就支持不了,列宁讲:“我们要公开承认错误,揭发这些错误的原因,分析错误的环境,产生的条件。”

第九条:“必须使得党善于把先进战斗员中的优秀分子选拔到基层的领导核心中去。”一切单位要注意领导核心的建立,没有核心就不好办了,不能每个人都是领导核心,没有领导核心怎样领导呢?领导核心是斗争中形成的,领导核心的条件,斯大林讲有两个斗争,就是十分忠诚,十分有经验。忠诚就是这样的忠诚,是以成为无产阶级的代表,经验是领导阶级斗争,因此要善于运用马列主义的策略和战略。朱、郭要独立性,就没有资格代表无产阶级。

第十条:斯大林说:“必须使党经常地改善自己组织的社会成分,清除那些腐化党的机会主义分子,以便达到最高限度的一元化。”讲改善自己机会主义成分,就是说要洗刷坏分子出党,经常清洗,经常吸收。

第十一条:“必须使党创立起无产阶级铁的纪律。”以此区别社会民主党,区别自由主义,这种自由主义严重的破坏了党的纪律,自由主义发展了闹独立性、小广播、讲价钱,调工作不动,不服从决议,讲了不作,见坏分子不批评,见坏思想不斗争,这种自由主义发展了,就没有纪律,创立这样的铁的纪律,斯大林讲要有这样的条件和基础,就是若干思想的统一,运动的目的明确。现在我们运动的目的,抗战,团结,部分人想打出去则与此相反,整顿三风也算一个运动的目标,整顿三风的目的有两个,一个是无产阶级思想。一个是克服暗藏的反革命,目标要明确整三风开始是搞了一个万人,后来目标讲清楚了整三风是整顿全党,这样就建立起纪律,要进行纪律,运动目标明确,设法整风,精兵简政也是一个运动,它实行在困难时期,制度要精简节约,统一效能、思想不要闹独立性,现在是困难要反官僚主义,这样就把精兵简政这个目标破了,斯大林讲纪律基础基于什么?纪律是逐渐凑成,不是天上掉下或下一个命令,而是思想统一、运动目标明确,实际行动统一,广大群众对运动要自觉性,作到这些就要有铁的纪律,否则就象豆腐。

第十二条:必须使得党有系统的检查自己的决定和指示的执行,斯大林把这二条放在最后,就是要我们经常检查工作。

这十二条是全世界共产主义运动一百年的经验,从一八四三年到明年(一九四三年)就是全世界共产主义运动一百年了,现在我们整顿三风,这些文件不仅总结了中国廿一年的经验,而且总结了共产主义运动的百年经验,斯大林在一九二五年写的这个文件也十多年了。没有这些经验联共党史结论的第三、四、五条就难写了。

关于第一条要有革命的党,要有领导一切的党,这一条相当于联共党史结束语的第一条,我们的“九一决定”,我们要统一党的领导,还有一个“增强党性的决定”,这个决定中讲,不要闹独立性,这次高干会议又整顿各级组织关系,叫做整顿关系,这是我们对于这条怎样实行的。

关于第二条,我们要精通马克思主义,精通实践联系,不脱离实际学马列主义,相当于结束语第二条,第二条讲得详细,这里只是一两句话,季米特洛夫论干部教育政策,在讲到教育政策时,告诉我们要教育些什么东西,他说有两种教育方法,一种是教育式的,一种是真马克思主义的,六中全会论“马克思主义中国化”有“中央关于干部学校的决定”有“在职干部教育决定”这两个决定都提到怎样学马克思主义。

第三条就是我们的口头指示,要根据具体的条件分析,我们有一个“调查研究的决定”,反对那种根据公式的公式主义,根据历史的类此来制定口号政策。第三、四条,是讲方法论,制定口号指示的方法论,如何指示,如何检查,第三条制定,第四条就是证明,釆取革命步骤,使群众自然而然的革命化,而不要那一种与宣传指南相反的那一种宣传,那一种态度。我们说反对党八股,我们还有“四三决定”,这个决定上面有很多步骤,决定讲的,各机关已实行了,在开始,同志站的是反革命立场,现在逐步转过来了。

第六条讲群众工作,讲群众工作的原则性同群众落后怎样适应,群众落后我们要讲原则,就是原则性,不要脱离群众,这是不要关门主义,讲原则不是关门,讲接近群众不是尾巴主义,对此中央有关“关于群众的决定”即联系第六条,边区如德源、陇东怎样坚持原则性,坚持原则性要密切联系群众,不是关门主义,又不是尾巴主义,把这提高一步,有了经验。

第七条不要冒险主义,又不要迁就行为,这就叫斗争形式和组织形式是讲统一战线。这个有左派幼稚病,有季米特洛夫七次大会的报告、有中央关于国共合作的各种条件,从“八一”宣言起,在反磨擦中,在反共高潮中,我们有许多经验。我们的“三三制”,我们的土地关系,对土地关系的处理,劳资关系的处理,除奸政策,各种政策等,施政纲领与斯大林所讲的这一条这一类性质,不要冒险也不要迁就,冒险主义不善于革命性;而最大限度的灵活性,决非尾巴主义,研究一下我们的工作,自己的历史,关于国共合作中期,最近三、二年,现在实行的“三三制”、土地、劳资、除奸政策,有许多同志不满意,我们应动不动,应捉的不捉、土匪闹的厉害,还要想宽大政策是斯大林讲的,十二条是没有的,土匪天天打,打到瓦窑堡、延安附近,还讲宽大政策,他那种宽大政策那来的?人家打得你要死,群众到处叫,还说宽大政策,施政纲领,这对觉悟的讲宽大,人家天天整你,你讲宽大,这样日本飞机还来加油,你还讲宽大,就是要检查这样的事。

第八条:要讲自我批评,改正错误,列斯都论它,这里引列宁的自我批评结束语第六条“四三决定”,决定告诉我们怎样作自我批评,有“论党内斗争”,有反对自由主义,党内几种不正确的倾向,都属于第八条,现在的整顿三风,属于这种性质,所谓正确的干部政策,究竟是什么?糊涂的团结,无原则的团结、缺乏自我批评的团结,是团结不了的,这样的党内宽大政策是非马列主义的,缺乏批评斗争性的,这样的干部政策是不正确的。

第九条讲党内要经常改善我们党内成分,清除机会主义,使党达到高度的一元性,季米特洛夫论干部教育政策,有四大标准,也可说是干部四大条件,第一要忠诚,一条是经验,忠诚自己讲不行,要在法庭上、战斗中、艰苦工作中考验。第二条是讲联系群众,嘴讲不行,要群众自己觉得你是他们的领导,只要眼睛多看一些、耳朵多听一些,夸夸其谈就少了,人家就相信你了。第三条是有独立工作能力。能即有经验,会领导无产阶级斗争。第四要守纪律。

第十条讲经常性清洗社会成分,洗刷机会主义分子,达到高度一元性,是结束语第四条,我们有“中央巩固党的决定”,“怎样做一个共产党员”,巩固党就要清洗坏分子,经党改善党组。党员应是什么人?党的文件回答了这个问题,要对党的敌人应该怎样与之斗争,要清洗。

第十一条,讲纪律,革命党没有纪律不成,斯大林讲思想的一致性,目标的明确性,实际工作的统一性,党员的自觉性,这些就生长在纪律中。列宁论党的纪律与民主,我们有这样的文章,是列在工人阶级的文章中。

第十二条讲检查,斯大林论领导及检查,他那样的文章在马克思主义时代是不可能的,如整党、整政、整军、整财、整×、整关系是有系统的,而且是有威信的人,而不是普通人,林老,贺老以及其他许多人,你们是各地方有威信的人,要各部门的主要领导去检查,“四三决定”上讲过这样的问题,因此十二条须得好好学习、研究、这是我们全党的圣经,不是教条,每一条都可以使用,可是有些情况,象社会民主党,我们中国没有,我们更可能使用这一条,那么我们更要布尔什维克化一步,这个“化”要自觉。

我讲这十二条,斯大林也讲了,而是在许多年前讲的,我们自觉的去作工作,学习马列主义,干部要有计划的读三、四十本书,不读则已,有了经验的人,读了就通了。我们是土包子怎能读呢?恰恰是土包子能读得通,容易懂。这次整顿三风的文件是一部经书,二十二个文件再加上三个,二十五个,不到十三万字,读八个月,有工作经验的人读时就钻牛角,为什么?是精神实质,要让他们去钻,再到工作中去,朱、郭在过去没有工作经验,过去大概读过几本书,到这里就大摇大摆为马列主义而斗争,这个文件有一、二万字,他的马列主义从那里来的?是那国的列宁呢?就等于没有读书,我们读不是装腔作势,借以吓人,因为我们不是朱、郭那样的人,有实践经验的人读了马列主义,就会同群众接近,不接近就是脱离群众,就是没有谈进去,那就是反对列宁。因此一定要有信心,有的人是走马观花,现在要下马看花,一年一本就是三四十本,在学校学习一星期,读一,二本,转来工作一段再翻,要今天翻一下,明天翻一下,慢慢的来,请同志们考虑。

