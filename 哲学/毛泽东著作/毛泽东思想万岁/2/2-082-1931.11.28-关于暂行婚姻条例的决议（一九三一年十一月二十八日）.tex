\section[关于暂行婚姻条例的决议(一九三一年十一月二十八日)]{关于暂行婚姻条例的决议}
\datesubtitle{(一九三一年十一月二十八日)}


中华苏维埃共和国中央执行委员会第一次会议通过

在封建统治之下,男女婚姻,野蛮到无人性,女子所受的压迫与痛苦,比男子更甚。只有工农革命胜利,男女从经济上得到第一步解放,男女婚姻关系才随着变更而得到自由,目前在苏区男女婚姻,已取得自由的基础,应确定婚姻以自由为原则,而废除一切封建的包办、强迫与买卖的婚姻制度。

但是女子刚从封建压迫之下解放出来,她们的身体许多受了很大的损害(如缠足)尚未恢复,她们的经济尚未能完全独立,所以关于离婚问题,应偏于保护女子,而把因离婚而引起的义务和责任。多交给男子负担。

小孩是新社会的主人,尤其在过去社会习惯上,不注意看护小孩,因此关于小孩的看护有特别的规定。

此条例在一九三一年十二月一日公布实行

<p align="right">中央执行委会主席

毛泽东

付主席:项×张××</p>

(婚姻条例略)

(注):在原稿中凡原稿涉及“苏维埃共和国”或“苏维埃政府”的地方,均改为“红色政权”易不提“苏维埃”,相应的“苏维埃代表”则改为“工农代表”。其原因见《毛选》一卷《中国红色政权为什么能够存在》一文中注释(6),请同志们注意。

