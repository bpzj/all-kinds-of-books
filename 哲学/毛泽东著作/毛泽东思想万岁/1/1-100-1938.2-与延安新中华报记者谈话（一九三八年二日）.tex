\section[与延安新中华报记者谈话(一九三八年二日)]{与延安新中华报记者谈话}
\datesubtitle{(一九三八年二日)}


问题一:苏联现在的政治制度是以党专政吗?

<b>答:</b>苏联是工人阶级专政的国家,即工农社会主义的国家,但不是以党专政的国家;从马克思列宁主义的观点看来,阶级的专政与党的专政是有区别的东西,党只是阶级的最觉悟的一部分人的组织,党应当而且只能在无产阶级专政的国家中起领导作用,党不应当而且不能代替阶级去实行专政。因此,苏联的政治制度的创造者,列宁斯大林从来未曾将党的专政与阶级的专政混为一谈,而他们有时借用所谓党的“专政”这个名词时,不过是指的党在国家系统中的领导作用这一点而已。为的使你明了和相信我的这种说法,我乐于将列宁和斯大林关于这个问题所说的话介绍一点给你。为了记忆和了解便利,我特将斯大林所着“关于列宁主义的问题”这本书上对这个问题所引证列宁的意见和他自己的意见介绍一点给你,在这本书里,在“无产阶级专政系统里的党和工人阶级”这一节中,斯大林曾将列宁对这个问题的意见加以详明的发挥,例如他说:“比方,在我们苏联这里,在无产阶级专政的国家里,应当把这样一件事实看作党领导作用的最高表现,这事实就是我们的苏维埃的或其他的群众组织,在解决任何一个重要的政治问题或组织问题的时候,都要根据党的指示”。在这个意义上可以说:“无产阶级的专政,在实质上,是他的先锋队的‘专政’是它的党的‘专政’,因为党是无产阶级的基本的领导力量。”请看列宁在共产国际第二次世界大会上关于这个问题所说的话:“田纳尔说:他是主张无产阶级专政的,但是他所了解的无产阶级专政并不是全和我们所了解的一样,他认为我们所了解的无产阶级专政,在实质上,就是无产阶级的那个有组织的有觉悟的少数的专政。而且真的是,在资本主义时代,当工人阶级是乃受着不断的剥削而不能发展自己的人类的才能的时候,工人政党的最大特点,正是在于这政党士能包括本阶级的少数,政党只能团结本阶级的少数,正好象在无论那个资本主义的社会里,真正觉悟的工人,只是占全体工人的少数。因此,我们不得不承认只有这个有觉悟的少数,才能够指导和引导广大的工人群众,而如果田纳尔同志说他是政党的仇敌,而同时又主张由工人阶级中少数最有组织的和最革命的分子来给整个无产阶级指导道路,那么,我就要说,在我们彼此中间,实际上是没有什么意见分歧的。”(请看列宁全集第二十五卷第三四七页)

“可是这是不是说,可以在无产阶级的专政和党的领导作用(党的‘专政’)之间放一个等号呢?可以将前者与后者看作一个东西呢?可以将后者代替前者呢?当然不是这样说。当然是不可以的。例如斯大林同志就说过是无产阶级的专政就是我们党的“专政”(请看关于“列宁的学说”P.95)这个说法显然是把“党的专政”和无产阶级的专政看作一个东西。从列宁主义立场上看来,可不可以认为这种把两个东西看作一个东西的事情是正确的呢?是不可以的。

由此可见,苏联实行的是工人阶级的专政,而不是共产党的专政。把阶级的专政和党的专政混为一谈,是既不合马克思列宁主义的思论,又不合乎苏联的实际。我想,对苏联这方面问题的了解,苏联政治制度创造者列宁斯大林自己的意见是最有权威和最有信任的意见。

问题二:虽然苏联不是以党专政,但苏联只有一个共产党存在的原因何在呢?

<b>答:</b>这一方面是由于过去俄国各党派长期斗争中人民×是选择的结果,即是在俄国人从长期斗争中厌弃了其他党派而只拥护共产党的结果,另一方面是由于在社会主义的苏联国家内部没有其他政党存在的社会基础。因而也就没有其他政党存在的必要与可能。为的使你能明了和相信我这种说法,最好也还是将斯大林自己关于这个问题的说法介绍给你。苏联只有一个共产党存在的这一点,斯大林在一九二七年九月九日与第一次美国工人代表团的谈话中,认得很明白。他说:“我们共产党的地位,全国唯一的公开政党的地位(一党垄断的地位)并不是造作而成的,也不是有意空想出来的,这种地位决不是故意造作地用行政的方法所能造作成的。我们共产党的垄断是从实际生活中自然地生长出来的,是在历史上逐渐形成的,这是社会革命党和孟塞维克党全党破产而下台的结果。”

在苏联,除共产党以外,无其他政党存在的社会基础,无其他政党存在的必要与可能,关于这一点,斯大林在一九三六年十一月五日关于苏联新宪法的报告中,特别解释过。他说:“未了,还有一类批评家,如果上边所述的一类批评家,责备苏联宪法草案是放弃工人阶级专政,那么,这一类批评家,恰恰相反,他们责备苏联宪法说它丝毫没有改变苏联的现状,说它没有放弃工人阶级专政,没有容许各种党派的自由,仍保存今日共产党在苏联的领导地位依然有效。同时这一类批评家认为苏联各种党派没有自由,乃是违反民生主义基础的一个标志。”

我承认苏联新宪法草案的确仍保留着工人阶级专政制度依然有效,同样也保全着苏联共产党目前的领导地位,而毫无变更。如果可敬的批评家认为这是苏联宪法草案的缺点,那么,只能对这一点表示惋惜而已。我们布尔什维党却认为这是苏联宪法的优点。

至于各种政党的自由,那么,我们对这个问题的观点略有不同。政党是社会阶级的一部分,是社会阶级的先锋部队,只有在有着对抗阶级,其利益互相敌对、彼此不能调合的社会里面,换一句话说,只有在有资本家和工人,有地主和农民,有富农和贫农等等社会里面,才能谈到数个政党和政党的自由的存在。在苏联,资本家、地主、富农等等这一类阶级早已没有了。现在苏联只有两个阶级,即工人和农民,这两个阶级的利益不仅不彼此敌对,而且恰恰相反,是互相融洽的。因之,在苏联并没有几个政党存在的基础,因之,也没有这些政党自由的基础。在苏联,只有一个政党一一共产党的基础,在苏联只有一个政党一一即勇敢和彻底保障工农利益的共产党才能存在。它对于这两个阶级的利益,保护得并不坏,这一点未必有人可以发生怀疑的。

人们都说民主,可是什么是民主呢?在资本主义国家里,有彼此对抗的阶级,那个的民主,归根到底,乃是对于有势力的人的民主,乃是对于有财产的少数人的民主。在苏联的民主,恰恰相反,乃是对于劳动者的民主,亦即对于一切人民的民主。由此可见,违反民主主义原则的,亦不是苏联新宪法,而是资产阶级的宪法。因此,我认为,苏联宪法,乃是世界上唯一彻底民主的宪法。

由此可见,一党存在的事实,只有在这党以外的其他党派已在革命过程中完全为人民大众所鄙弃,所推倒和没有其它党派存在的社会基础的国家一一如社会主义的苏联一样,才能够真正形成和做到。

问题三:为什么德国意大利都是一党专政呢?

<b>答:</b>首先必须把国内一党掌握政权与一国内是否只有一党存在的问题分别清楚。在德意两国的确只有一个政党掌握政权,但这不是说,在这些国家里只有一个当权的政党存在,在德国,固然今天当权的只有法西斯主义的国社党,但是在法西斯上台执政以前的其他各政党――曾在末次(一九三三年)国会选举时拥有千百万选民的社会民主党,拥有5—6百万选民的共产党,拥有数百万选民和有历史传统的各种民主制度的资产阶级各党派一一都是继续存在和活动着,不过是不公开不合法的存在和活动着而已。请任何人到德国的城市农村里去仔细考察一下,他便立刻可以看出集中监狱里虽然囚禁着几十万反政府党派的分子,但社会民主党,共产党和一切拥护民主政治反对法西斯专政的党派,都还潜在地英勇地斗争着,他们秘密地出版自己的报纸杂志,他们痛苦地进行教育和组织民众的工作。在法国、捷克斯拉夫、比利时等国的德国政治侨民正在联合国内自己的党派,为设立反对法西斯主义和拥护民主政治的人民阵线,在社会民主国际(即第二国际)在共产国际领导机关里,均派有德国社会民主党和德国共产党的代表。意大利的情形和德没有两样,共产党,社会民主党和一切民主主义的党派同样英勇地秘密地进行反法西斯党的活动,这些党派的代表,同样在第三国际和第二国际领导机关内占着重要的位置。由此可见,如果说在德意两国内只有一党掌握政权,只有一党有合法地位,这是对的;如果说在这些国家里只有法西斯蒂党存在,那就完全不合事实。

问题四:有些人说现在国民党应该实行“一党专政”,你对这个问题有什么意见呢?

<b>答:</b>对这个问题,我可以分两个方面回答你。

第一,如果你所谓的“一党专政”是指国民党一党掌握国民党党员,或国民党自愿指定的人物,中国的其他各党派,现在均无代表参加政府。我们中国共产党是国民党以外的我国第二大的政党。我们今天并不要求参加政府。关于这一点,我们党领导人陈绍禹同志,在其去年十二月二十五日与美国记者白得思先生的谈话中,以及我们党的另一位领导人张闻天(即洛甫)同志,在“解放”二十八期所发表的《巩固国共合作争取抗战胜利》一文中,已经代表中共中央坦白恳切的声明过。关于这一点,我可再代表中国共产党中央作一次郑重声明:我们中国共产党现在诚恳地帮助国民党对日抗战,但我们现在并不要求参加国民政府。

第二,一党掌握国家政权的所谓一党执政,并不一定要釆取“专政”的办法。如果所谓“一党专政”的办法,实际就是对于国民党过去十年所实行的政策,那我以为有考虑的必要。在我看来,今天国民党是可以维持一党掌握政权的局面,但为了集中抗日救国的人材和表现抗日救国的民意,似应当釆取相当的民主办法,当然这些民主办法是绝对有利抗战的民主办法,是绝对巩固政府和更增加人民对政府拥护信任的民主办法。

问题五:现在有人说,革命党应当再实行“党外无党”的政策,换句话说,即是不允许国民党以外有任何其他政党合法存在的政策,先生对此问题有何意见?

<b>答:</b>只允许国民党一党合法存在,不仅不承认共产党和其他政党(国家主义青年党,国家社会党等)的合法存在地位,而且企图以武装力量去消灭国民党以外的其他党派,这在中国不仅不是什么新理论,而且是曾经实行过十年的旧的事实。然而这个事实的惨痛的结果,是不仅内部纷争不已,而且招致来空前未有的外患。十年来中国的实际政治生活的痛苦经验告诉我们:国民党企图用武力消灭其他党派的政策,已经遭受失败;同时,中共在中国现有条件下企图造成一党领导的政权,也未收到预期的效果。第一次因国共合作而取得北伐胜利的经验,十年来因国共分裂而形成严重民族危机的事实,教训了国共两党的同志和全国人民的一件苦的真理,即是中国统一局面造成的真实有效的办法,不是以某一党派反对或企图消灭另一党派的内争,而是把各党派力量在共同政治基础上形成民族统一战线,一一首先是国共两党的亲密合作。六个月来我国之所以能够实行对日抗战的根本前提,便是由于国内民族力量的团结和统一,而我国民族力量团结和统一的具体方式和具体内容,便是以国共两党合作为基础的各党派形成的抗日民族统一战线的建立――而是国共两党不仅相互抛弃了过去互不承认和互相对立的立场,而且在抗日救国基础上实行携手合作,即是国民党放弃过去否认共产党和其他政党的立场,实行一切抗日党派团结御侮的方针,共产党和其他反日的党派也放弃过去对国民党的立场,实行与国民党合作,去进行抗日救国的共同事业,由此可见各党派力量结成的抗日民族统一战线是中国对日抗战的必要前提,没有这个前提,破坏了这个前提,便是实际上使中国继续内战,中国一有内战,便无法继续对日抗战,这是显而易见,人所公认的真理。也正是因为如此,所以日本法西斯军阀若干军事侵略之外时刻企图使用“以华制华”的毒计,首先是企图再挑拨中国各党派之间的内争,以便破裂抗日民族统一战线。

由此可见今天有些人宣扬的不许国民党以外的任何政党存在的理论,实际上是中国历史事实已经否定了的理论,是使中国回到团结抗战以前的纷争局面的企图,同时就是使中国已由抗日民族统一战线而形成的统一局面不能继续,因而也是使中国再形成无力对日抗战的局面。因此无论宣扬这种理论的人口头上如何空喊“国家统一”,如果他们的理论不幸见诸实行,实际上所得的结果,一定是破坏今日既经形成的统一局面;因而无论宣传这种理论的人口头上如何高呼“抗日”,如果他们的理论不幸而见诸实行,实际上所得的结果,一定是破坏抗日团结,使对日抗战不能继续。正因为如此,所以我深信,这种实际上要使中国退回到国内团结一致对外以前的悲惨局面的理论,一定会受到爱国人民的反对,一定会受到全国抗战军队的否认,同时也就一定会受到国民党贤明领导人物和一切为国为民而领导继续抗日的国民党同志的斥责和厌弃。这是毫不足奇的,首先因为这种理论完全不合于实际,违反着实际。这些人说:国民党之外加其他党派也有合法存在的地位,便因党争而妨碍抗日;事实的证明是:原来只许国民党一党合法存在而不许其他政党有合法地位时,中国确实有极大的党派内争,以致中国末曾抗日一一顶多也不过局部抗日,以致中国无力进行全面的对日抗战;恰恰相反,当国民党允许其他政党有合法存在的地位并与其他政党合作时,中国的确停止了内战,消释了党争,造成了空前未有的国家政权和军队的统一局面,因而才能进行空前未有的神圣的民族自卫战争以保障民族的生存和争取民族的解放。这些人说:国民党以外,如允许其他政党有合法存在地位时,则多党合法存在的事实,便成为妨碍国家统一和破坏国家统一的因素。中国的事实证明是,只国民党一党有合法地位而同时不允许其他政党有合法地位时,国内纷乱到不断的国内战争;恰恰相反,国民党一一允许其他政党有合法存在地位时,国内统一的局面立见,各政党之间,对某些问题即有争论,也绝不会用武器作批评,顶多也不过相互以友谊的批评作武器;英美法比捷瑞等国的事实证明:多党合法存在的事实,绝未曾妨碍或破坏这些国家的统一。由此可见,只允许国民党一党合法存在,中国才能统一才能抗日的理论,是不合中国实际生活的理论,是使中国既不能真正统一,又不能真正抗日的理论,这种理论绝对不为中国爱护统一和坚决抗日的军民所接受。

问题六:现在有人说共产党既宣布相信三民主义,便不能再相信共产主义,先生于此问题的意见怎样?

<b>答:</b>首先我要声明的是,有人说共产党员宣布为三民主义的实现而奋斗,就是等于放弃自己的共产主义的信仰,我可以再一次代表我们的党郑重声明:这只是拨弄是非者的谣言,这绝不合乎事实。我们是共产党员,我们宣布愿意与国民党同志们一起去为中山先生的未竟的革命事业一即为中国的国际地位平等,政治地泣平等和经济地位平等的三民主义的事业而奋斗,同时,我们绝不会放弃我们自己虔信多年,并为之牺牲奋斗多年的共产主义。其次,有人说,如果宣布你们为三民主义实现而奋斗,同时又信仰共产主义义,这便不合乎三民主义的立场,因为三民主义与共产主义根本不相容。关于这一点,我可以再一次郑重指出:这种说法完全不对的。这种说法完全既不合乎首创三民主义和国民党的革命导师孙中山先生的理论和行动,也不合乎中国革命的实际情形。从理论上看起来,三民主义的民族、民权、民生等主要内容,与共产主义所主张彻底推翻帝国主义压迫,使中国人民达到民族独立;彻底摧毁封建压迫,使中国人民得到民主自由;彻底改造中国经济制度,使中国人民达到民生幸福这些思想恰能相容的。因此,共产主义与三民主义绝不是不能相容的,因而本党一与国民党也不是不能合法并存和携手合作的。关于这一点,中山先生在民生主义演讲中不止一次地说:“所以一讲到社会问题,多数的青年便赞成共产党,要拿马克思主义在中国实行。到底赞成马克思主义的那般青年志士用心是什么样呢?他们的用心是很好的,他们的主张是要从根本上解决,以为政治社会问题,要正本清源,非从根本上解决不可,所以他们便极力组织共产党在中国来活动,我们国民党的旧同志,现在对共产党生出许多误会,以为国民党提倡三民主义,是与共产主义不相容的。”(见《中山全集》第一集《民生主义第二讲》四十二页)特别指明是:一般地是由于这般人不了解三民主义的互相结合性,特别是由于这般人不了解民生主义。因此,在解释这种原因以后,中山先生在这篇演讲里继续说:“为什么我敢说我们革命同志对民生主义还没有明白呢?就是由于这一次国民党改组许多同志因为反对共产党,便居然说共产主义与三民主义不同,在中国只要实行三民主义便够了,共产主义是不能容纳的。然则民主主义到底是什么东西呢?我在前一次演讲,有一点发明,是说社会的文明发达、经济组织的改良和道德的进步,都是以什么为重心呢?就是以民生为重心,民生就是社会一切活动中的原动力,因为民生不遂,所以社会的文明不能发达,经济组织不能改良和道德退步,以及发生种种不平的事情,象阶级战争和工人痛苦那些种种压迫,都是由于民生不远的问题不能解决,所以社会中的各种变态都是果,民生问题才是因,照这样判断,民生主义究竟是什么东西呢?民生主义就是共产主义,就是社会主义,所以我们对于共产主义,不但不能说是和民生主义相冲突,而且是一个好朋友,主张民生主义的人,应该要细心去研究的。”孙中山先生在演讲中接下去发问说:“共产主义既是民生主义的好朋友,为什么国民党员要去反对共产党呢?这个原因,或者是由于共产党员也不明白共产主义为何物?而幸有反对三民主义的言论,所以激成国民党之反感。但是这种无知妄作的党员,不得归咎于全党及其党中之主义,只可说是他们个人的行为;所以我们决不能够以共产党员个人不好的行为,便拿他们来做标准去反对共产党,即是不能以个人的行为,便反对全体主义。那么,我们同志中何以发生这种问题呢?原因就是不明白民生主义是什么东西,殊不知民生主义就是共产主义。”从中山先生这段演讲里面,我们应当得出那些结论来呢?结论应当是:第一,三民主义与共产主义是能够相容并存的,三民主义与共产主义是一个很好朋友的关系,第二,认为三民主义与共产主义不相容的人,实际也是不懂三民主义的人,特别是不知民生主义是何物的人,这些人应该是国民党中的极少数,同样,认为共产主义与三民主义不相容的人,实际上也是不了解共产主义的人,特别是不了解共产主义与三民主义相互关系的人,这种人也应该是共产党中的极少数。而对于这个问题的解决办法,中山先生认为这只是某一方面的个人行动,任何一方面不能因此来反对另一方面的全党及其主义;第三,中山先生认为,不仅主张共产主义的人应该了解三民主义,同时中山先生指示:主张民生主义的人,也应该细心去研究共产主义。这是一种光明磊落大公无私的态度。我们共产党员非常赞美中山先生这种态度。的确,世界上任何一个伟大的思想和主义,不曾是某一部分人或某一个党的专利品。它应当而且必然欢迎别的任何人和任何党来研究实行的。我们不仅愿意为实现三民主义而奋斗,同样,我们更欢迎任何人一一首先是国民党的同志们遵照中山先生的指示来细心研究共产主义。因为我们深信:凡是愿意为人类解放事业奋斗而又真正研究和懂得共产主义(或马克思主义)的人,一定会承认马克思主义是伟大社会问题的最高理想,是集千年来人类思想的大成,共产主义社会是全人类最美满最愉快最幸福的社会,绝没有什么可怕的东西,正因为如此,所以我国近代最伟大的革命家中山先生才说:“……从前人类战胜了天同兽之后,不久有金钱发生,近来又有机器创出;那些聪明的人,把世界物质都垄断起来,因他个人的私利,要一般人都做他的奴隶,于是变成人与人争极剧烈的时代。这种斗争要到什么时候才可以解决呢?必要回复到一种新共产主义时代,才可以解决。所谓人与人争,究竟是争什么呢?就是争面包,争饭碗,到了共产主义,大家都有面包和饭吃,便不至于争,便可以免去人与人争,所以共产主义就是最高尚的理想,来解决社会问题的。”(见《中山全集》第一集《民生主义》第一讲第三十八页)对于马克思主义的价值,中山先生曾说:“至于马克思所著的书和发明的学说,可以说是集几千年来人类思想的大成,所以他们学说一出来之后,便举世风从,各个学者都是信仰他,都是跟着他走。”

从以上所说的一切,你可以看出,在中山先生亲手著作的三民主义理论中,绝找不出个三民主义与共产主义不相容的指示来,至于讲到中山先生对这个问题,在行动中表现更是尽人皆知的事实。中国共产党正式成立于一九二一年,自成立以后,中山先生与共产党员便有着亲切的关系,所以到一九二四年中山先生决心改组国民党时,便公开与共产党合作,而且合作的方式是非常亲密的,即不仅建立国共两党的国民革命联盟,而且允许共产党员以个人资格加入国民党组织中去,共同担任革命的工作。当时共产党在全国还不过几百个党员,成立历史不过几年,而且共产党员并以个人资格去加入国民党,去共同为国民革命而斗争,即在那种情况下,中山先生是否曾向共产党员提出过除三民主义以外不允许同时相信共产主义的要求呢?没有,即在那种情况下,中山先生是否提出只允许国民党一党存在,不允许共产党同时会法存在的主张呢?没有!不仅没有;而且中山先生还坚决反对别人提这种要求和这种主张。这就是中山先生伟大远见私洞悉实情之处,因为中山先生深切懂得,任何主义和政党的产生绝不是偶然的,而是有其社会基础和历史根源的,任何人的信仰,绝不是可以强制或取消的,古今来为信仰而杀身成仁的,不可胜数,任何社会基础和群众拥护的组织,绝不是可以强制解散或消灭的,古今来多少革命团体在万重压迫极端镇压下依然存在和发展,而现今的情形,比当时还更不相同,中国共产党已经有十七年奋斗的革命历史,有数十万党员,有久经战斗的坚强组织,有为主义、为党、为革命、为中国人民解放和为全人类解放而百折不屈,英勇奋斗的领导和干部,有千百万群众的信仰和拥护,那里能谈得到取消共产党员的共产主义信仰和党的组织呢。如果说只是因为我们宣布了我们愿为三民主义的彻底实现而奋斗一层,就应当取消共产主义的信仰,这完全是误会,因为我们对三民主义与对共产主义的相互了解,的确如中山先生一样,即我们认为他们是好朋友是相容的,所以我们共产党员不仅在第一次国共合作时,曾以共产党员的资格去与国民党同志一起为三民主义的实现而奋斗,即在国共分家后,我们在自己单独的革命苦斗中,也还是为的中国的民族独立,民权自由和民生幸福的目的,即是与三民主义根本思想相符合的目的。今天和过去国共分裂时不同的,只是过去我们共产党员在国共合作破裂后,单独地为共产主义思想和革命的三民主义事业奋斗,今天国共重新合作时,我们共产党员又与国民党同志们在一起为共同的革命事业而奋斗。所以在这种情形下,如果有人要求共产党员放弃共产主义的思想和组织,这不仅不能为共产党所接受,而即有直接违反着中山先生的理论和行动。

从中国的实际情形看来,第一次园共两党的合作,造成了一九二五年一一九二七年的革命蓬勃发展和北伐军的伟大胜利,而国共关系破裂,便造成了十年来外患内忧空前严重的局面;现在国共两党一经合作,对外便能发动空前未有的光荣的民族自卫抗战,对内便能造成数十年来空前未有的政权的统一,便造成空前未有的同民族中党派各阶层力量的大团结,使全国同胞和全世界人尽感觉到,这是中华尽族生死关头中的唯一生机和希望。两个主义的两个党同时合法并存,对于中国人民和国家是有利或有害,应根据活的,人所共知的事实来判断,而不应该根据少数人的偏见和主观愿望为决断根据。中国的过去和现在的铁一般的事实一再证明:三民主义与共产主义互相合作,则国家统一,革命发展;三民主义与共产主义互相对立,互相否定,则国家分裂,革命困难。所以任何违反这种事实的意见,必然都是不能实现的,一旦或万幸而实现了,定造成国家民族的灾难。

由此可见,认为三民主义相共产主义不能相容的意见,以及由此而认为国民党与共产党不能同时合法并存的意见,都是毫无根据,而且有害国家民族解放事业的意见,这种意见既违背中山先生的理论和行动的遗教,又违反中国的实际情况,这种意见的实际危险,就是有把目前举国一致对外的良好政局转化成视线对内,增加内部困难的可能局面。

因此,在谈话结束时,对这个问题,我再重复郑重地告诉你几句话:如果所谓国民党一党专政的内容,即是说国民党一党掌握政府的政权,这是己成的事实,这没问题。我们共产党员虽然自己不参加政府,但我们对领导抗战的国民政府绝对拥护,今天只有日本法西斯军阀才否认国民政府,企图推翻国民政府,今天只有执行日寇特务机关意旨而行动的托洛斯基及其门徒们,才会造谣说国民政府是克伦斯基式的临时过渡政府,因而才对国民政府釆避“表面上虚与委蛇”,“实际上准备打倒”的汉奸政策。我们共产党员对国民政府的拥护和帮助,是绝对真诚的,是从国家民族根本利益的根本立场出发的。同时,如果把国民党一党专政的内容扩大到和曲解成除国民党外不允许共产党员和其他党派存在,除三民主义外不允许共产党员有自己的共产党员的信仰等等,这绝不是新的理论,而是旧的实际,这是再重新回转到十年来走不通的道路和做不到的办法,这一思想如不幸见诸实际行动,则既又害统一而又害抗日,即是使中国既经形成的抗日民族统一战线解体,因而也就是使抗战救国的事业发生危险。因此,这些思想客观上一定全被日本法西斯军阀和汉奸所利用。也正因为如此,所以我确难相信,这种意见能得到国民党中以国家民族根本利害为重的有识之士的同情。当然,更谈不上这种意见能得到宝贵统一和彻底抗日而无党派偏见的同情了。不过,这个问题即恰当着民族危机更加紧迫的关头,能公然提出,这就不能不唤起共产党员和其他一切抗日救国的党派和同胞们的严重警惕,这的确不只有关国共两党关系的问题,而且有关于各党各派和整个中华民族生死命运的问题。所以我希望全国各界同胞、各党派有远见的贤明之士,都严重注视这一问题的发展趋势,尽一切努力使以各党派合作的中国抗日民族统一战线不受威胁,使国家统一和抗日救国的一切事业不受危险。以达到抗战到底,争取国家民族最后胜利的任务。

<p align="right">抄自《毛泽东救国言论集》</p>

