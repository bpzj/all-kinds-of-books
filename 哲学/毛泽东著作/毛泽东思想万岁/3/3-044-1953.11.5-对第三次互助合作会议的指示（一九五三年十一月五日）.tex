\section[对第三次互助合作会议的指示(一九五三年十一月五日)]{对第三次互助合作会议的指示}
\datesubtitle{(一九五三年十一月五日)}


做一切工作,必须切合实际,不切合实际就错了,切合实际就是要看需要和可能,可能就是包括政治条件,经济条件和干部条件。发展农业生产合作社,现在是既需要,又可能,潜在力很大,如果不去发展,那就是稳步而不前进。(脚本来是走路的,站着不动那就错了)有条件成立,而去强迫解散它那就不对了,不管那一天这都是错的。纠正急躁冒进,总有一股风吧!吹下去了,也吹倒了一些不应该吹倒的,像大名府四百几十个社只剩下五十多个了,倒了几百个,其中有倒错的,查出来讲清楚,承认是错误,不然,那里的乡干部、积极分子就要蹩着一肚子气了。

要搞社会主义,确保私有是受了资产阶级的影响。“群居终日,言不及义,好行小惠,难矣哉。”言不及义就是言不及社会主义。不搞社会主义,在个体经济基础上搞农贷,发救济粮,依率计征,依法计免,打井开渠,小型水利,深耕密植,合理施肥,步犁水车,喷雾器,六六六,反对五多等等,这些都是好事,但是不靠社会主义,只从小农经济基础搞这一套,那就是对农民行小惠。这些好事跟社会主义总路线联系起来,就不是小惠了,不与社会主义联系起来,就是小惠。至于“确保私有”,“四大自由”,那更是小惠了,而且是惠及富裕中农和富农。不靠社会主义,想从小农经济做文章,靠在个体经济上行小惠而希望大增产粮食,解决粮食问题,解决国计民生的大计,那真是难矣哉!有句古语“纲举目张”,拿住纲,目才能张。纲就是主题,社会主义和资本主义的矛盾,并且逐步解决这个矛盾,这就是主题,就是纲。提起了这个纲,那么反对冒进,克服五多,以及一切帮助农民的政治工作,经济工作,一切都有统属了。

农业生产合作社,社内社外都有矛盾。合作社是半社会主义的,社外的个体农民是完全的私有制,这两者之间是有矛盾的(个体农民在供销社入了股,这一部分股金所有制上也有变化,他也有了一点社会主义)。互助组跟农业生产合作社不同,互助组只是集体劳动,并没有触及所有制。同时农业生产合作社,还是建立在私有制基础上的,个人所有的土地、大牲口、大农具入了股,在社内,社会主义和私有制也是有矛盾的,这个矛盾要逐步解决,到将来,由现在这样半公半私进到集体所有制,这矛盾就完全解决了。我们所采取的步骤是稳的。由社会主义萌芽的互助组,进到半社会主义的合作社(将来也叫农业生产合作社,不要叫集体农场)。一般讲互助组还是农业生产合作社的基础。

有一段时候,曾经有几个文件没有提到互助合作,我都加上了发展互助合作或者是进行必要的和可行的政治工作,经济工作这一类的话。想从小农经济做文章,因而就特别反对干涉农民过多,那个时候也确实干涉过多。上面五多,条条往下插,插得下面很乱。五多那一年也不行,不仅农村不行,城市也不行,军队也不行。主观主义一万年也是要不得的。发了几个文件,反对干涉过多,这有好处。但什么是干涉过多呢?不顾需要和可能,不切实际,主观主义的计划,或计划倒合实际,但用命令主义的办法去做,那就是干涉过多。主观主义,命令主义,一万年也要不得。不仅对于分散的小农经济要不得,就是对于不分散的合作社,主观主义、命令主义也是要不得的。不要把需要做,可能做的,做法又不是命令主义的,也叫做干涉过多。检查工作,应该用这个标准去检查。凡是主观主义的,不合实际的,都是错误的。凡是用命令主义去办事,都是错误的。稳步不前,右了;超过实际可能办到的程度勉强去办,“左”了,这都是主观主义吧!都是错误的,检查工作应该本着这个标准。冒进是错误的,可办的不办也是错误的吧,强迫解散也是错误的吧!

“农村苦,不大妙,措施不符合小农经济”,党内党外都有这种反映。农村是有一些苦,但是缺乏恰当的分析,其实并不是那样苦,也不过百分之十左右的缺粮户,其中百分之五是很困难的,鳏寡孤独,没有劳动力,但是互助组合作社可以给他帮点忙,比起国民党时代总是好得多了,总是分了田。农民是苦,但也发了救济粮。一般农民生活是好的,向上的,所以说百分之八十到百分之九十的农民欢欣鼓舞,拥护政府。农村人口中间,有百分之七左右的地主富农对我们不满,其中地主和富农也有区别,就是地主中间,也有一部分他的子女在城市工作,一个月寄十多万块钱(按:指旧币)回去,他的生活也并不苦。“农村苦,不得了了”。我历来就不是这样看的。对于个体经济实行社会主义改造,互助合作搞合作社,这不仅是方向而已,而且是当前的任务。

总路线的问题,没有七八月间的财经会议是没有解决的。七、八月的财经会议,主要是解决总路线问题,批评薄一波的错误,也是关于总路线的问题。总路线,概括的一句话就是:国家工业化,对农业,对手工业,对资本主义工商业的社会主义改造。这次实行粮食的计划收购,计划供应,对于社会主义也是很大的推动。接着又开展了这次互助合作会议。又是一次很大的推动。鉴于今年大半年互助合作运动缩了一下,所以这次会议要积极一些,但是改革政策要交待清楚,交待政策这件事很重要。你们说“积极领导稳步发展”,这句话很好。这大半年缩了一下,稳步而不前进,这不大妥当,但是也没多大问题,也有好处。比如打仗,打了一仗,休整一下,再开展第二个战役。其中有些阵地退多了一些,而是本来可以发展的没有发展,不让发展,不批准,成了非法的。世界上有许多新生的正确的东西,常常是非法的,我们过去就是非法的呀!国民党是合法的。可是这些非法的社坚持下来了,办好了,你不能不承认嘛?还得承认他是合法的,他还是胜利了。积极领导,稳步发展,你们讲了,但是还要估计到会还有些乱子出来。你说积极领导,做起来他会不积极领导,或者不稳步发展。积极稳步就是要有数字控制数字,派任务,以后再检查完成没有,这就是积极稳步。有可能完成,而不去完成,那是不行的,那就是对社会主义不热心。现在据检查,有百分之五到百分之十的社减产,办得不太好,这是没有积极领导的结果。当然有少数社没有办好,减了产,这也是难免的,但如果有百分之二十甚至更多的社减了产,那就是问题。

总路线就是逐步改变生产关系,斯大林说:生产关系的基础就是所有制。这一点同志们必须弄清楚。现在私有制和社会主义都是合法的,但私有制要逐步变为不合法,在三亩地上确保私有,搞四大自由,结果就是发展少数富农,走资本主义的路。

区干部,县干部的工作,要逐步转到农业生产合作社这方面来,转到搞社会主义这方面来(他们不办社会主义之事,他们做什么,办个体经济之事吗?)县委书记、区委书记要把办社会主义之事当作大事看。一定要书记负责,我就是中央书记,中央书记、省委书记、地委书记、县委书记、区委书记,各级书记都要负责,亲自动手,并不是要费很多唇舌。中央现有百分之七、八十的精力都集中在办社会主义之事上。办资本家之事,也是要把资本主义工商业改造为社会主义,比如现在全国工商联开代表会,也是办社会主义。各级农村工作部的同志,到会的人,要成为农业社会主义的专家,要成为懂得理论,懂得路线、懂得政策,懂得方法的专家。

蔬菜的生产供应,主要是有计划供应,不仅大城市,像鞍钢,新发展起来的城市,人口很集中,没有蔬菜吃,那能行呢?东北局农村工作部要负责解决这个问题。在城市郊区蔬菜生产,搞互助组,有先后矛盾不好解决,可以不经互助组,就搞合作社,甚至集体农庄,可以不可以研究一下?

发展计划提出来了,今冬明春,到明年秋以前,发展三万二千多个,五七年可到七十万个。可是要估计到,可能有时候突然发展一下,可能是一百万个,也许不止一百万个。总之,既要办多,又要办好,积极领导,稳步发展。这次会开得有成绩。现在不开,明年一月再开就迟了,今冬就错过去了。规定明年三月二十六日再开会,要检查这次计划执行得怎样。这次会决定了下一次会议的日期,并决定下次会检查这次会议执行情形,这个办法是公安部提出来的,很好。明年秋天还要开一次会,讨论规定明冬任务,早一点,九月二十六还要开一次会,提早一点,布置冬季工作。


