\section[视察天津时的谈话(汇集)(一九五八年八月十三日)]{视察天津时的谈话(汇集)}
\datesubtitle{(一九五八年八月十三日)}


(建立地方独立工业体系的指示)

地方应该想办法建立独立的工业体系。首先是协作区,然后是许多省,只要有条件,都应建立比较独立的,但是情况不同的工业体系。你们怎么样?

一个粮食,一个钢铁,有了这两个东西就什么都好办了。(视察天津大学时的指示)

(天津大学张××校长向主席汇报学校情况说,这个学校已有98%的学生参加了勤工俭学。今年下学期还准备搞几个班半工半读。)

主席说:这样很好,本来光读书本上的,没有亲自去做,有的连看也没有看过,用的时候,就做不出产品来。一搞勤工俭学、半工半读,这样有了学问,也就是劳动者了。河南省长葛县有的中学勤工俭学搞的好,学生进步快,升学的多。有的中学没有搞勤工俭学,就不好,没有学进去。把脑筋学坏了。不仅学生要搞勤工俭学,教师也要搞。机关干部也要办点附属工厂,不然光讲空的,脱离实际。

主席问王元之同志:天津中学有没有搞勤工俭学?(王元之回答:天津近百所中学,都已搞起勤工俭学来了,六十多所中学还建立了工厂或生产车间。)

主席说:好啊!学校是工厂,工厂也是学校,农业合作社也是学校,要好好办。

毛主席说:以后要学校办工厂,工厂办学校。老师也要参加劳动,不能光动嘴,不动手。

(毛主席对天大在短时期办起许多工厂很感兴趣。)

毛主席说:有些先生也得进步,形势逼着他们进步,他们动动手就行了。五十岁以上的教师可以不动手了,青年的、中年的都要动手。搞科学研究的人,也应该动动手,不然一辈子不动手也不好。

(张××校长汇报说:现在同学们搞技术革命的劲头很大。有的为了向国庆献礼,昼夜突击,说服他们也说服不了。)

主席很关切地说:连夜搞是否把人搞瘦了呢?还是注意有节奏的生产,有节奏的休息和劳动。

毛主席还指示:

高等学校应抓住三个东西:

一是党委领导;

二是群众路线;

三是教育与生产劳动相结合。


