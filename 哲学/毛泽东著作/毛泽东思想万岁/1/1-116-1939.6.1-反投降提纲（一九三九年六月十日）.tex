\section[反投降提纲(一九三九年六月十日)]{反投降提纲}
\datesubtitle{(一九三九年六月十日)}


一、目前形势的特点

目前形势的特点在于,国民党投降的可能已经成为最大的危险,而其反共活动则是投降的步骤,国民党投降可能是从抗战开始就存在的,不是今天突然发生的;但成为时局的最大危险,则是目前政局中的现象。国民党反共也是以统一战线建立时就存在的,不是今天突然发生的,但把反共作为直接投降的步骤,则是目前的实际。

甲、日本诱降政策的历史发展

(一)日本对华基本方针是灭亡中国,建立所谓:“东亚新秩序”是坚定的不变的。在日本侵略者看来,也是变不得的,今年一月二日东京《国民新闻》(军部机关报)的社论说,

“眼前现实的课题,已不容尺寸言迟,瞬间踌躇。事变入予水拾羽,万一国步冗刹有欺骗疏漏,则不但将夫失战果,且会影响国家的命运,战争的后期之比初期困难好几倍,历史有过教训,欧洲大战时代,德国和俄国不平的事例,为政者和国民须得深加洞察。’(世界知识二月一日出版的九卷一期)

做一句来说,就是:基本方针非贯彻不可,否则有德俄革命的危险。由此而知,以为日本可以根本上让步,可以不经长期战争而用英美压力恢复芦沟桥以前状态(蒋介石,国民党许多人)只是梦想。

(二)它的“灭亡中国,建立东亚新秩序”的基本方针(总路线)是坚定的,不会自己动摇的,但是它执行此基本方针的方法,(或策略)是软硬兼施的,有伸缩性的,并可做出某些暂时的,局部的,表面的让步,以求达到基本目的。

大体上,芦沟桥以前,政治诱降为主,芦沟桥至武汉,军事打击为主,武汉到现在,以政治诱降为主。

(三)为了明白敌人根本方针的坚定不变,而其施行策略的软硬兼施,只要看到下列各种材料就可知道:

A、芦沟桥前,规定硬的方针,并实行占领东三省,对全中国则釆取外交诱降政策。

1.满洲事变前的田中奏折,一九二六年(昭和二年)七月二十五日总理兼外务大臣田中义一承致宫内大臣一本喜法,请代奏明治天皇者,有云:“欲征服世界,必先征服支那。”规定了灭亡中国的方针。

2.一九三一年,九.一八占领满洲。

3.一九三四年四.一七的天羽声明“天羽所代表的外务省,发出声明有云。”东方和日本负责为之,反对中国与东方和平抵触之行为,反对他国任何妨碍东方和平之举动,公开表明其灭华方针。

4.一九三六年一月二十日的广田原则“广田在议会正式公布三条原则”即,第一,对日亲善,第二,承认满洲,第三,共同防共。

5.一九三七年春,日本特务机关长松金孝良的秘密之呈:结论六条,均表示实行方法之软硬兼施。

B、芦沟桥到武汉为执行其硬的方针,采取坚决的军事进攻方法为主,辅以政治诱降。

6。一九三七年,七七事变后的几天,首相广田宣布不扩张主义。我党“公开决定,曾指出这是掩护进攻的烟幕弹”。但同时含有“降则免打”之意。

7.一九三七年七月二十日陆相杉山在特别会议的战争演说:“断然决定,惩,陆军当局已采取重要措施”。表示方法是硬的;打为着降。

8.开战以后,日本提出和平条件,“据英国《标准晚报》称一日政府负责人曾以这些条件提蒋介石,内容六条:一、内蒙古独立;二、华北自治;三、上海占领地做为日本租界;四、山海关到安南公海的捕鱼权;五、中国退出国联;六、不得设空军”。(一九三七年一月十三日《解放日报》)表示所要的东西,至今大体上不出这个范围。

9.一九三七年十二月十三日占领南京后,于二十七日日本第三舰艇司令长谷川发表谈话:“日本帝国主义决不满足于今日的成功,盖距总局之前途尚远,帝国军人签本帝国之总意,为东洋永久之和平,希望达到彻底进步之目的”。表示要继续军事进攻。

10.广田的除夕(一九三七年)的演说:“友邦中国若能了解日本力求东亚和平重要观点,则能免除日之苦痛,日本政府深虑于一九三八年开始新的和平,解决一切问题”。表示方法可以用软的,只要降。

11.有名的一月十六日宣言,十二月十一日(南京占领前二日)商议决定,十二日通过御前会议,十三日占领南京,十七日蒋委员长发表告国人书,到一月十六日政府就发表此宣言。其大意谓:“今后不以国民政府为对手,期望新政府成立,尊重中国领土,主权与外国在华利益。”表示方针是硬的,方×也是硬的.

12.一九三八年一月二十日日驻华大使川樾对记者发表谈话,谓:“奉令回国,参赞日高留沪,但留沪目的,并非商谈与中国政府交涉,而是与外国人员接洽可能发生之事件,与国民政府重开之门户已经关闭,即使国民政府之重新考虑其态度,亦仅能与中国新组织交涉,不能以日本为对手。”从一月十六日宣言后,以战为主,以和为辅,打倒蒋介石,建立新政府,此谈话最露骨。

13.但战中仍有和,一月二十日路透电所传广田在七十三届议会中宣布德国调解的和平条件四条:一、放弃联共抗日,承认满洲。二、在若干地带成立作战扈,在此扈内设立管理机构。三、中、日、满经济合作。四、中国赔款。此外求低议和条件,即陶德曼交蒋之条件。蒋曾拟考虑接受,汪精卫的“举一个例”即指此。(蒋、二陈、何,还有孙、于,都参加讨论接受问题)最后蒋拒绝,广田乃在议会宣布。

14.一月下旬在七十三届议会中。

近卫的开幕演说:“战争之解决,尚须长久时日”。现在政府釆取解决之政策,嗣后不再与中国国民政府发生关系。

广田演说:宣布陶德曼提交之四条件(见前)。

民政党议员“要求政府保证”嗣后决不与国民党政府讨论讲和条件。

议员岛田俊雄,质问为什么不对华宣战。

日本政府宣布对华政策四要点:一、绝对不与国民党政府交谈;二、为阻止军火运华,仍可对华宣战;三、对华北新政府居监护人地位:四、绝对不容第三者调停。

但至一月二十九日那天,近卫又在本届议会表示:“宣战之举仍在考虑中,现中国表示态度为断。和平之门仍是开的”。

此乃反映日本政府中主和派与主战派之斗争,杉山为此曾于此时发表:“必须准备长期战争”之文告,与主和派对抗。

15.日本驻沪大使(维新政府的)谷正之,三月九日发表谈话谓:“蒋介石政府将行崩溃,但须继续加强战争,促使蒋介石政府倒台,并使第三国放弃援蒋企图。战争以久之法,如能使蒋氏议和,则耗费较少,效力更大。”一个人在谈话中主战又主和。

16.五月二日近卫演说准备长期战争,为推翻蒋政权,铲除东亚祸根,虽费数年光明,在所不惜。同时全国国民应积极援助华北、华中两亲日防共政权。

17.五月九日广田在“地方军会议”演说:主张慎重处理事变。“中国事变进到第二阶段,蒋介石政权宣传长期抗战,以事于国内之团结,但未能达到团结之目的,他方面,各国军需品之输入,苏联援助中国,这是事实。因此,帝国政府必须顾及此种事实,慎重处理此次事变;津浦线占领之时,临时维新两政府即刻实行合作,帝国政府对此极力加以支持,使能合并统一,并完满键全发展。”敌以军部与外务省,一个代表硬,一个代表软,或硬中带软(军部之特务机关)或软中带硬(外务省)装红白脸。

18。徐州陷后,敌声言打到昆明,五月十九日徐州失陷,二十二日敌前线指挥官前发谈话谓:“徐州攻陷,战争并未停止。战争步骤有三第一步,即徐州大战;第二步,进攻汉口;第三步,进攻重庆或昆明。”不亦硬哉!

19.六月十六日永口通信大臣发表谈话,谓必须打倒蒋介石政权。目前政府之对华政策,在于打倒蒋政权,援助新政权建设新中国,奠定东亚永远之和平。其他政策亦需照此路线进行。

20.张高峰事件时,外传日本向中国提出和平五条件:“日苏过境日烈,有成为大规模战争之可能”,中、日和平空气今日(八月五日)又盛传于香港,上午《字林西报》首传日本向中国求和五条件,下午各晚报均刊载,五条件是:一、日军退出占领区,但中国也不得驻军;二、承认满洲国;三、虹口,闸北、江湾租给日本,定期九十九年;四、赔偿此次战争损失;五、共同防共。(星岛日报,二十七年五月八日香港通讯)。

21.坂垣的强硬论“七月一日坂垣发表谈诘”“在蒋介石政权依然存在之时,日中间定无和平可能,今日日本不能再与蒋介石携手开和平谈判。”七月八日事变一周年纪念日,坂垣发表谈话:“今后战争无论延长多少年,帝国所走之路,只有一条(按指武力征服中国),除此以外,不能达到东洋万年之和平。”

22.设立对华院,总揽对华统治大权,总裁,副总裁,五相,不但占领区,而且非占领区,均归其管辖。

23.九月二十九日军部赶走宇垣,宇垣因对华政策与军部不合,愤之离职。

C、武汉到现在:策略改变,由硬到软。

24.十月十二日占领广州,二十五日占领武汉,看见蒋介石三十日发表主张长期抗战的告国民书,故日本择了十一月三日明治纪念日,即所谓天长节这一日,发表了一个重要发言,明显的表示改变策略,其要点为:一、国民政府已成为一个地方政府,如继续抗日,则该政府歼灭之前,决不停止军事行动,二、日本之目的,在于建立东亚长治久安之新秩序,即华、日、满提携,树立政治、经济、文化等项互相连环之关系;三、至于国民政府,只能抛弃从来错误政策,另由其他人员从事更生之建树秩序的维持,则帝国未不加拒绝”。政策已变,宣布废除不以国民政府为对手及必须建立的政权之政策,声明可以国民政府为对手,但须蒋下野。

25.十二月二十二日近卫声明,即汪精卫通电拥护,蒋介石演说痛驳之声明,其要点:“一、中、日、满三国应以建设东亚新秩序为共同目的互相结合,相互亲善,并实行共同防共中日经济提携;二、因此,中国必须与满洲国树立完全的国交关系;三、缔结中日防共协定在规定地点驻军防共,以内蒙为特殊防共地域;四、中日经济提携,帝国臣民在中国内地有居住营业之自由,在华北与内蒙予日本以开发与利用资源之便利;五、日本允许考虑交还租界,废治外法权。”重申十一月三日演说宣言的基本政策,如具体内容,只要降日,国民政府与蒋介石(允许蒋存在自此始)均可存在,华中,华南可以退出,华北是要的,但主权名义仍可不要。

26.平沼内阁继此方针,至今不变。

27.日本在中国策动广泛的“和平运动”,设立所谓“和平息战会”,到处开大会发传单,设立统一中央政府,很久不提了,它偶一提及,无外吓蒋,除非蒋抗战到底,他是留以待蒋的。

28.离开国共合作,利用三民主义的大阴谋,日本松本贞一做了一篇题为“争取支那大民族大众问题”的文章,其中说:“帝国行使武力之目的,第一在彻底打击抗日政权及军队,第二在要日、支两民族提携,融合及东洋和平之确立,或者乃达成后者之手段,然此两者,证诸支那之现实,实为对立之矛盾,试现事变以来,经过抗日政府之领导原理虽属误谬,但蒋政权已具有民族政权之性质,而有支那民族大多数的支持,现政权所以能得到中国共产党、中国青年党或抗日联合阵线,以及旧军人之热烈支持者,即缘于此。今认一方欲击灭蒋政权,他方欲与支那民族相提携,事实上诚不缘木求鱼。盖欲倒蒋,忽使支那民族拥蒋,要倒蒋就不能抓住民众,要抓住民众即不能倒蒋,故我国根木困难即在于此”。

“临时、维新两政府,尚无民众基础,有民众基础则南、北两政府,自不难合流,而新政府之纲领,应以国民党三民主义为旗帜,新政府之任务应在民族主义、民权主义、民生主义之实现,或曰敌人之旗帜与我国同,难免混淆不清,但吾人可告以彼等以一面抗战,一面实现三民主义为号召,吾人则以“一面与日提携,一面建设三民主义为号召,国共合作,乃抗日联合阵线之基于,实有加以打击之必要”。

“武汉作战之目的,乃在分散国共合作,故武汉占领在国共间插入一个旗子,战略政策双方都在分散国共力量”。(上文见《大公报》本年一月二十三日)

29.上海《导报》还在二月十七日就指出了远东慕尼黑的危险。一方面,他在实事求是,努力加强地方政权,组织所谓江汉政府之类。别方面,他在进行更大规模和有国际性的阴谋,这阴谋的主要企图是压迫和诱骗英法两国来用国际会议的形式(公开的或秘密的)强压中国与日本妥协,虽然日本声明不愿第三国参加解决中日冲突。并且说:“认为汪精卫完事大吉是不对的,他将来还可以起极恶劣的作用”。

以上我举的二十九件材料,说明日本对华诱降政策的历史发展,充分证明下列三点。

第一,日本灭亡中国的总方针是非常坚定的,决不改变的,他一定要把中国变为他的殖民地,一定要建立所谓“东亚新秩序”。

第二,但他实行的方法是软硬兼施的,并且依时有所侧重,武汉以打为主,但亦战中有和,武汉以后,以和为主,但仍略打一打,迫使投降。

第三,他还努力策动英、美、法召集远东和平会议,远东慕尼黑的危险就压在中国人头上。

(四)以上三节,第一节指出他的基本方针坚定不变,第二节指出他的实行办法软硬兼施;第三节指出他的基本方针之实行方法由硬到软的历史发展,证明投降是当前的最大危险。现在这个第四节从他的财政、经济、军事方面说明其主和的原因,日本已处于财(三年共一百二十亿元,第一年二十五亿五千万元,第二年四十八亿五千万元,第三年四十六亿五千万元)经济(输入大增,输出大减,军事工业打倒和平工业,物价大涨,人民生活恶化,现金用尽,外汇不补)军事(兵力不足,兵力分散),大见困难的境地,他还要准备应付国际战争,故极力策动中国投降,分裂抗日阵线,利用汪精卫劝降蒋介石,并准备利用三民主义与国民党,他的这个阴谋是十分恶毒的。敌人战不怕,和则十分危险,《大公报》、曾标为“缢鬼式之和平”,意为如果这样,中国将无疾而终。

(五)为表示和平之诚意,给国民党看,给英美看,用大力“扫荡”八路军(华北十五个师团),现又企图进攻陕北,作为促进国民党降日反共之步骤。

以上是日本的诱降政策。

乙、英、美、法的压力

中国投降成为当前最大危险的第二个因素是英、美、法投降主义者加以中国政府的压力。

英、美、法等非侵略国,对于侵略国所进行的侵略战争所釆取的放任政策,正如斯大林所指,不是由于他们力量不足,也不是单纯的由于他们畏惧革命,而是由于他们“坐山观虎斗”的阴谋计划,即所谓中立政策,或不干涉政策。

他们开始即鼓励日本进行战争,说什么“三个月就可以打败中国”。他们随即让出上海,使战争深入内地去打,他们宁可使香港受包围,让日本占领广州和海南岛。他们大量供给日本以军需品,使日本有可能进行消耗战争。

他们又声言援助中国,并且已实行有所援助,借点小款,借给军需,使中国有可能与日本进行消耗战争。他们要鼓吹“中国胜利”,使中国在消耗战争增加勇气。

一切这些,其中心的目的,在于消耗战争双方,等到筋疲力竭时,他们就以“健全的身体”出来喝令双方停战,使双方都听他们的话,他们容纳德、意在西方的侵略行动,同样是这个目的。

他们希望德国和苏联打,而他们从旁观战,然后乘其敝而掠夺之。

他们始终不赞成苏联提议的普遍安全计划者在此。

他们不愿保障波罗的海三国的安全者在此(开一缺口便于德国进攻苏联)。

战争的片面性(侵略战争危害英、美、法利益的,但英、美、法却取旁观态度)的原因在此。

鹬蚌相持,渔人得利,这就是英、美、法帝国主义者的现实政策。

这些非侵略国之间有深刻矛盾的,但不到一定程度他们是不会放弃其“渔人”的政策的。

莫洛托夫谓:“英、美、法有进步,但所谓进步是浮而不实的”。

斯大林谓:“不要被人利用”。即是英、法渔人政策并不弃,不要上当。

英、法、苏协定有成立可能,但目前尚难乐观,成立了以后,又有破裂可能,新慕尼黑危险并未消灭。

六中全会指出,英、美、法政府不可信,可信的只有其人民。英、美。法人民反战反法西斯势力正在逐步增长,只有这种形势才是最可信的。

苏联声明继续援助中国,决不赞成中国投降。

英、美、法策动的慕尼黑,现在接近了一个紧要时节。

他们正在做这种想法,希望中国再打半年,双方都正疲怠一点那时就得远东慕尼黑开幕了。

中国投降危险的第二个因素,就在这样的国际情况上面。

丙、中国地主资产阶级的动摇

(一)资产阶级叛变的必然性

A、一九三六年四月苏区代表大会曾经指出:

“在某种历史环境能够参加反帝反封建的中国资产阶级(正确点说,民族资产阶级),由于他政治上、经济上的软弱性,在另一种历史环境就会变节,这一规律,在中国历史上已经证明了。因此.中国反帝反封建制度的资产阶级革命任务,历史现在已经制定不能经过资产阶级的领导,而必须经过无产阶级的领导,才能达到目的”。

B、一九三七年八月十五日中央决定上说:

“由于当前的抗战,还存在着严重的弱点,所以在今后的抗战过程中可能发生许多挫败、退却,内部的分化叛变,暂时与局部的妥协等不利的情况。天津的丧失就是东四省丧失后最严重的教训,因此应该看到这一抗战是艰苦的持久战。”

C、六中全会指出妥协危险的严重存在,把反对投降政策放在第一条。指出国民党有光明的前途,但同时指出其前途还有障碍物,如不克服则没有什么光明前途。

很明显的,所谓一切党派在坚持抗战,坚持统一战线的大前提下都有光明的前途,是包括了克服了各党内部守旧倾向这种势力的,如果存在着不利抗战与统一战线的守旧倾向而任其发展下去,那就有断送其光明前途的危险,这不论是国民党也好,共产党也好,其他党派也好,都是一样。

国民党是以资产阶级为骨干的党,是在资产阶级指导之下的。

(二)五中全会以后的国民党。

A、五中全会还是以联共抗日为主要方向,但同时已包含反共投日的因素。

1、决定了依靠国际压力和平解决中日矛盾的方针。

2、决定了防共反共的方针(但不是战争),设立了“防共委员会”。

B、主和空气笼罩一时。

禁止反对远东慕尼黑,最近到的一期重庆出版的《文摘》把一篇题名“英国会在远东来一个慕尼黑吗”?的文章删去了。

许多刊物发表提倡依赖英美制裁日本的文章,其意即开国际调和会议解决问题。政府中党部中许多人是主和的,军队中也有这种人。

“国民党投降可能”这一点,历来就存在,但在今天已成了最大的危险。如不克服,中国抗战将受到极大挫折。

C、国民党已在进行真投降的主要准备工作,即是反共,反共是投降准备工作的主要的组成部分。

半年以来,华北,华中、华南、西北反共活动特别厉害,在华北八路军从日本手里收复失地,国民党从共产党手里“收复失地”,西北共产党没有超越边区寸土,国民党则用武力侵入边区许多地方,(镇原、宁县、构邑、靖边、瓦窑堡)原因在共产党是投降的最大障碍物,不反共则不能投降。

一切都是借口,进行投降准备工作则是实际。

制造了无数的假文件。

以上三个因素、三种原因,使目前局势处于投降派与抗战派严重斗争中,投降可能成为当前的最大危险,而反共正是准备投降的一个必然步骤。

三种原因中,以中国地主资产阶级的动摇为主要原因,如不自己动摇,则敌人的诱降政策无所施,国际的劝降压力无所用。

这种情况是与六中不同的。那时抗战正在高潮,十月三十日蒋之宣言,十二月二十六日蒋之驳斥近卫演说,就是证明。现在是在敌人以诱降为主业已生效的时候)是在英美加紧策动投降的时候,是国民党五中全会决定依借英美井执行反共政策之后。所以,那时,虽以反投降为第一条,但实际没有现在严重。

八个月以来,时局变到反共与投降最为严重的时候了。所以现在增加了新的具体任务,这就是全力反对投降。

准备投降各种的借口:

第一个借口一一共产党捣乱,这是不对的。伪造文件。边区的被攻,八路军的无钢。全国执行良好的统一政策,拥蒋拥政府。全国无土地革命,八路军新四军的英勇作战。

第二个借口一一苏联阴谋,这是不对的。一九二七年帮助中国革命最大。他没有阴谋侵略过任何一国。

第三个借口一一财政经济困难,这是不对的。

第四个借口一一人心厌战,这是不对的,日本正在作这种宣传,这样说不是响应日本宣传。人民不满兵役法,是怨恨方法不对,不是怨恨作战。

第五个借口一-军心厌战,这是不对的。污蔑抗日军,多数官兵的为民族解放而战,多数军兵的厌恶内战。

第六个借口一一国际无援,这是不对的。抗日主要靠已不靠人。苏联的有力援助,各国人民的同情。英美法政府本来不可靠。

第七个借口一一敌人太强,这是不对的。杀人已消耗了很大,三年要用一百二十万万元,兵力不足与分散。我之地形好,士气盛。

还有许多借口。

二、抗战的前途(大概估计,不能死看)

然而抗战一定要坚持下去的。抗日民族统一战线与国共合作,是一定要使之巩固发展的。三民主义旗帜与三民主义共和国口号是一定要努力坚持的,这是党的基本任务。

国民党投降和抗战继续可能是两个可能。有投降者,有抗战者,又是两种状况。

“可能发生许多挫败,退却,内部的分化叛变,暂时的局部的妥协”一一这是一方面。但我们相信,已经发动的抗战,必将因为全国人民的努力,冲破一切障碍物,而继续地向前与发展。只要真正组织千百万人员.进入抗日民族统一战线,抗日战争的胜利是无疑的。一一这又是一方面(均见“八月决定”)抗战可能有两个前途:大部抗战,小部投降;第二个前途:大部投降.小部抗战。

中国革命的长期曲折性,第一个前途,它是长期的曲折的;第二个前途,则更是长期的曲折的。

我们从来也没有设想过抗战是速胜论,直线论(一字论),而历来主张长期论与曲线论(之字论)。从来也没有主张过不发动全国人民,不实行国家民主化,可以克服投降,取得胜利。

克服投降,取得胜利,是要人民的大多数来干的,是要各党各派的一切爱国进步分子来干的。

不能设想,国民党整整齐齐一个不剩地投降。

一九二七年大革命失败,国民党全国投降帝国主义,举行反共战争。但那时的情况是:(1)没有一个帝国主义打进中国来;(2)所有帝国主义都赞助蒋介石反共,各国暂时稳定局面之下;(3)国民党没有吃过反共宣传的苦,能够动员军队;(4)全国人民没有尝过反共宣传的味道,没有受痛苦的教训,也没有受欺骗的经验;(5)共产党没有统一战线失败期间的经验,又没有武装力最(都失败了),主要的没有战争经验;(6)世界不是革命与战争时期,而是反动时期;(7)苏联没有今天强大。

然而还有邓演达等坚持革命,没有叛变。九一八以后,有冯玉祥,有蔡廷锴,有赵博生、孟根堂、季振同,有吉鸿昌、任应岐,有张学良、杨虎城,有陈济棠、孙科。

光在今年有两年抗战历史。

假如有米亚伽,有哈柴一一中国也不是西班牙、捷克。

①有投降者(这是必然的,已有汪精卫,还有张精卫、李精卫),也会有反对投降者。

②有坚决投降,一往不返者;也会有一时投降准备再战者。

⑧有先降后战者(有一些人要当一回汉奸才能变好),也会有先降后战再降者。

④人民和军官可能初受蒙蔽,后来觉悟反抗。如象“共产党捣乱”“八路军叛变”“国际情况不佳”,“财政异常困难”,“苏联要赤化中国”,“暂时屈服徐图反抗”等欺骗是必然会到来的。

⑤敌人必不让有一个全国统一下受干涉的投降集团存在。他们必然采取四分五裂政策。那时,可能有许多傀儡政权同时存在,以便其统治。各种武装投降之间可能发生内战一一中国是内战最可能(地大、经济分散)与最有经验的国家。

⑥敌人可能先让一步,然后干涉。当其让步时一一可能投降;当其干涉时一一又可能分裂。

①弗朗哥并没有收容米亚伽,希特勒也没有收容哈柴。日本最后只收容汪精卫一类。

③满洲人民与军队的经验,如果大家知道了(原文如此)。

中国军队不能消灭,这是肯定的。缴枪不打一一这也是肯定的。

总之,那时必是一个混乱局面,不是一个统一局面,那时,共产党将成为全国人民的救星,全国人员望共如渴,那时,中国人民对苏联希望必增加。那时,国际必是一个战争与革命局面。那时,日本与英美法的冲突是可能的。如果这种情况出现,正是中国革命长期性曲折性的实际。

不能设想,不民主的政府如现政府,能够“抗战到底”的。不能设想,这种政府能够争取最后胜利,我们一次也没有希望过,也没有说过。只有愿意动员全国人民的政府,包括共产党在内的政府,才是抗日胜利的政府。

旧东西的改造,经过叛变与克服过程。

那时可能发现抗日民族统一战线政府(包括共产党在内的)。

那时,中国政府可能有三种性质的:日本的傀儡政府、半傀儡政府、抗日政府。这几种政府间的斗争,将组成所谓相持阶段的局面,不能速胜也因这种情况。所谓相持阶段,可能正是这样情况的阶段。

不能速胜也因这种情况。

不能亡国也因这种情况。

将来抗战局面不外两种,如前所说:(1)大部抗战,小部投降;大部投降,小部抗战。如果是第二种前途,则情况可能如上所述。但第一种前途,目前并未绝望,原因在于:

在国内一一

A、有许多军人拥护抗战。B、国民党中的抗战派。C、共产党的反对投降。D、人民的反对投降。

在国际一一

A、苏联的援助;B、西班牙、捷克的教训;C欧美人民的舆论。

在日本一一

财政、经济、军事的困难。

但是要巩固扩大统一战线才能办到。

如果出现第二种前途,即就不但是政府的分裂,而正是三民主义与国民党分裂。

日本人的三民主义与国民党一一现在已做,经过汪精卫,这是假三民主义与假国民党。

中国行的三民主义与国民党一一现在萌芽,正在奔跑,这是半真半假的三民主义与国民党。

三民主义与国民党一一现已发生,正在发展,这是真三民主义的真国民党。

三民主义和国民党是否可以避免上述那样的分裂(分裂是必然的),要看国民党、共产党与全国人民的努力如何。

因此,相持阶段的可能情况,便有几种:

(甲)大部抗战,小部投降的相持阶段一一即克服投降可能,取得大部抗战。打下去,除非破坏日本的阴谋之外,还停止它的军事进攻,这是第一种情况的相持阶段。这是最理想最希望的。

(乙)大部投降,小部抗战的相持阶段(共产党与一切不愿意投降的人是要继续抗战的,即使是小部)一一即使大部投降了,剩下小部抗战,但只要能巩固抗战根据地,可能与敌相持。如同过去小部红军在根据地上与敌相持一样,不同的,过去是内战,今后是对日本,对汉奸,而不是内战。

在站住了脚,不是继续后退这一点说来,在他是全国唯一的抗日军这一点说来,这也是一种相持局面。这是第二种相持局面。

(丙)由小部再到大部的相持阶段一一这是第三种相持阶段。由于小部坚持抗战,坚持统一战线、坚持持久战,坚持国共合作,坚持三民主义,将投降派把持的阵地分化过来,再争取大部抗战。那时还不是反攻,则还是相持阶段。

我们力争第一种相持局面,不得已再是第二种。看其前途还有第三种。危险是存在的,但总的前途是光明的。在危险环境中表示绝望的人,在黑暗中看不见光明的人,只是懦夫与机会主义者。还在一九三七年苏区代表大会就指出了资产阶级叛变的必然性。还有一九三七年八月中央又指出:“抗战是艰苦的游击战”,“可能发生许多挫败、退却、内部的分化叛变”。十一月又指出民族投降主义与阶级投降主义的危险。这些我们都预料到了。它的发生并不是突然的。

抗战前途不外好坏两种,我们力求好的前途,同时准备即使很坏,我们也有办法。总的前途是光明的,但必须经过长期的斗争与曲折的斗争,短期的与直线的胜利是没有的。我们历来反对速胜论与亡国论,今天仍一样。以为地主资产阶级叛变,就会亡国,这是没有看到:1、共产党、八路军、新四军的存在;2、抗日友军中的抗日情绪;3,国民党中的抗战派;4、资产阶级中目前不愿投降者;5、沦陷区仍然受压迫的资产阶级;6、广大的小资产阶级与农民;7、苏联的援助。不看见这些正在生长的力量,只看见地主资产阶级投降了,就认为整个世界黑暗了,于是发生惊慌失措的现象,这是完全不对的。

六中全会说,相持阶段必是“更加困难,同时更加进步”的阶段.更加困难,就是除敌以外还加以投降叛变。更加进步就是经过各种形式的斗争(包括战争),抗日的国民党继续同共产党合作,国民党便能更加进步。

一部叛变,大部抗日一一这种情况是我们希望的,过去这样希望,现仍如此,要力争之。大部叛变(甚至一时表现为全体),一部抗日一一这种情况,不是我们希望的,但是可能的。如果出现,我们就要把他颠倒过来,这也是可能的,也要力争。

三、当前的任务

估计到上述好坏两种情况,特别是可能的坏的情况,党的任务是什么呢?

1、全党努力,从思想上组织上准备自己,并准备舆论,准备群众,随时可以对付事变――各种意料之外的袭击,各种大小事变。

2、全党努力,同一切爱国进步分子,一切爱国进步的国民党人员(上层的,中层的,下层的群众)亲密联合在一块,并和他们一道(如果在国民党统治区域,必须和他们一道,不是我们单独)去动员群众,开展反投降斗争,公开揭穿反共即准备投降的实质,以孤立投降派与反共分子,以便继续抗日。

在这里要加强统一战线的工作与人员。

3、不论何种情况,党的基本任务是巩固扩大抗日民族统一战线,坚持国共合作与三民主义,必须坚持这种方针,不能有任何的动摇,党的一般任务就是这样。

基于上述的任务,党应注意下列各问题了解,并进行必要的工作。

(一)六中全会的方针是正确的。六中全会在全国有很大的影响。我们的积极团结全国争取抗战胜利的方针,因为六中全会文件的发布,更加深入人心的了。我们的统一战线工作与党的组织工作,因为六中全会的指示,在全国更加发展了。

现在投降危险(……)阴谋当前主要危险,故需指出这种新的情况(过去还不是事实,只能一般指出.现在已有成为事实之可能,故需具体指出)才能克服投降危险,并准备如果投降由可能变为事实时及时的必要的对付政策,但基本方针仍然是六中全会的,并且指导适当,可能使抗战的实际内容发展到一个更高阶段。

(二)为什么此时要在思想上组织上准备自己,准备舆论,准备群众问题。这里要了解一九二七年大革命失败的经验。那时精神上没有准备,是失败的主要原因,现在必须有准备。

那时,不能再是民族阵线,但我们也有太绝对的地方,现在,必须坚持长期的民族阵线。

那时,是革命的整个失败,党的路线是退却然后进攻,现在,没有组织上的退却问题,现在继续进攻中发生局部的暂时的战术上的保守或隐蔽,期时准备进攻的问题(如果投降成为事实的话)。

那时,因为反动势力到来,它又没有准备,没有经验,以致队伍混乱,步骤错误(如盲动主义)。现在,则在沦陷区,并准备在投降去占领地(原稿如此一一抄者)应该作有秩序的退却(有的暂时的,有的长久的)或反攻,而且在其他区域则继续进攻方针。

那时,国际国内的情况是黑暗的,现在基本上是光明的。

那时,尚没有造成大批干部骨干,现在已经造成了。

以上,我做了八条比较,虽然如此,仍需在思想上、组织上准备自己,并准备舆论,准备群众,以便顺利的克服困难发展抗日战争与中国革命到一个更高的阶段,而避免可能的失败。

(三)为什么假如投降成为事实时还要坚持民族阵线,国共合作与三民主义问题。

主要理由已在说明抗战前途时说过了,这里只说:

1、在全部抗日过程中必须都是抗日民族统一战线,虽然由于叛变,许多地方资产阶级都走开了,并变成了敌人,但地主资产阶级的叛变是逐渐的,不是闪时的,留下的成分必须与之合作,忽视这一点,势将铸成大错。故终抗日过程,统一战线依然是各党派、各阶级、各民族、各集团、各军队的统一战线。

2、国共合作也是一样。若干国民党员离开了并成了敌人,但留下来的若干仍然是合作的。我们要用真国民党对抗假国民党,争取中间性的国民党。国共两党平等合作联盟的前途还是存在的,国共合作是统一战线的组织基础。忽视这点,也将铸成大错。

3、三民主义也是一样。它是统一战线的政治基础,是抗日过程中能够适用的原则,方针。我们要用真三民主义对抗假三民主义。争取中间性的三民主义,这是在几种三民主义存在的情况下应取的政策。三民主义与共产主义就在抗日中也是有区别的,但在抗日过程中两者有其一致点,即在把三民主义照着国民党第一次全国大会那样解释时,二者在资产阶级民主革命阶段的政纲上基本上是不相冲突的。因此,党内党外许多人轻视三民主义,认为它是根本反动的骗人的与空洞的思想和教条,是不对的,这种想法是由于没有把真三民主义、假三民主义与半三民主义加以区别而来。

4、在思想斗争问题上,两年来,尤其是半年来,代表国民党写文章的人,包括托派叶青等在内,发表了不但反对共产主义而且也反对真三民主义的“分歧错杂的思想”,亦此假三民主义或半三民主义的思想,应该加以严正的批驳.其中以所谓“国情论”与“统一论”之武断的叫唤为最嚣张。其实他们所谓只有三民主义与国民党为适合国情,乃是最不适合国情的假三民主义与假国民党,而共产主义与共产党乃是完全适合国情的,所谓边区与八路军的不统一,而欲将其取消,以完成其所谓统一,其实乃是反民族、反民主、反民生的完全违背抗战利益的假统一,完全违背统一中的差别或统一中的斗争这个社会历史事实的武断空话。在旧式半封建政权没有改变为真正革命民主政权以前,边区与八路军是必须存在的,只有在革命民主政权成立之时,反共危险消失之时,二者才可放弃现在这样的特殊性,否则只是反革命的要求而已。

(四)拥蒋问题

拥护蒋介石委员长的口号,过去是对的,现在还是对的,只要蒋领导抗战一天,我们还是拥护的(当然以抗战为条件),不应对蒋有不尊重的表示。

但蒋对抗战在某种情况下,不能坚持的可能是存在的。在那时,我们如何表示,还要慎重考虑。当然,那时不能不有表示,但须适当的姿态,当以有利多数团结抗战,有利国共继续合作为原则,而不能随便的轻率的恢复“反蒋”口号。

蒋对于共产党存在着敌意,这是他自己表示的事实,我们必须严防他及其部队破坏我党,这是毫无疑义的。

积极帮助蒋与督促蒋向好的一边走,仍然是我们的方针。对外不说“国民党投降”,应说“地主资产阶级投降派”。

(五)反汪问题

“认为汪已完事是不对的,汪还能起大的作用”,这是完全对的。汪之汉奸系统较之其他汉奸更大的危害中国,没有疑义。汪在政府中、党部中与一部分军队中有群众与同情者。

反投降必须联系于反汪,是更便利与更实际的。


(六)抗日除奸问题

在假如严重叛变事件发生之时,党应极力注意对付的适当,以不脱离国民党多数为原则。

那时抗日除奸的口号是必须的,抗日除奸的战争(同时抗日又除奸)也是无可避免的,但绝不可离开国民党多数而轻举妄动。只有在为多数人所了解与要求时才能(也必须)发动除奸战争。例如现在的反汪,是合乎这个原则的。

(七)民主民生问题

没有革命民主政府,要领导抗日胜利是不可能的,全国广大人民之渴望一个抗日的,给人民以自由的,民主集中制的与廉洁的政府,有如此切,故“争取民主”,应在今后“反对投降继续抗战”的运动中与之联结起来。

“改良民主”的与之联结起来也是一样。不过,只有抗日才有实行民主可能,只有抗日与民主才有改良民主可能,这是今日政治形势中的实际,应该明白。

(八)摩擦问题

国民党五中全会后,在河北、山东,特别在边区所举行的破坏性与准备投降性的摩擦及武装斗争,是必须给以坚决抵制的,这种抵抗是有用的。但必须站在自卫立场与决不能过此限度,给挑衅者以破坏统一战线之口实,这种自卫的防御的反摩擦斗争的目的,在于巩固国共合作,为此目的,一定条件下缓和退让,也是必要的。

统一不忘斗争,斗争不忘统一,二者不可偏废,但以统一为主,“摩而不裂”。

严防挑衅,不要上当。

(九)友军中不发展党的组织及从某些军队中撤退党的组织的问题

半年来经验,在抗日的与我合作的军队中不去发展党;这个六中全会的决定,根本上是对的,在中国军队中发展共产党,一般应限于汉奸与准备汉奸的军队,目的是在破坏之,其他则不应发展。任何军队只要变了汉奸军队,或直接准备变为汉奸军队,都应采取坚决的破坏方针。某些抗日军队中已有党的组织问题发生了有碍合作的问题时,应公开表示撤退,以求合作之继续。

(十)援助友党问题

1、极力拥护那些进步分子,但也应依程度不同而定援助程度之不同。

2、拒绝援助那些退步分子,无希望分子,以免其扩充力量后对我反噬,过去在这方面有了许多教训,一句话,不作无条件的援助。

3、极力加强对于一切爱国进步分子的工作,推动他们反对投降,必须要有很多有能力干部去工作,一刻不能放松。对中间分子也要加紧工作。


(十一)六中全会后,中央各部的工作

三年来,特别六中全会以来,逐渐恢复与建立了中央各个工作部门,除军委会、组织部、宣传部、敌区工作部,青委建设工作较久外,统战部、妇委、工委,财政经济部、干部教育部、总政治部、党报委员会、华北华中委员会、西北委员会、南方委员会都是新建立的,秘书处亦改进了组织,共十六个机关。这些中央工作部门的恢复与建立,是在长征破坏后的一大成绩,一大令人高兴的事。

(十二)中央组织问题

1、政治局会议以后每月一次。2.书记处处理日常事件。3、通知,大事用中央名义,小事用书记处。4、人员分配,以保证中央领导健全,同时加强地方党委为原则。5、中央各部既紧缩又加强。

(十三)干部教育

两年来,在中央直接指导下,建立的抗大、陕公、党校、马列学院、鲁艺、青训班、女大、工人学校、卫生学校、通讯学校、组织部训练班、行政人员训练班、边区党校、鲁迅师范、边区中学、鲁迅小学、儿童保育院等十七所学校,学生多的万余人,少的几百人,几十人,几千个干部从事干部工作,教育出来的及尚未出来的学生三万以上,这是一个很大的成绩,十八年未有过的现象。这些学生现在还不能看出他们大的工作成绩,但数年以后就可以看见了。今后仍应继续这个方针,……地办。去华北的指挥管理仍属中央,但委托北方局监督之。

(十四)学习运动

1、六中全会以后,中央发起的全党干部学习运动,提高全党干部的理论水平,有头等重要意义。

2、党政军民学各种机关的在职干部,均应一面工作,一面学习。

3、按其程度,文化与理论或并重或偏重。

4、是一种长期大学校。

5、每日二小时学习制。

6、一面工作,一面生产,一面学习。

7、自动与强制并重,理论与学习一致。

8、勤学者奖,怠惰者罚。

9、各级机关学校部队均设干部教育领导的机关与人员。

(十五)生产运动

一切可能地方,一切可能时机,一切可能种类,必须发展人民的与机关部队学校的农业,工业,合作社运动,用自己动手的方法解决吃饭、穿衣、住屋、用品问题之全部或一部,克服困难,以利抗战。

边区今年的生产运动是认真进行了的。须继续总结经验,达到解决困难之目的。

(十六)青年运动

六中全会以来,有成绩,青委和青年联合办事处集中指导青运是对的。

青运方针:坚持青年统一战线,继续采取“五四致三青团书”的态度,用以争取青年的多数,在继续抗战与民主共和国的旗帜之下。

(十七)妇女沄动

中央发布了妇女运动的方针,开始建立了妇委的工作,地方妇女工作指导机关亦正逐渐建立中。我们历来最缺乏的是妇女干部,妇女运动经验亦没有总结,这个缺点必须补救。没有一批而专职的妇女工作干部,要展开妇女运动是不可能的。

(十八)工人运动

发布了工运工作方针,开始建立了工委,开办了一个工人学校。工人运动在群众这动中是比较薄弱的,党的成分中工人亦太少,今后应该在这方面有进步。

(十九)锄奸工作

新环境下,敌人破坏我党我军与边区政府的阴谋,已逐渐显得严重,今后必更严重,大大增加了锄奸工作的重大性。因此全党必须增加这一方面的注意力,必须派出与训练有必要数量的有能力的干部到这个部门中去。

凡国民党员,因特务工作或武装向我进攻而被逮捕或俘虏时,一般以不杀不降为原则(不降为不强迫其写自首书,不强迫其放弃自己的信仰),以争取国民党的大多数。

(二十)党的组织问题

1、组织一一现在不是普遍发展的时候了,一般应停止发展,以精干为原则。2、严密。3、消除破坏分子,保卫党。4、阶级教育,马列主义。5、严格执行秘密训令。

(二十一)自力更生

党、军队,一切。准备对付最恶劣的环境。

(二十二)华北问题

华北局面有变到极严重的可能,敌人主力有进攻华北可能,因此,八路军与华北党必须严格注意这种情况的可能到来;而从军事、政治、财政,党的组织,统一战线各方面进行准备,以适合坚持游击战应付最困难为原则。

(二十三)华中问题

必须大大发展华中的党与游击战争工作,从复杂的环境建立自己的根据地,以为全国抗战的枢纽地带。必须有更好的统一战线工作,并与新四军工作的发展与进步联系起来。边区的干部以主要输送华中为原则。

(二十四)保卫边区

1、边区是基本根据地一一中央所在地,全国有威信的地方,必须坚决保卫之;

2、敌人进攻边区是促进中国投降的一步骤;

3、敌人有进攻延安之可能;

4、即使延安失守,仍然坚持边区,大家准备过游击战争生活,最艰苦但是最生动的生活;

5、吃饭是第一个问题,自力更生克服困难;

6、学生与工作人员大批上前线;

7、好好保护与教育青年学生、新干部、新党员;

8、保存一个领导集团,一个教育集团,一个军事集团,依靠很好地形,很好的民众,一定有办法。

(二十五)动员大批学生与工作人员上前线

1、加强教育,加强前线工作;

2、毕业后依然上前线,早上前线为强;

3、因为敌情;

4、因为经费;

5、决定去一万人;

6、政治上,组织上好好动员;

7、后方便于领导教育,与打游击战争,但人员还是不少,

8、干部须作适当的分配;

9、学生必须酌量多留一些于后方;

10、学校名义必须保存,前后方同。

(二十六)七次大会

1、八月一日前选举完毕;

2、十月开会;

3、为团结全党反对投降而斗争;

4、能够来与能够开的。

(二十七)两条路线斗争

1、反对右的。不看见投降与反共危险,屈服于国民党的压力,对时局无出路(失去前途),没有阶级立场,在艰苦斗争面前恐惧消沉。

2、反对“左”的黑暗的中国,黑暗的世界,黑暗的边区,黑暗的自己,准备破坏统一战线,否认国共合作,不要三民主义,自己孤立起来盲干。

3、老干部应做基干,带领广大新党员,新干部,向抗战的更高阶段前进,向困难阵地攻击。

(二十八)团结战胜一切

1、必须更加团结。

只要中央与高级干部是团结,全党必能团结。只要共产党团结,必能无坚不破。全国人民的救星是共产党。

2、必须更加集中:

减少不必要的民主。个人服从组织,少数服从多数,下级服从上级,全党服从中央。

3、全党团结一致,集中指挥,战胜一切。

结论

(六月三十日)

1、时局问题

(一)投降因素中中国地主资产阶级的动摇是主要的,这是对的。

(二)要克服两种投降一一(A)克服小部投降。(B)克服大部投降。A是目前的,是大部抗战之下的;B是将来的,是大部投降了的情况之下的。但做了目前工作即是便利于做.将来工作,二者统一不可分。

(三)接着六中全会的决议,敌我力量平衡才是相持阶段,那么,不但现在不能说相持阶段,如出现李精卫,则是敌人已组成了(他正在努力组织)他的战略预备军,并使用之于作战一一迂回我抗战阵地之后方(民族反革命是帝国主义战略同盟军,这是无疑的),这种时候,更不是敌我相持,而是我又打了一个大败仗,那时特点,打我的是日本帝国主义又加李精卫(中国弗朗哥),故较过去的战略退却更加严重,亦不是什么相持阶段。

(四)日本帝国主义正在组织两支同盟军,一支是国际投降主义者,一支是中国投降主义者,前者用以包围中国外部,后者用以迂回中国内部,我们的努力方向,就是要动员国际的国内的反投降势力,打败这二支敌人的同盟军。在没有打退之前,没有什么相持阶段。

(五)正面的敌人(日本帝国主义的主力)还可能进行军事攻击,说已经没有这种可能性了,是不对的。它在财政经济人力军力各方而虽困难,但依实力说,依时间说,都还有军事进攻的可能。所以不但一方面须要打仗,从侧面进攻敌人,一方面还要准备继续打败从正面进攻的敌人,才能取得敌我平衡,取得相持阶段的到来。

(六)但这不是说,要把这些正面的侧面的敌人在实力上打到各方面都与我抗日军相等了,才能出现相持阶段。不是实力相等,而是=我之实力加敌人弱点加国际控制相持阶段。

(七)于是相持阶段能在三种情况下出现:

1、地域大,人数多,阵线巩固,造成敌我相持。例如,象国内战的中期,欧战的中期。抗战到现在的中国,地域大,人数多,但阵线不固(政治腐败,投降者捣乱),故还没有相持阶段,如能努力克服投降危险,并在这种克服中改进政治,阵线巩固了,相持阶段就来了,这是第一种。

2、地域小(谓直接根据地小,但国度大),人数少,但阵线巩固,也能相持。例如十年红军战争,某种程度来说,一个时期内的东北。将来假定有大部叛变之事,留下的小部不同造成相持,只要能不断打破围剿,相持形势就有了。目前的华北局势,可以说是暂时的小局部的相持,把华北形势延长下去,就变成长期小局部相持了。

3、由上述小部相持到大部相持,经过统一战线的扩大与反围剿斗争的胜利而达到。如果那时还不能举行战略反攻(还是战略反攻期间),仍然是相持阶段。

(八)假如大部叛变之事不可免地要在抗战行程中出现的话,那么抗战的过程也许就要以下述公式组成整个相持阶段:武汉以后的暂时大部相持(目前局势在某种意义上,可以说是相持),但因阵线不固,故只能是暂时的――某时以后的小部相持一一某时以后再来一个大部相持。

这种变化的主要种点,就是由阵线不巩固到阵线巩固。要长期相持,就要阵线巩固;有此条件,虽小亦能相持,无之虽大也不能。所以应力争数量之大(人多、地大),但主要的还在质量之强。所谓阵线巩固,就是抗日根据地之巩固和统一战线之巩固。

(九)华北确有相持形势,但这不是暂时的,我们提出“坚持华北游击战争”是想把暂时的变为长期的.如果别处都黑暗了,剩下华北(同样,新四军、陕甘宁边区),那它虽在地域上人数上是局部的,但因已没有别的抗日军了,它成了唯一的抗日军,也就取得了战略相持的意义。这是困难局面,但我们也应准备。大半天天都黑了,剩下共产党抗日,我们也要干下去。这样干三年五年,唤醒人民,吸收友军,又逐渐形成新而大的抗日阵线,我们就成了抗日救国的核心与领导力量。

所以不论如何说,我们的前途是光明的,虽然道路是艰难的。

(十)目前,正是敌我关系(敌我力量)重新改组的时候,看谁争取多数,更看谁的质量较好,以决胜负谁属。谁胜谁败,这个问题目前未决,双方正重新准备条件。

阵线必然会改组,也必然要改组的。我之同盟军的一部(地主资产阶级一部)改变为敌人的同盟军,但我决不让其全调去,我必须争取其一部,也可能争取一部。我还有广大同盟军没有动员,中国农民、日本工农兵、国际无产阶级、苏联的力量,动员起来,便能制敌死命。

敌人是反革命的,我是革命的,这个质的不同,决定最后胜利之谁属。但指挥战争的策略,尤其决定胜利之谁属。没有后者,单靠本质优良,胜利仍然没有的。

2、转变问题
(一)我们说准备自己应付新的可能情况,应付可能的米亚伽,是在六中全会总路线之下的,在六中全会路线下,准备对付可能的米亚伽(或哈柴),与准备在可能的米亚伽占领地区采取暂时的退却政策。到此为止,不能设想对六中全会来一个一百八十度的转变。不是路线的转变,而是战术或策略之局部的暂时的转变,以防意外的袭击,这一点要首先弄清楚,我们要坚持过去的总路线。

(二)目前要极力争取的,用党的全力去争取的,是克服投降可能,争取多数抗日,拥护并帮助并监督并批评国民党与蒋,使之能够从反汪斗争中从今后发展中克服投降倾向,这是目前的中心任务。

(三)因此应该强调团结,强调统一,强调国共长期合作,而不是强调其他东西,也只有强调这些才能克服投降危机,也只能强调这些才能更好的准备自己应付可能的事变,一切能孤立投降而不是使自己孤立。

3、几个策略问题

(一)巩固党问题(组织上的紧缩政策)。去年三月会议决定大大发展党之后,党已在全国有了大数量的发展,现在的任务是巩固它,故须暂时的一般的停止发展,当然不是一个也不准入党,有些地方,还是应发展。停止发展为的是便于清理、除奸和教育,将来还是要再发展的。

(二)党应保护新党员新干部。主要责任在老党员老干部身上。新之党员新老干部之间一定要弄好,如弄得不好,老党员老干部负其主责。有完全的理由给新党员新干部以原谅,而对老党员老干部则完全没有这种理由。此问题过去存在着某些不正确观点,这事,与主力部队和地方游击部队的关系大略相同。

要从新党员中提拔大批的干部。


(三)加紧党内教育。阶级教育与民族教育统一,但目前应着重阶级教育同时不忽略统一战线教育。

要编中级课本,哲学问题上要着重唯物史观。

(四)要提拔地方干部,没有地方干部――中级的、高级的,不能建立巩固的根据地。

反对“钦差大臣”的倾向,反对轻视“土包子”。取消“土包子”这个口号。

(五)党应保护革命知识分子不蹈过去的复辙。没有革命知识分子革命不能胜利的。国民党和我们力争青年、军队.一定要收容大批革命知识分子。要说服工农干部,吃得下,不怕他们;工农没有革命知识分子帮忙不会提高自己;工作没有知识分子不能治国、治党、治军,政府中,党部中,民众运动中也要吸引革命知识分子。

(六)友党友军中停止发展党并撤退党问题。从大局看停止与撤退是有利的,否则将因此大事妨碍大局,破裂统一战线。不争取中国军队革命不能胜利,而争取,目前的是从政治上,若组织党则妨碍政治争取。靠嘴巴不靠组织(暂时情况下),靠上层不靠下层。一面撤退党,一面加强联络,同意统一战线工作的某种独立性。一般只在真投降者的军队中,政治党中取组织上的破坏政策,其余一般只取政治上的争取政策。某些特殊部门不执行撤退办法。

(七)不杀国民党员,有重要意义。不是不杀汉奸,不是不杀某些叛徒。(以下缺稿)

