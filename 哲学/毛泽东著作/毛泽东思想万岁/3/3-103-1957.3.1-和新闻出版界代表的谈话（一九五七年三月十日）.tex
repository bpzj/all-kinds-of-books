\section[和新闻出版界代表的谈话(一九五七年三月十日)]{和新闻出版界代表的谈话}
\datesubtitle{(一九五七年三月十日)}


三月十日下午三时至七时,毛主席在他的办公室和新闻出版界代表举行座谈。参加这次座谈的有:上海《新闻日报》和中国新闻社代表全中华,《大公报》代表王芸生,上海《家汇报》代表××…谈话开始,主席先请上海方面的光外人士发言。主席问上海的《申报》是什么时候取消的?听到答称解放后《申报》就没有出版,主席便说,《申报》取消没有道理,人家是出了几十年的老报纸嘛!现在敢做什么报呢?(有人答道:改做《解放日报》了。)主席说:改了就不好吧,如果把它改回来又好像是复辟了,但是这问题好好研究一下。

接着《文汇报》的×××提出问题。他说去年七月《人民日报》改版后,上海的报纸也改了一下,但是改了之后,问题也很多,大家都感到自己的马克思主义水平低,在社会主义社会办报心中无数。

主席说:现在心中无数,但是慢慢就会心中有数。一切事情开头的时候总是心中无数的。打游击战,未打之前,我们就连想也没有想过,逼上梁山,非打不可,只好硬着头皮打下去。当然,打仗这件事情不是好玩的但是打下去慢慢就熟悉了。对于新出现的问题,谁人心中有数呢?我也心中无数。就拿朝鲜战争来说吧,打美帝国主义就和打其名日本不相同,最初也是心中无数的,打了一两仗,心中就有数了。现在我们要处理人民内部矛盾问题,不像过去搞阶级斗争(当然也夹杂一切阶级斗争),心中无数是很自然的。无数并不要急我们可以把问题好好研究一下。谈社会主义的书,虽然出了那么多,教我们怎样去具体搞社会主义的书,在俄国搞社会主义革命的时候就还没有。也有些书把社会主义的什么东西都拿出来,但那是空想的社会主义,不是科学的社会主义。事情没有出现,虽然可以预料到,都不等于能够具体提出解决问题的方针和办法。

马克思主义修养不足,这是普遍的问题,解决这个问题,只有好好地学,当然,学是要自愿的。听说有些文学家十分不喜欢马克思主义这个东西,说有它,小说就不好写了。我看这也是“条件反射”。什么东西那是旧的习惯了,新的就钻不进去,因为旧的思想把它顶住了。说学了马克思主义,小说不好写,大概是因为马克思主义跟他的旧思想抵触,所以写不出东西来。

我们中国有500万左右的知识分子。知识分子多,有好处也有坏处。坏处坏在成堆。《人民日报》就是这个样子,知识分子堆在一起,毛病就多起来了;但是堆在一起也有好处-知识集中。在知识分子中提倡学习马克思主义是很有必要的,要提倡大家学它十年八年,马克思主义学的多了,就会把旧思想推了出去。但是学习马克思主义也要形成风气,没有风气是不会学好的。

目前思想偏向有两种,一种是陈其通、马寒冰他们几个那一类的教条主义,一种是钟惦棐那一类的右倾机会主义。右倾机会主义则把凡有怀疑的都一棒子打回去,肯定一切。听说陈其通这人还好,马寒冰就很霸道。他拿了文章跑到《人民日报》,一声“圣旨到”,邓拓就双膝跪下了。(××插话:当时他写了文章来,一进门,就说他们有些意见,要想争鸣一下,希望文章不要改动)马寒冰的文章十分教条主义,我就看不下去,简直强迫受训。钟惦棐这名字很古怪,他的文章倒能看下去。教条主义和右倾机会主义都是片面性,都是用形而上学的思想方法去片面、孤立地观察问题和了解问题。当然,要完全避免片面性也很难,但是思想方法片面、孤立,和没有好好学习马克思主义有关系。我们要用十年、八年的时间来努力学习马克思主义,逐步抛弃形而上学的思想方法,那样,我们的思想面貌就可能有很大的不同。

主席说到这里。叫×××再说下去,×××说:现在报上开展批评也有困难,反批评往往简单化,太粗暴。《文汇报》上开展电影问题的讨论,初时老演员、老导演参加讨论十分热烈,后来电影领导部门的同志写文章,一枪把批评打回去,他们就很后悔,认为自己出头发表意见,是上当了。

主席说:这次对电影的批评很有益,但是电影局开门不够,他们的文章有肯定一切的倾向,人能一批评很有益,现在的电影,我就不喜欢看,当然,也有好的,不要否定一切。批评凡是合乎事实的,电影局必须接受,否则不能改进。你们报上(指《文汇报》)发表的文章,第一个时期批评的多,第二个时期肯定的多,现在可以组织文章把他们统一起来,好的肯定,不好的批评。电影局不理是不对的。这次争论暴露了问题,对电影局和写文章的人都有益处。听说撤钟惦棐的职是不是?

(××答:是他自己要辞《文艺报》评论员的职。)主席说:我看也不必撤职,电影的问题本来就很多,老演员,老导演一肚子气,应该让他们发泄发泄。

(×××说:报上登的东西,读者意见不一致的,有些青年说《文汇报》登些琴棋书画之类的东西,他们不要看,写信来抗议。主席说:你们的报纸编得活泼,登些琴棋书画之类,我不爱看,青年不爱看可以不看。各有各的“条件反射”。一种东西,不一定所有人都爱看的。

(《新闻日报》金仲华提出意见说:日常生活问题出现紧张现象的时候,人民群众提出意见的很多,报纸一封也不登,群众有意见,报纸登了,政府和有关部门的人又有意见,说是刺激群众,造成更加紧张。不知该怎么办才对?主席说:可以试试看。(意思是说,也可以登一些出来看反映么样。)政府和有关的业务部门有意见,报馆可以和她们研究商量一下,在报上加以解释,再看结果如何。一点不登恐怕不大好,那样业务部门会官僚主义,不去改进工作。

(主席问到上海报纸的销路么样?金仲华答:《新民晚报》扩大报导面之后销数增加了很多。这时座中有人提到“心中有领导,编报无自由,读者不爱看;心中无领导,编报就自由,读者就爱看”这句“怪话”。王芸生接着说:这个说法是原则性的错误。)

毛主席说:这也要具体分析。报纸是要领导的,但是领导要适合客观情况,马克思主义是按情况办事的,情况就是包括客观效果。群众爱看,证明领导得好;群众不爱看,领导就不那么高明吧?有正确的领导,有不正确的领导。正确的领导按情况办事,符合实际,群众欢迎;不正确的领导,不按情况办事,脱离实际,脱离群众,使编报的人感到不自由,编出来的报纸群众不爱看,这个领导一定是教条主义的领导。我们要反对教条主义。中国革命也是这样。第三国际不灭亡,中国革命不会胜利。列宁在生的时候,第三国际是领导得好的。列宁死后,第三国际的领导是教条主义的领导,(领导入斯大林、布哈林就不大好。)只有季米特洛夫那一段领导得好。季米特洛夫所作的报告,是很讲道理的。当然,第三国际也有功劳,就是帮助各国建设党。后来教条主义不顾各国的特点,一切照搬俄国,中国就吃了大亏。我们用整风方式搞了十多年,批判了教条主义,独立自主地按马克思主义的精神实质办事,才取得中国革命的胜利。列宁也是不承认第二国际的,结果十月革命胜利了。我们不要再搞国际了。情报局成立之后,只做了一件事情,就是批评了南斯拉夫。(康生:还批评了法国和日本。)但也不是从此不要,但要就要像第三国际初期那样,各国独立自主,按情况办事,不要干涉别人。这些话,我和许多苏联同志谈过,和尤金、米高扬都谈过。

(有人问到报纸应不应该专业化的问题。)

主席说:有些专业化也好。好像《大公报》那样,开放自由市场的时候,我就爱看他因为他登这一类东西的多,又登得快。但是太专业化有时很枯燥,人家看的兴趣就少。专业的人也要看专业之外的东西。(有人提到现在报纸上的东西太硬,反映了最近上海对于报纸问题讨论到的一些意见。例如说“思想性多了,报纸就不活泼”,又有人提出“软些,软些,再软些”的口号等。)主席说:社会主义国家的报纸总比资本主义国家的报纸好。香港的报纸虽然没我们说的思想性,但也没有什么意思,说的话不真实,夸大,传播毒草。我们的报纸毒少,对人民有益。报上的文章,“短些,短些,再短些!”是对的,“软些,软些,再软些”要考虑一下。不要太硬,太硬了,人家不爱看,可以把软和硬两个东西统一起来。文章写得通俗、亲切,由小讲到大,由近讲到远,引人入胜,这就很好。你们赞成不赞成鲁迅?鲁迅的文章就不太软,但也不太硬,不难看。有人说杂文难写,难就难在这里。有人问,鲁迅现在活着会怎么样?我看鲁迅现在活着,他敢写也不敢写。在不正常的空气下面,他也会不写的。但更多的可能是会写的。俗语说得好,“舍得一身剐,敢把皇帝拉下马。”鲁迅是真正的马克思主义者是彻底的唯物论者。真正的马克思主义者,彻底的唯物论者是无所畏惧的,所以他会写。现在有些作家不敢写,有两种情况:一种情况是我们没有为他们创造敢写的环境,他们怕挨整;还有一种情况,就是他们本身唯物论未学通。是彻底的唯物论者就敢写。鲁迅的时代,挨整就是坐班房和杀头,但是鲁迅也不怕。现在的杂文怎样写,还没有经验,我看把鲁迅搬出来,大家向他学习,好好研究一下。他的杂文方面很多,政治、文学、艺术等等都讲,特别是后期,政治讲得最多,只是缺少讲经济的。鲁迅的东西,都是逼出来的。他的马克思主义也是逼着学的。他是书香门第出身。人家说他是封建余孽,说他不行。我的同乡成仿吾他们,对他就不好。国民党压他,我们上海的共产党员也整他,两面夹攻,但鲁迅还是写。现在经济方面的杂文也可以写。文章的好坏,要看效果,自古以来,都是看效果作结论的。

(说到这里,主席问《光明日报》的代表,问他是不是共产党员?那位同志回答说,“是共产党员”。)主席说:“共产党员替民主党派办报,这不好嘛”。主席又说:你们的报纸还可以看看,副刊多。

跟着,主席问新华总社的朱穆之同志,你们的新闻受不受欢迎?听说你们那里有人提出通讯社的消息有没有阶级性的问题。朱穆之同志回答之后,主席说:在阶级未消灭之前,不管通讯社抑成报纸新闻都有阶级性。说:“新闻自由”是骗人的。完全客观报导是没有的,美国的通讯社和报纸现在也报导一下新中国经济建设的情形。原因是它想做生意,所以故意做些姿态出来给人看,因为经济危机压迫着它。必要时,蒋介石不也会做出些姿态来的,他也放和谈空气。因为美国压迫他,要用更加亲美的人如胡适等来代替他,他放出和谈空气,使美国不敢压他压得太厉害,现在美国学了我们过去那一套,过去我们联合民主党派孤立蒋介石,现在美国联合胡适等去孤立蒋介石,拆蒋介石的台,倒过来,却由我们去“保护”蒋介石了。蒋介石不垮比垮了好,垮了,胡适等更加亲美的分子上台,那更不好。蒋介石放和谈空气,是为了抵住美国的压力,我们不要揭露他的批评他。他放我们也放嘛。当然,蒋介石还是反共的,还是要骂我们的,他不骂就没有资本了。

中华书局的舒新城(已退休3年,最近复出)反映了目前出版界的情况和他到湖南长沙一带视察所见的情况。他认为人们说目前出版的书又“缺”又“滥”,其实缺是主要的。因此他要求中央设法解决纸张供应问题,他又说目前中央各部都搞专业出版社,编辑,出版等整套业务都各搞一套,是否有此必要,可否只负责本部专业书刊的编辑工作,其他业务拨出来统一搞,以节省人力、物力,又各地大量搜集档案、旧书回炉,把大量资料性的档案、书报都毁掉了,文物保管委员会只管文物,不管资料,这样下去,将来要找过去的资料也困难。

主席对纸张供应问题说:轻工业部是否考虑多投资一点,这是不会赔本的。(文化部有一同志说:已请示过总理,计算了一次,增加投资无可能。)主席打断他的话说,人家谈的是要纸,你说没有纸,大家不要听,不要说吧。对于档案,旧书回炉问题,主席说,这是焚书的新形式,一看这样,别省也这样,恐怕浙江一带更多了,(向××说)这个问题值得注意一下。

谈到这里,××发言。他说:主席今天和出版界的谈话,将使中国报学史开一新纪元希望主席对于新闻工作的根本性问题给予指示。×××认为目前报纸工作的关键问题是如何正确地开展批评的问题。他说这个问题过去长期存在,但是会后在正确处理人民内部矛盾问题的时候报上开展批评搞得好不好,对处理人民内部矛盾关系很大,因此可否想出几条准则来规定一下?康生同志也插话说,现在报上开展批评碰到很多困难,不是有人反映说,“批评真军人,说你压制新生力量;批评老干部,说你立场不稳;批评民主人士,说你破坏统一战线”吗?

主席说:批评的时候,要为人家准备楼梯,否则群众包围起来,他就下了楼。反对官僚主义也是这样。“三反”的时候,有许多部长就是中央给他们端了梯子接下来的。“三反”初期,说打出那么多老虎,后来一查,又说只有2%到3%了。肃反也有类似情况。“三反”实际上是整共产党,“五反”才是整资本家。过去搞运动是必要的,不搞不行,但是一搞又伤人太多。我们应该接受教训。现在搞大民主不适合大多数人民的利益,如果你推翻人民政府,蒋介石就要回来。有些人对别人总想用大民主,想整人,到末了整自己,民主就越小越好,我看在文学、新闻等方面解决问题要用小民主,小民主之上再加一个“小”字,就是毛毛雨,下个不停。整风运动明年正式开始,今年先发一个指示,让大家有个准备,有一个非正式时期。在这个非正式时期,你有主观主义、官僚主义、宗派主义,自己检查出来,把它改了,以后就不追究。不久以前,陈伯达同志回到他的家乡福建搞基层选举,就是这样的,有些干部贪污了,叫他把钱吐出来,向群众认错,结果群众批评了他之后;干部仍旧当选,这个经验很好。共产党整风免不了会波及民主人士,但是千万不要整死人。要用小小的民主的方法,先整共产党。现在我们有些同志装腔作势,他们没有“本钱”,又要做官,不摆架子就不行。共产党整,大家就谦虚了。

说到办报,共产党不如党外人士,延安办报,历史也很短,全国性办报就没有经验,办学,搞出版,科学研究都是这样,全国有500万左右的知识分子,共产党员不过是一个小指头。我们拿手好戏是打仗,专政,文艺有一点,人也不多,说共产党不能领导科学,这话有一半真理。现在我们是外行领导内行,搞的是行政领导,政治领导,至于具体的科学技术,譬如地质学,共产党是不懂的,但是国民党也不懂。国执政20多年只造就200多个地质人材。我们解放七年,就选了一万多。行政领导就是“一个李先念加一个李富春”,一个管吃饭,一个管规划。现在是过渡时期。只好这样,这个情况,将来是要改变的。现在要争取80%以上的中间状态的知识分子来学习马克思主义,要求他们对马克思主义有个初步的了解,而不是要求他们一下子贯通。马克思主义的创造者是马克思本人,也不是一下子就会全部贯通的,1848年,《共产党宣言》出版,只是马克恩主义体系的开始,还不是马克思主义体系的完成。要求知识分子一下子都接受马克思主义,这个要求是不现实的。说懂得马克思主义,其实懂得的程度也不相同。我读马克思主义书籍也不多,不晓得他们写了多少,大概我们翻译过来的还不到一年。(人民出版社的曾彦修同志说:翻过来的47%)作为专家是要读多一点的,我们没有那么多功夫,读少一点也可以,主要的是注意研究方法。现在很多干部都没有读书的习惯,把剩余的精力放到打卜克、看戏、跳舞上面去。大家不应该把时间浪费掉。

你们这次提出的问题很多,大家回去研究一下,试试自己来求得解答。开会的时候,就是要党内外的人在一起。共产党不要关起门来开会,合起来开有好处,这是辩证法。可以把两方面的意见统一起来。(康生同志插话:这种开会方式大家还不习惯。党外人士有些顾虑,党内的人有些吞吞吐吐,主席说的是好办法。)主席说:你们回去试试行不行得通,行不通我就来个强迫命令。至于人数问题,我看党外人士可以占三分之一。共产党员占三分之二。

马寒冰他们几个人的文章,方针不对,方法也不对。他们的方针是反对党中央的方针,他们用的是压的方法,不能说服人。(康生:现在对于中央的这个方针,也有三派,陈其通、马寒冰他们的文章代表“左”的一派,怀疑这个方针不好,认为以后不要宣传这个方针。右的一派则不管三七二十一,认为在报上闹闹有好处,闹闹有好处,但是要看如何闹。还有中间的一派,承认方针好,但是心中还不大有数,有畏难情绪。)主席说:怕什么呢?如果批评青年人帮助了青年人,批评老干部帮助了老干部,批评了民主人士巩固了统一战线,这样不是很好吗?(康生同志又说:现在部队干那对这个方针思想混乱,认为部队进行的正面教育,只能开香花,不能放毒草。)主席说:事实上也有毒草,只不过不以毒草之名出现罢了。因为每个人都是把自己的意见当作香花开的。当然部队应该和其他部门有所不同,部队是下命令办事情的。但是我们的部队本来就有民主讨论,士兵可以批评军官,答复不满意可以再批评。部队不是也有贪污浪费和军阀主义吗?因此部队也要整风,不过搞得很乱就不成,要有领导的来开展。过去三反的时候,戏院“打虎”可以挂个牌牌出来,停几天不演戏,报纸可不能这样,有缺点可以开个会讨论,但是不能够明天不出版。(康生同志插话:对部队来说,应该“开言路,讲效果”。)主席说:这很好,各处都要这样。

关于新闻的快慢问题,在这次宣传会议上,新闻工作者参加了讨论。对于这个问题,主席说:对具体问题要作具体分析,新闻的快慢问题也是这样,比如禁烟运动的消息我们就不是快登慢登的问题,而是干脆不登的问题。因为美国在联合国大会上,污蔑我们卖鸦片,我们登了,不是正好供给他们宣传资料么?土改新闻也是这样,我们在报上不宣传,免得传播一些不成熟的、错误的经验,前年年底北京几天就实现了全行业公私合营,宣布进入社会主义。本来对这样的消息就要好好考虑,后来新华社一播,(新华总社朱穆之同志说,是广播电台先播。)各地不顾本身具体条件,一下子都干起来,就很被动。又如匈牙利事件当中纳吉登台这件事情,我们不明情况,又沉不住气,早登了三天,结果第一天登出的消息,没有说他好和坏,第二天的消息说他好,第三天又说他坏,群众弄得莫明。所以,本来情况不明,根本可以不登,这点法共就比我们高明,《人道报》在巴黎出版,四面都是资产阶级报纸,它未弄清情况,就是不登。

谈到再培养一些人写文章的时候,主席说:一定要找出能够写文章的人,而且现在已经找出了些。谈到批评人民内部事情的文章应不应该尖锐?主席说:对人民内部进行批评,锋芒也可以尖锐。我也想替报纸写些文章,但是要把主席这个职辞了才成。我可以在报上辟一个专栏,当专栏作家。文章要尖锐,刀利才能裁纸,但是尖锐得要帮了人而不是伤了人。

对于当地有关政策性的可不可以批评?主席说:有些也可以批评。听说新疆有一个县城,商业系统就开了24间公司,《大公报》你们不是搞商业报导的吗?你们就可以搞几个典型来批评一下。(《人民门报》的××说:报纸处理人民内部矛盾问题处理得不好,影响就不好。但是我们过去缺乏这方面的经验,文风生硬,现在情况不同了,文风不能适应。)主席说:过去就不适应了,有些文章使人看了不舒服,不讲道理,一出来就要压人。(××接着说:这个当然和我们报纸工作同志的思想方法和思想作风有关。刮风往往是由报纸刮起的。报纸自己也意识到这一点,极力设法避免刮风,避免片面性。譬如这次宣传节育和晚婚,《人民日报》一开头就注意到防止片面性,可是消息、文章一登多了,下面就发生问题。)主席说:文章一多了,就以为修改婚姻法,赶快去结婚。这样报纸也实在难办,在旧社会里,报纸上的东西,老百姓看了等于不看,现在报纸上一登情况可不同。(××又说:譬如百家争鸣问题,有人以为争鸣的主要限于学术思想,实际工作不好讨论。这两者之间的关系也很容易混淆,比如关于劳卫制和因材施教的讨论,报上文章一多,有些学生和教师的思想就引起混乱,恐怕要划个范围才好。)主席说:完全学术性的,争来争去不会有影响,至于政策性的,恐怕就分别一下情况,但是划范围也有困难,因为政策那么多。如果一发现节育晚婚的宣传产生一些不良后果,那么报上可以写文章来解释说明,我们的文章,就是往往不及时。至于范围怎样划法,各报可以自己回去研究。

(新华总社的朱穆之同志说:现在问题很多。是否可以专门开一次会来讨论总结一下呢?)

主席说:这次只是提出问题,过一会还要开会讨论。这次会议也是“五湖四海”,效果如何,将来再看。我在最高国务会议上讲话所谈的问题,本来在心里积累了很久,去年已经讲了几次,后来又看了些事情,看了陈其通、马寒冰他们的文章,想到会有人以为他们的文章是代表中央的意见,因此觉得有好好谈谈的必要。因为《再论无产阶级专政的历史经验》只解决了国际问题,现在我们国内(大规模疾风暴雨的)阶级斗争基本结束,人民内部矛盾突出,于是就有一股风,说批评多了,说人民闹事,惶惶不可终日;另外有些人又觉得还不过瘾,有些人要收,有些人要放。中央的方针到底怎样,大家都要来摸底。其实中央也没有什么另外的底,方针就是那么一个,不过有了新问题。罢工罢课都是人民内部的问题。罢课是因为去年招生太多,一招多,有些人恐怕招不够,于是就骗人。骗人,学生自然不满意。问题凑起来,就显得严重。这样的事情,今后还有。人民内部,绝大部分是小资产阶级,一部分是民族资产阶级,有许多民主党派,还有无党无派民主人士。现在是社会大变动,思想混乱就是反映了这个大变动,不反映出来倒是不可理解的。官僚主义是闹事的直接原因,因为官僚主义不肯改,群众就会闹事。中国人民是最守纪律的人民,上海副食品供应那么紧张,我们把情况摆出来,把道理说清楚,叫大家想办法,结果今年的春节不是过得很好吗?现在过渡时期还没有结束,每天都会发生大大小小问题。(×××说:我们过去和新闻界也联系不够。)主席说:不是没有联系吧?”今后我们每年都要开个会。我们今天还是过得比较好的,苏联1917年十月革命,到了1927年还很乱,文学艺术更乱,他们更穷,知识分子更叫,到现在已经搞了三十九年,商品也不见得比我们多。谈到这里,有人提到过去对苏联的宣传是否存在片面性的问题。

主席说:当然苏联有缺点,我们不要登,印度的缺点,我们也不登,现在就是两方的资本主义国家宣传。对于这些,我们就是要片面性。有人说西德的生产比东德快,你们为什么老是说西德不如东德。可是我们在宣传上还是不要说西德比东德好,不过也不要说东德什么都比西德好。我们社会主义国家也是有缺点的,因为我们历史短,马克思主义首先是在资本主义链条薄弱的地方突破,取得革命胜利的。我会经对德国的同志说过,现在马克思在东方很忙,暂时还不能回去,所以他们那里的革命不能成功。现在亚洲政治上比英美进步,因为亚洲人的生活比英美的差得多。我们过去受剥削,很穷,穷就要革命,他们生活水平高,文化水平高,就是不革命。自然,美国也有好处,历史短,没有历史上的负担,他们学起历史来,就不像我们那样费劲,也用不着去讨论历史分期的问题。有时坏事也会变成好事。现在是东方先进,西方落后,以后再过几十年,东方国家把西方的帝国主义赶走精光,到那时候,他们没有殖民地可以剥削,油水就不多,而我们东方国家却富起来了。他们的生活水平降低了。人民就会进步。现在亚洲闹起来了,非洲闹起来了,如果拉丁美洲也闹起来,那就好办了。


