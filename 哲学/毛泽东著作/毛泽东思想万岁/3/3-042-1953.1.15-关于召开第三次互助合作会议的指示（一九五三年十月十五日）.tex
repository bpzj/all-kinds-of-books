\section[关于召开第三次互助合作会议的指示(一九五三年十月十五日)]{关于召开第三次互助合作会议的指示}
\datesubtitle{(一九五三年十月十五日)}


办好合作社,即可带动互助组大发展。

在新疆,无论大中小县,都要在今冬明春,经过充分准备,办好一个到两个合作社,至少一个,一般一个到两个,至少三个,根据工作好坏而定。要分配数字,摊派多了冒进,少了右倾,有也可以,没有也可以,那就是自流了。不要超过三个?只要合乎条件,合乎章程,决议,是自愿的,有强的骨干(主要是两条:公道,能干),办得好,那是韩信收兵。责成地委,县委用大力去搞,一定要搞好。中央局,省市委农村工作部就要抓紧这件事,工作重点要放在这个问题上。

要有控制数字,摊派下去,摊派而不强迫不是命令主义。十月开会后,十一月、十二月,明年一、二月,北方还有三月,有四、五个月可搞,明年三月开会检查。这次就交代清楚,明年二月是要检查的,看看完成的情况怎样。

个别地方是少数民族区,又未完成土改,可以不搞。个别县,工作很坏的县,比如说落后乡占百分之三十至四十,县委书记很弱,一搞就要出乱子,可以暂缺,不派数字,但是省委地委要负责帮助整顿工作,准备条件,明年秋收以后,冬季要搞起来。

一般规律是经过互助再到合作,但是直接搞社也可以允许试一试。走直路,走得好,可以较快的搞起来,为什么不可以,可以的。

各级农村工作部,要把互助合作社这件事,看做极为重要的事。个别农民,增产有限,必须发展互助合作农村阵地,社会主义不去占领,资本主义必然去占领,难道说既不走资本主义,又不走社会主义?资本主义道路也可以增产,但时间要长,而且是痛苦的道路。我们不搞资本主义,如果又不去搞社会主义,那就要两头落空。

总路线,总纲领,工业化,社会主义改造,十月开会,你要讲一下。

“确保私有财产”“四大自由”,都是有利于富农和富裕中农的。为什么法律又要守呢?(法律是说保护私有财产,无确保字样)法律不能废除,现在农民卖地。办法就是合作化,互助组还不能阻止农民卖地,要合作化,要大合作社才行,大合作社也可以使得农民不必出租土地了,一二百户的大合作社代几户鳏寡孤独,问题就解决了。小合作社是否也可以带一点,应加以研究。该搞中的,就搞中的;能搞大的就搞大的,不要看见大的就不高兴。一二百户的社算大的了,甚至也可以是三百户。在大社之下设几个分社,这也是一种创造,不一定去解散大社。所谓办好,也不是完全都好。各种经验都要做好,不要用一个规格到处套。

老区应该多发展一些。有些区可能此老区发展得快,例如,关中可能比陕北发展得快,成都坝子可能比阜平那些地方发展得快,要打破新区一定慢的观点。东北其实不是老区,南海与关内后解放的地区也差不多。可能江苏杭州嘉湖一带赶过山东,华北的山地老区,而且应该赶过。新区慢慢来,一般可以这样讲,但有些地方干部强,人口集中,地势平坦,为了搞几个典型,可能一下子较快的发展起来。

华北、山东,华北现有六千,翻一翻——摊派,翻二翻——商量,一翻半或两翻,华北也是这样。控制数字不必太大,地方可以超过,超额完成,情绪很高。

发展合作社也要做到数多,质高,成本低。所谓成本低,就是不出废品;出了废品,浪费了农民的精力,落个影响很坏。政治上蚀了本,少打粮食。最后的结果是要多打粮食、棉花、甘蔗、蔬菜等等。不能多打粮食,是没有出路的,于国于民也都不利。

在城市郊区,要多产蔬菜,不能多产蔬菜也没有出路的,于国于民也都不利。城市郊区土地肥沃,土地平坦,又是公有的,可以首先搞大社(集体农庄)。当然要搞的细致,种菜不像种粮,粗糙更不行。要典型试办,不能冒进。

城市蔬菜供应,依靠个体农民进城卖菜来供应,这是不行的,生产上要想办法,供销合作社也要想办法。大城市蔬菜的供求,现在有极大的矛盾。

粮食、棉花的供求,也均有极大的矛盾,肉类,油脂不久也会出现极大的矛盾,需要大大超过,供应不上。

从解决这种供求矛盾出发,就要解决所有制与生产的矛盾问题。是私有所有制,还是集体所有制?是资本主义所有制,还是社会主义所有制?

个人所有制的生产关系与大量供应是完全冲突的。个人所有制,必须过渡到集体所有制,过渡到社会主义。合作社有低级的,土地入股;有高级的,土地归公,归合作社之公建。一个所有制有两种,劳动人民的和资产阶级的,改变为集体所有制和国营(经过社会主义和资本主义的合营,统一于社会主义)这才是提高生产力,国家工业化。生产力发展了,才能解决供求矛盾。


