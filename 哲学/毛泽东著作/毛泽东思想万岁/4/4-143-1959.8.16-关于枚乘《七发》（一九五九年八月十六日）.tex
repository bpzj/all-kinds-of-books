\section[关于枚乘《七发》(一九五九年八月十六日)]{关于枚乘《七发》}
\datesubtitle{(一九五九年八月十六日)}


此篇早已印发,可以一读。这是骚体流裔,又有所创发。骚体是有民主色彩的,属于浪漫主义流派,对腐败的统治者投以批判的匕首。屈原高居上游,宋玉、景差、贾谊、枚乘略逊一筹,然亦甚有可喜之处。你看《七发》的气氛,不是有颇多的批判色彩吗?“楚太子有疾,而吴客往问之”,一开头就痛骂上层统治者的腐败。“且夫出舆入辇,命曰蹶痿之机。洞房清宫,命曰寒热之媒。皓齿蛾眉,命曰伐性之斧。甘脆肥脓,命曰腐肠之药。”这些话一万年还将是有理。现在我国在共产党的领导之下,无论是知识分子,党、政、军工作人员,一定要做些劳动,走路、爬山、游泳、广播体操都在劳动之列,如巴甫洛夫那样,不必说下放参加做工、种地那种更踏实的劳动了。总之,一定要鼓足干劲,反右倾。枚乘直攻楚太子:“今太子肤色糜曼,四肢痿随,筋骨挺解,血脉淫濯,手足惰窳;越女待前,齐姬奉后,往而游宴,纵恣乎曲房隐间之中,此甘餐毒药,戏猛兽之爪牙也。所从来者至深远,淹滞永久而不废,虽令扁鹊治内,巫咸治外,尚何及哉!”枚乘所说,有些像我们的办法,对犯错误的同志,大喝一声:你病重极了,不治将死。然后,病人几天或者几个星期,或者几个月睡不着觉,心烦意乱,坐卧不宁,这样一来就有希望了。因为右倾或“左”倾机会主义这类毛病,是有历史根源和社会根源的,“所从来者至深远,淹滞永久而不废”。这个法子,我们叫作“批判从严”。客曰:“令太子之病,可无药石针刺灸疗而已,可以要言妙道说而去也,不欲闻之乎?”指出了要言妙道,这是本文的主题思想。此文首段是序言,下分七段,说些不务正业而又新奇可喜之事,是作者立题的反面。文好。广陵观潮一段达到了高峰。第九段是结论,归到要言妙道。于是太子高兴起来,“涊然汗出,霍然病已。”用说服而不用压服的方法,用摆事实讲道理的方法,见效甚快。这个法子有点像我们的“处理从宽”。首尾两段是主题,必读。如无兴趣,其余可以不读。我们应当请恩格斯、考茨基,普列汉洛夫,斯大林、李大剑、鲁迅、瞿秋白之徒,“使之论天下之精微,理万物之是非”,讲跃进之必要,说公社之原因,兼谈政治挂帅之极端重要性。马克思“览观”,列宁“持筹而算之,万不失一”。我少时读过此文,四十年不理它了。一日忽有所感,翻起来一看,如见故人。聊効野人献曝之诚,赠之于同志。枚乘所代表的是地主阶级的较低层,有一条争上游,鼓干劲的路线。当然,这是对于封建阶级上层、下层两下阶层说的。不是如同我们现在是对社会主义社会无产、资产两个阶级说的。我们争上游鼓干劲的路线代表无产阶级和几亿劳动农民的意志。枚乘所攻击的是那些泄气、悲观、糜烂、右倾的上层统治的人们。我们现在也还有这种人。枚乘,苏北淮阴人,汉文帝时为吴王刘濞的文学待臣。他写此文,是为给吴国贵族们看的。后来,“七体”繁兴,没有一篇好的。《昭明文选》所收曹植“七启”。张协“七命”作招隐之词,跟屈、宋、景、枚唱反调,索然无味了。


