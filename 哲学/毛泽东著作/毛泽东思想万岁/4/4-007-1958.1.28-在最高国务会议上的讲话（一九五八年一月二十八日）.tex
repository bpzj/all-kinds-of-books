\section[在最高国务会议上的讲话(一九五八年一月二十八日)]{在最高国务会议上的讲话}
\datesubtitle{(一九五八年一月二十八日)}


今天的国务会议是临时召集的。

大家不要因为上午八点开会就认为有大事。过去多在下午,这是我心血来潮,商量一个普通问题。

八年以来,讨论国家预算这一次是最早的一次。以后也要每年在这时候开会。

这次人代大会,要开得从容些,要开好一点。多开小组会,多做些准备工作.少开大会,真正把问题搞清楚,修正工作上的缺点和错误。做报告的人来没有?(答声,未了。)做了报告不要第二天就发表,报告了,让大家提修改意见,讨论修改后再发表。

我看了七、八年了,我看我们这个民族大有希望。特别是去年这一年,我们六亿人口的民族精神,大大发扬。经过大鸣大放大辩论,把许多问题搞清楚了,任务提得更恰当.如十五年左右可在钢铁和其它重工业方面赶上英国,多快好省;农业发展纲要四十条的修正重新发布等,给群众很大的鼓励。许多事情过去做不到的,现在能做到了。过去没有办法的,现在也有办法了,比如除四害,群众劲头很大。我这个人老鼠捉不到,苍蝇、蚊子可以捉它一下。平常总是苍蝇蚊子向我们进攻嘛!古代有这么一个人写了一篇提倡消灭老鼠的文章。现在我们要除四害,几千年来,包括孔夫子在内都没有除四害的志向,现在杭州市准备四年除去四害,有的提二年、三年、五年的。

所以我们这个民族的发展大有希望。悲观论是没有根据的,是不对的,要批判悲观论者。当然不要打架,要讲道理,是具有希望,不是中有希望,小有希望,更不是没有希望,而是大有希望,文章在“大”字上,日本人讲:“大大的有”。(笑声)

我们的民族在觉醒,像我们大家在早晨醒来一样。因为觉醒了,才打倒了几千年来的封建制度,以及帝国主义和官僚资本主义,执行了社会主义改造,现在整风、反右派又取得了胜利。

我们的国家是又穷又白,穷者一无所有,白者一张白纸,穷是好的,好革命,白做什么都可以,做文章,画图样,一张白纸好做文章。

要有股干劲,要使西方世界落在我们的后头,我们不是要整掉资本主义思想吗!西方要整掉资产阶级思想不知要多长时间。西方世界又富又文,他们就是太阔了,包袱甚重。资产阶级思想成堆。要是杜勒斯愿意整资产阶级的风,还要请我们做先生。(笑声)

一谈起来,我们国家这么多人口,地大物博,人口众多,四千多年历史,但现在生产与我们的地位完全不相称,钢铁生产还不如一个比利时。它有七百多万吨钢,我们只有五百二十万吨。总之。我们是个历史长久,优秀的民族,可是钢是那么低。粮食北方一百多斤,南方三百多斤。识字人那么少。比这些都不行。但是我们有股干劲。要赶上去,在十五年内赶上英国。

十五年要看头五年,头五年要看前三年,前三年要看头一年,头一年要看头一个月,更看前冬,去年中共三中全会就在水利、积肥上做了布置。

现在劲头鼓起来了,我们的民族是个热情的民族,现在有了热潮,正好有一比,我们民族像原子,把我们民族的原子核打破,释放热能,过去做不到的事,现在也能做到。我们这民族有这么一股劲,十五年要赶上英国,要搞四千万吨钢(现在五百多万吨),要搞五亿吨煤(现在是一亿吨),要搞四千万瓧电力(现在是四百万瓧),要发展十倍,所以要发展水电,不光发展火电。实现农业发展纲要四十条还有十年,看来不要十年,有的说五年,有的说三年,看来八年可以完成。

要达到这个目的,在这种形势下要有一股干劲。我在上海,一个教授和我谈《人民日报》社论《乘风破浪》,他说,要鼓起干劲,力争上游就是从上海上四川,上游得费点劲,不是下游。说得很对,我很欣赏这个人,这是好人,这人有正义感。有人批评我们“好大喜功,急功近利,卑视过去,迷信将来。”这几句话恰说到好处,“好大喜功”,看是好什么大,喜什么功?是革命派的好大喜功,还是反革命派的好大喜功?革命派里又有两种:是主观主义、形式主义的好大喜功,还是合乎实际的好大喜功。我们的古人都说:“福如东海,寿比南山。”都是好大喜功。我们是好六亿人民之大,喜社会主义之功,这有什么不好呀,急功近利,也不是不好呀!曾子曰:“吾日三省吾身。”这是圣人之长。大禹惜寸阴,陶侃惜分阴,像我们这样人要惜分阴,不能老开会,几个月不散会。急功近利,要看是搞个人突出、主观主义,还是搞合乎实际、可以达到的平均先进定额?要搞平均先进定额,如亩产量,有先进、中间、落后,都搞先进的为定额,以大力士为定额,那不行,是在先进定额中加以平均。

至于卑视过去,不是说过去没有好东西,过去是有好的东西,但是否对过去那么重视,老是天天想禹、汤、文、武、周公、孔子,我不赞成那样看历史。如过去用木船,现在就可以不用了,可以用轮船,郑州的建筑物太古老了,总是新的东西好,北京的房子,就不如青岛好。外国的好东西,为什么不可以搬来,铁路就是外国的嘛!这个东西(敲扩音器)也是外国的嘛!外国的好东西要学,应该保存的古董一定要保存,要挖,把它保存起来。推出午门以外斩首,那是老落后。有的认为城墙不要拆,有的主张可以拆,我看可以拆。用石头做工具才四千年不到五千年,那时发明细石器,像现在发明原子弹一样,是了不起了,那时的英雄可以骄傲得很,可是现在不能用石器。为什么要把古老的东西保持下来?石器起过进步作用,而且最大,是否现在要回到石器时代?我看人类历史是前进的,一代不如一代,前人不如后人。右派分子说“今不如昔”,应当倒过来!今天比过去好。有的人为了拆城墙伤心,哭出眼泪,我不赞成。但北京的城墙不拆也可以,南京、济南、长沙的城墙拆了我很高兴,有些老人就伤心啊!伤心哉,秦欤,汉欤,近代欤?北京的城墙保存一千年,一千年以后还是要拆。你们不要以为我这个人什么都轻视。在某种意义上不要对过去太重视。“迷信将来”,我们的目的是为了将来。如开会,现在讲,将来就是散会,老开会不行,人民代表大会,开上十几天就想散会了。我们把希望寄托于将来是对的,但不能迷信。

所以上边上海那个教授的话是对的。

陈铭枢说我“偏听偏信,好大喜功,喜怒无常,轻视古董”。好大喜功我已经讲过了,至于偏听偏信,陈铭枢是叫我听梁漱溟、陈铭枢的,我不能偏听右派的,是偏听共产党,还是偏听国民党、杜勒斯。君子群而不党,没有此事。孔夫子杀少正卯。就是有党。是因为少正卯同他争学生,孔夫子就给少正卯定了五条罪状。(问在座的人:那五条?有人答……)

我们对右派都不杀,所以不偏听偏信是不可能的。陈铭枢你过去好,我就喜欢,现在你成了右派,我就愤怒,这还让我喜什么?说我不像个主席的样子,我这个人就是不像个主席的样子。还说我轻视古董,古代的东西都好吗,我劝青年不要搞旧诗,不要那么重视古董。

人多好,人少好?人多一些好么,现在劳动需要人。但是要节育,现在是:第一条控制不够,第二条宣传不够,目前农民还不注意节育,恐怕将来搞到七亿人口时就要紧张起来。现在不要怕人多,有人怕没饭吃,那我们大家就少吃一点,人多一点,士气旺盛,这是我有点乐观,不是地大物博吗!但我不是说不要宣传节育,我是赞成节育的。要像日本、美国那样节育,不要像法国那样节育,越节越少。邵先生六道讲的对,现在不对,达到极点就趋向反面。人多没饭吃,就少吃点。据说东方人吃素对身体健康有益处,这是黄道之学(黄炎培)。。中国人平均每月吃肉三斤,二人六斤,匈牙利每人吃二十多公斤,这是我们社会主义阵营的国家,除匈牙利外,帝国主义国家吃肉多,都肉食者鄙。我们吃四钱油,五钱盐,也行。至于提倡吃素,我看不行,因为理论与实际脱节,可见黄道之学不学也可。过去孔夫子很讲究排场,食不厌精,每餐要吃点姜,闹脑溢血。我看还是少吃点好。吃那么多,把肚子胀那么大干啥。像漫画上画外国资本家那样。

我这都是说的一些问题,请大家考虑。

有两种领导方法,一种比较好一点,一种比较差一点。这两种方法,不是说杜勒斯一种,我们一种,而是都搞社会主义,有两种领导方法,两种作风。合作化问题,有人主张快点,有人主张慢慢来,拖到七、八年才搞。我认为前一种好,还是趁热打铁,一气呵成好点,不要拖拖拉拉。整风好,不整好?还是整风好,还是大鸣大放好。我们说鸣放,右派说大鸣大放,我们说鸣放是指学术上说的。他们要用于政治,所以“大鸣大放”这个提法是从右派那里借来的,可见小鸣小放不行,中鸣中放也不行,就是要大鸣大放。

要改掉官气,官是可以做的,但要打掉官气。最好根绝官气。我们都是做官的.都有点官气,官气是一种坏习惯,不是好习惯。不论什么大官,主席也好,总理也好,都应以普通劳动者姿态在人民中出现,使工人、农民感到和他们平等,我们自己说平等靠不住,要使对方感到平等。改掉官气不是很容易的,有官气就要改掉,先从共产党起,民主党派也可以逐渐改掉。湖北红安县的领导干部过去就有官气。世界上有个中国,中国有个湖北省,湖北省有个红安县,过去这个县叫黄安县,因为黄字不好改为红安县,这个县的干部以前官气十足,农民看不惯干部,还有三多,说皮鞋多,大氅多,自行车多,是否还有打扑克多。后来他们改了,穿草鞋到乡下去,农民很欢迎。现在干部下乡,山东的老百姓讲,“八路军又来了。”可见这六、七年来官气十足,做了官有了架子,因此要整风,要整掉官气,民主党派也要整风。写《水经注》这个人了不起,写得那么好。孔夫子也是官气十足,他有两匹马一辆车,每天坐在车子里摇摇摆摆,得了胃病,叫胃下垂,而且还要吃细的。类似狮子之类吧。他吃多了,有砂子,不干净,所以得了胃病。孔子到了齐国,人家骂他四体不勤,五谷不分。我看骂红安县以前有些干部也是这样,所以中央机关干部每年要有四个月要离开北京。北京不是好地方,历来出官僚的地方。为什么孙中山先生不建都在北京呢?大概是因为这个地方出官僚。北京不出产任何东西,我不是指北京这个地方,是指中央机关,中央机关不生产钢,不出水泥,不出粮食,也不出纸烟,什么也不产生。产生思想吗?也不产生,思想也是从群众中来的。不是北京出的。我说不产生任何东西,是指不产生任何原料。原材料是产生自工人、农民,章伯钧要搞政治设计院那不行,一切要从群众中来。原材料来自工农,我们是加工,我脑子里不产生任何东西,一跑出北京就取得了东西,产生出力量。

要鼓干劲!鼓舞士气,劲可鼓,而不可泄,应当鼓舞士气。合作化一搞,有人叫得不得了,说搞多了,要砍掉十万个,双轮双铧犁在南方名誉不太好,在湖北等四省还好。大家看过登徒子写的好色赋没有,就是攻其一点不及其余,说登徒子的老婆很丑,别人谁都不要的,脸上有麻子,耳朵很大,还有痔疮,结果生了五个儿子,宋玉以此证明登徒子好色。因为登徒子告了宋玉一状,说宋玉很漂亮,好色,请楚王注意。我这里不是替登徒子翻案,是讲这个方法不好。右派就是这样攻击我们的。但好人也有的这样看。我们大家都要注意,有那么一天,攻你们一点。比如王云五在国民党时期当财政部长时,他说:“我没有研究过财政,还想学习。”结果人家就说:你没学,你就不能当财政部长。

现在是一场新的战争,向自然界开火,要革命球的命,从我们这里到杜勒斯那里,直径12,500公里,乘3.1416……,要大家努力,现在是革命尚未完成,同志仍需努力。我们不能老整风,整风后目标要转向技术革命,我们只能革地球表面的命,空间还不行,现在我们抛卫星还不行,要改造地球表面,实现第二个五年计划还差一点,实现第三个五年计划就差不多了。要认真学习,要搞试验田,农业要搞,工业也要搞。工厂的干部每礼拜一天,半天,真正当个学徒工,这有什么困难呀,文学也要学一点。你是科学家文学家也要学,由郭沫若当老师,过去我不看《人民日报》,像蒋介石不看国民党《中央日报》一样,现在《人民日报》七整八整好了一点。

政治思想革命还要革,不能松劲,技术革命现在不登报,一登有的就会说,整风不要整了。要坚持整风,一鼓作气,再而衰,三而竭。放松整风不利于社会主义,不利于民主党派,不利于改进工作。社会革命还要天天革,整风还要整,六个月可告一段落,并不是说可改造好了,以后还要整。

关于右派分子,我想开个右派分子大会,你们赞成不赞成?今天我们约了个右派分子参加会议,费孝遖来了吗?(应声;来了。)请费孝通参加会,我是寄希望于他,最高国务会议请右派分手参加这像什么样子啊,最高国务会议请右派分子参加不违犯宪法,因为宪法有规定,开最高国务会议,主席要请什么人就请什么人。右派分子做了好事,就是他们说了假话。对右派分子,第一要感谢,感谢他们向党进攻,引起了人民的愤怒,感谢右派是因为他们当了教员;第二,是帮助(监督)。所谓帮助,是三七开,十个人有七个人可以改造,逐步转变过来,经过五年到十年的时间,其中大部分能够转变过来的,规定时间,给以帮助,多数是有可能变好的。如不相信多数,就没有信心了。对人民的事业丧失信心是不对的。但总有一部分人不变,不变的人,只有带到棺材里去。像章、罗,要像鲁迅说的:“横眉冷对于夫指,俯首甘为孺子牛。”不变也好,有它的用处,它的用处就是不变。我们不怕它,因为它人数少。我们对右派的批判必须是全面的、深刻的,对右派分子的斗争是严肃的,但处理要宽大点,不要宽大无边,要给他们留条路,这是为了教育中间分子,也是为了教育他本人。现在的大学生,百分之七十到百分之八十是剥削家庭出身的,但右派只占百分之二到百分之三,对他们除个别的以外,都不开除学籍,用这种政策可以把他们改造过来。

再就是共产党大改革,说干什么,就干什么,说整风,就整风。整风就大鸣大放。整得不够就再整,民主党派也要改革,人的思想是可以改变,整个社会都变了嘛。

我主张不断革命论,你们不要以为是托洛茨基的不断革命论,革命就要趁热打铁,一个革命接着一个革命,革命要不断前进,中间不使冷场。湖南人常说:“草鞋无样,边打边像”。托洛茨基主张民主革命未完成就进行社会主义革命,我们不是这样。如一九四九年解放,接着搞土改,土改刚结束,就搞互助组,接着又搞初级社,然后又搞高级社。七年来就合作化了,生产关系改变了。随着就搞整风,趁热,整风以后,就搞技术革命。像波兰、南斯拉夫建立民主主义秩序,搞七、八年,出了富农。可以不建立新民主主义秩序,还要团结一切可能团结的力量。“长期共存,互相监督”还要有。民革有人说,民革的右派占百分之十二,十个指头有八个半是好的,当然不会有半个。十个人有一个是右派,那么还有九个不是右派,并且就是右派,也是批评从严,处理从宽。

去年七月我与费孝通谈,他说他那时才感到孤立。你(指费说)现在还孤立吗?(费答:孤立。)知识分子在某一方面来讲是没有知识的,对知识分子的骄傲自满应该批判,知识分子像孙行者一样,不要把尾巴翘得像旗杆那么高。罗隆基说:“小知识分子不能领导大知识分子。”我看工人阶级小知识分子领导大知识分子,这是条真理,工农出知识。除马克思、列宁是大知识分子外,我不算。费孝通到过英国,我就没有条件到英国。我去年讲过:皮之不存,毛将焉附?帝国主义、封建主义、官僚资本主义,这三张皮都剥掉了,知识分子的毛就要附在工人阶级这张皮上,有时沾上来了,有时沾上一点,有时在天宫中,梁上君子。我看知识分子要恭恭敬敬夹起尾巴向无产阶级学习,所谓(罗隆基说)“三颐茅庐”、“礼贤下士”、“士为知己者死”、“士可杀不可辱”、“温良恭俭让”都是封建的东西。我们一直讲知识分子要改造,七、八年都这样讲。知识分子一面说共产党英明领导,一面向我们进攻。英明领导,猖狂进攻,口喊“万岁!”进攻,喊万岁时,总有人在那里骂娘,同仇敌忾。接受共产党的领导,宪法规定,各党派也承认,但是还要搞两套。过去很多人不相信,现在很多人相信了。傅作义先生相信了吗?现在要帮助他们,要互相帮助,要公开讲,不要背地讲。什么要结束共产党的领导,搞阴谋,这不行。我们釆取和平改变(转变),国际上没有先例。三、五反是场严重的斗争,资产阶级工商业者,他们谨慎了,比较老实一点。但是知识分子还骄傲得很,一跳跳到一万公尺那么高,这须扑登跌一下,很必要,使他们受教育。我们要右派分子向人民投降,写降表,但他们写假降书是不行的。

在统一战线内部,不管共产党和民主党派,要互相帮助,要讲直话。要当面讲,不要背后讲,要去掉疑心,每个人要把心交给别人,不要隔张纸,你心里想什么东西,交给别人。鲁迅的作品很好,他把他的心与读者交流。不能像蒋介石那样做法,总是叫人不摸底。“逢人只说三分话,未可轻抛一片心”,这不适合今天的社会的。我有点东西就先卖出去。

我开了支票,在人代会上再讲讲,我这支票也不一定兑现,如果代表们有兴趣,就讲讲。还讲这套。

知识分子失败一次没有坏处。

我们当年红军有三十万人,走了二万五千里,剩下二万多人,蒋介石把我赶到山上。他没有料到,他办了好事。我当时一看蒋介石手里有枪,我也要有,我要从你手里拿枪,蒋委员长就当了运输队长。

一九二七年蒋介石清党,赶我们“上山为寇”。后来,就是抗日战争时期,我们要求当一家人,大公报王芸生写了个《不要另起炉灶》。我们请蒋委员长封官,就可以不另起炉灶,你得给饭吃嘛!我说得加个但是,要是不给饭吃,就另起炉灶,你不封,我就自己封自己,上山为寇,落草为王。

第二次王明路线,害得我们两只脚,走了两万五千里。陈独秀是右的,王明是“左”的。你们听说过吧,唐朝有个什么诗人写的诗:“一朝权在手,便把令来行。’

这一次是二万五千里长征,严重的挫折才教育了我们。

知识分子不受严重的挫折是教育不过来的。你们民主党派,民主,很高明,我过去就说过,共产党还出高岗、饶漱石,你们就没有,你们总以为我说这话是怕你们出奸臣,以为看你们不起(一人插话:没有。),啊,也许我是以小人之腹,度君子之心。我把心交给你们了,你们没有交给我。现在我抓住你们的小辫子了,摆在人民面前的右派就不少。我们都是旧社会来的人,在座的恐怕都是清朝人吧,我看这里我们清朝人占优势哟!全国人民已振奋起来,我们这些人要适应这种情况,适应六亿人民的要求,相信能适应这种情况。因为全中国人民都在进步,有一股热气,在这样的环境中生活是有利于进步的。

十五年赶上英国是可能的,要鼓起干劲,力争上游。我就是偏听偏信,看听信谁的。要节省,要反浪费。我们一面要提高生活,一面要节省,反对浪费。一万年也要节省。反浪费大有文章可作。作官可以,不要官气,以普通劳动者的姿态出现。主要干部要四个月离开北京,去求神拜佛,到工农群众中去。工农群众出钢铁,出粮食,弄点东西同来就加工,成为政策法令,不要以老爷姿态出现,你们看过“四进士”的戏没有?四进士的戏,有我们老毛家的一个毛朋,就是神气十足,巡按出朝,地动天摇。

劲,可鼓而不可泄。有了缺点错误,用大鸣大放的方法来纠正,不要泼冷水。有人批评好大喜功,那么能好小喜过吗?能重视过去,轻视将来吗?要好大喜功。要鼓励士气。

检查工作,一年四次,有些可以一年检查十二次,一年十二个月嘛!老鼠、麻雀、蚊子,一年检查十二次,看你干不干。

革命尚未全成,同志仍须努力。

政治和业务要配合,要又红又专。红讲的是政治,专讲的是业务,要红色的业务家,不能要白色的业务家。你说你不是白色的,那么是灰色的,也不行;不是灰色的,是桃红色的,也不行。搞政治的人,如只红而不那么专,红也不那么真红,是空头政治家。当然有些人情况不同,比如年龄大,……凡情况许可的人都可以专,同时要更加红起来。在我们这个国家要有几百万、上千万的知识分子。苏联知识分子就那么多。美国就搞他不赢,据说美国博士也有那么好弄的,当然也有是用功的,如杨振宁。

我们搞上层建筑的,不出原材料,要到外边去取,我们加工。

要改造右派,要帮助。要改革,这是激烈的改革,各民主党派要注意。

要把心交给人。

要釆取不断革命的方法。

公私合营,敲锣打鼓,黄炎老你没料到,我也没料到。抗战后,民主革命才三年半的时间就把蒋委员长赶到台湾,我也没料到。世界是变化的,两个卫星上了天,谁也没料到,我就根本不懂。现在那边很被动,我们这边很主动。过去苏联面上有灰,两个卫星上了天,脸上也光彩了。双轮双铧犁能用,我要为他恢复名誉而奋斗。什么合作化不行,四十条不行,双轮双铧犁也抹黑了,这跟斯大林一样倒霉。

不讲了,大家讨论讨论,提出意见。

