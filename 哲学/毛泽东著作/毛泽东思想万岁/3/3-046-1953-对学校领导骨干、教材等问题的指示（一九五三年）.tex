\section[对学校领导骨干、教材等问题的指示(一九五三年)]{对学校领导骨干、教材等问题的指示}
\datesubtitle{(一九五三年)}


l、领导骨干:大专院校,中学必须有骨干。骨干配备大专由中央组织部负责,中学由地方负责,从宣传部门,青年团干部下去。有了好校长就有好教师。

2、教材问题:集中力量编,语文、历史等原则问题,由陈伯达同志等解决。

3、整顿小学:防止过左,教育发展的不平衡性,要照顾农村特点。私塾也要让其存在。小学有三类:(1)中心小学、完全小学;(2)一般小学;(3)速成小学。小学问题主要便利农民子弟入学。教师只要不是现行犯,不做反革命宣传就可存在。对超龄学生不应赶出,首先教好。民办可办,不要求太严,群众需要自愿,学费可收。五年一贯制暂缓执行,提早了。小学生不能都升学,这不叫失学,要加强劳动教育。

4、学生建康,高中大部分要有助学金,要增加营养,减轻负担。

5、教育工作重点和关键:重点,高等学校和高中,关键问题是师资。中学要重点办好一批。如何办好高中,这是我们的重点任务。

6、招生要注意成分。

7、速成识字法,在农村有局限性。应是群众自愿的业务教育。

8、中央教育部任务:(1)方针计划;(2)教材。

9、总的方针和工作重点:过去主要毛病是盲目冒进,不能适应建设,我们任务是培养干部和逐渐(适应)生产发展,目前在培养干部。


