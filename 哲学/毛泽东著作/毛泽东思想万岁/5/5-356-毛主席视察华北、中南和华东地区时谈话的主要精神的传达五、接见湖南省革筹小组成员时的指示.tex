\section[毛主席视察华北、中南和华东地区时谈话的主要精神的传达五、接见湖南省革筹小组成员时的指示]{毛主席视察华北、中南和华东地区时谈话的主要精神的传达五、接见湖南省革筹小组成员时的指示}


我们最最敬爱的伟大领袖毛主席九月十八日来到长沙,接见了湖南省革命委员会筹备小组黎源、华国锋、章伯森三同志,对湖南省的无产阶级文化大革命作了极其重要的指示。

毛主席号召我们迅速实现革命的大联合。毛主席说:工人阶级内部,没有根本的利害冲突,为什么要分两大派呢?两派要互相少讲别人的缺点,多作自我批评,才有利于革命的大联合。

毛主席教导我们要正确对待受蒙蔽的群众。毛主席说:两派都是工人,一派造反,一派保守。保守是上头有人蒙蔽了他们,对受蒙蔽的群众,不能压。

在汇报到湘潭情况时,毛主席说:这样多的产业工人,不会一辈子保皇。要正确对待,至于他们的头头,靠下面自己起来造反。

毛主席教导我们要正确对待干部。毛主席说:干部大多数是好的,我们要团结大多数,包括犯错误向群众作了检讨的,除了极少数坏人,打击面太宽了不好。对干部除投敌、叛变、自首者外,过去十几年,几十年总做过一些好事嘛!要扩大教育面,缩小打击面。进行批判斗争时,要用文斗,不要搞武斗,不要侮辱。

毛主席教导我们造反派要加强学习。毛主席说:对造反派也要进行教育,现在正是犯错误的时候。年青人,不要性急,现在紧张、严肃有余,团结、活泼不足,缺乏民主作风,不平等待人。打人、骂人、拍桌子,把我们的传统搞乱了,把我们坚持“团结——批评——团结”的公式搞乱了。

(以上根据黎源、华国锋、章伯森同志传达)

开始汇报湘潭问题时,毛主席说:这么多工人不能一辈子保皇,要正确对待,黑头头由他们受蒙蔽的群众自己起来造反。

谈到常德、安江问题时,汇报这些地方农民进城的问题。主席说:打打也好,受受教育嘛。许多农民进城不容易,现在是十五个工分,有的是抽签去打,有的是两元钱,有的是一百元打一次仗。

谈到内战问题时。有人反映,有的组织现在争人数,认为人多就要以它为核心。主席说:那也不一定。

谈到抢枪时。主席说:抢枪把你们吓得不得了。名义上是抢的,有的是发的,政干校就是把枪发给造反派了。主席还说:抢枪不要怕,民兵的枪就有××万条。

谈到解放军下工厂,造反派武装保护解放军时,主席说:这也是个经验。

谈到“高司”问题时。主席问到姓万的(指万达),群众不同意万达到革筹小组,主席说:等一段。

谈到军队有一段不能拿枪时,主席说:解放军也五不了。(此时原有四不了,现在是五不了)  (主席算了一下,湖南军队两个月没有拿枪)主席说:开大会军队可以武装参加。

谈到火车不通时,主席说:不通的反面就是通,社会上和内部有压力。

主席问湘潭王治国时,华国锋汇报了王被斗的情况,主席说:为什么斗得那么厉害呢?他身体不好嘛,军区错了影响到军分区,也影响到县人民武装部。

谈到公检法问题时,主席说:好像没有公检法就不得了,它一垮台我就高兴。

谈到干部问题时,主席说:清理干部要搞群众运动,这样多的军分区、人武部,不是文化大革命是搞不动的。主席又说,“……最后要团结大多数,包括犯了错误向群众作了检讨的,除极少数的坏人外,打击面太宽了不好。”

谈到对犯过错误的干部进行训练时,主席说:只要犯过错误的?军队的、党的机关都要训,要吸收红卫兵参加。

谈到红旗军问题时,主席说:再研究一下嘛,看一下,恢复了再说,世界上的事不要怕。  (张春桥同志插话,红旗军问题,林彪同志去年十一月批过文件,说服他们和别的组织联合,或参加到别的组织。)

谈到湘潭成立临时革筹小组,主席说:十月份都要搞起革筹小组来。 (张春桥同志插话说。江西也成立了临时小组,驻军召开了全省会议)主席说:开会也是训练。

谈到造反派团结时,主席说,两派要少讲别人的缺点,过去军队和地方的关系,军队老是先做自我批评,这样就好了嘛。 (张春桥同志插话说:主席语录中有这么一段话) 主席又说:“两派是工人,一派造反,一派保守,我想总是上头有人。对保守的不能压,越压越反抗,蒋介石压我们,我们就有希望了。一压就压出三十万共产党,三十万红军,后来是自己犯错误才有二万五千里长征。”

谈到长沙的碉堡工事时,主席说:不放心就保存一个时期嘛! (×× ×插话,水泥的不要拆,也是一个备战)。

陪同主席接见的有;张春桥、  ×××、汪东兴等同志。


