\section[关于外国朋友参加文化大革的指示(一九六七年一月二十八日)]{关于外国朋友参加文化大革的指示}
\datesubtitle{(一九六七年一月二十八日)}


外国朋友真正革命的可以参加无产阶级文化大革命运动。与周总理谈夺权问题

\kaoyouerziju{ (一九六七年一月)}

毛主席问周总理夺权怎么样?公安局是专政机关。

总理:才夺权一天多。

主席:要抓典型。总理:市局委开了会,夺权有几种形式;干部是当权派;(一),受黑帮影响很坏,变黑帮,(二)走资本主义道路当权派,(三)顽固坚持资产阶级反动路线,(四)承认错误,但还有严重错误,(五)有个别一般错误(这种人为多)。

主席。前两种要划小,要孤立打击极少数,接管本身就是革命。建立新的,根据不同情况,也有不同形式:(一)全部改组(上海:张春桥、姚文元);(二)接管后对当权派不同形式处理,边检讨边工作,监督留用(根据指示工作),(三)停职留用,(四)撤职留用,(五)撤职查办。

总理:哪种办法好,撤职一面斗争,一面留用,有了对立面,就可壮大队伍。把许多事压在身上(指革命造反派)也很被动,留用一面斗争,一面工作。科学院左派队伍壮大了,抓革命促生产搞得很好。

主席;让那些当权派扫街,扫完了就休息,睡大觉,太便宜他们了,便宜事都叫他们办了。不要把自己陷入事务之中,要注意这个问题。要掌握大权监督他们。一个单位几个战斗队观点不同不奇怪。有事商量比不商量奸。接管是大事情,会引起一系列的变动。要解决接管的目的,解决什么问题,接管的方法(遇到问题怎么处理。)要有具体政策(局、科、部、科员等怎么办?)现在夺权了,也许还会夺走。有的单位夺过来夺过去是个锻炼,要巩固住,主要靠左派力量壮大。左派力量小时,夺权小,夺过去很快要夺走,左派要壮大。我支持夺权的,夺权后一定要抓革命促生产。


