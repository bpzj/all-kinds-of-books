\section[在八届十一中全会闭幕式上的讲话(一九六六年八月十二日)]{在八届十一中全会闭幕式上的讲话}
\datesubtitle{(一九六六年八月十二日)}


关于第九次大会的问题,恐怕要准备一下。第九次大会什么时候召集的问题,要准备一下,已经多年了。八大二次会议到后年就十年了,现在需要开九次大会。大概是在明年适当的时候再开,现在要准备。建议委托中央政治局同它的常委来筹备这件事,好不好?

至于这次大会所决定的问题,究竟是正确的还是不正确的,要看以后的实践。我们所决定的那些东西,看来群众是欢迎的,比如中央的一个重要决定就是关于文化大革命,广大学生和革命的教师是支持我们的。而过去那些方针广大的革命学生跟革命教师是抵抗的,我们是根据这些抵抗来制定这个决定的。但是究竟这个决定能不能实行,还要靠我们在座的与不在座的各级领导去做。比如依靠群众吧,一种是实行群众路线的,一种是不实行群众路线。决不要以为决定上写了,所有的党委,所有的同志都会实行,总有一部分人不愿意实行,可能比过去好一些,因为过去没有这样公开的决定,并且这样的决定有组织上的保证。这回组织有些改变。政治局委员,政治局候补委员,书记处书记,常委的调整,就保证了中央这个决定以及公报的实行。

对犯错误的同志,也要给他出路,要准许他改正错误。不要认为别人犯了错误就不许他改正错误。我们的政策是“惩前毖后,治病救人”,“一看二帮”,“团结——批评——团结”。我们这个党不是党外无党,我们是党外有党,党内也有派,从来都是如此,这是正常现象。我们过去批评国民党,国民党说党外无党,党内无派。有人就说:“党外无党,帝王思想,党内无派,千奇百怪。”我们共产党也是这样,你说党内无派?它就是有,比如说对群众运动就有两派,不过是占多占少的问题。如果不开这次全会,再搞几个月,我看事情就要坏得多。所以我看这次会是开得好的,是有效果的。


