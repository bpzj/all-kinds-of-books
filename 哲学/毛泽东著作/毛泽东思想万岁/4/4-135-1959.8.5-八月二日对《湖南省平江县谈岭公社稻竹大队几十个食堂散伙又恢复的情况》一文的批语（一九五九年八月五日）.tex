\section[八月二日对《湖南省平江县谈岭公社稻竹大队几十个食堂散伙又恢复的情况》一文的批语(一九五九年八月五日)]{八月二日对《湖南省平江县谈岭公社稻竹大队几十个食堂散伙又恢复的情况》一文的批语}
\datesubtitle{(一九五九年八月五日)}


印发各同志。此件很值得一看。一个大队的几十个食堂,一下子都散了,过了一会儿又都恢复了,教训是:不应该在困难面前低头。像人民公社和公共食堂这一类的新鲜事物是有深远的社会经济根源的,一风吹是不应该的,也是不可能的,某些食堂可以一风吹掉。但是总有一部分人,乃至大部分人,又要办起来。或者在几天之后,或者在几十天之后,或者在几个月之后,或者在更长时间之后,总之又要吹回来的。孙中山说:“事有顺乎天理,应乎人情,适合世界之潮流,合乎人群之需要,而为先知先觉者决志行之,则断无不成者也。”这句话是正确的。我们的大跃进,人民公社,属于这一点。困难是有的,错误也一定要犯的。但是可以克服和改正。悲观主义的思潮是腐蚀党、腐蚀人民的一种极坏的思潮,是与无产阶级和贫苦农民的意志相违反的,是与马克思列宁主义相违反的。


