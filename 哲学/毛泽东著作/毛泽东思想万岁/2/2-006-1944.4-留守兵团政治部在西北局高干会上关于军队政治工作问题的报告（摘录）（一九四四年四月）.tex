\section[留守兵团政治部在西北局高干会上关于军队政治工作问题的报告(摘录)(一九四四年四月)]{留守兵团政治部在西北局高干会上关于军队政治工作问题的报告(摘录)}
\datesubtitle{(一九四四年四月)}


一、关于边区军队一年经验的总结(略)

二、关于发扬政治工作中的成绩与纠正政治工作中的缺点

中国共产党从它参加与领导中国民族民主革命以来,从它参加与领导为这个民族民主革命而战的革命军队以来,就创设了并发展了军队中的革命的政治工作,这种政治工作的基本原则是以民族民主革命为纲领教育军队,是以人民革命的精神教育军队,使革命军队内部趋于一致,使革命军队与革命人民、革命政府趋于一致,使革命军队完全服从革命政党的政治领导,提高军队的战斗力,并进行瓦解敌军协助友军的工作,达到团结自己,战胜敌人,解放民族解放人民的目的。这就是我们的军队和其他军队的原则区别。我们说,共产党领导的革命的政治工作是革命军队的生命赖,就是指的这个意思,拿了这种革命的政治工作去和革命的军事工作相配合,就成了革命军队的全部工作。北伐战争时期,共产党人在国民革命中所指导的政治工作是这样做的,国内战争时期与抗日战争时期我党在过去的红军中与现在的八路军新四军中所指导的政治工作,更是这样做的。共产党人在这三个时期的革命军队中、所做的军事工作与政治工作,都是收到了成效的。在中国历史上第一次出现了彻底拥护人民利益的军队,如果我们的军队没有共产党领导,如果没有共产党领导的革命的军事工作与革命的政治工作,那是不能设想的,没有共产党的领导,就不可能有彻底拥护人民利益的军事工作与政治工作,而如果没有这种军事工作与政治工作的军队,就不可能是彻底拥护人民利益的军队。八路军新四军在抗日战争中之所以能够如此英勇坚持,艰苦奋斗,再接再厉,百折不回,其根本原因就在这里。这种革命的政治工作,也不但是我们军队的政治工作人员做的,许多的军事工作人员及其他人员也都参加了这种工作。同样,许多的政治工作人员,也都参加了军事工作。这在内战时期是如此的,在抗战时期也是如此的。

以内战时期来说,我们军队在其初期是创造的时期,其中期是发展的时期,其后期是受到某些挫折但同时有某些成绩的情形,特别是在内战初期与中期,在对敌斗争的政治工作上,在协调军党关系、军政关系,官兵关系,军民关系,上下级关系、军事工作与政治工作关系,各部分军队间友好关系的政治工作上,总之一句说,在团结自己,战胜敌人的工作上是有很多的创造,以至于到今天还成为我们军队的一部份优良传统。

以抗战时期来说,其初期是生动活泼的,是有伟大成绩的。在这时期内,整个八路军新四军在党的领导下,发展了抗日游击战争,扩大了抗日武装力量,团结了广大人民群众于自己的周围,恢复了广大的失地建立了许多抗日民主根据地,抗击了大量的民族敌人,以至于能够协助处在正面阵地的国民党军队停止了敌人的战略进攻,保障了中华民族不被日本帝国主义以击败,这些都是做得很好的,很正确的。在抗战的中期,尤其是最近几年,整个军队继续执行了党的路线与党的许多政策,以至于能够抗击了日寇侵华军队的百分之五十八,伪军的百分之九十,以至于能够击破敌伪军无数次的反复的残酷的“扫荡”,能够有效地对付其杀光抢光烧光的所谓“三光”政策,能够使被敌人再度夺去的许多中国土地,又被我军再度夺回来,能够忍受史无前例的没有政府接济,缺乏武器、确乏弹药,缺乏药品、缺乏粮食被服的无限困苦,而为中华民族,为同盟各国,坚持了一条中国大陆上最重要最有力的抗日战线,这些也都是做得很好的,很正确的。

总之,不论在内战时期与抗战时期,整个军队的成绩是伟大的,军队政治工作的成绩也是伟大的。军队中从事政治工作的人员,也和从事军事工作的人员与从事后勤工作的人员一样,他们足有功劳的,是对中国人民有了光荣的贡献的。

但是,我们的政治工作(这里只说政治工作),是存在过与存在着缺点的,我们应该采取自我批评态度,检讨这些缺点,战胜敌人,打倒日本帝国主义的目的。

例如,当我们提到政治工作中的传统的时候,若干同志往往存在着一种模糊观念,似乎认为我们政治中一切传统都是好的。因此,就不免对于某些不好的东西与某些不适合于我们军队当时情况的东西,也都奉为天经地义,当作所谓优良传统在发扬。而政治工作历史上积累起来的许多真正好的经验、真正好的传统,却又被忽视,被抹杀。

我们的政治工作,在内战后期曾过这样的情形。就其积极方面来说,这时期的政治工作曾经执行了反对人民敌人的政策,曾有过许多成绩,主要表现在对敌斗争方面,例如关于鼓励官兵的英勇斗争,关于争取俘虏等。在团结军队内部与团结人民方面也有其成绩,例如,关于争取人民参加军队,军队的更加有秩序。军队中许多好的政治教育工作与文化教育工作,居民工作中的许多好的工作等(这种成绩主要是表现在中央苏区,及其他几个区域军队中,在另外某些军队中,由于个别同志在政治领导与政治工作领导士的绝对错误,就不但看不出这种成绩,反而摧毁了原有的成绩了。但是,就其消极方面来说,却又产生了几种不正确的作风,这就是政治工作中的教条主义作风,(这种教条主义作风,在军事工作特别是在军事教育工作中也是存在的,这里不去论列)。这种教条主义,在我们工作中的最典型的表现,就是爱好空谈,脱离实际;重视形式,轻视内容;团结少数,脱离多数。某些同志之所以或多或少地犯这种毛病的原因,在于他们的主观主义教条主义的方法论。这种方法的表现,就是他们在许多时候,只凭主观臆想办事,忽视中国革命的实际和当时军队的实际,机械地搬运外国经验,不适当地强调当时军队的正规化,割断我们军队的斗争史。这些同志喜欢从口头上,从形式上强调所谓无产阶级的领导作用,但是在实际上,却放松了无产阶级思想(即马列主义)对于从农民与小资产阶级出身的人们在思想上的改造工作和教育工作,实际上不甚重视与不甚强调我们军队所应该认真倡导与认真实行的真正军党一致,军政一致,军民一致,官兵一致,上下级一致,军事工作与政治工作一致,各部分军队间的一致等等基本思想,不是严肃地认真地把这些思想列为政治工作的基本内容与基本工作方向,而是把这些基本内容、基本方向,当作不甚重要的东西,却从形式方面重视了与强调了许多不应该重视不应该强调的东西。

这种作风发展下去,其结果势必产生这样的情形;争取群众的任务有些被轻视了。军队的三大任务,三项纪律,八项注意,改变了提法,或被认为不必要了。军队内部团结一致的工作也有些放松了。党政军民关系也不去力求亲密了。军队不尊重党政,军队打骂民众,侵犯民众利益的现象,以及军队内部官长对士兵的打骂现象,虽有指责也不甚得力了。各部分军队之间的友好关系也减弱了,山头主义,本位主义,在潜滋暗长了。由于山头主义与本位主义的发展,军队的这一部分可以不照顾那一部分,主力部队可以不照顾地方部队与民兵,模范队伍可以不照顾落后队伍,从而在主力部队的若干同志中(这里军事政治后勤三种工作同志都有),生长了一股骄气,自以为是,目空一切,只爱说成绩,不爱说缺点,只爱听恭维话,不爱听批评话,惧怕别人批评,也惧怕作自我批评,有些则更发展到对党与上级的领导机关闹独立性的恶劣现象。所有这些情形,有些在抗战时期有了改变,有些在抗战时期先有发展,后有改变,有些则至今还没有改变。

在党内生活上,内战后期曾经普遍釆用了过火斗争的方针,缺乏惩前毖后治病救人的精神。军队中的党内生活,也发生过这种不正常的情形,以致许多不应该受打击的同志受到了错误的打击。抗战时期这种情形有了改变,但其残余没有肃清,还发生过若干错误地打击不应该受打击的同志的情形。

在军事工作与政治工作的关系上,虽然基本上是团结的,但是存在过某些不协调的情形。在内战后期,曾经不适当地强调政治委员制度,不适当地强调政治工作的地位与权力,并且在实际上是在不适当地强调所谓军事工作人员与政治工作人员的党性的差别;对于军事工作者与政治工作者之间,缺乏合作精神的提倡,是这些不协调情形的来源之一。但是这种不协调不能一概责备政治工作人员方面,有些时候军事工作人员也要负责任。

在抗战初期,曾经一时迁就国民党,取消了政治委员制度,降低了政治工作的地位,这是错误的。后来改正了,恢复了政治委员制度,提高了政治工作的地位,这是很对的。政治工作在任何一部分革命军队中,都应有其适当的地位,都应适当地强调他的作用,否则这个部队的工作就要受到损失。特别是在那些政治工作比较薄弱的部队,这样的强调十分必要。

对于政治工作地位的过分强调是不对的,但是没有必要的强调,没有必要的地位,也是不对的。

整个军队的方向就是政治工作的方向。因此,政治工作的任务只能根据我军的基本任务(为反帝反封建而斗争,为战争、生产与群众工作而斗争等)与当前具体任务(如反扫荡、反蚕食、生产运动,整动J运动,防止奸细,整顿三风,统一领导,精兵简政,拥军爱民,改善军党、军政、军民、官兵、上下级各部分军队之间的关系等,依当前需要而作具体布置)去规定,不能在我军基本任务与当前具体任务以外再有所谓政治工作的独立任务。政治工作就是以革命精神教育军队,从思想上、政治上与组织上去保证这些任务的完成。如果在这方面强调政治工作的独立性,特殊性,把政治工作任务与整个军队任务分离起来,那便是不对的,那便会是产生政治工作与军事工作目标不一致,使政治工作脱离实际,显得空虚的原因。任何组织形式,工作方式,工作方法,都是根据情况与任务而产生的,是从属于一定情况与任务的,是应该依据情况与任务的变更而变更的,政治工作的组织形式,工作方式、工作方法,也是如此。所以,如果离开军队的具体任务去谈政治工作的组织形式,工作方式,工作方法,把组织形式,工作方式,工作方法与整个军队的具体任务相分离,并把它看成绝对的东西,那是不能不变成教条主义、形式主义的东西的。在这个问题上,过去有些同志是认识不清楚的,如果现在还有这种情形,便应加以纠正。

应该指出,所有上述一切缺点,无论在内战时期与抗战时期,都不是一切部队同样具备的,其间轻重大小,互不相同,正如各部队的优良成绩,优良传统,也是各有特点,互不相同一样。我们的军队始终是革命的军队,其有缺点,好似一个壮健的人有时生了一点疾病一样,很快就会治好的。

还应指出,上述这些缺点,也不能认为都是由于教条主义产生的,中国革命的特殊环境(小资阶级的广大与长期被敌人分割的农村根据地)是其产生的客观原因。教条主义思想的存在与没有被肃清,我们的教育工作做得不够,则是其主观的原因。一九四二年整顿三风以及实施各种改革以来,这些缺点就逐渐减轻了,某些整风深入的部队,这些缺点就更少了。

还应指出,军队中的教条主义和内战后期党的领导上的教条主义,有密切的关系,如果没有当时党的领导上的教条主义,则军队是不会单独产生怎样严重的教条主义的。

还应指出,当我们检讨工作中的缺点错误时,不应着重在某些个别同志的责任,而应着重在总结历史经验,以为今后工作的借鉴。还应知道,我们的军队工作与军队政治工作总结是向前发展的,我们的军事工作同志,政治工作同志与后勤工作同志的思想认识也向前发展的,某些犯过错误的同志或是已经进步了,或是正在进步中,或是一经指出就可能使之进步,根据中央惩前毖后治病救人的方针,我们应该团结一切从事军事工作,政治工作与后勤工作的同志,不分彼此,共同工作。

但是应使我们的同志深刻认识,如果我们不从中国革命实际与中国革命军队的具体任务出发,如果我们犯了主观主义教条主义的错误,我们就会无力以高度的革命精神去教育军队,就会无力巩固我们的军党关系,军政关系、军民关系与官兵关系,就会无力克服军队上下级之间,军队政治工作与军事工作之间,各部分军队之间以及军队与地方之间的不正常关系,也就无力克服军阀主主,本位主义,山头主义的恶劣偏向。反之,如果我们真正懂得了马列主义的方法论,如果我们善于从中国革命实际与中国军队的具体任务出发,善于调查研究,善于联系群众,那末、政治工作(与军事工作,后勤工作)的全盘任务就能完成,团结自己,战胜敌人的目的就能达到。中国革命的经验,北伐战争、国内战争及抗日战争三个时期的经验,都在证明这一论断。

长期革命斗争的经验证明,我们军队的任务,不只是一个单纯地对敌斗争的任务,四军古田决议中已经明确地规定了这一点。由于我们军队的特殊环境,我们在内战时期,曾经规定以作战,筹款与做群众工作作为军队的三大任务。自然,作战是被放在第一位,在战争时期,其他一切都是服从于战争的。但是,如果放弃后两项工作,就不适合于我们的环境,在内战的初期与中期,我们军队曾经艰苦地但是光荣地执行了上述三大任务,这也就是说,在根据地没有建立之前,曾经协同地方党政建立根据地,而在根据地建立之后,又曾经协助地方党政巩固根据地,这样的任务,就是在抗日战争中也没有改变的。大家知道,如果不是八路军新四军在华北华中协助当地人民协助当地党政建立各个抗日民主根据地,试问这样强大的民族敌人(全部侵华敌军的百分之五十八与全部伪军的百分之九十都担负在八路军新四军的身上),这样长期艰苦的战争,怎能支持呢?不过在抗战时期,筹款的方式改变了,我们提出了军队进行生产运动的任务,借以改善军队生活与减轻根据地人民的负担。于是我们八路军新四军今天的任务,就变成战争,生产与群众工作这样三大任务了。如果把这样三个任务,缩小成为一个单纯的作战任务,那就不能适合我们的特殊环境。

为着团结自己,战胜敌人,为着实现军队的三大任务,过去军队的三项纪律,八项注意的精神必须恢复。军党之间、军政之间、官兵之间、军民之间、上下级之间,军事工作与政治工作之间,以及各部分军队之间的关系,必须竭力改善,并建立在巩固基础之上。必须使八路军新四军,一切部队,无条件地服从共产党中央及其代表机关的政治领导,才能使军队不走偏向,达到协合全国军民,打倒日本帝国主义的目的。

革命军队的内部,必须是团结的方针,必须是合作互助的方针,不能是分门别户,各自为政的方针,不能是互相轻视,互相妨碍的方针。凡属革命的军队,在党的统一领导之下,军事工作同志,政治工作同志与后勤工作同志,除应强调专精自己本业,并指出如果不注意本业则是不正确倾向这一点之外,还应提倡,军事工作与后勤工作同志应学习政治工作,参加政治工作;政治工作同志则应学习军事工作与后勤工作,并在需要他们的时候准备去参加军事工作与后勤工作,或竟改变现有工作职位去作军事工作与后勤工作。而在军事工作与政治工作的关系上,则应强调彼此的合作。不错,这些工作各有其不同内容,因此才分为几种工作。但是这些工作都是为着团结自己战胜敌人一个目的,这些工作的进行又必须取得和谐的合作。因此,凡是能巩固这种合作的,即应受到赞扬,而如果是妨碍这种合作的,应即受到批评与指责。

在团结自己,战胜敌人的总方针下,革命军队内部的争论问题,只应该是思想倾向与政策方针的原则性的争论,只应该是正确原则克服不正确原则的争论,并且只是因为这种争论,是有助于巩固党的领导,巩固官兵关系,军民关系及其它各种关系,借以提高军队的战斗力,借以达到团结自己战胜敌人的目的。任何的争论都应该完全是在党的统一领导之下去进行。而如果是多少离开这种立场的争论与对立,都是无原则的,而不应该允许其存在的。

在我们军队中,应该把尊重工农出身的干部与尊重知识分子出身的干部,尊重老干部与尊重新干部的方针,同时提出来,并使这些干部很好的结合起来。应该使人们懂得工农干部的长处,我们军队中有大批的工农干部,他们已在历史上形成为军队中一部分起主要领导作用的骨干,这正是我们军队的优点与特点(在党务工作,政府工作,民兵工作中也是这样)。他们中的绝大多数是忠诚可靠的,并特别能吃苦耐劳,英勇奋斗。轻视工农干部的思想是不正确的。但是同时,又应该使人们懂得革命知识分子的长处,我们军队如果没有大批革命知识分子参加,就不能达到团结自己战胜敌人的目的。他们中的绝大多教,也是忠诚可靠的,他们的缺点,经过锻炼(特别是经过实际工作的锻炼)是能够改造的,轻视知识分子的思想也是不正确的(在党务工作、政府工作、民众工作与文化教育工作中,也是这样)。无论工农干部与知识分子干部,都有他们自己的短处,都不应该骄傲,而应互相学习其长处,去掉自己的短处。

军队中的老干部是军队的主要骨干,没有他们就不可能有如此坚强的军队。因此,我们应该尊重老干部,轻视老干部的观点是错误的。但是同时应该尊重新干部,因为新干部是我们军队的新鲜血液,没有他们,就没有我们军队的发展与壮大,轻视新干部的观点也是错误的。新老干部应该互相尊重,互相帮助,而老干部尤应随时随地,特别表示欢迎与尊重新干部的态度,以便于从各方面吸收新干部加入我们抗日的军队,壮大我们的抗日力量。

从地方工作中吸收到军队来工作的干部,新的与老的,工农出身的与知识分子出身的都有,这是扩大军队干部的一大来源,我们应该表示尊重,欢迎与和气地同他们一道工作的态度。他们初来时,在工作上有些不习惯,不熟练,是很自然的,我们应该善于帮助他们。

我们军队中,在工农干部与知识分子干部的关系、老干部与新干部的关系、原有军队干部与新来地方干部的关系这些问题上,是存在着不正确的观点的,在这些问题上存在的宗派主义观点,是应该纠正的。

在军队的政治教育中,要把培养高度的对敌仇恨与争取敌军俘虏二者区别而又统一起来。没有前者,就不能振起一往无前,杀敌致胜的士气,没有后者,就不能瓦解敌军官兵。应该使我们军队的指挥员战斗员都懂得,在战斗时,集中一切力量去压倒敌人,迫使敌人投降,如果敌人不投降,那就应坚决地歼灭他们,或俘虏他们,这就是我们军队的一往无前,杀敌致胜的革命精神,这种革命精神,是我们军队非常宝贵的历史传统,今后应该在我们军队中大大提倡,大大发扬。但是在战斗解决以后,对待俘虏的政策,就不是这样,这里应该转变为说服态度,从思想教育上,物质待遇上、政治态度上争取他们,将我们在战斗前与战斗中对于敌军的宣传变为事实,借以瓦解敌人的队伍。拿中国的老话说,如果前一种态度可以叫做“霸道”(革命的霸道),那么后一种态度就可以叫做“王道”(革命的王道)。如果拿前一种态度应用于后一种情况,那是不对的。分别而又同时发扬这两种态度,正是我们的历史传统,今后应把这一点更大地发扬起来。

如果说对敌人是用“霸道”,那么,对同志、士兵,对人民,对朋友,就是用“王道”,对前者是打击,是消灭,对后者是尊重,是说服。如果不去学会分别这两者,如果把对待敌人的态度有时稍微误用了去对待同志、士兵、人民与朋友,那就是犯了极大错误,严格地分别这两种态度正是我们的历史传统,今后同样应当予以更大的发扬。

在我们军队的军事教育与政治教育的比重上,应依照部队的具体情况去规定。依照具体的需要,在一个时期内,对于这一方面或那一方面有所侧重。例如,在部队政治教育特别缺乏时,应当多费时间,多费精力去着重进行政治教育,是完全必要的,但是一般地说来,在分配二者的时间上,军事教育应占的比较多些,政治教育及文化教育则应占的比较少些,不可拿许多不必要的繁琐的会议、汇报占去军事训练的时间,许多政治教育工作与政工会议,则多利用各种间隙与机会去做。这种时间上的分配,决不能被误解为是降低了政治工作的重要性,恰好相反,我们认为政治工作是我们军队的生命线,无此则不是真正的革命军队。必须使人们懂得,认真的政治工作具有非常重要的意义,任何轻视政治工作的观点,都是不对的,但是,现在有些部队中将政治教育、文化教育及其他政治工作活动占去了大多的时间,减弱了军事训练的时间,这是应当纠正的。

在军队的政治教育与文化教育的比重上,政治教育应占第一位,文化教育则占第二位,借以提高部队的政治情绪、政治认识与政治质量。但革命军队的文化水平,是应该提高的,不能误认为只要政治,不要文化,我们可以而且应该利用各种间隙的机会去进行识字运动及其它文化活动。在陕甘宁边区留守部队的情况,文化教育有更多的可能性,更应多加些力量去做。

我们的军队是人民的军队,是彻底拥护人民利益的,是以工农为其基本成分的。但是我们绝不拒绝从地主富农资本家阶级出身,而真心愿意献身于抗日战争,真心愿意服从共产党领导的那些人们,参加我们的军队。这些人们的缺点,甚至严重的缺点,那是会有的,他们这些缺点是不对的,但是这些缺点是可以纠正的。我们不应排斥或歧视他们,而应欢迎他们信任与尊重他们,和他们一起奋斗,在奋斗过程中逐渐地帮助他们克服其缺点。

对于军队中的二流子(流氓),应和对于地方上的二流子一样,釆取改造与感化的政策。听任二流子习惯存在,是不对的,单纯地排斥与洗刷二流子,也是不对的,绝大多数的二流子是可以经过教育变为好人的,只要我们的政策与方法不犯错误。

在改造落后分子的工作中,主要地应采取耐心感化的方法,禁用单纯惩办的方法。对于逃兵也不是单纯惩办方法所能解决问题的,必须釆用教育感化方法,定要绝对禁止枪毙逃兵。在这里,首长们从爱护观点出发的亲切谈话是最有效的,一定要把这种首长谈话与群众中自我检查自我教育配合起来,才能发生更大的效力。单纯的惩办,甚至打骂的方法,只是一种脱离群众而毫无效果的方法。军队中一定要废止打骂与肉刑,坚决执行一九二九年十二月古田会议的决议,一定要把从爱护观点出发从治病救人出发的勤勤恳恳好心好意的教育感化工作,去代替谩骂与肉刑(打人),去代替单纯的处罚方法,并且一定要把谩骂与肉刑以外的必要的,即按照实情必不可少的正当处罚方法,也减少到最低限度。非在十分必要的情形之下,决不轻易采用处罚方法。

对于会门与土匪,也应采取逐渐地有步骤地感化与改造他们的政策,不能是单纯地排斥与打击的政策。除非我们尽了一切力量去争取,而某些会门或土匪还是坚决站在敌人方面反对我们,才可以采取某种打击手段,但在打击之后,又应继之以争取。

抗战时期,各种对敌伪斗争的政治工作,争取敌伪军俘虏的政治工作,武装工作队的政治工作,争取广大民众加入军队的政治工作,组织民兵与游击队的政治工作,团结广大民众加入军队的政治工作,组织民兵与游击队的政治工作,团结广大知识分子到军队中来的政治工作,团结地主资本家阶级和我们一道抗战的政治工作,团结少数民族的政治工作,以及其他作得有成绩的工作,凡属有积极意义的,都应该继续坚持与加以发扬。

如果我们部队中对于上述各项问题发生了思想上的偏向,那就应当加以纠正。

此外,在我们军队的政治工作中(同时也在军事工作中),还有几个毛病,在前而虽已提到,但是没有着重说清楚的,需要在这里说清楚。这些毛病并不是每个部队都是一样地存在与一样地严重的,陕甘宁边区部队在去年一年中又已经给了许多的改造。但是这些:毛病是这样显著与这样引人注目,使我们不得不提出来加以分析,以期唤起同志们注意,加以彻底的克服。

哪些毛病呢?第一个毛病就是形式主义。常有这样的情形,当有些同志做工作计划时,他们的计划常是随时可用,随地可用的照例文章,他们只知道习惯了的一套老公式,不知道依情况的变化而变化,这是一种形式主义。

在我们的政治教育问题上(军事教育上边有某些相同),我们教的东西与战士们需要的东西,有些是不相符合的。例如,有些同志在战士面前专门喜欢谈论国际大势与国内大势,却不愿意谈论本地本军与本连本队的大势。把这种自己所最清楚的同时又是战士群众最需要的最欢迎的具体事物,抛弃不谈,却喜欢向战士们输送大批的抽象概念。又如,有些同志常喜欢凭空教导战士们要守纪律,要做模范,要英勇坚决,却不愿意把现实的、活的、动人的模范人物,英雄事迹,讲给战士们听。似乎这样的东西,没有讲的价值,只有大批概念,才是无价之宝。在这种情形下,教育的内容既是教条主义的,教授法、学习法,测验法,也是教条主义的。教的人不管被教的人懂不懂,只管教下去,为的是要“完成计划”。学的人也就只得把听课当做一种负担,当做一门差事,虚应故事,从来也不知道要联系自己与反省自己。在测验的时候,只知算问答的分数,不知看实际的行动。这种情形,在许多部队中存在着。在这些部队中,不但对战士的教育如此,对党员的教育也是如此,不但对政治教育如此,对文化教育也是如此,不重内容,但重形式的形式主义作风,成了我们宣教工作的一个严重毛病。这种情形,就陕甘宁边区说来,一九四三年有了很大的改进,但在干部的思想上还是没有彻底解决的。

我们的有些支部工作,也犯了形式主义的毛病。有些同志,当他们去检查支部工作,讨论支部工作的时候,当一个支部书记在布置自己工作的时候,开宗明义第一条就是会议与汇报,名曰“建立会议制度”、“建立汇报制度”。会议与汇报执行得好的,就是“模范”,不好的,就不是模范,几乎把会议与汇报当做中心,当做一切。因此,支部忙于开会,党员忙于到会,结果是大家讨厌开会。为什么如此?因为这种会议是毫无内容的,枯燥无味的。战士们讨厌这种汇报,因为这种汇报多是消极地向战士找岔子,反映所谓“怪话”,因此,引起群众不满。这种所谓会谈与汇报,只是使支委脱离党员,党员脱离群众。像这样的支部工作,是没有法子做好的,这也是一种形式主义。

如果不是按照每一部队的特殊机能(步兵、骑兵、炮兵、工兵、卫生队,运输队等等)及每一任务的特点(作战、训线、生产、做群众工作等等),去进行不同内容与不同形式的政治工作,而是对于什么都做成一样的政治工作,那就是一种很厉害的形式主义。这种形式主义,在若干部队中是存在的,是这些部队的政治工作所以缺乏生气的原因。

有些部队中写报告填表格的制度,也有类似的情形。例如,上级机关规定许多报告项目与表格形式,生硬地要求下级(特别是营连)去做去填。在此种要求下,下级只得拼命地做上去,填上去。但是这些东西的内容,大都千篇一律,很少实用,徒费精力。上级得了这种东西又往往堆在一起,不加分析整理,结果化为废纸,无所用之。这也是一种形式主义。

我们军队中的许多组织形式是不合实际需要的。例如,有些政治工作机关分了很多的部与科,收罗很多的人堆在机关里,结果人员越多,工作越少,忙的多是一些琐碎的形式上的工作,真正的领导工作反而做得少了。又如连队中的工作网,十人团、青年队、政治战士,也变成了形式上的好看实际上无用的东西。

总之,凡属不合实际需要,徒具形式的工作形式与组织形式,都属于这一类,这是第一个毛病。

第二个毛病,就是平均主义。这里所说的,是指我们有些同志做工作计划吋,不知道抓住中心;检查工作时,不知道找典型;解决问题时,不知道从某一个关节着手。以致他们的许多计划,常常是应有尽有,无所不包。早已成了日常习惯的业务,也要列入所谓计划之内,把一些不关痛痒的东西和主要的东西同时并列在一起,这样搅在一团毫无眉目的所谓“工作计划”,完全失去了工作计划的意义。有些部队的同志,不知道在一个相当时期内只应该有一项为当时情况所要求的中心工作,而以其他若干必不可少的工作附属之;而是同时发动做很多项工作,想在这很多项工作中获得成绩。但其结果则是一项也做得不彻底,一项也得不到满意的成绩。我们有些同志在检查工作时,可以把所有的连队都跑一遍,把政治工作的各个项目都考查一遍,但是结果呢?仍然没有发现问题,没有分析问题,变成了一句空话。这些都是很严重的平均主义。平均主义的工作方法,其原意是想样样做好,但其结果一样也做不好,这种平均主义方法,同形式主义一样,其性质是主观主义的,因为他是从主观臆想出发,不是从工作的实际需要出发。

第三个毛病,就是重号召不重组织,以一般号召代替具体指导,换句话讲,就是空喊。我们看到有些同志提出的号召是不适当的,他们没有估计那里的工作性质,没有顾及那里的群众情绪,他们所号召的东西,并不是那个部队实际上所需要的东西。即使号召是正确的,我们也看到我们有些同志对于自己发出的号召不负责任的。第一个号召发出之后,第二个号召接着又来,在那里,许多工作是在号召上打圈子,永远停留在号召的阶段,许多工作常常是有头无尾的。全是对工作的开玩笑态度,都只会损失威信,得不到任何的结果。与此相反,我们应该根据“首长负责,自己动手,领导骨干与广大群众相结合,一般号召与具体指导相结合”的原则,一经提出号召,就应组织实行,并在实行中,突破一点,取得经验,指导其余。如果没有具体指导,则所谓一般号召,就是自己给自己取消了。

第四个毛病,就是有些同志的孤立主义。不是人家孤立我们,而是我们自己孤立自己。这些同志,首先不是把政治工作看作是群众工作,不从群众观点出发,不采取群众路线,不组织群众行动,宁愿将政治工作锁在狭隘的圈子里,只有少数政治工作人员在做,在忙,广大群众却在那里闲着。许多事情可以而且应该协同军事工作同志一起去做的,但是我们不愿意去找人家,宁愿使工作老是抓不开。至于动员了群众的,也有这种情形,在那里,不知道照顾群众的需要,而是违背这种需要,不知道考察群众的情绪,而是忽视群众的情绪。从而关于领导群众的艺术问题,关于如何团结积极分子,巩固中间分子,提高落后分子的问题,关于党与非党群众的联系问题,关于宣传鼓动工作与组织工作互相配合的问题等等,都被认为是不重要的,或者完全被忽视。在这种情形下,我们的政治工作,就不能不一般化与简单化。在这种情形下,我们的政治工作,就变成了少数人脱离群众而孤立起来的所谓“工作”。此外,有些同志在解决问题时,不凭自己的本领,不凭自己勤于向同级,向下级、向群众去商量,去学习,借以取得知识,取得经验,而只凭自己的工作职位,只凭行政手段去解决问题,以至于不得不把事情弄坏,有时甚至毫无办法,只好拿大帽子压人。这种脱离群众的工作方法,就是孤立主义的方法。

总起来说,形式主义的作风,平均主义的作风、空喊的作风与孤立主义的作风,实质上都是小资产阶级的作风,都是主观主义、教条主义的作风。这种作风不是共产党人的作风,不是革命军队的作风。它同党的作风,革命军队的作风,是不能并存的。政治工作中存在着这样的作风,就会减弱它的革命性,即使党的路线是正确的,政治工作方向是正确的,如果这些作风不改变,仍然无力完成政治工作的任务。我们的政治工作曾经有过很好的传统,古田会议的决议,就是这种传统的一个具体表现。过去工作中,凡是继承了并发扬了这种传统的,那里的工作就充分地表现其革命性与创造性,反之,就表现死板僵化,死气沉沉,毫无生气,毫无力量。现在在陕甘宁边区部队中,在整个八路军新四军中,都有很多生动活泼的创造性。在长期抗战中,产生了很多的战斗英雄、劳动英雄与模范工作者。他们没有形式主义,也没有平均主义,也没空喊,也没有孤立主义,他们是实事求是与联系群众的。我们应该学习他们,应该表扬他们,应该展开每一个部队中学习战斗英雄、劳动英雄、模范工作者的运动。干部中群众中的积极性创造性是无限的,只要我们善于把任务提出来,善于启发他们,善于鼓励他们,他们就会蓬蓬勃勃地活跃起来,发扬自己的成绩,纠正自己的缺点。即使过去工作中曾经犯过错误的同志,也会在这个群众运动中改正过来。经验已经给我们证明,干部中群众中蕴藏的正气、热忱、创造性、积极性,一经被启发,就会是取之不尽用之不竭均源泉,就会变成长江大河,一泻千里,那时来看我们过去的缺点,就会显得不过是像太阳中的一些黑点了。我们的党,我们的军队,我们的人民本来是太阳,这个太阳是要照耀全世界的。

三、关于组织形式与工作制度中的一些规定(略)

(注:这是一九四四年经毛泽东同志亲自主持写成的留守兵团政治部在西北局高干会议工所作的政治工作报告。(见《中共中央军委扩大会议关于加强军队政治思想工作的决议》一九六○年十二月于北京)

<p align="right">(据南京革命教育学院革造指挥等单位辑印本)</p>

