\section[查田运动的第一步一一组织上的大规模动员(一九三三年六月)]{查田运动的第一步一一组织上的大规模动员}
\datesubtitle{(一九三三年六月)}


怎样进行查田运动呢?首先是组织上的动员。苏区内一切领导群众斗争的经验告诉我们,只有共产党苏维埃与革命群众团体三者,在党的领导之下,协同一致的行动起来,才能达到每个斗争任务的完满完成。查田运动是一个残酷激烈的阶级斗争,是一个群众的伟大革命运动,是党和苏维埃群众团体工作改善的根本,是目前工作中最主要的一环,只有整个党控个苏维埃工会都动员他们的全力,加入这个运动中去,才能发动开展与完成这个运动。关于党的动员中央局已有正确的指示,关于苏维埃的动员首先要反对过去把查田运动看做只是土地部的工作,不但财政部、军事部、国民经济部与教育部认为是与查田运动毫无关系的,就是工农检察部、裁判部,政治保卫局,也认为没有多大关系,甚至主席团也不去管理查田运动,这是完全不对的,须知整个苏维埃,没有那一部分可以脱离查田运动不管的。第一,是各级政府主席团要用最大的注意去领导整个的查田起动。第二,各级土地部,工农检察部,裁判部,国家政治保卫局及其特派员,是各级政府在查田运动中主要领导与工作的部门,要在查田运动中彻底解决土地问题,改造乡区县三级苏维埃,肃清农村中的反革命,这些政府部门,必须拿出他们最大的力量,财政部要注意从地主罚款富农捐款去进攻封建残余势力,同时增加国家的收入,军事部要注意在查田运动中整顿与扩大地方武装动员群众积极分子参加红军,国民经济部要注意从查田运动的发展中,去进行农业手工业生产的恢复与发展,合作社的发展,与生产品消费品的调剂,教育部不是没有工作的,他应为着开展查田运动供给一些简明通俗的课本与小册子,给予一切查田的干部与群众,他应跟着查田运动的发展,去发展群众的文化教育,因此现在苏区,工会会员,最大多数是在农村中,并与土地有密切的关系,因为查田运动这个伟大激烈的阶级斗争,无产阶级必须是最坚决的领导者,所以不论农业工业,手艺工会及其他工会,必须在全总执行局的领导之下,动员他们的最好干部与一切农村中的会员,加入查田运动,最主要的是在贫农团中起推动鼓励的作用,与对于查田查阶级改造政府提出积极坚决的主张。总之,查田连动不是一件寻常小工作,不是一日甚至不是半年所能彻底完成的,所以必须有党团政府工会各方面配合的大规模的动员。

这里论到教育干部的问题:第一,须召集下级负责人开会,给他们以关于查田运动中几个主要问题的充分的说明,过去各级苏维埃人员,多不了解查田运动,是一个紧急任务,不懂得实际去分阶级成份,不懂得争取群众发动斗争的路线与方法,所以没有法子去开展这个运动。第二,这样的教育还应施之于从当地的下级调来从一切先进的区域调来的干部,要为他们办短期查田运动训练班,省县两级政府要为了查田运动多次的开办这种训练班,一星期至两星期毕业:专门讲授查田运动几个主要的问题,省县区三级政府土地部工农检察部政治保卫局,每部应有他们一组工作员(区一级的不脱离生产)各部对于自己的这一组人,应给予充分的,关于查田工作的教育。第三,一种教育是应该施之于行动中的,这就是省县两级的派人出去巡视,与区一级的每五天至七天召集自己派出的工作员及乡苏主席贫农团主任开会,检阅他们的工作经过,有前两种方法,而没有后一种方法,是不能争取工作的最大成效的。

<p align="right">(原载一九三三年六月《红色中华》)</p>
