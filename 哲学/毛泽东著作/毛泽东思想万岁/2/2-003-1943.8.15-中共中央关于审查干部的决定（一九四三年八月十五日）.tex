\section[中共中央关于审查干部的决定(一九四三年八月十五日)]{中共中央关于审查干部的决定}
\datesubtitle{(一九四三年八月十五日)}


特务之多,原不足怪。在德意日法西斯国家及其附属国与占领地,欺骗与强迫千百万青年,加入法西斯组织,并与其服务。中国买办封建的法西斯化的国民党,虽有抵抗外国法西斯侵略的作用,但从二九二七年以来就是反共反人民的,设立了庞大的特务系统。抗战期间,虽则一面利用共产党抗日,但是一面又极力反共,欺骗与强迫广大青年加入其组织,并将其中一部分变为职业特务,从事子反共破坏活动。日本法西斯则利用中国人作特务,其数量亦是很多的。故特务是一个世界性群众性的问题,不认识此点,就不能釆取正确方针。

这一次我党在整风中审查干部、并准备进一步审查一切人员,不称为肃反,不采取将一切特务分子及可疑分子均交保卫机关处理的方针,而采取首长负责,自己动手,领导骨干与广大群众相结合,一般号召与个别指导相结合,调查研究,分清是非轻重,争取失足者,培养干部,教育群众的方针,就是因为这是一个群众性的问题。离开了机关、学校、部队、工厂、农村的广大群众及其各级联系群众的强有力的干部,就无法最妥善地最彻底地解决这个重要问题。

上述首长负责的整个方针,是和内战时期,曾经在许多地方犯过错误的肃反方针,根本对立的。这个错误的方针,简单地说来,就是逼供信三字,审讯人对特务分子及可疑分子采用肉刑,变相肉刑及其他威逼的办法,然后被审人随意乱供,诬陷好人,然后审讯人员及负责人不加思索地相信这种绝对不可靠的供词,乱捉、乱打,乱杀。这是完全主观主义的方针与办法。抗战时期,山东湖西地方的错误肃反事件,也是重复这样方针与方法的结果。这种错误思想的余毒,在许多干部中,特别是保卫工作干部中,至今还是严重的保存着,只有采取上述首长负责的整个方针,才有充分可能肃清这种主观主义的错误思想,而使这次审查干部,乃至审查一切人员,走到最妥善最彻底之目的。

首长负责,就是将大多数有问题的人,留在各机关学校部队,信任与责成当地各级行政首长负责审查(但若干被坏人掌握的部门不在此例,对于这些部门,必须首先改造领导)。延安是分普通、反省、保卫三种机关进行这个工作的。对于有问题的人,责成普通机关,即各党政军民学校机关自己处理的约占百分之八十;送反省机关,例如西北公学、行政学院办理的约占百分之十;送保卫机关,例如社会部、保安处、军法处处理(逮捕审讯)的约占百分之十。三省在整个审查过程中又互相交流,某些人由普通机关送入反省及保卫机关,某些人在反省与审查清楚并表示改悔之后,又由反省及保卫机关送回普通机关。凡是普通机关乃反省机关的特务分子及嫌疑分子,大多数均照常在原来的工作岗位及学习岗位上,仅为保障机密,防止破坏及自杀,才将某些人移动工作,或加以监视。各机关学校均严密地组织自己的自卫军,在一定时期内实行戒严,除可靠人员携带通行证及互相出入外,一切有问题的人都暂时禁止个人外出自由,只许随大众集体行动。

自己动手,就是从当地最高负责同志至各伙食单位的首长,均须亲身参加审查干部的大会、小会、劝说、询问及研究,以便收集经验,指导运动。空口指挥,坐着不动的官僚主义态度是错误的。

领导骨干与广大群众相结合,就是从每一伙食单位的全体人员中,以最可靠的一个行政首长为中心,围绕着他组织几个小核心,再围绕着小核心组织十几人,乃至几十人的中核心、大核心,围绕着这些核心的乃是广大群众。大多数有问题的人,均须经过核心组的谈话与群众的质问、劝说、斗争及展开热烈的坦白运动。各级核心,即是各级学习委员会及支部委员会与小组长,这些人须是完全可靠的,而且是从整风及审查干部过程中,逐渐的形成起来的。

一般号召与个别指导相结合,就是各级领导人员,除在当时当地一般地提出审查任务并推动大家去做之外,还必须选择个别单位,集中力量,给予突破,才能取得经验,造出范例,推动其余;然后再对所有各单位一个一个的给以具体的总结与指示。经常不断耐烦耐劳,审查工作才能彻底收效,不犯错误,和少犯错误。

调查研究,就是调查与研究每一个人的历史,找出其矛盾,发现其问题。每一单位,须由领导核心,根据所有人员平日的言行,经过慎重考虑,拟定两种名单,一种是估计无问题的,一种是估计有问题的,经过上级在慎重考虑后的批准,然后对于有问题的,一个一个的,予以实事求是的调查研究,禁止主观主义的逼供讯方法。

分清是非轻重,就是用调查研究方法,第一分清其是不是两条心的特务,或叛徒,或隐瞒自己参加过其他党派的人们,决不可把半条心的人(共产党员,但有非无产阶级思想及犯错误者)与二条心的人混为一谈;第二分清其犯罪之轻重,或者是情节很轻的普通分子(占多数)或者是情节较重的中等分子,或者是情节很重的头等分子(后二者占少数)。不可以为凡被提出的,一定都是特务,或都是重要特务。每一个被提出的人,虽被提出或被逮捕,但他究竟是不是特务及是轻是重,全靠我们用调查研究方法,收集材料,加以分析,才能清楚,这就是分清是非,分清轻重的任务。如果是被冤枉了的或被弄错了的,必须予以平反,逮捕的宣布无罪释放,未逮捕的宣布最后结论,恢复其名誉。在审查运动中,一定会有过左的行动发生,一定会犯逼供信错误(个人的逼供信与群众的逼供信),一定会有以非为是,以轻为重的情形发生,领导者必须精密注意,适时纠正。对于过左偏向,纠正太早与纠正太迟都不好,太早则无的放矢,妨碍运动的开展,太迟则造成错误,损伤元气;故以精密注意,适时纠正为原则。

争取失足者,就是对于一切大小特务,叛徒、或被日本被国民党一时利用的普通分子(占多数),原则上一律采取争取政策,即宽大政策。延安审查出二千多大(其中有一部分人被弄错了或被冤枉了,准备在最后清查时给了平反),至今未杀一人,其罪大恶极,反复无常,绝对坚决,不愿改悔者,自应处以极刑,但这种人是极少数的。前方及边境地区环境特殊,或对某些个别分子有早日镇压的必要,但总方针应是毫不动摇地,干方百计地,耐心地,热情地,争取他们。在大会小会及个别谈话中,向他们说明,世界法西斯末日已到,国民党腐败黑暗,决无前途,共产党则光明正大,前途无限,从思想上瓦解他们。延安经验证明,绝大多数这类分子是能够被我们争取转变为一条心的,许多人转变得很好。整风的任务就是将半条心的人们转变为一条心。审查干部的任务就是将两条心的人们变为一条心。日本及国民党很久以来,就釆取争取与软化共产党员为其服务的反革命方针,很少杀人。我党必须采取争取大部至全部特务分子为我们服务的方针,否则我们就是失败的。不要有怕特务跑掉的恐惧心理。当然不是故意放纵,让其跑掉,但是不可因怕跑掉而主张多杀。在某种情形下,宁可让他们跑掉,亦不可多杀人,跑掉是比杀掉为有利的。只有少捉不杀,或少捉少杀,才可保证最后不犯错误,留得人在,虽有冤枉,可以平反(确实冤枉的必须平反,绝无犹疑余地)。多捉多杀则一定会犯不可挽救的错误。

培养干部,就是应该从一切参加审查及被审查的人员中,培养出百分之十至百分之二十的人,学会调查、研究、侦察、讯问、审查等一全套的理论与技术。例如延安三万党政军,一万老百姓,应该培养出四千至八干人,善于这一套。从一九四三年四日起至八月止,延安头一期参加审查工作的一万干部数千杂务人员中,已经培养出二千多个这样的人。只有这样,才能打破保卫工作神秘化的观点,才能使特务不易再侵入,才能与将来准备大批的锄奸干部,肃清、争取并改造许多的特务破坏分子。在培养干部的任务中,应该包括党员与特务两部分人在内,就是说,不仅要注意培养共产党员(这当然是主要的),而且要着重注意,将反革命分子转变为革命的锄奸干部,愈是大特务转变过来就愈有用处。这个政策日本及国民党也是很早就釆取了,用以对付共产党,并且收到了成效。延安几个月来已经争取一大批特务分子很好地转变过来为我党服务,便利了我们的清查工作。

教育群众,就是此次在审查干部以及进一步审查一切人员中,一再要发动广大群众与核心骨干一起进行。这样不但培养了干部,而且使群众有了充分的经验,积极性发扬了,眼睛打开了,觉悟性提高了,党才真正巩固了。如果没有群众的发动、参加、受锻炼与提高觉悟,党的真正巩固是不可能的。脱离群众,少数人冷冷清清地审查干部,也一定是达不到目的的。我党在过去审查干部问题上所犯极端恶劣的形式主义的错误,其根本原因。就是在组织问题上的右倾观点与审查干部时的脱离群众。为了培养干部与教育群众,一般的上级不可代替下级,此地不可代替彼地。例如中央局或区党委,如果代替下级审查一切有问题的人,一则全部代替是不可能的,二则即使可能(例如代替下级审查大部分干部),也会使地委、县委、区委及乡村支部袖手旁观,毫无审查干部、审查党员、审查其他坏人肃清特务分子的经验,并使有问题的人脱离当地群众,脱离工作岗位,而不易审查清楚。从沦陷区调人来根据地审查是必要的,在根据地内抽调一部分人集中审查(例如进党校、抗大、开整风班)也是必要的;但一般的代替下级,代替他处,则是错误的。为了防止下级发生偏向,上级事前应作充分的思想准备与组织准备,并派人下去帮助,调人上来研究,密切注意,毫不放松,这样就可以解决偏向的问题。

以上指出了主要的方针与经验,希望各地同志研究采用,并依据你们的具体环境,创造你们自己的经验。

根据各地材料,各地整风须延长至一九四四年,审查干部可在整风中掺杂着进行。凡发现了特务活动并且有了思想准备与组织准备的地方,就可动手审查他们。先从一部分重要机关开始,取得经验,并巩固这些机关,然后逐渐推广于其他部门及其他地方,决不可普遍地同时进行。在那些还没有思想准备与组织准备的地方,在领导机关掌握在坏人手里的部门地方,便决不可轻易发动审查干部,这类地方,仍然应该着重整风或改造领导,准备审查干部的必要条件。

