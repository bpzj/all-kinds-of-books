\section[在延安各界庆祝斯大林六十寿辰大会上的讲话(一九三九年十二月二十一日)]{在延安各界庆祝斯大林六十寿辰大会上的讲话}
\datesubtitle{(一九三九年十二月二十一日)}


今天开大会,庆祝斯大林同志的六十诞辰。“人生七十古来稀”,世间六十岁也是难得的。但是我们为什么单单庆祝斯大林呢?而且这样的庆祝不仅在延安,而且在全国,而且在全世界只要晓得他今天生日的人,只要懂得他为人的,只要是受压迫的,都会庆祝他,原因就在于斯大林同志是一切被压迫者的救星。哪些人是反对庆祝他,而不希望庆祝他的呢?那就是自己不受压迫单单压迫他人的人,首先是帝国主义。

同志们,一个外国人,相隔万余里,大家庆祝他的生日,还是破天荒的第一回呢?

这就是他领导着伟大的苏联,因为他领导着伟大的共产国际,因为他领导着全人类的解放运动,帮助中国打日本。

现在世界正分为两条斗争阵线。一方面是帝国主义,这是压迫人民的阵线。一方面是社会主义,这是反抗压迫的阵线。有的觉得好象是站在中间,但是它的对头是帝国主义,它就不能不引社会主义为朋友,不能不属于反抗压迫者的革命阵营的一面。中国的顽固分了又想做婊子,又想立牌坊。一只手反共,一只手抗日,自称是中间派,但他们终究不成功的,如果不悔过,最后必要走向反革命方面去。革命与反革命阵线,都要有个做主的,都要有个指挥官。反革命阵线的指挥官是谁呢?就是帝国主义,就是张伯伦。革命阵线的指挥官是谁呢?就是社会主义,就是斯大林。斯大林同志是世界革命的领导者。这是一件非常重要的事情,全人类中间出了这位斯大林.这是一件大事,有了他,事情就好办了。你们知道马克思是死了,列宁也死了。如果没有一个斯大林,哪一个来发号施令呢?这真是幸事,现在世界上有了一个苏联,有了一个共产党,又有了一个斯大林,这世界上的事就好办了。革命指挥官干些什么事?使人人有饭吃,有衣穿,有房住,有书读,而要这样,就要领导十几万万人向压迫者作斗争,而使之得到最后的胜利,这就是斯大林要办的事。既然这样,那么,一切被压迫的人们,要不要庆祝他呢?我想是要的,是应该的,我们要庆祝他。拥护他,还要学习他。

我们要学习他的两方面。一个是道理方面,一个是事业方面。

马克思主义的道理千条万绪,归根结底就是一句话:“造反有理。”几千年来总是说压迫有理,剥削有理,造反无理。自从马克思主义出来,就把这个旧案翻过来了,这是个大功劳,这个道理是无产阶级从斗争中得来的,而马克思作了结论。根据这个道理,于是就反抗,就斗争,就干社会主义。斯大林有什么功劳呢?他发挥了这个道理,他发挥了马克思列宁主义,为全世界被压迫的人民,弄出一篇很清楚很具体很生动的道理来,这就是建立革命阵线,推翻帝国主义、推翻资本主义,建立社会主义社会的整个理论。

事业方面是把道理见之于实际。关于建设社会主义的事业,马克思、思格斯、列宁都没有完成。斯大林把他完成了,这是开天辟地的大事。在苏联两个五年计划之前,各个资产阶级的报纸,天天说苏联不得了,社会主义是靠不住,但是今天怎样呢?把张伯伦的口封住了。他把中国那些顽固派的口也封住了。他们也都承认苏联是胜利了。

斯大林除在道理方面帮助了我们的抗日战争外,他还给了我们事业上即物质上的帮助。由于斯大林事业的胜利,他帮助了我们很多的飞机,大炮,航空员,各战区的军事顾问,还有借款,世界上还有哪一个国家这样地帮助我们?世界上还有哪一个阶级哪一个党哪一个人所领导的国家,这样地帮助我们呢?除了苏联,除了无产阶级,除了共产党,除了斯大林,还有谁呢?

现在有些人,他们自称是我们的朋友,但是只能是属于唐朝李林甫一类的人物。这位李林甫先生,是个“口蜜腹剑”的人,帝国主义都是口蜜腹剑,张伯伦就是现在的李林甫,各国在中国的一切特权,什么驻军权、领导裁判权、治外法权等等,哪一个帝国主义国家废除了呢?只有一个,只有苏联是废除了。

在过去,马克思列宁主义,在理论上指导世界革命,现在加上一点东西,可以在物质上帮助世界革命了。这就是斯大林的大功劳。

我们庆祝斯大林的生日之后,还应该把这件事向全国宣传,向四万万五千万人讲清楚,使中国人民都懂得,只有社会主义的苏联,只有斯大林,才是我们中国的好朋友。

<p align="right">毛泽东

(原载《人民日报》一九四九年十二月二十日)</p>


