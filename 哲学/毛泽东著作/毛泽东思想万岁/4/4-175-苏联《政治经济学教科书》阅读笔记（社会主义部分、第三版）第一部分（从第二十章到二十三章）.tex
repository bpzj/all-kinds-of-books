\section[苏联《政治经济学教科书》阅读笔记(社会主义部分、第三版)第一部分(从第二十章到二十三章)]{苏联《政治经济学教科书》阅读笔记(社会主义部分、第三版)第一部分(从第二十章到二十三章)}


一、关于从资本主义到社会主义

教科书327—328页上说:社会主义“不可避免地”要代替资本主义,而且要用“革命手段”。帝国主义时代生产力和生产关系之间的冲突“达到了空前尖锐的程度”。无产阶级的社会主义革命是一种“客观必然性”。这些说法都很好,是应该这样说的。这个“客观必然性”很好,很令人喜欢。说是客观必然性,就是说它不依人们的意志为转移,不管你赞成不赞成,它总是要来的。

无产阶级要“把一切劳动者团结在自己周围来消灭资本主义”(327页)。这个说法对,但是在这里还应该说到夺取政权。“无产阶级革命遇不到现成的社会主义经济形式”,“社会主义经济成分不能在以私有制为基础的资产阶级社会内部成长起来”(328页)。其实不只

是“不能成长起来”,而且不能产生。在资本主义社会里,社会主义成分的合作经济和国营经济根本不能产生,当然也说不上成长。这是我们同修正主义者的主要分歧。修正主义者说,在资本主义社会中,像城市的公用事业是社会主义因素,说资本主义可以和平长入社会主义。这是对马克思主义的严重歪曲。

二、关于过渡时期

书中说:“过渡时期开始于无产阶级政权的建立,完成于社会主义革命任务的实现——建成社会主义即建成共产主义的第一个阶段”(328页)。究竟过渡时期包括什么阶段,要好好的研究。只包括从社会主义到共产主义,还是既包括从资本主义到社会主义,也包括从社会主义到共产主义?

这里引用了马克思的话:从资本主义到共产主义有一个革命转变时期。我们现在就是处在这个转变时期中。我们的人民公社要在若干年内,实现从基本队所有制到基本社所有制的转变,而且还要进一步转变为全民所有制。人民公社实现了基本社所有制的转变还是集体所有制。

在过渡时期中,要“进行一切社会关系的根本改造。”(328页)这个提法原则上对。所谓一切社会关系,应该包括生产关系和上层建筑,包括经济、政治、思想、文化等各方面的关系。

在过渡时期中要“使生产力得到保证社会主义胜利所必须的发展。”在我国来说,大约至少要一、二亿吨钢吧。今年以前,我们所做的事情,主要是为生产力的发展扫清道路。我国社会主义生产力的发展实际上刚开始,经过一九五八、一九五九年大跃进以后,六○年将是生产大发展的一年。

三、关于各国无产阶级革命的共同性和特殊性

书中说:十月革命“树立了一个榜样”,又说:“每一个国家具有自己特别的具体的社会主义建设的形式和方法”。(329页)这个提法好。在1848年有一个《共产党宣言》,在110年以后又有一个《共产党宣言》,这就是一九五七年各国共产党的莫斯科宣言。这个宣言就讲到了普遍规律和具体特点相结合的问题。

承认十月革命的榜样,承认任何国家无产阶级革命的“基本内容”都是一样的,这就和修正主义对立起来了。

革命为什么不首先在西方那些资本主义生产水平很高,无产阶级人数很多的国家成功,而首先在东方那些资本主义生产水平比较低,无产阶级人数比较少的国家成功,例如俄国和中国?这个问题值得研究。

为什么无产阶级首先在俄国取得胜利?教科书上说:“是由于俄国是帝国主义一切矛盾的集合点。”(329页)从过去的革命历史来看,革命的中心是由西方向东方转移。18世纪末革命中心在法国,当时法国成了世界政治生活的中心。19世纪中叶革命中心转到了德国,无产阶级走上了政治舞台,产生了马克思主义。20世纪初叶革命中心转到了俄国,产生了列宁主义,这是马克思主义的发展,没有列宁主义就没有俄国革命的胜利;二十世纪中叶,世界革命的中心又转到了中国,当然以后革命的中心还会转移。俄国革命的胜利,还因为有广大的农民群众做无产阶级革命的同盟军。教科书说:“俄国无产阶级和贫农结成联盟”(329页)。

农民中有几个阶层,无产阶级在农村中的依靠是贫农阶层。在革命开始时,中农总是动摇的,他要看一看。革命有没有力量,能不能站住,革命对他有没有好处,看得比较清楚了,他才转到无产阶级这方面来。十月革命是这样,我国的土地改革、合作化、人民公社化也都是这样。

俄国布尔什维克和孟什维克的分裂,从思想上、政治上、组织上准备了十月革命的胜利。如果没有布尔什维克和孟什维克的斗争,同第二国际修正主义的斗争,十月革命要取得胜利是不可能的。列宁主义是在反对一切修正主义和机会主义的斗争中产生和发展起来的,没有列宁主义也就没有俄国革命的胜利。

书中说:“无产阶级革命首先在俄国取得了胜利。革命前的俄国有足以使无产阶级革命取得胜利的资本主义发展水平。”(329页)无产阶级革命的胜利不一定要在资本主义发展水平很高的国家里。书中所引列宁的话很对,一直到现在,社会主义革命成功的国家,资本主义发展水平较高的只有东德和捷克,其它的国家资本主义发展水平都比较低。西方资本主义发展水平较高的国家革命都没有起来。列宁曾说过“革命首先从帝国主义薄弱的环节突破”。十月革命时的俄国是这样的薄弱环节,十月革命后的中国也是这样的薄弱环节。俄国和中国的共同点都是有相当数量的无产阶级,都有大量被压迫的痛苦的农民群众,都是大国。……在这些方面来说,印度也是相同的。那么,印度为什么不能像列宁、斯大林说的那样突破帝国主义的薄弱环节取得革命的胜利呢?因为印度是属于英国一个帝国主义国家的殖民地,这一点和中国不同。中国是几个帝国主义统治下的半殖民地。印度共产党没有积极参加他们国家的资产阶级民主革命,没有使无产阶级在民主革命中取得领导权。到了印度独立后,又没有坚持无产阶级的独立性。

中国和俄国的历史经验证明,要取得革命的胜利,有一个成熟的党,是一个很重要的条件。俄国布尔什维克党积极参加民主革命,在1905年提出了与资产阶级相区别的民主革命的纲领。这个纲领不只是要解决推翻沙皇的问题,而且要解决无产阶级在推翻沙皇的革命斗争中同立宪民主党争取领导权的问题。中国在1911年的资产阶级革命(辛亥革命)时还没有共产党。1921年中国共产党成立以后,立即积极参加民主革命,站在民主革命的前头。中国资产阶级的黄金时代是在1905—1917年,那时他们的革命活动很有生气。辛亥革命以后,国民党已经堕落,到了1924年没有办法只好找共产党,才看到前途。无产阶级代替了资产阶级的地位。无产阶级政党代替了资产阶级政党成为民主革命的领导者。我们常说中国共产党在1927年的时候还不成熟,从主要的意义上来说,就是指我们的党在同资产阶级联盟时没有看到资产阶级叛变革命的可能,并且也没有做应付这种叛变的准备。

教科书在这里(331页)还有这种意思:资本主义前的经济形式占优势的国家之所以能实现社会主义革命是由于先进的社会主义国家的帮助。这样说是不完全的。中国民主革命胜利后,能够走上社会主义道路主要的是由于我们推翻了帝国主义、封建主义和官僚资本主义的统治,国内的因素是主要的。已经胜利了的社会主义国家对我们的帮助是一个重要条件。但是这种帮助不能决定我们能不能走社会主义道路的问题。只能影响我们走上社会主义道路以后,前进得快点和慢点的问题。有帮助可以快一点,没有帮助会慢一点。所谓帮助,包括他们经济上的援助,同时也包括我们对他们成功和失败的正面和反面的经验的学习。

四、关于“和平过渡”

书上说:“某些资本主义国家和过去的殖民地国家中,工人阶级通过议会和平地取得政权是有可现实的可能性的。”(330页)这里的“某些”究竟是哪些呢?欧洲的主要资本主义国家北美洲的国家,现在都武装到了牙齿,他们能让你和平地取得政权吗?

每一个国家的共产党和革命力量都要准备两手,一手是和平方法取得胜利,一手是暴力取得政权,缺一不可。而且要看到,就总的趋势来说,资产阶级不愿意放弃政权,他们要挣扎,资产阶级在要命的时候他们为什么不用武力?十月革命和我国革命都曾是准备了两手的。俄国1917年7月以前,列宁曾经想用和平的方法取得胜利。七月事件表明了把政权和平地转入无产阶级手里已不可能,就转过来进行了三个月的武裴准备,才取得了十月革命的胜利。经过十月革命,无产阶级夺得政权以后,列宁还想用和平的方法,用“赎买”的方法实现社会主义改造,但是资产阶级勾结了十四个帝国主义国家,发动了反革命的武装暴动和武装干涉,在俄国党领导下进行了三年的武装斗争才巩固了十月革命的胜利。

五、关于从民主革命到社会主义革命转变问题

330页最后一段说到从民主革命到社会主义革命的转变,如何转变?没有讲清楚。十月革命是社会主义革命,它附带完成了资产阶级民主革命遗留下来的任务,十月革命胜利以后立即宣布土地国有令,但是完成土地问题的民主革命也还用了一段时间。

我国在解放战争中,解决民主革命的任务。1949年中华人民共和国的建立标志着民主革命的基本完成,和向社会主义过渡的开始。我们还用了三年的时间来完成土地改革,但是在中华人民共和国成立时,我们立即没收了占全国工业、运输业固定资产80%的官僚资本主义企业转为全民所有制。

我国在解放战争中除了提出反帝反封建的号召外,还提出了反对官僚资本主义。反对官僚资本主义的斗争包含着两重性,一方面是反官僚资本就是反买办资本是民主革命性质的,另一方面反官僚资本是反对大资产阶级又带有社会主义革命的性质。

官僚资本中的很大一部分是抗战胜利后,国民党从日本、德国、意大利手中接收过来的。官僚资本和民族资本的比重是8:2.我们解放后,全部没收了官僚资本,就把中国资本主义的主要部分消灭了。

如果以为在我们全国解放以后,“革命在最初阶段主要是资产阶级民主性质的,只是后来才逐渐地发展成为社会主义革命”。这是不对的。

六、关于暴力和无产阶级专政

333页上对暴力这个概念使用得不够确切。马克思、恩格斯总是讲“国家就是用来镇压敌对阶级的暴力机关”,那么,怎么能够说“无产阶级专政不仅仅是对剥削者使用暴力,甚至主要不是使用暴力?”

剥削阶级在要命的时候总是要动武的。而且只要他们看到革命一起来,他们就要用武力把革命扑灭。教科书说:“历史经验证明,剥削阶级不愿意把政权让给人民而使用武力反对人民政权”(333页),这个说法不完全。不仅在人民已经组织了革命政权以后,剥削阶级要用暴力来反对革命政权,而且当人民起来向他们夺取政权的时候,他们就用暴力来镇压革命的人民。

我们人民革命的自的是要发展社会生产力,为此,第一要推翻敌人,第二要镇压敌人的反抗,没有人民革命的暴力怎么能行呢?

书中这里还说到无产阶级专政的“实质”,说到工人阶级和劳动人民在社会主义革命中的“主要任务”,也说的不完全。没有说对敌人的镇压,也没有提到阶级的改造,地主、官僚、反革命分子、坏分子要改造,资产阶级、上层小资产阶级要改造,农民也要改造。我们的经验证明,改造是不容易的,不经过反复的多次的斗争,都是不能改造好的。彻底消灭资产阶级残余势力和他们的影响,至少要十年、二十年的时间,甚至要半个世纪。在农村来说,基本的社有制实行了,社有变国有了,全国布满了新的城市和大工业,全国交通运输都现代化了,经济情况真正全面改变了,农民的世界观才能逐步的以至完全的改变过来。(按书中在这里讲到“主要任务”时,引用列宁的话,与列宁的原意(注:原文为“愿意” 是不符合的)

说话,写文章都力求合乎敌人、帝国主义的口味,这是欺骗群众,其结果是敌人舒服,而自己的阶级被蒙蔽了。

七、关于无产阶级国家的形式问题

334页上说,无产阶级国家的形式可以有各种各样,这是对的。但是人民民主国家无产阶级专政的形式,同十月革命后在俄国建立的无产阶级专政的形式,其实质并没有多大区别。苏联的苏维埃和我国的人民代表大会都是代表会议,只不过是名称不同,我国的人民代表大会中有以资产阶级名义参加的代表,有从国民党分裂出来的代表,有其他民主人士的代表,他们都接受共产党的领导,其中有一部分人想闹事,但闹不起来。这种形式好像与苏维埃不同,但是十月革命后,苏维埃的代表中有孟什维克右派社会革命党,有托洛茨基派、布哈林派、季诺维也夫派等等。他们名义上是工人、农民的代表,实质上是资产阶级代表。那时(指十月革命后)无产阶级接受了克伦斯基的国家机关中的大量人员,这些人都是资产阶级分子。我国中央人民政府是在华北人民政府的基础上成立起来的,各部门的成员都是根据地里出来的,而且大多数的骨干都是共产党员。

八、关于对资本主义工商业的改造

335页上关于中国的资本主义所有制变成社会主义国家所有制的过程说得不对。只说了我们对民族资本的政策,没有说我们对官僚资本的政策(没收),对于官僚资本的财产,我们是采取没收的办法来实现公有化的。

339页上第二段的意思是把经过国家资本主义的形式来改造资本主义当成一种个别的特殊的经验,否定这种经验的普遍意义。西欧各国和美国资本主义发展水平很高,少数的垄断资本家占统治地位,但同时也还有大量的中小资本家,据说美国的资本是集中的,又是分散的。在这些国家革命成功以后,垄断资本要没收是没有问题的,但是中小资本家也一律没收吗?是不是也要采取国家资本主义的形式来改造他们呢?

我国的东北可以说是资本主义发展很高的地区,以上海和苏南为中心的江苏省也可以说是资本主义发展很高的地区,既然我国这些省区可以实行国家资本主义,那么世界上同我们这些省分类似的国家为什么不可以实行这个政策呢?

日本人过去在东北的办法是消灭当地的大资本家,把他们的企业变成日本的国营企业,或者垄断资本的企业,而对于当地的中小资本家,则用建立母子公司的办法来加以控制。

我们对于民族资本的改造经过三个步骤,即加工订货、统购包销、公私合营(单个企业的公私合营和全行业的公私合营)。就每个步骤说,也是逐步进行的,这种办法使生产没有遭到什么破坏,而且在改造过程中发展了。我们在国家资本主义问题上是有很多新的经验,公私合营以后,给资本家定息,就是一项新经验。

九、中国农民问题

我国土地改革后,土地不值钱,农民不敢“冒尖”。有的同志曾经认为这种情况不好,其实经过阶级斗争,搞臭了地主富农,农民以穷为荣,以富为耻,这是一种好现象,这说明贫农在政治上已经压倒了富农,而树立了自己在农村中的优势。

教科书中说“中农成了农业中的中心人物”(339页),这个说法不好。把中农吹成中心人物,捧到天上去,不敢得罪他们,会使过去的贫农脸上无光,其结果必然导致富裕中农掌握农村的领导权。

书中对于中农没有分析,我们把中农分成上中农、下中农,其中还有新、老之别,新的又比老的好些。历次运动经验证明:贫农、新下中农、老下中农三部分人政治态度较好,拥护人民公社的是他们,在上中农、富裕中农中,一部份人拥护人民公社,一部分人反对人民公社。河北省的材料全省共有四万多个生产队,其中50%完全拥护公社,没有动摇;35%的队基本拥护,在个别问题上有意见或者动摇;有15%的队或者反对,或者发生严重动摇。这些队所以发生严重动摇和反对,主要的原因是这些队的领导权被掌握在富裕中农手里,甚至掌握在坏分子手里。在这次两条道路斗争的教育中,这些队要展开辩论,首先要改变领导,可见对中农要进行分析。农村的领导权掌握在谁手里,对农村发展的方向关系极大。

339页中说:“从富农那里没收来交给贫农和中农的土地”。政府没收,然后政府把土地交给农民来分,这是一种恩赐观点,不搞阶级斗争,不搞群众运动。这种观点实质上是一种右倾观点。我们的办法是依靠贫农,联合大多数中农(下中农)向地主阶级夺取土地,党起引导作用,反对包办代替,并且有一套具体的作法,那就是访贫问苦,物色积极分子,扎根串联,团结核心,进行诉苦,组织阶级队伍,展开阶级斗争。

书上说(340页):“中农按其本性说来是两重性的。”对这问题也要作具体分析。贫农,下中农、上中农、富裕中农一方面都是劳动者,一方面又都是私有者,但是作为私有者来说,他们的私有观念是各不相同的。贫农、下中农可以说是半私有者,他们的私有观点比较容易改变,上中农和富裕中衣的私有观念就此较浓厚,历来他们对于合作化有抵触。

十、关于工农联盟

340页三、四段上讲了工农联盟的重要性,但是没有叙述工农联盟怎样才能发展和巩固。讲了对小生产者农民要进行改造,但没有讲进行改造的过程,没有讲这个过程中每个阶段中有什么矛盾,如何解决这些矛盾,也没有叙述整个改造过程的步骤和策略。

我们的工农联盟已经经过了两个阶段,第一是建立在土地革命的基础上,第二建立在合作化的基础上。不搞合作化,农民必然要两极分化。工农联盟就无法巩固,统购统销就无法坚持,只有在合作化的基础上统购统销的政策才能继续,才能彻底实行,现在我们的工农联盟要进一步建立在机械化的基础上,单有合作化、公社化而无机械化,工农联盟是不能巩固

的。就合作化来说,如果只是小合作化,工农联盟也是不巩固的。还必须从合作化发展到人民公社,还必须从人民公社基本队有发展到基本社有,再由社有发展到国有,在国有化和机械化互相结合的基础上,我们就能够把工农联盟真正的巩固起来,工农之间的差别就会逐步消失。

十一、关于知识分子的改造

341页专讲培养工农自己的知识分子,吸收资产阶级知识份子,参加社会主义建设,但没有讲对知识分子的改造,不但资产阶级知识分子要改造,就是工农出身的知识分子也因在各方面受资产阶级的影响而需要进行改造。文艺界的刘绍棠当了作家以后,大反社会主义就是证明。在知识分子中,世界观的问题常常表现在对知识的看法上,究竟知识是公有还是私有?有些人把知识看成自己的财产,待价而沽,没有高价钱就不出卖,他们只专不红,说党“外行”不能领导“内行”,搞电影的说党不能领导电影,搞歌舞的说党不能领导歌舞,搞原子能科学的说党不能领导原子能科学事业,总之说党不能领导一切。

在整个社会主义革命和社会主义建设的过程中,改造世界观的问题是一个极大的问题,不重视这个问题,对资产阶级的东西采取将就的态度当然是不对的。

同页上说过渡时期的经济的基本矛盾是社会主义和资本主义的矛盾,这是对的。但这里只说在经济生活上的一切领域中开展谁战胜谁的斗争,都是不完全的。我们的说法是在三条战线上即政治、经济、思想的战线上都要进行彻底的社会主义革命。

书上说我们吸收资产阶级分子参加企业管理和国家管理(421页也这样说),但我们提出了对资产阶级分子进行改造的任务,帮助他们改变自己的生活习惯、世界观以及个别问题上的观点,书上在这里都不提改造。

十二、关于工业化和农业集体化的关系

书上把社会主义工业化看成是农业集体化的前提,这种说法并不合乎苏联自己的情况。苏联基本上实现集体化是在1930年至1932年,那个时候,他们的拖拉机虽然比我们多,但是1932年机耕面积不到耕地面积的20.3%。集体化不完全决定于机械化。故工业化不是前提。

东欧的社会主义国家的农业集体化完成得很慢,主要的原因是在土地改革以后,没有趁热打铁,而是间歇了一个时期,我们的一些根据地也出现过一部分农民满足于土地改革而不愿再向前进的现象。问题并不在于有没有工业化。

十三、关于战争与革命

(352——354页)书上说,东欧各人民民主国家,“能够在没有国内战争和外国武装干涉的情况下建设社会主义”,又说,这些国家社会主义改造的实现“没有经过国内战争”。应当说,这些国家是通过国际战争的形式来进行国内战争的,是国际战争和国内战争合而为一的进行的,这些国家的反动派是苏联红军的铁犁犁掉的。说这些国家没有国内战争是只从形式上看问题,没有看到实质的说法。

教科书说东欧各国在革命后“议会成了广泛代表人民利益的机构”,其实这种议会同旧的资产阶级议会完全不同。只是形式上的相同。我们解放初期的政治协商会议,名义上同国民党时期的政治协商会议是一样的,同国民党谈判的时候,我们对政治协商会议不感兴趣,蒋介石却很有兴趣。解放以后,我们把这个招牌接过来,召开了中国人民政治协商会议,起了临时人民代表大会的作用。

教科书上说中国“在革命斗争的进程中,组成了人民民主统一战线”,(357页)为什么只提革命斗争,不提革命战争?从1927年起直到全国胜利我们进行了22年延续不断的革命战争,在这以前,从1911年的资产阶级革命开始,还有15年的战争,这里面有革命战争,也有在帝国主义指使下的军阀混战。如果从1911年算起,一直到抗美援朝战争,中国可以说连续进行了40年的战争,其中包括革命的战争和反革命的战争,我们党成立之后参加和领导的革命战争就有30年。

大革命不能不经过国内战争,这是一个法则。只看到战争的坏处,不看到战争的好处,这是战争问题上的片面性,片面的讲战争的毁灭性对于人民革命是不利的。

十四、落后国家的革命是否更困难?

在西方各国进行革命和建设有一个很大的困难,这就是资产阶级的毒害很厉害,已经渗透到各个角落里去了,我国的资产阶级还只有三代,而英、法国这些国家的资产阶级已经有了十几代了。他们资本主义发展的历史有二百五六十年至三百多年,资产阶级思想作风影响到各个方面各个阶层,所以英国的工人阶级不跟着共产党走而要跟工党走。

列宁说:“国家愈落后,它由旧的资本主义关系过渡到社会主义关系就愈困难”。(353页)这个说法现在看来不对。其实经济越落后,从资本主义过渡到社会主义愈容易,而不是越困难,人越穷,越要革命。西方资本主义国家就业人数比较多,工资水平比较高,劳动者受资产阶级的影响很深,在那些国家进行社会主义改造看来并不那么容易。这些国家机械化程度很高,革命成功后,进一步提高机械化,问题不大,重要的问题是人民的改造。在东方像俄国和中国这样的国家,原来都是落后的,贫穷的,现在不仅社会制度比西方先进得多,而且就生产力发展的速度也比他们快得多。就资本主义各国发展的历史来看也是落后的赶过先进的,例如在19世纪末叶,美国超过英国,后来二十世纪初德国又超过英国。

十五、大工业是社会主义改造的基础吗?

教科书说:“走上社会主义建设道路的国家面临着这样一项任务:以加快发展大工业(对经济进行社会主义改造的基础)的办法,最迅速地消除资本主义统治的这些后果。”(364页)这里把发展大工业说成是对经济进行社会主义改造的基础,说得不完全。一切革命的历史都证明并不是先有充分发展的新生产力,然后才能改造落后的生产关系。我们的革命开始于宣传马列主义,这是要造成新的社会舆论,以推行革命,在革命中推翻落后的上层建筑以后方有可能消灭旧的生产关系,旧的生产关系被消灭了,新的生产关系建立起来了,这就为新的社会生产力的发展开辟了道路,于是就可以大搞技术革命,大大发展社会生产力。在发展生产力的同时,还要继续进行生产关系的改造,进行思想改造。

这本教科书只讲物质前提,很少涉及上层建筑即阶级的国家,阶级的哲学,阶级的科学。经济学研究的对象主要是生产关系,但是政治经济学和唯物史观难得分家,不涉及上层建筑方面的问题,经济基础生产关系的问题,不容易说得很清楚。

十六、列宁论走向社会主义道路的特点

(375页)引用列宁的一段话,讲得很好,可以用来辩护我们的作法。他讲到:“居民的觉悟程度和实行这种计划或那种计划的尝试等等都一定会在走向社会主义道路的特点中反映出来。”我们的政治挂帅就是为了提高居民的觉悟程度,我们的大跃进就是实现这种计划或那种计划的尝试。

十七、工业化的高速度是个尖锐的板题

教科书上说:“工业化的速度对于苏联是一个很尖锐的问题。”(376页)现在我国的速度问题也是一个很尖锐的问题。原来工业越落后速度问题越尖锐,不但国与国之间比较是这样,就是一个国家内部,这个地区和那个地区比较起来也是这样。例如,我国的东北和上海.因为那里的基础比较好,国家对这些地区的投资增加相对地慢一些。而另外一些原有工业基础薄弱,而又迫切需要发展的地区,国家在这些地区的投资增加得很快。上海解放十年一共投资22亿元。其中包括资本家投资2亿多元,它原有工人50多万人,除了调出几十万工人外,现在全市有工人百多万人,只比过去增加一倍。这同一些职工大量增加的新城市比较就可以明显地看到工业基础差的地方速度问题更加尖锐。书上这段话只讲了政治环境要求高速度。没有讲到社会主义制度本身许可高速度。这是一种片面性,如果只有高速度的需要而没有可能,那么怎能做到高速度呢?

十八、大、中、小、并举是为了高速度

381页上虽然提到我们广泛发展中小型企业,但并没有正确的反映我们土洋并举,大中小并举的思想。说我们“规定广泛地发展中小型企业,这是由于国内技术经济十分落后,人口众多以及与此关连的就业问题”。问题并不在于技术落后,人口众多,增加就业。在大的主导下,大量的发展中小型,在洋的主导上普遍采用土法,主要是为了高速度。

十九、两种社会主义所有制可以长期并存吗?

教科书386页上说:“社会主义国家和社会主义建设不能在相当长的时期内建立在两个不同的基础上,就是说,不能建立在最巨大最统一的社会主义工业基础上和散漫而落后的农民小商品经济基础上。”这个说法当然是正确的。由此推论下去就可以合乎逻辑地得出这样的结论:社会主义国家和社会主义建设不能在相当长的时期内建立在全民所有制和集体所有制两个不同的所有制的基础上。

苏联的两种所有制并存的时间太长,全民所有制和集体所有制的矛盾实际上是工农的矛盾,教科书上不承认这个矛盾。

全民所有制和集体所有制长期并存下去,同样会越来越不能适应生产的发展,不能充分满足人民生活对农业生产不断增长的需要,不能充分满足工业对原料不断增长的需要。而要满足这种需要就不能不解决这两种所有制的矛盾,不能不把集体所有制变成全民所有制,不能不在全国单一的全民所有制的基础上来统一计划全国的工业和农业的生产与分配。

生产力和生产关系的矛盾是不断发展的,生产关系这个时候适合生产力,过一个时候就不适合了。我国在完成高级合作化以后,每个专区,每个县都出现了小社并大社的问题。

社会主义社会里按劳分配、商品生产、价值规律等等,现在是适合于生产力发展要求的,但是发展下去总有一天要不适合生产力的发展,总有一天要为生产力发展所突破,总有一天它们要完结自己的命运,能说社会主义社会的一些经济范畴是永恒不变的吗?能说按劳分配,集体所有制这些范畴是永久不变的,而不像其他范畴一样是历史范畴吗?

二十、农业的社会主义改造不能只靠机器

382页上说:“机器拖拉机站是对农业实行社会主义改造的重要工具。”教科书上很多地方强调机器对社会主义改造的作用,但是如果不提高农民的觉悟,不改造人的思想,只靠机器,怎么能行?两条道路斗争的问题,用社会主义思想训练人和改造人的问题在我国是个大问题。

397页上说:实行全盘集体化的初期的任务,提到和敌对富农分子的斗争等等,这当然是对的。但是教科书对合作化以后农村的情况的叙述,都不讲富裕阶层的问题,也不讲内部矛盾。例如国家和集体与个人之间的矛盾,积累和消费之间的矛盾等等。

402页上说:“由于农业合作化运动的发展,广大的中农群众不再动摇”,不能笼统地这样说。一部分富裕中农现在动摇,将来还会动摇。

二十一、所谓“彻底巩固”

“彻底巩固”集体农庄制度。”(407页)“彻底巩固”这四个字看了不舒服。任何东西的巩固都是相对的,怎么能彻底呢?如果自有人类以来,所有的人都不死,都“彻底巩固”下来,这个世界怎么得了?宇宙间、地球上的一切事物,都是不断发生、发展和死亡,都是不能彻底巩固的。就蚕的一生来说,不但它最后一定要死亡,而且在它的一生发展过程中要经过蚕子、蚕、蛹、飞娥这四个阶段,每一阶段都要进行到后一阶段,每一阶段都不能彻底巩固的。飞蛾最后死了,旧的质变成新的质(新下来很多蚕子),这是一个质的飞跃。但是从蚕子到蚕,到蛹,到飞蛾的发展中显然也不只是量变,而且有质的变化——是部分质变。人也是从生到死的这个过程中,经过童年、少年、青年、壮年、到老年这样不同的阶段。人从生到死是一个量变过程,同时也是不断地进行部分质变的过程。难道能够说,从小到大,从大到老只有量的增加,没有质的变化?人的机体里,细胞不断地分裂,不断有旧的细胞死亡,新的细胞的生长,人死了就达到整个的质变,这个质变是通过以往的不断的量变,通过量变中不断的部分的质变而完成的。量变和变质是对立的统一,量变中有部分质变,不能说量变中没有质变;质变中有量变,不能说质变中没有量变。

在一个长的过程中,在进入最后的质变以前,一定经过不断的量变和许多部分质变。如果没有部分质变,没有大量的量变,最后的质变也不能到来。例如一个工厂,厂房有了,规模有了,里面的机器设备部分地,部分地更新,这就是部分的质变。工厂的规模和外形都没有变,但工厂的内部变了。一个连队也一样,百多人打了一仗,伤亡了几十个人,要补充几十个人,不断地战斗,不断地补充,就是这样经过不断地部分的质变使这个连队不断地发展坚强起来。

打垮蒋介石是一个质变,这个质变是通过量变完成的。例如要有三年半的时间,要一部分,一部分地消灭蒋介石的军队和政权,而这个量变中间同样有部分质变。在解放战争期间,战争经过几个不同的阶段,每个新的阶段和旧的阶段比较都有若干性质的区别。从个体经济转变到集体经济是一个质的变化过程。这个过程在我国是通过互助组、初级合作社、高级合作社、人民公社这样一些不同阶段的部分质变而完成的。

我国目前的社会主义经济是由全民所有制和集体所有制两种不同的公有制组成的。这种社会主义经济,有它的发生发展过程,难道就没有它进一步变化的过程吗?难道我们能让这两种所有制长远地“彻底巩固”下去么?在社会主义社会里面的按劳分配、商品生产、价值规律这些经济范畴难道是永生不灭的么?难道是只有生长发展,而没有死亡变化么?难道不像其它的历史范畴一样都是历史的范畴么?

社会主义一定要向共产主义过渡,过渡到共产主义社会的时候,社会主义阶段的一些东西必然要死亡。共产主义时期也还是有不断发展的。共产主义社会可能要经过许多不同阶段,能够说到了共产主义社会就什么都不变了吗?就一切都“彻底巩固”下去了,就只有量变没有不断的部分质变么?

事物的发展是一个阶段接着一个阶段的不断的进行的。但是每个阶段总是有个“边”。我们每天读书,从四点钟开始,到七、八点钟结束,这就是“边”。拿思想改造来说,社会主义思想改造是长期的,但每一次思想改造运动,总是有个结束,就是有个“边”。在社会主义的思想改造战线上经过不断的量变,不断地部分质变,总有一天资本主义思想的影响完全肃清了,到那时,这种思想改造的质变也就完成了,然后又会开始新的质的基础上的量变过程。

建成社会主义也有一个“边’,要有笔账。例如:工业产品占多大比重,生产多少钢,人民生活水平多么高等等。说建成社会主义有个“边”当然不是说不要进一步过渡到共产主义。从资本主义过渡到共产主义有可能分成两个阶段,一个是由资本主义到社会主义,这可以叫做不发达的社会主义,二是由社会主义到共产主义,即由比较不发达的社会主义到比较发达的社会主义即共产主义,后一阶段可能比前一个阶段需要更长的时间。经过了后一阶段,物质产品,精神财富,都大为丰富,人的共产主义觉悟大为提高,就可以进入到共产主义高级阶段了。

409页上说到社会主义生产方式“确立”以后,生产不断迅速地扩大,生产率不断提高,讲了许多“不断”但只有量的变化,没有许多部分的质变。

二十二、关于战争与和平

408页上说到在资本主义社会中,“不可避免地要造成生产过剩的危机和使失业者的增加”,这就是酝酿着战争。难道马克思经济学原理忽然失灵了么?难道世界上还存在着资本主义制度的时候就能彻底消灭战争吗?

能不能说现在出现永远消灭战争的可能性?出现了把世界一切物力财力利用来为全人类服务的可能性?这种说法没有马列主义,没有阶级分析,没有把资产阶级统治下的情况与无产阶级统治下的情况区别开来。不消灭阶级怎么能消灭战争?世界大战打不打不决定于我们。即使签订了不打仗的协定,战争的可能性也还存在。帝国主义要打仗的时候,什么协定也不算数。至于打起来用不用原子弹、氢弹,那是另外一个问题。虽然有了化学武器,但是打仗的时候没有用,还是用常规武器的。即使在两个阵营之间不打仗也不能保证资本主义世界内部不打仗,帝国主义与帝国主义可能打,帝国主义国家内部资产阶级和无产阶级也可能打仗,帝国主义与殖民地半殖民地现在就在打。战争是阶级冲突的一种方法,只有经过战争才能消灭阶级。只有消灭阶级才能永远消灭战争。不进行革命战争,就不能消灭阶级。我们不相信,没有消灭阶级,要消灭战争武器,这不可能。在人类的阶级社会历史上,任何阶级,任何国家,都是注意实力地位的。搞实力地位是历史的必然趋势。军队是阶级实力的具体表现。只要有阶级对抗,就有军队,当然我们是不希望打仗的,我们是希望和平的,我们赞成用极大的努力来禁止原子战争,并且争取两个阵营签订互不侵犯的协定,争取十年、廿年的和平,是我们最早提出来的主张,如果能够实现这个主张,对整个社会主义阵营,对我国社会主义建设都是很有利的。

409页说:现在苏联已不再受资本主义的包围了,这个说法有使人睡觉的危险。当然现在的情况已经和只有一个社会主义国家的时候有很大的改变,在苏联的西方有了东欧各社会主义国家,在苏联的东方有我们,朝鲜、蒙古、越南这几个社会主义国家,但是导弹没有眼睛,它可以打几千公里,万把公里,在整个社会主义的周围布满了美国的军事基地,这些军事基地的箭头都是朝向苏联和社会主义各国的,能够说现在已经不在导弹的包围之中了吗?

二十三、“一致”是社会发展的动力吗?

413页上说社会主义“团结一致”,“十分稳定”,说一致就是“社会发展的动力”。

只承认团结一致,不承认社会主义社会内部有矛盾,不承认矛盾是社会发展的动力。这样一来矛盾的普遍性这个规律就被否定了,辩证法就中断了。没有矛盾就没有运动,社会总是运动发展的,在社会主义时代,矛盾仍然是社会发展的动力,因为不一致才有团结的任务,才需要为团结而斗争,如果总是十分一致那还有什么必要不断进行团结的工作呢?

二十四、关于社会主义制度下劳动者的权利

414页讲到劳动者享受的各种权利时,没有讲劳动者管理国家、管理各种企业、管理文化教育的权利。实际上这是社会主义制度下劳动者最大的权利,这是最根本的权利,没有这种权利,就没有工作权、受教育权、休息权等等。

社会主义民主的问题,首先就是劳动者有没有权利来克服各种敌对势力和它们的影响的问题,像报纸、刊物、广播、电影这类东西掌握在谁的手里,由谁来发议论,都是属于权利的问题。如果这些东西掌握在右倾机会主义分子这些少数人手里,那么全国绝大多数迫切需要大跃进的人在这些方面的权利就被剥夺了。如果电影掌握在钟惦棐这些人手里,人民又怎么能够在电影方面实现自己的权利呢?人民内部有各种派别,有党派性,一切机关、一切企业,掌握在哪一派的手里,对于保证人民的权利问题关系极大。掌握在马列主义者手里,绝大多数人民的权利就有保证了,掌握在右倾机会主义分子或者右派分子手里,这些机关这些企业就可能变质,人民对这些机关这些企业的权利就不能保证。总之人民必须有权管理上层建筑。我们不能够把人民的权利问题了解为国家只是由部分人管理,人民只能在某些人的管理下面享受劳动、教育、社会保险等等权利。

二十五、向共产主义过渡是不是革命

417页说:“社会主义制度下,没有同共产主义的利益相冲突的阶级和社会团体,所以向共产主义过渡是不通过社会革命完成的。”

向共产主义过渡,当然不是一个阶级推翻另一个阶级,但是不能说这不是社会革命。因为一种生产关系代替另一种生产关系,就是质的飞跃,就是革命。我国的个体经济变为集体经济,再从集体经济变为全民经济,都是生产关系方面的革命。由社会主义的按劳分配转变为共产主义的按需分配,也不能不说是生产关系方面的革命。当然按需分配是逐步实现的,可能是主要物资能充分供应了,首先对这些物资实行按需供应,然后根据生产力的发展推行到其它产品去。

拿我国的人民公社的发展来说,在从基本队有制转变为基本社有制的时候,在一部分人中间会不会发生抵触现象,这个问题值得研究。我们实现这个转变的一个决定性条件是社有经济的收入占全社总收入的一半以上,实现基本社有制,对于社员一般都是有利的。这样估计对绝大多数人不会抵触,但是原来的队干部,那个时候,他们不能像原来那样当家做主了,他们的管理权力势必相对缩小,他们对于这种改变会不会抵触呢?

共产主义社会虽然消灭了阶级,但是在发展过程中也会有某种“既得利益集团”的问题,他们安于已有的制度,不愿意改变这种制度,例如实行按劳分配,多劳多得,对他们很有利,在转到按需分配时,他们可能会不舒服。任何一种新制度的建立,总要对旧制度有所破坏,不能只有建设没有破坏。要破坏,就会引起一部分人的抵触。人这个动物就是怪,他有一点优越条件就有架子,……不注意很危险。

二十六、所谓“中国没有必要采用那样尖锐的阶级斗争形式”

419页说得很不对。

十月革命后的俄国资产阶级,看到那个时候俄国经济遭到严重破坏的情形,断定无产阶级不能改变这种情况,无产阶级没有力量保持自己的政权,认为只要他们动手,就能使无产阶级政权垮台,于是他们实行武装的反抗。这就逼着俄国的无产阶级不得不采取激烈的办法来没收他们的财产,那时候,资产阶级和无产阶级双方都还没有经验。

说中国的阶级斗争不尖锐,这不合实际,中国革命可尖锐呢。我们连续打了22年的仗。我们用战争打垮了资产阶级国民党的统治,没收了占整个资本主义经济80%的官僚资本,这样才使我们有可能对占20%的民族资本采取和平办法来加以改造,在改造过程中,还经过了“三反”、“五反”那样激烈的斗争。

420页上对资本主义工商业改造描写得不对。解放以后,民族资产阶级走上社会主义改造的道路是逼出来的。我们打倒了蒋介石,没收了官僚资本,完成了土地改革。进行了“三反”、“五反”,实现了合作化,从一开始就控制了市场,这一系列的变化,一步一步地逼着民族资产阶级不能不走上接受改造的道路。另一方面,共同纲领规定了各种经济成分各得其所,使资本家有利可图的政策,宪法又给了他们一张选票、一个饭碗的保证。这些又使他们看到接受改造就能够保持一定的地位,并且能够在经济上文化上发挥一定的作用。

在公私合营的企业中,资本家对企业没有实际上的管理权,并不是公方代表和资本家共同管理生产,不能说:“资本对劳动的剥削受到了限制”,而是受到了很大的限制。教科书上没有吸取我们所说的公私合营企业是3/4的社会主义这个意思,当然现在说来不是3/4,而是9/10.甚至更多了。

资本主义工商业的改造已基本完成,但是有机会他们还是要猖狂进攻的。1957年右派进攻被我们打退了,1959年又通过他们在党内的代表向我们进行了一次进攻。我们对民族资本家的策略是拉住他们又整住他们。

书中引用了列宁的话(421页)国家资本主义是“阶级斗争另一种形式的继续”这是对的。

二十七、关于建成社会主义的期限

423页上说:我们在1957年政治战线、思想战线“完成了”社会主义革命,我们不这样说,而是说取得了决定性胜利。

同页上说,要在十年或十五年内把中国变成强大的社会主义国家,这倒可以同意。这就是说,在第二个五年计划以后,再经过两个五年计划到1972年争取提前二、三年到1969年,除了实现工业现代化、农业现代化、科学文化现代化以外,还要国防现代化。在我们这样的国家,完成社会主义建设,是一个很艰巨的任务,建成社会主义不要讲早了。

二十八、再谈工业化和社会主义改造的关系

423页上说,“在中国的特殊条件下,社会主义能在国家工业化实现以前,就在所有制方面(包括农村在内)取得胜利,是因为有强的社会主义阵营存在,有苏联这样高度发展的工业国家的援助。”这种话讲得不对。在东欧这些国家同我们一样“都有强大的社会主义阵营存在,有苏联这样的高度发展的工业国家的援助”这样两个条件,为什么不能在工业化实现以前完成所有制方面(包括农村在内)的社会主义改造呢?至于工业化和社会主义改造的关系问题,实际上,苏联也是先解决了所有制的问题,然后才实现工业化的。

从世界的历史来看,资产阶级革命,资产阶级建立自己的国家也不是在工业革命之后,而是在工业革命以前,也是先把上层建筑改变了,有了国家机器,然后进行宣传取得实力,才大大推动生产关系的改变,生产关系搞好了,走上轨道了,也就为生产力的发展开辟了道路。当然生产关系的革命是生产力的一定发展所引起的,但是生产力的大发展总是在生产关系改变之后。拿资本主义发展的历史来说,先是简单的协作,然后发展为工场手工业,这时已经形成了资本主义的生产关系。但是工场手工业还不是用机器生产,这种资本主义生产关系产生了改进技术的需要,为采用机器创造了条件,在英国是在资产阶级革命以后(17世纪以后)才进行工业革命(18世纪末到19世纪初)。德、法、美、日也都是经过不同的形式,改变了上层建筑、生产关系以后,资本主义工业才大大发展起来。

首先造成舆论夺取政权,然后才解决所有制问题,再大大发展生产力,这也是一般规律,无产阶级革命同资产阶级革命虽然在这个问题上有所不同(在无产阶级革命以前,不存在社会主义的生产关系,而资本主义生产关系已经在封建社会中初步成长起来)但基本上是一致的。


