\section[在八大二次会议上的讲话(二)(一九五八年五月十七日下午)]{在八大二次会议上的讲话(二)(一九五八年五月十七日下午)}
\datesubtitle{(一九五八年五月十七日)}


一、国际形势

讲讲卫星上天吧。上天是好的。这个卫星比第二个大一倍以上,几个月以后;一年、二年、几年以后,也许再搞大一点的,两千公斤的。我在莫斯科议会上讲搞五万公斤的,搞到五千公斤只是十分之一。突破这一关就可以搞两万到三万公斤,这是很大的好事。

资本主义世界现在乱子很多,我们这个世界现在乱子比较少,我们团结巩固,南斯拉夫不在我们阵营,它不算,不是我们不要,是他自己不干。我们阵营十二个国家形势很好,形势从来就是好,没有那一天不好过,不过有时天上有些乌云,有人认为我们不行,人家行,我说我们行。我在莫斯科会议上讲了十条证据,证明我们从来就行。蒋介石在南京,我们在延安,究竟那个行?那时延安只有七千人,还包括郊区在内,南京那么大,南京上海等大城市,都在蒋介石手里,他们有几百万军队,我们只有几十万游击队,从来就是小的战胜大的,弱的战胜强的,小的弱的有生命力,大的强的没有生命力,总的形势很好,希特勒、蒋介石、美帝国主义不在话下,我们从来把美帝国主义看成纸老虎,美帝国主义可惜只有一个,再有十个也不在话下,迟早它是要灭亡的。

日本人在北京和我说:“很惭愧,过去打过你们。”我说,你们做了好事。正因为有了你们的侵略,占领了大半个中国,使我们团结起来,领导全国人民,打走了你们,到了北京。我们在延安时说,那一年才看到梅兰芳程砚秋的戏,有的人怕这一辈子看不到,可是我们看到了,革命形势发展的很快,七年来全国团结,就推翻了蒋介石,现在又要团结起来建设。七大有个纲领,这次会议也是团结的大会、胜利的大会,也有个共同纲领,全党一致制定了一个建设社会主义的总路线,也是全国人民的总路线,全党团结,全国人民团结一致这是国内形势。

国际上乱子很多,帝国主义内部吵架,世界不太平,法国、阿尔及利亚、拉丁美洲、印尼、黎巴嫩乱子都出在资本主义世界,但我们都有关系,凡是反帝国主义的东西,都对我们有利,帝国主义内部吵架,他们压迫印尼、黎巴嫩、拉丁美洲,还争夺阿尔及利亚(不详讲,看材料)。总而言之,有时似乎形势不好,天上有乌云,这种时候我们要有远见,不要被暂时的现象所迷惑,不要被暂时的黑暗所迷惑,以为我们就不好了,就觉得世界不好了,要倒霉了,没那个事!我们过去最不好的那一段是万里长征,前堵后追,军队少了,只剩下一点点,地方小了,党也小了,十个指头剩下了一个,那样的困难都克服了,得到了锻炼。以后机会来了,又发展了,又由一个指头发展到十个指头,一直发展到成立中华人民共和国,取得全国的胜利。苏共党史第一章第一页就讲到由小到大的辩证法,苏共由几个人开始的小组,发展成为苏维埃联邦的大党。他们当时一支枪也没有,而他们的敌人先是沙皇,后来是克伦斯基政府,都是全付武装的,是全付武装强,还是手无寸铁强?你说那个强,我说手无寸铁的人强。最后是谁战胜谁?我们党的情况也是一样。一九二一年我党成立,只有几十个人,第一次代表大会到的十二个代表,董老就有你呀!你参加了吧!参加这次大会的周佛海是个“好同志”(笑声)。有个陈公博又是个“好同志”(笑声),陈独秀没有到会,因为他有威望,选他当总书记,可是他不成材,他不成器,他是伯恩斯坦主义,民主革命他干,是激进派,社会主义他不懂,他不懂不断革命,犯了错误,想一想我们党的历史,我们经过了多少困难!有一段万里长征,还有一段三中全会到四中全会,四中全会在上海开,没有几个人了,危机存亡,党在分裂。

二万五千里长征的时候也是党在分裂,党经过的分裂,以后又团结。张国焘跑了,党恢复了团结,后来在延安,蒋介石和日本包围我们,将我们分割成十几块根据地,那样困难的局面,到底延安强些还是南京强些?我们强些还是蒋介石强些?现在证明是我们强,不然为什么现在我们能在怀仁堂开会呢?他为什么跑到台湾呢?是谁胜利?

中国是国际形势中的重要组成部分,讲到国际形势就要讲中国,举中国例子,就证明劳动人民被压迫者有生命力。现在社会主义有很大的同盟军,亚洲、非洲、拉丁美洲的民族独立运动是我们的同盟军。这些地区是帝国主义的后方,在它们后方有我们的同盟军,我们绕到帝国主义后方来了。列宁说:“先进的亚洲,落后的欧洲。”欧洲、英、法、意、西德、比利时、葡萄牙都落后了,美国也落后了。你看他们先进我们先进?斯大林懂得这一点,一九四九年六月×××率领我党代表团到苏联,斯大林在宴会上举杯祝贺中国将来要超过苏联,×××同志说:“这杯酒我们不能喝,你是先生嘛,我们是学生,我们赶上你,你又前进了。”斯大林说:“不对,学生不超过先生,那还算什么好学生,一定要喝。”僵了一二十分钟,最后×××同志还是喝了。先生教了学生,学生赶不上先生就不争气。这说明不仅列宁,连斯大林也看出了先进的东方。师高弟子强。我们不要狂妄,把尾巴翘到天上去,也不要有自卑感,妄自菲薄。要破除迷信,把自己放到恰当的地位。应当敢想、敢说、敢做,基础是马列主义。铁托也敢想、敢说、敢做,但他的基础是帝国主义,资本主义,而不是马列主义。我们的基础是马列主义,因此我们是正确的,所以敢想、敢说、敢做是不会出乱子的。

二、国内形势

讲讲国内问题。国内问题还是一个农民同盟军问题。中国革命始终是农民同盟军问题。工人阶级假如没有农民同盟军,就不能得到解放,就不能建设强大的国家.解放前。我国工人阶级数目只有四百万人(手工业除外),现在有一千二百万人,增加了两倍。连家属在内不过四千万左右。而农民则有五亿多,中国的问题始终是农民同盟军问题。有些同志对这个问题不很清楚,在农村混几十年也不清楚。一九五六年为什么犯反冒进的错误?主要原因就是在这个问题上。对农民思想情绪不太懂,因此就没有根,风浪一来就容易动摇。一九五六年我们出了一本农村社会主义高潮的书,搜集了各省、自治区一百九十几个合作社的资料,那一省都有几篇文章,只有西藏没有。其实不需要那么多。有一个河北省遵化县王国藩合作社的资料就差不多了。另外冀中有个穷棒子社,中农跑了,只剩下三户贫农不散,他们还是坚持下去。这三户指出了五亿农民的方向,每个省都有许多合作社增了产,一增产就是一倍,几倍,你还不相信吗?农业四十条一定能实现你还不相信吗?我看是能够实现的。在一九五五年、一九五六年、一九五七年上半年不相信的人相当多,所谓观潮派很多,从中央到各级都有那种人。现在还有×××说的秋后算账派,不去找积极因素,只找消极因素。听几个干部说农村不大妙,三四个人往耳朵内一吹,说合作社不好,眼前一片黑,农民吃不饱,说什么不增产,无余粮等。家里人写信为了要钱就说得很厉害,写得苦一点,说什么粮油布都没有了。不然你就不寄钱。这些你要加以分析,真的粮油布都没有了?柯庆施同志给我讲过,在江苏做过一次统计,一九五五年县、区、乡三级干部中百分之三十闹得最凶,替农民叫“苦”,说统购统销“统”多了,他们是些什么成分呢,这些干部的成分都是富裕中农,或者先是贫农,下中农,后来上升为富裕中农的。所谓喊农民苦,就是富裕中农苦。富裕中农想存粮,不想拿出粮来,想搞资本主义,就大叫农民苦。下边这样叫,地、省、中央一级没有人喊吗?没有人多多少少受家庭、农村的影响吗?问题是你站在那个立场上看问题。是站在工人阶级、贫、下中农立场上看问题呢?还是站在富裕中农立场上看问题。

现在比较好些,农村有了大跃进。经过整风,反右,干部参加劳动,工人参加一部分管理工作,城乡政治空气变了,农业“悲观论”,“没希望”、“四十条不能实现”,可以说一扫而光了。但是还有一部分“观潮派”,“秋后算账派”,这部分人还没有扫光,所以还要注意工作。谭××报告中提到要防止华而不实,浮而不深,粗而不细。这些话是江苏提出的。就是说要看出自己的缺点,十个指头,九个指头是光明的,一个指头是阴暗面。华者花也,不要只开花不结果。粗而不细,张飞粗中有细,我们就当张飞,要粗中有细,不要华而不实。粗而不细,以免秋后达不到指标的要求。各行业各部门同志们都要注意,不论什么工作,工业、农业、商业、文教、写小说…等工作都要注意。

国内形势很好,是一片光明。过去思想不统一,包括多快好省在内,没有信心。多快好省是讲工业、农业、交通各项工作,基本问题是农业,是对四十条的问题。现在信心高了。是由于农业生产大跃进.农业跃进压迫工业,使工业赶上去,一齐跃进,推动了整个工作。南宁会议上提出了一个问题,五年、七年、究竟几年地方工业的产值赶上或超过农业产值。各省就进行规划.这个纲一提起来,一月不算,二、三、四、只有三个月,省、县、乡的地方工业就蓬蓬勃勃搞起来了。现在许多同志都了解了,一九五六年下半年中央有些同志不大了解.经过一九五六年、一九五七年上半年,四,五、六几个月,现在解决了。去年六月。恩来同志往人代去上的报告很好,以无产阶级战士的姿态向资产阶级宣战,那篇文章很可再看一下。那时问题真正解决了。深刻了解还在以后。

现在中央规定中央负责同志一年下去四个月,解剖几个麻雀,几个工厂,几个合作社。把根扎在人民群众身上,把人民群众的根扎在脑子里面,不然总不深。感谢河南长葛县第一书记的发言。这个发言很好,我又看了一遍。一年把一百二十万亩地全部深翻一遍,深翻一尺五,争取亩产几百斤。这就提出一个新问题,各县是否都能做到。河南长葛县能做到,别的县难道不行吗?一年不行,两年不行,三年行不行,四年五年就可以了吧?五年总可以再翻一次吧?我看五年总可以,他们第二个五年计划把全县所有的地都翻一遍。没有好工具就用长葛县那样的工具,用他们那种办法。他们的办法是。先把熟土翻在一边,然后把肥料施在生土上,再用铁锹把二层生土翻开,与肥料搅拌,打碎坷拉后仍放在下层不动,挨着翻第二行,把第二行熟土翻在第一行生士上,依次翻下去,表层土不变。这是个大发明,深翻一遍增产一倍,至少增产百分之几十。增产的措施,土壤应当放在前边,土、肥、水、种秄,还有密植,要单列一项,要合理密植。广东一亩要搞三万垛,每垛插三根秧,每根秧发三根苗,结二十七万个穗,每穗平均六十粒,共一千六百二十万粒。两万粒一斤,一亩八百斤。亩产八百斤不就算出来了吗?北方的小麦、玉米、谷子,高粱、大豆等都可以这样算一算。密植就是充分利用空气和阳光。现在不是反浪费吗?就应该把空气和阳光的浪费也反掉。阳光每天辛辛苦苦的工作,你们都不利用,空气中的二氧化碳被植物吸收变成碳水化合物,经过光合作用,制造植物需要的东西,碳水化合物等于二氧化碳加阳光。粮食是热能储藏库,每个结构都是个小水库。这扯远了,主要是讲扎根串连,研究几个合作社,几个工厂,串连搞几个连队,教育搞几个学校,商业搞几个商店不要多了,总之各行各业都是要搞几个,抓几只麻雀,然后才有深刻的印象。要尊重唯物辩证法,首先要尊重唯物论。为什么要尊重唯物论?世界现、方法论、认识论、这三个东西是一个东西。人的思想是从哪里来的?生下来就有?还是实践之后才有,人的思想不是天赋的,是后来外界事物反映形成的概念。看见狗,看见人、小孩、树木、马、石头等概念,概念初步形成之后,才可以推理和判断。问三岁小孩子,你妈妈是狗还是人?他能回答是人不是狗,这就是小孩的判断。妈妈是个别的,入是一般的,这里面有同一性。这个个别与普遍的对立的统一,这就是辩证法。所以说三岁小孩就懂得矛盾统一,懂得辩证法。我们的思想只能由客观世界刺激感官而形成,是客观实践所形成。概念是从那里来的?是客观世界来的。现在的多快好省的概念是积累了许多经验才形成的,中国的经验,苏联的经验,根据地的经验,几年建设的经验。鼓足干劲,力争上游这两句话也是非要不可的没有这个不行。一个人,一群人,一个党,没有干劲,干劲不足就不好办事。上游当然要争,力争到四川,不争下游,下游是江西。这是借自然地理来谈问题。要向先进看齐……。

我们的同志要和群众联系,要真正懂得群众的感情,要使群众的感情深入到我们脑筋中来,群众的感情不深入我们的脑子,就容易动摇。深入了,工作上有问题,就有办法对付了。过去我们打仗也常遇到困难,到半夜十二点还无办法,睡一觉第二天办法就出来了。经常有困难的事,不容易的事。孙中山积四十年的经验,我们是积了几十年的经验,深知凡遇到困难的事就和群众商量一下、睡一觉,开个会,就可以解决问题。现在没有问题,没有困难吗?不要为一时的黑暗所吓倒。我们经常有两个因素,一是光明,一是黑暗。现在河北北部就干旱不下雨,你说河北同志不发愁?他们去年搞四十亿斤,今年搞八十亿斤,就是早也要增产五十亿到六十亿斤。国内形势很好,有黑暗不要怕,有两个侧面,光明和黑暗。犯过错误的同志,去年六月就了解了,现在更深刻地了解了。还有很多“观潮派”“秋后算账派”也不怕,多讲些道理,要好好说服他们,摆一摆国际国内形势进行教育。

三、除四害

讲个除四害。除四害好不好?我很感兴趣。《参考消息》说印度人也感兴趣,也想除四害。他们有猴子一害,吃很多粮食,谁也不敢打,说它是神。

我们不提“干部决定一切”、“技术决定一切”的口号,也不提“苏维埃加电气化,就是共产主义。”我们不提这个口号,是否就不电气化?一样的电气化,而且化的更厉害些。前两个口号是斯大林的提法,有片面性。“技术决定一切”一一政治呢?“干部决定一切”一一群众呢,在这里缺乏辩证法。斯大林对辩证法有时懂,有时不懂。这点我在莫斯科会议上讲过。

我们的口号,多些、.快些、好些、省些,我看我们的口号高明一些。应当高明些。因为先生教出学生,学生应当比先生强,青出于兰而胜于兰。后来居上。我看我们的共产主义可能提前到来。他“干部决定一切”,我们要干部么!他“技术决定一切”,我们要技术么!他“苏维埃加电气化”,我们要共产主义么,我看我们的共产主义可以提前到来。苏联的老底子在一九一三年时是四百万吨钢。那是在辛亥革命后两年。十月革命时工人四百万,从一九一七年到一九二○年三年内战不算,从一九二一年算起到一九四○年,一九四一年六月,共计二十年加半年,他们搞到一千八百万吨钢。德苏战争一九四一年六月开始,就拿这点钢打败了希特勒。苏联二十年加半年比老底子增加一千四百万吨钢。我们不要这么多时间,我们有苏联的帮助。有六亿人口,有苏联四十年经验。从他那里学,但是对的我们就学,不对的不学。几千万吨钢我兴趣不大。一九六二年我们三千万吨,一说三千五百万吨,还有一说四千万吨。八年加五年十三年。我们老底子不是四百万吨,只有九十万吨。这些钢主要是日本人搞的。其次是蒋委员长。蒋介石实在不高明,他搞了二十年加满清张之洞的老底子才搞了四万吨,蒋介石不灭亡实在无理。苏联从四百万吨钢,二十年增加了一千四百万吨,我们十三年不是增加一千四百万吨,而是三千万吨。所以说事在人为。六亿人口加苏联经验。几个并举,群众路线,真正的民主集中制。列宁讲党群关系讲的很好,斯大林这方面不会讲。列宁讲不管多大官,要以普通劳动者的身份出现。十三年三千万吨可能超过,数字不着急。总而言之,大大超过。为什么?六亿人口,群众路线,以普通劳动者姿态出现。我们发展了列宁的民主集中制。大家敢想、敢说、敢做。落后阶层也都发动起来了。富裕中农、贫农、工人中一部分落后的人也起来了。

做事要有紧张有休整。常常紧张不好。又紧张又松弛,太紧了也不行。河北、河南大办又红又专的学校,这很好。可是大家太累了。上课时有人打瞌睡,先生也累了。但不敢打瞌睡,硬挺着。太累了不行,总要有几天休息。我们要有张有弛。“张而不弛文武不能也,弛而不张文武不为也”。文王武王人家是圣人啊!尚且不能,我们能行吗?

有紧张,有松弛,有团结有斗争。只有团结没有斗争不行。斗争是为了团结,大中小结合,有张有弛,有民主有集中,那个地方都是一样的。对观潮派、秋后算账派要斗争。但目的是为了团结,不是不叫革命。阿Q最伤心的事是不准他革命。不帮助人家改过,一味批评不好。一斗二帮,要有好心,没有好心,居心不良可不好,无非是打倒你我来。多一个人好。少一个人好,人多一点好,要调动一切积极因素。

辩证法应该在中国得到发展,别的地方我们不管,中国由我们管。我们这一套比较合乎辩证法。比较合乎列宁。不太合乎斯大林。斯大林说。社会主义社会生产关系完全合乎生产力的发展,否认矛盾,他死前写了一篇文章否定了自己.说完全适合不是没有矛盾.处理不好也可能发展成为对抗性的矛盾,不能说斯大林没有辩证法,有,有几成,有迷信.有片面。但也依他的方法建成了社会主义,打败了敌人,有五千万吨钢,今年可能到五千五百万吨。三个卫星上了天,那是一种方法。我们是不是可以找别的一种方法?都是搞社会主义。都是马列主义。比如经济斗争,我们釆用列宁的,而不采用斯大林的。斯大林在论社会主义经济问题中说,革命后的政策是从上而下的和平政策,斯大林不搞自下而上的阶级斗争。对东欧,北朝鲜和平土改,没有斗地主,没有反右。只是自上而下的对资本家。不斗争。我们有从上而下。但又加了一个从下而上的扎根串连阶级斗争.我们在“五反”中斗争了资产阶级。现在搞建设,我们搞群众运动,从上而下的要一点,如政府的命令指示、规章制度等等。但大量的要群众自己来搞,反对恩赐观点,和平土改,东欧和朝鲜的办法,我们不要恩赐观点和平土改。没有阶级斗争。没斗地主,没斗资本家。路线不对,遗害无穷。

为什么我们比苏联的建设速度要快?四十年他们搞五千万吨钢.我们可能只要十五年就行,从今年超可能再要七年。王××提出一九六二年四千万吨。很有可能六三年达到五千万吨以上。是否如此,请大家想一想。讲十大关系时讲过。可否比苏联快一些?因为我们条件不同,六亿人口。苏联走过的道路。苏联的技术援助,应当比苏联发展得快一些。我们将十月革命的传统、列宁的群众路线加以发挥,依靠群众。农村依靠贫农,不过他没有这句话。

昨天有一位同志说跟着某一个人就不会错。某个人就指着我。这句话要修正一下,又跟又不跟,一个人有对有不对。对就跟,不对就不跟。不要糊里糊涂的跟。我们跟马克思、跟列宁。有些东西跟斯大林,真理在谁手里就跟谁,即使掏大类扫街的。只要他有真理。我们就跟。合作化我们跟贫下中农,多快好省是因为群众中出现了多快好省,工厂、农村、商店学校,军队……找先进的。那个好,真理在那。就跟。不要跟某某人。胡里胡涂跟某个人走很危险,要独立思考。

我们同志对十个指头。往往搞不清。一出事忘了十个指头.劳动人民内部矛盾。劳动人民犯错误总是九个指头与一个指头的问题.我们同志犯错误也是如此。我这不是讲,古大存李世农、×××、陈再励、李峰、吴芝圃同志发言很好。安徽发言为什么不讲李世农,浙江讲沙文汉也讲少了。要献宝。让大家见识见识.为啥不讲.他们这些人不是九个指头与一个指头的问题,沙文汉是十个黑指头,陈再励也是十个指头都黑了...李世农是九个黑指头。只有一个指头干净。现在讲的是在大风大浪中有动摇的人,这些同志是九个指头一个指头的问题。现在又看清楚了。他们与这些人不同。要团结所有这些人。要保护这些干部。要坚决保护各级积极分子,虽然有错误,但他们积极.他们怕大鸣大放。怕下不来台。坚决保护就下台了。他们的错误只是十分之一,在整风中要坚决保护这些干部.青岛会议文件上就讲了保护干部的问题。以前也讲过,劳动人民内部的矛盾一般是九个指头与一个指头的关系。个别例外。资产阶级中间派,中中是五个指头(五个指头是资本主义,五个指头是社会主义)。中左是六个到七个好指头,中右是六个到七个黑指头.资产阶级知识分子脑筋一下于洗不干净。需要几次反复。资产阶级还会反复,大的没有,小的可能……无产阶级也会起风浪,在十二级台风面前,我们有些同志还会动摇的。但有了去年一年的经验,全党经历了一次锻炼。就可以任凭风浪起,稳坐钓鱼船。去年那么大的风,我们的船没有翻。有人说《这是为什么》的社论写早了一点,也不早。再下去有些左派也要烂掉。实际上去年十二月以后还在小学教员中搞出了十几万右派,占全国三十万右派的三分之一,他们还是猖狂进攻。你说章罗划了右派就不能进攻吗?他照样进攻。只要温度适宜,达到三十七度到三十八度,那些东西照样会放出来的。

不要忘记九个指头与一个指头,一九五六年反冒进就是忘了这个问题。不从本质看问题,要从中吸取教训。

四、准备最后灾难

现在讲点黑暗,要准备火灾大难。赤地千里无非是大旱大涝。还要准备打大仗。战争疯子甩原子弹怎么办?甩就甩吧?战争贩子存在一天,就有这个可能。还要准备党搞的不好,要分裂。我们搞的好,不会分裂,搞得不好也会分为二。现在这样搞不会吧?但在某种情况下不能说不会分裂。苏联还不是分裂了吗?我和××说过我们有分歧,对斯大林问题,和平过渡问题都谈过,我们有些事,为大鸣大放你们也不一定赞成,有意见,但这都是九个指头与一个指头的问题。在莫斯科和××××、×××、×××××谈话,我们有×××参加,单独谈,把这些问题都提出来了。和平过渡问题,公开场合不谈,法宝留一点,个别谈都谈了。谈斯大林欠我们的债,我们有一肚子气,气拿出来帝国主义就兴趣。什么气?两笔账,一王明路线。二不许革命。王明路线实际是斯大林路线。抗战时、第二次王明路线也是如此。以后不许我们革命,不准打内战。雅尔塔会议上,罗斯福劝蒋介石、斯大林劝我,说打内战我们民族有毁灭的危险。说的过分。怎么毁灭呢?有那么容易?打原子仗,我们死一半还有三亿人口。在十二日会议上讲,气不多了,什么事我不讲。

战争与和平,和平的可能性大于战争的可能性,现在争取和平的可能性比过去大.社会主义阵营的力量比过去大,和平的可能性此第二次世界大战前大。苏联强大,民族独立运动是我们强大的同盟军,西方国家不稳定,工人阶级不愿打仗,资产阶级一部分人也不愿打仗,美国人也不愿打仗,和平的可能性大于战争的可能,但也有战争的可能性,要准备有疯子。帝国主义为了摆脱经济危机现在打原子战,时间会缩短,不要四年,只三年就可以了。要准备,真正打怎么办?要讲讲这个问题,要打就打,把帝国主义扫光,然后再来建设,从此就不会有世界大战了。既有可能打世界大战,就要准备,不能睡觉。打起来也不要大惊小怪,打起仗来无非就是死人。打仗死人我们见过,人口消灭一半在中国历史上有过好几次,汉武帝时五千万人口,到三国两晋南北朝,只剩下一千多万,一打几十年,连连续续几百年,三国两晋南北朝、宋、齐、梁、陈。唐朝人口开始是两千万。以后到唐明皇时又达到五千万,安禄山反了,分为五代十国,一两百年,一直到宋朝才统一,又剩下千把万。这个道理我和×××讲过,我说现代武器不如中国关云长的大刀厉害,他不信,两次世界大战死人并不多,第一次死一千万,第二次死两千万,我们一死就是四千万。你看那些大刀破坏性多大呀。原子仗现在没经验不知要死多少。最好剩一半。次好剩三分之一。二十几亿人口剩几亿,几个五年计划就发展起来,换来了一个资本主义全部灭亡。取得永久和平,这不是坏事。

假如党分裂,就会乱一阵子。假如有人不顾大局,如有人和×××高岗一样不顾大局,党就要分裂,他就要走到自己的反面,就会出现不平衡,当然最后还可以平衡,不平衡走向反面就平衡,你们要注意一下.中央委员更要注意顾全大局,谁不顾全大局谁就要跌跟头。××××不让××××革命,他不看中国小说,未看过阿Q正传。你们看过阿Q正传没有?这是本好书,没看的要看。高岗不准中央个别同志有个别缺点,不准革命。××××他们把一个指头的缺点说成十个指头,闹分裂,搬起石头打自己的脚。凡不顾全大局闹分裂的有什么好结果。罗章龙、张国焘闹分裂有什么好处?不应闹分裂,搞分裂是不对的,只有一种分裂是可以的。像第二国际时代德国社会民主党投帝国主义战争的票,列宁才和他们决裂。在以前,列宁和他们有斗争,但不决裂。我们要作合法斗争,来争取多数,不要搞分裂,不顾大局。山东的李峰,广东的古大存,冯白驹(比古好些,有进步),……古大存、李世农、沙文汉是闹分裂的问题,广西陈再励也是,冯白驹稍好一点。他们是站在错误的立场上,地主资产阶级的立场上。新疆也有一批干部闹分裂,不是各民族团结起来,而是要分裂出去。西兰也有人在闹。想分裂,不想合作。闹分裂的人都是会失败的。

我们是要调动六亿人民的力量,连右派我们都要做工作。分化他们。你们开了右派分子会议没有?使右派中有十个人有七个人改好,经过十年八年改好了。会站到我们方面来,摘掉右派帽子。再三五年再坏。再给他戴上。


