\section[毛主席在参观中国科学院时和钱学森同志的谈话(一九五八年十月二十七日下午)]{毛主席在参观中国科学院时和钱学森同志的谈话(一九五八年十月二十七日下午)}
\datesubtitle{(一九五八年十月二十七日)}


一九五八年十月二十七日下午,毛主席到中关村参观中国科学院自然科学跃进成果展览会。在参观过程中,毛主席看见了钱学森同志,和钱学森同志谈了话。

……主席看见了钱学森同志,主席说,“我们还是一九五六年在政协见的面。那一年,全国的干劲很大,第二年春天也还有劲,以后就泄气了。接着就是匈牙利事件,又来个反冒进,真是一股邪风。说‘马鞍形’是不错的。”

钱学森同志回答说:“我不懂农业,只是按照太阳能把它折中地计算了一下,至于如何达到这个数字,我也不知道。而且,现在发现那个计算方法也还有错误。”

主席笑着说:“原来你也是冒叫一声!”这句话把大家引得哈哈大笑。

可是主席接着说:“你的看法在主要方面上是对的,现在的灌溉问题基本上解决了。丰产的主要经验,就是深耕、施肥和密植。深耕可以更多地吸收太阳,让根部多吸收一些有机物,才能长得多,长得快。过去是浅耕粗作,广种薄收,现在要求深耕细作,少种多收。这可以省人工,省肥料,省水利。多下来的土地可以绿化。可以休闲,可以搞工厂。”


