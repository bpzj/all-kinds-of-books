\section[同蒙哥马利的谈话(一九六○年五月二十七日)]{同蒙哥马利的谈话(一九六○年五月二十七日)}


蒙哥马利(简称蒙):我来到中国发现西方对中国的看法完全错误,他们以为中国人是受压抑的,很不愉快,饿着肚皮。事实上,大家都很愉快,满面笑容,看起来都吃得很饱。今天我访问了一个人民公社。社长才三十岁,他是一个很聪明、很能干的人。他的公社办得很好。

主席:西方对我们的看法可以说是基本上错误的。我们的粮食还不够。按照平均人口计算,每人每年平均只有四百公斤粮食。

蒙:可是没有人饿着肚皮。

主席:这四百公斤包括口粮、种子、饲料和储备粮。当然比过去有很大的好转。比十年前好,比蒋介石统治时期好,就是比前几年也好。所以西方的观点基本上是错误的。

蒙:可是大家还是有足够吃的。

主席:相对来说是够的。

蒙:孩子们看起来吃得很饱。

主席:这是对的。

蒙:所有的人看起来都很健康。

主席:他们是很高兴的。人们都组织起来了,为建设自己的国家和改善生活而努力。

蒙:我去天津近郊看了你们的士兵。他们的身体都很健康。

主席:我们现在的日子还不能算是富族。还要等十年或者两个十年,那个时候我们每人每年可能有七百五十公斤到一千公斤粮食。

蒙:再过十年就增加了一亿五千万人口。

主席:一亿左右,这不要紧。

蒙:你们粮食的增长可以满足你们人口增长的需要。

主席:粮食增长快于人口增长,而且我们也在控制人口的增长。

蒙:你们每年人口的增长率是不是百分之二?

主席:百分之二左右。我们的死亡率减少了,平均年龄提高了。过去平均寿命只有三十岁。就是死得多死得早,现在的平均寿命已提到五十岁。

蒙:这是因为你们有了各种医药、卫生设备和抗生素等。主席:人民的生活改善了,我们也进行了防疫工作。在蒋介石统治时期,我们生产的钢铁每年很少。今年可能生产钢××××万吨到××××万吨。但是,这还是不够的。你们平均每人每年有半吨钢。要是我们按照六亿五千万人口计算,每人每年只有一点点,还差得远呢。

蒙:我们是一个高度组织起来的工业国家。

主席:你们是一个高度工业化的拍国家。

蒙:而且还在更加工业化。我们国家面积小,但是人口多。

主席:你们人口密度比中国大。

蒙:你是否去过英国?

主席:没有,可是去过香港,所以也可以说是去过英国。我去过二十多次。

蒙:最近的一次是在什么时候?

主席:最后的一次是一九二四年。

蒙:那是三十六年以前。三十年以前我曾经到过上海。当时上海是一个欧洲城市。现在仍然是欧洲的建筑物,但是欧洲人不在了。

主席:英国还有一些侨民,还有一些商业和企业在上海,例如英国还有一个毛线厂在上海。

蒙:那很好。请你给我讲一讲你对今天的世界局势有什么看法?

主席:国际局势很好,没有什么坏,无非是全世界反苏反华。

蒙:这是很坏的。

主席:这是美国制造的,不坏。

蒙:但这是很坏的。

主席:不坏,是好的。他们如果不反对我们,我们就同艾森豪威尔、杜勒斯一样了,所以照理应该反。他们这样做,是有间歇性的。去年一年反华,今年反苏。

蒙:那是美国做的,不是英国。

主席:主要是美国,它也策动在各国的走狗这样做。

蒙:因此,我认为局势是坏的。

主席:现在的局势我看不是热战破裂,也不是和平共处,而是第三,冷战共处。

蒙:困难就在这里。在冷战中相处是困难的。

主席:我们就要解决这个问题。

蒙:我们必须找到一个解决办法。

主席:但是我们要有两个方面的准备。一个是继续冷战,另一个是把冷战转为和平共处。所以你做转化工作,我们欢迎。

蒙:是的。我认为我们不能在这种紧张局势中生活下去。我们的孩子们是在冷战中长大的,这对孩子们是坏的。所以我们必须把这种情况转为和平共处。我不希望看到我的孩子长大以后认为世界必须一直存在着紧张。

主席:还要有分析。冷战有好的一面,也有坏的一面。坏的一面是它有可能转为热战。

蒙:有可能。

主席:好的一面是有可能转为和平共处。

蒙:这不能够称为是冷战的好处。

主席:我们说是好处,因为美国制造紧张局势,就制造更多反对它的人。例如在南朝鲜、日本、土耳其、拉丁美洲,很多国家都反对美国人的控制。这是美国人自己造成的。

蒙:我不能肯定美国在西方国家集团中制造了它的反对者,在西方集团中没有发生这种情况,虽然我希望发生这种情况。

主席:我不是指欧洲,欧洲是比较平静的。我是指南朝鲜、南越、日本、土耳其、古巴、其他拉丁美洲国家、非洲。非洲不能光责备美国,首先是欧洲的殖民主义者,但是,美国要在那里取欧洲殖民主义的地位而代之。因此我说好的一面就在于使这些国家反对美帝国主义。这正在动摇整个资本主义世界的基础。

蒙:我希望看到美国在西方阵营制造了它的反对者。我也谈谈关于动摇资主义世界的基础问题。你们用资本主义世界这个名称。我们用西方世界这个名称,这是一个比较容易说的名称。西方世界的领袖是美国。现在西方国家怕被这个领袖领到战争中去。这是个很奇怪的现象,因为在上两次大战中,美国都等到战争打到一半才参加进来。可是现在西方国家却怕美国把它们带入战争。我们必须把这种情况改变过来。现在的情况是,西方集团的领袖跟东方集团两个最大的国家根本谈不拢。由于这个原因,美国在西方的领导受到怀疑。

主席:只要美国的领导不削弱,而由英国、法国来加强,就不可能改变局势。

蒙:我相信必须产生这样的一种情况。

主席:你是英国人,你到法国跑过,你去过两次苏联,现在你来到了中国。有没有可能,英、法、苏、中在某些重大国际问题上取得一致意见?

蒙:是的,我想是可能的。但是,由于美国的领导,英、法会害怕这样做。

主席:慢慢来。我们希望你们强大一些,希望法国强大一些,希望你们的发言权大一些,那样事情就好办了,使美国、西德、日本有所约束。威胁你们和法国的是美国和西德,还有在远东的日本。威胁我们的也是这三个国家。我们不感到英国对我们是威胁。苏联也不感到英国是个威胁。我们也不认为法国对我们是个威胁。对我们的威胁来自美国和日本。

蒙:我觉得很重要的是,在这个非常复杂的局势中,我们应首先采取那一个步骤。我觉得首先应该从别国领土上撤走一切外国军队。这是需要时间的。

主席:主要是美国的势力,一部分在欧洲,一部分在亚洲,英国在德国只有四个师。

蒙:只有三个。

主席:而美国在国外有一百五十万军队,二百五十个军事基地,包括在西德、英国、土耳其,还有摩洛哥。在东方,美国也在日本、南朝鲜、台湾、菲律宾有军事基地。美国还在南越有军事人员,在泰国和巴基斯坦有空军基地。

蒙:主要的问题是大家应该回到本国去。如果我们能做两件事,我们就有可能和与缓紧张局势,第一,停止对欧洲的军事占领;第二,解决台湾问题。问题只能一个一个来。

主席:但是人民在做。南朝鲜人民、日本人民都在进行示威游行,还有土耳其人民。土耳其刚才发生了政变,这总不能说是共产党搞的吧。

蒙:要同时做一切事情是没有好处的。我是个军人,我了解这一点。你也是个军人,你也应该了解这一点。

主席:你有三十五年军龄,你比我长,我只有二十五年。

蒙:我有五十二年了。

主席:可是我还是共产党军事委员会主席。

蒙:那很好。我读过你关于军事的著作,写得很好。

主席:我不觉得有什么好。我是从你们那儿学来的。你学过克劳塞维茨,我也学过。他说战争是政治的另一种形式的继续。

蒙:我也学过成吉思汗,他强调机动性。

主席:你没有看过两千年以前我国的孙子兵法吧?里面很有些好东西。

蒙:是不是提到了更多的军事原则?

主席:一些很好的原则,一共有十三章。

蒙:我们应当从二千年以前回到现在了。你同意不同意,我回到伦敦以后,在结束欧洲的军事占领和解决台湾这两个问题上动员世界的舆论?你是否同意先从这两个问题开始?

主席:好,我赞成。

蒙:我可以使美国非常为难。

主席:这里也有两条。一条就是你这样做,另一条就是美国人非常自高自大,它是寸土不让的。

蒙:我可以使美国非常为难。

主席:有可能。

蒙:我跟美国人很熟,在美国有很多朋友,他们的看法跟我一样。

主席:我们的政策也是使美国为难。

蒙:在美国我有很多朋友会同意我的意见的。很多强大的报界人士也会同意我的。我过去从来也没有设法使美国为难,我想现在就要使他们为难了。

主席:美国现在很被动,有几百条绞索把美国捆起来,它在国外有二百五十个军事基地。

蒙:我想应该对美国人讲一些不客气的老实话。

主席:美国有一半的军队都捆在基地上。它有三百万军队,有一百五十万在海外,包括在你们的英国和中国的台湾。我们在国外没有一个军事基地,没有一个兵。

蒙:主席同意不同意我跟周恩来谈的关于美国应该遵守的那几条原则?那就是:第一,美国应该承认台湾是中国的一部分;第二,美国应该从台湾撤走;第三,台湾问题应该由中国和蒋介石谈判。

主席:我知道,我也同意。我们不要同美国用战争解决问题,同蒋介石就不同了,但是如果他不用武力,我们也不用武力。

蒙:这点我是同意的。

主席:美国人声明愿意通过和平谈判解决国际问题,而不使用武力或武力威胁。它这个话是否可靠还是个假定,还要等着看。可是蒋介石没有发表这样的声明,他反对同中国共产党谈判,而我们早就表示愿意同蒋介石谈判解决问题。

蒙:你认不认识蒋介石?主席:他是我的好朋友,我怎么不认识?蒋介石就是经过我们的帮助才掌权的。在他没有掌权以前,我们同孙中山打交道。

蒙:毛主席同蒋介石是否在抗日的时候合作过?

主席:抗日合作了八年,后来他又同美国合作来打我们。过去你们英国同日本有一个同盟。对付沙皇俄国。那时候远东是你们的天下。中国主要是你们的势力范围。这种情况是什么时候变的呢?第一次大战时开始变了。第二次大战后,日本你们就管不了啦,美国管了。英国还同美国订了一项君子协定,把中国让给美国。这是克里浦斯夫人到延安时告诉我的。她说,在中国问题上,英国没有发言权了。从此以后,中国人民对英国的仇恨就消除了,中国人民的仇恨转向美国。日本投降以后,在中国的美国军队有几万人。

蒙:可是过去的仇恨是针对英国的。

主席:过去是对英国的,同时也是对着日本。

蒙:我们曾经是最坏的洋鬼子。

主席:过去也有日本,后来就成为日本和美国。

蒙:你们反对美国,是不是因为美国派了马歇尔将军来中国干涉中国内政?

主席:日本就是在美国的帮助下才占了大半个中国。日本没有铁,没有石油,煤也很少。这三样东西,都是美国源源不断给日本送去的。但是,美国却扶植了一个力量,造成了一个珍珠港事件。

蒙:你们今天不怕日本了吧?

主席:还有点怕,因为美国扶植日本的军国主义。

蒙:日本是一个高度组织起来的工业国家。

主席:美国在东方的主要基地是日本。本月十九号日本在国会中强行通过了同美国的军事同盟条约。

蒙:日本对中国有没有什么坏的意图?

主席:我看是有。

蒙:什么样的意图?

主席:当然主要是美国。日美条约上有一条,把中国沿海地区也包括在日本所解释的远东范围之内。我读过艾登的回忆录。他讲到苏彝士问题、埃及问题和伊朗问题,也谈到东南亚条约组织问题。他说美国在组织东南亚条约组织的时候,英国希望印度参加,美国坚决反对。美国说如果英国要印度参加,美国就要蒋介石和日本参加。

蒙:印度是不会参加的。

主席:那个时候,艾登想让印度参加来对付美国。艾登在回忆录中说,他想不通蒋介石怎么能同尼赫鲁相提并论。

蒙:尼赫鲁永远不会参加任何集团。

主席:我不是那么肯定。

蒙:尼赫鲁说,他永远也不会参加任何集团和联盟。

主席:现在是这样,将来怎么样呢?

蒙:我跟尼赫鲁是很熟的。只要尼赫鲁还活着,印度就不会参加任何集团。

主席:只要尼赫鲁还活着,尼赫鲁多大年纪了?

蒙:他七十岁了。他还可以工作十年。

主席:就是等到他八十岁的时候,那么他八十一岁时候怎么办呢?

蒙:尼赫鲁不在以后,就难说了。

主席:英国在印度的资本同美国在印度的资本的比例,正在下降,而且英国现在不能增加对印度的投资,美国却在大量增加对印度的投资和借款。

蒙:讨论尼赫鲁不在以后印度会怎样和阿登纳不在以后德国会怎样,是没有好处的。毛主席不在以后,中国会怎样呢?

主席:资本是一个实际问题。美国的资本在印度大大增加,压倒了英国的资本,这是事实。

蒙:这点我同意。但是印度是一个英联邦国家。

主席:名义上是英联邦国家,实际上是美国的势力范围。

蒙:毛主席不在以后。中国会怎样?

主席:我们的制度不同,我们是靠制度,不靠个人。

蒙:我觉得领袖永远是很重要的。

主席:有一部分责任。我现在和你差不多。

蒙:你还相当年轻。

主席:我不是国家主席,不是内阁总理,不是部长,任何国家职务我都没有。你还是元帅,是上议院议员,我也是人大代表。

蒙:我是议员。但不是被选的。

主席:你是被任命的,我是被选的。我和你有点不同,你不是英国保守党主席,我是中国共产党主席。

蒙:我不是一个政客,我是一个军人。

主席:你是一个军人和政治家。

蒙:政治家和政客之间有很大的不同。我有一个有趣的问题问一下主席:中国大概需要五

十年,一切事情就办得差不多了。人民生活会有大大的改善,房屋问题,教育问题和建设问题都解决了。这大概需要五十年。到那时候,你看中国的前途将会怎样?

主席:你的看法是,那时候我们会侵略,是不是?

蒙:不,至少我希望你们不会。

主席:你怕我们会侵略。

蒙:我觉得当一个国家强大起来以后,它应该很小心,不进行侵略。看看美国就知道了。

主席:对,很对,也可以看一看英国。第一次世界大战以前,世界上最强大的国家就是英帝国。一百八十年前的美国呢,只是英国的殖民地。

蒙:历史的教训是,当一个国家非常强大的时候,就倾向于侵略。

主席:要向外侵略,就会被打回来。到底是华盛顿的北美强大,还是英帝国强大?但是,华盛顿用几支短枪,打了八年,把英帝国赶回去了。

蒙:美国革命是件好事。革命往往是件好事。如果不是美国革命,加拿大就不是今天的加拿大。中国的革命也是好的。所以革命可以是好的。

主席:你很开明!

蒙:我是个军人。

主席:外国是外国人住的地方,别人不能去。没有权利也没有理由硬挤进去。

蒙:我同意。

主席:如果去,就要被赶走,这是历史教训。华盛顿不是一个共产党人吧?你们英国开发了澳大利亚和新西兰,该算是仁政吧。现在澳大利亚有八百万人口,新西兰有二百万人口。

蒙:澳大亚利有一千六百万到一千八百万人口,新西兰是二百万人口。

主席:根据我的地理知识,澳大利亚只有八百万人,也许我了解错了。他们实际上都是英国人。

蒙:是英国人的后代。

主席:一旦他们自己能够制造钢铁、飞机、轮船、坦克,他们就半独立了,成为自治领。

蒙:他们是完全独立的。

主席:他们背着英国同美国订了一个条约,而条约规定不许英国参加。

蒙:他们是完全独立的国家。

主席:他们亲美比亲英还多,或者说一半对一半。

蒙:他们是觉得,如果发生战争,英国离得太远,最大的保护还是来自美国。这是唯一的原因。

主席:美国不买他们的羊毛。

蒙:你们倒买澳大利亚的羊毛。但是你还没有回答我的问题,五十年以后中国的命运怎样?那时中国是世界上最强大的国家了。

主席:那不一定。五十年以后,中国的命运还是九百六十万平方公里。中国没有上帝,有个玉皇大帝。五十年以后,玉皇大帝管的范围还是九百六十万平方公里。如果我们占人家一寸土地,我们就是侵略者。实际上,我们是被侵略者,美国还占着我们的台湾。可是联合国却给我们一个封号,叫我们是侵略者。你在同一个侵略者说话,你知道不知道?在你对面坐着一个侵略者,你怕不怕?

蒙:革命前,你们曾经遭受过我们的侵略。

主席:过去有过,现在那种仇恨没有了,只留下一点尾巴了。你们的政府只要改善一点你们的态度,我们就可以同你们建立正式外交关系,互派大使。

蒙:我希望如此。

主席:你们为什么不稍稍改善一点你们的态度呢?基本问题已解决了,你们同台湾没有正式外交关系,同意北京政府代表中国,基本事情你们已经做了。只剩下个别的问题,就是:(一)在联合国讨论蒋介石的代表权问题的时候,同美国站在一起;(二)在台湾你们还有领事;(三)你们的政府比较亲台湾而对中国疏远,有很多人从台湾到伦敦,你们外交部接待。在西藏问题上你们同美国站在一起。西藏叛乱分子嘉乐顿珠到伦敦,你们外交部的负责人接见。

蒙:这我不知道。西藏是在中国之内的。

主席:你们外交部做的很多事情,你是不知道的。所以我看来,我们不能轻易地把正式代表权给英国,不能同英国正式互换大使。

蒙:这是需要时间和等待的。

主席:你们只要少许改善一下态度,我们的关系就会改善。

蒙:我觉得你提到的关于英、法、苏、中这个问题是很有趣的。我同麦克米伦和戴高乐是很熟的。戴高乐曾要我下个月到巴黎去同他会见。我将把这一点告诉他。戴高乐是一个很好的人。

主席:我们对戴高乐有两方面的感觉。第一,他还不错;第二,他有缺点。

蒙:人人都有缺点。

主席:说他还不错是因为他有勇气同美国闹独立性。他不完全听美国的指挥棒。他不准美国在法国建立空军基地。他的陆军也由他指挥而不由美国指挥。

蒙:海军也是这样。

主席:法国在地中海的舰队原由美国指挥,现在他也把指挥权收回了。这几点我们都很欣赏。另一方面,他的缺点很大。他把他的军队的一半放在阿尔及利亚,进行战争,使他的脚被捆住。

蒙;戴高乐会说,阿尔及利亚是法国的一个省份,而在法律上戴高乐这样说是对的。

主席:阿尔及利亚人可不同意,他们要求独立。

蒙:麻烦就在这里,所以必须解决。但是,法律上阿尔及利亚是法国的一个省份。这个问题必须解决。

主席:阿尔及利亚问题应该解决。阿尔及利亚人告诉我:法国在阿尔及利亚有九十万军队,我觉得没有这么多。大概有五、六十万。每天、每月、每年,法国都在阿尔及利亚消耗大量军费,这对法国很不利。

蒙:这个问题必须解决。

主席:是必须解决。法国人不能打仗,在越南他们打不过胡志明。

蒙:这个问题必须解决。

主席:他们在阿尔及利亚打了六年。开头阿尔及利亚只有三千名游击队,在存已经发展到十万人的军队了。

蒙:这个问题必须解决。戴高乐的地位在很大程度上取决于他能否解决这个问题,如果他解决不了,他可能被迫下台。

主席:也会决定他是否能够同英国和美国一道在欧洲有平等的权利。

蒙:他已经得到了。他曾经坚持这一点。

主席:不完全如此,美国人不干。我们看到麦克米伦到法国访问,戴高乐到伦敦访问时受到隆重接待,我们感到很高兴,我们希望你们两个国家能够合作。

蒙:麦克米伦可能是西方世界最好的政治领袖。

主席:可能,至少他比艾森豪威尔好。

蒙:谁会比他更好呢?我是指在西方世界里。

主席:我们希望英国能够更强大。

蒙:他在西方集团是最聪明、最老实的人了。

主席:人们可以看出,他比较有章法。

蒙:我衡量一个政治领袖的标准是看他是否会为了地位而牺牲他的原则。你同意不同意这样一种标准?如果一个领袖为了取得很高的地位而牺牲他的原则,他就不是一个好人。

主席:我的意见是这样的,一个领袖应该是绝大多数人的代言人。

蒙:但是他不能牺牲他的原则啊!

主席:这就是一个原则。他应该代表人民的愿望。

蒙:他必须带领人民去做最有利的事。

主席:他必须是为了人民的利益。

蒙:但是人民并不经常知道什么对他们最有利,领袖必须带领他们去做对他们有利的事情。

主席:人民是懂事情的。终究还是人民决定问题。正因为克伦威尔代表人民,所以国王才被迫让步。

蒙:克伦威尔只代表少数人。

主席:他是代表资产阶级反对封建主。

蒙:但是他失败了。克伦威尔死掉,并且埋葬以后,过了几年,人家又把他的尸体挖出来,砍掉他的脑袋,并且把他的头在议会大厦屋顶上挂了好几年。

主席:但是在历史上克伦威尔是有威信的。

蒙:如果不是克伦威尔的话,英国就不是今天的英国了。

主席:耶稣是在十字架上被钉死的,但是耶稣有威信。

蒙:那是在他死以后。在他活的时候,他没有很多的跟随者。

主席:华盛顿是代表美国人民的。

蒙:可是他被暗杀了。

主席:印度的甘地也是被暗杀的,但是他代表印度人民的。

蒙:印度的问题是这样的,中国和印度都是很大的国家,两国都进行了革命。但是印度的革命是依靠对甘地的个人迷信。他们没有思想意识,没有统一的目标,没有纪律。中国的革命有强有力的思想意识,统一的思想和目的,有非常严格的纪律,并且有很好的领导人。你们的革命不是依靠对个人的迷信,而是为了改善人民的处境,反对一个非常腐败的帝王制度。

主席:印度取得独立比我们早,但是去年它才有一百七十万吨钢。

蒙:那是另外一个因素,尼赫鲁没有能够把三亿农民团结起来,像你们作到的那样。你们把农民团结起来,使他们有共同的目标,但是,尼赫鲁没有作到。

主席:十年前,在一九四九年,我们从蒋介石手中只继承了四万吨钢,但是在去年我们就生产了××××多万吨钢。

蒙:我不想谈数字的问题。你们的革命是以人民为基础的。你们的人民都有统一的意志和统一的目标,所以你们就进步了,你们作的是对的。

主席:这不是我,我并没有生产什么钢铁,也没有耕地,这是人民组织起来搞的。

蒙:问题是如果革命是以人民为基础的,就必然有统一的意志和目标,而这一点你们是有的。

主席:精神力量产生物质力量。

蒙:当然。

主席:可是,如果英国没有二千二百万吨钢的话,英国就抬不起头了。

蒙:我现在感到兴趣的是道义上的力量。如果人民道义上是团结的,就可以做出伟大的事情来,而这一点你们是有的。但是,印度的三亿农民是不团结的,也没有统一的目标。

主席:这是因为印度没有解决封建剥削的问题。

蒙:是的。尼赫鲁感到很难团结他的人民。

主席:他搞了一个土改法,但是没有实行。

蒙:尼赫鲁已经七十了,他去世以后,印度怎么样呢?

主席:他没有准备好继承人……。

蒙:……权力在总理手里。

主席:不对……。

蒙:×××是否军人?

主席:他打过几十年仗……。

蒙:我为什么没有见到他?

主席:你没有要求见他。

蒙:我从来没有听说过他。

主席:你来以前没有调查清楚,现在我向你解释了。中国这么大的国家,你为什么这样忙,只访问四、五天?

蒙:周总理要我再来中国,作一次较长的访问,多看着,我答应明年十月再来,访问一个月。我现在必须回去,就我和中国领袖所谈的几个问题动员世界舆论。我明年十月再来访问一个月。

主席:欢迎。

蒙:十月份气候如何?

主席:还可以。可是为什么不在九月下旬来,可以参加我们的国庆。

蒙:那我就从九月中到十月中,访问一个月吧!

主席:好。你想到什么地方去,就可以到什么地方去,想同谁谈,就可以同谁谈。

蒙:明年我将经过莫斯科来,而不是经过香港来。在访问时,希望有人陪我,因为我大概在一年内学不好中文。

主席:欢迎你来,我们会有人陪你的。我们先吃饭好不好?等一下再谈。

蒙:我想问一下,你为什么发现我开明而感到奇怪?

主席:我并不感到奇怪,我以前没有见到过你,只听到关于你的事情。你说革命是好事,可是有各种不同的革命。有资产阶级民主革命,例如克伦威尔的革命,华盛顿的革命,法国大革命,孙中山的革命,还有共产党领导的革命,例如中国的革命。你说你也赞成中国的革命,这是出乎我意料之外的。

吃饭时,没有作详细记录,双方谈到如下问题:

(一)蒙请主席谈了一下长征的情况。

(二)主席提到,美国制造紧张局势对我们没有任何害处,使全世界反对美国,而反美的事情总是使我们高兴的。蒙问:那么紧张局势对社会主义阵营有好处吗?主席说,不是我们要紧张,是美国要紧张,我们怕紧张有什么用呢?紧张损害了我们一根毫毛没有?蒙就说不谈这个问题了。

(三)蒙提起尼赫鲁,主席说尼赫鲁对我们不友好,就因为一个叫达赖喇嘛的人是他的好朋友。但是,我们同印度人民是友好的。蒙说,他认为尼赫鲁是不友好的。

(四)蒙问印尼如何?苏加诺这个人好不好?主席说印尼不坏,并说苏加诺不算坏,也不很好。主席说印尼人民是起来了,但权力是在苏加诺他们这批人手中。

(五)从抽烟和不抽烟的人之间要“和平共处”谈起,蒙说,我对毛主席说的我们四个国家要一致这一点非常感兴趣,我回去以后一定会推行这方面的工作。

主席:这是决定世界大局的问题。

蒙:如果我们能够实现这一目标的话,今后就不再会有什么麻烦了。

主席:问题就是美国,它是不同任何人商量办事的。它国内有一亿吨钢,在国外有二百五十个军事基地,一百五十万军队。

蒙:他们利用他们在别国领土上的基地来进行间谍飞行是很坏的。如果他们一定要进行这种飞行,他们应该从他们本国领土上起飞。

主席:在很多重要问题上我们和你们可能是一致的。

蒙:我会推行关于四国这一思想,我将同戴高乐谈,也准备同赫鲁晓夫谈。我认为这个思想是非常好的。周恩来没有提过这一点,这是很好的一点。

主席:我们这四国是一条线,从东到西,从西到东。

(六)蒙提到对中国军队的良好印象,从这里谈起哪些国家的军队能打仗。主席说,在朝鲜我们打过十六国军队。起初,英国在朝鲜有一个旅,后来增加为一个师,这很好,使英国有了发言权。到后来,我们故意不同英国军队作战,集中打美国军队。在英女王加冕那天,我们停止向英国军队开火,英国司令员还为此特别来信向我们表示感谢。

