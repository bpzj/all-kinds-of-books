\section[接见阿尔巴尼亚劳动青年联盟代表团、新闻工作者代表团、工会代表团和档案工作者代表团的谈话(一九六三年五月四日)]{接见阿尔巴尼亚劳动青年联盟代表团、新闻工作者代表团、工会代表团和档案工作者代表团的谈话}
\datesubtitle{(一九六三年五月四日)}


主席:你们到中国有多久了?是一起来的吗?

各代表团团长:新闻代表团四月二十一日到中国,劳动青年联盟代表团和工会代表团都是四月二十六日,档案代表团四月三日。

主席:你们的党是很好的党,你们的国家是很好的国家。我们两党、两国的关系也很好。我们两党、两国,修正主义都攻击我们。对修正主义的攻击要分析:第一是坏,攻击我们当然不好,第二也好;修正主义骂我们,这对我们有好处,修正主义不骂我们就不好了。修正主义不援助我们,经济上不援助我们,撤走专家,这也要分析,第一是不好,撤走专家,不帮助了,当然不好;但也是好事,让我们自己干,自力更生,对吗?要自力更生,要不怕困难。

你们到过××吗?有机会最好到那个国家去看看,不一定就是你们去。……他们建立了重工业,他们自力更生。我们的农业还没有过关,他们过关了,他们的农业比我们好。\marginpar{\footnotesize 39}我们的朋友不少,帝国主义、反动派、修正主义要孤立我们,但孤立不了。

我们是多数,他们没有多少人,多数人民并不赞成帝国主义、反动派和修正主义,各个修正主义领导国家的人民,并不见得很欢迎修正主义。这方面的情况你们知道一些。人民是好的,包括修正主义领导国家的人民,干部中也不都是坏的,不是一块铁板,并不都是修正主义者,是修正主义领导集团坏。

你们在中国还有多久?

各代表团团长:工会代表团还有一个月,新闻代表团和青年联盟代表团还有二十天,档案代表团六月七日走。

主席:你们可以到处走走。前几年我们的情况不好,最近一、二年比较好一些,现在政治、经济情况都比较更好些了。但还有困难,困难不少。社会上和党内还有些问题。困难可以克服,正在克服中,问题在解决。

社会主义国家经常会生长资本主义因素。有些共产党员挂了党员的招牌,实际上是资产阶级分子,这不是多数,但有一部分是如此,搞投机倒把、贪污盗窃、铺张浪费。浪费问题很大,这个问题如能适当解决,就可以搞几十亿美元。美国人不会借款给我们,社会主义国家也不借给我们,我们有个办法,就是向官僚主义者借款。(外宾笑了)

马马基(新闻工作者代表团团长):美国借款给他们——所谓建设社会主义的南斯拉夫。

主席:你们有一个“好”邻居(指南斯拉夫),他们锻炼你们,他们反对你们,南斯拉夫和其他资本主义国家包围着你们。

马马基:但是我们的朋友很多,我们的人民很坚强,因为和中国人民在一起。

主席:即使中国是资本主义国家,你们也不会灭亡。

卢鲍尼亚(劳动青年联盟代表团团长):但我们感到幸运,中国给了我们国际主义的援助,鼓舞了我们。

主席:但是援助很少,有些科学技术关键问题,我们自己也没有解决,要从资本主义国家购买技术,所以不能完全满足你们的要求,我们过意不去。再过五年到十年,我们会好一些。

马马基:修正主义企图破坏我们的五年计划,而你们是帮助我们。

主席:你们原定的五年计划有没有修改?

吉贝罗(工会代表团团长):有的,一般说项目有些增加,大型项目有增加。

马马基:五年计划有些改动,但总的说增加的比减少的多。去年冬季的气候对农业的影响很大,雨下得多了,涝了。

主席:春季生产怎样了?

马马基:直到三月还在下雨。

主席:你们雨下多了,我们有些地方少了。

马马基:如果有可能,我们可以把我们的雨送一些给你们。(全场笑)

卢鲍尼亚:沈阳很久没有下雨,我们劳动青年联盟代表团一到就下雨了。

主席:很好。你们可以到郊外去看看。这里(指上海)春季生产还可以,有麦、油菜、蚕豆和豌豆。

卢鲍尼亚:我们从北京坐火车到上海,看到一路上庄稼长得很好,尤其是从山东到上海这一段,很深刻的印象是农民没有荒废一寸土地。

主席:\marginpar{\footnotesize 40}这是因为我们土地不够,平均每人只有五分之一公顷,长江以南每人只有十五分之一公顷。我们是寸土必争,否则没有饭吃。这里一年要种两次,冬季种麦,夏季种稻。南方有的地方不种麦,种两季稻子,早稻和晚稻。东北只有一季麦子,只有一百二十天无霜期。这些地方(指上海附近)比较好。上海是北纬三十一度,你们在北纬多少度?

卢鲍尼亚:我们和北京一样,纪诺卡斯特城(霍查同志的故乡)和北京在一个北纬度。

主席:北京是北纬四十度。请你们问候霍查同志和谢胡、卡博、巴卢库、阿里雅和凯莱齐等同志。这些同志我都认识。

卢鲍尼亚:一定转达您的问候。我们出国时,同志们也一再要我们向毛主席致最衷心的问候,祝您身体健康。

主席:谢谢。

卢鲍尼亚:您身体健康,这不仅有利于中国人民,而且有利于国际共产主义运动。

主席:帝国主义、反动派、修正主义骂我们,我们要和全世界百分之九十以上的人民一起,反对他们。团结百分之九十以上的人民,难道还孤立吗?印度还是反动派统治,但印度人民对我们很好。

卢鲍尼亚:我们有些同志过去来过中国,这次再来,看到中国发生了巨大的变化,感到无限鼓舞。我一九五七年来过中国,这次再来,北京不认识了。十大工程是世界上绝妙的杰作,只有像中国人民这样有才能的人才建造得出来。也看到上海的工业展览会,使我们十分惊奇,生产出很精密的仪器,还有重工业,这使我们十分鼓舞,欢欣愉快。我们在以霍查同志为首的阿尔巴尼亚劳动党的领导下,正在进行反对帝国主义和反对修正主义的斗争,你们党以自己的英雄气概鼓舞了我们。我们劳动党和霍查同志也一再指示我们要向你们学习。我们劳动青年联盟代表团和中国青年的会面,使我们受到了革命的鼓舞。

主席:不是一国支持一国,而是互相支持。不是一国帮助一国,而是互相帮助。有了你们站起来反对修正主义,全世界人民都高兴,不仅中国人民。

卢鲍尼亚:这是我们的国际主义义务。

主席:都是国际主义义务,你们是国际主义义务,我们也是国际主义义务。国际主义应当如此,应当坚持马列主义,应当互相支持。

你们今天还有什么活动吗?

阿外宾:主席的接见就是最重要的节目。

主席:今天谈得不多,以后还会有机会的。

卢鲍尼亚:主席接见我们,不仅是给我们,而且是给我们阿尔巴尼亚人民、劳动党和青年的荣誉,是我们一生中最难忘的大事。

主席:我要谢谢你们来看我。\marginpar{\footnotesize 41}

