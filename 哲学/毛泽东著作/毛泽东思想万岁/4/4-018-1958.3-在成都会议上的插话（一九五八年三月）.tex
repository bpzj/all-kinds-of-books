\section[在成都会议上的插话(一九五八年三月)]{在成都会议上的插话}
\datesubtitle{(一九五八年三月)}


(×××发言时的插话)

国家、自治区、合作社三者之间的关系要搞好。

说清楚,和汉族要密切,要相信马克思主义,各族互相相信。使蒙汉两族合作。不管什么民族,看真理在谁的方面。马克思是犹太人,斯大林是少数民族。蒋介石是汉族,但很坏,我们要坚决反对。不要一定是本省人执权,不管那里人人,南方,北方,这族,那族,只问那个为共产主义,马克思作书记,你赞成不赞成?他也不是本地人。汉人的头子,要向少数民族干部讲清楚。

汉族开始并非大族,而是由许多民族混合起来。汉人在历史上征服过少数民族,许多地方被赶上山去,应从历史上看中国的民族问题。究竟吃民族主义的饭还是共产主义的饭,吃地方主义的饭还是共产主义的饭,首先要吃共产主义。民族要,地方要,但不要主义。

(×××发言时的插话)

要破除迷信,“人多了不得了,地少了不得了。”多年来认为耕地太少,其实每人二点五亩就够了。宣传人多造成悲观空气,也不对,应看到人多是好事,实际人到七亿五到八亿再控制。现在还是人口少,现在很难要农民节育。少数民族、黑龙江、吉林、江西、陕西、甘肃不节育。其他地方可以试办节育。一要乐观,不要悲观,二要控制。到赶上英国时人只有文化了,就会控制了。

许多事外行比内行高明,唱戏如此,改良戏要靠观众。靠外行。

大烟,国内每年用××万两,云南现有三十万两,烟土不要烧,收起来。技术革命开禁,不一定到七月一日。对整风无害。文化大革命也可开禁。

改良土壤有二法:一为深翻,一为调换。可四至五年轮流深翻一遍。山东若县大山农业社就是如此。

现在中心问题就是地方工业,既是解决机器的问题。地方工业有四大任务;一为农业服务(基本的),一为大工业服务,一为城市人民生活服务,一为出口服务。

一切正义的、有生命的事,开始都是违法的。

化肥厂,南宁会议谈到统一由专区办,现在看每县都可办。

我们有些人有错觉,认为农业品出口容易,换回工业机器不容易。其实相反,死东西容易搞,活东西不容易,农业不容易。应把这种看法改变过来,农产品很贵重。

(××发言时的插话)

要把薯类、洋芋的名誉提高,列入正粮,不要叫什么杂粮。

三年内不要减少自留地和个人养猪。可以说一下。增加合作社的积累,分的少了,应该让农民发展一些付业,增加一些收入。自留地减少大家要多养猪,两头猪死不了。千斤社庆丰收,这不同于婚丧,吃一顿,每人×××,不必泼冷水。

大字报在农村可以推广。有四条好处:一、可及时议论国家大事,二、干部能听话,三、群众便于说话,四、不怕报复。这次会议作出一条决议,发一个指示。农村普遍贴大宇报。中国自有了大字报。

(×××发言时的插话)

北京城墙可以挖,先不全挖,而是挖得稀烂。

打开通天河、白龙河与洮河,借长江济黄,丹江口引汉济黄,引黄济卫,同北京连起来。

定息不能取消。资本家要求取消,我们就不取消。资本家要求取消定息想去掉帽子。资本家自动不领定息是可以的,但不摘帽子,也不宣传。资本家劳动可以。

(毛主席插话)

为什么不做政治工作?各部可否设立政治委员?设政治委员是设立对立面。逼部长进步。管业和管人是两面。

规章制度,各方面都布置些问题,工厂报表要大减少。由几人小组负责整理一下。下次会议提出汇报。并且提出一个革命的办法。实现规章制度革命。各地来个专题鸣放。

(×××发言时的插话)

无产阶级之风压倒资产阶级之风,正风压倒邪风。

现在有些虚,不是(实)计划,要措施,不要措施。工人没有信心。许多事要有具体措施,才有保证。计划要和措施结合,否则计划会落空。地方工会、产业工会应下放由省、市管。

没有办法时,睡一觉起来开会就有办法。

对六十条你们要提出意见,取消什么?增加什么?

“酒、色、财、气”,酒是粮食,色是生育,财是财金,气是干劲。一样不能少。

(毛主席插话)

工业方面,全国平衡,超产部分,地方与中央分成,由地方调动。地方协作也可以平衡。六十条。加一条协作关系。

(×××发言时的插话)

八年中只有两个半年,大家很值得注意,肇源县去年百分之六十的亩产达到四百斤,东北、华北、西北地区为什么不能?

乌克兰称为苏联的谷仓,为什么东三省不能称为中国的谷仓?

成都灭鼠经验,不搞就不搞,要搞就两礼拜消灭。

各省的第一书记和参加会议的部长同志。大家要读一读威廉斯著的土壤学。从那里面可以清楚为什么会增长。土壤学是农业的基础科学.好象医生的解剖学。日本农业并不高明,

我们苦战三年就可以赶上去。不要请他们来插手。要请社会主义国家的。我们对帝国主义国家决不开门。日本现在跃跃欲试。

协作会议应多开,一月一次或两月一次,不超过三个月。每次两天就够了。

“农业机械化(包括拖拉机)靠地方制造为主。还是靠中央为主,恐怕要靠地方,地方自办为主,国家支援为辅。头两年所需油料,钢材和高级技术人员.也可以地方为主,中央帮助。以后自己解决。

拖拉机社有或大部分社有。

“苦战三年,基本改变本省面貌。在七年内实现农业四十条。实现农业机械化。争取五年完成。”各省可不可以这样提?特别是农业机械化问题,各省可以议一下。

对工业化不要看得太神秘了。看农业机械化看得太神秘了,但忘记了一条,有××,许多事情就好办了,有葫芦,照样子一画就行了。机械化了,合作化就可以最后巩固起来。我国农业机械化,可以很快实现。

小社势必会合并一些。合并后仍然不能搞的(指机械化)可以联社搞。

(毛主席插话)

中国实现社会主义不要一百年。可以五十年。个别行业可以试办一些办法和经验。可不可以先由一个省先进入共产主义?

整个中国农业机械化,要打破陈旧观念.可以试办,可以缩短时间。外行解决问题来得快。还得内行跟着外行跑。恐怕是个原则。今年修水利,不是谭××等同志,靠些内行一百年也修不出来。

学习苏联和一切国家先进经验,是我们任何时候都要做的一件事。更要坚持。同时又要独创,独立思考,但要防止不学外国,防止两极化。

农业机械化的所有制如何?现在苏联已改变。过去苏联是耕者无其机。是否以社有或大社所有。合作社买不起的。恐怕也要贷点款。

(×××发言时的插话)

省的工作应该从三分之二的人口出发。作到粮食自给。

工业和农业同时并举。是在一九五六年四月“十大关系”中提出的。在以前我们也没有认识这个问题。辽宁工业为主,八年吃了这个亏。一开始就提出并举,可不可以?这个问题也可研究。提出并举的时间也许迟了一点,但是宣传上有很大偏差,一直是讲工业化。没有把农业放在恰当的位置上。总路线宣传提纲中强调了工业化,未强调农业。对农业机械化,过去也讲得很远,现在看有两个到三个五年计划就可以实现。过去有忽视农业的思想,认为农业落后似乎是应该的。

中国的社会主义建设路线,是在八年内逐步形成起来的。时间不算很长。中国革命路线是经过多少年才形成的。一九三五年遵义会议未完全形成,一九四一年到一九四五年才完成。建党、北伐、内战时期未形成中国自己的政治路线。那时有“左”倾又有右倾。即陈独秀右倾路线,三次“左”倾路线,抗日时期王明路线,这就没有可能形成。从一九:一年到一九四二年共二十一年。到‘七大’时才形成了一条完整的政治路线。社会主义建设的路线,八年不算长。还不能算形成。再有五年就差不多了。苦战三年也可能形成。过去革命中损失很大,八年建设中也受了一些损失,但损失不大。同时这么时期也顾不上,抽不出手来抓建设。如去年春季到夏季右派进攻,一九五○年到一九五三年抗美大部分力量在朝鲜,一九五五年合作化高潮,也难得抓建设。对事物的认识,对客观规律的认识,是在实践中才能认识清楚。现在切实抓一下,苦战三年,建设路线就可以形成。没有陈独秀主义、王明路线,就没有比较。一九五六年下半年,斯大林问题发生,我们每天开会,一篇文章写了一个月。又发生了波匈事件,注意力又集中到国际方面。现在才有可能抽出时间来研究建设。开始摸工业.现在要苦战三年,邢成一条中国社会主义建设的路线。

电气化这个名词不好,叫电力化好。

(谈到三年实现亩产四百斤时)不要吹得太大,还是五年计划争取三年完成。这么快法有点发愁。可以活动一点,再看一看。

解决相互关系要分析一下。一种是剥削者与被剥削者的关系,右派、中间派与工人是剥削与被剥削关系,另一种是劳动者内部关系。党政工团和工人农民的关系,不同于一般劳动者之间的关系,因为有“五气”,不是平等关系,不是普通劳动者的关系,是官与民的关系。在这一点说来,同国民党一样。“五气”是资产阶级给我们的,我们从旧社会来,当然有。单是所有制改变,工人、农民不感到与我们是平等的,不批评我们。整风反右解决了这两个方面,反了右派,也批评了干部,干部整改,解决了领导与被领导的相互关系问题。使工人党得真解放了。工人生产情绪大增。过去是为官做工,“计件打冲锋,计时磨洋工”,和国民党一样,为五大件奋斗。一日所有制,一日相互关系,一日分配,这是经济学。所有制和分配改变了。相互关系未改。工人觉悟都大大提高。说“(不清楚)”“八路又回未了”。要抓住这件事,凡是做得不彻底的要继续搞。

(吴德发言时的插话)

(讲到要克服动口不动手的官气。安于现状的暮气。怨天怨地的怨气和制度,执行计划是春天必大,秋天必小时)计划不合实际,很值得注意。去年粮食三干七百亿斤,今年四千七百亿斤。靠住靠不住。有暮气,值得注意。

(×××发言时的插话)

山上到处搞梯田,搞鱼鳞坑。

是否民主革命较早的老区,对社会主义革命不那么积极?两年前河北,西北都有些情况,十年前陕北即此情况。过去曾发生老土改区社会主义的劲差一些。原因是新区土改后接着搞合作社,群众没有习惯于“新民主主义秩序”——实际是资本主义民主秩序,发展资本主义。不断革命就是从这里来的。去年以来有变化,是好现象。

全国有三个一千七百万(指人口)即陕西、江西,广西。对那些(搞指标过高的)也需要压缩空气。

应普遍提高人工翻地。一年翻一部分。三、五年翻完,可保持三年到五年丰收,这是改良土壤的基本建设。《人民日报》应读把土壤学宣传一下。

农具展览(包括人力的,不只是耕作的,而且要有加工的,运输的)在今年四月间搞起来。

苏联技术出口.我们依样画葫芦。并不那么神秘,工业化,机械化不要那么迷信。也不要迷信科学家。科学家的脑筋中总有一部分不科学的。

大家同去找一个大学当教授,发聘书.每月讲一次,一年讲几次,学柯庆施,都要有著作。在座的同志,中央委员,一年作两篇文章,一业务,一政治,专深红透。

中国历来男人是农民,女人是工人。女人是食品制造,纺织……男人造原料,所以男人心粗。

(谈到农村搞工业问题时)比较大的最好是乡政府搞。国、社、私三者合营。国家也不一定投资,赚的钱多少可以分一点红。

中等技术学校都归地方管,学生分三分之一给地方,或三分之二,或二分之一。农学院完全归地方,医学院三分之一归中央。

(毛主席插话)

(谈到三车辆精简机构问题时)这是劳动组织问题。两种形式那种好?这不忙作结论,铁道部也不要说全世界都查了,没有这样的。

(谈到勤工俭学时)资本主义国家就是半工半读,专读书就是最坏的,见书不见物,脱离实际,四体不勤。不一定会自给,有半自给,四分之一自给。方针是不要全读书,一定要又读又劳动。我们民族又穷又白,省下来的钱多办学校,中小学可大办,农业学校也要发展。只有教育发展了,才能赶上英国。靠文盲建设不起社会主义。

苏联有几百万知识分子,我们要上千万的知识分子,美国就怕这一点。

(毛主席插话)

(谈到水利局反对修东渠的问题时)科学家不科学。水利局应登报检讨。种草很重要,要加进去,覆盖面上要有草。

只要提出问题,各地就想办法解决,南宁会议只提出若干地方工业赶上农业总产值,并没有提出办法。可是现在各地解决了。

西北四省、山西、黑龙江、吉林、蒙古及其他少数民族地区,不要提倡节育.本省也要有地区的分别。

麦子的穗太短,如何研究培养一种新的品种,穗很长的,那就很好。

(谈到群众集资问题时)不搞就不搞,一搞就搞这么多,我们这个民族就是这样。

(谈到除四害、扫盲时).大鸣大放,提出问题,几个星期,面貌大变。农民不见得那样保守。




(××发言时的插话)

公私合营全国取消定息,内蒙、新疆、青海等少数民族地区也不取。

(王恩茂发言时的插话)

牧区的改造经验很好。因为在社会主义包围中,他是不安心的,这和西藏不安心是一样的。

中国创造了一条经验,合作化增多。农业,牧业都如此。……

下次会议,要把工业当成中心,大家要摸一下。六、七月开这样一次会,再下一次讨论一下文教,请大家准备。商业也要讨论一下。中央和地方合署办公。

新疆地区分散,加工工厂必须分散办。流动加工厂、轧花、面粉、榨油、化肥。这个办法可以在人广地稀的地方应用。二万五千元可以搞一个流动汽车加工厂。水多的地方可以搞船上流动加工厂。

不要以为天下太平。不太平是正常现象,也不过是个把指头。但不能任其泛滥,不及早注意就会传染,一变二,二变三,发展下去就会天下大乱。

(当谈到民族主义者希望出匈牙利事件时)实质就在这里。三中全会议题之一就是民族主义,过去我们反对大汉族主义,现在就主动了。小平报告中提出这个问题,引起异议。

许多人过去看不清楚,如李世农过去就未看出来。山东八个地委,两个反对省委,两个拥护,四个中间动摇,后来摇过来了。但未彻底搞,问题未解决。现在省委指挥不灵,也是一条经验。

首先是阶级消灭,而后是国家消灭,而后是民族消亡,全世界都是如此。

(×××发言时的插话)

分两种情况。一种有反党集团,广东、广西、安徽、浙江、山东、新疆、青海,八省区都有,要推翻领导自己挂帅。也有另一种情况,像四川那样做,是右派活动。是不是各省都有大同小异,这是阶级斗争的规律。阶级斗争发展到这个阶段,隐藏在党内的资产阶级分子一定会暴露出来,不出来反而是怪事。党内思想动向值得注意。阶级斗争情况如何?可谈一谈。

摸工业、摸农业,摸阶级斗争,就是要找马克思主义。当然按业务来讲,还有文教和商业,文教和商业有相当一部分属于阶级斗争。要逐步研究马列主义,要研究理论,结合解决当前实际来研究马克思主义。

双铧犁不能用,是因为思想不能用,脑子不能用,不是客观不能用。可见思想是统帅。思想动态要当成一个阶级斗争问题。应首先抓。在委员中应经常座谈,平常不谈不好,平常没有意识到就不好了。有的省对思想讲的少,不在意识,山东一开会就发现。

这里是否有两条路线的问题:一条多快好省,一条少慢差费。是否有?明显地有:一为排、大、国,一为蓄、小、群,这不是两条路线吗?把水排走是大禹的路线,从大出发。依靠国家(过去依靠国家修了好多水库),现在是蓄为主,小型为主,群众自办为主。河南的水利就是两条路线的斗争。

农业比工业更难些。盲目性是慢慢克服的。所以盲目性就是对客观的必然性不认识,因而也没有自由。什么叫自由?就是对客观世界的认识,对客观必然性的认识,自由是对必然的了解。自由和必然是对立的。所谓盲目性即对必然没有认识。农民上肥料,知其然不知其所以然。对农业不了解,就不自由。对客观世界的认识是逐步的,不可能一下子就认识。例如治淮排涝是曾希圣发明的。他是曾颜。在他以前山西太行山的和尚张凤林,在高阳县发明了治水的方法,他和一个雇农发明了鱼鳞坑。现在全国推广。他是蓄、小、群,不是排、大、国。当然他并不那么系统。经过我们许多同志一帮忙,就系统化了。把漭河等经验一总结,总结出了葡萄串,满天星。蓄小群为主,当然也要排大国。社会主义建设路线,也是逐步形成,现在不能说已经形成,至少还有五年,苦战三年再加两年,如工农业不大出乱子,路线就差不多了,就可以说形成了。五年加八年,共十三年,付出一部分代价,无疑是浪费一点,群众痛苦,时间延长,苦闷一点,但成绩总是主要的。

十五年赶上英国,二十年赶上美国,那就自由了。学苏联首先在路线上学。斯大林基本上正确,但有错误。他们不工农并举,反对大中小。我们是大中小结合,基础放在小的上,靠地方,靠小的。中央是标准设计,干部、技术。盲目性是慢慢克服的,对客观必然性是逐步认识的。没有克服以前,那就是盲目性,就是自然界的奴隶。对于社会斗争,去年反右以前,我们也是奴隶,因为你对右派这个客观现象不太认识嘛!不认识社会主义国家的阶级斗争,不了解客观世界,就是客观世界的奴隶。

(谈到实现农业发展纲要时)辽宁三年,广东五年,是左派,三年恐怕有困难,可以三年到五年。上面打长一点,让下面超过。

(谈到劳动中工伤事故时)工业也有,花这一点代价赶上英国,也是要付的。各省准备死五百人,一年一万多人,十年十万人,要有准备。

化肥太多破坏土质,还是以自然肥料为主。

河南水利全国第一,达四千八百万亩。

这次会议应对时间取得一致看法,除四害可三年到五年,一年突击,三年推广,第五年扫尾。粮食是五年至十年。绿化三年到五年。这样两本账,有伸缩,好些。

工业发展必然同购买力相符合,否则,像匈牙利工业产品没有出路怎办。工业产品必须和购买力平衡,这是一条原则。

党、政、军、商业机构缩小,技术机构扩大。

(谢富治发言时的插话)

云南明朝以前是少数民族,以后才开始的。

(谈到水利化时)现在算成三年,大修水利。现在搞政治运动,为了多打些粮食。社现在可考虑除了地广人稀的地区外,搞大型社,可议一下。当然不是回去就并,而是五年之内。要逐渐并。

农村房子很不卫生,在十年内应改为砖房,不要茅房,这不发生爱国不爱国的问题。青岛、长春最好,成都就不如重庆,开封不如青岛。应有一个计划,十年内改变。房子样子搞好一点,不要封建主义,应搞些标准设计,采取因地制宜。几年丰收的合作社,可以逐步建筑农民的房子。苦战七年到十年,改变农村房子的基本面貌。拆除城墙,北京应当向天津、上海看齐。

报纸是一个材料部,它反映很快,也很经常给我们提出问题。

过去许多资产阶级分子在办事,反右派后,工厂并未搞坏,反而更好了,这是生产关系的改革。

中央的人没有上课,总有一天要比输的。比输也好,我们下去你们上来,一直下到当老百姓。

(周林发言时的插话)

说人民内部矛盾经典作家没有讲,这个话不对,列宁说:“一个郑重的党,对于自己的错误,向群众公开承认,找出错误的原因,加以克服。”这就是处理人民内部矛盾问题,但未这样来说。工人阶级、共产党内部经常不统一的,参差不齐的。我们这些人那么统一了,三个月不开会就不统一了。因为各人所得的情报,材料和观点不同,就不统一。开会就是为了达到一致,不统一才开会,统一了还开什么会。

整风没有整好的要补课。不然,总有一天要暴露出来的。

军队中废除肉刑——打骂枪毙,不是处理人民内部矛盾吗?三大纪律,八项注意,不是调整矛盾的政策吗?当时推行这些政策不是长期工作之后才行通的吗?

各省市要准备出一点乱子,群众中出了乱子,领导中出了乱子,要有精神准备。不要采取赫鲁晓夫式的答复:没有矛盾。杜勒斯看到很多人对人民内部矛盾的报告有幻想,他读了几遍,他看见抱有幻想很危险,他就以全世界资产阶级总司令的资格说话,指出这个危险。英国报纸有些观念是对的,但他也摸到点气候。去年春季波兰拥护,中国右派拥护那篇文章,而左派摇头。苏联现在敢于说人民内部矛盾,但不说领导与被领导的矛盾。杜勒斯很赏识我那篇文章,他们注意七月一日的社论。他们很注意我们这些东西。资产阶级很注意研究我们这些东西,我们干部为什么不注意研究呢?美帝为什么注意我们的动向,因为它将要灭亡,总想看到我们的弱点,把芦苇当渡船。

请各省、市抓一下工业,抓一个月,没有一个月抓两个礼拜,然后到北京去开会.还要抓思想,抓理论,这是纲。以后口里要触到马列主义,现在是不讲政治经济学,不讲辩证法,也不讲自然科学,只要部门经济学。以后要略带一些理论色彩,报纸的社论,也应略带一些理论色彩,以此为荣。

大社可以办一些加工厂,最后由乡办,或几个乡联合办,县办社助,手工业社办工矿。

(谈到工商户要把股票给商联时)接受了被动,对内好,对外不好。每年只拿一百元,要把资产阶级的政治资本剥夺干净,帽子戴起来,对我们有利。要强迫他们要。他们交了股票,手里无股票,头上无帽子,政治资本就在他们手里。

群策群力。群策,即大鸣大放,大家出主意;群力,即大家动手。这个路线古来就有。

现在不科学的风气要转变一下。

(谈到民主党派誓师问题时)可以搞,交心可以,不要交服,另外要帮助他们动员知识分子参加。

(谈到工厂劳动强度问题,白天工作晚上出大字报出工伤事故时)应通报。着重要技术革命,大字报数量,不要追求。

(谈到少数民族闹事时)应当用解决人民内部矛盾的办法去解决。而不用战争去解决。贵州的事与川西不同,川西有五万枝枪,他先攻我。

去到少数民族地区,要批评过去欺负少数民族不对,解放后我们也有一个指头不对,不经常与群众说这一条,群众就会改变态度。

对四川西部藏族叛乱,八擒八纵,百擒百纵,比孔明的七擒七纵多九十三擒九十三纵,对杜聿明、王跃武,也准备纵。

(柯庆施发言时的插话)

(谈到工业的生产竞赛时)根本解决这个问题,要推翻资本主义、帝国主义,搞个世界政府,地球政府,五年计划分工合作。

(谈到整改工作时)这与去春不同,去春夹杂着敌我矛盾,看不清楚。现在反保守,比反浪费目标鲜明,批评领导,领导觉得越批评越舒服。

知识分子,科学家的态度也在改,这些人要就不改,要就突变。

现在有些过去常写东西的人,现在不写东西,因为他们还在过渡状态中,旧的破了,新的未建立起来。资也破得差不多,无还没有建立起来,有一点也不多,所以难写。过些时候就会写出来。

(谈到整风中工人当中阿飞和“废品大王”态度有很大转变时)过去许多资产阶级思想还统治着我们党,工人、农民,还没有兴起来。现在变了。无兴起来了,无的自由就扩大了,横竖自由××××,一定要把资产阶级思想灭掉。有些人感到不自由,是资产阶级思想还没破尽。资灭多少,无的自由就有多少,有你的自由,就无我的自由。“废品大王”本来是资产阶级思想统治着的,第五天资灭掉,无就兴起来了。

中间派现在是不敢动笔的。只要把无产阶级兴起来,他就有大的自由,才能写出东西。

现在是过渡时期,需要的小说不是大作品,而是写一些及时反映现实的中篇,短篇,像鲁迅的那些作品。鲁迅并没有写什么大作品嘛!现在是兵荒马乱时期,大家忙的很,知识分子还未改造好。大作品是写不出来的,我们也一样,没有创造一件,都是把群众和下级创造出来的东西加以提倡,不接近群众如何能提倡好的东西。创作也是一样。也必须和人民接近。听人民和下级干部的话。

没有民主哪来的集中?过去国际范围内的民主集中是一句假话。

因为集中是建立在民主之上,没有民主就不可能有很好的集中。民主集中制首先是民主,然后是集中,没有真的民主,群众的热情和创造怎么能发挥出来呢?

右派帽子也可以摘掉,全靠自己改造。右派这个对立面转过来将我们的军,也是一种推进工作的力量。

学制,课程要由各省市去研究改变,有了典型,教育部门才能改出来。历来统一的东西,都是由典型到普遍的。

孔子是一个学派,是许多学派中的一个,到汉朝的时候,政府才加以提倡和推广,之一学派得到发展。

在取消定息问题上,我们准备处于被动,总是不松口,这样于我们有利。

对资本家的薪金部分工资高一些,是为了“赎买”,目的是把他的政治资本完全剥夺净尽。必须向工人作解释工作。工人阶级不要和资产阶级比,不可比,比不得,这是两个不可比的阶级。资本家工资太高的也可以不动为好,一动就不好了,就给他们增加了政治资本。他们吃“五个菜”政治上就被动,他们的薪金高,说话声音小。

要和中间派作朋友,也要找几个右派分子交朋友,作工作,现在连我们这些中央委员都怕沾右派的边。那怎么行?怎么了解他们呢?

有些左派,例如邓初民在理论问题上是真左派,在政治问题并非真左派。

(谈到培养理论干部时)现在已经是理论落后于实际。

(陶铸发言时的插话)

总路线就全党来说,是逐步完备的。开始提出工业化是不太完备的。没有这次社会主义革命,要把小学教师中的反革命分子搞出来是不可能的。民主革命时期没有解决这个问题。

这次山东的一个教训是没有划右派,没有搞臭、搞透,是非没有分清楚。鸣放不够,以致现在指挥不灵。广东问题较彻底。

全国青工、青农、青年学生、社会青年,确实需要很好的教育,要在鸣放中让他们知道天有多高,地有多厚。

雪花膏按马克思主义原则可以搞,但是苦战三年,不搞也可。

反对地方主义教育,全国各省市都需要进行。

对地方主义者,实际是右派,是资产阶级在党内的代表。

王明究竟怎么处理?开除不行,拿出讨论也不必要,还是让他住在苏联有利,再拖二十年,赶上英国再说。

右派开除党籍,地方民族主义者不开除党籍咋行?

教育主义是资产阶级性质,比较容易改,右派是资产阶级中的反动派,不容易改,两者不同。

没有南方的布尔什维克到东北、华北、西北建立根据地,先取北方后取南方,革命咋能胜利?现在把南方干部北调,各地干部互相渗透,对工作有好处。

对地方主义不要让步,要派一批外地人去广东,广东干部可调一批到北京来。泥里掺沙,沙里掺泥,改良土壤。天下人写八宝饭,不能单打一万和九万。掺的政策是有利的政策,区乡不在内,可以清一色,县上掺外来干部。现在省、专的负责人,大部分是外地去的,对反地方主义感到理不直,气不壮。应采取列宁的办法:“与其你专政,不如我专政。”

这次实际是一次清党,一千二百万党员中清除二十万、百把万,五、六十万,不算多。这比苏联几次清党人数大,方式好,经过群众,民主。

销售点多设,排队购买的现象是可以消灭的。

总路线、规律,总是经过反复才得出来的,规律就是经常出现的东西。美国的经济状况,二月份增加失业者七十万人,达到五百二十万人。衰退——萧条——危机。苏联二十次大会,对资本主义的估价是有毛病的。

七届二中全会对社会主义问题是讲清楚的,当时没有公开讲,直到一九五三年才讲,原因是抗美援朝,恢复经济,土地改革,但是作的百分之八十是社会主义的,百分之二十是半社会主义的,当时不讲有个策略问题,例如孙行者,糖衣炮弹,这些不好公开讲,邓老根本不管“七届二中全会”,他搞“四大自由”,他说是河南取来的经验,但为什么不从西北坡取经,而从河南取经?

二中全会决不是突如其来的,是在整个民主革命过程中有了思想准备,在民主革命过程中我们就看出社会主义因素,如在江西打土豪分田地,就看出其中有很多社会主义因素,当时红军拿着武器,但是跟老百姓讲,平等,那是社会主义;群众耕田队,那也是社会主义萌芽,当时在陕北就讲那是另一种革命。当时在安寨发现一个安全集中的合作社,我们很感兴趣,并发展了互助组,这些为二中全会作了准备。但是没有唤起更多的人注意。例如邓老仍靠“四大自由”,也不跟中央商量,我说他资产阶级思想根深蒂固,他死不承认,直到七月三十日(一九五五年)他才缴出武器。因此说,邓老有资产阶级思想,但这个人是好的,可以改造的,作为思想问题,经过严肃对待,坚持原则,改得彻底。但是有些同志对此却避开锋芒,表示宽宏大量,无非是怕不好混,不好共事,或者怕失掉选举票。马克思主义者不要隐瞒自己的观点。我对邓老讲过,要改造你的思想,不是撤你的副总理的职,不开除你的中央委员,但对许多错误思想党内要作严肃斗争。在原则问题上,共产党员要有明确的态度,但有的人怕“打击别人,抬高自己”,有相当庸俗的空气。思想阵地你不插旗子,他就插旗子。

革命路线吃过苦,经济建设路线不能迷信苏联,不破除迷信要妨碍正确贯彻执行建设路线。

(王任重发言时的插话)

一九五八年的劲头,开始于三中全会,许多事没有料到的,如一九五六年斯大林问题,匈、波事件和一九五六年反冒进。当时对于社会主义革命以为只是所有制问题,而没有弄清那只是小部分,还有生产关系的其他方面。

右派、反冒进都是对我们有压力,人民内部干群关系中也存在问题,心情并不舒服。经过整风反右派,关系改变了,大家的思想不得到解放,如铁道工人的节约,就可修六千公里的铁路。

技术决定一切,政治思想不要了?干部决定一切,群众不要了?全面的提法就是又红又专。领导和群众相结合,要技术又要政治思想,要干部又要群众,要民主又要集中。


