\section[中央农民运动讲习所开学宣言(一九二六年十月)]{中央农民运动讲习所开学宣言}
\datesubtitle{(一九二六年十月)}


同志们、同胞们:


我们中央农民运动讲习所,继续二次北伐的高潮中,举行开学典礼,这实在有非常的意义,其意义之重大,简直等于两次北伐誓师。

因为国民革命的目的,是打倒外国帝国主义,及铲除国内封建政治势力,所谓封建政治势力,即军阀及地主阶级土豪劣绅,我们晓得军阀及土豪劣绅就是帝国主义的两只手,帝国主义有了这两只手,就可尽量地销售洋货,贱价收买原料,从而剥削中国广大的农民群众,那么广大的农民群众,不能起来,形成一个伟大的斗争力量,则中国革命是不可能的。换言之,不号召广大农民群众起来,实行农村革命,帝国主义在中国的基础,不能铲除。所以,农民问题,成了国民革命的中心问题,成了大家不能忽视的革命问题,农民问题之重大,到现在是没有哪一个有勇气敢来否认。谁反对农民运动,谁就是反革命。要肃清党内与帝国主义军阀妥协的破坏农工政策的右倾分子。为了解决农民问题,要继续两次北伐,扩大农村革命。

中央农民运动讲习所的使命,是要训练一般能领导农村革命的人材出来。对于农民运动有深切的认识,详细的研究,正确解决的方法,更锻炼着有农运的决心。几个月后,都跑到乡间,号召广大农民群众起来实行农村革命,推翻封建势力。中央农民运动讲习所可以说是国民革命大本营。

今天的开学,可以说是我们的誓师,我们从今天起决为农民奋斗牺牲,除了农民运动,没有第二条路走,可是这种重大的运动,一切革命群众都要负起责任,都与之有关系。我们在这奋斗的途中,谨盼望各革命的民众,予以诚恳的指导。我们应该呼:

扩大农村革命,铲除封建势力!继续北伐工作!消灭奉系军阀!拥护总理农工政策!反抗一切破坏农工政策的一切反革命派!

农民解放万岁!

国民革命成功万岁!

世界革命成功万岁!


(翻印者说明:一九二六年秋,北伐战争胜利发展,农民运动日益高涨,急需加强对农民运动的领导。在广州举办的农民运动讲习所因革命形势的发展而结束。学员派回原籍,领导农民运动,掀起了一个农村斗争新高潮。十月,北伐军占领武汉,革命中心已由广州移至武汉,这时毛主席又到武汉筹办了这所《中央农民运动讲习所》,继续培养革命干部。)

