\section[给××,陈×,富春等同志的一封信(一九五八年九月二日)]{给××,陈×,富春等同志的一封信}
\datesubtitle{(一九五八年九月二日)}


这个文件看了两篇(注:指计划说明要点)觉得不大满意,不能动员群众。这次会议各类文件,以农村类文件为最好,每样文件交待清楚,前后次序有逻辑性,文字通顺,一般具有鲜明性和准确性,特别是人民公社一件为最好。其次,如×××同志论教育的文章,虽较长,理论水平较高,逻辑性、准确性、鲜明性,三者都具。其次民兵决议写得很好,使人读得下去,读过后很舒服。其次是商业类文件,也不错,可读。工业类文件算考下等了。有一个较好的,就是那个不长的意见书,读得下去,提纲领。“说明要点”最差,我读了两遍,不大懂,读后脑中无印象。将一些观点凑合起来,聚沙成堆,缺乏逻辑,准确性、鲜明性,都看不见,文字又不通顺,更无高屋建瓴,势如破竹之态。其原因,不大懂辩证逻辑,也不大懂形式逻辑,不大懂文法学,也不大懂修辞学,我疑心作者对工业还不是内行,还不大懂,如果真懂,不至于不能用文字表现出来。所谓不大懂辩证逻辑,就工业来说,就是不大懂工业中的对立统一,内部联系,主要矛盾与次要矛盾的分别。因此,杨思写文,不可能有长江大河,势如破竹之势,讲了一万次了,依然文风不动,灵台如岗之岩,笔下若立冰之冻。那一年,稍稍松劲一点,使读者感觉有些忘,因而免于早上天堂略为延长一年,两年寿命呢!我对作者是很喜欢的,从文件内容看来,他是一个促进派,力争上游、多快好省的坚决拥护者,政治路线是正确的,甚为不足,是在理论与文风。我的意思痛切一道,引起注意(过去我所做的一万次唠叨历史,是当作一阵西北风。)如不同意,可用通信方法,鸣放辩论。我写的是一张大写报,你们也写吧。如果同意,请你们会谈一下。我看你们的心意,把这类事当作芝麻,你们注意西瓜去了。却是写出文件叫人不愿看,你们是下决心不叫人看的,是不是呢?建议:重写一遍、二遍、三遍,以至多遍。写得同人民公社那样好。你们研究一下吧?你们做工业官,有工业就是不用心思,毫无理论研究,以致文件写成这样。建议;你们有空写工业纲要四十条,那样好就动手吧。写一篇好文章出来,为五年接近美国。七年超过英国,这个目标而奋斗吧!

“说明要点”,无年月日,无署名,不知谁人写的,表中有好几张,除作者外,恐怕谁也看不懂。为什么如此呢?
