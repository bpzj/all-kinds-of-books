\section[人民日报要搞理论工作(一九六四年三月二十二日)]{人民日报要搞理论工作}
\datesubtitle{(一九六四年三月二十二日)}


毛主席听了×××汇报学术讨论,学习《毛泽东选集》的宣传情况时,谈了一些意见。汇报时周恩来同志,×××、×××也在。

毛主席说:人民日报要搞理论工作,不能只搞政治。主席在听了×××汇报人民日报组织一些学术讨论之后说,这样做好。他又问,人民日报对教学改革发表过文章没有。文汇报上一篇文章《不可能什么都懂》,可以看看,可以转载。

关于《毛泽东选集》的学习,林总提的还是对的(指“带着问题学,活学活用,学用结合,急用先学,立竿见影”。)对群众不能像对理论工作者那样要求,许多是可以立竿见影的。

恩来同志谈到,四川基层干部学习《愚公移山》《为人民服务》等文章,效果很好。\marginpar{\footnotesize 97}

谈到人民日报上“桌子的哲学”的讨论,主席说:观念是从实践来的。人们从土堆、石堆,或者别的东西中,逐渐有了桌子的想法。做桌子,把想法提高了一步,是一个飞跃。而做出来的桌子同原来想法已经不同,又进了一步,这又是一个飞跃。王若水同志可以写篇东西来补正一下。

×××谈到这个讨论时说:如像有的文章说的,人造的东西都是先有观念,这不对,这就成了唯心主义了。

××说,报纸要把两方面的意见都登出来。

(据×××说:报纸理论版近来登了些讨论,但主席还没有形成已经有改进的印象。因此康生同志提出,决定在学术版挂起“学术研究”的牌子,并准备一周出两次。)

