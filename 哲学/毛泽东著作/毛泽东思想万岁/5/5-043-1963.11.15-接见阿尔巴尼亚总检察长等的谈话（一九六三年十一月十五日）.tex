\section[接见阿尔巴尼亚总检察长等的谈话(一九六三年十一月十五日)]{接见阿尔巴尼亚总检察长等的谈话}
\datesubtitle{(一九六三年十一月十五日)}


主席:很欢迎同志们。你们来了几天了?

阿拉尼特·切拉(以下简称切拉):十天了。

主席:走北路来的,还是走南路来的?

切拉:最近是从朝鲜来的,我们住朝鲜住了一个月,在朝鲜是休假。到朝鲜是从北路走的。

主席:他们让你们过?

切拉:让我们经过了,但对我们冷遇。

主席:冷遇啊!请抽烟。(外宾说,不会抽),朝鲜的同志们很好,他们的工作做得很好。

切拉:我们也是这样认为的。

主席:这几个月在全世界,反对修正主义斗争更加发展了。你们坚决地站稳了立场,并且取得了胜利。你们的国家是被他们包围的。你们对全世界真正的马克思列宁主义者是一个很大的鼓舞。

回去时,问候你们的领导同志们好,问候霍查同志、谢胡同志,还有其他同志。

切拉:一定转达。

主席:请喝点茶。除了问候霍查同志、谢胡同志,还有卡博同志、阿利雅、巴卢库等其他同志,也替我转达问候他们。

切拉:一定转达。

主席:你们今年的收成怎么样?

切拉:我们今年的收成情况是:去年冬天雨下得多了,造成今年夏收不好;但今年春耕春种搞的好,所以今年秋收是好的。可以说今年的年成是个好的年成。

索弗克利·巴巴华西里(以下简称巴巴华西里):今年的气候对我们的秋耕秋种是有利的。

主席:很好。今年我们有点灾,一般说来是增产的。如果没有南边的旱灾和北边的水灾,那今年是个大丰收。去年比前年增产一千万吨。今年有好多社会主义国家的农业不好。

切拉:我们来的时候,经过布达佩斯。听说那里的人民意见很大,有抱怨情绪。他们今年的收成不好,政府向美国买粮食。我们去的时候,还没有告诉人民,现在也许告诉了。

主席:你们是否最近就要回国?

切拉:现在预定二十六日离开中国。

主席:今天是十五日,还要到外边去2

张××:还要到上海、杭州、广州、昆明,从那里离开中国回去。

主席:好,到那些地方去看看。

切拉:我们将会看到很多东西。你们的同志对我们的帮助很大。

主席:交换意见嘛。

切拉:是帮助了我们。

主席:互相交换经验。你们阿尔巴尼亚同志到中同来,中国同志都是很欢迎灼。

切拉:我们具体地看到了。虽然在阿尔巴尼亚早已知道你们会欢迎我们的,到这里我们亲眼看到了。

主席:你们是第一次来中国?

切拉:是第一次。

主席:你们两位都是做政法工作的吗?

切拉:不,我是司法工作者,他(指巴巴华西里)是在党中央当视察员。

主席:没有去东北看看?

切拉:时问有限。我们在朝鲜停了一个月,余下的时间就不多了。现在想利用这些时间到中国南方看看。要到中国各个地方都走一趟,这是一件难事。

主席:我刚才从南方回来。南方的秋收还没有完全结束,现在大概差不多了,广东可能还没有收完。你们这次到不到广东去?

黄火星:要去广州。

主席:对付反革命分子,对付贪污浪费分子,单是用行政的办法,法律的办法是不行的,要依靠群众的力量。检察院、法院和公安部门,同党的工作,同群众的工作配合起来,这样比较好一些。比如讲,铺张浪费、贪污分子,一般说靠行政是整不好的,他们就是怕群众。叫做上下夹攻,他们就无路可走了。

要隔几年就整顿一次。即使不是一年一次,几年就要整一次。比如,一个机关,几十人、几百人的机关,过几年就会发生一些问题。我们国家仍然存在着相当严重的阶级斗争。我们过去十年没有抓这个问题了。从去年起,我们准备用几年的时间,把阶级斗争的问题和其他有关的问题抓一下,不然,就很不好搞。有旧的资产阶级残余存在,又产生新的资产阶级分子,就是做投机生意的,贪污的等等。这些人就是修正主义的社会基础,如果现在不整,再过十几年,中国会出修正主义。当然,他们的人数比较是少数,大概百分之几的样子。

我们主要不靠捉人、杀人,主要靠批评教育。但不是说一个也不捉,一个也不杀。

对罪大恶极的,罪恶很大,人民群众要求把他们捉起来,就非捉起来不可;有破坏行为,如杀人放火,破坏工厂、破坏桥梁等少数分子。就是那些普通的破坏分子,他们反对社会主义,比如讲,放谣言啊等等,不是严重的破坏分子都不捉,依靠群众来监督他们,在劳动中去改造。看来,这个方法可能是一个比较好的方法。我们的经验供你们参考,各国的情况不同,你们根据你们的实际情况办事,相信你们会做得更好。

司法工作是不容易做的。检察院,法院和公安部门都是专政的工具。

切拉:毛泽东同志,我们同你们的同志淡了些问题,他们还把我们带到北京监狱去看了,他们向我们介绍了很多情况。我们感到你们教育人的工作做得很出色。

主席:不是每个地方都做得好的。

切拉:也可能你们的工作还有缺点,但基础是正确的。

主席;就是用教育的方法改造人。

巴巴华西里:对于你们用的这种方法,感到受益不少。

主席:第一条,我们要相信群众;第二条,就是这些反革命分子是劳动力。如果把他们捉起来,杀掉,他们的家庭和生产队就丧失了这些劳动力。第二条,对于他们的子女不好做工作,他们的子女要恨我们。所以,用教育的方法来改造,就可以避免了。我们相信依靠群众是可以把他们教育改造好的,他们又是一些劳动力,可以参加社会生产。这样又可以做好他们的子女和家属的工作,使他们不恨我们。

但是并不是每一个地方的工作都做得好。有那么一些同志性急,喜欢用简单的方法解决问题。动不动就把人抓起来或者要求把他杀掉。我们这些同志是把矛盾上交,从下面交到上面来。把矛盾上交的方法并不是一个奸的方法,上面不好处理,还不如放在群众中间,一面教育,一面让他们劳动好。至于有少数分子,你们不是看了北京监狱吗?那是要抓起来的,但也是采取教育的方法进行改造。

他们看了哪个监狱?

黄火星:北京市的监狱。

主席:那些人有工作做吗?

黄火星:那里有塑料厂、鞋厂、袜厂等等。

主席:他们学了技术,放出去以后好劳动。

切拉:我们认为这样做是很对的。我们国家的劳改营里也有些劳动,但没有你们开展得这样广泛。我们监狱的工作是薄弱的,虽然,在我们监狱中关的人很少,是那些非常危险的分子。尽管这样,对这种人还是采取教育的方法。

主席:对!我们第一要相信人是可以改造过来的,在一定的条件下,在无产阶级专政的条件下,一般说是可以把人改造过来的。只有个别人改造不过来。那也不要紧,刑期满了放回去,有破坏活动就再捉回来。有的放出去一次,他照样破坏;放二次,他再破坏;放三次,他再要破坏。是有这样的人,那我们只好把他长期养下去,把他关在监狱的工厂工作。或者把他们家属也搬来,有些刑满了不愿意回去的就把家属也接来。

张××:刑满了可以把家属搬来,安置就业。

主席:对,就是安置就业。有些人是自己不愿意回去的,因为回到当地名誉不好,他在这里已经有很多熟人了,这样就可以把他的家属也搬来,等于迁居了。这样的也不少。

黄火星:北京市那个监狱,也有就业的,有四百多人。

主席:我们把一个皇帝也改造得差不多了。

切拉:我们听说过,他叫溥仪。

主席:我在这里见过他。他现在有五十几岁了,他现在有职业了,听说还重新结了婚。

切拉:听说他还写了本书,叫《我的前半生》。

主席:现在这本书还没有公开发行。我们觉得他这木书写得不怎么好。他把自己说得太坏了,好像一切责任都是他的。其实,应当说这是一种社会制度下的一种情况。在那样的旧的社会制度下产生这样一个皇帝,那是合乎情理的。不过对这个人,我们也还要看。

谈到这里好不好。现在让我们照相吧!


