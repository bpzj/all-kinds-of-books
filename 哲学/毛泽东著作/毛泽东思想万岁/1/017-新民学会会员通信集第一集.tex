\section{新民学会会员通信集第一集}
\datesubtitle{新民学会致各会发的信}



各会友均鉴:

本会出版物,分“会员通信集”与“会务报告”之二,除会务报告叙述公务状况年出二册外,会员通信集,为会员发抒所见相与杨榷讨论的场所。凡会友与会友间往来信稿,不论新旧长短,凡是可以公开的,均望将原稿和腾正稿寄来本会,以便采登第四期以后的通信集。不登之稿不退还。已登之稿声明要退还的也可以退还。稿寄长沙潮宗街文化书社毛泽东君为荷

\begin{flushright}新民学会启\end{flushright}
\subsection{发刊的意思及条例}
第一集所收多前一、二年旧信,然于学会颇关重要,因多属于团体事业之进行与发展的。留法活动一事此集只能载蔡君给各会友的信,各会友给蔡君的信,其重要者尚望蔡君付来选印,通信集之发刊,所以联聚同人精神,商榷修学,立身,与改造世界诸方法。发刊不定期,大约每两月可有一本。同人个人人格及会务固宜取绝对公开态度,俱不宜标榜,故通信集以会友人得一本为主,此外多印了几十本,以便会外同志之爱看者取去。因学会极穷,不论会友非会友,都要纳一点印刷费,集内凡关讨论问题的信,每集出后,总望各会友对之再有批评及讨论,使通信集成为一个会友的论坛,一集比一集丰富,深刻,进步,就好极了。
\subsection{给陶毅的信}
※“联军”

※“同志的分配”

※“自修大学”

※“女子留俄勤工勤学”
<p><font face="宋体">斯咏先生:

(上略)

我觉得我们要结合一个高尚纯粹勇猛精进的同志团体。我们同志,在准备时代,都要存一个“向外发展”的志。我于这个问题,颇有好些感想。我觉得好多人讲改造,却只是空泛的一个目标。究竟要改造到那一步田地(即终极目的)?用什么方法达到?自己或同志从那一个地方下手?这些问题,有详细研究的却很少。在一个人,或者还有;团体的,共同的,那就少了。个人虽有一种计划,象“我要怎样研究”,“怎样准备”“怎样破坏”,“怎样建设”,然多有陷于错误的。错误之故,因为系成立于一个人的冥想,这样的冥想,一个人虽觉得好,然拿到社会上,多行不通。这是一个弊病。一个人所想的办法,尽管好,然知道的限于一个人,研究准备进行的限于一个人。这种现象,是“人自为战”,是,“浪战”,是“用力而多成功少”,是“最不经济”。要治这种弊,有一个法子,就是“共同的讨论”。共同的讨论有两点:一、讨论共同的目的;二、讨论共同的方法。目的同方法讨论好了,再讨论方法怎样实践,要这样的共同讨论,将来才有共同的研究(此指学问),共同的准备,共同的破坏和共同的建设。要这样才有具体的效果可观。“浪战”,是招致失败的,是最没有效果的。共同讨论,共同进行是“联军”,是“同盟军”,是可以橾战胜攻取的左券的。我们非得力戒浪战不可,我们非得组织联军共同作战不可。

上述之问题,是一个大问题。至今尚有一个问题,也很重大,就是“留学或作事的分配”。我们想要达到一种目的(改造),非讲究适当的方法不可,这方法中间,有一种是人怎样分配。原来在现在这样“才难”的时候,人才最要讲究经济。不然,便重迭了,推积了,废置了。有几位在巴黎同志,发狠的招人到巴黎去。要扯一般人到巴黎去是好事;多扯同志去,不免错了一些。我们同志,应该散于世界各处去考察,天涯海角都要去人,不应该堆积在一处。最好是一个人或几个人担任开辟一个方面。各方面的“阵”,都要打开。各方面都应该去打先锋的人。

我们几十个人,结识的很晚,结识以来,为期又浅(新民学会是七年四月发生的),未能将这些问题,彻底研究(或并未曾研究)。即我,历来狠懵懂狠不成材,也很少研究。这一次出游,观察多方而情形,会晤得一些人,思索得一些事,觉得这几种问题,很有研究的价值。外边各处的人,好多也和我一样未曾研究。一样的睡在鼓里,很是可叹!你是很明达很有运志的人,不知对我所陈述的这一层话,有甚么感想?我料得或者比我先见到了好久了。

以上的话还空,我们可再实际一些讲:

新民学会会友,或旭旦学会会友,应该常时开谈话会,讨论吾济共目的目的,及达到目的之方法,一会友的留学及做事,应该受一种合宜的分配,担当一部分的责任,为有意识的有组织的活动。在目的地方面,宜有一种预计,怎样在彼地别开新局面?怎样可以引来或取得新同志?怎样可以创造自己的新生命?你是如此,魏国劳诸君也是如此,其他在长沙的同志及已出外的同志也应该如此,我自己将来,也很想照办。

以上所写是一些大意,以下再胡乱写些琐碎。

会友张田基君安顿赴南洋,我很赞成他去。在上海的肖子璋君等十余人准备赴法,也很好!彭璜君等数人在上海组织工读互助团,也是一件好事。彭璜君和我,都不想往法。安顿往俄。何叔衡想留法,我劝他不必留法,不如留俄。我一己的计划,一星期外将赴上海。湘事平了,回长沙。想和同志成一“自由研究社”(或经名自修大学),预计一年或二年,必将古今中外学术的大纲,弄个清楚,好作出洋考察的工具(不然,不能考察)。然后组一留俄队,赴俄勤工俭学。至于女子赴俄,并无障碍,逆料俄罗斯的女同志,必会特别欢迎。“女子留俄勤工俭学会”,继“女子留法勤工俭学会”而起,也并不是不可能的事。这庄事(留俄),我正和李大钊君等为商量。所说上海复旦教授汤寿军君(即前商专校长)也有意去。我为这件事,脑子里装满引俞快和希望,所以我特地告诉你!好象你曾说过杨润余君入了我们的学会,近日翻阅旧的大公报,见他的著作,真好!不杨君近日作何重活?有暇可以告诉我吗?今日的女子工读团,稻田新来了四人,该团建前共八人,湖南占六人,其属一韩人一苏人,觉得很有趣味!但将来的成绩怎样?还要看他们的能力和道德力如何,也许终究失败(男子组大概可以说已经失败了)。北京女高师,学生方面很有自动的活泼的精神,教职方面不免黑暗。接李一纯君函,说将在周南教一课,不知已来了否?再谈。

 毛泽东
<blockquote>
九月二日在北京
</blockquote>
\subsection{给周世钊的信}
※国内研究出国围研究的先后问题

※团体事业准备工夫

※自修大学
<p><font face="宋体">谆元吾兄:

接张君文亮的信,惊悉兄的母亲病故,这是人生一个痛苦之关。象吾等长日在外未能略尽奉养之力的人,尤其发生“欲报之德吴罔极”之痛,这一点我和你的境遇,算是一个样的!


早前承你寄我一个长信。很对不住,我没有看完,便失掉了,但你信的大意,已大体明白。我想你现时在家,必正纲缪将来进行的计划,我很希望我的计划和你的计划能够完全一致,因此你我的行动已能够一致。我现在觉得你是一个真能爱我,又真能于我有益的人,倘然你我的计划和行动能够一致,那便是很好的了。

我现在极愿将我的感想和你讨论,随便将他写在下面,有些也许是从前和你谈过来的。

我觉得求学实在没有“必要在什么地方”的理,“出洋”两字,在好些人只是一种“迷”。中国出过洋的总不下几万乃至几十万,好的实在很少,多数呢?仍旧是“糊涂’,仍旧是“莫名其妙”,这便是一个具体的证据。我曾以此问过胡适之和黎邵西两位,他们都以我的意见为然,胡适之并且作过一篇“非留学篇”。

因此,我想暂不出国去,暂时在国内研究各种学问的纲要。我觉得暂时在国内研究,有下列几种好处:

1、看译本比较原本快迅得多,可于较短的时间求得较多的知识。

3、世界文明分东西两流,东方文明在世界文明内,要占个半壁的地位。然东方文明可以说就是中国文明。吾人拟从先研究过吾国古今学说制度的大要,再到西洋留学才有可资比较的东西。

3、吾人如果要在现今的世界稍微尽一点力,当然脱不开“中国”这个地盘。关于这地盘内的情形,拟不可不加以实地的调查及研究。这层工夫,如果留在在出洋回来的时候做,因人事及生活的关系,恐怕有些困难。不如在现在做了,一来无方才所说的困难,二来又可携带些经验到西洋去,考察时可以借资比致。

老实说,现在我于种种主义,种种学说,都还没有得到一个比较明了的概念,想从译本及时贤所作的报章杂志,将中外古今的学说刺取精华,使他们各构成一个明了的概念。有工夫能将所刺取的编成一本书,更好。所以我对于上列三条的第一条,认为更属紧要。

以上是就“个人”的方面和“知”的方面说。以下再就“团体”的方面和“行”的方面说:

我们是脱不了社会的生活的,都是预备将来要稍微有所作为的。那么,我们现在便应该和同志的人合力来做一点准备工夫。我看这一层好些人不大注意,我则以为很是一个问题,不但是随便无意的放任的去准备,实在要有意的有组织的头准备,必如此才算经济,才能于较短的时间(人生百年)发生较大的效果。我想(一)结合同志,(二)在经济的可能的范围内成立为他日所必要的基础事业,我觉得这两样是我们现在十分要注意的。

上述二层(个人的方面和团体的方面),应以第一为主,第二为辅。第一应占时间的大部分;第二占一小部分。总时间定三年(至多),地点长沙。

因此我于你所说的巴黎南洋北京各节,都不赞成,而大大赞成你“在长沙”的那个主张。

我想我们在长沙要创造一种新的生活,可以邀合同志,租一所房子,办一个自修大学(这个名字是×××先生造的)。我们在这个大学里实行共产的生活,关于生活费用取得的方法,约可定为下列几种:

1、教课。(每人每周六小时及至十小时)

2、投稿。(论文稿或新闻稿)

3、编书。(编一种或数种可以卖稿的书)

4、劳动的工作,(此项以不消费为主,如自炊自濯等)

所得收入,完全公共,多得的人补助少得的人,以够消费为止。我想我们两个如果决行,何叔衡和邵泮清或者也会加入。这种组织,也可以叫做“工读互相团”。这组织里最要紧的是要成立一个“学术谈话会”,每周至少要为学术的谈话两头或三次。

以上是说暂不出洋在国内研究的话。但我不是绝对反对留学的人,而且是一个主张大留学政策的人,我觉得我们一些人都要过一回“出洋”的瘾才对。我觉得俄国是世界第一个文明国,我想两三年后,我们要组织一个游俄队。这是后话,暂时尚可不提及他。

出杂志一项,我觉得很不容易。如果自修大学成了,自修有了成绩,可以看情形出一本杂志(此间的人,多以恢复湘江评论为言)其余会务进行,留待面谈,暂不多说,有暇请简幅一信。

 弟泽东

 一九二○年三月十四日

 北京北长街九十九号
\subsection{给罗学瓒的信}<p><font face="宋体">荣熙兄:

兄此信我自接到,先后看了多次。今天再看一次,尤有感动。你的话我没有不以为然的。我已经决定了一种求学的方法,暂时也不必说,只是你的话我一定要行就是。你奋勉的志气很可敬。你现处环境很好,可以从事周将的观察和深湛的思考。听说你已离学校在工厂做工,西洋工厂里的情况,可由此明了,并且可以得到托尔斯泰所谓“由劳动{得来的生活是真快乐”。我现在很想作工,在上海,李声(水鲜)辩君劝我入工厂,我颇心动,我现在颇感觉专门用口用脑的生活是苦极了的生活,我想总要有一个时期专用体力去做工就好。李君声鲜以一师范学生在江南造船厂打铁,居然一两个月后,打铁的工作样样如意。由没有工钱以渐得到每月工钱十二元。他现寓上海法界渔阳里二号,帮助陈仲甫先生等组织机器工会,你可以和他通信。启民在太安里周南女校。惇元在理问街道俗报。湘潭教育腐败已极,旅省诸人组织“湘潭教育促进会”从事促进,尚无大效。一师湘潭学发会亦将有所兴作,兄信尚未转去,稍迟当转去。兄沿途寄稿均登湘南日报,无转阅必要了。此信曾经复了一次,今再附识近来感想于此。

 弟泽东

 九年十一月廿六日

