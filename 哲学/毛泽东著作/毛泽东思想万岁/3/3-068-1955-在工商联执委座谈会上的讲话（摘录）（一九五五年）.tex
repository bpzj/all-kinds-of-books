\section[在工商联执委座谈会上的讲话(摘录)(一九五五年)]{在工商联执委座谈会上的讲话(摘录)}
\datesubtitle{(一九五五年)}


前途如何?趋向如何?你们为什么老是不相信?为什么发生这样的问题?因为你们有一点财产,你们的想法和工农不一样。你们中有一部分人也不这样想。你们应自己掌握自己的命运,看清社会发展规律。现在社会主义的趋势和前途是肯定的,在改造两种私有制的时候(资本主义私有制和小农私有制),许多人总心不定,总要问这个问题。这就是说,不能掌握自己的命运。我们看社会发展,可以把握,可以安定,但还要常常讲。

资本主义工商业改造不是少数人,统一战线要不要,不是随便可以废除的,不是中共中央要不要,而要看统一战线对劳动人民社会主义建设有没有利,有利就要,无利就不要。从中国革命长期的经验来看,有统一战线比没有统一战线好。我们的事业和蒋介石日暮途穷恰恰相反,各项建设都在开始。大家不要十五个吊桶打水,七上八下。我来替你减少吊桶,增加抽水机。我们要使全国大为发展大为富强,我国是大国,但不富强。飞机大炮都不能自己造,谈不上繁荣富强,但按照我们计划办事,可以富强,这是大家的富,大家的强。

……

听说各大城市工商业界有学习小组,令人高兴。要有几个核心人物,是先知先觉。有无可能?过去有些懂得政策方针的人,可以设想能组成这种队伍。

有人计算全国工商界和城市小资产者共约三千万人,这是很大的队伍。过去宣传不够,我看总的这几年工商界是有进步的,这个总的估计是必要的;工商界有坏人,共产党内何尝没有坏人。我上面的估计是有事实根据的(公私合营、买公债……)。否则就没有信心,下文就难办了。讲成绩当然不抹煞缺点。要全面估计。

肃反也如此,一件东西总有这件东西的性质。老账不要算,我们算新账。

共产这件事,我看是好事,你们以后会看得见。经常说就不可怕了,私有制大大妨碍了全国统筹,妨碍我国走富强的道路。不要说了睡不着觉,要说好几年,不可能靠一个早上。我们商量来办事,现在只讲宣传,核心分子做宣传,大家不要感到可怕,要感到可喜有利。

我们现在提出的改造政策,就是马克思列宁提出过的“赎买”政策。大概十五年左右,这个期间,工人阶级为资本家每年生产的利润(一年几个亿)十年就有几十个亿之多,所以实际上是工人阶级给资本家出钱收买了工厂。

我们在安排上做了工作及政治地位的安排,我们对资产阶级采取和地主不同的政策。过去你们说共产党是洪水猛兽,现在你们不愿叫资本家,愿叫工人,这是可以达到的。我国地大物博,现在每年只有二百多万吨钢,实在太不像话。我们要全国努力,工商界也要努力,四、五十年总行吧,我们要争这口气,超过美国。李富春说超过美国用不到一百年,我同意。

领导好些,工作就好些,领导不好,工作就差些。我们要注意领导的规划,要考虑用什么办法使大家更愉快更起劲些。有朋友说有些小企业还漏税,这是学习不好,也是没有教育好。

宣传问题,再讲几句:不要一阵风,像台风一样不好,说马上要共产,这会引起不必要的损失,因大、中、小进展不一。不要引起损失,要有准备,要有秩序有步骤的前进。为此核心分子很重要。我们是有成绩的,柜台上没有断过商品,这是了不起的。我看这是有国际意义的,有世界意义的。中国资本家走在前头,给世界各国资本家做个榜样,这是很光荣的。日本资本家很关心,认为我们搞得好,人和物品都没有损失。古话说:“做好事越多,前途越好”。不要你们是先进,就来一个高潮,不要只看到你们几百人,要看到几千人。农民参加合作社,有犹豫有怀疑。资本家也不例外,不要以为怀仁堂一会,就解决了问题。十五个吊桶是逐渐取消的,否则那来的这许多抽水机。“瓜熟蒂落,水到渠成”,我看这是一句好话。


