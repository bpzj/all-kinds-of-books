\section[历史教训(一九四二年十月十四日)]{历史教训}
\datesubtitle{(一九四二年十月十四日)}


十月四日,斯大林在答复美国新闻记者关于“苏联所有的抵抗力量如何”的询问时称:“我认为苏联抵抗德国匪徒的能力,如果不是较大的话,也不小于法西斯德国或其他任何追求世界统治权的侵略国。”十月五日莫斯科真理报社论:“斯大林格勒的抗战,粉碎了希特勒的巨大计划。说计划原订在迅速攻下斯大林格勒后,即向莫斯科、巴库前进。”这些话到十月八、九两日就证明了。十月八日德军发言人宣布:“不再使用炮兵与工兵强袭斯大林格勒。”原因之一,是如十月九日苏联情报局所宣布的:“苏军业已突破斯大林格勒工业区的包围线,并据守新阵地。”原因之二,是如十月十日路透电所称:“昨日德军被迫转调进击斯大林格勒的部分兵力至西北区,因该处正遭苏联解救军的不断攻击。”所有这些,就是说:红军由城内正面与北部侧面两方夹击,迫使希特勒绝望于该城的进攻,而不得不在事实上一天一天把自己转入防御地位。还在一个多月前,一些人们就在匆匆忙忙地讨论高加索失守后的局势,他们对于苏德两军的力量都是判断错了的。希特勒的“巨大计划”是有的,但这个计划真如真理报所说:被斯大林格勒的抗战所粉碎了。

希特勒的实力和他的野心之间的矛盾,是他失败的重要原因。这个矛盾,表现在他釆取避实击虚政策上面。列宁格勒、莫斯科是被认为应当避开的,他就集中力量向着南线一隅。七月间他曾拚命争夺沃罗湼兹,打不开,又避开它。拚命争夺克列特斯卡雅,又打不开,又避开它。于是攻击点集中到斯大林格勒与高加索北麓了,这是无可避开的了,又是打不开,又是要避开了。但这是最后的避开,就是说要被迫放弃攻势,转入防御地位,希特勒现在就是处在这样情况中,希特勒今天还没有发出停止进攻的一般声明,他也许还想最后挣扎一下,但大势已去,无可挽回了。一切他所避开的地点,却成了红军向他进攻的出发点,目前红军就是从克列特斯卡雅到斯城北角一线向德军进攻的。这样将迫使希特勒最后的放弃他的一切战略进攻。

保卫斯城的直接战斗从八月二十三日德军渡过顿河河曲开始,但河曲战斗对保卫斯城的意义是其极重大的。整个七月份,德军由卡尔科夫打到顿河。这个期间,红军采取战略退却,德军毫无所得。七月三十日,柏林宣布德军在齐姆兰斯卡雅至罗斯多夫二百五十公里战线及顿河下游,渡过顿河。不知是德军原先计划的,还是因渡河胜利冲昏了德将波克头脑,他就从整个野战军中分出二十个师,向高加索方向猛追退却中的红军,八月中旬即已达到库班河流域,但出于德军意外地遇到了红军的顽强抵抗。直到今天(十月十四)德军只得到迈科普一个小油田。高加索北麓赶有一个较大的油田叫格列斯尼的,同盟社还在八月十三日就说:“通至格列斯尼的进路非常平坦,目前所关心的问题,只是德军的进攻能否给予红军以充分的间彻底破坏该处油田。”可是从说这话到今天已整整两个月,希特勒对这个油田还是可望而不可即至于为着想达到巴库一带吃这块最肥的天鹅肉,德军确曾冤枉地……高加索山隘。八月二十三日海通社宣称:“八月二十一日晨十时,德山岳部队在高达5630公尺的伊尔布鲁斯山升起德军战旗。”表示法西斯吸血鬼们的狂喜。可是这面战旗,不知是送给红军了,还是法西斯们自己拖着溜下山去的,总之是一场空欢喜。听说波克现在被希特勒撤职了(十月九日路透电),那么,他的错误在哪里呢?或者第一条就是向高加索分兵吧。红军一方面坚决扼守顿河河曲,一方面却又使波克祸水分一股流向库班河去,因此就减轻了对斯大林格勒的压力。

河曲战斗是从七八月之交打起的,红军以极其英勇的奋战,直打到八月二十三日才放弃吨河东岸克拉赤一带阵地。如果没有这一战,如果德军没有河曲二十三天的阻碍与极大的消耗,则斯城的直接保卫将是困难的。

从八月二十三日德军渡河至九月十五日德军打入斯城工业区,红军又在顿河与斯城间纵深五十至六十公里地带,消磨了德军二十三天,没有这二十三天的消磨,斯城的保卫也是困难的。


从九月十五日德军打入工、业区,至十月九日红军路丁什夫部队突破该区德军阵线,二十四天中,是极端猛烈的巷战期间,红军以城内的巷战与北部的压力粉碎了德军的进攻。

九月三十日,希特勒在柏林体育宫演说中说到红军时,他好这样说:“那是一个不知道慈悲的敌人,他们不是人类而是野兽”,表示他似乎曾经希望过红军向他讲慈悲似的。整个苏德战争已经证明:只要人们不对法西斯讲慈悲,就是说,多一点勇气,法西斯就会失败的,这就是历史的教训。日本法西斯看了希特勒的惨败,将作什么感想呢?还是要到海参威一带碰一碰吗?那里靠得住,又是不讲慈悲的。日本的实力与他的野心之间的矛盾,也是一定要把日本法西斯压得粉碎。

<p align="right">(一九四二年十月十四日《解放日报》)</p>

