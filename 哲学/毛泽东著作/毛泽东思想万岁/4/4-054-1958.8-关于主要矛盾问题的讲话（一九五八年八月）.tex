\section[关于主要矛盾问题的讲话(一九五八年八月)]{关于主要矛盾问题的讲话}
\datesubtitle{(一九五八年八月)}


关于主要矛盾这个问题,提不提,提了有好处没有好处,(康生:和××,柯庆施同志谈过,人们认为过渡时期是资产阶级和无产阶级矛盾为主,但也提出一点疑问,是否影响整改,其次是否引起对八大几句话的争论:“先进的社会制度和落后的生产力的矛盾”。现在两条路线的斗争是主要的,即社会主义与资本主义两条道路的矛盾。)我们进行了两次革命,一次是反帝反封建的民主革命,对于资产阶级和个体经济是不动的,第二次革命是无产阶级性质的社会主义革命,第一次革命中有两条道路,即民主主义与封建主义两条道路,与现在两条道路不同。现在是无产阶级性质的社会主义革命,是资本主义与社会主义两条道路的矛盾。从理论上讲是没有问题了。

社会主义革命已经进行了一段了,从一九五三年全国财经会议宣布总路线以后,到冬季又让中宣部写了个总路线宣传提纲。如果不算今年,到社会主义改造高潮只有三年半时间,算今年只有四年半,给了资产阶级以严重的打击,个体农民问题也解决了。这种情况反映在八大,八大说社会主义改造己基本胜利,大规模的群众性的阶级斗争已经基本结束,能说不对吗?所有制解决了,人家服服贴贴打锣打鼓吗!八大指出在经济制度上也没有完全解决(资本家拿定息),在政治上也没有完全解决(思想斗争)还要继续改造。民主党派的一部分一一右派分子,资产阶级知识分子和部分富裕中农还对社会主义改造不满意,这在八大时并不是完全没有划清,八大并没有放松对他们的思想改造,当时他们服服贴贴,现在他们要造反嘛。

青岛这篇文章(《一九五七年夏季的形势》)即讲清楚了,今后为了策略,还是青岛的讲法好,即城乡都存在两条道路的斗争,阶级斗争没有熄灭等都讲了。这个问题到会的人知道就算了。不要因为“主要矛盾”那两个字,闹得天翻地覆。

×××讲的重庆那个工厂干部被工人斗得过不下去了。新工人有资产阶级思想,我们干部有官僚主义、宗派主义、主观主义,这些都是资产阶级思想,都排到资产阶级账上,这些都是人民内部矛盾。人民内部矛盾包括两种,一种是剥削人民的人,一种是不剥削人民的人。最大量的是中间派,没有中间派就不行了。这个问题最近不在报上搞,过几个月以理论文章的形式写东西。几十年以后没有人剥削人了,总不能再排到资产阶级账上了。但还是有先进与落后的矛盾。明朝朱元璋,南北朝时代宋刘裕也是那样的人。政治势力和意识形态还没有完全解决,党内三个主义也属于意识形态。

大鸣大放是最好的革命形势,革命是要取得经验的,有人要大吵大闹就让他闹,革命这么多年就没有发明这个办法,今年和右派合作发明了这个办法,大鸣大放大字报。大鸣大放这个办法是他们提出,我们接过来的。在延安时有兵的报和轻骑队,当时这个办法没有提倡。百家争鸣。百花齐放是在艺术、学术方面讲的,在政治方面右派提出来大鸣大放,我们接过来这很好嘛。××同志去过看过新乡工厂,从那个工厂的情况和现在的报告看来,用压的方法是不行的。

在过渡时期资产阶级和无产阶级的矛盾是主要矛盾,这一个是肯定对的了,第二个在几个月内不在报上宣传主要矛盾,免得引起新的混乱,惹起麻烦,影响整改,报纸上只宣传两条道路斗争。

所谓人民内部矛盾有几种人,无产阶级、小资产阶级、资产阶级。党内也有几种人。实际上人民内部矛盾,就有阶级矛盾。所谓敌我矛盾是对抗性的阶级矛盾。资产阶级知识分子是人民,但有对抗的一面。现在的主要矛盾,已经不是与地主的矛盾,而是三部分人民的矛盾,这三部分人民之间,内部有一部分暗藏的对抗性的阶级矛盾,如章伯钧等。今年把他们暴露了,我们用剥笋政策,今年是剥不完的。现在主要矛盾不是与地主的矛盾。湖南捉了七千人,没有人民反对,如捉章伯钧就不行。今天敌我矛盾是次要的了。社会主义革命的主要对象是资产阶级、资产阶级知识分子和小资产阶级。资产阶级加上家属有几千万人,小资产阶级是几亿,对这些人主要是改造问题,资产阶级和小资产阶级有大量的中间派,对这些人不能说是对抗性的矛盾。如章伯钧之类是对抗性的。百分之九十是人民内部矛盾,人民内部矛盾包括阶级矛盾(××敌我矛盾包括地主、富农、反革命、坏分子和右派),工农也有矛盾,工农矛盾也算两条道路的矛盾。

右派有多少人呢?最多有十五万左右,不是那么多,不能说主要矛盾,估计还要分化一部分出来,对我们有利,特别是有知识的。过渡时期的基本矛盾是资产阶级和无产阶级两条道路的矛盾。

八大讲先进的社会制度与落后的生产力的矛盾,那是讲生产问题,不是讲人与人的关系问题。人与人的生产关系问题已经解决了,但还没有完全解决(查看八大文件第四页),提出社会主义制度是否适合生产力的发展,我们讲大体适合,斯大林讲完全适合有毛病。将来若干年以后生产力发展了,集体所有制和发展生产会发生矛盾的。现在的生产关系是适合的,为什么适合,合作社是发展生产的嘛。我们这个制度比起印度来,印度第一个五年计划增加三百万吨钢,我们增加了四百万吨,你说我们的制度不好嘛。我们的生产关系基本适合生产力的发展的,但也有缺点。到几十年以后,生产力发展了,价值法则没有用了,货币可以不要了。

八大那句话(先进的社会制度与落后的生产力的矛盾)没有什么害处,不妨害整风、生产、反右派、改进工作。这句话是好话,意思是让我们发展生产,充实我们的物质基础。不是讲人民之间的矛盾,这是和外国比,和我国以前比。(康生:原来写这句话时,当时考虑写不写?反复考虑了,套了列宁的一句话。)这句话有语病的,但没有坏处,实际上没有发生坏作用,这句话不必去改了,将来在适当时机讲一下。当时本来想改,已经印发出去了。


