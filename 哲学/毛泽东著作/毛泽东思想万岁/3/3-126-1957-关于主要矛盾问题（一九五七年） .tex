\section[关于主要矛盾问题(一九五七年) ]{关于主要矛盾问题 }
\datesubtitle{(一九五七年)}


关于主要矛盾这个问题,提不提?提了有好处没有好处?(康生:和××、庆施同志谈过,人们认为过渡时期是资产阶级和无产阶级矛盾为主。但也提出一点疑问,是否影响整改,其次,是否引起对八大几句话的争论:“先进的社会制度和落后的生产力的矛盾”。现在两条路线的斗争的是主要的,即社会主义和资本主义两条道路的矛盾。)我们进行了两次革命,一次是反帝反封建的民主革命,对于资产阶级和个体经济是不动的;第二次革命是无产阶级性质的社会主义革命。第一次革命中有两条道路,即民主主义与封建主义两条道路,与现在两条道路性质不同。现在是无产阶级性质的社会主义革命,是资本主义和社会主义两条道路的矛盾。从理论上讲是没有问题了。社会主义革命已经进行了一段了,从一九五三年全国财经会议宣布总路线以后,到冬季又让中宣部写了个总路线宣传提纲。如果不算今年,到社会主义改造高潮,只有三年半时间,算今年只有四年半,给了资产阶级以严重打击,个体农民问题也解决了。这种情况反映在八大,八人说社会主义改造已基本胜利,大规模的群众性阶级斗争已经基本结束,能说不对吗?所有制解决了,人家服服贴贴打锣打鼓嘛,八人指出在经济制度上也还没有完全解决(资本家拿定息),在政治斗争上也没有完全解决(思想斗争);还要继续改造,民主党派的一部分——右派分子、资产阶级知识分子和部分富裕中农,对社会主义改造不满意,这在八大时并不是完全没有划清。八人并没有放松对他们的思想改造,当时他们服服帖帖,现在他们要造反嘛,青岛这篇文章(《一九五七年夏季的形势》)即讲清楚了。今后为了策略,还是青岛的讲法好,即城乡都存在两条道路的斗争,阶级斗争没有熄灭等都讲了。这个问题到会的人知道就算了。不要因为“主要矛盾”那个字,闹得天翻地覆。

×××讲的重庆那个工厂干部被工人斗得过不下去了,新工人有资产阶级思想,我们干部有官僚主义、宗派主义、主观主义,这些都是资产阶级思想,都挂到资产阶级账上,这些都是人民内部矛盾。人民内部包括两种人,一种是剥削人民的人,一种是不剥削人的人民。最大量的是中间派,没有中间派就不行了。这个问题最近不在报上搞,过几个月以理论文章的形式写东西。几十年以后没有人剥削人了,总不能再挂到资产阶级账上了。但还有先进与落后的矛盾。(××:湖北有个刘介梅就不是资产阶级,他想走资本主义道路。)明朝朱元璋、南北朝时代宋刘裕也是那样的人。政治势力和意识形态还没有完全解决,党内三个主义也属于意识形态。

大鸣大放是最好的革命形式。革命是要取得经验的,有人要大吵大闹就让他闹。革命这么多年就没有发明这个办法,今年和右派合作发明了这个办法:大鸣、大放、大字报。大鸣大放这个办法,是他们提出,我们接过来的。在延安时有兵的报和轻骑队,当时这个办法没提倡。百家争鸣、百花齐放,是在艺术、学术方面讲的。在政治方面右派提出来大鸣大放,我们接过来这很好嘛。××同志过去看过新乡工厂,从那个工厂的情况和现在的报告来看,用后的方法是不行的。在过渡时期资产阶级和无产阶级的是主要矛盾,这一个是肯定对的了。第二个,在几个月内不在报纸上宣传主要矛盾,免得引起新的混乱,惹起麻烦,影响整改。报纸上只宣传两条道路斗争。

所谓人民内部有几种人:无产阶级、小资产阶级、资产阶级,党内也有几种人。实际上人民内部矛盾,就有阶级矛盾。所谓敌我矛盾是对抗性的阶级矛盾。资产阶级知识分子是人民,但有对抗的一面。现在的主要矛盾,已经不是与地主的矛盾。而是三部分人民的矛盾。这三部分人民之间,内部有一部分暗藏的对抗性的阶级矛盾,如章伯钧等。今年把他们暴露了。我们用剥笋政策,今年是剥不完的。现在主要矛盾不是与地主的矛盾。湖南捉了七千人,没有人民反对。如捉章伯钧就不行。今天敌我矛盾是次要的了。社会主义革命的主要对象是资产阶级、资产阶级知识分子和小资产阶级。资产阶级加上家属有几千万人,小资产阶级是几亿,对这些人主要是改造问题。资产阶级、小资产阶级有大量的中间派。对这些人不能说是对抗性的矛盾。如章伯钓之流是对抗性的。百分之九十是人民内部矛盾。人民内部矛盾包括阶级矛盾。(××:敌我矛盾包括地主、富农、反革命、坏分子和右派)工农也有矛盾,工农矛盾也称两条道路的矛盾。右派有多少呢?最多有十五万左右,不是那么多,不能说主要矛盾。估计还要分化一部分出来,对我们有利,特别是有知识的。过渡时期的基本矛盾是资产阶级和无产阶级两条道路的矛盾。

八大讲先进的社会制度与落后的生产力的矛盾,那是讲生产问题,不是讲人与人的关系问题。人与人的生产关系问题已经解决了,但还没有完全解决。提出社会主义制度是否适合生产力的发展,我们讲是大体适合,斯大林讲完全适合有毛病。将来若干年之后生产力发展了,集体所有制和发展生产矛盾的。现在的生产关系是适合的。为什么适合?合作社是发展生产的嘛!我们这个制度比起印度来,印度第一个五年计划增加三百一十万吨钢,我们增加了四百多万吨,你说我们的制度不好吗?我们的生产关系基本适合生产力的发展的,但也有缺点。到几十年以后,生产力发展了,价值法则没有用了。货币可以不要了。

八大那句话(先进的社会制度与落后的生产力的矛盾)没有什么害处,不妨害整风、生产、反右派,改进工作。这句话是好话,意思是让我们发展生产,充实我们的物质基础。不是讲人民间的矛盾,这是和外国比、和我国以前比。(康生:原来写这句话时,当时考虑写不写?反复考虑了列宁的一句话。)这句话有语病的,但没有坏处,实际上没有发生坏作用,这句话不必去改了。将来在适当时机讲一下子。当时本来想改,已印发出去了。</p>

