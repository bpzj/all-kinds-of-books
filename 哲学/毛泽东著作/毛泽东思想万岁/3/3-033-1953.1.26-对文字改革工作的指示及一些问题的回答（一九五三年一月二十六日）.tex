\section[对文字改革工作的指示及一些问题的回答(一九五三年一月二十六日)]{对文字改革工作的指示及一些问题的回答}
\datesubtitle{(一九五三年一月二十六日)}


中国的拼音文字字母和七百简体字材料这两个材料,已经送传毛主席看过。他翻阅了简体字。他认为这种工作不必太急,不要因为人家催就操切办事。事情关系很大,影响到几万万人,是长远的事情。不要受了外界压力,轻易作出,反而使自己陷于被动。可以多征求意见。已经有了方案,可以作为继续研究改进的材料。

关于简体字:毛主席认为现在的简体字不够简,有些字与原字只差一两笔,有些字笔划仍多。笔划形状仍然不顺利,写起来别扭。如“氵”可以用“讠”(电子版制作者注:两点水)代替,有了写两点的时间,就比点三点要快。作简体字可以更多利用草体。草体难记。但是可以把一些草体字规律化,作为基本形体、类似日本的片假名,字的基本形体数目少,规律化,学习书写起来就方便。这样就和现在汉字的秩序不同,汉字的数量必须大大缩简,那就要把两三个汉字合并为一个,即一个字可以代用好几个字。如圆周的“圆”即用“元”。一元、二元、花园也用“元”字。这样才叫简化,可以利用假借,以外利用简单容易的草体,不用笔划仍繁难的草体。汉字的数目减少,字的笔划也大大减少;难于减少的,利用假借;这样就能做到的确是简化了。笔化少,写着就顺利。

他举了个例子:如“氵”字旁,用一笔作“し”。要把字形规律化,就不能按原来通行的。因为原来的简体字并没有什么规律。

关于拼音文字:毛主席认为我们用的拼音字母,笔划太繁,有些比注音字母还难写。

毛主席认为何必搞成方块,方块字千年来害人不浅。字母的笔划要向一面倒,“永字八法”要废除,那样写法不合现代化的要求。现在这些字母有的笔划多,有的写着不顺,笔划方向复杂,纠缠太多,向四面八方发展,所以难写。汉字就因为笔划方向乱,所以产生了草书。草本就是打破方块体式的。

因为原来汉字笔划可恶,所以非打碎不可。主席赞成把注音字母多改一些,也赞成改用两拼,以便代替汉字。比如现在这套字母就不好代替汉字。因为有些字母和音书比汉字还难写,比赛不过,这种拼音字母就要落选。

又如注音字母也不能代替汉字,三个字母写不到一起,与汉字合,显不出来简单。拼音字母如果不比汉字简易,在群众中立足就会有困难,主席对这套字母不满意,无论如何要简单,要利用原有汉字简单笔划和草体,笔势要一边倒,这样才能为人民所接受。“注音字母”谁都没考虑它作文字。要制作拼音文字,就是再费一年时间,也算是很快的了。如果照现在的样子拿出来,恐怕人人会阻碍它。在制作字母方面要花工夫。

总起来说,主席希望文字要能真正作到简单,如果不简单,就不能推行。因为文字主要是要写,不仅要数量少,而且要便于写。例如,“云”不好写,改作“士”也难写,草书“e”稍顺一点。我们的字母一定要比日本字母好,不要比他坏。现在的注音字母和简体字,还不如日本假名,要多征求各方意见,加以改进。

这次郭沬若主任在莫斯科曾经谒见斯大林同志。斯大林很关心我们的文字改革,问到,“你们的文字改革搞得怎么样了?”郭老提到,在方言不统一情况下,汉字仍然有用处。斯大林说,这是我们知识分子的话,对于文盲,汉字是无意义的。郭老说一定要改。

改革文字是费时间的,费十年时间也可以,不必求快,但一定要好。

……

现在无人要求废除汉字,但也不能走到另一极端,替汉字辩护。汉字学起来还是困难。……我们的儿童学字的时间,要比拼音文字国家多两年。我们的小学语文教学就发生问题。比如五年一贯制的语文课本第一册的程度比拼音文字国家的浅得多,但是大家仍嫌生字多,教学起来,学生觉到吃力,各方面都提意见。我们小学生的语文水平和欧洲国家比相差太远。我从大连回来,火车上看到李立三的孩子(三年级学生)看俄文大仲马的小说“侠隐记”,三天可以看完。中国的三年级学生根本不能看这样深的书。算术,外国小学一年级就有文字题,我们到三年级才有。这妨碍教育多大!如果我们的子孙世世代代都比别人多学两年,损失简直无法计算。所以不能不改用拼音文字。如果考虑到打字、印刷、打电报等问题,而也使国家工作效率受到很大的损失。发电报,人家用电动打字机,排字,人家也用机器,在这上面耗费的时间的损失不能计算。至于字典难编,尤其余事。但是鲁莽的方法,认为汉字可以立刻废除,也是不对的。首先要有妥当的方案,其次要有实行的步骤,过去把文字改革想得太容易,现在又反转过来替汉字辩护,都是不对的。毛主席剧烈地攻击方块汉字,认为汉字千年来害人不浅,必须改革。《中国语文》出了半年,但是很少看到充分说明文字必须改革的文章。比如说,就没有一篇比较汉字与拼音文字教科书等的优劣的文章。反对方块字,要拿出真凭实据,汉字的电报,打字和印刷等都可以拿出来和拼音文字作比较。不这样做,有时候连提倡文字改革的人都有点动摇。这主要因为我们改革文字的主要理由说的不清楚,文章的逻辑不强。

……

我们的纲领是文字必须改革,决不能有不用改革或怀疑改革的话,但是如何改革,改革的理由是什么,必须说清。

……

提倡文字改革是对的,缺点在于没有作详细的调查研究和比较,在文章的背后没有事实来支持他。往往是议论多,科学的研究少。比如打字问题要收集最熟练的,相当熟练的,不熟练华文打字员的成绩,拿来和英文、俄文的同样的打字成绩作具体的比较。这样才能说服人。

……

所谓字的笔划要“一边倒”就是笔势要顺。比如“水”字就不好写,方向变化多。“三”字方向虽顺,起大太多,也不算好写,像“水”则必须改造。

……

一边倒,不是每字笔划都是一个方向,如“千”字就可以不改。因为它和阿拉伯数字“4”相似,比“干”好写。主要是不要起笔太多。要写起来流利,并不是说任何由右向左的方向都不要了。笔划的组织也要简单。

毛主席的意思是要大改,这样的工程就要复杂一点,必须与代用字一起考虑。

拼音文字要和汉字配合得上,不能设想某日公布后立即就要实行,要准备它跟汉字掺用一个时期。拼音方案,简体字中有些符号,就用拼音字母代替也好,特别是虚字,如“的”字就可以用字母代,虚字要极简单,每字一笔两笔,不一定必须有通行简体的才简化。例如“的”字本来就没有简体,但用的次数最多,必须简化。

……

有些难写的地方名称也要改,比如盩厔、鄜县、鄠郿等,都可以用拼音字母代替。

<p align="right">(根据首都文字革命造反联络总部供稿)</p>


