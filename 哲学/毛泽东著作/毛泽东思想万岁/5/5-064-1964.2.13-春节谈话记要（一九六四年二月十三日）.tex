\section[春节谈话记要(一九六四年二月十三日)]{春节谈话记要}
\datesubtitle{(一九六四年二月十三日)}


主席:今天是春节,开个座谈会,谈谈国际问题,国内问题……

你们看我们国家会不会倒掉?帝国主义、修正主义联合打到国境了,民主人士怕不怕原子弹?原子弹一摔无非是重新回延安,整个陕甘宁边区有一百五十万人,延安城有三万人。人总要被骂才好公开答复,国民党倒有一个时期聪明,不公开骂,发了一个文件,限制异党办法,限制共产党,你知道吗?

章士钊:不知道。

主席:你们消息不灵。一九四一年一月,国民党发动了皖南事变,我们牺牲了一万七千多人,以后又搞了几次反共高潮,教育了党。蒋介石不是好人,一有机会就是整我们。抗战后,讲谈和平,叫我去重庆谈判,也是各下各的令。就在谈判期间打了一个上党战役,消灭了高树勋三个师。

×××:高已入党,人是会变的。

康生:宣统皇帝来拜年了(在政协)。

主席:宣统皇帝应好好团结,光绪、宣统都是我顶头上司。宣统薪水一百多元太少了,人家是个皇帝。

章士钊:宣统的叔叔载涛生活苦。

主席:载涛这个人是陆军大臣,到过法国留过学,我知道他,但不熟悉。是否通过你帮助他,生活有所改善,食无鱼不出,还是让他改善生活。

当走狗不好当,尼赫鲁太不行了,帝国主义修正主义输空了。修正主义到处碰壁,在罗马尼亚碰壁,在波兰不听,古巴是听一半不听一半,听一半是无可奈何,不出石油不出武器。帝国主义日子也不好过。日本反美,反美不仅是日本共产党,日本人民,还有大资本家。不久前北×制铁所拒绝美国调查。戴高乐反美也是资产阶级要求。与中国建交也是他们主动。中国反美,北京过去有个沈崇,全国反美帝国主义。\marginpar{\footnotesize 91}赫鲁晓夫修正主义骂我们宗派主义、假革命,骂得好。不久以前,苏共中央给中共中央来信提出四点:1、停止公开论战;2、再派专家来;3、中苏边界谈判;4、扩大贸易。边界可以谈,二月二十五日就开始。生意可以做一点,不能太多,苏联的物品笨重、价贵,还要留一手。

康生:质量差。

主席:一笨二贵三差四留一手,不如同法国资产阶级好办,还有一点商业道德。

过去工作中有错误,第一是瞎指挥,第二是高征购,现已改正。现在走到反面,由瞎指挥到不指挥,这就没有干劲了。所以要学解放军,学石油部大庆。大庆油田××多投资,三年时间建成××万吨油田,××万吨炼油厂,投资少,时间短,成效高,文学海赋值得看看。

每一个部都应学石油部,学解放军,搞一套好经验,对敌人是战斗队,对自己是工作队。大学生也要学习解放军。要发扬成绩,树立标兵,多表扬,同时也批评错误。以表扬为主,以批评为辅。在我们事业中有很多好人,有很多好典型需要表扬。

去年河北大灾,南方干旱,本来年成好,下了暴雨损失了二百亿斤粮食,去年总计还是增产一百多亿斤,今年还要搞得更好。现在学解放军,学石油部,学习城市、乡村、工厂、学校、机关的典型,克服工作中的错误,把今年的工作搞得更好些。

今天开个座谈会,谈了国际问题。国内问题是根本,国内搞不好,国外就不好谈了。现在有些国家要与我国建交,如刚果,卢蒙巴的刚果搞起了游击战,并没有什么新式武器,关公的青龙偃月刀,张飞的丈八长矛。

×××:还有黄忠的箭。

主席:无非是关张赵马黄的武器,没有新式武器。我们过去也是没有的。南昌起义两个师丢了,××、陈毅、林彪带残部上了井冈山。我根本不会打仗,一九一八年在北大图书馆,八块大洋一个月,不管衣食住行。章士钊不愿当袁世凯的官,让他当北大校长,他跑到北大办报。黄炎老,你是立宪派的人?

黄炎培:我是革命派,不是立宪派,参加同盟会的。

章士钊:他是革命派的。

主席:陈叙老,你是研究系,章士钊二次革命,一九二五年当总长,现在你们都跟我们一起了,在新中国参加社会主义建设。我们今年的工作想法做得更好一些,不但是中央希望,也是你们的希望。许德珩,你是工业部的?

×××:他的部大有希望。

主席:黄老,你的家如像各党各派,民盟、民进、共青团,你的儿子黄万里写的“贺新郎”词写得好,我欣赏。九三学社中一人诗写得好,也欣赏。孩子十几人不大认识,你是郭子仪。

毛主席:各部门都要学习解放军,搞政治部,加强政治工作。要发扬成绩,树立标兵,多表扬,同时还要批评错误。以表扬为主,批评为辅。我们事业有好多好人好事,需要表扬。

今天想谈谈教育问题。现在工业有了进步,我看教育工作也要改一改,现在还不行。我看教育路线、方针是正确的,但方法不对,要改变。今天有中央同志、党内同志、党外的同志,科学院的同志。现在×××同志谈谈。

×××:现在教育中一个迫切的问题是学制的问题,就是学制太长了。现在七岁上学,小学六年,中学六年,大学有的六年,一般的五年,共十七、八年,\marginpar{\footnotesize 92}到二十四、五岁才大学毕业,然后再劳动一年,见习一年,出来已廿六、七岁了,比苏联多二、三年,苏联中、小学十年,大学四、五年,廿三、四岁进工作岗位。年岁大了,学文的问题还不大,学自然科学的就显得太长了。特别是搞原子能科学的,搞尖端科学的,毕业的年岁就太大了。根据世界各国的经验,学自然科学的到廿四、五岁就可以作出贡献。例如美国苏联搞自然科学,搞原子能有成绩的人,一般都是廿四、五岁,这个年龄脑子最好使,而这个年龄我们的学生还在大学,未进入工作岗位。廿六、七岁才工作,对于发展科学不利。学制特别长,应考虑学制问题。

毛主席:可以缩短一些。

×××:最近××同志有个意见,小学五年,中学四年,十六岁中学毕业。如果小学六年,十七岁中学毕业。问题是设备不行,每年大学只招十二、三万人到十五万人。其他的人十六岁就可以就业。中学毕业后搞二年职业教育,十八岁到工厂、农村就业,就比较接近。或搞二年预科,这样就可以和大学衔接起来,到二十四、五岁就可以工作。总之要搞的短一些。现在中央专门研究学制,建立了小组,由××同志负责。采取这样的意见完成国民教育,一般是十五、六岁就可以毕业了。不过有个问题,是当兵,不够年龄,但可以当预备兵。

毛主席:这不要紧,不够当兵年龄也可以过军事生活,不仅男生,女生也可以当兵,搞红色娘子军。十六、七岁的女孩子可以过半年到一年的军事生活,十七岁也可以当兵。

×××:这样文科学校问题不大,理工科问题大一些。大学搞一到二年的预科,中学毕业后可升到大学预科,或者进修职业学校,受二年教育,到十八岁再到工厂、农村参加生产就此较接近。如考理工科也比较接近,到二十三、四岁毕业走上工作的岗位。

毛主席:现在书念多了害死人。现在的课程太多,负担太重,使中学生大学生天天处于紧张状态。中小学生近视眼成倍增加,这样非改不行。

×××:课程多而繁重,老师作业留得多,学生无法应付,紧张得不得了,没有课外活动和阅读时间。

毛主席:课程可以砍掉一半。学生要有娱乐、游泳、打球、课外自由阅读时间。孔子教学生的课程只有六门:礼、乐、射、御、书、数。就这样还教出了颜回、曾子……孟子等四大贤人。学生只是成天读书,不搞点文化娱乐,体育活动、游泳,不能跑跑跳跳,又不看课外读物……等,那是不行的。

×××:学生紧张得不得了。我在家时,小孩子说门门五分没有用。

毛主席:历史上的状元很少有出息的。唐朝有名诗人李白、杜甫既非进士,又不是翰林。韩愈、柳宗元还是二等进士。王实甫、关汉卿、罗贯中、蒲松令、曹雪芹也都不是进士翰林。蒲松令是一个提升的秀才,要高一等,还不是举人。凡是当了进士、翰林的都是不成功的。明朝搞得好的只有明太祖、明成祖两个皇帝,一个不识字,一个识字不多。以后到了嘉靖知识分子当权,反而不行了,就出了内乱。汉武帝、李后主文化多了亡了国。可见书念多了要害死人。刘秀是个大学生,而刘邦是个草包。

×××:课程过多、作业多,学生不能独立思考。现在的考试办法……

毛主席:现在的考试办法是对付敌人的办法,而不是对人民的办法。实行突然袭击,出偏题,出古怪题,还是考八股文章的办法,我不赞成,要彻底改革。我主张公开出考题,向同学公布,让同学自己看书,自己研究,看书去作。例如对《红楼梦》出二十道题,有的学生作出一半,但其中有几个题目答得很出色,有创造性,可以一百分。\marginpar{\footnotesize 93}另外有些学生二十道题都答了,是照书本上背下来的,按老师讲的答对了,但没有创造性的,只能给五十分或六十分,考试可以交头接耳,甚至冒名顶替。冒名顶替的也不过是照人家的抄一遍,我不会,你写了,我抄一遍,也可以有些心得,可以试点,要搞得活一些,不要搞得人死。先生讲课有的啰啰嗦嗦,允许学生打瞌睡,你讲的不好,还一定让人家听,与其睁着眼睛听着没味道,还不如睡觉,可以养养精神,可以不听,稀稀拉拉,休息一下脑筋。

×××:学制缩短了,可以抽出时间搞劳动或当兵。可以考虑优秀生跳班,不能老压在那里。我的小孩同一个班有一个同学,原来是优秀生,后来跳了班还是优秀生,可见跳班是可能的。关于学制问题,请××同志搞个专门小组研究。

毛主席:让××、×××都参加这个小组。现在我们搞得太死了,课程太多,考得太死,我们不赞成。现在的教育办法是摧残人材,摧残青年。我不赞成读那么多书。考试办法是对付敌人,害死人,要停止。

×××:现在教育厅长正在开会,有两个问题要研究:一是学生负担太重,门门有课外作业;二是教育学三套办法:孔夫子一套,苏联一套,杜威一套。

毛主席:孔夫子可不是这样。我们丢掉了孔夫子的主流,他只有六门课,礼、乐、射、御、书、数。(毛主席问×××:书是书法还是历史?)

×××:是书法吧。

毛主席:是历史吧。如书经、汉书。

×××:现在中小学以升学为唯一目标,毕业后不肯劳动,问题很大,要解决一下。要实行教育与生产劳动相结合,其次还要两条腿走路。河北省去年发大水,教育厅很紧张,很多房屋塌了,想来想去办简易学校,结果中小学人数反而增加了。

毛主席:大水冲垮了教条主义,洋教条、土教条都要搞掉。

×××:别的地方搞正规化,单式教学,不肯搞复式教学,学生人数下降,贫下中农人数下降,贫下中农失学的人数很多。河北省有了好经验。广东省新会县调查了十几所农业中学,普通中学。普通中学培养一个学生,国家一年化一百二十元;农业中学培养一个学生,一年只化六元八角。农业中学毕业生就业没有问题,普通中学毕业生考不上大学,就业就麻烦得很,所以中小学都要两条腿走路,同时要注意提高质量。以前就是苏联一套办法,一九五八年冲击了一下,劳动多了一些,又忽视了学习,改了就好。文艺也是如此,现在水平较高,如果没五八年,就没有现在水平。

毛主席:要把唱戏的、写诗的、文学家、戏剧家赶出城,统统都轰下去。都要分期分批到农村去,到工厂去。不要让作家住在机关里。不下去写不出东西来,谁不下去不给他开饭,下去了再开饭。

×××:现在中小学教师中有百分之二点几的坏分子,中小学还有出名的坏分子。

毛主席:那不要紧,可以转业。

×××:现在最坏的学生上师范,好学生进理工。今后可考虑师范文科不直接招高中毕业生,可招高中毕业后劳动过一、二年的学生。学自然科学的学生也要下去。哈尔滨××学校有经验,把教师下放一、二年,原来不好的劳动回来后都不错,成了骨干。

毛主席:应该下去。现在有些人不重视下乡劳动。明朝李时珍就是跑来跑去,上山采药。祖冲之也没有上过中学、大学,孔夫子出身于贫农,放过羊,也没有进过中学、大学,是个吹鼓手,他什么都干过,人家死了,他给人家吹吹打打,\marginpar{\footnotesize 94}也可能做过会计,会弹琴赶车,骑马射箭,“御”是驾车,就是当汽车司机。教出了颜回、曾子等七十二贤人,有弟子三千。他自小由群众中来,了解一些群众的疾苦。后来他在鲁国当了官,也不太大。鲁国有一百多万人口,长期人家瞧不起他,周游列国时,人家骂他,这个人爱说老实话,说他吃不了苦,挨不了骂。后来子路做了孔子的侍从保镖,他不准人家说孔夫子坏话,谁说了他就揍人家,从此不好的声音不再入耳了,群众不敢接近。孔夫子的传统不要,丢了。我们的方针正确,方法不对。现在的学制、课程、教学方法、考试方法都有不少问题,这一套都要改。这是摧残人的。

×××:小学五年是有把握的。

毛主席:小学也不要念得太长。高尔基只读过二年小学,学问完全是自学。美国的富兰克林是卖报出身,发明了电,瓦特是工人,发明了蒸气机。在古今中外许多科学家都是在实践中自修成的。

××:将来学制经过教改,学生到了二十三、四岁走上工作岗位是可以的,七岁入学太晚,可以提到六岁,就造房子有问题。小学改为五年可以解决一些房子。中学四年,预科一、二年,大学因各科性质不同,可以多样化,大学每年招生十四万到十五万人,可以办一、二年的预科。

×××:入大学前可拿出一段时间,进工厂,到农村劳动劳动。

毛主席:还有到军队去锻炼。

××:文科可以,但理科有数理化问题,劳动二年恐怕忘掉了。

××:苏联中学毕业后劳动二年后进理化科,不衔接。

××:大学如个别学校外,分三种学制:六年主要是医,五年制理工科,四年制文科。多数大学四年就行了,将来学制要多样化形式,多种学制。城市中学办两种,一种是升大学的,一种是毕业进专科,两年就毕业。

毛主席:对了,要多样化。

××:课程问题主要是不集中,还有过去研究的那个问题,好些课程是学好几遍,中学每学期八、九门课,考试多,很紧张。

毛主席:现在一是课多,二是书多,压得太重,有些课不一定要考。如高中学点逻辑、语法,不要考,真正理解要到工作中慢慢体会,知道什么是语法,什么是逻辑就行了。

××:现在是灌输、死记、死背。

×××:现在有两派意见:一是主张当堂讲深讲透,另一派是主张当堂能学懂,学会,学少点。现在不少学校就是前一派,前者不是办不到的,主张那么搞,把思想僵化了。

毛主席:这是繁琐哲学。四书、五经的注释很繁琐,现在都消化不了。繁琐哲学总是要灭亡的。如经学搞那么多注释现在统统消灭了。我看用这种办法教出来的学生,无论中国也好美国也好苏联也好都要消灭,都要走向自己的反面。如佛经那么多,唐玄奘考证的金刚经就比较简化,只有一千多字,现在还有。另一个鸠摩罗什考证的字太多了,灭亡了。五经、十三经不是也行不通吗?注释得很多。结果没人读,十四、五世纪搞了繁琐哲学,十七、十八、十九世纪才进入启蒙时期,出现了文艺复兴。书不能读得太多,马克思主义的书要读,也不能读得太多,读十几本就行。读多了就会走向反面,成为书呆子,成为教条主义、修正主义。孔夫子的书里没有农业知识,因此他的学生四体不勤、五等不分。这方面我们要想办法。\marginpar{\footnotesize 95}

×××:还有一个是政治问题,学生的伙食问题,需要改善。每月吃十二元五,要多花四千万元。

毛主席:多花四千万元也可。

×××:多增二至四元。

毛主席:念书多了,念死。梁武帝早年不错,以后书念多了就不行了,饿死在台城。

