\section[关于政府、军事系统审查旧人员中答复“包下来”问题的指示(一九五一年四月二十日)]{关于政府、军事系统审查旧人员中答复“包下来”问题的指示}
\datesubtitle{(一九五一年四月二十日)}


各中央局,并转分局、省委、区党委、大中市委;各大军区,并转各军区;中央各直属部门,各特种兵团,中央政府各党组:

(一)关于在政府系统中和军事系统中审查旧人员和知识分子的工作,各地各部门现在开始着手进行。在答复“包下来”的问题时,我们看见有说得不适当的。例如,有的人说:“解放接收时原封不动包下来了,今天又来处理,是否有矛盾?”答复:“没有矛盾。镇压与宽大结合政策,并无变动。以前包下来是为了迅速恢复生产,安定工作,特务分子不积极生产,而积极破坏生产,今天既然已了解他是坚决与人民为敌的罪大恶极的反革命分子,为什么还不处理?”(膝代远同志在铁路工会的报告)。这样答复是不妥当的。一九四九年四月中国人民革命军事委员会发表约法三章的布告,内称:“凡有一技之长而无严重的反动行为或严重的劣迹者,人民政府准予分别录用。”人民解放军和人民政府从来也没有说可以把“有严重的反动行为或严重的劣迹”的人们包下来。现在审查旧人员(还有新知识分子),就是要将那些混入军事系统和政府系统(包括公营工矿)中的有“严重的反动行为或严重的劣迹”的人们清理出来,分别加以惩办或淘汰。同时对于那些并无严重的反动行为或严重劣迹的人们(这些人占大多数)则因为清出了那些有严重的反动行为或严重劣迹的人们,而使他们显得历史清白,或虽有问题但不严重,利于团结和改造。这后一类人又分为两部分,第一部分是历史清白,没有问题的;第二部分是有些问题,但不严重,只要坦白承认错误,可以继续工作的(其中有些人须调动工作岗位)。

(二)所谓有严重反动行为,包括恶霸、匪首、惯匪、特务、欺压过许多人的行政官吏(这些人多属国民党及三青团的重要分子),反动的军官及反动会道门的头子等。

(三)下面是李××日给中央的信,其中所说的工厂和矿山的恶霸应予惩办,这是毫无疑议的,请你们加以注意。


