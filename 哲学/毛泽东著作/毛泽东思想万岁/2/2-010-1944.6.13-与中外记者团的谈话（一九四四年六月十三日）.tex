\section[与中外记者团的谈话(一九四四年六月十三日)]{与中外记者团的谈话}
\datesubtitle{(一九四四年六月十三日)}


我十分欢迎各位记者来到延安,我们的目的是共同的,就是打倒日本军阀与一切法西斯,全中国,全世界,都在这个共同的基础上团结起来。

各位到延安时,正遇欧洲开辟了第二战场,我们表示极大的庆祝。第二战场的开辟,其影响不仅在欧洲,而且将及于太平洋与中国。中国要前进,我们必须取得最后胜利。

第二战场的开辟,是经过长期发展的结果,是经过莫斯科,德黑兰会议发展而来的,在这些会议上决定了从东、西、南三面打击敌人。第二战场现在是实现了,三面打击希特勒的计划是实现了,我们谨祝罗斯福总统、丘吉尔首相、斯大林元帅的健康!

全中国所有抗战的人们,应该集中目标,努力工作,配合欧洲的决战,打倒日本军阀,现在时机是很好的。

关于中国国内情况,诸位先生是十分关心的,我在这里必须讲几句。关于国共关系,中国共产党对此问题的态度,早已见于中共中央历次文告及其报纸,今乘诸位先生来延之便,特再申述如下:坚持国共与全国人民的合作,为着打倒日本帝国主义,建立独立民主的中国而奋斗,中国共产党此种政策始终不变,抗战前期如此,抗战中期如此,今天还是如此,因为这是全中国人民所希望的。

但是中国是有缺点的,而且是很大的缺点,这种缺点,一言以蔽之,就是缺乏民主。中国人民非常需要民主,因为只有民主,抗战才有力量,中国国内关系和对外关系,才能走上轨道,才能取得抗战的胜利,才能建设一个好的国家,亦只有民主才能使中国在战后继续团结,中国缺乏民主,是在座诸位所深知的,只有加上民主,中国才能前进一步。


问题与答复

问题:

斯坦因先生问:毛主席是否谈一谈,林伯渠先生在重庆谈判的情况?

夏南汗神父问:上述问题为大家所关心,可否尽先答复?

斯坦因先生问:为使问题明了起见,我请毛主席将一九三六年国共谈判的情形与今日谈判情形做一比较。

爱卜斯坦先生问:第二战场的开辟是否引起了一个新阶段?中共中央对此是否准备发表宣言,阐明中共中央之政策?

谢爽秋先生问:为着加强团结,中国共产党希望于各方面的是什么?

赵炳良先生问:为使问题明了起见,我增加问,中国共产党希望国民政府、国民党及其他各党派做些什么?中国共产党本身又准备做什么?

毛主席答:

诸位的问题可综合为三个:

第一个问题:关于国共谈判,谈判已进行了许久,但是今天还在谈判中,我们希望谈判有进步,并能获得结果。其他今天还无可奉告。

第二个问题:关于第二战场。日前解放日报社论已说明是一个新阶段,我们不准备再发表宣言。第二战场的开辟是同盟国战争合作的发展。其总的性质,现在与过去比较,是没有变化的。但是第二战场的开辟有与斯大林格勒反攻某种相等的意义。一九四二年十一月以前,是法西斯凶焰高涨,反法西斯力量被打与退却的时候,赖有苏联的进攻结束了过去的阶段,开辟了新阶段。接着,北非与太平洋相继有了进攻,这是同盟国从防御到进攻的一个大转变。第二战场开辟,在进攻中又前进了一大步,如果没有它,就不能打倒希特勒,现在欧洲已进到了决战阶段了,在这个意义上说,它是一个新阶段,特别在军事方面,我已说过,第二战场开辟的影响会是很广泛的,直接影响欧洲,将亦会影响到太平洋与中国。但就目前来说,对中国的影响似乎不会很大,你们可以看见,外面的情况虽然甚好,但是,中国的问题还靠中国人民自己去努力,单有国外情况的好转,是不能解决问题的。

第三个问题,关于中央的希望和它自己的工作。为了打倒共同的敌人以及为了建立一个很好的和平的国内关系,及一个很好的和平的国外关系,我们所希望于国民政府、国民党及一切党派的,就是从各方面实现民主。全世界都在抗战中,欧洲已进入了决战阶段,只有民主,抗战才能够有力量,这是苏联、美国、英国的经验都证明了的,中国几十年以来及抗战七年以来的经验,也证明了这一点。民主必须是各方面的,是政治上的、军事上的、经济上的、文化上的、党务上的以及国际关系上的,一切这些都需要民主。毫无疑问,无论什么都需要统一,都必须统一。但是,这统一应该建筑在民主基础上。政治需要统一,但是只有建立在言论出版集会结社的自由与民主选举政府的基础上面,才是有力的政治。统一在军事上尤为需要,但是军事的统一,亦应建筑在民主基础上,在军官与士兵之间,军队与人民之间,各部军队相互之间,如果没有一种民主生活、民主关系,这种军队是不能统一作战的。经济民主,就是经济制度要不是妨碍广大人民的生产、交换与消费的发展。党务民主,就是在政党的内部关系上与党的相互关系上,都应该是一种民主的关系,在国际关系上,各国都应该是民主国家,并发展民主的相互关系,我们希望外国在中国的朋友以民主的态度对待我们,我们也应该以民主的态度对待外国及外国朋友。我重复说一句,我们很需要统一,但是只有建筑在民主基础上的统一,才是真统一。国内如此,新的国际联盟亦将如此,只有民主的统一,才能打倒法西斯,才能建设新中国与新世界。我们赞成大西洋宪章及莫斯科、开罗、德黑兰的会议决议,就是基于这个观点的。我们希望于国民政府、国民党及各党派、各人民团体的,主要的就是这些。中国共产党所已做和所要做的,也就是这些。先生们来到边区已经十几天了,今后还要有若干时间留在边区,你们可以看到,我们共产党人为着打倒日本帝国主义而做的一切工作,都贯彻着一个民主统一或民主集中的精神。其有不足的,必须继续做。如果有缺点必须克服这些缺点。我们认为全中国只有民主制度,民主作风,目前才能胜敌,将来才能建立一个很好的和平的国内关系与国际关系。对于德意日等法西斯国家,在法西斯被打倒以后,我们所希望于他们的,也是如此。持此观点来看许多问题,没有不可以说通与做通的。今天时间已晚,今后还可以相互交换意见。我要说的,就是如此。

