\section[在第十六次最高国务会议上的讲话纪要(一九五九年四月十五日)]{在第十六次最高国务会议上的讲话纪要}
\datesubtitle{(一九五九年四月十五日)}


有许多人对西藏寄于同情。但是他们同情少数人,不同情多数人,一百个人里头,同情几个人,就是那些叛变分子,不同情百分之九十几。在外国,有那么一些人,他们对西藏人就是只同情一两万人,顶多三、四万人。西藏(包括昌都、前藏、后藏三个区域)大概是一百二十万人。一百二十万人,用减法除掉几万人,还是一百十几万人,世界上有许多人对他们不同情。我们的同情相反,我们同情一百一十几万人,而不同情那少数人。

那少数人是一些什么人呢?那少数人就是剥削压迫分子。讲贵族,××和阿沛也算贵族。但是贵族有两种:一种是进步的贵族,一种是反动的贵族。进步分子主张改革,旧制度不要了,舍掉它算了。旧制度不好,对西藏人民不利,一不人兴,二不财旺。西藏地方大,人太少了,要发展起来。这个事情,我跟达赖讲过,我说,你们要发展人口。我还说,你们的佛教,就是喇嘛教,我是不信的,我赞成你们信。但是,有些规矩可不可以稍微改一下子?你们一百二十万人里头,有八万喇嘛,这八万喇嘛是不生产的,一不生产物资,二不生产人口,你看,基督教是结婚的,回教是结婚的,天主教多数是结婚的,只有少数不结婚。(周恩来:印度教也结婚,日本的佛教,除少数人以外,也结婚。)就是西藏的佛教不结婚,不生人,不生产后代。这是不是可以改一改,来一个近代化?同时,教徒(喇嘛)要从事生产,搞农业、搞工业,这样可以维持长久。你们不是要天长地久,永远信佛教吗?我是不赞成永远信佛信教的。但是你们要,那有什么办法?我们是毫无办法的,信不信宗教的事情,只能个人自己决定。这些话,我跟达赖一个人谈过。我说,你们为了长久之计,是不是可以加以改革。

至于贵族,对那些站在进步方面主张改革的革命的贵族,以及还不那么革命,站在中间动动摇摇,不站在反革命方面的中间派,我们采取什么态度呢?我个人的意见:对于他们的土地,他们的庄园,是不是可以用我们对待民族资产阶级的办法:实行赎买政策,使它不吃亏。比如我们中央人民政府,把他们的生活包下来。你横直剥削农奴也是得那么一点,中央政府也给你们一点,你为什么一定要剥削农奴才舒服呢?

我看西藏是个农奴制度,就是我们春秋战国时代那个庄园制度。奴隶不是奴隶,自由农民不是自由农民,是介乎这两者之间的一种农奴制度。坐在农奴制度的火山上是不稳固的,每天都觉得地球要地震,何不舍掉算了,不要那个农奴制度了,不要那个庄园制度了,那一点土地不要了送给农民。但是吃什么呢?我看,对革命的贵族,革命的庄园主,还有中间派的贵族,只要他不站在反革命那方面,用赎买政策,我跟大家商量一下,看是不是可以?现在是平叛,来不及改革,将来改革的时候,凡是革命的贵族,以及中间派,动动摇摇的,总而言之,只要不站在反革命那边,我们不使他吃亏,就是照我们现在对待资本家的办法。并且,他这一辈子我们都包到。资本家也是一辈子都包到。几年定息过后,你得包下去,你得给他工作,你得给他薪水,你得给他就业,一辈子都包下去。这样有利,这样一来,农民就不恨这些贵族了,把仇就解开了。而农民(百分之九十五以上的人口)得了土地。

日本有个报纸哇哇吗,讲了一篇,它说,共产党在西藏问题上打了一个大败仗,全世界都反对共产党。说我们打了大败仗,谁人打了大胜仗呢?总有一个打了大胜仗的吧,又有人打了大败仗,又没有人打了大胜仗,哪有那事,既然我们打了大败仗,那么就有打了大胜仗的。你们讲,究竟胜负如何?假定我们中国人在西藏问题上打了大败仗,那么,谁人打了大胜仗呢?是不是可以说印度干涉者打了大胜仗?我看也很难说。你打了大胜仗,为什么那么痛哭流涕,如丧考妣呢?你们看我这个话有一点道理没有?

还有个美国人,名字叫艾尔奈普,写专栏文章的,他隔那么远。认真地写一篇文章,说西藏这个地方没有二十万军队是平定不了的;而这二十万军队,每天要一万吨的物资,不可能运这么多。西藏那个山高得不得了,共产党的军队难得去。因此,他断定,叛乱分子灭不了。叛乱分子灭得了灭不了呀?我看大家都有这个问题。因为究竟灭得了灭不了,没有亲临其境,没有打过游击战争的人,是不知道。这是个大家的疑问。我这里回答:不要二十万军队,只要××军队,只要二十万的×分之一。一九五六年以前就是××(包括干部)。一九五六年那一年我们撤退××多,剩下××多。那个时候,我们确实认真地这样搞了,宣布六年不改,六年以后,如果你们还不赞成,我们还可以推迟,是这样讲的。你们晓得,西藏整个民族不是一百二十万人,而是三百万人。刚才讲的西藏本部(昌都、前藏、后藏)是一百二十万人,其他在哪里呢?主要在四川西部,就是原来西康区域以及川西北,就是毛儿盖、松番、阿赛那些地方。这个地方最多。第二是青海,有五十万人。第三是甘肃南部。第四是云南西北部。这四个区域合计一百八十万人。四川省人民代表大会开会,跟他们商量,搞点民主改革,听了一点风,立刻就传到原西康这个区域,就举行叛乱,武装斗争。现在在青海、甘肃、四川、云南都改革了,人民武装起来了。藏人扛起枪来,组织自卫武装,非常勇敢。这四个区域能够把叛乱肃清,为什么西藏不能肃清呢?你讲复杂,西康这个区域是非常复杂的,为什么现在有许多所谓康边人去了西藏呢?就是西康的叛乱分子打败了,跑到那里去的。他们跑到那里,奸淫虏掠,抢得一塌糊涂,他要吃饭,就得抢。于是康人同藏人就发生了矛盾。西康跑出去的,青海跑出去的有一万多人。一万多人要不要吃饭?那里来呢?就在这一百二十万人中间吃来吃去。从去年七月算起,差不多已经吃了一年了。这回我们把叛乱分子打下来,把那些枪收缴了。比如日喀则,把那个地方政府的队伍的枪收缴了,江孜也收缴了,亚东也收缴了。收缴了枪的地方,群众非常高兴。老百姓怕他们三个东西:第一是怕他那印,就是怕那个图章;第二怕那个枪;第三,还有一条法鞭,老百姓很怕。把这三者一收,群众皆大欢喜,非常高兴,帮助我们搬枪枝弹药。西藏的老百姓痛苦得不得了。那里的农奴主对老百姓硬是挖眼,硬是抽筋,甚至把十六岁女孩子的脚骨拿来作乐器。还有拿人头作饮器,喝酒。这样野蛮透顶的叛乱分子完全能够灭掉,不需要二十万军队,只需要××军队,可以灭得干干净净。灭掉是不是都杀掉呢?不要。所谓灭掉者,并不是把他们杀掉,而是把他们捉起来教育改造,包括反动派,比如索康那种人。这样的人,如果他们回来,悔过自新,我们不杀他。

再讲一个中国人的议论。此人在台湾,名为胡适,他讲,据他看,这个“革命军”(就是叛国分子)灭不了。他说,他是徽州人,日本人打中国的时候,占领了安徽,但是没有去徽州。什么道理呢?徽州山太多了,地形复杂,日本人连徽州的山都不敢去,西藏那个山共产党敢去?我说,胡适之这个方法论就不对,他那个“大胆假设”是危险的。他大胆假设,他推理,他说徽州山小,日本人尚且不敢去,那么西藏的山大得多,高得多,共产党难道敢去吗?因此,结论:共产党一定不敢去,共产党灭不了那个地方的叛乱武装。现在要批评胡适之这个方法论,我看他是要输的,他并不“小心求证”。只有“大胆假设”。

有些人,像印度资产阶级,比如尼赫鲁总理这些人,又不同一点,他们是两面性:一面非常不高兴,非常反对我们这种政策,非常反对我们三月二十号以后开始的坚决镇压叛乱。


