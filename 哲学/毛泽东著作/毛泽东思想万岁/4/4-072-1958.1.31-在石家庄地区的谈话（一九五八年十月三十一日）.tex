\section[在石家庄地区的谈话(一九五八年十月三十一日)]{在石家庄地区的谈话}
\datesubtitle{(一九五八年十月三十一日)}


你们这里是河北省水利化先进的地区?今年的水库用上了没有?(×××:有的用上了,有的没有用上。)

这里有什么铁矿没有?(×××:井陉、平山、获路等十几个县都有铁矿,估计有×××吨。)可以在这里搞个大钢铁厂了,你们不一定在邯郸搞一个大钢厂吧,在这里也可以搞一个吧。

今年的麦子种的怎么样?每亩下种多少斤?犁多深?(×××:一亩下种三十多斤,耕一尺多深。)今年一尺多,去年才有三、四寸,那就不错吧,是否准备大面积丰产,五千斤到一万斤有没有?(××:有一个县搞一万斤。)你们今年小麦平均多少?(×××:270斤。)今年丰产,有上帝帮忙,究竟不算。白薯亩产千斤的有几个县?(×××:有十四个县。)那还有十几个没达到。今年小孩摘棉花,将来农业是妇女和娃娃们的事情。

究竟是否需要化肥,要它干什么?我看就搞土化肥好了,不搞洋化肥怎么样?拖拉机是否需要?(××:拖拉机搞深翻地还有用处,带三华五华改成带一个犁可以深耕。)这样子你们要拖拉机了。化学肥料不要可以吧?(×××:有些还是好。)搞土化肥,洋化肥是否少搞一些?你们这里有拖拉机厂没有?(×××:有一个,现在搞鼓风机。)暂时休息,只搞鼓风机,不搞拖拉机。

人民公社搞的怎么样?只是搭起架子吧,食堂办了没有?是在一起吃饭,还是打回家去吃?(×××:有几种,有的在一起吃,有的打回家去吃。)打回家去不冷了吧,食堂里作不作菜?你们食堂有办的好的吧。比较一下,分一、二、三类,都要向一类看齐,人民是否欢迎吃大锅饭?过去不欢迎,现在欢迎了。(×××:正定县妇女罢了一天工。食堂就办起来了。)正定罢了一天工,真有其事,罢一天工有什么不好,进步的罢工,是对落后分子最好的批评。

一个食堂,一个托儿所,两个事注意搞好。搞不好影响很大,影响生产。饭吃不好就生产不好,小孩带不好就影响后一代。保育员要像母亲那样关心孩子,你们有没有人管这个事情?这个问题很值得研究,对孩子一叫一闹就打不好,要叫小孩子吃的好,穿的好,玩的好,睡的好,要了解他们的心理状态。

每个人民公社都要种商品作物,如果只种粮食那就不行了,那就不能发工资。山区可以种核桃、梨,可以养羊子,拿到外面去交换。河北是否可以分这样几类县,第一类饭不够吃的有几个县,第二类有饭吃,可以实行吃饭不要钱,工资一个也没有,你们这里有没有?(×××:有两个公社。)这个要研究一下,看能否有些商品出卖?粮食这个商品出路不大了,可以搞些核桃、枣子,是种核桃好还是种枣子好?(×××:这两个社都有核桃。)这两个社有核桃,为什么不能发工资?(×××:这两个社是建屏县高山上的两个社。)你们听说过李顺达那个社吧?他这个社就是在太行山上,七扶八扶起来了。不能发工资的社要把它扶起来。种些核桃,核桃是高级油料。将来普通油料是吃不开的,菜子油是吃不开的,要种些芝麻。叫人吃香油么,此外要喂羊子,喂羊和林业有矛盾,吃光了山。好处是,一个拉肥料,一个是羊毛。我们这里有多少羊?(×××:48只。)你们(指×××)那里羊多一些,(×××:张家口地区多。)第三类发工资很少的,三、五元。第四类发的工资比较多,安国一年就是一百多元,平均大人小孩一百多元。(×××:大人小孩平均110元。)你们要摸一下底。看生活水平情况怎么样,有高有低吗?安国比徐水高多了,徐水是70元,安国是110元。安国很值得注意,你们这里有没有像安国这样的县?(×××:正定县好,棠城比正定还好。)正定这个县比较好,棠城比正定还好,离这里有多远?深县是你们这里的吧,他们种的蜜桃很好,把深县这个县都种成蜜桃可以不可以?

吃饭不要钱都实行了吧?(×××:已经有五个县都实行了。计划在十一月分全部实行。)人家不要求实行,你计划实行怎么能行?有人说劳多人少的不赞成.这部分占15%,他们感到吃亏,发工资是否多发一些,是否应当多发些?不然,他就不舒服。一家五口人,四个劳力,另一家五口只一个劳力,这两家就是不同了,恐怕要照顾一下劳力多些的。现在是社会主义,价值法则还是存在的,有些政治觉悟不高也不在乎。

干部里也有不痛快的吧,徐水怎么样?实行供给制能不能持久?年把垮台还不如谨慎些好,现在还是依靠这些干部么?向干部讲清楚,不要同群众过于悬殊。

石家庄有唱戏班子没有?有很的没有?他们拿多少薪水?(王力:有个奚哨伯,过去是八百元,现在成了右派降到五百元。)到五百元他还唱不唱?(王力:评剧团有个郭彦芳,每月400元,要求实行供给制。)降低太多了不好,降低八元就不好。城市公社石家庄闹了没有?(王力:正在搞。)可以慢慢研究,不要那么忙。


