\section[接见日本社会党人士佐佐木更三、黑田寿男、细迫兼光等的谈话(一九六四年七月十日)]{接见日本社会党人士佐佐木更三、黑田寿男、细迫兼光等的谈话}
\datesubtitle{(一九六四年七月十日)}


\begin{list}{}{
    \setlength{\topsep}{0pt}        % 列表与正文的垂直距离
    \setlength{\partopsep}{0pt}     % 
    \setlength{\parsep}{\parskip}   % 一个 item 内有多段,段落间距
    \setlength{\itemsep}{\lineskip}       % 两个 item 之间,减去 \parsep 的距离
    \setlength{\labelsep}{0pt}%
    \setlength{\labelwidth}{3em}%
    \setlength{\itemindent}{0pt}%
    \setlength\listparindent{\parindent}
    \setlength{\leftmargin}{3em}
    \setlength{\rightmargin}{0pt}
    }

\item[\textbf{主席:}] 欢迎朋友们。对日本朋友,十分欢迎。我们两国人民应当团结,反对共同敌人。在经济上互相帮助,使人民的生活有所改善。文化上也要互相帮助。你们是经济、文化、技术都比较我们发展的国家,所以,恐怕谈不上我们帮助你们。是你们帮助我们的多。

谈到政治上,难道我们在政治上不要互相支援吗?而是互相对立吗?像几十年前那样互相对立吗?那种对立的结果,对你们没有好处,对我们也没有好处。同时,另外讲一句相反的话:对你们有好处,对我们也有好处。二十年前那种对立,教育了日本人民,也教育了中国人民。

我曾经跟日本朋友谈过。他们说,很对不起,日本皇军侵略了中国。我说:不!没有你们皇军侵略大半个中国,中国人民就不能团结起来对付你们,中国共产党就夺取不了政权。所以,日本皇军对我们是一个很好的教员,也是你们的教员。结果日本的命运那么样呢?还不是被美帝控制吗?同样的命运在我们的台、港,在南朝鲜、在菲律宾、在南越、在泰国。美国人的手伸到我们整个西太平洋、东南亚,它这个手伸得太长了。第七舰队是美国最大的舰队,它有十二只航空母舰,第七舰队就占了一半——六只。它还有一个第六舰队在地中海。当一九五八年我们在金门打炮时,美国人慌了,把第三舰队的一部分向东调。美国人控制欧洲,控制加拿大,控制除古巴以外的整个拉丁美洲。现在伸到非洲去了,在刚果打仗。你们怕不怕美国人?\marginpar{\footnotesize 138}

\item[\textbf{佐佐木:}] 让我代表访问中国的五个团体简单地讲几句话。

\item[\textbf{主席:}] 好。

\item[\textbf{佐佐木:}] 感谢主席在百忙中接见我们,并作了有益的谈话。我看到主席很健康,为中国社会主义的跃进,为领导全世界的社会主义事业日夜奋斗,在此向主席表示敬意。

\item[\textbf{主席:}] 谢谢!

\item[\textbf{佐佐木:}] 今天听到了毛主席非常宽宏大量的讲话。过去,日本军国主义侵略中国,给你们带来了很大的损害,我们大家感到非常抱歉。

\item[\textbf{主席:}] 没有什么抱歉。日本军国主义给中国带来了很大的利益,使中国人民夺取了政权。没有你们的皇军,我们不可能夺取政权。这一点,我和你们有不同的意见,我们两个人有矛盾。(众笑,会场活跃)

\item[\textbf{佐佐木:}] 谢谢。

\item[\textbf{主席:}] 不要讲过去那一套了。过去那一套也可以说是好事,帮了我们的忙。请看,中国人民夺取了政权。同时,你们的垄断资本、军国主义也帮了你们的忙。日本人民成百万、成千万地觉醒起来。包括在中国打仗的一部分将军,他们现在变成我们的朋友了。有一千一百多人(指战犯——编者)回到日本,写来了信。除了一个人之外,都对中国友好。世界上的事就是这么怪的。这一个人叫什么名字?

\item[\textbf{赵安博:}] 叫饭森,现在当法官。

\item[\textbf{主席:}] 一千一百多人,只有一个人反对中国,同时也是反对日本人民。这件事值得深思,很可以想一想。你(指佐佐木)的话没讲完,请再讲。

\item[\textbf{佐佐木:}] 毛主席问我们怕不怕美国人。中国已经完成了社会主义革命,现在正在为彻底实现社会主义而工作。而日本,今后才搞革命,才搞社会主义。要使日本革命成功,就必须击败事实上控制日本的政治、军事、经济的美国。因此,我们不仅不怕美国,而且必须同它斗争。

\item[\textbf{主席:}] 说得好!

\item[\textbf{佐佐木:}] 这次我们来中国,同周恩来总理、廖承志先生、赵安博先生以及其他中国朋友一起,就日中问题,就围绕日中问题的亚非形势和世界形势,世界的帝国主义、新旧殖民主义等问题,交换了意见,得到了教益,并且找到了许多共同点。我们回国以后,一定要促使日本社会主义的发展,加强日中两国的合作关系。

\item[\textbf{主席:}] 这个好!

\item[\textbf{佐佐木:}] 日本社会党和日本的人民群众认为,日本是亚洲的一员,因此,它必须同关系很深的中国保持密切的关系,希望中国把日本当作亚洲的一员,同我们进行合作。

\item[\textbf{主席:}] 一定,互相合作。整个亚洲、非洲、拉丁美洲的人民都反对美帝国主义。欧洲、北美、大洋洲也有许多人反对(美)帝国主义。帝国主义者也反对(美)帝国主义。戴高乐反对美国就是证明。我们现在提出这么一个看法,就是两个中间地带。亚洲、非洲、拉丁美洲是第一个中间地带。欧洲、北美、大洋洲是第二个中间地带。日本的垄断资本也属于第二个中间地带。你们的垄断资本是你们反对的,可是他们也不满意美国。现在已经有一部分人公开反对美国。另一部分依靠美国。我看,随着时间的延长,这一部分人中的许多人也会把骑在头上的美国人赶掉。因为的确日本是一个伟大的民族。它敢于跟美国作战,跟英国作战,跟法国作战;\marginpar{\footnotesize 139}曾经轰炸过珍珠港,曾经占领过菲律宾,占领过越南、泰国、缅甸、马来亚、印度尼西亚;曾经打到印度的东部,就是因为那个地方夏天蚊子很多,台风很大,没有深入进去,打了败仗。日本军队在那里损失了二十万人。这样一个垄断资本让美帝国主义稳稳地骑往自己的头上,我就不相信。在这里,我不是赞成再轰炸珍珠港,(众笑)也不是赞成占领菲律宾、越南、泰国、缅甸、印度尼西亚、马来亚,当然,我也不赞成再去打朝鲜和中国了。日本完全独立起来,和整个亚洲、非洲、拉丁美洲、和欧洲的愿意反对美国帝国主义的人们, 建立友好关系,解决经济方面的问题,互相往来,建立兄弟关系,岂不好吗?

刚才你说到你们日本要革命,将来要走社会主义道路,这个话讲得很正确。全世界人民都要走你所讲的这条道路。把帝国主义、垄断资本埋葬到坟墓中去。

还有朋友提问题吗?什么问题都可以提出来,我们商量商量,这是座谈会。你们不是有五个团体吗?

\item[\textbf{佐佐木:}] (对日本人说)各团出一个代表讲话吧!

\item[\textbf{黑田:}] 我与其说是提出问题,勿宁说是谈一谈日本的日中友好运动。

\item[\textbf{主席:}] 好!

\item[\textbf{黑田:}] 日中友好运动,开始时只有社会主义者和从事工人运动的人参加。最近,逐渐包括了广大的各阶层人民。这是日中友好运动的变化、特征,也是一个前进,值得注意。从政党来说,过去参加日中友好运动的是革新政党(在日本革新政党包括社会党、共产党),现在保守党中的一部分人也下决心参加日中友好运动了。从国民的阶层来看,过去参加日中友好运动的有工人、农民、学生、知识分子和中小企业者。最近,连垄断资本中的一部分人,也要日中友好,特别是下决心搞日中贸易。

\item[\textbf{主席:}] 我也知道,是个很大的变化。单是搞中小贸易,不搞大贸易,不和垄断资本搞贸易,意义就不完全,也不算大。

\item[\textbf{黑田:}] 保守党内和垄断资本中有一部分人也开始搞日中友好和日中贸易,当然也有跟美国走的,因此在保守党和垄断资本内部发生了矛盾和分裂。这是最近的突出的情况。而且,这一部分垄断资本和保守党,不能和我们完全一样,这样要求他们是不可能的。

因此,这里就必须有斗争,那些没有决心向前看的一部分垄断资本家和保守党的背后,有美国的力量。美国在操纵他们。因此,同这部分反动的保守党和反动的垄断资本家进行斗争,实际上也是和美国进行斗争。整个说来,要求恢复日中邦交的运动,成了国民运动。日中友好运动的另一个特点是,日本人民对中国抱有亲近感,有的表现出来,有的潜在着。这样一种感情是促进日中友好,恢复日中邦交的一个很大的力量。日本人对美国没有这种感情,对英国、苏联也没有这种感情,对中国却有特殊的感情。

\item[\textbf{主席:}] 中国人民也是这样,高兴和日本人民的代表们亲近,关心我们两国的关系。你们可以看到,到中国什么地方都可遇到中国人民对你们是友好的。他们知道时代不同了,情况变化了。中国的情况变了,日本的情况变了,世界的情况变了。昨天我接待了几十位亚洲、非洲的朋友,也在这个地方(指接见的场所)。有十五位非洲的黑人和阿拉伯人,有十五位亚洲朋友,有一位澳洲朋友。今天你们是三十位朋友,昨天是三十一位。其中有日本朋友,就是他(指西园寺公一)。有两个泰国的代表。这个国家跟我们现在是对立的。这个国家来了两位代表参加平壤的经济讨论会。但是没有印度人。\marginpar{\footnotesize 140}(会场活跃)你们以为印度人都是反对中国人的吗?不是。印度广大的人民同中国广大的人民是互相友好的。我相信,印度的广大人民也是和日本的广大人民友好的。就是他们的政府被帝国主义、修正主义控制,受帝国主义、修正主义的影响很大。有三个国家援助印度以武器来打我们。这就是美国、英国、苏联。你说怪不怪?苏联过去与我们是很好的。自从一九五六年二十大以后,就开始不好了。后来就越来越不好。把在中国的专家一千多人统统撤退。几百个合同统统撕毁。首先公开反对中国共产党。既然你反对,我们就要辩论。他们现在又要求停止公开辩论,那怕停止三个月也好。我们说三天也不行。(众笑)我们说,我们过去打二十五年仗,这里包括国内战争、中日战争二十二年,朝鲜战争三年,一共二十五年。我说,我这个人是不会打仗的,我的职业是教小学生的小学教师。谁人教会我打仗呢?第一个是蒋介石,第二个是日本皇军,第三个是美帝国主义。对这三个教员我们要感谢。打仗,并没有什么奥妙的,我打了二十五年仗,我也没有受过伤。从完全不懂到懂,从不会到学会打仗。打仗是要死人的,在这二十五年中,我们的军队和中国人民死伤总有几百万、几千万。那么,中国人不是越打越少吗?不!你看,现在我们有六亿多人口,太多了。要打文仗,打笔墨官司,公开辩论,是不会死人的。打了几年了,一个人也没有死。我说我们也准备打二十五年。我们请罗马尼亚代表团转告苏联朋友。罗马尼亚代表团就是来作这工作的,要停止公开争论。听说现在罗马尼亚和苏联也打起笔墨官司来了。(笑)问题就是一个大国要控制许多小国,一个要控制,一个就反控制,等于美国控制日本和东方各国,日本和东方各国势必就要反控制一样。世界上两个大国交朋友,一个美国,一个苏联,企图控制整个世界。我是不赞成的,也许你们赞成,让他们控制吧?(外宾表示不赞成)

\item[\textbf{细迫:}] 我曾经长期坐过监狱。像我这样善良的好人被关在监狱,对有病的妻子,也不能照料。对这样恶劣的政府,我没有办法像主席那样宽大。这次来中国访问是从神户坐中国的“燎原”号货轮来的。日本的友好团体租了小船,打旗、奏乐来欢送。但日本警察方面的小船也在那里转来转去,采取了另外一种行动。我们来中国后,中国的政府要人和人民一道来欢迎我们。希望日本也能早日成为一个政府和人民能一起欢迎中国朋友的国家。

\item[\textbf{主席:}] 你们从上海登岸的?

\item[\textbf{细迫:}] 是的。像日本政府那样的坏政府应当早日打倒,建立一个人民政府,否则就实现不了真正的友好。我不能宽恕欺负我的政府。我年纪大了,想在我的遗嘱里告诉我的孩子,要他们打倒政府。

\item[\textbf{主席:}] 多大年纪了?

\item[\textbf{细迫:}] 六十七岁。

\item[\textbf{主席:}] 比我小嘛!你活到一百岁,所有帝国主义都垮台了。你们恨日本政府、日本的亲美派,跟我们过去恨国民党政府亲美派——蒋介石是一样的。蒋介石是一个什么人物呢?曾经和我们合作过,举行过北伐战争,这是一九二六年到一九二七年的事。到一九二七年他就杀共产党,把几百万人的工会、几千万人的农会,一扫而光。蒋介石是第一位教会我们打仗的人,就是指这一次。一打就打了十年。我们从没有军队,发展到有三十万人的军队,结果我们自己犯错误,这不能怪蒋介石,把南方根据地统统失掉,\marginpar{\footnotesize 141}只好进行二万五千里长征。在座的,有我,还有廖承志同志。剩下的军队有多少呢?从三十万减到二万五千人。我们为什么要感谢日本皇军呢?就是日本皇军来了,我们和日本皇军打,才又和蒋介石合作。二万五千军队,打了八年,我们又发展到一百二十万军队,有一亿人口的根据地。你们说要不要感谢呀!

\item[\textbf{荒哲夫:}] 我提一个问题。先生刚才说两大国要控制世界。现在,日本有一个奇妙的现象。日本的冲绳和小笠原群岛被美国占领,但在北方,在我居住的北海道的左边有个千岛群岛,被苏联占领了。从我们这方面来说是被占领的。据说,千岛是根据我们没有参加的波茨坦公告划归苏联的。我们长期同苏联交涉,要求归还,但是没有结果。很想听听毛主席对这个问题的想法。

\item[\textbf{主席:}] 苏联占的地方太多了。在雅尔塔会议上就让外蒙古名义上独立,名义上从中国划出去,实际上就是受苏联控制。外蒙古的领土,比你们千岛的面积要大得多。我们曾经提过把外蒙古归还中国是不是可以。他们说不可以。就是同赫鲁晓夫、布尔加宁提的,一九五四年他们在中国访问的时候。他们又从罗马尼亚划了一块地方,叫做比萨拉比亚。又在德国划了一块地方,就是东部德国的一部分。把那里所有的德国人都赶到西部去了。他们也在波兰划了一块归白俄罗斯。又从德国划了一块归波兰,以补偿从波兰划给白俄罗斯的地方。他们还在芬兰划了一块。凡是能够划过去的,他都要划。有人说,他们还要把中国的新疆、黑龙江划过去。他们在边境增加了兵力。我的意见就是都不要划。苏联领土已经够大了,有二千多万平方公里,而人口只有两亿。

你们日本人口有一亿,可是面积只有三十七万平方公里。一百多年前,把贝加尔湖以东,包括伯力、海参崴、勘察加半岛都划过去了。那个账是算不清的。我们还没跟他们算这个账。所以你们那个千岛群岛,对我们来说,是不成问题的,应当还给你们的。

\item[\textbf{曾我:}] 在三十个人当中,我们这一批人(社会主义研究所代表团)最年青,都是在第一线活动的。我们很想了解革命政党的建党和党风。我们都是社会党的左派。我们同社会党中央的改良主义者、结构改革论者进行斗争。

\item[\textbf{主席:}] 你们有多少人?

\item[\textbf{曾我:}] 全团十一人。从我们年青人看来,我们觉得社会党的干部、议员行动迟钝。也许因为他们年老。(主席插话:包括我在内了。)我们很想了解中国共产党的干部作风和党风,请讲讲。

\item[\textbf{主席:}] 这个问题应该说我比较熟悉。我们这一批人参加过一九一一年的资产阶级民主革命——孙中山领导的,当过兵。从那时和那时以后,我读过十三年书,有六年读的是孔夫子,有七年是读资本主义。干过学生运动,反对过当时的政府。干过群众运动,反对过外国侵略。就是没有准备组织什么党。既不知道马克思,也不知道列宁。因此就没有准备组织什么共产党。我相信过唯心主义,相信过孔夫子,相信过康德的二元论。后来,形势变化了,一九二一年组织了共产党。当时全国有七十个党员,选出十二个代表,在一九二一年开了第一次代表大会,我是代表之一。其中还有两个,一个是周佛海,一个叫陈公博,后来他们都脱离了共产党,参加了汪精卫政权。另一个,后来成了托派。这个人现在住在北京,还活着。我活着,那个托派还活着,第三个活着的就是董必武副主席。其他的都牺牲了,或者是背叛了。\marginpar{\footnotesize 142}从一九二一年组织党到一九二七年北伐,只晓得要革命,但怎么革命,方法、路线、政策,啥也不懂。后来初步懂得,这是在斗争中学会的。比如土地问题吧,我是花了十年功夫研究农村阶级关系。战争嘛,也是花了十年,打了十年仗,才学会战争。党内出右派的时候,我就是左派。党内出“左”倾机会主义时,我就被称为右倾机会主义。啥人也不理我,就剩我一个孤家寡人。我说,有一个菩萨,本来很灵,但被扔到茅坑里去,搞得很臭。后来,在长征中间,我们举行了一次会议,叫遵义会议,我这个臭的菩萨,才开始香了起来。后来,又花了十年时间。从一九三四年到一九四四年,我们又用整风的办法,我们叫做“惩前毖后,治病救人”,“团结——批评——团结”的路线,说服那些犯错误的同志。以后在一九四五年上半年的七次党代会上,终于将党的思想统一起来了。所以我们才能够在美帝国主义和蒋介石发动进攻时,用四年的工夫把他们打败。

你们的问题是党的作风吗?首先是政策问题——政治方面的政策,军事方面的政策,经济方面的政策,文化方面的政策,组织路线、组织方面的政策。单有简单的口号,没有具体、细致的政策是不行的。

我说我的历史是从不觉悟到觉悟,从唯心主义到唯物主义,从有神论到无神论。如果说我一开始就是马列主义者,那是不正确约。如果说我什么都懂,也不正确。我今年七十一岁了,有很多东西不懂,每天都在学习。不学习、不调查研究,就没有政策,就没有正确的政策。可见,我并不是一开始就很完善,曾相信过唯心论,有神论,而且我打过许多败仗,也犯过不少错误。这些败仗、错误教育了我,别人的错误也教育了我。就是那些整我的人,教育了我。难道要把他们都抛掉吗?不!我们统统团结了。比如陈绍禹(王明),他还是中央委员,他相信修正主义,住在莫斯科。比如李立三,你们有人会知道,他现在还是中央委员。我们这个党,几朝领袖都是犯错误的。第一代,陈独秀,后来叛变了变成了托派。第二代,向仲发和李立三,是“左”倾机会主义。向仲发叛变,逃跑了。第三代就是陈绍禹,他统治的时间最长——四年,为什么把南方根据地统统失掉,三十万红军变成了二万五千,就是因为他的错误路线。第四代是张闻天,现在是政治局候补委员,当过驻苏大使,当过外交部副部长,后来搞得不好,相信修正主义。以后就是轮到我了。我要说明一个什么问题呢?这么四代,那么危险的环境,我们党垮了没有呢?并没有垮。因为人民要革命,党员、干部大多数要革命。有了适合情况的比较,正确的政治方面的政策,军事方面的政策,经济方面的政策,文化方面的政策,组织路线的政策,党就可以前进,可以发展。如果政策不对,不管你的名称叫共产党也好,叫什么党也好,总是要失败的。现在,世界上的共产党有一大批被修正主义领导人控制着。世界上有一百多个共产党,现在分成两种共产党,一种是修正主义共产党,一种是马列主义共产党。他们骂我们是教条主义。我看那些修正主义的共产党还不如你们,你们反对结构改革论,他们赞成结构改革论。我们和他们讲不来,和你们讲得来。

\item[\textbf{佐佐木:}] 毛主席在百忙之中,对我们进行了有意义的谈话,谢谢。

\item[\textbf{主席:}] 我讲了多久啊?两个多小时啦。

\item[\textbf{细迫:}] 谢谢毛主席进行了富于教益的谈话。上次我随铃木茂三郎来时,毛主席说没有看过孙子兵法。日本有一句谚语:“虽读论语,却不知论语之所以然。”由于毛主席贤明,所以虽然没有看过孙子兵法,但是也懂得兵法,我们是无法和毛主席相比的,\marginpar{\footnotesize 143}不过,听了主席的谈话,我想,不读马克思主义的书,也可以从我们周围许多教员那里学习。

\item[\textbf{主席:}] 特别是美帝国主义和日本的垄断资本是你们的很好的教员,逼你们想问题,开动脑筋。不过马克思主义也要读几本,修正主义的书也要读,唯心论也要读,美国实用主义也要读。不然我们就无法比较。你们如果不读结构改革论的文章和书,你们就不懂结构改革论。什么叫结构?就是上层建筑。上层建筑的第一项,根本的、主要的,就是军队。你要改革它,怎么改革?意大利人发明了这个理论,说要改革结构。意大利有几十万军警,怎么改法?第二个是国会。今天在座的许多人都是国会议员。国会,实际上是政府和垄断资本的代表占大多数。如果你们占了多数,他们会想办法的,什么修改选举法等等,它是有办法的。比如,发签证不发签证,还不是你们的政府管。你们管不了,我们也管不了。我们发,他不发。今年八月六日的禁止原子弹、氢弹的大会,有个是不是发签证的问题。并不是向你们发不发的问题,你们已经来了,还不是发了。我和你们一样,不相信结构改革论,也不相信什么三国条约。全世界差不多百分之九十以上的国家的政府都签了字,只有几个国家的政府没有签字。有时候多数是错误的,少数是正确的。四百年前,哥白尼在天文学上说地球是转动的,当时全欧洲人没有一个人相信。意大利的伽利略相信这个天文学,他也是物理学家。结果,和你(指细迫)一样,被关在监狱里。他是怎么出来的呢?签了一个字,说地球是不转动的。他刚出了班房,就说地球还是转动的。你(指细迫)没签字,你比他好。至于你对你的妻子没能照顾,那样的事多得很。我有兄弟三个,有两个被国民党杀死了。我的老婆也被国民党杀死了,我有个妹妹也被国民党杀死了。有个侄儿也被国民党杀死了,有个儿子被美帝国主义炸死在朝鲜。我这个家庭差不多都被消灭完了,可是我没有被消灭,剩下了我一个人。中国家庭被蒋介石消灭的不知有多少,整个家庭被消灭的也有。所以你(指细迫)不要悲伤,要看到前途是光明的。(大家热烈鼓掌)
\end{list}

