\section[在上海各界人士会议上的讲话(记录要点)(一九五七年七月八日)]{在上海各界人士会议上的讲话(记录要点)}
\datesubtitle{(一九五七年七月八日)}


同志们:

你们好。

今年三月下旬,就是一百天以前,我在这里同党内的同志们讲过一次话。那时我是来点火的。这一百天左右的时间内,局势有很大的变化,人民觉悟有了相当大的提高。就是打了仗嘛!当的我们是料到这些事情的,我在这里就讲过,不过当时没说过烧火这句话,而是说人家批评,我们要硬着头皮听,听一个时期以后,加以分析,加以答复,说得对的就接受,说得不对的就加以批评。我们总要相信多数人是好人。全世界也好,全国六亿人口也好,共产党也好,青年团也好,民主党派也好,知识分子也好,工商界也好,学生也好,工人、农民也好(工人、农民是我们的基本群众),多数人都是好人。他们的心都是善良的,是诚实的,不是狡猾的,不是别有用心的。所谓多数人,不是百分之五十一,而是百分之九十以上。比如这一次,以学生来说,北京大学有七千多人,教授与学生一起,右派只有百分之一、二、三。什么叫一、二、三呢?就是坚决的骨干分子百分之一,只有七十几个人。经常闹得天翻地覆的始终只有五十几个人,还不到百分之一。他们组织小团体,什么“百花学社”、“爱智者学社”、“黑格尔——恩格斯学社”、“孔夫子——毛泽东学社”,后来觉得还是不大妥当,还是叫“百花学社”。学生的“领袖”名叫谭天荣,现在是全国有名的人物了。这回可出了些英雄!左派、右派都出了英雄。

放火烧身,可不容易,听说你们这个地方有些人后悔了,没有放得厉害。我看上海放得差不多,就是有点不够,有点不过瘾。早知这么妙,何不大开放?让那些毒草长出来,让那些牛鬼蛇神出来,怕它干什么呢?那时,我们讲不要怕,可是我们党内有一些同志,如×××等,忠心耿耿,为党为国,就是怕天下大乱。就是没有看到这个大局面,就是没有估计到大多数人、百分之九十几是好人,他们是跟我们一块儿的,用不到怕。他们可以骂我们,但是他们不打我们,他们用口骂,但是不用拳头打。至于那些极少数的人,譬如刚才讲北京大学学生中,百分之一不到的右派骨干分子,另外还有百分之一、二跟他们拍拍掌,拥护他们的。教授付教授中间,情况就不同一些,大概有百分之十左右的右派,有百分之十左右是左派,这两方面旗鼓相当,中间派占百分之八十左右。对这些中间派,我们不要怕。我们有一些同志怕房子塌下来,又怕天塌下来。从古以来只有少数人怕天塌下来。就是河南的“杞人忧天”,除了他以外,所有的人都不怕天塌下来。百分之九十几的人是我们的朋友、同志,不要怕。怕群众是没有道理的。什么叫领导人物?小组长、班长、学校里的校长、教授、助教、讲师、党委书记、党委委员、支部书记,包括我们在内,都是领导人物。我们这些人,总有一点政治资本,就是替人民多少做了一点事。现在把火放起来烧,就是要把我们烧好。我们每个人都有点毛病,包括我在内。“人非圣贤,孰能无过”,所以要定期放火。以后我看至少三年一荧,五年再荧,一个五年计划里头至少放火烧二次。孙悟空在太上老君的八卦炉里头一锻炼,不就更好了吗。我们不是讲要锻炼吗?锻,是打铁,炼,是高炉里面炼铁,平炉里面炼钢,炼出来的钢还要锻。那个汽锤可厉害,我在苏联看见过了千吨的汽锤,一万吨的汽锤。我们这些人也要锻炼。人人都说要锻炼一番,平时讲锻炼舒服得很。“我有缺点,很想去锻炼一下。”但真正要锻炼,他就不干了。这回应该锻炼一番了,虽不是万吨汽锤,至少也有五千吨吧!一个时期天昏地黑,日月无光。这是二股风,一个是大多数好人的批评,我们是欢迎的。他们批评共产党的缺点,要共产党改。另外有极少数的右派,他们是向我们进攻的。多数人的进攻是应该的,攻得对,这是一种锻炼。右派进攻对我们来说也是一种锻炼。真正讲锻炼,还是要感谢右派的进攻。对于我们党、对于广大群众、各民主党派、青年学生、工人阶级、农民,右派对我们的教育最大。对这些右派,现在我们是“围剿”。每个城市都有一些右派,他们是要打倒我们的。

革命,是人民的革命,是无产阶级领导的六亿人民的革命,一个党怎么革得起来呢?民主革命是人民的事业,社会主义革命是人民的事业,社会主义建设是人民的事业。他们要否定人民事业的成绩,这是第一。第二,走什么方向,走社会主义还是走资本主义这个方向?第三,要搞社会主义谁来领导?是无产阶级领导,还是资产阶级领导?无产阶级那么多,先锋队是共产党。资产阶级也有一群,它也组织政党。是共产党领导。还是右派领导?共产党好不好?要不要?人民说要,右派说不要。我看在三个问题上进行一场大辩论,很好。革命对不对?建设对不对?有没有成绩?成绩是不是主要的?还是错误是主要的?这是第一个问题。现在不是展开大辩论吗?这个问题是没有辩论过的。民主革命是经过长期辩论的,民主革俞从清朝的末年起,经过辛亥革命,到反袁世凯,到北伐战争、抗日战争,一直是在辩论的,抗日战争时,要不要抗日?也经过辩论的,一派人说不能抗,因为中国的枪不够,这是唯武器论。另一派人说,不怕,还是人为主,武器不如人,我们还是可以打,以后重庆谈判,旧政协,南京的谈判,这都是辩论。蒋介石一刻都不停地要打,打的结果是他输了。所以,那一场民主革命是经过辩论的。是经过长时期精神准备的。社会主义革命是短促突击,在六、七年之内,社会制度的改革已基本上完成,人的改造也改造了一些,但还差。社会主义改造有两方面,一方面是制度的改造,一方面是人的改造。制度不但是所有制,而且有上层建筑,就是政府、政府机关、意识形态。譬如报纸是属于意识形态的。有人说报纸没有阶级性,报纸不是阶级斗争的工具。这句话讲得不对了。至少在这几十年内,全世界帝国主义没有消灭以前,这样讲不好的。报纸以及别的东西,如哲学、意识形态,它们都是反映阶级关系的。学校、教育事业、文学艺术都是意识形态、上层建筑。自然科学分两部分,纯自然科学,它是不分阶级的,但是利用自然科学,谁利用自然科学,这是有阶级性的。北京大学“百花学社”的首领谭天荣,他就是物理系四年级学生。现在讲物理学的人唯心论可多啦!一个中文系,一个历史系,唯心论最多,办报纸的唯心论也最多。你们不要以为只是这一些,社会科学这一方面,哲学、政治经济学方面的唯心论也多,而且自然科学里头也有许多唯心论,他们的世界观是唯心的。你若说水是什么东西构成的,那他是唯物论,水是两种元素构成的,他是照实际情形办事的。你讲社会怎么改造,共产党怎么整风,他就要消灭共产党。我们说要整好共产党,他说要消灭共产党。当时我们的政策是这样的,就是只听不说,说几个星期内硬着头皮,但把耳朵扯长一些,自己一句不说,我们也不通知党员,也不通知支部书记,也不通知支部干事会,也不通知团员,让他们混战一场,各人自己打主意。清华大学党委会内就有敌人,你这里一开会,他就报告敌人了,叫做“起义”分子,共产党员“起义”这一件事,两方面都高兴。北京大学学生党员里头崩溃了百分之五,团内崩溃的多一些,也许百分之十,或者还多一些。这些崩溃我说是天公地道。百分之十也好,百分之二十也好,百分之三十也好,百分之四十也好,总而言之,崩溃了,我们高兴就是了。那种资产阶级思想,唯心论,钻进共产党、共青团,名为共产主义,实际上是反共产主义,或者是摇摆分子。他们“起义”我们高兴,不要我们清理,他自己跑出去了。敌人方面,也很高兴。我们把右派一包围,事情就反过来了,许多跟右派有联系,但并非右派的人来揭露他,不是起义了吗?还有一些右派也要起义的。现在右派也不好混了。

几个月以前我在这里讲话,到今天不到一百天,时局起这么大的变化。这个斗争主要是政治斗争,斗争的性质是阶级斗争。有各种形式的阶级斗争,这一次主要是政治斗争,不是军事斗争,不是经济斗争。思想斗争成份有没有?有,但政治斗争占主要的成份。思想斗争还在下一阶段,要和风细雨,共产党整风,青年团员也整风,经过思想斗争,提高一步,真正学习马列主义,真正的互相帮助。主观主义有没有?官僚主义有没有?真正用脑筋想一想,写点笔记,搞那么几个月,就把马克思主义水平、政治水平,思想水平提高一步。

斗争还要个把月,右派分子尽是在报纸上登,今年登一年,明年登一年,后年登一年,那不好办事了,也没有那么多东西登了,右派就是那么多,登得也差不多了,以后就阴登一点,阳登一点,有就登一点,没有就不登。七月,是反右派紧张的一个月,过了七月到了八月就要和风细雨了。右派最喜欢急风暴雨,最不喜欢和风细雨,而我们主张和风细雨的。他们说共产党不公道,你们从前整我们是急风暴雨,现在你们整自己就和风细雨了。即使我们从前搞的思想改造,包括批评胡适、梁漱溟在内,我们党内下的指示,都是要和风细雨的。世界上的事情,总是曲折地前进的。社会的运动总是采取螺旋形前进的。对右派要挖,现在还要挖,不能松劲。这个时候的右派,那里有一根草,他就想抓了,因为他要沉下去了。现在他才晓得和风细雨的好处。以前他们要来一个急风暴雨,说和风细雨,天天下黄梅雨,秧菽烂掉了,粮食就没有了,就要闹灾,不如急风暴雨简便。现在是夏季,是暴雨天,到了八月,可以和风细雨了,因为没有多少东西可挖了。

我们中国历来受到两方面的教育,正面的教育跟反面的教育。日本帝国主义又是第一个大好“教员”,从前有清朝、有袁世凯、有北洋军阀,以后有蒋介石,都是我们很好的“教员”。没有他们,中国人民教育不过来。单是共产党来当教员不够。我们有许多话,中间人士不听,要另搞一套。譬如“团结——批评——团结”他就不听。譬如讲肃反成绩是主要的,他又不听。譬如讲民主集中制,他又不听。讲无产阶级领导的人民民主专政,他又不听。讲要联合苏联和社会主义国家以及爱好和平的各国人民,他又不听。还有一条,他特别不听,就是说“毒草要锄掉。”毒草要锄掉,牛鬼蛇神让他出来。让大家展览,展览之后,大家认为这些牛鬼蛇神不好,要打倒。毒草长出来,就要锄,农民每年都锄草。锄掉可以做肥料。这些话讲过没有?还不是讲过吗?可是毒草还是要出来。农民每年锄草,就是跟他讲话,可是,草根本不听的,它明年还要长。锄了一万年,一万年还要长,一万万年,年年要长草,认为毒草是我们,他自己是香花。因此,他并非被除之列。他要把我们锄掉,他就没有想到他正是应该锄的东西。

社会主义来的急促。总路线各方面都学习过,但没有辩论。党内没有辩论,社会上也没有辩论,像牛吃草一样,先吃下去,然后慢慢再回头来嚼。我们的革命,在制度方面已经基本上改革过了,首先是经济基础,就是生产资料的所有制;第二是上层建筑,就是权力机关,意识形态等,这些都基本上改了,但是没有展开辩论,这回经过报纸,经过座谈会,经过大会,经过大字报,展开大辩论。大字报是个好东西,我看要传下去。你看孔夫子的《伦语》传下来了,《五经》、《十三经》传下来了,《二十四史》传下来了,“十五贯”也传下来了,大字报我看也要传,譬如讲,工厂里整风,我看用大字报好,越多越好。如果是一万张,那是头等,如果是五千张,就是二等,如果只有两千张,就是三等,如果稀稀拉拉只有几张,吃丁等。大字报是没有阶级性的。等于语言没有阶级性。白话文没有阶级性,无产阶级也讲白话,资产阶级也讲白话,无产阶级也有话剧,资产阶级也有话剧,汉奸也有话剧,抗日时期也有话剧。无产阶级可以用大字报,资产阶级也可以用大字报,我们相信多数人是站在无产阶级这一边的,因此,大字报这个工具,是有利于无产阶级的,并不利于资产阶级。一个时候天昏地黑,日月无光,好像是利于资产阶级。所谓一个时候,是两个星朗、三个星期,只有那么一点。所谓硬着头皮,也就是那么二、三个星期,睡不着觉,吃不下饭。你不是讲要锻炼吗?人生在世有几个星期睡不着觉,吃不下饭,这就是锻炼。并非真的把你送到高炉里去烧。有许多中间人士动摇一下,这也很好,动摇一下,他得到经验。中间派的特点就是动摇,不然为什么叫中间派?一头是无产阶级,另一头是资产阶级,还有很多中间派。两头小,中间大,但是归根结蒂中间派都是好人,是无产阶级的同盟军。资产阶级想争取他们做同盟军,一个时候有点像。无产阶级争取他们做同盟军,一个时候也有点像。中间派也批评我们,但是,是好心的批评。右派的批评是借着这个事来捣乱。中间派就搞糊涂了。刚才讲大字报,是方式的问题,是作战的武器之一,像步枪、短枪、机关枪,是轻武器,像文汇、光明日报,还有些别的报纸,是飞机大炮。光明日报,文汇报这次得了很深刻的教育,过去他们不知道什么是无产阶级报纸,什么是资产阶级报纸;什么叫社会主义报纸,什么叫资本主义报纸,分不清楚。一个时候即使分的清楚,可是这些报纸的领导人要把它办成一个资产阶级报纸。他们仇恨无产阶级报纸,仇恨社会主义报纸。一个学校把学生引导到社会主义方向还是把学生领导到资本主义方向?工商界还是把这些工商业者(大、中、小资本家)引向无产阶级方向,还是把他们引向资产阶级方向?要不要改造?有人非常怕这个改造,说改造就有那么一阵自卑感,越改造越自卑。我看不应该这么解释。应该是越改造越自尊,应该说是自尊感,因为自己有觉悟,才要改造。有些人,自认为有很高的阶级觉悟,认为自己不要改造,相反要改造无产阶级,要按照他们的面貌来改造这个世界,而无产阶级要按照自己的面目来改造。我看多数人百分之九十以上是愿意改造的,当然,中间要经过踌躇、考虑、不断犹豫、摇摆的过程。越改造,他就越觉得要改造。共产党整风就是改造。将来还要整风,三年一整,五年再整风,你说整了这一次风就不整了。难道整了这次风,就没有官僚主义了?只要过了二、三年,他都忘记了,官僚主义又来了。人就容易忘记。所以过了一个时候还要整。资产阶级旧社会过来的知识分子。难道就不需要整风?不要改造?你说不要改造,调个名字,叫整风也可以。现在各民主党派不在整风吗?整个社会整整风,为什么不好?

现在民主党派是整路线问题,整资产阶级的右倾机会主义路线。我看整得对了。共产党不是路线问题,而是作风问题。民主党派现在作风问题在其次,主要是走那一条路,走章乃器、章伯钧、罗隆基、陈仁炳、彭文应、陆治、孙大雨那条路线,还是走什么路线?首先要把这个问题搞清楚。这三个问题要搞清楚:革命的成绩、建设的成绩问题、几亿人做的事,究竟做得好不好?将来的方向,是社会主义还是资本主义?要走社会主义的话,那么要受那一个党的领导?还是要章、罗同盟来领导?还是要共产党来领导?来它一个人辩论,把路线搞清楚。共产党内有一个路线问题,就是那些“起义分子”。那些“起义分子”是共产党、青年团里头的右派,对他们来说这是个路线问题。教条主义现在不是个路线问题,因为它没有形成路线。我们历史上有一次教条主义是形成路线的,因为它形成制度、形成政策,形成纲领。现在的教条主义没有形成制度、政策、纲领,它是有那么一些硬性的东西,现在打上一锤子,火这么一烧,它也就软了一点。各个机关里头,学校里头,工厂里头不是在讲下楼吗?不要国民党作风、老爷习气,合作社主任下田同群众一起耕田,工厂的厂长、党委书记到车间里头去,这样官僚主义就大为减少。出大字报、开会、开座谈会,把应该改正的,应该批评的问题,分类来解决。再学点马克思主义,提高一步。

我们中国民族,是个好民族。这个民族是很讲道理的,很热情的,很聪明的,很勇敢的。我们希望造成这么一种局面,就是既集中统一,又是生动活泼;有民主,又有集中;有自由,又有纪律。两方面都有,不只是一方面。不要把人家的嘴巴封住,不准人家讲话。应该提倡讲,应该生动活泼。对大多数人是言者无罪,不管你怎么样尖锐,怎么痛骂一阵,也没有罪。不受整,不给穿“小鞋”。“小鞋”要给右派穿。不要怕群众,要跟群众在一起,你们游水不游水?只要一百天,每天一小时,不间断地搞,你本来一点不会游水的,保证你会游水,一不要先生,也不要那个橡皮圈,有了橡皮圈就学不会,人民就像水一样的,打比方,领导者从各级小组长起一直到我们这些同志,就是像游水的人一样,不要离开水,不要逆那个水,你要顺那个水,顺着水性。不要去骂群众,群众是不能骂的。不要和群众对立,总要跟群众一道。群众也可能犯错误,犯错误的时候要好好讲,他不听,你就等一下,有了机会再讲,就是不要脱离他。等于我们游水不要脱离水,不要逆水,要顺着水性。

智慧都是从群众那里来的。我历来讲,知识分子是最无知的。知识分子把尾巴一翘,认为老子不算天下第一,也算天下第二。工人、农民算什么?你们这些阿斗,又不认得几个字。可是决定问题的,不是知识分子而是劳动者,是劳动人民中最先进的一部分,就是工人阶级决定问题。无产阶级领导资产阶级?还是资产阶级领导无产阶级?无产阶级领导知识分子?还是知识分子领导无产阶级?知识分子应该成为无产阶级知识分子,没有别的出路。我说过“皮之不存,毛将焉附?”这是上海有个资本家讲的,我是引伸他的话。他讲的跟我讲的意思不同。他说自己的东西都交出去了,公私合营了,“皮之不存,毛将焉附?”还说我是资本家?还说我是剥削者?知识分子从旧社会中来,就是吃五张皮的饭。过去知识分子的毛,是附在这五张皮上面,帝国主义所有制,封建主义所有制,官僚资本主义所有制,还有民族资本主义所有制,小生产所有制。过去或者附在前三张皮上,或者附在后两张皮上,现在是“皮之不存”皮没有了,帝国主义跑了,东西都拿过来了,封建主义打倒了,土地归农民,现在归合作社了;官僚资本主义企业归国有了,民族资本主义企业公私合营了,基本上变成社会主义了;小生产(农民、手工业者)所有制现在也改变了,变为集体所有制了。虽然现在还不巩固,还要几年,才能巩固下来。尤其是人的改造,人的改造时间更要长一些了,因为这五张皮影响着这些资本家,影响着这些知识分子,他们脑筋里老是记着这些东西,做梦也记着。旧轨道过来的人,就是留恋那个旧生活习惯,这是人之常情。现在知识分子附在什么皮上?就是附在公有制这个皮上,附在无产阶级身上。谁给他饭吃?就是工人、农民。知识分子是无产阶级请的先生,可是你要教你的那一套,要教八股文,要教孔夫子,要教资本主义,让你吃饭拿薪水,那工人阶级是不干的。知识分子已经丧失了社会经济基础,也就是那五张皮没有了,现在他除非落在新皮上。现在有些知识分子在天上飞。十五个吊桶打水,七上八下。上不着天,下不着地,在空中飞。五张皮没有了,老家回不去了。可是他又不甘心情愿附在无产阶级身上。要附在无产阶级身上,就要有无产阶级思想,要跟无产阶级有点感情,要跟工人搞好,要拉朋友。可是他不干,他还想那个旧的东西。我们现在劝他们,经过这一场大批评,我看他们多少会觉悟的。我们现在在劝中间形态的人,中间形态的人应该觉悟,尾巴不应该翘得太高。你的知识是有限的,是知识分子,又不是知识分子,叫半知识分子比较妥当。因为你的那个知识只有那么多,讲起大道理来你就犯错误。你那么多知识,为什么犯错误?为什么动摇?“墙上一根草,风吹两边倒”,你为什么动摇?现在不去讲右派的知识分子,那是根本错误的。中间派知识分子也犯错误,他犯的错误就是动摇,看不清楚方向,一个时候迷失方向,头脑不清醒,可见你知识不太多。在这个方面知识多的是工人。农民里头的过去的半无产阶级。他一看就知道孙大雨这一套东西,他一看就知道不对。只要谈三句话,他就知道不对。用不看写这么长的文章。你看谁的知识高?还是那个不识字的人知识高,决定大局,决定大方向是听无产阶级的。我就是这么一个人。我们这些人办什么事,要决定什么大计,就非问他们不可,就非到各个地方跑一跑,跟他们谈一谈,看此事能行不能行,跟他们商量,以及跟他们接近的干部商量,就要到地方上来。北京是什么东西也不出的,他没有原料,原料都是工人、农民那里拿去的,都是地方拿去的。中共中央是一个加工厂。就是把这些原料制造好,制造不好就要犯错误。知识的来源,是出于群众,归根到底是群众路线四个字。什么叫真正解决人民内部矛盾?就是实事求是,群众路线。就是讲不要脱离群众的。像鱼跟水的关系,游泳者跟水的关系一样。

右派是不是要一棍子打死,孙大雨那些人怎么办?打他几棍子是必要的。攻得他想回头,切实地攻,使得他完全孤立,那就有可能争取他们。因为他们还是知识分子,而且是大知识分子,这样的人争取过来是有用处的,多少可以做点事情。而且他们这一回头又帮了大忙,给我们当了“教员”,教育了人民。他是以反面的方法,从反面教育了我们。我们并不准备把他拋到黄浦江里,还是要治病救人。也许这些人是不愿意过来的,他不愿意过来那末也好,那就带到棺材里头去,孙大雨现在多少年纪?算他活一百岁,还有五十年他坚决不改,巩固得很,这个堡垒攻不破,也就算了。尽攻他,我们没有那么多的气,我们现在要办事,天天攻,攻他五十年,那怎么了得。有那么一部分人不肯改,就让他带到棺材里面去见阎王,他可以跟阎王说:“我可是有骨气,我是‘五张皮’的坚持者。我眼这些王八蛋、共产党、中国人左翼、同广大群众作过斗争,要我检讨,我都抵抗过来了。”可是现在阴间的阎王也改了,阎王第一个是马克思,第二个是恩格斯,第三个是列宁。现在分两个地狱,资本主义世界地狱是老的,社会主义世界的地狱就是这些人当阎王。我看这些人到阎王那里也是要挨整的。

讲的多了,不讲了。谢谢你们听我这个讲话。

<p align="right">中共上海市委办公厅整理一九五七年七月十日

(据新北大“雷达”战斗队翻印稿翻印)</p>


