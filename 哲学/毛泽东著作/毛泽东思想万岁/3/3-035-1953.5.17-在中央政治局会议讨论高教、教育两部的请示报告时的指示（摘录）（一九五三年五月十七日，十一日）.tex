\section[在中央政治局会议讨论高教、教育两部的请示报告时的指示(摘录)(一九五三年五月十七日,十一日)]{在中央政治局会议讨论高教、教育两部的请示报告时的指示(摘录)(一九五三年五月十七日,十一日)}
\datesubtitle{(一九五三年五月十七日)}


毛主席指出,办好学校,首先要解决学校的领导骨干问题,而且首先要解决大学的领导骨干问题。同意总理意见,从宣传部门与青年团抽调一批干部,去充实大学的领导。关于中、小学的领导骨干,应该由地方在几年内逐渐加以解决。毛主席认为训练与提高中学领导骨干很重要,东北的经验是好的。毛主席说:有了坚强的校长,就会产生好的教员。

毛主席对教材问题特别重视。认为目前三十个编辑太少了,增加到三百人也不算多。宁可把别的摊子缩小一点,必须抽调大批干部编教材。确定补充一百五十个编辑干部,由中央组织部负责解决配备。

所谓教学改革,就是教育内容与教学方法的改革。因此应该改编教材,编辑教学法。

关于整顿小学问题,毛主席认为,“整顿巩固、保证质量、重点发展、稳步前进”的方针很好,但是整不要整过了头。

××同志认为,中国政治,经济发展不平衡,教育也发展不平衡,不可能把全国小学都办成一样,不可能整齐划一,不应该过分强调正规化。农村小学可分为三类:一,中心小学;二,不正规的小学;三,速成小学,编速成教材。农村小学应该便于农民子女上学。毛主席同意××同志的意见,并指出应该容许私塾式、改良式、不正规的小学存在。只要不是现行反革命分手,只要不进行反革命宣传,就可以给他一本教科书去教。甚至和尚、尼姑,只要不宣传迷信,也可以让他们教书。

用多种多样办法办学,不强求一律。

关于超龄生问题,中央同志都认为不能把超龄生送出学校。毛主席说:“你们对七、八岁儿童这么爱好,对超龄生就可以不管?”并且说:“应首先把超龄生教好,让七、八岁孩子迟几年上学并不大要紧。”这就告诉我们先教育超龄生,对发展生产有利。适龄儿童迟几年上学,影响不太大。

小学城乡比例一样,不对。城市小学比例应该大些。

关于小学经费问题,毛主席指出,税种税率必须统一,收支也必须由国家统一掌管。不能像过去那样,搞什么教育经费独立,指定一定的税作教育经费。毛主席同意中央统一支拨,预算由省(市)县拟造。

民办小学,毛主席说,可以容许。不限定几年,能办几年就办几年,横直公家不出钱。学田,也可以允许留,但是要在有机动土地的地方,并有可能拿出来的条件下,才能给学田。

学费也可以收。

小学教育,应该强调劳动教育。关于五年一贯制问题,毛主席认为既(然)贯不下去,只好缓行。五年一贯制的文件,并不错,只是实行过早,应该推迟。

毛主席对青年健康,非常注意,指示增加大、中学生助学金。学生健康不好的原因,是伙食不好,卫生不好,功课重,课外负担过重,太忙。毛主席认为要增进学生健康,要增加营养,搞好卫生,减少负担,少紧张些。毛主席说,要吃得饱,学得太多,可少学点,要克服忙的现象。要一方面增加收入,一面减少消耗。因此,要增加助学金。改善伙食。另一方面克服忙乱现象。因此,决定大学生增加到十七万元,高中生除少数有钱子女外,一律发给助学金。初中生也增加助学金。

关于设置重点中学问题,毛主席认为要办,具体数字要按实际情况办。可以研究一下。留苏预备学校十所,要分配到各大区。办留苏预备学校,并不是说排除其他学校学生投考留苏;同时,留苏预备学校学生也并不一定全体一律留苏,还要经过考试和审查。重点学校一百五十至二百所。

毛主席指出:“培养工人出身的干部要特别着重一些。”周总理指示,可以从部队机关抽调干部学习。

初中招生,毛主席和××同志都认为,能多招一些,就多招一些。今年还是招六十五万人。

<p align="right">(根据前教育部党组一位同志记录,未经核对)</p>


