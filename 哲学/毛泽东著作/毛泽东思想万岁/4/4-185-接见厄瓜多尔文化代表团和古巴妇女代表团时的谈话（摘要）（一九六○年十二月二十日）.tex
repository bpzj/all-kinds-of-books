\section[接见厄瓜多尔文化代表团和古巴妇女代表团时的谈话(摘要)(一九六○年十二月二十日)]{接见厄瓜多尔文化代表团和古巴妇女代表团时的谈话(摘要)(一九六○年十二月二十日)}


(厄瓜多尔朋友谈到“中国有几千年的文学遗产,应该利用它们为社会主义服务。”)

应该充分地利用遗产,要批判地利用遗产。所谓几千年的文化,是封建时代的文化,但并不全是封建的东西,有人民的东西。有反封建的东西。要把封建主义的东西与非封建主义的东西区别开来。封建主义的东西也不完全是坏的,也有它发生、发展和灭亡的时期。我们要注意区别发生、发展和灭亡时期的东西。当封建主义还在发生和发展的时候。它有很多东西还是不错的。反封建主义的文化,也不是全部可以无批判地利用的.封建时代的民间作品.也多少带着若干封建统治阶级的影响,你说是不是?

我们应当善于分析。应当把封建主义发生,发展和灭亡时期的文化分别出来。应当批判地利用封建主义文化。我们不能不批判地加以利用。反封建主义的文化当然要比封建主义的好,但也要有批判、有区别地加以利用。我们了解的是这样,我们现在的方针是这样。至于充分地利用它,我们现在还没有做到。古典著作多得很。现在是分门别类地去整理。重新再版,用现代科学观点逐步整理出来。

(一位外国朋友谈到。“我们感到惊奇的是,看到中国的画家在抄袭西方的画法。”)

这种抄袭已经有几十年、近百年了。特别是抄袭欧洲的东西。他们看不起自己国家的文化遗产,拚命地去抄袭西方。我们批评这个东西也有一段时间了。这个风气是不好的。

不单是绘画,还有音乐,都有这样一批人抄袭西方。他们看不起自己的东西。文学方面也是如此,但要好一些。在这方面我们进行过批评。批评后小说好些,诗的问题还没有解决。我很久未接触艺术界了。你们(指楚图南同志和×××)和他们谈谈吧。

(古巴代表团团长。“想问一个问题,中国与古巴签订的文化贸易协定,对两国之间有什么利益?”)

就是友好,互相支持。你们支持我们,我们支持你们。你们的斗争支持我们。我们的斗争支持你们。整个拉丁美洲各国罢工、示威,反帝反封建,我们对此都很高兴。

刚才讲的文化方面的问题。各国人民应该根据民族特点,有所贡献。有共同点,但也有差别。共同点是同一时代,都处于二十世纪下半个世纪,总有共同点。但是如果都画一样的画,唱一样的曲调,千篇一律就不好,就没有人看,没有人听,没有人欣赏。……普遍性就在特殊性里。

(毛主席在接见两个代表团后,还向楚图南同志和×××作了如下指示.)

你们应该和××谈一下,我们的艺术如绘画、音乐、文学等,在创作方法上,要向外国学。但不是主要的。我们大多数人都应该学我们自己的东西,有这么几个人学外国的就行了。这是个老问题,说了也很久了。但始终也没有解决。你们和××谈一下。这个问题要解决。


