\section[中央政府关于查田运动的训令(一九三三年六月一日)]{中央政府关于查田运动的训令}
\datesubtitle{(一九三三年六月一日)}


现在各苏区,尤其是中央苏区尚有广大区域没有彻底解决土地问题,这种区域在中央区差不多占百分之八十的面积,群众在二百万以上,为瑞金(除武阳区)、会昌、寻邬、安运、信丰、云都(除新陂区)、乐安、宜黄、广昌、石城、建宁、黎川、宁化、长汀、武平十五个全县,博生、胜利、永丰的大部分,公路、万大、上杭、永定、新泉的一部分,兴国也还有均材,黄塘两区,所有这些,都是没有彻底解决土地问题的地方,这些地方的农民群众还没有最广泛地发动起来。封建残余势力还没有最后的克服下去,苏维埃政府中,群众团体中,地方武装中还有不少阶级异己分子在暗藏活动着,还有不少的反革命秘密组织在各地活动破坏革命。为了这个原故,这些地方的战争动员与文化经济建设都落在先进区域(兴国差不多全县、胜利、赣县、万大、公略、永丰、上杭的一部分,博生的黄坡、安枢区,瑞金的武阳区,永定的溪南区等等)之后。在这个广大区域内进行普遍的深入的查田运动,在二百万以上群众中发展最高度的阶级斗争,向着封建势力作最后一次的进攻而把他们完全消灭,是各地苏维埃一刻不能再缓的任务。关于查田运动的具体进行事项,人民委员会特决定如下:

(一)责成各级政府主席用最大注意去领导整个查田运动。

(二)责成各级土地部,检察部、裁判部、国家政治保卫局及特派员用他们的全力于查田运动中,彻底解决土地问题,改造地方苏维埃,肃清农村中的反革命,中央土地人民委员部、工农检察人民委员会部,司法人民委员部、国家政治保卫局,应用全力指导各下级机关,切实完成这些任务。

(三)责成中央财政人民委员部指导各级财政部,从地主罚款富农捐款去进攻封建半封建势力,同时增加国家的收入。责成中央军事人民委员部指导各级军事部在查田运动中整顿与扩大地方武装,动员群众参加红军,责成中央国民经济人民委员部指导各级国民经济部从查田运动的发展中去进行农业手工业生产的恢复与发展,合作社的发展与生产品消费品的调济,责成中央教育人民委员部指导各级教育部为着开展查田运动供给各种简明通俗的课本与小册子给一切查田干部与群众,应跟着查田运动的发展去发展群众的文化教育。

(四)省县两级政府应该从查田区域及一切先进的较先进的区域征调干部开办短期的查田运动训练班。县苏区应每月召集区苏负责人开会,区苏应每十天召集乡苏主席贫农团主任开会,检阅查田运动的经过。

(五)应首先召集瑞金、会昌、博生、云都、胜利、石城、宁化、长汀八县以上苏维埃主要负责人会议及八县贫农团代表大会在中央政府开会,发动八县的查田运动。

(六)查田运动要坚决执行阶级路线,以农村中工人阶级为领导,依靠贫农,坚决联合中农,向着封建半封建势力作坚决的进攻,把一切冒称“中农”“贫农”的地主富农完全清查出来,没收地主阶级的一切土地财产,没收富农的土地及多余的耕牛、农具、房屋,分配给过去分田不够及尚未分到田的工人、贫农、中农,富农则分与较坏的劳动分田。

(七)查田运动中要充分注意发动群众的大多数起来向着封建残余作斗争,首先要经过广泛的宣传与鼓动一切调查地主富农成分,通过这些成分,没收这些成分的土地财产,均要经过尽可能多数群众的同意与参加。没收来的财物除现款外,应全部发给极贫苦的群众,特别注意发给贫苦的红军家属,并要多发给于财物所出的村庄的群众。

(八)贫农团是查田运动中极重要的群众团体。乡苏维埃要极力指导贫农团,洗刷其中的坏分子,吸收多数积极分子加入,贫农团中的工人小组应是贫农团的积极领导者。

(九)查田运动中要集中大力注意于一切落后的尤其是最落后的区乡及村子。要注意在落后的区多,尤其在大村子中开展查田远动,必须发动本村贫农群众自己起来与本村地主富农斗争,要极力避免可能引起民族地方斗争的一切错误行为。

(十)要在查田运动中肃清一切反革命的组织与活动,要防止揭破地主富农的造谣与破坏。

(十一)要在查田运动中改造地方苏维埃,洗刷地方苏维埃中一切阶级异己分子及其他坏分子出来,引进大批革命积极分子进苏维埃来。

(十二)各级苏维埃中一切直接间接阻碍查田运动的人,都应该受到严厉的惩罚。

只有坚决执行上述的决定,才能广泛而深入的开展农村中的阶级斗争,激发最广大群众的积极性,彻底消灭农村中一切封建半封建势力,完成查田运动的任务。此令。

<p align="right">主席毛泽东

付主席项×张××</p>

