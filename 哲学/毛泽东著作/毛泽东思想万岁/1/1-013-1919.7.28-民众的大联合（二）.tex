\section{民众的大联合(二)}
\datesubtitle{(一九一九年七月廿八日)}



以小联合作基础

上一回本报,已说完了“民众的大联合”的可能及必要。今回且说怎样是进行大联合的办法?就是“民众的小联合”。

原来我们想要有一种大联合,以与立在我们对面的强权者害人者相对抗,而求到我们的利益。就不可不有种种做他基础的小联合,我们人类本有联合的天才,就是能群的天才,能够组织社会的天才。群和“社会”就是我所说的“联合”。有大群,有小群,有大社会,有小社会,有小联合,有大联合,是一样的东西换却名称。所以要有群,要有社会,要有联合,是因为想要求到我们的共同利益,共同的利益因为我们的境遇和职业不同,其范围也就有大小的不同。共同利益有大小的不同,于是求到共同利益的方法,(联合)也就有大小的不同。

诸君!我们是农夫。我们就要和我们种田的同类,结成一个联合,以谋我们种田人的种种利益。我们种田人的利益,是要我们种田人自己去求。别人不种田的,他和我们利益不同,决不会帮我们去求。种田的诸君!田主怎样待遇我们?租税是重是轻?我们的房子适不适?肚子饱不饱?田不少吗?村里没有没田作的人吗?这许多问题,我们应该时时去求解答。应该和我们的同类结成一个联合,切切实实彰明较著的去求解答。

诸君!我们是工人。我们要和我们做工的同类结成一个联合,谋我们工人的种种利益。关于我们做工的各种问题,工值的多少?工时的长短?红利的均分与否?娱乐的增进与否?……均不可不求一个解答。不可不和我们的同类结成一个联合,切切实实彰明较著的去求一个解答。

诸君!我们是学生,我们好苦,教我们的先生们,待我们做寇仇,欺我们做奴隶,闲镇我们做囚犯。我们教室的窗子那么矮小光线照不到黑板,使我们成了“近视”,桌子太不合式,坐久了便成“脊柱弯曲症”,先生们只顾要我们多看书,我们看的真多,但我们都不懂,白费了记忆。我们眼睛花了,脑筋昏了,精血亏了,面血灰白的使我们成了“贫血症”’成了“神经衰弱症”。我们何以这么呆板?这么不活泼?这么萎缩?呵!都是先生们迫着我们不许动,不许声的原故。我们便成了“僵死症”。身体上的痛苦还次,诸君!你看我们的实验室呵!那么窄小!那么贫乏--几件坏仪器,使我们试验不得。我们的国文先生那么顽固,满嘴里“诗云”“子曰”,清底却是一字不通。他们不知道现今已到了二十世纪,还迫着我们行“古礼”守“古法”,一大堆古典式死尸式的臭文章,迫着向我们脑子里灌,我们板书室是空的,我们游戏场是秽的。国家要亡了,他们还贴着布告,禁止我们爱国,象这一次救国运动,受到他们的恩赐其多呢!唉!谁使我们的身体,精神,受摧折,不愉快?我们不联合起来,讲究我们的“自教育”,还待何时!我们已经陷在苦海,我们要求讲自救:卢梭所发明的“自教育”,正用得着。我们尽可结合同志,自己研究。咬人的先生们,不要靠他。遇着事情发生一一象这回日本强权者和国内强权者的跋扈一一我们就列起队伍向他们作有力的大呼。

诸君!我们是女子。我们更沉沦在苦海!我们都是人,为什么不许我们参政?我们都是人,为什么不许我交际?我们一窟一窟的聚着,连大门都不能跨出。无耻的男子,无赖的男子,拿着我们做玩具,教我们对他长期卖淫,破坏恋爱自由的恶魔!破坏恋爱神圣的恶糜,整天的对我们围着,什么“贞操’却限于我女子,“烈女嗣”遍天下,“贞童庙’又在那里?我们中有些一窟的聚重在一女子学校,教我们的又是一些无耻无赖的男子,整天说什么“贤妻良母”,无非是教我们长期卖淫专一卖淫。怕我们不受约束,更好好的加以教练,苦!苦!自由之神,你在那里,快救我们!我们于今醒了!我们要进行我们女子的联合!要扫荡一般强奸我们破坏我们精神自由的恶魔!

诸君,我们是小学教师,我们整天的教课,忙的真很!整天的吃粉条屑,没处可以游散舒吐。这么一个大城里的小学教师,总不下几千几百,却没有一个专为我们而设的娱乐场。我们教课,要随时长进学问,却没有一个为我们而设的研究机关。死板板的上课钟点,那么多,并没有余时,没有余力,一一精神来不及!一一去研究学问。于是乎我们变成了留声器,整天演唱的不外昔日先生们教给我们的真传讲义。我们肚子是饿的。月薪十元八元,还要折扣,有些校长先生,更仿照“克减军粮”的办法,将政府发下的钱,上到他们的腰包去了。我们为着没钱,我们便做了有妇的鳏夫。我和我的亲爱的妇人隔过几百里几十里的孤住着,相望着,教育学上讲的小学教师是终身事业,难道便要我们做终身的鳏夫和寡妇?教育学上原说学校应该有教员的家庭住着,才能做学生的模范,于今却是不能。我们为着没钱,便不能买书,便不能游历考察。不要说了!小学教师直是奴隶罢了,我们想要不做奴隶,除非联结我们的同类,成功一个小学教师的联合。

诸君!我们是警察。我们也要结合我们同类,成功一个有益我们身心的联合。日本人说,最苦的是乞丐,小学教员和警察,我们也有点感觉。

诸君!我们是车夫,整天的拉得汗如雨下!车主的赁钱那么多!得到的车费这么少!何能过活,我们也有什么联合的方法么?

上面是农夫、工人、学生、女子、小学教师、车夫、各色人等的一片哀声。他们受苦不过,就想成功于他们利害的各种小联合。

上面所说的小联合,象那工人的联合,还是一个很大很笼统的名目,过细说来,象下列的:

铁路工人的联合,

矿工的联合,

电报司员的联合,

电话司员的联合,

造船业工人的联合,航业工人的联合,

五金业工人的联合,

纺织业工人的联合,

电车夫的联合,

街车夫的联合,

建筑业工人的联合,

方是最下一级联合,西洋各国的工人,都有各行各业的小联合会,如运输工人的联合会,电车工人联合会之类,到处都有,由许多小的联合,进为一个大联合,由许多大的联合,进为一个最大的联合。于是什么“协会”,什么“同盟”,接踵而起,因为共同利益只限于一部分人,故所成立的为小联合,许多的小联合彼此间利益有共同之点,故可以立为大联合,象研究学问是我们学生分内的事,就组成我们研究学问的联合厶象要求解放要求自由,是无论何人都有分的事,就应联合各种各色的人,组成一个大联合。

所以大联合必要从小联合入手,我们应当起而仿效别国的同胞们,我们应该多多进行我们的小联合。

\begin{flushright}《湘江评论》第三期一九一九年七月二八日出版\end{flushright}

