\section[在延安大学开学典礼上的讲话(报导)(一九四四年五月二十四日)]{在延安大学开学典礼上的讲话(报导)}
\datesubtitle{(一九四四年五月二十四日)}


“现在边区教育已经开始走上轨道,而这是与边区各个抗日根据地工作的进步有联系的。”

“所有我们一切工作,只有一个目标,就是打倒日本帝国主义。要把日本打出去,没有根据地就不行。”

“今后延大的具体任务:在政治上要学习统一战线,三三制,精兵简政的方针,要学习各政策与方法。在经济上要学习如何发展工业、农业、商业、运输;要帮助三十五万家农民做到耕三余一,要帮助老百姓订一个植树计划,十年内要把历史遗留给我们的秃山都植上树,还要使边区工业做到全面自给,达到每年出产三十一万匹布,四百七十万斤铁。还有文化建设,要使边区老百姓每一个人至少识一千个字,要提倡卫生,要使边区一千多个乡每乡设立一个小医务所,还要教会老百姓闹秧歌,唱歌。要达到每个区有一个秧歌队,家家有新内容的年画、春联。

“要为实际服务,不要闹教条主义,人总要落在一个地点,像飞机早上出去、晚上也得回来,落在一个地点,不能到处飞不落地。教条主义就是不落地的,它是永远挂在空中。

“共产党人工作中有缺点错误,一经发觉,就会改正,他们应该不怕自我批评,有缺点就公开讲出是缺点,有错误就公开讲出是错误,一经纠正之后,缺点就不再是缺点,错误也就变成正确了”。(一九四四年五月三十一日《解放日报》)

