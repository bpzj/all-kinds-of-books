\section[和《江峡》轮船员的谈话(一九五八年三月二十八日)]{和《江峡》轮船员的谈话}
\datesubtitle{(一九五八年三月二十八日)}


毛主席对江峡轮三副、女青年石若仪说:“当你对一件事物还不了解时,往往是害怕的。正如蛇一样,当人们还不了解它,没有掌握它的特性时,感到十分害怕,但是一旦了解了它,掌握了它的特性和弱点,就不再害怕了,而且可以捉住它。”接着,毛主席又问石若仪:“你在船上工作了多久?”当石若仪说到有四年多了的时候,毛主席转过去问杨大副:“你呢?”杨大副回答说:“三十多年了。”毛主席慈祥地对石若仪说:“要好好向他们学习,他们这些老工人是你的好师傅,水上经验都很丰富,许多知识是书本上学不到的。”

毛主席还说:“有些地方航道的很不好,在三峡修一个大水闸,又发电又便利航运,还可以防洪、灌溉……。”

毛主席对江峡轮船长说:“你的经验是丰富的,要多带徒弟,把技术传给青年人。”

<p align="right">(据一九五八年四月十五日《人民日报》有关报导)</p>

