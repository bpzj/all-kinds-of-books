\section[接见老挝爱国战钱党文工团团长、副团长和主要成员时的谈话(一九六四年九月四日武汉)]{接见老挝爱国战钱党文工团团长、副团长和主要成员时的谈话}
\datesubtitle{(一九六四年九月四日 武汉)}

\begin{duihua}

\item[\textbf{主席:}] 欢迎你们啊!你们是今天到的?

\item[\textbf{宋西·德沙坎布(团长,以下简称宋西):}] 我们是今天到武汉的。

\item[\textbf{主席:}] 这里气候很热,习惯吗?

\item[\textbf{宋西:}] 请让我向主席报告,这里的气候并不热,因为我们是南方人。

请让我们这些子孙向您问候!请问主席身体健康吗? 

\item[\textbf{主席:}] 勉勉强强过得去,眼看就不行了。快要去见马克思了。(众笑)

你们好,都是年轻人。你们的斗争是英勇的,你们是在前线,是在反对美帝国主义的前线。你们学会了走群众路线,能够团结大多数人——工人、农民以及爱国人士共同来反对美帝国主义。你们一定能够取得胜利。

要做群众工作就要和群众一样。要跟群众交朋友,首先就要精神一样。然后就是穿衣服要一样,他们穿什么衣服,你也穿什么衣服。吃饭要一样,他们吃什么,你们也吃什么。同他们一起劳动。不然,他们要怕你们的。你们是知识分子,是他们的朋友还是敌人?他们就搞不清。如果是他们的朋友,他们穿衣、吃饭、住房子,劳动就得一样。有一、两个月就熟悉了。你们这样就可以团结他们,反对美帝国主义。

我不是单讲你们文工团,军队也一样,你们能够做到,帝国主义就不能。反动派尽是剥削群众的,压迫群众的。军队要作战,也要做群众工作。我们的军队就是这样。我们搞了几十年,订出三大纪律,八项注意。第一条,一切行动听指挥,不听指挥各干各的就不行。你们文工团也有组织纪律吗?也听指挥吗?

\item[\textbf{××:}] 他们的纪律很好。

\item[\textbf{主席:}] 没有纪律,文工团就搞不好嘛!第二条不拿群众一针一线。那么军队怎么办呢?穿什么?吃什么?不能到工人、农民那里去要哇。除了向敌人要而外,我们政府还要收一点税,收点粮食税,收点商业税。不收一点税就不行。\marginpar{\footnotesize 176}收来后,一部分归军队,一部分还要归老百姓,老姓百利益。我们的党,我们的政府,我们的军队,是工人农民的党,是工人农民的政府,是工人农民的军队。我看你们也是这样。你们是工人农民的文工团,革命的文工团。不是反革命的文工团。你们的党,你们的军队,是革命的党,革命的军队。让大多数人——工人、农民和爱国人士团结起来,反对美帝国主义和它的走狗。你们看美国那么大的势力,在印度支那有四、五十万军队,现在美国也为难。在南越有人民解放军,他们的军民关系好。祝他们胜利,也祝你们胜利!再有十年,十几年就会取得胜利的。

我们搞了几十年,取得了胜利。可是我们犯过几次错误。比如右倾错误就犯了两次。“左”倾错误犯了三次。你们好,你们没有犯我们犯过的错误。

\item[\textbf{宋西:}] 因为我们吸收了犯过错误的同志的经验,所以就没有犯类似的错误。

\item[\textbf{主席:}] 犯错误,看犯什么错误。政治路线上犯错误,损失就大。比如一九二七年犯过大错误,损失很大,五万党员剩不到一万了。要纠正错误,就是拿起枪来打仗,这样我们就有活路了。有了几块根据地,三十万军队,这时候头脑发昏了,又犯了“左”倾机会主义错误。把南方根据地统统丢了。开始万里长征到北方。然后三十万军队剩下两万。这个时候就舒服了。为什么舒服了?就是犯错误的人抬不起头来了。我们用说服方法,就是通过整风运动把他们团结起来。一个也没有丢掉。最后取得了今天的胜利。你们到中国来看到一些好的东西。但我们的错误也要看到。不了解我们的错误,对你们不利。胜利了,搞社会主义建设,搞了十五年,我们的文化界比不上你们。有几百万人,都是国民党留下来的资产阶级知识分子。教育界大学教授、中学教师、小学教师也有不少是资产阶级知识分子。文化界唱戏的、画画的、唱歌的都有,新闻界好一些,电影界也有。现在他们受不了了。现在又整风,把资产阶级知识分子整他一年、两年睡不着觉我就高兴。

你们从南方来,认为中国一切都是好的,没有那么回事。我就不相信。你们年轻,经验不多,觉得什么都好。一个社会有黑暗一面,有光明一面,当然我们光明面是主要的。我们的军队、政府和党都比较好。现在全国有地主、资产阶级(包括它的知识分子)三千五百万。比你们全国人口还多。所以很难说找不出缺点来。你们可能会问,为什么我这个人这么糊涂?搞了十五年,还没有搞好?就是因为我糊涂,并不高明,并不比你们高明,信不信?不信啊?

\item[\textbf{宋西:}] 我很难理解。

\item[\textbf{主席:}] 我这个人有缺点、有错误。二十年前我就讲过,文艺要为工农兵服务。可是这十五年我们没有很好抓,这还不是怪我不行?现在我改正错误。

过去忙了那方面的事情,就忽略了这方面的工作,现在我要来抓一抓。今年文化界可不太平噢。整风不是把他们统统丢掉,改正错误就可以。

少数人不改怎么办?不改也可以。为什么可以呢?因为是少数人。我看一百年也有人不会改的。他不改也不能把他枪毙。我们如何争取?一百人有三十人为工农兵服务就行了。现在右派和反革命分子最多不过百分之五,有百分之六十五是中间派,是大多数。我们要好好做工作,争取他们。我们也有文工团,文工团也不都是好的。演戏的全国有多少?

\item[\textbf{×××:}] 全国约有三千多个剧团。\marginpar{\footnotesize 177}

\item[\textbf{主席:}] 统通演古的。表演的戏尽是帝王将相、才子佳人。就是缺乏反映工农兵的,反帝的。\\
你们看过我们的京戏没有?

\item[\textbf{宋西:}] 我们从前看过中国戏。这次到中国来,我们在北京也看过。

\item[\textbf{主席:}] 是京戏吗?

\item[\textbf{宋西:}] 我们在北京看的是京剧现代戏。

\item[\textbf{主席:}] 这是少数,是开始。这方面我不相信样样都好。

\item[\textbf{宋西:}] 我不容易理解。

\item[\textbf{主席:}] 事实就是这样。看过旧京戏没有?

\item[\textbf{××:}] 没有看过。

\item[\textbf{主席:}] 帝王将相搞点看看嘛。就比较嘛,也有一部分绘画、电影、音乐、照相是好的,可惜不多。

我们这个党也是不纯的。有人做官当老爷,有大老爷,有小老爷。有的支部书记,那是老爷,在一个乡当支部书记像个土皇帝,可厉害吶。特别严重。我们站在大多数农民方面,不站在少数地主、富农方面。可是他们实际上是站在地主、富农方面的。建设社会主义十五年了,还有国民党。你们两位(指宋西和副团长巴巴)年纪大一点,是能够理解的。他们(指主要演员)年纪轻,不容易理解。就是有这样的事情嘛,还不少呢!三个指头中就有一个指头。我们现在已在进行社会主义教育,我们还要搞几十年。把老的改造了,又产生新的出来,有些贪污分子,今天说不贪污了,退了赃,可是明天还照样贪污。

\item[\textbf{主席:}] (面向外宾)现在把希望寄托在你们南方:老挝、泰国、南越、柬埔寨、印尼、缅甸、马来亚、日本、南朝鲜……,你们帮了我们的忙,帮我们建设社会主义,你们帮了全世界人民的忙。

你们表演的什么节目?

\item[\textbf{×××:}] 昨天晚上,×××看了。

\item[\textbf{××:}] 节目都是战斗性的。

\item[\textbf{主席:}] 好!

你们不要看不起自己,以为是小国,小国又怎么样呢?小国同样出英雄。你们知道印度尼西亚共产党主席叫什么名字?

\item[\textbf{宋西:}] 艾地。

\item[\textbf{主席:}] 这个同志,我问过他是什么地方人?他说他是苏门答腊西南一个小岛的人,是个少数民族。你说地方那么小,为什么能当印尼党主席呢?他对我说,你不要看我们的地方小啊,印尼的语言是以它那个地方为标准,那是印尼共产党最活跃的地方。马克思就是少数民族,他是个犹太人。耶苏也是个犹太人。过去犹太人就是一个少数民族。中国的孔夫子住在鲁国,也只有几十万人。他办了中国历史上第一个学校,大家都不理他,后来到各国去找工作,人家也不理采他。没有办法,就只好流浪。他是宣传地主阶级的封建道德,那时候人人都说他是个圣人。他是中国第一个教育家。……讲得太远了,离开题目了。

\item[\textbf{宋西:}] 刚才主席所讲的问题我们很关心。如果不提出这些例子来谈,就不容易理解。因为它关系到执行文艺路线的问题。\marginpar{\footnotesize 178}

\item[\textbf{主席:}] 旧社会的知识分子不改造不行。过去我们没有抓紧。谁战胜谁的问题,是无产阶级战胜资产阶级,还是资产阶级战胜无产阶级?这个问题还没有解决。有些人不懂,赫鲁晓夫就是这样。你们看苏联搞了四十多年现在资本主义复辟了。列宁建立的党,列宁建立的苏联,四十多年资本主义复辟,搞修正主义。我们还只搞了十五年,将来马列主义会胜利。教育青年是个大问题。如果我们麻痹睡大觉,自以为是,资产阶级就会起来夺取政权,资本主义复辟。马克思主义不克服修正主义,修正主义就克服马克思主义,资本主义进行复辟。挂共产主义的招牌,实行资本主义政策。你们要知道,这个问题十年几十年也不好解决。

请你们回去向你们党中央转达,我们是有希望的。赫鲁晓夫不是好人,但他帮了我们的忙,帮我们认识苏联——第一个社会主义国家怎样变成修正主义的。他不仅帮了中国人的忙,也帮了你们的忙,帮了全世界革命人民的忙。

世界上有三种坏人:一种是帝国主义,第二种是修正主义,第三种是各国反动派。你们那里反动派还有力量。叫什么呀?叫什么富米·诺萨万、西何·库帕拉西……,富马看来也是靠不住的,但还要拉他一把。所以苏发努冯同志去巴黎开会是正确的。我认识你们的领导人很少。一个是苏发努冯,一个是凯山。

差不多了吧?你们讲,你们讲。

\item[\textbf{宋西:}] 首先让我向主席报告,我们见到您感到非常荣幸。我们听了您对我们讲的话,将它当作一种教育。

\item[\textbf{主席:}] 我讲的不是教育,是经验。你们回去不能硬搬中国这一套。

\item[\textbf{宋西:}] 我们将您的讲话看作对子孙教育一样。

\item[\textbf{主席:}] 是对同志。

\item[\textbf{宋西:}] 我讲的话是从内心讲的。今天有机会来见主席是毕生最荣幸的象征。我们很久以来就听到您的名字。想找机会见您。我们觉得您是六亿五千万人民的领袖,也是为反对殖民主义争取解放斗争的人民的领袖。

\item[\textbf{主席:}] 不是领袖,是朋友。

\item[\textbf{宋西:}] 我们知道帝国主义是外来的,而修正主义是内部最危险的敌人。

我们今天有机会拜访您,感到十分荣幸。也是老挝人民的荣幸。我们正在遵循以苏发努冯为首的党中央的领导进行斗争。我们在抗法斗争中经过八、九年。后来美帝国主义进来了,因此,我们又起来进行反美斗争。在正确的路线指引下,我们的斗争是正义的。因此得到了全世界爱好和平的人民,社会主义各国人民,尤其是伟大的中国人民的支持和帮助,一定能够取得胜利。

\item[\textbf{主席:}] 讲得好。

\item[\textbf{宋西:}] 我们今天有机会见到您,也就是由于我们革命人民和革命先烈二十多年斗争给我们的恩惠。我们的同志没有机会来亲自见到主席,但他们为我们见到您创造了条件。我们还很年轻,我们不会辜负先烈们所给我们的恩惠。所以我们在这里宣誓,要做革命的接班人!

\item[\textbf{主席:}] 对的。

\item[\textbf{宋西:}] 目前我们的任务很重,因为美帝国主义从各方面来破坏我们。我们是文艺工作者。我们认为文艺是一条革命路线,是从政治思想上向敌人斗争,为人民服务的。\marginpar{\footnotesize 179}我将尽最大的努力从事这项工作。

今天非常高兴,因为拜访了主席,从主席的谈话中获得了很多经验。文艺战线上不进行革命是不行的。因为要通过艺术来进行思想革命工作。旧文艺服务于旧社会。

而新文艺则为新社会服务,怎么使旧文艺为新社会服务呢?我看我们在北京看到的京剧现代戏,内容都是革命的。没有看旧戏,因为帝王将相、才子佳人是为旧社会服务的艺术。这一点是我们刚学到的。

这是毛主席给我们提出的,新任务,我们为接受这个新任务而感到荣幸。这是为什么呢?因为主席工作忙,有党务工作,有国家工作,但是还抽出时间来抓文艺工作。过去在老挝有人看不起演戏的,认为演戏的不是好的。

\item[\textbf{主席:}] 你们自己认为怎么样?

\item[\textbf{宋西:}] 现在我们有些人还是看不起这种工作,甚至他们的父母也不让自己的子女来演戏。主席:我们中国也是这样。

\item[\textbf{宋西:}] 我们这个文工团从建立起来已有三年时间,在党中央和苏发努冯主席领导、关怀和教育下,我们很年轻,也贡献了一点力量。群众赞扬说:我们文工团是一个有民族性、革命性、斗争性的文工团。

\item[\textbf{主席:}] 这个好。

\item[\textbf{宋西:}] 尤其是到中国来表演,受到中国同志热烈欢迎。

\item[\textbf{主席:}] 那就好。

\item[\textbf{宋西:}] 在中国我们受到了男女老少热烈地欢迎。从基层干部到国家领导人都支持我们的演出。

\item[\textbf{主席:}] 你们来多久了?

\item[\textbf{宋西:}] 我们到中国已经一个多月了。我们前几天还到过沈阳、鞍山,虽然在鞍山只演两个晚上,场子小,票子少,可是有许多观众在戏院外听我们的广播。

\item[\textbf{主席:}] 好哇。

\item[\textbf{宋西:}] 我们看见中国人民在中国共产党和毛主席领导下,他们一致支持我们。

\item[\textbf{主席:}] 应该支持。不支持是错误的。不支持的无非是帝国主义、修正主义和各国反动派。

\item[\textbf{宋西:}] 我们正处在激烈的斗争中,没有什么比知心朋友更宝贵,没有任何东西比朋友的支持更为宝贵。因此,我们认为应该把感想带回老挝去,向老挝人民进行宣传。我们将把在中国学习的经验带回去工作,把您的话带回去,好好向党中央汇报。

\item[\textbf{××:}] 好吧,是否照个相留作纪念?

\item[\textbf{主席:}] 好哇。

(合影后,主席请外宾向在北京这次没有来的团员们问候。当主席送老挝同志们出门口的时候,宋西和其他客人恋恋不舍地紧紧握着毛主席的手,并说:“祝毛主席健康长寿!”)\marginpar{\footnotesize 180}
\end{duihua}
