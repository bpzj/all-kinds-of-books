\section[关于第三个五年计划在中央工作会议上的讲话(一九六四年六月六日)]{关于第三个五年计划在中央工作会议上的讲话}
\datesubtitle{(一九六四年六月六日)}


制定计划的方法,过去基本上是学苏联的,比较容易做:先定下来多少钢,然后根据这来计算要多少煤,多少电,多少运输力量,等等;根据这些再计算增加多少城市人口、多少生活福利,是摇计算机的办法。钢的产量一变少,别的一律跟着削减。这种方法是一种不合实际的方法,行不通。这样计算把老天爷就计划不进去。天灾来了,偏不给你那么多粮食,城市人口不能增加那么多,别的就都落空。打仗,也计划不进去。我们不是美国的参谋长,不晓得他什么时候要打。还有各国的革命,也难计划进去。有的国家的人民革命成功了,就需要我们的经济援助,这如何能预计到?

要改变计划方法。这是一个革命。学上了苏联的方法以后,成了习惯势力,似乎很难改变。

这几年,我们摸索出来了一些方法。我们的方针是;以农业为基础,以工业为主导。按照这个方针,制定计划时先看可能生产多少粮食,再看需要多少化肥、农药、机械、钢铁……。

年成,如何计划?五年中,按一丰、二平、三欠来定。这样比较切实可靠。先确定,在这样能够生产的粮食、棉花和其他经济作物的基础上,可能搞多少工业。如果年成好些,那就更好。

还要考虑到打仗。要有战略部署,各地党委,不可只管文不管武,只管钱不管枪。只要有帝国主义存在,就有战争危险。要建立战略后方。……沿海不是不要了,也要好好安排,发挥支援建设新基地的作用。

两个拳头,一个屁股。基础工业是一个拳头,国防是一个拳头。要使拳头有劲,屁股就要坐稳屁股就是农业。

基础工业,现在主要解决品种、质量问题。去年钢的数量虽然比过去少了,但品种比过去多了,质量比过去好了,用处比过去还大。关键不在数量上。苏联就是以数量为标准,如果钢的数量标准完不成,就好像整个社会主义建设就不行了。他们年年要增加产量指标,年年搞虚夸。其实数量计划完不成,国家垮不了台。有一定的数量,品种更多了,质量更好了,基础就更巩固了。

农业主要靠大寨精神,自力更生。这不是说可以不要工业支援。水利、化肥、农药都是需要基础工业的。

要按照我们掌握的客观的比例关系安排计划。

计划不能只靠加、减、乘、除。计算出来了,各部门、各地区,就分数字、争人、争钱、打官司……要政治挂帅,要有全局观点,不是根据那个地区自己的愿望,而是根据客观存在,事物本身的规律,来安排计划。

不要老是争钱,争来了钱,就乱花钱。周信芳一个月一千七百元工资,不演多少戏,还存钱在香港。有的年青演员就作“十年的计划”,要赶上周信芳……。对资产阶级知识分子,按政策,必要时可以收买,对无产阶级知识分子,为什么要收买?钱多了一定要腐化自己,腐化一家人和周围的人……。苏联的高薪阶层,先出在文艺界。

争取几年内做到不再进口粮食,节省下外汇来多买技术设备,技术资料。……

不能乱花钱。不要看到情况好转了,又随便“大办”。“留有余地”过去说了多少次,不照办。这两年照办了。不要情况好了又不照办了。

机关工作人员,大部分可以做到半工作半劳动。这办法值得提倡。懒是出修正主义的根源之一。

文艺界为什么弄那么多协会摆在北京?无所事事,或者办些乱七八糟的事。文艺会演,军队的第一,地方的第二,北京(中央)的最糟。这个协会,那个协会,这一套也是从苏联搬来的,中央文艺团体,还是洋人,死人统治着。……一定要深入生活。老搞死人洋人,我们的国家是要亡的。要为工人、贫下中农服务。体育,也要对革命斗争和建设有益处的。

一般干部中,“三门”干部很多(出家门、进学校门、进机关门),“三门”不能很好培养干部。国家将来靠这种干部掌握,就危险。靠“小学门、中学门、大学门”干部也不行。不读书不行,读书太多了也不行。本事,光靠读书不行,要靠实践。我们的国家主要靠在实践中读书的干部掌握。

各省都要搞军事工业。要从工业、农业、文教挤出钱来。不要办那么多正规学校。清华,学生一万多,教职员、家属四万多。这样,领导精神会大大浪费。

院士、博士,不一定要搞。


