\section[研究沦陷区(一九三九年十月一日)]{研究沦陷区}
\datesubtitle{(一九三九年十月一日)}


中国沦陷区问题,是日本帝国主义生死问题。在目前阶段内,集中精力侵略中国,是日本帝国主义的确定政策。

在目前阶段内,敌人侵略中国的方式,正面的军事进攻,大规模的战略进攻(某种程度的战役进攻不在内),如同大举进攻武汉那样的行动,其可能性已经不大了。敌人侵略的方式,基本上已经转到政治进攻与经济进攻两方面。所谓政治进攻,就是分裂中国的抗日统一战线,制造国共磨擦,引诱中国投降。所谓经济进攻,就是经营中国沦陷区,发展沦陷区的工商业,并用以破坏我国的抗战经济。

为达其经济进攻之目的,彼需要举行对我游击战争的“扫荡战争”,需要建立统一的伪政府,需要消灭我沦陷区人民的民族精神。

所以沦陷区问题,成了抗战第二阶段一一敌我相持阶段的极端重要的问题。

敌我相持阶段,在敌人是确保占领地并准备进一步灭亡全中国的阶段,在中国,是确保未失地并准备收复沦陷区的阶段。敌人为了确保占领地,为了消灭全中国,他就用经营沦陷区来准备条件,我们为了确保未失地,为了收复沦陷区,不能不从各方面有所准备。而最积极的支持游击战争,改革国内政治,乃是破坏敌人计划,实现我们计划的两个大政方针。

在这种情况下,沦陷区问题研究是刻不容缓了。在这个问题上,有敌人的一方面与我们的一方面。在我们的一方面,是如何支持游击战争的问题,研究这个问题,不待说,是十分重要的。在敌人的一方面,是敌人在沦陷区已经干了些什么并将怎样干,研究这个问题,乃是研究前一问题的起点,不了解敌人的情况,我们对付它的方法是无从说起的。

为了研究一切重要的时事问题,延安组织了一个“时事问题研究会”,同志们除了研究讨论外,还着手编辑“时事问题丛书”。分为日本问题、沦陷区问题、国际问题、抗战的中国问题这样四个问题来研究,分别搜集材料,用综合文摘体裁出版参考书。本年七月间出版的“战争中的日本帝国主义”,算是研究日本帝国主义本国情况的第一本书,亦即是时事问题丛书的第一集。现在出版的“日本帝国主义在中国沦陷区”(简称“日本在沦陷区”),则作为研究沦陷区情况的第一本书,亦即是时事问题丛书第二集。其余两个问题,亦将继续出版。

这些系统的研究问题,并为一切抗战干部们供给材料,实在是必要与重要的。“瞎子摸鱼”,闭起眼睛瞎说一顿,这种作风,是应该废弃的了。“没有调查就没有发言权”,或者说:“研究时事问题须先详细占有材料”,这是科学方法论的起码的一点,并不是什么“狭隘经验论”。

最后要指明的,这一类的时事问题丛书,仅仅是材料书,它是重要的材料,但仅仅是材料,而且还是不完全的材料,问题是没有解决的。要解决问题就需要研究,需要从材料中引出结论,这是另外一种工作,而在这类书里面是没有解决的。

