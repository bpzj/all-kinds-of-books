\section[关于学术思想批判的指示——与陆××、周×的谈话(一九五四年)]{关于学术思想批判的指示——与陆××、周×的谈话}
\datesubtitle{(一九五四年)}


毛主席说:不是没有警觉性。我们有些党员就是保护那些仇视马克思主义的。自己不反对马克思主义。还反对别人反对。

俞伯平的唯心论是很明显的,把家谱孤立起来看,他的批评标准是主观主义和相对主义,是客观唯心论和相对论。

对资产阶级合作,政治上是朋友,但不能把他捧为父亲,不能放弃原则,接受他们的意见是对我们有利的才接受。如共同纲领是在党的领导下搞的,而古典文学是他们领着我们搞的。

要在学术各部门清除唯心论。

整个学术界和文艺界主要是投降主义,光是反自由主义还不够,因为不能包括一个阶级思想反对另一个阶级思想。很多战线上没有战士,没有斗争。很多党员是吃、睡、休息。

第二个问题:学术界缺乏批评。“官书”为什么不能批评呢?联共党史结束语第二条说马克思主义的个别原理也还可以批评哩。鲁迅之所以便人感到亲切,就是把自己思想告诉读者。

怕批评后抬不起头是不对的,总是有些人抬不起头来。当然改正错误后他可以抬头。

进行批评应有充分理由,若一律抹煞、武断,这是军阀主义的,我们要这样就要失败。有些有待证明的都可以讨论。思想问题不是一个人的问题。

不能采取少数服从多数。少数人的意见也可以坚持,也不一定是错误的,开始时对的总是少数。各家意见都可以暴露,特别是我们都缺少学问。红楼梦问题不要急于做结论。学术问题要开学术会议来解决,不能由中宣部来做。

第三,关于扶植新生力量。几个部门都不注意这一点,这不是马克思主义。老干部要好好帮助青年作战。老干部往往缺乏生命力。注意青年,招募新战士,这是领导的作用。

要注意“小人物”。任何有成就的人,过去都是“小人物”。在我们生活中,文化工作中没有“小人物”是不成的。……在大学中要培养助教,重点放在培养青年。要相信一代好似一代。

《清官秘史》很多人以为是好的,毛主席以为是错的,到义和团时我们拥护光绪就是拥护帝国主义了。还有梁思成的建筑,既不经济,也不好看,穿西装戴瓜皮帽。印刷机和广播器没有民族形式。

文字是要拼音的,但现在注意字母的样子太不好看了。若用民族形式就好推行了。这有个民族心理问题。毛主席最后说:我今天讲的是个意见,不要看成不能动的。更不能用作打击别人的力量。

对俞平伯的斗争应该停止了,应转到批判胡适的思想。特别借此向全国人民宣传唯物论思想。唯心论,唯物论要作些通俗宣传。

这次斗争要分清敌、我、友。对敌要打击,敌方就是有些进步的东西也不要去鼓励,我方的缺点也不要急于批评,对朋友要保护。不要把战线弄乱,一团混乱。

斗争由中宣部领导,郭老挂帅。


