\section[在中国共产党第六届中央委员会扩大的第六次全体会议上讲话(摘要)(一九三八年十一月六日)]{在中国共产党第六届中央委员会扩大的第六次全体会议上讲话(摘要)}
\datesubtitle{(一九三八年十一月六日)}


第一,允许蒙、回、藏,苗,瑶,彝、番各民族与汉族有平等权利,在共同对日原则之下,有自己管理自己事务之权,同时与汉族联合建立统一的国家。第二,各少数民族与汉族杂居的地方,当地政府须设置由当地少数民族的人员组成的委员会,作为省县政府的一部门,管理和他们有关事务,调节各族间的关系,在省县政府委员中应有他们的位置。第三,尊重少数民族的文化、宗教,习惯,不但不应强迫他们学汉文汉语,而且应赞助他们发展用各族自己语言文字的文化教育。第四,纠正存在着的大汉族主义,提倡汉人用平等态度和各族接触,使日益亲善密切起来,同时禁止任何对他们带侮辱性与轻视性的言语、文字与行动。

上述政策,一方面,各少数民族应自己团结起来争取实现,一方面应由政府自动实施。才能彻底改善国内各族的相互关系,真正达到团结对外之目的,怀柔与羁縻的老办法是行不通了的。

<p align="right">(转摘自《人民日报》一九五三年九月九日的社论《进一步贯彻民族区域自治的政策》)</p>




