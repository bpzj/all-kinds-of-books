\section[中共中央关于目前形势与党的任务的决定(一九三七年八月二十五日,洛川会议)]{中共中央关于目前形势与党的任务的决定(一九三七年八月二十五日,洛川会议)}
\datesubtitle{(一九三七年八月二十五日)}


(一)芦沟桥的挑战与平津的占领不过是日寇大举进攻中国本部的整个的开始。日寇已经开始了全国的战时动员。他们一切所谓不求扩大的宣传不过是掩护进攻的烟幕弹。

(二)南京政府在日寇进攻与人心愤激的压迫下已经开始定了抗战的决心。整个的国防部署与各地的实际抗战也已经开始,中日大战不可避免。七月七日芦沟桥的抗战,已经成了中国全国性抗战的起点。

(三)中国的政治形势从此开始了一个新的阶段,这就是实行抗战的阶段。抗战的准备阶段已经过去了,在这一阶段内的最中心的任务,是动员一切力量争取抗战的胜利。过去阶段中,由于国民党的不愿意和民众的动员不够,因而没有完成争取民主的任务,这必须在今后争取抗战胜利的过程中去完成。

(四)在这一新阶段内,我们同国民党及其他抗日派别的区别与争论,已经不是应否抗战的问题,而是如向争取抗战胜利的问题。

(五)今天争取抗战胜利的中心关键,是在使已经发动的抗战发展为全面全民族的抗战。只有这种全面的全民族的抗战,才能使抗战得到最后胜利。本党今天所提出的抗日救国的十大纲领,即是争取抗战最后胜利的具体的道路。

(六)今天所发动的抗战,中间包含着极大的危险性。这主要的是由于国民党还不愿意发动全国人民参加抗战。相反的,他们把抗战看成只是政府的事,处处惧怕与限制人民的参战运动,阻障政府军队与民众结合起来,不给人民以抗日救国的民主权利,不去彻底改革政治机构,使政府成为全民族的国防政府。这种抗战可能取得局部的胜利,然而决不能取得最后胜利。相反的,这种抗战存在着严重失败的可能。

(七)由于当前的抗战还存在着上述的严重弱点,所以在今后抗战过程中,可能发生许多挫败,退却,内部的分化叛变,暂时与局部的妥协等不利的情况。平津的丧失就是东四省丧失后最严重教训,因此,应该看到这一抗战是艰苦的持久战。但我们相信已经发动的抗战必将因为我党与全国人民的努力,冲破一切障碍物,而继续的前进与发展。我们应该克服一切困难,为实现本党所提出的争取抗战胜利的十大纲领而坚决奋斗。坚决反对与此纲领相违背的一切错误方针,同时反对悲观失望的民族失败主义。

(八)共产党员及所领导的民众和武装力量,应该最积极地站在斗争的最前线,应该使自己成为全国抗战的核心,应该用极大的力量发展抗日的群众运动。不放松一刻工夫一个机会去宣传群众组织武装群众,只要真的组织千百万群众进入抗日民族统一战线,抗日战争的胜利是无疑的。

<p align="right">(原载《中国共产党历史参考资料》(四)中共中央高级党校编第四一五页)</p>

