\section[在天津市党员干部会议上的讲话(一九五七年三月十七日)]{在天津市党员干部会议上的讲话}
\datesubtitle{(一九五七年三月十七日)}


(黄火青、吴砚农、×××、李耕涛四同志陪同毛主席出现在主席台,全体起立,长时间热烈鼓掌)

黄火青:同志们,现在请毛主席给我们讲话。(鼓掌)

毛主席:你们有什么问题?

黄火青:你们有什么问题需要主席解答的,可以写出来,现在主席先给大家讲。

讲什么东西?同志们,你们有什么问题?

很多同志没见过面,今天和同志谈一点事情。希望同志们提一点问题给我,你们的问题还没有来,那末,还有时间,我就先讲几句咯,就讲“百花齐放,百家争鸣”吧!好不好啊?(鼓掌)

因为这个问题是我们全党关心的问题,是社会上大家关心的,社会上民主人士,各界人士是关心的。对这个问题在我们同志中意见不完全一致。有些同志觉得这个方针可以,赞成,有一些同志看样子还可以,但是心里总有那么一点不舒服;“百花齐放”那么多花,恐怕有不好的东西出来了怎么得了啊!“百家争鸣”,咱们共产党员只算一“家”,九十九家包围我们,怎么得了啊!有没有这个问题,同志们?就是有比较了解这个问题的,有了解一些但是又不很多的,还有怀疑的,有不赞成的。有各种意见在我们党内。

过去我们党做的工作主要是什么工作呢?现在不是有建设工作吗,在过去我们几十年,主要的工作就是阶级斗争工作。阶级斗争,还不是建设工作。阶级斗争就是要推翻几种制度,就是帝国主义、官僚资本主义、封建主义那些制度。那是一场革命;社会主义就推翻资本主义制度。这些都是由于阶级斗争范围。打蒋介石、抗美援朝、镇压反革命、土地改革,城市里还有一个民主改革,还有社会主义改造。这些都是属于阶级斗争范围。那个声势很大,是不是?

过去几十年,从有我们这个党以来,一直到去年上半年,社会主义改造,就是前年下半年跟去年上半年这个高潮,那个时候锣鼓喧天,热闹得很。我们党的精力,主要是放在这一方面。这个斗争时间很长,它从我们祖宗鸦片战争那个时候算起,一八四零年,从那个时候反帝国主义算起,算到一九五零年,就一百年了,到去年一九五六年,就有一百一十六年之久,才把上层建筑、生产关系——把生产关系,旧的生产关系改变为新的生产关系,旧的上层建筑改变为新的上层建筑。生产关系主要是讲所有制,现在我们的所有制,是社会主义所有制。上层建筑就是政府、国家权力机关、军队,都改了。这是一个很大的斗争。在过去作这个斗争的时候,人们对于我们,在开头也是不相信的。谁人相信共产党能够成功呢?开始有共产党的时候是不相信的,中间也是不相信的,因为中间我们经过失败。经过北伐战争时候的失败,后头,土地革命的失败,经过两次很大的失败,人们都不相信的。现在呢?现在人们就相信了,说共产党行了,行在什么地方呢?说政治你们行、军事你们行、你们有这两门。人们说共产党有这两门,还可以考赢了。还是帝国主义考赢了?跟我们考,跟我们比,蒋介石,国民党跟我们比,谁比赢了?谁胜谁败?我们赢了。你赢了嘛,那就算嘛,赢是好的嘛,你赢了嘛,还有什么话讲,那好嘛,建设呢?历来人们就是说共产党恐怕是不行的,现在怎么样呢?我们一面革命,一面建设。我刚才是讲为“主”,阶级斗争为主,不是讲没建设,还是有建设,这一方面,我们是不会。要讲不会,也会一点,因为过去作阶级斗争的时候,在根据地也多多少少学了一点。但是盖大工厂,什么设计、施工、安装,哪一套东西,我们就不会了。科学、自然科学、工程技术、要教大学生,大学教授我们就很少了。在座有没有多少大学教授?(起立向台下问),有没有?大学教授请举手!一千都没有?有没有大学教授?总得有几个吧?(台下有同志举手)一个,那里有一个,少得可怜啊!同志们,(黄火青:有的没举手,毛主席又回原座继续说)有几个没举手?是啊,你们势力不大,你就不敢举手。

现在人些人说,共产党搞科学不行,共产党大学里头教书不行,医院里头当医生不行,工厂里头搞工程,当工程师,当技术人员不行。有几不行就是喽!这个话怎么样呢?你们听得到没呢?同志们!我听到了一些这样的话喽,我说这个话讲得对,讲得合乎事实。就是我们没有科学家、工程技术人员、医生、大学教授。中学里头当教员的也少。文学艺术方面有点儿也是三七开;就是像斯大林犯错误一样,斯大林不是有三分错误有七分正确吗?我们有三分会七分不会。文学艺术方面。优势还是共产党以外,大学教授几乎全是共产党以外的,医生几乎全部都是共产党以外的。是不是?教育界有二百万人,大、中、小学,所谓公教人员的“教”有二百万人之多,共产党干什么事情呢?共产党就是在学校里名为领导,实际上就是不能领导,因为你不懂嘛!所以应该承认,这是我们不行的一方面。他们讲得对,但是他们这个话也是不全面的,我说讲对了一半,还有一半不对。共产党不能领导,这是对的,共产党也能领导,他们没给讲。为什么也能领导呢?我说就是我不懂请你们干嘛!国民党也是一不懂,“蒋委员长”蒋介石他也不懂,国民党那个党对刚才讲过的这几门也是跟我们差不多,他也是做阶级斗争的这个党,他是专搞阶级斗争的,他就不搞建设。我们一面搞阶级斗争,这七年搞了点建设。他搞二十年,只有几万吨钢,我们搞了七年,今年八年,有多少吨钢呢?可能达到五百万吨钢,按照计划四百一十二万吨,可能超过。也许不到五百万,就四百多万吨。那么他二十年只有几万吨,我们八年有四百万吨,这就是说我们搞计划,此外,财政部给饭给这些人吃。国民党一无计划,二不给饭。同志们!不吃饭我也不能讲话,你们也不能听,科学家也好,大学教授也好,医生也好,是要吃饭的,国民党有饭不给那些人吃,他专搞阶级斗争,给军队吃了,给那些政客吃了。我说李富春,一个李先念,世界上还有这两个人嘛,一个管计划,一个管财政,把这些科学家、大学教授、工程师、医生统统放到计划里头,做出什么长远计划,年度计划,在这上面领导他们,以计划去领导他们,此外,还有一条什么可以领导他们呢?就是以政治去领导他们,就是以马克思列宁主义去领导他们。但是无论如何,我们现在是不会的,但是世界上的事情是可以学会的。我们现在要学,现在阶级斗争,这件工作基本上结束。所谓基本上结束就不是完全结束了,大规模的群众性的阶级斗争基本结束,我们“八大”上面说了,我们全党要求搞这个建设,要学科学,要学会大学里头当教授,要学会在科学机关里头做试验,研究科学。要学会当工程师,当技术人员,当医生,跟自然界作斗争,率领整个社会跟自然界作斗争,要把中国这个面貌加以改变。政治面貌加以改变之后,必须要使经济面貌加以改变。为了改变政治面貌我们花了几十年的时间,要改变经济面貌,大体上改变一下,跟过去有所不同,也要花几十年时间。像我们过去不会作阶级斗争一样,阶级斗争谁会呢?没有哪个人会的,比如说我就不会,我是当小学教员的,进共产党也没有想到,相信资本家对我们宣传的那一套,进的是资产阶级学校,相信那一套,后头,大概是逼上梁山了。你们各位呢?老早就打定主意了,母亲生下来,就当共产党?是不是?交代任务就是共产党,我妈妈没有交代这个任务给我,那么是学的,是不是?什么土改,也是不会,各种都是不会的,打仗也是不会的,打了多少败仗,经过多少变化,跟着经验的增多,我们学会了,善于作阶级斗争。是不是?经过许多失败,现在我们搞建设也要几十年,是不是可以比我们学习阶级斗争时所付的代价少一些?因为我们付的代价很大,有几次革命失败,一九二七年革命失败,万里长征以前的南方根据地都失掉了,白区也没多少工作了,那是因为“左”。一九二七年的失败是因为右。阶级斗争右了,后头是因为“左”。有了这两条我们就学会了。所以陈独秀,王明(陈独秀就到敌人那方面去了,后头这个人也死了。王明还没有到敌人那方面去)不管怎样,这样的人,都给我们很大的益处。不是他们个别的人,而是他们领导的这个运动,在一个时候失败了。这给全党以很大教育,给全国人民以很大的教育,那么代价是很大。现在我们的建设,是不是也付那么大的代价?如匈牙利从前的领导者,它的阶级斗争也是失败的。它的建设工作也是失败的,因为-闹嘛——去年十月一闹嘛,我(说)是闹得很好,有些人很不高兴这个匈牙利,我是很高兴。坏事就是好事。匈牙利还是闹好还是不闹好?无所谓闹不闹,总而言之是要闹的就是了,一个脓疮总是要穿的喽,我看可以争取,因为我们曾经学阶级斗争要付很大的代价才能学到,那么现在我们如果是不搞右倾机会主义。为什么曾经付很大的代价呢?不是搞左倾机会主义吃的亏嘛,搞教条主义、左倾机会主义吃了亏嘛,如果我们现在不重复以前的错误,在建设的时候,那么我们就可以比较少的少付代价,可以避免像匈牙利那样的事情(有些语句不通,原文如此)

现在我们好好的想想,很需要想一想,有这么一种情况,作阶级斗争结束,这么一件事。过去阶级斗争高潮的时候,比如社会主义改造,镇压反革命,高潮的时候,我们的缺点,人们不大看得见,我们的缺点,我们还懂,我们在建设方面,科学,办学校,这些事情人们要原谅,人们比较原谅我们。社会上现在那个戏不唱了,×××,有没有?那就是镇压反革命,他的戏不多了,有还是有,在座的有军官,有军事干部,如果没有戏唱,还要他们干什么呢?当然还是有戏唱的,但是现在不唱,故养兵千日,现在还没有听见炮响,于是就把现在社会上许多问题暴露出来了,这个问题是有了,因为这个锣鼓,阶级斗争的锣鼓,热闹得很,自从去年上半年锣鼓喧天,以后“八大”做了结论,那么这个东西就浮到上面来了,就摆到议事日程上来了,摆到我们党的议事日程上来了,人们就更加要求我们了,你共产党是干什么的,我们说:“同志,我也有一套本领叫阶级斗争喽!哈哈!你们不要老是看我们不起吧!老子也是干了几十年的哟!”(笑声)但是我们老是讲这个话,就不那么好了。因为人们,承认你这条。他说你政治行,军事行,你是干了几十年,你是辛辛苦苦,那是没问题,功劳薄上有我们的名字,但是,同志啊,现在大学教书是怎么教啊,医院里头是怎么开刀啊,我是没学呀,中学怎么办呀?什么科学问题怎么解决呀,原子物理,什么东西、工程师、设计、施工、安装、运转,这一套我们就不会,我们才开始学习,这种情况的改变,得要有一个时间,大概要远须三个五年计划,至于还要有十五年的时间,才会有一个改变,更大的改变,时间还要多一点,因为这个东西需要学,需要时间,那么能不能学到,我看是一定能学到,没什么巧,自然科学,开刀之类,你没学就不会开,我去开刀,就是像一个相声讲开刀一样的,一定会当那样的医生就是了。但是只要学就可以学到。现在有没有人学呢?现在有人学,我们还要派人去学。比如现在的大学生,现在的共产党员,共青团员,他们就在学,过十五年,他们就是大学教授,就是工程师,也许不要十五年,有十五年他毕业的时候,还有现有的科学家,工程师,大学教授,中学教员有一部分人,他们愿意加入共产党,条件适合的也可以接受他们加入共产党。所以再有三个五年计划是可以学到的。阶级斗争,那个容易学习呢?我看还是这个东西比较容易学习。总比那个打仗,比那个肃反,像那个肃反,看都看不见,你说没有,又有,你说有,他又没写名字反革命。抗美援朝打美国人那样的仗,事实没有多少把握美国这样大堆,我们拿这么一个指头去打他。我看还是就要是中学毕业以后进五年大学以后出来做五年到十年工作,就可以当大学教授,就可以当工程师或工程技术人员开刀也可以。所以有些人看不到这些变化,大概共产党永远学不到。我看是可以学到的而现在我们的任务就是要学到。那么先生是谁?先生就是现在的党外民主人士。他们是我们的先生,我们要跟他们学点东西,只要我们态度好,不摆官架子,不摆那个老子多少年,革命时候你在什么地方啊?不把这一套摆出来,我们把这套装起来,因为那套摆得没有意思,什么几十年嘛,那个是历史家去写的。我们每天见面就把两句话讲完了,老子革命几十年,此外,就没有了,就等于没有干一样,因为那个事情完了嘛,那个事情基本完了嘛,阶级斗争的问题。咱们现在来讲科学,你有什么长处,我跟你学习。我懂不懂呢?我是一窍不通。首先承认自己一窍不通。那你过去干什么去啦?那么就是因为过去搞阶级斗争喽,就是忙一点喽。这样子答复一句那可以的了,就是少年失修嘛!少年失修,所以不懂。现在你可不可以教呢?有些教员还不那么爽快的哩。但是只要我们好好请教,他会教的。我说从前拜师付的时候要烧香,又磕三个头,现在天津还有没有啊?恐怕也还有吧!还有磕头的?(黄火青:不磕头了)不磕头了,(黄火青:订师徒合同)订师徒合同。现在磕头这一灾难免了,我说假如要的话,怎么办?一定要那个规矩呢?还有要磕三个头才教我们,磕不磕?这个时候发生问题了,我说咱们应该磕三个头。你要学本领嘛,那他的规矩是要你磕头嘛,头现在是可以不磕,但是诚心诚意向他学,要尊重师付啊,要尊重他啊,要努力学啊,这也相当于磕头了。这个精神还是要的。现在我们党里头是有种风气是不大好的,就是脑筋里头还是装满了过去那个几十年,新的风气没有养成,没有事情就是,你们这里是打麻将还是打纸牌啊?或者是看戏,或者是跳舞,横直是没有事情做,没有养成阅读看书的习惯。没有把我们剩余精力放在学习哈面去。这是一般喽。至于在学校里头,在工厂里头,在科学研究机关,在医院里头,我们的工作人员凡能学的,你能学那么一部分也好,了解一下内容也好。你完全不懂,你又要在那里做领导工作,就是无怪乎人家说我们不行了。你完全不懂,你学,你又要摆那么一套架子,这就不好了,应该学习。百花齐放,关于艺术方面,这样的方针也是这样一个时期则特别显着。百花齐放,百家争鸣,这是竞赛。放出坏事来怎么办?现在有许多坏事,有许多怪议论。我说不要紧,这些怪议论批评它就是了。人们很怕开出来的花不好看,这个花有毒。百家争鸣,说是共产党只是一家,其它九十几家把我包围了。当然不是这样的意思。在社会科学,在世界观这一方面的问题上,不是什么百家争鸣。是两家争鸣。无产阶级一家,资产阶级一家,这百家里头按其性质分类,可分为两家争鸣。在现在这个世界上,无产阶级一家,资产阶级一家,无产阶级思想,资产阶级思想,这两家的思想斗争。在社会科学领域,小资产阶级算不算一家?当然也可以算一家。但是小资产阶级在基本点上,它是同资产阶级是一家的,它属于资产阶级那个范畴里头的。在共产党这一家里头,有没有争鸣的?事实上有,历史上第二国际这是一家,头上顶着马克思主义,实际上修正主义,现在还有修正主义--南斯拉夫,我们国家有没有呢?我们国家也有这样,可以找到这样的人这种人可以叫做右倾机会主义,在中国讲有修正主义,不必去讲,好像我们是有个修正主义派别一样,现在没有。但是过去有,陈独秀就是。陈独秀就是修正主义右倾机会主义;后头王明时期,第二次王明路线错误,同志们知道不知道?第二次,就是打日本初期抗日初期那就是修正主义。现在钟惦棐不是写了篇文章吗?关于电影问题,否定一切,对于过去的成绩。另外有些同志就肯定一切,看不到缺点错误,人家批评不得我们工作中的缺点错误,这是什么人呢?就是陈圩、马寒冰,一月七日《人民日报》有篇文章,陈其通、马寒冰,我说这些同志好心好意忠心耿耿,追求正义,他们是保护党的,看到外面锣鼓喧天反对我们,就是不得命令自动出去作战这个自动出去作战的这些人们,他们是有一股精神他仇恨那些看不惯的那些东西,应该承认他们这点是好的,但是他们的方针同方法是错误的。天津有没有这样的事情你们可以找一找。“左”的右的,肯定一切的否定一切的这是两种片面性把我们的工作一切都肯定不加分析,这个东西不对,以前教条主义者就是这样,拉科西就是这样,斯大林就是这样,斯大林,说他完全是教条主义者?也不能讲。这个人,他作了许多事,但是他有教条主义,他这个教条主义影响中国,使得我们一个时期革命就是失败,如果照他的办,后头的革命也办不成,现在我们就不在这里开会,这个房子什么人盖的,也还不是我们盖的,我们就没有这个机会,因为这还是国民党帝国主义统治嘛。他是两方面都有,又有教条主义。一切搬苏联,我们必须学苏联,苏联的东西很值得我们学,错误和成绩这两方面,我们现在提的口号是学习苏联先进经验,没有讲要学习他的落后经验,那一天提过这样的口号呢?但是虽然没有提,有那么一些东西跟着搬过来了,在最后七年里头,但是大体上说,我们不算是完全不加选择,这就是硬搬,因为我们曾经对教条主义有所批评,而教条主义的来源就来自斯大林。

对于社会上各种不同意见,因为阶级斗争基本结束而显露出来的各种东西,各种不满意,不满意共产党,说我们不行,这个是我刚才讲了的,本来不行的就应该承认不行。说我们不能领导科学,就具体的业务来说,我们是不能领导;就整个科学的前进的这方面我们能够领导,就是政治去领导吧!国家计划去领导他们,所以我们有学习的任务,对这些错误的议论,社会上有许多错误议论,我们采取什么方针,我们应该采取“百花齐放,百家争鸣”的方针,在讨论中,在辩论中去解决。究竟那个是对的,那个是不对,我们只有这样一个方法,别的方法都是不要,而现在我们党内有一种情绪呢,就是记着继续过去那种方法,或者叫做军法从事,你不听话呀,那么就正军法了,拉出去简单的砍了(正军法嘛。因为我们习惯这种简单的方法)×××的办法(指镇压反革命-校记录者注)这个办法不行了,这是对付阶级敌人的,对付阶级敌人呀,对付那些有血债什么东西的那些人们,就出去砍了吧,正军法嘛。因为我们习惯这种方法,我们习惯了。搞了几十年了,在阶级斗争里头,其实也不那么简单,还有许多细致的方法喽,但是因为是对付敌人,我们很有一股劲,在现在呢,不是对付敌人,而是对付人民内部的问题,民主党派,无党派民主人士,民族资本家,这么许多人,大学教授、医生,这个简单的办法,就是不行了。这是另外一套。我们要经过学习,他们是对的,辟如科学、技术这些东西,那么我们只有一条出路,就是向他们学习,而学习是完全可以学到的。有十年到十五年,就可以学到,不仅在政治上领导他们,而且在技术上领导他们。至于各种不同意见,分两类,一类是科学方面的,不论我们懂不懂,我们现在不懂,将来懂了,也是这个方针,不能“军法从事”的,×××的那个办法是不行的。只能采取“百家争鸣”这个方针。大家讨论,一门科学可以有几个学派,去争,以后会得出真理,社会科学的问题,也是这样,凡属于科学的,都是用辩论的方法,而不是用我们看不顺眼就去整一下的办法,而要改造这样一个情况,需要经过说服的过程。这些议论完全没什么可怕,有什么可怕呀,胡风怎么样呀,胡风不是抓起来了吗?那是国内他搞秘密的小组织,胡风这个人还活着的了,总有一天要把他放了的了,坐班房坐了一个时期,他的错误,他的罪就满了嘛。但是呢,胡风的思想并没有死,在社会上,胡风的思想活在许多人里头,这就是资产阶级思想。凡是科学方面的问题,思想方面的问题,精神的问题,譬如宗教,马列主义,资产阶级世界观还是无产阶级世界观,艺术方面的问题,这些东西都不能用粗暴的方法。两个方法,一个叫压服,一个吗说服,这两个方法里头取一个,还是采取压服的方法?还是采取说服的方法?现在我们有些同志,等不及了,大有去想要压一下,这个压么就不服了,压不服的人;只会把问题压得使我们处于不利的地位,而再来个,他挂笔账共产党犯了错误。你对美国人打过三八线,你用压服的方法,那是对的,反革命你用压服的方法,是对的。对国民党蒋介石、地主阶级使用压服的方法,你为什么用压服的方法对付科学呢?如果我们用压服的方法,我们就没理,就站不脚住,我们就输了,所以应该采取说服的方法。那么说服我们又不会说怎么办呢?问题是那么就学嘛。我们南方有一句话叫化子打狗,叫化子你们叫要饭的,他打狗是专门技术,“叫化子打狗,操习一门”,我们乡下有这么一句话。现在我们是不了解的,不会应付,不会说理,我们要学会说理,要学会写说理文章,学会做说理的报告。老子就是一冲。曾经有过一个同志跟我讲:“搞那么多道理,老手就是搞不惯,老子就是一冲。”他说就是一冲,我说你一冲呀,不能解决问题,你三冲两冲还是不能解决问题。要加以分析,要研究,写出文章来有说服力。至于各种错误的意见在报纸上刊物上发表,开座谈会,评论,会不会把我们的天下搞乱,把人民政府搞倒?我说完全不会,因为他们不是反革命。他们不是特务,他们愿意跟我们合作,大多数,也就是极少数人是仇视我们的,也不是特务,但是他是仇视我们的。许多人还不相信马克思主义的世界观,学了一点,但实际上不相信,再有一些人,相信了一些,不完全,这里头包括共产党员,我们党里头的同志有一些,但是没有完全学通马克思主义的,还不懂,你看两种片面性的人都有嘛,教条主义的,肯定一切的,别的人说不得坏话,本来我们有缺点,不许人说一说;也有另外一种人。什么都是坏的,否定一切。这两种片面性,这两种形而上学,可见得是存在的。

现在我们发行《参考消息》在座的同志都看过这个东西吧,没有这么多吧,有呀。这个东西人们就会说了,是共产党、人民政府替帝国主义尽义务,一个铜板都不要,给他出份报纸来骂自己,共产党无代价替帝国主义出份报纸骂共产党,是不是这样的?看样子是这样的。我每天都看,许多东西就是骂我们呀,现在要扩大发行三十万份,可能有一百万人看,发到县一级范围看的,城市到什么一级呀?来锻炼我们,党内党外都应该受锻炼,应该见世面,锻炼知道一点世界上的事情,敌人怎么样骂我们,敌人家里的事情是怎么样。说会乱,不会乱的。这就是把我们关在房子里,把眼睛封起来,把耳朵封起来,那就很危险,所以有人这么说,这个《参考消息》一扩大发行,那就会使反动气焰嚣张,因此那位同志(这是你们市委黄火青同志告诉我的),希望每一条国际消息都加一按语,那就麻烦了!同志!都要加按语?我们就是叫人们自己去思考,去开座谈会,可能有许多怪言论出来的,怪言论我看是越多越好,就是不要把自己封锁起来,马克思主义,是同他的敌对力量作斗争创造出来的,发展起来的,现在还要发展,譬如我们中国办事情,如果我们不发展,那么我们事情就办不好。马克思主义的原理、原则,到中国实行的时候,就要带有中国的色彩,就要按具体问题的情况来解决。对百家争鸣没信心,对百花齐放怕放出毒来。我看完全不是这样。若采取压服的办法,不让百家争鸣百花齐放,那就会使我们的民族不活泼、简单化、不讲理,使我们的党不去研究说理,不去学会说理,至于马克思主义可不可以批评?人民政府可不可以批评?共产党可不可以批评?老干部可不可以批评?我说没一样不可以批评的,只要谁愿意批评。什么人怕批评呢?就是蒋介石那样的党,蒋介石那样的法西斯主义。我们是马列主义,他们是国民党,我们是共产党。蒋介石那样的“老干部”,他们也是“老干部”,那些“老干部”可是批评不得呀,因为他们的脚是站不稳的,我们本钱大。这个地方没有摆一摆-一摆一摆几十年,我们做工作的时候,不要天天挂在面上,说老子干了几十年,但是人民还是知道的。老干部就是那样容易批评,一批评就批评倒了,吹一口气就倒了?你们都是老干部啊,有没有新干部啊?大概也有吧!现在专讲老干部,我看是吹不倒的,普通的风;除非是十二级台风,十二级台风可以吹倒树,吹倒房子,但是要吹倒老干部我看也还是不行。十二级台风要吹倒共产党,人民政府,马克思主义,老干部不行,新干部也不行,只要他是正确的。至于你有错误,正是要有点风吹一吹,要刮点风。这房子没有风扇哪?这是因为冬天喽!到夏天得搞点风扇,大概夏天人就要有点毛病吧。我们用洗脸打比喻你们是三天洗一次面还是一天洗一次呀?这样一件事情谁也不去研究它,不去研究为什么人们要一天洗一次睑,有的甚至一天洗两次,这个就是因为有灰尘嘛!既然这个皮肤不仅与外面是一条界限-“三八线”,它是与空气接触,它是“黄河为界”,它搞了一付面孔,它而且是个排泄机关,皮肤是个很大的排泄机关,它排泄那么多东西,人的面部我看周身就是它最脏,因为它有七个洞,出了些东西,另外还有许多小的排泄机关,因此每天要洗。那么党就不要洗吗,党也要洗脸。毒草不可怕,反马列主义思想暴露出来不可怕,而且正有作用,我们需要那些东西跟我们来见面,以便和它作斗争,使我们发展起来,这就等于我们种牛痘一样。刚才讲的《参考消息》,要扩大发行。现在不是提倡种牛痘吗?牛痘是一种什么东西呢?是一种病毒,还是一种微生物,种那么点东西放在人里头,使你两个人打架,就产生一种免疫力量,以后就不出痘子啦,出了麻子就不出麻子了,因为他产生了“免疫”了;一辈子不害病的是很危险的。有个人写文章说我讲什么东西,说是小毛病什么东西,我是讲这一辈子不害病的人,就是有一天他要害病了,就是危险,就是了。因为细菌病毒钻他钻得太少了。经常害病的人呢,那些人比较牢固,因为他作斗争嘛!无产阶级思想他跟资产阶级的思想作斗争,马克思主义要跟非马克思主义作斗争才能发展起来。百花齐放,百家争鸣,所以需要就是这个道理。各种艺术的发展,需要百花齐放互相竞争,来比较。百家争鸣,互相竞争,批评讨论,正确的东西才能发展起来。因此我们不要怕这个东西,不要怕,我们要到空气里头去,不要老在有暖气设备的房子里头。


