\section[关于增强党性的决定(一九四一年七月一日)]{关于增强党性的决定}
\datesubtitle{(一九四一年七月一日)}


(甲)中国共产党经过二十年的革命锻炼,现在已经成为全国政治生活中的重要的决定的因素,然而放在我们面前的仍然是伟大而艰难的革命事业。这样,就要求我们更进一步的成为思想上、政治上、组织上完全巩固的布尔什维克党,要求全体党员和党的各个组织部分都在统一意志,统一行动和统一纪律下面,团结起来,成为有组织的整体,没有这样坚强的统一的集中的党,便不能应付革命过程中长期残酷复杂的斗争,便不能实现我们所担负的伟大历史任务。因此今天巩固党的主要工作是要求全党党员,尤其是干部党员,更加增强自己党性的锻炼,把个人利益服从于全党的利益,把个别党的组成部分的利益服从全党的利益,使全党能够团结得象一个人一样。

(乙)我们的党,虽然已有廿年英勇奋斗的历史,虽然已经是和广大群众密切联系的布尔塞维克化的党;但是必须指出:我们的环境,是广大农村的环境,是长期分散的独立活动的游击战争的环境,党内小生产者及知识分子的成份占据很大的比重,因此,容易产生某些党员的“个人主义”、“英雄主义”、“无组织的状态”、“独立主义”与“反集中的分散主义”等等违反党性的倾向。这些倾向,假如听其发展,便会破坏党的统一意志,统一行动和统一纪律,可能发展到小组活动与派别斗争,一直到公开反党使党与革命受到极大的损害。而有这些倾向的个人如不改正,亦会身败名裂。叛徒张国焘的结局,便是明显的历史教训。这些缺乏党性的倾向,今天在党内虽然还不是一个普遍的现象,但在某些个别部分中,在某些同志中确实存在着的,上述的这些倾向,具体的表现在下列各方面:

(1)在政治上自由行动,不请示中央或上级意见,不尊重中央上级的决定,随便发言,标新立异,以感想代替政策,独断独行,或借故推脱,两面态度,阳奉阴违,对党隐瞒。

(2)在组织上自成系统,自成局面,强调独立活动,反对集中领导。本位主义,调不动人,目无组织,只有个人。实行家长统制,只要下面服从纪律,而自己可以不遵守。反抗中央,轻视上级,超越直接领导机关去解决问题。多数决议可以不服从。打击别人,抬高自己。在干部政策上毫无原则,随时提拔,随便打击,感情拉拢,互相包庇,秘密勾搭,派别活动。

(3)在思想意识上,是发展小资产阶级的个人主义,来反对无产阶级的集体主义,一切从个人出发,一切都表现个人,个人利益高于一切,自高自大,自命不凡,个人突出,提高自己,喜人奉承,吹牛夸大,风头主义。不实事求是的了解具体情况,不严肃慎重的对待问题,铺张求表面,不肯埋头苦干,不与群众真正密切联系。

(丙)为了纠正上述违反党性的倾向,必须釆取以下办法:

(1)应当在党内更加强调全党的统一性、集中性和服从中央领导的重要性。不允许任何党员与任何地方党部,有标新立异、自成系统及对全国性问题任意对外发表主张的现象。要求各个独立作区域领导人员,特别注意在今天比任何时候更需要相信与服从中央的领导。应当在党内开展反对“分散主义”、“独立主义”、“个人主义”的斗争。

(2)更严格地检查一切决议决定之执行,坚决肃清阳奉阴违的两面性的现象。

(3)即时发现,即时纠正,不纵容错误继续发展,才更能挽救干部和不使工作受到损失。反对当面客气,背后指斥,一切批评应当是正面坦白诚恳的提出,目的是为挽救,而不是为了打击。应当强调党内团结互助,爱护干部,帮助干部在政治上的进步。但对于屡说不改者,必须及时预防,加以纪律制裁。

(4)要在全党加强纪律教育,因为统一纪律,是革命胜利的必要条件。要严格遵守:个人服从组织,少数服从多数,下级服从上级,全党服从中央的基本原则。无论是普通党员和干部党员,都必须如此.

(5)要用自我批评的武器和加强学习的方法,来改造自己,使适合于党与革命的需要。要求每个党员,特别是每个负责的领导的干部,都深刻反省自己的弱点。把党的利益看得高于一切,任何人都不应有自满自足、自私自利的观念,要提倡大公无私,忠实朴素,埋头苦干,眼睛向下实事求是,力戒骄傲,力戒肤浅的作风.要改造那些把理论与实践、学习与工作完全脱节的现象,这样来更加坚定自己的阶级立场、党的立场与党性。

(6)最后决定从中央委员以至每个党部的负责领导者,都必须参加支部组织,过一定的组织生活,虚心听取党员群众对自己的批评,增强自己党性的锻炼。

