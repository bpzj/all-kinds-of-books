\section[接见几内亚教育代表团、总检察长时的谈话(一九六五年八月八日)]{接见几内亚教育代表团、总检察长时的谈话}
\datesubtitle{(一九六五年八月八日)}

\begin{list}{}{
    \setlength{\topsep}{0pt}        % 列表与正文的垂直距离
    \setlength{\partopsep}{0pt}     % 
    \setlength{\parsep}{\parskip}   % 一个 item 内有多段,段落间距
    \setlength{\itemsep}{\lineskip}       % 两个 item 之间,减去 \parsep 的距离
    \setlength{\labelsep}{0pt}%
    \setlength{\labelwidth}{3em}%
    \setlength{\itemindent}{0pt}%
    \setlength\listparindent{\parindent}
    \setlength{\leftmargin}{3em}
    \setlength{\rightmargin}{0pt}
    }

\item[\textbf{主席:}] 你们是从几内亚来的?

\item[\textbf{贡代·塞杜:}] (几内亚教育部长)是的。我们之中有些人来了十天。我们是几内亚政府和党派出的两个代表团,来中国和中国朋友们接触。我们来了以后,学习了很多东西。我们知道帝国主义想尽一切办法孤立中国,中国是一个大国,伟大的中国人民是孤立不了的。我们把中国的胜利看成是我们自己的胜利,我们要加强同中国友好的甚至兄弟般的关系。

\item[\textbf{主席:}] 我们都是友好的国家,有你们帮助我们,我们就不怕了。你们非洲有两亿多人口,至少百分之九十以上的人都是反对帝国主义的,我们两国是走在一个方向的。你们都是头一次到中国吗?法廸亚拉:(几内亚总检察长)我已经来过一次,享受过中国好客的接待。

\item[\textbf{主席:}] 你来过一次了!

\item[\textbf{法廸亚拉:}] 我是在一九六〇年来的,当时我参加了亚非拉法律工作者代表团,我很高兴也很荣幸地访问过中国,当时受到了主席的接见,我还保存着接见时的照片,这张照片是很好的纪念。

\item[\textbf{主席:}] 你是跟总统一起来的吗?

\item[\textbf{法廸亚拉:}] 我是在总统进行国事访问离开中国几天后参加亚非拉法律工作者代表团来中国访问的,随后我又和韩幽桐同志和她的爱人一起参加了索非亚国际法律工作者会议。一九六〇年访问中国,得到了深刻的印象。中国在一九六〇年——一九六五年又获得了大跃进。

\item[\textbf{主席:}] 没有大跃进,小小的进步。

\item[\textbf{法廸亚拉:}] 还是大跃进。

\item[\textbf{主席:}] (分别问贡代·塞杜和法迪亚拉)你先到,还是你先到的呢?

\item[\textbf{法廸亚拉:}] 贡代·塞杜部长先到。上一次,我访问中同呆一个月,准备十三号去索非亚。这一次,我计划在中国呆一个月,准备十三号去外地参观,增加对中国的了解。

\item[\textbf{主席:}] 到处走走好。

\item[\textbf{法迪亚拉:}] 谢谢,我已经跟我的朋友们说过,非常感谢中国人民,中国政府和主席的邀请。

\item[\textbf{主席:}] 邀请你们,只要你们愿意来看,我们都邀请。不过你们要注意,中国的经验不都是好的,有一部分是好的,有一部分是坏的。

\item[\textbf{法廸亚拉:}] 主席很谦虚。

\item[\textbf{主席:}] 不是谦虚,这是实际问题。世界上没有哪一块地方,哪一个国家只有优点没有缺点。没有哪一个人不犯点错误,也许只有上帝不犯错误,因为我们都没有看见过他。我们的工作,无论哪一项工作,都正在改造过程中,教育工作也是如此。\marginpar{\footnotesize 240}我们过去自己没有大学教授、中学教员、小学教员,我们把国民党留下来的人统统收下来,逐步加以改造。有一部分人改好了,另一部分人还是照他们的老样子。你叫改造,他们不听你的。法院、检察工作也是一样,到现在还没有颁布民法、刑法、诉讼法。(主席问韩幽桐同志:“现在搞了没有?”韩回答:“正在搞”。)大概还要十五年。

\item[\textbf{法廸亚拉:}] 在我看来,规定不重要,重要的是精神,有了精神,办法就有了。规定不过是把已经做过的工作明确下来,规定是次要的。主席:你这个讲得对。现在正在做些工作,譬如改造反革命分子,改造刑事犯,我们有几十年的经验,不只十五年,过去根据地也有些经验。法廸亚拉:在这一方面,一九六〇年我和中国检察长、政法学会会长谈过这个问题,中国重视战犯的改造问题。我们几内亚也有同样的情况。把战犯改造成为对社会有用的人,需要动员人民,把司法机关和人民结合起来,我们两国的问题是相同的,当然其结果是你们取得了很大的成功,而我们现在还在试验阶段。你们无论在研究工作和实际工作方面都取得了巨大的成就,例如你们把最后一个皇帝改造成为公民,使他为人民的事业而工作。主席先生,你信任人民,认为改造人是可能的,这一点是完全正确的。我们两国的社会条件有所不同,但是目的是一致的。

\item[\textbf{主席:}] (面向贡代·塞杜)你是搞教育的。犯了罪的人也要教育。动物也可以教育嘛!牛可以教育它耕田,马可以教育它耕田、打仗,为什么人不可以教育他有所进步呢?问题是方针和政策的问题,还有方法问题。采取教育的政策,还是采取丢了不要的政策;采取帮助他们的方法,还是采取镇压他们的方法。采取镇压、压迫的方法,他们宁可死。你如果采取帮助他们的方法,慢慢来,不性急,一年、两年、十年、八年,绝大多数的人是可以进步的。

\item[\textbf{贡代·塞杜:}] 非常正确。

\item[\textbf{主席:}] 要相信这一点,如果有些人不相信,可以试点。(主席对韩幽桐同志说:“将来把这些写进法典里去,民法、刑法都要这样写。”)要把犯罪的人当作人,对他们有点希望,对他有所帮助,当然也要有所批评。譬如劳改工厂、劳改农场就不能以生产为第一,就要以政治改造为第一。要做人的工作,要在政治上启发人的觉悟,发挥他的积极性,劳改工厂、劳改农场就会办得更好。不仅犯人自己能够自给,而且还能给家里寄点钱。现在我们的劳改工作还有缺点,主要是我们的管理干部不太强,有些地方的方针不对。

\item[\textbf{法廸亚拉:}] 我看他们还是很强的。这个工作不是立竿见影的,已经取得的成就使人充满着希望。因为改变一个机构比较容易,要改造人们的思想比较困难。

\item[\textbf{主席:}] 这个问题不决定于罪犯,而决定于我们。我们有些干部不懂得要把改造人放在第一位,不要把劳动和生产放在第一位。不要赚犯人的钱。

\item[\textbf{法迪亚拉:}] 这点同意。在我们那里有同样的问题,做一件事首先要教育干部才能收到效果。

\item[\textbf{主席:}] 办教育也要看干部,一个学校办得好不好,要看学校的校长和党委究竟是怎么样,他们的政治水平如何来决定。

\item[\textbf{贡代·塞杜:}] 这很正确。

\item[\textbf{主席:}] 学校的校长、教员是为学生服务的,不是学生为校长、教员服务的。我们的法院工作、检察工作是为犯人服务的,不是要犯人为我们老爷服务的。\marginpar{\footnotesize 241}

\item[\textbf{贡代·塞社:}] 这是正确的,我们很同意。

\item[\textbf{主席:}] 整个来说,我们的政府是为人民服务的,人民给我们饭吃,吃了饭不为人民服务,干什么?大使是哪一年来的?

\item[\textbf{卡马拉·马马廸:}] 一九六三年来的,来了两年半了。我刚来中国不到一个月的时候,陪一个代表团去上海,在上海受到了主席的接见。

\item[\textbf{主席:}] (微笑)你这次穿的是白衣服,我不认得你了!

\item[\textbf{卡马拉·马马廸:}] 是的。上次在上海我穿了一身全黑的衣服。

\item[\textbf{主席:}] 是不是谈到这里。我也没有什么道理跟你们讲。你们回去后,请向你们的领导人杜尔总统问候。
\end{list}
