\section[对东北和河南两件报告的批示(一九六三年五月八日)]{对东北和河南两件报告的批示}
\datesubtitle{(一九六三年五月八日)}


宋××同志报告一分,河南省委报告一分,都可以供各地同志参考。河南报告说明,他们在中央二月会议以前是没有根据十中全会指示的精神,认真地进行社会主义教育工作的,或者是没有抓住问题的要点,没有采用适当的方法。二月会议以后,他们抓起了这个工作,并且抓住了问题的要点,采取了适当的方法。第一步,只用了二十几天的时间就训练了十五万多干部。第二步,还要训练一百五十万干部和贫下中农积极分子。然后才是全面铺开,作为第三步。在采取头两个步骤时,并经过试点。这种分步骤的进行工作并经过试点的方法,是正确的。报告所说的其他各项政策也是对的。总之,必须团结绝大多数(百分之九十五以上)的干部和群众,适当地解决人民内部矛盾,即解决程度不同的不正常的干群关系问题,组成有领导的广大干群队伍,以便一致对敌。对坏人坏事,也要有分析。轻重不同,处理的方法也不同。必须以教育为主,以惩办为辅。真正要惩办的,是群众和领导都认为非惩办不可的极少数人。宋××同志所讲的用讲村史、家史、社史的方法教育青年群众这件事,是普遍可行的。社会主义教育是一件大事,请你们检查一下自己在这方面的认识和工作,检查一下是不是抓住了要点和采取的方法是否适当,查一查是否还有很多的地、县、社没有抓住这方面的工作。如果有的话(看来一定是有的),应当在农忙间隙,在不误生产的条件下,抓住进行。上半年作不完,可以在下半年作。同年作不完,可以在明年作。特别要注意分步骤的方法,试点的方法和团结大多数、孤立极少数的政策。

