\section[论合作社――在边区高干会上的讲话(一九四三年十月)]{论合作社――在边区高干会上的讲话}
\datesubtitle{(一九四三年十月)}


今年边区在发展生产上,又来了一个革命,这就是合作社方式。把公社劳动力组织起来,发动了办众生产的积极性,提高了劳动效率,大大发展了生产。

在过去,束缚边区生产力使之不能发展的,是边区的封建剥削关系的一半地区,经过土地革命,把这种封建束缚打破了,一半地区经过了减租减息以后,封建束缚削弱了,这样合起来,整个地区就破坏了封建剥削关系的一大半,这是第一个革命。

但是,如果不从个体劳动转移到集体劳动的生产方式的改革,则生产力还不能获得进一步的发展。因此,建设在以个体经济为基础(不破坏个体的私有财产基础)的劳动互助组织,即农民的农业生产合作社,就是非常需要了,只有这样,生产力才可以大大提高,现在陕甘宁边区的经验,一般的变工札工劳动是二入可抵三人,模范的变工札工劳动是一人可抵二人,甚至二人以上。如果全体农民的劳动力都组织在集体互助劳动之中,那么,现有全边区的生产力就可以提高百分之五十到百分之一百。这办法,可行之于各抗日根据地,将来可以行之于全国,这在中国经济史上是要大书特书的。这样的改革,生产工具根本没有变化,生产的成果也不是归公而是归私的,但人与人的生产关系变化了,这就是生产制度上的革命。这是第二个革命。

全边区现有全劳动力35万个,今年经常组织在集体劳动中的变工队札工队的已有三万余人,占全劳动力总数的十分之一,而临时性的劳动互助,就延安县说,有百分之七十,明年还可发展。如果各县经常的集体劳动组织能由十分之一提高到十分之二、三,可达十万人左右,再加上临时性的劳动互助组织能向延安看齐,还有半劳动力也参加组织,这就是一种很大的劳动力。

我们的部队、机关、学校的生产也带着合作社的性质,比如一个连,就是一个小合作社,一个旅就是一个大合作社。在各种部队、机关、学校合作生产之中,扬家岭运输队在改组前,有大车八辆,驮骡十六头。照普通情形,每日最低限度应运输物品27万斤,但实际运输的只有19万斤,经费开支,则除照一般的供给标准外,每月还要贴六万元,今年在公私两利的原则下,把运输队改为运输合作社,公家以大车八辆,驮骡十六头(后增为23头)及全部用具作为80股,运输员二十名以身份股名义作为二十股,共l00股,每月按股数二、八分红,一切人员、牲口、装置等费用开支,均由合作社自行解决,给公家运输物品依照里程远近按斤给运费,运输员的生活由运输合作社适当改良。这办法经过解释后,全体运输员一致赞成,执行结果,运输量由每个月19万斤增为38万9千斤,增加了百分之百,超过了普通的运输力量百分之三十。同时大大提高了运输员对工作的责任心和积极性,节省了许多经费和工具,又更加爱护牲口。比方,过去装粉的口袋破了碗大的洞无人管,现在运输员随身带着针线,缝补口袋。过去贪污马料是公开的秘密,现在却没有这种贪污了,过去车马用具稍一损坏,就要求公家补充新的,现在只要能凑合着用,就对付着用下去,对牲口,过去是粗心大意的,现在也逐渐喂好了;运费开支,改组后比以前减少三分之一,过去除照顾供给标准外,每年还要六万元,现在不用半文津贴,还每月获利数×元。

各机关釆用这种办法后,也得到很大效果。管理局运输营130头牲口,23辆大车,在未组织合作社之前,每月运输量只有120万斤,改组为合作社后,每月运输量提高到185万斤,增加了百分之五十,因此,请大家考虑这种合作办法,是否可以广泛运用于我们的公营工厂及公营农场。

这一套办法,资本主义国家及国民党是不能做的,只有我们才能做。因为我们不以剥削人民为目的,我们贯彻公私兼顾的方针。军队中如三五九旅战士的纺毛线,用柳榆树条编成各种用具。规定:凡动用公家工具的手工劳动,以其结果五分之四归公,五分之一归私,凡不动用公家工具的,则以三分之二归公,三分之一归私。这种办法,一方面解决了公用品的需要,同时无疑增加了战士的津贴,也含有合作社的因素。

合作社性质,就是为群众服务,这就是处处要想到群众,为群众打算,把群众的利益放在第一位,这是我们与国民党的根本区别,也是共产党员员革命的出发点和归宿。从群众中来到群众中去。想问题从群众出发,而又以群众为归宿,那就什么都好办。因此,我们每个共产党员都要替人民着想,部队的负责同志要替战士着想,机关学校的负责同志要替大厨房着想,替杂务人员着想。这种群众观点的生产学说,打破了过去各种不正确的“学说”。也只有这种为群众的学说,才能把生产搞得好。

我愿各地同志注意提倡合作社的生产,部队机关学校的生产是一种合作社,农村的集体互助劳动又是一种合作社。此外还有包含各种业务在内的综合性合作社,被称为运盐队的运输合作社,工人们集体互助的手工业合作社,把这许多样式的合作社都发展起来,全体公私群众就会变为富裕的人。在敌后各根据地的目前困难情况,也就能够克服了。

