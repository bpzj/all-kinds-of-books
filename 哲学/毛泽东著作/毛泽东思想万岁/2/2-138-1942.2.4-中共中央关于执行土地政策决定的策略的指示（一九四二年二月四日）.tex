\section[中共中央关于执行土地政策决定的策略的指示(一九四二年二月四日)]{中共中央关于执行土地政策决定的策略的指示}
\datesubtitle{(一九四二年二月四日)}


(一)中央政治局一月二十八日所通过的关于抗日根据地土地政策的决定,是综合五年来各地经验而得到的结论。它的基本精神是先要能够把广大的农民群众发动起来。如果群众不能起来,则一切无从说起。在群众真正发动起来后,又要注地主能够生存下去。所以在经济上只是削弱(一定要削弱)封建势力而不是消灭封建势力。对富农则是削弱其封建部分,而奖励其资本主义部分。在经济上,目前我党的政策系奖励资本主义生产为主,但同时保存地主的若干权利,可以说是一个七分资本三分封建的政策。在政治上则实行三三制,使地主资产阶级觉得还有前途。所有这些,都是为着拆散地主资产阶级与敌人及顽固派的联合,争取地主资产阶级的大多数站在抗日民主政权方面,而不跑到敌人与顽固派方面去,跑去了你则可以争取回来。

(二)联合地主抗日是我党的战略方针。但在实行这个战略方针时,必须釆取先打后拉,一打一拉,打中有拉,拉中有打的策略方针。当广大群众还未发动起来的时候,一般地主阶级是坚决反对减租减息与民主政治的,在这种时候我们必须积极援助群众打击地主,摧毁其在农村中的反动统治,树立群众力量的优势,才能使地主阶级感觉除了服从我们的政策,便不能保持他们的利益,便无其他出路。在这种广大群众的热烈斗争中不可避免地要发生一些过左的行动,而这些过左的行动,如果真正是广大群众自觉自愿的行动,而不是少数入脱离群众蛮干的(这是绝对不许可的原则问题),则不但无害,而且有益,因为这足以达到削弱封建势力发动群众之目的。在这时,如果畏首畏尾,害怕群众参加,那就是右倾错误。这是策略斗争的第一阶段(打的阶段)。但是这个阶段应被联合抗日的战略所限制,不能听其自然发展下去,以致迫使地主阶级跑到敌、顽方面坚决反对我们,或跑出去了也不愿回来,妨害抗日战争与抗日根据地的巩固。因此,党的策略不是在事先防止这些过左行动的发生,以致妨害群众之充分发动与充分起来,而是在群众已经充分发动充分起来后,能够及时的说服群众。纠正过左行动,给与地主以交租交息及政治上的三三制,保证地主的人权政权地权财权,使其感恩怀德,愿与我们合作,达到抗战之战略目的。这就是策略斗争的第二阶段(拉的阶段)。在策略斗争的第一阶段中,也不是一切打倒,而是争取一部分倾向我们的地主(打中有拉),中立、麻痹一部分动摇不定的地主,集中火火力打击一部分最顽固的地主(但与内战时期打击地主的内容与形式都不相同)。在策略斗争的第二阶段中,我们必须表现极力宽大,认真实行三三制与交租交息,认真保障人权政权地权财权,公开批评内部宗派主义(关门主义),纠正过火行动。在这种时候,如不着重说服党员说服农民争取地主,就不能拆散地主与敌顽的联合,就有使我党与农民陷入孤立以至失败的危险。但在纠正过火行动与作自我批评时,必须同时注意保护干部与群众的积极性、热烈情绪或热气,须知我们正确的批评过火行动与宗派主义,决不是向这种热气泼冷水,使干部造成消极、群众失望、地主反攻的局面。在正常的斗争过程中,应该有一种酝酿斗争的准备阶段。在这个阶段中,地主还是优势,农民正在准备斗争,如果把这个作为第一阶段,则实行斗争(打)为第二阶段,团结抗日(拉)为第二阶段。在晋察冀区域除了雁北及平西尚未普遍与彻底按照我党政策解决土地问题以外,其它基本区域都经过了酝酿,斗争、团结这三个阶段。这是正常的策略模范。在其它若干根据地中,也有这种模范。所有这些都是为着执行联合抗日这个战略方针的总过程中,应该极力注意的策略阶段。

(三)各地过去在执行土地政策中所发生的过左错误,大体已经纠正。在今天一切为广大群众所拥护,而地主又已不至严重争议的事件,应作为已经解决,不再变动,维持良好的抗日秩序。但对三三制之没有彻底执行,及地主农民间尚有重大争议的事件,仍须着重纠正错误。

(四)目前严重的问题是有许多地区并没有认真实行发动群众向地主的斗争,党员与群众的热气都未发动起来,这是严重的右倾错误。这种错误不但在较差的根据地中是严重的存在着,就是在最好的根据地中,也有一部分区域尚未实行减租减息与发动群众斗争。因此目前需要强调反对这种右倾,要求一切没有实行减租减息,没有发动群众热情的地区,在广大农民群众自觉自愿而不是少数人包办蛮干的基础之上,迅速实行减租减息,迅速地把群众热情发动起来。各地党部必须检查此问题,如有些人釆取漠不关心与官僚主义的态度;就须向他们指明加以纠正。

(五)在农民已经充分发动,彻底执行了减租减息,经过了“打”的阶段,因而进入了“拉”的阶段的地区,由于我们开展自我批评纠正过火行动,彻底实行三三制与保障地主的人权政权地权财权,地主阶级必然要抓住新政策之有利于自己方面加以扩大和农民作斗争。这一阶段(拉的阶段)各阶级的争议,只能采取民主的合作的合理的方式去进行,而文化落后的农民群众,甚至区多干部,遂容易被老奸巨猾的地主所欺骗,或被地主收买操纵区村政权,或被地主打击而不敢回击。因此必须教育县区村三级干部学会与地主作合法斗争的本领,熟悉政府的法令,熟悉拉中有打的策略,以便对付某些奸猾地主的无理进攻,必须防止被收买。


(六)减租是减今后的,不是减过去的,减息则减过去的,不是减今后的。大体上以抗战前后为界限。在减息问题上:第一,应当允许农民清算旧账(包括公账、私账),以此作为发动群众的手段,到了群众已经充分发动,才把双方争论加以调停,使归平息。第二,抗战以后是借不到钱的问题,不是限制息额的问题。各根据地都未认清这个道理,强制规定,如息额不得超过一分或一分半,这是害自己的政策。今后应该听任农民自己处理,不应规定息额。目前农民只要有钱贷,即使利息是三分四分,明知其属于高利贷性质,但子农民有济急之益。同时政府每年的建设费中应以百分之七十至百分之八十投于农村,作为对农民的低利贷款(包括合作社贷款在内),以发展各根据地基本的农业经济,而以百分之二十至百分之三十投入公营工商业及私人商业。须知发展农业不但是农民的利益,而且就是扩大政府税收的最好与最可靠的来源,就是解决财政部门问题的基本政策。

(七)中央关于抗日根据地土地政策决定及三个附件都已公开发表,各地应立即公布广为宣传,认真实行。这是我党在新民主主义革命阶段的最长时期的土地政策,不但今天必须实行,而且还有很长时期要实行的。至于本指示则是专门对党内,不得公开发表。每一个根据地内应利用会议,党校、文件,使党的基本干部懂得党的战略与策略方针。然后经过他们使下及干部懂得,使这种具体的策略教育确收到成效。

