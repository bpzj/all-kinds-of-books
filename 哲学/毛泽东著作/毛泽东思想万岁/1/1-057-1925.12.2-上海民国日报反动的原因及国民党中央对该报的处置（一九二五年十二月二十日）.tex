\section[上海民国日报反动的原因及国民党中央对该报的处置(一九二五年十二月二十日)]{上海民国日报反动的原因及国民党中央对该报的处置}
\datesubtitle{(一九二五年十二月二十日)}


国民党右派叶楚伦等主持的上海民国日报,从十一月二十日起,即从登载北京右派会议的通电之日起,已经宣告做了反动派的机关,宣告脱离了革命的国民党,宣告与帝国主义军阀从此妥协,宣告做了帝国主义军阀宣传机关之一种。上海民国日报的反动是毫不足奇的,因为这个报从前是叶楚伦等的私人报,去年第一次全国大会后才收归党办,但是自始即不能作为国民党的言论机关。该报常常不登或删节反帝国主义及军阀文字之该报所执为‘过火者’替帝国主义军阀隐恶扬言无微不至,对国民党国民政府的革命策略丝毫不能宣传;去年江浙战争该报完全丢掉国民党地位做了安福系卢永祥的机关;南洋烟草公司压迫工人流离失所时该报为资本家大登其压迫工人有理的广大工人的新闻则拒绝登载或删节而后登载;本年五卅运动中该报反帝国主义的是清远不要研究系的机关报,近数月来广东的反帝国主义运动反军阀运动肃清反革命派等重要消息该报都不肯登载,而对于孙传芳的军事消息则占满了篇幅,凡此该该报反动的予兆,个个革命派,党员都早已十分不满意该报,该报此时不过借北京右派会议内“正式反动”的宣告而已。该报自从十一月二十日表明了“正式反动”以后,连日做了许多攻击左派的宣传,这自然是该报的本分。因为该报如果不做攻击左派的宣传,便不能取得右派机关报的资格,便不能向上海工部局和北京段祺瑞邀功使他们承认该报的反革命地位,便不能达到该报反动的目的,上海民国日报的反动我们觉得十分有理由。这个证明了国民党左派之强固,证明了中国反帝国主义反军阀运动之发展,证明了中国革命派反革命派已到了短兵相结的时候,证明了中国国民革命之成功已是快要到来。在中国现在的时候一切中立派的人中立派的报都一定迅速变化其态度,或者向左跑入革命派,或者向右跑入反革命派,从前灰色的中立的面具现在是不能再载自了。明白了这点,我们便可以明白上海民国日报何以要反动,晨报醒狮周报等可以近来攻击国民党,共产党此以前特别激烈,以及国民党右派领袖们何以必须于此时在北京开会以反对广州左派领袖的中央执行委员会和国民政府。关于上海民国日报之处置,国民党中央执行委员会业经议决派员查办并通电各地各级党部申明该报之反动荒谬行为。兹将电文录下:

必送北京翠花胡同八号,并转直隶,山东、河南、山西、陕西、察哈尔、绥远、热河、哈尔滨各党部.上海望志路永吉里江苏省党部,杭州浙江省党部,长沙湖南省党部,南昌江西省党部,武昌湖北省党部,广州广东省党部,并转各级党部,北京天津上海汉口各报馆,海外各地党部,各同志均鉴:上海民国日报,近为反动分子所盘踞,议论荒谬,大悖党义,已派员查办。谢闻。中国国民党中央执行委员会。

<p align="right">(抄自“政治周报”第3期第5页,1925年12月20日广州政治周报出版社出版)</p>

