\section[在八大二次会议上的讲话(三)(一九五八年五月二十日下午)]{在八大二次会议上的讲话(三)(一九五八年五月二十日下午)}
\datesubtitle{(一九五八年五月二十日)}


(一)再讲破除迷信。

第一机械部发了一个材料,不知印发了没有?搞了四十一科学家、发明家的小传,都是比较穷苦的。其中只有七个是工程师,比较有社会地位的,其他都是贫苦的,或工人出身。农民出身的。如瓦特就是工人。这批材料很有用处,已经印发给同志们,希望各部门都搞一下这种材料。这个材料是从十八世纪搞起的。是一百多年的事。一百多年也好,二百多年也好。无论从何时搞起,对破坏迷信很有好处,对我们很有帮助,可以帮助我们破除迷信,打掉自卑感。工农、小知识分子有自卑感,可以破除。上回来讲农林水(工业交通)、卫生应该加上。农林水,政法文教,卫生各部门,都可搞这方面的材料。

(二)再讲讲以普通劳动者姿态出现的问题。

这个问题很重要,之所以重要,是因为有一些人,老子天下第一,看不起人。不是平等待人,靠老资格吃饭,特别是做了大官的,靠做大官吃饭,不是以普通劳动者的姿态出现。提出这个问题,要靠大多数人做到这一点,事情就好办了。过去好多官僚主义者,不以普通劳动者的姿态出现。

“你是我管的”。就靠这个吃饭,妨害创造性的发展。要破除这种东西,在大部分人中扫除官气。只看谁有真理就服从谁,不管他是挑大粪的,挖煤的,扫大街的,贫穷的农民,真理在谁手里,就服从谁。官做的再大,真理不在他手里,就没有理由服从他。多数人扫掉了官气,剩下少数人就孤立了,就不敢作怪了。应该说,官气是一种低级趣味,不是高级趣味。不是共产主义精神。相反,以普通劳动者姿态出现才是高级趣味。这样一来,我们所要防止的大国沙文主义就可能防止。如果全党大多数。特别是领导干部.都谦虚(科学谦虚)。就可以防止.出了也不可怕。

(三)再讲一个外行领导内行问题。

外行领导内行,是一般规律。差不多可以说,只有外行才能领导内行。过去右派提出了这个问题。闹得天翻地覆,说外行不能领导内行。

只有外行才能领导内行,是否可以这样讲呢7在这个问题上,我们是处于被动地位。过去报纸在这个问题上,对右派的批判不系统,讲的不透。为什么说外行领导内行是一般规律?因为人人是内行,人人是外行。世界上一万个行业,一万行科学技术。每人只精通一行。如梅兰芳会唱戏,但只会青衣。而旦角就是青衣、花旦,老旦就不如李多奎。此外还有其他角色,老生、小生……。一万行里头每人只精一行。所以说人人是内行,人人可能成为内行,但是人人又是外行,对九千九百九十九行是外行。一个人精通两三行或四五行,就很厉害了。就算十八般武艺俱全,和薛仁贵一样,对一万行是九千九百八十二行是外行,隔行如隔山,内行少,外行多,岂不是人人是外行7做领导工作,除了本行外,对其他行业也应当知道些、摸一摸,略熟一门,有点常识是必要的。如做党的工作的,熟悉工业,农业等是必要的。但要熟悉多是不可能的。我就只会坐飞机,不会开飞机。中学有十几门科学,大学就更多。许多事情是由业余转化的。如孙中山,开始是被人看不起的,当个小医生。二十岁搞革命是不合法的,开始当医生他是内行,搞政治是副业,后来搞革命.政治转化为正业,不行医了,医又是副业,甚至不干了,变成外行了。但是,这时可以管医生了。政治家是搞人与人的相互关系的,是搞群众路线的。这个问题我们要很好研究一下。因为有许多工程师,科学家看我们不起,我们有些人也看不起自己,硬说外行领导内行很难。要有点道理驳他。我说外行领导内行是一般规律。如梅兰芳叫他当总统就不行,他只会唱戏。

(四)再讲一个插红旗,辨风向的问题。红旗就是我们的五星红旗。插什么旗子?插红旗还是插白旗?除了南北极,世界上任何地方都是要插旗子的,从南极到北极都是要插旗子,现在南北极也在插旗子,美国插了,苏联也插了。可惜我们还未去。北极南极都没去。将来有一天我们也开一只船到南极北极去一趟。凡是有人的地方,都要插旗子的,不是红旗子,就是白旗子,或者还有灰旗子。不是无产阶级插红旗,就是资产阶级插白旗。去年五六月间,机关、学校、工厂、某些合作社,究竟插什么旗,右派和我们双方都在争夺,资产阶级要插白的,我们要插红的。现在还有少数落后的工厂或工厂的一个车间,合作社,学校,连队或其中的一部分,那里插的什么旗子?不是白旗就是灰旗。我们要到那里走一走,到落后的地方走一走,发动群众大鸣大放,贴大宇报,把红旗插起来。一个生产队也要有个旗子插起来。

庸俗的谦虚,就是不插红旗。不插红旗就是低级趣味,虚伪的谦虚。“闭口道士”,不吹吹搭搭,这种谦虚应当批判。有这社会舆论,奖励这种作风,不挺身而出,不敢想敢说敢作,这是从《儒林外史》那里学来的。为了插旗子,就要提高嗅觉,学会辨别风向,看刮什么风。不是东风压倒西风,就是西凤压倒东风。这是苏州姑娘林黛玉讲的。世界上总是分党派的。社会上的人总是分左、中、右三种,有的处在先进状态,有的处在中间状态或者落后状态。现在的任务,就是依靠先进分子争取中间状态的人,带动落后分子。要争取中间分子站到左边来,即插起红旗。右派插的白旗,是资产阶级的旗子,中间分子插的旗子是灰的白的。唐朝有个刘知机,说写历史的人要有三个条件:才、学,识。才是才干,学是学问,识不是不是指知识,是指善于辨别风向。我特别请同志们注意的是“识”的问题,不讲前面两者,要善于识别风向,要有识别力。识别力有其极端的重要性,尽管有些人很有才,很有学问,但对识别风向很迟钝。斯大林讲,要有预见性。预见性是指的识别风向,未刮风,刮小风时就知道刮大风.站到看台上。什么东西看不到,是不好的。没有预见性,已经相当普遍存在了,还看不到,这种状态给右派可乘之机。你看不到,位置由他们占领,他就来了。

要驳右派,插红旗。随时随地,不要怕插红旗,凡应该插红旗的地方赶快去插。每一个山头、平原、村落,都要把红旗插起来,每小党委、机关、部队、工厂,合作社,都应把红旗插起来。哪里没有红旗,哪里就要插。现在许多地方并非都是红旗,参差不齐。有的刚刚插起红旗,过几年又不红了。又落后了,不红了。经常变化,这也是自然状态,旗子变了,就要换。

(五)讲一个红白喜事.上次讲对付可能的灾害,主娄是讲的战争和党内分裂,灾难有大、中、小。我讲的是大的战争分裂。

中国人把结婚叫红喜事,死人叫做白喜事,合起来叫红白喜事,我看很有道理。中国人是懂得辩证法的。结婚可以生小孩,母亲分裂出孩子来.是个突变,是喜事。一个母亲分数出三个、两个,一个小人出来。多子女的分裂出六个、七个,七个、八个,甚至十个,像航空母舰一样。我不是不赞成节育,我是讲辩证法,是说新事物的发生,人的生产,这是喜事,是变化,一个变两个,两个变四个。至于死亡,老百姓也叫喜事。一方面并追悼会,哭鼻子,要送葬,人之常情。另一方面是喜事。也确实是喜事。你们设想,如果孔夫子还在,也在怀仁堂开会,他二千多岁了,就很不妙。

讲辩证法而又不赞成灭亡,是形而上学。有灾难,是社会现象。灾变,是宇宙根本的规律。生是突变,死也是突变。由生到死几十年的渐变。假如蒋介石死了。我们都会鼓掌。杜勒斯死了,我们没有掉眼泪。这是因为旧社会事物的灭亡是好事,大家都希望。新事物的产生是好事,新事物的灭亡当然不好。如一九零五举俄国革命的失败。南方我们根据地的丢失,等于现在的苗子被雹子和暴雨打掉,这当然不好,这就发生补苗问题。我们共产党人希望事物变化的,所以跃进,就是和过去不同……突变优于量变。没有质变,不可能突变。没有量变不行,否定量变就会冒险主义。平衡的破坏是跃进。平衡的破坏优于平衡。不平衡,大伤脑筋是好事。如一机部,冶金部,地质部等,日子不好过,大家压他.压得很紧,都要大大发展.这是好事。平衡,量变,团结是暂时的,相对的。不平衡,突变,不团结则是绝对的,永远的。许多不团结被克服成为团结。团结任务的提出。就是因为有不团结。一个人团结了,两个人就有不团结。我们党有一千二百万党员,各种出身的人。要常开会就团结。我们有南宁,成都会议作准备,有去冬今春水利积肥运动等。大跃进,城乡结合,工农业并举。中央地方工业并举,火中小结合。都出来了。所以年年讲团结,就是因为年年有不团结。每人想法不同.党员水平不同。就必须开会。常任代表制搞对了。过去没有每年开一次代表大会的制度。开别的会。现在每年开一次极好。不开会,想法不同。开会就把比较合理的意见采纳了,会上作出决议,作个报告发表出来,全国一致。这种会议。有些地委、县委书记参加。使我们的会更好了,他们讲了很多好的意见。

不仅年年要讲团结.每天都要讲团结。因为每天都有分裂。细胞分裂。新陈代谢。旧的不死.对小孩发育不利。新陈代谢是姓陈的走了,姓新的来了,姓新的把他赶出去了。不是赶陈伯达。老的作揖打躬。新的把旧的赶走了。长江后浪推前浪,世上新人换旧人。事物都是变化的。没有不变的事物。现在有一百零二种元素,原来开头还没有这么多。是后来变化的,再过几万万年。就可能不是一百零二种了。可能是:百多种元素。事物是要变化的.要转化到他的反面。我们一千二百万党员,每天总有出党的,每天总有斗争,有受批评的..湖北省有哥妹俩贴大字报,哥哥老资格有官气,不是以普通劳动者的姿态对待妹妹,妹妹请人写了一张大字报,只好贴,真理在妹妹手里,结果哥哥输了,妹妹赢了。可见学问少年纪小的比较有真理。浙江父亲儿子争论密植。儿子赞成,父亲反对,结果父亲输了,儿子赢了。这是一般规律。做父亲哥哥总是有相当危险就是了。比输了,也没大关系,出路一条,就是检讨投降。这就好了,团结起来了。无非是兄妹开荒,哥哥比输了。团结了,父亲和儿子比要不要密植结果父亲说:我服了你。向妹妹、儿子认输就是了。要以普通劳动者姿态出现。免得危险。

我讲的是要防止不利于人民、不利于党的大灾难。如世界大战,党内分裂。像×××、高岗那样的分裂,我们党有四次分裂。一是陈独秀,二是罗章龙,三是张国焘,四是高岗。由中央,整下去了。王明三次“左倾”路线,是以合法形式出现的,我们对他采取治病救人,经过批评达到团结的态度。容许他们继续工作。只要有党,新的分裂是可能有的。只要有党,就有可能分裂,一百年后还会有。我们的办法是团结一一批评一一团结,惩前毖后,治病救人。这个党顶多百把年。也许几十年就要改变.大概到二十一世纪,现在到二十一世纪只有四十二年,世界会有很大的变化。四十二年要出多少煤、钢、电,十五年赶上美国。还有苏联赶过美国。我看苏联不要十五年。

这样讲大家可能不舒服。我就讲了才舒服。讲了大家有思想准备。南斯拉夫不是搞分裂吗?还有美国的福斯特。我们过去陈独秀、罗章龙、张国焘、高岗搞分裂,最近有了丁玲,山东的李峰,广东的古大存,广西陈再励,安徽李世农,河南×××。青海孙作宾。新疆拉甫古也夫,浙江沙文汉……也搞分裂。北京政法系统垮了,文艺界人类灵魂工程师垮的更厉害。这些垮了有什么不好?世界上总是有分裂的。新陈代谢嘛,年年有分裂,月月有分裂,日日有分裂,像细胞死亡一样,年年有团结,月月有团结,日日有团结,像细胞生长一样。第一国际,第二国际,第三国际都有发生、发展和死亡的过程。情报局也没有了。现在可以用莫斯科会议的方式来代替,十二国相约,苏联为会议召集人,有事开会。新的方式出现了,订了一个内部协定,波兰不赞成公开发表,未公布。所以一、二、三国际都有发生、发展、灭亡的过程。

(六)设立对立面。设置对立面有两种。一种是社会上本来存在的。如右派本来就存在。放不放是政策问题。我们决心放,大鸣大放,放出来作为对立面,发动人民起来与他辩论。与他对抗。把他搞下去。小学教员有很多右派,在三十万右派中有十万。三十万右派的对立面是存在的。放出来教育了六亿人民,对我们有利。

另外一种是自然界不存在的,带有物质条件。如修水坝,可以用人为的办法设对立面。抬高位置再让水流。使它有个落差,可以发电,可以行船。如开工厂。也是设置对立面。鞍钢是日本人修的,长春汽车厂是新的。是人工设置的对立面。自然界没有的,可以人为地造,但有物质基础。卫星上天是人为的。找到规律就上去了。

我们是乐观主义者,不怕分裂,分裂是自然现象,××××对苏联有帮助,陈独秀、罗章龙、张国焘、高岗分裂对我们也有帮助。两次王明路线。内战时期三次“左”倾路线,抗日战争时期的右倾路线。教育了我们党。这许多对立面都有好处。当然,要是人为的造一个×××、陈独秀、高岗也难造,也不必要,只要有一定气候,他就由来,没有什么可怕,出来了是不是要替他们开庆祝会?我们不开。克服这种修正主义者,我们开庆祝会。这种事发生我们也有忧愁。至少,一个月总有件把事忧愁。

乐观主义是我们的主导方面。忧愁的也有。右派出来时,大家能不发愁?柯庆施是乐观主义者,右派进攻不着急,我就有点发急,着急就要想办法。如天天高兴,没有什么事。就会被右派打倒,这就要讲领导艺术。领导得好,分裂由坏事变成好事。早早预见到,也可以使之不发生,消患于未然。像锄草一样,农民有预见,农民积累多年的经验,深知禾苗生长的好,必须除杂草。有一千二百万党员中,二、三万人有更高的觉悟就不怕。在座的只有一千多人,经过我们团结更多的人。如一万、二万,三万人,有更高的觉悟,就是能有预见性。搞好一点就不怕分裂.怕什么,怕也不行。世界大战我们要作准备,我们争取不打,但打也不怕。打了再建设。

我们坚持惩前毖后,治病救人。对犯了错误的,要允许他改正。除非到丁玲那种地步。潘××犯了路线错误.但要允许他改过。现在我们很团结,没有什么大事,中央地方都很好。经过整风。反冒进的问题现在搞清楚了。从团结的愿望出发,经过批评。在新的基础上达到新的团结.扩大一点讲一讲为的是使大家自觉起来,有精神准备,引起大家注意,我们是乐观主义者。

昨天××讲民歌讲的很好。在座的一直到支部,每个乡可出一集。九万个乡出九万集。如果太多了,少出一点,一两万集也好,出万把集是必要的,不但新民歌还有老民歌,革命的,一般社会上流行的都要,办法是发纸,一个人发三张纸。不够,发五张,不会写就请哥哥、妹妹,不行,请柯庆施写,他是提倡教育文化,乡乡办大学的。我说工、农、兵、学、商、思,黑龙江把思想旗到第一位也好,思想悬实际的反映。以虚带实,以政治带业务,以红带专。这就是“思、工、农、兵、学、商。”斯大林的两个口号缺乏辩证法,讲“技术决定一切”,政治呢?讲“干部决定一切”,群众呢?列宁游得好,“苏维埃加电气化就是共产主义”,是对的。苏维埃是政权,有了人民政府才有可能。如果北京是蒋介石政权,我们就不会在这里开会。苏维埃是政治。电气化是技术。是动力,苏维埃和电气化结婚,政治和业务结婚,生的儿子就是共产主义。政治与业务是对立的统一,但它俩结了婚,就会产生儿子,我们首先产一个七年超过英国,再有八年超过美国。第一个儿子叫超英,第二个儿子叫超美。

这两个月要抓一下,有的省委书记建议七月不开会,搞一九五八年计划,八月五日开会好,那时可决定农业的丰欠,开半个月二十天。再开三天散会。


