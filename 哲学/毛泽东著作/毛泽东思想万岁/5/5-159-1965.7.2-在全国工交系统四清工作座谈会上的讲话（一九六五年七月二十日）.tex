\section[在全国工交系统四清工作座谈会上的讲话(一九六五年七月二十日)]{在全国工交系统四清工作座谈会上的讲话}
\datesubtitle{(一九六五年七月二十日)}


社会的渣滓,也是不可少的,社会上没有这些渣滓才怪呢!我看一万年也会有的,不然就没有正确的了,真理是对谬误而言,唯物论是对唯心论而言,辩证法是对形而上学而言。一万年以后,形而上学,唯心论还是有的,不然社会上就没有矛盾了,斯大林晚年就犯了这个错误,他只是强调苏联社会各阶级人民的一致性,而否认了不一致性。他以为社会主义社会就没有矛盾了,这是指一九三四年到一九三七年,一九三八年,实际上隐藏了深刻的矛盾。他不承认社会主义社会存在阶级和阶级斗争,结果事物走向了反面,肯定变成了否定。当然,否定又变成了否定之否定,这就是说,苏联人民不可能长期被修正主义统治下去。


