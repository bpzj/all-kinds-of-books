\section[在四个副总理汇报时的插话(一九六四年五月)]{在四个副总理汇报时的插话}
\datesubtitle{(一九六四年五月)}


一、一定要很好地注意阶级斗争。农村四清是阶级斗争,城市五反也是阶级斗争。城市不要吹牛,五反工作不能在今冬明春结束,要准备三、五年才能结束。城市也要划成份。至于如何划法,将来作时要定出标准。不能唯成份论,马、恩、列、斯出身都不是工人阶级。

二、关于第三个五个计划。一定要把干劲鼓足,一定要把后备留到,不能凭我们的年龄来订计划。计划一定要有客观根据。我七十多岁了,此你们大些,但是不能凭着我们在临死以前看到共产主义来订计划。第三个五年计划,我看还是要注意数量多了,而质量没有更多的注意。

计划绝不能凭主观愿望,一定要有客观根据,要切实可靠。

三、自力更生问题。自力更生十分重要,不仅一个国要自力更生,就是一个工厂,一个人民公社、生产队也都要自力更生。在人民公社管理工作中,真正有成绩的是靠自力更生的那些公社,凡是有贷款的公社和生产队办的就要差些。现在我们全国真正自力更生的公社有三个,一个是江苏的陈永康公社,一个是山西的陈永贵公社;另一个是山东的曲阜的陈××公社,他们从来没有向国家要一个钱,完全靠自己力量搞起来的。

四、干部参加劳动问题。干部一定要参加劳动。现在这个问题还没有很好解决。领导干部要蹲点,不能只靠听汇报。部长都要蹲点,不然就不开会。……

五、今年的小麦估计可以比去年增加五十亿斤。看来今春多雨,是利多害少的。

六、第三个五年计划要从农村搞那么多人进城当工人,不是个办法。


