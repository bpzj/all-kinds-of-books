\section[从我国社会主义建设看《苏联社会主义经济问题》一书(一九五八年)]{从我国社会主义建设看《苏联社会主义经济问题》一书}
\datesubtitle{(一九五八年)}


一、关于社会主义制度下经济法则性质的问题

1、作者摸出了国民经济有计划按比例发展的法则与政策有区别,这一点很好,主观的计划应力求适应客观的法则。他提出了问题,但没有展开,可能他自己不大清楚。他们的计划,在更大的程度上反映客观法则,值得研究。

2.社会(主义政治)政治学多研究它的必然性,对客观的世界要逐步认识,不到时候没有展开矛盾,不反映到人们头脑中来,才能认识,几年不晓得以钢为纲,今年四月才抓住了主要矛盾就带动了一切。

3.“必须研究这个经济法则,必须掌握它,必须学会熟练地运用它,必须制定辩证的能完全反映这个法则的要求的计划。”作者这段话很好,我们还没有充分地认识、学习和熟练地掌握这个法则。

4.关于认识和掌握运用客观经济法则,作者谈到这里为止,这个问题没有展开,他研究到什么程度我们是怀疑的。为什么不两条腿走路呢?为什么重工业那么多规章制度呢?

二、关于社会主义制度下商品生产间题

1.不能孤立看商品生产,作者完全正确(P13),商品生产看它与什么经济联系,商品生产和资本主义生产联系多,就是资本主义,和社会主义联系,不是资本主义,(而)是社会主义。

2.只要存在两种所有制。商品生产就极其重要,极为有用。

3.不要以为自给自足就是有名誉的,商品生产不是名誉的。要扩大商品生产,扩大商品交换,否则不能发工资。

4.我们向两面发展,一是扩大调拨,一是扩大商品生产。不如此就不能提高生活(水平)。

5.作者说生产资料不是商品,消费资料是商品,运行不通。商品不只限于消费品。还有农业工具,手工业工具也是商品。

6.我们现在之商品生产不为价值法则所指挥。而是为计划所指挥。

7.不完成“两化”(公社工业化、农业工业化)商品不能丰富,不可直接交换。不能废除商品交换。

8.商品流通之重要是为要共产主义所考虑的,必须在商品充分发展以后。当有权支配一切商品的时候,才可能使商品不必要,而趋于消失。

三、关子社会主义制度的价值法则问题

价值法则是一个工具,只起计划作用,不起调节生产作用,必须发展社会主义商品。并且利用价值法则的形式在建设时期作(为)经济核算工具。以利逐步过渡(共产)主义。

四、关于社会主义所有制间题

1.他们宣布土地国有化,实际是社有,我们没有宣布国有化,实际上是比较彻底。

2.由集体所有制到全民所有制的标志,只能够有条件地调拨商品。不能作全国的调拨,不算全民所有制。

3.这一著作的最后一封信,认为机器交给集体农庄是倒退。这是彻底错误的。把国家和集体对立起来。

4.两种所有制如何过渡,作者自己没有解决,他很聪明地说,要单独地讨论。

五、关于社会主义社会向共产主义过渡的问题

1.作者说农民只愿意商品交换,不愿调拨,这是因为不要不断革命,要巩固社会主义秩序。

2.社会主义秩序是不能巩固的。我们要破坏一切社会主义秩序,实行部分供给制。这是为了破坏这种秩序。

3.他们社会主义是两种所有制。就是向共产主义过渡。但共产主义因素不提倡,不向全民所有制过渡,要割裂轻重工业,公开提出不着重消费资料的生产,几个差别也扩大了。

4.由按劳分配到按需分配的过渡,我们现在已经开始,吃饭不要钱就是萌芽。

5.供给制是共产主义过渡的形式,这不造成障碍。

6.作者对于两个过渡没有找出方法来。没有解决以集体所有制到全民所有制的出路。

7.作者提出向共产主义过渡的三个条件是好的,但缺少个政治条件,这条件的基本点就是增加生产,发展生产力。但没有一条办法,没有几个并举,没有设法和逐步破除资产阶级的法权的斗争,作者的三个条件是不容易办到的。

六、其他问题

1.政治经济学谈经济关系,不谈政治。报纸上讲“忘我劳动”。在他们的经济学里,其实每一段里都没有“忘我”,是冷冷清清,凄凄惨惨,阴森森的,好处是提出了问题。

2.他的批评方法里,雅罗申柯是对的,但他不谈上层建筑和经济基础的关系。没有谈过上层建筑如何适应经济基础,这是一个重大问题。

3.资产阶级法权、法权思想、法权制度等不谈,教育制度、组织也是资产阶级式的。没有共产主义劳动,如何到共产主义?作者见物不见人,只见干部、技术,不见群众。


