\section{湘江评论创刊宣言}
\datesubtitle{(一九一九年七月十四日)}



自“世界革命”的呼声大倡。“人类解放”的运动猛进。从前吾人所不置疑的问题。所不递取的方法。多说畏缩的说话。于今都要一改旧观。不疑者疑。不取者取。多畏缩者不畏缩了。这种潮流。任是什么力量。不能阻住。任是什么人物。不能不受他的软化。

世界什么问题最大?吃饭问题最大。什么力量最强?民众联合的力量最强。什么不要怕?天不要怕。鬼不要怕。死人不要怕。官僚不要怕。军阀不要怕。资本家不要怕。

自文艺复兴。思想解放。“人类应如何生活”。成了一个绝大的问题。从这个问题。加以研究。我深了“应该那样生活”“不应该这样生活”的结论。一些学者倡之。大多民众和之。就成功或将要成功许多方面的改革。

见于宗教方面为“宗教改革”。结果得了信教自由。见于文学方面。由贵族的文学。古典的文学。死形的文学。变为学民的文学。现代的文学。有生命的文学。见于政治方面。由独裁政治。变为代议政治。由有限制的选举。变为没限制的选举。见于社会方面。由少数阶级专制的黑暗社会。变为全体人民自由发展的光明社会。见于教育方面。为平民教育主义。见于经济方面。为劳获平均主义。见于欲想方面。为实验主义。见于国际方面。为国际同盟。

各种改革。一言蔽之。“由强权得自由”而已。各种对抗强权的根本主义。为平民主义。(兑莫克拉西。——作民本主义。民主主义。庶民主义。)宗教的强权。文学的强权。政治的强权。社会的强权。教育的强权。经济的强权。思想的强权。国际的强权。丝毫没有存在的余地。都要借平民主义的高呼。将他打倒。

如何打倒的方法。则有两说.一急烈的。一温和的。两种方法。我们应有一番选择。

(一)我们承认强权者都是人。都是我们的同类。滥用强权。是他们不自觉的误谬与不幸。

(二)用强权打倒强权。结果仍然得到强权。不但自相矛盾。并且毫无效力。欧洲的“同盟”。“协约”战争。我国的“南”“北”战争。都是这一类。

所以我们的见解。在学术方面。主张彻底研究。不受一切传说和迷信的束缚。要寻着什么真理?在对人的方面。主张群众联合。向强权者为持续的“忠告运动”。实行“呼声革命”——面包的呼声。自由的呼声。平等的呼声。——“无血革命”。不至张起大扰乱。行那没效果的“炸弹革命”“有血革命”。

国际的强权。迫上了我们的眉睫。就是日本。罢课。罢市。罢工。排货。种种运动。就是直接间接对付强权日本有效的方法。

至于湘江。乃地球上东半球东方的一条江。他的水很清。他的流很长。住在这江上和他邻近的民族。浑浑噩噩。世界上事情。很少懂得。他们没有有组织的社会。人人自营散处。只知有最狭的一己。和最短的一时。共同生活。久远观念。多半未曾梦见。他们的政治没有和意和彻底的解决。只知道私争。他们被外界的大潮卷急了。也办了些教育。却无甚效力。一般官僚式教育家。死死盘踞。把学校当监狱。待学生如囚徒。他们的产业没有开发。他们中也有一些有用人材。在各国各地学好了学问和艺术。但没有给他们用武的余地。闭锁一个洞庭湖。将他们轻轻挡住。他们的部落思想又很厉害。实行湖南饭湖南人吃的主义。教育实业界不能多多容纳异材。他们的脑子贫弱而又腐败。有增益改良的必要。没有提倡。他们正在求学的青年。很多。很有为。没人用有效的方法。将种种有益的新知识新艺术启导他们。咳,湘江湘江!你真枉存在于地球上。

时机到了!世界的大潮卷得更急了!洞庭湖的闸门动了。且开了!浩浩荡荡的新思潮业已奔腾澎湃于湘江两岸了!顺他的生。逆他的死。如何承受他?如何传播他?如何研究他?如何施行他?是我们全体湘人最切最要的大问题。即是“湘江”出世最切最要的大任务。

\subsection{西方大事述评—各国的罢工风潮}

法英美三国的官阀和财阀,倾注全力于巴黎和会,用高压手段对付败北的德奥,正在兴高釆烈时候,他们的国内,忽然发生了罢工风潮。罢工在他们国里,原是一件常事,政府和财阀,虽然不敢十分轻视劳动者。每当劳动者拿着劳获不均,工时太久,住屋不适,失职无归,种种怨愤不平的问题,联合同类,愤起罢工的时候,也不得不小小给他们一点恩惠。正如小儿哭饿,看着十分伤心,大人也不得不笑着给他一个饼子。但终是杯水车薪,济得甚事。所以广义派人,都笑英法的工人是小见识。从老虎口里讨碎肉,是不能够的。

此回各国的罢工风潮,英国因为在大战初了时候,(去年十二月)英伦加苏格兰各埠交通机关,燃料业,矿山业,造船业等已演了一大罢工。故此次罢工,未发生于英伦本土。法国罢工情形,初颇严重。终亦以小惠收场,没闹出什么好结果。广义派人有乘机在巴黎实行政治运动之说,亦未见诸事实。美国一部分电报电话人员的罢工,乃在附和议员多数派反对加入国际联盟。与英法罢工,异其目的。意大利之罢工乃社会觉嫉恶其政府所为的一种运动。德国自去冬少数社会党大失败,各处大罢工,亦随之而没得好结果。多数社会党掌握政权以来,早已噤若寒蝉,不敢出声。此次为和约签字问题,有激起罢工的形势。但施特满内阁倒了,继任巴安内阁,仍是前内阁的同调。抵御外侮不足,防备家贼有余的武力,紧握在手,谁敢予侮。广义一派的猛断政略,暂时决没有发动的机会。罢工不能成为事实,亦无足怪。匈牙利所受罢工影响不大,其原因则全在缺粮没饭吃。今将一月以来各国罢工情形,分述于下——

法国\quad{}六月三日,罢工风潮发生后,蔓延甚速。巴黎一区,男女工人赋闲者,二十万人。所要求各业不同,而一致主张每日工作八小时。四日蔓延更广,推算当有五十万人罢工。五日,洗衣工人罢工。自后罢工的人数更多。地底铁道,电车,街车的用人决议继续罢工。地底铁道工人要求工值每月至少四百五十佛朗。(以每佛朗当四角合我一百八十元)满五十岁须给养老金。服役十五年后亦须给养老金若干。七日巴黎罢工现象有转机,五金业与机器业,工人与雇主,已商妥数事。十一日,五金业及地底铁道工人上工。当道已取必要万法对付铁道罢工。煤矿工人有全体罢工的形势。十二日,国会通过矿工每日工作八时议案。但矿工会议,仍不满意,决定从十六日,全体罢工。水夫联合会,也决计于十六日罢工。工人联合会,言及生活代价奇昂,(记者按,近有从巴黎回者,举一物贵实例,一个旧牙刷,价二佛朗。一双皮鞋,价六十佛胡。)谓非洲各口岸,堆集麦粮千百吨,任其朽腐。各埠存货如山,轮船火车,宁闲置不远载。这样的政府,可要快快废止他的消耗,欺骗,和垄断!十四日,风潮渐平。极端派有乘机推翻克勒满沙强权政府的运动。路工联合会拒绝之。但矿工因解释政府每日工作八小时议案,未能满意,定十六日全体罢工。恐怕路矿运输联合会工人,也会罢工,表示同情。克勒满沙老头子急了,和运输公司及运输工人代表会商,恳请彼等在国家危机时候,发出些爱国热忱。工人吃了他的浓米汤,已老老实实决议上工了。

英国\quad{}伦敦五月三十日电,全国警察拟于三日罢工的气象,正在酝酿中,政府已允增给薪资,优加待遇。但不承认警察联合会及收用已革除的警察。英国的属地澳洲,坎拿大,苏依士,昔有罢工风潮。六月四日,坎拿大维克斯兵工厂工人罢工,要求每星期工作四十小时。六月五日,苏依士运河工人罢工,局势狠恶,六月九日,澳港航务罢工,势头狠烈,各项工业,都受窒碍。十×日,风潮仍严重,他业工人,因此赋闲的逐日增多。

美国\quad{}六月七日,芝加哥电报司员联合会一时罢工,共约六万人。内有二万五千人系属电报司员联合会.该会会长康能堪氏正计划全国罢工办法。同日,全国电话司员奉命于十六日起罢工,表同情于电报司员。八日,电报人员联合会干事,向全体电报人员宣布,连收发电报生在内,全体罢工。目的在停止威尔逊总统每日在巴黎往来的电报,使他注意国民不赞成他在和会的主张。十二日,各电报公司报告,电报司员罢工没成。

意国\quad{}六月十三日,意大利斯贝齐亚地方,因粮食昂贵,发生暴动,捣毁商店。十四日,热那亚工界示威,被捕者数百人,银行商店闭门,电车不走。杜林工人此日多停工,纪念德国斯巴达团领袖卢森堡氏。米兰工人罢工,抗议热那亚与斯贝齐亚当道的行动。

德国\quad{}六月十三日,大柏林公民会议秘密会,决议罢工。各职业及军界中人,均赞助停止各项实业工作的计划。有人料此举将促成国内战争。中等社会将得政权。

匈国\quad{}五月三十一日,匈京饥饿的工人,发生暴动.红旗军奉共产政府命令到各工厂制乱。匈京几无粮食。

\subsection{东方人事述评—陈独秀之被捕及营救}

前北京大学文科学长陈独秀,于六月十一日,在北京新世界被捕。被捕的原因,据警厅方而的布告,系因这日晚上,有人在新世界散布市民宣言的传单,被密探拘去。到警厅诘问,方知是陈氏。今录中美通讯社所述什么北京市民宣言的传单于下——

一、取消欧战期内一切中日秘约。

二、免除徐树铮曹汝霖章宗群陆宗舆段芝贵王怀庆职。并即驱逐出京。

三、取消步军统领衙门,及警备总司令。

四、北京保安队,由商民组织。

五、促进南北和议。

六、人民有绝对的言论出版集会和自由权。

以上六条,乃人民对于政府最低之要求,乃希望以和平方法达此目的。倘政府不俯顺民意,则北京市民,惟有直接行动,图根本之改造。

上文是北京市民宣言传单,我们看了,也没有什么大不了处。政府将陈氏捉了,各报所载,很受虐待。北京学生全体有一个公函呈到警厅。请求释放。下面是公函的原文――

警察总监钧鉴,敬启者,近闻军警逮捕北京大学前文科学长陈独秀,拟加重究,学生等期期以为不可。特举出两要点于下:(一)陈先生夙负学界重望,其言论思想,皆见称于国内外。倘此次以嫌疑遽加之罪,恐激动全国学界再起波澜。当此学潮紧急之时,殊非息事宁人之计。(二)陈先生向以提倡新文学现代思想见异于一般守旧者。此次忽被逮捕,诚恐国内外人士,疑军警当局,有意罗织,以为摧残近代思想之步。现今各种问题,已极复杂,岂可再生枝节,以滋纠纷?基此二种理由,学生等特请贵厅,将陈独秀早予保释。

北京学生又有致上海各报各学校各界一电——

陈独秀氏为提倡近代思想最力之人,实学界重镇。忽于真日被逮。住宅亦披抄查。群清无任惶骇。除设法援救外,并希国人注意。

上海工业学会也有请求释放陈氏的电。有“以北京学潮,迁怒陈氏一人,大乱之机,将从此开始”的话。政府尚未昏聩到全不知外间大事,可料不久就会放出。若说硬要兴一文字狱,与举世披靡的近代思潮,拚一死战,吾恐政府也没有这么大胆子。章行严与陈君为多年旧交,陈在大学任文科学长时,章亦在大学任图书馆长及研究所逻辑教授。于陈君被捕,即有一电给京里的王克敏,要他转达别厅,立予释放。大要说——

……陈君向以讲学为务,平生不含政治党派的臭味。此次虽因文字失当,亦何至遽兴大狱,视若囚犯,至断绝家常往来。且值学潮甫息之秋,訑可忽兴文绵,重激众怒。甚为诸公所不取。……

章氏又致代总理龚心湛一函。说得更加激切——

仙舟先生执事,久违矩教,结念为劳。兹有恳者,前北京大学文利学长陈独秀,闻因牵涉传单之嫌,致被逮捕,迄今末释。其事实如何,远道未能详悉。惟念陈君平日,专以讲学为务。虽其提倡新思潮,著书立论,或不无过甚之词,然范围实仅及于文字方面,决不舍有政治臭味,则固皎然可征。方今国家多事,且值学潮甫息之后,讵可蹈腹诽之殊,师监谤之策,而愈激动人之心理耶。窃为诸公所不取。故就历史论,执政因文字小故而专与文人为难,致兴文字之狱。幸而胜之,是为不武。不胜人心瓦解,政纽摧崩,虽有善者。莫之能挽。试观古今中外,每当文网最甚之秋,正其国运衰歇之候。以明末为殷鉴,可为寒心。今日谣琢萦兴,清流危惧。乃遽有此罪及文人之举,是露国家不祥之象,天下大乱之基也。杜渐防微,用敢望诸当事。且陈君英姿挺秀,学贯中西。皖省地绾南北,每产材武之士,如斯学者,诚叹难能。执事平视同乡诸贤,谅有同感。远而一国,近而一省,育一人才,至为不易。又焉忍遽而残之耶。特专函奉达,请即饬警厅速将陈君释放。钊与陈君总角旧交,同衿大学。于其人品行谊,知之甚深,敢保无他,愿为左证。……

\begin{flushright}章士钊拜启六月二十二日\end{flushright}

我们对于陈君,认他为思想界的明星。陈君所说的话,头脑稍微清楚的听得,莫不人人各如其意中所欲出。现在的中国,可谓危险极了。不是兵力不强财用不足的危险,也不是内乱相寻四分五裂的危险。危险在全国人民思想界空虚腐败到十二分。中国的四万万人,差不多有三万万九千万是迷信家。迷信鬼神,迷信物象,迷信运命,迷信强权。全然不认有个人,不认有自己,不认有真理。这是科学思想不发达的结果。中国名为共和,实则专制。愈弄愈糟,甲仆乙代,这是群众心里没有民主的影子,不晓得民主究竟是甚么的结果。陈君平日所标揭的,就是这两样。他曾说,我们所以得罪于社会,无非是为着“赛因斯”(科学)和“兑莫克拉西”(民主)。陈君为这两件东西得罪于社会,社会居然就把逮捕和禁锢报给他,也可算是罪罚相敌了,凡思想是没有畛域的,去年十二月德国的广义派社会党首领卢森堡被民主派政府杀了,上月中旬,德国仇敌的意大利一个都林地方的人员,举行了一个大示威以纪念他。瑞士的苏里克,也有个同样的示威给他做纪念。仇敌尚且如此,况在非仇敌。异国尚且如此,况在本国。陈君之被逮,决不能损及陈君的毫末。并且是留着大大的一个纪念于新思潮,使他越发光辉远大。政府决没有胆子将陈君处死,就是死了,也不能损及陈君至坚至高精神的毫末。陈君原自说过,出实验室,即入监狱。出监狱,即入实验室。又说,死是不怕的。陈君可以夺验其言了。我祝陈君万岁!我祝陈君至坚至高的精神万岁!

\subsection{世界杂评}

强叫化前月的初间,日本米价顶贵时候,每石超四十元。日当局有狼狈之状。报纸证言粮食的危机已迫。可怜的日本!你肠将饥断,还要向施主逞强。天下那有强叫化续得多施的理。

研究过激党阿富汗侵印度,俄过激党为之主谋,过激党到了南亚洲。高丽的“呼声革命”正盛吋,亦有过激党参与之说,则已到了东亚。过激党这么厉害!各位也要研究研究,到底是个什么东西?切不可闭着眼睛,只管瞎说,“等于洪水猛兽”“抵制”“拒绝”等等的空话。一光眼,过激觉布备了全国,相惊而走,已没得走处了!

实行封锁前月巴黎高等经济会议议决,实行封锁匈牙利,说理直到匈政府宣言遵从民意时为止。这要分两层观察,一、协约国看错了匈政府与匈国民志愿不合。匈政府与匈国民之少数有产阶级,绅士阶级,志愿不合是有的,若与大多数无产阶级,平民阶级,没有志愿不合的理。因为匈政府,原是他们所组织的。二、实行封锁,这是帮助过激主义的传播。吾恐怕协约国也会要卷入这个漩涡。果然,则这实行封锁,真是“罪莫大焉”了。

证明协约国的平等正义德国复文和会,要求德国陆军减少之后,协约国也须同减。这话谁人敢说错了?协约国满嘴的平等主义,我们且看协约国以后的军备如何?就可求个证明。

阿富汗执戈而起一个很小的阿富汗,同一个很大的海上王英国开战,其中必有重大原因。但据英国一方面的电传,是靠不住的。土耳其要被一些虎狼分吞了。印度舍死助英,赚得一个红巾照烂给人出丑的议和代表。印民的要求是没得允许。印民的政治运动,是要平兵力平压。阿富汗是个回教国,狐死冤悲,那得不执戈而起?

来因共和国是丑国协约国要划来因流域为自己挡敌的长城,必先使之脱离德国的关系,别成一国。听说已在威萨登成立临时政府,一位这登博士做总统。这位道登阵士不知果然高兴到甚么样?金人立了刘豫,契丹立了石敬塘,我们中国也曾有几个这样的国呢。

好个民族自决波兰捷克复国,都所以制德国的死命,协约因尽力援助之,称之为“民族自决”。亚刺伯有分裂士耳其的好处,故许他半自立。犹太欲在巴力斯坦复国,因为于协约国没大关系,故不能成功。西伯利亚政府有攻击过激党的功绩,故加以正式承认.日本欲伸足西伯利亚,不得不有所示好,故首先提议承认。朝鲜呼号独立,死了多少人民,乱了多少地方,和会只是不理。好个民族自决!我们认为直是不要脸!

可怜的威尔逊威尔逊在巴黎,好象热锅上的蚂蚁,不知怎样才好?四围包满了克勒满沙,路易乔治,牧野伸显,欧兰杜一类的强盗。所听的,不外得到若干土地,收赔若干金钱。所做的,不外不能伸出己见的种种会议。有一天的路透电说:“威尔逊总统卒已赞成克勒满沙不使德国加入国际同盟的意见”。我看了“卒已赞成”四字,为他气闷了大半天。可怜的威尔逊!

炸弹暴举人人知道很文明很富足的美国,有“炸弹暴举”同时在八城发生。无政府党蔓延甚广。炸弹爆炸的附近,有匿名揭贴说,“阶级战争”业已发生,必得国际劳动界完全胜利,始能停止,炸弹往往埋藏在一些官员的住宅,屋顶上发现人头。可怕可怕!我只挂牵官员人家的一些小姐小孩子,他们晚上如何睡得着?议院里一些钱多因而票多票多因而当选的议员,还在那里痛诋暴动者,通过严惩案。我正式告诉诸位,诸位的“末日审判”将要到了!诸位要想留着生命,并想相当的吃一点饭,穿一点衣,除非大大的将脑子洗洗,将高帽子除下,将大礼服收起,和你们国里的平民,一同进工厂做工,到乡下种田。

不许实业专制美国工党首领戈泊斯演说曰,“工党决计于善后事业中有发言权,不许实业专制。”美国为地球上第一实业专制国,托辣斯的恶制,即起于此。几个人享福,千万人要哭。实业越发达,要哭的人越多。戈泊斯的“不许”,办法怎样?还不知道。但既有人倡言“不许”,即是好现象。由一人口说“不许”,推而至于千万人都说“不许”,由低声的“不许”,推而至于高声的很高声的狂呼的“不许”,这才是人类真得解放的一日。

割地赔偿不两全德国答复协约国,说,如失去西里细亚及萨尔煤矿,则无力行赔偿。我料协约国听了一定很烦脑。何以故?地可割,赔偿也可得,最为两全。据德国的说,两样便成了反比例,如之何不烦脑?虽然,奉劝协约国的衮衮诸公,天下那有两全的好事!

为社会党造成流血之地奥总代表任纳博士答复和会,说,“奥国今已坐食其较前大减之资本,若再加以摧残,必为社会党造成流血之地”,蠢哉任纳博士,你还不知道协约国一年以来之真目的,你专为造成社会党流血之地吗?

彭斯坦德博士彭斯坦演说曰,“媾和条件”乃野蛮战争的结果,德国最宜负责,和约条件十九为必要的”。我们固然反对协约国的强迫和约,但博士这话,系专对野蛮战争而发,听了倒很爽快。

各国没有明伦堂康有为因为广州修马路,要拆毁明伦堂,发了肝火。打电给岭伍,斥为“侮圣灭伦。”说,“遍游各国,未之前闻”。康先生的话真不错,遍游各国,那里寻得出什么孔子,更寻不出什么明伦堂。

什么是民国所宜?康先生又说,“强要拆毁,非民国所宜”。这才是怪!难道定要留看那“君为臣纲”,“君君臣臣”的事,才算是“民国所宜”吗?

大略不是人邓镕在新国会云,“尊孔不必设专官,节省经费”。张元奇云,“内务部祀孔,由茶房录事办理.次长司长不理,要设专官。”内务部的茶房录事,大略不是人。要说是人,怎么连祀孔都不行呢?我想孔老爹的官气到了这么久的年载,谅也减少了一点。

走昆仑山到欧洲张元奇又说,“什么讲求新学,顺应潮流,本席以为应尊孔逆挽潮流。”不错不错!张先生果然有此力量,那么,扬子江里的潮流。会从昆仑山翻过去,我们到欧洲的,就坐船走昆仑山罢。

\subsection{湘江杂评}

好计策一个学校的同学对我说,我们学校里办事人和教习,怕我们学到了他们还未学到的新学说,将图书室看×了。外面送来的杂志新闻纸和书籍,凡是稍新一点的,都没得见。我听了为之点首叹服。他们的计策真妙!岂仅某学校,通湖南的学校,千篇一律都象联了盟似的。

摇身一变一些官僚式教育家,为世界的大潮卷急了,不提防就会将他们的饭碗冲破。摇身一变,把前日的烂调官腔,轻轻收拾。一些其有所感而改变的,很可佩服。一些则是假变,容易露出他们的马脚。这类人我很为他羞!很为他危!

我们饿极了我们关在洞庭湖大门里的青年,实在是饿极了!我们的肚子固然是饿,我们的脑筋尤饿!替我们办理食物的厨师们,太没本钱。我们无法!我们惟有起而自办!这是我们饿极了哀声!千不要看错!

难道走路是男子专有的一个女学校里的办事人,把学生看做文契似的收藏起来,怕他们出外见识了甚么邪样。新青年一类的邪书,尤不准他们寓目。此次惊天动地的学生潮,北京的女学生聚诉新华门。贫儿院的小女孩子,愿到监狱替男学生抵罪。这个女学校的学生独深闭固拒,一步也不出外,好象走路是男子专有似的。

哈哈!青岛问题发生,湖南学生大激动,新剧演说,一时风行。有一位朋友对我说,一位老先生,因为他的儿子化装演剧,气得了不得。走到学校问先生,开口便说,“我的命运如何这么乖?养大的儿子竟做出那么下流事”?我听了这话,忍不住卧的一声,哈哈!

女子革命军或问女子的头和男子的头,实在是一样。女子的腰和男子的腰实在是一样。为什么女子头上偏要高竖那招摇畏风的髻?女子腰间偏要紧缚娜拖泥带水的裙?我道,女子本来是罪人,高髻长裙,是男子加于他们的刑具。还有那脸上的脂粉,就是黔文。手上的饰物,就是桎梏。穿耳包脚为肉刑。学校家庭为牢狱,痛之不敢声,闭之不敢出。或问如何脱离这弊?我道,惟有起女子革命军。

