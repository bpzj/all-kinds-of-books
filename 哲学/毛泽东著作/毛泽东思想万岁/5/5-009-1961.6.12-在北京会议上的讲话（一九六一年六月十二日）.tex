\section{在北京会议上的讲话(一九六一年六月十二日)}


人民公社问题,在一九五八年的北戴河会议以后,开了两次郑州会议。第一次会议解决集体所有制和全民所有制的界线问题,社会主义和共产主义的界线问题。第二次会议解决公社内部三级所有制的界线问题。这两次会议的基本方向是正确的。但是,会议开得很仓促,参加会议的同志没有真正在思想上解决问题,对于社会主义建设的客观规律,开始懂得了一些,还是懂得不多。在一九五九年三月的上海会议上,通过了关于人民公社的十八个问题的纪要。后来,我给小队以上的干部写了一封“党内通讯”,对农业方面的六个问题提了意见。在这一段时间内,普遍地对人民公社进行了整顿,使工作中的缺点和错误逐步地得到纠正。不过,由于各级干部还不真正懂得什么是社会主义,什么是按劳分配,什么是等价交换,他们对党中央关于人民公社的许多意见和规定,还没有认识清楚,他们的思想问题还没有得到解决。一九五九年夏季庐山会议上,右倾机会主义分子向党进攻,我们举行反击,获得胜利。反右以后。工作中出现了一些假象。有些地方有些同志以为从此再不要根据两次郑州会议的精神,继续克服工作中存在的缺点和错误了。一九六〇年春,我看出“共产风”又来了,批转了广东省委关于当前人民公社工作中几个问题的指示。在广州,召集中南各省的同志开了三小时的会,接着在杭州又召集华东、西南各省的同志开了三、四天会,后来又在天津召集东北、华北、西北各省同志开了会。这些会,都因为时间短,谈的问题很多,没有把反“一平二调”、反“共产风”的问题作为中心突出来,结果没有解决问题。几个大办一来,糟糕,那不是“共产风”又来了吗,一九六〇年北戴河会议,用百分之七、八十的时间谈国际问题,只是在会议快结束的时候,谈了一下粮食问题,没有接触到人民公社内部的平均主义问题。同年十月,中央发了关于人民公社十二条的指示,从此开始认真纠正“一平二调”的错误,但是仍然坚持供给制、公共食堂、粮食到堂的作法。而且,在执行中,只对三类县、社、队进行了比较认真的整顿,对于一、二类县、社、队的“五风”基本上没有触动,放过去了。一九六一年一月九中全会以后,经过农村调查,在广州开会,强调提出人民公社内部存在着必须解决的两个平均主义问题,起草了农村人民公社六十条。这次会议,启发了思想,解放了思想,然而还不彻底,继续保留了三七开(即供给部分三成,按劳部分七成的分配办法)、公共食堂、粮食到堂的尾巴。经过会后的试点和调查,到这次会议,大家的思想彻底解放了,上面所说几个问题的尾巴最后解决丁,大家对社会主义建设规律的认识,也比过去清楚得多了。由此可见,对客观世界的认识是逐步深入的,任何人不能例外。

<p align="center">×××</p>

我们党从一九二一年成立,经过了陈独秀右倾机会主义和三次“左”倾机会主义的严重挫折,经过了万里长征,经过延安整风,通过了“关于若干历史问题的决议”,到一九四五年的七次代表大会,共用了二十四年的时间,本形成了思想上真正的统一,并且在政治、军事、经济、文化和党的建设等方面形成了一整套实现民主革命总路线的具体政策,保证了抗日战争和解放战争的胜利。建设社会主义社会,我们过去谁也没有干过,必须在实践中才能逐步学会。我们已经搞了十一年,有了社会主义建设总路线,积累了很多经验。只有总路线还不够,还必须有一整套具体政策。现在要好好地总结经验,逐步地把各方面的具体政策制定出来。我们已经有可能这样做,并且已经制定了人民公社六十条。

最近林彪同志下连队做调查研究,一次是在广州,一次是在杨村,了解了很多情况,发现了我们部队建设中一些重要的问题,提出了几个很好的部队建设的措施。要搞具体政策,没有调查研究是不行的。

形成一整套的具体政策,看来还需要一段时间,也许还需要十多年。这是一种设想。如果大家都觉悟了,也可能缩短一些。

<p align="center">×××</p>

现在的重要问题是要重新教育干部。干部教育好了,我们的事业就大有希望。不教育好干部,我们就毫无出路。我们要利用人民公社六十条等文件,作为教材,用延安整风的方法,去教育干部。这次参加会议的同志,思想通了,就要去教育地、县的干部,他们的思想通了,再由他们去教育社、队的干部,使大家具正懂得什么叫社会主义,什么叫按劳分配、等价交换。教育干部的事情,今年一定要做出一点成绩来,并且一定要长期地做下去。搞民主革命,我们长期地教育干部,搞社会主义也必须如此。


