\section{纸烟税}


我们时常论:中国政府是洋大人的贩属,或者有人不相信;我们不时常说:外国人(尤其是英美)的假亲善只是想借“亲善’两字好多量压榨中国人的膏血,或者也有人不相信。自禁棉出口令遭洋大人反对而取消,可不能不有些相信了,现在又来了洋大人压迫政府取消浙江等省的纸烟税,可更不能不有些相信了。申报八月二十八日北京电:“阁议,英美公使抗议加征地方烟捐,结果电令答该省停止征收”,征收纸烟损到底是怎么一回事呢?我们看六月二十日杭州总商会致北京政府电:

“窃维奢侈征税为各国之通例。近年卷烟盛行,以吾浙论,每岁销场竟逾千万,消耗之钜,骇人听闻,流毒之列小减鸦片。当局有鉴于此,特命设局征收,化无益之消耗,作修路之正用。乃闻外商借口条约权利垒与政府交涉,殊不知此种特税,完全取诸。及尹,烟商毫不相涉,纯粹浙人所输之捐,外商又劳干预!且系同内行政主权,断不落外人侵犯,自立据理力争,忽任有所借口,主权等甚。”原来美英据了协定关税条约,不许中国对外奢侈品自由抽税,任凭是“纯粹浙人所输之损’是“国内行政主权”,只因为是对外国货,所以到底不许抽税。

英美烟公司所出的纸烟,一小部分是英美日本国这来的,一大部分是英美烟商用中国的烟叶雇中国的劳力在上海汉口等处中国内地设厂制造的,制造出厂时照“条约”出了一点轻微的税,大批达到各省,以后就再不许中国“自由”抽税了,浙江一省销纸烟价“年逾千万”,全国每年销纸烟总额无确数,照浙江一省推算起来,至少在二万万元以上,真是“骇人听闻”!请四万万同胞想一想,外国人同我们“亲善”到底是为什么?

中国政府的“阁议”,真是又敏捷又爽快,洋大人打了一个屁都是好的“香气”,洋大人要拿棉花去,阁议就把禁棉出口令取消,洋大人要还纸烟来,阁议就“电令各该省停止征收纸烟税”再请四万万同胞想一想,中国政府是洋大人的帐房这句话到底对不对。

