\section[在武汉和各协作区主任的讲话(一)(一九五八年十一月三十日下午)]{在武汉和各协作区主任的讲话(一)(一九五八年十一月三十日下午)}
\datesubtitle{(一九五八年十一月三十日)}


作什么事情都要中央与地方、条条与块块相结合,这么一种民主集中制,群众路线的方法,否则搞不好。这次是不是泼李富春的冷水?计划要积极可靠,放在稳妥的基础上,还要鼓气,不要挫伤群众的积极性,接受一九五六年的教训。地方也有条条与块块的关系。第一书记是块块,分工的书记是条条,也要结合。第一书记要和工业书记结合。北戴河会议一股热情,三千万吨,当中一千万吨是主观主义,事非经过不知难。(总理插话:确非神秘、并不简单。××:钢铁的指标各地还可研究,减一点对国家的计划不会受影响。钢、电、交按第二个方案。其他指标按第一个方案。)

六号、七号开中央全会,要不要提纲挈领讲一下,计划搞一个时期再看看。明年七月一日再定。粮食原来并没有计划翻一番,开了几次会议就提上去了。农业的“八字宪法”只管地,不管天(管不了日照和空气),天、地是对立的统一。好的东西不要字数太多,老子一辈子只写了五千多字。工业与农业不同,工业方面的相互关系牵制很多,搞钢必须搞煤、电等等,缺一环也不行。农业方面相互关系牵制比较少一点。党、群众、技术人员三结合,试验田、高产是对人类的一大解放。人类对自然界的认识。三三制打破了许为保险系数,还是写进去好,时间上再灵活一些,再修改一下,全国大多数地方每人一亩左右。

人民公社再讨论两天,作好修改。这次很多问题展开了,回答了城市要不要办公社,肯定要办。民兵和家庭问题,杜勒斯攻击我们,说我们是奴隶劳动,破坏了家庭。资本主义早破坏了家庭,金钱关系,父不认子,各人开解。我们现在公社会养老,“老吾老以及人之老,幼吾幼以及人之幼。”工资差额略为展开一点四清左右或者更多一些。认为农民有平均主义倾向,但也不能过重悬殊,但也不能没有差额。苏联的工资差额悬殊太大,我们不照样学。将来,这样一点工资算什么?十五元算什么?总得有三十元,四十五元了,都提高到几十元差别就没有了,这是指的乡村。城市的差额还会更多一些。这是必要的,城市里不要砍掉黄炎培、梅兰芳、教授的工资,将来社会产品丰富起来,低工资提起来,完全接近了,就进入共产主义了。所讲按劳取酬和各取所需的问题,如何平均?由下长上去。

作风问题,以半天时间谈一下。现在的问题主要、是强迫与报假,枣阳县一个文盲未扫除,报告说是消灭了。强迫有两部分人,一部分是阶级异己分子,一部分是蠢人。强迫命令的人究竟有多少?百分之一、百分之五、还是百分之十,今年十二月或明年一月,各地都要开党代表大会,谈谈作风问题。

今年下半年出现了两个大问题,一为人民公社,二为以钢为纲。大家有点紧张,现在搞条例,心情就舒畅了。人民公社文件是从郑州会议搞起来的,有所准备。计划会议是条条搞的。华东现在走下坡路。过去没有想那些条件.有煤炭也运不出来。

一九五九年的农业生产,今年粮食是七千五百亿斤。明年增到三千亿斤,达到一万零五百亿斤,每人平均一千五百斤以上。苦战三年。达到每人两千斤。薯面也还要有一点。今年来一个以多报少方针,留有余地。棉花今年报六千七百万担,明年一亿担。粮食收成到底有多少?是不是增加一倍左右?可以写增加百分之九十左右比较妥当。吃饭问题。究竟把话说满好,还是留有余地?明年春天会不会有的地方吃不到三顿干饭?广东下命令一天三顿吃干的。山东的群众说,现在吃煎饼,明年春天怎么办?现在粮食摸不到底,是否现在少吃些,以后多吃些。各地议一议。

国际形势。赫鲁晓夫开记者招待会,在柏林问题上搞了一手,你不撤,我撤。赫鲁晓夫也懂得了搞紧张局势。我们也搞点紧张局势,使西方要求我们不要紧张了。让西方怕搞紧张局势,对我们有利。中苏会谈公报发表以后,台湾就开紧急会议,其实会上没有谈一句台湾局势问题。四国首脑会议不召开了,也说是受中国影响。其实会上也没有说四国首脑会议的问题。如果出远门,是不是安全?斯大林神经不健全,从前哪里也不去。各种材料证明帝国主义采取守势。一点攻势也没有了。杜勒斯十八目的谈话说,你们共产党人搞人民公社不要出那个范围,你们只管你们的事,不要管你们以外的事,我们就放心了。你不犯我,我不犯你。杜勒斯说我们搞奴隶劳劫,搞集权主义,说我们积累太多。他说总收入扣除工资,就是积累。他把这种积累叫做资本。殖民主义——民族主义——共产主义,是列宁的公式。好好看着杜勒斯十八日的讲话。他承认我们的积累多,组织性强,哲学搞不赢我们。杜勒斯谈话的调子低,战争边缘不讲了,实力地位不说了,杜勒斯比较有章程,是美国掌舵的。英国人老奸巨猾,美国人比较暴躁。英国人经常作战略和战术指导。杜勒斯讲世界五大问题:民族主义,南北两极,原子能,外层空间、共产主义。这个人是想问题的人。要看他的讲话,一个字一个字地看。要翻英文字典。世界秩序研究会议,三千七百方教徒发出一封信。主张承认我们。杜勒斯说教会只要规定道德原则,细节不要管。

北戴河会议谈的八个观点灵不灵?还是灵的。北大西洋公约,向民族主义和本国共产主义进攻(重心是进攻中间地带亚洲、非洲、拉丁美洲),对社会主义阵营是防御的,除非出了匈牙利事件。但我们在宣传上是另一回事,还要说他是进攻的。不要自己被自己蒙蔽。李普曼写了一篇文章,说不是进攻,但说服不了苏联人民。谁怕谁多一点?李普曼主张把印度扶植起来抵制我们,看来惧怕我们,怕我们在亚洲、非洲争取领导权,很怕我们经济高涨。紧张局势归根到底对我们有利。戴高乐出现,横竖要出现,出现了,比较对法规无产阶级有利。中东美军早撤好,还是晚撤好?只有一个多月就走得光光的,证明他们撤走。台湾打炮有好处。不然民兵不能组织这样快。禁运、进联合国、和、战、打原子弹问题,比较对谁有利?怕好越是不怕好?横起一条心,不怕反而好。在一起议一议就不怕鬼了。杜勒斯好战。挨骂是假象,不是真相。杜勒斯是真正掌舵的,省委要指定专人看参考资料。

赫鲁晓夫过于谨慎,过于不平衡,是铁拐李。不是两条腿走路。人民生活才二百卢布,比我们稍微高一点。重工业偏大。偏于中央不重地方。偏于行政,缺乏群众路线。

这次会议,开得比较长一点。松一点。比较集中,只搞两个文件。纸老虎问题,党内外尚有许多人不了解。有人说既然是纸老虎,为什么不打台湾,为什么还要提出赶上和超越英国。我写一篇短文回答这个问题。是真又是假,暂时现象是真的,长远看来纸的。我们从来就是说,战术上要重视它,战略上要藐视它。不但对阶级斗争应当这样,对自然斗争也应这样,除四害、扫盲、绿化、血吸虫,不是一年可以实现的。要几年去搞才行。


