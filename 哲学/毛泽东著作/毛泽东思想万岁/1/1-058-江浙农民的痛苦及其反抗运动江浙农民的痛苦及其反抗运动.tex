\section[江浙农民的痛苦及其反抗运动江浙农民的痛苦及其反抗运动]{江浙农民的痛苦及其反抗运动江浙农民的痛苦及其反抗运动}


江浙两省在中国为工商业特别发达之区,因此工人商人的地位容容被人重视,至于这两省农民,便很少有人重视其地位,而且多以为两省乃太平富庶之区,农民并无多大痛苦。其实这种见解完全是皮相,完全是不明白江渐农村实际状况,我们试一考察江浙农村之实际状况,便知道实际情形与思想完全相反,以下所述各县具体事实,又算我们近来得到的一极小部分材科,然已与证明江浙农民并不是一般人想象的那样太平富庶无多痛苦的农民了。

崇明 长江口之一岛,岛之全城为崇明县,均长江坭沙沉淀冲积而成,岛之四周年涨新沙,因此沙田甚多,佃农甚多。今举上沙一地为例,此地地主剥削佃农非常利害,每千步田要纳保证金五十元,这种田完全是新涨的沙田,农民逐渐替地主们经营成熟,成熟后,地主管田底所有权,农民管田面的权,每年耕种所用人工,肥料、农具、种子等均归农民自备。秋收后每千步田要纳租谷五百斤甚者五百斤以上,地主到农民家里的时候,农民要请他们吃好酒饭,不然便难免加租,收租的称,大概都在二十两以上。农民稍有反抗,马上送县究办。农民若今年欠了五元租,明年就要还你十元二十元,又不得不还,于是农民破产者年年有之,此地农民曾在民国十一年起了一个暴动。并没有什么赤党过激党煽动他们,他们成群的起来打毁警察局,割去地主陶某的耳朵,并大闹县署要求减租,后因团结不固,首领被捕,以致失败。今年江苏遭了普遍的旱灾,田亩减收,上沙地方每千步田农民只收谷三四百斤,而地主缴租则坚持要照例缴五百斤,地主且以“佃业维持会议决”以欺农民(佃业维持会系十一年地主组织以欺农民的),于是农民恨地主益深,暴动又将发生了。

江阴从无锡乘轮船前往,到一处地方,叫顿山镇,这顿山镇在江阴常熟无锡三县之间,三县大地主很多,压迫佃农很利害,去年秋天,有一个日本留学生顾山人周水平(周原在无锡省立师范毕业)回到本乡,看不过眼,乃劝佃农组织团体,名曰‘佃农合作自救社”,周往来各村。宣讲农民痛苦声泪俱下,顾山农民从者极众,江常锡三县各界农民都为煽动,如出而起,反对为富不仁之种种大地主,一致要求减租,但农反尚未完全联起来之时,劣绅地主早已联合起来,江阴,常熟,无锡三县,绅土地主同时动作,文电如雪片传到孙传芳,孙传芳那有不听劣绅地主的话,于去年十一月便把“佃农合作社”解散了,把周水平捕获,今年一月便把他枪毙了,减租运动算是一时镇压下来。当周水平灵柩回到顾山安置在他家里时,农民们每日成群到他灵前磕头,他们说“周先生是为我们死的,我们要给他报仇。”今年大旱,稻收不好,农民又想起要求减租,可见他们并不怕死,他们知道只有团结奋斗,以减少贪暴地主的剥削,才是他们的出路。只江阻东乡有一名叫沙洲的地方,亦有农民反对地主的事,此地主苛例为交上期例,江苏人所谓寅交卯种,是一件于农民经济上很痛苦的事。现在农民要求种田还租,正在那里奋斗。

丹阳这里叙述丹阳县吕城镇的两件事,(这吕城镇在丹阳县东乡,靠近沪宁铁路)一件是反抗当铺欺削农民,事在今年下间,吕城镇上有一家当铺,一天被马王仁残部在县西茅山为匪者到境抢了一回,但所抢不多。当铺主人即呜报失,说典当的衣物都抢去了,同时密将衣物乘空移藏他处,这些衣物的当户即近镇各村农民,闻迅,邀截于路,得原物之一部,但已被藏之部未得,典物的农民乃起而组织“当铺联合会”,向吕城当铺算账,结果当铺赔偿损失一部,即每人赔偿等于当价之数目,共赔了九百块践,此事证明农民有团结便可得胜利,设这回没有团结,便让奸狡的当商欺剥了全部的当物去了。又一件是反抗劣绅富农强迫农民缴钱戽水。事在今年夏秋,一事到现在尚未了结。江苏各县农村的河里,现在很普遍地采用一种机器戽水,叫作“戽水机器”,以代替旧法的手车脚车车水,吕城领附近八个村子的农民,感觉到要戽水改用机器,但这个地方的劣绅及富农便乘机图利,抢头先做,组织一个“机器戽水公司”,集合资本一千四百元,买来一架机器,安装在河里,用公司名义,要农民按亩出钱,若不出钱不准戽水,农民们一打算:所有这几村的四一年按亩交纳的钱,即够买一架机器,若集资自买一架,一次出钱,年年可用,用公司的机器,则年要缴这样多钱,于是大反对劣绅富营的公司。本地有几位少学教员顿帮农民的忙,帮他们组织一个团体,叫做“农民促进会”,在这个团体内,组织一个“机器戽水合作社”,办法也是农民按亩出钱,也凑足一千四百元,买了一架戽水机器,于是河里有两个戽水机器,一架是公司的,一架是合作社的,公司的一架完全摆在那里没人理会,劣绅气极了,用种种诬词告到孙传芳那里,结果派兵下乡,大索过激党,捕去四人,通缉三人,声言不用公司机器的人都要重办,当兵来时,村中壮年男女都躲在木丛中,只留老弱妇女小孩见士兵的的面,这些犯了重罪的农民,单是送兵太爷恶捕礼就还了一千元,其余被搜掠者不在内,此案到现在还没有完结。好在现在已是孙传芳不甚如意之时,吕城的劣绅们或且也稍稍感到难尽如意,亦未可知。

无锡寓无锡县城十里之徐苔镇,不久之前也出了一件小小乱子。此地大商兼大地主的荣德生,也要在此地修一条马路经过农村,折毁镇上房屋,兼价收买过路田亩,此事直接损害农民经济,故农民们组织农民俱乐部反抗荣德生,结果荣德生屈服,允许田二百元一亩,新植的桑苗一角一株,镇上房子不折。

青浦沪抗路侧之靖浦县,上月内发生农民反对重价买荒之事。本县荒地,农民缴价买荒,历来定价每亩三元,此次劣绅县长林员一组织一公司,以每亩三元领得荒地,而以每亩十二元卖给农民。农民组织垦务联合会对抗劣绅官厅则多方恐吓,现在仍在争持中。

泰兴东乡王家庄地方,今年因旱少收,农民要求减租,与地主起了激烈的斗争,地主不但不肯减租,反压迫农民,农民之中一人因恨极图杀一万恶之地主,地主报县,捕三十人入狱。

泰县泰县森森庄地方之农民,今年因旱请求减租,起了一个运动。地主压迫,捕去为首数人。

徐州江苏农民江北北徐海一带算是最苦,江枪会连庄会到处皆是,农村各种斗,此他处更多,继述不尽,铜山县东乡北乡等处地势洼下,去年禾稼淹没殆尽,所幸二麦已种,农民尚有“转过荒年有熟年”之希望,今秋淫雨连绵,田鼠禾苗终日浸在水中,田萎黄而腐烂,农民辛勤半载,溶得两手朴空,此时地中仍是积水片片,二麦播种无期,怨声载道,莫不表现一种扰惨愁苦的状态。天灾之处,同时还有横征暴敛之军阀贪官与重租重利之劣绅地主,层层敲剥,因此农民流为匪者极多,徐州一带所以成了著名之匪区以此。

慈衍慈衍属浙江,在宁波之西,近日本县山北地方曾发生一次大的暴动,这山地、方的农民本是很强悍的,时常有械斗的事发生,和以近年官僚警察无理的压迫,劣绅地主加倍剥削,农里极愤已深,恰巧今年睛雨不均,稻和棉花都没有收成,那地主铁租又一些儿都不肯减,农民的闹荒暴动就因此爆发了,“农民的暴动一爆动一爆发,一般游民无产阶级都很勇敢地参加进来”九月十三日上午,聚积两千多人到警察局报荒,和警察冲突起来,他们把警察署焚毁了,把警察的枪械也缴了,又转至多绅地主家“吃大户”。吃了以后,因愤乡绅地主的凶恶,把他们的屏画古董门窗壁格捣毁净尽,每天都是这样,他们也不大听人劝导,只是这样发泄他们的怒气,隔日乡绅逃至城内告发,军警陆续下乡大搜农民,农民领袖多已逃撤。“犯法”“犯罪”己成了普遍的宣传,农民由此胆却起来,这个暴动就镇压下去了,这次暴动失败的原因,在群众完全没有组织,又没有指导,所以成了原始的暴动而至于失败。

<p align="right">(《响导周报》第五集第一百七十九期)</p>

