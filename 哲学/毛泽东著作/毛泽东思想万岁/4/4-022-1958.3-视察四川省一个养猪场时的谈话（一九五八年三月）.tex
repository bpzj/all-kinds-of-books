\section[视察四川省一个养猪场时的谈话(一九五八年三月)]{视察四川省一个养猪场时的谈话}
\datesubtitle{(一九五八年三月)}


毛主席:(对饲养员)辛苦了,好同志!(握手)祝你们获得更大成功!

饲养员:全靠你老人家的教导!

毛主席:主要是你们自己的努力。

(对社长)能使你们全社妇女都当上模范吗?

社长:保证全能当上,毛主席!

毛主席:全当上模范未必得行吧!只能在二、三年内做到有一半妇女当上模范就不错了。你说我这个看法保守吗?(众笑)

毛主席:参观一下你们的养猪场好吗?

饲养员:欢迎,欢迎!毛主席多指教!

毛主席:别什么都要我指教吧,我和××都是来向你们学习的,在许多事情上,你们比我们内行多了,是吧?

毛主席:这是什么?

饲养员:是糖化牛粪,拿来喂猪的,猪很爱吃。

毛主席:真是新鲜事儿,牛粪也能喂猪!怎么个制法,介绍一下吧,(摸出日记本来作记录,向××说)来,我们都记上吧,这是群众的创造!从前我们就没听见说过。看来我们中国那句老话:“做到老,学到老”实在不错!

毛主席:(对一个回乡知识青年)回家喂猪有意见吗?

饲养员:有啥意见呀!我一辈子也不愿意离开这养猪场哩!

毛主席:很好!可要把你学的科学,传授给群众才更好。

毛主席:……多看看先进的东西,眼光就会更开阔些。

(亲自给一头猪打了蛋清针,然后对××说)

这里的经验真不少,特别是代饲料,如果全国都推广,一年要节约好几十亿斤粮食哩!请记者同志在报纸上介绍一下好吗?

×××:办得到。

毛主席:最好后天见报。

(对饲养员)谢谢你们!对我们教育太大了!

饲养员:可有人看不起我们哩!

毛主席:谁?谁是顽固派?

(饲养员介绍一些人如何从轻视妇女到称赞妇女的成绩。)

毛主席:看不起妇女的人虽不多,但哪里总有几个,这不完全怪他们,过去封建制度对他们影响太深了,脑筋一下不容易转过弯来,又不能用飞机大炮来对付他们。那怎么办呢?我看除了加强教育外,最好的办法,就是妇女同志们做出成绩来,多拿事实给他们看,看得多了,他们脑子里的那个封建王国就会不攻自破的。

<p align="right">(见《中国妇女》1958年第7期)</p>

