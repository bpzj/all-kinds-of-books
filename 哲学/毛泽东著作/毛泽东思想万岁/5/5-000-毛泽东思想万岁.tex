\documentclass[b5paper,oneside,12pt]{ctexbook}
\usepackage[hmargin=0.25in,vmargin=0.5in]{geometry} 
\usepackage[]{hyperref}
\pagestyle{plain} %整书页眉页脚设置
\ctexset{chapter/numbering=false}
\ctexset{
    section={numbering=false, afterskip = 0ex},
    subsection={format=\large\heiti\centering,numbering=false,beforeskip=1ex,afterskip = 1.75ex}
}
\newcommand\datesubtitle[1]{{\centering\large #1\par\vspace{1ex}}}  %自定义日期副标题格式,为了保险,最好使用两层大括号

\newenvironment{yinyong}{%
    \begin{list}{}{\parsep\parskip
        \setlength\topsep{0pt}
        \setlength\itemindent{2em}%
        \setlength\parindent{2em}
        \setlength\listparindent{2em}
        \setlength{\leftmargin}{2em}
        \setlength{\rightmargin}{2em}
        \kaishu
    }
    \item[]
}{
  \end{list}
}



\title{毛泽东思想万岁5}
\author{毛泽东}
\date{}

\begin{document}

\frontmatter
\maketitle
\tableofcontents

\mainmatter
% \chapter{}
\section[在中央工作会议上的讲话(一九六一年一月十三日)]{在中央工作会议上的讲话}\datesubtitle{(一九六一年一月十三日)\footnote{据《毛泽东年谱》,1月13日,在中南海怀仁堂主持中央工作会议全体会议。}}

这次工作会议,据我看比过去几次都要好,大家头脑比过去清醒了些,冷热结合得好了一些。过去总是冷得不够,热得多了些,这次比过去有了进步,对问题有了分析,对情况比较摸底了。当然,有许多情况还是不摸底。中央和省市都有这种情况,比如对一、二、三类的县、社、队比较摸底,一类是好的,执行政策,不刮共产风。二类也比较好,三类是落后的,不好的,有的领导权被地、富、反、坏分子篡夺了,实际上是打着共产党的招牌,干国民党、地主阶级的事情,是国民党、地主阶级的复辟。全国县、社、队有百分之三十是好的,百分之五十是一般的,百分之二十是坏的。在一个具体地方,坏的有超过百分之二十的,有不到百分之二十的。但是究竟情况怎样,也不是完全清楚,也不完全准确,只能说大体上是这样。不要以为一、二类社、队都是好的,其中同样也有坏人,三类队中也有好人。××同志批了河南灵宝县的一个报告,指出了一、二类社里也有问题,群众发动以后,谁是好人,谁是坏人,群众是摸底的,公社是摸底的,就是我们不太摸底。总的看好的和较好的占百分之八十,还是好的多,群众知道好坏,就是领导不摸底。我们要有决心,这些地方没有强有力的领导,如果不派大批干部深入发动群众,找出贫农和下中农中的积极分子,采取两头压的办法,是不能解决问题的。

灵宝县一、二类社尚有许多问题,也还有坏人,何况三类社?现在我们虽然还不完全摸底,但已向这个方向进了一步,今后好好地进行调查研究,就可以更摸底。譬如粮食产量究竟有多少?现在比较摸底了,口粮搞低标准,瓜菜代,粮食过秤入库,比较摸了底。但也有地方不摸底,河北省还有百分之××的县、社、队不摸底。口粮标准有的不按省里规定吃,吃多了。

至于城市工业问题,比较接近实际。今年钢只定××××万吨,煤、木材、矿石、运输还得搞那么多。煤的指标要增加,不但冬季烧煤不够,而且发电用煤也不够。今年着重在搞质量、规格、品种。钢的产量已居世界第×位,数量不算少,目前是质量不够,所以今年不着重发展吨数。

省委书记、常委,包括地委第一书记,他们究竟摸不摸底?他们不摸底就成问题了。应该说现在比过去进一步,也在动了。要用试点方法去了解情况,调查问题。调查不需要很多,全国有通海口一个就行了,但现在也只有这么一个报告。三类社、队的问题,有信阳地区的整顿经验的报告,那么整三类社、队的问题就够了。还有河北保定的一个材料很有说服力,这个报告说什么时候刮共产风,如何纠正,如何整顿组织,如何改进领导,以及怎样实现大生产。现在河南出了好事,出了信阳文件,纪登奎的报告。希望大家回去后,把别的事放开,带一两个助手,调查一两个社、队,在城市也要彻底调查一两个工厂、城市人民公社。

省委第一书记只有那么一个人,怎么能又搞农村又搞城市呢?因此要有个助手,分头去调查,使自己心里有底。心中没底是不能行动的。过去打仗,心中有底,靠什么?解放战争初期,中央直接指挥的经验少,但有两个办法;\marginpar{1}一靠陕北打胡宗南的经验,到四七年四、五月间,就靠各地区前方的报告,这是阳的,还靠阴的,即各方面的情报,所以情况很清楚。现在这些情报没有了,死官僚又封锁了消息,中央就得不到更多的消息。

我们下去搞调查研究,检查工作,要用眼睛去看,用耳朵去听,用手去摸,用嘴去讲,要开座谈会。看粮食是否增了产?够不够吃?要察颜观色,看看是否面有菜色,骨瘦如柴。这是眼睛可以看得出来的。保定的办法是请老农、干部开座谈会,与总支书、支书谈,群众也发言议论,这些意见是有钱买不到的东西。

这些年来,这种调查研究工作不大作了。我们的同志不作调查研究工作,没有基础,没有底,凭感想和估计办事。劝同志们要大兴调查研究之风,一切从实际出发,没有把握就不要乱发言,不要下决心。作调查研究也并不那么困难,人不要那样多,时间也不要那么长,在农村有一两个社队,在城市有一两个工厂,一两个学校,一两个商店,合起来有七、八个,十来个,也就行了。也不必都自己亲自去搞,自己搞一两个,其他就组织班子去搞,亲自加以领导。保定的报告是农村工作部搞的,是个大功劳;通海口是省委抽人下去的,灵宝县的报告是纪登奎同志下去搞的,信阳的报告是搞造后的地委下去搞的。

调查研究这件事极为重要,要教会许多人。所有省委书记、常委、各部门负责人、地委、县委、公社党委,都要进行调查研究,不做,情况就不清楚。公社有多少部门,第一书记不一定知道,一个公社,有三十多个队,公社党委只要摸透好、中、坏三个队就行。做工作要有三条:一是情况明,二是决心大,三是方法对。这里情况明是第一条,这是一切的基础。情况不明,一切都无从谈起,这就要搞调查研究。资产阶级是讲调查研究的。美国发言人总是说胡志明的军队进入老挝,但究竟进去什么兵,什么官,什么兵种,他们不说。资产阶级比我们老实,不知道就不讲。我们有时没有底,哇里哇啦一套。但是资产阶级也有冒失鬼,资本主义国家有个杂志说:从五一年到六〇年,就把苏联和一切社会主义国家都消灭掉。

这次会议,情况逐渐明朗,决心逐步大。但是决心还是参差不齐的。有的同志讲刮共产风要破产还债,听起来不好听,但实际上是要破产还债。县、区、社两级通通破掉就好了,破掉以后再来真正的白手起家。……我们是马克思列宁主义者,不能剥夺劳动者,只能剥夺剥夺者,这条是马克思列宁主义的基本原则。……资产阶级、地主阶级剥夺劳动人民,马克思列宁主义者不能剥夺劳动人民。资产阶级、地主阶级的方法比我们还高明,他们是逐步使劳动者破产欠债,我们是一下子平掉,用这种办法建立社有经济、国营经济。我们的国营经济赚钱太多,到农村中去收购,常常压级压价,剥夺农民,交换非常不等价,这就使工人阶级脱离他们的同盟者。这个道理,同志们也懂得,话也好讲,但实行起来决心不大,不那么容易。是不是所有的省委书记都有那么大的决心破产还债,还得看看。这也是不平衡的,各省也会是参差不齐的。可能有的省决心大,彻底一些,把群众团结在自己的周围。有些省决心不大,作的差一些。一省之内,几十个县也会是不平衡的,因为领导人的情况不同。一类县、社、队有百分之三十共产风刮了一下,停的早,五九年郑州会议后就停下来了,他们懂得不能剥夺农民,不能黑手起家,决心大,退赔的彻底,以后就不再刮了。有些搞变得不彻底,一次再一次刮共产风。去年春季,中央情况不明,以为共产风不很严重,所以搞得不彻底。其实去年春季就应该开这样的会,纠正共产风,可是没有开。我们对情况不够明,问题不集中,决心不大,方法也不大那样对头,不是像现在信阳、通海口、保定、灵宝的方法。所以这件事是个大事情,这是一场大斗争,要在实践与斗争中认识问题,解决问题。农忙过后还要再搞,一、二类社、队也还不少,还要抓紧搞,下决心搞彻底。\marginpar{2}总而言之,过去抗战时期、解放战争时期,调查研究比较认真,实事求是,从实际出发,情况明了,决心就大,方法就对头,解决问题的措施也较有利。只有正确的方针政策,但情况不明,决心不大,方法不对,还是等于零。郑州会议讲不能一平二调,方针是对的,说不算账、不退赔,这点不对。上海会议十八条讲了要退赔,紧接着我批了浙江、麻城的经验报告。五九年三、四月,我批了两万多字的东西。现在看来,光打笔墨官司,不那么顶用。他封锁你,你情况不明,有什么办法?那时省委、地委的同志也不那么认识共产风的危害性。有的同志讲郑州会议是压服,不是说服,思想还有距离,所以决心不大,搞的不够彻底。

工业开始摸了一些底,还要继续摸底。要缩短工业战线,重工业战线,特别是基本建设战线。要延长农业战线,轻工业要发展。重工业除煤炭、矿山、木材、运输之外,不搞新的基本建设,过去搞了的,有些还要搞,但有些也不搞,癞了头就让它癞头去吧。

长远计划现在搞不出来,我们要再搞十年,从六〇年到六九年,这是个革命。中国的封建主义搞了那么多年,民主革命也搞了那么多年,没有民主革命的胜利,就没有社会主义。搞社会主义建设不能那么急,十分急搞不成,要波浪式前进。陈伯达同志提出,社会主义建设是否也有个周期率,若干年发展较快,有几年较低,如同行军一样,有大休息、中休息、小休息,要劳逸结合,两个战役间要休整。这次工作会议也有劳逸,决议文件也不多,譬如郑州会议就只搞了那么一个决议嘛!还是看情况明不明,决心大不大,方法对不对头。

现在看一个材料说:西德钢产量去年是三千四百万吨,英国二千四百万吨,西德六〇年比五九年增加百分之十五,法国是一千七百万吨,日本是二千二百万吨。但他们的生产率是长期积累的,搞了那么多年,才那么多,我们才几年,就××××万吨。今、明、后年,搞几年慢腾腾,搞扎实一些,然后再上去。指标不要那么高,把质量搞上去,让帝国主义说我们大跃进垮台了,这样对我们比较有利,不要务虚名而得实祸。要提高质量、规格、品种,提高管理水平,提高劳动生产率。现在我们劳动生产率很低。五七年我们职工有二千四百多万人,现在有五千多万人,还要下放。不然,五六个人围着一台机器,一个人做,几个人看,这不行。解决这个问题也是要情况明,决心大,方法对。

陈伯达同志有个材料,美国一个农民劳动力养活三十个人,英国二十六个人,苏联六个人,我们只有三个半人。有人说我们也可以养四个人,那就看你怎样养了,如果一天只吃几两米,那不行。

国际形势我看也是很好的。原来我们讲要硬着头皮顶,准备顶它十年。从前年西藏闹事到现在,不过二十多个月,现在反华的空气大为稀薄了,但空气还是有,有时还有寒流。莫斯科会议以后,空气还好一些。

今年搞一个实事求是年。实事求是是汉朝的班固在汉书上说的,一直流传到现在。我党有实事求是的传统,但最近几年来不大了解情况,大概是官做大了,摸不了底了。今年要摸它一个工厂、一个学校、一个商店、一个连队、一个城市人民公社,不搞典型就不好工作。这次会议以后,我就下去搞调查研究工作。总而言之,现在摸到这个方向,大家都要进行。不要只讲人家的坏话,有的地方工作有错误,人家搞了,就要欢迎人家。\marginpar{3}


\section[在八届九中全会上的讲话(一九六一年一月十八日)]{在八届九中全会上的讲话}
\datesubtitle{(一九六一年一月十八日)\footnote{据《毛泽东年谱》,1月18日 在中南海怀仁堂主持中共八届九中全会全体会议}}

这次会议,因为经过二十天的工作会议的准备,开得比较顺利。今天想讲一讲工作会议上讲过的调查研究问题,别的问题也讲一下。

我们在反帝反封建的民主革命时期,提倡调查研究,那时全党调查研究工作作风比较好,解放后十一年来就较差了。什么原因?要进行分析。在民主革命时期,犯过几次路线错误,在解放后又出过高岗路线。右的不搞调查研究,“左”的也不搞调查研究。那时,中国是什么情况,应采取什么战略方针和策略方针才适合中国实际情况,长时期没有得到解决。自从我们党一九二一年成立起,到一九三五年遵义会议,十四年间,有正确的时候,也有错误的时候。大革命遭到了损失,第二次国内革命战争也遭到了损失,长征损失也很大,在遵义会议后到了延安,我们党经过了整风,七大时,……王明路线基本上克服了。抗战八年我们积蓄了力量,因此在一九四九年,我们取得了全国革命的胜利,夺取了政权。解放战争时期,我们和蒋介石作战,情况比较清楚,比较注意搞调查研究,对革命一套比较熟悉,那时情况也比较单纯。胜利后有了全国政权,几亿人口情况比较复杂了。我们过去有过几次错误,陈独秀机会主义错误,立三路线的错误,王明路线的错误等等,有了几个比较,几个反复,容易教育全党。近几年来,我们也进行了一些调查研究,但比较少,情况不甚了解。譬如农村中,地主阶级复辟问题,不是我们有意识给他挂上这笔账,而是事实是这样,他们打着共产党的旗子,实际上搞地主阶级复辟。在出了乱子以后,我们才逐步认识在农村中的阶级斗争是地主阶级复辟。凡是三类社、队,大体都是与反革命有关系,这里边也有死官僚。死官僚实际上是帮助了反革命,帮助了敌人,是地、富、反、坏、蜕化变质分子的同盟军。因为死官僚不关心人民生活,不管主观愿望如何,实际上帮助了敌人,是反革命的同盟军。还有一部分是糊涂人,不懂得什么叫三级所有,队为基础,不懂得共产风刮不得。反革命、坏分子、蜕化变质分子就利用死官僚、糊涂人把坏事做尽。一九五九年,有一个省,本来只有××××亿斤粮食,硬说有××××亿斤,估得高,报得高。出现了四高,就是高指标、高估产、高征购、高用粮,直到去年北戴河工作会议,才把情况摸清楚。现在事情又走到了反面,是搞低标准,瓜菜代。经过调查研究,从不实际走到比较合乎实际。

农、轻、重,工农业并举,两条腿走路,我讲了五年,庐山会议也讲了,但去年没有实行。看来今年可能实行,我只说可能实行,因为现在还没有兑现。一九六一年国民经济计划已经反映了这一点,注意了农、轻、重,就可能变成现实。

对地主的复辟,我们也缺少调查研究。我们进城了,对城市反革命分子比较注意,比较有底。一九五六年匈牙利事件以后,我们让他们分散的大鸣大放,出了几万个小匈牙利。这样把情况弄清楚了,就进行了反右斗争。整出了×××万个右派,搞得比较好,底摸清了,决心就转大了。农村那年也整了一下,没有料到地主阶级复辟问题。当然,抽象的讲是料到的。\marginpar{4}过去我们总是提出国内矛盾是资产阶级和无产阶级的矛盾,是社会主义与资本主义两条道路的矛盾,基本矛盾是阶级矛盾。是资本主义的天下,还是社会主义的天下,是地主的天下,还是人民的天下?

没有调查研究,情况不明,决心就不大。一九五九年就反对刮共产风,由于情况不明,决心就不大,中间又加上了一个庐山会议,反右倾机会主义。本来庐山会议要纠正“左”倾错误,总结工作,可是被右倾机会主义进攻打断了,反右是非反不可的。会后,共产风又刮起来了,急于过渡,搞了几个大办。大办社有经济、大办水利、大办养猪、大办县社企业、大办土铁路。同时要这么些个大办,如养猪什么也不给,这就刮起共产风来了。当然大办水利、大办工业取得了很多的成绩,不可抹煞。还有大办文化、大办教育、大办卫生等等,不考虑能不能做。共产风问题,反革命复辟问题,死官僚问题,糊涂人问题,干部情况问题,县、社、队分为一、二、三类,各占百分之××、百分之××、百分之××问题,这些问题以前我们就没有搞清楚,有的摸了,我们也没有讲清楚,或讲清楚了也不灵。郑州会议反共产风,只灵了六个月,庐山会议后冬天又刮起共产风。庐山会议前,“左”的情况还没有搞清楚,党内又从右边刮来一股风。彭德怀等人与国际修正主义分子、国内右派相呼应,打乱了我们纠“左”步骤。

去年一年国际情况比较清楚,对国内问题也应该聚精会神调查研究,工人阶级要团结农民大多数,首先是贫农、下中农和较好的中农,依靠他们对付地主反革命。三类社、队要成立贫下中农委员会,在党的领导下,主持整风整社,并临时代行社、队管理委员会的职权。我们党内也有代表地主,资产阶级,小资产阶级的人,应该纯洁党的组织,经过整风、整顿组织,使党纯洁起来,使绝大多数党员都代表贫农、下中农的利益,同时也不损害富裕中农的利益,坚持不剥夺农民利益的马克思列宁主义原则。刮共产风是非常错误的,是剥夺农民的,是反马克思列宁主义的,必须坚决退赔。经验证明,只要退赔,群众就满意了,情况就改善了。

这次工业计划比较切合实际,缩短了基本建设战线,延长了农业、轻工业战线。与农业有关的基本建设还要搞,有的重工业,像煤、木材、矿石、铁路还要搞。上下一本找账,不搞两本账。不要层层加码。总之,要实事求是,使一切从实际出发。粮食要过秤入库,不搞四高,搞低标准,瓜菜代,坚决退赔,整顿五风,不准不赔,不准不退。

城市也要整风,正在搞试点,还要一、二个月才能搞出来,也要搞十二条。

今年计划看来比去年高不了多少。有人建议钢仍然搞×××××××万吨到××××万吨,也增加不了多少;这个提法有道理。第二个五年计划钢的指标,早己超额完成,还剩两年,就是要搞质量、规格、品种,在质量上好好跃进一下,数量上不准备多搞。帝国主义者、修正主义者,会说大跃进垮台了,他们要讲就让他们讲,他们讲坏话也好,讲我们好反而不好。实际上我们现在就是要搞质量、规格、品种,搞企业管理制度、技术措施,提高劳动生产率,降低成本,成龙配套,要搞调整、充实、提高,就是要在这方面努力。英国、日本的钢,暂时还比我们的多,再有×年,我们总会赶上他们,并且还会超过他们。能否超过西德,还要看一看。讲打仗,斗地主,我们有一套经验,搞建设还比较缺乏经验,我与斯诺谈话就谈到这一点。凡是规律总要经过几次反复才能找到,我们只希望不要像民主革命花二十八年才成功。其实二十八年也不算很长,许多国家的党同我们同年产生,现在也还没有成功。搞建设是不是可以二十年取得验验,我们搞了十一年,看再有九年行不行。曾想缩短很多,看来不行。凡是没有被认识的东西,你就没法改造它。\marginpar{5}

工业还是要鼓干劲,不然几次会议一开,劲就没有了。泄了二、三个月的气,然后再开一次鼓干劲的会,反右倾。大家回去以后,要实实在在的干,不要老算账。搞计划要好好调查研究,搞清情况,鼓足干劲,力争上游,多快好省,坚持总路线。有人说现在不用多快了,这不对,搞粮食就要多快嘛,搞工业讲质量、成龙配套等等,也是要搞多快嘛!

团结问题。中央委员会的团结是全党团结的核心。庐山会议有少数人闹不团结,我们希望和他们团结,不管他们的错误有多大,只要他们能改。他们讲你们也有错误,不错,错误人人皆有,但错误大小轻重不同,性质不同,数量质量不同。不要一犯错误就抬不起头来。有的同志工作职位降低了,降低了也好。一年来有进步,不管真假,总是值得欢迎。地方工作的同志有的也犯了错误,欢迎他们改正。

河南、甘肃、山东三省问题比较严重,情况不明,决心不大,方法不对。现在情况明了就好了。有些地方政权也夺回来了,面貌已经开始一新。甘肃也开始好转,其他各省也总要烂掉若干县、社、队,大体是百分之××左右,严重的超过百分之××,好的不到百分之××。不光是因为粮食问题。林彪同志讲,军队有××个单位,烂掉了××个,占百分之×,这并不是因为粮食问题。这种情况在城市、工厂、学校一定会有。对××类干部,要按政策清洗出去,死官僚要改造,变成活官僚,长久活不起来的也要清洗。这些人是少数,合起来也不过百分之几,百分之九十以上是好人,其中也有糊涂人。我们也糊涂过,不然在民主革命时期,大革命为什么失败,南方根据地丧失,白区力量丧失,要长征,是因为不了解情况。

现在搞社会主义建设,是个新问题,我们缺少经验,要开训练班,把县、社、队干部轮训一遍,使他们懂得政策。如果一个省只有一个县委书记能讲清政策,不训练干部怎么行?每个县都要有一个县委书记真正能懂政策,弄清政策,那就好了。现在中央下放了八千多个干部帮助农村整风整社。大多数农村干部是好的,可靠的。如果大多数是国民党,我们还能在这里安心开会吗?所有一切可团结的人要团结。就是对反革命分子也不能都杀,不杀不足以平民愤的才杀,有的要关起来,管起来。杀人要谨慎,切不可重复过去所犯过的错误,如过去搞根据地时杀人多了一些。延安时规定一条,干部一个不杀。现在还关了一个潘汉年,绝对不杀。杀了就要比,这个杀了,那个杀不杀?总是不开杀戒。但是不是说社会上一个不杀?有些不杀不足以平民愤的人,民愤很大的人,不能不杀几个。至于中央委员犯了错误就不牵涉杀不杀的问题,还是留在中央委员会工作。要与各兄弟党团结,要和苏联的党团结,要和八十一个共产党和工人党团结,我们要采取团结的方针。

过去我们吃了亏,就是不注意调查研究,只讲普遍真理。六一年要成为调查研究年,在实践中调查研究,专门进行调查研究。

\section[《反对本本主义》说明(一九六一年三月十一日)]{《反对本本主义》说明\\{\large(原名:“关于调查研究”)}}
\datesubtitle{(一九六一年三月十一日)\footnote{据《毛泽东年谱》,3月10日—13日 在广州小岛招待所主持召开三南会议,主要讨论人民公社体制和工作条例问题。同时,刘少奇、周恩来、陈云、邓小平于三月十一日至十三日在北京主持召开有中共中央西北局、东北局、华北局及所属各省、市、自治区负责人参加的工作会议(即三北会议),讨论的问题与三南会议相同。后来,北京方面向毛泽东建议,两边合起来开会,得到毛泽东同意。十四日,参加三北会议的同志到达广州。3月11日 同胡乔木、田家英谈《调查工作》一文修改问题。同日 为印发《调查工作》一文给三南会议写如下批语:“这是一篇老文章,是为了反对当时红军中的教条主义思想而写的。那时没有用‘教条主义’这个名称,我们叫它做‘本本主义’。写作时间大约在一九三〇年春季,已经三十年不见了。一九六一年一月,忽然从中央革命博物馆里找到,而中央革命博物馆是从福建龙岩地委找到的。看来还有些用处,印若干份供同志们参考。”并批注:“送林彪同志阅,一九三〇年的,从闽西找出来的。阅后退毛。”毛泽东在印发这篇文章时,对正文作了一些文字修改,将标题改为《关于调查工作》。}}

这是一篇老文章,是为了反对当时红军中的教条主义思想而写的。那时没有用“教条主义”这个名称,我们叫它做“本本主义”。\marginpar{6}写作时间大约在一九三〇年春季,已经三十年不见了。一九六一年一月,忽然从中央革命博物馆里找到。而中央革命博物馆是从福建龙岩地找到的。看来还有些用处。印若干分供同志们参考。

\kaitiqianming{毛泽东}
\kaoyouriqi{一九六一年三月十一日}


\section[《中央关于认真进行调查研究工作问题给各中央局、省、市、自治区党委一封信》(摘录)(一九六一年三月二十七日)]{《中央关于认真进行调查研究工作问题给各中央局、省、市、自治区党委一封信》(摘录)}
\datesubtitle{(一九六一年三月二十七日)}

毛主席提倡哲学要走出课堂,走出书斋。毛主席讲:真理在谁手里,我们就跟谁走,挑大粪的人有真理,我们就跟挑大粪的人走。

\section[在广州会议上的讲话(节录)(一九六一年三月)]{在广州会议上的讲话(节录)}
\datesubtitle{(一九六一年三月)}

主要是两个平均主义的问题。一个是生产大队(即原来的生产队)内部,各生产队(即原来的生产小队)与生产队之间的平均主义,一个是生产队内部,社员与社员之间的平均主义。这两个问题很大,不彻底解决,不可能真正地全部地调动群众的积极性。

人民公社是生产大队的联合组织。

公社、生产大队不能瞎指挥,县、地、省、中央也不能瞎指挥。

不能用领导工业的办法来领导农业,也不能用领导农业的方法来领导工业。

\datesubtitle{(三月二十三日)}

几年来出的问题,大体上都是因为胸中无数,情况不明,政策就不对,决心就不大,方法也就不对头。最近几年吃情况的亏很大,付出的代价很大。

要作系统的由历史到现状的调查研究,省、地、县、社的第一书记,都要亲自动手。不做好调查工作,一切工作都无法做好。第一书记亲身调查很重要,足以影响全局。今后我们必须摆脱一部分事务工作,交别人去做。报告也要看,但是不要满足于看报告。最重要的是亲自作典型调查,走马观花只能是辅助的方法。

只有原理原则,没有具体政策不能解决问题。没有调查研究,就不能产生正确的具体政策。

调查研究的态度,不可以先入为主,不可以自以为是,不可以老爷式的,决不可以当钦差大臣。而要讨论式的,同志式的,商(量)的。

不要怕听不同的意见,原来的判断和决定,经过实际检验,是不对的,也不要怕推翻。



\section[在接见亚洲、非洲外宾时的谈话(一九六一年四月二十八日)]{在接见亚洲、非洲外宾时的谈话}
\datesubtitle{(一九六一年四月二十八日)}

毛主席对非洲、阿拉伯各国人民反对帝国主义的斗争表示深切的同情和支持。毛主席指出当前的国际形势对非洲、亚洲、拉丁美洲人民的反帝斗争是非常有利的。毛主席说,对帝国主义进行斗争当中,采取正确的路线,依靠工人、农民,团结广大的革命知识分子,小资产阶级和反对帝国主义的民族资产阶级以及一切爱国反帝力量,紧紧地联系群众,就有可能取得胜利。毛主席指出,革命政党和力量,在开始时都是处于少数地位的,但最有前途的就是他们。

毛主席严厉地谴责美帝国主义对古巴的侵略,并指出,美国帝国主义迫不及待地进攻古巴,再一次在全世界面前揭露了它的真面目,说明了肯尼迪政府只能比艾森豪威尔政府更坏些,而不是更好些。美国帝国主义利用联合国作为工具侵略刚果和杀害芦蒙巴的罪行,将非洲人民对美国帝国主义的认识进一步提高了。

毛主席表示,最近万隆亚非团结会议上表达的亚非人民同拉丁美洲人民加强团结的愿望,是好的。对世界人民反对帝国主义斗争的共同事业是有益的。

毛主席说,中国人民把亚洲、非洲、拉丁美洲人民的反帝国主义斗争的胜利看作是自己的胜利,并对他们的一切反帝国主义;反殖民主义的斗争给以热烈的同情和支持。

\kaoyouriqi{(《人民日报》一九六一年四月二十九日)}



\section{在接见古巴文化代表团时的谈话(一九六一年四月十九日)}


中国和古巴是两个友好的国家,我们相互帮助,相互支持。我们的目标是一个。反对帝国主义。美国帝国主义是帝国主义中最大的一个,它不但压迫我们,也压迫你们,它压迫全世界人民。在一些不是帝国主义的国家中,都有一些帝国主义的走狗。我们不但要反对帝国主义,也要反对帝国主义的走狗。

毛主席最后在送别古巴朋友的时候,祝古巴人民在反对美国帝国主义侵略的斗争中取得胜利,致以亲切的问候。并祝古巴人民庄反对美国帝国主义侵略的斗争中取得胜利。



\section[调查成灾一例(一九六一年五月三十日)]{调查成灾一例\\{\large——对“关于‘调查研究’的调查”的批示}}
\datesubtitle{(一九六一年五月三十日)}

如果还是如同去长辛店铁道机车车辆制造厂做调查的那些人们实行官僚主义的老爷式的使人厌恶得透顶的那种调查方法,党委有权教育他们。死官僚不听话的,党委有权把他们轰走。\marginpar{8}\footnote{5月28日 阅田家英报送的戚本禹五月十二日写的材料《关于“调查研究”的调查》和田家英报送这个材料的信。田家英的信中说:秘书室工作人员戚本禹,去年六月下放到长辛店机车车辆工厂劳动。最近他寄了一份材料给我,反映一些机关、学校人员到工厂作调查的情况。这个材料提出了一些在大兴调查研究之风中间值得注意的问题。戚本禹的材料说,他们利用业余时间摸了一下各级领导机关到长辛店机车车辆厂做调查研究工作的情况,认为在二十几个调查组的工作里,比较普遍地存在着“十多十少”的问题。毛泽东为戚本禹的材料拟了一个题目《调查成灾的一例》。批示:“此件印发工作会议各同志。同时印发中央及国家机关各部门各党组。派调查组下去,无论城乡,无论人多人少,都应先有训练,讲明政策、态度和方法,不使调查达不到目的,引起基层同志反感,使调查这样一件好事,反而成了灾难。”30日,毛泽东对这个材料再次批示:“此件,请中央及国家机关各部门各党组,各中央局,各省、市、区党委,一直发到县、社两级党委,城市工厂、矿山、交通运输基层党委,财贸基层党委,文教基层党委,军队团级党委,予以讨论,引起他们注意,帮助下去调查的人们,增强十少,避免十多。如果还是如同下去长辛店铁道机车车辆制造工厂做调查的那些人们,实行官僚主义的老爷式的使人厌恶得透顶的那种调查法,党委有权教育他们。死官僚不听话的,党委有权把他们轰走。同时,请将这个文件,作为训练调查组的教材之一。”}\footnote{见链接《毛泽东年谱(1949—1976)》http://dangshi.people.com.cn/n/2013/0527/c85037-21626561-7.html}

\section{在北京会议上的讲话(一九六一年六月十二日)}


人民公社问题,在一九五八年的北戴河会议以后,开了两次郑州会议。第一次会议解决集体所有制和全民所有制的界线问题,社会主义和共产主义的界线问题。第二次会议解决公社内部三级所有制的界线问题。这两次会议的基本方向是正确的。但是,会议开得很仓促,参加会议的同志没有真正在思想上解决问题,对于社会主义建设的客观规律,开始懂得了一些,还是懂得不多。在一九五九年三月的上海会议上,通过了关于人民公社的十八个问题的纪要。后来,我给小队以上的干部写了一封“党内通讯”,对农业方面的六个问题提了意见。在这一段时间内,普遍地对人民公社进行了整顿,使工作中的缺点和错误逐步地得到纠正。不过,由于各级干部还不真正懂得什么是社会主义,什么是按劳分配,什么是等价交换,他们对党中央关于人民公社的许多意见和规定,还没有认识清楚,他们的思想问题还没有得到解决。一九五九年夏季庐山会议上,右倾机会主义分子向党进攻,我们举行反击,获得胜利。反右以后。工作中出现了一些假象。有些地方有些同志以为从此再不要根据两次郑州会议的精神,继续克服工作中存在的缺点和错误了。一九六〇年春,我看出“共产风”又来了,批转了广东省委关于当前人民公社工作中几个问题的指示。在广州,召集中南各省的同志开了三小时的会,接着在杭州又召集华东、西南各省的同志开了三、四天会,后来又在天津召集东北、华北、西北各省同志开了会。这些会,都因为时间短,谈的问题很多,没有把反“一平二调”、反“共产风”的问题作为中心突出来,结果没有解决问题。几个大办一来,糟糕,那不是“共产风”又来了吗,一九六〇年北戴河会议,用百分之七、八十的时间谈国际问题,只是在会议快结束的时候,谈了一下粮食问题,没有接触到人民公社内部的平均主义问题。同年十月,中央发了关于人民公社十二条的指示,从此开始认真纠正“一平二调”的错误,但是仍然坚持供给制、公共食堂、粮食到堂的作法。而且,在执行中,只对三类县、社、队进行了比较认真的整顿,对于一、二类县、社、队的“五风”基本上没有触动,放过去了。一九六一年一月九中全会以后,经过农村调查,在广州开会,强调提出人民公社内部存在着必须解决的两个平均主义问题,起草了农村人民公社六十条。这次会议,启发了思想,解放了思想,然而还不彻底,继续保留了三七开(即供给部分三成,按劳部分七成的分配办法)、公共食堂、粮食到堂的尾巴。经过会后的试点和调查,到这次会议,大家的思想彻底解放了,上面所说几个问题的尾巴最后解决丁,大家对社会主义建设规律的认识,也比过去清楚得多了。由此可见,对客观世界的认识是逐步深入的,任何人不能例外。

<p align="center">×××</p>

我们党从一九二一年成立,经过了陈独秀右倾机会主义和三次“左”倾机会主义的严重挫折,经过了万里长征,经过延安整风,通过了“关于若干历史问题的决议”,到一九四五年的七次代表大会,共用了二十四年的时间,本形成了思想上真正的统一,并且在政治、军事、经济、文化和党的建设等方面形成了一整套实现民主革命总路线的具体政策,保证了抗日战争和解放战争的胜利。建设社会主义社会,我们过去谁也没有干过,必须在实践中才能逐步学会。我们已经搞了十一年,有了社会主义建设总路线,积累了很多经验。只有总路线还不够,还必须有一整套具体政策。现在要好好地总结经验,逐步地把各方面的具体政策制定出来。我们已经有可能这样做,并且已经制定了人民公社六十条。

最近林彪同志下连队做调查研究,一次是在广州,一次是在杨村,了解了很多情况,发现了我们部队建设中一些重要的问题,提出了几个很好的部队建设的措施。要搞具体政策,没有调查研究是不行的。

形成一整套的具体政策,看来还需要一段时间,也许还需要十多年。这是一种设想。如果大家都觉悟了,也可能缩短一些。

<p align="center">×××</p>

现在的重要问题是要重新教育干部。干部教育好了,我们的事业就大有希望。不教育好干部,我们就毫无出路。我们要利用人民公社六十条等文件,作为教材,用延安整风的方法,去教育干部。这次参加会议的同志,思想通了,就要去教育地、县的干部,他们的思想通了,再由他们去教育社、队的干部,使大家具正懂得什么叫社会主义,什么叫按劳分配、等价交换。教育干部的事情,今年一定要做出一点成绩来,并且一定要长期地做下去。搞民主革命,我们长期地教育干部,搞社会主义也必须如此。



\section[给江西共产主义劳动大学的一封信(一九六一年七月三十日)]{给江西共产主义劳动大学的一封信}
\datesubtitle{(一九六一年七月三十日)}

{\noindent 同志们:}

你们的事业我是完全赞成的。半工半读,勤工俭学,不要国家一文钱,小学、中学、大学都有,分散在全省各个山头,少数在平地。这样的学校确是很好的。在校的青年居多,也有一部分中年干部。我希望不但在江西有这样的学校,各省也应有这样的学校。各省应派有能力有见识的负责同志到江西来考察,吸收经验,回去试办。初时学生宜少,逐渐增多,至江西这样有五万人之多。再则党、政、民(工、青、妇)机关,也要办学校,半工半学。不过同江西这类的半工半学不同。江西的工,是农业、林业、牧业这一类的工,学是农、林、牧这一类的学。而党、政、民机关的工,则是党、政、民机关的工,学是文化科学、时事、马列主义理论这样一些学,所以两者是不同的。中央机关已办的两个学校,一个是中央警卫团的,办了六、七年了,战士、干部们从初识文字进小学,然后进中学,然后进大学,一九六零年他们已进大学部门了。他们很高兴,写了一封信给我,这封信,可以印给你们看一看。另一个是去年(一九六零年)办起的。是中南海的各种机关办的,同样是半工半读。工是机关的工,无非是机要人员、生活服务人员、招待人员、医务人员、保卫人员及其他人员。警卫团是军队,他们也有警卫职务,即是站岗守卫,这是他们的工。他们还有严格的军事训练。这些,与文职机关的学校是不同的。

一九六一年八月一日,江西共产主义劳动大学三周年纪念,主持者要我写几个字。这是一件大事,因此为他们写了如上的一些话。

{\raggedleft 毛泽东\\一九六一年七月三十日\par}



\section[对《各地贯彻执行六十条的情况和问题》的批语(一九六一年九月六日)]{对《各地贯彻执行六十条的情况和问题》的批语}
\datesubtitle{(一九六一年九月六日)}

此件很好,印发各同志。并带同去,印发县、市、区党委一级的委员同志们,开一次省委扩大会,有地委同志参加,对此件第二部分所提出的十个问题,作一次认真的解决。时间越早越好,以便在秋收、秋耕、秋种和秋收分配时间政策实行兑现,争取明年丰收。冬春两季六个月整风整社,训练干部,也在这一次省委扩大会上作出布置,主动权就更大了。生产、征购、生活安排,同时并举,就更加主动了。

\section[给政治局常委及有关同志的信]{给政治局常委及有关同志的信(一九六一年九月二十九日)}
\datesubtitle{(一九六一年九月二十九日)}

我们对农业方面的严重平均主义的问题,至今还没有完全解决,还留下一个问题。农民说,六十条就是缺了这一条。这一条是什么呢?就是生产权在小队,分配权却在大队,即所谓“三包一奖”的问题。这个问题不解决,农、林、牧、副、渔的大发展即仍然受束缚,群众的生产积极性仍然要受影响。如果要使一九六二年的农业比较一九六一年有一个较大的增长,我们就应在今年十二月工作会议上解决这个问题。我的意见是:“三级所有,队为基础”。即基本核算单位是队而不是大队。所谓大队“统一领导”要规定界限,河北同志规定了九条。如果不作这种规定,队的九种有许多是空的,还是被大队抓去了。此问题,我在今年三月广州会议上,曾印发山东一个暴露这个严重矛盾的材料,又印了广东一个什么公社包死任务的材料,并在这个材料上面批了几句话:可否在全国各地推行。结果,没有通过。待你们看了湖北、河北这两批材料,并且我们一起讨论过了之后,我建议。把这些材料,并附中央一信发下去,请各中央局、省、市、区党委、地委及若干县委亲身下去,并派有力的调查研究组下去,作两星期调查工作,同县、社、大队、队、社员代表开几次座谈会。看究竟那样办好。由大队实行“三包一奖”好,还是队为基础(河北人叫做分配大包干)好?要调动群众对集体生产的积极性,要在明年一年及以后几年,大量增产粮、棉、油、麻、丝、茶、糖、菜,烟、果、药、杂以及猪、马、牛、羊、鸡、鸭、鹅等类产品,我以为非走此路不可。在这个问题上,我们过去过了六年之久的糊涂日子(一九五六年高级社成立时起),第七年应该醒过来了吧!



\section{在接见日本朋友时的谈话(一九六一年十月七日)}


日本除了亲美的垄断资本家和军国主义军阀之外,广大人民都是我们的真正朋友。你们也会感到中国人民是你们的真正朋友。朋友有真有假,但通过实践可以看清谁是真朋友,谁是假朋友。

中国有句古话,物以类聚,人以群分。日本的岸信介和池田勇人是美帝国主义和蒋介石集团的好朋友。日本人民同中国人民是好朋友。

是美帝国主义迫使我们中日两国人民团结起来。我们两国人民都遭受美帝国主义的压迫,我们有着共同的遭遇,就团结起来了。我们要扩大团结的范围,把全亚洲、非洲、拉丁美洲以及全世界除了帝国主义和各国反动派以外的百分之九十以上的人民团结在一起。

尽管斗争道路是曲折的,但是日本人民的前途是光明的。中国革命经过无数次的曲折,胜利、失败、再胜利、再失败,最后的胜利属于人民。日本人民是有希望的。

<p align="right">(原载《新华月报》1961年第十一期)</p>



\section{关于科学研究十四条的指示(一九六二年一月)}


……科学研究工作十四条,这些条例草案已经在实行或者试行,以后要修改,有些还可能大改,“应当好好地总结经验,制定一整套的方针政策和办法,使他们在正确的轨道上前进”。,“在总路线指导下,制定一整套的方针、政策和办法,必须通过从群众中来的方法,通过做系统的、周密的调查研究的方法,对群众中的成功经验和失败经验,作历史的考察,才能找出客观事物所固有的而不是人们主观臆造的规律,才能制定适合情况的各种条例。这事很重要,请同志们注意到这点”。



\section[给郭沫若回信中的几句话(一九六二年一月)]{给郭沫若回信中的几句话}
\datesubtitle{(一九六二年一月)}

一九六一年,郭沫若看了《孙悟空三打白骨精》后,写了一首诗\footnote{郭沫若诗:《七律·孙悟空三打白骨精》人妖颠倒是非淆,对敌慈悲对友刁。咒念金箍闻万遍,精逃白骨累三遭。千刀当剐唐僧肉,一拔何亏大圣毛。教育及时堪赞赏,猪犹智慧胜愚曹。},毛主席十一月十七日写了一首七律《和郭沫若同志》\footnote{一从大地起风雷,便有精生白骨堆。僧是愚氓犹可训,妖为鬼蜮必成灾。金猴奋起千钧棒,玉宇澄清万里埃。今日欢呼孙大圣,只缘妖雾又重来。},站得更高,看得更远。以后一九六二年一月六日郭沫若又作了一首和主席的诗,诗曰:

赖有晴空霹雳雷,不教白骨聚成堆。九天四海澄迷雾,八十一番弭大灾。僧受折磨知悔恨,猪期振奋报涓埃。金睛火眼无容赦,哪怕妖精亿度来。

该和诗由康生同志转交主席,主席回信郭沫若说“和诗好,不是‘千刀当剐唐僧肉’。对中间派采取了统一战线政策,这就好了。”

\section[在扩大的中央工作会议上的讲话(一九六二年一月三十日)]{在扩大的中央工作会议上的讲话}
\datesubtitle{(一九六二年一月三十日)}


各中央局、各省、市、自治区党委、中央各部委、国家机关和人民团体各党委、党组、总政治部:

毛泽东同志一九六二年一月三十曰《在扩大的中央工作会议上的讲话》是一个十分重兵的马克思列宁主义的文件。中央决定将这个本件发给你们,供党内县团级以上干部学习。毛泽东同志在这个讲话中,着重讲了民主集中制的问题。这个问题是我们党的生活中一个根本性的问题。在我们党掌握了全国政权之后,这个问题尤其重要。毛泽东同志最近指出:“看来问题很大,真要实现民主集中制,是要经过认真的教育、试点和推广,并且经过长期反复进行,才能实现的,否则在大多数同志们当中,始终不过是一句空话。

望各地区、各部门根据毛泽东同志的指示,认真地学习这个文件,发扬批评和自我批评的精神,教育广大干部,特别是领导干部,认真贯彻实行民主集中制和纠正违反民主集中制的各种不良倾向。

《发至县团党委,不登党刊》

<p align="right">中央

一九六六年二月十二日</p>

同志们:

我现在讲几点意见。(热烈鼓掌)一共讲六点,中心是讲一个民主集中制的问题,同时也讲到一些其他问题。

第一点、这次会议的开会方法

这次扩大的中央工作会议,到会的有七千多人。在这次会议开始的时候,×××同志和别的几位同志,准备了4个报告稿子。这个稿子,还没有经过中央政治局讨论,我就向他们建议。不要先开中央政治局会议讨论了,立即发给参加大会的同志们,请大家评论,提意见。同志们,你们有各方面的人,各地方的人,有各省委、地委、县委的人,有企业党委的人,有中央各部门的人。你们当中的多数人是比较接近下层的,你们应当比我们中央常委、中央政治局和书记处的同志更加了解情况和问题。还有,你们站在各种不同的岗位,可以从各种角度提出问题,因此要请你们提意见。报告稿子发给你们了,果然议论纷纷,除了中央提出的基本方针以外,还提出许多意见。后来又由××同志主持,组织了二十一个人的起草委员会,这里有各中央局的负责同志参加,经过八天讨论,写出了书面报告的第二稿。应当说,报告第二稿是中央集中了七千多人议论的结果。如果没有你们的意见,这个第二稿不能写成。在第二稿里面,第一部分和第二部分有很大修改,这是你们的功劳。听说大家对第二稿的评价不坏,认为它是比较好的。如果不是采用这种方法。而是采用通常那种开会方法,就是先来一篇报告,然后进行讨论,大家举手赞成,那就不可能做到这样好。

这是一个开会的方法问题。先把报告草稿发下去,请到会的人提意见,加以修改,然后再作报告。报告的时候不是照着本子念,而是讲一些补充意见,作一些解释。这样,就更能充分地发扬民主,集中各方面的智慧,对各种不同的看法有所比较,会也开得活泼一些。我们这次会议是要总结十二年的工作经验,特别是要总结最近四年来的工作经验,问题很多,意见也会很多,宜于采用这种办法。是不是所有的会议郡可以采用这种方法呢?那也不是,采用这种方法,要有充裕的时间。我们的人民代表大会的会议,有时也可以采用这种方法,省委、地委、县委的同志们,你们以后召集会议,如果有条件的话,也可以采用这种方法。当然你们的工作忙,一般地不能用很长的时间开会,但是在有条件的时候,不妨试一试看。

这个方法是一个什么方法呢?是一个民主集中制的方法,是一个群众路线的方法。先民主,后集中,从群众中来,到群众中去,领导同群众相结合的方法。这是我讲的第一点。

第二点、民主集中制问题

看起来,我们有些同志,对于马克思、列宁所说的民主集中制,还不理解。有些同志已经是老革命了,“三八式”的或者别的什么式的,总之,已经工作了几十年的共产党员了,但是他们还不懂得这个问题。他们怕群众,怕群众讲话,怕群众批评。哪有马克思列宁主义者怕群众的道理呢?有了错误,自己不讲,又怕群众讲。越怕越有鬼。我看不应当怕。有什么可怕的呢?我们的态度是:坚持真理,随时修正错误。我们的工作中是和非的问题,正确和错误的问题,这是属于人民内部矛盾的问题。解决人民内部矛盾,不能用咒骂,也不能用拳头,更不能用刀枪,只能用讨论的方法,说理的方法,批评和自我批评的方法,一句话,只能用民主的方法,让群众讲话的方法。

不论党内党外,都要有充分的民主生活,就是说,都要认真实行民主集中制。要真正把问题敞开,让群众讲话,那怕是骂自己的话,也要让人家讲,骂的结果,无非是自己倒台,不能做这项工作了,降到下级机关去做工作,或者调到别的地方去做工作,那又有什么不可以呢?一个人为什么只能上升不能下降呢?为什么只能做这个地方的工作而不能调到别个地方去呢?我认为这种下降和调动不论正确与否,都是有益处的,可以锻炼革命意志,可以调查和研究许多新鲜情况,增加有益的知识。我自己就有这一方面的经验,得到很大益处。不信你们不妨试一试看。司马迁说过:“文王拘而演周易,仲尼厄而作春秋。届原放逐,乃赋离骚。左丘失明,厥有国语。孙子膑足,兵法修列。不韦迁蜀,世传吕览。韩非囚秦,说难孤愤。诗三百篇,大抵圣贤发愤之所为作也”,这几句话当中,所谓文王演周易,孔子作春秋,究竟有无其事,近人已有怀疑,我们可以不去理它,让专家们去研究吧,但是司马迁是相信有其事的。文王拘、仲尼厄的确有其事,司马迁讲的这些事情,除左丘的一例之外,都是指当时领导对他们作了错误处理的。我们过去也错误地处理了一些干部,对这些人不论是全部处理错的,或者是部分处理错的,都应当按照具休情况,加以甄别和平反。但是,一般地说,这种错误处理,让他们下降,或者调动工作,对他们的革命意志总是一种锻炼,而且可以从人民群众中吸取许多新知识。我在这里申明,我不是提倡对于部对同志,对任何人,可以不分青红皂白,作出错误处理,像古代人拘文王,厄孔子,放逐屈原,去掉孙膑的膝盖骨那样,我不是提倡这样做,而是反对这样做的。我是说,人类的各个历史阶段,总是有这样处理错误的事实。在阶级社会,这样的事实很多。在社会主义社会,也在所难免。不论在正确路线的领导时期,还是在错误路线领导时期,都在所难免。不过有一个区别,在正确路线领导时期,一经发现有错误处理的,就能甄别,平反,向他们赔礼道歉,使他们心情舒畅,重新抬起头来,而在错误路线的领导时期,则不可能这样做,只能由代表正确路线的人们,在适当的时机,通过民主集中制的方法,起来纠正错误。至于自己犯了错误.经过同志们的批评和上级的鉴定,作出正确处理,因而下降或调动工作的人,这种下降或调动,对于他们改正错误,获得新的知识,会有益处,那就不待说了。

现在有些同志,很怕同志开展讨论,怕他们提出同领导机关,领导者意见不同的意见。一讨论问题就压制群众的积极性,不许人家讲话,这种态度非常恶劣。民主集中制是上了我们的党章的。上了我们宪法的,他们就是不实行。同志们,我们是干革命的,如果真正犯了错误,这种错误是不利于党的事业,不利于人民的事业的,就应当征求人民群众和同志们的意见,并且自己做检讨,这种检讨,有的时候,要有若干次,一次不行,大家不满意,再来第二次,还有不满意,再来第三次,一直到大家没有意见了,才不再做检讨。有的省委就是这样做的。有一些省比较主动,让大家讲话,早的,一九五九年就开始做自我批评,晚的,也在一九六一年开始做自我批评。还有一些省委是被迫做检讨的,像河南、甘肃、青海。另外一些省,有人反映,好像现在才刚刚开始作自我批评。不管是主动的,被动的,早做检讨,晚做检讨,只要正视错误,肯承认错误,肯改正错误,肯让群众批评,只要采取了这种态度,都应当欢迎。

批评和自我批评是一种方法,是解决人民内部矛盾的方法,而且是唯一的方法。除此以外,没有别的方法。但是如果没有充分的民主生活,没有真正实行民主集中制,就不可能实行批评和自我批评这种方法。

我们现在不是有许多困难吗?不依靠群众,不发动群众和干部的积极性,就不可能克服困难。但是,如果不向群众和干部说明情况,不向群众和干部交心,不让他们说出自己的意见,他们还对你感到害怕,不敢讲话,就不可能发动他们的积极性。我在一九五七年这样说过:要造成“又有集中,又有民主,又有纪律,又有自由,又有统一意志,又有个人心情舒畅,生动活泼那样一种政治局面。”党内党外都应当有这样的政治局面。没有这样的政治局面,群众的积极性是不可能发动起来的。克服困难,没有民主不行。当然没有集中更不行,但是没有民主就没有集中。

没有民主,不可能有正确的集中,因为大家意见分歧,没有统一的认识,集中制就建立不起来。什么吗集中?首先是要集中正确的意见。在集中正确意见的基础上,做到统一认识,统一政策,统一计划,统一指挥,统一行动,叫做集中统一。如果大家对问题不了解,有意见还没有发表,有气还没出,你这个集中统一怎么建立得起来呢?没有民主,就不可能正确地总结经验。没有民主,意见不是从群众中来,就不可能制定出好的路线方针、政策和办法。我们的领导机关,就制定路线、方针、政策和办法这一方面说来,只是个加工工厂。大家知道,工厂没有原料就不可能进行加工。没有数量上充分的、质量上适当的原料,就不可能制造出好的成品来。如果没有民主,不了解下情,情况不明,不充分搜集各方面的意见,不使上下通气,只由上级领导机关凭着片面的或者不真实的材料决定问题,那就难免不是主观主义的,也就不可能达到统一认识,统一行动,不可能实现真正的集中。我们这次会议的主要议题,不是要反对分散主义,加强集中统一吗?如果离开充分发扬民主,这种集中,这种统一是真的还是假的,是实的还是空的?是正确的还是错误的?当然只能是假的、空的、错误的。

我们的集中制,是建立在民主基础上的集中制。无产阶级的集中,是在广泛民主基础上的集中。各级党委是执行集中领导的机关,但是,党委的领导,是集体的领导,不是第一书记个人独断。在党委会内部只应当实行民主集中制。第一书记同其他书记和委员之间的关系是少数服从多数。拿中央常委或者政治局来说,常常有这样的事情,我讲的话,不管是对的还是不对的,只要大家不赞成,我就得服从他们的意见,因为他们是多数。听说现在有一些省委、地委、县委,有这样的情况。一切事情,第一书记一个人说了就算数。这是很错误的。哪有一个人说了就算数的道理呢?我这是指大事,不是指有了决议后的日常工作。只要是大事,就得集体讨论,认真地听取不同的意见,认真地对于复杂的情况和不同的意见加以分析。要想到事情的几种可能性,估计情况的几个方面,好的和坏的,顺利的和困难的,可能办到的和不可能办到的。尽可能地慎重一些,周一些。如果不是这样,就是一人称霸。这样的第一书记,应当叫做霸王,不是民主集中制的“班长”。以前有个项羽,叫做西楚霸王,他就不爱听别人的不同意见。他那里有个范增,给他出过主意,可是项羽不听范增的话。另外一个人叫刘邦,就是汉高祖,他比较能够采纳不同意见。有个知识分子叫郦食其,去见刘邦。初一报,说是读书人,孔夫子一派的。回答说:现在军事时期,不见儒生。这个郦食其就发了火,他向管门房的人说:你给我滚进去报告,老子是高阳酒徒,不是儒生。管门房的人进去照样报告了一遍。好,请。请了进去,刘邦正在洗脚,连忙起来欢迎。郦食其因为刘邦不见儒生的事,心中还有火,批评了刘邦一顿。他说,你究竟要不要取天下,你为什么轻视长者!这时候,郦食其已经六十多岁了,刘邦比他年轻,所以他自称长者。刘邦一听,向他道歉,立即采纳了郦食其夺取陈留县的意见。此事见《史记》郦食其和朱建传。刘邦是在封建时代被历史家称为“豁达大度”“从谏如流”的英雄人物。刘邦同项羽打了好几年仗,结果刘邦胜了,项羽败了,不是偶然的。我们现在有一些第一书记,连封建时代的刘邦都不如,倒有点像项羽。这些同志不改,最后要垮台的。不是有一出戏叫《霸王别姬》吗?这些同志如果不改,难免有一天要“别姬”就是了。(笑声)我为什么讲得这样厉害呢?是想讲的挖苦一点,对于一些同志戳得痛一点,让这些同志好好地想一想,最好有两天睡不着觉。如果他们睡得着觉,我就不高兴,因为他们还没有被戳痛。

我们有些同志,听不得相反的意见,批评不得。这是很不对的。在我们这次会议中间,有一个省,会本来是开得生动活泼的,省委书记到那里一坐,鸦雀无声,大家不讲话了。这位省委书记同志,你坐到那里去干什么呢?为什么不坐到自己房子里想一想问题,让人家去纷纷议论呢?本来养成了这样一股风气,当着你的面不敢讲话,那末,你就应当回避一下。有了错误,一定要做自我批评,要让人家讲话,让人批评。去年六月十二号,在中央北京工作会议的最后一天,我讲了自己的缺点和错误。我说,请同志们传达到各省、各地方去。事后知道,许多地方没有传达。似乎我的错误就可以隐瞒,而且应当隐瞒。同志们,不能隐瞒。凡是中央犯的错误,直接的归我负责,间接的我也有份。因为我是中央主席,我不是要别人推卸责任,其他一些同志也有责任,但是第一个负责的应当是我。我们的省委书记、地委书记、县委书记直到区委书记、企业党委书记、公社党委书记,既然做了第…书记,对于工作中的缺点错误,就要担起责任。不负责任,怕负责任,不许人讲话,老虎屁股摸不得,凡是采取这种态度的人,十个就有十个要失败。人家总是要讲的,你老虎屁股真是摸不得吗?偏要摸。

在我们国家,如果不充分发扬人民民主和党内民主,不充分实行无产阶级的民主制,就不能有真正的无产阶级的集中制。没有高度的民主,不可能有高度的集中,而没有高度的集中,就不可能建立社会主义经济。我们的国家,如果不建立社会主义经济,那会是一种什么状态呢?就会变成南斯拉夫那样的国家,变成实际上是资产阶级的国家,无产阶级专政就会转化成资产阶级专政,而且会是反动的、法西斯式的专政。这是一个十分值得警惕的问题,希望同志们好好想一想。

没有民主集中制,无产阶级专政不可能巩固。在人民内部实行民主,对人民的敌人实行专政,这两个方面是分不开的,把这两个方面结合起来,就是无产阶级专政,或者叫人民民主专政。我们的口号是:“无产阶级领导的,以工农联盟为基础的人民民主专政。”无产阶级怎样实行领导呢?经过共产党来领导。共产党是无产阶级的先进部队。无产阶级团结一切赞成、拥护和参加社会主义革命和社会主义建设的阶级和阶层。对反动阶级,或者说,对反动阶级的残余实行专政。在我们国内,人剥削人的制度已经消灭,地主阶级和资产阶级的经济基础已经消灭,现在反动阶级已经没有过去那末厉害了,比如说,已经没有一九四九年人民共和过刚建立的时候那么厉害了,也没有一九五七年资产阶级右派猖狂进攻的时候那末厉害了,所以我们说是反动阶级的残余。但是对于这个残余,千万不可轻视,必须继续同他们做斗争,已经被推翻的阶级,还企图复辟。在社会主义社会,还会产生新的资产阶级分子。整个社会主义阶段,存在着阶级和阶级斗争,这种阶级斗争是长期的、复杂的、有时甚至是很激烈的。我们的专政工具不能削弱,还应当加强。我们的公安系统是掌握在正确的同志的手里的。也可能有个别地方的公安部门,是掌握在坏人手里。还有一些作公安工作的同志,不依靠群众,不依靠党,在肃反工作中不是执行在党委领导下通过群众肃反的路线,只依靠秘密工作,只依靠所谓专业工作。专业工作是需要的,对于反革命分子,侦察、审讯是完全必要的,但是,主要是实行党委领导下的群众路线。特别是对于整个反动阶级的专政,必须依靠群众,依靠党。对于反动阶级实行专政,这并不是说把一切反动阶级分子统统消灭掉,而是要改造他们,用适当的方法改造他们,使他们成为新人。没有广泛的人民民主,无产阶级专政不能巩固,政权会不稳。没有民主,没有把群众发动起来,没有群众的监督,就不可能对反动分子和坏分子实行有效的专政,也不可能对他们实行有效的改造,他们就会继续捣乱,还有复辟的可能,这个问题应当警惕,也希望同志们好好想一想。

第三点、我们应当联合那一些阶级?压迫那一些阶级?这是一个根本立场的问题。

工人阶级应当联合农民阶级,城市小资产阶级,爱国的民族资产阶级,首先要联合的是农民阶级。知识分子,例如科学家、工程技术人员、教授、作家、艺术家、演员、医务工作者、新闻工作者,他们不是一个阶级,他们或者附属于资产阶级,或者附属于无产阶级。对于知识分子,是不是只有革命的我们才去团结呢?不是的。只要他们爱国,我们就要团结他们,并且要让他们好好工作。工人、农民、城市小资产阶级分子、爱国的知识分子、爱国的资本家和其他爱国的人士,这些人占全入口的百分之九十五以上。这些人,在我们人民民主专政下面,都属于人民的范围。在人民的内部,要实行民主。人民民主专政要压迫的是地主、富农、反革命分子、坏分子和反共的右派分子。反革命分子、坏分子和反共的右派分子,他们代表的阶级是地主阶级和反动的资产阶级。这些阶级和坏人,大约占全人口的百分之四、五。这些人是我们要强迫改造的。他们是人民民主专政的对象。

我们站在哪一边?站在占全人口百分之九十五以上的人民群众一边?还是站在占全人口百分之四、五的地、富、反、坏、右一边呢?必须站在人民群众这一边,绝不能站到人民敌人那一边去。这是一个马克思列宁主义者的根本立场问题。

在国内是如此,在国际范围内也是如此,各国的人民,占人口总数的百分之九十以上的人民大众,总是要革命的,总是会拥护马克思列宁主义的。他们不会拥护修正主义,有些人暂时拥护,将来终究会抛弃它。他们总会逐步地觉醒起来,总会反对帝国主义和各国反动派,总会反对修正主义。一个真正的马克思列宁主义者,必须坚定地站在占世界人口百分之九十以上的人民大众这一边。

第四点、关于认识客现世界的问题

人对客观世界的认识,由必然王国到自由王国的飞跃,要有一个过程,例如对于在中国如何进行民主革命的问题,从一九二一年党的建立直到一九四五年党的第七次代表大会,一共二十四年,我们全党的认识才完全统一起来。中间经过一次全党范围的整风,从一九四二年春天到一九四五年夏天,有三年半的时间。那是一次细致的整风。采用的方法是民主的方法:就是说,不管什么人犯了错误,只要认识了,改正了,就好了,而且大家帮助他认识,帮助他改正,叫做“惩前毖后,治病救人”,“从团结的愿望出发,经过批评或者斗争,分清是非,在新的基础上达到新的团结”。“团结——批评——团结”,这个公式就是在那个时候产生的。那次整风帮助全党同志,统一了认识。对于当时的民主革命应当怎么办,党的总路线和各项具体政策应当怎样定,这些问题,都是在那个时期,特别是在整风之后,才得到完全解决。

从党的建立到抗日时期,中间有北伐战争和十年土地革命战争。我们经过了两次胜利,两次失败。北伐战争胜利了,但是到一九二七年,革命遭到了失败。土地革命战争曾经取得了很大的胜利,红军发展到三十万人,后来又遭到挫折,经过长征,这三十万人缩小到两万多人,到陕北以后补充了一点,还是不到三万人,就是说,不到三十万人的十分之一。究竟是那三十万人的军队强些,还是这不到三万人的军队强些?我们受了那样大的挫折,吃过那样大的苦头,就得到锻炼,有了经验,纠正了错误路线,恢复了正确路线,所以这不到三万人的军队,比起过去那个三十万人的军队来,要更强些。×××同志在报告里说,最近四年,我们的路线是正确的,成绩是主要的,我们在实际工作中犯过一些错误,吃了苦头,有了经验了,因此我们更强了,而不是更弱了。情况正是这样。在民主革命时期,经过胜利,失败,再胜利,再失败,两次比较,我们才认识了中国这个客观世界。在抗日战争前夜和抗日战争时期,我写了一些论文,例如《中国革命战争的战略问题》,《论持久战》,《新民主主义论》,《共产党人发刊词》,替中央起草过一些关于政策、策略的文件,都是革命经验的总结。那些论文和文件,只有在那个时候才能产生,在以前不可能,因为没有经过大风大浪,没有两次胜利和两次失败的比较,还没有充分的经验,还不能充分认识中国革命的规律。

中国这个客观世界,整个地说来,是由中国认识的,不是在共产国际管中国问题的同志们认识的。共产国际的这些同志就不了解或者说不很了解中国社会、中国民族。对于中国这个客观世界,我们自己在很长时间内都认识不清楚,何况外国同志呢?

在抗日时期,我们才制定了合乎情况的党的总路线和一整套具体政策。这时候,我们已经干了二十来年的革命,过去那么多年的革命工作,是带着很大的盲目性的。如果有人说,有那一位同志,比如前中央的任何同志,比如说我自己,对于中国革命的规律,在一开始的时候就完全认识了,那是吹牛,我们切记不要相信,没有那同事。过去,特别是开始时期,我们只是一股劲儿要革命,至于怎么革法,革些什么,那些先革,那些后革,那些要到下一阶段才革,在一个相当长的时间内,都没有弄清楚,或者说没有完全弄清楚。我讲我们中国共产党人在民主革命时期艰难地但是成功地认识中国革命规律这一段历史情况的目的,是想引导同志们理解这样一件事:对建设社会主义的规律的认识,必须有一个过程。必须从实践出发,从没有经验到有经验,从有较少的经验,到有较多的经验,从建设社会主义这个未被认识的必然王国,到逐步地克服盲目性,认识客观规律,从而获得自由,在认识上出现一个飞跃,到达自由王国。

对于社会主义建设我们还缺乏经验。我和好几个国家的兄弟党的代表团谈过这个问题,我说:对建设社会主义经济,我们没有经验。这个问题我也向一些资本主义国家的新闻记者谈过,其中有一个美国人叫斯诺,他老要来中国,一九六零年让他来了。我同他谈过一次话,我说:“你知道,对于政治、军事,对于阶级底级斗争,我们有一套经验,有一套方针、政策和办法,至于社会主义建设,过去没有干过,还没有经验。你会说,不是已经干十一年了吗?是干了十一年了,可是还缺乏知识,还缺乏经验,就算开始有了一点,也还不多。”斯诺要我讲讲中国建设的长期计划。我说:“不晓得。”他说“你讲话太谨慎。”我说:“不是什么谨慎不谨慎,我就是不晓得事呀!就是没有经验呀。”同志们,也真是不晓得,我们确实还缺乏经验,确实还没有这样一个长期计划。一九六○年,那正是我们碰了许多钉子的时候。一九六一年,我同蒙哥马利谈话,也说到上面那些意见。他说:“再过五十年,你们就了不起了。”他的意思是说,过了五十年,我们就会壮大起来,而且会“侵略”人家,五十年内还不会,他的这种看法,一九六零年他来中国的时候就对我说过。我说:“我们是马克思列宁主义者,我们的国家是社会主义国家,不是资本主义国家,因此,一百年,一万年,我们也不会侵略别人。至于建设强大的社会主义经济,在中国,五十年不行,会要一百年,或者更多的时间。在你们国家,资本主义的发展,经过了好几百年。十六世纪不算,那还是中世纪。从十七世纪到现在,已经有三百六十多年。在我国,要建设起强大的社会主义经济,我估计要花一百多年。”十七世纪是什么时代呢?那是中国的明朝末年和清朝初年。再过一世纪,到十八世纪的上半期,就是清朝乾隆时代,《红楼梦》的作者曹雪芹就生活在那个时代,就是产生贾宝玉这种不满意封建制度的小说人物的时代。乾隆时代,中国已经有了一些资本主义生产关系的萌芽,但是还是封建社会。这就是出现大观园里那一群小说人物的社会背景。在那个时候以前,在十七世纪,欧洲的一些国家已经在发展资本主义了,经过三百多年,资本主义的生产力有了现在这样子。社会主义和资本主义比较,有许多优越性,我们国家经济的发展,会比资本主义国家快得多。可是,中国的人口多,底子薄,经济落后,要使生产力很大地发展起来,要赶上和超过世界上最先进的资本主义国家,没有一百多年的时间,我看是不行的。也许只要几十年,例如有些人所设想的五十年,就能做到。果然这样,谢天谢地,岂不甚好。但是我劝同志们宁肯把困难想得多一点,因而把时间设想得长一点。三百几十年建设了强大的资本主义经济,在我国,五十年内外到一百年内外,建设起强大的社会主义经济,那又有什么不好呢?从现在起,五十年内外到一百年内外,是世界上社会制度彻底变化的伟大时代,是一个翻天覆地的时代,是过去任何一个历史时代都不能比拟的。处在这样一个时代,我们必须准备进行同过去时代的斗争形式有着许多不同特点的伟大斗争。为了这个事业,我们必须把马克思列宁主义的普遍真理同中国社会主义建设的具体实际,并同今后世界革命的具体实际,尽可能好一些地结合起来,从实践中一步一步地认识斗争的客观规律。要准备着由盲目性遭到许多的失败和挫折,从而取得经验,取得最后胜利。由这点出发,把时间设想得长一点,是有许多好处的,设想得短了反而有害。

在社会主义建设上,我们还有很大的盲目性。社会主义经济,对于我们来说,还有许多未被认识的必然王国。拿我来说,经济建设工作中的许多问题还不懂得。工业、商业,我就不大懂。对于农业我懂得一点。但是也是比较地懂得,还是懂得不多。要较多地懂得农业,还要懂得土壤学、植物学、作物栽培学、农业化学、农业机械等等,还要懂得农业内部各个专业部门,例如粮、棉、油、麻、丝、茶、糖、菜、烟、果、药、杂等等,还有畜牧业,还有林业。我是相信苏联威廉氏土壤学的,在威廉氏的土壤学著作里,主张农、林、牧三结合。我认为必须要有这种三结合,否则对于农业不利。所有这些农业生产问题,我劝同志们,在工作之暇,认真研究一下,我也还想研究一点。但是到现在为止,这些方面,我的知识很少。我注意的较多的是制度方面的问题,生产关系方面的问题,至于生产方面,我们知识很少。社会主义建设,从我们全党来说,知识都非常不够。我们应当在今后一段时间内,积累经验,努力学习,在实践中间逐步地加深对它的认识,弄清楚它的规律,一定要下一番苦功,要切切实实地去调查它,研究它。要下去蹲点,到生产大队、生产队,到工厂,到商店,去蹲点。调查研究,我们从前做得比较好,可是进城以后,不认真做了,一九六一年我们又重新提倡,现在情况已经有所改变。但是,在领导干部中间,特别是在高级领导干部中间,有一些地方部门和企业,至今还没有形成风气。有一些省委书记,到现在还没有下去蹲过点,如果省委书记不去,怎么能叫地委书记、县委书记下去蹲点呢,这个现象不好,必须改变过来。

从中华人民共和国成立到现在已经十二年了。这十二年分前八年和后四年,一九五零年到一九五七年底是前八年。一九五八年到现在,是后四年,我们这次会议已经初步总结了过去工作的经验,主要是后四年的经验。这个总结.反映在×××同志的报告里面。我们已经制定或者正在制定,或者将要制定各个方面的具体政策。已经制定了的,例如农村工作六十条,工业企业七十条,高等教育六十条,科学研究工作十四条,这些条例草案已经在实行或者试行,以后还要修改,有些还可能大改。正在制定的,例如商业工作条例。将要制定的,例如中小学教育条例。我们的党政机关和群众团体的工作,也应当制定一些条例。军队已经制定了一些条例。总之,工、农、商、学、兵、政、党这七个方面的工作,都应当好好地总结经验,制定一整套的方针、政策和方法,使它们在正确的轨道上前进。

有了总路线还不够,还必须在总路线指导下,在工、农、商、学、兵、政、党各个方面,有一整套适合情况的具体的方针、政策和办法,才有可能说服群众和干部,并且把这些当作教材去教育他们,使他们有一个统一的认识和统一的行动,然后才有可能取得革命事业和建设事业的胜利,否则是不可能的。对于这一点,我们在抗日时期就有了深刻的认识。在那时候,我们这样做了,就使得干部和群众对于民主革命时期的一整套具体的方针、政策和办法,有了统一的认识,因而有了统一的行动,使当时的民主革命事业取得了胜利,这是大家知道的。在社会主义革命和社会主义建设的时期,头几年内,我们的革命任务,在农村是完成对封建主义的土地制度的改革和接着实现农业合作化,在城市是实现对资本主义商业的社会主义改造。在经济建设方面,那时候的任务是恢复经济和实现第一个五年计划。不论在革命方面和建设方面,那时候都有一条适合客观情况的,有充分说服力的总路线,以及在总路线指导下的一整套方针,政策和办法,因此教育了干部和群众,统一了他们的认识,工作也就比较做得好。这也是大家知道的。但是,那时候有这样一种情况,因为我们没有经验,在经济建设方面,我们只是照抄苏联,特别是在重工业方面,几乎一切都抄苏联。自己的创造性很少。这在当时是完全必要的,同时又是一个缺点,缺乏创造性,缺乏独立自主的能力。这当然不应当是长久之计。从一九五八年起,我们就确立了自力更生为主,争取外援为辅的方针。在一九五八年党的八大二次会议上,通过了“鼓足干劲,力争上游,多快好省地建设社会主义”的总路线,在那一年又办起了人民公社,提出了大跃进的口号。在提出社会主义建设总路线的一个相当时期内,我们还没有来得及,也没有可能规定一整套适合情况的具体的方针、政策和办法,因为经验还不足。在这种情况下,干部和群众,还得不到一整套的教材,得不到系统的政策教育,也就不可能真正有统一的认识和统一的行动。要经过一段时间,碰到一些钉子,有了正、反两方面的经验,才有这样的可能,现在好了,有了这些东西了,或者正在制定这些东西。这样,我们就可以更加妥善地进行社会主义革命和社会主义建设。在总路线指导之下,制定一整套具体的方针、政策和办法,必须通过从群众中来的方法,通过系统的、周密的调查研究的方法,对工作中的成功经验和失败经验,作历史的考察,才能找出客现事物所固有的而不是人们主观臆造的规律,才能制定适合情况的各种条例。这种事很重要,请同志们注意到这点。

工、农、商、学、兵、政、党,这七个方面,党是领导一切的。党要领导工业、农业、商业、文化教育、军队和政府。我们的党,一般说来是很好的,我们党员的成分,主要的是工人和贫苦农民,我们的绝大多数干部都是好的,他们都在辛辛苦苦地工作。但是,也要看到,我们党内还存在一些问题,不要想象我们党的情况什么都好,我们现在有一千七百多万党员,这里面差不多有百分之八十的人是建国以后入党的,五十年代入党的。建国以前入党的只占百分之二十。在这百分之二十的人里面,一九三零年以前入党的,二十年代入党的,据前八年计算,有八百多人,这两年死了一些,恐怕只有七百多人了。不论在老的和新的党员里面,特别是在新党员里面,都有一些品质不纯和作风不纯的人。他们是个人主义者,官僚主义者,主观主义者,甚至是变了质的分子。还有些人挂着共产党员的招牌,但是并不代表工人阶级,而是代表资产阶级。党内并不纯洁,这一点必须看到,否则我们是要吃亏的。

上面是我讲的第四点。就是讲。我们对于客观世界的认识要有一个过程。先是不认识,或者不完全认识,经过反复的实践,在实践里面得到成绩,有了胜利,又翻过筋斗,碰了钉子,有了成功和失败的比较,然后才有可能发展成为完全的认识或者比较完全的认识。在那个时候,我们就比较主动了,比较自由了,就变成比较聪明一些的人了。自由是对必然的辩证规律。所谓必然,就是客观存在的规律性。在没有认识它以前,我们的行为总是不自觉的,带有盲目性的。这时候,我们是一些蠢人。最近几年,我们不是干过许多蠢事吗!第五点,关于国际共产主义运动这个问题,我只简单地讲几句。

不论在中国,在世界各国,总而言之,百分之九十以上的人终究是会拥护马克思列宁主义的。在世界上,现在还有许多人,在社会民主党的欺骗之下,在修正主义的欺骗之下,在帝国主义的欺骗之下,在各国反动派的欺骗之下,他们还不觉悟。但是他们总会逐步地觉悟过来,总会拥护马克思列宁主义。马克思列宁主义这个真理,是不可抗拒的,人民群众是要革命的。世界革命总是要胜利的。不准革命,像鲁迅所写的赵太爷,钱太爷,假洋鬼子不准阿Q革命那样,总是要失败的。

苏联是第一个社会主义国家,苏联共产党是列宁创造的党。虽然苏联的党和国家的领导现在被修正主义篡夺了,但是,我们劝同志们坚决相信,苏联广大的人民,广大的党员和干部是好的,是革命的,修正主义的统治是不会长久的。无论什么时候,现在,将来,我们这一辈子,我们的子孙,都要向苏联学习,学习苏联的经验。不学习苏联要犯错误。人们会问:苏联被修正主义统治了,还要学吗?我们学习的是苏联的好人好事,苏联党的好经验.至于苏联的坏人坏事,苏联的修正主义者,我们应当看作反面教员,从他们那里吸取教训。

我们永远要坚持无产阶级的国际主义团结的原则,我们始终主张社会主义和世界共产主义运动一定要在马克思列宁主义的基础上巩固地团结起来。

国际修正主义者在不断地骂我们。我们的态度是:由他骂去。在必要的时候,给予适当的回答。我们这个党是被人家骂惯了的。从前骂的不说,现在呢,在国外,帝国主义者骂我们,反动的民族主义者骂我们,修正主义者骂我们。在国内蒋介石骂我们,地、富、反、坏、右骂我们。历来就是这么骂的。……我们是不是孤立的呢?我就不感觉孤立。我们在座的有七千多人,七千多人还孤立吗?世界各国人民群众已经或者将要同我们站到一起,我们会是孤立的吗?

最后一点,第六点,要团结全党和全体人民。

这个问题我只简单地讲几句。

要把党内、党外的先进分子,积极分子团结起来,把中间分子团结起来,去带动落后分子,这样就可以使全党、全民团结起来。只有依靠这些团结,我们才能够做好工作,克服困难,把中国建设好。要团结全党、全民,这并不是说我们没有倾向性。有些人说共产党是“全民的党”,我们不这样看。我们的党是无产阶级政党,是无产阶级的先进部队,是用马克思列宁主义武装起来的战斗部队。我们是站在占总人口百分之九十五以上的人民大众一边,绝不站在占总人口百分之四、五的地、富、反、坏、右那一边。在国际范围也是这样,我们是同一切马克思列宁主义者,一切革命人民、全体人民讲团结的,绝不同反共反人民的帝国主义者和各国反动派讲什么团结。只要有可能,我们也要同这些人建立外交关系,争取在五项原则的基础上和平共处。但是这些事,跟我们和各国人民的团结是不同范畴的两同事。

要使全党全民团结起来,就必须发扬民主,让人讲话。在党内是这样,在党外也是这样。省委的同志、地委的同志、县委的同志,你们回去,一定要让人讲话。在座的同志们要这样做,不在座的同志们也要这样做。一切党的领导人员都要发扬民主,让人讲话。界限是什么呢?一个是遵守党的纪律,少数服从多数,全党服从中央。另一个是,不准组织秘密集团。我们不怕公开反对派,只怕秘密的反对派,这种人当面不讲真话,当面讲的尽是些假的,骗人的话,真正的目的不讲出来。只要不是违犯纪律的。只要不是搞秘密集团活动的,我们都允许他讲话,而且讲错了也不要处罚,讲错了话可以批评,但要用道理说服人家。说而不服怎么办:让他保留意见。只要服从决议,服从多数人决定的东西,少数人可以保留不同意见。在党内、党外,允许少数人保留意见,是有好处的,错误的意见,让他暂时保留,将来他会改的。许多时候,少数人的意见倒是正确的。历史上常常有这样的事实,起初,真理不是在多数人手里,而是在少数人手里。马克思、恩格斯手里有真理,可是他们在开始的时候是少数。列宁在很长一个时期内也是少数。我们党内也有这样的经验,在陈独秀统治的时候,在“左”倾路线统治的时候,真理都不在领导机关的多数人手里,而是在少数人手里。历史上的自然科学家,例如:哥白尼、伽利略、达尔文,他们的学说曾经在一个长时间内不被多数人承认,反而被看作错误的东西,当时,他们是少数。我们党在1921年成立的时候,只有几十个党员,也是少数人。可是这几十个人代表了真理,代表了中国的命运。

有一个捕人、杀人的问题,我还想讲一下。在现在的时候,在革命胜利还只有十几年的时候。在被打倒了的反动阶级还没有被改造好,有些人并且企图阴谋复辟的时候,人总会要捕一点,杀一点的,否则不能平民愤,不能巩固人民的专政。但是,不要轻于捕人,尤其不要轻于杀人。有一些坏人,钻到我们队伍里面的坏分子,蜕化变质分子,这些人,骑在人民头上拉屎拉尿,穷凶极恶,严重地违法乱纪,这是些小蒋介石。对于这种人得有个处理,罪大恶极的,也要捕一些,还要杀几个。因为对这样的人,完全不捕、不杀,不足以平民愤。这就是所谓的“不可不捕,不可不杀。”但是绝不可多捕、多杀。凡是可捕可不捕的,可杀可不杀的,都要坚决不捕,不杀。有个潘汉年,此人当过上海市副市长,过去秘密投降了国民党,是个CC派人物,现在关在班房里头,我们没有杀他。像潘汉年这样的人,只要杀一个,杀戒一开,类似的人都得杀。还有个王实味,是个暗藏的国民党探子。在延安的时候,他写过一篇文章,题名《野百合花》,攻击革命,诬蔑共产党。后来把他抓起来,杀掉了。那是保安机关在行军中间,自己杀的,不是中央的决定。对于这件事,我们总是提出批评,认为不应当杀。他当特务,写文章骂我们,又死不肯改,就把他放在那里吧,让他劳动去吧,杀了不好。人要少捕,少杀。动不动就捕人、杀人,会弄得人人自危,不敢讲话。在这种风气下面,就不会有多少民主。

还不要给人乱戴帽子。我们有些同志惯于拿帽子压人,一张口就是帽子满天飞,吓得人不敢讲话。当然,帽子总是有的,×××同志的报告里面不是就有许多帽子吗?“分散主义”不是帽子吗?但是不要动不动就给人戴在头上,弄得张三分散主义,李四分散主义,什么人都是分散主义。帽子最好由人家自己戴,而且要戴得合适,最好不要由别人去戴。他自己戴了几回,大家不同意他戴了,那就取消了。这样,就会有很好的民主空气。我们提倡不抓辫子,不戴帽子,不打棍子,目的就是要使人心里不怕,敢于讲意见。

对于犯了错误的人,对于那些不让别人讲话的人,要采取善意帮助的态度。不要有这样的空气,似乎犯不得错误,一犯错误,从此不得翻身。一个人犯了错误,只要他真心愿意搞正,只要他确实有了自我批评,我们就要表示欢迎。头一、二次自我批评,我们不要要求过高,检查得还不彻底,不彻底也可以,让他再想一想,善意地帮助他。人是要有人帮助的。应当帮助那些犯错误的同志认识错误。如果人家诚恳地作了自我批评,愿意改正错误,我们就要宽恕他,对他采取宽大的政策。只要他的工作成绩还是主要的,能力也还行,就还可以让他在那里继续工作。

我在这个讲话里批评了一些现象,批评了一些同志,但是没有指名道姓,没有指出张三、李四来。你们自己心里有数。(笑声)我们这几年工作中的缺点、错误,第一笔账,首先是中央负责,中央又是我首先负责;第二笔账,是省委、市委、自治区党委的;第三笔账,是地委一级的;第四笔账,是县委一级的,第五笔账,就算到企业党委,公社党委的了。总之,各有各的账。

同志们,你们回去,一定要把民主集中制健全起来。县委的同志,要领导公社把民主集中制健全起来。首先要建立和加强集体领导,不要再实行长期固定的“分片包干”的领导方法了,那个方法,党委书记和委员们各搞各的,不能真正的集体讨论,不能有真正的集体领导,要发扬民主,要启发人家批评,要听人家的批评。自己要经得起批评。应当争取主动,首先作自我批评。有什么就检讨什么,一个钟头,顶多两个钟头,倾箱倒筐而出,无非是那么多。如果人家认为不够,请他提出来,如果说得对,我就接受。让人讲话,是采取主动好,还是被动好?当然是主动好。已经处在被动地位了怎么办?过去不民主,现在陷入被动,那也不要紧,就请大家批评吧。白天出气,晚上不看戏,白天晚上都请你们批评。(掌声)这个时候,我坐下来,冷静地想一想,两三天晚上睡不着党,想好了,想通了,然后诚诚恳恳地作一篇检查。这不就好了吗?总之,让人讲话,天不会塌下来,自己也不会垮台。不让人家讲话呢?那就难免有一天要垮台。

我今天的讲话就讲这一些。中心是讲了一个实行民主集中制的问题,在党内、党外发扬民主的问题。我向同志们建议,仔细考虑一下这个问题。有些同志还没有民主集中制的思想,现在要开始建立这个思想,开始认识这个问题。我们充分地发扬了民主,就能把党内党外广大群众的积极性调动起来,就能使占总人口百分之九十五以上的人民大众团结起来。做到了这些,我们的工作就会越做越好,我们遇到的困难就会较快地得到克服,我们事业的发展就会顺利得多。(热烈鼓掌)



\section[接见几内亚政府经济代表团和妇女代表团的谈话(一九六二年五月三日)]{接见几内亚政府经济代表团和妇女代表团的谈话}
\datesubtitle{(一九六二年五月三日)}


主席:你们是来自友好国家、友好政府的代表团,欢迎你们。所有非洲的朋友,都受到中国人民的欢迎。我们与所有非洲国家人民的关系都是好的,不管是独立或没有独立正在斗争中的人民。非洲正出现一个很大的争取民族独立,反对帝国主义、反对殖民主义的革命运动。非洲有多少人口?二亿吧?二亿人民要翻身,不管已经站起来或者将要站起来。还有拉丁美洲也是二亿人口,亚洲的十几亿人口和全世界的革命人民。我们不是孤立的,到处都有我们的朋友,你们也不是孤立的。你们来中国可以感到中国人民是十分欢迎你们的。来了几天了?

凯塔(几政府经济代表团团长)。我们是四月十九日末的。

主席。她们呢?(指妇女代表团)

凯塔:她们是四月二十五日来的。

主席;听说你们明天要走了。

凯塔。她们不走。

主席:欢迎。他(指柯庆施同志)是上海的主人,柯庆施是中共中央政治局委员,同你们是民主党的中央政治局委员一样,对吗?

柯庆施:你们为何不多住几天?

凯塔。我们的日程排得很紧,国内工作很多,五月十五日以前要完成改组党的各级机构,如有时间,我们很愿意在中国访问一个月。

主席:你们的觉是很好的党,是个联系群众的党,有纪律的党,是一个有以反对帝国主义、反对殖民主义和建立民族经济作为纲领的党,一个独立自主国家的领导的党。我们感到同你们是很接近的。我们两国、两党互相帮助,互相支持,你们不捣我们的鬼,我们也不捣你们的鬼。如果我们有人在你们那里做坏事,你们就对我们讲。例如看不起你们,自高自大,表现大国沙文主义态度,有没有这种人?

凯塔:没有。

主席:如有这种人,我们要处分他们。

凯塔:有些国家的技术人员有这种情况,中国专家没有这种情况,他们都工作得很好。

主席:是不是有比你们几内亚专家薪水高、特殊化的情况?(对叶××说)恐怕有,要检查,待遇要一样,最好低一些。(叶××:周总理正在要方×同志检查。)

凯塔:这个问题是值得研究的,但直到现在中国专家并没有过分的要求,有些国家的专家要比几专家高二、三倍,相反,中国的专家没有过分的要求。

主席:驻几内亚大使是谁?是柯华吗?(旁人答是的)

凯塔:只有中国专家和越南专家待遇一样。

主席:是否有人损害你们的民族利益?搞颠覆活动?

凯塔:有,但不是中国人。对搞颠覆活动的人,我们也不是听任他们去搞的,发生这种情况,我们要迅速采取措施加以回击,我们不愿意做人家的尾巴。过去发生的事件你们是知道的,我们对这些事件的态度你们也是知道的。

主席:你们做得对。凡有人在你们那里称王称霸,不服从你们的法律,搞颠覆活动,应把他们赶掉。我们希望你们站住脚,不仅在政治上,而且要在经济上站住脚,不要被人颠覆掉了。你们站住脚我们高兴。你们倒台我们不高兴。因为你们是一个革命的党,是一个革命的政府,在非洲有很大的影响。经过你们,可以在非洲许多国家做工作,使他们得到解放。你们也有这个责任,不要自己独立就不管别人了。我们也一样,不能因为自己独立了就不管别人了。所谓管别人是指友好的支持、帮忙。你们知道我们现在还有些困难,帮忙不大。再过五年、十年我们的情况可能好一些,那时的帮助可能多一些。我们的国家,有一个很大的缺点,人太多,这么多人要吃饭,要穿衣,所以现在还有不少困难,但这些困难不是不可克服的,而是能够克服的,正在采取措施克服。我国的经济、文化与你们差不多,差不多是在没有什么遗产的情况下搞起来的。你们是法国的殖民地,我们是几个国家的殖民地。你们与法国建立了外交关系吗?

凯塔:关于跟法国建交的问题,还有一些悬而未决的问题至今没有解决。我们独立后与法国双方互派过代表,进行过谈判,想解决这些问题,但这些问题至今还未解决,我们希望能够解决。

主席:你们与阿尔及利亚的关系好吗?

凯塔:好的。

主席:与马里呢?

凯塔:非常好。

主席:与索马里呢?

凯塔:差一些,还没有外交关系,往来较少。

主席:与加纳呢?

凯塔:好的。

主席:与摩洛哥和突尼斯呢?

凯塔:跟摩洛哥和突尼斯也好,但有些不同。非洲有两个不同的集团,即卡萨布兰卡集团和蒙罗维亚集团。

主席:蒙罗维亚集闭?

凯塔:卡萨布兰卡集团是阿尔及利亚、加纳、几内亚、马里、利比里亚和摩洛哥六国集团。蒙罗维亚集团是过去非洲马尔加什联盟的国家。卡集团较进步,蒙集团不大进步,与殖民主义联系较多。

主席:蒙集团是否属于法属共同体?

凯塔:是的。我们与卡萨布兰卡集团关系好…些。跟蒙罗维亚集团的有些国家,如塞内加尔、利比里亚、象牙海岸等国,边境相连,遭遇和问题都差不多。我们认为非洲分为这样两个集团并不符合非洲人民的利益,所以杜尔总统向所有非洲国家采取外交措施,创议召开非洲国家首脑会议,五月份要开会,要协调相互的立场,取得一些共同点,取得合作。正如主席所说,进步力量应该支持邻国人民,非洲许多国家与殖民主义势力有联系,与欧洲共同市场有联系,而不是与兄弟的邻国有联系。我们预备开会讨论非洲共同市场问题,以便发展非洲自己的经济,摆脱非洲殖民主义势力的控制。非洲所有国家首脑会议五月份在埃塞俄比亚首都亚的斯亚贝巴召开。如果几内亚创议的非洲首脑会议有结果,可以使非洲国家的关系进一步密切。

主席:非洲国家要联合,另外还要一个更大的联合,即亚、非、拉美三大洲的联合。

凯塔:我们也意识到这种大联合的重要性,因此首先非洲国家自己要联合,以便在大联合中起积极作用。我们不少非洲国家正在受痛苦,还在受殖民主义的痛苦,特别是经济上受新殖民主义的痛苦。要实现大联合,以反对帝国主义和殖民主义。

主席:几内亚有多少人口?

凯塔:几内亚国家很小,有四百万人口。

主席:土地面积多少?

凯塔:二十五万平方公里,平均每平方公里十二――十三人。

主席:很大的土地,有很大的发展前途。有森林吗?

凯塔:很多,特别是矿产的前途很大,我国有丰富的铁矿、铝矿、铬矿。是非洲矿产最丰富的国家,还有丰富的水力资源,可以利用来发电,以便在当地自己提炼矿砂。

主席:听说你们在建造一座大水坝?

凯塔:在法国殖民统治时期,法国人已有此计划。法国想在孔库雷河上建造一座大水坝,每年可发六十亿度的电,用此电力提炼铝。法国组织了国际公司,并与国际银行建立了关系,以便取得资金。一九五八年几内亚独立了,法国认为不安全,就放弃了这个计划。独立后,几内亚政府想搞,苏联原则上同意与东欧几个社会主义国家一起援助几建设水坝,但现在苏、几政治关系复杂化了,恐怕不准备搞了。

主席:还没有搞吗?

凯塔:没有搞,杜尔总统访华时经过莫斯科,苏原则同意援助,但至今没有动静。

主席:听说有个货币问题,解决了没有?

凯塔:我们在一九六零年建立了几内亚法郎,目的是退出法郎区,建立独立的货币区。这种

凯塔:法郎不能兑换外币,以避免殖民主义者掌握大量几内亚货币兴风作浪。但有些邻国以几法郎投机,他们从几带出大量货币,换美元和英镑,或以低于官价出售几法郎,压低币值,或贩运货物到几内亚来换取几币搞投机。所以几政府在今年四月决定取消旧币换新币,在外国的旧币一律作废。

主席:你们自己能印钞票吗?

凯塔:不能。

主席:在哪里印呢?

凯塔:起初在捷克,最近一次在英国印。现在正设法自己弄到印钞票的机器,以便保证不断地印自己的钞票。

主席:几内亚妇女有选举权吗?

卡玛拉(几妇女代表团团长):有的,在党内、政府内都有。

主席。党里有妇女领导人吗?

卡玛拉:党的街道委员会,村委员会和省委员会都有妇女领导人,恩廸阿依、贡代二位都得到了独立勋章。

主席。你们的革命是群众性的,党也是群众性的。我们的党中央员会女的太少了,女的有,但比例是男的多,女的少,地方党委也是如此。你们走在我们前面去了。

凯塔:我们的比例也少,十七个政治局委员中只有二个女的,政府中只有一个部长是女的,就是卡玛拉夫人。我们那里也仅仅是开始,正如周恩来总理所说,妇女受到双重压迫,不仅有帝国主义、殖民主义和封建主义的压迫,而且男女不平等,女的上学机会不多,革命胜利后男的还有封建思想,女的积极斗争,现在有女的市长、村长等。

主席;慢慢来。

凯塔:她(指卡玛拉夫人)想一下子解决问题。

卡玛拉;他想阻拦。(全场笑)

主席:我们与你们的情况差不多,比较接近,所以我们同你们谈得来,没有感到我欺侮你,你欺侮我,没有什么优越感,都是有色人种。有人想欺侮我们,认为我们生来就不行,认为我们没有办法,命运注定了,一万年该受帝国主义的压迫,不会管理国家,不会搞工业,不能解决吃饭问题,科学文化也不行。他们不想一想,这种状况是他们造成的,经济、文化水平低是他们造成的。管理国家过去是他们代替我们管理的。本国人讲管理是可以的,但要学,学多少年,慢慢来,可是你们不是慢慢来,而是一下子就取得政权,我们也是,夺取了政权再学嘛!不会管理慢慢就会管理了,有错误就改嘛!难道只有我们有错误,西方国家没有错误?他们的错误比我们更大,他们犯了反革命的错误。我们根本上没有错误,我们是革命。没有工业可以逐步搞工业,没有现代化的农业可以逐步搞现代化的农业,科学文化水平也能一年一年提高,例如地质人员,你们现在开始有了吗?

凯塔:以前的地质人员都是别国的,现在有许多几留学生在别的国家培养。

主席:我们国民党、蒋介石遗留下来的地质人员只有二百人,现在十三年来有了十几、二十万人。(问柯庆施同志:各省都有吗?柯说有。)能搞起来的。难道只有西方国家能搞,我们就不能搞起来吗?

凯塔:杜尔总统也认为争取独立首先要自己相信自己,管理国家也是一样。只有打铁,才能成为铁匠,只有学了才会,管理国家慢慢能学会。如一九五八年独立时,几学生只有三万二千人(指在学校的),现在有了十二万。

主席:增加了三倍多。你们过去可能没有大学。

凯塔:没有,很少有机会上大学,上大学要到巴黎去。当时殖民主义者对培养当地的干部没有兴趣,只有二百个大学生,现在有一万五千个大学生,各省都有。

主席:比例不小,四百万人口中有一万五千人。

凯塔:现在科纳克里正在搞技术高等学校,培养地质、农业等技术人材。一方面继续向外国派,另一方面尽量在国内培养高等学校的学生,使能适合本国的条件。

主席:今晚你们有何活动?柯庆施:晚上我们要举行宴会欢迎他们。

主席:他(指柯)是主人了,我们这就告一段落,好吗?

凯塔;我们没有别的话。在北京已与许多负责同志讲过,现再一次向主席表示:几内亚对于中国的友谊和合作寄以极大的希望,对中国为几所作的一切表示感谢。这次谈判印象很深,中国政府领导人很谅解我们,谈判进行得顺利,得到了积极的成果。再一次表示两国关系是巩固的,代表几政府和人民向主席表示感谢。

主席:我们感谢你们,这是互相支持,我们很抱歉,不能完全满足你们的要求。

凯塔:你们已尽了你们的能力。非洲有句话:援助的方式比援助的东西更重要。

主席:我们的关系是平等的,友好、坦率、诚恳、不讲假话,讲老实话。以后继续往来。你们来过中国吗?

凯塔:他们都是第一次,我一九六零年陪杜尔总统一起来过,到过北京、武汉、广州、上海等地。主席在北京接见过我们。到上海时柯市长举行了宴会,我们还参观过上海汽轮机厂,宋庆龄副主席在上海接见了我们。

主席:你们回去后替我问候塞古.杜尔总统,问候你们中央的各位领导人,祝他们好。

凯塔:在我们离开这儿前,再一次祝主席身体健康,祝主席在建设社会主义的事业中取得新的更大成就。



\section[在北戴河中央工作会议上的讲话(一九六二年八月六日)]{在北戴河中央工作会议上的讲话}
\datesubtitle{(一九六二年八月六日)}


先开工作会议,为中央全会准备文件。从明天起,开始讨论,刘××建议成立核心小组,还有许多小组,解决六个大组不能畅所欲言的问题。核心小组有常委、书记处,再加大区第一书记,中央各口负责同志,共二十三人:毛、刘××、周、朱×、邓××、彭×、富春、先念、谭××、伯达、陆××、富治、谷牧、罗××、陈毅、杨××,加上各大区第一书记。

一、社会主义国家,究竟存在不存在阶级?在外国有人讲,没有阶级了,因此党是全民的党,不是阶级的工具,无产阶级的党了。无产阶级专政不存在了,全民专政没有对象了,只有对外矛盾了。像我们这样的国家是否也适应?这个问题是否谈一下。我同几个大区的同志都谈了话,了解到有的人听说,国内还有阶级存在,大吃一惊。资产阶级右派从来不承认有阶级存在,认为没有阶级了,不要改造,不承认阶级斗争,说阶级斗争是马克思捏造出来的。资产阶级不承认阶级斗争,孙中山就不讲阶级,说只有大贫小贫之分。有没有阶级,这是个基本问题。

二、形势问题,也要谈一下,国际问题要找几个人准备一下,究竟是什么情况?帝国主义、修正主义、反动的民族主义、广大人民群众各阶层、民族资产阶级、农民、城市小资产阶级……

国内形势谈一谈。究竟这两年如何?有什么经验?过去几年有许多工作没搞好,有许多还是搞好了,如工业建设、农业建设、水利等等。有人说,农村去年比前年好,今年比去年好,这个说法对不对?工业上半年不那样好,有主客观原因,下半年怎样,还要看一看。有些同志过去曾经认为是一片光明,现在是一片黑暗,没有光明了。是不是一片黑暗?两种看法那种对?如果都不对,是不是应有第三种看法?不是一片黑暗,基本光明,有黑暗,问题不少,确实很大。回到一九五九年庐山会议的三句话。“成绩很大,问题不少,前途光明”。两年调整,彻底调整、巩固、充实、提高的方针做得不那么好。以农业为基础,讲了三年,一九五九年至一九六二年,四个年头,实际上没有实行。中央的东西,有些没有下去,有些成了废品。所谓没有实行,就是没有认真做,个别做了,或者做得很不好。形势问题,我倾向于不那么悲观,不是一片黑暗。现在一片光明的看法没有了,不存在。有些人思想混乱,没有前途,丧失信心,不对。

三、矛盾问题。有些什么矛盾?一类是敌我矛盾,一类是人民内部矛盾。人民内部矛盾有两类。有一种矛盾,对资产阶级的矛盾,实质上敌对的,是社会主义与资本主义的矛盾,我们当作人民内部矛盾处理,如果承认国内阶级还存在,就应该承认社会主义与资本主义的矛盾是存在的。阶级的残余是长期的,矛盾也是长期存在的,不是几十年,我想是几百年,究竟那一年进入社会主义,进入了社会主义是不是就没有矛盾了?没有阶级,就没有马克思主义了,就成了无矛盾论,无冲突论了。现在有一部分农民闹单干,究竟有百分之几十,有说百分之二十,安徽更多。就全国来说,这时期比较突出。究竞走社会主义还是走资本主义道路?农村合作化要不要?“包产到户”还是集体化?已经“包产到户”的,不要强迫纠正,要做工作。为什么要搞这么多文件?为了巩固集体经济。现在就有闹单干之风,越到上层越大,有阶级就有阶层,地、富残余还存在着,闹单干的是富裕阶层、中农阶层、地富残余,资产阶级争夺小资产阶级闹单干,如果无产阶级不注意领导,不做工作,就无法巩固集体经济,就可能搞资本主义。有些人也是要闹单干的。

再有,生产和分配的矛盾,积累与消费的矛盾。积累过多,消费就少了。

再有,集中与分散的矛盾,七千人大会之后,我看没有解决,还要继续做工作,民主与集中的矛盾,要用民主的方法达到集中的目的。要让人家讲话,不民主,集中不起来,还要做工作。

社会主义与资本主义,本质上是敌对矛盾,我们当作人民内部矛盾来处理。积累过多,民主与集中还要做工作。阶级存在不存在?国内形势如何?矛盾,一个是敌我矛盾,一个是人民内部矛盾。敌我矛盾有个肃反问题,还有反革命存在,要看到,看不到不好,看得太严重也不合乎事实。


\section[在北戴河中央工作会议中心小组会上的讲话(一九六二年八月九日)]{在北戴河中央工作会议中心小组会上的讲话}
\datesubtitle{(一九六二年八月九日)}


今天单讲共产党垮得了垮不了的问题。共产党垮了谁来?反正两大党,我们垮了,国民党来。国民党干了二十三年,垮了台,我们还有几年。

农民本来已经发动起来,但是还有资产阶级、右派分子、地主、富农复辟的问题。还有南斯拉夫的方向。(有人插话:国民党在台湾搞了一个政纲,土地收为农民所有,但又保护地主)各地方、各部专搞那些具体问题,而对最普遍、最大量的方向问题不去搞。

单干势必引起两极分化,两年也不要,一年就要分化。

(李××同志揭露邓子恢的问题)派干部下去,而思想不“定一”,不讨论就走,这种办法不好。为什么不请邓子恢来?他不来,我们对台戏唱不成。建议中心小组再加五个人:邓子恢、王××、康生、吴××、胡×。

资本主义思想,几十年、几百年都存在,不说几千年,讲那么长吓人。社会主义才几十年,就搞得干干净净?历代都是如此。苏联到现在几十年,还有修正主义,为国际资本主义服务,实际是反革命。

《农村社会主义高潮》一书,有一段按语讲资产阶级消灭了,只有资本主义思想残余的影响,讲错了,要更正。

有困难,对集体经济是个考验,这是一种大考验,将来还要经受更重大的考验,苏联经过两次大战的大考验,走了许多曲折的道路,现在还出修正主义。我们的困难比苏联的困难更多。

全世界合作化,我们搞得最好。因为从全国说,土改比较彻底,但也有和平土改的地方。政权中混进了不少坏分子与马步芳分子。改变了生产资料所有制,不等于解决了意识的反映。社会主义改造消灭了剥削阶级的所有制,不等于政治上、思想上的斗争没有了。思想意识方面的影响是长期的。高级合作化、一九五六年社会主义改造,完成了消灭资产阶级的所有制,一九五七年提出思想政治革命,补充了不足。资产阶级是可以新生的,苏联就是这个情况。

苏联从一九二一年到一九二八年单干了近十年,没有出路,斯大林才提出搞集体化。一九三五年才取消各种购物券,他们的购物券并不比我们少。

找几个同志把苏联由困难到发展的过程,整理一个资料。这事由康生同志负责,搞一个经济资料。

动摇分子坚决闹单干,以后看形势不行又要求回来。最好不批准,让他们单干二、三年再说,他们早回来,对集体经济不会起积极作用。

要有分析,不要讲一片光明,也不能讲一片黑暗,一九六○年以来,不讲一片光明了,只讲一片黑暗,或者大部分黑暗。思想混乱,于是提出任务:单干,全部或者大部单干。据说只有这样才能增产粮食,否则农业就没有办法。包产百分之四十到户,单干、集体两下竞赛,这实质上叫大部分单干。任务提得很明确,两极分化,贪污盗窃,投机倒把,讨小老婆,放高利贷,一边富裕,而军、烈、工、干四属,五保(户)这边就要贫困。

赫鲁晓夫还不敢公开解散集体农场。

(康生同志插话:现在的价格,低出高进,不利于集体经济。)

内务部一个司长,到凤城宣传安徽包产到户的经验。中央派下去的人常出毛病,要注意。中央下去的干部,要对下面有所帮助,不能瞎出主意,不能随便提出个人意见。政策只能中央制定,所有东西都应由中央批准,再特殊也不能自立政策。

思想上有了分歧,领导要有个态度,否则错误东西泛滥。反正有三个主义:封建主义、资本主义、社会主义。资本主义有买办阶级,美国资本主义农场,平均每个场有十六户,我们一个生产队二十多户。包产到户,大户还要分家,父母无人管饭,为天下中农谋福利。

河北胡开明,有这么一个人,“开明”,但就是个“胡”开明,是个副省长。听了批评“一片黑暗”的论调的传达,感到压力,你压了我那么久,从一九六○年以来,讲两年多了,我也可以压你一下么。

有没有阶级斗争?广州有人说,听火车轰隆轰隆的声音,往南去的像是“走向光明”,“走向光明”,往北开的像是“没有希望”,“没有希望”。

有人发国难财,发国家困难之财,贪污盗窃。党内有这么一部分人,并不是共产主义,而是资本主义、封建主义。

每一个省都有那么一种地方,所谓后解放区,实际上是民主革命不彻底。

党员成分,有大量小资产阶级,有一部分富裕农民及其子弟,有一批知识分子。还有一批未改造过的坏人,实际上不是共产党。名为共产党,实为国民党。对这部分人的民主革命还不彻底,明显的贪污、腐化,这部分人好办。知识分子、地富子弟,有马克思主义化了的,有根本未化的,有的程度不好的。这些人对社会主义革命没有精神准备,我们没有来得及对他们进行教育。资产阶级知识分子,全部把帽子摘掉?资产阶级知识分子,阳过来,阴过去,阴魂未散,要作分析。

民主革命二十八年,在人民中宣传反帝、反封建,宣传力量比较集中,妇孺皆知,深入人心。社会主义才十年,去年提出对干部重新进行教育,是个重要任务。“六大”只说资产阶级不好,但是对资产阶级加了具体分析,反对的是官僚、买办资产阶级,对别的资产阶级就反得不多,三反五反搞了一下。没收国民党、大资本家、帝国主义的财产,这些拿到我们手上,就是社会主义性质,拿到别人手上是资本主义性质。一九五三、一九五四年搞合作社,开始搞社会主义。互助组、合作化、初级社、高级社,一直发展下来。真正社会主义革命是从一九五三年开始的。以后经过多次运动,社会主义建设与社会主义改造在全国展开。一九五八年已有些精神不对,中间有些工作有错娱,最主要的是高征购,瞎指挥,共产风,几个大办,安徽“三改”,引黄灌溉(本来是好的,不晓得盐碱化)。因此四个矛盾再加上一个矛盾,正确与错误的矛盾。高指标,高征购,这是认识上的错误,不是什么两条道路的问题。好人犯错误同走资本主义道路的完全不同,与混进来的及封建主义等更不相同。如基本建设多招了二千万人,没看准,农民就没有饭吃,就要浮肿,现在又减人。

有些同志一有风吹草动,就发生动摇,那是对社会主义革命没有精神准备,和没有马克思主义。没有思想准备,没有马列主义,一有风就顶不住。对这些人应让他们讲话,让他们讲出来,讲比不讲好,言者无罪,但我们要心中有数,行动要少数服从多数,要有领导。××同志的报告中说:“要正确处理单干,纪律处分,开除党籍……”。我看带头的可处分,绝大多数是教育问题,不是纪律处分,但不排除对带头搞分裂的纪律处分。

大家都分析一下原因。

这是无产阶级和富裕农民之间的矛盾。地主、富农不好讲话,富裕农民就不然,他们敢出来讲话。上层影响要估计到。有的地委、省委书记(如曾希圣),就要代表富裕农民。

要花几年功夫,对干部进行教育,把干部轮训搞好,办高级党校,中级党校,不然搞一辈子革命,却搞了资本主义,搞了修正主义,怎么行?

我们这政权包了很多人下来,也包了大批国民党下来,都是包下来的。

罗隆基说,我们现在采取的办法,都是治标的办法。治本的办法是不搞阶级斗争。我们要搞一万年的阶级斗争,不然,我们岂不变成国民党、修正主义分子了。

和平过渡,就是稳不过渡,永远不过渡。

我在大会上只出了个题目,还没有讲完,有的只露了一点意思,过两天可能顺成章。

三年解放战争,猛烈土地改革。土改后,对两种资本主义的改造很顺利。有的地区的民主革命还是不彻底,比如潘汉年、饶漱石,长期没发现。

修正主义的国内根源是资本主义残余,国外是屈从帝国主义的压迫,莫斯科宣言上这两句话是我加的。

一九五七年国际上有一点小风波,风乍起吹皱一池春水。六月刮起十二级台风,他们准备接管政府,我们来个反攻,所有学校的阵地都拿过来了。反右后,五八年算半年,五九年、六○年大跃进。六一年开始搞十二条,六○年搞工业七十条,农业六十条。

过去分田是农民跟地主打架,现在是农民跟农民打架,强劳动力压弱劳动力。

有这样一种农民,两方面都要争夺,地富要争夺,我们要争夺。

小资产阶级、农民有两重性,碰到困难就动摇,所以我们要争夺无产阶级领导权,就是共产党领导。农村的事,依靠贫农,还要争取中农,我们是按劳分配,但要照顾四属、五保。

二千万人呼之则来,挥之则去。不是共产党当权,哪个能办到。五八年十一月第一次郑州会议,提出的商业政策,没执行,按劳分配的政策,也不执行,不是促进农业,集体经济的发展,反而起了不利的影响。商业部应该改个名字,叫“破坏部”,同志们听了不高兴,我故意讲得厉害一点,以便引起注意。商业政策、办法,要从根本上研究。这几年兔、羊、鹅有发展,这是因为这几样东西不征购。打击集体,有利单干,这次无论如何得解决这个问题。

中央有事情总是同各省、市和各部商量,可是有些部就是不同中央商量,中央有些部作得好,像军事、外交,有些部门像计委、经委,还有财贸办、农业办等口子,问题总是不能解决。中央大权独揽,情况不清楚,怎样独揽?人吃了饭要革命,不一定要在一个部门闹革命,为什么不可以到别的部门或下面去革命呢?我是湖南人,在上海、广州、江西七、八年,陕北十三年。不一定在一个地区干,永远如此。中央、地方部门之间,干部交流,再给试一年,看能否解决,陈伯达同志说不能再给了。

财经各部委,从不做报告,事前不请示,事后不报告,独立王国,四时八节,强迫签字,上不联系中央,下不联系群众。

谢天谢地,最近组织部来了一个报告。

外国的事我们都晓得,甚至肯尼廸要干什么也晓得,但是北京各个部,谁晓得他们在干些什么?几个主要经济部门的情况,我就不知道。不知道,怎么出主意?据说各省也有这个问题。



\section{对中共中央组织部的批评(一九六二年八月十二日)}


中共中央组织部从来不向中央作报告,以至中央同志对组织部同志的活动一无所知,全部封锁,成了一个独立王国。



\section[在八届十中全会上的讲话(一九六二年九月二十四日上午怀仁堂)]{在八届十中全会上的讲话}
\datesubtitle{(一九六二年九月二十四日上午怀仁堂)}


现在是十点,开会。

这次中央全会解决了几个重大问题:一是农业问题;二是商业问题,这是两个重要问题,还有工业问题,计划问题,这是第二位的问题,第三个是党内团结问题。有几位同志讲话,农业问题由陈伯达同志说明,商业问题由李先念同志说明,工业计划问题由李富春、薄××说明。另外,还有监察委员会扩大名额问题,干部上下左右交流问题。

会议不是今天开始的,这个会开了两个多月了,在北戴河开了一个月,到北京差不多也是一个月。实际问题在八、九两月,各个小组(在座的人都参加了)经过小组,实际上是大组,都讨论清楚了,现在开大会不需要很多时间了,大约三天就够了,二十七号不够就开到二十八号,至迟二十八号要结束。

我在北戴河提出三个问题:阶级、形势、矛盾。阶级问题,提出这个问题,因为阶级问题没有解决。国内不要讲了。国际形势,有帝国主义、民族主义、修正主义存在,那是资产阶级国家,是没有解决阶级问题的,所以我们有反帝任务,有支持民族解放运动的任务,就是要支持亚、非、拉三大洲广大的人民群众,包括工人、农民、革命的民族资产阶级和革命的知识分子。我们要团结这么多的人,但不包括反动的民族资产阶级,如尼赫鲁,也不包括反动的资产阶级知识分子,如日共叛徒春日庄次郎,主张结构改革论,有七、八个人。

那末,社会主义国家有没有阶级存在?有没有阶级斗争?现在可以肯定,社会主义国家有阶级存在,阶级斗争肯定是存在的。列宁曾经说,革命胜利后,本国被推翻的阶级,因为国际上有资产阶级存在,国内还有资产阶级残余,小资产阶级的存在,不断产生资产阶级,因此,被推翻了的阶级还是长期存在的,甚至要复辟的。欧洲资产阶级革命,如英国、法国等都曾几次反复。社会主义国家也可能出现这种反复,如南斯拉夫就变质了,是修正主义了,由工人、农民的国家变成一个反动的民族主义分子统治的国家。我们这个国家就要好好掌握,好好认识,好好研究这个问题。要承认阶级长期存在,承认阶级与阶级斗争,反动阶级可能复辟。要提高警惕,要好好教育青年人,教育于部,教育群众,教育中层和基层干部,老干部也要研究,教育。不然,我们这样的国家还会走向反面。走向反面也没有什么要紧,还要来个否定的否定,以后又会走向反面。\marginpar{\footnotesize 34}如果我们的儿子一代搞修正主义,走向反面,虽然名为社会主义,实际是资本主义,我们的孙子肯定会起来暴动的,推翻他们的老子,因为群众不满意。所以我们从现在起就必须年年讲,月月讲,天天讲,开大会讲,开党代会讲,开全会讲,开一次会就讲,使我们对这个问题有一条比较清醒的马克思列宁主义的路线。

国内形势:过去几年不大好,现在已经开始好转。一九五九年、一九六〇年,因为办错了一些事情,主要由于认识问题,多数人没有经验。主要是高征购,没有那么多粮食,硬说有。瞎指挥,农业、工业都有瞎指挥。还有几个大办的错误。一九六〇年下半年就开始纠正。说起来就早了,一九五八年十月第一次郑州会议开始了,然后十一月、十二月武昌会议,一九五九年二、三月第二次郑州会议,然后四月上海会议,就注意纠正。这中间,一九六〇年有一段时间对这个问题讲的不够,因为修正主义来了,压我们,注意反对赫鲁晓夫了。从一九五八年下半年开始,他就想封锁中国海岸,要在我们国家搞共同舰队,控制沿海,要封锁我们。赫来我国就是为了这个问题。然后是一九五九年九月中印边界问题,赫支持尼攻击我们,塔斯社发表声明。以后赫来,十月在我国国庆十周年宴会上,在我们讲坛上攻击我们。然后一九六〇年布加勒斯特会议围剿我们。然后两党会议,二十六国起草委员会,八十一国莫斯科会议,还有一个华沙会议,都是马列主义与修正主义的争论,一九六〇年一年,与赫打仗。你看,社会主义国家,马列主义中出现这样的问题,其实根子很远,事情很早就发生了,就是不许中国革命。那是一九四五年,斯大林就是阻止中国革命,说不能打内战,要与蒋介石合作,否则中华民族就要灭亡。当时我们没有执行,革命胜利了。革命胜利后,又怀疑中国是南斯拉夫,我就变成铁托。以后到莫斯科,签订中苏同盟互助条约,也是经过一场斗争的,他不愿签,经过两个月的谈判最后签了。斯大林相信我们是从什么时候起呢?是从抗美援朝起。一九五〇年冬季,相信我们不是铁托,不是南斯拉夫了。但是,现在我们又变成“左倾冒险主义”、“民族主义”、“教条主义”、“宗派主义”者了。而南斯拉夫倒变成“马列主义”者了。现在南斯拉夫可行啊,吃得开了,听说变成了“社会主义”,所以社会主义阵营内部也是复杂的,其实也是简单的。道理就是一条,就是阶级斗争问题。无产阶级与资产阶级的斗争问题,马列主义与反马列主义的斗争问题,马列主义与修正主义之间的斗争的问题。

至于形势,无论国际、国内都是好的。开国初期,包括我在内,还有刘××,曾经有这个看法,认为亚洲的党和工会、非洲党,恐怕受摧残。后来证明,这个看法是不正确的,不是我们所想的。第二次世界大战后,蓬蓬勃勃的民族解放斗争,无论亚洲、非洲、拉丁美洲都是一年比一年地发展的。出现了古巴革命,出现了阿尔及利亚独立,出现了印尼亚洲运动会、几万人示威,打烂印度领事馆,印度孤立,西伊里安荷兰交出来了,出现了越南南部的武装斗争,那是很好的武装斗争,出现了苏伊士运河事件,埃及独立,阿联偏右,出现了伊拉克,两个都是中间偏右的,但它反帝。阿尔及利亚不到一千万人口,法国八十万军队,打了七、八年之久,结果阿尔及利亚胜利了。所以,国际形势很好。陈毅同志作了一个很好的报告。

所谓矛盾,是我们同帝国主义的矛盾,全世界人民同帝国主义的矛盾,是主要的。各国人民反对反动资产阶级,各国人民反对反动的民族主义,各国人民同修正主义的矛盾,帝国主义国家之间的矛盾,民族主义国家与帝国主义国家之间的矛盾,帝国主义国家内部的矛盾,社会主义与帝国主义之间的矛盾。中国的右倾机会主义,看来改个名字好,叫做中国的修正主义。从北戴河到北京的两个月会议,是两种性质的问题,一种是工作问题,一种是阶级斗争的问题,就是马克思主义与修正主义的斗争。\marginpar{\footnotesize 35}工作问题也是与资产阶级思想斗争的问题。工作问题有几个文件,有工业的、农业的、商业的等,有几个同志讲话。

关于党如何对待国内、党内的修正主义问题,资产阶级问题,我看还是照我们原来的方针不变。不论犯了什么错误的同志,还是一九四二年到一九四五年整风时的那个路线,只要认真改变,都表示欢迎,就要团结他,要团结,治病救人,惩前毖后,团结——批评——团结。但是,是非要搞清楚,不能吞吞吐吐,敲一下吐一点,不能采取这样的态度。为什么和尚念经要敲木鱼?《西游记》里讲,取回的经被黑鱼精吃了,敲一下吐一个字,就是这么来的。不要采取这种态度,和黑鱼精一样,要好好想想。犯了错误的同志,只要认识错误,回到马克思主义的立场方面来,我们就与你团结。在座的几位同志,我欢迎,不要犯了错误见不得人。我们允许犯错误,你已经犯了嘛!也允许改正错误。不要不允许犯错误,不许改正错误。有许多同志改的好,改好了就好嘛!李××的发言就是现身说法。李××的错误改了,我们信任嘛!一看二帮,我们坚决这样做。还有很多同志,我也犯过错误,去年我就讲了,你们也要允许我犯错误,允许我改正错误,改了,你们也欢迎。去年我讲,对人是要分析的,人是不能不犯错误的。所谓圣人,说圣人没有缺点是形而上学的观点,而不是马克思主义、辩证唯物主义的观点。任何事物都是可以分析的,我劝同志们,无论是里通外国的也好,搞什么秘密反党小集团的也好,只要把那一套统统倒出来,真正实事求是讲出来,我们就欢迎,还给工作做,决不采取不理他们的态度,更不采取杀头的办法。杀戒不可开,许多反革命都没有杀,潘汉年是一个反革命嘛,胡风、饶漱石也是反革命嘛,我们都没有杀嘛。宣统皇帝是不是反革命?还有王耀武、康泽、杜聿明、杨广等战犯,也有一大批没杀。多少人改正了错误就赦免他们嘛,我们也没有杀。右派改了的摘了帽子嘛。近日平反之风不对,真正错了才平反,搞对了不能平反。真错了的平反,全错全平反,部分错了部分平反,没有错的不平反,不能一律都平反。

工作问题,还请同志们注意,阶级斗争不要影响了我们的工作。一九五九年第一次庐山会议本来是搞工作的,后来出了彭德怀,说:“你操了我四十天娘,我操你二十天娘不行?”这一操,就被扰乱了,工作受到影响。二十天还不够,我们把工作丢了。这次可不能,这次传达要注意,各地、各部们要把工作放在第一位,工作与阶级斗争要平行,阶级斗争不要放在很突出的地位(抄者按:此指对敌斗争)。现在已经组成两个专案审查委员会,把问题搞清楚,潘汉年是一个反革命嘛!胡风、饶漱石也是反革命嘛,我们也没有杀嘛。不要因阶级斗争干扰我们的工作,等下次或再下次全会再未搞清楚。把问题说清楚,要说服人。阶级斗争要搞,但要有专门人搞这个工作。公安部门是专搞阶级斗争的,它的主要任务是对付敌人的破坏。有人搞破坏工作,我们开杀戒,只是对那些破坏工厂,破坏桥梁,在广州边界搞爆破案,杀人放火的人。保卫工作要保卫我们的事业,保卫工厂、企业、公社、生产队、学校、政府、军队、党、群众团体,还有文化机关,包括报馆、刊物、新闻社。保卫上层建筑。

现在不是写小说盛行吗?利用写小说搞反党活动,是一大发明。凡是要想推翻一个政权,先要制造舆论,要搞意识形态,搞上层建筑,革命如此,反革命也如此。我们的意识形态是搞革命的,马克思的学说,列宁的学说,马列主义普遍真理和中国革命具体实践相结合。结合的好,问题就解决的好些,结合的不好就失败受挫折。讲社会主义建设时,也是普遍真理与建设相结合,现在是结合好了还是没有结合好?我们正在解决这个问题。军事建设也是如此。如前几年的军事路线与这几年的军事路线就不同。叶剑英同志搞了部著作,很尖锐,大关节是不糊涂的,我一向批评你不尖锐,这次可尖锐了。\marginpar{\footnotesize 36}我送你两句话:“诸葛一生唯谨慎,吕端大事不糊涂。”

请邓××宣布那几个人不参加全会。政治局常委决定五人不参加。

(×××宣布:政治局常委决定五个同志不参加全会:彭、习、张、黄、周,是被审查的主要分子,在审查期间,没有资格参加会议。)

因为他们的罪恶实在太大了,没有审查清楚以前,没有资格参加这次会议,也不参加重要会议,也不要他们上天安门。主要分子与非主要分子要有分析,是有区别的。非主要分子今天参加了会议。非主要分子彻底改正错误,给他们工作。主要分子如果彻底改正错误,也给工作。特别寄希望于非主要分子觉悟,当然也希望主要分子觉悟。

从现在起以后要年年讲阶级斗争,月月讲,开大会讲,党代会要讲,开一次会要讲一次,以使我们有清醒的马列主义的头脑。

一九五九年八届九中全会胜利粉碎了彭反党集团向党的进攻。十中全会又一次揭露彭反党活动一高饶反党分子成员习仲勋。


官僚主义小则误国误民,大则害国害民。

第一、不调查,不研究,脱离实际,脱离群众的官僚主义。

第二、主观瞎指挥的官僚主义。

第三、忙忙碌碌,不抓政治,迷失方向的官僚主义。

官僚主义发展严重了,一种革命意志衰退、腐化、堕落。一种是互相勾结,敌我不分,官僚主义是修正主义的温床。

治官僚主义的办法:接触群众,接触实际。

三自:固步自封,骄傲自满,夜郎自大。

一高:高官厚禄。

一爱:爱形而上学,爱好资产阶级思想方法。

爱好形而上学,缺乏两分法,这表现在爱讲成绩,不讲缺点,爱听表扬,不爱听批评,老虎屁股摸不得。

遇事不做全面分析,扶得东来西又倒。


\section[在八届十中全会上对工业支援农业的指示(一九六二年七月二十四日)]{在八届十中全会上对工业支援农业的指示}
\datesubtitle{(一九六二年七月二十四日)}


〔在谈到坚持集体化方向,工业支援农业时说〕:搞了四年,以农业为基础的方针不大自觉,不大愿意。巩固集体经济有两方面:一是政策,二是支援农业。从长远来说是农业的技术改造,总的是如何搞社会主义建设的问题。

科学的研究,没有抓农业。科学院党组书记说:科学院是搞尖端的。

要抓农业技术改造。

民主与集中,集中与分散,分散主义首先是中央(指综合部门)。农业机械化要搞个文件,二十五年左右实现机械化,同时实现工业化。有的同志在困难面前躺下。是消极好,还是鼓足干劲好?是单干,还是集体好,要提起注意。

对困难发生消极情绪是不是思想问题?

计委、经委、工交各部要加强支援农业。要抓紧改,这是机会,还来得及。

计委搞两个文件:支援农业的报告,三个部分:方向,面向农业,支援农业的方针。



\section[关于电台的指示(一九六二年十月一日)]{关于电台的指示}
\datesubtitle{(一九六二年十月一日)}


中近东许多国家发生政变。搞政变的人开始就要夺取电台,向全国和全世界说话,原政府的声音人们就听不到了。我们的电台怎么样?是否掌握在可靠的人手里?要从部队调一个强的干部去。




\section{革命委员会的三条基本经验}\datesubtitle{(一九六八年三月三十日)}

革命委员会的基本经验有三条:一条是有革命干部的代表,一条是有军队的代表,一条是有革命群众的代表,实现了革命的三结合。革命委员会要实行一元化的领导,打破重迭的行政机构,精兵简政,组织起一个革命化的联系群众的领导班子。
{\raggedleft (转摘自《人民日报》、《红旗》杂志、《解放军报》一九六八年三月三十日社论《革命委员会好》\par}
\section{支持美国黑人抗暴斗争的声明}\datesubtitle{(一九六八年四月十六日)}

最近,美国黑人牧师马丁·路德·金突然被美帝国主义者暗杀。马丁·路德·金是一个非暴力主义者,但美帝国主义者并没有因此对他宽容,而是使用反革命的暴力,对他进行血腥的镇压。这一件事,深刻地教训了美国的广大黑人群众\footnote{原文为“黑大群众”,更正为“黑人群众”。},激起了他们抗暴斗争的新风暴,席卷了美国一百几十个城市,是美国历史上前所未有的。它显示了在两千多万美国黑人中,蕴藏着极其强大的革命力量。

这场黑人的斗争风暴发生在美国国内,是美帝国主义当前整个政治危机和经济危机的一个突出表现。它给陷于内外交困的美帝国主义以沉重的打击。

美国黑人的斗争,不仅是被剥削、被压迫的黑人争取自由解放的斗争,而且是整个被剥削、被压迫的美国人民反对垄断资产阶级残暴统治的新号角。它对于全世界人民反对美帝国主义的斗争,对于越南人民反对美帝国主义的斗争,是一个巨大的支援和鼓舞。我代表中国人民,对美国黑人的正义斗争,表示坚决的支持。

美国的种族歧视,是殖民主义、帝国主义制度的土产物。美国广大黑人同美国统治集团之间的矛盾,是阶级矛盾。只有推翻美国垄断资产阶级的反动统治,摧毁殖民主义、帝国主义制度,美国黑人才能够取得彻底解放。美国广大黑人同美国白人中的广大劳动人民,有着共同的利益和共同的斗争目标。因此,美国黑人的斗争正在获得越来越多的美国白色人种中的劳动人民和进步人士的同情和支持。美国黑人斗争必将同美国工人运动相结合,最终结束美国垄断资产阶级的罪恶统治。

我在一九六三年《支持美国黑人反对美帝国主义种族歧视的正义斗争的声明》中说过:“万恶的殖民主义、帝国主义制度是随着奴役和贩卖黑人而兴旺起来的,它也必将随着黑色人种的彻底解放而告终。”我现在仍然坚持这个观点。

当前,世界革命进入了一个伟大的新时代。美国黑人争取解放的斗争,是全世界人民反对美帝国主义的总斗争的一个组成部分,是当代世界革命的一个组成部分。我呼吁:世界各国的工人、农民,革命知识分子和一切愿意反对美帝国主义的人们,行动起来,给予美国黑人的斗争以强大的声援!全世界人民更紧密地团结起来,向着我们的共同敌人美帝国主义其帮凶们发动持久的猛烈的进攻!可以肯定,殖民主义、帝国主义和一切剥削制度的彻底崩溃,世界上一切被压迫人民、被压迫民族的彻底翻身,已经为期不远了。
\section[对派性要进行阶级分析——几段最新指示]{对派性要进行阶级分析\\{\large——几段最新指示}}\datesubtitle{(一九六八年四、五月)}

对派性要进行\footnote[1]{原文为“进在”,据《人民报纸》更正为“进行”}阶级分析。

{\raggedleft《人民日报》、《解放军报》一九六八年四月二十日社论:\\《无产阶级革命派的胜利》\par}

党外有党,党内有派,历来如此。

除了沙漠,凡有人群的地方,都有左、中、右,一万年以后还是这样。

{\raggedleft《红旗》杂志一九六八年四月二十六日评论员文章:\\《对派性要进行阶级分析》\par}

派别是阶级的一翼。

{\raggedleft《人民日报》、《红旗》杂志、《解放军报》五一社论:\\《乘胜前进》\par}

\end{document}