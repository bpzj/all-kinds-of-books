第二次帝国主义战争讲演提纲

(一九三九年九月十四日在延安干部大会上的讲演)



一、战争的新阶段:

第二次帝国主义战争早已开始了,已经打了好几年了。日本帝国主义侵略中国是从一九三一年开始的,它首先占领了东三省,然后又于一九三七年大规模侵略中国。意大利帝国主义则于一九三五年侵略阿比西尼亚。一九三六年德意两国联合侵略西班牙。一九三八年德国占领奥捷两国,在这一时期中,东方与西方,共有五万万以上人口卷入了战争,而美英法帝国主义国家却没有参加。因此,我们称这一时期为第二次帝国主义战争的第一阶段。而从现在开始,从英法等帝国主义国家参加战争的现在开始,则称之为第二阶段,因为情况已经完全不同了。如果过去还不能称为世界大战,它还没有世界规模,那么现在就不同了,现在还是帝国主义的世界大战,这是帝国主义战争的新阶段。

二、战争的原因:

根据历史事实与列宁的学说,在帝国主义时代,资本主义的发展是极端不平衡的。因此,帝国主义各国之间的战争是不可避免的。重分世界的第一次帝国主义战争结束之后,不到二十年功夫,第二次帝国主义战争又爆发了,帝国主义各国又来了一次重分世界的战争。、这个新的战争的开始阶段,它的第一个阶段,是建立在上一次全世界经济危机之上的,一九二九年至一九三三年的空前的世界经济危机过去之后,接着的不是繁荣而是物种的萧条。一部分帝国主义国家,或者认为自己在第一次世界大战中分赃过少的国家,如象日本与意大利,或者认为自己在上次战争中是损失了赃物的国家,如象德国,就全副武装跑上了战争的舞台,企图用战争来摆脱经济危机,避免资本主义崩溃,第二次帝国主义战争就是这样开始爆发的,并形成了它的第一阶段。第二次帝国主义战争的第二阶段,德英法波等国发动了大规模战争,世界资本主义各国均将直接间接的卷入战争,这是建立于新的经济危机之上的。从一九三七年开始的新的世界经济危机,这几年来是净淫于英法美这些所谓“和平”国家之内的,并且正向德日意发展。在经济危机之上,又造成了严重的政治危机,人民不满意,资本主义与资产阶级专政,不论在早已法西斯化了的国家或现在介乎的战争的实行法西斯化的国家,这种政治危机,这种人民不满,却日益尖锐化了。另一方面,社会主义的苏联又强大到不可侵犯。在这种情势下,一切帝国主义国家的资产阶级就认为除了扩大战争,除了把片面性的战争扩大为全面性的战争,除了打坏它们的帝国主义朋友,不能逃脱经济危机与政治危机,不能避免自己的死亡。

所有这一切,就是世界各国资产阶级在它临死前夜的打算。至于这种打算一一用重分世界的战争来逃脱经济危机与政治危机,是一定要加速自己死亡的日子,它们就不能设想了。它们象疯狗一样,已经疯了,被资本主义制度把它们弄得完全疯了。它们就不得不向它们的敌人,向世界的壁墙,乱撞乱碰,这就是今天世界各国资产阶级的实际生活。一群疯狗打架…这就是今天的帝国主义战争。

三、战争的目的:

战争是政治的继续。帝国主义的本性是掠夺,帝国主义国家在“和平”时期的政策也无时不是为了掠夺。但如果一些帝国主义国家的掠夺政策遇到了另一些帝国主义国家的阻碍而不能用和平方法冲破这种阻碍时,就使用战争方法去冲破这种阻碍,以便继续其掠夺政策。所以掠夺一一这就是帝国主义战争的唯一的政治目的。第二次帝国主义战争的目的,和第一次帝国主义战争的目的同样,是为了重分世界,就是说,为了重分殖民地半殖民地与势力范围,为了掠夺世界人民,为了争夺对世界人民的统治权。第二次帝国主义战争的这种目的,过去阶段与现在阶段,是完全一样的。除了这种目的以外,还有什么别的目的没有呢?还有什么善良的目的没有呢?一点也没有。不论是德日意,不论是英美法,一切直接间接参加战争的帝国主义国家,只有这一个反革命的目的。掠夺人民的目的,帝国主义的目的。日本帝国主义的“永久和平”,希特勒的“民族自决”,张伯伦的“反对国社主义”,达拉第的“援助波兰”,其实都是“掠夺”二字。不过为了说的好听起见,为了欺骗人民起见,命令他们的秘书制造出几个别致一点的代名词罢了。

四、战争的性质:

战争的性质是根据于战争的目的而定的。一切战争分为两类:照斯大林同志的说法,战争分为:(1)正义的,非掠夺的谋解放的战争;(2)非正义的掠夺的战争。第二次帝国主义战争同第一次帝国主义战争一样,是属于第二类性质的战争。因为这两次战争的目的,都是为了掠夺,丝毫也没有其他的目的,丝毫也不利于其本国与他国的人民。这就是战争的掠夺性,非正义性与帝国主义性。现在战争的双方,为了欺骗人民,为了动员舆论,都不顾羞耻地宣称自己是正义的,而称对方是非正义。其实,这只是一种滑稽,一种欺骗,只有民族解放战争与人民解放战争,以及社会主义国家为了援助这两种战争而战的战争,才是正义的战争。这一次,许多人又弄糊涂了,他们以为德国固然是非正义的,英法却是民主国家反对法西斯国家,波兰则是民族自卫战争,以为英法波方面总多少带有一点进行性,这是极端糊涂的见解,这种糊涂是由于没有弄清战争的目的而来的,也由于没有弄清战争第一阶段与第二阶段的不同特点而来。

五、战争第一阶段的特点:

第二次帝国主义战争第一阶段的特点,是在于:(1)就帝国主义各国之间的关系来说,是一部分帝国主义国家,德日意三个法西斯国家,举行了疯狂的侵略战争,侵犯各弱小民族的利益,侵犯各民主国的利益,并在各民主国内部,策动法西斯威胁,于是全世界人民都有反抗侵略保卫民主的要求,都要求另一部分帝国主义国家,即所谓英美法民主国家,出来干涉侵略战争,并允许人民保存一点仅剩的民主,苏联则屡次提议愿意与各个所谓民主国家建立反侵略统一战线。如果这些所谓民主国家在当时出来干涉侵略者,如果在当时发生了干涉侵略者的战争,例如能与苏联一道,援助西班牙政府军制止德意的侵略,援助中国制止日本的侵略,那么这种行动,这种战争,就是正义的,就有进步性,可是,这些所谓的民主国家,都并不出来干涉,他们采取了“不干涉”政策,他们的目的,在于使侵略国与被侵略国的双方都在战争中消耗起来,然后自己出来干涉,借收渔人之利,至于英法把奥捷两国奉给德国,这是作为交换条件,就是说,作为交换德国向苏联进攻的条件的。英、法、美想使苏德两国冲突起来,借刀杀人,两败俱伤,然后他们就好独霸世界。由于他们的这种“不干涉”政策,在战争中就表现为只有一部分帝国主义国家出面打仗,而另一部分帝国主义国家却“坐山观虎斗”,表现为战斗的片面性,不普遍性,不干涉性。民族国家的资产阶级政府的反革命的“不干涉”政策,没有能够被人民力量的逼迫而废除掉,因而出现了这种片面性,这就是战争第一阶段第一个特点。(2)但在这一时期中,除了德意日帝国主义举行了非正义的掠夺的战争,而所谓民主国家却纵容这种战争,这一种情况之外,还有另一种情况,还有民族解放战争,这就是阿比西尼亚的抗意战争,西班牙共和国的抗德意战争,与伟大的中国抗日战争,全世界人民与社会主义的苏联却真诚地援助了这种战争。这后一种战争,乃是正义的非掠夺的谋解放战争,这就是战争第一阶段的第二个特点。

帝国主义战争的片面性与反帝国主义战争的存在一一这就是第二次帝国主义战争第一阶段的两个特点。

六,我们在战争第一阶段中的革命政策:

据战争第一阶段的特点,这个阶段的革命政策,毫无疑义的,是组织被侵略国家的反侵略统一战线,以抵抗侵略者的进攻;组织各民主国家内人民群众的高涨着的反法西斯斗争,以保卫民主;同时还不放弃组织苏联与各民主国政府之间制止侵略进一步发展的斗争。在这最后的一点上,在组织苏联与各民主国政府间的反侵略统一战线这一点上,就是在慕尼黑之后,由于西班牙失败与捷克灭亡所引起的,在英法两国内部广大人民中甚至资产阶级左派中的愤怒情绪,还有逼迫张伯伦达拉第政府放弃不干涉政策而与苏联缔结反侵略统一战线之可能,这种可能性在当时还没有完全丧失。总之,这一时期内,革命的总任务是把全世界一切可能的力量都组织到反法西斯反侵略的统一战线内,用以抵抗三个法西斯国家的疯狂侵略与各国内部法西斯的袭击。因此,在当时这个统一战线,有下列四个可能的组织成分:(甲)在资本主义国家内是人民统一战线,无产阶级与小资产阶级的统一战线;(乙)在殖民地半殖民地的国家内是民族统一战线,无产阶级至以资产阶级的统一战线;(丙)社会主义的苏联;(丁)各个民主国家的资产阶级及其政府。这第四个组织部分,在当时之所以还有可能性,是因为这些民主国家的资产阶级及其所谓民主政府,同他们本国人民之间,同他们的殖民地半殖民地的人民之间,同苏联之间,在各法西斯国家的疯狂侵略之下,在各国内部法西斯势力威胁之下,以及存在着的反苏危险之下,是有某种程度的共同利益。这些成分中的主要的力量是苏联,如同苏联同各个所谓民主国政府能够组成真正有效的统一战线,配合着各国的人民统一战线与殖民地半殖民地的民族统一战线,是能够制止各个法西斯国家的进一步侵略,延缓大战爆发的日子;如果发生战争,是能够战胜各个法西斯国家的,如果这样做,那是真正有利于世界人民,有益于侵略者以外的世界各国,而为国际无产阶级所赞助的,因此,苏联加入了国际联盟,缔结了苏德苏捷两个互助条约,最后还进行了英法苏三国的谈判,这种革命政策,是适合于当时国际形势的,是必要的,是正确的,是只能这样做而不能有其他做法的。

七、英法苏谈判的破裂与战争第二阶段的开始:

所谓民主国家的资产阶级,他们是一面怕法西斯国家侵略他们的利益,一面更怕革命势力的发展,他们怕苏联,怕自己国家的人民解放运动,怕殖民地半殖民地民族解放运动。因此,他们拒绝了苏联参加在内的真正反侵略的统一战线与真正反侵略的战争而自己单独组成了反革命的统一战线,单独进行了掠夺的强盗战争。

英法苏谈判从四月十五到八月二十三日,进行了四个多月,在苏联方面已经尽到了一切的忍耐,而英法始终不赞成平等互惠的原则,只要求苏联保证他们的安全,而他们确不保证苏联的安全,不肯保证小波罗的海诸国,以便开一个缺口让德国进兵,并且不让苏联军队通过波兰去反对侵略者,英法提议的这样一种丝毫不适合于革命目的,而仅仅适合于反革命目的条约,苏联当然不愿意订,而苏联愿意订立的,根据平等互惠原则,而真正有益于制止侵略者的发展,真正有益于世界和平,英法却死也不愿意订。这就是英法苏谈判破裂的根本原因,在这个时间中,德国放弃了反苏立场,他愿意实际放弃所谓“防共协定”,承认了苏联的边疆不可侵略,于是德苏互不侵犯条约就订立了。

英法在与苏联谈判中的毫无诚意,毫无真正制止侵略的诚意,决心破裂三国谈判,不是证明别的,只是证明张伯伦决心作战了,所以大战的爆发,不但是希特勒要打的,而且是张伯伦要打的,因为如果真要避免战争,就一定要苏联参加才行。这一点,就是英国的路易乔治,这个资产阶级的代表,也是懂得的。可是一些傻子至今不懂得,还以为张伯伦的作战不是事先准备好了的。不懂得张伯伦之所以参加三国谈判,仅仅是为了便于动员舆论,便于向民众说:我们英法政府是仁至义尽了,三国联合既不成功,就只好向德国开火了。

在这种情况下,九月初旬,德国与英法波三量之间的战争就爆发了,于是开始了第二次帝国主义战争的新阶段一一第二阶段。

八、战争第二阶段的特点:

在现时,在大战爆发之后,情况已经根本改变了,过去关于法西斯国家与民主国家的划分,已经失去了意义,在现时,按照性质来划分,只能是:(一)进行非正义的掠夺的帝国主义的战争之诸国家,以及实际赞助这种战争的诸国家,这是第一类.(二)进行正义的非掠夺的民族解放战争与人民解放战争,以及援助这种战争的国家,这是第二类。现在应从新的情况做新的划分,抛弃过去的那种划法,因为情况已经变化了,各民主国家的资产阶级已经最后地拒绝与人民妥协,拒绝与苏联妥协,并且举行了掠夺战争了。现在世界上最反动的国家,已经转到英国方面,反苏反共反民主反人民反弱小民族的第一名魁首,已经是张伯伦了。

在这种情况下,一切共产党员应该懂得,争取所谓民主国家的资产阶级及其政府,同苏联、同各国人民,同各个殖民地半殖民地国家一道,建立反战争反侵略的统一战线。现在是少了一个所谓民主国家的资产阶级,这一成分之由动摇而最后转到敌人营垒,变成帝国主义战争的两个营垒中的一个营垒,乃是一个大变化。因而新的统一战线之可能的组织成分,就由四个变成了三个了。这种变化,第一,就是第二次帝国主义战争由片面变为全面;(由反动的片面变为反动的全面)第二,就使新的反战争反侵略的统一战线之组织成分由复杂变为单纯了一一这就是战争第二阶段的两个特点。

许多同志不去注意情况的变化.不去注意新事变的特点,以为事情还是和过去一样,把自己的思想停留在过去阶段,拿过去的观点来看新的事变,遗弃了新事变中有包藏的性质上的变化,因而不自觉的陷入了完全错误的境地,陷入了社会民主党的境地。我希望犯了这种错误的同志很快的改变过来,我也相信,只要一经提醒,就能改变过来。

九、我们在战争第二阶段的革命政策:

根据战争第二阶段的特点,无产阶级尤其是共产党的革命政策,应该是怎样呢?

我以为是下面这样的:

(甲)在各帝国主义交战国,是号召人民起来反帝国主义战争,揭穿这种战争的帝国主义性质,不管是战争的甲方或乙方,把他们看作一样的强盗,特别反对英国帝国主义这个强盗魁首,唤醒人民不要上帝国主义强盗的当,向人民宣传变帝国主义战争为革命的国内战争,建立反帝国主义战争的人民统一战线。

在各个交战国内,如果有共产党议员的地方,都要对于战争预算投反对票,好象第一次帝国主义大战吋,德国无产阶级的领袖李卜克内西同志在德国议同中所做的英勇坚定的斗争那样,而决不应该投票拥护战争,决不应该使自己的面目,同社会民主党混同起来。因为这些国家内的社会民主党,正在蹈袭第一次帝国主义大战时的覆辙,在所谓“保卫祖国”(资产阶级强盗集团的祖国)的口号之下,无耻的拥护战争。日本社会民主党,即所谓社会大众党,早已叛变了无产阶级,拥护了日本军阀的侵略战争,英法社会民主党,现在又正在张伯伦,达拉第的威迫利诱之下,叛卖英法无产阶级,拥护英法帝国主义的强盗战争。波兰社会民主党则拥护被张伯伦收买的走狗一一波兰资产阶级反动政府,叛卖波兰的民族利益。在波兰,毫无疑义,是应当动员全民并联合苏联抵抗德国帝国主义的侵略,为保卫波兰与解放波兰民族而战。但是波兰法西斯政府则压制了波兰的民族解放运动,拒绝了苏联援助,甘愿带领波兰人民充当英法财政资本的炮灰,甘愿把波兰变成国际财政资本反动战线的一个组织部分。没有问题,我们同情波兰人反,但我们决不同情波兰反动政府,波兰社会民主党拥护这样的政府,同样是不能容许的,因此,不论在德日意,不论在英法波,凡属交战国内共产党只有揭露社会民主党的这种叛卖性,才能争取群众,组织革命统一战线,准备用革命战争打倒反革命战争。

(乙)在各中立国内,好象在美国,共产党员应该在人民面前,揭穿资产阶级政府的帝国主义政策,就是说,名义上中立,实际上赞助战争,并图在战争中大发其洋财的那种政策,美国帝国主义在两年的中日战争中,在中立的假面具之下,已经发了一笔洋财,现在他又想在新的战争中大发其洋财。我在两个星期之前,在九月一日的谈话中,还以为美国资产阶级暂时还不至于在国内放弃民主政治,与平时的经济生活,那知他在这短短的几天之内,宣布了所谓“局部紧急状态”,这样一来,他已经在步英法的后尘,一步一步的走向反动化与战争化了。共产党必须反对这种实际上援助帝国主义战争的假中立,反对这些国家政治上的法西斯化,反对这些国家中的社会民主党的叛卖行为,反对这些国家卷入战争趋于不至无限制的扩大。

(丙)在各种民地半殖民地国家,则是民族统一战线或名抵抗侵略者(如在中国),或不反对宗主国(如在印度),用以达到民族独立之目的。要反对这些国内的民族叛徒出卖民族利益的行为,才能发展统一战线,才能战胜敌人。在各交战国的殖民地内,必须反对民族叛徒们拥护宗王国战争的叛卖行动,反对动员殖民地人民参加宗主国的战线,要把参加第一次帝国主义战争的痛苦经验告诉殖民地人民。在殖民地半殖民地国家中,如果不反对民族叛变,民族解放运动是没有希望的。

十、战争的前途:

这次战争是持久的战争。我们赞同伏罗希洛夫同志在苏联共产党第十八次代表大会的演说,他说:“现在的战争,将来一定是持久的,绵延不绝的,消耗的战争。”他又说:“无疑问的、在必不可避免的总的军事冲突中,要来一个破天荒的与你死我活的关头。”他的估计是很对的。这种持久性,是包括帝国主义战争与反帝国主义战争,包括反革命战争与革命战争,包括战争的继续与暂时的局部的停顿,包括参战阵线的改组变化,参战国家的灭亡与兴建种种情况在内的,这样种种错综复杂曲折变化的情况,就组成了战争的持久性。第二次帝国主义世界大战是人类空前的大灾难,死亡、疾病、饥饿、失业、失学、妻离子散,家破人亡,各种悲惨现象将充满于全世界。在这种情况下,毫无疑义的,将激起所有各资本主义国家的被压迫人民,所有各殖民地半殖民地的被压迫民族,觉醒起来团结起来,反对帝国主义战争,组织革命战争,其规模将比起第一次世界大战时候要大得多。在第一次世界犬战时,除俄国外,各国都没有共产党。现在已经不同了,共产党已经在几十个国家中分布着,并且在政治上组织上强健起来了。在第一次世界大战时,没有社会主义国家,现在已经不同了,苏联不但存在,而且成了世界上第一等的强国。他是坚决反对帝国主义战争,而坚决援助人民解放战争与民族解放战争的,他将在这个战争中起其维护人类利益干涉帝国主义的伟大作用。现在世界上已经分得清清楚楚,一切直接间接参加帝国主义战争的资产阶级都是反动派,组织反动营垒,现在的帝国主义战争就是这个大反动营垒里面两派帝国主义集团之间的战争。这个大反动营垒里面两个大反动派的冲突,并不妨碍他们将来会联起来反对苏联,反对各国人民解放运动,反对殖民地半殖民地的民族解放运动,反对世界革命,如果以为他们会永远打下去,很容易的就被革命人民推翻,那是幼稚的见解。这是一方面一一世界反动战线方面。另一方面,则有苏联有各资本主义国家的人民解放运动,有殖民地半殖民地人民的民族解放运动。所有这些,组成革命的战争,革命的营垒,其目的,就是推翻世界反动营垒,用革命战争打倒帝国主义战争,打倒战争祸首,推翻资产阶级,把全世界被压迫人民被压迫民族从资本主义压迫之下解放出来,从帝国主义战争中解放出来。这是一个伟大的斗争过程,艰难的持久作战的过程,教育人民,唤醒人民,并领导人民向资产阶级战斗的过程。资本主义经济已经走到尽头了,大变化大革命的时代已经到来了。现在的时代就是战争与革命的新时代,把黑暗世界整个儿的改造为光明世界的时代,我们是正处在这个时代中。

进行了两年抗战的中华民族,是属于世界革命营垒中的一个组成部分,一个重要的有机组成部分。四万万五千万人民的民族解放战争,一定会在世界改造过程中起其伟大的作用。帝国主义战争对于世界对于中国都是不利的,但是苏联的存在与发展,全世界各个资本主义国家人民解放运动的存在与发展,各个殖民地半殖民地民族解放运动的在与发展,都是中国的好朋友,都是中国抗战的可靠的援助者。中国、苏联,各国人民解放运动,各国民族解放运动,应该组成坚固的统一战线,这是革命的统一战线,用以对抗反革命的统一战线。帝国主义之间的战争,帝国主义之间的互相削弱,在这一点上说来对于各国人民解放运动,对各国民族解放运动,对于中国的抗战,对于苏联的建立共产主义社会,又是一个有力的条件。这样说来,世界的黑暗是暂时的,世界的前途是光明的,帝国主义一定会死灭下去,被压迫人民与被压迫民族的解放,是没有疑义的。中国的前途也是光阴的,只要中国的抗日民族统一战线更加巩固起来,在坚持抗战,反对投降,坚持团结,反对分裂,坚持进步,反对倒退的口号之下,努力奋斗,我们的敌人也是一定会死灭下去,一个自由独立的新中国就要出来了。

(《新中华报》1939年9月19日)

