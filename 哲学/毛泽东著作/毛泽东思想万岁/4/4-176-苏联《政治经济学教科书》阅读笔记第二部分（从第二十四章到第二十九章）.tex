\section[苏联《政治经济学教科书》阅读笔记第二部分(从第二十四章到第二十九章)]{苏联《政治经济学教科书》阅读笔记第二部分(从第二十四章到第二十九章)}


二十九、关于社会主义生产关系与生产力的矛盾

433页上只是讲生产关系与生产力的“相互作用”,而没有讲社会主义制度下,生产关系和生产力的矛盾。生产关系包括生产资料所有制,劳动中人与人之间的关系,分配制度这三个方面。所有制方面的革命可以说是有底的,例如集体所有制过渡到全民所有制,整个国民经济变成了单一的全民所有制以后,在相当长的时间内,总还是全民所有制,但同样是全民所有制的企业,实行不实行中央与地方分权,哪些企业由谁去管理,这仍然是重要问题。1958年有些基本建设单位实行了投资包干制。就大大地发挥了这些单位的积极性,中央不能只靠自己的积极性,必须发挥企业和地方的积极性,妨碍这种积极性就不利于发展生产.可见在全民所有制的生产关系中也还有要解决的矛盾,至于劳动中人与人的相互关系和分配关系更是要不断地改进。这方面很难说有什么底。在劳动过程中,人与人的关系问题上。例如,领导人采取平等态度,改变某些规章制度,“两参三结合”等等,有很多文章可作。原始公社的公有制时间很长,但是人们在劳动过程中的相互关系却有很多变化。

三十、集体所有制必然要过渡到全民所有制

435页上只讲“公有制的两种形式的存在是客观的必然性’,没有讲集体所有制过渡到全民所有制也是客观必然性。集体所有制过渡到全民所有制是不可避免的客观过程。现在我国有些地方就已明显地看出来,河北省成安县的一个材料。说有些经济作物区的公社现在很富,积累提高到了45%,农民的生活水平很高。这种情况如果继续发展下去,不让集体所有制改变为全民所有制来解决这个矛盾。农民的生活水平就会比工人高,对于工业和农业的发展是不利的。

438—439页上说“国营企业和合作(集体)轻济之间的差别不是根本性的差别....两种形式的公有制……是肿圣不可侵犯的”。

同资本主义比较起来,集体所有制和全民所有制之间的差别不是根本性的差别。就社会主义经济内部来说,两者之间的差别文是根本性的差别。教科书把这两种公有制的形式说成“是神圣不可侵犯的”,如果就敌对势力来讲是可以的,如果要就它本身的发展过程来说,那就错了。任何东西都不能看作是永恒的。两种所有制的并存不能是永恒的。全民所有制本身也有自己变化的过程。

若千年以后,人民公社社有制变为全民所有制以后,全国就出现单一全民所有制,这会大大促进生产力的发展。它在一个时期内仍然是社会主义性质的全民所有制,在经过一定的时期才进而为单一的共产主义全民所有制。所以全民所有制也有一个按劳分配到按需分配的变化过程。

三十一、关于个人财产

439页说,“另一部分是消费品按每个工作者的劳动数量和质量进行分配。成为劳动者的个人财产”。这种说法使人以为社会产品中属于消费品的部分都要分配给劳动者成为个人财产,这是不对的。消费品中一部分是个人财产,一部分是公共财产,如文化教育设备,公共医疗、体育设备、公园等等。而且这部分公共财产会越来越多,当然这一部分也归每个劳动者享受,但它不是个人财产。

440页上把劳动收入、储蓄、住宅、家庭日用品,个人消费品及其他日常设备等平列起来,不好。因为储蓄、住宅等,都是劳动人民收入转化而来的。

这本书在不少地方只谈个人消费。不讲社会消费,如公共文化、福利事业、卫生等,这是一种片面性。我们农村中的房屋还很不像样子的,要有步骤地改变农村的居住条件。我国居民房屋的建设,特别是城市居民的房屋。主要是应该用集体的社会力量来搞。不应该靠个人力量。社会主义社会如果不搞社会集体事业,还成什么社会主义:有人说,社会主义比资本主义更注意物质刺激,这种说法简直是不像样子,

教科书在另一段说;集体农庄农户的财产,包括个人财产,包括个人付业,对于这种个人付业不提公有化的问题,这样农民就会永远是农民。一定的社会制度在一定的时间之内需要巩固它。但这种巩固必须要有一个限制,不能永远地巩固下去。否则就会使反映这种制度的意识形态僵化起来。使人们的思想不能适应新的变化。

同一页上说到个人利益和集体利益结合的问题;“这种结合是用按照社会成员的劳动数量和质量支付劳动报酬的方法,贯彻个人物质利益的原则来实现的。”这里没有讲必要的扣留,而且把个人利益放在这种结合的第一位,就把个人物质利益的原则片面化了。

接着44l页上承认公共利益和个人利益有矛盾,很好。但说到公共利益和个人利益之间的矛盾不是对抗性的,可以得到逐步的解决。说得很空,不能解决问题。像我们这样的国家,人民内部矛盾如果不是一二年整一次风,是永远得不到解决的。

三十二、矛盾是社会主义社会发展的动力

443页第五段承认社会主义社会中,生产力和生产关系的矛盾存在,也讲要克服这种矛盾,但是并不承认矛盾是动力。

接下去的一段讲得还好。但是在社会主义制度下,不只是人与人的关系的某些方面和领导经济的某些形式会妨碍生产力发展,而且所有制方面(例如两种所有制)也存在着妨碍生产力发展的问题。

再下面一段的说法就很有问题。说“社会主义制度下,这种矛盾不是对抗性的、不可调和的矛盾”,这种说法不合乎辩证法。一切矛盾都是不可调和的。哪里有什么可以调和的矛盾呢?有的矛盾是对抗性的。有的矛盾是非对抗性的,但不能说有不可调和的矛盾和可以调和的矛盾。

社会主义制度下(可能是共产主义制度下之误一一原抄者)虽然没有战争,但是还有斗争,有人民内部各派的斗争。社会主义制度下虽然没有一个阶级推翻另一个阶级的革命,但是还有革命.从社会主义过渡到共产主义是革命,从共产主义的这一阶段过渡到另一阶段也是革命。还有技术革命,文化革命。共产主义一定会要经过很多阶段,也一定会有许多革命。

这里说到依靠群众的“积极活动”来及时克服矛盾(444页)所谓“积极活动”就应该包括复杂的斗争。

“在社会主义制度下,没有力图保存腐朽的经济关系的阶级”,(444页)这个说法对。但是在社会主义社会里还有保守的阶层,还有类似的“既得利益集团”,还存在着脑力劳动和体力劳动、城市和乡村、工人和农民之间的差别。虽然这些是非对抗性的矛盾,但是也要经过斗争才能解决矛盾。’我们的干部子弟很令人躭心,他们没有生活经验和社会经验,可是架子很大,有很大的优越感,要教育他们不要靠父母.不要靠先烈,要完全靠自己。

在社会主义社会里总还有先进的人和落后的人。有对集体事业忠心耿耿、勤勤恳恳、朝气勃勃的人,有为名为利、为私为己、暮气沉沉的人。社会主义发展过程中,每个时期,都会有一部分人乐于保存落后的生产关系和社会制度.农村中的富裕中农在许多问题上都有他们自己的观点,他们不能适应新的变化。并且其中有一部分人对这种变化进行抵抗。广东在农村中同富裕中农展开八字宪法的辩论就是证明。

453页第三段倒是讲了社会主义社会内部的斗争,讲得有点生气。但是接着后段中说,“批评与自我批评……是社会主义社会发展的强大动力。”这个说法不对.矛盾是动力,批评与自我批评是解决矛盾的方法。

三十三、认识的辩证过程

446页三段上说:随着生产资料的社会主义公有化“人们成为自己社会经济关系的主人”,“能够完全自觉地掌握和利用这些规律”。应该看到这里要有一个过程。认识规律总是开始少数人认识,然后是多数人认识,从不认识到认识要经过实践的过程和学习的过程,任何人开始总是不懂的,从来没有什么先知先觉。人们要经过实践取得成绩,发生问题,遇到失败。在这样的过程中才能使认识逐步推进,要认识事物发展的客观规律,必须经过实践,必须采取马列主义的态度,而且必须经过成功与失败的比较,反复实践,反复学习,经过多次的胜利和失败,并且进行认真的研究,才能逐步使自己的认识合乎规律。只看见胜利。没看见失败,要认识规律也是不行的。

所谓“完全自觉地掌握和利用规律”这是不容易的。不经过一定的过程是不能实现的.446页上引恩格斯的话,“开始完全自觉地自己改造自己的历史,……才会在很大的程度上和愈宋愈大的程度上产生他们所希望的结果。”说是“开始”,是“愈来愈大”,这就比较准确。

教科书不承认现象和本质的矛盾。本质总是藏在现象的后面,只有通过现象,才能揭露本质。教科书没有写出人认识规律要有一个过程,先锋队也不能例外。

三十四、关于工会和一长制

452页上说到工会的使命时,不讲工会的主要任务是发展生产,不讲如何加强政治教育。.只偏重讲福利。

全文提到“依据一长制原则管理生产”。一切资本主义国家企业都是实行一长制的,社会主义企业管理的原则应当同资本主义企业有根本的区别.我们所实行的在党委领导下厂长负责制就使我们同资本主义企业的管理制度严格地区别开来。

三十五、从原理原则出发不是马列主义的方法

从第二十章以后列举了许多规律。

《资本论》对资本主义经济的分析是从现象出发,找出本质,然后用本质解释现象,因此能够提纲挈领。而教科书的方法是不进行分析,文章写得很乱,它总是从规律、原理、原则、定义出发,.这是马列主义从来反对的方法。原理、原则的结果是要经过分析,经过研究才能得出的。人的认识总是先接触现象,从现象出发找出原理原则来。而教科书却与此相反,它所用的方法不是分析法。而是演绎法。形式逻辑说。“人都要死,张三是人,所以张三要死。”这是从人都要死这个大前提出发得出的结论,这是演绎法。教科书对每个问题总是先下定义,然后把这个定义作为大前提来进行推理。他们不懂得大前提应当是研究问题的结果。必须经过具体分析。才能够发现和证明原理原则。

兰十六、先进经验能毫无阻碍地推广吗?

461页三段上说。’在社会主义国民经济中,能够在一切企业里毫无阻碍地推广科学的最新成就、技术发明和先进经验。”可不一定。在社会主义社会中,还有“学阀”,他们控制科学研究机关,压制新生力量。因此科学的最新成就也不是毫无阻碍的得到推广的。这种说法实际上不承认社会主义社会中有矛盾。任何新的东西出釆,或者因为人们不习惯,或者因为人们不了解。或者因为同一部分人的利益相抵触,就会遇到阻碍。例如我们的密植、深翻这种事情本身没有阶级性.但还是受到了一部分入的反对和抵制。当然社会主义社会里阻碍新的东西的情形资本主义社会是根本不同的。

三十七、关于计划工作

465页上引用恩格斯的话,在社会主义制度下,“按照预定计划进行社会生产就成为可能”这是对的,资本主义社会里.国民经济的平衡是经过经济危机达到的。社会主义社会里有可能经过计划来实现平衡。但是也不能因此就否认我们对必要比例的认识要有一个过程,教科书在这里说,“自发性同自流性同生产资料公有制的存在不相容的。”(46页)但是不能认为社会主义社会里就没有自发性和自流性,我们对于规律的认识不是一开始就很完善.实际工作告诉我们在一个时期内可以有这样的计划。也可以有那样的计划。可以有这些人计划。也可以有那些人的计划。不能说这些人的计划都是合乎规律的,一定是有些计划合乎规律。或是基本上合乎规律,有些计划就不合乎规律或者基本上不合乎规律。

认为对比例关系的认识不要有个过程,不要经过成功或失败的比较,不要经过曲折的发展,都是形而上学的看法。自由是对必然的认识,但必然不是一眼就能看透的,世界上没有天生的圣人,到了社会主义社会也不是所有的人都成了“先知先党”。为什么教科书过去没有出版,为什么出版了要一次再一次地修改?还不是因为过去认识不清楚.现在也还认识不完善吗?拿我们自己的经验来说。开始我们也不懂得搞社会主义。以后在实践中逐步有了认识.认识了一些,但也不是认识够了。如果说够了。那就没有事做了。

466页上说社会主义的特点是“经常地直觉地保持着比例”,这是一个任务,一个要求。要实现这个任务是不容易的。斯大林就说过苏联的计划还不能说已经完全反映了规律的要求。

经常保持比例同时也就是经常出理不平街.因为不成比例了才提出按比例的任务。社会主义经济发展过程中,经常出现不按比例、不平衡的情况,要求我们按比例和综合平衡。例如经济发展了就到处都感到技术人员不够。干部太少.于是就出现干部的需要和干部分配的矛盾。就促进我们多办学校,多培养干部来解决这个矛盾,出现了不平衡,出现了不成比例,人们也就进一步认识客观规律。

在计划工作上。如果什么账都不算,一切听其自然或四平八稳,要求丝毫的漏洞都没有,这两种做法都是不对的。其结果都会破坏比例。

计划是意识形态.意识是实际的反映.又对实际起反作用.过去我们计划规定沿海地区不建设新的工业。1857年以前没有进行什么建设。耽误了七年的时间。1958年以后才开始大的建设.两年中得到很大的发展,这就说明像计划之类意识形态的东西.对经济的发展或不发展.对经济发展的快慢有着多么大的作用。

三十八、关于生产资料生产优先增长和工农业并举

466页上说到生产资料生产优先增长的问题

生产资料生产优先增长是一切社会扩大再生产的共同经济规律。资本主义社会如果不是生产资料生产优先增长也不能扩大再生产。在斯大林时期.由于特别强调了重工业的优先发展,结果在计划中把农业忽略了。前几年.东欧也有过同样的问题.我们的办法是在优先发展重工业的条件下,实行工农业同时并举和其他几个并举,每个并举中又有主要的方面。农业不上去。许多问题不得解决。我们提出工农业并举已经四年了,真正实行是在1960年。重视农业就表现在拨给农业钢材的数量上。1959年给农业的钢材只有59万吨,今年包括水利建设一共是130万吨。这算正是工农业并举了。

这里谈到1925---1957年苏联生产资料生产增长93倍.消费资料生产增长17.5倍。问题是93倍同17.5倍的比例是否对重工业发展有利。要使重工业迅速发展,就要使大家都有积极性,大家都高兴,而要与这样。就必须使工业农业同时并举。轻重工业同时并举。

只要我们使农业。轻工业、重工业都同时高速度地向前发展。我们就可以保证在迅速发展重工业的同时.适当地改善人民的生活。苏联与我们的经验都证明农业不发展。轻工业不发展.对重工业的发展是不利的。

三十九、分配决定论的错误观点

二十章中说,“利用工人个人对发展社会主义生产的物质利益的关心。是国营工业高涨的必要条件。”(348页)又说.“彻底采用经济核算。彻底应用按劳分配的经济规律.把劳动者个人的物质利益同社会生产的利益结合起来在争取国家工业化的斗争中起了重要的作用。”(348页)“社会主义的生产的目的……使工作人员从物质上关心自己劳动的成果。这是社会主义生产力增长的强大动力.”(456页)像这样地把“个人物质利益的关心”绝对化起来。只会带来发展个人主义的危险。

474页上又说按劳分配规律“使工作者从物质利益上关心执行劳动生产率提高的计划。它是发展社会主义生产的一种决定性动力。”人们不能不问。既然社会主义的基本经济规律决定了社会主义生产发展的方向.怎么又把个人物质利益说成生产的决定性的动力?把消费品的分配问题当作决定性的动力。这是一种分配决定论的错误观点。按照马克思在《哥达纲领批判》中所说的分配首先应当是生产资料的分配,生产资料在谁手里,这是决定性的问题。生产资料的分配。决定消费品的分配。把消费品的分配当着决定性的动力,是对马克思上述的正确观点的一种修正,这是一种理论上的错误。

四十、政治挂帅和物资鼓励

475页第三段中把党组织放在地方经济机关之后,地方经济机关成了头,由中央政府直接管理。地方党组织就不能当地挂帅。党组织不挂帅要在当地充分动员一切积极力量是不行的。(475页)虽然承认群众的创造活动。但是说.“群众积极参加完成和超额完成国民经济发展计划的斗争。这是加快共产主义社会建设速度的最重要条件之一。”477页也说“庄员的主动性是发展农业的决定因素之一。”这里把群众的斗争只看作“重要条件之一”的说法违背了人民群众是历史的创造者这个原理。无论如何,不能认为历史是计划工作者创造的,而不是群众创造的。

紧接着又提到,首先要利用物质鼓励因素。好像群众的创造性活动是要靠物质利益鼓励出来的。这本书一有机会就讲个人物质利益,好像总是想用这个东西来引人入胜,这反映了相当多的经济工作人员和领导人员的精神状态,也反映了不重视政治思想工作的情况。在这种情况下,不靠物质鼓励,就没有别的办法了。各尽所能,按劳分配,前半句是说要尽最大努力来生产。为什么要把这两句话分开,总是片面地讲物质鼓励呢?像这样地宣传物质利益,资本主义成了不可战胜的了。

四十一、关于平衡和不平衡

482页上的一段写得不对。资本主义技术的发展,有不平衡的方面,也有平衡的方面。问题是它们这种平衡和不平衡同社会主义制度下技术发展的平衡和不平衡在性质上不同。在社会主义制度下,有技术发展的平衡,也有不平衡。例如解放初期,我们的地质工作人员只有二百多人,地质勘探的情况同国民经济发展的需要极不平衡,经过几年来加强工作,这种不平衡已经走上平衡,但是技术发展的新的不平衡又出现了。目前我国手工劳动还占很大比重,同发展生产、提高劳动生产率的需要不平衡。因此有必要广泛开展技术革新和技术革命来解决这个不平衡。每逢一个新的技术部门出现以后。技术发展不平衡的状况又会特别显著.例如我们现在要搞尖端技术,也就感到许多东西不相适应。书中的这段话既否认了资本主义下某种平衡,也否认了社会主义制度下的某种不平衡。

技术的发展是这样,经济的发展也是这样。这本教科书中没有接触到社会主义生产发展的波浪式前进,说社会主义经济的发展一点波浪都没有,这是不能设想的。任何发展都不是直线的.而是波浪式的,螺旋式的。我们读书也是波浪式的,读书之前做别的事情,读了几个钟头以后要休息,不能无日无夜的读下去,今天读得多,明天读得少,而且每天读的时候,有时议论多,有时议论少,这些都是波浪式,都是起伏。平衡是对不平衡说的,没有不平衡就没有平衡。事物的发展总是不平衡的,因此有平衡的要求。平衡与不平衡的矛盾在各个方面、各部门、各个部门的各个环节都存在。不断地产生。不断地解决,有了头年的计划,又要有第二年的计划.有了年度的计划,又要有季度的计划。有了季度的计划还要有月的计划。一年十二个月,月月要解决平衡和不平衡的矛盾.计划常常要修改就是因为新的不平衡的情况又出来了。

教科书中没有充分运用辩证法,对各个问题没有用辩证法来研究。关于国民经济有计划按比例发展的规律的这一章写得很长,但没有提出平衡和不平衡的矛盾。

社会主义国家的经济能够有计划按比例的发展,使不平衡得到调节,但是不平衡并不消失“物之不齐,物之情也”。因为消灭私有制,可以有计划的组织经济。所以就有可能自觉地掌握和利用不平衡的客观规律,以造成许多相对的、暂时的平衡。

生产力跑得快造成了生产关系不适应生产力,上层建筑不适应生产关系的情况,于是就要改变生产关系和上层建筑,求得适应。上层建筑适应生产关系,生产关系适应生产力,或者说它们之间达到平衡是相对的.生产力总是要不断前进,所以总是不平衡。平衡和不平衡是矛盾的两个侧面。其中不平衡是绝对的,平衡是相对的,否则生产力、生产关系、上层建筑就不能发展了,就固定了。平衡是相对的,不平衡是绝对的,这是普遍的规律;这个普遍的规律难道独不能适用于社会主义社会吗,应当说社会主义社会同样适用这个规律。矛盾斗争是绝对的,统一、一致、团结是过渡的,有条件的。因而是相对的。计划工作中的各种平衡也是暂时的、过渡的、有条件的,因而是相对的。不能设想有一种平衡是没有条件的,是永远的。

我们要以生产力和生产关系的平衡和不平衡,生产关系和上层建筑的平衡和不平衡做为纲,来研究社会主义的经济问题。

政治经济学研究的主要对象是生产关系。但是要研究清楚生产关系。就必须一方面联系研究生产力,另一方面联系研究上层建筑对生产关系的积极作用和消极作用。这本书提到了国家,但未加研究。这是本书的缺点之一。当然在政治经济学的研究中,这两个方面的研究不能太发展了。生产力的研究太发展了,就成为技术科学。自然科学,上层建筑的研究太发展了。就成了国家论,阶级斗争论了。马克思主义三个组成部分中的社会主义部分所研究的是阶级斗争学说:国家论,革命论,党论,战略。策略等等。

世界上没有不能分析的事物。只是;一是情况不同。二是性质不同。许多基本范畴和规律.例如矛盾的统一都是适用的。这样来研究问题。看问题,就有了一定的、完整的世界观和方法论。

四十二、关于所谓“物质刺激”

486页上说:在社会主义阶段。“劳动尚未成为社会一切成员的生活的第一需要’,所以对劳动的物质刺激具有重大的意义。这段的“一切成员”讲得太笼统了。列宁也是社会成员之一,能够说他的劳动没有成为生活的第一需要吗?

486页提出,在社会主义社会中有两部分人,绝大多数忠实地履行自己的义务,有些工作者却不老实对待自己的义务。这个分析得很对。但是要使一部分不老实对待自己义务的人转变过来,也不能光靠物质刺激,还必须经过批评教育,提高他们的觉悟。

书中这段说到较为勤勉积极的工作者,在同样的条件下能创造出更多的产品。是否勤勉积极,显然是决定于政治觉悟,而不决定于文化水平的高低。有些人文化技术水平高。可是不勤勉。不积极,另外有些人文化技术水平较低些。可是很勤勉。很积极,原因是前一种人觉悟低些。后一种人觉悟高些。

书上说对劳动的物质刺激“使生产增加”。(486页),是刺激生产发展的决定因素之一。(487页)但是物质刺激不一定每年都变化。人不一定天天、月月、年年都需要物质刺激。在困难的时候,减少一些物质刺激,人们也要干,而且干得很好,教科书把物质刺激片面化,绝对化,不把提高觉悟放在重要地位。他们不能解释同级工资中为什么人们的劳动有几种不同的情况。比如说,都是五级工,可是有一部分人就干得很好。有一部分人就干得很不好。还有一部分人干得大体上还好。物质刺激都是一样,为什么有这样不同呢?照他们的道理是解释不通的。

即使承认物质刺激是一个重要的原则,但总不是唯一的原则,总还要有另一个原则.在政治思想方面的革命精神鼓励的原则。同时,物质刺激不能单讲个人利益,还应该讲集体利益。应该讲个人利益服从集体利益,暂时利益服从长远利益,局部利益服从全体利益。

在“劳动的物质刺激,社会主义竞赛”的一节(501页)中关于竞赛。有些地方写得还不错,缺点是没有讲政治。

一不死人,二不使身体弱下去,并且逐步略有增强.这两条是基本的。有了这两条,其他东西有也可以,没有也可以。我们要使人民有此觉悟。教科书上对于为前途、为后代总不强调,只强调个人物质利益,常常把物质利益的原则,一下子变成个人物质利益的原则。有一点偷天换日的味道。

他们不讲全体人民的利益解决了。个人利益也就解决了,他们所强调的个人物质利益,实际上是近视眼的个人主义。这种倾向,是无产阶级同资产阶级斗争时的经济主义在社会主义建设时期的表现。在资产阶级革命时期,许多资产阶级革命家的英勇牺牲,也并不是为个人眼前利益,而是为这个阶级的利益,为这个阶级的后代的利益。

在根据地的时候,我们实行供给制,人们还健康些,并不为了追求待遇而吵架,解放后实行工资制了。评了级。反而问题发生得多,有些人常常为了争级别而吵架,要做很多的说服工作。

我们党是连续打了二十多年仗的党,长期实行供给制。当然当时根据地里,整个社会并不是实行供给制,但是实行供给制的人员,内战时期多的时候有几十万人,少的时候也有几万人,抗战时期从一百多万人增加到几百万人,一直到解放初期,大致过的是平均主义的生活。工作都很努力,打仗都很勇敢,完全不是靠什么物质刺激,而是靠革命精神的鼓励。第二次国内战争后期,打了败仗。在这以前打了胜仗,在这以后还是打了胜仗,这并不是因为没有或者有物质刺激,而是因为政治路线,军事路线错误或正确,这些历史经验。对于我们解决社会主义建设的问题有很大意义。

<p align="center">×××</p>

二十八章说。“社会主义企业中的工作者从物质利益上关心自己的劳动成果。是社会主义生产发展的动力。(510页)

二十九章说.”熟练劳动的报酬较高,…这就刺激劳动者提高文化和技术水平。使脑力劳动和体力劳动间的本质差别逐渐消失.(523,524页)

这段说熟练劳动的报酬较高,促使非熟练劳动者不断上进,以便进入熟练工人的行列。这个意思是说,为了多挣钱就来学文化,学技术。在社会主义社会里。每个人在学校学技术,学文化,首先应该是为了建设社会主义社会,为了工业化。为了为人民服务,为了集体利益,而不应该是首先为了得高工资。

说按劳分配“是推动生产发展的强大力量。”(524页)而在526页的第三段.在说明社会主义制度下,工资不断提高以后,未修订的主版本.还有“社会主义比资本主义根本优越的地方就在这里”这样的话。说社会主义比资本主义根本优越的地方就在于工资的不断提高,很不对。工资是消费品的分配,没有生产资料的分配。就不会有产品的分配.不会有产品的分配。不会有消费品的分配。前者是决定后者的。

四十三、关-于社会企业中人与人的关系

500页说,“在社会主义制度下,经济领导人员的威信取决于他们的联系群众的程度。取决于人民对他们的信任。”这句话讲得好。但是要达到这个目的,必须做工作。我们的经验一一如果干部不放下架子,不同工人打成一片,工人就往往不把工厂看成是自己的,而看成干部的。干部的老爷态度使工人不愿意自觉地遵守劳动纪律。不能以为在社会主义制度下,不用做工作,自然会出现劳动者和企业领导人员的创造性的合作。

既然体力劳动者和企业领导人员是统一的生产集体的成员,为什么社会主义企业必须实行一长制而不能实行集体领导下的首长制,即党委领导下的厂长负责制呢?

政治性弱,就只好讲物质刺激了,所以接下去马上就说;“彻底实行从工作者个人物质利益上关心劳动成果的原则……是进一步提高劳动生产率的必要条件。”(506页)

四十四、关于突击和赶任务

s05页上说。未修订的三版上这句话是。“要同突击现象作斗争,要按预定进度表均衡地工作。’根本否认突击。赶任务,讲得太绝对了。“消除赶任务的现象,按图表均衡的进行生产。”

不能完全否认突击,突击和不突击是对立的统一。在自然界中有和风细雨,也有疾风暴雨。突击和不突击也是波浪式起伏,在生产方面的技术革命,常常发生需要突击的情形,农业生产要抢季节。唱戏要有高潮,否定了突击,实际上就是不承认高潮。苏联要赶美国。我们想不用苏联那么多时间达到苏联这样的水平。这些也都是突击。

社会主义竞赛就是落后者赶上先进者。就是要经过突击才能达到的。人与人,组与组。企业与企业,国家与国家,都是要竞赛。要赶先进,也就会有突击。用行政命令的办法搞建设.搞革命,例如依靠行政命令进行土改。合作化,会造成减产的损失.这是因为不发动群众的缘故。不是因为突击的缘故。

四十五、关于价值规律与计划工作

520页用小字印的一段,正确。有批评,有议论。

价值规律作为计划工作的工具。这是好的。但是不能把价值规律作为计划工作的主要依据。我们搞大跃进就不是根据价值规律的要求,而是依据社会主义基本经济规律,依据我们扩大生产的需要。如果单从价值规律的观点来看,我们的大跃进就必然得出得不偿失的结论,就必然把去年大办钢铁说成是无效劳动,土钢质量低、国家补贴多,经跻效果差等等。从局部短期来看,大办钢铁好像吃了亏的,但从整体和长远来看,这是很值得的。因为大办钢铁的运动把我国整个经济建设的局面打开了,在全国建立了许多新的钢铁基地和其他工业基点。这样就使我们有可能大大加快我们的速度。

一九五九年冬,全国参加搞水利的有七五OO多万人,用搞这样两次大规模的运动的办法。我们的水利问题就可以基本上得到解决。从一年、二年、或者三年来看,花这么多的劳动,粮食单位产品的价值当然很高,但是从长远来看,粮食更可以增加得多,增加得快.农业生产可以更稳定,那么每个单位产品的价值,也就更便宜,也就更能够满足人民对粮食的需要。

多发展农业和轻工业。多为重工业创造一些积累.从长远来看,对人民是有利的。只要农民和全国人民了解到国家“赚或者是赔了钱”是用来干什么的,他们就会赞成,不会反对。农民中自己已经提出了支援工业的口号.就是证明。列宁和斯大林都说过。在社会主义建设时期.农民要向国家“进贡”。我国绝大多数农民是极积“进贡”的。只有富裕中农里面15%的人不高兴。他们反对大跃进和人民公社这一套。

总之,我们是计划第一,价格第二。当然价格问题是我们要注意的。前几年我们曾经提高生猪的收购价格。对于发展养猪有积极作用。但是像现在这样大量的普遍的养猪,主要还是靠计划。

521页上说到集体农庄市场上的价格问题,他们那种集体农庄市场自由太大l,对这种市场的价格只用国家的经济力量来调节是不够的,还要有领导。有控制。我国初级市场的价格由国家规定一定的幅度,不让小自由成为大自由。

522页上说;“由于掌握了价值规律,它在社会主义经济中所起的作用不会带来像在资本主义制度下发生危机的那种毁灭性后果。”这种说法,把价值规律的作用夸大了。在社会主义社会里,不发生危机,主要不是由于我们掌握了价值规律,而是由于社会主义的所有制,社会主义的基本经济规律,全国有计划的进行生产和分配,没有自由竞争和无政府状态等等。资本主义的经济危机当然也是由于它的所有制决定的。

四十六、关于工资形式

530页讲工资形式,主张以计件工资为主,计时工资为辅。我们是计时工资为主。计件工资为辅,片面强调计件工资会造成新老工人之间、强弱劳动之间、轻重劳动之间的矛盾。助长部分工人中“为挣大件而斗争”的心理,不是首先关心集体事业。而是首先关心个人收入。有的材料说明计件工资制还有碍于技术革新和机械化的采用。

书上承认在生产自动化的情况下。不宜于实行计件工资。一面说要广泛发展生产自动化,一面又说要广泛采用计件工资形式,这就自相矛盾起来了。

我们实行计时工资和加奖励。这两年的年终跃进奖,就是这种奖励,除了国家工作人员,教育工作人员以外,其他职工中普遍有年终跃进奖,谁发多,谁发少。由每个单位的职工自己评定。

四十七、关于价格的两个问题

有两个问题值得研究。

一个消费品的价格问题。书上说.“社会主义一贯实行的降低人民消费品价格的政策”。(535页)我们的办法是稳定物价,一般不涨也不降。我国工资水平虽然比较低.但是普遍的就业,物价低。房价低,职工生活水平并不很坏。究竟是不断降低物价好。还是不涨不降好,这是值得研究的问题。

另一个是重工业和轻工业品的价格问题,相对说来,他们是重工业品的价格低.轻工业品的价值高。我们是重工业品的价格高、轻工业品的价格低。为什么这样,究竟怎样才好,也值得研究。


