\section[在最高国务会议上讲话要点(一九五八年一月二十八、三十日)]{在最高国务会议上讲话要点(一九五八年一月二十八、三十日)}
\datesubtitle{(一九五八年一月二十八)}


一、八年来第一次在一月讨论国家预算和国民经济计划,以后也要每年在这时候开会,这次人大要开的从容一些,多开小组会,大会可以少开,工作缺点看到的要加以批评,准备工作不太好,一方面开,一方面准备,文件可以在讨论后再修改,再发表。

二、我们这个民族,七、八年来看来是有希望,特别是去年一年,几亿人口经过大鸣大放大辩论,把许多问题搞清楚了,到处发扬了积极性,任务提得更恰当,如十五年在钢铁和其他主要工业方面赶上英国,多快好省,农业发展纲要40条修正,重新发布等。过去做不到的事现在可以做到了,过去没有办法的事,现在也有办法了,如除四害,全民族大有希望,悲观者不对。是大有希望,不是中有希望,小有希望,更不是没有希望,文章就在大字。

我们民族还在逐渐觉悟,因为觉了,打倒了帝国主义、封建主义、官僚资本主义,才进行了社会主义改造,才进行整风,反右派,中国又穷又白,穷就要革命,一张白纸好做文章,西方世界又富又文,他们就是太阔了,包袱甚重,资产阶级思想成堆。

现在生产与我们地位完全不相称,历史甚久,但钢铁生产比不上比利时,它有七百多万吨,我们只有五百二十万吨,群众热情甚好,它完全有把握十五年赶上英国,十五年看五年,五年看三年,三年看头年,头年看头月,更看前冬,去年中共三中全会就在水利、积肥上做了布置。现在群众热潮好像原子能,发出了热力,十五年后,要搞出四千万吨钢,五亿吨煤,四千万瓧电力,农业发展纲要40条,看起来,八年可以完成,为达到这个目的要有干劲,要鼓足勇气,力争上游。

三、有一个朋友说我们:“好大喜功,急功近利,轻视过去,迷信将来”。这几句话恰说到好处,“好大喜功”看是好什么大,喜什么功?是反动派的好大喜功,还是革命派的好大喜功?革命派里又有两种:是主观主义的好大喜功,还是符合实际的好大喜功?我们是好六万万人之大,喜社会主义之功。“急功近利”看是否搞个人突中,是否搞主观主义,还是符合实际,可以达到的平均先进定额。过去不轻视不行,大家每天都想禹、汤、文、武、周公、孔子是不行的,对过去不能过于重视,但不是根本不要,外国的好东西要学,应该保存的古董也要保存,南京、济南、长沙的城墙拆了很好。北京、开封的旧房子最好全部变成新房子,“迷信将来”,人人都是如此,希望寄托在将来,这四句话提得很好。

还有一个右派说我:“好大喜功,偏听偏信,喜怒无常,轻视古董”,“好大喜功”前面已说过。偏听偏信,不可不偏,我们不能偏听右派的话,要偏听社会主义之言,君子群而不党,没有此事,孔夫子杀少正卯,就是有党。“喜怒无常”,是的,我们只能喜好人,当你当了右派时,我们就喜不起未了,就要怒了。“轻视古董”,有些古董如小脚、太监、臭虫等,不要轻视吗?

四、人多好还是人少好?现在还是人多好,目前农民还不注意节育,恐怕要到七亿人口时,人们才会紧张,要看到严重性,但不要怕,要节省,一方面节育,一方面节省,要成为风气。

五、工作方法有两种:一种比较做得好些,一种做得比较差些,也就是两种作风,譬如合作化,一种搞得快些,一种拖到七,八年才搞。我看趁热打铁,一气呵成为好,整风中大鸣大放很好,这是右派发明末后我们搞的,现在全民中用大鸣大放来整风了。

官风、官气要打掉,最好根除。像除四害一样,官风、官气也是一种迷信。要破除迷信,部长也好,总理也好,只能以一个普通劳动者的姿态在人民中出现,要使普通劳动者感到在我们面前是平等的,自己感觉平等是靠不住的,要使对方感觉平等,湖北红安县的干部,1956年上半年官气十足,农民很不高兴,下半年他们改了,穿草鞋到乡下去,农民很欢迎,山东干部下放农村,农民说:“八路军又来了”,这几年官气大长,共产党要改,各党派也要改,共产党的负责人,除了病老以外,每年要有四个月的时间离开北京,向劳动人民取经,回来加工制造,这样可以打掉官气。各党派和民主人士酌情办理,身体不行的可不去。北京不在地方好坏,而在中央机关不产生任何东西,即不生产任何东西,中央只是加工厂,一切原料出自工人、农民那里,我在北京住久了就觉得脑子空了,一出北京就有了东西。

六、劲可鼓而不可泄,有时没有注意,给群众以挫折。一个时期一些问题上发生了错误,如合作社曾有人说搞多了要砍掉十万个,双轮双铧犁在南方名誉好,举登徒子好色为例,宋玉攻其一点,不及其余,这个办法不好。右派就用这个办法攻击我们的,但好人有时也这样看,共产党也有这样的人。共产党也好,民主党派、工商界、知识分子也好,多数人是可以进步的,就是右派,多数也是可能变好的,如不相信多数,就没有信心了,对人民的事业丧失信心是不好的。现在的大学生,百分之七十至八十是剥削家庭出身的,但右派只占百分之二至三,对他们除个别的以外都不开除学籍,用这种政策,可以把他们改造过来。

七,现在是一场新的战争,向自然界开火。“革命尚未成功,同志仍需努力”。要革地球的命,现在我们只能革地球表面的命,在整风以后,要准备把注意力逐渐引向技术革命,要认真学习,搞试验田,到工厂当学徒,要学自然科学、技术科学、社会科学、文学等。但社会革命还要天天革,整风还要整,不能松劲,六月可告一段落,但并不是说改造好了,将来还要整。

要讲不断革命论,解放后搞土改,土改后搞互助组、合作社。一九五六年是公私合营和手工业合作化,接着五七年搞整风,再接着就要搞技术革命,一个接一个,趁热打铁,中间不使冷场,在这里,要团结一个可能团结的人。

八、共产党准备大改。整风和反省,各党派也可以搞,现在已在搞,有很大的成绩。人的思想是可以改变的,作风也可以改变的。全国人民已振奋起来,我们这些人要适应这种情况,适应六亿人民的要求。相当能适应这种情况,各党派在进步,整风在继续,但不要勉强。要把事情搞好,把人整好,不是整坏,整风对共产党要求严格,对民主党派不要太严格了。不太严格不是不整,整整也好,试试看。目的是整得适合人民要求,把人整好不是整坏,相信会整得更好,因为全中国人民都在进步,有一股热气,在这样环境中生活,是有利于进步的。

很值得高兴,民主党派主要负责人成右派的不多,参加最高国务会议的人成右派还不到十人,但也给了我们以教训,去年四月三十日的最高国务会议上,我们说过资产阶级的知识分子要改造。“皮之不存、毛将焉附”,知识分子要附到工人阶级的皮上来,否则变成梁上君子,但章伯钧、罗隆基等听不进去。他们要取消学校党委制,要同共产党轮流坐庄。他们很高兴“长期共存”,但他们变成“短期共存”了。

口头喊万岁,切记不要都信,有些人大喊万岁,接受领导,但实际上却猖狂进攻。

在统一战线内部,不管共产党和民主党派,要互相帮助,要讲直话,要去掉疑心。要将心交给人家,要当面讲,不要在后面讲。“逢人且说三分话,不可全抛一片心”,这是旧社会的话,现在不适用,逐步做到说真话。

九、对资产阶级知识分子,我们总是要说改造,从未说不要改造,知识分子要向劳动人民投降,知识分子在某一点说是最无知识,知识分子不失败一次,不会翻身。我们党失败过多次,从右的和“左”的两大错误中取得了教训,就全面了,民主党派不见得更高明,中共出了高、饶,你们就没有?我们都是旧社会来的,人要经过严格考验,才能取得教训。

政治和业务要配合,要又红又专,红是政治、专是业务,不红只专是白色专家,搞政治的,如只红不专,不熟悉业务,不懂得实际,红是假红,是空头政治家。搞政治的,要钻业务,搞科技的要红起来。十五年赶上英国,要有成百万上千万忠于无产阶级的知识分子。

十、要开一个右派分子大会,在大会中,第一向他们致感谢,第二想帮助他们。所谓感谢,是指他们向工人和党进攻,当了教员。帮助他们,是想在其中使五成到七成的人,经过五年到十年时间,逐步变过来,为人民服务。总有不变的人,即使如此,也有用处,用处就是在他不变,容许社会上有一部分人,不变不强迫,对右派批判必须严肃、深刻、全面,处理要比较宽大,当然宽大无边是不好的,要有处分,但要留一条路让他们走。第一是为了许多思想上还未解决问题的中间分子,第二是为了这些右派本身,使他们有可能间到人民的队伍里来,当然首先要他们自己下决心,但是还要我们帮助。

右派大会要开,那一天开,要研究,不只是北京开,各地也要开,先开小的,然后开大的。

(注:为便利阅读,把前后两次谈话按问题整理在一起,问题排列次序也略有变动,项目也是记录者所加——中央统战部)


