\section[在庆祝党的二十二周年晚会上的讲话(一九四三年七月三日)]{在庆祝党的二十二周年晚会上的讲话}
\datesubtitle{(一九四三年七月三日)}


今天是纪念党的二十二年和抗战的六周年,现在,全世界全中国一切反法西斯力量、全世界各国共产党和我们中国共产党,任务都只有一个,这就是打败人类公敌法西斯侵略者德意日。

毛泽东同志总结六年的抗战说:中国抗战已经六年,就时间来说,比别国却会更长一些。他指出:一年以来,世界战争的形势已经有了根本的变化。过去是全世界各国被法西斯进攻,法西斯则主动的进攻世界的一切国家,并且在进攻中打胜仗和压迫反法西斯的国家,这就是过去的情况,就是说,是一种不利,艰难困恶的情况。现在的情况就根本不同了,起了根本的变化了。这个变化是在过去一年中发生的。苏联冬季攻势的胜利,英美在北非的胜利,太平洋上英美的胜利,和中国抗战的坚持六年,就是造成了这个根本变化的原因。其中特别是斯大林格勒的大胜利,起了转变形势的主要决定作用。过去,法西斯侵略者非常猖獗,主动权握在他的手里,现在,法西斯侵略者已经丧失了主动权,主动权到了同盟国的手里了。毛泽东同志以肯定的语调说,今后的问题,要解决法西斯。这个解决要分两步,先解决德国,然后解决日本。往后的一年,是欧洲决战的一年。毛泽东同志指出:去年我曾经说过,欧洲在一九四二年即可决战,但是因为欧洲第二条战线没有开辟,因此没有实现。今年的关键还是这个欧洲第二条战线,如果愈早建立起来,胜利愈早到来。

毛泽东同志更进一步解释道:过去我们指出趋势,指出可能性,我们共产党时时这样指出,为的是他大家在困难中看到光明的前途。现在的盟国的被动状况已经结束,转到了主动,所以过去仅是可能性的东西,今天就是转变为现实的东西了。今天欧洲还没有第二条战线,但此事成为现实,一定是不会很久的了,这是今天世界人类努力的目标。

对于中国战场,毛泽东同志说,打倒了大头子希特勒,则二头子日本法西斯亦一定被打倒。大后方有一部分人,弄不清楚,不赞成先打倒希特勒,这是不对的。现在全世界结成了整个反法西斯战线,任何国家都非孤立作战,所以在决定战略的时候,不应从一个单独国家的眼前利益来看,要先看打什么对于整个反法西斯战线最为有利。这样一看,就可以知道,打倒了希特勒,解决日本会是很顺利的。

这样比较了过去六年抗战与目前形势之后,毛泽东同志结语说:可以断定,过去我们所指出的光明前途,现在接近实现了。

毛泽东同志继之总结中国共产党的二十二年,他把第一次世界大战时的世界与第二次世界大战时的世界作一比较,又把二十二年前的中国与现在的中国作一比较。他指出:从这种比较中得出的结论,就使我们对光明的新世界和光明的新中国加强信心,我们就知道世界将向那个方向去,中国将向那个方向去,是否世界将很快成为光明的世界,中国将很快成为光明的中国。

毛泽东同志首先比较了两次世界大战。三十年来世界上爆发了两次空前规模的战争,乃是世界经济发展的必然结果。第一次世界大战曾被列宁正确地预见了,而预见第二次世界大战的乃是斯大林。他们的正确预见,证明马列主义乃是其正的科学真理。

第一次世界大战,是非正义的战争,是帝国主义战争,那时世界上还只有一个俄国的布尔什维克党和别国的很少数人是真正的共产主义者,而那时的各国社会民主党,其领袖是赞成帝国主义战争的,其党员则最大多数是还没觉悟的。这是三十年前的情况。

第二次世界大战,世界的面貌全变了,同盟国中有社会主义的国家,有资本主义的国家,有殖民地半殖民地,同盟国各国的共产党参加了这个战争,这是一个正义战争。

仅在第一次世界大战时的第三年才由俄国的布尔什维克党建立的社会主义国家一一苏联,相隔不过二十多年,现在变成了全世界人类反法西斯战争中的主角,没有苏联红军、苏联人民和苏联领导者斯大林,没有斯大林格勒一战,人类的命运还在不知之数。有了一个列宁斯大林所领导的布尔什维克党,旧的俄国就变成社会主义的苏联。世界上有了这一块社会主义国家的领土,就影响整个世界,以至现在成为世界人类反法西斯的主角。

再就全世界共产主义运动来说,一九一七年十月革命胜利之后,一九一九年共产国际成立,于是在东方,中国共产党于一九二一年成立,日本共产党于一九二二年成立,印尼共产党于一九三三年成立。我们中国共产党在二十二年前举行第一次代表大会,到会的只有十二个代表,现在却已经成了这样的大政党。

中国共产党在其自己经历的二十二年中已经干了三次大的革命运动,现在则协同全国人民组织统一战线,集中力量对付日寇。我们现在绝非只有孤立的一个党,而有全国的人民,有全世界人民和我们一起反对日本法西斯的。中国的四万万五千万人,印度的四万万人,南洋的一万万人,日本朝鲜台湾的一万万人,实际上都是反对日本法西斯的。毛泽东同志对日本北产党的领袖与日本人民的代表岗野进同志表示热烈的欢迎,他说,我们久处山中,希望来延的冈野进同志多多指教我们,现在是为了联合打倒日本法西斯的需要,将来是为了建设新中国与日本。

毛泽东同志转到比较现在的中国与第一次世界大战时的中国。他说,那时的中国,没有共产党,人民不觉悟到这样的程度,以致外国人有好坏两种还分不清,还不知道将外国人区分为帝国主义者与善良的工农。那时马克思已经产生了七十年,但我们还不知道它,学校里教员讲哲学、经济学、社会学时,连马克思的名字也不知道,那时中国已经有了旧民主主义的文化,但新民主主义的文化,新民主主义的文学、艺术,则还没有,那时甚至还不承认白话文,第一次世界大战,中国也是参战国之一,可是只替帝国主义者帮忙,自己的半殖民地地位却不但未改变,反而加深了。

现在的中国,则是抗战的中国,人民觉悟到参加反法西斯的国际战线来与法西斯作战,有了共产党,文化也进步了。我们常常把我们眼前的许多进行事情当作家常便饭,其实,此起从前来,乃是改朝换代的大变化。

由此可见,不仅在抗战的问题上,六年来起了根本的变化,而且就三十年前后的世界与中国来说,也有着两样根本不同的情景,那是翻天覆地的大进步。这种进步,是人类用自己的手造出来的。这一次的反法西斯战争,必然要造出一个更进步的世界,一个更加进步的中国来。法西斯拉着世界往后退,那是不行的。向前进步,这就是我们的大方向。

“有了方向还要有政策”,毛泽东同志说。于是他转到政策的问题,他说,政策可分全国的和边区与敌后抗日根据地两部分来说。他说,关于全国抗战政策,党中央抗战六周年的宣言中,提出了四条向政府建议,这就是“加强作战”“加强团结”、“改良政治”“发展生产”。至于抗战胜利之后怎么办,我党去年七七宣言中已经说得很清楚,我们希望与各党派继续合作,共同建国。

于是毛泽东同志讲到党在边区与敌后的政策。他说,我们抗战,现在有正面与敌后两个战场,敌后抗战的斗争非常残酷,我们共产党在那里是作了工作的,几年以来,我们创造了许多新东西,例如反“扫荡”,“反蚕食”,精兵简政、拥政爱民、拥护军队、生产运动、整顿三风等等都是。毛泽东同志指出,对于别的地方,看见有缺点,我们只有建议,但是边区与敌后则不然,我们可以自己动手,所以应当把工作做得更好些。

毛泽东同志把抗战六年来党在敌后与边区的政策,分做两个时期来作总结。第一个时期:是抗战开始后的四年半(到一九四一年底为止),第二时期,是最近的一年半。

在第一时期中,党的注意力,放在下列问题上:如何组成抗日民族统一战线,如何发动群众,如何与友军抵抗日军的战略进攻,如何创造敌后抗日根据地,以及制定各种政策,如土地政策、劳动政策、三三制政策等等,这些都是那时迫切需要解决的问题,在这四年半的后一年半中,还曾被迫去对付反共分子的两次大磨擦。

在第二时期,即最近的一年半中,除了继续执行上述各项工作而外,又进行了整顿三风,精兵简政,拥政爱民与拥军运动。毛泽东同志特别详细地讲到整顿三风,说这件事保证了党在思想上政治上的一致,和党的组织成份的纯洁。毛泽东同志指出这些工作必须继续不懈地进行下去,用以保证抗战的胜利,一切为的战胜敌人,为了克服现在的困难,迎接将来的光明。

<p align="right">(1943.7.3《解放日报》)</p>

