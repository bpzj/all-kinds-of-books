\section[给林彪、荣臻等同志的信(一九六三年十一月十六日)]{给林彪、荣臻等同志的信}
\datesubtitle{(一九六三年十一月十六日)}


林彪、荣臻、××同志:

国家工业各个部门现在有人提议从上至下(即从部到厂矿),都学解放军,都设政治部、政治处和政治指导员,实行四个第一和三八作风。我并建议从解放军调几批好的干部去工业部门那里去作政治工作(分几年完成,一年调一批人),如同石油部那样。据×××同志说:现在已有水利电力部、冶金工业、化学工业部正在学习石油部学解放军的办法在做。我已收到冶金部学解放军的详细报告,他们主张从上到下设政治部、处和指导员。看来不这样做是不行的,是不能振起整个工业部门(还有商业部,还有农业部门)成百万成千万的干部和工人的革命精神的。要这样做,政治干部的来源,我想有四个办法解决:一是从解放军中调出一部分强的而又可能调出的政治干部和懂政治的军事干部送到工、商、农部门中去(先着重工业部门);二是由工业及其他部门派得力同志到解放军的军师团去学习几个月;三是由他们派人到现在莫文骅管的政治学院去当学生,按期毕业,回去工作;四是他们自己抓起来做,将解放军一套思想政治工作条例办法,拿去略加改变(必须适合各个部门的情况),做为自己的东西去实行,现在已有四个部这样做厂。看来这第四项办法是主要的,因为解放军不可能调出很多的干部。但解放军要给他们以帮助,是肯定的,请你们考虑一下是否可行,然后我和中央常委同志问你们淡一下(有个别管工业的同志参加。林有病可不出席),把方针确定下来。这个问题我考虑了几年了,现在因为工业部门主动提出学解放军,并有石油部的伟大成绩可以说服人,这就到了普遍实行的时候了。\marginpar{\footnotesize 65}解放军的思想政治工作和军事工作,经林彪同志提出四个第一、三八作风之后,比较过去有了一个很大的发展,更具体化又更理论化了,因而更便于工业部门采用和学习了。

\kaoyouriqi{一九六三午十一月十六日}
