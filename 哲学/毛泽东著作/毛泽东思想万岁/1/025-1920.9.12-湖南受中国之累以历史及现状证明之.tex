\section{湖南受中国之累以历史及现状证明之}
\datesubtitle{(一九二〇年九月十二日)}



自有中国,就有湖南。湖南在古为蛮地,在周为楚国,在汉为长沙国,唐为节度使地,宋为荆湖南道!至元建为行省,明清仍之,迄今不变。莽莽四千年,人类总是进化的,而湖南进化的地方在那里?春秋时,荆楚屈兴,凡欲和中原大国挈长较短。其时则上无中央政府,诸国并立,各得遂其发展,虽迷于竟争侵略,用事者野心英雄的君臣而无与于小百姓,然声光赫濯,得发展一中分之特性,较之奴隶于专制黑暗的总组织者,胜得多多。不过所谓荆楚,其中心不在湖南而在湖北,潇湘片土,对于江汉,犹是卑职之于上司。所以湖南在当时之中国,仍算不得什么。长沙国,以小弱见全,可怜的国,一非自主自决的国。节度史地,荆湖南道,一言蔽之,被治之奴隶耳。五代曾为马殷割据,陋孺殊不足道。至于行省,乃皇帝行巡宫府,举湖南而为一王之奴隶。之明历德,长夜漫漫,所得的只是至痛极惨。由此以观,四千年历史中,湖南人未尝伸过腰,吐过气,湖南的历史,只是黑暗的历史,湖南的文明,只是灰色的文明。这是四千年来湖南受中国之累,不能遂其自然发展的结果。

中国维新,湖南最早。丁酋戊戌之秋,湖南人生气勃发,新学术之研究,新教育之建设,谭嗣同熊希龄辈领袖其间,全国无出湖南之右。乃未与而熊遂谭杀,亡清政府以其官方施于湖南,新镜顿挫,事业全亡。这又是湖南受中国之累,不能遂其自然发展的结果。

湖南有黄克强,中国乃有实行的革命家,甲辰一役,萍醴丧亡,黄克强出遁,马福盖骈首,清廷以其暴力,戳辱湘人,湖南不走克辛亥而推倒满清,早脱臣妾之羁勒,这又是湖南受中国之累,不能遂其自然发展的结果。

民国成立,分数论不胜集权论,袁盗当国,汤屠到湘,湖南于是第一次被征服。湘人驱汤,而北方段祺瑞又欲达其力征统一之迷梦,传良佑以湘人而凭借北势,被命督湘,湖南于是第二次被征服。湘人起而逐传,兵到岳阳,骤遇大敌,张敬尧连陷长宝,湖南于是第三次被征服。今借湘人自决的力,奋起驱张,恢复全宇,然九年三被征服,屡践北人马蹄,假中央统一之名,行地方睬躏之实,运不更是近事之中,湖南受中国之累,不能逐其自然发展的结果吗?

反之,湖南不受中国之累,得遂其自然发展,岂犹是今日的湖南吗?小组织受束于大组织,事事要问过中央,事享要听命别人,致造成今日之恶结果。假使湖南人早能自决自治,远且不言,丁戊以云新之气,居全国之先,使无所谓中央者为之宰割,不早已造成了一个新湖南吗?次之辛亥革命,湖南首应,湘人治湘,行之二载,使无所谓中央者为之宰割,加以人民能自觉悟,奋其创造建设之力,三被征服之惨祸不作,不又早已造成了一个新湖南吗?我赏思之,重思之;前者所以未能,固由湘人无力,亦象机会未来。现在呢?机会来了,会实实在在来了。全中无政府,全中大乱而特乱,我料定这种现象至少尚要延长七八年,以后中国当大分裂大糜烂,武人更横行,政治更毁败,然在这当中必定要发生一种新现象。什么新现象呢?就是由武人官僚的割据垄断,变为各省人民的各省自治。各省人民,因受武人官僚专制统治垄断之毒,奋起而争自由,从湘人自决,粤人自决,川人自决,以至直人自决,本人自次,这是必至之势。如此者十年乃至二十年后,再有异军苍头特起,乃是彻底的总革命。

湖南人呵,我们的使命实在重大,我们的机会实在佳胜,我们应该努力,先以湖南共和国为目标,实施新理想,创造新生活,在潇湘片士开辟一个新天地,为二十七个小中国的首倡。湖南人呵!我们应该一齐努力!

(同上)

