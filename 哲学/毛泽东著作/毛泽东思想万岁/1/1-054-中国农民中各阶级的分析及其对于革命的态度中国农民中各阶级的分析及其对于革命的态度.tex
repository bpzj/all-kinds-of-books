\section[中国农民中各阶级的分析及其对于革命的态度中国农民中各阶级的分析及其对于革命的态度]{中国农民中各阶级的分析及其对于革命的态度中国农民中各阶级的分析及其对于革命的态度}


你跑到农村中间去,无论你走至甚么地方,只要你留心去看,你便看得见有下列八种不同的人:

大地主,

小地主,

自耕农,

半自耕农,

半益农,

贫农,

雇农及乡村手工业者,

游民。

这八种人分成八个阶级,其经济地位各不同,其生活状况各不同,因而影响于其心理即其对于革命的观念也各不同。

中国大地主的来源,一大部分是前清官僚贵族后裔及现在的官僚军阀,一个部分是城市富商置买土地,力田起家成大地主者极少,其利益建筑于对自耕农、半自耕农、半益农、贫农、雇农五种农民的严重剥削之上。其剥削方法分为五种:第一种重租,自百分之五十至百分之八十。乃对于半自耕农、半益农、贫农的剥削,此种剥削极其普遍而惨酷;第二种高利贷,月息自百分之三至百分之七,年息自百分之三十六至百分之八十四,也是对于半自耕农、半益农、贫农的剥削,这种剥削之惨,有的较重租更甚,往往有因借债累息,数年即完全破产者;第三种重捐,乃用一种压力强迫自耕农、半自耕农按亩出捐,以充团防局经费。此团防局(或名民团)乃地主阶级的武装,为镇压农民暴动维持地主阶级剥削制度之必要的设备;第四种为对于雇农的剥削,即剥削其剩余劳动。但中国尚少资本主义的农业,大地主多不亲自经营土地,故此项剥削小地主较多,大地主较少;第五种为与军阀及贪官污吏合作,本年包缴预征田赋,而来年索取重息于完粮之农民。合这五种剥削加于农民的惨酷,真是不可形容。所以中国的大地主是中国农民的死敌。是乡村中真正统治者,是帝国主义军阀的真实基础,是封建宗法社会的唯一坚垒,是一切反革命势力发生的最后原因。大地主阶级人数,以收管业五百亩以上者计算,在农民中大概约占千分之一(包括其家属在内),在全国三万万二千万农民中(以全人口百分之八十计)约占三十二万人。

小地主数目比大地主多,全国至少在二百万以上。其来源大部分为力田起家即自耕农升上来者,亦有一部分都市商人购置土地,又有一部分为官僚后裔之衰败者及现在之小官僚。其剥削方法为重租,高利贷,和剥削剩余劳动三种。此种人颇受军阀及大地主的压迫,故颇有反抗性,然又怕“共产”,故对于现代的革命取了矛盾的态度,国内高等知识分子如大学专门学校教员学生以及东西洋留学生一大半都是小地主子弟,所谓国家主义乃自他们口中倡导出来。盖小地主为中国的中产阶级,其欲望为欲达到大资产阶级地位,建设一个一阶级统治的国家,然受外资打击军阀压迫不能发展,故需要革命。但因现代的中国革命运动,在国内有本国无产阶级的勇猛参加,在国外有国际无产阶级的积极援助,对于其欲达到大资产阶级地位建设国家主义国家的阶级的发展及存在,感觉着威胁,又怀疑革命。有一个戴季陶的真实信徒(其自称如此)在北京晨报上发表议论说:“举起你的左手,打倒帝国主义;举起你的右手,打倒共产党”,乃活画出这个阶级的矛盾惶遽态度。他们反对以阶级斗争说解释民生主义,反对国民党联俄及容纳共产党分子。这一班人乃中国中产阶级的右派,他们颇有跑向反革命地位的倾向。但中产阶级中有一个左派,在相当时候可以引向革命的路。如在农民协会运动气焰高涨时,小地主中间的左派,分子可以引其帮助农民协会们忙。但其性质极易妥协,其血统到底与小地主右派及大地主要亲,与农民协会要疏,断不能望其勇敢地跑上革命的路,跟着其余阶级忠实地做革命事业,除开少数历史上和环境上都有特别情况的人。

自耕农属于小资产阶级,其中又分三种。第一种自耕农是有余钱剩米的。即每年劳动所得,除自给外,还有剩余,因以造成所谓资本的初步积累。这种人“发财”观念极重,虽不妄想发大财,却总想爬上那小地主地位。他们看见那些受人尊敬的小财东,往往垂着一尺长涎水,对于赵公元帅礼拜最勤。这种人胆于极小,他们怕官,也有点怕革命。因为他们的经济地位与中产阶级的小地主颇接近,故对于乡村小地主们中那些什么“老”什么“会”什么“胡子”的“谨防过激党”“谨防共产”的宣传颇相信,自然这些谨防的话又是从大地主中那些什么“大人”什么“老爷”的咀里出来的。这班余钱剩米派,乃是小资产阶级的右翼,他们对于现代的革命在他们没有明了真相以前,取了怀疑的态度。但这部分人在自耕农中占少数,大概不及全数百分之十。中国的自耕农人数,有人说超过佃农、雇农的总数。但把半自耕农除外,一定只占农民的少数,大概自一万万至一万万二千万。自耕农中的富裕部分约占其中百分之十,计一千二百万人。第二种自耕农是恰足自给,每年收支怕是相抵,不多也不少。这种自耕农比较第一种自耕农大不相同,他们也想发财,但是赵公元帅总不让他们发财。随着近年帝国主义军阀地主阶级的压迫和剥削,使他们感觉现在的世界已经不是从前的世界。他们感觉得现在如果只使用从前相等的劳力,就会不能维持生活。必须增加劳动时间,即每天起早散晚,对于职业加倍注意,才能维持生活。他们有点骂人了,他们骂洋人叫“鬼子”,骂军阀叫“抢钱司令”,骂土豪劣绅叫“为富不仁”。对于反帝国主义反军阀的运动仅怀疑其未必成功(理由是洋人和司令的来头那么大),不肯贸然参加,取了中立态度,但绝不反对革命。这一部人数甚多,大概占自耕农的一半,约六千万。第三种自耕农是每年要亏本的。这种自耕农好些是原先本来是所谓殷实人家,渐渐变得只能保守,渐渐变得要亏本了,他们每逢年终结账一次就吃惊一次,说“咳!又亏了”。这种人因为他们从前过着好日子,后来逐年下降,负债渐多,渐次过着凄凉的日子,真是“瞻念前途,不寒而栗”,这种人在精神上感的痛苦比较大,因为他们有一个从前与现在相反的比较。这种人在革命运动中颇要紧,颇有推进革命的力量。其人数约占自耕农中百分之四十,即四千八百万――一个不小的群众,乃小资产阶级的左翼。以上说三种自耕农村于现代中国革命的态度,在平时各不相同。但一到战时,即革命潮流高涨可以看得见胜利的曙光时,不但第三种左倾的自耕农马上参加革命;第二种中立的自耕农亦可参加革命。即第一种右倾的自耕农受了佃农及自耕农左翼的革命大潮所裹挟,也只得附和着革命。所以小资产阶级的自耕农是全部可以倾向革命的。


半自耕农半益农贫农这三种农民的数目,在中国农民中大概自一万五千万至一万七千万。分开来说半自耕农大概为五千万,半益农贫农各占六千万,乃农村中一个极大的群众。所谓农民问题,一大半就是他们的问题。这三种农民是同属半无产阶级,然经济状况大有分别。在半自耕农其生增苦于自耕农,因其食粮,每年有一半不够,须租别人田地,或者作工或营小商以资弥补。春夏之间,青黄不接,高利向别人借债,重价向别人籴粮,较之自耕农之不求于人,自然境遇要苦。然优于半益农。因半益农无土地,每年耕种只得收获之一半:半自耕农则租于别人的部分是只收获一半或者不是一半,然自有的部分却可全获。故半自耕农之革命性优于自耕农而不及半益农。

半益农与贫农都是乡村的佃农,同受地主的剥削,然经济地位颇有分别。半益农无土地,然有比较充足的农具及相当数目之流动资本。此等农人每年劳动结果自己可以得到一半。不足部分,种杂粮,捞鱼虾,饲鸡豕,免强维持其生活,于艰难竭蹶之中,存聊以卒岁之想。故其生活苦于半自耕农,然较贫农为优。其革命性则优于半日耕农而不及贫农。

贫农既无充足的农具,又无流动的资产。肥料不足,田亩歉收,送租以外,所得无几。荒时暴月,向亲友乞哀告怜,借得几斗几升,敷衍三日五日,债务丛集,如牛负重。乃农民中之极艰苦者,极易接受革命的宣传。

雇农乃农业的无产阶级,有长工,月工,零工三种。此等雇农,不仅无土地,无农具,又无丝毫流动资本,故只得营工度日。其劳动时间之长,工资之少,待遇之薄,职业之不安定,超过其他工人。此种人乡村中甚感痛苦者,做农民运动极要注意。乡村手工业工人地位比雇农要高,因其自有工具,且系一种自由职业。但因家庭负之重工资与生活物价之不相称,时有贫困的压迫与失业的恐慌,亦与雇农差不甚远。

游民无产阶级为帝国主义军阀地主阶级之剥削压迫及水旱天灾因而失了土地的农人与失了工作机会的手工业工人。分为兵、匪、盗、丐、娼妓。这五种人名目不同,社会看待他们贵贱各别,然他们之为一个“人”,他们之有五官四肢则一。他们谋生的方法,兵为“打”,匪为“扣”,盗为“偷”,丐为“讨”,娼妓为“媚”,名下相同,谋然生弄饭吃则一。他们乃人类中生活最不安定者。他们在各地都有秘密组织,如闽粤的三合会,湘鄂黔蜀的哥老会,皖豫鲁等省的大刀会,直隶及东三省的在理会,上海等处的青帮,做了他们政治和经济斗争的互助机关。处置这一批人乃中国最大最难的问题。中国有两个问题,一个是食,又一个是失业。故若解决了失业问题,就算解决了中国问题的一半。中国游民无产阶级人数说来吓人,大概在二千万以上。这一批人,很能勇敢奋斗,引导得法,可以变为一种革命力量。

我们组织农民乃系组织自耕农、半自益农、贫农、居农、及手工业工人五种农民于一个组织之下。对于地主阶级在原则上用斗争的方法,请他们在经济上在政治上让步,在特别情形上,即使遇了如海丰广宁等处最反动最凶恶极端鱼肉人民的土豪劣绅时,则须完全打倒他。对于游民无产阶级则劝他们帮助农民协会一边,加入革命的大运动,以求失业问题的解决,且不可迫其跑入敌人那一边,做了反革命派的力量。

<p align="right">(《中国农民》第一期,一九二六年一月一日)</p>

