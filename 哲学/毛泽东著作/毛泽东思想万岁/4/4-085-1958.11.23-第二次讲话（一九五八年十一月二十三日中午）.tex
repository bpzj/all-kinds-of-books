\section[第二次讲话(一九五八年十一月二十三日中午)]{第二次讲话(一九五八年十一月二十三日中午)}
\datesubtitle{(一九五八年十一月二十三日)}


(一)从写文章讲起。中央十二个部的同志写了十二个报告。请所有到会的中委,候补中委看一看,议一议,作些修改。文章我看了很高兴。路线还是那个路线,精神还是那个精神。但是指标的根据不充分。只讲可能,没有讲根据。讲四亿吨是可能的,为什么是可能

的?指标要切实研究一下。搞得要扎实些。电力的报告写得很好。是谁写的,李葆华,刘澜涛?刘澜涛不在。在座的没有电力一切事情搞不成。中委都要看一看。还可以发给十八个重点企业的党委书记、厂长,让他们都看一看,使他们有全局观点。有的文章修改后甚至可以在报上发表,让人民知道,这没有什么秘密,我说要压缩空气,不是减少空气,物质不灭。空气是那么多,只不过压缩一下而已,成为液体和固体状态。没有过关的问题,再搞清楚一些,说明什么时候可以过关。什么时候可以过去?明年三月四月五月,说出个理由和根据(比如,冶金设备的两头设备——采矿和轧钢设备还没有过关)。机械配套,为什么配不起来?究竟什么时候配得齐?有什么根据?与二把手商量一下。又如洋炉子土铁的技术,什么时候,用什么办法解决?又如电力不足怎么办?现在找到一条出路,就是自造、自建、自备电厂,工厂、矿山、机关、学校、部队都自搞电力,水、火、风、气(沼气)都利用起来,这是东北搞出来的名堂,各地是否采取同样的办法?解决多少?

是不是对十二个报告再议论两三天,然后再动手修改。补充根据主要要求切实可靠。把指标再修改一下。

现在我们兴了个规矩,一年抓四次,中央和地方在一起检查,共同商量。明年的事今年安排,一年的架子先搭起来。明年到了春夏秋冬各抓一次。今年南宁会议,成都会议,八大二次会议,北戴河会议,郑州会议和这次武昌会议,算是抓了六次。南宁会议是夸夸其谈,解决相互关系。成都会议就有具体东西了,解决一些具体问题。武昌会议是成都会议的继续。

(二)关于各省、市、自治区党委的同志写报告的问题。中央各部的同志写了十二个报告。各省市委的同志,你们一个也不写是不行的,要压一压。每人写一个是否可以?大家不言语。这次逼,可能逼死人。是不是下次每人写一篇。五、六千字或七、八千宇,片面性、全面性都可以,就是第一书记亲自动手,即使不动手,也要动脑。动口,修修补补。中央各部的报告是不是部长亲自动手写的啊?下次会,明年二月一日开,这些文章在一月二十五日前送到,以便审查,会上印发,在会场上可以讨论修改。各省要开党代会总结一下。问题太多了不行,搞一百个问题就没有人看了。去掉九十九个,写几个问题或一个问题,最多不超过十个问题。要有突出的地方。人有各个系统,(吸收系统、生殖系统…),地方工作也有许多系统,因此,有些可以不讲,有的要带几笔。有的要突出起来讲。

我们的路线,还是鼓足干劲,力争上游,多快好省地建设社会主义的路线。办法仍然是政治挂帅,群众路线,几个并举,加上土洋结合等等。

(三)谈一谈昨天晚上的问题。以钢为纲带动一切。钢的指标,究竟定多少为好?北戴河会议定为二千七百万吨至三千万吨,那是建议性的,这次要决定。钢二千七百万吨,我赞成,三千万吨,我也赞成,更多也好,问题是办到办不到,有没有根据?北戴河会议没有确定这个问题,因为没有成熟。去年五百三十五万吨,都是好钢,今年翻一番,一千○七十万钢,是冒险的计划。结果六千万人上阵,别的都让路,搞得很紧张。湖北有一个县,有一批猪运到襄阳专区,运不走,放下就走。襄阳有很多土特产和铁运不出,农民需要的工业品运不进。钢帅自己也不能过路。北戴河会议后,约三个月来的经验,对我们很有用。明年定为二千七百万吨至三千万吨,办不到。可不可以把指标降低?我主张明年不翻二番,只翻一番,搞二千二百万吨有无把握?前天晚上,我找李富春、×××、×××等几个同志研究,研究一千八百万吨有无把握。现在说的那些根据我还不能服。我已经是站在机会主义的立场,并为此奋斗。打我屁股与你们无关,无非是将来又搞个马鞍形。过去大家反我的冒进,今天我在这里不反人家的冒进。昨晚谈的似乎一千八百万吨是有把握的,这努力可以达到。不叫冒进。明年搞好钢一千八百万吨,今年一千一百万吨钢,只有八百五十万吨好的、八百五十万吨翻一番,是一千七百万吨,一千八百万吨钢比翻一番还多,这样说是机会主义吗?你说我是机会主义,马克思会为我辩护的,会说我不是机会主义。要他说了才算数,还说我是大冒进。不是大跃进我不服,一千八百万吨,我觉得还是根据不足,好些关未过。你们作文章,要说明什么时候过什么关,选矿之关,釆矿之关,破碎之关,冶炼之关,运输之关,质量之关。有的明年一月二月或三月四月五月六月才能过关。现在有的地方已无隔宿之粮(煤、铁、矿石),有些厂子因运输困难,目前搞得送不上饭,这是以钢为例,其他部门也都如此。有些关究竟何时能过,如果没有把握,还得下压,一千五百万吨也可以。有把握,即一千八百万吨,再有把握,二千二百万吨,再有把握,二千五百万吨,三千万吨,我都赞成。问题在于有无把握,昨天同志们赞成一千八百万吨,说是有把握的。东北去年三百五十万吨左右,今年原定六百万吨,完成五百万吨。明年只准备搞七百一十五万吨,又说经过努力可以搞八百万吨。我看要讲机会主义,他才是机会主义。可是在苏联,他是要得势的,因为今年只有五百万吨,明年八百万吨,增加了百分之六十嘛,增加了半倍多,是半机会主义。华北去年只有六十万吨,今年一百五十万吨,明年打算四百万吨,今年增加一点七倍。这是马列主义。明年打算四百万吨,这是几个马列主义了,你办得了吗?你把根据讲出来,为什么明年搞这么多?华东去年二十二万吨,今年一百二十万吨,(加上坏钢是一百六十万吨),明年四百万吨,增加二倍多。上海真正是无产阶级,一无煤,二无铁,只有五万人。华中去年十七万吨,今年五十万吨,明年二百万吨,增加三倍。此人原先气魄很大,打算搞三百万吨,只要大家努力,过那些关,能成功。无人反对,并且开庆祝会。西南去年二十万吨,今年七十万吨,明年二百万吨,增加两倍。西北去年只有一万四千吨,比蒋介石少一点,今年五万吨,超过蒋介石,明年七十万吨,增加了十三倍,这里头有机会主义吗?华南去年两千吨,今年六万吨,增加三十倍,马克思主义越到南方越高,明年六十万吨,增长十倍。

这些数字。还要核实一下,要各有根据,请富春同志核实一下,今年多少,明年多少,不是冒叫一声。这些数字,无非证明并非机会主义,没有开除党籍的危险,各地合计,明年是二千一百三十万吨,问题是.是否能确实办到。要搞许多保险系数,一千八百万吨作为第一本账,在人民代表大会通过,确实为此奋斗.还要做思想准备。如果只能搞到一千五百万吨好钢。另外有三百万吨土钢,我也满意。

第一本账,一千八百万吨,第二本账,两千二百万吨,以此为例,各部门的指标,都要相应地减下来。例如发电,搞小土群,可以自发自用。强迫命令,已搞的,要采取不发饷的办法。又如铁路,原定五年只搞两万公里,现在几年就搞两万公里。需要是需要,但能不能搞这么多?(×××。明年第一季度,只有二百九十万吨钢材,加上进口,不过三百万吨.不够分配。开口要三万吨,只能给一万吨。)吕正操,没有钢怎么办,(吕:可以搞球墨铸铁。)成都会议是五年二万公里,现在一九五八年就搞了二万公里,×××的气魄很大。我很高兴,问题是能不能办到,有没有把握。要找出根据,你有什么办法?(他的办法就是要各地自己造。)(×××:几个月我们都是见物不见人,看看部的报告,吓一跳,写不出来。)有矛盾。×××,你真是思想解决了置中央可以夸海口,担子则压在地方身上。(××:任务是第三本账,武钢要七万五千吨,共十五万五千吨,而中央只给七万吨,所以那些项目是建不成的。)不给米,巧妇难做无米之炊。农民就有各种办法抵制我们。例如,区上为填表报,专设一个假报员,专门填写表报。因为上面一定要报,而且报少了不像样子。一路报上去,上面信以为真,其实根本没有。我看现在不少这样的问题。今年究竟有不有八百五十万吨好钢,是真有还是报上来的,没有假的吗,调不上来的就是虚假。我看实际没有这样多。(×××、×××:好钢不敢虚报,小土群靠不住。)

(四)作假问题。郑州会议的公社问题决议要改为指示,要把作假问题专写一条。原有两句,两句不够。要专搞一条。放在工作方法一起,人家不注意。现在横竖要放“卫星”,争名誉,管他作假不作假。没那么多东西,就要假造。有一个社,自己只有一百条猪,为了应付参观,借两百条大猪,看后送回。有一百条就是一百,没有就没有,搞假干什么?过去打仗出捷报,讲俘虏多少,也有这样的事。虚报成绩,以壮声势。老百姓看了舒服,敌人看了好笑。后来我们反对,三令五申,多次教育要老实,才不敢作假了。其实,就那么老实?人心不齐,我看还是有点假,世界上的人有的就不那么老实。建议跟县委书记、公社党委书记切实谈一下,要老老实实,不要作假。本来不行,就是人家写,脸上无光也不要紧,不要去争虚荣。如扫盲,说什么半年、一年扫光,两年扫光我都不太相信,第二个五年扫除了就不错.还有绿化,年年化,年年没有化,越化越见不到树,说消灭了四害,是“四无村”实际上是“四有村”。上面有任务,他总说完成了。世界上的事,没有一项没有假,有真必有假。没有假的比较,那有真的?这是人之常情。现在的严重问题.不仅是下面作假,而且是我们相信,从中央、省、地到县都相信,主要是前三级相信,这就危险。如事事不相信,那就成机会主义了。群众确实作出了成绩,为什么要抹杀群众的成绩?但相信作假也要犯错误。比如一千一百万吨钢,就说一万吨也没有,那当然不对了。但是真有那么多吗?又如粮食,究竟有多少了×××,你是元帅,算了账的。有人说九千亿斤,究竟有没有,(×××:七千五百亿斤到八千亿斤,差不多。)(×××,说九千亿斤,已经打了七折。)(李先念:七千五百亿斤是有的。)去年三千七百亿斤,今年七千五百亿斤,就翻了番,那就了不起。丢掉不要紧,物质不灭,变了肥料。农民很爱惜,听说又收第二道。

要比,结果就造假,不比,那就不竞赛。要订个竞赛办法。要查,要检验,要像出口物质那样检验,用显微镜照。一斤粮,含水量多少?有多少虫子,不合规格不行。经济事业要越搞越细,越深入,越实际,越科学。这个东西跟作诗是两间事,要懂得作诗和办经济事业的区别。“端起巢湖当水飘”,这是诗。我没有端过,大概你们安徽人端过。怎么端得起来?检查也要注意作风,也要估计里头有假,有些假你查也查不出来,开个会,就布置好了,希望中央、省、地三级要懂得这个道理,要有清醒头脑。现在一般来说,对于报的成绩,打个折扣。三七开,十分中打三分假七分真可不可以,这是否对成绩估计不足?对于干部、群众不信任?要有一部分不信任,至少不少于一成假,有的是百分之百的假。有时候事情还没有办,他说办好了。(江渭清:群众知道。)你讲一县一省的,群众只知道本村的。这是不好的造假。另一种是优良的造假,值得高兴的造假。比如瞒产,是个矛盾,有好处。干部要多报,老百姓要瞒产。奸就好在这点上。有些地方吃了亏,报多了。上面要得多,他说没有了。再有一种假,也是造得好的,是对主观主义、强迫命令的。中南海有个下放的干部,写信回来说,合作社规定拔掉三百亩苞谷,种红薯,每亩红薯要种一百五十万株,而包谷已长到人头高,群众觉得可惜,所以不拔。只拔了三十亩,而报了三百亩。这种假报是好的。×××说他的家乡年初一浇麦子,不让休息。老百姓有什么办法,只得作假,夜间在地里打了灯笼,实际上人在家里休息。干部看到遍地灯光,认为没有休息。湖北省有一个县,要日夜苦战。夜间不睡觉,但群众要睡觉,让小孩子放哨,看见干部来了,大家起来哄哄,干部走了又睡觉。这也是好的造假。总之这样的事。我看不少。一要有清醒的头脑,一要进行教育,不要受骗,不要强迫命令。现在有种空气,只讲成绩,不讲缺点,有缺点脸上无光,讲实话没人听。讲牛尾巴长在屁股后面,没人听,讲长在头上,就是新闻了。造假,讲得多,有光彩。要教育。讲清楚,要老老实实,几年之内做到就好。我看经过若干年.走上轨道。就可以比较踏实。

正在建的,已经建的钢铁重点,列个表,那一省多少。多少数量。我想把我们过去想的,回头再提一下,也许机会主义,过去想。明年三千万吨,后年六千万吨钢,六一至六二年达到一亿吨。现在回来想,假若明年只搞一千八百万吨,后年三千万吨,苦战三年,超过西德,变成世界第三位,那就很好。六一年六二年每年多少?如果每年增加一千万吨,第二个五年计划就达到五千万吨,或者五千三百五十万吨,比一九五七年增加十倍,还能叫机会主义吗?如果马克思还要骂我们机会主义,我们就不承认他了。需要和可能,需要是个问题,可能也是个问题。五年计划要做几个方案,三千万吨还不能作第一个方案,要看明年的结果。假如大家努力,领导正确,破除迷信,土洋结合,大中小结合,鼓足干劲,五年能达到三千万吨就很好了,如果超过一点就更好了。……整个说来,技术关、什么关明年一年我看过不了,至少要一年,如果都过了关,当然很好。机床,第一个五年计划二十万台,今年八万台,明年十八万台到二十万台,后年二十五万台到三十万台,就是把原定的明年计划推迟一年,苦战三年,总数达到八十万台,超过日本。一九六一年六二年再搞六十万台,可以达到一百四十万台,就是由二十六万台增加到一百五十万台,那就很好了。如果钢只有五千万吨,不要一百五十万台,有一百一十万台就差不多了。

钢材的分配要有一个排队,机器制造第一位(其中工作母机第一位,机器设备第二位)铁路交通运输第二位,农业第三位。

这种设想,把盘子放低一些,很有必要的。两个五年加三年达五千多万吨。我们十三年,相当于苏联的四十年,他到一九三九年,二十年只搞一千八百元吨钢。我们五千万吨钢和一百一十万台机床,这就大为优胜,其原因:1)大国,人口多,2)三个并举,党的路线,3)苏联经验。没有第三条是不行的。它二十年搞一千八百万吨,我们十三年搞五千万吨。这样一想还是划得来。机会主义有一点,也不多,可能比较切实一些。

农业快得很,明年再搞一年。就粮食而论,搞到一万五千亿斤,农民就可以休息了。就可以放一年假。粮食多了吃不完,棉花当然不行。(××:农业有个政策问题,粮食每人搞到一千五百斤到二千斤,还不够吗?为巩固公社,要搞些能交换的东西,重点就可以放到经济作物方面来,可以多搞一些商品。)(曾希圣:我们躭心农作物的出路问题。)(××:油料作物有出路。)对。河北从一千一百斤搞到一千五百斤。粤、赣、皖从一千五百斤搞到两千就行了。经济作物要订合同,就在这次会议上订,我们这个会上就作生意。中央、省,县、乡要订四级合同,全国各省要分工,竹、木、丝、茶、油、麻搞多了是没问题的。

(五)破除迷信,不要把科学当作迷信破除了。比如,人是要吃饭的,这是科学,不能废除。没有人证明可以不吃饭。“张会辟谷”,但吃肉。现在不放手让群众吃,大概是报多了。搞七、八千亿人斤.还不愿意人家吃的多.可能就是报的多了。吃不饱饭的,就没有跃进。人是要睡觉的,这也是科学。动物总是要休息,细菌也是要休息。人的心脏一分钟跳七十二次,一天跳十万多次。一不吃饭,二不睡觉,破除这两条就要死人。此外。还有不少迷信在那里破除,破的结果,人被机器压死。人去压迫自然界。拿上工具作用于生产对象。自然这个对象再来个抵抗,反作用一下,这是一条科学。人在地球上走路,地球有个反抗。就不能走路。过草地不太抵抗,不好走,泥内陷下去拔不起来,这种田要渗沙土。自然界有个抵抗力,这是一条科学。你不承认,他就要把你整伤砸死。破除迷信一来,效力极大,敢想敢说敢做。但有一小部分破得过分了,把科学真理也破了。这是不能破的。如说睡觉一小时就够了。方针是破除迷信,但科学是不能破的。

成绩与虚报,要有个估计。到底有多少?要议一下,三七还是二八,可带回去与地委、县委同志研究一下。把假的估计多了,不相信群众,要犯错误,要泄气,不估计到假也要犯错误。这是说一般。就个别说来,有全部是真的,也有全部是假的。例如扫盲,除四害,文盲成堆也说扫除了,根本没有绿化,报绿化了,四有报四无,如此类推。加以分析,凡迷信一定要破除,凡真理一定要保护,资产阶级法权只能破除一部分,如三风五气,过分悬殊,老爷态度,旧关系,一定要破除,越彻底越好。另一部分。如工资等级,上下级关系,国家一定的强制,还不能破。六千万人上阵,阜阳五万人口,无煤无铁,还不是听共产党的话没错。命令六千万人搞钢是有强制性的,是北戴河会议、四次电话会议逼上梁山的。这种强制性,强制分配劳动,在现在还不能没有。如果自由报告,自由找职业,谁愿意钓鱼就钓鱼,画画就画画,唱歌就唱歌,跳舞就跳舞,如果一亿人唱歌,一亿人画画,还会有粮食啊?那就要灭亡了。资产阶级法权一部分要破,有一部分在社会主义还是有用的,必须保护,使之为社会主义服务。把它们打的体无完肤(像过去内战时期肃反一样,捉了好人,打得一身烂),会有错,我们要陷于被动,要承认错误。对有用的部分,你打烂了,搞错了,还要道歉,还要扶起来。要有分析,那些有用,那些要破除。苏联应破者未破,还相当顽固。我们应该破者破,有用的部分保护。

(六)四十条。这次不搞为好,现在没有根据,不好议。

(七)、谁先进入共产主义?苏联先进入还是我国先进入?赫鲁晓夫提出在十二年内是准备进入共产主义的条件,他们很谨慎,我们在这个问题上,也要谨慎一些。有人说,两、三年,三、四年、五年、七年进入共产主义,是否可能?要进,鞍钢先进,辽宁后进(他二千四百万人中有八百万在城市),而不是别省。再其次是老柯、上海,如果他们还要等待别人,不能算独进。×××,秦张、范县就要进,那不太快了吗?派了陈伯达同志去调查,说难于进。现在专区、省还没有人说先进,想谨慎,就是县有些打先锋的,整个中国进入共产主义。要多少时间,现在谁也不知道,难以设想,十年,十五年?二十年?三十年?苏联四十一年,再加上十二年共五十三年,还说是准备条件。中国就那么厉害,我们还只有九年,就起野心,这可能不可能?从全世界无产阶级利益考虑,也是苏联先进为好,也许在巴黎公社百年纪念时(一九七一年)苏联进入共产主义,我们十二年怎么样?也许可能,我看不可能,即或十年到一九六八年我们已经准备好。也不进。至少等苏联进入二、三年后再进,免得列宁的党,十月革命的国家脸上无光,本来可进而不进,也是可以的,有这么多本领。又不宣布。又不登报说进入共产主义,这不是有意作假吗?这不要紧。有许多人想,中国可能先进入,因为我们找到人民公社这条路,这里有个不可能,也有个不应该,(××:吃薯怎么进入共产主义。)一块钱工资怎么进入?这些问题不好公开讨论,但这些思想要在党内讲清楚。


