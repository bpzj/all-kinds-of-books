\section[相持阶段中的形势与任务(一九四〇年二月)]{相持阶段中的形势与任务(一九四〇年二月)}


<h3 style="text-align: center">第四节共产党的十大任务</h3>

<p align="center">(一九四○年二月)</p>


我们要坚决反对顽固派.我们要坚决团结进步派,我们要用这样的政策去达到争取时局好转,打胜日本强的总目的。为了这个,我们要作许多工作,不久几天,共产党中央开了一个会决定了争取时局好转的十个任务,那十大任务呢?我现在向同志们宣读一下共产党中央的决议案。

为了力争时局的好转,克服逆流危险,必须强调抗战、团结、进步,三者不可缺一,并在这个基础上坚决执行下列的十大任务。

第一便要普遍扩大反汪反汉奸的宣传,坚决揭穿一切分裂投降的阴谋,从思想上政治上打击投降派与反共派。坚决说明与具体地证明反共是投降派准备投降的反革命步骤。

第二便要猛力发展全国党、政、军、民、学各方面的统一战线,组织进步势力,同国民党的大多数亲密合作,用以对抗投降派与反共派。

第三便要广泛开展宪政运动,力争民主政治。没有民主政治,抗日胜利只是幻想。

第四便要抵抗一切投降反共势力的进攻,对任何投降派反共派顽固派的进攻,均须在自卫的原则下,在人不犯我我不犯人的,人若犯我我必犯人的原则下,坚决反抗之,否则任其猖獗,统一战线就会破裂,抗日战争就要失败。

第五便要大大发展抗日的民众运动,团结一切抗日的知识分子,并使知识分子与抗日的民众运动,抗日游击战争相结合,否则就没有力量打击投降派、反共派与顽固派。

第六便要认真实行减租、减息、减税与改善工人生活,给民众以经济上的援助,才能发动民众的抗日积极性,否则是不可能的。

第七便要巩固与扩大各抗日根据地,在这些根据地上建设完全民选的没有任何投降反共分子参加的抗日民主政权,这种政权不是工农小资产阶级的政权,而是一切赞成抗日又赞成民主的人的政权,是抗日民族统一战线的政权,是几个革命阶级联合的民主专政,对一切破坏抗日根据地的阴谋,必须加以坚决打击,对一切暗藏在抗日政权抗日团体中的汉奸反共分子,必须加以肃清。

第八便要巩固与扩大进步的军队,没有这种军队,中国就会亡国。

第九便要广泛抗日文化运动,提高抗日人民、抗日军队与抗日干部的文化水平与理论水平,没有抗日文化战线上的斗争以与总的抗日斗争相配合,抗日也是不能胜利。

第十便要巩固共产党的组织,在无党或党弱的地方要发展党的组织,没有一个强大的共产党,就不能解决抗日救国的任何重要问题。

如果能解决的并具体正确的执行上述十大任务,就一定能够强固抗日进步力量,克服投降倒退力量,争取时局好转,避免时局逆转。击破大资产阶级分子破坏抗战与破坏统一战线的阴谋。

同志们,这就是共产党中央最近决定的十大任务,你觉得这十大任务好不好呢?这是对症下药的十大任务。这是起死回生的十大任务,但是做起来不能单是共产党来做,还要全国人民来做。当然这个十大任务,日本帝国主义是不喜欢的,汪精卫是不喜欢的;反共派、顽固派是不喜欢的。他们不喜欢是应该的,我们没有必要让他们的喜欢,他们不喜欢也就罢了,我们今天开大会,我们要向全国人民各党各派叫出我们的吼声,我们要挽救危难中的祖国,我们要创造一个新中国我们要赶跑日本帝国主义,我们要打倒汪精卫,我们要打倒那批反共“英雄”,我们要高呼下列口号:

拥护抗战到底的政策,反对汪精卫的卖国协定,

全国人民团结起来,拥护蒋委员长,打倒汉奸汪精卫,

拥护国民政府,打倒汪精卫的中央,

拥护国共合作,打倒汪精卫的反共政府,

反共就是汪精卫分裂统一战线的阴谋,打倒一切反共的汉奸,

加紧全国的团结,消灭内部磨擦,

革新内政,开展宪政运动,树立民主政治,

开放党禁,允许抗日党派的合法存在权.

人民有抗日救国的言论、出版、集会、结社自由权,

发展民众运动,实行减租、减息、减税、改善工人生活,

巩固抗日根据地,反对汉奸反共派与顽固派的阴谋破坏,

拥护抗日有功的军队,充分接济前线,

发展抗日文化,保护进步青年,取缔汉奸言论,

中华民族解放万岁!

