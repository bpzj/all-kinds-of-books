\section[中国社会各阶级的分析(一九二六年)]{中国社会各阶级的分析}
\datesubtitle{(一九二六年)}


谁是我们的敌人?谁是我们的朋友?分不清敌人与朋友,必不是个革命分子,要分清敌人与朋友呢,却并不是立场。中国革命且三十年而成效甚少,并不是目的错、完全是策略错。所谓策略错,就是不能团结真正的朋友,以攻击真正的敌人。所以不能如此,乃是未分清谁是敌人谁是朋友。革命党是群众的向导。在军队中,未有他的响导领错了路而可以打胜仗,在革命运动中,未有革命党领错了路而这个革命可以不失败。我们都是革命党,都是给群众领路的人,都是群众的向导。但我们不可不自问一问:我们有这个本领没有?我们不至于领了群众到失败的路上去么?我们可以一定完成功吗?要有:“不错了路”和“一定成功”的把握不可不致谨于一个重要的策略,要决定这个策略,就要分清谁是敌人谁是朋友。国民党第一次全国大会宣言,就是宣告这个策略的决定和敌友的分辨。但这个宣言极其简单。我们要认识这重要的策略,要分辨那真正的敌友,不可不将中国社会各阶级的经济地位,阶级性,人数及其对于革命的态度,作一个大概的分析。

无论那一个国内,大概地说,都有三等人,上等、中等、下等。详细分析则有五等,大资产阶级、中产阶级、小资产阶级、半无产阶级、无产阶级。拿农村说:大地主是大资产阶级,小,地主是中产阶级,自耕农是小资产阶级,半自耕佃农是半无产阶级,雇农是无产阶级。拿都市说:大银行家、大商业家、大工业家是大资产阶级、钱庄主、中等商人、小工厂主是中产阶级,小商人、手工业主是小资产阶级,店员小贩手工业工人是半无产阶级,产业工人苦力是无产阶级。五种人各有不同的经济地位,各有不同的阶级性。因此对于现代的革命,乃发生反革命,半反革命,对革命守中立,参加革命和向革命的主力军种种不同的态度。

小国各阶级对于民族革命的态度,与西欧资本主义国家的各阶级对社会革命的态度,几乎完全一样。看来好似奇怪,实际并不奇怪。因为现代的革命本是一个,其目的与手段均相同,即同以打倒国际资本帝国主义为目的。同以被压迫民族被压迫阶级联合作战为手段。这是现代革命异于历史上一切革命之最大的特点。

我们试看中国社会各阶级是怎么样。


第一、大资产阶级

经济落后半殖民地的中国,大资产阶级完全是国际资产阶级的附庸。其生存和发展的条件,即附属于帝国主义。如买办阶级――与外资有密切关系之银行家(陆宋与东兼伯等)商业家(如唐绍仪何东等),工业瘃(如张寒盛恩照等)。

大地主(如张作霖陈恭受等)。

官僚(如孙宝琦颜惠庆等)。

军阀(如张作霖曹锡等)。

反动派知识阶级一一上列四种人之附属物,如买办性质的银行工商业高等员司,财阀,政府之高等事务员,政客,一部分在西洋留学生,一部分大学枝专门学校的教授和学生,大律师等都是这一类。

这一个阶级与民族革命之目的完全不相容,始终站在帝国主义一边,乃极端的反革命派。其人数大概不出一百万即四万万人中四百分之一,乃民族革命运动中之死敌。

第二、中产阶级

华资银行工商阶级(因在经济落后的中国,本国资本银行工商业的发展尚限在中产阶级地位。所谓银行业乃指小银行及钱庄,工业乃指小规模工厂,商业乃指国货商。凡是大规模银行工商业无不与外国资本有关系,只能算入买办阶级内。)

小地主

许多高等知识分子一华商银行工商业之从业员,大部分东西洋留学生,大部分大学校专门学校教授和学生,小律师等都是这一类。这个阶级的欲望为欲达到大资产阶级的地位,然受外资打击军阀压迫不能发展。这个阶级对于民族革命历取了矛盾的态度。即其受外资打击,军阀压迫感受痛苦时,需要革命,赞成反帝国主义反军阀的革命运动。但国内现在的革命运动,在国内有本国无产阶级的勇猛参加,在国外有国际无产阶级的积极援助,对于其欲达到大资产阶级地位的阶级的发展及存在,感觉着威胁又怀疑革命。这个阶级即所谓民族资产阶级,其政治的主张为国家主义一一实现民族资产阶级一阶级统治的国家。有一个戴季陶的,“真实信徒”(其自称如此)在北京晨报上发表议论说:“举起你的左手打倒帝国主义!举起你的右手打倒共产党!”乃活画出这个阶级的矛盾惶遽态度。他们反对以阶级斗争说解释民主主义,反对国民党联俄及容纳共产派分子。但这个阶级的企图一一实现民族资产阶级统治的国家,是完全不行,因为现在世界上局面,乃革命反革命两大势力作最后争斗的局面,这两大势力竖起两面大旗:一面是赤色的革命的大旗,第三国际高举着,号召全世界被压迫民族被压迫阶级集合子其旗帜之下,站在一边;一面是白色的反革命大旗,国际联盟高举着,号召全世界反革命分子都集于其旗帜之下,站在另一边。那些中间阶级,在西洋如所谓第二国际等类,在中国如所谓国家主义派等类,必须赶快的分化,或者向左跑入革命派或者向右跑入反革命派,没有他们“独立”革命思想,仅仅是个幻想。他们现在虽还站在半反革命的地位,他们现在虽然还不是我们正面的敌人,但到他们感觉工农阶级的威胁日甚时,即是为了工农阶级的利益迫他们让步稍多时(如农村中的减租运动,都市中的罢工运动)他们或他们的一部分(中产阶级右翼)一定会站入帝国主义一边,一定变为完全的反革命,一定要成为我们正面的敌人。本来买办阶级与非买办阶级,有一部是未能截然划分清楚的。以商业论,固然许多商人是洋货商土贷商划分得很清楚,但是在有些商店的店门内,是一部分摆设着土货,一部分又摆设着洋货。以知识阶级论,以小地主子弟的资格赴东洋资本主义国家读书的留学生,固然是明显的除了半身土气之外,又带上了半身洋气。即从小地主子弟的资格在国内专门学校大学校读书受着那半土半洋回国留学生的熏陶,仍然不免是些半身土气半身洋气的脚色。在这类人并不是纯民族的资产阶级性质,可以叫他们做“半民族资产阶级”。这种乃是中产阶级右翼,只要国民革命的斗争加深;这种人一定很快地跑入帝国主义军阀的队伍里,和买办阶级做着(资料不清)。中产险级左翼即与帝国主义完全无缘者,此派在某种情况(如抵制外国朝流高潮时)有革命性。及其死持之空虚的“和平”观念极破,而且对于所谓“赤化”时时怀着恐慌。故其对于反革命极为妥协,不能持久,故中国的中产阶级,无论其右翼,及其左翼,也包括许多危险成分,断不能望其勇敢地跑上革命的路,跟着其余的阶级忠实地做革命事业,除开少数历史和环境都有特别情况的人。中产阶级的人,在全国内至多每百个人里有一个(百分之一)即四百万人。

第三小资产阶级

如:自耕农、小商、手工业主、小知识阶级――小员司、小事务员、中学学生、及中小学教员、小律师等都属于这类。这一个阶级,在人数上,在阶级性上,都值得大大注意。小资产的人数,单是自耕农就有一万万至一万万二千万,小商人手工业主,知识阶级,大概自二千万至三千万,合计达到一万万三千万,这个阶级虽然同在小资产阶级之经济地位,但实有三个不同的部分。第一部分:是有余钱剩米的,即用其体力或脑力劳动所得,除自给外,每年还有余剩,用以造成所谓资本的初步积累。这种人“发财”观念极重,虽不妄想发大财。却总想爬上那中产阶级地位。他们看只那些受人尊敬的小财东,往往垂着一尺长涎水,对于赵公元帅(俗财神)礼拜最劝。这种人胆子极小,他们怕官,也有点革命。因为他们的经济地位与中产阶级颇接近,故对于中产阶级的宣传颇相信,对于革命取怀疑的态度。但是这一部分人在小资产中占少数,大概不及小资产阶级全数百分之十约一千五百万,乃小一部资产阶级的右翼。第二部分是恰足自给的,每年收支恰足相抵,不多不少,这部分人比较第一部分人大不相同,他们也想发财,但是赵公元帅总不让他们岁财,随着近年帝国主义军阀大中资产阶级的压迫和剥削,使他们感觉现在的世界已经不是从前的世界。他们感觉现在如果只使用从前相等的劳力,就会不能维持生活。必须增加劳动时间,即每天起早晨,对手职业加倍注意,才能维持生活。他们有点骂人了,他们骂人叫“洋鬼子”,骂军阀叫“抢钱司令”,骂土豪劣绅叫“为富不仁”。对于反帝国主义反军阀的运动,仅怀疑其未能成功,理由是“洋人和司令的来头那么大”,不肯贸然参加,取了中立的态度,但绝不反对革命。这一部分人数甚多,大概小资产的一半:“十分之五”即七千五百万。第三部分是每年要亏本的。这一部分人,好些大概原先本是所谓殷实人家,渐渐变的仅仅保守,渐渐的要亏本了。他们每逢年终结账一次,就吃惊一次,说“咳,又亏了!”这种人因为他们从前过着次日子,后来逐年下降,负债渐多,渐次过着凄凉的日子,“瞻念前途,不寒而栗”。这种人,在精神上感觉的痛苦比一切人大,因为他们有一个从前与现在相反的比较。这种人在革命运动中颇要紧,颇有推动革命的力量。其人数占小资产阶级中百分之四十即六千万一个不小的群众,乃小资产阶级的左翼。以上说小资产阶级的三部分对于革命的态度在平时各不相同,但到战时即革命潮流高涨可以看得见胜利的曙光时,不但小资产阶级的左派参加革命,中派亦可以参加革命,即左派分子受了无产阶级及小资产阶级左派的革命大潮所裹挟,也只得附着革命。我们从工业运动和历年来各地农民运动的经验看来,这个鉴定是不错的。

第四、半无产阶级。此处所谓半无产阶级,乃包含:

(1)半自耕农(2)半益农(3)贫农(4)手业工工人(5)店员(6)小贩,之六种。半自耕农的数目,在中国农民中大概占五千万,半益农贫农大概各占六千万,三种共计一万万七千万,乃农村中一个颇大群众,所谓农民问题,一大半就是他们的问题。这三种农民有同属半无产阶级,然从经济状况仍有上中下三个细别:在半自耕农其生活苦于自耕农,因其食粮每年有一半不够,须租种别人田地或者作工或营小商以资弥补。春夏之间,青黄不接虽高利向别人借贷,重价向别人籴粮,较之自耕农之不求于人,自然说过要苦,然优于半益农,因半益农无土地,每年耕种只得收获之一半,半自耕农则租于别人的部分虽只收获一半,或且不足一半,然自有的部分都可全获,故半自耕农之革命性优于自耕农而不及半益农。半益农与贫农都是乡村的佃农,同受地主的剥削,然经济地位颇有分别,半益农无土地,然有比较充足之农具及相当数目的流动资本,此种农人,每年劳动结果自己可以得到一半,不足部分,种杂粮,捞鱼虾,饲鸡豕,勉强维持其生活,于艰难竭蹶之中,存聊以卒岁,想,故其生活苦于半自耕农,然较贫农为优,其革命性则优于半自耕农而不及贫农。贫农之无充足的农具,又无流动的资本,肥料不足,田亩歉收,送租之外,所得无几,荒时暴月向亲友乞哀告怜,借得几斗几升敷衍三日五日,债务丛集,如牛负重,乃农民中之极艰苦者,极易接近革命的宣传。手工业工人所以称以半无产阶级,因其自有工具,且系一种自由职业,其经济地位略与农业中半益农相当,因家庭负担之重,工资与生活物价之不相称,时有贫困的压迫或失业的恐慌,与半益农亦大致相当。店员为中小商人的雇员,以微簿的薪资,供事人畜的费用,物价年年增涨,而薪例往往须数年一更,偶与此辈倾谈,便见叫苦不迭,其地位与手工业工人不相上下,对于革命宣传极易接受。小贩不论是肩挑叫卖或于畔摊售,总之本小利微吃着不够,其地位与贫农不相上下,其需要一个变更现代的革命也和贫农相同。手工业工人人数,大概占全人口百分之六即二千四百万人,店员大概有五百万,小贩大概有一百万,合起半自耕农,半益农、佃农人数半无产阶级人数共计约二万万占全国人口之一半。

第五、无产阶级。其种类及人数如左,

工业无产阶级约二百万;

都市苦力约三百万;

农业无产阶级约二千万;

共约四千五百万。中国因经济落后,故产业工人(工业无产阶级)不多,二百万产业工人之中,主要为铁路、矿山、海运、纺织、造船五种产业,而大多数在外资产业之下,故工业无产阶级虽不多,却做了民族革命运动的主力。我们看四年以来的罢工运动,如海员罢工、铁路罢工,开源及焦作煤矿罢工,及五四后,上海、香港两处之大罢工所表现的力量,就可知工业无产阶级在民族革命中所处地位的重要。他们所以能如此,第一个原因是集中,无论那种人都不如他们“有组织的集中”,第二个原因,是经济地位低下。他们失去了工具,剩用两只手,绝了发财的望,又受着帝国主义、军阀买办阶级极惨酷的待遇,所以他们特别能奋斗。都市苦力的力量也很可注意,以码头搬运夫以人力车夫占大多数,粪夫清道夫等都属于这一类。他们除了一双手外,别无长物,其经济地位与产业工人相似,惟不及其有组织的集中及在生产力上的重要。中国尚少新式资本主义的农业,所谓农业无产阶级,乃指长工、月工、零工等雇农而言。此等雇农,不仅无土地,无农具,又无丝毫流动资本,故只得营工度日。其劳动时间之长,工资之少,待遇之薄,职业之不安定,超过其他工人,此种人在乡村中乃最感困难者,在农民运动中,与贫农处于同一要紧的地位。游民无产阶级为失了土地的农民与失了工作的机会的手工业工人,其人数在二千万以上,乃国内兵争匪祸的根源。此游民无产阶级中最多者为匪,其次为兵,次为乞丐,次为盗贼与娼妓。他们乃人类生活中最不安定者,他们在各地都有秘密的组织,如闽粤的“三合会”,湘、鄂、黔、蜀的“哥老会”,皖、豫、鲁等省的“大刀会”,直隶及东三省的“在理会”,上海等处的青红帮,都做了他们政治和经济争斗的互助机关。处置这一批人乃中国最大最难的问题,一个是贫乏,又一个是失业,故若解决了失业问题,就算是解决了中国问题的一半。这一批人很能勇敢奋斗,引导得法可以变成一种革命力量。

谁是敌人谁是朋友,我们现在可以答复了,一切勾结帝国主义的军阀官僚,买办阶级,大地主,反动的知识阶级即所谓中国大资产阶级,乃是我们的敌人,乃是我们真正的敌人;一切小资产阶级、半无产阶级、无产阶级乃是我们的朋友,乃是我们真正的朋友;那动摇不定的中产阶级,其右翼应该把他当做我们的敌人,即使时非敌人也去敌人不远,其左翼可以把他们当做我们的朋友,但不是真正的朋友,我们要时常提防他,不要让他乱了我们的阵线,我们的真正敌人有多少?有四百万!让这四百万算做敌人,也不枉妄他们有一个五百万人的团体,依然抵不住三万万九千五百万人的一铺唾沫。三万万九千五百万人团结起来!

<p align="right">(“农民运动参考书”)</p>

