\section[对中央首长的讲话(一九六六年七月)]{对中央首长的讲话}
\datesubtitle{(一九六六年七月)}


主席说:五月二十五日聂元梓大字报是二十世纪六十年代的中国巴黎公社的宣言书。意义超过巴黎公社。这种大字报我们写不出来。

(几个少先队员给他爸爸贴大字报说爸爸忘了过去,没有给我们讲毛泽东思想,而是问我们在学校的分数,好的给奖品。)

主席叫陈伯达同志转告这些小朋友,大字报写得很好!主席说:我向大家讲,青年是文化大革命的大军!要把他们充分发挥出来。

回到北京后,感到很难过,冷冷清清,有些学校大门都关起来了,甚至镇压学生运动。谁镇压学生运动?只有北洋军阀!共产党怕学生运动是反马克思主义。有人天天说走群众路线,为人民服务,而实际上是走资产阶级路线,为资产阶级服务。团中央应该站在学生这边,可是他站在镇压同学那边,谁反对文化大革命?美帝、苏修、日本反动派。

借“内外有别”是怕革命。大字报贴出来,又盖起来,这样的情况不能允许,这是方向错误,赶快扭转。把一切框框打个稀巴烂!

我们相信群众,做群众的学生,才能当群众的先生,现在这个文化大革命是个惊天动地的事情。能不能,敢不敢过社会主义这一关?这一关是最后消灭阶级,缩短三大差别。

反对,特别是资产阶级“权威”思想,这就是破,如果没有这个破,社会主义就立不起来,要做到一斗、二批、三改也是不可能的。坐办公室听汇报不行,只有依靠群众,相信群众,闹到底,准备革命革到自己头上来。党政领导,党员负责同志,应当有这个准备。现在要把革命闹到底,从这方面锻炼自己,改造自己,这样才能赶上,不然就只能靠在外面。

有些同志斗别人很凶,斗自己不行,这样自己永远过不了关。

靠你们自己引火烧身,煽风点火,敢不敢?因为是烧到自己头上,同志们这样回答,准备好,不行就自己罢自己的官,生为共产党员,死为共产党员,坐沙发吹风扇生活不行。

给群众定框框不行,北京大学看到学生起来,定框框,美其名曰:“纳入正轨”,其实是纳入邪轨。

有些学校给学生戴“反革命分子”的帽子。(外办的张彦跑到外面给人扣了二十几个反革命帽子)这样就把群众放到对立面去了。不怕坏人,究竟坏人有多少?广大的学生大多数是好人。

(有人提出,乱的时候,打乱档案怎么办?)

怕什么?坏人来证明是坏人,好人你怕什么,要将一个怕字换成一个敢字。要最后证明社会主义关是不是过。

你们要政治挂帅,要到群众里面去,和群众在一起,把无产阶级文化大革命搞得更好。


