\section[关于《山西省昔阳县干部参加劳动已形成社会风尚》一文的批语(一九六三年五月)]{关于《山西省昔阳县干部参加劳动已形成社会风尚》一文的批语}
\datesubtitle{(一九六三年五月)}


《山西省昔阳县干部参加劳动已形成社会风尚》一文和省委的批语都很好,一并发给你们参考。干部参加劳动,是党的优良传统之一,是党在社会主义建设时期的一项极为重要的政策。认真贯彻执行这项政策,对于农村工作来说,其重要性是很明显的。农业合作化以来的无数事例证明:凡是办得好的社、队,无例外的都具备有社、队的领导干部经常和社员在一起积极参加劳动的特点。反之,凡是办得不好的社、队,往往具有一个相反的特点,即这些社、队的领导干部不愿意和社员在一起积极参加劳动,因而脱离群众,不能抵抗剥削阶“思想的侵袭,生活特殊化,贪污、多占群众的劳动果实,有的甚至逐步蜕化变质,堕落成为富裕农民和资本主义分子利益的代言人,修正主义的社会基础。

人民公社工作条例(修正草案)对于人民公社各级干部参加劳动问题,已经作出明确规定,可是直到现在,不少地方还没有认真贯彻执行。有的县委和公社党委对这一规定的重大意义认识不足,甚至认为大队和生产队干部的补贴工分不得超过生产队工分总数百分之二的规定根本行不通。应该请他们好好读一读昔阳县的经验。昔阳县的经验证明了:这项政策能否得到正确执行的根本关键,恰恰在于县委和公社党委是否有决心,是否以身作则。这个县的县、社两级干部一九六二年在生产队作的劳动日,县级每人平均六十二个,公社级每人平均八十二个。他们到那里下乡工作,就在那里参加劳动,并且一直坚持不懈,经过几年的努力,才逐步形成风气。应该说,昔阳县的同志们能够这样做,所有各县也可以这样做的。

中央要求各省、市、自治区党委、地委,要认真帮助县委弄通道理,结合整风、整社工作,把人民公社工作条例关于干部参加劳动和补贴工分的规定,抓紧加以解决,以利于人民公社的巩固和健全发展。

