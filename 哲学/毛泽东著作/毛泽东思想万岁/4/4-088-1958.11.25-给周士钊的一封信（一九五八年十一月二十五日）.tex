\section[给周士钊的一封信(一九五八年十一月二十五日)]{给周士钊的一封信}
\datesubtitle{(一九五八年十一月二十五日)}


惇元兄:

赐书收到。十月十七日的,读了高兴。

受任新职,不要拈轻怕重,而要拈重鄙轻。古人有云,贤者在位,能者在职。二者不可得而兼,我看你这个人是可以兼的。年年月月,日日时时。感觉自己能力不行,实则是因为一不甚认识自己,二不甚理解客观事物——那些留学生们,大教授们,人事纠纷,复杂心理,看不起你,口中不说,目笑存之,如此等类这些社会常态,几乎人人都要经历的。此外自己缺乏从政经验,临事而惧,陈力而后就列,这是好的,这些都是事实。可以理解的。我认为,聪明、老实二义,足以解决一切困难问题,这点似乎同你说过。聪谓多问多思,实谓实事求是,持之有恒,行之有素,总是比较能够做好事情的。你的勇气看来比过去大有增加,士别三日应当刮目相看了。我又讲了这一大篇,无非是加一点油,添一点醋而已。

“坐地日行八万里”,蒋竹如讲得不对,是有数据的,地球直径约一万二千五百公里,以圆周率三点一四一六乘之约得四万公里即八万华里,这是地球的自转(即一天时间)里程。坐火车轮船汽车要付代价,叫做旅行,坐地球不付代价(即不买车票),日行八万华里,问人这是旅行么?答曰,不是,我一动也没动,真是岂有此理!囿于习俗,迷信未除。完全的日常生活,许多人却以为怪。“巡天”,即谓我们这个太阳系(地球在内)每日每时却在银河系里穿来穿去。银河何也?河则无限。“一千”言其多而已,我们人类只是“巡”在一条河中,“看”则可以无数。牛郎晋人。血吸虫病,蛊病,俗称鼓胀病。周、秦、汉累欠书传。牛郎自然关心他的湘人,要问瘟神,情况如何了。大熊星座,俗名牛郎星(是否记错了?),属银河系,这些解释请向竹如道之。有不同意见。可以辩论。十二月我不一定在京,不见也可吧!


