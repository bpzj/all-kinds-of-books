\section[关于农村社会主义教育等问题的指示(一九六三年五月)]{关于农村社会主义教育等问题的指示}
\datesubtitle{(一九六三年五月)}


\subsection{一、关于社会主义社会的阶级斗争}

在社会主义社会中,有没有阶级?有没有阶级斗争?外国有一种说法:他们国内没有阶级了,他们的党是全民的党了;无产阶级专政也没有对象了,他们的国家是全民的国家了。我们国内也有类似的说法。资产阶级每天在斗争无产阶级,就是不承认有阶级存在,就是不承认有阶级斗争,说阶级斗争是马克思捏造的。不光是外国的修正主义者和国内的资产阶级不承认有阶级和阶级斗争存在,我们有许多干部、党员、对于敌情的严重性也是认识不足的,甚至熟视无睹的。

以上这些看法对不对?完全不对。已被推翻的反动统治阶级是不甘心于死亡的,他们总是企图复辟的。同时,资产阶级分子会新生,反革命分子也会新生。而在这些阶级敌人的后面,还站着帝国主义,现代修正主义和反动的民族主义。因此,党的八届十中全会公报指出:“在无产阶级革命和无产阶级专政的整个历史时期内(这个时期需要几十年,甚至更多的时间),存在着社会主义和资本主义这两条道路的斗争。”

〔翻印者注:公报上原文为:“在无产阶级革命和无产阶级专政的整个历史时期,在由资本主义过渡到共产主义的整个历史时期(这个时期需要几十年,甚至更多的时间)存在着无产阶级和资产阶级之间的阶级斗争,存在着社会主义和资本主义这两条道路的斗争。”〕

社会上的阶级斗争,一定要反映到我们党内来。我们这样大的国家,又存在阶级,在党内不反映资产阶级思想、封建阶级思想、富裕农民思想那才是怪事!阶级斗争所以会反映到党内来,还有一个重要根源。从党内成份来看,我们党内主要是工人、贫雇农、下中农,主要成份是好的。但是党内有大量的小资产阶级,其中有的是城乡上层小资产阶级分子,也有一批是知识分子,还有相当数量的地主、富农的子女。这些人,有的马列主义化了,有的化了一点,没有全部马列主义化,有的完全没有化,组织上入了党,思想上没有入党。\marginpar{\footnotesize 46}这些人对社会主义革命没有思想准备。另外,这几年还钻进一些坏人,他们贪污腐化,严重违法乱纪。民主革命不彻底,坏人钻进来,这个问题要注意。但是比较好处理。主要问题是没有改造好的小资产阶级分子,知识分子和地主、富农子女,对这些人需要做更多的工作。因此,对党员、干部要进行教育,再教育,这是一个重要任务。

\subsection{二、关于社会主义教育运动问题}

这次社会主义教育运动,是一次伟大的革命运动。“革命尚未成功”这是孙中山的话。我们现在是:社会主义革命正在进行,有些地方民主革命尚未成功。社会主义革命没有完成,就要继续进行社会主义革命;民主革命没有成功,就要进行民主革命的补课,还有一些地方,地主根本没有打倒,那些地方是重新革命的问题。

资产阶级右派和中农分子把希望寄托在自留地、自由市场、自负盈亏和包产到户,这“三自一包”上面。我们搞社会主义革命,在城市搞“五反”,在农村搞“四清”,就是挖资产阶级的社会基础,挖资本主义的根子,挖修正主义的根子。

党的八届十中全会以后,有些地方比较认真执行了中央关于社会主义教育的指示,做得很好,不仅制止了“单干风”,而且把农村中阶级斗争的盖子揭开了,把各种矛盾揭开了,把各种破坏社会主义的牛鬼蛇神揭露出来了。可见,阶级斗争一抓就灵,也有些地方,虽然进行了社会主义教育,但是没有抓住要点,没有找到正确的方法。今后,还需要抓住要点,采取正确的方法和步骤,进一步开展社会主义教育运动。

这次社会主义教育的要点是什么?要点就是阶级和阶级斗争,干部洗手洗澡,依靠贫、下中衣,“四清”,干部参加集体劳动这样一套。凡是社会主义教育一般化,不触及洗手洗澡,不触及贪污盗窃的地方,就不能抓住主要问题。

方法是什么?方法就是说服教育,洗手洗澡,轻装上阵,团结对敌,就是要团结百分之九十五以上的群众和干部,同阶级敌人作斗争,对百分之九十五以上的人,不抓辫子,不打棍子,不戴帽子,还要加上不追不逼,不打不骂。有错误的人,只要彻底坦白悔改,就算在百分之九十五以内。手脚不干净,要批评,要洗手洗澡,还要继续做工作。伤人不要过多,但少数人是要伤的。要奖励一些好人,处理少数坏人,组织处理一定要经过批准手续。有的地方采取讲社史、村史、家史的办法,对青年进行阶级教育。这个办法很值得推行。贫农受剥削、受压迫的家史可以讲,贫农富裕起来的家史也可以讲。地主、富农的家史也可以作为反面教材讲给贫、下中衣听,讲他们是怎样剥削压迫人的。

步骤是什么?就是:经过试点,分期、分批、分地区地进行。一个县之内,也要分期、分批进行。要注意到各地区的不同情况,允许有先有后,允许参差不齐,开始训练县级干部,再训练公社和大队的干部,然后训练生产队干部和贫、下中农的积极分子。试点很重要,各地都要搞试点,经过试点把情况弄清楚。有的同志,开始对阶级斗争、社会主义教育不大相信?后来他去试点以后,就相信了,相信了就抓起来了。

有人对社会主义教育有顾虑,无非是两条:一是怕耽误生产,一是怕“伤人”太多,阶级斗争和社会主义教育一定会有利于增加生产。“伤人”不能过多,但少数人是要伤的。解决人民内部矛盾问题也要有点“紧张”,精神轻松愉快,这是就其结果说的。不是说在社会主义教育过程中没有一点紧张。\marginpar{\footnotesize 47}只有搞了社会主义教育,“五反”、“四清”,组织贫、下中农阶级队伍,才能做到心情舒畅。贫、下中农起来,几股黑风不打倒,干部不洗手洗澡,能够心情舒畅吗?干部心情不舒畅,就要搞我们这一套。贫、下中农心情不舒畅(此处遗漏了一段)才能出现真正的心情舒畅的局面。

当然也不要急躁,不要蛮干。过去社会主义教育搞得不深的地方,要从搞得不深的实际情况出发,要跳起来,一哄而起。不打无把握之仗,要准备好了再打。没有试点,情况不明,或者兵没有练好,干部和贫、下中农没有训练好,就不要急急忙忙的、大规模地开展运动。对干部要着重说服教育,口头说不服的,就用事实材料去说服。要派强有力的干部去领导运动,不会打仗的人,不要他当指挥官。没有蚂蚁的地方不要硬找蚂蚁。例如,过去有些一类社、队过去注意了阶级斗争,注意了社会主义教育,就不一定完全采取现在这一套办法来搞。但是人民内部矛盾是普遍存在着的,在一类社、队,也要解决人民内部矛盾。

总之,这次运动是一次大考验,干部好不好,行不行,都要在这次运动中受到考验。只要我们分期、分批、分地区去搞,经过试点,认真对待,再加上保护百分之九十五以上的干部和群众,大毛病是可以避免的。

\subsection{三、关于四清问题}

在农村中,不搞“五反”,只搞“四清”。“四清”就是清理账目,清理工分,清理仓库,清理财物。其中又主要是清理账目、清理工分两项。首先应当发动群众把1962年以来的账目、仓库、财物、工分,同时把国家投资、银行贷款和商业部门赊销所添置的资产,全面彻底地清查一次。

这是一项同社会主义教育运动相结合的大规模的群众运动,主要解决人民内部矛盾,但对于贪污、盗窃、投机倒把、蜕化变质分子来说,也是一场严重的阶级斗争,农村中的“四清”运动,同城市中正在进行的“五反”运动一样,都是打击和粉碎资本主义势力猖狂进攻的社会主义革命斗争。

我们在农村中十年没有搞阶级斗争了。1952年搞“三反”、“五反”,是在城市。在农村,1957年搞了一次全民整风,但不是用现在这个方法搞的。合作化以来,农村中的现金、工分、财物、仓库就没有清理过,没有向社员全面公布账目。“四清”能搞出很多贪污盗窃、牛鬼蛇神。公安机关搞不出来,“四清”能搞出来。

进行“四清”方法要对,要采取扎根串连,依靠贫、下中农,这一套办法,放手发动群众,有些干部不听领导的话,他们却不能不听群众的话,把群众发动起来,事情就好办了。要把百分之九十五以上的人团结教育过来,打击极少数严重贪污盗窃分子。有些人坦白了,退赔了,就可以不戴贪污分子的帽子,要使多数洗温水澡,轻装上阵,团结对敌。“四清”中一切问题的处理都要发动群众充分讨论。赃款、赃物,不退不行,但要退得合情合理。不退群众不允许,太挖苦了,有些干部过不去,群众过些时候也会同情他们的。对贪污盗窃分子,一般不采用群众大会斗争的方式,可以一方面采用背靠背的方式(在必要的时候,也可以在较小范围的群众会上),让群众充分揭发和批判;一方面组织专门小组清理账目,进行调查研究,然后根据确凿的证据,核实定案。要严防敷敷衍衍“走过场”,也要防止逼、供、信,严禁打人、骂人和任何变相的体罚。已经搞过的要复查。凡是搞得不好,真不彻底的,必须重搞,一些必要的制度还没有建立的,必须建立起来。\marginpar{\footnotesize 48}

\subsection{四、关于组织贫、下中农革命的阶级队伍}

不论在革命中,或者在社会主义建设中,都必须解决依靠谁的问题。(翻印者注:上文与下文联系起来看,这里一定遗漏了一段,请读者注意),因为那时还有唯物论和唯心论,还有先进和落后。没有阶级差别了,总还有左、中、右。

在农村中依靠谁?

不是依靠全民,而是依靠贫农、下中农。贫农、下中农大约占农村人口百分之五十到七十左右。他们是农村中的多数。是农村中的无产阶级和半无产阶级。我们要同地、富、反、坏作斗争,就要团结大多数,首先是依靠贫农、下中农。要多数,还是少数,要无产阶级专政,还是牛鬼蛇神专政?真正的马列主义者就是依靠多数,修正主义者名曰依靠全民,实际上是依靠少数。一些农村干部有一种说法,说:“地主听话,中农好办,贫农糊涂。”其实,地主是要你听他的话,县社干部都不注意听贫农的话,那么贫农一辈子都翻不了身。

依靠贫农、下中农,树立贫、下中农在农村的优势,是进行社会主义革命和社会主义建设的重要问题也是巩固无产阶级专政的重要问题。在整个社会主义历史阶段,一直到进入共产主义以前,我们在农村都要依靠贫、下中农。

要依靠贫、下中农,就必须建立贫、下中农的阶级组织。组织起来,就有了中心。贫、下中农组织起来就可以更好地团结中农。有些地方,贫农一经组织起来,中农就打听消息,表示要走社会主义道路,还不是大家都组织起来了。

建立贫下中农阶级组织,要在斗争中发动群众去建立,不要形式主义地建立。搞形式有什么用?形式上是社会主义,实际上不是。南斯拉夫还不是挂着社会主义招牌。

开始组织,不一定数量很多。比如一个生产队有二十户,先组织两三户,以后四、五户,有十一、二户组织起来了,就很起作用。要像滚雪球一样逐步搞起来,不要一哄而起,一步一步搞,扎扎实实搞。

有些地方提出贫、下中农委员会的主任,由党支部副书记兼任。这不好,要由群众选,可以选支部书记兼任,也可以不选支部书记兼任,不能硬性规定,保谁当选。

在农村中,无产阶级和资产阶级都在争夺农民中的富裕阶层,这个阶层本身就是产生资产阶级的东西,也容易接受资产阶级的影响。但是,对这个阶层,要作具体分析。例如,要看生活上升,还要看政治表现,只要不是剥削者,思想上又赞成社会主义,积极劳动,还是要把他们团结起来,共同建设社会主义。

\subsection{五、关于干部参加集体生产劳动问题}

干部参加集体生产劳动的问题,对于社会主义制度来说,是带根本性的一件大事,干部不参加集体生产劳动,势必脱离广大的劳动群众,势必出修正主义。

我们的党是无产阶级的党,是劳动群众的先进的党。我们党的基层组织必须掌握在劳动积极的先进分子手里。农村中的党支部书记不但在政治上应当是最先进的分子,而且必须力争成为生产能手,成为劳动模范。县社以上干部也要认真参加集体劳动。干部不劳动了就会慢慢变质,甚至变成国民党,修正主义就有基础了。\marginpar{\footnotesize 49}浙江省就有一位大队支部书记应四官说:“不参加劳动,工作就像浮萍一样浮在水面,摸不到底。”参加劳动,就可以解决这个问题,至少可以减少贪污、多占问题,可以了解农、林、牧、副、渔这一套,支部书记参加了,大队长、队长、会计就要参加,整党整团就好办了。这样,修正主义就少了。

现在有些群众对干部作官很有意见。群众说:“大队干部了不起,一吃二用三送礼。”大队干部就这样大?现在给他一个“四清”,一个劳动。你不干,就当老百姓。这是一个很大的斗争,没有一大批积极分子起来是搞不好的。

山西昔阳县干部参加劳动的情况是个好榜样。昔阳县的干部既然能够长期坚持,其他县的干部也应当是能够办到的。我们争取在三年之内,分期分批,使农村支部书记认真参加劳动。比如一百个支部,第一年先有三分之一参加劳动,第二年又有三分之一,就有百分之六十的人。这样声势就大了,其他的人就要参加进去了。

有些劳动模范现在不参加劳动了,不参加劳动,还作什么模范?不能参加劳动,理由无非是怕耽误工作,会议太多。公社以上的领导机关要减少开会。一个县,每年只开一、两次三级干部会议就够了。必须开的会要事先作好充分准备,有些会还可以到下边去开。

\subsection{六、关于运用马克思主义科学方法进行调查研究的问题}

调查研究有两种方法。一种是大胆的主观的假设,小心的主观主义的求证。这是个很坏的方法。一种是马克思主义的科学方法,河北省保定地委关于“四清”的调查就是这种方法。保定地委开始并不是去搞“四清”,是去搞分配问题的。群众不同意,提出搞“四清”。保定地委听了群众的意见,改变了计划,搞了“四清”。这才是真正的调查研究。

调查研究的范围,一个是生产斗争,一个是阶级斗争,一个是科学实验。不对这方面进行调查研究,哪有马克思主义?浙江省清田县搞试验田,带点科学试验性质。他们试验到山里去了。那里人民开头不赞成冬水田,经过试验,冬水田第二年收成好,贫农看到以后就接受了。所以要调查,要试验。礼会主义教育为什么有人不相信?就是没有试点,没有认真调查研究。比如走路,像平常那样走,什么也看不见,弯下腰来细看,就可以看到地上的蚂蚁很多,就能看到很多东西。否则,不仅新鲜的萌芽的东西看不见,就是大量普遍存在的现象也看不见。例如阶级斗争和干部不参加劳动,是大量存在的现象,有些人都看不见。

当然,这些事情也是要逐步认识的,要从现象到本质。比如干部不参加劳动,势必产生修正主义,有许多同志就看不清。干部不参加劳动,了解和反映的情况就不会真实。比如打仗不亲自参加战斗,还不是纸上谈兵,怎么能懂得打仗呢?单是进军事学校也不行。

为了造成调查研究的风气,做好我们的工作,各级党委在日常工作中讲哲学,对干部进行马列主义认识论的教育。唯物论、唯心论、世界观、辩证法,都是讲的认识论。物质可以变为精神,精神可以变为物质。这些道理,应当让干部懂得、群众懂得。让哲学从哲学家的课堂上和书本里解放出来,变为群众的尖锐武器。

人的思想是从哪里来的,我们有些同志是不知道的,对于精神可以变为物质,有些同志就更糊涂了,但是不识字的农民是懂得推理的。比如农民认为地主是人,是剥削压迫他们的人。人、地主是两个概念,农民把这两个概念联结起来,进行判断推理,得出结论说:地主是剥削人的人。农民的这种认识,是从生活中来的,不一定识字才懂得,所以要破除迷信(当然不要破除了科学),不要把哲学看得那么神秘,那么困难。\marginpar{\footnotesize 50}哲学是可以学到的。雷锋那样年青的同志就懂得一点哲学。

总之,这一次社会主义教育运动是一次伟大的革命运动,不但包括阶级斗争问题,而且包括了干部参加劳动的问题,而且包括严格的科学态度,经过试验,学会在企业和农业中解决一批问题这样的工作。看起来很困难,实际上只要认真对待,并不难解决。这一场斗争是重新教育人的斗争,是重新组织革命的阶级队伍,向着正对我们猖狂进攻的资本主义势力和封建势力作尖锐的针锋相对的斗争,把他们的反革命气焰压下去,把这些势力中间的节大多数人改造成为新人的伟大运动,又是干部和群众一道参加生产劳动和科学实验,使我们的干部成为既懂政治、又懂业务,又红又专,而不是浮在上面,做官当老爷,脱离群众,而是同群众打成一片,受群众拥护的真正好干部。这一次教育运动完成以后,全国将会出现一种欣欣向荣的气象。差不多占地球四分之一的人类出现了这样的气象,我们的国际主义的贡献也就会更大了。

