\section[在西北青年救国会上的讲演(一九三七年)]{在西北青年救国会上的讲演}
\datesubtitle{(一九三七年)}


同志们:

今天西北青年救国代表开会,将号召西北及全国青年群众与各界联合,走上团结抗敌的道路,我就来说一点关于共产党的团结抗敌的政策。共产党过去的政策是国内战争与土地革命。现在政策是和平团结,对日抗战。这是什么原故呢?为什么我们过去和现在有这样不同的政策呢?原因就在过去若干年间,我们不能不那样做,也只容许我们那样做。过去是两个政权对立的斗争,阶级的武装斗争,因为那时候国民党和资产阶级背叛革命,拿着枪,杀革命人民,革命的人民除此以外,没有任何第二种办法和出路。那时候的那样干,完全必要与正当的。

如果说共产党过去错了,那末请问除了这条道路之外,还有什么别的道路?难道我们应和国民党一道投降帝国主义与封建势力吗?不投降帝国主义与封建势力,就得反抗帝国主义与封建势力。二者之间没有中间的道路。

九一八以后特别是华北事变以后,日本要单独独占中国,灭亡中国,使得中国内部的阶级关系变化了,使得英法关等帝国主义国家对中国的关系也变化了一些。资产阶级同国民党里面都发生了分化,国家处在危亡面前,我们的政策就有改变旧的采取新的必要与可能,因此我们釆取了抗日的民族统一战线的政策,并提出停止内战一致抗日的口号,我们统一战线的目标是抗日的,而不是反对一切帝国主义的;我们的统一战线是民族的,不分阶级与党派的联合战线,与外国的人民阵线也不相同,这是我们统一战线的特点。

自从一九三五年十二月九号平津学生反日运动以来,经过许多事变直至西安事变与三中全会,中国内部已经由一种状态开始走上另一种状态,即是由内战独裁与不抵抗开始走上和平民主与抗战的阶段。共产党的八一宣言,十二月决议,致国民党书,九月决议,以及西安事变中的和平统一政策,给三中全会通电中的和平政策,都是国内达到新阶段的号召与领导的力量,这些东西都表示我们过去提出了并实行了粉碎国内统治阶级的围剿这个口号,那末,现在就不是如此了,现在是要团结全民族来粉碎日本为首的法西斯蒂对全中国的围剿。倘若内外都有围剿,那中国就唯有灭亡,因此我们提议内面的围剿必须和解,并且建立全民族的大团体来对付外面的围剿,现在这新任务已经开始走了一步,往后就是第二步,第三步,一直粉碎法西斯围剿而后止。

我们停止土地革命是否就不顾民族利益了呢?不是的,恰恰相反,我们更要为民众的利益而斗争,我们坚决要求国民党实行民主改革与民主改良就是为的这种目的,因为今天的民主改革,正是抗日人民大众的需要,因为没有民主改革,没有人民的自由权利,抗日会只是一句空话。抗日与民主二者是不能分离的。

以前的不和平不统一,国共两党之间如此,全国各省之间也是如此。抗日救国必须和平统一,相打分裂是不能抗日救国的,因此和平统一成为全国必须的要求。过去一年的工作是坚决要求和争取这种和平统一,现在算是初步的达到了,但和平还没有巩固,要巩固和平并实现真正的统一,就得实行民主政治。

现在全国人民,除了苏区以外,都没有民主自由,说话有罪,走路也有罪,爱国有罪,如沈钓儒等爱国领袖因为爱国而被关起来,现在听说还要起诉。为向日本帝国主义示威而在街上走路就要被捉,出报纸讲爱国话就要被封。这种情形怎能真正抗日呢?所以我们说,没有和平与没有民主,都不能统一团结,便都不能抗日,所以巩固和平,争取民主,是实现抗日战争的必要条件,也是今日的中心任务,是今日必须要做的文章。

我们曾为扩大红军而努力,现在也不相同了。中国人与中国人从前相打的现在应该改为好朋友,大家去打共同的敌人,这样,红军也可改为国民革命军,一切中国的军队都要变为抗日军,我们应该为扩大与训练真正的抗日军而斗争。

过去国民党背叛革命,所以共产党不得不单独负起革命的责任。理在呢?国民党又开始转变到抗日的方面来,所以我们极力主张国共合作,主张恢复孙中山先生的三民主义的革命精神,国共两党与全国人民,大家为民族独立,民权自由,民生幸福而斗争。

有些同志,以为我们红帽子戴了十年,今天又戴了三民主义帽子,就表示老不愿意。这个思想在过去是很对的,因为那时三民主义帽子确实戴不得。但如果旧帽子换上新内容,那事情就变化了,不是不可戴的,反而变为可戴的了。苏维埃改制,红军改名,并受南京国民政府指挥,就是为了这个意义。

西北青年有团结全国青年的责任,全国青年都团结起来,跟成年、幼年、老年就一道结成一个几万万人的大团体,那末日本帝国主义凶悍怎么样,我们是能够打倒他的。(长时间的鼓掌)

