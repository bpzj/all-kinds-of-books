\section[工作方法十六条(一九五九年四月)]{工作方法十六条}
\datesubtitle{(一九五九年四月)}


党的总路线,大家都赞成。去年政治上的主要标志是总路线的制定,但是还不能达到预期的效果,这是工作方法问题。

我们要实现总路线,必须有好的工作方法,没有好的工作方法,我们的总路线是不能完全贯彻的。有了总路线,还必须有好的工作方法,才能实现多快好省、几个并举的方针。所谓方法,无非是思想方法和工作方法。思想方法和工作方法是互相结合的,思想方法不对头,工作方法也就不对头。现在中心问题是工作方法问题。

一、多谋善断。这句话的重点在谋字上。谋的目的是为了断。曹操有个参谋叫郭嘉,他批评袁绍多谋寡断,有谋无断,没有决心,不果断,结果官渡之战就打了败仗。所以有谋还要善断。要多谋,少谋是不行的,要与多方面商量。我们了解情况要多听各方面的意见,多看看各种材料,各种方案,善于判断,善于下决心。现在我们有些人就是少谋武断。根本不同人家商量,这是武断。有些同志不大愿意听不同的意见,只愿听相同的意见。与相同意见的谋得多,与相反意见的谋得少;与干部谋得多,与生产人员谋得少。他不是听听别人不同的意见,看看别人不同意见的材料。来做出判断,听听不同的意见才有好处。不一定相同的意见才正确,不同的意见就不正确。不同的意见也可能是正确的。有些同志在订经济计划时不同别人交谈,不去多谋。为什么不跟秘书谋一下,不跟工厂的厂长谋一下?可以谋你左右的干部,也可以谋工人、农民,可以谋提出不同意见的同志。他既然提过不同的意见,你就谋谋他,看看他的意见怎样。有些同志很主观武断,认为自己的意见就是正确的。实践的结果,不行,还可能造成虚假现象。要听听各方面的意见,特别是反面的意见,不同的意见,将各方面的意见集中起来分析研究,才能多谋善断,订的计划才能正确。谋是基础,只有多谋才能善断。多谋的方法很多,如开调查会、座谈会。

二、留有余地。俗话说要有后手。一切工作都要留有余地。我们在安排工作计划时,要留有余地。给下面点积极性。不给下面留有余地,就是不给自己留有余地。留有余地上下都有好处。如农村包产问题,包产指标二千斤,就是没有给下面留有余地,也是没有给上面留有余地。过去我们打仗也是一样,要留有余地,集中兵力打击敌人,还要有个预备队,必要时把预备队拉出去。现在搞生产就忘掉了。计划工作要留有余地。长短期计划都应如此。要让实际工作去超过,要给群众超过计划的余地。生产队长说:“不怕一万,只怕万一。”过去工作中间起码是个泄劲的办法。让群众超过反而会鼓舞群众的干劲。我们有些生产计划、经济计划,满打满算,不留一点余地,很容易造成虚假现象。我怀疑搞工业的同志是否懂得工业。留有余地是政治问题,也是工作方法问题。我第一次访问苏联与斯大林谈话时,问他们第一个五年计划经济建设有什么经验。苏联同志说:经济建设有二条经验。第一是留有余地,苏联第一个五年计划留有百分之二十的余地,他公布的数目就少百分之二十。苏共二十一次代表大会公布七年计划,钢的生产指标九千多万吨,它实际生产不止这个数。农业生产也好,其他也好,都是留有余地,实际超过了,人民群众心情更加舒畅;一点余地不留,将来完不成计划,就造成悲观失望。苏联经济建设的第二条经验是抓住重点。有重点才有政策,没有重点就没有政策。我们要按政策办事情。一个时期有一个时期的重点。打仗也是一样,要有重点,有个主攻方向,有个箝制方向,这样才能打歼灭战。不仅建设工作要有重点,打仗要有重点,就是舞台艺术、写文章、做诗也要有重点,留有余地。舞台艺术也要给观众留有余地,不要把戏都演完,演完戏群众还会想想,这样的戏演得才算成功。现在我们有些戏用不着去看,都是一个公式,开个群众大会,开个斗争大会,喊了几句口号就收场了。没有给群众留有余地就结束了。所以写文章、做诗、演戏都要留有余地,不要一下子什么都做完,要让群众去想想。

三、波浪式前进。过去讲马鞍形,其实不是批评马鞍形,主要是反对冒进。那时有人公开在群众中反冒进,反多快好省路线。一九五七年把基本建设压缩了一下是对的,因为当时只有那么多,只能办那么多的事。今年计划也稍微降低了些。按速度来看,是增加了百分之××,今年要确保××万吨钢,煤有信心完成。农业指标,粮棉要很大努力才能完成,明年数字稍微再低一些。六一年再来个大跃进。社会主义建设中要懂得波浪式前进。“天增岁月人增寿,春满乾坤福满门。”不能天天搞高潮。我不反对波浪式,反对冒进。但是在群众中公开反冒进是不对的,这是泼冷水,泄气的办法。马鞍形将来还会有的,主要是不要反冒进。美国从一八六○年到一九五八年的九十九年中间,也不过是七次生产高潮,不是逐年高潮,也是波浪式前进,并不是每年都是高潮。我们的经济建设,按实际情况,可以高些,可以低些。比如钢的生产到一亿六千万吨,你还要翻一番就不容易了。少的时候翻一番可以,多了翻一番就困难,也没有这个必要。这是波浪式前进。波浪式前进也是个工作方法。凡是运动就有波,在自然科学中有声波、电波。凡是运动就是波浪式前进,这就是事物发展的规律,是客观存在,不以人的意志为转移的。我们做工作,订计划,也要照顾到这一点。波浪式前进是客观法则、客观规律,不能老是翻一番。

四、实事求是。不同意在群众中公开反冒进,但是要落实计划,要按照形势改变计划。形势变了,情况变了,人的思想也要跟着变,我们就要改变计划,计划不是不能改变的,武昌会议到现在就改变了。不根据情况改变计划,工作就被动。经过第一季度的实践,认为非改变不可,不改变就会造成困难,造成混乱,任务不可能完成。要根据情况变化,适应情况变化,按情况办事。脑子不要硬化,订计划要有多少材料,多少人,订多大的计划,不要主观地订计划。

五、要善于观察形势。脑子不要僵化,要注意观察形势,观察动态,了解情况。要提倡嗅一嗅政治形势,嗅一嗅经济形势。所谓政治形势,就是观察各阶级思想,观察他们立场变化。书记要观察,各委员也要观察。各委员不但要做好分管的工作,也要做好集体工作。北戴河会议时,指标订得高,后来我走了河北、山东,感到不行。六中全会决定(钢)降低为××万吨,上海会议又压了一下,一步步落实。

六、要当机立断。只有观察形势正确,才能当机立断。把握形势的变化,来改变我们的计划。有的同志上不摸底,下不摸底,有的工作也有断,但断得不适当。优柔寡断是不对的,断的时候要下决心。机不可失,时不再来。公社这个问题很明显,究竟几级管理,去年没有清楚,(再)经过一个过程,一月二十七日,就提出这个问题。要感谢内部参考。内部参考给我很大帮助,登了一些反映农村情况的材料。后来看了×××同志的报告,又到天津找了×××,到了山东找了×××,到了吕鸿宾社,发现了以下问题,一条杆子,一杆秤,不同意的一顶帽子。一把钥匙,一张布告,一个楼梯。从这时才发觉队的所有制。这次会议明确地解决了这个问题。有些同志三分怕上级,七分怕下级,因为他摸到了上面的底,有什么错误缺点,上级总是要掌握团结——批评——团结、与人为善、治病救人的方针;他怕的是群众,你要他开五、六级干部大会,几万群众参加,他就怕了,他就经不起检查。以后要来一个上级、下级夹攻中层。很多县委、地委的话不可听,尽是好话,不把事实情况让你知道。在工业方面也要注意这个问题。中级干部的话往往不能听,要听也得上下结合才能做出正确的结论。召开五、六级干部会,基层干部占多数,这样的会才开得好。要善于分析情况,抓紧时机,当机立断,下定决心,这样才能得出正确的方针政策。对党内一些不良倾向,也要当机立断。

七、与人通气。上下左右,左邻右合,上上下下,都要通气。中央与地方商量通气。党委委员互相商量通气,与书记要通气。我们过去通气少了些,要想办法通气,现在釆取了写信的办法。一个月写一次,这就是通气的办法。通气有好处,不通怎么了解情况?怎么指导工作?做工作,会议,不要一点空气没有,讨论之前,应该事先有个酝酿。有些会连题也没有。我讲这个问题讲了一百次,不通气不好,不要把问题独立起来,不要使人不摸底。决定问题要有个充分酝酿。

八、解除封锁。平时不向中央反映情况,开会才拿一大堆材料来,平时不下毛毛雨,到时就下倾盆大雨。他就是不给你反映情况,不汇报,不请示,就是不给你知道,平时纸片只字不给你,开会就给你一大堆材料,要你做决定,下决心。要解除封锁,不要封锁中央,要把封锁消息的同志狠狠批评一顿,让他几夜睡不好觉,以后就会好了。杜勒斯封锁我们,共产党内部也有封锁情况的,不要封锁政治局,不要把中央委员和主席当做跑龙套的。对省委书记也不要封锁情况,报告中要有观点,一个事情要提出几种方案,要说明你那里的基本情况、不同意见、核心问题是什么。要把工作情况如实反映上来,不要封锁。

九、一个人有时胜过多数,因为真理往往在他一个人手里。多数的时候是多数人胜过少数人,有时候又是少数人胜过多数人。就是说,有时候真理不在多数人这边,而在少数人或个别人这边。你们不要看开大会,大多数人赞成就是正确,就是真理。那不见得。往往真理是在少数人手里。如马克思主义,就是在他(马克思)一个人手里。你们看过联共党史,有一次会议大家都举手了,就是列宁一个人不举手。列宁未举手是真理,其他人都不对。列宁讲,要有反潮流的精神。各级党委要考虑多方面的意见。要听多数人的意见,也要听少数人和个别人的意见。在党内要造成有话讲、有缺点要改进的空气。批评缺点往往就有点痛苦的,但批评之后,改了,就好了。有些同志不把自己心里话说出来,中庸之道太多了。不敢讲话无非是六怕:怕警告,怕降级,怕没有面子,怕开除党籍,怕杀头,怕离婚。杀头,岳飞就是杀头才出名的。要言者无罪,按照党章可以保留自己的意见。过去朝廷有廷杖制度,不知打死多少人,但还有很多人死在朝廷(原稿如此——抄者)。有些同志深怕对自己不利,这就要提倡敢想敢说的共产主义风格。王熙凤舍得一身剐,敢把皇帝拉下马。当然,这是痛苦的。但错误的意见提出来,还可以受教育。领导干部对极少数人的意见,应该很好地考虑,注意分析这些意见,不要马上顶回去,看看里边有没有真理。

十、要历史地观察问题。计划变动,要经过一个历史过程。北戴河、(第一次)郑州、武昌、上海几次会议,原订的计划不是有了改变了吗?上海会议比较稳当,切合实际,可能会超额完成。粮食、棉花能不能超过,要做很大的努力。粮食把杂粮、地瓜加上去也可能完成。现在公布的四大指标不改变。去年放的“卫星”有的有好处,有的未放起就掉下来了,为什么去年不去更正呢?一更正就会给群众泄气。但去年大跃进是确实的。每年增产百分之十是跃进,百分之二十是大跃进,百分之三十是特大跃进。事物是从发展中逐步认识的,不到今年一月下旬,也不认识人民公社的所有制问题。武昌会议还没有提出所有制问题。也是(第二次)郑州会议才提出来的。

十一、凡是看不懂的文件禁止拿出来。有些工业部门拿出来的文件,别人看不懂,问他,他自己也不懂。写文章、写报告,不能用加减乘的方法,即形而上学的方法,一定要有情况、有分析,切合实际。我怀疑搞经济工作的同志是不是懂得经济,有的经济学家是不是真懂得经济,如果真懂得了,一定会从意识形态上表现出来。自己还不懂,写出来的东西别人当然也看不懂。为什么写的文章别人看不懂?就是没有钻进去,没有掌握材料,没有把每个问题都交待清楚。新华社关于西藏叛乱的公报,就有个来龙去脉。(按:此公报是在主席亲自主持下写的。)写文章是给人看的,一切问题都要有个交待,交代不出不要勉强,勉强写出来就不能说服人家。为什么会勉强呢?就是对事物没有真正的了解。有些文章没有说服力,说明你对业务本身不了解,不认识,不了解群众心理。唐朝名作家韩愈,以散文者称于世。他是河南修武人。他主张用师之意,不用师之辞。他主张不要因循守旧,要有独特风格。不要怕毁誉,潘宗词写骈体文,不好读,专叫人看不懂。美国的新闻报导值得我们学习。美国新闻通讯社的新闻报导,凡是提到某一个议员的名字,必注解他是美国某州的议员。每次都这样注解,多少年也是如此,重复了多少次也是如此,就是怕其他国家的人看不懂。我们的新闻报导却不管人家懂不懂,自己知道,人家不知,又不注解,又不注释。有的人写文章不用口语。鲁迅写文章就口语化,《阿Q正传》里的阿Q说:儿子打老子,这就是口语,一讲出去人家就懂。我不止讲过一万次了,从今天起要改过来。我们写文章是要给全国人民看的,要给一千三百万党员看的,要一看就懂,看不懂就退回去。这次工业部门有几份材料都退回去了。

十二、权要集中。权力集中在常委会和书记处。以后凡是小问题,政治局、常委会签字是可以的,凡是国家重大问题,一定要经过中央全会讨论。各经济部门的各种计划,先要通过中央全会讨论,决定方针,然后才订计划。不能先动手,后做计划。(造成)既成事实,才送上来签字。

一朝权在手,便把令来行。有权在手,就可以把令下。要勇于负责。要服从领导。

十三、要解放思想。不要怕鬼,不要扭扭捏捏,要有骨有肉。有些同志中间空气不健全,前怕狼,后怕虎,思想没解放,怕挨整。干部要有坚持真理的勇气,不要连封建时代的人物都不如。不要怕穿小鞋,怕失掉选票。就是杀了头也好么,你是为了国家,为了坚持真理,为了坚持自己正确的意见。

十四、关于批评。我们都是好同志,对同志的批评也是为了把工作做好,找到好的工作方法,希望同志们敢于提出各种不同的意见。要是你有缺点,我不批评,我有缺点,你不批评,造成这种自由主义的空气,不好。有些同志报喜不报忧,不把真实的情况反映上来。人家本来有困难,你反映上来说人家没有困难,人家有不同的意见,不好。(可能有漏字——抄者)我们又不打击,又不报复,为什么不敢大胆批评,不向别人提意见?明明看到不正确的,也不批评,不斗争,这是庸俗作风。我们前代无冤,后代无仇,不打不相识么。

一个人如果没有人恨,就是不可设想的。

批评、自我批评是党教育人民的武器。

十五、集体领导。中央开会有了核心,各地都应办到。

十六、和各部门的联系,特别是和工业部门的联系要加强。和三委(计委、经委、建委)、两部(冶金、一机)的联系要密切。

(说明:主席在一九五九年四月在八届七中全会上讲过工作方法九条,此后在另一次会议上的讲话增为十六条。以上是根据九条的传印稿和十六条的两种传抄稿整理的。)


