\section[为东渡黄河给杨立三同志的指示信(一九四六年二月二十一日)]{为东渡黄河给杨立三同志的指示信}
\datesubtitle{(一九四六年二月二十一日)}


立三同志:

(一)作战命令一份付上,可了解红军行动方向及任务。

(二)杨森、蔡树藩、赖传珠领导之游击队五六百人由河口过河后,以义虹镇为指挥中心,其任务是:(1)维持石楼、义蝶、河口的交通;(2)拆毁沿河堡垒,消灭残敌;(3)发动新关、老娃关、清水关、义蝶镇四点之间的群众斗争,组织山西本地游击队;(4)保持主要渡口。

(三)已下令从绥德、清间、延水三县动员三千人当担架队,本周底集中一千人。但三县人口多少不等,这个数平均分配恐不适当,应以按照可能实情为动员原则,首先集中一千人,看前方需要情况,等候命令再集中二、三两批。办法,(1)有适当组织,有好干部带领;(2)有支部组织;(3)有伙食组织;(4)带衣、毯、碗、筷;(5)服务期一个半月;(6)每两人一付担架;(7)均到河口你处集中,待命前送。

(四)戴秀英负责加造船十二只,计马灰样三只,老娃关三只,河口六只(此六只准备开赴下游适当渡口应用)。请与联络。

(五)你须与清间县委及河边各苏区密切联系。

(六)周付主席日内来河边主持,你的工作向他请示,带有电台。

(七)我由义蝶镇向石楼前进。

<p align="right">毛泽东

二月二十一日十二时于河口</p>

