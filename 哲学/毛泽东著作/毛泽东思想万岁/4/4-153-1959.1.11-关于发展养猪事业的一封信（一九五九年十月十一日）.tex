\section[关于发展养猪事业的一封信(一九五九年十月十一日)]{关于发展养猪事业的一封信}
\datesubtitle{(一九五九年十月十一日)}


×××同志:

此件很好。请在新华社内部参考发表。看来养猪事业必须有一个大发展。除少数禁猪的少数民族以外,全国都应当仿照河北省吴桥县王谦寺人民公社的方法办理。在吴桥县,集资容易,政策正确,干劲甚高,发展很快。关键在于一个很大的干劲,拖拖沓沓,困难重重,这也不可能,那也办不到,这些都是懦夫和懒汉的世界观,半点马克思列宁主义者的雄心壮志都没有。这些人离一个真正共产主义者的风格大约还有十万八千里。我劝这些同志好好地想一想,将不正确的世界观改过来。我建议,共产党的省委(市委、自治区党委)。地委、县委、公社党委、以及管理区、生产队、生产小队的党组织,将养猪业、养牛、养羊、养驴、养骡、养马、养鸡、养鸭、养鹅、养兔等项事业,认真地考虑研究,计划和采取具体措施,并且组织一个畜牧业,家禽业的委员会或小组,以三人、五人至九人组成,以一位对于此事有干劲有脑筋,而又善于办事的同志充当委员会或小组的领导责任。就是说派一个强有力的人去领导,大搞饲料生产,有各种精粗饲料,看来包谷是饲料之王。美国就是这样办的。苏联现在也开始大办了。中国的河北省吴桥县现在也已经开始办了,使人看了极为高兴。各地养猪不亚于吴桥的,一定还有很多。全国都应大办而特办。要把此事看得同粮食同等重要,把包谷升到主粮的地位。有人建议,把养猪升到六畜之首,不是“马牛羊鸡犬猪”,我举双手赞成。猪占首要地位,实在天公地道。苏联伟大土壤学家和农学家威廉斯强调说,农林畜三者互相依赖,缺一不可,要把三者放在同等地位。这是完全正确的。我们认为农林业是发展畜牧业的祖宗,畜牧业是农林业的儿子了。然后,畜牧业又是农、林业(主要是农业)的祖宗,农、林业又变为儿子了。这就是三者平衡地位互相依赖的道理。美国的种植业和畜牧业并重。我国也一定要走这条路线。因为这是证实了确有成效的经验。我国的肥料来源第一是养猪及大牲畜,一人一猪,一亩一猪,如果办到了,肥料的主要来源就解决了。这是有机化学肥料,比无机化学肥料优胜十倍。一头猪就是一个小型有机化肥工厂。而且猪又有肉,又有鬃,又有皮,又有骨,又有内脏(可以作制药原料),我们何乐而不为呢?肥料是植物的粮食,植物是动物的粮食,动物是人类的粮食。由此观之,大养特养其猪,以及其他牲畜,肯定是有道理的。以一个至两个五年计划完成这个光荣而伟大的任务,看来是有可能的。用机械装备农业,是农、林、牧三结合大发展的决定性条件。今年已成立了农业机械部,农业机械化的实现,看来,为期不远了。


