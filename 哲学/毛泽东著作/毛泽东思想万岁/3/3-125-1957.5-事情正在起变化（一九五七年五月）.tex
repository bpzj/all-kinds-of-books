\section[事情正在起变化(一九五七年五月)]{事情正在起变化}
\datesubtitle{(一九五七年五月)}


对立面的统一和斗争是社会生活中普遍存在的,斗争的结果是走向自己的反面,建立新的统一,社会生活就前进了一步。

共产党整风是一个统一体,两种作风之间的斗争。在共产党内部如此,在整个人民的内部也是如此。

在共产党内部,有各种人,有马克思主义者,这是大多数。他们也有缺点,但不严重。有一部分人有教条主义错误思想。这些人大都是忠心耿耿的,为党为国的,就是看问题的方法有“左”的片面性。克服了这种片面性,他们就会大进一步。又有一部分人,又修正主义或右倾机会主义错误思想。这些人比较危险,因为他们的思想是资产阶级思想在党内的反映,他们向往资产阶级自由主义,否定一切。他们与社会上资产阶级知识分子有千丝万缕的联系。几个月以来,人们都在批判教条主义,却放过了修正主义,修正主义应当批判。不批判教条主义,许多错事不能改正。现在应当开始注意批评修正主义。教条主义走向反面,或者是马克思主义,或者是修正主义。就我党的经验来说,前者为多,后者只是个别的,因为他们是无产阶级的一个思想派别,沾染了小资产阶级的狂热观点。有些被攻击的“教条主义”,实际上是一些工作上的错误。有些被攻击的“教条主义”实际上是马克思主义,被一些人误认为“教条主义”而加以攻击。真正的教条主义分子觉得“左”比右好是有原因的,因为他们要革命。但是对于革命事业的损失来说,“左”比右并没什么好,因此应当坚决改正。有些错误因为是执行中央的方针犯的,不应当过多的责备下级。我党有大批知识分子新党员(青年团就更多),其中有一部分确实具有相当严重的修正主义思想。他们否认报纸的党性和阶级性。他们混同无产阶级新闻事业与资产阶级新闻事业的原则区别,他们混同反映社会主义国家集体经济的新闻事业与反映资本主义国家无政府状态和集团竞争的经济的新闻事业,他们欣赏资产阶级自由主义,反对党的领导。他们赞成民主,反对集中。他们反对为了实行计划经济所必须的对文化教育事业(包括新闻事业在内的)必须的但不是过分集中的领导、计划和控制。他们跟社会上的右翼知识分子,互相呼应,联成一起,亲如兄弟。批判教条主义的有各种人,有共产党人一一马克思主义者;有括弧内的“共产党人”,即共产党的右派一一修正主义者;有社会上的左派、中派和右派;社会上的中间派是大量的,他们大约占全体党外知识分子70%左右,左派大约是20%,右派大约占l%、3%、5%到10%,依情况而不同,最近这个时期在民主党派中和在高等学校中右派表现最坚决,最猖狂,他们认为中间派是他们的人,不会跟共产党走了,其实是做梦。中间派有一些是动摇的,是可左可右的。现在右派的进攻还没有达到顶点,他们正在兴高采烈。党内党外的右派都不懂辩证法。物极必反。我们还要让他们猖狂一个时期,让他们走到顶点,他们越猖狂对于我们越有利益。他们说:怕钓鱼。或者说诱敌深入聚而歼之。现在大批的浮到水面上来了,并不要钓。这种鱼不是普遍的鱼,大概是鲨鱼吧,具有利牙喜欢吃人,人们吃的鱼翅就是这种鱼的浮游工具。我们和右派的斗争,集中在争夺中间派,中间派是可以争取过来的。什么拥护人民民主专政、拥护人民政府、拥护社会主义、拥护共产党的领导,对于右派来说,都是假的,切记不要相信。不论是民主党派的右派,教育界的右派,文学艺术界、新闻界、工商界的右派,都是如此。

有两派最坚决,左派和右派。他们互相争夺对中间派的领导权。右派的企图是先争局部后争全局,先争新闻界、教育界、文艺界、科学界的领导权。他们知道共产党在这方面不如他们,情况也正是如此。他们是“国宝”,是惹不得的。过去的肃反、三反、思想改造,岂有此理,太岁头上动土。他们又知道许多大学生属于地主、富农、资产阶级的儿女,认为这些是可以听右派号召起来的群众。有一部分有右倾思想的学生,有些可能。对于大多数学生,这样设想则是做梦。新闻界的右派还有号召工农群众反对政府的迹象,他们反对扣帽,这是反对共产党扣他们的帽子。他们扣共产党的帽子,扣民主党派中的左派、中派和社会各界左派和中派的帽子则是可以的。

几个月以来,报纸上从右派手上飞出来多少帽子,中间派反对扣帽子是真实的,凡是对中间派过去所扣的一切不适当的帽子都要取掉,以后也不要乱扣。在肃反中,在三反中,在思想改造中,某些真正作错了,都要公开纠正,不论对什么人的。只有扣帽子一事对右派当别论。但是也要扣得对,确实是右派才给他扣上右派这顶帽子。除个别例外,不必具体指名,给他们留个回旋余地,以利在适当条件下妥协下来。所谓1%、3%、5%、10%的右派是一种估计,可能多些,可能少些。在各个单位内,情况又互相区别,必须确有证据,实事求是,不可过分,过分就是错误。

资产阶级和曾经为旧社会服务的知识分子的许多人总是要强烈地表现他们自己的,总是留恋他们的旧世界,对新世界总有些格格不入,要改造他们,需要很长时间,而且不可用粗暴的方法。但是必须估计到他们的大多数较之解放初期是大有进步了。他们对我们提出的大多数批评是对的,必须接受,只有一部分不对,应当解释。他们要求信任,要求有职有权是对的。必须信任他们。必须给予职权。右派的批评也有一些是对的,不能一概抹杀。凡是对的就应该接纳。右派的特征是他们的政治态度右。他们同我们有一种形式上的合作,实际上不合作,有些事合作,有些事不合作,一遇空子可钻,如像现在这样时机,就在实际上不合作了,他们违背愿意接受共产党领导的诺言,他们企图摆脱这种领导。而只要没有这种领导,社会主义就不能完成,我们的民族就要受到绝大的灾难。

全国有几百万资产阶级和曾为旧社会服务的知识分子,我们需要这些人为我们工作,我必须进一步改善同他们的联系,以便让他们更有效的为社会主义事业服务,以便进一步改造他们,使他们逐步的工人阶级化,走向现状的反面。大多数人一定可以达到这个日的。改造就是又团结又斗争,以斗争之手段达到团结之目的。

斗争是互相斗争,现在是许多人向我们进行斗争的时候了。多数的批评合理或基本上合理,包括××大学××教授那种尖锐的没有在报纸上发表的批评在内。这些人批评的目的就是希望改善互相关系,他们的批评是善意的。右派的批评往往是恶意的,他们怀着敌对情绪。善意恶意不是猜想的,是可以看得出来的。

这次批评运动和整风运动是共产党发动的。香花与毒草同生,牛鬼蛇神与鳞凤龟龙并长,这是我们所料到的。也是我们所希望的。毕竟好的是多数,坏的是少数。人们说钓大鱼,我们说除毒草,事情一样,说法不同。有反共情绪的右派分子,为了达到他们的企图,他们不顾一切想在中国这块土地上,刮起一阵害庄稼毁房屋的七级以上的台风。他们越作得不合理,就会越快地把他们抛到过去假装合作,假装接受共产党领导的反面,让人们认识他们不过是一小撮反共反人民的牛鬼蛇神而已,那时他们就会把自己埋葬起来,这有什么不好呢?

右派有二条出路,一条夹紧尾巴,改邪归正,一条继续胡闹,自取灭亡。右派先生们,何去何从?主动权(一个短时间内)在你们手里。

在我们国家里,鉴别资产阶级知识分子在政治上其伪善恶,有几条标准。主要是看人们是真正要社会主义和真正接受共产党的领导。这两条他们早就承认了,现在有些想翻案,那不行。只要他们翻这条案,中华人民共和国就没有他们的位置。那是西方世界(一名自由国家)的理想。还是请你们到那里去吧?

大量的反动的乌烟瘴气的言论,为什么允许登在报纸上?这是为了让人民见识这些毒草、毒气,以便除掉它,灭掉它。

你们这篇话为什么不早讲?为什么没有早讲,我们不是早已讲了,一切毒草必须锄掉吗?你们把人们划为左、中、右,未免不合理吧?除了沙漠,凡有群众的地方都有左、中、右,一万年以后还会这样。为什么不合情理呢?划分了使群众有一个观察人们的方向,便于争取中间,孤立右派。

为什么不争取右派?要争取的。只有他们感到孤立的时候,才有争取的可能。现在他们尾巴翘到天上去了,他们企图灭掉共产党,那肯就范?孤立就会起分化,我们必须分化右派,我们从来就是把人群分为左中右,或叫进步、中间、落后,不自今日始,有些人健忘罢了。

是不是大“整”?要看右派先生们今后的行为作决定,毒草是要锄的,这是意识形态上的锄毒草,“整”人又是一件事,不到某人“严重违法乱纪”是不会受整的。什么叫严重违法乱纪?就是国家利益和人民利益受到严重损害,而这种损害是在屡诫不听,一意孤行的情况下引起的。其他普通犯错误的人,更加是“治病救人”。这是一个恰当的限度,党内党外一切如此。“整”也是治病救人。

要多少时间才可以把党的整风任务完成,现在情况进展甚速,党群关系迅速改善,看样子几个星期,有的几个月,有的一年左右(例如农村)就可完成,至于学习马列主义提高思想水平,则时间就要长些。

我们同资产阶级知识分子又团结又斗争将是长期的,但有两三个五年计划估计就差不多了。共产党整风告一段落以后,我们将建议各民主党派和社会各界实行整风,这样将加速他们的进步,更易孤立少数右翼分子。现在是党外人士帮助我们整风,过一会我们帮助党外人士整风。达就是互相帮助,使歪风整掉,走向反面,人们正是这样希望我们的,我们应当满足人民的希望。


