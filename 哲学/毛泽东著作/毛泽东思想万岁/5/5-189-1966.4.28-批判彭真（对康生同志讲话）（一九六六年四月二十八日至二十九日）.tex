\section[批判彭真(对康生同志讲话)(一九六六年四月二十八日至二十九日)]{批判彭真(对康生同志讲话)(一九六六年四月二十八日至二十九日)}
\datesubtitle{(一九六六年四月二十八日)}


北京一根针也插不进去,一滴水也滴不进去。彭真要按他的世界观改造党,事物是向他的反面发展的,他自己为自己准备了垮台的条件。这是必然的事,是从偶然中暴露出来的,一步一步深入的。历史教训并不是人人都引以为戒的。这是阶级斗争的规律,是不以人们的意志为转移的。凡是在中央有人搞鬼,我就号召地方起来攻他们,叫孙悟空大闹天宫,并要搞那些保“玉皇大帝”的人。彭真是混到党内的渺小人物,没有什么了不起,一个指头就通倒他。“西风落叶下长安”,告诉同志们不要无穷地忧虑。“灰尘不扫不走,阶级敌人不斗不倒。”

赞成鲁迅的意见,书不可不读,不可多读。不读人家会欺骗你。

现象是看得见的,本质是隐蔽的。本质也会通过现象表现出来。彭真的本质隐藏了三十年。

要不要告诉阿尔巴尼亚同志?没有什么不可告人的。


