\section[关于增产节约的讲话(纪录稿)(一九五五年)]{关于增产节约的讲话(纪录稿)}
\datesubtitle{(一九五五年)}


我们是人口众多经济落后的大国,经济水平生活水平均很低,建设社会主义,十五年内是过渡时期,中心是工业化,打物质基础。在现有条件下,继续贯彻党的艰苦奋斗传统。资金唯一来源靠自己积累。第一个五年十个指头,只做了一个。二、三个五年资金需要更大,不厉行节约无法完成任务。如果非生产建设投资突出了,会破坏工农联盟和全国团结,农民为什么不进城?十五年内,不可能样样现代化,样样社会主义。是打基础阶段。非重点建设,非近代技术装备的工厂,第一个五年感到困难,第二第三个更要大。削国防不可能,削行政人员,也非安置生产,非加强生产不可。不可设想削减建设国防人员解决问题。只有全面节约解决问题。

过去认识不深,贯彻不透。如工厂,福利设施均社会主义设施。已无福利可增加。先抓生产、开工。

农场也太近代化了。牛场120万平方米,先进社会。

工房、宿舍、高楼、大厦,百把万,住不起。长春汽车厂、乌拉尔厂,宿舍太漂亮了。厨房,西太后出来也要说话了。这是不了解建设困难,忘记传统,违背工业化方针及与农民共艰苦患难的精神。

生产和生活不平衡。节约为了增加积累。

一、除近代化技术设备以及和设备必要配合的厂房(国外的按国外设计,国内的按可能削减),其它一律降低造价。和延安窑洞看齐,办公室、教室、农场、宿舍、车站等按过渡时期标准,用十五年、二十年。一律砖木结构,甚至土坯子,竹泥墙。

一类是现有大城市,可以砖木。

二类新的单独城市、厂矿,要用土坯。

办公室、学校50—70元

宿舍20—50元

不同宿舍不同房租。

仓库40—50元

车站50—70元

首都盖大楼装样子,须国务院批准。

工程部六—八月拟出标准。

二、新建工厂、铁道、农场、福利、设施,根据生产、利润、业务的情况,逐年设施。先抓重点,如宿舍、卫生所无利润以前不建设。不可一下子社会主义,是远景,不是一下子社会主义。首先礼堂不可搞。

三、新建企业其它方面厉行节约。轻工业至多一年的筹备机关,以前筹备由部局直接筹备。

设计,照标准。重工业部门到苏联的人太多。

实习生的待遇要降低。

每个环节,严加控制。厂内办公室,沙发一律不准,也不买收音机。不要厂长和职工生活脱离太远。

四、现有新城,计划均要修改,不要搞社会主义化,要利用旧城。工业要相当疏散。

(1)不搞高大建筑。一层两层。

(2)非工业集中的城市,根本不建城市。如职工用水,可用地下水,用自来水者公共使用水龙头。

马路要,绿化面积慢慢来。

先工业化,后“社会主义化”。二十年后,再用新规划。

五、今年准备建筑的高楼大厂,未动工的一律停止。

六、地质勘探:

(1)勘探量不必增加。

(2)希望增加的,增加50%工作量。

七、建筑部六、七、八个月交出标准。

(1)生产方面:

1.降低成本:工业6%加到7.5%;铁路2.5——3.5%;商业2.9——13.9%;

2.节约原料,轻工业、棉花、卷烟量。

3.重工业完成新产品试制计划,上半年49%,不及一半。

4.一律停止收新工人,先内部清理调整,现在都核不清。现在人多了。

5.每一环节,直至清理垃圾。鞍钢月搞五万吨钢铁,指标到车间。

(2)生活消费:

1.除招待外宾外,一律只供清茶一杯,一律不请客、会餐。

2.办公室不买沙发,首长办公室,两个小沙发,用会议桌。

3.公文表报节约,非国家统计局批不可发。

4.减少汽车使用。


