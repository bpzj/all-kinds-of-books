\section[在延安纪念“一二.九”四周年大会上的讲话(节录)(一九三九年一二月九日)]{在延安纪念“一二.九”四周年大会上的讲话(节录)}
\datesubtitle{(一九三九年一二月九日)}


“一二·九”运动有着重大的历史意义,“一二·九”运动是伟大的抗日战争的准备,这和“五四”运动是第一次大革命的准备一样。“一二.九”运动推动了“七七”抗战;准备了“七七”抗战。当“五四’运动以后,到“五卅”运动,中国形成了一个全民的运动。全国各地工人罢工,学生罢课,商人罢市。这个时期,“五四”运动准备了舆论,准备了干部,准备了思想和准备了人心,到后来才有一九二五一一九二七年的革命。

“一二.九”是抗战动员的运动,是准备思想和干部的运动,是动员全民族的运动。“一二.九”发生在红军北上抗日到达了西北之时,这说明“一二.九”学生运动与红军北上抗日两件事情的结合,这两件事帮助了全民抗战的发动。中共中央在“八一宣言”之后,他们反对当时对青年的压迫,他们反对日本帝国主义侵略中国,他们要求停止内战,一致抗日,这个运动的发生,轰动了全国。“一二.九”运动将为历史上一个大的纪念。

中国是永远向前发展的,中国人民现在已经懂得了很多道理,他们认识清楚,他们要进步,他们要民主,要参政;这民主和宪政改革的真正实现还需要大家起来奋斗。现在,抗战中有一批专门倒退的人,他们拖住中国要倒退,这是现在中国的黑晴势力压迫光明势力,光明与黑暗的斗争,这叫做压迫自由,然而全国青年全中国工农大众也有另外一种自由,这叫做反抗黑暗势力压迫的自由,他们不准倒退,他们要坚持抗战,反对投降,坚持团结,反对分裂;坚持进步,反对倒退。

自“五四”运动起,共产党就与知识分子结合在一起,“一二。九”运动中,共产党起了骨干的作用。这些说明知识分子要与共产党结合,要与广大的工农群众结合,要与革命的武装队伍结合,要与八路军、新四军结合。共产党非常欢迎知识分子,反对的是那少数坏知识分了。现在很多青年知识分子没有自由,没有走路之权,知识分子一定要与革命军队结合起来,笔与枪结合起来,打倒日本帝国主义,抗战胜利,建立一个民主共和国出来!

