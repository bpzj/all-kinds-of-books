\section[在八届六中全会上的讲话(一九五八年十二月四日下午)]{在八届六中全会上的讲话(一九五八年十二月四日下午)}
\datesubtitle{(一九五八年十二月四日)}


帝国主义很乱,我们很好。一乱一好,就这样下去。

二五计划没有四年。我们再搞四年,过去九年加四年,十三年。过去九年不能说是全党办工业,全党办工业只靠工业厅办,这算全党办?,全党一办。就有把握了,我们办政治三十年了。

二五计划搞××万吨,一年加××万吨。……

要这么多钢,要这么多铝,从哪里来?能不能搞那么多?需不需要那么多钢?有钢无铜、铝。就搞不到那么多电,因此缺电力,钢摆下没用。一年××万吨.史无前例。苏联战后每年才三百万吨钢。铁路、轮船不要电行。有蒸气机、内燃机也行嘛!

横直要看。因为我们人多。到处敲敲打打,中央、地方。大、中、小,条件就大变.一、人多。自然规律,并非马克思主义,二、办法。算是马克思主义了,学了苏联.你不搞.我就搞。与斯大林唱反调。他不强调农业。重工业太重,轻工业也注意得不够。我们强调工农业并举,他不强调政治也单纯强调了技术,我们强调政治挂帅,他采用群众路线不够,我们就强调群众路线。我们要稳步。假如每年能增加××万吨钢,这是一种想法。夏季是高潮,现在要定案,不能只讲精神。还有一种可能,办不到,放在少于××万吨,××万吨。超过××万吨有几个方案.最高一九五九年到××万吨,一九六○年到××万吨,一九六二年到××××万吨.铁路选线有个办法:要选六条线,从其中选一条最合理最可靠的.现在我看也有几条嘛!今年增加××万吨钢算一条。插个××××万吨也算一条,××万吨算一条,××多万吨也算一条,选嘛!

中央全会二十八号开。他们(指帝国主义报纸)给我们开出题目来了,讨论打金门是否成功,日本问题,支持××××对西柏林意见,还有一千零十万吨钢,人民公社,等等。

总之,有两句对联。“轻重缓急要排队,自力更生小土群”。政治挂帅,对嘛!“挡箭牌”,排队是个挡箭牌,还有个挡箭牌就是小土群。政治挂帅,这也都对嘛!

我们现在中央对地方,比打仗时好些。那时,不给粮食,不给军饷。不给衣服,什么都不给嘛!现在包钢还给些材料嘛!

内蒙古写了个报告,那样客气。“可否”?“考虑”?不是“可否’考虑’,硬是要请求。


