\section[与世界学联代表团柯乐满先生雅德先生傅路德先生雷克难先生之谈话(一九三八年七月二日)]{与世界学联代表团柯乐满先生雅德先生傅路德先生雷克难先生之谈话}
\datesubtitle{(一九三八年七月二日)}


第一个问题:目前边区在中国的意义与作用是什么?请你告诉我们,我们很需要明白这点。

答:明白了边区的性质,才能明白它在中国的意义与作用。边区是一个什么性质的地方呢?一句话说完,是一个民主的抗日根据地。

首先,在民众方面,都有他们自己的组织。边区人民,只要在抗日原则下,都有他们言论出版集会结社之自由,不论是工人、农民、商人、学生、知识分子、妇女、儿童以及宗教团体、自由职业者的团体都有这种自由。共产党与边区政府从而积极地扶植他们,帮助他们,使他们更普遍的发展下去。这里仅仅限制汉奸的活动,对于汉奸,是不给任何自由的。

其次,边区已成了直接抗战的区域,这里为八路军的一部分,还有地方武装部队。这些军队,其内部官长与士兵的关系,其对外与人民的关系,也都有一种民主的精神,能够官兵打成一片,军民打成一片,使在抗日战争中表现不能被战胜的力量。

其次,边区的教育,同样是抗日的与民主的,你们从抗日军政大学与陕北公学,就可看出这种精神。

其次,经济方面也是这样,以有利抗战为主旨,而以民主精神经营之。例如这里颇为发达的合作社,得了土地后的农民之于农业,都是按照这种精神,使人民的生活得到改良,同时又有利于抗日事业。这里是有租税的,但没有苛捐杂税,实行一种统一的累进税,符合于上述的主旨。

还有,也是最重要的,就是边区各级政府都是由人民投票选举的。这里说明一点,就是有些人说:“知识落后的农民,不能实行选举制度”,这是不符合事实的。这里实行民选的结果并不坏,每个有眼睛的人都能看到。当人民选举他们所欢喜的人去办政府的事的时候,办得很不错,这比派官办事制度要好得多,对于动员人马力量参加抗日战争,特别积极而有效。与过去苏维埃不同的是推广了选举与被选举的范围,即不论工人、农民、妇女、知识分子、学生、商人、有产者,只要不反对抗日而年满十八岁者,都有选举与被选举权。

以上所述的各方面,把抗日战争与民主制度结合起来,都能取得很大的效果。在这个制度之下,无论那一种职业的人,无论从事什么活动,都能发挥他们的天才,有什么样的人都可以表现出来。

这就是边区的性质,边区的特点,明白了这种性质与特点,就可以明白它在全国的意义与作用是什么了,全国也应采取这个制度,应把抗日战争与民主制度结合起来,以民主制度的普遍实行去争取抗日战争的胜利。如果全国人民都有言论出版集会结社的充分自由,全国军队中官兵打成一片,军民又打成一片,全国教育也以民主精神实行之,全国经济建设发动了人民的力量并与改良人民生活相结合,全国各级政府都实行选举制度,置有各级人民的代议机关,而一切这些都是为了争取抗战的胜利,那战胜日本就指日可期了。民主制度在外国已是历史上的东西,中国则现在还未实行,边区的作用,就在做一个榜样给全国人民看,他们懂得这种制度是最于抗日救国有利的,是抗日救国唯一正确的道路,这就是边区在全国的意义与作用。这种制度要全国采取,是需经过全国人民切实了解,认为可行,然后才能实行的,所以我们欢迎各党各派与无党无派的人都来看一看,来看的人也不少,青年学生尤多,除了少数人说这种制度不好之外,大多数人都是说好的,这是可以引为庆幸的事。边区是中国之一部分,在中央政府领导下,与中国其他部分是一样的,但有一点不同,这里是实行了民主制度的区域,这就是边区的特点,我们希望这个特点普及于中国。

现在有些人对于边区有两种不正确的观点:一种说边区什么都不好,有少数顽固分子这样说,这种说法显然不合事实。另一种说这里象个神圣的天堂,什么缺点都没有,这种说法也过分了。正确的说应该是这样:这里的民主制度与抗日精神是很好的,值得供全国人民仿效与参考。但这里的工作还受某些条件的限制,例如,物质困难即其一例,许多工作还得继续努力,方能更好,并不是样样都好得不得了,不须再求进步了。因此,欢迎外界的批评,加重了这里工作人里的努力,这也是要指出的。承诸位好意来延安参观,我就欢迎诸位的批评,指出这里的缺点,以便加以改正,使之更利于抗日救国的伟大事业。

第二个问题:目前中共在全中国的作用是什么?

答:这个问题很简单。坚持抗战,坚持统一战线,坚持持久战,这就是目前中共的基本主张,它在全国的工作与作用也就在这里。

什么叫做坚持抗战?妥协还是坚持抗战,这是存在着的问题。我们是主张抗战到底,反对任何妥协的,我们愿和国民党及其他党派与全国人民一道,坚持抗战,绝不动摇,直至收复失地打到鸭绿江为止。

什么叫做坚持统一战线?就是全国团结到底,只有坚持全国的团结,才能坚持抗战。现在虽然已有了全国的团结,但还要更加团结,不但团结几个党派就行了,还须团结全国人民。只有全国各界人民都团结在一定的组织之中,都发动了抗战的积极性,是算巩固与扩大了统一战线。

什么叫做坚持持久战?中国现在有两种人,一种人说:“中国会亡,不能作持久战。”另一种人说:“中国很快可以把日本帝国主义赶出去,也无需于持久战。”我们认为这两种意见是不对的。首先,中国决不会亡。理由是日本虽强,但它先天不足,国内外矛盾很多;中国虽弱,但是国大,又有许多国内外的优良条件。因此,中国虽在战争的一定的时期损失了许多地方,但仍然坚持战争,取得最后的胜利。但要很快的打胜日本也困难,因为虽有争取胜利的可能条件,但不能很快的全部的实现,这不论中国的进步也好,日本的内溃也好,国际的援助也好,都非有相当长的时间不能达到目的。所以我们应准备长期战争,不能希望即别胜利。

这样,现在的方针应该是:第一,坚持抗战,第二,坚持抗日统一战线,第三,坚持持久战。

中国共产党在全国干些什么事情呢?就是干的这些事情。中央愿与各党各派及全国人民紧密的团结起来,一定要把这个方针贯彻下去,这就是中共在全国的作用。

第三个问题:中国是否有什么条件可以缩短这一持久战时间呢?

答:要缩短战争时间,必须加强三个条件。第一个条件,是中国的统一战线巩固和扩大,这是基本的,在统一战线的方针下,各项工作须大大的发展与进步,这些工作多发展与多进步一分,战争期间就能缩短一分。第二个条件,是日本国内人民的帮助。现在这种帮助已经开始,譬如他们的士兵不愿意战争,有自杀的,有投降的,有发反战争传单的,日本人民反战思想也在发展中。如果日本的士兵和人民更多的觉悟一分,战争期间也就可以缩短一分。还有一个条件,是世界各国的帮助。我们需要世界学生的帮助,需要世界人民的帮助,也需要各国政府的帮助。如这些帮助更多更快,那末我们的战争也就会缩短。这三个条件是互相关联的,如果中国加快进步,加快团结,就能使日本国内的助我力量加快发展,也能使世界各国帮助我力量加快发展。如果日本及世界各国的助我力量加快发展,也能使我们国内的抗日力量加快的发展与进步。中国给日本帝国主义的更大的打击,也就是给了日本人民的帮助,将使日本人民的解放斗争发展更快,对世界也是一样,中国的抗战,同时也就是帮助世界人民反对共同的敌人。所以中国、日本、世界各种反法西斯的势力是互相影响的,互相帮助的。世界和平不能分割,世界是一个整体,这是现在世界政治的特点。这三个条件多具备一分,则战争时间就能缩短一分,这是中国共产党与各抗日党派及全体人民的任务,同时也是日本及世界各个先进政党与全体人民的任务。我们的战争是持久战,但我们应极力争取尽可能缩短时间的条件,如没有这些条件,则缩短时间只是空想。

第四个问题:抗战获得最后胜利之后,中共的主要任务将是什么?

答:抗战胜利后,共产党的主要任务,一句话,是建立一个自由平等的民主国家。在这个国家内,有一个独立的民主政府,有一个代表人民的国会,有一个适合人民要求的宪法,在这个国家内的各个民族是平等的,在平等的原则下,建立联合的关系。在这个国家内,经济是向上发展的,农业、工业、商业都大大的发展,并由国家与人民合作去经营,订定八小时工作制,农民应该有土地,实行统一的累进税,对外国和平通商,订立互利的协定,在这个国家内,人民有言论、出版、集会、结社、信仰的完全自由,各种优秀人民的天才都能发展,科学与一般文化都能提高,全国没有文盲。在这个国家内,军队不是与人员对立的而是与人民打成一片的。这样的国家,还不是社会主义的国家,这样的政府,也不是苏维埃政府,乃是实行彻底的民主制度与破坏私有财产原则下的国家与政府。这就是中国的现代国家,中国很需要这样一个国家。有了这样一个国家,中国就离开了半殖民地与半封建的地位,变成了自由平等的国家,离开了旧中国变成了新中国。共产党愿意联合全国的一切党派与人民,大家努力建立这样一个国家,是全国人民几十年来所希望所力争的,也是孙中山先生所希望所力争的。进行建立这样一个国家,也是把日本赶到鸭绿江之后的第二天才开始的,抗战过程中的各种工作,就都与建立这样的国家有关系。不过许多重要工作是要在抗战胜利之后才能完成,例如基本的经济建设等等,在抗战中,是要具备建立这样一个国家的先决主要条件,目的是把日本帝国主义赶出去。这样的任务,不但是共产党的,也应该是国民党与其他革命党派的,同时是全国人民的,这是中国的历史任务。

第五个问题:你以为现在中国的学生及青年在抗战中的主要任务是什么?世界学生与青年的援华运动中的主要任务是什么?

答:中国青年的任务,可以分为一般的与特殊的,一般的任务,与前面所讲的相同,就是坚持抗战,坚持统一战线、坚持持久战,驱逐日本帝国主义,建立自由平等的民主共和国,这是中国任何年龄,任何职业的人民之共同任务,没有什么分别的,有分别的,是其特殊的任务。中国青年们的特殊任务是什么?就是争取自身的特殊利益。例如改良教育与学习,在学习中有参加救亡运动的权利,有组织学生与青年团体及组织救亡团体的权利,十八岁以上的青年有选举权与被选举权。贫苦学生有免费入学之权,青年应大批上前线等等,说到世界青年们帮助中国抗战的主要任务,我想,首先是经过世界学联,使世界大多数青年与学生了解要共同反对日本帮助中国的必要,是使他们了解这种工作是与他们自身的利益有联系的。因为和平不可分割,法西斯的世界侵略是世界和平的仇敌。其次,学生是联络人民的桥梁,经过学生,使各民族的人民懂得反对日本帮助中国的必要以及这种工作与他们自身利益的关系.至于具体的帮助方法,例如用口头和文字作宣传,劝告人民与政府给我们的物质上的帮助,不买日本货与不卖货给日本,直至组织国际抗日义勇军,准备于适当时机来华参加战争等等。诸位代表着广大的国际学生团体来华视察,给我们以广大的同情,全中国人都感谢你们,我代表中国共产党与中国人员向你们致敬,希望你们回去之后,把中国伟大抗日战争的真相带给世界学生与人民。我们与你们永远团结起来,为中国的自由平等而战,为世界的永久和平永久幸福而战。

