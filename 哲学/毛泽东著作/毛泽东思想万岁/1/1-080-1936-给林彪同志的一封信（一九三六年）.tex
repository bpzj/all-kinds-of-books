\section[给林彪同志的一封信(一九三六年)]{给林彪同志的一封信}
\datesubtitle{(一九三六年)}


林彪同志:

你的来信我完全同意,还有一点,就三科的文化教育(识字、作文、看书报等能力的养成)是整个教育计划中最重要最根本的部分之一,如果你所说的实际与理论并重,文化工具就是实际的一部分。如果你所说的实际与理论联系,文化工具乃是能够而且必须使用了去把二者联系起来的。如果一切课程都学好了,但不能看书作文,那他们出校后的发展仍是很有限的。如果一切课学了许多,但不算很多,也不算很精,但会看书作文,那他们出校后的发展就有了一种常常用得的基础工具了。如果你同意此意见,那我想应在二、三两科,在以后的四个月中,把文化课(识字`看报、作文三门)更增加些,我想把它增加到全学习时间(包括自修时间)的四分之一到三分之一,请你考虑这个问题,定期检查时,文化应是重要的检查之标准之一。

布礼

<p align="right">毛泽东

<p align="right">二十六日十四时

