\section[在郑州会议上的讲话(三)(一九五九年二月二十八日)]{在郑州会议上的讲话(三)}
\datesubtitle{(一九五九年二月二十八日)}


我们和农民的关系有点紧张。一是粮食问题,二是供应问题。在北京看了一些材料。就想这个问题。在天津、郑州找省委、地委同志谈,各地都在解决这个问题。反本位主义、反个人主义,情有可原,赦你无错,不给处分。农民瞒产私分是完全有理由的,不瞒产私分不得了。去年11月以来,这股“共产风”白天吃萝卜,晚上吃大米,几亿农民和小队长联合起来抵制党委,中央、省、地、县是一方,那边是几亿农民和他们的队长领袖作为一方。管理区生产队队长是中间派,动摇于两者之间。就是我们手伸得太长,拿得太多,他们就不得不瞒产私分。不上调粮食,不给予处分,实际上是承认他们有权。从九月起,有一个很大的冒险主义错误。这个问题不很好解决,很可能会犯斯大林的错误。农业不能发展。河南公社生产费20%,积累、税收50%,农民只分到30%,瞒15%,实际拿45%,猪归公社,大白菜也归公社,平均主义就是冒险主义。我们的决议提了按劳分配,至于如何实行,没有讲,生产责任制提了。如何实行。也没有讲。谁料到大丰收出粮食问题。今年要出个安民布告,生产多少,征购多少,吃多少。生产队养的猪归谁?卖东西的钱归谁?一盘棋大部分是五亿农民,第一是安排社员的生活。第二是安排积累,公社积累18%;,加上国家税收7%,共25%,现在很多地方超过了这个比例,是很危险的,就会犯斯大林的错误。现在统得太多,公社至少有十统。一统税收,二统购,三统积累,四统生产费,五统公益金,六统管理费,七统工业,八统文教,九统供给、工资……。我说,本位主义只能是部分的本位主义,不能都戴本位主义的帽子,几亿农民都戴这顶帽子不舒服,要去掉这顶帽子。能完成征购任务而不完成,可以按个本位主义,基本上大部分是基本权利,不是本位主义。

讲四个问题:一是所有制问题,二是劳动问题,三是分配问题,四是干部下放当社员。

一、所有制问题。公社集体所有制,少则三、四年,多则五、六年,或者更多一些时间逐步完成由基本上是生产队(即过去的高级社)的所有制过渡到公社所有制。大集体和小集体的矛盾,要承认它合法,现在基本上是他们的所有制,公社所有不了,他们就瞒产私分。目前只能是部分的公社所有制,即基本队有,部分社有,过去没有搞清楚。农民有两面性,农民还是农民。上次郑州会议前,讲农民觉悟高,大兵团作战,共产主义风格。秋收以后,瞒产私分,名誉很坏,共产主义风格那里去了?农民还是农民,农民只有如此,应该如此。一下子搞共产主义不可能。有人说,这是向农民让步问题,在某种意义上来说,是向农民让步,但基本上不是让步,是我们要得太多。把卖猪卖大白菜的钱交给公社去了,不给生产队。农民怕共产。当然他们就杀猪、吃菜。实际上大批公社的鸡都共产了。所以把公鸡杀掉,母鸡藏了。

现在的公社是联邦政府。要由联邦政府逐步过渡到统一政府。变秦始皇就危险,十三年亡国。隋炀帝三十一年灭亡。一不能统一拉平分配,二积累、社办事业不能过多,要有个过渡。现在社办工业太多。社揽的事情太多。羊毛出在羊身上,羊是农民和生产队,要在农民和生产队上刮羊毛,所以产生对抗,站岗放哨。不要砍富队补穷队。而是要帮助穷队向富队看齐,这就需要时间。我反对平均主义和左倾冒险主义。手伸得太长,用的劳动力太多,工业办得太多,竭泽而渔,可能影响农业三十年不能发展。所有制只能基本队有,部分社有,逐步转过来变为基本社有,部分队有。由互助组到高级社,没有过渡不行。这样作,基本上不是向农民让步的问题,而是一个逐步发展的过程。公社所有制只能经过几年引导农民一步一步地去完成,而不能在目前一下子去完成,要办就违背客观规律,请你自己缩手。由互助组到高级社,经过了四年(1953年到1956年),由高级合作社集体所有制到公社集体所有制,可能也要经过三、四年,或者更多一点时间。公社一成立,就完成公社所有制。这种想法是错误的。问题是将穷队提高到富队的生产水平,这样一个过程,所以要有较多时间。

再一个问题,就是公社工业化、机械化、电气化、文化教育事业等,只能逐步发展,逐步有所发展,不能一口气办得很多很大,否则会犯冒险主义错误。扶助穷队向富队看齐拉苏联二千万吨钢来补中国,生产者会反对的。这个过程就是公社工业化、农业机械化、电气化,国家工业化、人民社会主义共产主义觉悟程度和道德品质的提高、文化教育技术水平提高过程。当然,这还是第一阶段,以后还有几个阶段,才能完成社会主义建设任务。只有这样,才能作到公社所有制,也即接近全民所有制。在这整个过程中,其性质还是社会主义的,其分配原则还是按劳分配的。但是在这个过程的第一阶段内,从1958年算起,少则三、四年,多则五、六年,人民公社集体所有制完成了。现在是基本上队有,社只有部分所有。假如现在什么都归县,什么都由公社统,就要统翻几亿农民。在三四年,五六年内,人民公社集体所有制完成了可能有一部分或大部分转到全民所有制。1958年,粮、棉、油、麻大丰收,但是,却在最近四个月大闹粮食、油料不足的风潮,中央、省、地、县、社、管理区六级党委大批评生产队、生产小队的本位主义(反本位主义,我走了三个省,觉得是保护正当权利,幸得有此一手,情有可原,或者是初犯,或者是宣传工作没有赶上),即所谓瞒产私分。另一方面生产队、生产小队则普遍一致瞒产私分,深藏密窖,站岗放哨,进行反抗,保卫他们的产品,反批评公社同上级的平均主义、抢产共产。我以为生产队和群众的作法基本上是合理的。而且合理的。他们基本上不是不合法的本位主义,而是合法的正当权利.因为土地劳力是他们的,劳动结果——产品,应当是他们的。

这里有两个问题;一是穷队、富队拉平的平均主义分配方法,是由穷队无偿占有别的一部分劳动成果,这是违反按劳分配原则的。二是国家农村税收只占农业总产值7%左右。不算太多。农民是赞成的,但是很多公社和县从公社的总收入中抽出的积累太多。例如河南积累占26%,如税收7%,共33%,占总收入的三分之一。这是农民对国家的投资。这还不算修铁路、水库等义务劳动,以及很低的工资(如修三门峡)。再扣除1959年生产费20%。再加上公益金、管理费,就达53%以上,社员个人所得只有47%以下,我认为这个数目太少了。

公社1958年秋季成立,刮起一股“共产凤”。一是穷富拉平,二是积累太多,三是共各种产,其中有猪、鸡、鸭无偿归社,还有部分的桌、椅、板凳、锅、盆、刀子、碗、筷归公共食堂(还能算废铁无偿收去),以及自留地归公。这几项“公”,应当加以分析。有些是正确的,如大部分自留地归社,这是正常的,有些是不得不借用的,食堂房屋和桌椅、板凳,有些则是不应当归社而归社的,如全部的猪、鸡、鸭。有一部分猪作价归社是可以的。这样一来,共产之风就刮起来了。无偿占有他人劳动成果,这是不许可的。我们曾经无偿剥夺过帝国主义的财产,但只限于德、日,意,英美是打日本的同盟国.并没有剥夺过。其中有些是征用的,有些是挤垮的。我们曾经没收过地主的生产资料,侵犯过地主的一部分生活资料

(粮食、房屋)。所有这些都是劳动人民的劳动成果,不过拿回来而已,所以不叫侵犯劳动成果。对民族资产阶级的生产资料,不是采取无偿剥夺的办法,而采取赎买政策。对富裕农民更要谨慎,我们怎么可以对农民采取无偿占有呢?当然,公共积累不是对消费资料的无偿占有,而是为了扩大再生产。

我的基本思想不是给队给农民戴本位主义的帽子,使县社干部不顶牛,而是去掉包袱,团结一心,讲明道理,不算错误,把政策搞清楚,这是关系到联系几亿农民的小社以上干部的情绪问题。中央、省、地三级比较超然,而县、社首当其冲,下面是大队、小队和广大群众。我们拿多了一点,也要讲清楚,是好心建设社会主义。主意不好,过分的那一部分,得承认手伸得长,其性质是冒险主义.办法是要开六级干部会议。

讲讲党的历史。我们党中央实际上是一个联合委员会,山头很多,一军团三个山头,四方面军四个山头,二方面军两个山头,陕北两个山头,其他各根据地、白区又各有小山头。在延安曾说,要认识山头,承认山头,照顾山头,然后才有可能最后消灭山头,不要骂人家是保守派主义。现在的山头是生产队(过去是穷村、富村)。

公社搞什么,一、拿出几百万吨钢装备农业。七年可以机械化,二、搞公社工业,三、搞多种经营:林、牧、渔。这些全民性部分,将来是会发展起来的.三、四、五、六年之后这些东西多了,相形之下,队生产的东西就少了。

山东吕鸿宾社先以条子、秤、“帽”子去对付,后以一把钥匙(思想),讲明政策,一个楼梯、双方下楼,用这三个办法去对付。

历来讲国家、集体、个人,实际应该是个人、集体、国家。一盘棋应该先安排五亿农民安排适当的粮食。

我们党中央逐步建立权利,从前教条主义,强制执行,实际脱离群众,并没有实权,想多统。统不了,把革命统垮。中央有权是一个过程。工业过去统得太死太多,十大关系提出以后,才逐步调整。适当的集中,适当的统一。要逐步。不要希望一步就集中起来。半路中间,怎么来个这样的干老子——公社。工业也要分级管理,才有地方的积极性.反对绝对集中统一。不要乱戴本位主义的帽子。

富队、穷队还有中间的队。吃饭标准、工资标准应该不同。吃粮食四、五、六、百斤,工资按劳分配。也允许有多有少。如河南省有富队,按劳能分220元,结果只分给130元,砍了90元。这就是无偿占有了人的劳动成果。

二、劳动问题。土地、人力、产品,三种东西,现在名义上归公社所有,而实际上基本上仍然只能是归生产队(即原合作社)所有,现在(1959年以及以后还有一段时间)只有部分的归公社所有。就是说,社的积累,社办工矿场的固定或半固定工人,此外还有一批公益金,一批管理费如此而已,还有一批生产费,不过是过过手而已。这里讲的是人、物。没有讲计划。社的权利还包括统一计划等。雄心不要太大,不要揽权太多,他们的权力只有这样多。我主张权力只搞这样多,要教会公社书记这样作。希望也就在这里。因为年年增加积累,年年扩大社办工业,公社有大、中型的农业机械,社办电气站,社办学校等等。有个三、五、七年,就可以将现在实际所有状况反转过来,由基本上队有,部分的社有。变为基本上社有,部分的队有。就接近于全民所有制。那时当然还会拖一个个人生产资料所有制的尾巴,如极小部分的宅房土地,果树、小农具、家畜家禽等,还为个人所有。公社范围有个人所有,有小集体、大集体,而房屋在公共宿舍大规模建立起来以前,当然是私有的。现在农民一样不怕二样怕。不怕公社拉走土地,因为知道搬不走.怕的是人力产品随便被人拿走——共产,农民就叫“共产”,虽然我们说的是社会主义。现在是要人要财,这是争执的问题。

现在劳动力分配极不合理。农业(农、林、牧。副、渔)劳动力分得太少,工业、服务业、文工团、学校、行政人员分配得太多。一个太少,一个太多。太多的部分必须坚决减下来充实农业。工业方面多了20—30%,山西有一个公社立即减少了30%。服务业人员要大减,一百个人中十个人的比例太大,有的一个伙夫烧十个人的饭。行政人员只允许千分之几,而不是百分之几。山东历城十二万人的东郊人民公社,只有十三人脱产,十五个管理区每区五人,154个生产队每队三人不脱产(不包括财贸人员)。公社不允许有脱产的文工团、体育队、业余的还是可以。生产队与社办工业、与县、与国家争人力,石家庄一个公社跑出去一万一千人。争人的问题是一个严重的问题。重心是把向城里、工业、服务业跑的人赶回来,加强农业战线。

三、分配问题——消费资料的分配问题。队有三等——穷、中、富。粮食、工资的分配应该有差别。社办专业队的工资应该统一。工资可以“死级活评”一月评一次,上死下活。今年要严格规定一个收粮、管粮、用粮的制度,要严格的杜绝浪费,大反浪费。新乡收棉籽号召谁收谁有,结果一天收光。滦县收花生放假三天,谁收谁有,一手交钱,一手交货,就解决了。还是“人不为己,天诛地灭”。去年丰收,反而用粮不足,去年粮食收得粗糙。主要是分配制度问题,反本位主义反不动,制度一万年,还是需要的。要分出国库、社库、队库、堂(公共食堂)库,都必须有制度。一般说来,1958年公社积累搞多了一点,有鉴于此1959年应向群众宣布:公社积累不超过18%,加国家税收7%左右总共不超过25%左右(占工农业总收入),以安人心,以利于提高农民的生产积极性,以利春耕。

四、干部下放当社员、工人的问题。各级干部分期分批下放生产队当社员,舒同当了九天。每年至少一个月到一个半月。一部分下放到工厂当工人,也是一个月到一个半月。中央、省、地、县、社、区六级,要讲清楚六级只有几百万人,另一级是几亿农民及其领袖小队长和生产队长,是大多数,这两方面要打成一片。在若干年内基本实行队所有,分期分批作到公社所有。这样一来,就一定可以达到发展生产,改善关系的两大目的。目前的紧张关系是队和社,有点“国际紧张形势”,主要怕共产。一经济,一政治,以便舒舒服服搞生产,两方面下楼梯,区以上干部左了一点,生产队小队长一般无罪,我们要向公社党委和小队长讲清楚,帽子只扣一部分,该卖给国家的不卖,是本位主义,这样就可以取得广大群众的同情,剩下来的观潮派、算账派就会孤立起来。

三月十五日开会不变。同志和地委同志和县委同志研究讨论,提出意见。我的意见是松一下,让农民多生产,也就会更愿意多出一些。


