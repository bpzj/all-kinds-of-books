\section[中华苏维埃共和国临时政府及工农红军革命军事委员会宣言(注)(一九三三年一月十日)]{中华苏维埃共和国临时政府及工农红军革命军事委员会宣言(注)}
\datesubtitle{(一九三三年一月十日)}


全中国的民众们:

日本帝国主义在法美帝国主义及国际联盟的公开援助之下,已经开始侵入华北。这是更进一步地完全瓜分中国及奴役整个中国。帝国主义强盗的更进一步的侵略,造成了和平居民的整批惨杀,城市与乡村的毁灭,痛苦与饥荒的增加。上海与满洲的惨状,在大部的中国土地上要更残酷地重复着。

因为国民党政府的不抵抗,中国士兵整千整万的被惨杀,在蒋介石命令之下,国民党的军阀们步步撤退,这样帮助日本及其他帝国主义者的更进一步的侵略。同时国民党用一切力量镇压反帝国主义抵货运动的斗争,反对组织反日义勇军。

国民党政府及其军阀政客们解释他们的罪恶行为与卖国勾当的理由之一,就是说中华苏维埃的存在,使他们不能动员一切力量来进行国防。蒋介石就不愿意与日本军阀作战,而用八十万大军去进攻已经创立了自己的苏维埃政府的中国工农,但是中国民众愿意自己保卫自己。许多部队与几十万国民党军队的士兵反对屠杀自己的兄弟姊妹,赞成武装抵抗日本帝国主义。他们开始了解:只有武装民众的民族革命战争,能够胜利地抵抗日本帝国主义的侵略。

中华苏维埃政府与革命军事委员会唾骂国民党的解释是蠢笨的谎话,他们想用这种蠢笨的谎话,在全国民众的面前掩盖自己的卖国行为,中华苏维埃政府再一次提醒中国民众,在去年四月我们已经是号召全中国民众与我们一起共同的进行反对日本帝国主义的民族战争。而蒋介石对于这个号召的回答,却是动员一切军队进攻中国工农而不去反抗日本帝国主义。

中华苏维埃政府与工农红军革命军事委员会在全中国民众面前宣言:

在下列几个最简单的,但是为真正组织和进行民族革命战争所必要的条件之下,中国红军准备与任何武装队伍订立战斗的协定,以便共同用武装斗争来反对日本帝国主义的侵略。

(一)立即停止进攻苏维埃区域。

(二)立即保证民众的民主权利(集会、结社、出版言论之自由等)

(三)立即武装民众,创立武装的义勇军队伍以保卫中国及争取中国的独立、统一与领土的完整。

我们要求中国民众及士兵拥护这个号召,进行联合一致的民族革命战争,争取中国的独立,统一与领土完整。

将反对日本及一切帝国主义的斗争与反对帝国主义走狗一一国民党的卖国与投降的斗争联结起来!

开展武装民众的民族革命战争反对日本及一切帝国主义!

<p align="right">中华苏维埃共和国中央临时政府主席毛泽东

付主席项×张××

中华苏维埃共和国革命军事委员会主席

兼工农红军总司令朱德

一九三三年一月十日</p>

(注):

《毛选》三卷《关于若干历史问题的历史决议》中提到过这个宣言。

