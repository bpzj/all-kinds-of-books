\section{新民学会会员通信集第二集}
\datesubtitle{新民学会致各会友的信}



各会友均鉴:

本会出版物。分“会员通信集”与“会务报告”之二,除会务报告叙述会务状况年出二册外,会员通信集,为会员发抒所见相与杨权讨论的场所。凡会友与会友间往来信稿,不论新、旧,长、短,凡是可以公开的,均望将原稿或誊正稿寄来本会,舅便釆登第四期以后的通信集,不登之稿可退还。已登之稿声明要退还的也可退还。稿寄长沙文书社毛泽东君为荷。

\begin{flushright}新民学会启\end{flushright}
\subsection{序}

这是新民学会会员通信集“第二集”,釆集会员通信计二十八封,重要者如下:向警予讨论女子发展的计划一封;讨论女子留法勤工俭学问题及女同志联系问题一封;欧阳洋讨论新民学会共同的精神一封;毛泽东主张新民学会应取潜在的态度一封;肖子障报告男女公友在法生活状况一封;易礼容主张吾人进行要有准备一封;罗璈阶希望学会反抗靡俗估量最高价值一封;毛泽东讨论湖南自治并主张学会同人应为主义的结会一封;张国基讨论会务进行一封;报告湘人在南泽情形一封;毛泽东讨论湘人往南洋应取的态度希望会友往南洋为文化动动和南洋建国运动一封;张国基述南洋的奇闻一封;罗学瓒希望会友注意锻炼身外祛除四种迷惑解决家庭问题一封;毛泽东主张组织拒婚同盟一封。以上各信,或于身心之修养有益,或学术之讨论问题之研究有益,或于会务之进行有益,并且都是在能引起会员团体生活的兴味的。其余的信,或商量进止,或讨论事宜,或报告个人状况,具载于此,以见一班。

\begin{flushright}一千九百二十年一月二十号编者\end{flushright}
\subsection{给向警予的信}

※湖南问题

来信久到,未能即复,幸谅!湘事去冬在沪,姐首慷慨论之。一年以来,弟和萌柏等也曾间接为力,但无大效者,教育未行,民智未启,多数之湘人,犹在睡梦,号称有知识之人,又绝无理论计划。弟和萌柏主张湖南自主为国,务与不进化之北方各省及情势不同之南方各省离异,打破空洞无组织的大中国,直接与世界有觉悟之民族携手,而知者绝少。自治问题发生,空气至为黯谈自由湖南革命政府召集湖南人民宪法会议制定湖南宪法以建设新湖南之说出,声势稍振,而多数人莫明其妙,甚或大惊小怪,诧为奇离。湖南人脑筋不清晰,无理想,无远计,几个月来,已看透了。政治界暮气已深,腐败已甚,政治改良一涂,可谓绝无希望。君人惟有不理一切,另辟道路,另造环境一法。教育系我职业,顿湘两年,世已决计。惟办事则不能求学,于自身牺牲太大耳。湘省女子教育绝少进步(男子教育亦然),希望你能引大批女同志出外,多引一人,即多救一人。此颂

进步!

健豪伯母及戍熙姐同此问好

 弟泽东

 九年十一月二十五日
\subsection{给欧阳泽的信}

※学会应取潜在的态度
<p><font face="宋体">玉先生:

共同的精神四项,弟样样赞成。会员加入不限省界,也是极端赞成的。岂但省界,国界也不要限。弟在京所以有那么一说,是因为新民学会现在尚没有深固的基础,在这个时候,宜注意于固有同志之联络,以道义为中心,互相劝勉谅解,使人人如亲生的兄弟姐妹一样。然后进而联络全中国的同志,进而联络全世界的同志,以共谋解决人类各种问题。弟意凡事不可不注重基础,弟见好些团体,象没有经验的商店,货还没有办好,招牌早已高挂了,广告早四出了,结果离不开失败,离不开一个“倒”。半松园会议,都主张本会进行应取“潜在的态度”,弟是十二分赞成,兄也是赞成的一个,长沙同人,亦同此意。弟以为这是新民学会一个好现象,可大可久的事业,其基础即筑在这种“潜在的态度”之上。

你的信我在上海接到。彭、周、劳、魏都转给他们看了,我七月回湘,一回多忆,未能作答,幸谅!你现状谅好,我忘记你在芬丹白露,抑在蒙达尔尼?来信幸告。近因极倦,游觉到萍,旋中作书,言不尽意。

 弟泽东

 九年十一月二十五日萍乡旋中
\subsection{对易礼容来信的评论}

礼容这一封信,讨论吾人进行办法,主张要有预备,极忠极切。我的意见,于致陶斯咏姐及周惇之兄函中已具体表现,于归湘途中和礼容也当面说过几次。我觉得去年的驱张运动和今年的自治运动,在我们一班人看来,实在不是由我们去实行做一种政治运动。我们做这两种运动的意义,驱张运动只是简单的反抗张敬尧这个太令人过意不下去的强权者。自治运动是简单的希望在湖南能够特别定出一个办法(湖南宪法),将湖南造成一个较好的环境,我们好于这种环境之内,实现我们的具体准备工夫,彻底言之,这两种运动,都只是应付目前环境的一种权宜之计,决不是我们根本的主张,我们的主张远在这些运动之外,说到这里,诚哉如礼容所书,‘准备’要紧,不过准备的‘方法’怎样?又待研究。去年在京,陈赞周即对于‘驱张’怀疑,他说我们既相信世界主义和根本改造,就不要顾及目前的小问题,小事实,就不要‘驱张’。他的话当然也有理,但我意稍有不同,‘驱张’运动和自治运动等,也是达到根本改造的一种手段,是对付‘目前环境’最经济最有效的一种手段。但有一条件,即我们自始至终(从这种运动之发起至结局),只宜立于‘促进’的地位,明言之,即我们决不跳上政治舞台去做当局。我意我们新民学会会友于以后进行方法应分几种:一种是出国的,可分为二,一是专门从事学术研究,多造成有根底的学者,如罗荣熙肖子升之主张。一是从事于根本改造之计划和组织,确立一个改造的基础,如蔡和森所主张的共产党。一种是未出国,亦分为二,一是在省内及国内学校求学的,当然以求学储能做本位。一是从事社会运动的,可以从各而发起并实行各种有价值之社会运动及社会事业。其政治运动之认为最经济最有效者,如‘自治运动’,‘普选运动’等,亦可从旁尽一点促进之力,惟千万不要沾染旧社会习气’尢其不要忘记我们根本的共同的理想和计划。至礼容所说的结合同志,自然十分要紧,惟我们的结合,是一种互相的结合,人格要公开,目的要共同,我们总不要使我们意识中有一个不得其所的真同志就好。

 泽东
\subsection{复肖子嶂的信}
<p><font face="宋体">子嶂兄:

你沿途给我的信和照片,都收到,很感谢你的厚意。我竟没有一个信给你,很对你不住,‘半松园会议’的结果,既由他和赞周等带到了欧洲,‘蒙达尔尼会议’的情形,又由你和子升递回了亚洲,手升并且自己回国,不日就可见而了,真是乐阿!你现在谅好,谅还在学校。我意你在法宜研究一门学问,择你性之所宜者至少一门,这一门使要将他研究透彻。我近觉得仅仅常识是靠不住的,深慨自己学问无专精,两年来为事所扰,学问来能用功,实深抱恨,望你有以教我。学会出版分‘通信集’与‘会务报告’为二,通信集本年至少可出三集,请设法收集在法诸友间新旧信稿邮递‘长沙文化书社’交弟,作第四集第五集材料,至感。(请告诺友以后通信均写长沙文化书社)

 弟泽东
\subsection{给罗璈阶的信}

※叛湖南问题

※叛主义的结合

※湖南问题

※主义的结合

※学生自决
<p><font face="宋体">章龙兄:

昨信接到。重翻你七月二十五日的信,我昨信竟没有一句答复你信内的话,真对不住。今再奉复大意如下。我虽然不反对零碎解决,但我不赞成没有主义头痛医头脚痛医脚的解决。我主张湖南人不与闻外事,专把湖南一省弄好,有两个意思:一是中国太大了,各省的感情利害和民智程度又至不齐,要弄好他也无从着手,从康梁维新至孙黄革命,(两者亦自有他们相当的价值当别论)都只在这组织上用功,结果均归失败。急应改涂易辙,从各省小组织下手。湖南人便应以湖南一省为全国倡,各省小组织好了,全国总组织怕他不好,一是湖南的地理民性,均很有为。杂在全国的总组织者中,既消磨特长,复阻碍进步。独立自治,可以定出一种较进步的办法(湖南宪法),内之自庄严璀睽其河山,处之与世界有觉悟的民族直接携手,共为世界的大改造。全国各省也可因此而激厉进化。所以弟直主张湖南应自立为国,湖南完全自治,丝毫不受外力干涉,不要再为不中用的‘中国’所累,这实是进于总解决的一个紧要手段,而非和有些人所谓零碎解决实则是不痛不痒的解决相同,此意前函未尽,今再补陈于此。

兄所谓善良的有势力的士气,确是要紧。中国坏空气太深太厚,吾们戒哉要造成一种有势力的新空气,才可以将他划换过来。我想这种空气,固然要有一班刻苦励志的‘人’,尤其要有一种为大家共同信守的‘主义’,没有主义,是造不成空气的。我想我们学会,不可徒然做人的聚集,感情的结合,要变为主义的结合才好。主譬如一面旗子,旗子立起了,大家才有所指望,才知道趋赴,兄以为何如?

“会务报告”专纪会务,不载论述文字,尚未着手编写,大略每季一册尽够了。此外会友通信,发刊通信专集,为会友相互讨论商讨的场所,兄处有无会友们往还信稿,不论新旧,请检查寄弟。

“湘江”尚未出版,固因事忙,亦怡出而不好,到底出否,尚待斟酌。

弟本期在城南附小办一点事,杂以他务,自修时间很少,读‘岁月易逝无法挽回’,‘思想学术节节僵化’诸语,使我不寒而栗,我回湘时,原想无论如何每天要有一点钟看报,两点钟看书,竟不能实践。我想忙过今冬,从明年起,一定要实践这个条件才好。求学程序计,略有一点,迟后当可奉告。

讲到湖南教育,真是欲哭无泪。我于湖南教育共有两个希望:一个是希望至今还存在的一班造孽的教育家死尽,这个希望是做不到的。一是希望学生自决,我唯一的希望在此。怪不得人家说‘湖南学生的思想幼稚’,(沈仲九的话)从没有人供给过他们以思想,也没有自决的想将来自己的思想开发过,思想怎么会不幼稚呢?望时赐信为感!

 弟泽东

 (九年十一月二十五日)
\subsection{给钦文(李思安)的信(一九二〇年十一月二十五)}
<p><font face="宋体">钦文姐:

你这信我八月里就收到了。后来还接到你在星加坡寄来的一封长信,并一些印刷物。很感谢你的厚意。我因事忙,没有即答,想能原谅我吧。湘江尚未出版。湖南虽有一些志士从事实际的改造,你莫以为是几篇文章所能弄得好的。大伟人虽没有十分巩固,小伟人(政客)却很巩固了。我想对付他们的法子,最好是不理他们,由我们另想办法,另造环境,长期的预备,精密的计划。实力养成了,效果自然会见,倒不必和他们争一日的长短,你以为然么?你事务谅是忙的,我劝你总要有时间看一点新书报。并且希望你能够继续省察自己,能够知道自己的短处。你前信嘱转集虚,已转他看了。有暇望告我以近状。

 弟泽东

 十一月二十五日
\subsection{给张国基的信(一九二〇年十一月二十五)}

※会务问题

※湘人往南洋应取之态度

※南洋文化运动和南洋建国运动
<p><font face="宋体">颐生兄:

两信先后本悉,久未作复,甚歉!所言会务六项,弟大体均赞成。第一项发行会报,现已决发刊会员通讯集和会务报告两种。第二项会友加入宜郑重。第三项会友加入不要有男女老幼等区别。弟忙夏间在上海与焜甫,赞周,子障,荫柏、望成、韫广、敦洋、冀儒,玉生,等在半淞园会商,及回长沙再和长沙会友商酌,多主会友加入,要准备下列三个条件,(一)纯洁,(二)诚恳,(三)向上,并须有五人介绍,经评议部通过,然后再郑重通告全体会员,正与你的主张相合。(南洋方面同志,当然应该联络),第四项,会所的确定,也是要事,不久总要在长沙觅到一个相当的会所。书报的设置与会友研究有关,长沙巴黎南洋应分别设备,其经费可由各地会员“分”任,第五项,经费的筹措,我意只要会员常年费交齐,普通用费已够。此外只有印通信集和会务报告须款,但也不多,可由会员临时分任。会友录即印在会务报告之内,本年总可以印出一本。南洋通讯社组织极要。惟弟对于湘人往南洋有一意见,即湘人往南洋要学李石曾先生等介绍学生往法国之用意,取世界主义,而不釆殖民政策。世界主义,愿自己好,也愿别人好,质言之,即愿大家好的主义。殖民政策,只愿自己好,不愿别人好,质言之,即损人利己的政策,苟是世界主义,无地不可自容,李石曾等便是一个例。苟是殖民政策,则无地可以自容,日本人便是一个例,南洋文化闭塞,湘人往南洋者,宜以发达文化己任。兄等苟能在南洋为新文化运动,使国内发生之新文化,汇往南洋,南洋人(不必单言华侨)将受赐不浅。又南洋建国运动,亟须发起,苟有志士从事于此种运动,拯救千万无告之人民出水火而登袵席,其为大业,何以加兹。弟意我们会员宜有多人往南做教育运动,和文化运动,俟有成效,即进而联络华侨士著各地各界,鼓吹建国,世界大同,必以各地民族自决为基,南洋民族而能自决,即是促进大同的一个条件。有暇望时通信。

 弟泽东

 九年十一月二十五日给
\subsection{罗学瓒的信}

※养成读书和游戏并行的习惯

※理论上的错误

※拒婚同盟
<p><font face="宋体">荣熙兄:

兄七月十四日的信,所论各节,透彻之到身体诚哉是一个大问题。你谓中国读书人,以身殉学,是由于家庭、社会和学校的环境太坏造成的,这是客观方而的原因,诚哉不错。尚有主观方面的原因,就是心理上的惰性。如读书成了习惯,便一直谈下去不知休息。照卫生的法则,用脑一点钟,应休息一分钟,弟则常常接连三四点钟不休息,甚或夜以继日,并非乐此不渡,实是疲而不舍。我看中国下力人身体并不弱,身体弱只有读书人。要矫正这弊病,社会方面,须设造成好的环境。个人方面,须养成工读并行的习惯;至少也要养成读书和游戏并行的习惯。我的生活实在太劳了,怀中先生在时,曾屡劝我要节势,要多休息,但我总不能信他的话,现在我决定在城市住两个月,必要到乡村住一个星期。这次便是因休息到萍乡,以后拟每两个月要出游一次。

四种迷,说得最透彻,安得将你的话印刷四万万张遍中国人每人给一张就好。感情的生活,在人身原是很要紧,但不可令感情来论事。以部份概全体,是空间的误认。以一时概永久,是时间的误认。以主观概客观,是感情和空间合同的误认。四者通是犯了理论的错误。我近来常和朋友发生激然的争辩,均不出四者范围。我自信我于后三者的错误尚少。惟感情一项,颇不现能免。惟我的感情不是你所指的那些例,乃是对人的问题。我常觉得有站在言论界上的人就不佩服他,或发现他人格上有缺点。他发出来的议论,我便有些不大信用。以人废言,我自知这是我一个短处,日后要是矫正。我于后之者于说话高兴时或激烈的也常错误,不过自己却知道这是错误,所谓明知故犯罢了,(作文时也有)。

“工学励志会”,听说改成了“工学世界社”,详情我小知,请你将组织,进行,事务等,告我一信。通信尚未到。交换报一节弟可办到。请陆续将稿寄来(寄长沙文化书社交弟)。

以资本主义做基础的婚姻制度,是一种绝对要不得的事,在理论上是以法律保护最不合理的强奸,而禁止最合理的自由恋爱;在事实上,天下无数男女的怨声,乃均发现于这种婚姻制度的下面。我想现在反对婚姻制度已经有好多人说了,就只没有人实行。所以不实行,就只是“怕”,我听得“向蔡同盟”的事,为之一喜,向蔡已经打破了“怕”实行不要婚姻,我想我们正好奉向蔡做首领,组织一个“拒婚同盟’,已有婚约的,解除婚约(我反对人道主义)。没有婚约的,实行不要婚约。同盟组成了,同盟的各员立刻组成同盟军。开初只取消极的态度。对外‘防御’反对我们的敌人,对内好生正理内部的秩序,务使同盟内的各员,都实践“废婚姻”这条盟约。稍后,就可取积极的态度开始向世界“宣传”,开始“攻击”反对我们的敌人,务使全人类对于婚姻制度都得解放,都纳入同盟做同盟的一员。我这些话好象是笑话,实则兄所愈感的那些“家庭之吉”,非用这种好笑的办法,无可避免。假如没有人赞成我的办法,我“一个人的同盟”是已经结起了的。我觉得凡在婚姻制度底下的男女,只是一个“强奸团”,我是早已宣言不愿加入这个强奸团的。你如不赞成我的意见,便请你将反对的意见写出。此祝进步。

 弟泽东

 九年十一月二十六日

