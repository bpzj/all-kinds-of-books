\section[中央农民运动讲习所规约(一九二六年十月)]{中央农民运动讲习所规约}
\datesubtitle{(一九二六年十月)}


同志们:我们为什么要进来这里?我们的目的是什么?我们的责任是什么7我们进来这里的目的并不和他们进普遍的中学大学一样,我们的责任自然也和他们并不一致。因此我们进所之后一切的生活自然和他们不能相同。

同志们,我们的目的是什么?我们的责任是什么呢?我们不是受殖民地的被压迫的三亿二千万的贫苦农民群众中的一个分子吗?我们现在的唯一出路不是要唤起运广大的被压迫的农民群众,向我们的敌人帝国主义、军阀贪官官污吏土豪劣绅、起来反抗吗?我们除了革命,除了和我们的敌人拼命,还有什么是我们的出路?所以我们进来这里的唯一目的,是研究革命的理论和行动。我们的责任是唤起广大的农民群众,领导他们起来,打倒我们的敌人,解除农民群众的痛苦。

但是,我们也要知道我们的敌人的状况,他们几年来占据了社会上优越的地位。这便是说,他们几千年来都是一个统治阶级、压迫阶级,他们的基础十分稳固,他们有他们的组织,他们有优越的经济能力,有凶狠的武装(军士警察)、有维护特殊阶级利益的法律,习惯和一切封建制度,做他的护符,所以他们是很不易的。这两年来,在广东、在湖南、在湖北,在北方各省,农村发生了莫大的冲突、莫大的流血,结果我们的农民群众、虽得许多进步,然而我们的牺牲也真不小,这都是足以证明我们的强敌,是很不容易打倒的、并且在我们自己的战线上,同时出现了许多毛病,许多足以减弱我们斗争能力的毛病。农民的生活方式,是颇散漫的,几千年来,老生活把他们弄成不问政治,只管真命天子出世救人的依赖思想,他们对于地方观念,家族姓氏的观念很深,一切封建社会的坏毛病都应封建地主阶级的需要,而澎湃发展,便之笼罩在农民身上,现在要农民起来,反对他们根深蒂固的强敌,真是谈何容易?便是我们自己何尝不是一个封建社会里出来的人,也何曾能够免除了封建制度的影响?我们现在虽然决心要做一个农民阶级的急先锋,要为农民的利益牺牲一切,然而在,我们的身上,已经从旧社会里带来了不少毒菌,这些毒菌随地都有阻障我们前进的可能。

同志们!我们从旧社会带来的毒物是什么?我们不是过惯了个人主义的浪漫生活么?我们不是不大喜欢受团体的干涉么?我们不是不喜欢有秩序有规律的生活么?我们的队伍中有许多人不免可借自由平等的名词来掩护他们的一切罪过,这些人喜欢崇拜英雄主义,喜欢个人的自由活动,这些人的一切行动,一切的批评,常常从他自己个人做出发点,个人的利害关系,随时在他的后头做他的指挥者,他的眼睛每每只看到人和人的关系,看不见群众和客观的现实,这样的最容易变成一个唯心论者,目(口)中常常发出许多良心活,道德话。

同志们!我们许多是从小有产者的阶级出来的,所以一切小有产阶级的毛病都容易丛集在我们的身上。上述那些毛病,我们不是时常犯着么?这些毛病,最能损害我们的革命工作。我们虽然是一个具有热烈革命性的青年,仍多少免不了这些毛病在身上作怪,现在许多反革命的势力,都是从这些封建制度,封建的思想习惯产生出来的。我们的党最近以前有了一种个人独裁的倾向,有了老朽昏庸腐败的分子在里面作怪,大部分的原因,是因为这个封建制度,封建思想存在的原因。

同志们!我们的敌人是这样凶狠顽强,我们的责任是这般重大,我们非把我们从旧社会带来的毛病,痛切的扫除,我们更难免要做个落伍的分子,难免有右倾的危险。

同志们!我们跑来这里,从消极方面,要把我们这些从封建社会里带来的毛病扫除,从积极方面,我们要学习革命的理论和实际,这样我们才可以变成一个很好的有效的革命的工具,才能达到我们的目的,完成我们的责任。同志们:我们为着便利我们在所内从消极和积极两方面把我们训练成一个好的的革命工具,我们应该照下面规定的各条去努力,这些规定我们不要看作一种普通的学校规则,乃是使我们达到变成一个好的革命的党员的一种必需的方法,这些条文我们大家都经过一番讨论发表过意见,都认为是我们必须的东西,经过最后的决定颁布,已成为本所的纪律,我们一定更要一致,因为这是为了我们自己的利益而制定的纪律。这种纪律是自觉的纪律,是达到我们学习革命的理论和实际,造成好的革命党员之唯一方法。

