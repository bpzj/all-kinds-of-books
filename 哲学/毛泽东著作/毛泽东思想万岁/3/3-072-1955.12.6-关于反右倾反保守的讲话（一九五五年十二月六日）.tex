\section[关于反右倾反保守的讲话(一九五五年十二月六日)]{关于反右倾反保守的讲话}
\datesubtitle{(一九五五年十二月六日)}


“左”比右好,不对;右比“左”好,也不对。有些人思想落后在实际后边,头上没有角,没有斗争性。两条道路斗争,不进行批评,批评不尖锐。怕批评,怕丢选票,均右倾之表现,党内有反之必要。农村部经验主义右倾,六中全会做了批评,领导思想落后于实际是严重问题。资本主义改造、镇反、合作化均如此。对自己力量估计不足,过去农业落后于工业,十七条出来赶上来了,压迫工业进步,如初级社,三、四年不提高,不利的。五九年可能会社会主义。十七条思想反保守主义,全面规划加强领导,基本措施,先进经验,半年前不考虑,现在变了。出现了极大变化,极大的生产力,像发现新大陆,大陆本存在,看不到。工商业社会主义改造,五七年百分之九十,手工业也要快,计划太少了。五七年百分之七十至百分之八十,工商业国有化原说六二年,也许提前。合营了,变国又不难了,合作化十八年改为十年,证明落后实际。群众潜力极大,可多办事,应该反右倾反保守,提前完成改造。建议,十五年以前超额完成。改掉“大约”、“基本上”、“十五年左右”等不定语气。可争取的争取,如订货要求提早,利用在休战期间,加快速度,完成总任务。这就是“八大”思想。如做到提早完成过渡时期的总任务,战场好办。赶快搞极有利,要打台(?)好打,但不完成建设任务则困难,要加快,各项工作均又快又多又好,较短时期得到较好成绩。办得到有条件,群众要求,办得多而好,即稳步前进。过去反盲目冒进出毛病,反掉群众干部的积极性,不对的。扫盲,反冒进反掉了,正气不升,邪气高涨,干部群众均没劲。这样问题:中央未能及早提出,是中央失职。群众说:“跟共产党走,没错。”大方向说没错,但具体说,我们错了,群众也跟错了。前进有几条路,上中下三策,

比较正确的合理的路钱即站在前头,鼓舞群众前进,而非站在后边泼冷水。较短时间内取得较大成绩,按这,即稳步前进。按常规办事,时间长工作少,即保守路线。要克服,有两种方法:全面规划,接近群众,非坐在办公室办法,要接近群众找出新事物、新经验,先进经验,推广。不一定捉麻雀好多才知肝胆。知浙、皖可快合作化,即知全国可搞。保守主义不是那一个人的问题。抓先进的批评落后的,发现新的生产力,群众潜力,这是一个领导法则,是领导方法,才能说服人。坐办公室,不接近群众,不抓先进,不行。公事要办,但只办公事是不行的,出去跑也要抓先进的,不能只搞落后的回来。一个负责人七至十周,接近群众。办公室式,只能常规法。先进经验是要突出常规的。客观事物,天天突破,不平衡是经常的,平衡是暂时的,这是前进规律。平衡了不可改变的观点是不对的,不平衡中抓先进的,带动其他,才能前进。不怕突破、出矛盾。永远太太平平必出错误。冲破而又求平,才是辩证法。这才能鼓舞群众积极性,提早社会主义建设,快、多、好,十五年以前超额完成,稳步前进。

我国和苏联比较:(1)我是二十多年根据地经验,三次革命战争的演习,经验极其丰富。胜利前,各个方面均有经验,左右好几次,很快组成国家,完成革命任务。(苏是新起家,十月革命,无军,无政,党员少)。(2)我有苏联及其它民主国家之助。(3)人口众多,地位很好,勤苦耐劳。不合作化,农民无出路。中国农民比英美工人还好,因此可以更多、更好、更快地进到社会主义,不要老比苏联。我三个五年即可搞二千四百万吨钢,就比苏快。现在是两翼高涨,主体可能落后。两翼易骄,但最易骄者为工业,首先是重工业。鞍钢即有骄。很可能国家社会主义化了,但非工业化(比例不过六十,体系未建立),可否让农民等一下?那是不行的。农民社会主义化,并不妨害工业化。不能让他们等。两翼工作做了严格的检查。此外,对工业问题,也应做一次严格的检查,也按期过过大运动。财经、贸易、文教均要检查。八大做一次总结性的检查,这就是中心问题。反保守、右倾、消极、骄傲,发现先进经验,改变领导方法,做到又快又好。动员群众批评,自我批评,克服保守主义,做出来更多成绩,全党准备,全人民准备。


