\section[中国共产党全国代表会议闭幕词(摘录)(一九五五年三月)]{中国共产党全国代表会议闭幕词(摘录)}
\datesubtitle{(一九五五年三月)}


马克思、恩格斯在《共产党宣言》上说过:“共产党人认为隐瞒自己的观点和意图是可鄙的事。”我们是共产党人,更不待说是党的高级干部,政治上都要光明磊落,应该随时公开说出自己的政治见解,对于每一重大的政治问题表示自己或者赞成或者反对的态度,而绝对不可以学高岗、饶漱石那样玩弄阴谋手段。

鉴于种种历史教训,鉴于个人的智慧必须和集体的智慧相结合,才能发挥较好的作用和使我们在工作中少犯错误。

同志们:我们现在是处在新的历史时期,一个六万万人口的东方国家举行社会主义革命,要在这个国家里改变历史方向和国家面貌,要在大约三个五年计划期间,使国家基本工业化。

同志们!我们共产党人是以不怕困难著名的。

……种种困难,遇到共产党人,它们就只好退却,真是“高山也要低头,河水也要让路。”

但对待困难:

(一)在战术上必须重视一切困难,对于每一个具体困难,我们都要采取认真对待的态度,创造必要的条件,讲究对付的方法,一个一个地,一批一批地将它们克服下去。

(二)在战略方面,我们要藐视困难,就是说在总的方面,不管任何巨大的困难,我们一眼就看透了它的底子。所谓困难,无非是敌人和自然界给予我们的,——而这却不过是垂死的力量,我们则是新生力量,真理是在我们方面。对于他们,我们从来就是不可战胜的,——不论在自然界和在社会上,一切新生力量,就其性质来说,从来就是不可战胜的。因此我们可以藐视而且必须藐视人世遭遇的任何巨大困难,把他们放在“不在话下”的话里,这是我们的乐观主义。


