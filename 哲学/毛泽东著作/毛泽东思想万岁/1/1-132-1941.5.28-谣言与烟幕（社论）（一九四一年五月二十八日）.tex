\section[谣言与烟幕(社论)(一九四一年五月二十八日)]{谣言与烟幕(社论)}
\datesubtitle{(一九四一年五月二十八日)}


造谣惑众是市井无赖的惯技,而政治流氓就把这种卑劣的手段用之于政治斗争中,竟自称曰:“散布合理的流言。”

在社会政治斗争中,先进的阶级和先进的政党,它们手中握着正义和真理之旗,不害怕公开和坦白的向着广大的群众宣布自己的政纲,目的,表明白己的政治任务和政治行动的方向,进行公开的严肃的斗争来反对当前主要的敌人,他们用不着玩阴谋、耍手段、造谣言之类的下流无耻的办法,因为他们有自信:真理只在他们的方面的,它们的前途是光明的。所以,它们的行动,光明磊落、坦白严肃。而那些政治流氓们,因为要维护其违反真理和正义的私利,而自知他们的主张、纲领、政治任务和政治目的是得不到广大人民的欢迎的,所以不敢堂堂正正的行动,只能鼠窃狗偷,鬼鬼祟祟,因而玩阴谋,耍手段,造谣言之类,就成了他们的拿手好戏。造谣言一一或者“散布合理的流言”,就为他们的所癖爱,因为在公开坦白的政治斗争中既斗不过对手们,那么只好乞援于造谣了。何况,古书上说过,曾参杀人,慈母投杼,诗人亦唱过,“三人成市虎,浸溃能胶漆”,其效力不是很大么?

不仅如此,谣言也还有别的作用,就是遮眼罩和烟幕弹的作用。不是有这么一个故事么?当一个小偷偷了东西之后,被人发觉,大呼捉贼。此君急中生智,亦高叫“贼在那边,在那边,”竟得脱身。政治上的没落人物在做亏心事的时候,亦常常借谣言来转移视线,来隐身的。

这种“造谣术”的最近例子,就是所谓:“十八集团军集中晋北不与友军协同作战”的广泛地有计划地发布的流言。这个谣言的最初散布者乃是专务此道的个中老手一一同盟社。日寇于本月初发动了一个小规模的军事攻势,而同时却发动了一个大规模的谣言攻势。同盟社的广播连篇累牍地散布各种谣言,尤以八路军决不与中央军协同作战,八路军集中陕北准备乘机向西安出动,八路军乘机扩张势力收缴中央军枪械之类为特多。日寇这种军事攻势和谣言攻势的目的是很明显的。一个字足以尽之,曰:吓!或者说吓降,军事攻势在炫耀其兵力,其愿若曰:你若不降,我将占你的故乡,占你的一切海口,歼灭你在中条山上的几十万军队,进占你的洛阳、西安、昆明、重庆,你怎么办?谣言攻势在挑拨国共关系,描出一幅错误的画图来吓你说,你看国共关系恶劣至此,自力更生,还有什么希望呢?快降吧!“八路军不打日本”之类的谣言是烟幕,吓降诱降是目的。其技至浅,其理至明。

奇怪的是某些中国人,不是汪精卫之类双料的汉奸,而是抗战营垒中统治阶层里的某些风云人物,虽然亦一字一句地抄写同盟社的广播来替它作义务的转播。象大公报和中央社这类新闻机关,居然一方面说:“敌方所传大部出乎揑造自不能信,”另一方面又重复着“敌方所传”,称“十八集团军集中晋北迄今尚未与友军协同作战则为事实。”日本人一钱不花,就有中国人义务地替他的谣言当留声机和见证人,宁非怪事!

可是“怪事”实际上是没有的。只要懂得我们上述故事中的贼的急智,就可以理解这次谣言唱和中的“机”了。重庆军委发言人在本月二十二号就多少泄露了这个“机”的一部分,他说:“上周寇军全面发动,总计达三十万人之多,其结果不过如此,以此种方式而后谈解决‘中国事变’,不但世无相信之人,即敌寇亦自如其不可能也。”这不就是说,此种(军事进攻)方式是不能解决的,换种方式吧!果然同盟社接着的纷纷报告日各战线军事当局均称第一期作战全部结束。谣唱和,这次竟完成了红叶题诗式的媒介作用。可惜的是,香港的一部分参政员竟为这个小小手法所迷惑。

我们认为在这里无须再事“辟谣”了,因为八路军共产党人对于抗战的坚持,对民族的忠忱,是决非谣言所能摇撼的事实,是不怕火烧的真金。尽管新四军被宣布为叛军,八路军两年没有领到一颗子弹,五个月没有领到半文制钱的饷项,然而新四军、八路军的将士们,没有一分钟停止过和敌人苦战,而为着策应晋南作战,八路军在华北正在全演出动浴血酣战,这是连造谣的人和传布谣言的人心里都深知的事。其所以造谣和传谣,都是别有怀抱的。

其怀疑为何?在造谣言者为诱降,在传谣言者为投降,而都想在共产党身上做文章,共产党成了他们题诗的红叶。日本人当春心大发时,题诗一首于其上,从长江飘将下来,这边的人儿得了诗,果然打动春心,跃跃欲试,拾起红叶,转题一首,又从长江飘将下去,这就是同盟社与中央社近日抓着共产党问题一唱一和的由来。还有德国的海通社,美国的合众社,英国的路透社,为着各自不同的目的,将双方的情报到处传播,其目的均在“催装”,不过英美是为着反共反苏反德,德国是为着反英美,这个不同而已。“你们赶快结婚罢,好去发动太平洋战争”一一这就是德国的目的。“你们赶快结婚罢,好造成反共反苏反德的东方慕尼黑”一一这就是英美的目的。我们不能不忠告中国国民党的领导人员们,这种结婚乃是“劫婚”,将来是不好过日子的,理应拒婚为上。从共产党身上做文章,也是做不出好文章的。不信,你们瞧罢!

全中国的同胞们!注意在这种谣言烟幕遮盖下的投降危机呀,远东新慕尼黑的极大危险在一天天的增长。以“吓蒋投降”为目的的这次“军事攻势”,现在是暂时的过去了,继之而来的必然是诱降。这虽是日本人“一打一拉,又打又拉”的老把戏,但却包含着新意义,因为正在1941年5月至10月的时机中。全国同胞起来揭破他,粉碎他,乃是民族生(存)所关的巨大任务!

<p align="right">(一九四一年五月二十八日《解放日报》社论)</p>

