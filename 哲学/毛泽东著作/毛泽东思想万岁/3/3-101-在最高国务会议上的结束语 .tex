\section[在最高国务会议上的结束语 ]{在最高国务会议上的结束语 }


同志们:

我没有多少话讲了,有几点说一下。

批评的问题。这个问题刚才有人谈到老干部批评的问题。还有人提到(在小组会的时候)马列主义能不能批评?马列主义,这个马列主义是不是可以不规定为领导的思想,或者叫指导思想?因为规定他为指导思想,使得有些人有点害怕。关于这个问题么,我们这个国家现在有个很大的改变。我是讲由过去的那种领导,旧社会国民党,蒋介石嘛,那么一种领导,改变为让位于以工人阶级为领导的以工农联盟为基础的人民民主专政,为我们的国家制度。这个工人阶级为领导末,他就是有个党。现在还是有党派的时代。这个党、这个阶级,这个党有一种思想,就是马克思主义——共产主义思想咯。那末,问题是他如何领导,如何指导。并不是说让一切人都进共产党,都相信共产党的道理,去讲唯物辩证法的世界观。这些东西,如讲世界观这个问题,只能够是逐步地使人了解,开头是很少的人,以后多一点人,以后又多了一点人,不能一下子要求很多人相信,就是不能强迫人相信这一点嘛!不能强迫人信什么东西,前天我提到了。精神上东西不能强迫人家相信,也不能强迫家人不相信。比如强迫人不信教就不行咯,强迫人要信这种教不信那种教是不行的。强迫人相信马克思主义的世界观行不行呢?马克思主义?也是不行的。但是马克思世界观是在一天天发展,相信的人一天天更多一些。但是无论怎样,几十年之后,我相信总还有许多人不相信的,不那信那个世界观。马克思主义是无神论的,有神论者不信。甚至几十年后,多少年后,还有相当一些人不信怎么办?原来就是在马列主义领导里头应该安排的,就是应该认识有些人不相信,事实上或许要如此,也只能如此。

能不能批评?马克思主义者是不怕批评的。马克思主义如果能够批评倒他就没有用,能够证明马克思主义不是真理,那么这个东西就不行了。马克思主义如果还怕批评,那就不行了。所以不发生马克思主义可不可以批评。事实上唯心论他就批评马克思主义,宗教界也就批评马克思主义,有些人不说就是了。有人把马克思主义和宗教结合起来,说成为与他那个宗教结合。老干部如果批评倒了是该批评的。你一批就倒嘛,你是纸糊的、棉扎的,纸糊棉扎风一吹就倒。应该是不怕批评的。所以这时候,提出怕批评,就是有弱点就是了。这个弱点有没有?我看老干部、新干部都有弱点,弱点方面都应当批评,无论那种干部、政府,缺点、错误应该接受批评的。并且成为习惯,人民政府成了习惯。批评就没有事;批评对了当然很好,批评不对没有事,这就是言者无罪嘛!人民范围之内的事,人民是有批评的权利的,我们不要把这个权利交给反革命,宪法是应该实行的。言论集会、结社自由、言论出版。

长期共存,互相监督,有人说我讲的不够。讲的这种也是一种批评,前天没大讲这个问题,今天有朋友讲到咯。什么叫长期?我想这个问题答复比较容易,长期么就是你共产党有多长,他就有多长,就是,照办就是了。共产党有多少寿命么,就是民主党派有多少寿命。今天郭老说我们这些人统要向他看齐,无党派寿命最长咯。共产党有天要灭亡的,我们希望这一天不迟不早地到来。大概太早不行,太迟了没有必要。凡发生的事物,世界上的事物都是发生的,因此都有他一定的寿命,共产党就是一个事物。如何监督?就是属于批评、建议,用这些方法监督嘛。我们有各种机会嘛,比如今天就是机会,还有各种机会进行批评,主要的方法就是批评,批评缺点,从团结的愿望出发,通过批评达到团结,把工作改善。

说,你又讲大民主不适当,大民主是对付阶级敌人的,但是又说可以罢工、罢课,岂不是自相矛盾吗?这样一个问题。所谓大民主就是群众运动,这是我们过去所做的,包括肃反中间的某些作法。现在有人建议不要大民主了,现在的作法已经改了,现在遗留下来的一些问题是老的。工商业改造也不是大民生咯,知识分子的思想改造也不是大民主咯,都是小民生咯,有些一个“小”字还不够,再加一个“小”字,叫做“小小”民主,就是和风细雨比较好。现在所讲的就是那个不实行小民主,任何民主都没有,那样的地方,那样的工厂,那样的合作社,那样的学校,那样的商店,那样的机关。那样的单位怎么办?它任何的民主都没有,大民主没有小民主没有,小小民主也没有。简直是官僚主义,一万年也不能解决问题,就是问题不得解决,这样逼出一个大民主来了,于是乎罢工、罢课,对这种官僚主义如果不允许他我看是不好的,应当允许他,虽然宪法上没有罢工的条文。这并不是说,我们在全国范围内提倡罢工、罢课,我们不提提倡这个,我们提倡反对官僚主义,提倡人民范围之内的所用批评的方法来解决。罢工算不算一种斗争?也算一种斗争,用批评斗争,没有办法的时候,对严重的顽固的官僚主义者,用这种方法是应该允许的。那么,先生管不了学生怎么办?应该是有纪律的,无论工厂、学校、商店都应该有政治思想教育。我们中国人民是有纪律的,是很守纪律的;问题是我们没有进行工作,政治思想教育很缺乏,很多方面缺乏。官僚主义非常严重,有些地方,相当多的学校不能解决,没有教育。纪律也应该进行教育,应该先生学生打成一片,工厂厂长与工人打成一片,关心他们。那么这样一来的时候,那么还有什么罢工呢?凡是这样做了,就不会有罢工的。所以第一条是反对官僚主义。以后下面三条,就是第二条咯,如果个别地方、局部地方、个别工厂、个别合作社、个别学校,官僚主义十分严重,这个时候,不问事实上有罢工,事实上有罢课就是因为连小民主也没有,在这样一种范围内允许。有些地方的同志,回到各个地方,不要说北京开了会的,从此全国可以大罢工(笑声)全国可以大罢课,说是我说的,就不是这样的。我们以军队为例,军队我们曾经进行过尖锐的批评,战士批评军官,使军官很难受,战士批评连长,干部会批评军长、师长。大批评以后,这个师长、这个军长、这个连长他的工作更好做了。战士们是拿着枪的,可以用这样的批评的方法,批评与自我批评,然后你就要检讨,你当连长的要检讨,你犯了错误!不能打人,官长不打士兵。军队这样做了,我们解放军里头有很大的民主,为什么我们学校不能做?我们工厂不能做?我们合作社不能做?我们合作社现在的时候,许多地方命令主义很严重。为什么我们机关不能做?是不是这样一做,天下就会造反,天下就会大乱?为什么军队没有乱,还这么打胜仗?那么打胜仗靠什么?一没有原子弹,二没有氢弹,洲际导弹,一个也没有,飞机也没有,那个时候武器不如敌人,如何团结起来?就靠肃清官僚主义,减少官僚主义,跟群众打成一片,走群众路线。

至于要服从领导,工厂要服从厂长的领导、指挥,学生要服从先生的指挥,要有学习的纪律,但是先生与学生要打成一片,关心学生解决学生的问题。现在学校有许多问题还没有解决。比如课太多,学生负担太重。我看我们中国人民就是这样,要么一个字不认识叫文盲,要认识字一天给你堆一大堆。所以讲到这个罢工、罢课,游行示威,请愿这个事情,我们要作调整社会秩序的方法之一,一种补充方法,经常的方法应该是克服批评官僚主义,但是如果办不到,用这种方法调整我们的社会秩序作一种补充方法也是可以的,也是应该允许的,是克服人民内部矛盾,调整社会秩序的一种补充方法。

要见世面,要了解国际情况、敌人的情况。我们准备发行一个刊物《参考消息》,在座的大概都看到了,《参考消息》过去发多少?发二千份,现在准备扩大到三十万份一个大报纸,比如《大公报》他是发二十万份,这个报纸发行比他多,有三十万份。由二千份准备跃进到三十万份,发到县一级,就是要花钱买就是咯。凡是愿意购买这个刊物的,党内党外都可以看这个刊物,那么人们就要说替帝国主义出钱办报纸,共产党、人民政府替帝固主义无条件件地办报纸。我说这个话也可以这么讲,那就是这么一回事。就是要把你们骂我们的话,如何骂我们,你那里发生了的乱子,登在报上,作为内部刊物,有三十万人可以看到,一份不止看一人,还不止,什么“不得遗失”,那个东西人家遗失了怎么办?那有什么要紧(笑)(周总理:现在没有了)没有了,不要那么禁令,就是要我们的人见世面,要懂得外界的事情。至于是不是受他的宣传,受帝国主义的影响而变成他们的人,我看也有可能一、二个变,变一、二个,我们中国人多,刚才马寅老也讲了六亿人口嘛,变那么几个有什么要紧。蒋委员长这个问题,蒋介石他讲了许多东西,有许多著作、演说,听说有这么多长,我就赞成出全集,有人反对出全集,这个当然不公开发行,公开发行行不行?现在还不好公开发行吧,图书馆里里头愿意买,愿意看看,你要研究研究历史他是历史人物,是社会存在的一个反映,他的意识形态,蒋介石是社会存在的反映。我们要批判那些东西,他的文章都没有读过,那怎么批判法?那是不是读了他的文章统统就要到台湾进他那个党呢?是不是有这个危险呢?我看没有这个危险。有那么几个人要进也可以咯,这一条就是要见世面,要见风雨,不要坐在暖室里头,暖室里头长大的东西是不牢固的。那么百家争鸣,百花齐放就更加懂得咯,为什么要百家争鸣、百花齐放?这里面总要放出一些祸出来就是咯!要鸣出一些不好的东西,不好的东西应该怎样看?我前天说了,不好的东西他另外有一个作用,一方面是不好,一方面是好,毒草有他的好处。它不单是一个毒,我举了种牛痘的那例子,这个细菌,病毒是坏东西,有些也有好的作用,它能使人产生免疫。

其他的东西还有什么?

比如匈牙利,有人批评我,说我讲的不够,没有讲清道理。匈牙利究竟犯了什么错误?什么错误那天也讲了一点,没有时间讲那么多,我们也真不太清楚。过去那个拉科西,那个同志干了些什么事情也不大清楚,苏联许多东西我们也不太清楚,苏联比较好,拉科西那个时候搞得岂有此理。大概有这么几条,无非有官僚主义、教条主义、脱离群众,工业方针错误,那天我不是讲了吗?工业方针没有原料,没有销场,办大工厂,工人的薪水降低百分之二十,资本家简单地打倒。如讲我们的荣段仁先生把他打倒,到申新九厂当工人,一个匈牙利事件一起,他就出来组织工人委员会,组织什么会?组织工人委员会。他就那么简单,简单明了。(笑)知识分子没有改造,根本不谈知识分子改造,于是乎裴多菲俱乐部,记者协会,学生联合会就出来了。还有反革命分子没有镇压,没有群众跟反革命作斗争,而是少数人在那里斗,大批的真正的反革命没有受损伤。匈牙利来了大民主啦!这一次,大民主时间不长,可是要恢复要多少时候呢?有人说要三年,他们自己讲,他们的人跟我们讲要三年才能恢复到过去十月二十三日,一二个月的大民主,大民主好是好,可是要三年才能恢复元气。波兰的大民主就减产一半,十月十一月,十月减产一半,现在不知道怎么样。多少恢复一点吧!所以还有搞小民主好。搞大民主,我们在座的人相当多的人也受不了,是不是。

人口控制在六亿,一个也不多啦?(笑)这是一种假设,就是讲有一个时期,比如讲条件没有具备,无非是粮食、衣服、房子、教育等等,现在每一年生一千多万,你要他不增,很难讲,因为现在是无政府主义嘛,必然王国还没有变自由王国咯!在这方面这个人类完全不自觉,没有想出办法来,我们可以研究这个问题,应该研究。政府应该设一个部门,那天我讲了,政府应该设一个部门或一个委员会,人民团体可以广泛研究这个问题,可以想出办法来,人类总而言之要自己控制自己就是了,有的时候使它能够增加一点,有时候停顿一下子,是不是可以搞成有计划的生产,(笑)这是一种设想。这一条马寅老讲得很好,今天讲的好哇!我跟他是同志,以前他的意见百花齐放没有放出来,准备放就是人家反对,说是不要他讲,今天算是畅所欲言了!但这个问题还很值得研究,政府应该设机关,还有一些办法。人民会不会有这个要求,还是我们主观的,人民是要求这个东西的,不是每个人要求,而是很多人要求,比如农民要求这个,人口太多了家庭,他要求节育。城市里头农村里头都有这个要求,说没有要求是不适当的。邵先生咯:嘿,你们两个坐在一起了。(笑)

不要向科学泼冷水,这一条好不好?我看好,不要泼冷水,科学家怎么能泼冷水。向科学家泼冷水这当然不好。一切家,政治家、艺术家、文学家也不要泼冷水。科学家不要泼冷水,积极分子只要他积极,有时候工作做的不好,我们也不要泼冷水,帮助他改正错误,只有那个别的、极顽固的、无可救药的分子另外处理,无可救药的,科学家不要泼冷水。参观科学院,今天郭老将了我一军,这个恐怕没有办法,你既然将起我的军来了,不然你已经封了我的封号“官僚主义者。”(笑)

单纯技术观点抬头的问题,同志们说了这样一个问题,说得好,应该教育我们的干部、科学家、技术人员、技术干部,学生们不要单纯技术观点,技术观点是好的,应该有技术观点,没有技术观点怎么搞技术?观点还没有哪有技术?就是要发展科学技术,要有科学技术观点,要发展到大家热心。但是单纯就不好咯!可以搞到替社会主义可以服务,替资本主义也可以服务,因为我们现在希望我们的同志们,对于新的政治关心,对于新的大局关心,但是我们要做好我们的工作。这不能完全怪他们,这是我们的工作没有做好,有些东西他们不能接受,就是我们所谓思想政治工作教育工作使人不能接受,他就没有兴趣来接受你这个东西,无非是一种教条主义,没有说服力,引不起他们的兴趣,所以应该改善我们这个政治工作,在学校里头,在科学研究机关,在工厂,一切有科学技术人员存在的地方,学生们中间加强并且改善我们的思想政治工作。

关于邦有道怎么样?贫且贱焉耻也是这个问题,这个问题的确不是个别的问题,这是所谓安排问题。这有两个问题,一个是有些人没有安排;一个是安排不适当。所谓不适当者就是有职无权没事做,或者就是安排的位置跟他那个学问才干不适合。人们承认邦有道,这个好,邦有道,邦么就是中华人民共和国,道是什么东西呢?无非是社会主义嘛!辩证法嘛,(笑)贫且贱焉贫就是薪水不够;贱就是没有工作,大概是不要讲他不注意劳动呐,没有工作或者安排得不恰当。他是用孔夫子来批评我们。所以孔夫子有时还有用处,(笑)这个是关于统一战线工作方面缺点的问题,这个缺点很多,中共中央准备今年开一次会,开一次中央全会专门讨论统一战线方面的问题,希望各党派无党派的同志们关于这方而的问题有所准备,提意见给我们,什么话都说。那一些人搜罗起来贫且贱焉的有多少,其人姓甚名谁,何方人士。

还有我们有许多缺点,我们在前进中间有很多困难,但是不要忘记外国也有困难,比如讲美国他有他的困难。红楼梦里斗著名人物王熙凤有言“大有人的难处”。因为人家给她借钱,他就说(诉苦)“大有大的难处”,刘老老借钱,因为他这么一说,刘老老就冷了半截,的确大有大的难处,美国的事情并不那么好办,据我看,我看,经济危机要来了,美国、法国、西欧、自由世界、西方世界,西方国家他们都矛盾很大,经济危机要来,这个东西恐怕不可避免的。美国的月亮恐怕也不一定那么好,还是要证明。有时刻他们的那么几个原子弹,多了几斤钢铁,他现在强点,这一点要不要承认,要承认,我们骂他纸老虎,人们不了解,为什么人家有那么多东西,我们说他是纸老虎呢?就是说他这个东西是建立在不很稳定的基础上,谁人的基础比较巩固,还是我们的基础比较巩固,还是社会主义的基础。社会主义阵营不那么巩固,我们也有毛病,也有缺点,我们的人民对我们有许多不满意,我们的经济还落后,文化还落后,我们的人民对我们有许多不满意,我们的经济还落后,文化还落后,整个社会主义跟他比较起来我们还落后,可是他们是建立在一个矛盾更多更大的基础上的,不要忘记这一点。

最后我这个讲话,前天这个讲话,有些地方补充一点,修改一点,今天讲的同志们提了许多意见,这个东西不准备全部公开发表,因为有些问题比如讲罢工罢课这个问题,一公开发表,那好吧,全国就罢起来了吧!(笑)而我们的干部没有准备,我们的同志没有准备,地方来的同志开过会回去的时候,请你们要使他们有准备,现在这个事情我们好讲,我们坐在这里好讲,身当共冲的是他们,要使他们有准备。局部的、若干个别的,是那些极端的官僚主义长期不能解决问题,是指那些范围。最后如果修改一下子,调整一下子,我讲的这一篇,准备发到县一级,党内党外都可以看到。几天之后我们准备开一次宣传教育会议、宣传会议、宣传工作、报纸的会议,准备跟他们再谈一次。完了。(鼓掌)


