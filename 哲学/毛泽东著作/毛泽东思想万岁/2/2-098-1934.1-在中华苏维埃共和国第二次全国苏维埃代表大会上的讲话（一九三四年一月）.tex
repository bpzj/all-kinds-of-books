\section[在中华苏维埃共和国第二次全国苏维埃代表大会上的讲话(一九三四年一月)]{在中华苏维埃共和国第二次全国苏维埃代表大会上的讲话}
\datesubtitle{(一九三四年一月)}


同志们,现在我代表中华苏维埃共和国中央执行委员会宣布第二次全国苏维埃代表大会的开幕了。

同志们,今天是第二次全国工农兵苏维埃代表大会开幕的日子,我代表中央执行委员会向全体代表们致革命的敬礼!

同志们:第一次全国苏维埃代表大会以来,已经有两年又两个月了。两年以来,全国事变的发展,完全显示了反革命统治阶级是更进一步动摇崩溃,而苏维埃运动与全国革命斗争则是大大的发展了。

中国是一个被帝国主义压迫侵略的国家,是一个受地主资产阶级国民党政府屠杀、压迫、奴役的国家,全国广大的领土是很多被国民党送给帝国主义者了,使全中国受着帝国主义瓜分共管的威胁,使中国快到了完全灭亡的地位。在这种情形之下,中国人民,在中国共产党领导之下,自己团结起来,武装起来,创立了自己的政府与军队。我们在第一次全苏大会宣告了中华苏维埃人民共和国的成立。从此全国就是两个政权的极端尖锐的对立了。

两年来革命的发展,使全国革命形势更加开展了。广大民众团结在苏维埃旗帜之下,向我们的敌人帝国主义国民党进攻,两年来我们得到了极大的胜利。帝国主义国民党在我们胜利的面前发抖起来了,他们继续四次“围剿”之后,组织了五次至六次的“围剿”。但是我们打破了敌人五次的“围剿”,在粉碎第六次“围剿”中间,我们已经取得了第一步胜利。现在我们是处在对六次“围剿”的决战中,是处在最紧急的关头。

两年来,全国红军在浴血的战斗中,取得了伟大的胜利,在这中间,我们许多同志为着苏维埃流尽最后一滴血,而光荣地牺牲了。许多在国民党区域在白色区域领导革命斗争被帝国主义和国民党屠杀了。这些同志中间如黄公略,赵博生,韦拔群,恽代英,蔡和森,罗登贤,邓中夏,陈原道,何子述,鲁易,王良,关阿林,张锡龙,吴高群,彭鳌,孙小宝,傅作玉,童长荣,伯阳等等,他们是在前线上,在各方面的战线上,在敌人的枪弹下屠刀下光荣地牺牲了。我建议我们静默三分钟向这些同志表示我们的哀悼和敬仰(全体代表起立静默三分钟)。

第二次全国苏维埃代表大会的庄务是要彻底粉碎敌人的六次“围剿”,是要把苏维埃运运推到全国去,是要反对帝国主义国民党灭亡中国的阴谋毒计。我们的大会是担负着很大的责任,我们的大会应该号召全中国民众为着扩大一百万铁的红军而斗争,号召全苏区全中国民众武装起来,集中一切力量,粉碎六次“围剿”,争取革命战争的最大的彻底胜利,实行抗日讨蒋,争取革命在全中国的胜利。

我们的大会是全国最高的政治机关,他有着极大的力量来解决这些问题,我们的大会将使六次“围剿”得到彻底粉碎。把革命发展到全中国去,把苏维埃版图扩大到一切帝国主义和国民党统治的地方去,把红旗插到全国去,让我们高呼:

第二次全国苏维埃代表大会万岁!

苏维埃新中国万岁!

(压倒一切的鼓掌声中全体起立欢呼!)


<B>报告</B>
(一月二十二日)

(一)目前形势与苏维埃运动的胜利。

(二)帝国主义的进攻与苏维埃政府对于反帝运动的领导。

(三)帝国主义国民党的“围剿”与苏维埃政权反对“围剿”的斗争。

因为中国苏维埃区域是全中国反帝国主义的革命根据地,中国工农红军是全中国反帝国主义的主力军,因为苏维埃运动与革命战争均猛烈向前发展,所以国民党一次、二次、三次、四次、五次,以五、六次在帝国主义的直接帮助之下,集中一切力量,向着苏维埃与红军进行绝望的进攻,企图消灭中国革命势力,为帝国主义们瓜分中国担负肃清道路的责任。

但是帝国主义国民党的每一次进攻,都遭到了严重的失败,中华苏维埃与工农红军在全中国万众的拥护之下,由于中国共产党的正确的领导,已经成了不可战胜的力量,同时苏维埃与红军的胜利,更加兴奋了全中国的劳苦大众,使他们认识只有苏维埃与红军才是真正为了民族的独立自由而战!只有苏维埃与红军才能救中国。

当敌人四次“围剿”开始的时候,正是国民党出卖了东三省,签订了上海停战协定之后,卖国的国民党不但不以一兵一卒去抵抗日本帝国主义的侵略,不仅不管苏维埃中央政府与红军的屡次宣言愿意与真正抗日军队订立抗日的战斗协定的提议,相反的国民党卖国罪魁蒋介石,在四次“围剿”失败后,立刻接着举行五次“围剿”,集中了数十万军队进攻鄂豫皖与湘鄂苏区,压迫红军离开围绕武汉的区域。在我们方面,虽然因为要避免与过于强大均敌人力量作战,因为我们主观上某些策略的错误,红四方面军不能不退出鄂豫皖苏区,作了有名的远征,但红四军在四川南江、宣汉、绥定一带创造了的均广大的苏维埃根据地。由于红四方面军的远征,在辽远的中国西北部开展了广泛的群众革命斗争,把苏维埃的种子广播到革命形势比较落后的区域中去了,红四方面军的英勇善战,在不足一年之内,已经在二十余县建立了苏维埃政权,已经发展了十倍以上的红军队伍,号召了整个四川的工农劳动群众与白军兵士倾向于苏维埃革命,在中国西北部建立下苏维埃革命新的强有力的根据地。川陕苏区是中华苏维埃共和国的第二个大区域,川陕苏区有地理上、富源上、战略上和社会条件上的许多优势,川陕苏区是扬子江南北两岸和中国南北两部间的苏维埃革命发展的桥梁,川陕苏区在争取苏维埃新中国伟大战斗中具有非常巨大的作用和意义。这使蒋介石与四川军阀都不得不在红四方面军伟大的胜利面前前发抖起来。同时从洪湖根据地退出的红军第二军团,不但主力没有遭到遭大的损失,而且在川鄂湘边,配合着红四方面军积极行动,取得了新的胜利。即洪湖一带,亦尚有游击队的存在。鄂豫皖苏区方面,我们的根据地虽然受了部分的损失,但留守的红军部队和游击队,英勇地向四周发展了游击战争。

(“报告”的第一、二部分见《毛泽东思想万岁》第一集,144—247页)

至于中央苏区,这里是苏维埃中央政府所在地,是全国苏维埃运动的大本营,因此,当然也就是帝国主义和蒋介石主要进攻的目标,国民党集中了全国大部分的兵力和我们作了顽抗的战斗,他调动了所谓“中央军”、蒋蔡军阀、两广军阀与湖南军阀从四面包围中央苏区及其临近各苏区,然而经过一年的艰苦奋斗,我们获得了空前的胜利,最大的胜利是在一九三三年的上半年,单只这半个年头,中央苏区红军消灭了白军二十四个团,六个营,二个连,击溃白军三个师、十二个团、五个营、二个连,缴获步枪二万枝左右,机关枪、短枪一千枝左右。尤其是在东黄坡战役,消灭了敌人最顽强的基本纵队,使敌人的五次“围剿”遭受了最后的惨败。

在粉碎五次“围剿”的伟大胜利中,红军不但数量上扩大了,而且质量上增强了,红军指挥员、战斗员政治上的坚定,军事技术上的提高,比五次战役以前是有了长足的进步。苏维埃领土扩大了,除四川广大苏区外,在福建的西北部,江西的东部,扩大了广大的苏区,增加了近百万的人口,建立了新的闽赣省,旧的苏区更加巩固了,这表现在苏维埃工作的改善,工农群众革命积极性的提高,农村中阶级斗争的发展,以及苏区中残杀的反革命势力遭受到严厉的镇压。同时,这一胜利影响国民党区域非常之大,广大白区工农群众,在这一胜利影响之下,更加提高了他们斗争的勇气。一切参加围剿的国民党军队,不但兵士中发生了普遍的动摇,甚至干部中亦发生了决大的恐怖情绪,甚至使蒋介石不得不公开宣布:“不剿匪抗日者杀勿赦&quot;的绝望命令。

然而这些胜利的取得,决不是偶然的。他依靠了中国共产党政治路线的正确,依靠了苏维埃政府的领导集中,与他政策设施的适当,依靠了红军的英勇善战,依靠了广大工农群众的热烈的拥护并且还依靠了白区的工农群众的日常斗争和反帝反国民党运动的开展,依靠了全世界无产阶级与殖民地被压迫民众的同情与援助。这些都是战胜敌人的基本条件,没有这些条件,是决不能够取得胜利的。

国民党军阀在五次“围剿”惨败之后,唯一的出路上更加无耻的投降帝国主义,从帝国主义取的大批的借款与军械,聘请大批外国顾问,收集一切旧有力量,组织新的力量(训练新兵,训练新的航空队,训练兰衣社军官团等)总之,集中一切反革命的势力,在帝国主义指挥之下,对于苏维埃与红军进行第六次“围剿”。

苏维埃对于六次“围剿”的斗争,是决定中国或者被帝国主义瓜分而成为完全的殖民地或者独立自由领土完整的苏维埃中国这个伟大斗争的主要环节之一。

苏维埃应该号召一切苏区中白区中参加斗争的群众,明白认识这一斗争的严重性,只有团结一切革命力量用百倍积极百倍坚毅的精神统一于苏维埃指挥之下,才能争取这一斗争的完全胜利。

苏维埃应该指给一切参加斗争的群众:在粉碎五次“围剿”之后我们有着战胜敌人六次“围剿”的一切基本条件。党与苏维埃的正确的领导,红军的坚强与扩大,苏区与白区广大工农群众的斗争积极性,这一切都是我们战胜敌人的基础。

由于我们的努力与统治阶级内部矛盾的发展,已经使帝国主义及国民党的新的大举进攻受到了我们的严重打击,敌人的原定定计划已经失败了,不得不在新的阵地与新的计划之下,向着我们作绝望的进攻,我们是处在六次“围剿”的最后决战面前。国民党军阀的堡垒政策与经济封锁政策虽然极其残酷,但并不是不可战胜的铜墙铁壁。提高我们的军事技术,加强我们的群众工作和士兵工作,改进我们的军事策略,集中我们一切力量去克服这些困难,胜利是属于我们的。

我们应该指出,敌人的困难是大大超过了我们,白军士兵的动摇,敌人统治下工人农民以及广大的小资产阶级的愤恨与不满,统治阶级各派军阀之间的斗争与分裂,援助国民党的各个帝国主义者间的矛盾与冲突,以及国民党财政经济的破产,所有这些,都是革命取得胜利的客观方面的条件。

这里应该指出:当着帝国主义国民党进行六次“围剿”之际,福建出现了一个“人民革命政府”,这个“人民革命政府”的出现,表现国民党系统的进一步的破裂,由于苏维埃运动的伟大胜利与国民党在全国民众面前破产,使得中国反动统治阶级的一部分不得不釆取新的方式,企图于国民党道路与苏维埃道路之外寻找第三条道路,以保持反动统治阶级垂死的命运。然而这一企图只是徒劳,因为如果“人民革命政府”这一类的组织不从真正中国人民利益出发,而坚决承认苏维埃政府还在去年四月间即已宣布了的三个条件,而与苏维埃政府订立并真正执行反帝反国民党的协定,而且是止于欺骗与讲空话,那么,广大的革命民众,不会于“人民革命政府”与国民党政府采取任何不同的态度,它必然要遭受悲惨的失败也是可以预言的。而苏维埃在全国民众对它的信仰日益增加中,在国民党及一切反革命派别欺骗的日益破产中,将坚决的粉碎六次“围剿”,以便努力阻止帝国主义殖民化化中国的道路,努力争取在全国范围内苏维埃革命的胜利,在事实上证实这一句名言:“只有苏维埃才能够救中国”。

(四)两年来苏维埃各种基本政策的设施

当着我们来说苏维埃各种基本政策时候,首先要问什么是这些政策的出发点呢?答复这个问题,应该明白苏维埃过去与现在所处的环境和从这种环境所产生的任务。

苏维埃的过去时期,他是生长于游击战争中,他是从许多极小的地方生长起来。这些地是各自独立没有联合,每一个苏区的四周都是敌人的世界,敌人对于苏区是每时每刻的摧残与压迫。然而,他能够战胜这些敌人,他是从战胜这些敌人无数次压迫中间生长和发展起来的。这就是苏维埃的环境。

苏维埃现在所处的环境,同过去有许多的不同;他有广大的领土,有了广大的群众,有了坚强的红军,他已将许多散漫的力量集中起来(虽然还没确完全集中起来),他已经组织成为一个国家,这就是我们的中华苏维埃共和国。这个国家已经有了他的地方与中央的组织,已经建立临时中央政府,这个政府是一个集中的权利机关,他依靠着广大的民众,依靠民众武装的力量一红军。这个政府是工农的政府,他实行了工人与农民的革命民主专政,他对于工农和广大民众是广大的民主。同时他是一个专政,是对占人民中极少数的军阀、官僚、地主、豪绅和资产阶级的专政,而且已经是一个具有极大权力的专政,这个专政已经向着全国范围扩大他的影响,他在广大民众中间有了很大的信仰,他与过去游击战争时代情形大不相同了。然而战争仍旧是经常的生活,并且更加广大与激烈,原因是这个专政与国民党地主资产阶级专政之间的对立一天一天尖锐起来,现在已经进到两方面快要决定胜负的时期,帝国主义国民党大规模的“围剿”是摆在他的面前。则就是苏维埃现在的环境。

这种环境决定了他的任务,就是他必须用全部力量去动员民众,组织民众与武装民众,必须一时不停地去进攻他的敌人去粉碎敌人对于他的“围剿”。他的任务是革命战争,是集中一切力量去开展革命战争,用革命战争么打倒敌人,并且还要打倒强大的帝国主义的统治,因为帝国主义是敌人那一个专政的拥护者与指挥者。打倒帝国主义与国民党的目的,为的是要解放中国人民,为的要把中国四万万人民从日本帝国主义和其他帝国主义的奴役和蹂躏之下解放出来,为的要使几万万中国劳苦同胞从军阀、官僚、豪绅、地主资产阶级压迫之下翻过身来,为的使中国民众能够学习苏联工农一样将来地在中国共产党领导之下建立光明的,幸福幸的人类新生活的社会主义社会。这就是苏维埃的根本任务。

从此,我们明白苏维埃在这种环境与任务之下,施行各联基本政策,是为了什么呢?为了巩固已经胜利了的工农民主专政,为了发展这种专政到全国范围内去,为了动员组织武装全苏区全中国的工农劳苦群众以坚决的革命战争推翻帝国主义与国民党的统治,来巩固与发展这个专政,并且为了从现时工农民主专政,准备将来变到社会主义的无产阶级专政去,这就是苏维埃一切政策的出发点。

中央执行委员会与人民委员会秉承第一次全苏大会的指示,两年以来坚持这种政策的总方向,取得了伟大的成绩。已经从经验上证明给中国全体民众看:只有苏维埃政府的政策,才是为了民众政权与民众利益的政策,才是与帝国主义国民党的反革命政策坚决对抗,推翻帝国主义国民党在全国的统治,挽救全体民众出于危亡,解放全体民众出于水火的唯一的政策。

不待说,在两政权尖锐对立的中国,苏维埃每一具体的施政,要立即取得广大民众的拥护。在帝国主义国民党的反革命政策之下,受尽压迫剥削的民众对于苏维埃每一具体政策的施政,简直如同铁屑之追随于滋石。这种情形成了反动统治的极大恐慌,反动统治阶级因此不惜以一切最无耻的造谣来污蔑苏维埃的施政。然而铁的事实,是给无耻造谣的有力回答,每一个有眼睛的中国人,只要不是丧心病狂的帝国主义者和国民党地主资本家,便不能不承认苏维埃政府的政策与国民党政府的政策有何等天渊之别。

一、苏维埃的武装民众与建设红军

我们首先来说苏维埃的武装民众与建设红军。

为着反对敌人的“围剿”,为着进行革命战争,为着保卫中国民族和国家,苏维埃的第一个任务就是武装民众与组织坚强铁的红军,组织地方部队和游击队,组织关于进行战争的给养与运输。两年来在与敌人四次五次以及六次“围剿”坚决斗争中,苏维埃在这一方面的努力大大取得了成功。

首先是中央革命军事委员会的建立,统一了全国红军的领导,使各个苏区各个战线的红军部队,开始在统一的战略意志之下,互相呼应与互相配合行起来,这是在散漫的游击队的行动进到正规的大规模的红军部队的行动的主要关键。两年来革命军事委员会领导着全中国红军,首先是中央苏区红军,进行了光荣的胜利战争,粉碎了敌人的五次“围剿”,并且取得了反对六次“围剿”的第一步胜利。而四川红军在中央革命军事委员会领导之下,已经取得了光荣伟大的新胜利。

红军在两年内是迅速扩大了。是比两年前扩大了几倍。这一方面之所以得到了成功,是依靠于广大工农群众参加革命战争的积极性,并且还依靠于动员方法的进步与为苏维埃优待红军法令的执行。在一九三三年五月一个月中间,中央苏区的若干个县中扩大了近两万的新战士。很多的地方,工农群众潮水一般地涌进红军中去。一切以为群众不愿意当红军,或者以为在新苏区边区等处不能扩大红军的机会主义的说法,事实已经证明是错误的。然而动员方法之正确,苏维埃优待红军法令之彻底执行,是迅速完成动员计划的关键。废弃一切强迫命令,实行充分的宣传说服,制裁破坏扩大红军以及领导开小差的阶级异己分子,是动员方法的重要节目。提高红军战士的社会地位到最光荣的标准,给予红军战士一切可能与必需的精神上与物质上的待遇,分配外籍红军战士以土地,而发动群众代替他们耕种。为每一红军战士的家属很好的耕种土地,实行消费合作社对于红军家属百分之五的廉价,实行红军家属开办供给日用必需品的专门商店,实行在国家企业与合作社的盈利中抽百分之十供给红军家属,号召群众为红军家属的疾病困难募捐接济,号召群众对于红军战士及其家属给予精神上物质上的慰劳,所有一切关于优待红军战士及其家属的法令与办法,实际与彻底的执行,是保障红军踊跃的上前线去及巩固其在前线上的战斗决心的必要与重要的步骤。这些工作,在苏区各地存在着很多的模范,在这些地方的广大工农群众,以手执武器保卫苏区与发展苏区为自己的神圣职责,而大批的不断的涌向前线去。其中,如江西的长闹乡十六岁至四十五岁的全部青年成年男子四○七人中,出外当红军做工作的去了三百二十人,留在乡间的八十七人,去的与留的成为百分之八十与二十之比。福建的上才溪乡,全部青年成年男子五百五十四人中,出外当红军做工作的四百八十五入,留在乡间只有六十七人,去与留的比例为百分之八十八与十二。这些乡中的壮丁这样大数量的英勇的上前线去,然而乡中的生产,家庭的生活怎么样呢?不但不发生不好的影响,而且更加扩大了改良了。什么原因?因为劳动互助社,耕田队,及其他一切的办法,有组织有计划的调济乡村的劳动力,解决了红军家属每一个困难的问题。我想这种光荣的教训是值得全苏区学习的。

红军铁一样的巩固,应使之与红军的扩大密切连接起来,两年以来这一个方面的工作,同样的得到了好的成绩。现在的红军,已经走上了铁的正规的革命武装队伍的道路,这表现在于:(一)成分提高了,实现工农劳苦群众才有手执武器的光荣权利,而坚决驱逐那些混进来的阶级异己分子。(二)工人干部增加了,政治委员制度普遍建立了,红军掌握在可能的指挥者手中。(三)政治教育进步了,坚定了红军战士为苏维埃斗争到底的决心。提高了阶级自觉的纪律,密切了红军与广大民众之间的联系。(四)军事技术提高了,现在的红军虽然还缺乏最新式武器的釆用及其使用方法的练习,然而一般的军事技术,是比过去时期大大进步了。(五)编制改变了,使红军在组织上增加了力量。所有这些,大大提高了红军的战斗力,成为不可战胜的苏维埃武装力量。

广泛的扩大赤卫队与游击队,是苏维埃武装民众进行革命战争的极端重要的事业。赤卫军少先队是前线红军的现成后备军,是保卫苏区的地方部队,并且是从现在的自愿兵役制转变到将来实行义务兵役的桥梁。而游击队则是新苏区的创造者,是主力红军不可缺少的支队。两年来每个苏区中是发展了这些部队。他们的军事政治训练也相当的加强了。他们加入红军,他们之保卫地方,他们之袭敌扰敌,在历次粉碎“围剿”的战斗中,显示了他们极其伟大的成绩,致使敌人惊为奇迹,成为敌人侵入苏区的绝大困难。这在中央苏区与闽浙赣苏区,是特别表现了他的作用。把这个制度广布到一切新开辟的苏区去,极大的扩大他们的组织,加强他们的训练,使这些部队成为红军在革命战争中最可靠的兄弟,这是苏维埃的重要责任。

充实红军的给养与供给,组织联络前线与后方的军事运输,组织军事的卫生治疗,同时于革命战争有决定意义的事业。在我们还没有取得若干中心城市与敌人经济封锁的情况之下,这一任务的进行是极其艰难的,然而两年以来,凭借了苏区与白区广大工农的积极性,使我们对这事业亦已建立了相当的基础。我们已经在这一方面保证了红军在过去长期中的给养供给与运输,这不能不说是一个极大的成绩。但是当前粉碎六次“围剿”的决战,及以后更加扩大的战争,需要我们用更大努力增加这一方面的力量,保证这一方面更加充分的供给。

更大规模的革命战争是在我们的面前,苏维埃武装民众政策更加显示了他的绝对重要性。一刻不放松去武装民众,去从切实的工作中以最快速度,实现一百万铁的红军的创造,是苏维埃基本的战斗的任务。

苏维埃的基本任务是革命战争,是动员一切民众力量去进行战争。环绕着这个基本任务,苏维埃就有着许多迫切的任务。他应该对广大民众施行广泛的民主。他应该坚决镇压内部的反革命。他应该启发工人的阶级斗争,发展农民的土地革命,在工农联盟以工人阶级领导的原则下,提高工农的积极性。他应该执行正确的财政经济政策,保证革命战争的物质需要。他应该实行文化革命,武装工农群众的头脑,以及其他的许多基本政策。所有这一切,都是为着一个目的,以革命战争去推翻帝国主义国民党的统治,巩固与发展工农民主专政,并准备转变到无产阶级专政的阶段去。

二、苏维埃的民主制度

现在我们来说苏维埃的民主制度。

工农民主专政的苏维埃,他是民众自己的政权,他直接依靠于民众。他与民众的关系必须保持最高程度的密切,然后才能发挥他的作用。苏维埃具有绝大的力量,他已经成为革命战争的组织者与领导者,而且也是群众生活的组织者与领导者,他的力量的伟大。是历史上任何资产阶级国家形式所不能比拟的。但他的力量完全依靠于民众,他不能够一刻离开民众。苏维埃政权需要使强力去对付一切阶级敌人,但对于自己的阶级一一工农、贫民、职员、革命知识分子等大多数民众,则不能使用强力,而他表现出来的只是最宽泛的民主主义。

苏维埃最宽泛的民主,首先表现于自己的选举。苏维埃给予一切过去被剥削被压迫的民众以完全的选举权与被选举权,使女子的权利与男子同等。工农劳动群众对这种权利的取得,只是中国历史上的第一次。总结两年来各地苏维埃的选举经验,一般来说是有很大的成绩。第一,关于选民登记。用红榜白榜的办法、将有选举权的居民与无选举权的居民实行严格的划分。以不准任何剥削分子参加的选民大会的选举,代替了过去开群众大全选举的办法。第二,关于成份比例。为了保证无产阶级在苏维埃政权中的领导干部,采用了工人及其家属十三名选举代表一人,农民及贫民五十人选举代表一人的办法,拿了这样的成份去组织市乡代表会议。从区到中央,各级的代表大会与执行委员会,工人与农民的代表都有适当的比例。这样便在苏维埃政权的组织上保证了工人与农民的联盟,并使工人站在领导的地位。第三,关于选举单位,为了保证多数的选民参加选举,并使工人能够选举他们的适当的代表进苏维埃,一九三三年九月中央执行委员会颁布的新的选举法,规定每个乡苏和市苏,分戍几个选举单位进行选举。即是农民以村为单位进行选举,工人则单独为一单位进行选举。这样就便民众参加选举十分便利了。第四,关于参加选举的人数。苏维埃选举运动的发展,使选举群众极大的认识了选举与自己生活的关系,过去不积极参加选举的民众,现在许多都积极起来了。一九三二年两次选举与一九三三年下半年的选举,许多地方达到了百分之八十以上的选民,有些地方仅只害病的生育的以及担任警戒的入不会参加选举会。第五,关于选举名单。一九三三年下半年进行的选举,实行了候选名单制度,使选民在选举之;先就有应否选举其人的准备。第六,关于妇女的当选。现在多数的城乡苏维埃,妇女当选为代表的占百分之二十五以上。部分地方如上杭的上才溪乡,七十五个代中妇女四十三个,占了百分之六十。下才溪乡九十一个代表中妇女五十九个,占百分之六十六。广大的劳动妇女是参加国家的管理了。第七,关于工作报告。即是由乡苏市苏在选举以前,召集选民开会,报告苏维埃的干作,并引导选民批评这种报告。这一办法,一九三三年下半年进行的选举,也比较上一年实行得更加普遍了。所有这些,都使民众对于行使管理国家机关的权利的基本步骤一一苏维埃的选举,有了完整的办法,保证了苏维埃政权巩固的基础。

其次,苏维埃的民主,见立于市与乡的代表会议。市乡代表会议制度是苏维埃组成的基础,是使苏维埃密切接近于广大民众的机关,两年来的进步,使我们的这一制度更加完满了。其最显著的特点在于,(一)为着使乡苏市苏的代表与当地居民密切联系,便于吸收居民的意见,并便于领导工作起见,依照代表与居民住所接近,将全体居民适当分配于各个代表的领导之下,(通常以居民三十八至七十人置于一个代表的领导之下),使各个代表对于其领导下的居民发生固定的关系。这样便使民众与苏维埃在组织上连成一片了。(二),乡苏市苏的代表,按其住所接近,在三个至七个代表之中选举一人为代表主任,其任务是在乡苏及市苏主席团领导之下,分配和指导其领导下各代表的工作,传达主席团的遍知于各个代表,召集其领导下的居民开会,解决其领导下居民中的较小的问题。一村之内,并须有二个总的代表主任,负领导全村工作之责。这样便使市乡主席团与代表之间密切的联系起来,并使村的工作得到了有力的领导。(三)在乡苏维埃与市苏维埃之下,组织各种经常的及临时的委员会,如优待红军委员会、水利委员会、教育委员会、粮食委员会、卫生委员会等,其数可以多至数十,吸收群众中大批的积极分子参加这些委员会的工作。不但乡有委员会,村亦应该有某些必要的委员会。这样便把苏维埃工作组成了网,使广大民众直接参加了苏维埃的工作。(四)乡苏及市苏的选举,规定每半年举行一次(区苏亦半年一次,县苏省苏则每年一次),这样便使民众的新的意见容易涌现到苏维埃来。(五)在两次选举之间,代表有犯重大错误的,得由选民十人以上的提议,经选民半数以上之同意撤回之,或由代表会议通过开除之。这样便使不良分子不能长期驻足于苏维埃机关了。所有这些,都是苏区中许多地方正在实行着的市乡苏维埃的特点。大家都可以看见苏维埃政权的民主发展到了这样的程度,实在是历史上任何政治制度所不曾有的。而苏维埃依靠这一制度,同广大民众结合起来,他能使苏维埃成为最能发扬民众创造力的机关,使苏维埃成为最能动员民众以适应国内战争,适应革命建设的机关,这也是历史上除苏联外无论什么政府所做不到的。区以上各级苏维埃政权机关完全建筑于市乡苏维埃的基础之上,由各级的工农兵代表大会与执行委员会而组成,政府工作人员,由选举而任职,不胜任的由公意而撤换,一切问题的讨论解决根据于民意,所以苏维埃政权是真正广大民众的政权。

再次,苏维埃的民主,还见之于给予一切革命民众以完全的集会结社言论出版与罢工的自由。当着国民党统治区域剥夺一切革命民众的自由权利,执行疯狂的法西斯蒂恐怖的时候,苏维埃政府下每个革命的人民,都有发表自己意见的权利,苏维埃并且给予一切可能的物质条件上的便利(会场、纸张、印刷机关等等),一切为了反对帝国主义国民党的集会结社与言论出版,苏维埃总是极力的领导者,苏维埃所不允许的,只是那些压迫剥削分子的反革命自由。

不但如此,为了巩固工农专政,苏维埃必须吸引广大民众对于自己工作的监督与批评。每个革命的民众都有揭发苏维埃工作人员的错误和缺点之权。当着国民党贪官污吏布满全国,人民敢怒而不敢言的时候,苏维埃制度之下则绝对不容许此种现象。苏维埃工作人员中,如果发现了贪污腐化消极怠工以及官僚主义的分子,民众可以立即揭发这种人员的错误,而苏维埃立即惩办他们,决不姑息,这种充分民主的精神,也只有苏维埃制度之下方能存在。

最后,苏维埃的民主精神,还见之于其行政区域的划分。苏维埃取消了旧的官僚主义的大而无当的行政区域,把从省至乡各级苏维埃的管辖境界都改小了。这有什么意义?这是使苏维埃密切接近于民众,使苏维埃因管辖地方不大,得以周知民众的要求,使民众的意见迅速反映到苏维埃来,迅速得到讨论与解决,使动员民众为了战争为了苏维埃建设成为十分便利。国民党军阀利用封建时代的大省大县大庄乡制度,这仅仅便利于隔绝民众,苏维埃政府是用不着的。这里应该指出:关于村的划分是重要的一节,因为乡苏维埃之下,执行苏维埃工作的最便利的方法,是以村为单位去动员民众,依靠了村的适当的划分,村的民众组织的建立,村的代表与代表主任对于全村的有力的领导,乡村的工作才能收到最大的成效。

三、苏维埃对于地主资产阶级的态度

其次说到苏维埃对于地主资产阶级的态度。

苏维埃实现了最完满的民主制度,他是为广大民众直接参加的,他给予广大民众一切民主的权利,他对于民众绝对不使用也绝不需要使用任何的强力。

但是,地主资产阶级,即一切被革命民众所推翻的剥削分子,苏维埃对之则是另一种态度。

地主资产阶级,因为他们是剥削者,因为他们是过去的统治者,所以他们对于苏维埃是怀着极端的仇恨的。因为他们虽被推翻但并未消灭,他们还有根深蒂固的社会基础,他们还有优越的智识与技术,所以他们虽被推翻,却时时企图复辟,企图推翻苏维埃政权,恢复原来的剥削制度。特别是在国内战争时代,敌人对于苏区不断举行军事的进攻,更使这些被推翻的剥削者时刻企图以反革命行动响应进攻的敌人。因此苏维埃政权不能不从各方面对于这些分子施行严厉的制裁与镇压。

苏维埃制裁剥削分子的政策,第一件是拒绝他们于政权之外。苏维埃的宪法规定,对于地主资产阶级及其他一切与革命为敌的人,完全取消他们的选举权,取消他们在红军中在地方部队中服兵役的权利。但这些分子却总是千方百计企图混进苏维埃机关中红军中与地方部队中来,特别在新开辟的苏区,群众斗争的发展还不充分,这些分子更容易利用机会混了进来。过去的经验完全证明,同这些阶级异己分子混入革命政权的活动作残酷无情的斗争,是苏维埃的一个十分重要的任务。

第二件,是剥夺一切地主资产阶级的言论出版集会结社的自由。这种自由苏维埃仅只给予革命的民众,而不给予任何地主资产阶级分子。因为地主资产阶级分子必然的要利用这种自由作为他们反革命的工具,剥夺这些分子的自由是绝对必要的,苏维埃之所以日益巩固,剥夺了这些阶级敌人的自由,减少了他们借以活动的机会,也是重要的原因之一。

第三件,利用革命武力与革命法庭镇压一切反革命活动。苏维埃基于武装民众的任务,建立了坚强的红军与广泛的地方部队,这是苏维埃直接依靠的铁的力量,苏维埃依靠了他,才能战胜帝国主义国民党的武力,才能镇压苏区内部的反革命活动。但是苏维埃还有一个与此相连的镇压反革命的重要的武器,这就是苏维埃法庭。苏维埃法庭直接依靠于武装力量,依靠于国家政治保卫局的活动,依靠于人民的阶级斗争,使苏区中一切反革命企图受到严厉的镇压。数年来各个苏区中都发生了反革命的严重的活动,如中央苏区与湘赣苏区等处的AB团,福建的社会民主党,湘鄂西,鄂豫皖,闽浙赣,闽浙与闽赣等地的改组派,湘鄂赣的托陈取消派等,都曾经企图甚至已经实行他们的反革命暴动。但结果都受到苏维埃法庭的严厉的镇压,克服了他们的暴动阴谋,巩固了苏维埃政权。在这一方面政治保卫局与苏维埃法庭已经聚集了丰富的经验,纠正了过去许多地方没有执行明确阶级路线的错误。苏维埃法庭的群众化,即苏维埃法庭的制裁反革命应该同广大群众的肃反斗争联系起来,现在也更加进步了,巡回法庭的普遍使用就是证明。

总结起来看,苏维埃具备着对于广大民众的十分宽泛的革命的民主主义,但同时就在这种民主主义中间构成了他绝大的权力一一建筑于千百万工农民众坚固的信仰与自觉的需要之上的权力。苏维埃运用这种权力,形成了自己的专政,组织了革命战争,组织了苏维埃法庭。向着阶级敌人开展各方面的激烈的进攻,、而苏维埃法庭则在苏维埃领土之内起了他镇压反革命活动的伟大的作用。

如果拿工农专政的苏维埃法庭与地主资产阶级专政的国民党法庭相比较,那又是一幅绝妙的图画。

苏维埃法庭以镇压地主资产阶级为目的,对于工农分子的犯罪则一般处置从轻,国民党法庭以镇压工农阶级为目的,对于地主资产阶级的犯罪则一般处置从轻,法庭的作用完全给政权的阶级性决定了。

苏维埃法庭一方面严厉镇压反革命分子的活动,苏维埃对于这些分子绝不应该有丝毫的姑息。但是另一方面。对于已经就逮的犯人,却是禁止一切不人道的待遇,苏维埃中央政府已经明令宣布废止肉刑。这亦是历史上的绝大的改革,而国民党法庭则至今充满着中世纪惨无人道的酷刑。

苏维埃的监狱对于死刑以外的罪犯是采取感化主义,即是用共产主义的精神与劳动纪律去教育犯人,改变犯人犯罪的本质。而国民党监狱则是纯粹的封建野蛮的虐杀,法西斯蒂的酷刑,劳苦群众与革命者的人间地狱。

消灭敌对阶级的反革命阴谋,建立苏维埃领土内的革命秩序,而废除司法范围内一切野蛮封建的遗迹,这是苏维埃法庭的目的,苏维埃这一方面的所有的改革,同样是有他的历史意义的。

四、苏维埃的劳动

其次说到苏维埃的劳动政策。

苏维埃在于他的政权的阶级性,基于武装劳动民众以革命战争打倒帝国主义国民党的伟大任务,必须坚决的发展工人阶级斗争,保证工人的日常利益,发展工人的革命积极性,组织工人的这种阶积极性到伟大的革命战争中来,并且使工人成为革命战争的积极领导者,成为巩固与发展苏维埃政权的柱石,这就是苏维埃劳动政策的出发点。

在苏维埃劳动政策之下,工人的利益得到了完全的保护,他与过去的国民党统治时代及现在的国民党区域比较起来,具有天堂地狱之别。

当着苏区还是国民党统治的时代,工人是在做雇主的奴隶,工作时间之长,工钱之少,待遇之残酷,工人地位之没有任何法律保障,这是每个工人所永远不能忘记的。所有这些,在现在的国民党区域,不但是一样的存在着,而且变本加厉了。最近的情形,白区工人的实际工资减少了百分之五十以上,而减工、裁工、关厂,则成为资本家进攻工人的普通方法。因此造成了广大的失业,单以产业工人说,失业人数达百分之六十以上。一切国民党统治区域,罢工是犯罪的行为,一九三三年三月国民党在汉口公开宣布罢工者处死刑。一切工人与资本家的争议,国民党无不是站在资本家方面向着工人压迫的。

但是,这些罪恶,在苏区中便一扫而空了。

苏维埃政权之下,工人是主人翁,工人领导着广大的农民担负了巩固苏维埃的伟大的责任。因此苏维埃的劳动政策的原则是保护工人阶级的利益,巩固与发展苏维埃的政权。根据这种原则,一九三一年十二月颁布了劳动法,一九三三年加以修改,重新颁布。此次修改的劳动法,对于城市与乡村,对于大企业与小企业都能使之应用适当。

现在苏区是一般的实行了八小时工作制,订立了劳动合同与集体合同。在城市内与许多的乡村内已经普遍建立了劳动检查所与检查员,目的是检查雇主是否有违背苏维埃劳动法的行为。对于雇主犯法行为的制裁,则属于专门设立的法庭。为了防止资本家对于劳动者的操纵,为了保护帝国主义国民党长期“围剿”所造成的一部分失业工人,苏维埃垄断了劳动介绍权,一切资本家请工,必须到苏维埃设立的劳动介绍所去。失业救济机关的设立,现在也日益推广了,失业工人一般得到了具体的救济,农村工人又都分配了土地。社会保险制度是确立了,社会保险局已建立在苏区各个城市中。所有这些都是工人们在国民党政权下所丝毫不能得到的。然而苏维埃政权则以为这些政策的实行乃是自己最大的责任。

由于苏维埃坚决执行自己的政策,苏区工人生活得到了很大的改善。

首先是关于工资。苏区各地的实际工资,比较革命前,是一般的增加了。下面是汀洲的一个例子:(略)

从这个表看来,汀洲市工人的工资比战前最少的增加了百分之三十二(木匠),最多的竟增加了百分之一千四百五十,即增加十四倍半(布业工人),这种惊人的增加,完全是反映了国民党时代的惊人的低落。当然汀洲工人的工资是比较其他苏区城市的工资要特别高一些(并且工人吃的伙食是在算内的),但是其他城市的工资也是增加了的,如瑞金城市的泥水木匠工人,在最近一时期,从革命前每日两角五分,增加至每日四角五分,增加了百分之八十。

不但城市、农村中工资也增加了。……

工资的交付,一般都办到了按期付清,因为苏维埃的监督,雇主拖欠工资的事情是很少了,少数顽固的资本家,经过劳动法庭的裁判,也不敢与工人为难了。

关于法定工作时间――八小时制的实施,两年来在苏区的一切城市是普遍实现了。农村中的雇佣劳动者,每日的实在工作时间也少有超过八小时。十六岁至十八岁的雇佣劳动者的时间,是一般比较成年为少。

关于妇女及未成年人的保护,如同工同酬,产前产后休息,十四岁以下童工的禁止,也是一般的实行了。

关于学徒的保护,则一般的缩短了学徒的年限,改良了学徒的待遇,扫除了对学徒的封建压迫。学徒的生活是相当的改善,学徒的工资是增加了(如江西方面,学徒每年至少有十五元的津贴,多的则有每月二元的)。

关于一般待遇,在城市中,特别在国家企业里,工人的卫生与伙食有了很大的改善,各个城市工人的伙食普遍是每月六元以上。农村工人的伙食,与雇主同等。

苏区工人是组织了坚强的阶级工会。这种工会是苏维埃政权的柱石,是保护工人利益的堡垒,同时它又成为广大工人群众学习共产主义的学校。苏维埃对于工会,在法律上保障了它的权利,因此工会会员极大的发展起来。据中华全国总工会的统计,现在苏区工会会员数,仅以中央苏区及其附近几个苏区计算,共有二十二万九千人,其分布:中央苏区十一万人,湘赣二万三千人,湘鄂赣四万人,闽浙赣二万五千人,闽赣六千人,闽北五千八,根据中央苏区的材料,没有加入工会的工人,仅只有三千六百七十六人,不足全体工人的百分之五,即是说百分之九十五的工人是加入工会了。一部分地方如兴国,加入工会的竟达百分之九十八。请问这是国民党区域能够梦见的事情吗?不但中国,全世界除苏联外,那一个帝国主义国家是有这种情形的?

总之,两年以来苏维埃的劳动法,已在苏区所有城市中实行了,乡村中也实现了他的主要条文。在两年当中,虽然遇到了不少的资本家富农对于劳动法的抵抗,但因工人群众的积极斗争与苏维埃的严厉监督,使这种抵抗归于无效。同时对于独立生产者与中农贫农雇用工人,有时发生违犯劳动法的事情,则应该经过恳切的劝告使之明了而自动的拥护劳动法。因此工人的生活得到了极大的改善,工人的革命积极性大大发扬起来,工人在革命战争中在苏维埃建设中,是起了他的伟大的作用。

根据中央苏区公略、万太、龙岗、兴国、胜利、西江、雩都、寻邬、上杭、宁化、长汀新泉十二县的统计,工会会员七万零五百八十人中,现在红军及游击队服务的一万九千九百六十人,等于会员的百分之二十八。参加苏维埃等革命机关工作的六千七百五十二人,等于会员的百分之十;他们大部分是公苏维埃机关负责。以上两项共计二万六干七百一十二人,占会员总数百分之三十八。现在还在家的会员则为四万三千八百六十八人。这十二县在家的工会会员,退还第二期公债四万三千八百五十五元,最近购买经济建设公债十九万七千八百另三元,在家会员平均每人买了四元五角。在家会员现在是党团员的一万二千四百三十五人,占在家会员总数百分之二十八。从这些统计,证明了工人群众积极的加入红军,参加与拥护革命战争,拥护中国共产党。然而这些都是苏维埃保护了工人的利益,发扬了工人的积极性得来的,那些说工人在革命后没有得到什么东西,说工人的积极性没有发扬起来,只可算作完全的胡说。

五、苏维埃的土地革命

现在我们来说苏区的土地革命。

中国的苏维埃与红军,是从土地革命中生长与发展起来的。广大的农民群众在地主阶级与国民党军阀的残酷压迫剥削之下,只有土地革命才能解放他们。苏维埃的土地政策的原则,就在于完全推翻地主阶级与国民党军阀一切封建与半封建的剥削与压迫。

一切过去及现在的国民党区域,农村中是吓人的地租(百分之六十到百分之八十),吓人的高利贷(百分之三十到百分之百),与吓人的苛捐杂税(全国计一千七百余种之多),结果土地集中于地主阶级与富农的手里,绝大多数农民失去土地,陷于求生不得求死不能的惨境。因为土地上面的无情掠夺,农民失掉防御灾荒的能力,结果使水旱灾荒遍于全国,一九三一年被灾区域达八百另九县,被灾人口达四千四百余万人。因为层层的被掠夺,农民缺乏再生产能力,许多耕地变得很瘠,许多简直变成荒地,同时农民仅有的一点出产,又被帝国主义的农产物品倾销所压倒,田此中国农村经济陷于完全的破产状态。农村中土地革命的火焰,就在这种基础上强有力的爆发起来了。

苏区土地革命的威力,扫荡了一切封建的残迹,千百万农民群众从长期的黑暗中惊醒起来,夺取地主阶级的全部土地财产,没收了富农的好田,废除了高利贷,取消了苛捐杂税,打倒了一切与革命为敌的人,而建立了自己的政权,农民第一次从地狱中出来,取得了主人翁的资格,这就是苏维埃政权下与国民党政权下农村状态的根本区别。

第一次全苏大会颁布了土地法,使得全国土地问题的解决有了正确的依据。因为农村中阶级斗争的尖锐化,在分析阶级问题上发生了许多的争论,人民委员会根据过去土地斗争的经验,作出了“关于土地斗争中一些问题的决定”,将地主富农游民等许多问题给予了正确的解决,农村斗争将更有力的发展起来。关于分配土地方法上面的许多问题,如像距离、肥瘠、青苗、山林、池塘等等,还急待收集各地经验作成必要的决定,这在新区分配土地的领导上是非常的必要的。

为着彻底消灭封建残余势力,使土地革命的果实完全落在雇农贫农的手里,中央政府发动了广泛而深入的查田这动。根据一九三三年七、八、九三个月的统计,中央苏区江西福建粤赣三省共计查出地主六千九百八十八家,查出富农六千六百三十八家,从这些被查出的地主富农等手中收回土地三十一万七千五百三十九亩,没收地主现款与富农捐款共计六万另六千九百十六元。农民群众的革命积极性更加发扬起来了,雇农工会与贫农团成为苏维埃在农村中的柱石。三个月中期得到了如此伟大的成绩,证明农村阶级斗争还需要苏维埃予以充分地注意,而查田运动是继续发展农村斗争彻底消灭封建残余的有力的方法,也是完全证明了。

土地斗争的阶级路线,是依靠雇农贫农,联合中农,限制富农与消灭地主。这一路线的正确应用,是保证土地斗争胜利发展的关键,是苏维埃每一对于农村的具体政策的基础。因此苏维埃政府对于那些侵犯中农(主要是侵犯富裕中农)及消灭富农的错误倾向,是应该严厉的给予制裁,同时决不应该放松对于那些同地主富农图谋妥协的错误,土地斗争才能走上正确的轨道。

土地革命斗争的群众工作,两年以来得到了不少的经验。总结其要点则为:(一)分配土地的运动与查田运动,均必须以全力去动员广大贫农中农以及农村工人群众,自己动手向着地主富农作斗争。分田与查田的工作,都必须经过群众的同意。每一阶级成分的处理,必须通过于群众的会议中。苏维埃人员单独的少数人的进行分配土地与清查阶级,那便是有降低群众斗争热情的危险。(二)没收地主阶级土地以外的财产与富农多余的耕牛农具房屋必须以大部分分配于贫苦的群众,如果不是这样做而归之于少数人员使用,那么也就要降低群众的情绪而有利于剥削分子的反抗。(三)土地的分配不宜在长期不定的状态之中,应当在相当短促的期间内分配妥当,使之固定在农民的手中。以后非有当地多数群众的要求,不应轻易再行分配。如果不这样做,那么违反农民的意见,不但影响农民于土地生产的积极性,而且同样将为剥削份子利用了去阻碍土地斗争的发展。(四)查田运动的目的是为了清查剥削成分,而不是为了清查被剥削成分,因此不应该按家按亩去查,而应该动员最广大群众清查那些暗藏着的地主富农分子。(五)必须打击那些阻碍分田与查田的反革命分子,在群众的同意之下用最严厉的办法处治他们,从逮捕监禁,群众公审直至枪决,这是完全必要的。如果不是这样做,那么土地斗争就要受到极大的障碍。(六)应该极力发展阶级斗争,而避免地方斗争与姓族斗争,地主阶级与富农,却是极力想拿地方斗争与姓族斗争来代替阶级斗争以便阻碍土地革命的前进,苏维埃人士不应该去上地主富农的当。(七)土地革命的发展,依靠于农村基本群众的阶级觉悟与组织程度的提高,因此,苏维埃人员必须在农村中进行广泛而深入的宣传,必须健全贫农团与雇农工会的组织。

土地革命不但使农民得到土地,而且要使农民发展土地上面的生产力。由于苏维埃的领导与农民劳动热忱的提高,苏区的农业生产在广大的地方是恢复了,有些并且更加发展了。

在这个基础之上,农民的生活是有了很大的改良。农民推翻了地主与国民党的剥削,生产结果落在自己的手里,因此现在农民的生活比较国民党时代是至少改良了一倍。农民的大多数,过去一年中有许多时候吃不饱饭,困难的时候有些竟要吃树皮,吃糠秕,现在则一般不但没有饥饿的事,而且生活一年此一年丰足了。过去大多数农民每年很少吃肉的时候,现在吃肉的时候多起来了。过去大多数农民衣服穿得很烂,现在一般改良,有些好了一倍,有些竟好了两倍。

那一种生活那一种政权是农民群众愿意的呢?让一切国民党区域的农民群众自己答复这个问题吧!

六、苏维埃的财政政策

再说苏维埃的财政政策。

苏维埃财政的目的,在于保证革命战争的给养与供给,保证苏维埃一切革命费用的支出。但是有着广大的革命战费与革命工作费用的支出的苏维埃共和国,当他还是处在全国范围内的较小部分,又是一些经济比较落后的地方,并且实行着便利于广大民众的税收政策,许多外边人们竟不知道苏维埃财政的出路在什么地方。而国民党占据着广大的区域,大数量的搜刮民脂民膏,为什么反弄到财政破产?

没有什么奇怪,苏维埃的财政政策与财政的使用,同国民党是根本的不同。

苏维埃的财政政策,建筑于阶级的与革命的原则之上。因此,苏维埃的财政来源乃是:(一)向一切封建剥削者进行没收或征发。(二)税收。(三)国民经济事业的发展。

所谓向封建剥削者没收征发,即是向苏区与白区地主富农筹款,根据过去的经验恰好相反:苏维埃把主要财政负担放在剥削者身上,国民党则把主要财政负担放在工农劳苦群众身上。

苏维埃的税收,是统一的累进税,现在简单的两方面实行,这就是商业税与农业税,税收的基本原则,同样是重担归于剥削者。

商业税的征收,分为关税与营业税。关税是以按照苏区的需要程度统制货物的进出口为目的,因此税率有完全免征的,有高至百分之百的。在中国境内,只有苏维埃实行了完全自主的关税制,不受任何外国政府的干涉,一切货物在边境税关纳税之后通行全苏区,无第二次之征税,一扫国民党厘金关卡层层抽剥的虐政。

营业税即是商业所得税(工业税现在没有收)按照商店资本大小盈余多少,征收统一的累进税,资本小盈余少的税轻,资本大盈余多的税重。资本在百元以下,群众的合作社以及农民直接卖出其剩余生产品,这些都实行免税。

农业税依靠于农民的革命热忱,使之自愿的纳税,同样是累进原则的征收法。家中人口少分田少的税轻,家中人口多分田多的税重,贫农中农税轻,富农税重。雇农民红军家属免税,被灾区域按灾情轻重减税或免税。

苏维埃采取统一的累进税法,乃是世界上最优良的税法,而一切资本主义国家所不敢釆用或者不敢彻底采用的。至于国民党的税收,则是一篇绝大的糊涂账。其税收原则是主要取之农民及其他小有产阶级。正税之外,有无数的附加税。据天津大公报一九三三年三月二十一日的统计,国民党区域内捐税名目共有一千七百五十六种之多,而四川的田赋预征到了一九八七年,陕西的田赋比国民党未到时增加了二十五倍,这是国民党对于劳苦民众的“恩德”!

从发展国民经济来增加苏维埃财政的收入,是苏维埃财政政策的重要部分,明显的效验已在闽浙赣苏区表现出来,在中央苏区也表现出来了。这一方面的着重的进行,是苏维埃财政机关与经济机关的责任。这里应该指出;国家银行发行纸币的原则,应该根据于国民经济发展的需要,财政的需要只能放在次要的地方,这一方面的充分注意是绝对必需的。

至于财政的使用,应该根据节省的方法,应该使一切苏维埃人民明白,贪污和浪费是极大的犯罪。向着贪污浪费作坚决的斗争,过去虽有了些成绩,以后还应加倍的用力,节省每个铜板为着战争与革命事业,是苏维埃会计制度的原则。苏维埃对于财政的使用,应该与国民党的使用有绝对的差别。

苏维埃的财政不是没有困难的,红军的扩大,战争的发展,使苏维埃面前有着它财政上面的困难。但是困难的克服,即包含于困难本身之中,开展我们的革命战争,改善我们的苏维埃工作,向着一切国民党区域去扩大我们的财政收入,向着一切剥削份子的肩上安放着苏维埃财政的担子,向着国民经济的发展去增加苏维埃的收入,这就是克服困难的方法。

七、苏维埃的经济政策

(见《毛选》一卷《我们的经济政策》全文)

八、苏维埃的文化教育

现在要说到苏维埃的文化教育了。

为着革命战争的胜利,为着苏维埃政权的巩固与发展,为着动员民众一切力量,加入于伟大的革命斗争,为着创造革命的新时代,苏维埃必须实行文化教育的改革,解除反动统治阶级所加在工农群众精神上的桎梏,而创造新的工农的苏维埃文化。

谁都知道,国民党统治下一切文化教育机关,是操在地主资产阶级手里的。他们的教育政策,是一方面实行反动的武断宣传,以消灭被压迫阶级的革命思想,一方面实行愚民政策,将工农群众排除于教育之外。反革命的国民党把教育经费拿了作为进攻革命的军费,学校大部分停办,学生大部分失学。因此在国民党统治之下,造成了人民愚昧无知。全国文盲数目占全国人口百分之八十以上。对于革命文化思想则釆取极端残酷的白色恐怖。任何进步的文学家,社会科学家一切文化教育机关中的革命份子,都要受到国民党法西斯蒂的摧残。使一切文化教育机关变成黑暗的地狱,这就是国民党的教育政策。

谁要是跑到我们苏区来看一看,那就立刻看见是一个自由光明的新天地。

这里一切文化教育机关,是操在工农劳苦群众的手里,工农及其子女有享受教育的优先权。苏维埃政府用一切方法来提高工农的文化水平。为了这个目的,给予群众政治上与物质条件上的一切可能的帮助。因为现在的苏维埃区域,虽然是处在残酷的国内战争环境,并且大都是过去文化很落后的地方,但是已经在加速度的进行着革命文化建设了。

根据江西福建粤赣三省的统计在二千九百三十二个乡中,有列宁小学三千零五十二所,学生八万九干七百一十人,有补习夜校六千四百六十二所,学生九万四千五百一十七人,有识字组(此项只算到江西粤赣两省,福建未计)三万二千三百八十八组,组员十五万五千三百七十一人,有俱乐部一千六百五十六个,工作员四万九千六百六十八人。这是中央苏区一部分统计。

苏区中许多地方,学龄儿童的多数是进入了列宁小学校,例如兴国学龄儿童总数二万零九百六十九人(内男一万二千零七十六,女八千八百九十三),进入列宁小学的一万二千八百零六人(内男八千八百二十五,女生三千九百八十一),失学的八千一六十三人,(内男生三千二百五十一,女生四千九百十二),入学与失学的此例为百分之六十与四十,而在国民党时代,入学儿童不到百分之十。苏区很多地方的儿童们,现在是用了大部分时间受教育,做游艺,只小分时间参加家庭的劳动,这同国民党时代恰好相反了。儿童们同时又组织在红色儿童团之内,这种儿童团,同样是儿童们学习共产主义的学校。

妇女群众要求教育的热烈,实为从来所未见。兴国夜校学生一刀五千七百四十人中,男子四千九百八十八人,占百分之三十一,女子一万另七百五十二人,占百分之六十九。兴国识字组组员二万二千五百十九人中,男子九千人,占百分之四十,一女子一万三于五百一十九人,占百分之六十。在兴国等地妇女从文盲中得到了初步的解放,因此妇女的活动十分积极起来。妇女不但自己受教育,而且已在主持教育,许多妇女是在作小学与夜校的校长,作教育委员会与识字委员会的委员了。女工农妇代表会在苏区是一种普遍的组织,它注意于劳动妇女群众的整个利益,妇女教育当然是他们注意的一部分。

群众识字的人数是迅速增加,识字的办法有夜校,识字远动与识字牌,夜校有一定的地点,识字组在群众的家里,识字牌在道路的旁边。领导识字运动的机关则为乡村的识字运动委员会。拿兴国来说,全县有一百三十个乡的识字运动总会,五百六十一个村的识字运动分会,三干三百八十七个分会下面的识字小组,二万二千五百二十九个加入识字小组的组员。这是扫除文盲的极大规模的群众运动,这种运动应该使之向着全苏区一切城市与乡村中间开展去。

苏区群众文化运动的迅速发展,我们看报纸的发行也可以知道。中央苏区已有大小报纸三十四种,其中如《红色中华》从三千份增加到四、五万份以上,《青年实话》发行二万八千份,《斗争》只在江西苏区每期至少要销二万七千一百份,《红星》一万七千三百份,证明群众文化水平是迅速提高了。

苏区中群众的革命的艺术,亦在开始创造中,工农剧社与工农歌舞团的运动,农村中俱乐部运动,是在广泛的发展着。

群众的红色体育运动,也是迅速发展的,现虽偏僻乡村中也有了田径赛,而运动场则在许多地方都设备了。

苏区还缺乏完备的专门教育的建设。但为了革命斗争领导干部的创造,我们已设立了红军大学,苏维埃大学及马克思共产主义大学,及教育部领导下的许多教育干部学校,中等教育与专门教育之应该跟着普通教育的发展而使之发展起来,无疑的应该成为教育计划中的一部分。

为了造就革命的知识分子,为了发展文化教育,利用地主资产阶级出身的知识分子为苏维埃服务,这是苏维埃文化政策中不能忽视的一点。

苏维埃文化教育的总方针在什么地方呢?在于以共产主义的精神来教育广大的劳苦民众,在于使文化教育为革命战争与阶级斗争服务,在于使教育与劳动联系起来,在于使广大中国民众都成为享受文明幸福的人。

苏维埃文化建设的中心任务是什么?是履行全部的义务教育,是发展广泛的社会教育,是努力扫除文盲,是创造大批领导斗争的高级干部。

每个人都明白,所有这些方针与任务,只有在苏维埃政权之下才有实现的可能,因为这是阶级斗争极端尖锐的表征,这是人类精神解放绝大的胜利。

九、苏维埃的婚姻制度

现在说苏维埃的婚姻制度。

为了解放妇女于野蛮封建的婚姻制度之下,为了实行真正男女平等的婚姻制度,还在一九三一年的十一月,中央执行委员会即颁布了苏维埃的婚姻条例,在这里,确定了结婚与离婚的完全自由,废除了包办强迫买卖的婚姻制度,禁止蓄带童养媳。两年来在一切苏维埃管理区域是一般的实行了这一法令,凡非亲族血统在五代以内,非神经病与危险性的传染病,男子年满二十、女子满十八,经双方同意,并在乡苏与市苏举行登记,即可以实行结婚,离婚则只要男女一方提出要求,经过多苏或市苏登记就行了。

这种民主主义的婚姻制度,打碎了中国四千年束缚人类尤其是束缚女子的封建锁链,建立适合人性的新规律,这也是人类历史上伟大的胜利之一。

但是,这一胜利,是附属于工农民主专政的胜利之后的,因为工农劳苦群众婚姻制度的解放,必须首先推翻地主资产阶级的专政,实行土地革命,男女劳动群众尤其是妇女第一有政治上的自由,第二也有了经济上的相当的自由,然后婚姻自由才有最后的保障。苏区中劳动妇女同男子一样有了选举权,并且分配了土地和工作,所以新制度是能够完全的实行了。

因为数千年来婚姻关系野蛮得无人性,女人所受压迫此男子更甚,所以现时苏维埃的婚姻法令着重于保护女子,把因离婚而起的义务更多的给了男子负担。

因为小孩子是革命的新后代,过去社会习惯上不甚注意小孩子的保护,所以关于保护小孩,婚姻法法令上有了单独的规定。其中关于私生子地位的承认与私生子的保护,是给了特别的注意的。

这婚姻制度的实行,使苏维埃取得了广大的群众的拥护,广大群众不但在政治上经济上得到解放,而且在男女关系上也得到解放。

就拿婚姻制度一件事来说,苏维埃区域与国民党区域,也是两个绝对相反的世界。

十、苏维埃的民族政策

最后,关于民族政策。

争取一切被压迫的少数民族环绕于苏维埃的周围,增加反帝国主义与反国民党的革命力量,使一切被压迫民族得到自由与解放,是苏维埃民族政策的出发点。

国内的许多少数民族如蒙古人,西藏入,新疆维回,甘肃回民,高丽人,安南人,苗人,黎人等等都受着帝国主义和中国历来封建皇帝与封建军阀的剥削和统治,国民党所谓“五族共和”只是欺骗人的鬼话,而且冯玉祥的屠杀甘肃回民,白崇禧的屠杀广西苗族,乃是国民党最近的“赏赐”。另方面,少数民族自己内部的统治阶级,如王公活佛喇嘛土司等,与英日帝国主义及国民党军阀相结合,使这些民族的广大劳苦民众遭受更加利害的压迫与剥削,或者他们(王公、活佛、喇嘛土司等)直接投降于帝国主义,引导帝国主义迅速的殖民地化这些区域(如西藏、内蒙)更进一步的掠夺民众,这是少数民族过去与现在生活的实质。

苏维埃政府坚决反对一切帝国主义与国民党军阀对于少数民族的统治与掠夺,一九三一年十一月第一次全国苏维埃代表大会,在其颁布的宪法大纲的第十四条宣言:

“中华苏维埃政权,承认中国境内的少数民族自决权,直到各民族脱离中国建立自己的独立自由国家。蒙古回藏苗黎高丽人等,凡属住在中国境内者他们加入中国苏维埃联邦,或者脱离苏维埃联邦,或者建立自己的区域,均由各民族照自己的意志去决定。中国苏维埃政权在现在必须努力,帮助这些弱小民族,使他们脱离帝国主义国民党军阀王公喇嘛土司等压迫统治,而得到完全的解放。苏维埃政权应在这些民族中间发展他们自己的民族文化和民族言语。”

这是对于全世界帝国主义和中国国民党实行民族压迫的响亮的回答,中国广大工农群众及其苏维埃政府,不但自己正在用坚决的民族革命战争以求脱离帝国主义的羁绊,而且号召国内一切弱小民族同时脱离中国统治阶级以及帝国主义的羁绊,直到这些民族的完全独立。不但如此,苏维埃宪法大纲第十五条又说:

“中国苏维埃政权对于凡因革命行动受到反动统治迫害的中国各民族以及世界各国的革命战士,给以托庇苏维埃区域的权利。并且帮助他们重新恢复斗争的力量,直到这些民族与国家的革命运动得到完全胜利为止。”

苏区的许多高丽、台湾与安南革命同志的寄居,第一次全苏大会高丽代表的出席,这次大会的几位高丽、台湾、安南与爪哇的代表出席,都证明了苏维埃这一宣言的真实。

共同的革命利益,使中国劳动民众与一切少数民族的劳动民众真诚地结合起来了。

民族的压迫基于民族的剥削,推翻这个民族剥削制度,民族的自由联合就代替民族的压迫。然而这只有中国苏维埃政权的彻底胜利才有可能,赞助中国苏维埃政权在全国范围的胜利,同样是各少数民族的责任。

从苏维埃政府所决定的这些政策和实施的这些成绩来看,很明显的苏维埃中国内的民众已经得到了许多中国历史上空前未有过的政治上,经济上、社会生活上的解放和胜利,然而谁晓得,我们苏区目前所作的一切,还只不过是苏维埃革命伟大解施政纲领的部分实现的开始,中华苏维埃共和国自始即有一贯的明显的施政纲领和前途,苏维埃中央政府早已向全中国和全世界明白宣言过,他要完全消灭帝国主义和国民党在中国的统治,他要使中国成为一个完全新式的独立自主的国家,使中国人民过着真正的幸福的新生活,在将来向前发展过程中,它将实现国家工业化政策,它将实现国家农业集体化,它将逐步实行完全有计划的国民经济,它将根本消灭失业痛苦,它将根本消除贫穷的现象,它将使人民享受优裕丰富的生活,它将有不可战胜的国防力,它将有豁然独立的国际地位,它将使中国变成永远没有人压迫和人剥削人的幸福光荣的社会主义社会,换句话说,今日社会主义胜利建设的苏联,这是我们中华苏维埃共和国向前发展的活榜样!

(五)苏维埃在彻底粉碎六次“围剿”争取全国革命胜利面前的具体战斗任务

当作我们说到了目前新的形势,说到了两年来苏维埃政权反对帝国主义反对国民党“围剿”的斗争,说到了苏维埃实施的基本政策的时候,就便我们得到一种确定的结论,即是说苏维埃运动是大踏步前进了。两年以来,苏维埃运动的胜利,明显地变动了敌我两方的力量,敌人加强了他们的动摇与崩溃,而苏维埃运动则在猛烈的发展中,革命的力量是更加壮大了,革命的阵地是更加巩固了。民族战争与革命的国内战争已在中国广大范围中开展着,红军已成为不可战胜的力量,工农民主专政的基础已经确立了,苏维埃工作已经在各方面得到伟大的成绩,苏维埃中央政府的集中领导不但在苏区树立了坚固的基础,而且在国民党统治区构成为广大民众的革命旗帜了,所有这些都已经成为现时生活实际,成为不可否认的客现存在的事实了。

然而革命的前进,要求我们估计到另外一些情形,要求我们以深刻的自我批评精神检查革命的战线中存在着的弱点,这是我们不能放弃的责任。

估计我们的弱点,首先必须明白,现在苏维埃已经胜利的区域虽然是很广大了,但在全国说来,则还是处在较为狭小的范围内,还是处在一些经济比较落后的地方。反革命还保存着他们的广大的区域,还占据着各个重要的城市。因此,苏维埃争取全国胜利的任务是最严重地放在我们的肩上。日益加紧两个政权之间决死的斗争,要求我们以极大的努力去解决这个问题,而不容许丝毫的自满自足观念留存在我们革命队伍中间,也不容许表现任何微小的疲倦态度。

第二,在过去的两年中间,虽然全国民众的反帝运动是广大的开展了,苏维埃中央政府对于运动的领导也有不少的成绩,然而若拿了与当前民族危机的严重性比较,与阻止帝国主义侵略和国民党投降卖国的严重任务相比较,则现时发展着的反帝斗争力量,显然还是异常不够的。苏维埃还没有采用更多的办法去发展广大群众的民族的与阶级觉悟,去组织民众的反帝斗争,而且就在民众自发的反帝斗争中,苏维埃政府直接的帮助与领导还是非常不充分。广大白色区域的工人民资产阶级的斗争,农民反地主的斗争,苏维埃还没有充分的尽其组织与领导的责任。即在苏区周围的国民党区域,也还未用最大力量去组织群众的斗争,使这些区域造成迅速变为苏区的条件,使红军在这些区域作战得到群众更多的配合,特别是在白军土兵中道成暴动响应红军的局面。

第三,红军的数量与质量,虽然是迅速扩大和增强了,但对于执行战胜帝国主义国民党整个军事力量,争取革命在全国胜利伟大任务上,倒还是相差很远。后方扩大红军的工作,还能适应前线的要求。赤卫军,少先队的编制与训练,在许多地方还是差得很。游击队的组织与行动,一般还是很不够的,优待红军家属的工作,在有些地方做得非常之不好,所有这些使得革命战争的发展,还只限于过去的成绩,使得我们每一次冲破敌人“围剿”之后,还不能乘胜直进,争取更加伟大的胜利。

第四,在一切为了战争的任务下,我们还不能使一切苏维埃工作完全适合于革命战争的要求。这不论在土地斗争方面,在经济建设方面,在财政方面,在肃反方面,以及在文化教育方面,都有他的弱点的存在,指出其一般的弱点,即是说,革命战争要求这些工作以很快的速度,争取大的成绩;然而在各地执行起来,都是参差不齐的。很多地方真正达到了所谓很快速度和最大成绩的标准,依靠了这些地方的工作,使革命战争得到了很大的帮助,但在另一些地方,则不但工作进行非常之慢,甚至经过很长的时期,还不能得到应有的成绩,特别在有些新区与边区中间,那里的工作更加差些。这种情形的主要原因,在于这些地方的苏维埃机关中存在着一些不了解甚至不愿意执行苏维埃法令政策的份子。这些份子当中,有些是严重的机会主义者与官僚主义者,有些简直是地主资产阶级派遣进来的暗探。他们不是推进苏维埃的工作,而是妨障了苏维埃的工作。他们不使苏维埃工作服从战争,而是使苏维埃工作离开战争,他们不愿意去开展群众的斗争,而使群众斗争停顿起来。他们不会对于广大群众的动员,不曾对于群众的说服教育,去执行苏维埃工作,而是用空谈空喊,甚至强迫命令的官僚主义去执行苏维埃工作。他们不去了解下层的情形,不去教育新进的干部,不去听取群众的意见,而只是机会主义的诬蔑下级干部不好,那里的群众没有革命积极性。在这些地方苏维埃的民主是没有得到充分发展的,没有吸收最广大群众来参加苏维埃的选举,没有吸收群众中大批积极分子来参加苏维埃的工作,这些地方的市乡代表会议制度,还没有很好的建立,还没有使苏维埃成为真正广大群众自己的政权机关。在这些原因之下,就使得许多苏维埃工作,在这些地方缺乏应有的成绩,使他不能适应革命战争的迫切要求。应该明白指出,这是苏维埃工作中间一个很严重的弱点。

所有这些弱点的存在,给了我们一种深刻的警觉:就是必须克服了这些弱点,苏维埃运动才能适应一切客观的有利条件,而釆取更大规模向着更大范围内发展去。

我们已经有了一个伟大的力量,这个力量成为我们发展的基础。但是革命形势的需要,超过我们的力量,我们的力量还不够,我们必须增加力量。

为彻底粉碎帝国主义国民党的“围剿”争取革命在全国的胜利,第二次全国苏维埃代麦大会必须号召全苏区全中国的一切革命民众,坚决执行下列各个具体战斗任务:

一、在红军建设方面

把中央革命军事委员会,对于全国红军的领导,更进一步地强健起来,使全国红军的行动比较过去更加能够在统一战略意志下互相呼应与互相配合,使各地方军事机关更加能够在中央领导之下充分的执行他们自己的职务。

把扩大一百万铁的红军的口号普遍的深入的传播到全苏区全中国广大工农群众中去,号召群众在很短促的时间之内,为着实现这个最低限度的口号而斗争。要使群众明白,摆在面前的苏维埃政权与国民党政权之间的决定胜负的战斗,苏维埃政权与帝国主义之闻的直接广大的冲突,依靠我们数百万大红军的创立,因此一百万红军的首先创立,是苏维埃与每个工农群众神圣的责任。中央军事委员会与各级地方苏维埃应该负责,收集两年尤其是去年红五月以来,各地扩大红军的丰富经验。着重的指出以充分的政治鼓动,去代替强迫方法。以残酷的阶级斗争与苏维埃在这一方面的法令,去对付破坏扩大红军与领导开小差的阶级异己分子与不良分子,以充分执行苏维埃优待红军战士及某家属的一切法令与办法,去提高红军战战士们的社会地位,去增加红军战士及其家属的精神上的安慰,去解决红军战士及其家属一切物质生活上的困难,是扩大红军的重要方法。还要指出,为红军家属耕种土地以及日用必需品的供给,是优待工作的重要部分。一切对于优待红军战士及其家属怠工消极与阳奉阴违的分子,应该受到苏维埃法律的裁判。

应该把巩固红军放在红军建设的重要地位,使红军不但能够很快扩大,而且能够很快强健起来。应该更进一步提高红军战士的政治教育,使每个红军战士都自觉的为了苏维埃新中国而奋斗到底,使红军成为苏维埃的宣传者与组织者,成为创造新苏区的执行者,使红军战士与广大苏区白区的工农劳苦群众之间发生更加密切的联系。要从政治教育去提高红军的自觉的纪律,使他们明白这是保证战争胜利的重要武器。政治委员制度,应该建立到一切红军部队、地方部队和游击队里面去。应该提拔更多的工人成为各级军事的与政治的指挥员。红军学校应当使之成为比较过去更能训练大批高级的与初级的军事政治干部的学校。注意红军中战士的考察,对于地主资产阶级分子混进红军中来破坏红军的企图,应该给予严重的打击。巩固红军使红军成为铁军的工作,与政治工作同等重要而为现时红军所迫切需要的,就是军事技术的提高。这一任务的解决,在战争的规模日益扩大,在帝国主义国民党军队日益采用新的军事技术的前面,对于我们是绝顶重要的。“学会与提高新的军事技术”的口号,应该深入到每个红军战士中去,红军学校应该为了这一目的去尽他最大的努力。

应该把赤卫军少先队的新编制方法推广到苏区的一切地方去,把一切劳动的青年成年男女,全部武装起来。红军后备军与地方守卫部队的作用与责任,应该使每个赤少队员清楚的认识。义务兵役制在将来更大规模的国内战争中的需要与作用,现在就应该向一切劳苦群众与赤少队员适当的宣传起来。应当用大力进行一切可能的与必要的军事训练与政治训练,野营演习方法应当尽可能推广到一切地方的赤少队中间去。在敌人进攻与苏区剥削分子企图捣乱的情形之下,赤少队保卫地方的责任应该特别的加重,许多地方赤色戒严的松懈现象,应该给予迅速的纠正。动员模范赤少队整连整队的加入红军中去与动员之后立即重新编制起这些队伍来,是扩大红军最好的方法之一。与红军作战不可分离起而其伟大支队作用的,是新区边区以及白区中间的红色游击队。加强和扩大现有的游击队,最广泛的繁殖新的游击队,收集过去游击战争的丰富经验,极大的加强游击战争的教育与指导,把千百支游击队伸出白区去,伸出到敌人的侧方与后方去,在这些地方袭敌击人,发展群众斗争,创造游击区域,以致发展到创造新苏区,特别在淘未连成一片的各个苏区之间去做这些工作,与主力红军的行动互相配合起来,是苏维埃十分迫切的任务。

应该用一切办法去保障红军的给养、供给与运输,苏维埃的财政机关与经济机关,军事系统中的供给运输与卫生机关,应该为着这个共同目标而努力。运输队的动员,应该克服过去的弱点,使红军不致因缺乏运输而妨碍了运动与作战。一切牺牲一切努力给予战争,是每个苏维埃人员每个革命分子的责任。

二、在经济建设方面

为着冲破敌人封锁,抵制奸商操纵,保证革命战争的需要,改良苏区民众的生活,苏维埃必须有计划的进行各种必要的与可能的经济建设。

首先是发展苏区广大的农业生产。苏维埃应该用一切方法去提高农民群众的生产热忱。应该乘着春耕夏耕秋收各个重要的农事季节,进行提高生产的普遍而广大的运动,动员整个农村民众一齐进入生产的战线中。普遍组织劳动互助社与耕田队,有计划的调剂乡村劳动力,动员广大妇女群众参加生产,是扩大生产的重要方法。应该领导并帮助农民去解决耕牛农具肥料种子水利以及防止害虫等及农业上而的具体重要的问题。耕牛合作社应当普遍组织。根据去年春耕夏耕运动的经验,“完全消灭荒田”,“增加今年二成收获”,应当成为战斗的口号。

应当收集种棉经验。发展棉区的棉花生产。应当发起植树运动,号召农村中每人植树十株,牲畜的增殖,苏维埃应给予注意。某些重要农业部门如粮食棉花等,中央国民经济人民委员、部及各省国民经济部,应当作出具体的实施的计划。苏维埃粮食部、粮食调剂局与群众粮食合作社,应当在工作上密切联系起来。为了完全保证红军与民众的粮食供给而努力。

苏区广大手工业的恢复,军事必须工业的建立,是苏维埃经济建设的重要任务。苏维埃恢复与发展工业的计划,应当放在战争需要,苏区民众需要,以及白区出口的可能基础之上。钨砂、煤、铁、石灰、农具、黄烟、纸、布匹、糖、药材、硝盐、樟脑、木材等项,应当是主要部门。应该用极大的努力去发展对于这些工业的群众的生产合作社,将失业工人,独立劳动者与农民,尽量组织到生产合作社来。同时应该容许并奖励私人资本家的投资,扩大苏区的这些生产。苏维埃在目前不应当企图垄断所有的生产事业,但创办并发展一些特别需要与特别有利的国有企业,则是可以而且应当的。提高劳动热忱,发展生产竞赛,奖励生产战线上的成绩显著者,是提高生产的重要方法。

打破敌人的经济封锁,发展苏区的对外贸易,以苏区多余的生产品(谷米、钨砂、木材、烟、纸等)与白区的工业品(食盐、布匹、洋油等)实行交换,是发展国民经济的枢纽。苏维埃对外贸易局与各种商业机关,必须更加健全起来。同时奖励私人商业,使他们为输出与输入各种必要商品而努力。而普遍的发展消费合作社,把广大工农群众组织在这种合作社内,使群众能够廉价的买进白区的必需品,高价的卖出苏区的生产品,则在苏维埃贸易与整个经济建设上都占有特别重要的位置。苏维埃对于消费合作社、中央总社与各省总社的领导,应当极大的加强起来。还没有建立省县总社的地方,应当迅速的建立。

经济建设中资本问题的解决,主要是吸收群众资本,把他们组织在生产的消费的与信用的合作社之内,应该注意信用合作社的发展,使在打倒高利贷资本之后能够成为他的代替物。经过经济建设公债及银行招股存款等方式,把群众资本吸收到建设国家企业,发展对外贸易,与帮助合作社事业等方面来,同样是要紧的办法。应该在苏维埃法律范围以内,尽量鼓励私人资本家的投资,使苏区资本更加活泼。应该尽量发挥苏维埃银行的作用,按照市场需要的原则,发行适当数目的纸币吸收群众的存款,贷款给有利的生产事业,有计划的调剂整个苏区金融,领导群众的合作社与投机商人作斗争,这些都是银行的任务。

三、在苏维埃建设方面

苏维埃中央政府的建立,使全国苏维埃运动得着总的领导机关,对于中国革命有绝大的意义。两年以来,在领导反帝国主义反国民党的斗争中间,得到了光荣伟大的胜利。我们应该指出中央政府在自己的组织上与工作上还存在着许多不健全与不充分的地方。为着加强中央政府对于各苏区与全国革命的总领导,必须使中央执行委员会与人民委员会在工作上划分开来,必须健全中央执行委员会主席团的组织与工作,必须充实各人民委员部的工作人员,并改善他们的工作方法,必须增设必要的人民委员部即如粮食委员部等,使中央政府在革价形势更加开展的面前,能够充分地发挥他总的发动机作用。

省苏维埃是地方政府最高领导机关,是中央政府与各县区苏维埃之间的连锁,必须极大的加强中央政府对于各个省苏的领导,密切中央政府与各个省苏之间的联系(中央区各省与中央区以外各省),产密检查各个省苏的工作。必须积力改善各个省苏的工作。必须改善各个省苏的工作方法,实行集体讨论,精确分工与个人负责的制度。加紧对于各县苏区工作的检查。极力纠正过去有些省工作上松懈与不集中的现象。

乡苏市苏是苏维埃的基本组织,因此用极大努力改善乡苏与市苏的工作。必须在一切尚未建立代表会议制度的地方,把这个制度建立起来。必须进一步加强各地代表会议的工作。应该设立他的主席团。应该设立他的许多委员会,并把委员会制度建立到村里去,吸收大批工农积极分子参加委员会的工作。应该建立每个代表与一定数目的居民发生关系的制度。应该建立主任制。每个村里要有一个主持全村工作的代表主任,应该准许他能够召集一村的代表与居民去开讨论村中工作的会议。乡苏与市苏是动员群众执行苏维埃工作的直接的负责机关,他的工作中心是如何向最健全地最充分地动员全市全乡的民众,为着苏维埃的每一任务每一工作的完满实现面斗争。乡苏与市区苏维埃必须把极大的注意力,放到各村各街道的实际工作上去。必须对于各村与各街道的工作,实行定期检查制度。各村或各街道之间的革命工作竞赛,是争取工作速度的有效方法。乡苏与市区苏维埃切实而迅速的改善,依靠于区苏与市苏正确具体的领导。区苏市苏的注意力应该全部放在各个乡苏各个市区苏维埃的工作改善上面,充分地解释,频繁的巡视,切实的检查,与民众中间的考验,是区苏市苏领导方法的要点。县苏对于区苏工作的考查,也应该以这些为标准。

各省苏必须把自己的注意力极力地投到新开辟的苏区去,把在新区建立并加强革命委员会的工作,看成自己重要的职务。革命委员会的组织形式与工作内容,都与市乡苏维埃有许多不同,一切白色区域变为苏维埃区域,经过革命委员会的过程。因此,健全革命委员会的组织与工作,使革命委员会能够负担起来武装民众,发动民众斗争,肃清反动势力,迅速转变到建立苏维埃政权,是各省苏区各新边区区县苏应该极大的注意的。

苏维埃的民主虽然发展了,但应该指出许多地方还是异常不够的。必须严厉地开展反对官僚主义的斗争,把那些遮塞在苏维埃与民众之间的废物拔开去,这些废物就是官僚主义与命令专义。苏维埃人员应该从对于民众的动员对于民众的说服去执行苏维埃工作,而不应该用强迫命令的办法去执行苏维埃工作。苏维族人员应该注意民众的每一要求与每一提议,而不应该忽视这些要求与提议,苏维埃人员特别是工农检查委员会。应吸收广大民众对于存在在苏维埃机关中的不良份子,开展广大的批评斗争,直至用苏维埃法律严厉制裁他们。保证苏维埃与民众中间良好的关系。为了健全苏维埃的成份,必须实行苏维埃选举的群众化。必须向群众解释选举的意义,吸引最广大选民来参加选举。在选举中绝对的屏除那些阶级异己分子,贪污浪费与官僚主义分子。选举大批的工农积极分子,来管理国家工作.在这里,依照选举法的规定,引进大批工人干部,加强工人在苏维埃政权中的领导地位,是健全苏维埃工作的重要关节。为了苏维埃工作的群众化,苏维埃必须与工会、贫农团、女工农妇代表会、合作社及其他一切民众团体发生密切的联系,经过这些团体去动员广大民众,执行苏维埃的工作。

为了争取苏维埃工作的速度与质量,使一切苏维埃工作适合于革命战争的要求,必须用极大的力量除掉苏维埃工作人员中间的松懈不紧张的现象。绝大的提高苏维埃人员的工作热忱,使好个工作人员都自觉的为工农民主专政的国家工作而努力,同时必须森严工作纪律。一切对于工作不积极,疏忽与废弛职务,把苏维埃工作放在不要紧的位置等等的分子,应该向之作严厉的斗争,、直至开除他们的工作。必须反对贪污浪费现象,因为这种现象,不但是苏维埃财政经济的损失,并且足以腐化苏维埃工作人员,使他们对于工作失去热忱与振奋精神的元素。必须把“一切工作服从战争”“争取工作的速度与质量”的口号提到全部苏维埃人员的前面去,在这一方面,各级苏维埃的主要负责人,特别是工农检查委员会,应该对于苏维埃工作人员,进行充分说服教育工作。

应该把苏维埃法令政策的彻底与忠实的执行,移在全部苏维埃人员的肩上去,应把违反苏维埃法令政策的行为,首先是苏维埃人员自己的违反放在严厉责罚的地位。

必须充分执行劳动法,把劳动法的每一条文解释给广大工人群众听。八小时工作制的实行,最低工资的规定,是保证工人利益的中心与起码的部分。劳动检查所与劳动法庭,必须使之起完全的作用。必须向着那些忽视工人利益而企图与资本家妥协的人员作坚决的斗争。必须对于失业工人实行具体的与及时的救济,失业救济委员会必须在一切有失业工人的地方组织起来,社会保险制度,必须在一切可能实行的地方真实的实行。必须给予社会保险局的工作以应有的注意,必须避免过去有些地方对于保险金支配上的错误。为了这些工作的充分执行,应该把苏维埃劳动部健全起来,劳动部与工会之间应该发生密切的关系。

充分的实行土地法,实行关于土地斗争的一切法令,向着全国范围开展广大的土地革命,这是苏维埃中心任务之一。没收地主阶级和大私有者土地的斗争,应该着重地猛烈的使之在一切新收入的苏维埃版图中开展起来。应该收集过去土地分配方法的许多经验,普遍应用到一切新区去。应该把查田运动开展到一切土地问题尚未彻底解决的地方去,把那些地方封建残余势力给以迅速的肃清。土地斗争中正确的阶级路线与充分的群众工作,是保证土地革命彻底胜利的先决条件。

执行苏维埃的文化教育政策,开展苏维埃领土上的文化革命,用共产主义武装工农群众的头脑,提高群众的文化水平,实施义务教育制度,增加革命战争中动员民众的力量,同样是苏维埃的重要任务。

苏维埃制裁剥削分子及镇压反革命分子的政策,必须坚决执行,国家政治保卫局与苏维埃法庭,必须提高自己的警觉性,对于反苏维埃法令的剥削阶级分子,及一切进行反革命活动的分子,实行严厉的制裁与镇压。在这里政治保卫局工作与苏维埃法庭群众化,动员广大群众参加肃反斗争,是非常必要的。

争取苏维埃工作时速度与质量,使一切苏维埃工作,完全适合子革命战争的要求,这是苏维埃工作的总方向。

四、关于领导反帝斗争与白区工作

为了坚决的反对帝国主义的侵略,为了猛烈的发展全国工农斗争,为了使苏维埃区域扩大到全国去,苏维埃政府必须加强对于全国反帝斗争与国民党区域工农革命斗争的领导。对于这一方面的消极,就是放纵了帝国主义的强盗侵略,就是延长了国民党反动统治的寿命,就是限制了苏区发展的速度与范围。苏维埃中央政府及各省苏维埃,必须把自己的眼光放大到广大的国民党区域去,不但要领导每一个群众自发的反帝运动,而且要在广大工农群众以及小资产阶级群众中间,利用帝国主义的侵略国民党投降卖国的每一具体事实,启发群众的民族觉悟与阶级觉悟,号召他们组织与武装起来,为驱逐帝国主义保卫中国领土而斗争,特别在东三省、热河、察哈尔,华北等日本帝国主义进攻地带,组织人民革命军义勇军,领导旧有的义勇军,使之脱离国民党的反动影响,而与日本帝国主义坚决的作战。苏维埃政府对于工人的每一反帝罢工,农民以及小资产阶级每一反帝斗争,必须尽可能的给予精种上与物质上的帮助。

对于国民党区域工人民资产阶级的斗争,农民反地主的斗争,一切革命民众反帝国主义和国民党的群众斗争,苏维埃政府必须用一切办法去组织、去援助,去领导。一切苏维埃人员应该明白,要想把苏维埃运动发展到全国去,要想把比较苏区庞大若干倍的国民党区域造成转变为苏区的条件。要想创造新苏区,要想在反对帝国主义国民党大规模的“围剿”中间能够得到白区群众的援助,必须把极大的注意放到白区去。必须从苏区派遣人员,准备一切必须的物质帮助,去组织与领导白区群众斗争。对于这一方面的消极,就是对于扩大苏区与发展革命战争的消极。特别是苏区附近的国民党区域,这些地方的群众受苏维埃的影响最大,受国民党军事奴役食盐公卖等等的压迫最甚。苏维埃尤其是各个省苏及边县苏区政府,必须利用各种时机取得与这些群众的联系,组织他们的日常斗争,发展到游击战争,到群众暴动,到建立苏区与老苏区联结一片。在这里苏区与白区交界地带的工作,应该看得非常之重要。在这些地带苏维埃(或革命委员会)与游击队,必须完全遵守苏维埃的基本政策,禁止一切不分阶级乱打土豪的行为。没收地主阶级及反动派的财物,必须大量的发给当地群众。此外关于赤白对立问题,逃跑群众问题、食盐封锁问题,被难群众问题等,必须根据阶级的与群众路线,很好的给予解决。必须把造成赤白对立与群众逃跑的原因除掉了去。交界地带工作的改善,是争取白区变为苏区的重要关节。

同志们!我们苏维埃和红军,正在担负着挽救中国民族危亡的重大责任,我们要完成这个责任,就必须完成苏维埃第二次全国代表大会所指出和规定的任务。同时,同志们都知道:中国苏维埃革命的胜利,不仅是四万万中国民众的解放,而且是整个东方被压迫民族脱离帝国主义锁链的先导,是给日本和其他帝国主义在太平洋上爆发世界帝国主义大战的计划一个致命的打击,是使日本和其他帝国主义从东方战线上进攻苏联的计划受到摧毁,是使世界无产阶级革命胜利的时期大大的缩短逼近,我们的任务是何等的光荣和伟大!

同志们!努力,最后的胜利是属于我们的!(演讲完,随着雷鸣一样的掌声,全体代表起立,欢迎毛泽东同志!欢唱国际歌)</font></p>
<p><font face="黑体">结论(一月二十七日)

同志们!关于我代表中央执行委员会及人为委员会向大会所作的报告,同志们已经讨论了两天了。昨天的分组会,今天的大会,在这两天中间同志们发表了很多的意见,从各方面发挥了我们工作中的经验与教训,总括起来对于我的报告可以说是一致承认的。对于目前的形势,对于从这一形势产生的任务,对于两年来苏维埃政策的各方面的实施,以及对于我们工作中存在的弱点,在昨天今天同志们的讨论中一般是同意了我的报告,同志们的发言,一般都是非常正确的,这是应该首先指出的。

但在昨天今天两天的讨论中间,主要是在昨天的分组会中间,有个别同志的发言包含着不正确的观点,我也应该在结论中指出。这里主要是关于“围剿”的意见。关于这个问题大多数同志都承认我在报告中说的:我们是取得了对于六次“围剿”的第一步胜利,但是六次“围剿”的最后决战却还是严重地摆在我们的面前,号召广大群众,团结一切力量,争取对于六次“围剿”最后决战的胜利,是我们当前的最严重任务。因此在讨论中间有同志说:“六次围剿已经完全粉碎了”。这种意见显然是不对的。又有同志说:“六次围剿我们仅在准备粉碎中。”这种意见也是不对的。照前一说法,是过分估计了自己的胜利,把苏维埃最后粉碎“围剿”的严重任务轻轻取消了,而实际上,蒋介石正在集中一切力量最后向我们大举进攻,所以这种估计是不正确的,并且是非常危险的。照后一说赞,是看不到几个月来红军从艰苦战争中已经给了敌人以相当严重的打击,已经取得了第一步胜利,这种胜利,同粉碎五次“围剿”的伟大胜利合起来,就成为我们彻底粉碎六次“围剿”坚固的基础。对于自己成绩估计不足,同样是很危险的。

有一个同志对于所谓的所谓人民革命政府,说他带有多少革命性不是完全的反革命,这种意见也是不对的。我在报告中已经指出;“人民革命政府”的出现,是反动统治阶级的一部分,为着挽救自己将死命运而搞的一个欺骗民众的新花样,他们感觉到苏维埃是他们的仇敌,而国民党这块招牌又太烂了,所以弄个什么“人民革命政府”,以第三条道路为号召,这样来欺骗民众,没有真正革命意义,现在的事实已经证明了。

关于婚姻问题,我在报告中曾经说到男女两方有一方坚决要求离婚,苏维埃政府应该准许离婚。但应该指出红军家属是例外。为了巩固红军战大的战斗决心,中央政府曾经规定,红军战士之妻要求离婚,必须取得其夫同意,只有在两年内还得不到丈夫音信,才可以由妻子一方提出离婚。关于结婚的年龄问题,不少同志主张降低,这种意见,我觉得是不妥当的。为了种族的与阶级的利益,结婚年龄不应该低于男二十,女十八以下。应该明白早婚是极大害处的。同志们!要耐烦一下子啊!(全场轰笑)以前在地主资产阶级统治之下,贫苦工农有到四五十岁还不能结婚的,为什么现在一两年都等不及呢?(全场又大笑)

以上是我的结论的一部分,但是结论主要部分还在下面。

※主要部分以“关心群众生活,注意工作方法”为题,收入《毛泽东选集》第一卷。

(原载《苏维埃中国》,一九五七年中国现代史资料编辑委员会翻印第235-305页)。

