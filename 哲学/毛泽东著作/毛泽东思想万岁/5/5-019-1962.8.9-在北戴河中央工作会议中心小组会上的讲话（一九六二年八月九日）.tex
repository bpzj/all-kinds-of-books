\section[在北戴河中央工作会议中心小组会上的讲话(一九六二年八月九日)]{在北戴河中央工作会议中心小组会上的讲话}
\datesubtitle{(一九六二年八月九日)}


今天单讲共产党垮得了垮不了的问题。共产党垮了谁来?反正两大党,我们垮了,国民党来。国民党干了二十三年,垮了台,我们还有几年。

农民本来已经发动起来,但是还有资产阶级、右派分子、地主、富农复辟的问题。还有南斯拉夫的方向。(有人插话:国民党在台湾搞了一个政纲,土地收为农民所有,但又保护地主)各地方、各部专搞那些具体问题,而对最普遍、最大量的方向问题不去搞。

单干势必引起两极分化,两年也不要,一年就要分化。

(李××同志揭露邓子恢的问题)派干部下去,而思想不“定一”,不讨论就走,这种办法不好。为什么不请邓子恢来?他不来,我们对台戏唱不成。建议中心小组再加五个人:邓子恢、王××、康生、吴××、胡×。

资本主义思想,几十年、几百年都存在,不说几千年,讲那么长吓人。社会主义才几十年,就搞得干干净净?历代都是如此。苏联到现在几十年,还有修正主义,为国际资本主义服务,实际是反革命。

《农村社会主义高潮》一书,有一段按语讲资产阶级消灭了,只有资本主义思想残余的影响,讲错了,要更正。

有困难,对集体经济是个考验,这是一种大考验,将来还要经受更重大的考验,苏联经过两次大战的大考验,走了许多曲折的道路,现在还出修正主义。我们的困难比苏联的困难更多。

全世界合作化,我们搞得最好。因为从全国说,土改比较彻底,但也有和平土改的地方。政权中混进了不少坏分子与马步芳分子。改变了生产资料所有制,不等于解决了意识的反映。社会主义改造消灭了剥削阶级的所有制,不等于政治上、思想上的斗争没有了。思想意识方面的影响是长期的。高级合作化、一九五六年社会主义改造,完成了消灭资产阶级的所有制,一九五七年提出思想政治革命,补充了不足。资产阶级是可以新生的,苏联就是这个情况。

苏联从一九二一年到一九二八年单干了近十年,没有出路,斯大林才提出搞集体化。一九三五年才取消各种购物券,他们的购物券并不比我们少。

找几个同志把苏联由困难到发展的过程,整理一个资料。这事由康生同志负责,搞一个经济资料。

动摇分子坚决闹单干,以后看形势不行又要求回来。最好不批准,让他们单干二、三年再说,他们早回来,对集体经济不会起积极作用。

要有分析,不要讲一片光明,也不能讲一片黑暗,一九六〇年以来,不讲一片光明了,只讲一片黑暗,或者大部分黑暗。思想混乱,于是提出任务:单干,全部或者大部单干。据说只有这样才能增产粮食,否则农业就没有办法。包产百分之四十到户,单干、集体两下竞赛,这实质上叫大部分单干。任务提得很明确,两极分化,贪污盗窃,投机倒把,讨小老婆,放高利贷,一边富裕,而军、烈、工、干四属,五保(户)这边就要贫困。

赫鲁晓夫还不敢公开解散集体农场。

(康生同志插话:现在的价格,低出高进,不利于集体经济。)

内务部一个司长,到凤城宣传安徽包产到户的经验。中央派下去的人常出毛病,要注意。中央下去的干部,要对下面有所帮助,不能瞎出主意,不能随便提出个人意见。政策只能中央制定,所有东西都应由中央批准,再特殊也不能自立政策。

思想上有了分歧,领导要有个态度,否则错误东西泛滥。反正有三个主义:封建主义、资本主义、社会主义。资本主义有买办阶级,美国资本主义农场,平均每个场有十六户,我们一个生产队二十多户。包产到户,大户还要分家,父母无人管饭,为天下中农谋福利。

河北胡开明,有这么一个人,“开明”,但就是个“胡”开明,是个副省长。听了批评“一片黑暗”的论调的传达,感到压力,你压了我那么久,从一九六〇年以来,讲两年多了,我也可以压你一下么。

有没有阶级斗争?广州有人说,听火车轰隆轰隆的声音,往南去的像是“走向光明”,“走向光明”,往北开的像是“没有希望”,“没有希望”。

有人发国难财,发国家困难之财,贪污盗窃。党内有这么一部分人,并不是共产主义,而是资本主义、封建主义。

每一个省都有那么一种地方,所谓后解放区,实际上是民主革命不彻底。

党员成分,有大量小资产阶级,有一部分富裕农民及其子弟,有一批知识分子。还有一批未改造过的坏人,实际上不是共产党。名为共产党,实为国民党。对这部分人的民主革命还不彻底,明显的贪污、腐化,这部分人好办。知识分子、地富子弟,有马克思主义化了的,有根本未化的,有的程度不好的。这些人对社会主义革命没有精神准备,我们没有来得及对他们进行教育。资产阶级知识分子,全部把帽子摘掉?资产阶级知识分子,阳过来,阴过去,阴魂未散,要作分析。

民主革命二十八年,在人民中宣传反帝、反封建,宣传力量比较集中,妇孺皆知,深入人心。社会主义才十年,去年提出对干部重新进行教育,是个重要任务。“六大”只说资产阶级不好,但是对资产阶级加了具体分析,反对的是官僚、买办资产阶级,对别的资产阶级就反得不多,三反五反搞了一下。没收国民党、大资本家、帝国主义的财产,这些拿到我们手上,就是社会主义性质,拿到别人手上是资本主义性质。一九五三、一九五四年搞合作社,开始搞社会主义。互助组、合作化、初级社、高级社,一直发展下来。真正社会主义革命是从一九五三年开始的。以后经过多次运动,社会主义建设与社会主义改造在全国展开。一九五八年已有些精神不对,中间有些工作有错娱,最主要的是高征购,瞎指挥,共产风,几个大办,安徽“三改”,引黄灌溉(本来是好的,不晓得盐碱化)。因此四个矛盾再加上一个矛盾,正确与错误的矛盾。高指标,高征购,这是认识上的错误,不是什么两条道路的问题。好人犯错误同走资本主义道路的完全不同,与混进来的及封建主义等更不相同。如基本建设多招了二千万人,没看准,农民就没有饭吃,就要浮肿,现在又减人。

有些同志一有风吹草动,就发生动摇,那是对社会主义革命没有精神准备,和没有马克思主义。没有思想准备,没有马列主义,一有风就顶不住。对这些人应让他们讲话,让他们讲出来,讲比不讲好,言者无罪,但我们要心中有数,行动要少数服从多数,要有领导。××同志的报告中说:“要正确处理单干,纪律处分,开除党籍……”。我看带头的可处分,绝大多数是教育问题,不是纪律处分,但不排除对带头搞分裂的纪律处分。

大家都分析一下原因。

这是无产阶级和富裕农民之间的矛盾。地主、富农不好讲话,富裕农民就不然,他们敢出来讲话。上层影响要估计到。有的地委、省委书记(如曾希圣),就要代表富裕农民。

要花几年功夫,对干部进行教育,把干部轮训搞好,办高级党校,中级党校,不然搞一辈子革命,却搞了资本主义,搞了修正主义,怎么行?

我们这政权包了很多人下来,也包了大批国民党下来,都是包下来的。

罗隆基说,我们现在采取的办法,都是治标的办法。治本的办法是不搞阶级斗争。我们要搞一万年的阶级斗争,不然,我们岂不变成国民党、修正主义分子了。

和平过渡,就是稳不过渡,永远不过渡。

我在大会上只出了个题目,还没有讲完,有的只露了一点意思,过两天可能顺成章。

三年解放战争,猛烈土地改革。土改后,对两种资本主义的改造很顺利。有的地区的民主革命还是不彻底,比如潘汉年、饶漱石,长期没发现。

修正主义的国内根源是资本主义残余,国外是屈从帝国主义的压迫,莫斯科宣言上这两句话是我加的。

一九五七年国际上有一点小风波,风乍起吹皱一池春水。六月刮起十二级台风,他们准备接管政府,我们来个反攻,所有学校的阵地都拿过来了。反右后,五八年算半年,五九年、六〇年大跃进。六一年开始搞十二条,六〇年搞工业七十条,农业六十条。

过去分田是农民跟地主打架,现在是农民跟农民打架,强劳动力压弱劳动力。

有这样一种农民,两方面都要争夺,地富要争夺,我们要争夺。

小资产阶级、农民有两重性,碰到困难就动摇,所以我们要争夺无产阶级领导权,就是共产党领导。农村的事,依靠贫农,还要争取中农,我们是按劳分配,但要照顾四属、五保。

二千万人呼之则来,挥之则去。不是共产党当权,哪个能办到。五八年十一月第一次郑州会议,提出的商业政策,没执行,按劳分配的政策,也不执行,不是促进农业,集体经济的发展,反而起了不利的影响。商业部应该改个名字,叫“破坏部”,同志们听了不高兴,我故意讲得厉害一点,以便引起注意。商业政策、办法,要从根本上研究。这几年兔、羊、鹅有发展,这是因为这几样东西不征购。打击集体,有利单干,这次无论如何得解决这个问题。

中央有事情总是同各省、市和各部商量,可是有些部就是不同中央商量,中央有些部作得好,像军事、外交,有些部门像计委、经委,还有财贸办、农业办等口子,问题总是不能解决。中央大权独揽,情况不清楚,怎样独揽?人吃了饭要革命,不一定要在一个部门闹革命,为什么不可以到别的部门或下面去革命呢?我是湖南人,在上海、广州、江西七、八年,陕北十三年。不一定在一个地区干,永远如此。中央、地方部门之间,干部交流,再给试一年,看能否解决,陈伯达同志说不能再给了。

财经各部委,从不做报告,事前不请示,事后不报告,独立王国,四时八节,强迫签字,上不联系中央,下不联系群众。

谢天谢地,最近组织部来了一个报告。

外国的事我们都晓得,甚至肯尼廸要干什么也晓得,但是北京各个部,谁晓得他们在干些什么?几个主要经济部门的情况,我就不知道。不知道,怎么出主意?据说各省也有这个问题。

