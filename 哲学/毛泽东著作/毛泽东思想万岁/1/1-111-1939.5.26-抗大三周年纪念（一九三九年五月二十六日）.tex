\section[抗大三周年纪念(一九三九年五月二十六日)]{抗大三周年纪念}
\datesubtitle{(一九三九年五月二十六日)}


抗大为什么全国闻名、全世界闻名,就是因为他比较其它的军事学校最革命最进步,最能为民族解放与社会解放斗争,到延安参观的人们,所以十分注意去看抗大,我想不外这个道理。

抗大的革命与进步,是因为他们的职员教员与课程是革命的进步的,又因为他们的学生是革命的进步的,没有这两方面的革命性进步性,抗大决不能成为全国与全世界称赞的抗大。

一部分人是反对抗大的,就是投降派与顽固派。这一点正是表明抗大是一个最革命、最进步的抗大,如若不然,他们就不会反对了。投降派顽固派人们之起劲的反对抗大,证明抗大的革命性进步性,增加了抗大的光荣。抗大之所以是个光荣的军事学校,不但因为大多数人拥护他称赞他,也还因为投降派顽固派人们在那里起劲地反对他,污蔑他。

抗大三年来有其贡献于国家民族社会的大成绩,这就是他教成了几万个青年有为与进步革命的学生,抗大今后必能继续有所贡献于国家民族与社会,因为他还要造就大批青年有为与进步革命的学生。昔日之黄埔,今日之抗大,是先后辉映,彼此竞美的。

抗大的教育方针是:坚定正确的政治方向,艰苦奋斗的工作作风,灵活机动的战略战术,这三者,是造成一个抗日的革命的军人所不可缺一的,抗大的职员、教员、学生、都是根据这三者去进行教育与从事学习的。

抗大在其逐年的改良进步中间,伴来了若干缺点,他发展了,但困难也来了。主要的是经费不足、教员不足、教材不足这几项,然而共产党领导的抗大,是不怕困难与一定能够克服困难的。在共产党面前无困难,就是因为他能克服困难。

抗大三周年后,改正其缺点使之更加进步,这是我的希望,也是全国全世界的希望。

抗大的教职员们、学生们努力啊!

