\section[在庆祝十月革命节干部晚会上的讲话(一九四三年十一月七日)]{在庆祝十月革命节干部晚会上的讲话}
\datesubtitle{(一九四三年十一月七日)}


今天延安开会庆祝苏联十月革命二十六周年。我们去年开庆祝会时,红军还正在为保卫斯大林格勒而奋斗,但在一年之中,红军的胜利就转变了战争的全局,由伏尔加河打到了德涅伯河,即将到来的冬季攻势,又将取得更大的胜利。没有红军,战争的局面是不能设想的。红军的胜利,关系于整个人类的命运,这个真理,早已是明明白白的了。同时,这一年中,英美法联军肃清了北非,西西里及意大利南部的敌人,空军采取了攻势,配合了红军的作战。在东方,中国的军队与英美的军队也打击了日本法西斯。

一星期前,英美苏三国在莫斯科所开的会议,胜利的完成了任务,这也是值得大大庆祝的。这次会议解决了许多军事政治问题,于十一月一日发表了三国联合公报,签订了几个有历史重要性的宣言。这次会议所讨论与解决的问题中,据联合公报所说,首先最重要的,是为着缩短战争时间,解决了确切的军事行动计划,关于此种行动,已经有所准备。由此,我们可以想到,不久的时间内,我们将看得见第二战场的实际开辟,从东西两面夹击希特勒而打败它,决定地解决欧洲问题。欧洲问题解决,就是折断了整个法西斯的脊骨与右手,剩下日本帝国主义这个左手,也就不难打断了。

莫斯科会议的决议中,有中国参加的四国宣言,在抗战到底的决心下,全面地规定了保障战后和平与安全的整个重要纲领,其中最重要的,是规定四国在战争中的合作,将使之在战争之后也继续实行,这样就打破了德日法西斯及各国内部的投降主义者离间英美苏中的阴谋,四国是更加紧密的团结起来了,战后的和平与安全有了保障了。宣言中又规定战后将组织包括一切大小国家在内而以主权平等为原则的新的国际联盟,作为保障和平安全的组织形式,我们可以想到,这种新的国际联盟,将和战前的老的国际联盟(虽然在其后期有苏联参加在内,但那时不容许苏联起重要作用)大不相同,它将是真能保障和平安全的联合机构。

莫斯科会议决定了对意大利的基本原则与具体政策。其基本原则是:“法西斯主义及其所有恶势力及其所产生的事物应予完全消灭,而予意大利人民以每一机会,建立以民主原则为基础的政府机构及其它机构。”其具体政策的第一条是:“意大利政府应容纳始终反对法西斯主义的意大利人员团体的代表,使其更加民主化”,第二条是:“意大利人民应完全恢复言论、宗教信仰、政治信仰、出版与公共集会的自由,意大利人民并得成立反法西斯的政治团体。”此外有几条是关于彻底消灭法西斯主义残余的,有一条是关于建立地方民主政府的。根据这些条文,消灭一切法西斯遗迹与建立有共产党参加的广泛的新民主主义意大利的方针是确定了。对意大利宣言的末尾还作了一个声明,这声明说:“本决议的内容,决非付诸实行以反对意大利人民最后选择其政治制度的权利”,这是一个原则性的声明,这就是说将来意大利人民选择民主制度,还是选择其它更进步的制度,他们是有权利的。莫斯科会议对意大利宣言是一个范例,将来将以此对待一切法西斯国家。这是完全区别于第一次大战的东西,历史上凡尔赛的帝国主义精神全扫除了,给了战败国人民以自由解放的光明道路,这是苏联人民英美人民及各国人民的伟大国际主义精神的集中表现。

莫斯科会议宣布了奥地利脱离奥国,同时责成奥国人民要为反希特勒战争而努力。这也是一个范例,一切被法西斯吞并的国家或地方,均获得解放。

在三国会议上宣布了罗斯福总统、丘吉尔首相与斯大林委员长三人的宣言,在这个宣言中,规定了彻底惩办法西斯凶手的原则,一切法西斯刽子手将不能饶免。同时,宣言号召:“目前尚未沾染无辜人民的血迹的人民,切勿和那些凶手们同流合污,盖三国必将追寻他们至天涯海角,务使归案法办”。彻以瓦解法西斯营垒,像这样的带着深刻革命意义的宣言,也是第一次世界大战所不能有的。

总之,三国会议的成功,确实是划时代的,它将深刻地影响到战争及战后的人类生活,人类解放的曙光已经看得见了。那些对中国人民前途及世界人民前途抱悲观见解的人们,那些抱投降思想,抱无原则妥协思想的人们,已经证明是完全错误的了。

我们庆祝苏联诞生的二十六周年,我们庆祝苏联红军的伟大胜利,我们庆祝斯大林元帅的英明领导,我们庆祝莫斯科三国会议的划时代的成就,我们庆祝中国参加了伟大的四国宣言!我们共产党人团结全中国一切爱国力量,打倒日本帝国主义,建立自由平等的新国家,以这种新国家的资格参加到新的国际合作与国际建设中,造就是我们的期望。

<p align="right">(1943年11月7日《解放日报》)</p>

