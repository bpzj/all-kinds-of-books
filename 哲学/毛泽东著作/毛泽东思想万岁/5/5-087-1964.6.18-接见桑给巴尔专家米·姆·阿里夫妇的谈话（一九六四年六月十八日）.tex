\section[接见桑给巴尔专家米·姆·阿里夫妇的谈话(一九六四年六月十八日)]{接见桑给巴尔专家米·姆·阿里夫妇的谈话}
\datesubtitle{(一九六四年六月十八日)}

\begin{duihua}

\item[\textbf{主席:}] 先照个像吧?\\(照相,然后坐下)

\item[\textbf{主席:}] 你们是非洲来的,桑给巴尔的?

\item[\textbf{阿里:}] 是的。

\item[\textbf{主席:}] (对江××)你讲的是什么?

\item[\textbf{江××:}] 英文。

\item[\textbf{主席:}] (对阿里)听说你在中国有几年了。

\item[\textbf{阿里:}] 是的,有四年了。

\item[\textbf{主席:}] 你为我们做了很多工作,帮助了中国人民搞广播事业。

\item[\textbf{金××:}] 他帮助我们办了斯瓦希里语广播,帮助我们培养了斯瓦希里语的干部。

\item[\textbf{主席:}] 好!

\item[\textbf{阿里:}] 北京电台也帮助了我们的人民,帮助他们了解世界情况。

\item[\textbf{主席:}] 听得到吗?

\item[\textbf{阿里:}] 听得很好,不仅桑给巴尔听得到,而且整个斯瓦希里语地区都听得很清楚。

\item[\textbf{主席:}] 有几个国家?

\item[\textbf{阿里:}] 有坦噶尼喀、肯尼亚、乌干达一部分,还有刚果。

\item[\textbf{主席:}] 哪个刚果?大刚果?

\item[\textbf{阿里:}] 几乎两个刚果。

\item[\textbf{主席:}] 啊,肯尼亚、坦噶尼喀、桑给巴尔。两个刚果。\\(阿里给主席点烟)

\item[\textbf{主席:}] 谢谢你!(用英文讲)\\你为什么要回去?

\item[\textbf{阿里:}] 这是国家的需要。

\item[\textbf{主席:}] 国家要你回去。你们这回来的联合共和国代表团里面,有你们的一个部长,你遇到过他吗?

\item[\textbf{阿里:}] 是巴布,我见过他。

\item[\textbf{主席:}] 我是头一次见到他。他很高。

\item[\textbf{阿里:}] 是的。

\item[\textbf{主席:}] 他现在在坦噶尼喀的首都工作重作?

\item[\textbf{阿里:}] 他没有去,将来可能去。

\item[\textbf{主席:}] 你们过去是朋友?

\item[\textbf{阿里:}] 是的。实际上是他把我介绍到中国来的。

\item[\textbf{主席:}] 你走了,就没有人啰。

\item[\textbf{阿里:}] 还有,还有六个桑给巴尔人工作。

\item[\textbf{主席:}] 你们两个人走了,还有四个?

\item[\textbf{阿里:}] 不,我是说还有六个。一个在电台,四个在外交出版局,一个在外语学院。

\item[\textbf{主席:}] 都是桑给巴尔的,有没有坦噶尼喀的?

\item[\textbf{阿里:}] 有一个,他在中国画版社工作,翻译斯瓦希里文。

\item[\textbf{主席:}] 你们那里的气候同我们这里的不同吧?

\item[\textbf{阿里:}] 是的,但我们已经习惯了。我们那里不下雪。

\item[\textbf{主席:}] 几个冬天了!

\item[\textbf{阿里:}] 可是我已经习惯了。

\item[\textbf{主席:}] 你们那里是南半球还是北半球,南纬度还是北纬度?

\item[\textbf{阿里:}] 实际上是在赤道。

\item[\textbf{主席:}] 在赤道上不是很热吗?

\item[\textbf{阿里:}] 是的,但是我们那儿只是一个小岛,不太热。

\item[\textbf{主席:}] 海洋性气候。

\item[\textbf{阿里:}] 是的。

\item[\textbf{主席:}] 你有中国朋友吗?

\item[\textbf{阿里:}] 很多,很多。

\item[\textbf{主席:}] 到外地去参观访问过没有?

\item[\textbf{阿里:}] 去过。一九六一年去了哈尔滨、广州、上海、杭州以及其他地方。\marginpar{\footnotesize 122}最近还荣幸到井冈山去了一趟。

\item[\textbf{主席:}] 哦,爬到山上去了。

\item[\textbf{阿里:}] 瑞金我也去了。

\item[\textbf{主席:}] 哦。

\item[\textbf{阿里:}] 我看到了第一次苏维埃政权的所在地。我们同井冈山的人民进行了交谈,同老干部,老区人民交谈。他们给我介绍了许多情况。

\item[\textbf{主席:}] 一九二七年到一九二八年我们在那里。到现在三十七年了!后来转移到瑞金去了。瑞金地区比较大,有几百万人口——不只是瑞金一个县,有几十个县,在那里打过很多胜仗。后来万里长征到了北方。一九三四年到一九三五年,到了陕西北部。甘肃也到过。也到过山西,过黄河到太原附近。山西靠近河北省。后来打日本,主要以延安为中心,在长江以北各省。后来发展到满洲。日本走了,蒋介石又来了,蒋介石打我们,我们就同他打,打了三年半,打垮了蒋介石大部分军队,百分之九十的军队。剩下的都跑到台湾去了。他历来是靠美国保护的。现在还是靠他们的(美国的)第七舰队。所以美国同我们还和不了。美帝国主义是很凶恶的帝国主义,也是最大的帝国主义,它对你们也有影响。

\item[\textbf{阿里:}] 是的,美国现在正想一切办法渗入桑给巴尔。

\item[\textbf{主席:}] 坦噶尼喀、桑给巴尔过去是英国的殖民地还是半殖民地?

\item[\textbf{阿里:}] 英国把桑给巴尔殖民地了,把它叫做“保护地”。

\item[\textbf{主席:}] 有个国王,叫苏丹。

\item[\textbf{阿里:}] 正是因为有个苏丹,所以叫“保护地”。

\item[\textbf{主席:}] 坦噶尼喀呢?

\item[\textbf{阿里:}] 叫做“领地”。

\item[\textbf{主席:}] 那就没有什么国王啰?是英国直接管辖?

\item[\textbf{阿里:}] 是的。

\item[\textbf{主席:}] 还有肯尼亚、乌干达呢?

\item[\textbf{阿里:}] 肯尼亚是殖民地,乌干达有国王,也叫做“保护地”。

\item[\textbf{主席:}] 南北罗得西亚呢?

\item[\textbf{阿里:}] 没有国王,是殖民地。

\item[\textbf{主席:}] 现在那里白人还不少啰?

\item[\textbf{阿里:}] 是的,在坦噶尼喀、肯尼亚有移民。住肯尼亚,因为气候比较凉,有许多移民。

\item[\textbf{主席:}] 有多少,听说有几十万。

\item[\textbf{阿里:}] 是的,有几十万。

\item[\textbf{主席:}] 听说有三十万。

\item[\textbf{阿里:}] 是的。

\item[\textbf{主席:}] 肯尼亚有多少人口?三百万?

\item[\textbf{阿里:}] 八百五十万。

\item[\textbf{主席:}] 有这么多?

\item[\textbf{阿里:}] 是的,坦噶尼喀的人口还要多,有九百万。

\item[\textbf{主席:}] 有一千万。\marginpar{\footnotesize 123}

\item[\textbf{阿里:}] 可能,我的数字是很久以前的人口统计。

\item[\textbf{主席:}] 你去过坦噶尼喀吗?

\item[\textbf{阿里:}] 只是路过。

\item[\textbf{主席:}] 到过肯尼亚吗?

\item[\textbf{阿里:}] 到乌干达去时路过。

\item[\textbf{主席:}] 现在回去走哪一条道?

\item[\textbf{阿里:}] 经过巴基斯坦、肯尼亚,也可能经过坦噶尼喀再到桑给巴尔。那儿有两条航线,一条直达线从肯尼亚到桑给巴尔,另一条从肯尼亚经过坦噶尼喀到桑给巴尔。

\item[\textbf{主席:}] 你的皮肤颜色看起来同坦噶尼喀人有点不同。

\item[\textbf{阿里:}] 是的,坦噶尼喀人更黑一些。

\item[\textbf{主席:}] 还有个马达加斯,那里的人的皮肤同非洲其他地方的人也不一样。

\item[\textbf{阿里:}] 嗯。

\item[\textbf{主席:}] 希望你们今后有机会再到中国来。

\item[\textbf{阿里:}] 中国已经是我们的家了。

\item[\textbf{主席:}] 来旅行,观光。就谈到这里吧?你还有什么问题没有?

\item[\textbf{阿里:}] 有,我想请问您几个问题。现在非洲人民的斗争,正在蓬勃发展,斗争越发展,我们对帝国主义的打击越大。但是这个斗争还要走很长一段道路。我虽然读了不少文件,但是还希望您谈谈您对非洲人民斗争的前景有些什么看法。

\item[\textbf{主席:}] 我对非洲的情况不太熟悉。但依我看,过去十年、十一年,从一九五二年埃及推翻法鲁克王朝起,非洲的变化是很大的。英国人,法国人,他们不甘心被打败的,进行了对苏伊士运动的攻击。另一个地方是阿尔及利亚,打了八年仗。阿尔及利亚以少数军队抵抗几十万法国军队。结果,法帝国主义失败了,阿尔及利亚胜利了。最近不久,你们国家也有变化,你们国家只有三十万人口,敢于起来推翻帝国主义的走狗,帝国主义也不敢怎么样。坦噶尼喀也独立了,英国军也走了。肯尼亚呢?

\item[\textbf{阿里:}] 肯尼亚也独立了,但兵变以后英国军队还在。

\item[\textbf{主席:}] 还有吗?听说非洲国家军队去了。

\item[\textbf{阿里:}] 这是在坦噶尼喀。

\item[\textbf{主席:}] 哦,在坦噶尼咯。

\item[\textbf{阿里:}] 在肯尼亚情况有点不同。肯尼亚与英国有协定,英在肯有基地。英国军队到年底才撤走。

\item[\textbf{主席:}] 他们最终是要走的。

\item[\textbf{阿里:}] 对!

\item[\textbf{主席:}] 在刚果,我说的是大刚果,有个卢蒙巴,是个民族英雄,被整死了,但斗争还在发展。在最近大半年,斗争有发展。在西南非洲,安哥拉,葡属殖民地,斗争也在进行。我对非洲虽然不熟,但我看来,根据过去十年的情况,可以说,在今后十年会有更大的变化。可能你们也是这样看。我们要从历史看,从发展看嘛!困难一些的是南非。那个地方有三百多万白人。他们是不愿意走的。那个地方要解放。恐怕时间要长一点。

亚洲、非洲、拉丁美洲,这三大洲,现在都有革命形势\marginpar{\footnotesize 124},这三大洲占世界人口的最大多数。这是事实。这是世界的大多数,欧洲、新西兰、澳洲和北美洲是少数。\\(阿里给主席敬烟)

\item[\textbf{阿里:}] 现在非洲没有共产党。您认为在非洲建立共产党的时机是否成熟?您对非洲的统一战线有什么看法?

\item[\textbf{主席:}] 建立共产党的问题,要看那个地方有没有产业工人。我看,在非洲有工业,很多国家有工业,有的是帝国主义建立起来的,有的是非洲人自己建立起来的,有矿山、铁路、公路以及其他工业。现在虽然没有共产党,但总有一天会有的。现在也不是没有共产党,阿尔及利亚有,摩洛哥有,南非有。阿尔及利亚共产党不是革命的党,是修正主义的党。修正主义的党,如阿尔及利亚的党,还不如民族解放阵线,因为他们进行民族解放战争,阿尔及利亚共产党是反对解放战争的,它听法国共产党的命令。阿尔及利亚共产党是反对我们的,反华的,阿尔及利亚政府,阿尔及利亚民族解放阵线是同我们合作的。不知道什么理由他们反对我们,有什么利害关系反对我们,我们不懂。

还有个例子,亚洲的伊拉克共产党,也是反华的,只注意反对中国共产党,而不注意他们自己面临着政变的危机。就是去年,来了一次政变,把卡塞姆杀了,把党的总书记也杀了。你知道这件事吗?

\item[\textbf{阿里:}] 知道,在报上读到过。

\item[\textbf{主席:}] 杀了许多共产党,杀了许多修正主义,也杀了许多进步人士。你说,为什么伊拉克共产党反对我们?

\item[\textbf{阿里:}] 听指挥棒。

\item[\textbf{主席:}] 听指挥棒,搞和平过渡。

再有一个是巴西,也不赞成我们,因为我们不同意和平过渡。几个月以前发生了政变,把总统赶跑了。修正主义党的领袖被判了八年徒刑。这个党的领袖到中国来过,叫普列斯特,是个很有名的共产党员,后来成了修正主义者。美帝国主义和它的走狗,不管你是修正主义,不是修正主义,他们是不管的。有九个中国人被捕,六个是贸易工作者,三个是新闻记者。

这就是说,修正主义不反对帝国主义,同帝国主义、反动派妥协。非洲工人阶级会得到教训的。可能出现一些修正主义的党,也可能出现一些马克思主义的党。

统一战线的问题,是反帝不反帝的问题。反帝的都要团结起来。在资产阶级民主革命的范畴来讲,就是看他反帝不反帝。至于建立真正的社会主义国家(不是名义上的),建立无产阶级领导的全民所有制、集体所有制的经济,那就是另外一件事了,这不仅是触动帝国主义的利益,而且要触动资产阶级的利益。譬如讲,现在,阿尔及利亚有可能走社会主义。老的一批人跟不上,包括临时政府的总理阿巴斯,贝勒卡塞姆,他们跟不上人家。

阶级斗争,真正的马克思列宁主义者是讲阶级斗争的,社会上有阶级斗争。我们同国民党有两次统一战线。一次是北伐,那是一九二七年。第二次是打日本的时候,第一次统一战线,北伐打到长江流域,国民党得到了政权,就反对我们,我们只好同它打,上了井冈山,后来到了瑞金。

后来,日本人打进来了。蒋介石感觉得再要同我们打下去不行了。\marginpar{\footnotesize 125}就建立了第二次统一战线。这次统一战线有八年之久。一方面国民党同共产党团结,反对日本,另一方面,国民党又每天反对我们。我们怎么办?这一边有日本,那一边又有国民党。所以我们采取了又团结又斗争的政策,以团结为主。这样,同国民党维持了八年。日本投降了,国民党打我们,统一战线就破裂了。破了就破了嘛,我们打胜了,他们打败了。我们没有大城市,没有外国的援助,我们军队人少,没有空军,没有海军,没有飞机,没有大炮,只有轻武器,不是我们自己造的,是我们缴来的。

这样岂不是没有统一战线了吗?把他赶到台湾去了,但是,还有统一战线。其实,我们的统一战线更广泛了。我们中国有八个民主党派。国民党在的时候,知识分子,大学教授,中小学教员与我们接触不很广泛。在解放以后,他们都不走。我们把他们都团结起来。在北京的大学教授,如北京大学、清华大学的教授,和在上海、广州,大学教授都不走,他们感觉到跟着国民党没有前途。

基本的统一战线是同工人、农民的统一战线。也是在解放后,工人和农民的联盟才在全国范围内得到实现。

国民党代表大资产阶级、买办阶级和封建地主阶级。我是讲它的后期。国民党曾经是代表民族资产阶级和广大人民的。那时,以孙中山先生为代表,是中国唯一的、最进步的政党。那时还没有共产党。共产党是后来才有的,一九二一年才有共产党。后来共产党和国民党建立了第一次的统一战线。

后来,国民党反对共产党。打了十年仗。它变成了帝国主义、美国、英国帝国主义的代理人。为什么它变成大资产阶级、大地主阶级的代理人,我们还可以同它形成第二条统一战线呢?因为日本打进来了。

日本侵入东北的时候,国民党还打我们。只是在日本打进关内,向大陆进攻时,它感到同共产党不讲和不行了,所以才形成第二次共产党和国民党的统一战线。

蒋介石是站在美、英、法一边的,反对日本、希特勒、墨索里尼,一派帝国主义,打另一派帝国主义。德、意、日三个国家变成了战败国。要看什么条件,那时美、英、法,我们也可以同它们合作。在战后就发生了变化,美国想控制世界。日本变成了战败国,意大利、德国变成了战败国。英法削弱了。非洲为什么起来呢?就是因为帝国主义削弱了,英法削弱了。

大概非洲……,对英、美、比利时、葡萄牙、西班牙,对广大人民来说,都不会有什么好感的。为什么我们同你们非洲人、黑人讲得来呢?我们有共同之点。

\item[\textbf{阿里:}] 我们非洲人与帝国主义进行了长期的斗争。我们看到中国解放了。中国人民的斗争给了我们很大的鼓舞。在中国解放之后,我们更加了解中国。

我们的斗争不断发展,因为中国给了我们许多经验,中国给了非洲人民很大的支持和鼓励,我们非常感谢。中国发表了许多支持我们的声明。近几年来,我们能到中国来,参观了许多地方,对我们很有帮助。

苏联修正主义告诉我们要和平共处,裁军,说这是我们的主要任务,说要把裁军省下来的钱来援助我们。但是,我们的斗争要靠自己的力量。

\item[\textbf{主席:}] 对!

\item[\textbf{阿里:}] 在这方面,修正主义越来越同帝国主义勾结在一起。在您看来,他们勾结到什么程度?\marginpar{\footnotesize 126}

\item[\textbf{主席:}] 可能进一步勾结。帝国主义同修正主义又勾结,又有矛盾。修正主义同修正主义也有矛盾。修正主义有几十个党,但并不是很团结的。帝国主义之间也不是很团结的。你看,法国同英国就不是很团结的。日本垄断资本家,日本政府首先打了美国的珍珠港,以后占领了菲律宾、越南、泰国、马来亚、印度尼西亚,打到印度的东部,占领了大半个中国,朝鲜就不再讲啰,本来就是它的殖民地。现在这些地方都独立了,有的还在美国的控制之下。在美国控制下的有南朝鲜、南越、菲律宾。日本也是在美国半控制之下。你说,日本,不要说人民,就是大资产阶级,他们会舒服吗?我不相信。我不相信美国帝国主义同日本垄断资产阶级没有矛盾。

我们说有两个中间地带。亚洲、非洲、拉丁美洲是第一个中间地带。欧洲、加拿大、澳洲新西兰、日本是第二个中间地带。日本的垄断资本家受美国欺侮,我们反对欺侮。很有一些人听得进去中间地带的说法。

这个话不是现在才讲的,是一九四六年就讲了。那时候没分第一、第二,只讲了中间地带,讲苏联同美国之间是中间地带,包括中国在内。一九四六、一九五六、一九六四,……十八年了,话讲了十八年了。那时我们在延安,是同美国记者讲的,她叫斯特朗。

\item[\textbf{阿里:}] 我认识她。

\item[\textbf{主席:}] 她七十几岁了!那时美国代替了德、意、日,想控制世界,它的目的是侵略中间地带,不是打苏联。反苏是个口号,是烟幕。与反华的性质一样,其目的是要整中间地带,以反华为口号。

\item[\textbf{阿里:}] 我耽心主席的时间。请允许我表达我的感情。到中国来以后就一直盼望着有这一天,今天终于实现了。我的感情是无法用言语来表达的。

\item[\textbf{主席:}] 你看过马克思列宁主义的书籍没有?

\item[\textbf{阿里:}] 看过,也看过您的著作。

\item[\textbf{主席:}] 我是从马克思、列宁那里学来的。

\item[\textbf{阿里:}] 您发展了马克思列宁主义。您的著作比马克思、恩格斯、列宁的更加易懂。

\item[\textbf{主席:}] 比较通俗一些。

\item[\textbf{阿里:}] 这是我的感觉,您的著作写得很通俗。

\item[\textbf{主席:}] 我也没有多少著作。

\item[\textbf{阿里:}] 不,很多。

\item[\textbf{主席:}] 好,就说到这里吧!

\item[\textbf{阿里、阿里夫人:}] 再见!

\item[\textbf{主席:}] 再见!\marginpar{\footnotesize 127}

\end{duihua}