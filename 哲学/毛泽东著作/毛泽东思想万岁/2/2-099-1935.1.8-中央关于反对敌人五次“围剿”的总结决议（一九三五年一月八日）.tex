\section[中央关于反对敌人五次“围剿”的总结决议(一九三五年一月八日)]{中央关于反对敌人五次“围剿”的总结决议}
\datesubtitle{(一九三五年一月八日)}


(九)在持久战与速决战问题上,单纯防御路线的领导者的了解也是错误的。必须明白中国国内战争不是二个短时期的战争,而是长期的持久的战争,苏维埃革命,就在不断的粉碎敌人的“围剿”中发展与巩固起来的。因此在有利的条件之下,我们完全应该从防御转入反攻与进攻,消灭敌人,粉碎“围剿”(如一二三四次战争及五次战争广昌战役以前)。在不利的条件下,我们可以暂时的退却,以保持我们的有生力量,在另一有利的条件下转入反攻与进攻(如五次战争广昌战役以后)。这是第一个基本原则。但同时必须了解另一个原则,即为了进行长期的持久战,对于海一次“围剿”与每一个战役,必须极力争取战局的速决。因为在现时,敌我力量的对比上,对于一次“围剿”与每一个战役釆取持久战的方针,对于我们是极端不利的。当着敌人以持久战来对付我们的时候(如五次“围剿”)我们必须运用正确的战略方针,打破敌人这种计划,在我们可以支持的时间之内取得决定的胜利,以粉碎敌人的“围剿”。拿我们的人力物力及军火补充的数量同国民党所有的去比较(即所谓同敌人拚消耗,见××同志红星报的文章),这种持久战的了解是根本错误的。在这些方面现时我们正处在绝对的劣势,这种数目字的比较只能证明相反的结论,即持久战对我们是没有胜利前途的。

正因为要进行长期国内战争的持久战,同时对每一次“围剿”与每一战役却要进行速决战,所以我们特别要谨慎的决定我们的战略战役计划。五次战争中单纯防御战略是根本错误的,在这种错误战略之下,进行许多拼命主义的战斗(如毛丁山、三溪坊、平疗、广昌等战役)同样是错误的,红军一定要避免那种没有胜利把握的战斗。即使作战的决定在当时是正确的,但当形势变化不利于我们时,我们立即应拒绝这种战斗。要玩弄暴动是极大的罪恶,玩弄战斗同样是非恶。

正因为要进行战争的持久战与战役的速决战,所以我们一定要给予红军以必须的休养兵力与教育训练的时间,这是争取战争胜利的必要条件。以为五次战争中没有休息训练的可能的说法是不对的,那只是单纯防御短促突击主义者必然的结论。以为红军行动积极化,便是使他经常的不停止的打仗而不必注重休息训练,也是不对的,须知没有必要的休息和训练,就不能好好的打胜仗。红军的编制,一定要适合现时国内战争的环境。主力还未充实就去建立许多新的师团单位是不对的。应该于充实主力之后,再去建立新单位。拿全无教育训练又无战斗经验的新兵师团去单独作战是不对的,应该使新兵师团中有老军骨干,尽可能在初期使他们在老的兵团指挥之下,训练出战斗经验来。那种不必要的笨重的与上重下轻的组织与装备是不对的,应该是尽可能的轻装,必须充实连队与加强师以下的领导。

正因为要进行战争的持久战与战役的速决战,必须反对那种把保持有生力量与保卫苏区互相对立起来的理论。为了进行胜利的战斗,红军的英勇牺牲是完全必要的,这是阶级武装的特质,是革命战争胜利的基础,这种牺牲换得了胜利,这种牺牲是有代价的。但这不能适用于无代价的拚命主义的战斗,须知只有保持有生力量,我们才能真正的保卫苏区。没有坚强的红军,苏区即无法保存。有了坚强的红军,即使苏区暂时遭到部分的损失,也终究能够恢复,并且新的苏区也只有依靠红红军才能创造出来。

在战争持久战的原则之下,要反对当敌人的“围剿”被我们用反攻战斗粉碎了之后可能发生两种错误倾向:一种是对于疲劳情绪与过高估计敌人力量所产生的保守主义。这种保守主义使我们懈怠消极,使我们停顿不动,使我们不能由反攻转入进攻,消灭更多敌人,发展更大苏区,扩大红军力量,使我们不能在敌人新的“围剿”到来之前取得粉碎新的“围剿”的充分条件。另一种是由于对自己过分估计与对敌人力量估计不足所产生的冒险主义。这种冒险主义使我们进攻得不到胜利(如无把握的及在当时无必要的进攻中心城市等),甚至于使反攻中已经得到的胜利归于消灭或抛弃,使红军有生力量过分牺牲,使扩大红军扩大苏区争取战略地区的发展与巩固的任务放弃不顾,这同样使我们不能在敌人新的“围剿”到来之前取得继续粉碎它的充分条件。因此,反对这两种错误倾向,是党在战略的进攻问题上即在敌人两次“围剿”之间的严重任务。

在战役速决战的原则下,要反对根源于恐慌情绪的仓猝应战,或对战路上初战的不慎重,或企图先发制敌一成不胜就认为没有办法,或借口速决战而不作充分准备,即对于敌人的“围剿”不作必要的与尽可能支持的时期内的一切准备,等筹机会主义的倾向。速决战是要求具备一切必要条件(战略的优胜,战役领导的正确,运动战,不失时机,集中兵力等等)去消灭敌人部队,只有消灭了敌人的部队才能使战局速决,才能使敌人的进攻与“围剿”归于粉碎。

(十)利用反革命内部的每一冲突,从积极方面扩大他们的内部的裂痕,使我们利于转入反攻与进攻,是我们粉碎敌人“围剿”重要战略之一。福建十九路军事变是粉碎敌人五次“围剿”的重要关键,党中央当时釆取了利用国民党内部这一矛盾的正确的政治路线,同十九路军订立了停战协定,来推动十九路军去反对日本帝国主义与蒋介石。然而当时的××同志等却在左的空谈之下,在战略上采取了相反的方针,根本不了解在政治上军事上同时利用十九路军事变是粉碎五次“围剿”的重要关键之一。相反的以为红军继续在东线行动打击进攻十九路军的蒋介石部的侧后方是等于帮助了十九路军,因此把红军主力西调劳而无功的攻击永丰地域的堡垒。失去了这一宝贵的机会,根本不了解十九路军人民政府当时的存在对于我们是有利益的,在军事上突击蒋介石的侧后方以直接配合十九路军的行动,这正是为了我们自己的利益,为了粉碎五次“围剿”。这并不是因为十九路军是革命的军队,相反的这不过是反革命内部的一个派别,这个派别企图用更多的欺骗与武断宣传甚至社会主义之类的名词来维持地主资产阶级的统治,只有我们在实际行动中表现给在十九路军欺骗下的工农士兵群众看,我们帮助任何别派反日反将的斗争我们才能更容易揭破十九路军军阀的欺骗,在共同反日反蒋的战争中,争取他们到我们方面来。只有我们军事上釆取与十九路军直接配合的方针,才能使我们在当时这一重要关键上不失去消灭蒋介石主力的机会,这种有利条件,是过去历次战斗中所没有的。然而在我们军事上没有去利用,这对于单纯防御路线的领导者原是不足为怪的,因为他们的目的,原来不过为了抵御敌人的前进,至于利用敌人内部的矛盾冲突使自己转入反攻与进攻,在他们看来是冒险的行动。

(十一)在战略转变与实行突围的问题上,同样是犯了原则性的错误。首先应该说的:当我们看到在中央苏区继续在内线作战取得决定的胜利已经极少可能以至最后完全没有可能时(一九三四年五月至七月间,即广昌战役以后),我们应毫不迟疑的转变我们的战略方针,实行战略上的退却,以保持我们的主力红军的有生力量,在广大堡垒要地区,寻求有利时机,转入反攻,粉碎“围剿”,创造新苏区,以保卫老苏区。国际六月二十五日来电曾经这样的指出:“动员新的武装力量,这在中央苏区并未枯竭,红军各部队的抵抗力及后方环境等,亦未足使我们惊慌失措”。甚至说到对苏区主力红军退出的事情,这唯一的只是为了保存活的力量,以免遭受敌人可能的打击。在讨论国际十三次全会和五中全会的议案时,关于斗争的前途及目前国际的情形以及红军灵活的策略,首先是趋于保存活的力量及在新的条件下来巩固和扩大自己,以待时机进行广大的进攻,以反对帝国主义、国民党。在这个重要关节上,我们的战略方针显然也是错误的。在“五、六、七三个月战略计划”上,根本没有从小这一问题。在“八、九、十三个月战略计划”上虽是提出了这一问题,而且开始了退出苏区的直接准备,然而新的计划的基本原则依然同当时应取的战略方针相反,“用一切力量继续捍卫苏区来求得战役上大的胜利”,“发展游击战争,加强补助方向的活动,来求得战略上的情况的变更”,这些依然是新计划基本原则的第一部第二条。关于有生力量的保存问题,完全忽视。而这正是决定退出苏区的战略方针的基础。这一战役时机上的错误,再加上阵地战的发扬,给了红军以很大的损害。这种一方面预备突围,一方面又“用一切力量继续捍卫中央苏区”的矛盾态度,正是单纯防御路线的领导者到了转变关头必然的惊慌失措的表现。

其次,更加重要的,就是我们突围的行动,在华夫同志的心目中,基本上不是坚决的与战斗的,而是一种惊慌失措的逃跑的以及搬家式的行动。正因为如此,所以这种巨大的转变不但没有依照国际指示,在干部中与红军指战员中进行解释的工作。而且甚至在政治局的会议上也没有提出讨论。把数百万人的群众行动的政治目标,认为不是重要的问题。在主力红军方面,从苏区转移到白区去,从阵地战场转移到起动战场去,不给以必要的休养兵力与整顿训练,而只是仓猝的出动。关于为什么退出中央苏区?当前任务怎样?到何处去?等基本的任务与方向问题,始终秘而不宣。因此在军事上,特别在政治上,不能提高红军战的热情与积极性,这不能不是严重错误。庞大的军委纵队及各军团后方部的组织,使行军作战受到极大的困难,使所有的战斗部队,都成了掩护队,使行动迟缓,失去到达原定地区的先机。这是根本忘记了红军的战略转变将遇到敌人严重的反对,忘记了红军在长途运动中,将要同所有追堵截击的敌人作许多艰苦的决斗,才能达到自己的目的。所有这些军事上政治上组织上的错误,特别战略方针不放在争取于必要与有利时机同敌人决战的原则上,就使得自己差不多经常处于被动地位,经常遭受敌人打击,而不能有力的打击敌人。就使得三个月的突围战役,差不多处处成为掩护战,而没有主动的放手的攻击战。就使得口头上虽然常说“备战”,而实际上除掩护战而外,却经常是“避战”。就使得红军士气不能发扬,过分疲劳,得不到片刻的休息,因而减员到空前的程度。就使得“反攻”的正确口号在实际上变成了××同志等的避战主义的掩盖物,而不准备于必要时与有利时机争取真正反攻的胜利。就使得以红军战略转变,迫使敌人转变其进攻中央苏区的整个计划,以保卫中央苏区,以粉碎五次“围剿”,以建立湖南的根据地,乃至高度保持红军有生力量的基本任务,都不能完成。所以这些,都是基本的战略方针釆取了避战主义的必然结果。这种战略避战主义是从一种错误观点出发,即是说红军一定要达到了指定地区(湘西),放下了行李,然后才举行反攻消灭敌人,否则是不可能的。对追击敌人(如周、薛两纵队),就在他们分离时与疲弊时也是不敢作战的。而这种错误观点的泉源,则在于不明了当前的环境是不允许我们这样简单地轻巧地尽情直遂地干的,在于对追击敌人的力量的过分估计。殊不知这种简单的轻巧的与径直的干法,在短短的环境不严重的与小部队的行动,或者是可能的,而在数千里的五次“围剿”环境中的主力红军的巨大战略的转移则是不可能的。对不必要的与敌人无隙可乘的那种战斗,是应该避免的,而对于必要的与敌人有隙可乘的战斗,则是不应该避免的。此次突围行动,没有完成自己的任务,其主要原因正在这里。这一原则上的错误,一直发展到突围战役的最后阶段,当红军到了湘黔边境,在当时不利于我的情况下,却还是机械的要向二六军团地区前进,而不知按照已经变化的情况来改变自己的行动与方针。红军到了乌江地域,又不知按照新的情况变化,提出在川黔边转入反转消灭蒋介石追击部队的任务,而只是看见消灭小部黔敌以及消灭所谓土匪的任务。虽则最后两次错误因政治局大多数同志坚决的反对而纠正了,而在华夫同志等则适足表现其战略上一贯的机会主义的倾向。

单纯防御路线发展的前途:或者是不顾一切的拚命主义,或者是逃跑主义,此外决不能有别的东西。

(十二)政治局扩大会议认为一切事实证明我们在军事上的单纯防御路线,是我们不能粉碎敌人五次“围剿”的主要原因。一切企图拿党的正确路线来为军事上的错误路线做辩护(如××同志的报告,华夫同志的发言)是劳而无功的。

政治局扩大会议认为这种军事上的单纯防御路线,是一种具体的右倾机会主义的表现。他的来源是由于对敌人的力量估计不足,是由于对客现的困难特别是持久战保垒主义的困难有了过分的估计,是由于对自己主观的力量特别是苏区与红军的力量估计不足,是由于对于中国革命战争的特点不了解。因此政治局扩大会议认为反对军事上的单纯防御路线的斗争,是由于党内具体的右倾机会主义的斗争,这种斗争在全党内应该开展与深入下去。一切把这一斗争转变为无原则的个人纠纷的企图,必须受到严厉的打击。

(十二)此外政治局扩大会议认为××同志特别是华夫同志的领导方式是极端的恶劣,军委的一切工作为华夫同志一个人包办,把军委的集体领导完全取消,惩办主义有了极大的发展,自我批评丝毫没有,对军事上一切不同意见不但完全忽视,而且采取各种压制的方法,下层指挥员的机断专行与创造性是被扼杀了。在转变战略战术的名义之下,把过去革命战斗中许多宝贵经验与教训完全拋弃,并谓之为“游击主义”,虽是军委内部大多数同志曾经不止一次提出了正确的意见,而且曾经发生过许多剧烈的争论,然而这对于华夫同志与××同志是徒然的。一切这些,造成了军委内部极不经常的现象。

同时政治局更认为过去书记处与政治局自己对于军委领导是非常不够。书记处与政治局最大部分的注意力是集中在扩大红军与保障红军的物质供给方面,因此在这些方面,得到了空前伟大的成绩,然而对于战略战术方面则极少注意,而把这一责任放在极少数的同志身上,首先是××同志与华夫同志。我们没有清楚的了解,战争的指挥问题,关系战争胜负的全局。战争指挥的错误,可以使最好的后方工作的成绩化为乌有。政治局对于这一问题上所犯的错误是自己应该承认的,书记处的所有同志,在这方面应该负更多的责任,因为有些重要的决定或战略计划是经过书记处批准的。

然而政治局扩大会议特别指出××同志在这方面的严重错误,他代表中央领导军委工作,他对华夫同志在作战指挥上所犯的路线上的错误以及军委内部不正常的现象,不但没有及时的去纠正,而且积极的拥护了助长了这种错误的发展。政治局扩大会议认为××同志在这方面应负主要的责任,而××同志在他的结论中对于绝大多数同志的批评与自己的错误是没有承认的。必须指出,这种错误对于××同志不是整个政治路线的错误,而是部分的严重的政治错误。但这一错误如果坚持下去,发展下去,则必然走到整个政治路线的错误。

政治局扩大会议认为为了粉碎敌人新的围攻,创造新苏区,必须彻底纠正过去军事领导上所犯的错误,并改善军委领导方式。

(十四)最后,政治局扩大会认为:虽是由于我们过去军事上的错误领导使我们没有能够在中央苏区粉碎五次“围剿”,使我们主力红军不能不退出苏区并遭受到部分的损失,然而我们英勇的红军主力依然存在着,我们有着优良的群众条件,我们有着党的正确领导,我们有着物质上地形上的比较良好的地方,我们有着全国广大的群众拥护,与红四方面军和二二六军团的胜利配合,再加上正确的作战指挥,我们相信,这些困难,在我们全体同志与红军指战员努力之下,是可以克服的。同时,敌人方面的困难是大大的增加了,我们活动地区远远的离开了南京政府反革命的根据地,蒋介石几年经营的堡垒地带的依靠是没有了,军阀内部的矛盾与不统一有了进一步的增加,我们主要的敌人蒋介石的主力在五次“围剿”中是削弱了,尤其是帝国主义瓜分中国与国民党的卖国政策,全国国民经济的空前崩溃,使全同民众更清楚的看到只有苏维埃才能救中国,而更加同情与拥护苏维埃革命运动以至直接为苏维埃政权而斗争。这些都是我们粉碎敌人新的围攻,创造新的苏区根据地,发扬全国苏维埃运动的有利条件。必须指出,目前的环境在党与红军面前提出了严重任务:这就是因为帝国主义与反革命国民党军阀在任何时候都不会放松我们。我们在敌人新的围攻的前面,中央红军现在是在云南贵州地区,这里没有现存的苏区而要我们重新去创造,我们的胜利要在自己艰苦奋斗中取得,新苏区的创造,不经过血战是不可以成功的。当前的中心问题,是怎样战胜川、滇、黔蒋这些敌人的军队。为了战胜这些敌人,红军的行动,须有高度的机动性,革命战争的基本原则是确定了,完成作战任务则必须灵活的使用这些原则。红军运动的特长,在五次战斗中是被长期的阵地战相当的减弱了,而在目前正要求红军各级指挥员具有高度的运动战战术。因此从阵地战战术(短促突击)到运动战战术的坚决的迅速的转变,是严重的工作的。对战斗员尤其是新战士,则须进行必要的技术教育。在政治工作上,一切须适应目前运动战的需要,以保证每一个战斗任务的完成。红军更要从作战中修养与整理自己,并大量的扩大自己。严肃自己的纪律,红军对广大劳苦工农群众的联系,必须更加密切与打成一片,极大约加强对地方居民的工作,红军应该是苏维埃的宣传者与组织者。目前的环境要求党与红军的领导者用一切努力,具体的确实的解决这些基本的问题。

白区党的工作,必须建立与加强。对白区群众斗争的领导方式,必须有彻底的转变。瓦解白军工作必须真正开始。广泛的发展游击战争,是党目前最中心的任务之一。在中央苏区湘赣湘鄂赣苏区与闽浙赣苏区,党必须坚持对游击战争的领导,转变过去的工作方式,来适合于新的环境。最后,同二六军团及四方面军必须取得更密切的关系并加强对于他们的领导,以求得全国红军的一致行动与互相配合。

政治局扩大会议相信放在我们前面的这些严重的任务,我们是能够完成的。完成这些任务是以后革命战争的新的胜利的保证。新的革命战争的胜利,将使我们中央红军在云贵川三省广大地区中创造出新的苏区根据地,将使我们恢复老苏区,将使全国各地的红军与苏区打成联系的一片,并将使全国工农群众的斗争转到胜利的大革命。

政治局扩大会议相信,中国苏维埃革命有他雄厚的历史泉源,他是不能消灭的,他是不能被敌人战胜的。中央苏区湘赣苏区与闽浙赣苏区的变为游击区,不过是整个苏维埃革命运动中部分的挫折。这种挫折丝毫也不足以使我们对于中国苏维埃革命的前进表示张皇失措,实际上帝国主义国民党就是想暂时停止苏维埃革命运动的发展也是不可能的。二六军团与四方面军的胜利,中央红军在云贵川三省内的活跃,以及全国工农群众的革命斗争,证明整个中国苏维埃革命正在前进中。

政治局扩大会议指出过去党在军事领导上的错误,对于我党的整个路线说来不过是个部分的错误。这种错误不足以便我们悲观失望。党勇敢的揭发了这种错误,从错误中教育了自己,学习了如何更好的领导革命战斗到彻底的胜利。党在揭发了这种错误之后,不是削弱而是加强了。

政治局扩大会议号召全体同志,以布尔塞维克的坚定性,反对一切张皇失措悲观失望的右倾机会主义,首先反对单纯防御路线。政治局扩大会议更号召全党同志像一个人一样团结在中央的周围,为党中央的总路线奋斗到底,胜利必然是我们的。

