\section[几段插话(一九六四年一月)]{几段插话}
\datesubtitle{(一九六四年一月)}


(谈到工业问题,要建立“托拉斯”问题时)

目前这种按行政方法管理经济的方法,不好,要改。比如说,企业里用了那么多的人,干什么!人是要吃饭的,要消耗的,不像孙猴子吃铁砂,拉铁屎。用那么多人,就是不按经济法则办事。

生产出来的物质,必须按合同收购。商业部门说是因为计划变了,不收购;是谁变的计划?国家变的,就由国家收购。总之,不能积压在工厂里头。

(谈到企业管理不好的原因时)

解放军一道命令,可以通到底,行得通。说解放军所以搞得好,是由于共产党领导。那经济也是共产党领导的呀,为什么搞得四分五裂?抢钱(利润分成)、抢物质(物质分成)、发生冲突(闹关系),多年不得解决?商业为什么不能按经济渠道经营管理,为什么只能按行政设置机构?打破省、专、县界嘛!就是要按经济渠道办事。企业跟军队一样,一通到底嘛!党委管思想、管政治、管仲裁(冲突)、管人、实行监察嘛!

(谈到专业化和协作、制造主机和辅机的关系时)

主机和辅机的矛盾,是对立物的统一。资本家办一个工厂,管工厂的资本家总是少数,做工的工人总是多数。少数和多数的矛盾,也是对立物的统一。

(讲到树立标兵,领导上不要埋没英雄时)

我说高官厚禄、固步自封有的是,比如有的部的报告就说他们也有多少模范、典型、标兵,但多年没有发现,这不是被固步自封的官僚主义所压住了?!现在被压住的,没有发现的还很多。要改变这种情况,抓活人活事。

(讲到报纸应该怎样宣传时)

文汇报、光明日报办得比较好,有些议论,也有科学研究、哲学、历史研究等方面的文章。人民日报单纯些。(它担负人家不担负的任务。)报导国际消息,一礼拜几次也就可以了。要写点新鲜事物,活人活事。加上一版,专门报导学习解放军,学习石油部。写“活”哲学,不要写“死”哲学。现在写文章,引语太多,看了就心烦,少引一些可以。我写文章很少引马、恩、列、斯。要写活的哲学。许多老粗懂得哲学,最近解放军发现一个炊事员写的文章说,他从前烧饭每顿二斤四两煤炭,后来经过调查研究,掌握了煤炭的客观规律,每顿只用六两煤炭。

(讲到石油部的经验时)

过去封建皇帝时代,还可以据理力争。我们解放军有一条,真理在谁手里,就服从谁,在伙夫手里就听伙夫的,在班长手里就服从班长的。真理不是谁的官大、官小来决定。

(在和日本人谈到对垄断资本反美的态度时)

(一)日本垄断资本现在有变化,八番钢铁托拉斯,富士钢铁托拉斯,东海渔业托拉斯,反美反得很紧张。过去只纺织业来反,现在垄断资本家也反了。这一件事,对日本共产党来讲,是个新问题。过去讲得太死了,切记不要讲死。我讲了法国的经验,戴高乐掌握了民族独立的旗帜,掌握了反美的旗帜,而法共则去欢迎艾森豪威尔,为肯尼廸流泪,所以戴高乐上台,倒不了。苏加诺也是大资本家,但他掌握了反美和民族独立的旗帜,所以他能维持。而印尼共产党就把这些接过来。希特勒上台后,宣布废除凡尔赛条约、收复失地,抓民族独立的旗帜,德国共产党台尔曼未表态,失败了。日本垄断资产阶级是两个拳头作战,一手反美,一手反共。你们要给他放松一头。日本垄断资本家提出反美,要美国撤走基地,要反控制,你们应该把这些接过来。日本目前的形势是民族矛盾超过了阶级矛盾。就是要跟垄断资本家在反美问题上搞统一战线。我们也给四大家族(中国垄断资本家)搞过统一战线。这样做,可能麻痹了工人。但我们也要想想工人的心理,资本家反对美国控制,工人有事做,工人是同意的。我们跟蒋介石的经验也是。日本侵略中国,民族矛盾上升为主要矛盾。我们同蒋介石是采取两手:又团结,又斗争,斗争也是为了团结,以斗争求团结。反共高潮只三次,一打,他退回去了,我们还是联合。不联合,哪能发展这样快,这样大,从二万五千人发展到一百二十万人?如果日本不投降,这一政策还要继续下去。我们在这中间发展,我们要趁这个空子发展嘛!

(二)为什么苏联出了修正主义?这个问题是带普遍性的,许多人脑子里有这个问题。解答这一问题,还是要用阶级、阶级分析。这是从斯大林时候就包下来的。联共党史写了,宪法也写了,只提工人、农民、知识分子全民一致,不提工人、农民、知识分子以外的不一致,不提还有资本主义分子,还有未改造的知识分子;此外,也不提还会产生新的资产阶级分子,高薪阶层,工人贵族。问题不在于赫鲁晓夫一个人,而在于这个基础,基本问题,即有新的资本主义产生的基地。所以,只说反赫鲁晓夫不行,打倒一个,还有第二、第三、第四个,……。不只苏联出了修正主义,欧洲十几个国家都出了修正主义,代表什么?代表工人贵族。我说工人阶级的广大贫苦阶层出马克思列宁主义,少数工人贵族出修正主义。

(讲到不用武力来解决领土争端问题时)

台湾海峡,两重性。这是国内问题,谁人都不得干涉。我们也是两手,和平解放或者武力解放。不管那一手都是内政问题,谁也不能干涉。我们用武力解决,并未说死。

(讲到苏联现在请南斯拉夫以观察员身份列席经互会议时)

经互会要去,南斯拉夫参加,我也去。现在形势变了,赫鲁晓夫的指挥棒不灵了。南斯拉夫你参加,我也参加。将来要把经互会转到抵抗苏联的控制。大西洋公约国家反对美国控制,以法国为代表嘛!东欧国家也反对苏联控制,并且正在发展。……现在指挥棒不灵了……。要看到这一形势。

(谈到介绍工人中间的模范人物时)

老粗出人物。我们军区司令百分之九十都是老粗,行伍出身。但是,没有几个知识分子也不行。自古以来,能干的皇帝大多是老粗。汉朝刘邦是封建皇帝里边最厉害的一个。刘静劝他不要建都洛阳,要建都长安,他立刻去长安;鸿沟划界,项羽引退,他也想到长安休息,张良说什么条约不条约,要进攻,他立即听了张良的话,向东进。韩信要求封假齐王,刘邦说不行,张良踢了他一脚,他立即改口说:“他妈的,要封就是真齐王,何必假的。”而项羽则有三次错误,鸿门宴不听范增的话,那时他有四十万军队,刘邦只有十万人。鸿沟协定他认真了,建都徐州(那时叫彭城)。南北朝宋、齐、梁、陈,五代梁、唐、晋、汉、周,很有几个老粗。文的也有几个好的,如李世民。我们中央上过大学的也很少,过去上了大学的就算做官了,还革什么命。现在有许多新的好的典型,要提倡。


