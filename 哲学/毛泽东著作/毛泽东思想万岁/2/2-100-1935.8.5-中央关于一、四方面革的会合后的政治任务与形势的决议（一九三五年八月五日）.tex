\section[中央关于一、四方面革的会合后的政治任务与形势的决议(一九三五年八月五日)]{中央关于一、四方面革的会合后的政治任务与形势的决议}
\datesubtitle{(一九三五年八月五日)}


一、目前形势与特点

帝国主义的更进一步侵略中国,特别是日本帝国主义的占领华北,造成“华北国”的实际行动,国民经济总崩溃的深刻化,全中国的水旱灾荒,农村经济的崩溃与毁灭,造成了中国政治上的严重形势。国民党反动派统治不能消灭或暂时削弱造成中国革命的基本原因,相反的,他使这些原因更进一步地紧张化了。国民党的统治是在削弱和崩溃中。

白区工农群众的革命斗争是继续存在的,虽然许多斗争往往带着自然性,虽然目前尚未广泛地爆发带有全国性的大的群众斗争,然而这种斗争是普遍地蕴藏着,造成了伟大事变立刻就要来的形势。

苏维埃革命行动,虽是由于长江下游的几个苏区暂时变为游击区,而遭到部分损失,然而这些苏区中广大的游击战争是继续坚持着。湘赣与福建沿海的红军得到许多胜利。红二六军团击破了陈渠珍一路之后,现在又消灭了张振汉一路,巩固了原有苏区根据地。红二十五、二十六军及二十九军在川、陕、甘三省的活跃,尤其是一、四方面军两大主力在川南北的会合,造成了中国苏维埃运动在西北开展极大胜利的前途。一切这些,说明了中国革命形势的依然存在,证明苏维埃革命并未低落,而是继续发展着。

国民党正在以空前的仇恨与疯狂,向着白区工农群众的革命斗争尤其是苏维埃革命进攻。国民党反动统治力量的绝对的削弱,促使它以更大的积极性更进一步地出卖中国,在帝国主义直接指挥下,进攻苏维埃革命。一、四方面军在西北的发展,使华北的军阀更直接的与积极的参加围攻苏维埃与红军的斗争。有计划的追击、截击,特别是堵击我们,企图消耗我们的有生力量,用堡垒主义封锁我们,使我们陷入不利地区内,然后寻求我主力决战,这是蒋介石反革命的基本战略方针。

但是敌人向我们进攻中,遇到了很大的困难;部队的远离军事政治经济的中心城市,交通的不便,给养的困难,内部的不统一与冲突减员、疲劳、冻饿、不满意,与失败情绪的滋长,财政支出的空前不敷等。而在另一方面,全国民众的革命斗争,各地苏维埃红军的发展,尤其是一、四方面军的会合,大大兴奋了全中国的工农劳苦群众,坚强了他们对于革命胜利的信心。由于一、四方面军的会合,革命战争经验的交换,指挥的统一,红军战斗力不但在数量上增加而且在质量上也增加了,一、四方面军在中国西北部的活动,将大大推动西北少数民族对帝国主义与反国民党的斗争,使西北广大地区土地革命的斗争进一步的尖锐化,使共产党苏维埃红军的影响大大的扩大,同时西北各省是中国反动统治及帝国主义力量最薄弱的地区,在地理上又接近世界无产阶级祖国苏联及蒙古人民共和国,这更造成苏维埃与红军发展的有利条件。共产党中央的正确领导与适当的战略战术方针,更使我们坚信,、我们一定能够取得彻底粉碎敌人对于我们的进攻,创造和巩固西北苏区根据地。

二、一、四方面军会合后的基本任务

一、四方面军会合后,大大的增强了苏维埃革命的武装力量,展开了苏维埃革命伟大胜利的前途。六月十八日中央政治局曾经决定:“在一、四方面军会合后,我们的战略方针是集中主力向北进攻,在运动战中大量消灭敌人,首先取得甘肃南部,以创造川陕甘苏区根据地,使中国苏维埃运动放在更巩固更广大的基础上,以争取中国西北各省,以至全中国的胜利。”这一决定无疑的是正确的。

创造川陕甘的苏区根据地,是放在一、四方面军前面的历史任务。这个根据地的造成,不但是红军作战的后方,而且是推动整个中国革命前进与发展的苏维埃国家的领土。它的存在,是以鼓励全中国被压迫的工农劳苦群众起来同帝国主义国民党作斗争。它的模范的作用,给全国民众指出了政治经济解放的道路。它是一个团结全中国革命力量的核心,与散布革命种子到全中国去的发源地。

红军基本的严重的责任就是在川陕甘及广大西北地区创造出这样的一个根据地,彻底地击破蒋介石国民党的包围与封锁,大量的消灭敌人的有生力量,是创造这个根据地的先决条件。和平创造新苏区是不可能的。把一切努力和牺牲去争取革命战争的胜利,把一切利益服从于革命战争的最高利益,才能创造出西北苏区根据地,才能取得苏维埃在全中国的胜利。

为了建立巩固的苏区根据地,必须深入农民土地斗争,彻底解决土地问题,必须经过革命委员会的阶段建立真正工农民众的苏维埃政权,必须普遍建立赤卫军少先队独立师团游击队等民众自卫的武装,必须严厉镇压反革命,要使反革命统治区域变为真正革命的苏维埃区域,必须坚持地执行上述各项政策。

三、加强党在红军中的领导

为了创造川陕甘苏区的历史任务,必须在一、四方面军中更进一步的加强党的绝对领导,提高党中央在红军中的威信,中国工农红军是在中国共产党中央的唯一的绝对的领导之下生长与发展起来的,没有中国共产党就没有中国工农红军,就没有苏维埃革命运动。

在一、四方面军会合后,红军中的个别同志,因为看到中央苏区的变为游击区,看到一方面军的减员,看到党在某些工作中的错误和弱点,而认为是党中央政治路线的不正确,这种意见是完全错误的,但政治局认为对于其它个别同志的不了解与怀疑党应给以明确的解释与教育。

五中全会(一九三四年一月开的)在他的决议案中曾经清楚的指出:“四中全会以后,中央政治局在报告的环境之下,忠实的执行着共产国际与四中全会的路线,坚持地进行了反对一切机会主义的倾向与动摇,粉碎了各种机会主义,在实际工作中间开始了党的全部工作的彻底转变,得到许多重要的成功与胜利。”在思想方面指出:“自从四中全会以来,党在坚决的两条路线斗争中坚强的锻炼了自己,获得了思想上的布尔什维克的坚定性与一致,最后走上了布尔什维克化的道路。”必须指出五中全会的决议是得到了共产国际的同意的。

遵义政治局扩大会议重申指出了党中央的政治路线的正确:党中央根据于自己的正确估计,定出了反对敌人五次“围剿”的具体任务。一年半反对“围剿”的艰苦斗争,证明党中央的政治路线无疑义的是正确的,特别中央苏区的党,在中央直接领导之下,在动员广大群众参加革命战争方面得到了空前的成绩。扩大会议同时指出五次“围剿”不能在中央苏区粉碎,使主力红军退出苏区,受到损失的主要原因,是由于党在军事上犯了单纯防御的错误。这种军事上的错误,不但对于党中央的总的政治路线说来是个别的错误,即对于这一错误的主要负责者,也不是整个政治路线的错误而是部分的严重的错误。

遵义政治局扩大会议纠正了党中央在军事上所犯的错误以后,在军事领导上无疑义是完全正确的。因此,一方面军在遵义会议后得到了许多伟大的胜利,完成了党中央预定的战略方针。

必须使每一个同志清楚的了解,党的总路线是否正确,要看党是否正确的估计了中国革命的许多基本问题,是否正确的估计了当前的形势,是否正确的提出了并执行了策略上与战略上的各种任务。军事指挥问题是这许多问题中的一个问题,如果党在中国革命的理论与实际上是基本正确的,是马克思列宁主义的,而在军事问题上却在一个时期中犯了错误,那这一错误对于党只是部分的错误,虽是严重的政治错误。

关于一方面军的减员的原因,党在遵义会议上已经在全党内充分的发展了自我批评。党对于自己所犯的错误没有丝毫的隐蔽。遵义会议后党在军事指挥上固然有了很大的进步,但实际工作中还有某些弱点,是没有问题的。这里应该特别指出的,是部队中政治工作的薄弱,首先是总政治部没有尽一切可能给下级政治部以适时的指示与推动,使下级政治部得不到坚强的领导。反右倾的斗争没有很大的开展起来。在一、四方面军会合后,没有利用机会来及时整顿一方面军。

显然的,把党的部分的错误误解为全部的错误,把党在实际工作中的某些弱点误解为路线的错误,而对于党所成就的空前的伟大的事业不给以应有的估计,是不正确的。因此对于这种误解,党必须给以及时的解释与纠正,使全体党员与红军指战员像一个人一样团结在党中央的周围,这是以后胜利的保障。

四、一、四方面军兄弟般的团结

一、四方面军的兄弟的团结,是完成创造川甘陕苏区,建立中华苏维埃共和国的历史任务的必要任务的必要条件,一切有意无意的破坏一、四方面军团结一致的倾向,都是对于红军有害,对于敌人有利的。

目前在一、四方面军内部产生的某些个别问题,主要的是由于相互了解的不够,缺乏对一、四方军的正确的估计。

一方面军一万八千里的长征是中国历史上的空前的伟大事业,为蒋介石以及七八省国民党军队所包围追击截击与堵击,完全没有休息的长途行军,历尽艰难困苦与饥饿寒冷,然而一方面军的全体指战员,在党中央与军委领导之下,始终以惊人的英勇与坚决,同敌人作无数次的血战,突破了敌人的包围,击破了敌人的追击截击与堵击,消灭了蒋介石等军阀的许多部队,渡过了天险的湘江、乌江、金沙江与大渡河,最后达到了与四方面军会合的预定目的,使蒋介石等进攻我们的计划完全失败。

然而这丝毫也不能否认一方面军在一万八千里长征中所给予他的损失。一方面军在脱离中央苏区后,这一时期(十个月)不但在数量上极大的减员(遵义会议前军事领导的错误负最大的责任),即在质量上由于肉体上的疲劳,由于休息时间的缺乏,更由于政治工作的不深入,而受到了相当的损失。这表现在:部队组织的松懈,纪律性的薄弱,游击主义倾向与军阀习气的部分生长,在某些干部中发展着疲倦,不负责任以及右倾的悲观失望的情绪与思想。这就使部队的战斗力相当的削弱,不看到一方面军的这些弱点或夸大这些弱点与不去分析这些弱点的来源,必然会产生对于一方面军过左或过右的估计。过左的估计可以掩盖目前必须整顿的部队,加紧反右倾的斗争,严禁纪律的实际工作的消极,而过右的估计则可以产生对于一方面军力量的不相信。一方面军的同志应该以最大的努力整顿自己的部队,学习四方面军的英勇善战,坚决相信只要取得相当休息整理的时间与补充扩大,完全可以得到很大的进步,决不要因为目前的相当减员与部分损失而气馁。而四方面军的同志应该给一方面军以最切实的兄弟的帮助。

四方面军的党的领导在基本路线上是正确的,是执行了四中全会后国际与中央的路线的。正因为如此所以创造了强大的与坚强的红四方面军,取得了许多次战争的伟大的胜利,创造了鄂豫皖与南巴赤区,四方面军英勇善战,不怕困难,吃苦耐劳,服从命令,遵守纪律等许多特长,特别是部队中旺盛的攻击精神与战斗情绪,是现在一方面军应该学习的。但四方面决不应该以此自满,而应更加发扬自己的特长,应吸收一方面军在战略战术方面与红军建设方面所有丰富的经验。以来得自己更大的进步,成为铁的工农红军。

必须使一、四方面军的每一个同志了解一、四方面军部是中国工农红军的一部分,都是中国共产党中央所领导的。在我们中间只有阶级的友爱和互助而没有分歧和对立,只有这样一、四方军的团结一致才是巩固的与永久的,才能溶成一片的去消灭敌人。

五、关于少数民族中党的基本方针

一、四方面军的会合,正在少数民族番夷民占多数的区域,红军今后在中国的西北部活动,也到处不能同少数民族脱离关系,因此争取少数民族在中国共产党与中国苏维埃政府领导之下,对于中国革命胜利前途有决定的意义。

中国共产党和中国苏维埃政府在少数民族中的基本方针,是无条件的承认他们省民族自决权,即在政治上有随意脱离压迫民族即汉族而独立的自由权,中国共产党和中国苏维埃政府在实际上帮助他们的民族独立与解放运动,反对帝国主义与国民党,反对他们的内奸卖国贼、土司喇嘛与他们自己的剥削阶级。

估计到少数民族中阶级分化程度与社会经济发展的条件,我们不能到处用苏维埃方式去组织民族的政权。在有些民族中,在斗争开始的阶段上,除少数上层分子外,还有民族统一战线的可能,在这种情况下,可以采取人民共和国及人民革命政府的形式。在另外一种民族中,或在阶级斗争深入的阶段中,则可采取组织工农苏维埃或劳动苏维埃的形式。一般的组织工农民主专政苏维埃是不适当的。

在一、四方面军没有会合以前,四方面军在帮助番民组织游击队,在建立革命政权上,发动番民内部的阶级斗争上,得到了相当的成绩。但目前建立西北苏维埃联邦政府是过早的。因为目前在少数民族中的基本方针,应首先帮助他们的独立运动,成立他们的独立国家。中华苏维埃共和国中央政府应公开号召蒙、回、藏、等民族起来为成立他们自己的独立国家而斗争,并给这种斗争以具体的实际的帮助。在他们成立了独立国家之后,则可以而且应该根据他们自愿的原则,同中华苏维埃共和国联合成立真正的民族平等与民族团结的中华苏维埃联邦,在这个时候,联邦的策略才是正确的。

在许多其它问题上,马克思、列宁、斯大林关于民族问题的理论与方法,是我们解决少数民族问题的最可靠的武器,只有根据这种理论与方法,我们在工作上才能有明确的方针与路线,学习马克思、列宁、斯大林关于民族问题的理论与方法,是目前我们全党的迫切任务。

六、目前的中心工作

为了创造川陕甘新苏区,目前我们的中心工作应该是:

(一)立即在一、四方面军中进行宣传鼓动,提高部队的战斗情绪,胜利信心,与刻苦耐劳的精神,准备大量消灭当前敌人,取得北进战略中各个战役的完全胜利,以实现党中央的战略方针。

(二)利用并争取时间整顿部队,进行军事政治的教育训练,以加强部队的战斗力。

(三)为了加强红军党的领导,必须使政治委员制度更加确定,加强红军中党的组织及政治部的工作,把政治工作的重点深入到连队与支部中去。

(四)大大提高与严紧一方面军的纪律,必须采取严厉办法保障纪律的执行。同时应使四方面军的同志了解红军的纪律,主要的不是依靠于强迫,而是依靠于阶级的觉悟,极大发扬党员间与红军指战员间阶级友爱与服从纪律的精神。

(五)加紧对于全体党员与红军指战员间的基本的阶级教育,使他们能够在各种复杂的与变化的环境下坚决不动摇的为苏维埃革命斗争到底,设立红军大学与高级党校,大批培养军事的与政治的干部。

(六)番民中的工作必须有迅速的转变。总政治部应收集各地番地工作的经验与教训,以教育自己的干部。用一切办法争取番民群众回家,组织番民游击队,发动番民斗争,建立番民革命政府等。必须挑选一部分优秀的番民给以阶级的与民族的教育,以造成他们自己的干部。红军主力到甘陕青宁等区域后,对回蒙民疾须作更大的努力。

(七)广大的白区的工作,首先是邻近白区工作,目前应该有计划的开始,发动白区广大工农群众的斗争与游击战争,响应与配合红军的行动,创造游击区与新苏区是我们目前的迫切任务。党中央必须利用一切方法去加强川陕甘三者白区党的领导。同时对长江下游及华北华南各中心城市,产业区域及农村中的斗争,应该在过去工作的基础上及新的有利的环境中极力加强自己组织与领导的力量。

(八)必须立刻开始白军中的士兵的工作,以瓦解国民党的部队。当地白军工作委员会,必须广泛的×××工作组以×××的材料(××军委标语)适合于当前环境的简单指示,使每个红军指战员了解瓦解白军工作××××部,这一工作的重要,与如何进行白军工作。(原稿如此)

(九)必须加强川康省与宁夏省军委的工作,使我们能够真正集中地方与游击战争的领导。要使这个地区成为川陕甘苏区之一部,红军到达陕北后,须更大的建立与加强当地×××的工作。

(十)建立保卫局的组织系统,加强同反革命的斗争。

(十一)吸收四方面军党的最好干部,参加党中央及其它军事政治机关的负责工作。

(十二)采取必要方法,加强对于其他苏区与游击区的领导,使各方面的行动更能取得互相的呼应与配合。

七、苏维埃革命胜利的前途与两条路线的斗争

在创建川陕甘新苏区的过程中,我们必须要碰到许多困难,敌人决不放松我们,他必将加强他的战斗力向我们进攻,同时高山河流草地……必会给我们许多困难。虽然如此,只要我们执行党中央路线,发扬我们的创造清神,我们是能够克服这些困难的。

必须在部队中坚决反对各种右倾机会主义的动摇,这种动摇是由于对于敌人力量的过分估计,夸大敌人的力量,看不到敌人内部力量的削弱,而同时对于自己估计不足所产生的。这种动摇具体的表现在对于党中央所决定的战略方针表现怀疑,不敢大胆的前进,而企图远离敌人避免战斗,对创造新根据地没有信心,惧怕少数民族中工作的困难,没有决心在少数民族中进行艰苦的工作。这种动摇具体的表现在对于一、四方面军力量的不信任,不了解一、四方面军会合的伟大意义。甚至根本怀疑到自己部队的战斗力。这种动摇具体的表现在碰到某些困难即表示悲观失望,消极怠工,不负责任与自暴自弃,这种动摇更表现在对于目前时局估计不正确,怀疑到革命形势的存在,推想到苏维埃运动的低落。因而对革命前途悲观失望。

这些右倾机会主义的动摇,明显的在部队中存在着,而且部分的生长着。这对于创建根据地的任务是最大的危硷。开展反右倾机会主义的斗争,是当前中心任务之一。在这种斗争中,必须向每一个党员与红军指战员细心的解释目前的形势与我们胜利的前途,使他们确信苏维埃革命现在虽是遭遇到一些困难,但苏维埃革命必然要胜利。必须使反倾向的斗争同最具体的实际工作的转变密切的联系起来,必须在反倾向斗争中加紧肃反的工作,反对任何对反革命活动的放任和宽恕。

在反右倾的斗争中丝毫也不要放松“左”倾的空谈,这种“左”倾的具体于以“吹牛皮”来代替敌我力量的正确的分析。因此便仍然造成对目前形势的过“左”的估计,这种“左”的估计的结果,或者走到轻敌的冒险主义,或者掩盖自己惧怕敌人的“右”倾机会主义实质。

党必须用马克思列宁主义来教育我们的党员与红军指战员,冷静的来估计敌我力量的对比与目前的形势。只有这种正确的估计,才能使我们正确的提出党的任务与口号。马克思列宁主义者拥护真理而反对欺骗。

因为两条路线的斗争的主要目的,是在教育全体党员与红军指战员,以一切努力与牺牲来完成他们所担负的历史任务,只有对于不可救药的机会主义者,党才迟疑的采取纪律的制裁。

放在我们面前的任务是艰巨的,但我们确信,一、四方面军在党中央及军委领导之下,我们必然能够完成这些任务,能够创造川陕甘西北苏区,取得苏维埃革命在全中国的胜利。

