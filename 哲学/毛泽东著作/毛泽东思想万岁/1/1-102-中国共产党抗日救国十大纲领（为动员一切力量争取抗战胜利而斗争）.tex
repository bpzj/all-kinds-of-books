\section[中国共产党抗日救国十大纲领(为动员一切力量争取抗战胜利而斗争)]{中国共产党抗日救国十大纲领(为动员一切力量争取抗战胜利而斗争)}


一、打倒日本帝国主义:

对日绝交,驱逐日本官吏,逮捕日本侦探,接收日本帝国主义在华财产,否认日本外债,废除日本条约,收回日本租界。

为保卫华北与沿海各地而血战到底。

为收复平津与东北而血战到底。

反对任何的动摇妥协。

二、全国军事的总动员:

动员全国海陆空军实行全国抗战。

反对单纯防御的消极作战方针,釆取独立自主的积极作战方针。

建立经常的国防会议,讨论与决定国防计划与作战方针。

武装人民,发展抗日的游击战争,配合主力军作战。

改革军队的政治工作,使指挥员与战斗员团结一致,军队与人民团结一致。发扬军队的积极性。

援助东北人民革命军东北义勇军。破坏敌人后方。

实现一切抗日军队的平等待遇。

建立全国、各地军区,动员全民族参战,以便从雇佣兵役制转变为义务兵制。

三、全国人民的总动员:

全国人民除汉奸外,皆有抗日救国的言论、出版、集会、结社,及武装抗敌之自由。废除一切束缚人民爱国运动的旧法令,颁布革命的新法令。释放一切爱国的革命的政治犯,开放党禁。全国人民动员起来武装起来,参加抗战,实行有力出力,有钱出钱,有枪出枪,有知识出知识。

动员蒙民回民及其他一切少数民族,在民族自决和民族自治的原则下,共同抗日。

四、改革政治机构

召集真正代表人民的国民大会,通过真正的民主宪法,决定抗日救国方针,选举国防政府。

国防政府必须吸收各党各派及人民团体的革命分子,驱逐亲日分子。

国防政府采取民主集中制,他是民主的,但又是集中的。

国防政府执行抗日救国的革命政策。实行地方自治,铲除贪官污吏,建立廉洁政府。

五、抗日的外交政策:

在不丧失领土主权的范围内,与一切反对日本侵略主义的国家订立反侵略的同盟,及抗日的军事互相协定。

拥护和平阵线,反对德日意侵略阵线。

联合朝鲜台湾及日本国内的工农人民反对日本帝国主义。

六、战时的财政经济政策:

财政政策以有钱出钱及没收汉奸财产作抗日经费为原则,经济政策是整顿与扩大国防生产,发展农村经济,保证战时农产品的自给,提倡国货改良土产,禁绝日货,取缔奸商,反对投机操纵。


七、改良人民生活:

改良工人农民职员教员及抗日军人的待遇。


优待抗日军人的家属。

废除苛捐杂税,减租减息;救济失业;调节粮食;脤济灾荒。

八、抗日的教育政策:

改变教育的旧制度旧课程,实行以抗日救国为目标的新制度新课程。

实施普及的义务的免费的教育方案,提高人民民族觉悟的程度。

实行全国学生的武装训练。

九、肃清汉奸卖国贼亲日派。巩固后方。

十、抗日的民族团结:

在国共两党彻底合作的基础上,建立全国各党各派各界各军的抗日民族统一战线,领导抗日战争,精诚团结、共赴国难。

