\section[在郑州会议上的讲话(一)(一九五九年二月二十七日)]{在郑州会议上的讲话(一)}
\datesubtitle{(一九五九年二月二十七日)}


1958年,我们在各个战线上取得了伟大的成绩,不论在思想政治战线上,工业战线上,农业战线上,交通运输战线上,商业战线上,文教战线上,国防战线上,以及其他方面。都是如此。特别显着的。是工业和农业生产方面有了一个伟大的跃进。1958年.在全国农村中普遍建立了人民公社。

人民公社的建立使农村中原来的生产资料集体所有制扩大和提高了,并且开始带有若干全民所有制的成份,人民公社的规模比农业生产合作社大得多,并且实行了工农商学兵,农林牧副渔的结合,这就有力的促进了农业生产和整个农村经济的发展,广大的农民。尤其是贫农和下中农。对于人民公社表现了热烈的欢迎。广大干部在人民公社运动中做了大量的有益的工作。他们表现了作为一个共产主义者的极大的积极性,这是非常宝贵的,没有他们这种积极性。要取得这样伟大的成绩是不可能的。当然,我们的工作中,不但有伟大的成绩。”也有一些缺点。在一个新的、像人民公社这样缺乏经验的前无古人的几亿人民的社会运动中。人民和他们的领导者们都只能从他们的实践中逐步取得经验,对事物的本质逐步加深他们的认识,揭露事物的矛盾,解决这些矛盾,肯定工作中的成绩,克服工作中的缺点,谁要说一个广大的社会运动能够完全没有缺点,那他不过就是一个空想家,或者是一个观潮派算账派,或者简直是敌对分子。我们的成绩和缺点的关系,正是我们所常说的,只是十个指头中九个指头和一个指头的关系。有些人怀疑或者否认1958年的大跃进。怀疑或者否认人民公社的优越性,这种观点显然是完全错误的。

人民公社现在正在进行整顿巩固工作,就是说整社,已经或者正在辩论1958年有无大跃进和人民公社有无优越性两个问题。各级党委正在整社工作中,按着六中全会的方针,采取了首先肯定大跃进的成绩,肯定人民公社的优越性,然后才能指出工作中的缺点错误这种次序,这种作法是完全恰当的。这样作,可以保护广大干部和群众的积极性。就干部来说。90%几都是好的。都是应当加以坚决保护的。

现在我来说一点人民公社的问题.我认为人民公社现在有一个矛盾。一个可以说相当严重的矛盾。还没有被许多同志所认识。它的性质还没有被揭露,因而没有被解决。而这个矛盾我认为必须迅速的解决,才有利调动广大人民群众更高的积极性。才有利于改善我们和基层干部的关系,这主要是县委、公社党委和基层干部之间的关系。

究竟什么样一种矛盾呢?大家看到,目前我们跟农民的关系,在一些事情上存在着一种相当紧张的状态,突出的现象是在1958年农业大丰收以后,粮食、棉花、油料等等农产品的收购至今还有一部分没有完成任务。再则全国(除少数灾区外),几乎普遍地发生瞒产私分.大闹粮食,油料、猪肉、蔬菜“不足”的风潮,其规模之大,较之1953年和1955年那两次粮食风潮都有过之无不及。同志们。请你们想一想,究竟是什么一回事呢?我认为,我们应当透过这种现象看出问题的本质即主要矛盾在什么地方。这里面有几方面的原因。但是我以为主要地应当从我们对农村人民公社所有制的认识和我们所采取的政策方法去寻找答案。

农村人民公社所有制要不要有一个发展过程?是不是公社一成立,马上就有了完全的公社所有制,马上就可以消灭生产队的所有制呢?我这是说的生产队,有些地方是生产大队即管理区,总之大体上相当于原来的农业生产合作社。现在有许多人还不认识公社所有制必须有一个发展过程,在公社内,由队的小集体所有制到社的大集体所有制,需要一个过程,这个过程要有几年时间才能完成。他们误认人民公社一成立,各生产队的生产资料、人力、产品,就都可以由公社领导机关直接支配。他们误认社会主义为共产主义。误认按劳分配为按需分配,误认集体所有制为全民所有制。他们在许多地方否认价值法则,否认等价交换。因此,他们在公社范围内,实行贫富拉平,平均分配,对生产队的某些财产无代价地上调,银行方面,也把许多农村中的贷款一律收回。

“一平,二调,三收款”,引起广大农民的很大恐慌。这就是我们目前同农民关系中的一个最根本的问题。

公社成立了,我们有了公社所有制。如北戴河决议和六中全会决议所说,这种一大二公的公社有极大的优越性,是我们的农村由社会主义的集体所有制过渡到社会主义的全民所有制的最好形式,也是我们由社会主义社会过渡到共产主义社会的最好形式。这是毫无疑问的,这是完全肯定了的。如果对于这样一个根本问题发生怀疑,那就是完全错娱的。那就是右倾机会主义的。问题是目前公社所有制除了有公社直接所有的部分以外,还存在着生产大队(管理区)所有制和生产队所有制。要基本上消灭这三级所有制之间的区别,把三级所有制基本上变为一级所有制,即由不完全的公社所有制发展成为完全的、基本上单一的公社所有制,需要公社有更强大的经济力量,需要各个生产队的经济发展水平大体趋于平衡,而这就需要几年时间。目前的问题是必须承认这个必不少的发展过程,而不是什么向农民让步的问题。在没有实现农村的全民所有制以前,农民总还是农民,他们在社会主义的道路上总还有一定的两面性。我们只能一步一步地引导农民脱离较小的集体所有制,通过较大的集体所有制走向全民所有制,而不能要求一下子完成这个过程,正如我们以前只能一步一步地引导农民脱离个体所有制而走向集体所有制一样。由不完全的公社所有制走向完全的、单一的公社所有制,是一个把较穷的生产队提高到较富的生产队的生产水平的过程,又是一个扩大公社的积累,发展公社的工业,实现农业机械化、电气化,实现公社工业化和国家工业化的过程。目前公社直接所有的东西还不多,如社办企业、社办事业,由社支配的公积金、公益金等。虽然如此,我们伟大的、光明灿烂的希望也就在这里。因为公社年年可以由队抽取积累,由社办企业的利润增加积累,加上国家的投资,其发展将不是很慢而是很快的。

关于国家投资问题,我建议国家在七年内向公社投资几十亿到百多亿元人民币,帮助公社发展工业帮助穷队发展生产。我认为,穷社穷队,不要很久,就可以向富社、富队看齐,大大发展起来。

公社有了强大的经济力量,就可以实现完全的公社所有制,也就可以进而实现全民所有制。时间大约需要两个五年计划,急了不行,欲速则不达。这也就是北戴河决议所说的,将需要三四年、五六年或者更长一些的时间。然后,再经过几个发展阶段,在十五年、二十年或者更多一些的时间以后,社会主义的公社就将发展成为共产主义的公社。

六中全会的决议写明了集体所有制过渡到全民所有制和社会主义过渡到共产主义所必须经过的发展阶段,但是没有写明公社的集体所有制也需要有一个发展过程,这是一个缺点,因为那时我们还不认识这个问题。这样。下面的同志也就把公社,生产大队、生产队三级所有制之间的区别模糊了,实际上否认了目前还存在于公社中并且具有极大重要性的生产队(或者生产大队,大体上相当于原来的高级社)的所有制。而这就不可避免要引起广大农民的坚决抵制。从1958年秋收以后全国性的粮食、油料、猪肉、蔬菜“不足”的风潮,就是这种反抗的集中表现。一方面,中央、省、地、县、社五级(如果加管理区就是六级)党委大批评生产队,生产小队的本位主义、瞒产私分,另一方面,生产队、生产小队却几乎普遍地瞒产私分,甚至深藏密窖,站岗放哨。以保卫他们的产品。我认为。产品本来有余,应该向国家交售而不交售的这种本位主义确实是有的,犯本位主义的党员干部是应该受到批评的。但是有很多情况并不能称之为本位主义。即令本位主义属实,应该加以批评,在实行这种批评之前,我们也必须首先检查和纠正自己的两种倾向,即平均主义倾向和过分集中倾向。所谓平均主义倾向,即是否认各个生产队和各个人的收入应当有所差别。而否认这种差别。就是否认按劳分配,多劳多得的社会主义原则。所谓过分集中倾向,即否认生产队的所有制,否认生产队应有的权利,任意把生产队的财产上调到公社来。同时,许多公社和县从生产队抽取的积累太多,公社的管理费又包括很大的浪费(例如有一些大社竟有成千工作人员不劳而食,甚至还有脱产文工团)。上述两种倾向。都包含有否认价值法则。否认等价交换的思想在内,这当然是不对的.凡此一切。都不能不引起各生产队和广大社员的不满。

目前我们的任务,就是要向广大干部讲清道理,经过充分的酝酿和讨论,使他们得到真正正的了解,然后我们和他们一起,共同妥善地坚决地纠正这些倾向。克服平均主义,改变权力、财力、人力过分集中于公社一级的状态。公社在统一决定分配的时候.要承认队和队,社员和社员的收入有合理的差别。穷队和富队的伙食和工资应当有所不同。工资应当实行死级活评。公社应当实行权力下放,三级管理,三级核算。并且以队的核算为基础。在社与队、队与队之间要实行等价交换。公社的积累应当适合情况。不要太高。必须坚决克服公社管理中的浪费现象。只有这样,我们才能够有效地去克服那种确实存在于一部分人中的本位主义,巩固公社制度。这样做了以后。公社一级的权力并不是很小,仍然是相当大的。公社一级领导机关并不是没有事做,仍然有很多事做,并且要用很大的努力才能把事情做好。

公社在1958年秋季成立之后,刮起了一阵“共产风”。主要内容有三条:一是穷富拉平,二是积累太多,义务劳动太多,三是“共”各种“产”。所谓“共”各种“产”,其中有各种不同情况。有些是应当归社的,如大部分自留地。有些是不得不借用的,如公社公共事业所需要的部分房屋桌椅板凳和食堂所需要的刀锅碗筷等。有些是不应当归社而归了社的,如鸡鸭和部分的猪归社而未作价。这样一来。共产风就刮起来了。在某种范围内。实际上造成了一部分无偿占有别人劳动成果的情况。当然,不包括公共积累,集体福利。经全体社员同意和上级党组织批准的某些统一分配办法,如粮食供给制等,这些都不属于无偿占有性质。无偿占有劳动的情况,是我们所不许可的。看看我们只是无偿剥夺了日德意帝国主义的、封建主义的、官僚资本主义的生产资料,和地主的一部分房屋、粮食等生活资料。所有这些都不是侵占别人劳动成果。因为这些被剥夺的人都是不劳而获的。对于民族资产阶级的生产资料,我们没有采取无偿剥夺的办法,而是实行赎买政策。因为他们虽然是剥削者,但是他们曾经是民主革命的同盟者,现在又不反对社会主义改造.我们采取赎买政策,就使我们在政治上获得主动,经济上也有利。同志们,我们对于剥削阶级的政策尚且如此,那么,我们对于劳动人民的劳动成果,又怎么可以无偿占有呢?

我们指出这一点,是为了说明勉强把穷富拉平,任意抽调产生队的财产是不对的,而不是为了要在群众中间去提倡算旧账。相反,我们认为旧账一般地不应当算。无论如何,较穷的社,较穷的队和较穷的户,依靠自己的努力,公社的照顾和国家的支持,自力更生为主,争取社和国家的帮助为辅,有个三五七年,就可以摆脱目前的比较困难的境地。完全用不着依靠占别人的便宜来解决问题。我们穷人,就是说,占农村人口大多数的贫农和下中农,应当有志气,如像河北省遵化县鸡鸣村区的被人称为“穷棒子社”的王国藩社那样,站立起来,用我们的双手艰苦奋斗,改变我们的世界,将我们现在还落后的乡村建设成为一个繁荣昌盛的乐园。这一天肯定会到来的,大家看吧。

除了平均主义倾向和过分集中倾向以外,目前农村劳动力的分配也有很不合理的地方。这就是用于农业(包括农林牧副渔各业)的劳动力一般太少,而用于工业,服务业的行政人员一般太多。这后面三种人员必须加以缩减。公社人力的分配是一个重大问题。争人力,是目前生产队同社、县和县以上国家机关的重要矛盾之一,必须按农业、工业、运输业、服务业和其他各方面的正当需要,加以统筹,务使各方面的劳动分配达到应有的平衡。公社和县兴办工业是必要的,但是不可一下子办得太多。各种工业企业都必须节约人力,不允许浪费人力。服务业方面的人员,凡是多了的,必须减下来。行政人员只允许占公社人数的千分之几。文教事业的发展,应当注意不要占用过多的劳动力。公社不允许有脱产的文工团、体育队等等。

我们必须把安排人民生活,安排公社积累和安排国家需要这三个方面的工作,同时统筹兼顾。这样,才算真的作到了全国一盘棋。否则所谓一盘棋,实际上只是半盘棋,或者是不完全的一盘棋。一般说来,1958年公社的积累多了一点。因此,各地应当根据具体情况,规定1959年公社积累的一个适当限度,并且向群众宣布,以利安定人心,提高广大群众的生产积极性。

人民公社一定要坚持勤俭办社的方针,一定要反对浪费。在粮食工作方面,鉴于最近的经验,今后必须严格规定一个收粮、管粮,用粮的制度,一定要把公社的粮食收好,管好、用好。社会对于粮食的需要总是会不断增长的,因此,至少在几年内不要宣传粮食问题“解决”了。

最近各省都有干部下去当社员,这个办法很好。我提议各级干部分期分批下放当社员,少则一个月,多则一个半月。一部分干部可以下厂矿当工人。这个办法在去年已经行之有效。今年要更好的加以推广。总之,一定要不断地巩固我们同广大群众的联系。

采取以上所说的方针和办法,我认为,我们目前同农民和基层干部的关系一定会很快的改善。广大农民从公社运动和1958年的大跃进已经得到了巨大的利益,他们坚决要求继续跃进和巩固公社制度。这是事实,不是任何观潮派、算账派所能推翻的。我们的干部在去年一年中做了很多很好的工作,得到了伟大的成绩,广大群众是亲眼看到的。问题只是我们在生产关系的改进方面,即是说,在公社所有制问题方面,前进得过远了一点。很明显,这种缺点只是十个指头中一个指头的问题。而且这首先是由于中央没有更早地作出具体的指示,以致下级干部一时没有掌握好分寸。如我在前面所说过的,这种情况在人民公社化这样一个复杂的和史无前例的事业中是难以避免的。只要我们向广大群众公开说明这一点,并且在实际行动中克服过去一段时间内工作中的缺点,那么,主动权就完全掌握在我们手里,广大群众就一定会同我们站在一起。必须估计到,一方面,那些观潮派、算账派,将会出来讥笑我们,另一方面,那些地主、富农、反革命分子、坏分子将会进行破坏。但是,我们要告诉干部和群众,当着这些情况出现的时候,对于我们丝毫没有什么可怕。我们应该沉得住气,在一段时间内,不声不响,硬着头皮顶住,让那些人去充分暴露他们自己。到了这种时候,广大的群众一定会很快分清是非,分清敌我,他们将会起来粉碎那些落后分子的嘲笑和敌对分子的进攻。经过这样一个整顿和巩固人民公社的过程,我们同群众的团结将会更加紧密。在伟大的中国共产党的领导下,五亿农民一定会更加心情舒畅,更加充满干劲。我们一定能够在1959年实现更大的跃进。人民公社的事业,一定能够在巩固的基础上蒸蒸日上,胜利一定是我们的。


