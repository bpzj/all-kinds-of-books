\section[青少年的教育(一九五三年)]{青少年的教育}
\datesubtitle{(一九五三年)}


鲁迅有一篇文章,题目是“从孩子的照相说起”,鲁迅很懂得这个事情,他提倡孩子要活泼,要顽皮。他说“驯良之类并不是恶德,但发展开去,对一切事无不驯良,却决不是美德,也许简直倒是没出息。”(《鲁迅全集》第六卷第84页)在有些地方要让我们的后一代学我们;但有些地方就决不要后代再学我们,再学我们就糟糕。我们总是中孔夫子的毒太深了。不仅学校如此,就是托儿所也要很好注意。现在不把孩子送到托儿所还好,一送到托儿所就“机械化”了。在这些方面,我们要下很大的功夫,用很多方法,把我们的青年、儿量变成朝气蓬勃,生龙活虎一样。有这样的人民,建立这样一个国家,那我们中国就了不起。要懂得人在幼年、青年的时候,正是性格、品质慢慢形成的时候,如果在这方面不会教育他们,将来的损失就很大。


