\section[国防部长告台湾同胞书(一九五八年十月六日)]{国防部长告台湾同胞书}
\datesubtitle{(一九五八年十月六日)}


台湾、澎湖、金门、马祖军民同胞们:

我们都是中国人,三十六计,和为上计。金门战斗,属于惩罚性质。你们的领导者们过去长时期太猖狂了,命令飞机向大陆乱钻,远及云、贵、川、康、青海,发传单、丢特务、炸福州、扰江浙。是可忍、孰不可忍?因此打一些炮,引起你们注意。台、澎、金、马是中国的领土。这一点你们是同意的,见之于你们领导人的文告,确实不是美国的领土。台、澎金、马是中国的一部分,不是另一个国家。世界上只有一个中国,没有两个中国。这一点也是你们同意的。见之于你们领导人的文告。你们领导人与美国人订立军事协定,是片面的,我们不承认,应予废除。美国人总有一天肯定要抛弃你们的,你们不信吗?历史的巨人会要做出证明的。杜勒斯九月三十日的谈话,端倪已见。站在你们的地位,能不寒心?归根结底,美帝国主义是我们共同的敌人。十三万金门军民供应缺乏,饥寒交迫,难为久计。为了人道主义,我已命令福建前线,从十月六日起暂以七天为期,禁止炮击,你们可以充分地自由地输送供应品,但以没有美国人护航为条件。如有护航不在此例。你们与我们之间的战争,三十余年了,尚未结束,这是不好的。建议举行谈判,实行和平解决。这一点,周总理已在几年前告诉你们了。这是中国内部贵我两方有关问题,不是中美两国有关问题。美国侵略台、澎、与台湾海峡,这是中美两国有关问题,应当由两国谈判解决,目前正在华沙举行。美国人总是要走的,不走是不行的。早走于美国有利,因为他们可以取得主动。迟走不利,因为他老是被动。一个东太平洋国家,为什么跑到西太平洋来了呢?西太平洋是西太平洋人的西太平洋,正如东太平洋是东太平洋人的东太平洋一样,这一点是常识,美国人应当懂得。中华人民共和国与美国之间并无战争,无所谓停火。无火而谈停火,岂非废话了?台湾的朋友们,我们之间是有战火的,应当停止,并予熄灭。这就需要谈判。当然。再打三十年,也不是什么了不起的大事,但是究竟以早日和平解决为妥善。何去何从,请你们酌定。


