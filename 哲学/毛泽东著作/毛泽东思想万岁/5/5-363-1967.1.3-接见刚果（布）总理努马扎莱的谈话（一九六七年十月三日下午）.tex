\section[接见刚果(布)总理努马扎莱的谈话(一九六七年十月三日下午)]{接见刚果(布)总理努马扎莱的谈话(一九六七年十月三日下午)}
\datesubtitle{(一九六七年十月三日)}


努马:我看到中国无产阶级文化大革命的伟大胜利,使中国人民政治觉悟大大提高了。

主席:无政府主义也大大发展了。

努马:也许是这样,但是我们还没有看到这些。

主席:有那么个思潮暴露出来好教育。

努马:你们的干部很谦虚。

主席:非谦虚不可,否则群众斗他们。

努马:你们的干部与我们的干部有很大的区别。

主席:没有多大区别。都是官大了,薪水多了,坐小汽车了。大官还得有人做,大官没人做还得了!薪水多一点,房子好一点,坐汽车也可以,但不要摆架子,和工农群众平等相待。不要动不动就训人,骂人。有的大队书记,薪水不多,房子不好,没坐小汽车,官也不大,就是官架子不小。运动一开始,结果把他们吓了一跳。

努马:外国讲中国很乱,我们怎么没看到?

主席:乱一点,你们可以到处走走,乱了以后就不乱了。不闹够就不行。这时候差不多了。我们准备再乱一年。

努马:什么叫越乱越好呢?

主席:不乱胜负不分,湖南、湖北、江西、安徽、浙江,除安徽省外都好。到中国得一条经验。湖南煤矿动刀动枪了。生产几万吨下降到几吨,现在已产二万吨。

努马:这样的矛盾,怎么解决得这么好呢?

主席:后台揪出来了,群众打够了,这时中央讲几句话就行了。×××吹中国怎么好,不要听那一套,非洲人架子小,所以我们希望你们来。欧洲、亚洲就不行。

努马:我们也开始反官架子。

主席:我不建议你们也搞文化大革命。我们建军四十周年,建国十八年,打了二十二年,拥有打了几十年仗的解放军,所以搞文化大革命。

努马:我们不搞文化大革命,但我们要研究文化大革命的理论和世界意义。

主席:这次文化大革命要改变国家部分机构,包括军队。思克鲁玛那次来,没有料到推翻他政权的就是他的军队。我看你还是早点回去。

努马:我们家里还有人,但我尽量早点回去。


