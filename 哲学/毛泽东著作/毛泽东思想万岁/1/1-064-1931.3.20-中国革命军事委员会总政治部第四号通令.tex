中国革命军事委员会总政治部第四号通令

(一九三一年三月二十日)

※ 二次战争的意义目前敌我的形势和争取二次胜利的准备工作
(一九三一年三月二十日)



红军各级政治委员、政治部主任、地方各级政府:

(一)争取二次胜利的中心意义在于转变敌我形势。

自从军阀战争暂时停顿敌人向革命进攻以来,敌人是一种攻势,我们在大体上说始终是一种守势,龙冈战争,虽然取得大的胜利,但仍未能转变这种攻守形势。因此,转变敌我攻守的形势,成为二次战争的中心任务。我们要转变敌我攻守的形势,就要给第二次进攻的敌人以严重的打击,取得比龙冈战争更大的胜利。必须这样,才能使敌人溃败下去,退守中心城市,把敌人的形势,强迫着他变成一种守势。必须使敌人退守到中心城市去了,我们才能彻底解决苏维埃区域的巩固问题。必须使敌人退守到中心城市去了,我们才能使现有苏维埃区域广大的发展一步。必须在敌人溃退下去,苏维埃区域有了切实的巩固,尤其在有了广大的发展了,才能使红军得到进一步的扩大与精练。必须在苏维埃区域切实巩固与广大发展,红军进一步扩大与进一步精练的形势下,才能给全国政治局面一个大的影响。一方面促进统治阶级内部各派争夺领导权冲突的迅速的爆发,使各派的力量更加削弱,使各派之间的无政府状态更加发展,造成革命的客观形势的有利条件,一方面促进反动统治区域工农兵士贫民斗争情绪的高涨,斗争少质的加深,斗争范围的推广,扩大旧有的红色区域与红军,创造新的红色区域与红军,使革命发展在全国内更加扩大起来,由此去争取一省几省首先胜利直至于发展到全国的胜利。假如我们不能争取这次战争的更大胜利,便不能转变敌我攻守形势。那与我们上述的任务是违背的。所以我们必须努力去争取第二次战争的伟大的胜利,哪一个对于争取这一次胜利不努力,哪一个就是革命的罪人。

(二)目前敌我形势和我们二次战争的胜利条件。

龙冈东韶战争以后的政治形势:在反动统治方面,军阀战争的危机一天一天继续发展。如最近山西编制问题的就绪,张学良对于××旧部的拉拢,北方七省财政会议的召集,蒋介石极力拉拢黄绍纯解决两广问题,召集军官讲话团结干部,派人到各省拉拢小军阀,四川云南等省小军阀的冲突不得解决等等,都是蒋张冲突不可避免的要到来的证据。另一方面蒋介石对苏区和红军第一次的进攻在全国范围内部是失败。第一军在鄂豫皖边界有胜无败,最近打下黄梅等处,蒋介石派七师兵力去打,无可奈何。贺龙等领导的第四军团在湘西鄂边活动蒋介石也曾派四师兵力打过,也奈何他不得。赣东北的第十军湘鄂边的第十八军很活动,十六军曾在通城解决敌人整个一团,缴枪千枝。闽西龙岩虽失,红军现又向汀州发展,打了胜仗。广西第七军驰突于桂湘粤三省之交,最近到上犹、崇义,不日既可和我们汇合。至于我第一方面军第一次战争的伟大胜利,更是影响全国,给反动阶级一大威吓,给各地工农以一大兴奋。据近日上海来人说,上海工人曾起来热烈庆祝这次胜利。至于江西红色区域,除一些城池被敌人占领,接近这些城池的地方有一部分群众反水之外,有些地方还有发展。特别是广昌、石城、宁都、三都、七保以至永丰、乐安、南丰等地,新争取了几十万群众,都在不怕疲劳的与敌人不断斗争。所以上面这些,一方面证明军阀的冲突策略,一方面又证明革命势力的顽强。这样使得蒋介石对于南方各省革命势力的镇压,尤其对于第一方面军二次进攻的企图不得不更为积极些。

蒋介石在江西的兵力,除原有的公、罗、蒋、蔡、毛、许等部外,虽然增加了王金珏,孙连仲两部,数目较前加多,但后方交通须要许多兵力维持;红军第七军来赣,又要抽兵对付,因此实际用上来的兵力得不能超过九师,仍较上次兵力,差不很多。加以张谭惨败兵必较前更加动摇,赤区进兵较前更加困难。敌人方面的弱点,原来是很多的,我们方而虽然在数量上比不上敌军,但我们有一个最大的优胜条件,就是军心团结,大家磨拳擦掌只想打仗。第二、龙冈战争之后,我军加强了实力(枪弹政治军事训练等)。第三、增加了地方武装,武装了广大群众,统一了地方武装的指挥。第四、富田事变的解决,争回了被欺骗的群众。第五、争取了三都、七保这个最反动的区域,扩大了宁都、石城、广昌、南丰、永丰等地的群众斗争,肃清了部分的地主武装,巩固了战争的后方。第六、红军第七军来到江西,给群众一个兴奋,给二次战争一个帮助。所以第二决战争的胜利,是必然归于我们的,只要我们大家下决心,努力的去争取。

(三)逼在目前的第二次战争与我们的准备工作。

目前敌人已开始移动,南丰增加四团敌人,王金珏已到南昌,设督办署,王部有到南城说,郝梦龄部到达吉水,吉安有部队增加,并极力拉夫。二次战争的爆发,已一天一天逼近了。加强准备二次战争的一切工作,是争取二次胜利的必要条件,除关于军事训练工作已有参谋部指示外,本部特将政治方面的准备工作分为红军与地方两个方面指示如下:

第一,红军的准备工作,

(1)以军为单位召集政治工作人员会议,除政治委员、政治部科长以上人员到会外,宣传队长、士兵会主席均应到会。会的主要内容是报告政治形势、二次战争的意义和军中宣传鼓动计划。报告之后,须充分讨论注意纠正各种不正确的见解与实际宣传鼓动的执行方法.

(2)。以师为单位,召集士兵大会,主要是报告政治形势与二次战争的意义,对争取二次胜利作一大鼓动。讲话的人除师长、师政治委员之外,军长、军政治委员、军政治部主任要分别出席。

(3)指导士兵会开士兵会,主要是提出争取二次胜利的各种问题来讨论,使士兵对于政治方面的问题,能进一步的了解,对于准备的实际工作促使士兵自觉的做起来。

(4)以军为单位,在大部队集中后开誓师大会。部队之外要发动当地群众参加,主要意义是鼓动士兵群众,提高士兵群众的作战勇气。在大会中讲话的内容和技术,要简短有鼓动性,讲话中要插入少数口号,鼓起全场的热气;并须有准备的推动士兵提出一些关于作战方面的问题如连坐法等,重复在大会上通过。

(5)在群众会、士兵会、誓师会中都要讲到军事方面的问题(如作战注意,作战经验,报告各部队的特殊优点与缺点等)。

(6)组织并鼓励作战附近地区的群众,作各种参战及鼓动红军的工作。如举行赤色戒严,派代表团对红军慰劳,组织担架队,欢送红军作战,送茶水、粥饭等。

(7)政治部应即进行对俘虏的宣传工作及宣传品的准备,办到每个俘虏有一种好的宣传品。

第二,地方政府与工农群众的准备工作:

(1)以县为单位召集各级政府主席、游击队长、民众团体负责人的联席会,报告政治形势与二次战争的意义,计划白军前进时坚壁清野及参战的工作。

(2)以区为单位开群众大会,主要是起群众热烈参战。

(3)选举有说话技能、观念比较正确的同志组织地方慰劳红军代表团,到红军中鼓励红红军作战。来时最好有布置:红旗上面用文字写鼓励红军的口号,写明送给某军,某某地方工农群众或某团体赠送;同时有宣传单告红军战士,能带些慰劳品更好。

(4)准备敌人经过地方的宣传工作,各村各乡应满写着宣传敌人士兵的标语,散布许多简单明了(如几句口号凑拢去)的传单。但口号条数,不要多,到处多写相同的几个主要口号。

(5)各乡村组织士兵运动委员会,做白军士兵运动。

(6)动员群众准备欢送白军俘虏的欢送工作。如俘虏经过的地方政府应招待饭餐,杀猪他们吃沿途送茶水饭粥,高呼欢送新同志的口号,由群众送食物给俘虏,同时,对他们宣传鼓动。

(7)地方武装部队参照红军政治鼓动工作准备。

二次作战中红军及地方游突队早晚点名口号:

一、勇敢冲锋!

二、拚命杀敌!

三、拥护共产党!

四、拥护苏维埃!

五、活捉何应钦!

六、打倒蒋介石!

七、二次战争胜利万岁!

八、工农解放万岁!

对白军宣传鼓动口号:

一、白军弟兄是工农出身不要拿枪打工农!

二、白军士兵要发清欠饷只有暴动起来!

三、白军弟兄杀死反革命的官长自己选出官长成立红军!

四、白军士兵不上前线打仗不替军阀当炮灰!

五、白军士兵出身不替军阀屠杀工农!

六、欢迎白军士兵下级官长打土豪分田地!

七、优待白军俘虏兵!

八、医治白军伤病员!

九、欢迎白军弟兄下级官长来当红军!

十、穷人不打穷人,

十一、士兵不打士兵,

十二、白军是军阀的军队,红军是工农的军队!


主任毛泽东

(抄自中国革命军事博物馆)



