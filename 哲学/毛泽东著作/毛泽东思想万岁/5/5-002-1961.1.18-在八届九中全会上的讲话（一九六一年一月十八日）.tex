\section[在八届九中全会上的讲话(一九六一年一月十八日)]{在八届九中全会上的讲话}
\datesubtitle{(一九六一年一月十八日)\footnote{据《毛泽东年谱》,1月18日 在中南海怀仁堂主持中共八届九中全会全体会议}}

这次会议,因为经过二十天的工作会议的准备,开得比较顺利。今天想讲一讲工作会议上讲过的调查研究问题,别的问题也讲一下。

我们在反帝反封建的民主革命时期,提倡调查研究,那时全党调查研究工作作风比较好,解放后十一年来就较差了。什么原因?要进行分析。在民主革命时期,犯过几次路线错误,在解放后又出过高岗路线。右的不搞调查研究,“左”的也不搞调查研究。那时,中国是什么情况,应采取什么战略方针和策略方针才适合中国实际情况,长时期没有得到解决。自从我们党一九二一年成立起,到一九三五年遵义会议,十四年间,有正确的时候,也有错误的时候。大革命遭到了损失,第二次国内革命战争也遭到了损失,长征损失也很大,在遵义会议后到了延安,我们党经过了整风,七大时,……王明路线基本上克服了。抗战八年我们积蓄了力量,因此在一九四九年,我们取得了全国革命的胜利,夺取了政权。解放战争时期,我们和蒋介石作战,情况比较清楚,比较注意搞调查研究,对革命一套比较熟悉,那时情况也比较单纯。胜利后有了全国政权,几亿人口情况比较复杂了。我们过去有过几次错误,陈独秀机会主义错误,立三路线的错误,王明路线的错误等等,有了几个比较,几个反复,容易教育全党。近几年来,我们也进行了一些调查研究,但比较少,情况不甚了解。譬如农村中,地主阶级复辟问题,不是我们有意识给他挂上这笔账,而是事实是这样,他们打着共产党的旗子,实际上搞地主阶级复辟。在出了乱子以后,我们才逐步认识在农村中的阶级斗争是地主阶级复辟。凡是三类社、队,大体都是与反革命有关系,这里边也有死官僚。死官僚实际上是帮助了反革命,帮助了敌人,是地、富、反、坏、蜕化变质分子的同盟军。因为死官僚不关心人民生活,不管主观愿望如何,实际上帮助了敌人,是反革命的同盟军。还有一部分是糊涂人,不懂得什么叫三级所有,队为基础,不懂得共产风刮不得。反革命、坏分子、蜕化变质分子就利用死官僚、糊涂人把坏事做尽。一九五九年,有一个省,本来只有××××亿斤粮食,硬说有××××亿斤,估得高,报得高。出现了四高,就是高指标、高估产、高征购、高用粮,直到去年北戴河工作会议,才把情况摸清楚。现在事情又走到了反面,是搞低标准,瓜菜代。经过调查研究,从不实际走到比较合乎实际。

农、轻、重,工农业并举,两条腿走路,我讲了五年,庐山会议也讲了,但去年没有实行。看来今年可能实行,我只说可能实行,因为现在还没有兑现。一九六一年国民经济计划已经反映了这一点,注意了农、轻、重,就可能变成现实。

对地主的复辟,我们也缺少调查研究。我们进城了,对城市反革命分子比较注意,比较有底。一九五六年匈牙利事件以后,我们让他们分散的大鸣大放,出了几万个小匈牙利。这样把情况弄清楚了,就进行了反右斗争。整出了×××万个右派,搞得比较好,底摸清了,决心就转大了。农村那年也整了一下,没有料到地主阶级复辟问题。当然,抽象的讲是料到的。过去我们总是提出国内矛盾是资产阶级和无产阶级的矛盾,是社会主义与资本主义两条道路的矛盾,基本矛盾是阶级矛盾。是资本主义的天下,还是社会主义的天下,是地主的天下,还是人民的天下?

没有调查研究,情况不明,决心就不大。一九五九年就反对刮共产风,由于情况不明,决心就不大,中间又加上了一个庐山会议,反右倾机会主义。本来庐山会议要纠正“左”倾错误,总结工作,可是被右倾机会主义进攻打断了,反右是非反不可的。会后,共产风又刮起来了,急于过渡,搞了几个大办。大办社有经济、大办水利、大办养猪、大办县社企业、大办土铁路。同时要这么些个大办,如养猪什么也不给,这就刮起共产风来了。当然大办水利、大办工业取得了很多的成绩,不可抹煞。还有大办文化、大办教育、大办卫生等等,不考虑能不能做。共产风问题,反革命复辟问题,死官僚问题,糊涂人问题,干部情况问题,县、社、队分为一、二、三类,各占百分之××、百分之××、百分之××问题,这些问题以前我们就没有搞清楚,有的摸了,我们也没有讲清楚,或讲清楚了也不灵。郑州会议反共产风,只灵了六个月,庐山会议后冬天又刮起共产风。庐山会议前,“左”的情况还没有搞清楚,党内又从右边刮来一股风。彭德怀等人与国际修正主义分子、国内右派相呼应,打乱了我们纠“左”步骤。

去年一年国际情况比较清楚,对国内问题也应该聚精会神调查研究,工人阶级要团结农民大多数,首先是贫农、下中农和较好的中农,依靠他们对付地主反革命。三类社、队要成立贫下中农委员会,在党的领导下,主持整风整社,并临时代行社、队管理委员会的职权。我们党内也有代表地主,资产阶级,小资产阶级的人,应该纯洁党的组织,经过整风、整顿组织,使党纯洁起来,使绝大多数党员都代表贫农、下中农的利益,同时也不损害富裕中农的利益,坚持不剥夺农民利益的马克思列宁主义原则。刮共产风是非常错误的,是剥夺农民的,是反马克思列宁主义的,必须坚决退赔。经验证明,只要退赔,群众就满意了,情况就改善了。

这次工业计划比较切合实际,缩短了基本建设战线,延长了农业、轻工业战线。与农业有关的基本建设还要搞,有的重工业,像煤、木材、矿石、铁路还要搞。上下一本找账,不搞两本账。不要层层加码。总之,要实事求是,使一切从实际出发。粮食要过秤入库,不搞四高,搞低标准,瓜菜代,坚决退赔,整顿五风,不准不赔,不准不退。

城市也要整风,正在搞试点,还要一、二个月才能搞出来,也要搞十二条。

今年计划看来比去年高不了多少。有人建议钢仍然搞×××××××万吨到××××万吨,也增加不了多少;这个提法有道理。第二个五年计划钢的指标,早己超额完成,还剩两年,就是要搞质量、规格、品种,在质量上好好跃进一下,数量上不准备多搞。帝国主义者、修正主义者,会说大跃进垮台了,他们要讲就让他们讲,他们讲坏话也好,讲我们好反而不好。实际上我们现在就是要搞质量、规格、品种,搞企业管理制度、技术措施,提高劳动生产率,降低成本,成龙配套,要搞调整、充实、提高,就是要在这方面努力。英国、日本的钢,暂时还比我们的多,再有×年,我们总会赶上他们,并且还会超过他们。能否超过西德,还要看一看。讲打仗,斗地主,我们有一套经验,搞建设还比较缺乏经验,我与斯诺谈话就谈到这一点。凡是规律总要经过几次反复才能找到,我们只希望不要像民主革命花二十八年才成功。其实二十八年也不算很长,许多国家的党同我们同年产生,现在也还没有成功。搞建设是不是可以二十年取得验验,我们搞了十一年,看再有九年行不行。曾想缩短很多,看来不行。凡是没有被认识的东西,你就没法改造它。

工业还是要鼓干劲,不然几次会议一开,劲就没有了。泄了二、三个月的气,然后再开一次鼓干劲的会,反右倾。大家回去以后,要实实在在的干,不要老算账。搞计划要好好调查研究,搞清情况,鼓足干劲,力争上游,多快好省,坚持总路线。有人说现在不用多快了,这不对,搞粮食就要多快嘛,搞工业讲质量、成龙配套等等,也是要搞多快嘛!

团结问题。中央委员会的团结是全党团结的核心。庐山会议有少数人闹不团结,我们希望和他们团结,不管他们的错误有多大,只要他们能改。他们讲你们也有错误,不错,错误人人皆有,但错误大小轻重不同,性质不同,数量质量不同。不要一犯错误就抬不起头来。有的同志工作职位降低了,降低了也好。一年来有进步,不管真假,总是值得欢迎。地方工作的同志有的也犯了错误,欢迎他们改正。

河南、甘肃、山东三省问题比较严重,情况不明,决心不大,方法不对。现在情况明了就好了。有些地方政权也夺回来了,面貌已经开始一新。甘肃也开始好转,其他各省也总要烂掉若干县、社、队,大体是百分之××左右,严重的超过百分之××,好的不到百分之××。不光是因为粮食问题。林彪同志讲,军队有××个单位,烂掉了××个,占百分之×,这并不是因为粮食问题。这种情况在城市、工厂、学校一定会有。对××类干部,要按政策清洗出去,死官僚要改造,变成活官僚,长久活不起来的也要清洗。这些人是少数,合起来也不过百分之几,百分之九十以上是好人,其中也有糊涂人。我们也糊涂过,不然在民主革命时期,大革命为什么失败,南方根据地丧失,白区力量丧失,要长征,是因为不了解情况。

现在搞社会主义建设,是个新问题,我们缺少经验,要开训练班,把县、社、队干部轮训一遍,使他们懂得政策。如果一个省只有一个县委书记能讲清政策,不训练干部怎么行?每个县都要有一个县委书记真正能懂政策,弄清政策,那就好了。现在中央下放了八千多个干部帮助农村整风整社。大多数农村干部是好的,可靠的。如果大多数是国民党,我们还能在这里安心开会吗?所有一切可团结的人要团结。就是对反革命分子也不能都杀,不杀不足以平民愤的才杀,有的要关起来,管起来。杀人要谨慎,切不可重复过去所犯过的错误,如过去搞根据地时杀人多了一些。延安时规定一条,干部一个不杀。现在还关了一个潘汉年,绝对不杀。杀了就要比,这个杀了,那个杀不杀?总是不开杀戒。但是不是说社会上一个不杀?有些不杀不足以平民愤的人,民愤很大的人,不能不杀几个。至于中央委员犯了错误就不牵涉杀不杀的问题,还是留在中央委员会工作。要与各兄弟党团结,要和苏联的党团结,要和八十一个共产党和工人党团结,我们要采取团结的方针。

过去我们吃了亏,就是不注意调查研究,只讲普遍真理。六一年要成为调查研究年,在实践中调查研究,专门进行调查研究。
