\section{在广州会议上的讲话(节录)(一九六一年三月)}


主要是两个平均主义的问题。一个是生产大队(即原来的生产队)内部,各生产队(即原来的生产小队)与生产队之间的平均主义,一个是生产队内部,社员与社员之间的平均主义。这两个问题很大,不彻底解决,不可能真正地全部地调动群众的积极性。

人民公社是生产大队的联合组织。

公社、生产大队不能瞎指挥,县、地、省、中央也不能瞎指挥。

不能用领导工业的办法来领导农业,也不能用领导农业的方法来领导工业。

<p align="right">(三月二十三日)</p>

几年来出的问题,大体上都是因为胸中无数,情况不明,政策就不对,决心就不大,方法也就不对头。最近几年吃情况的亏很大,付出的代价很大。

要作系统的由历史到现状的调查研究,省、地、县、社的第一书记,都要亲自动手。不做好调查工作,一切工作都无法做好。第一书记亲身调查很重要,足以影响全局。今后我们必须摆脱一部分事务工作,交别人去做。报告也要看,但是不要满足于看报告。最重要的是亲自作典型调查,走马观花只能是辅助的方法。

只有原理原则,没有具体政策不能解决问题。没有调查研究,就不能产生正确的具体政策。

调查研究的态度,不可以先入为主,不可以自以为是,不可以老爷式的,决不可以当钦差大臣。而要讨论式的,同志式的,商(量)的。

不要怕听不同的意见,原来的判断和决定,经过实际检验,是不对的,也不要怕推翻。


