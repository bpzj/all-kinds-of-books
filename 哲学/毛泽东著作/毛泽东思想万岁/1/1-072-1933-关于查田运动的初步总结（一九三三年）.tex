\section[关于查田运动的初步总结(一九三三年)]{关于查田运动的初步总结}
\datesubtitle{(一九三三年)}


一、伟大的胜利

查田运动在党与中央政府的号召之下,已经广泛地开展起来了。如果说,查田运动过去还仅仅在开始的阶段上,那末在六月的八县查田大会后,七月一个月的工作,便已超过了去冬以来大半年中所做的成绩。一般说,在开会的八个县中,查田运动已经进入了一个新形势。查田运动已经成了一个广大的群众运动。瑞金与博生成绩最大,两县共查出了二千八百家地主、富农;胜利、零都、会昌、汀乐、长汀、五城、宁化各县,亦都有了初步成绩,在一切查田有成绩的区乡,广大的群众斗争发动了,苏维埃工作中,党的工作中许多过去停顿着的状态现在都活跃起来了。苏维埃中的坏分子许多被洗刷出去了.暗藏在农村中的反革命分子,遭到严厉的镇压。一句话,封建残余势力.在广大群众面前遭受了惨败。在这个基础上,各种工作更加开展了。在查田有成绩的区域,扩大红军与扩大地方武装,推销经济建设公债与发展合作社,秋收秋耕与发展劳动互助社。以及俱乐部夜校小学文化建设事业,都得到了极大的成绩。一切工作进行更加顺利了。在群众活跃的基础上,大批积极分子自己创造成为各种工作干部,许多工农积极分子加入了党,被吸收到苏维埃工作中来,最好的例子是瑞金的壬田区,壬田区的查田运动在中央政府工作团帮助之下,五十五天中发动了全区的群众,彻底消灭了封建残余,查出了地主富农三百家,枪决了群众所谓“大老虎”的十二个反革命分子,镇压了反革命活动。在群众面前检举了苏维埃工作人员中犯了严重错误的一些分子,清洗了一些混进苏维埃里来的阶级异己分子出去。全区查出了土地二万七千担,全区二万余劳苦群众差不多平均每人重新得了一担二斗咎土地,分配了豪绅地主的无数财物给予群众。依靠于群众积极性的空前提高,五十五天中扩大红军七百余人进瑞金模范师去,没有一个开小差,节省咎予卖给红军,达到一千九百余担,全县没有任何区比得上它,在各乡的要求下,全区担任推销经济建设公债四万元,地主罚款富农捐款已收得七千五百元,承认继续去筹备一万元。合作社迅速地发展了,文化教育建设如俱乐部,识字班,夜学也增加了,党员数量扩大,党的领导加强了,工会工作也进步了。全区另换了一种新气象,由瑞金的一个落后区在五十五天中换了地位,成了与武阳区相等的一等区了。我们现在要问:“壬田区为什么得到这样大的成绩呢?那我们应该指出,由于他们认识了查田任务的重要,由于他们的动员方式,阶级路线与群众工作,都是坚决执行了中央局与中央政府的正确指示,他们做了真正布尔什维克的工作,在任务的认识上,他们懂得查田运动与革命战争有密切联系,因此他们认真的做这工作,抓紧了查田运动的领导,有计划的去布置当地的工作,在动员方式上,他们在全区十一个乡中,抓紧了最落后和比较落后的七个乡的工作,在这些乡中动员了党,动员了团,动员了乡苏,动员了工会与贫农团及其他群众团体,经过他们去动员广大的群众。在阶级路线上,坚决执行了依靠雇农贫农与联合中农的策略,广大的发展了几个中农他们说了查田不是分田,查阶级不是查中农贫农雇农阶级,他们“讲阶级”的工作做得很充分。当着鹅凤乡的地主,恐吓中农,使部分中农发生恐慌的时候,我们的同志找了几个中农来作个别谈话,经过他们去传达其它中农,鹅凤乡的中农立即稳定起来,接着积极拥护苏维埃的查田政策,从受地主、富农的欺骗转到与贫农工人一致进攻地主富农,壬田乡的同志亦曾错误地处置了几家成分,但他们迅速地改正了错误,关于领导群众斗争的工作,首先他们做了广泛的宣传,不是开的全区全乡会,他们开的是村子屋子会。这样去接近更广大群众,向他们做了多少次的宣传鼓动,所以全区群众都明了查田查阶级是自己的责任,自己的利益,其次调查阶级的成分,发动了多数人去查,详细搜查了各个成分的材料证明给群众看,所以没有发生群众不满意的事,其次通过阶级成分,是首先通过了查田委员会的分析决定,提交贫农团讨论通过,送达苏区批准再到被没收人的村子里召集本村群众大会,解释明白,举手赞成,然后举行没收,在分配财物与土地上,工作人员都能了解自己应做模范,不拿东西,而把东西完全分给群众,没收的村子多分,其它的村子少分。得到了群众的完全满意。土地也迅速地分配了。别地拖延很久,不分的现象,壬田区是没有的,所以迅速地发动了群众。他们打进落后大村子的办法也是正确的,他们不是畏惧这些大村子,他们也不用蛮干的办法来应付,他们对大村子是集中了火力,做了更多的宣传,从争取当地的积极分子着手,团结他们,教育他们。经过他们去动员其它的群众,他们很耐心地去做这种村子里的工作,表面上看是迟慢的,但实际上是迅速的,他们在五十五天中把壬田区所有的落后大村子一概发动了,很短时间消灭了这些村里的落后,抗柏坑乡的一个村子打不进去,原来那里有两个著名的“大老虎”一向在作怪,他们就釆取不同的办法,首先捉到这两个坏东西在当地开巡回法庭审判他们,经过群众热烈拥护枪决他们,那里的群众斗争就象烈火一样燃烧起来了。它们开了十次群众审判大会,三次巡回法庭都是经过了极广泛的群众路线,本乡的人都到,别乡的人每村派代表,小乡十几人,大乡四五十人,所以每一次公审与裁判的结果,都立即传播到了全区各乡各村去,不但使各个乡村的群众都觉得今天审判的这个人该罚该杀,而且使他们立即想到自己地方那个同样作恶分子也该处治他。壬田区的查田运动真算得全苏区的模范!瑞金九堡区的查田运动工作也有极大成绩,他们首先是抓紧三个乡去做,召集各乡的查田委员会委员,到苏区开了三天训练班,讲明了动员方式,阶级路线与争取群众的方法,他们在没收分配问题上创造了一个好办法,他们的办法是:要没收一家地主了,就号召本村本屋的群众一同去,在群众大会上选举了没收分配委员会,在群众的监视下进行收费,没收的东西堆在一个大坪上,再经过群众同意立即分配给应得东西的群众。吃得的东西又是一个处置,就是杀猪煮饭群众大会吃一顿。这个办法在九堡区收了极大的成功,他们没收分配委员会不是经常组织的,而是临时组织的,更加密切联系了群众(关于土地的没收分配,仍有经常的土地委员会负责)。一切东西不挑到乡苏去。不得集中了若干家然后分配,免去了拖延时日与被别人偷去了的毛病,九堡区在其他的路线上方法上,一般也是正确的,所以能够发动广大群众自己动手查阶级。他们办到了没有一个地主富农到苏维埃来闹,成分与过去查田时的情况完全相反,过去总有许多被查的地主、富农到苏维埃来横闹,说把他们查错了,不但本人,甚至有时有个别乡代表贫农团员负责人也有来替他们求请讨保的,这次当然不是地主富农不闹了,而是闹也闹不起来了,他们的同宗,他们的亲戚,没有一个人赞助他们,地主富农软了劲,没有闹的可能了,这件事说明了九堡区发动群众的工作做得极充分,不然是办不到这个地步的。

所有这些光荣模范的例子(这种例子别的地方还有不少),给了党与中央政府的号召以布尔什维克的回答,证明中央局与中央政府的指示的绝对正确性,那一处完全执行了这些指示,那里就立即取得伟大胜利,但是谁违背了这些指示,忽视了这些指示,那里的工作就犯了错误,没有成绩。或者成绩很微弱,让我再拿事实来证明。

二、有些地方放弃查田运动的领导

查田运动的斗争任务,在中央局查田决议发出之后,在中央政府训令及召集八县查田大会之后,查田运动在各县的开展,并没有普及到一切地方。这如福建全省查田的成绩,还只当得博生一县的成绩;胜利、雩都、会昌、石城每县的成绩,还只当得瑞金最好一个区至两个区的成绩;各地有许多区的查田委员会没有开过一次会,甚至县查田委员会,亦还有几个县没有去抓紧全县的查田工作(会昌、雩都、石城,宁化);许多区与乡的查田委员会,区乡主席不做主任,借口别的事忙,放弃查田不管。党对查田的领导,在一切查田有成绩的地方都明显地表示出党的坚强领导作用,党员群众的大多数,在支部与区委的领导下做了很多勇敢战斗的工作。但在一切查田没有成绩及成绩微弱的地方,就表示出党部忽视查田运动,这与会昌县委在中央局查田决议发出后,差不多两个月没有讨论过一次查田工作,直到七月底才开了一次会讨论查田;瑞金的下属区委一个时期中,对查田运动完全不管;瑞金市委里在查田开过一次会,却没有推动四郊支部去注意查田的领导,各个支部没有为了查田运动开过会的。在别的地方)如雩都、胜利、石城、宁化县委与许多区委,同样没有用大力去注意查田工作。党与中央政府说:“查田运动成为发动群众深入农村中的阶级斗争,彻底解决土地问题与肃清封建半封建的有力方法!(中央局决议)。查田运动是各地苏维埃一刻不容再缓的任务(中央政府训令)。查田运动是目前工作中心最重要的一环(八县大会总结)”。我们的许多同志却在说:“忙得很,没有工夫照顾查田运动。”党的决议说:“一切以官僚主义与形式主义的敷衍态度来对付查田运动,是最有害的。”这些同志却还是以官僚主义,形式主义来对待查田运动。

三、有些地方竟对地主富农投降

在查田运动开展了的地方,也还表现许多个别的仍是严重的错误,那就是这些地方党部中苏维埃中还常常有个别的同志,在查田运动这个激烈的阶级斗争面前,表现了他们机会主义的动摇,这主要在当着查田运动激烈发展的时候,他们丢弃不了姓旗与地方的关系,包庇同姓同村的地主富农成分。或者错误的分析阶级成分,把地主当富农,把富农当中农,有些裁判部的工作同志,在他们的极端疏忽接受了地主富农假冒群众名义对于查田积极分子的诬告,另一方面我们有些保卫与裁判部的同志。又没有跟着群众查阶级斗争的开始,去积极镇压反革命,甚至当着群众请求捉拿与枪决抵抗查田运动的地主富农分子时,还有不接受群众请求的。瑞金的裁判部就是做了许多这样错误的例子。

四、侵犯中农的倾向是最严重的危险

“左”的机会主义倾向,在七月查田运动中又在很多地方发生了。这里应该着重指出的就是侵犯中农的倾向,虽然在中央局的决议上早已明白的写着:“必须特别注意与中农群众的联盟,中农是革命后苏维埃农村中最广大的基本群众,一切我们的处置与策略,必须获得他们的赞助与拥护。每一个贫农团与苏维埃的决定,必须是在一村或一屋的群众会议上得到中农群众的拥护,一切中农群众呼声,必须注意听,并须严厉的打击任何侵犯中农利益的企图。”在八县大会的决议上指出:“查田的目标是查阶级而不是再分田。”“联合中农应从不侵犯中农的利益做起。”“在查田的开始,应普遍宣传苏维埃联合中农不侵犯中农的政策。”在查田过程中,应审慎决定。介在中农与富农之间的疑似成分,不使弄错。但是这样的指示,并没有为许多地方的同志所注意。瑞金城区查田,一起始即按亩去查,查得中波恐慌。竟有中农跑到苏维埃来请求改变自己成份,他们请求改变为贫农。他们说:“中农危险得很,推上就是富农,改为贫农咧,隔富农就远一点。”这样沉痛的呼声,还不值得我们倾听吗?黄柏区浑古乡的同志向群众说:“查阶级不查别的,只查中农富农地主阶级。”踏逞区的同志,插起牌子遍查,查得一部分中农恐慌逃躲到山上。博生县的某些乡中,同样是有插牌子的意见,是不对的。但全未为这些同志所注意。这种插牌子遍查的方法,每县都有发生,这是异常严重的情形。他们把查田与分田混合了。不错,分田是应该插牌子的,是要一丘一丘查清数目,然后拿去分了;但若把这个办法应用到查田运动上来,那就混乱了农村中的斗争目标。过去我们曾经指出:“查田与分田必须严格分别,这种分别,不但为了巩固农民的土地所有权,使他们不起分田不定的恐慌,而且是为了查阶级斗争的胜利,必须集中全力、特别是联合中农,去对付地主富农的反抗,这种时候,决不应在农民自己的队伍内发生任何纷扰。”(八县大会结论)这样的策略,是我们领导查田斗争中整个策略最重要的一部份,可是还为许多同志忽视了。这种忽视,一刻也不能再忍耐下去。那些经过指导还故意在做这些错误的。当地的上级苏维埃须给他以严厉的处罚。要在党内团内开展思想斗争,反对任何党员团员侵犯中农利益,违反联合中农策略的思想与行为。已经做了错事,如已经没收了中农的土地财产的地方,苏维埃人员要向当地中农群众公开承认自己的错误,把土地财产赔还他。去年兴国曾赔还许多中农的土地,取得了中农群众的满意,是一个宝贵的教训。

五、贫农团的关门主义与忽视雇农的领导作用是错误的

“贫农群众是党和无产阶级在农村的支柱,彻底地进行土地革命的坚决拥护者。”“依靠贫农”是我们查田运动及一切土地斗争的重要策略之一,而贫农团则是查田运动中有极大作用的团体。八县贫农团代表大会已经指出过去贫农团的关门主义倾向是错误的,应该废除介绍制,向贫农工人打开大门,一切男女老少的贫农工人均可报名加入。但是许多地力仍然沿着旧办法不改,仍然非有介绍不能进贫农团,甚至在瑞金踏径区当着尚未入会的贫农群众跑来参加贫农团的会议的时候,贫农团的负责人居然拒绝他们参加。博生的竹柞芨区,七月一个月中贫农团没有发展一个人。在一切查田有成绩的区乡,贫农团是广大发展了,而在没有成绩或成绩微弱的区乡,那里的一个表征,就是贫农团的关门主义状态。雇农在查田运动中的伟大领导作用,同样为许多同志所不认识。党的决议说:“雇农群众是城市无产阶级在农村中的兄弟,是土地革命中的先锋队。因此苏维埃的工作人员必须与工会取得密切联系,经过工会发展与组织工人群众的积极性,使他们成为查田运动先锋队。”我们的同志,依照这个指示去做的,仍然不是多数。这里主要方法,是使农村工人加入贫农团,而在其中成立车独工人小组;经过这些工人小组,去团结贫农积极分子,发展贫农团,推动查田运动前进。黄柏区山河乡的经验是可宝贵的,当我们的同志两次召集贫农团,开会不成的时候,就去开了一个农业工会与手艺工会的会员群众联合会,发动了几十个工人积极起来,每人带领一个贫农分子加入贫农团,山河乡的这个经验应该把它运用到一切农村中去。这里工会的上级领导机关,应该给下级工会以积极的指导,要把查田运动看成为工会的重要任务之一。

六、关于富农问题的不正确观点

农村斗争中整个我们的策略,是依靠贫农,坚决的联合中农,并使雇农起先锋队作用,团结所有一切力量,去消灭地主阶级与反对富农。关于富农问题,党已经正确说了:“必须把地主富农分别清楚,在无情的消灭封建残余斗争中,决不容许任何消灭富农的企图。七月查田中,虽然还没有公开主张消灭富农好理论,但是把富农分子当做地主全部没收了他的家产的,就已经在许多地方发现了。这一错误的来源,是由于抹煞富农的劳动力,当着我们说:“没有劳动或只有附带劳动而有地租等等剥削的是地主。”的时候,有些地方就把在生产中用了相当多的劳动分子,认为是“附带劳动”一类,把他当地主看待;有些地方则把富农兼有高利贷剥削的认为是“高利贷者”而照着“消灭高利贷者”的办法去对付这种富农;有些算陈账,算到革命前若干年上去,一人在革命前五六年甚至十几年前请过长工的,也把他当做富农,或者仅仅只请过一年两年长工的而前后没有请过的富裕中农分子,也放在富农一类。更严重的是过去兴国某地方的例子,那里的办法,拿剥削的种数去分别地主与富农的成分,三种剥削的叫地主,两种剥削的叫富农,比如请了长工,收了租,又放了债,则不管他家里有几个人劳动,总之他家里就是地主了。还有“反动富农”这个问题,在许多地方弄得颇糊涂。武阳区一家富农兼商人,七个人吃饭;因过去他家有一人参加AB团,在两年前被杀了,两年之后那里同志一定要全家没收他。在别的许多地方,同样发生许多这样的事,把富农在暴动前不十分严重的反革命行为如参加“收三成租谷”之类(瑞金),暴动后几年他也没有做过反革命的活动,群众的多数对他不要惩办,而我们有些则一定要没收他。正确的说,我们对待这类分子的办法,在巩固了的区域与尚未巩固的边区,应该有策略上的不同。在边区,无疑要釆用严厉办法,镇压一切包括富农在内的反革命分子,在中心区,则要分别情形决定,暴动前有过严重的反革命行为,或暴动后还在做反革命活动的,自然坚决的没收他,否则不应该没收。有些一家中只没收他本人以及同他反革命行为直接关系的分子,其他则不没收。这样处置才是正确处置。

七、工农检察部没有负起自己应有的责任有些并且做出了错误

我们的工农检察部的同志,许多不认识查田运动开展中正是开展思想斗争,反对官僚主义,反对贪污腐化消极怠工,驱逐阶级异己分子出苏维埃去的最好时机,因此,对于这些工作做得异常不够。许多我们工农检察部的同志,在这个剧烈的阶级斗争面前,表现了自己的消极动摇。自己的官僚主义与形式主义。甚至如瑞金市苏工农检察科长,包庇市苏裁判科长的极大贪污行为(私用公款千余元),工农检察科长对于区苏主席放弃查田不管,也不去批评他与检举他。工农检察部的渡头区的检举运动,有的地方又走到了别的错误方向。把恋爱问题当作腐化,把拿了地主的东西当作贪污,对于这样的分子进行检举甚至公审,有些地方把犯了轻微错误的开除职务。不去有系统的发动自我批评、开展思想斗争,把这种艰苦的工作省去,而代之以简单的惩办主义。不消说,苏维埃工作人员中那些犯了长期与严重错误的分子是应该坚决洗刷出去的;但是错误不到这种程度的也给予撤职处分,则是过分了。关于阶级异己分子的问题,普遍的只讲成分,不讲工作,只要是出身坏,不管他有怎样长久的斗争历史,过去与现在怎样正确执行党与苏维埃的路线政策,一律叫做阶级异己分子,开除出去了事。完全不错,我们要坚决洗刷那些阶级异己分子,那些成分坏又加工作坏的(包括地主富农,消极怠工、贪污腐化等等),他无疑应该洗刷干净;但如果不是这样也把他洗刷出去,那就过分的了。

八、关于查田斗争的领导艺术

了解了任务与路线,但没有群众斗争的布尔什维克的艺术,查田运动是仍然不能开展的。.本文的开头已经说了壬田区等处许多很好领导斗争的例子,但在另外许多地方却在这个问题上犯了不少的错误。有些地方对于地主富农集中的落后大村子,不知道用各种方法去发动斗争,即如有种大村子只有首先捉拿著名凶恶即群众所谓“大老虎”的那些豪绅地主分子,才能开展当地的斗争,我们的同志却没有这样去做。关于用分配没收来的财物去发动斗争这个最好的方法,还有许多不知道采用。在瑞金踏径区的瓦子乡,甚至把东西只发给查田干部与贫农田会员,其他不发,理由是自己不积极的不应分东西。有些地方没收的东西分配得很慢,甚至没收了一个多月还未发给群众。没收的土地分配得更慢,不少地方的同志不知道动员苏维埃的各部,动员各个群众团结,不知道动员所有的党员团员,在各个群众团体中,各个村屋中,去起核心领导作用;工作推不动,就说这里本来没有办法的。有些地方当着群众查阶级的斗争已经发动起来了,许多群众都来报告地主富农,请求去查田没收时,我们的同志不能立即抓紧群众这个热潮去领导群众开展斗争,把查阶级工作开展到各个村子里去。有些地方,则在查了一番之后,群众的斗争热情不能继续向上高涨,表现了停顿状态的时候,我们的同志不能用各种方法鼓励群众,使斗争继续高涨,一直领导到消灭封建残余的地步。许多地方当着经过了查田运动,群众斗争热情蓬勃发展起来了的时候,不知道把这种热情组织到别的战线上去,比如当着群众得了东西,得了土地的时候,即不在那时的群众大会上,或在其它一切有利的时机,鼓动群众去当红军,去买公债票,去进合作社;鼓动群众去加紧秋收,秋耕的工作,去建立俱乐部,识字班,发展夜学与小学。把这种好的时机放过去,另等上级对于这些工作督促来了,才又重新开头去做宣传鼓动。这种落在群众斗争热情之后的尾巴主义的领导,乃是最有害于革命工作的。


另一方面,在许多别的地方,又发生了少数人蛮干的恶劣现象。我们曾经着重的指出:要反对对于争取大多数群众的忽视与命令主义的工作方式。只有耐心的艰苦的去作发动群众争取群众的工作,才能取得大多数群众的拥护,达到消灭封建残余的目的。这样的群众工作,是执行阶级路线的唯一保障。查田运动的开始,必须在一切村屋中做广大的宣传,向群众说明查田这动的必要,说明查田不是分田,查阶级不是查中农贫农雇农的阶级,特别重要的是把什么叫做地主富农中农,向群众分析清楚。村屋的群众大会应该不只开一次,特别是那些落后的村屋应该多去开几次,应该不让一个劳动分子不听到我们的宣传。为了达到这个目的,应该首先在乡代表会、工会、贫农团、女工农妇代表会,及其它群众团体中,向一切积极分子解释明白,通过他们去向广大群众作宣传。查阶级应该不是少数人去查,要发动多人数去查。通过阶级成分不但在贫农团、乡苏、区苏通过,而且要征被没收人的村屋中开群众大会,取得群众的同意,才能进行没收,分发财物,要发给本村本屋群众,取得本村本屋群众的满意。所以要这样,都是为了争取群众的大多数。党与苏维埃工作人员每个时候,每件工作,都不要忘记群众大多数。我们的工作要深入群众,要深入一切大的、小的村庄,一切大的小的市镇里的群众。要严厉反对少数人干的关门主义,命令主义的错误办法。可是我们的许多地方并不是这样去做的,在瑞金,许多区乡中没有向群众讲过一次阶级(没有分析过什么是地主富农中农)。在瑞金及别的县的许多地方,都发现不经过宣传就动手去查,以致地主跑出来造谣,欺骗中农,说什么本乡地主富农多得很,或者说本乡有几百家地主富农要查(瑞金);而我们的同志,还不知用明白的“讲阶级”办法,去识破这种造谣。广昌的地主说:“查田运动是中央政府要过去欠债的人把债还给中央。”我们同志没有立即去揭穿这种鬼话。在踏径区的几个乡中,除了不做宣传之外,查阶级只是查用委员会的几个查,通过阶级不但不经过群众大会,连贫农团也不经过,他们说:“群众靠不住,群众不会分析阶级,阶级成分提到群众大会去通过是要发生纠纷的,还是只由委员会通过是靠得住的。”这个踏径区同志的理论,真算得是天下奇闻,好几个地方没收地主时不在白天而在晚上,唯一的理由是怕地主跑掉了。有一处地方开群众大会,用了“全区大会”的办法,可是到的不上两百人,三个人轮流从上午演说到下午,不让群众休息,也不让群众喝水吃饭,说是怕群众跑掉了。这样蠢干的办法,又算奇闻的一种。

九、开展两条战线斗争克服自己错误争取查田运动的彻底胜利

无疑的,查田运动是在广大地区的内开展了。但当这个运动前进的时候,当我们正确估计了已得成绩,并奠定了运动发展的基础的时候,我们还要警觉地注视运动中途的障碍物。只有发动两条战线斗争的火力去清除这些障碍,才能推进查田运动更加迅速的前进。开展反右倾的思想斗争,反对对查田运动严重意义的估计不足,及对地主富农的妥协投降,反对对群众斗争的尾巴主义,是每个共产党员的责任。同时要把侵犯中农的危险唤起全体党员的注意,要严厉打击任何侵犯中农利益的企图,因为这是目前查田工作中已经明显表现出来了的十分严重的危险。对富农不正确观念,也无疑要影响到中农上去。一切命令主义的蛮干,对于联合中农是最大的危害。用两条战线斗争的火力,来扫荡查田运动道路上的一切障碍物,查田运动就可以大踏步前进,他的彻底胜利就有了充分保障了。

