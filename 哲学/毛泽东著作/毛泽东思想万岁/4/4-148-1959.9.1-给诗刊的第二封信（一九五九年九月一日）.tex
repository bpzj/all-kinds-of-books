\section[给诗刊的第二封信(一九五九年九月一日)]{给诗刊的第二封信}
\datesubtitle{(一九五九年九月一日)}


×××、××二位同志:

信收到,近日写了两首七律,录上呈改。如以为可,可以上诗刊。

近日右倾机会主义猖狂进攻,说人民事业这也不好,那也不好。全世界反华反共分子以及我国无产阶级内部、党的内部,过去混进来的资产阶级、小资产阶级的投机分子,他们里应外合,一起猖狂进攻。好家伙,简直要把昆仑山脉推下去了。同志,且慢!国内挂着共产主义招牌的一小撮投机分子,不过捡起几片鸡毛蒜皮,当作旗帜,向党的总路线、大跃进、人民公社举行攻击,真是“蚍蜉撼大树,可笑不自量”了。全世界反动派从去年起,咒骂我们狗血喷头。照我看,好得很。六亿五千万伟大人民的伟大事业而不被帝国主义及其在各国走狗大骂而特骂,那就是不可理解的了。他们越骂得凶,我就越高兴。让他们骂上半个世纪吧!那时再看,究竟谁败谁胜。我这两首诗,也算是答复那些忘八蛋的。
<p align="right">毛泽东
九月一日</p>

附两首律诗:
七律到韶山一九五九年六月

一九五九年六月二十五日到韶山,离别这个地方已有三十二周年了。

别梦依稀咒逝川,
故园三十二年前。
红旗卷起农奴戟,
黑手高悬霸主鞭。
为有牺牲多壮志,
敢教日月换新天。
喜看稻菽千重浪,
遍地英雄下夕烟。


七律登庐山;一九五九年七月一日

一山飞峙大江边,
跃上葱茏四百旋。
冷眼向洋看世界,
热风吹雨洒江天。
云横九派浮黄鹤,
浪下三吴起白烟。
陶令不知何处去,
桃花园里可耕田?


