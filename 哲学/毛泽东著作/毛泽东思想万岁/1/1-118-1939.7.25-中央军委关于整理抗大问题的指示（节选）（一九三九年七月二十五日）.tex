\section[中央军委关于整理抗大问题的指示(节选)(一九三九年七月二十五日)]{中央军委关于整理抗大问题的指示(节选)}
\datesubtitle{(一九三九年七月二十五日)}


抗大以致一切由知识分子所组成的军政学校及教导队之办理方针,应当如下:(一)把知识青年训练成为无产阶级的战士或同情者,把他们训练成为八路军的干部,确是一个艰苦的工作,我们应努力转变他们的思想,注意于领导他们思想转变的过程,用适当的方式组织学生中的思想上的争论与辩论,实际上这样的学校中一定有资产阶级思想与无产阶级思想的斗争。

(二)学校一切工作都是为了转变学生的思想,政治教育是中心的一环,课目不宜太多,阶级教育、党的教育与工作必须大大加强。抗大不是统一战线的学校,而是党领导下的八路军干部学校。

(三)教育知识青年的原则是:

1、教育他们掌握马列主义,克服资产阶级及小资产阶级的思想意识;

2、教育他们有纪律性、组织性,反对组织上的无政府主义与自由主义;

3、教育他们决心深入下层实际工作,反对轻视实际经验;

4、教育他们接近工农,决心为他们服务,反对看不起工农的意识。

