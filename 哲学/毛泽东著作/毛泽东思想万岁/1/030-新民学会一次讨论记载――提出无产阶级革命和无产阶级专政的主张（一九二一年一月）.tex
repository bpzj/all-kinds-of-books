\section{新民学会一次讨论记载――提出无产阶级革命和无产阶级专政的主张(一九二一年一月)}


《达到目的须釆用什么方法?》

首由毛润之报告巴黎方面蔡和森君的提议。并云:世界解决社会问题的方法大概有下列几种:

1.社会政策

2.社会民主主义

3.激烈方法的共产主义(列宁的主义)

4.温和方法的共产主义(罗素的主义)

5.无政府主义

我们可以拿来参考以决定自己的方法。

何叔衡:主张过激主义,一次的扰乱,抵得二十年的教育,我深信这些。

毛润之:我的意见与何君大体相同。社会政策,是补苴罅漏的政策,不成办法。社会民主主义,借议会为改造工具,但事实上议会的立法总是保护有产阶级的。无政府主义否认权力,这种主义,恐怕永世都做不到。温和方法的共产主义,如罗素所主张的极端的自由,放任资本家,亦是永世做不到的。急烈方法的共产主义,即所谓劳农主义,用阶级专政的方法,是可以预计效果的。故最宜釆用。

