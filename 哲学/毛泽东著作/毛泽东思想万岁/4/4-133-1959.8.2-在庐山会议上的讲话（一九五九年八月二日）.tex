\section[在庐山会议上的讲话(一九五九年八月二日)]{在庐山会议上的讲话}
\datesubtitle{(一九五九年八月二日)}


中委、候补中委191人,到会147人,列席15人,共165人,会议议程:

(一)改指标问题:武汉六中全会决定了今年的指标,上海七中全会有人主张改指标,多数不同意,看来改也改不彻底,现在还有五个月,改了好经过人大常委会,高指标是自己定的,自己立了个菩萨自己拜,现在还得打破,打破了不符合实际的指标,钢、煤、粮、棉等。

(二)路线问题:有些同志发生怀疑,究竟对不对?上庐山前不清楚,上庐山后有部分人要求民主,要求自由,说不敢讲话,有压力,当时摸不着头脑,不知所说的不民主是为的什么?前半个月是神仙会议,没有紧张局势。他们说没有自由,就是要攻击总路线,破坏总路线,他们要自由,就是破坏总路线的自由,要批评总路线的言论自由,他们要求紧张的局势。以批评去年为主,也批评今年的工作。说去年的工作做坏了,自去年十一月第一次郑州会议以来,纠正了“刮共产风”,纠正了“一平二调三提款”等一些“左”的倾向。他们对于九个月来的工作,看不到,不满意,要求重新议论,否则就认为压制民主。他们对政治局扩大会议嫌不过瘾,说民主少了,现在开全会,民主大些,准备明年开党代表大会。看形势,如需要,今年九、十月开也可以。五七年不是要求大民主,大鸣、大放、大辩论吗?庐山会议已经开了一个月了,新来的同志不知道怎么一回事,先开几天小会,再开大会,最后作决议。

开会的方法,用大家所赞成的方法,从团结的愿望出发。中央全会的团结,关系到中国社会主义的命运。在我们看来,我们应该团结,现在有一种分裂的倾向。去年八大我说过,危险无非是:一、世界大战,二、党分裂,当时还没有显着的迹象,现在有这种迹象了。团结的方法,从团结的愿望出发,经过批评与自我批评,在新的基础上达到新的团结的目的。对犯错误的同志,采取惩前毖后,治病救人的方针,给犯错误的同志一条出路,允许犯错误的同志改正错误,继续革命,不要像“阿Q正传”上的赵太爷,不许阿Q革命。对犯错误的同志要一看二帮,只看不帮,不作工作是不好的。我们反对错误,毒药吃不得,我们不是欣赏错误的臭味。批评斗争他们是使他们离我们近一点,使缺点错误离我们越远越好。对于犯错误的同志要有分析,无非是两种可能,一个是能改,一个是不能改。所谓看,就是看能不能改,所谓帮,就是帮助他改。有些同志一时跟到那边去,经过批评说服,加上客观情况的改变,许多同志改变过来了,又脱离了那些人。立三路线、王明路线,遵义会议上纠正了,以后经过十年时间,一直到七大,中间经过了整风,经过十年是必要的。一个人要改正错误要有几个过程。你强迫一下改正不行。马克思说:“商品是经过千百次交换才认识其两重性的。”洛甫开始不承认路线错误,七大经过斗争,洛甫承认了路线错误。那场斗争,王明没有改,洛甫也没有改,又旧病复发,他还在发疟疾,一有机会出来了。大多数同志改好了。从路线错误来说,历史事实证明是可以改变的,要有这种信心。不能改的是个别的。可见釆取治病救人的方针是见效的,要有好心帮助他们。对人有情,对错误的东西应当无情的,那是毒药,要有深恶痛绝的态度,但不用武松、鲁智深、李逵的方法。他们很坚决。可以参加共产党,他们的缺点是不大策略,不会作政治工作。要釆取摆事实讲道理的方法,大辩论,大字报,中字报,庐山会议的简报。

上山讲了三句话:成绩很大,问题不少,前途光明。后来问题不少一句出了问题,是右倾机会主义向党猖狂进攻的问题,刮“共产风”的问题没有了,“一平二调三提款”没有了,浮夸也没有了,现在不是反“左”而是反右。是右倾机会主义向党、向六亿人民、向轰轰烈烈的社会主义运动猖狂进攻的问题。现在要指标,越落实越好,反了几个月的“左倾”右倾必然出来,缺点和错误确是存在的,但已经改了,他们还要求改。他们抓住这些东西来攻击总路线。把总路线引导到错误的方面去。


