\section[在中央工作会议上陈伯达同志发言时的插话(一九六四年十二月二十七日)]{在中央工作会议上陈伯达同志发言时的插话}
\datesubtitle{(一九六四年十二月二十七日)}

(注:方括号内为陈伯达同志发言的有关内容)

〔主要矛盾是什么?主席根据大家的意见作了总结。主要矛盾是社会主义和资本主义的矛盾。四清与四不清不能说明问题时性质。封建社会就是清官与贪官的问题。《四进士》就是反贪官的嘛!〕

巡抚出朝,地动山摇,可了不起哩!

〔封建时代的清官实质上是假的。“三年清知府,十万雪花银。”清,在不同社会有不同的阶级内容。资本主义社会也有所谓清官,那些清官都是大财阀。〕

清朝刘锷的《老残游记》中说,清官害人,比贪官害人还厉害。后来一查,南北朝史中的《魏史》就有此说。

〔内部矛盾那个时代没有?党内党外矛盾交叉,党内有党。国民党也有此问题。〕

我们党内至少有两派,一个是社会主义派,一个是资本主义派。

〔主席强调,要听各方面的话。好话,坏话,特别是反对的话,要耐心听。这是工作做的好坏的准则。〕

讲话长了怎么办?李雪峰同志说的话,讲长了打零分。讲长了,让他讲长嘛!横直没人听嘛!

〔许多人忘记自己是从那里来的。不能忘本嘛!我这个人不参加革命,顶多是个小学教员、中学教员而已。〕

大官是从小官来的,小官是从老百姓来的。我们都是从老百姓中来的,还是老百姓嘛!“蒋委员长”不姓蒋,姓郑,叫郑三发子,河南人氏,只知有母,不知有父,还不是老百姓变的?

〔主席常说,不要自以为是。乡干部当权,就以为自己的意见对。〕

不要当了权,就是自己的意见对。自以为是,没有自以为不是的。为什么要开会?就是意贝不一致。一致了开会干什么?

〔不怕官,只怕管嘛!〕

小官怕大官,大官怕洋人。


