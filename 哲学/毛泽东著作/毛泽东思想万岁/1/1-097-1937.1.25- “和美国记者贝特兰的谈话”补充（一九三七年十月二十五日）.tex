\section[ “和美国记者贝特兰的谈话”补充(一九三七年十月二十五日)]{ “和美国记者贝特兰的谈话”补充}
\datesubtitle{(一九三七年十月二十五日)}


<b>问:</b>你们对于将被毒气攻击一事有所准备没有?

<b>答:</b>因为物质困难,现将还没有什么防御武器,但我们正从蒋委员长要求援助。朱总司令已准备发表宣言答复日本华北军司令部,这个宣言指出,日本法西斯这种灭绝人道的举动,必然促成他们自己的毁灭。

<b>问:</b>共产党不是主张普选制的吗?

<b>答:</b>我们是主张普选制的,因为只有普选制才能彻底的表现民意。临时国民大会是一种过渡方式,因为在紧急情况下不许可从容的选举,目前也没有保证选举出民意的条件。此时求其能够比较的代表民意,还是依照孙中山先生在世时候的主张,由政党军队与民众团体各自推出代表的好。这种办法,当然与普选制有区别,我们一定要达到普选制,这个办法是到普选制的桥梁,然而他是当前比较好的办法。

<b>问:</b>据你估量,国民党是否能够同意?

答;困难至此,国民党中许多明达人士早已有了这种呼声,蒋介石先生亦有声明要实行民权主义,社会各界人士,抗战军队的将倾,亦有多人感到这种必要了,我想国民党没有理由拒绝这个提议吧!假如他们同意的话,共产党将与国民党来到亲密合作的阶段,这是一个国利民福的前提。

