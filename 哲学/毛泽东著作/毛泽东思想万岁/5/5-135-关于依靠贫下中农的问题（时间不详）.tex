\section[关于依靠贫下中农的问题(时间不详)]{关于依靠贫下中农的问题(时间不详)}


要依靠大多数,依靠贫下中农,把他们组织起来,看你们站在百分之九十五的人这一边,还是站在百分之五的人那边。剥削者不过一、二、三、四、五,按七亿人口计算,百分之五就是三千五百万人,剥削六亿六千五百万人口,要算这个基本账,到底站在哪一方面。不管赫鲁晓夫说我们是小资产阶级,总之,他是修正主义,他站在百分之五的人那一边,我们站在工人阶级、贫下中农这一边。

我赞成省召开贫下中农代表会议,各级有工会,就是没有农会,共产党又不代表它。妇女有妇联,青年有青年团,省应该开贫下中农代表会议。

贫下中农代表会议,也要有一部分中农的积极分子参加,使他们感到有他们的份,湖南就是这样开的。

我们这辈子忘不了贫下中农,有时只要提醒一下就行了,干部子弟恐怕就忘记了。我们许多人中间,有的地委书记也忘了,他们现在丰衣足食了。作计划工作的,也要注意绝大多数,注意贫下中农。贫下中农有权,能管中农,也能管地主、富农。

修正主义跟我们不同,我们依靠工人、农民中的大多数,就算工人中有百分之八到十五的坏人,坏人还是少数,而且要加以分析。农村中贫下中农占百分之七十左右。可以带动中农,改造地富中的好的,再加上地富子女,使少数人孤立起来,其中有反革命分子、破坏分子。


