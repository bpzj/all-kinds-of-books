\section[在八届三中全会上的讲话(一九五七年十月七日)]{在八届三中全会上的讲话}
\datesubtitle{(一九五七年十月七日)}


中国两次革命,在民主革命时期是反帝反封建反官僚资本主义,对民族资产阶级的资本主义思想只是在党内作斗争。当时有两条道路,解放道路和殖民地道路。社会主义革命是消灭阶级、消灭剥削,是无产阶级革命。说资产阶级与无产阶级是主要矛盾,我认为在理论上是没有问题的。

一九五三年在财经会议上提总路线,开始还不敢在全党宣传,先讲到县一级,五三年底在政协上讲开了。跟着宣传部起草了一个总路线宣传提纲。三年半以来,给资产阶级以严重打击,对个体经济也给了打击。因此反映到八大决议上说资产阶级与无产阶级的矛盾基本上解决了。这句话也没有说错。基本解决不等于完全解决。政权问题解决了,所有制问题基本上解决了,但经济上和政治上没有完全解决。

资产阶级及资产阶级知识分子,民主党派中的右派,一部分富裕中农,站在人民中反映人民。那时不是完全看得那么清楚,但也不是完全没有看到(那时还是提了改造)。当时他们服服帖帖,所以说基本解决了。今天强调这个矛盾是因为他们要造反。到今年青岛会议时就看清楚了,提出了城市和农村还有两条道路的斗争,这种阶级斗争没有熄灭,这次右派分子疯狂进攻,就应说资产阶级与无产阶级的矛盾是主要的。但策略上还是按青岛文件上那样讲的好,到会的人晓得主要次要就行了。很长时间不讲了,如果现在加上去,闹得天翻地覆这不好。现在应再按青岛的讲法说他三个月。

工人中也有资产阶级思想。党内的三大主义也挂在资产阶级账上。两条道路——资产阶级和无产阶级、社会主义与资本主义——是过渡时期的主要矛盾,暂时不在报上讲。讲了有无可能把大量人民内部矛盾冲淡了;另外是内部的官僚主义、宗派主义、主观主义,任白戈讲的,可以从理论上写这个问题。

劳动人民内部的关系——是党群关系、干群关系、个人与集体、青年与老年、工人内部等矛盾,是大量的东西,突然提出资产阶级与无产阶级矛盾是主要的,是否会影响鸣放?也不一定。但是要影响劳动人民内部吵架。

人民有两部分,一部分剥削过人,一部分没有剥削过人。受资产阶级影响,一部分人少,一部分人多。过几百年后就不能再挂在资产阶级账上,那就是先进与落后的斗争。大规模阶级斗争基本结束,矛盾基本解决,说的政治制度和所有制问题。但上层建筑、意识形态、政治势力的问题大量的还没有解决。个人主义、官僚主义、唯心主义也是上层建筑,也得解决。

去年资本家敲锣打鼓之后,马上提出反对资产阶级,说不出口,不得人心,于我不利。以后鸣起来了,就好办了,取得经验就好办了。我们提百花齐放、百家争鸣,他们就放出来了。过去资产阶级是服服贴贴,现在是大吵大闹,我们提鸣放,右派提大鸣大放。我们说在文学艺术和学术上鸣放,他们要发展到政治上。今年共产党与右派合作找出一个好办法,大鸣、大放、大字报、大辩论,找到一个比较适合的形式。在延安我们没有这样大的胆子,没有经验,没有禁止,也没有放。社会主义革命,我们没有干过,没有经验,这次大鸣大放增加了我们的经验。将来还是鸣放的。百花齐放不包括反革命在内。一年一放还会放出来。把人民当敌人压是很危险的。论人民矛盾正是防止采取压服的办法。

第一条:在过渡时期肯定主要矛盾是资产阶级与无产阶级矛盾。

第二条:在一定时期在报纸上不讲,继续宣传两条路线的斗争,不加这两字,免得引起许多麻烦。

劳动人民内部矛盾,目前正在大鸣大放,大辩论中解决,一提出阶级矛盾是主要的,将影响整改。

人民内部包括三个阶级:无产阶级、资产阶级、小资产阶级,阶级矛盾存在这三部分人中,这个矛盾也是人民内部矛盾,也是阶级矛盾。阶级矛盾和敌我矛盾有区别。人民内部矛盾一般说是非对抗性的,与资产阶级的矛盾有对抗性的一面。中心问题是三部分人之间的矛盾。其中有一部分是暗藏对抗的,如和章伯钧等的矛盾就是对抗性的矛盾。对待这种对抗性的矛盾是采取剥笋政策,每年剥一点,今年剥了一些笋皮,但是剥不完。有两年不宣传社会主义就出来了。以后还要剥的。正确处理人民内部矛盾那篇文章没有错误,但是没有青岛的一篇就不够。现在主要问题不是封建残余、帝国主义残余(矛盾还是存在的),湖南捉了七千地富谁也不说话,但捉一个章伯钧就有问题。社会主义革命,是资产阶级、小资产阶级问题。资产阶级、资产阶级知识分子连家属三千万,是大问题,工人阶级三口人家,最多四千万人。社会主义革命对象是资产阶级、资产阶级知识分子、上层小资产阶级(农村富裕中农)。资产阶级及资产阶级知识分子都有左派,大量是中间派,右翼只有百分之一、二。百分之九十以上是教育问题,批评问题。我们所提人民内部矛盾包括阶级矛盾。资本家还有公民权。社会主义不能说是反帝反封建了,但帝国主义和封建势力的残余是资产阶级右派的同盟军。所以地主爱文汇报,是反社会主义的。

右派分子现在六万(一作五写——编者),将来顶多十五万到二十万,其中可以分化,应该去分化。如对一些工程技术人员、自然科学家、学者,能分化出来更好,要对他们进行工作。对一部分人批评从严,处理从宽,如荣毅仁等。

现在明确:在从资本主义到社会主义的过渡时期,主要(或基本)矛盾是无产阶级与资产阶级,社会主义与资本主义的矛盾,这是社会关系,人与人的关系。基本上解决了,但没有完全解决。地富反坏赞成资本主义,剥削人的人赞成资本主义,也是资产阶级与无产阶级的矛盾。两条道路斗争是经过长期斗争来解决的。“主要”和“基本”是一个意思。

八大决议说主要矛盾是先进的社会制度和落后的生产力之间的矛盾,讲道理不能这样讲。现在矛盾,将来也还是要矛盾的,到合作社都改为国营农场发薪水的时候,也还是有矛盾的。社会主义由两部分组成、公有制、集体制,将来二者也要发生矛盾。社会主义制度与生产力基本适合,也有不完全适合的地方,还有缺点,讲完全适合不对,斯大林提完全适合,就出了问题(报告第十四页)。宗教这种意识形态就不适合社会主义,但是还要修庙,修庙是为了达到毁庙的目的。为什么说大体适合呢?因为可以发展生产力。印度搞五年,增加了三十万吨钢,我们增加四百万吨。我们的制度不妨碍生产力发展。几十年后,集体与国营矛盾解决了,还会有新的矛盾。到共产主义价值法则不要了,军队不要了,当然也要国际环境许可。八大决议这句话,马、恩、列没讲过,但也没有害处。意思是要赶快发展生产,充实社会主义的物质基础,只是没有讲清楚,带语病的性质,没有认真的讲清矛盾,是比外国、此将来,这句话现在也不必改,现在可不谈这个问题。列宁讲过苏联政权与落后的技术有矛盾。现在不讲,以后解释清楚就行了。严格地讲,说社会主义制度与生产力不适合当然是不对的,我们恰好就是社会主义制度发展了生产力。许多经济学家说我们的制度与生产力有矛盾,说社会主义制度落后于生产力,这种说法不好。

一九五七年十月九日

领导方法,基本上有两种,按照一种方法领导,我们的事业可以进行得多快好省,按照另外一种方法领导,我们就会进行得少慢差费。

不仅革命工作有两种方法,建设工作也有两种方法。我国第一个五年经济建设计划,是个正确的计划,它取得了很大的胜利。可是,这个计划能否再进一步作得更好一些?看来是可能的。例如,我们如果一方面充分发展大型企业,另一方面积极发展中型小型企业,一方面抓中央管理的企业,另一方面充分发挥地方企业的积极性,速度就可能更快一些。现在看来,我们能否以十八年的时间,即三年恢复时期加上三个五年计划时期,达到年产钢一千八百万吨,甚至两千万吨呢?如果我们的方法正确,经过努力,有此可能。

多快好省和少慢差费,是互相矛盾、互相对立的两种方法。不进行斗争,不反对少慢差费的方法,多快好省的方法就不能实现。我们在一九五六年一月除了提出“多快好省”的口号,还提出了全国农业发展纲要草案四十条,这个纲要,是一个多快好省发展我国社会主义农业的纲领。我们还提出了一个“促进派”的概念,就是说,大家都应当作促进派,不作促退派。由于有了这几个东西,一九五六年我国整个经济文化事业有了一个很大的跃进。当然,也出现了一些缺点,多用了点钱,多招收了一些工人,市场供应有某些紧张。这个缺点并不大,很容易克服。可是,有些同志低估了成绩,夸大了缺点,说一九五六年“冒进”,了。吹起一股风,把“多快好省”的口号、“四十条”、“促进会”这几个东西都吹掉了。结果就影响今年经济建设的进展,特别是农业的进展。这个经验教训很大,不能不接受。总之,给群众泼了凉水,损害了他们的职极性,这是不对的。我们不论作什么事总要发扬群众的积极性,保护群众的积极性。对于工作中的缺点也要批评纠正,但是要在保护干部和群众积极性的条件下批评纠正,既批评他们的缺点也批评我们的缺点,这样他们就有一股劲了,我们的工作就可以作得更好了。我看,一个“多快好省”,一个“四十条”,还有一个“促进派”,都是好东西,不能吹掉,必须恢复。

我高兴的是这个会上有同志讲到“多快好省”的口号。

多和快有了好和省的限制,就没有弊病了。多和快,就是要求多办事。好,就是要求质量好。省,就是要求少用钱,既要质量好,又要少用钱,那么,多和快就没有毛病了。当然,“多快好省”应该实事求是的,合乎实际的,不应该是主观主义的,不切合实际的。我们建设社会主义,不是在主观主义的条件下,总应该尽可能争取多一点、快一点。


