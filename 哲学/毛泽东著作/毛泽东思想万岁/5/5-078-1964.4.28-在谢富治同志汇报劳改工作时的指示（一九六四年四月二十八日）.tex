\section[在谢富治同志汇报劳改工作时的指示(一九六四年四月二十八日)]{在谢富治同志汇报劳改工作时的指示}
\datesubtitle{(一九六四年四月二十八日)}


谢富治同志汇报说:去年我们着重抓了改造,然而生产是近年来最好的一年。但劳改工作中改造与生产的关系问题,迄今还没有解决。

主席说:究竟是人的改造为主,还是劳改生产为主,还是两者并重?是重人?重物?还是两者并重?有些同志就是只重物,不重人。其实人的工作做好了,物也就有了。

谢富治同志说:我在浙江省第一监狱守硕中队,向犯人宣读了双十条,工作组的其他同志也在乔司农场五大队宣读。读后,绝大多数原来不认罪的犯人认罪了,许多顽固犯人也有转变。

主席:大概那些人是比较有用的。他们为什么对双十条感兴趣?

谢富治同志说:他们懂了党的政策,感到他们自己特别是家庭和子女有了前途。

主席说:是啊!有前途改造就有信心,不然,一片黑暗,改造就没有信心了!

谢富治同志说:许多干部起初反对向犯人宣读双十条,但读了以后,犯人反而好管了,因而干部也就改变了。

主席说:许多干部不赞成读双十条,是怕读了以后,他那一套不灵了。他不相信能把绝大多数犯人改造成新人,过去红军军官带兵靠打人、骂人、关禁闭,枪毙等等。当连长、排长如果不打人,不骂人,不摆威风,他就没有法子带兵了。这样事情搞了多少年,后来总结了经验,逐渐改变,兵反而好带了。做人的工作,就是不能压服,要说服。现在,你的那一套在劳改中开始见效,但才是个开始,也要搞多少年才行。

主席说:对,原有的劳改干部水平不变。

谢富治同志说:劳改干部质量较弱,但任务重,劳改工作中阶级斗争、生产斗争、科学实验都有。

主席说:是啊,你一样都不行,怎么能改造人?(谢富治同志说了经过蹲点研究,提出劳改工作的“四个第一”,“二个为主”,“二个从宽从严”。对刑满就业人员的处理,提出“四留、四不留”,要求劳改干部对罪犯要“四知道”时)。

主席说:这很好,其他地方怎么办呢?

谢富治同志说:准备经过试点,逐步推广。日本战犯和中国战犯的改造工作都作的较好,释放后,除个别人外,绝大多数都表现很好。

主席说:在一定条件,在敌人放下武器,缴械投降以后,敌人中的绝大多数是可以改造的,但要有好的政策,好的方法,要他们自觉改造,不只靠强迫压服。

{\raggedleft (1964年4月28日,同谢富治同志的谈话)\par}


