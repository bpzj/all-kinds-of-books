\section[批评人民日报上的东西有很多大方向不对(一九五八年九月四日×××传达)]{批评人民日报上的东西有很多大方向不对(一九五八年九月四日×××传达)}
\datesubtitle{(一九五八年九月四日)}


对国际问题应该有研究,有一定的看法,不要临时抱佛脚,发表感想式的意见。现在对国际问题的意见,有些是感想式。对许多国际问题要有基本的看法,应该有比较深刻的议论。

搞新闻工作,光务实,不务虚,不好。有了看法,有了意见,就要找机会,找题目发挥。去年整风的时候,我就预先考虑到反攻的问题。右派进攻了,要不要反攻?什么时候反攻?抓什么题目反攻?后来抓到了卢郁文这个题目。

文汇报的资产阶级方向应当批判。在什么时候批判,抓什么题目开火?后来新民晚报进行了自我批评,就抓住这个题目开火了。

对国内外问题都应该这样。

报纸宣传,报纸编辑工作,最近一个月比较杂,看不出方向究竟搞什么?去年反右派的时候,突出,一气呵成。今年春天,在发表《梅林看全国》的社论以后,也很突出。最近一两个月,东西很多,方向不太明了。报纸应该有方向,在最近就要转变过来,把工业首先是钢铁、机械放在第一位。省报也应当如此。

一个时期要有一个方向,把大家的注意力集中起来。


