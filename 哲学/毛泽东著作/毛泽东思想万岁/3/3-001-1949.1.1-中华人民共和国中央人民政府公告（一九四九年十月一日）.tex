\section[中华人民共和国中央人民政府公告(一九四九年十月一日)]{中华人民共和国中央人民政府公告}
\datesubtitle{(一九四九年十月一日)}


自蒋介石国民党反动政府背叛祖国,勾结帝国主义,发动反革命战争以来,全国人民处于水深火热的情况之中。幸赖我人民解放军在全国人民的援助之下,为保卫祖国的领土主权,为保卫人民的生命财产,为解除人民的痛苦和争取人民的权利,奋不顾身,英勇作战,得以消灭反动军队,推翻国民党政府的反动统治。现在人民解放战争业已取得基本的胜利,全国大多数人民业已获得解放。在此基础之上,由全国各民主党派、各人民团体、人民解放军、各地区、各民族、国外华侨及其他爱国民主分子的代表们所组成的中国人民政治协商会议第一届全体会议业已集会,代表全国人民的意志,制定了中华人民共和国中央人民政府组织法,选举了毛泽东为中央人民政府主席,朱德、刘少奇、宋庆龄、李济深、张澜、高岗为副主席,陈毅、贺龙、李立三、林伯渠、叶剑英、何香凝、林彪、彭德怀、刘伯承、吴玉章、徐向前、彭真、薄一波、聂荣臻、周恩来、董必武、赛福鼎、饶漱石、陈嘉庚、罗荣桓、邓子恢、乌兰夫、徐特立、蔡畅、刘格平、马寅初、陈云、康生、林枫、马叙伦、郭沬若.张云逸、邓小平、高来民、沈钧儒、沈雁冰,陈叔通、司徒美堂、李锡九、黄炎培、蔡廷锴、习仲勋、彭泽民、张治中、傅作义、李烛、李章达、章伯钧、程潜、张奚若、陈铭、谭平山、张先、柳亚子、张东荪、龙云为委员,组成中央人民政府委员会,宣告中华人民共和国的成立,并决定北京为中华人民共和国的首都。中华人民共和国中央人民政府委员定于本日在首都就职,一致决议:宣告中华人民共和国中央人民政府的成立,接受中国人民政治协商会议共同纲领为本政府的施政方针,再选林伯渠为中央人民政府委员会秘书长,任命周恩来为中央人民政府政务院总理兼外交部部长,毛泽东为中央人民政府革命军事委员会主席,朱德为人民解放军总司令,沈钧儒为中央人民政府最高人民法院院长,罗荣桓为中央人民政府最高人民检察署检察长,并责成他们从速组成各项政府机关,推行各项政府工作。同时决议:向各国政府宣布,本政府为代表中华人民共和国全国人民唯一合法政府,凡愿遵守平等、互利及互相尊重领土主权等项原则的任何外国政府,本政府均愿与之建立外交关系。特此公告。

<p align="right">中华人民共和国中央人民政府主席毛泽东

一九四九年十日一日

抄自《新华日报》一九四九年十一月――一九五零年一月</p>


