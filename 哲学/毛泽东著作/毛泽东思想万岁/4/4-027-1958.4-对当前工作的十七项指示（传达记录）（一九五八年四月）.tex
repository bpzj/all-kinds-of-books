\section[对当前工作的十七项指示(传达记录)(一九五八年四月)]{对当前工作的十七项指示(传达记录)}
\datesubtitle{(一九五八年四月)}


当前有十七个工作:

(一)十年农业规划:1、水利规划:全国大兴修水利,甘肃省五八年一千一百多万亩,我省原定三十二万亩,有保守,现为九十二万亩(解放前全省为九十四万亩)认为差不多,现在看还落后,五八年要修九十四万到一百万亩,解决水利。2、肥料:要作到上万斤肥,打千斤粮。3、土壤改良:盐碱地的改良(如柴达木等地)。4、选择优良种子。5、改制、改良(农业技术改良,深耕细作)。6、病虫害的消灭。7、推广新式农具(包括农业机械在内)。8、副业发展:副业与农业的关系,发展副业为了支援农业。9、发展耕畜。10、绿化。11、除四害。12、消灭严重疾病(传染病)。

这十二条中央、省、地、县、区、乡、农业生产合作社都要作出规划。

(二)另有十二个规划:1、工业规划(主要地方工业);2、手工业;3、农业;4、副业;5、森林;6、渔业;7、牧业;8、交通;9、商业;10、文教;11、科学;12、卫生。中央、省、地、县都要作出规划。共为二十四条。

(三)反浪费问题。要开展反浪费运动,国务院将发指示,贪污也要反,贪污不是大量,浪费是大量的,特别在建设方面浪费最大。×××同志讲:“浪费很大,各个厂(场)矿都有浪费,反掉浪费就能积累建设资金,他说柴达木建设(工薪)不降,不管有多少油,我主张不降。”增产是两个方面,即增产,节约。增产不节约就浪费了增产。

(四)正确解决农村积累和消费问题。在合作化开始一二年,为了显示合作化的优越性提出多扣少分。现在是要多积累合作社的资金,一个合作社的积累也是巩固社的,没有社的积累,社就无法巩固。可考虑50%作为社的积累,50%分配。合作化后,农村的情况是26%的农户已达到富裕中农的生产水平。现在不强调生活水准,主要发展生产,现在不是要吃好,穿好,而是要真正苦干五年,艰苦奋斗,如果一人少用十元就六十亿,这对国家有好处,有了积累才能建设,有了建设才有希望。积累资金的办法:一种是国家的积累(包括企业事业公益金)。二种是合作社积累(合作社的公积金、公益金)。三种是国家税收积累。

全国取得一条经验,一些建设国家投资就搞不好,凡是自己动手就能搞好建设,要大量的依靠农业人口的劳动力来建设水利。自己搞水利建设就搞好了,靠国家投资搞水利就没搞好。

农业合作社里,可以搞社员个人向合作社投资,作成合作社的积累,来建设农业,集中生产,但不能强迫社员投资。青海一九五八年搞一至三千万积累,有些人是有钱的。

(五)搞试验田问题。各个党委要搞试验田。黄安县搞的好(土地不好)亩产八百斤,原因是人的作用,经营的好,黄安县提出千斤县,主席指出:中央到县都要搞试验田。牧业区搞牧业试验场,要抓紧搞先进,不搞落后。人的作用是主要的。要具体领导,什么工作都要具体领导,搞农业就要搞试验田。县、区、乡都搞,就能搞几十个,几百个场子。工业也要搞试验,这方面的干部不要住在楼上办公,应到工厂(场)去办公。搞出经验来,办学校也搞试验,文教厅的干部可搬到学校办公。真正深入实际,这是领导的根本问题。

(六)红与专的问题。红与专这是矛盾的对立统一,红专两个方面都搞(红是政治,专是专业技术)要批判空头政治家。批判不重视政治的技术家,只讲空话是不行的,有些人不红,就是白的。

(七)打掉“官”风(全国反官僚主义一般化)。要提倡干部有革命的干劲,干部要有朝气。现在的干部不论新老,应是越老,于劲就越大。

(八)二十四条规划,省、地委在六月报中央,县委的规划除报地委外,直报省委,乡区的规划好的典型与不好的典型选择报省委。

(九)除四害:毛主席指示,要求开展以除四害为中心的爱国卫生运动,它的意义不仅是除四害,它关系着人的健康问题。

(十)绿化问题。主席讲天天绿化,大规模的搞运动,如木林、经济林。有些省提倡上山,河北省飞地,农业社抽人上山搞生产,搞牧、农场。

(十一)认真地搞地方工业,第二个五年计划,争取地方工业的总产值要超过农业总产值。王震参观日本的农业发现许多工业分散在农村,我们要学习日本这种方法。省、地、县都应搞工业。

(十二)开会方法,地区联系问题。开会的方法我们党内有许多好形式。主席讲大鸣大放大字报是群众路线的新发展,几级干部会除讲工作外,还要讲理论、思想问题。这样就能提高干部的思想性、原则性。这次反右派是理论问题,又红又专。地方民族主义实质是资产阶级思想问题,不从理论讲,永远不能解决问题,浙江省委书记××同志的报告是标准,我们要看七、八次,报告中第八、九部分是敌我矛盾,政治、思想、革命问题。上海市委的报告提到了思想问题。

地区联系,更好地互相配合起来。中央提倡互相联系,过去大区撤消是必要的,大区撤消后密切了上下联系(省与中央),地区的联系是根据经济建设的需要。各省可以互相订合同。

中央初步考虑了七个点。1.以上海为中心,华东五省。2.武汉,两湖、河南、安徽。3.广东,包括两广、湖南、福建、江西。4.成都,包括西南三省、陕西。5.西安、西北五省陕、甘、青(宁夏回族自治区)、山西、河南。6.天津,与河北合并,河北、内蒙、山西、河南。7.沈阳为中心,东北三省。

各省开党代会,互派代表参加。

我们也要搞中心联系,农业社也要互相联系。

(十三)中央领导同志下乡问题,一年至少四个月下去。

(十四)中央同志下去,不迎接,不请客,不唱戏,以免躭误工作。

(十五)两类不同性质的矛盾问题。××报告谈得很明白,反右派以来有人认为人民内部的矛盾不存在了。在社会主义过渡时期主要是敌我矛盾问题(二中全会决议指出)。两个阶级:无产阶级与资产阶级的矛盾。两条道路:社会主义与资本主义道路的矛盾。有两种不同性质的矛盾(右派是代表敌人的),但是敌人一天天减少。右派十五万要超过,这些右派搞了,敌人的量减少了,但是矛盾仍然存在,敌我斗争还有。但是敌我矛盾不是主要的。人民内部矛盾占了主要的,敌我矛盾2%,同时是分散的。敌我矛盾有一部分也可以采取人民内部矛盾处理。反右派中一部分是敌我矛盾,大部分是人民内部矛盾,领导与被领导,上下之间,这是长期的大量的,因此,不能把人民内部矛盾看成敌我矛盾,或者看不到敌我矛盾。

(十六)不断革命,(这是对工作而言)工作以不断革命的精神进行,今后是一个接一个运动,经过这次整风,在政治战线上,在思想战线上,谁战胜谁的问题基本解决了(无产阶级战胜了资产阶级),今后是解决技术问题,十五年后铁的产量要赶上英国。十五年,党的任务放在技术问题上,政治是统帅,灵魂,只有红不专,当空头政治家是不行的。工作要比赛,比先进与落后,全党要钻研技术,不钻研技术不能完成社会主义建设,但要红就要学习政治,要有共产主义思想,提拔干部有的人说不要德,有才就行,德是坚强的革命意志,没有是不行的。另外,要有技术。十五年后要消灭阶级,那时有没有革命,有没有斗争?还是有的。解决好了就是人民内部矛盾,解决不好就是敌我矛盾。苏联两个人造卫星的上天,证明科学技术超过了美国,因此,我们革命的干劲应赶上先进派。

(十七)对立面的统一。事物是客观存在的,事物有好与坏,才能比较出先进与落后。报纸也要宣传好的、坏的,大量的和主要的。比较的方法实际是积极性的比较。不但经济工作上要比较,而且政治思想工作也要比较,主席讲一年比三次(党代表大会比,其他会议比)办法是:推广好的,批评差的。


