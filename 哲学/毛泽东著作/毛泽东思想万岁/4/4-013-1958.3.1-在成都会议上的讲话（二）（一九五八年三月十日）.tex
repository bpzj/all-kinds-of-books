\section[在成都会议上的讲话(二)(一九五八年三月十日)]{在成都会议上的讲话(二)}
\datesubtitle{(一九五八年三月十日)}


规章制度是一个问题,借此为例,讲一讲思想方法问题——坚持原则与独创精神。

国际方面,要和苏联、一切人民民主国家和各国共产党、工人阶级友好,讲国际主义,学习苏联及其他外国的长处,这是一个原则。但是学习有两种方法,一种是专门模仿,一种是独创精神,学习应和独创相结合,硬搬苏联的规章制度,就是缺之独创精神。

我党从建党时期到北伐时期(一九二一年到一九二七年),虽有陈独秀披着马克思主义外衣的资产阶级思想,但比较生动活泼。十月革命胜利后的第三年,我们建了党,参加党的人都是参加“五四”运动和受其影响的青年人。十月革命后,列宁在世,阶级斗争很尖锐,斯大林尚未上台,他们也是生动活泼的。陈独秀主义来源于国外社会民主党和国内资产阶级。这个时期,虽发生了陈独秀主义的错误,一般说没有教条主义。

内战时期到遵义会议(一九二七年到一九三五年)中国党发生了三次“左”倾路线,而在一九三四年至一九三五年最厉害,当时苏联反托派胜利了,在理论上只战胜了德波林学派,中国“左”倾机会主义者差不多都是在苏联受到影响的,当然也不是所有去莫斯科的人,都是教条主义者。当时在苏联的许多人当中,有些人是教条主义,有些人不是,有些人联系实际,有些人不联系实际,只看外国。加上斯大林的统治开始巩固(大巩固是在肃反后);共产国际当时是布哈林、皮可夫、季诺维也夫,东方部长是库西宁,远东部长是米夫。×××是个好同志,善良,有独创精神,就是太老实了些,米夫的作用大了,这些条件使教条主义得以形成,有些中国同志也受到影响,“左”倾在知识青年中也有。当时王明等搞了个所谓“二十八个半布尔什维克”,几百人在苏联学习,为什么只有二十八个半呢?就是他们“左”得要命,自己整自己,使自己孤立,缩小了党的圈子。

中国的教条主义有中国的特色,表现在战争中,表现在富农问题上,因为富农人数很少,决定原则上不动,向农民让步。但是“左”派不赞成,他们主张“富农分坏田,地主不分田”,结果地主没有饭吃,一部分被迫上山,搞绿色游击队。在资产阶级问题上,他们主张一概打倒,不仅政治上消灭,经济上也消灭,混淆了民主革命和社会主义革命。对帝国主义也不加分析,认为是铁板一块,不可分割,都支持国民党。

全国解放后(一九五○年到一九五七年)在经济工作和文教工作中产生了教条主义,军事工作中搬了一部(分)教条,基本原则坚持了,还不能说是教条主义。经济工作教条主义主要表现在重工业、计划工作、银行工作、统计工作,特别是重工业和计划方面,因为我们不懂,完全没有经验,横竖自己不晓得,只好搬。统计工作几乎是抄苏联的;教育方面也相当厉害,例如五分制,小学五年一贯制等,甚至不考虑解放区的教育经验。卫生工作也是,害得我三年不能吃鸡蛋,不能吃鸡汤,因苏联有篇文章说不能吃鸡蛋和鸡汤,后来又说能吃。不管文章正确不正确,中国人都听了,都奉行。总之,是苏联第一。商业少些,因中央接触较多,批转文件较多,轻工业工作中的教条主义也少些,社会主义革命和农业合作化未受教条主义影响,因为中央直接抓,中央这几年主要抓革命和农业,商业也抓了一点。

教条主义的情况也有不同,需要分析比较,找原因:

一、重工业的设计、施工、安装自己都不行,没有经验,中国没有专家,部长是外行,只好抄外国的,抄了也不会鉴别。而且还要借苏联的经验和苏联专家,破中国的旧专家的资产阶级 注:原文为“想” 思想,苏联的设计用到中国大部分正确,一部分不正确,是硬搬。

二、我们对整个经济情况不了解,对苏联和中国的经济情况的不同更不了解,只好盲目服从。现在情况变了,大企业的设计施工,一般说来,可以自己搞了;设备,再有五年就可以自己造了,对苏联、对中国的情况,都有些了解了。

三、在精神上没有压力了,因为破除了迷信。菩萨比人大好几倍,是为了吓人,戏台上的英雄豪杰出来,与众不同,斯大林就是那样的人,中国人当奴隶当惯了,似乎还要当下去,中国艺术家画我和斯大林的像,总比斯大林矮一些,盲目屈服于那时苏联的精神压力,马列主义对任何人都是平等的,应该平等待人。赫鲁晓夫一棍子打死斯大林也是一种压力,中国党内多数人是不同意的。还有一些人屈服于这种压力,要打倒个人崇拜。有些人对反对个人崇拜很感兴趣,个人崇拜有两种:一种是正确的,如对马克思、恩格斯、列宁、斯大林正确的东西,我们必须崇拜,永远崇拜,不崇拜不得了,真理在他们手里,为什么不崇拜呢?我们相信真理,真理是客观存在的反映,一个班必须崇拜班长,不崇拜不得了。另一种是不正确的崇拜,不加分析,盲目服从,这就不对了。反个人崇拜的目的也有两种:一种是反对不正确的崇拜,一种是反对崇拜别人,要求崇拜自己,问题不在于个人崇拜,而在于是否是真理。是真理就要崇拜,不是真理就是集体领导也不成。我们党在历史上就是强调个人作用和集体领导相结合的。打死斯大林有些人有共鸣,有个人目的,就是为了想让别人崇拜自己,有人反对列宁,说列宁独裁,列宁回答很干脆:与其让你独裁,不如我独裁好。斯大林很欣赏高岗,专送一辆汽车,高岗每年“八·一五”都给斯大林打贺电,现在各省也有这样的例子;是江华独裁,还沙文汉独裁?广东、内蒙、新疆、青海、甘肃、安徽、山东等地都发生过这样的问题,你不要以为天下太平,时局是不稳定的,“脚踏实地”是踏不稳的,有一天大陆会下沉,太平洋会变成陆地,我们就得搬家。轻微的地震是经常会有的,高饶事件是八级地震……

四、忘记了历史经验教训,不懂得比较法,不懂得树立对立面。我昨天已经讲过,对许多规章制度,我们许多同志不去设想有没有另外一种方案,择其合乎中国情况者应用,不合适者,另拟。也不作分析,不动脑筋,不加比较。过去我们反对教条主义,他们的“布尔什维克”刊物把自己说成百分之百的正确,自己吹嘘自己,其办法是。攻其一点或几点,不及其余,“实话报”攻击中央苏区五大错误,不讲一条好处。

一九五六年四月提出“十大关系”,开始提出自己的建设路线,原则和苏联相同,但有我们的一套内容。“十大关系”中,工业和农业,沿海和内地,中央和地方,国家、集体和个人,国防建设和经济建设,这五条是主要的。国防费在和平时期要少,行政费任何时期都要少。

一九五六年,斯大林受批判,我们一则以喜,一则以惧。揭掉盖子,破除迷信,去掉压力,解放思想,完全必要。但一棍子打死,我们就不赞成,他们不挂像,我们挂像。一九五○年,我和斯大林在莫斯科吵了两个月,对于互助同盟条约,中长路,合股公司,国境问题,我们的态度:一条是你提出,我不同意者要争,一条是你一定要坚持,我接受。这是因为顾全社会主义利益。还有两块“殖民地”,即东北和新疆,不准第三个国的人住在那里,现在取消了。批判斯大林后,使那些迷信的人清醒了一些。要使我们的同志认识到,老祖宗也有缺点,要加以分析,不要那样迷信。对苏联经验,一切好的应该接受,不好的应拒绝。现在我们已学会了一些本领,对苏联有了些了解,对自己也了解了。

一九五七年,在“正确处理人民内部矛盾的报告”中,提出了工、农业同时并举,工业化的道路,合作化、节育等问题。这一年发生了一件大事,就是全民整风、反右派,群众性的对我们工作的批评,对人民思想的启发很大。

一九五八年在杭州、南宁、成都开了三次会。会上大家提了很多意见,开动脑筋,总结八年的经验,对思想有很大启发,南宁会议提出了一个问题,就是国务院各部门的规章制度,可以改,而且应当改。一个办法是和群众见面,一个办法是搞大字报。另一个问题是地方分权,现在已经开始实行,中央集权和地方分权同时存在,能集则集,能分则分,这是去年三中全会后定下来的。分权当然不能是资产阶级民主,资产阶级民主在社会主义之前是进步的,到社会主义时期是反动。苏联俄罗斯族占百分之五十,少数民族占百分之五十,而中国汉族占百分之九十四,少数民族占百分之六,故不能搞加盟共和国。

中国的革命是违背斯大林的意志而取得胜利的,假洋鬼子“不许革命”。“七大”提出放手发动群众,壮大一切革命力量,建立新中国。与王明的争论,从一九三七年开始,到一九三八年八月为止,我们提十大纲领,王明提六十纲领。按照王明即斯大林的作法,中国革命是不能成功的。我们革命成功了,斯大林又说是假的,我们不辩护,抗美援朝一打就真了。可是到我们提出“正确处理人民内部矛盾”时,我们讲,他们不讲,还说我们是搞自由主义,好像又是不真了。这个报告公布后,纽约时报全文登载,并发表了文章说是“中国自由化”。资产阶级要灭亡,见了芦苇当渡船,那是很自然的事。但资产阶级的政治家也不是没有见解的人,如杜勒斯听到我们的文章,说要看看,不到半月他便作出结论:中国坏透了,苏联还好些。但当时苏联看不清,给我们一个照会,怕我们向右转。反右派一起,当然“自由化”没有了。

总之,基本路线是普遍真理,但各有枝叶不同。各国如此,各省也如此。有一致,也有矛盾,苏联强调一致,不讲矛盾,特别是领导与被领导的矛盾。


