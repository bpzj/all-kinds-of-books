\section[关于春耕问题的训令(节录)(一九三二年二月八日)]{关于春耕问题的训令(节录)}
\datesubtitle{(一九三二年二月八日)}


春天到了,春耕在即,这一问题是苏区国民经济的主要部分,不但关系苏区工农群众日常生活的需要和改善,同时关系经济的发展与巩固。因为生产增加,是巩固与加强苏区和红军向外发展的力量。目前正当着帝国主义积极瓜分中国的时候,而国民党反革命政府正无耻的出卖中国,使中国无数万的群众牺牲于帝国主义宰割之下;同时全国民日反帝反国民党的运动,工人大罢工猛烈的发展着,各地群众自动组织义勇军,有些白军士兵已自动起来与日本帝国主义作战,各地苏区与红军得到新的伟大的胜利和发展,这些事实完全证明目前中国的形势又到了革命最紧张的时候。我们主要大大地向外发展,配合和领导全国工农民一切革命群众的反日反帝国主义反国民党的运动,以组织广大民众的民族革命战争,来争取中国的民族解放和国家独立。因此发展和提高苏区生产,强固苏区向外发展的经济力量,在目前便有极伟大的意义。

在发动和加强春耕运动中,特别要注意对于去年被白匪摧残区域,应当用多种方法去帮助群众去解决春耕中各种问题,这是苏维埃政府一个实际任务。对于广大工农群众,苏维埃政府要努力领导他们来解决春耕中各种困难问题,使今年粮食收获大大的增加,来改善群众生活,来加强向外发展的力量,来充分供给红军的给养,以开展和帮助目前积极进行的革命战争。临时中央政府委员会现在决定如下办法,希各级政府切实地更具体立即执行:

(具体办法略)

