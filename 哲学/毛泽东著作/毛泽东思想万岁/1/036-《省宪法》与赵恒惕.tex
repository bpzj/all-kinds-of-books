\section{《省宪法》与赵恒惕}


“……自湘西问题发生,赵氏地位动摇,不能不假省宪以自卫,故前数日即立其党羽方克刚李济民等以金钱实公团通电,或作一群众运动。……寓拥护赵氏于维持省宪之中……定五号午前八时举行游街,……彼辈发信以工人团体为多,以为工人头脑简单,必有千余人可来,是日候至十时,不持无一团体到会,即私人参与者亦只彼辈亲朋数人而已,彼等焦急万分,又派人四出;至素为匪彼包办之中华工会及养济院佛化讲演院等处,每人大洋五角。并备点心一餐,雇请百数十分,于十二时出发游街,游街之先,在教育处幻灯场开会,一堂和尚叫化等共一百二三十人。首由:

(1)和尚炽培主席,报告开会宗旨,略谓:省宪法不谓之者宪经,如佛家之有佛经……坐中和尚叫化即两手乱舞。

(2)散会后出发:

前有大旗两面一一书省宪维会一一书售民请愿大会;

第一队为长衫马褂队,俨然省民代表,约二三千人;

第二队-一中华工会,约五六人;

第三队一一佛化讲演团,老头博衣,约四五十人;

第四队――养济院,约四五十人;

五光十色……演此丑剧,尽滑稽之能事……,

见民国日报

我们向来反对联省自治,因为他不是联省自治,乃是联督割据,我们历来反对军阀烂政客假窃名义的省宪,因为他不能做人民的保障,仅做了军阀烂政客争权争利的保障,湖南最是个明例。赵恒惕现在堂哉皇哉兴“护宪之师”了,而他两年来一一有省宪以来一一掺杀劳工(茧庞等)勘封报馆(大公报,自治新报,新湘报)剥夺人民书信自由(邮电检查员未曾撤去一日)剥夺集会结小自由(封力车工会,封碾谷工会,封外交后援会,多次禁止学会工人的集会),庇护其军队皈买各种鸦片烟,贿买选举(派遣荣员私造省宪的总投票,用钱私造省议会,用钱买得省长),向商人勒捐(长沙总商会会长,屡被迫逃),向农人提征田赋(有些县提征到民国十七年),被破法定的预算总此例减教育实业费增加军费,勾结吴佩孚伪耀南,何一不是戴省宪假面具与人民为敌,他这次兴师动众,也是和森钜犹争鸦片税(所谓持税)起始。赵恒惕这样万恶不赦的东西,居然还在那里假借各义大吹大擂“护宪”,真不怕羞死湖南人!

