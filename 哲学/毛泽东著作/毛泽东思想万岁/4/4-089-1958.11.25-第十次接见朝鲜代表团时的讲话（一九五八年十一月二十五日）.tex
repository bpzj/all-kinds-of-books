\section[第十次接见朝鲜代表团时的讲话(一九五八年十一月二十五日)]{第十次接见朝鲜代表团时的讲话}
\datesubtitle{(一九五八年十一月二十五日)}


越搞越熟了。对于一个党,一个民族,相互之间的认识都要有一个过程。个人与个人之间也不是一下子就认识清楚的。

一个民族,一个党,两个民族,两个党,兄弟国家有十二个党,十二个民族,相互之间的认识是有一个过程的。十月革命四十周年,十二国、六十四国共产党工人党会议.对于我们之间的认识。对于十二国之间的认识进了一步。

我们对你们的认识有一个过程,也同你们认识我们有一个过程一样。

看问题要看本质,看主流。马克思主义告诉我们看问题要看本质,看路线。就是它在国内是不是搞社会主义,在国际上是不是反对帝国主义。在社会主义阵营内是不是讲国际主义。这三点就构成一条路线。

中国共产党也是反帝的、社会主义的、国际主义的党。其它社会主义国家也是。这些方面表现了马克思主义路线的本质,这些都是表现。例如反对帝国主义这就是表现。是不是坚决,可以作个比较。像铁托是不是坚决?他的东西,我看三条都缺少。反帝国主义。他是不怎么反的。总是讲帝国主义好。


