\documentclass[a5paper,oneside,12pt]{ctexbook}
\ctexset{
section/name = {第,节},
section/number = \chinese{section}
}
\usepackage[hmargin=0.25in,vmargin=0.5in]{geometry} 
\usepackage[]{hyperref}
\usepackage[inline]{enumitem}
\usepackage{tikz,ifthen,etoolbox}
\usetikzlibrary{arrows.meta}
\newcommand*{\circled}[1]{\ifthenelse{#1 > 10}{\lower.7ex\hbox{\tikz\draw (0pt, 0pt) circle (.5em) node {\makebox[1em][c]{\footnotesize #1}};}}{\lower.7ex\hbox{\tikz\draw (0pt, 0pt) circle (.5em) node {\makebox[1em][c]{\small #1}};}}}
\robustify{\circled}

\setcounter{secnumdepth}{3} % 在report, book结构中,subsubsection默认没有编号,重设
\ctexset{
subsubsection/number = \arabic{subsubsection}
}

\newcounter{defNum}[subsubsection] % 一个名为defNum的计数器,以subsubsection为排序单位
\newcommand{\circledNum}[1]{\refstepcounter{defNum}\mbox{\circled{\arabic{defNum}}}#1\hspace{1em}} %定义了一个\circledNum命令, 每调用一次命令, 计数器命令\refstepcounter就会将 defNum 的值加一

\usepackage{float}
\usepackage{wrapfig}
\usepackage{array}
\usepackage{caption}
\usepackage{diagbox}
\usepackage{amsmath}
\usepackage{changepage}


% 居中,楷体,与上/下段无间距
\newenvironment{tightcenter}{%
  \setlength\topsep{0pt}
  \setlength\parskip{0pt}
  \begin{center}\kaishu 
}{%
  \end{center}
}

\newenvironment{yinyong}{%
    \begin{list}{}{\parsep\parskip
        \setlength\topsep{0pt}
        \setlength\itemindent{2em}%
        \setlength\parindent{2em}
        \setlength\listparindent{2em}
        \setlength{\leftmargin}{2em}
        \setlength{\rightmargin}{2em}
        \kaishu
    }
    \item[]
}{
  \end{list}
}


\pagestyle{plain} %整书页眉页脚设置


\title{中国古代算命术}
\author{洪丕谟姜玉珍}
\date{}

\begin{document}

\frontmatter
\maketitle
\tableofcontents
\chapter{《中国古代算命术》序}
\begin{center}
张荣明
\end{center}

在古人的思想现念中,人们的富贵贫贱、吉凶祸福,以及死生寿夭、穷通得失,乃至科场中举、货殖营利,无一不取决于冥冥之中的非人类自身所能把握的一种力量,即命运是也。

命运的观点,在古代源远流长。这里且不谈夏商周,只是考察一下先秦诸子的观点。

孔子弟子子夏说:“死生有命,富贵在天”(《论语•颜渊》)。可见孔门弟子是信奉命运的。孔子进一步指出:“富而可求也,虽执鞭之士,吾亦为之。如不可求,从吾所好”(《论语•述而》)。宋国的桓魅有一次想谋害他,孔子声称:“天生德于予,桓魋其如予何”(同上)!总之,在孔子看来,一个人的生死存亡、富贵贫贱完全与高悬于天的命运有关,绝非尘世碌碌众生的力量所能改变。故孔子又说:“不知命,无以为君子也”(《论语•尧曰》)。

墨子似乎是反对信奉命运的。墨子说:“王公大人,蒉若(意即假如)信有命而致行之,则必怠乎听狱治政矣,卿大夫必怠乎治官府矣,农夫必怠乎耕稼树艺矣,妇人必怠乎纺绩织絍矣”(《墨子•非命下》)。如果天下之人皆相信个人的成败得失完全取决于预定的命运,那么个人自然不必勤奋努力,只须静候祸福光临罢了。这样一来,王公贵族必然会怠倦于治理政务,农夫、妇女必然会怠倦于耕种、纺织,结果就会导致“天下必乱矣”。显而易见,墨子反对信奉命运,主要是从功利主义的观点出发的。

道家老子讲“道法自然”(《道德经》二十五章),似乎排斥了天命在人世间的干预或主宰的作用。然而庄子却说:“天下有大戒二:其一,命也,……知其不可奈何而安之若命,德之至也”(《庄子•人间世》)。由于是命定的,故虽然知道这事没有希望,但还是硬着头皮去做。这话显然有点宿命论的气息。

西汉时代流行谶纬神学,各种方术迷信,诸如巫蛊、择日、禁忌、符应、望气、卜相、杂祀、星占蓬勃兴起。在这种社会思潮的推波助澜之下,命运的观点更是不胫而走。董仲舒、扬雄其人,《淮南子》、《白虎通》诸书,无一不是命运观点的提倡者或信奉者。甚而至于连东汉杰出的无神论者王充,他虽然批判了卜筮、祭祀、鬼神等等世俗迷信,但是对于命运的观点却深信不疑。王充说:“自王公逮庶人,圣贤及下愚,凡有首目之类,含血之属,莫不有命;”“贵贱在命,不在智愚”(《论衡•命禄》)。

古人既然信奉命运,自然就殚精竭虑地设计出种种预测命运的方法,以便能趋吉避凶,把握命运。这样,算命术也就应运而生。

中国算命术大抵滥觞于汉代,迭经魏晋南朝的推衍发挥,于唐代始告确立。

唐人李虚中是算命发展史上的一位重要人物。韩愈评论他是:“无所不通,最深于五行书,以人之始生年月日所值日辰枝干,相生胜衰死相王斟酌,推人寿夭、贵贱利不利,辄先起其年时,百不失一二。其说汪洋奥义,关节开解,万墙千绪,参错重出。”(《韩昌黎文集•李虚中墓志》)

后来,五代徐子平进一步发展了李氏算命术,使之更加周详完备。自此以后,算命术在宋元明清蓬勃发展,命理著作层出不穷,算命术士代有新人。近代东海乐吾、袁树珊诸人皆是个中佼佼者。

然而古代也有怀疑与不信奉命运的明达之士。除墨子之外,屈原曾说:“天命反侧,何罚何佑?”(《楚辞•天问》)意即命运之神颠三倒四,何尝能罚恶佑呢?道教人物指出:“我命在我,不在天地。”认为通过掌握道教独特的气功修炼方法,完全能主宰自身的命运。至于唐代吕才、宋代费衮、清代袁枚其人皆目光如炬,洞烛其谬,纷纷援笔作文抨击算命之术,鞭辟入里,锐不可当。近代著名学者林惠祥先生著有《算命的研究和批判》,对于古代算命术的荒谬之处进行深入细致的分析与批判。

算命术当然是一种迷信,在科学昌盛的今日,可不待言而自明。但是自从算命术诞生之后,人们对于婚姻嫁娶、工商谋富、赴考求官,乃至用兵打仗、施政方略,自帝王贵族以至于黎民百姓,无不求助于算命术而企图预测吉凶。一千多年来,此道盛行不绝,愈演愈烈。即此可见,算命术对于中华民族的观念文化、心理结构的彩响是极其深刻的。

洪兄丕谟攻古法侓,擅书法,精歧黄,多才多艺,驰誉沪上,近日与其夫人姜君玉珍合著《中国古代算命术》一书,以大纲见寄,嘱为序言。自古医易相通,中医理论与命理学说皆以五行相生相克之理立说,科目虽别,其理一也。洪兄伉俪以其中医学殖,自能深解命理学说,发其底蕴,抉其谬误。然则该书对于当前传统文化之研究,恐不为无补云。是为序。
\begin{flushright}
一九八八年八月十五日写于华东化工学院文化研究所
\end{flushright}

\mainmatter
% \chapter{}
\chapter{算命术的缘起}
\section{天命观的产生和先秦诸子的天命观}

上古时期,在人类生产力极不发达,认识水平极度低下的悄况下,由于不能解释神秘的自然现象和完全把握自己的穷通寿夭,于是就萌发了一种似乎上天有着一种不可抗拒的神秘的力,在支配着世上一切的一切,安排着人类一切的一切的思想。这种思想观点,就是进入阶级社会后古文献上记载的“天”能致命于人,决定人类命运的天命观,或者说是命运观。这用《易经》的话来说,叫做“乾道变化,各正性命”。对于这里的命,后人注释道:“命者,人所禀受,若贵贱夭寿之属也。”

出土的甲骨卜辞、彝器铭文看,“受命于天”刻辞的不只一次出现,说明早在殷周时期,这种天命观就已经在一些人的头脑里扎根了。后来的儒家祖师爷孔子,就是位信命的老夫子。按理说,孔子是个知识渊博的大儒,对于人类社会有者深刻的认识,怎么鱿会信起命来呢?原来,他早年风尘仆仆,奔走列国,到处推销自己的政治主张,很想干一番轰轰烈烈的事业。可是到了后来,当他碰了一鼻子灰以后,才深深地省悟到,命运之神竟然是如此这般的厉害,然而这时他已是个五十左右的人了。“五十而知天命”就是他从不知命到知命这一思想转化过程的最好说明。与此同时,他不仅“知命”,并且还是个怕命的人,不然,他就不会说出“畏天命”这样的话来了。有意味的是,除了自己“知命”、“畏天命”外,他和他弟子还不遗余力大肆宣扬“死生有命,富贵在天”,“不知命,无以为君子”,“君子居易以俟命,小人行险以侥幸”的思想。这里他的说教是,一个人的生死贫富,都是命里早就注定了的,作为一个君子来说,非得“知命”不可,否则就够不上做“君子”的资格。正因为君子是“知命”的,所以他能安份守己,服从老天爷的安排,但是小人却不这样,他们不肯听从天命,往往冒险强求,希望有幸能得个好结果。

当然,看问題也不能攻其一点,不及其余。《孔子集语》记录孔子的话说:“古圣人君子博学深谋不遇时者众矣,岂独丘(我孔丘)哉!贤不肖者才也,为不为者人也,遇不遇者时也,死生者命也。”这里,他认为贤和不肖是根据才华来划分的,干和不干是人们自己可以把握的,至于机遇好和不好,是死还是活,那就只得看时运和老天的旨意了。可见他在主张服从天命的同时,又是主张发挥人的主观能动性的。因为人的才华和努力是一回事,境遇和死生的命运安排毕竟又是一回事。

作为儒家学派的创始人,孔子的这种天命思想,又在后来大儒孟子身上得到了新的反映。《孟子•万章》上篇说:“莫之为而为者,天也;莫之致而至者,命也。”意思就是,没有人叫他干,而他竟干了,这就是天意;没有人叫他来,而他竟来了,就是命运。同时他还举例说明,尧、舜的儿子都不肖,是因为舜、禹为相的时间太长,所以尧、舜的儿子不有天下;禹的儿子启贤能,而益为相的时间又短,所以启能得到天下。以上这些,都不是人力所为而自为,不是人力所致而自至。从理来说,这属于天意,对人来说,这属于命运。天和命,实在是一致的。在《孟子•尽心上》中,孟子还说:“夭寿不贰,修身以俟之,所以立命也。”又说:“莫非命也,顺受其正。是故知命者不立乎岩墙之下。尽其道而死者,正命也;桎梏死者,非正命也。”前者是说,不管命短命长,我都不三心两意,只是培养身心,等待天命,这就是安身立命的方法。后者是说,天底下人的吉凶祸福,无一不是命运,只要顺理而行,接着的就是正命。所以懂得命运的人不站立在有倾倒危险的墙壁下面。因此,尽力行道而死的人所受的是天的正命,犯罪而死的人所受的不是天的正命。这里,孟子虽然认为天命的力量是无可抗拒的,但是不管怎样,我还是应该按照我的仁义而行,不能无缘无故地白白送死。无疑,这对孔子的天命观来说,是有着补充一面的。

那个御风而行,洒脱自在得很的列子,也是个坚信命运的人。在《列子•力命篇》里,他巧妙地通过“力”和“命”的对话,宣传了自己的这一思想。力对命说:“你的功劳怎么比得上我呢?”命回答说:“你对世间事物有什么了不起的功劳,竟想和我老子比起髙低来?”力说,一个人的寿夭穷通,贵贱贫富,正是我力所能及的。”命听了随即反驳道:“彭祖的智恝不及尧舜,而活了八百岁。颜回的才华超出众人,可寿只三十二岁。孔老夫子的德不比诸侯来得差,却在陈、蔡等国遭了难。商纣王的行够不上仁人,但坐上了帝王的宝座。吴国贤公子季札没能在吴国做上官,坏蛋田恒却篡了齐国的政。商朝忠臣叔齐和伯夷饿死在首阳山,鲁国权臣季氏的钱又远远超过了贤士展禽。假如说你力能发挥作用,又为什么要让彭祖长寿颜回夭折,孔子困厄纣王登基,季札低贱田恒高升,伯夷叔齐展禽贫困而季氏富贵呢?”力被命一驳,楞了好一会儿,才开口道:“按照你的说法,我虽然对世间事物没有功劳,可社会上搞得这般模样,你又为什么不出来制止呢?”命从容回答说,世界上的一切事物都得听凭他们自己去变化,长寿的长寿,夭折的夭折,困厄的困厄,通达的通达,贵的贵,贱的贱,富的富,穷的穷,又难道是我所能够改变的吗?”

通过这段风趣的对话,列子表达了这样一种思想,就是世界上种种不合情理的事,都不是人力所能够解决的,因为在命运安排面前,人力原是很渺小有限的。

此外,先秦诸子信命的还很多,而以儒家的势力为最大。这种思想发展到汉代,随若儒家学说的风行天下,儒家天命观的思想,就更深入人心了。淮南王刘安认为,“仁鄙在时不在行,利害在命不在智”,扬雄《法言》也说:“遇不遇,命也。”有人向他问命,他说:“命是上天决定的,不是人为的。人为的称不上命,上天决定的命是逃避不了的。”

王充是东汉著名学者,他不信鬼神,一生反对迷信,可是对于命运,他却坚决主张是客观存在着的。他说:“凡人遇偶(碰上好运)及道累害(遭受灾祸),皆由命也。有死生寿夭之命,亦有贵贱贫富之命。”在他所著《论衡》一书的《命禄》、《气寿》、《幸偶》、《命义》、《无形》、《偶会》、《初禀》等篇章里,随时都可碰上对于命运的论述。可见,人们对于命运的信仰,不只是后来算命家们故弄玄虚所捣的鬼,这里面还有一批大学问家的参与、坚信和鼓吹,可谓盘根错节,阵容坚强。

在这种社会历史背景下,加上人们对种种不合理的社会现象如不该飞黄腾达的反而飞黄腾达,不该久处人下的反而久处人下,不该富贵荣华的反而富贵荣华,不该穷困潦倒的反而穷困潦倒,以及自己难以把握自己的未来、难以把握自己的生活、难以把握自己的生命等种种困惑,得不到使人信服的合理解释,因此便就很自然地信奉起那些大知识分子的说教来。而统治阶级为了坐稳他们的交椅,也乐于接受这一套,并且不择手段地加以推广利用。从此以后,那些时运不济的达人,常常会用“比上不足,比下有余”,“乐天知命”的话来安慰自己,安于现状;而那些想不通的,除了怨天尤人之外,只好自讨不服命的苦吃了。

\section{算命术的起源、发展和成熟}
天命观经过先秦学者的一阵鼓吹,其时从上到下,从统治者到平民百姓,信命的风气一时很盛。还早在殷商时期,迷信的统治者们,就已习惯于在每做一件事之前,总要先占卜一下天意如何,是凶是吉?后来,又由于人与天地相应观念的影响,更使得人们普遍认为,整个天下的命运和每个个人的命运,都和天时星象有关。《周礼•春官》记载,“冯相氏掌十有二岁,十有二月,十有二辰,十日,二十有八星之位,辨其叙事,以会天位。”保章氏掌天星,以志星辰日月之变动,以观天下之迁,辨其吉凶。以星土辨九州之地所封,封域皆有分星,以观妖祥,这是说冯相氏和保章氏,是专管岁时星象,并从而推测人间吉凶祸福的一种职官。

用占卜、占星来测候人事的吉凶祸福,这种方法,只能根据已经发生了的现象进行分析,并且测候的事情,也只能局限在近阶段所要做的事里。随着人们对预知术越来越高的要求和越来越大的胃口,这种原始占卜和占星的方法,已经远远不能满足人们要求预知一生吉凶祸福,贵贱寿夭的难填欲壑。于是,标榜着能预测人们一生征途,包括过去未来的算命术就应运而生了。算命术的产生,是人类社会发展到一定阶段的必然产物,中国如此,外国也如此,只不过是戏法人人会变,各有巧妙不同罢了。

算命的产生既然有着这种历史的必然性,那末,中国算命术产生的理论依据又是什么?自先秦两汉以来,伴随着哲学上阴阳五行学说的确立盛行,哲学家们认为,天地间一切事物的发生、发展和变化,都是阴阳对立平衡,金、木、水、火、土五行相互生发制约的结果。既然天地间事物的发生、发展阴阳五行有关,那末,作为一个小天地的人身,假如有可能推知他与生俱来的五行禀赋,不就可以测知他一生的发展前途了吗?在这种思想支配下,阴阳五行便就自然成了算命家算命的重要理论依据了。

在时间上说确切点,我国算命术的真正起源,大概始于两汉。这反映在文字上的,主要见载在《白虎通义》和王充《论衡》等著作里。王充就宵说过“五行之气,天生万物,以万物含五行之气,五行之气更相贼害”,“欲为之用,故令相贼害,贼害相成也。故天用五行之气生万物,人用万物作万事。不能相制,不能相使,不相贼害,不成为用。金不贼木,木不成用。火不烁金,金不成器。”“且一人之身,含五行之气,故一人之行有五常(五行)之操”,以及“寅木也,其禽(生肖)虎也;戌土也,其禽犬也;丑未亦土也,丑禽牛,未禽羊也。木胜土,故犬与牛羊为虎所服也。亥水也,其禽豕也;巳火也,其禽蛇也;子亦水也,其贪鼠也;午亦火也,其禽马也。水胜火,故豕食蛇;火为水所害,故马食鼠屎而腹胀。”“世曰:‘男女早死者,夫贼妻,妻害夫。’非相贼害,命自然也”(见《物势》、《偶会》等篇)等一系列有关算命的话。这里,王充不但提出了五行算命的根据,并且还进一步触及到了生肖相克和夫妻贼害都是命的理论实践,从而成了我国算命发展史中实际上的先驱者。

但是,从后代人的著述看,一般都把战国时的鬼谷子、珞琭子等人推上算命术祖师爷时宝座。因为没有确切的证据可以证明这种说法,加上先人们有着好古的习惯,一切都喜欢假托古人,并且越古越好,所以这种说法多半出于附会,大致可以肯定下来。

不过,从东汉以后直到六朝的很长一段时间里,算命的方法很是粗糙简单,有的仅仅只就生日那天所碰上的星象来作推测,所以还形不成一个完整的体系。然而,也正由于经过了三国魏晋南北朝一段时期算命家们共同的摸索探讨,到了唐代,才有了一次大的发展,质的飞跃。原因是阴阳五行和一个人出生年月日进一步紧密结合,以推断一生休咎的学说,在李唐年间获得了方法上的确认。同时,随着中外文化的空前交流,印度、西域的占星术也相继传进中土,促进了中土算命术的发展。

从宋元人文献记载得知,按照星象历法推算人命,始于唐代贞元年间,也就是公元785年到805年的一段时间里,当时有西域康居国来的一个名叫李弼乾的术士,传来了印度婆罗门术《聿斯经》。有了原来的气候土壤,再加上外来术数的一促进,中国原有算命术的发展,就振翅起飞了。

在唐代算命术飞速发展,并且正式确立体系的过程中,起关键核心作用的有李虚中、僧一行、桑道茂等人。其中魏郡(治所在今河北大名)人李虚中,字常容。唐德宗贞元年间,李虚中科考顺利,中了进士,后来一直做到殿中侍御史的官。平时,他精究阴阳五行,能够根据一个人出生年、月、日的天干地支,来推定他一生中的贵贱寿夭,吉凶祸福;并且竟然“百不失一”。以上这些,都是唐朝大儒韩愈白纸黑字,在《殿中侍御史李君墓志铭》里明白告诉我们的。由于李虚中本人的本事,再加上韩愈一吹,后人就把他尊为命理学家开山祖师了。因为东汉王充那一套,和李虚中“汪洋奥美,关节开解,万端千绪,参错重出”的学说比较起来,毕竟粗糙得很,还形不成一套完整的体系。李虚中留下的著述,有署名为鬼谷子撰的《命书》三卷中的注释部分。但是后人考证结果,《命书》既不是鬼谷子所撰,就是连注释也不一定出于李虚中之手。

李虚中这种以出生年、月、日天干地支对一个人一生的吉凶祸福进行推测的方法,经过五代宋初人徐子平的进一步发展完善,中国算命术才正式进入成熟完备阶段,并为后来的命理学家所广泛取法。

史书记载,徐子平名居易,曾和当时看相大师麻衣道人陈图南一起隐居华山,精研命理之学。他在算命术上的最大贡献,就是把李虚中推算年、月、日干支的办法,进一步演进为年、月、日、时同时测算的“四柱”法。所谓“四柱”,就是以出生年份的天干地支为第一柱,月份的天干地支为第二柱,日期的天干地支为第三柱,时辰的天干地支为第四柱。这样,每一柱天干一字,地支一字,共两个宇,四柱天干地支加起来的总字数就是八个字。然后再按照这八个字中所蕴含着的阴阳五行进行演算,就可推知一个人一生命运的大致情况了。他的著述,据说有《徐子珞琭子赋注》二卷,见《说郛•巳疟篇》。后来,人们为了纪念他,又有把算命术称作“子平术”的。

徐子平以后,宋代文人信命的记载较之前代,有了更加增多的趋向。苏轼《东坡志林》说到命的有三条,其中一条说韩愈以磨竭为身宫,而他本人也以“磨竭为命,平生多得谤誉,殆是同病”。王辟之《渑水燕谈录》记载渑州三灵山人程惟象,年轻时碰上异人传授命相要诀,后来精益求精,为人推算贵贱寿夭准确率很髙。那时有个名叫张宣徽的,问他一个丁酉人命,他说:“天宾星行初度,不当作内臣,寿数五十四岁。”结果果然被他说中。有趣的是,释文莹在《玉壶清话》中,还提到了一则瞎子算命的事。那时有个刘童子,从小瞎了眼睹,擅长于“声骨及命术”。荆南人夏侯嘉正向他求教,刘童子说:“你将来一定及第,并且有清职。收入除了薪俸,还会有百金的横财。可惜有了横财,寿数就终了。”后来也果然应了他的话。这些记载,都从另一角度,反映了当时学者如苏轼、王辟之、释文莹等人都是信命的,并且在他们的笔记中,似乎还带着明显拔高命术的倾向。

在方法上,徐子平所创立的“四柱”法从宋代开始,已渐渐地大行于天下了。那时,算命不仅是命理学家的事,就是通儒学者,也大多精通命理。如南宋著名理学家朱熹的老朋友徐端叔,就是一个读书也精通命理的人。对于徐端叔命理的大致情况,朱裹在《赠徐端叔命序》中说:“世以人生年月日时所值枝(支)干纳音(五行的一种),推知其人吉凶寿夭穷达者,其术虽若浅近,然学之者亦往往不能造其精微。盖天地所以生物之机,不越乎阴阳五行而已,其屈伸消息,错综变化,固已不可深究,而物之所赋,贤愚贵贱之不同,特昏明厚薄毫厘之差耳,而可易知其说哉。徐君尝为儒,则尝知其说矣。其用志之密微而言之多中也固宜。世之君子倘一过而问焉,岂惟足以信徐君之术而振业之,亦足以知夫得于有生之初者,其赋予分量固已如是,富贵荣显,固非贪慕所得致,而贫贱祸患固非巧力所可辞也。”

从其时著作流传看,现在仍在港台流行着的《渊海子平》就是宋代徐子升根据徐子平命理研究成果,纂辑而成的一部重要著述。

元朝时期,上层统治者虽然由汉贵族换成了蒙古贵族,可是汉族社会中算命的风气却依然盛行不衰。据元末明初陶宗仪《辍耕录》的记栽,元代有个旷达不羁的富家子弟,好几个算命先生为他推算命程的结果,都说他的寿元只有三十岁。富家子弟听后,知道自己将不久人世,就把家里的资财都慷慨地接济了穷人。后来在一个风浪险恶的渡头,这富家子弟又救了个伫立江边,正要纵身跳下翻滚波涛自寻短见的丫髮。事情过去一年以后,富家子一行又来到这风恶浪险的渡头摆渡,正巧上次被他救起的那个丫髮从旁走来,坚决要遨他回家和丈夫一起拜谢他的救命之恩。原来这丫髮后来被主家辞退,已嫁了人。没奈何,富家子只得让其他同行的二十八人先摆渡过去,自己则跟着丫髮来到她家。待到吃过茶点,辞别主人出来,只听得街上沸沸扬扬,才知道刚才摆渡的船只,已被风浪卷没,而同行的二十八人,全都无一生还。这里,陶宗仪在宣扬信命的同时,又掺进了佛家积德可以扭转命运的因果报应思想,色彩极为瑰奇。由于元朝享国不长,命理学著述较少,只李钦夫所撰《子平三命渊源注》少数几种而已。

到了明代,中国算命术在社会流传上,达到了一个前所未有的髙峰。那时的开国功臣宋濂,曾写《禄命辨》一文,第一次系统地总结了我国命理学的历史渊源。而一时间有关的命理著述,也如雨后春笋大量涌现。其中《滴天髄》是否出于明初重臣刘基之手,虽然人们看法不一,可是它的参考价值,却是有目共睹的。此外沈孝瞻的《子平真铨》、万育吾的《三命通会》、张神峰的《神峰通考命理真宗》等,也都是一时的佼佼者。

至清一代,算命术依然盛行不歇,那时社会上人们不管富贵贫贱,男女老少,也不管碰上婚姻、赴考、经营等什么事,都存在着一种请人算算的心理上的强烈欲望。虽然有时他们中的一些人嘴上总会挂起“君子问凶不问吉”的堂皇话,可心里想的,则最好是吉星髙照,洪运亨通。由于知识份子如纪昀、俞樾等人的推波助澜和介入,社会上研究命理的风气十分浓厚,主要著述有陈素庵的《子平约言》、《滴天髄辑要》,任铁樵的《订正滴天髄征义》,以及无名氏所撰先名为《拦江网》,后来又改名为《穷通宝鉴》等多种。

综括以上所说,算命术在东汉末年确立概念,形成雏形以后,经过唐朝李虚中的发展,基本形成了体系,后来又经过徐子平创“四柱”法,从而标志了中国算命术的瓜熟蒂落,正式成熟。徐子平以后,宋元明清,人们寄希望于命运,相信算命术的好似堤岸决口,洪波汹涌,多得不可收拾。就是连天下至尊的皇帝,也还要请人算算,看看往后的结局如何?寿数长还不长?这就证明了即使坐着皇帝宝座的,心里也并不那么踏实,更不要说是深受剥削压迫,生活没有保障的黎民百姓了。

民国以来,大军阀和大官僚们,包括蒋介石本人,也都相信算命。百姓由于生活在水深火热之中,受着寄希望于晚年转运的心理支配,请教算命先生的更是多得难以计数。即使到了现在,在科学经济都已十分昌明繁荣的香港、台湾,人们相信命运的仍然趋之若鸯。

由于习惯势力的根深蒂固,积重难返,即使在大陆上,经过解放多年以来的历史唯物主义和辩证唯物主义教育,要是进行一次民意测验,人们相信命运的肯定还大有人在。当前农村里迷信活动的重新抬头,算命风水先生由冷落而再次走红,就很足以说明这种在祖国土地上盛行了一千多年的算命风气,是多么的难以彻底根除。



\chapter{算命术的基础理论}
\section{天干地支}
李虚中、徐子平创立算命术以后,算命的都少不了要和天干地支打交道,否则便就寸步难行。

“天干地支”,简称“干支”,又称“干枝”。前人有云:“天干,犹木之干,强而为阳;支,犹木之枝,弱而为阴。”可见称为干支的原始用意。

天干的数目有十位,它们的依次顺序是:甲、乙、丙、丁、戊、己、庚、辛、壬、癸。

地支的数目有十二位,它们的依次顺序是:子、丑、寅、卯、辰、、午、未、申、酉、戌、亥。

因为天干地支原是取意于树木,所以,对于它们的原始意义,有这样有趣的说法:

1.天干

〔甲〕象草木破土而萌,阳在内而被阴包裹。

〔乙〕草木初生,枝叶柔软屈曲。

〔丙〕丙,炳也,如赫赫太阳,炎炎火光,万物皆炳然著见而明。

〔丁〕草木成长壮实,好比人的成丁。

〔戊〕茂也,象征大地草木茂盛。

〔己〕起也,纪也,万物抑屈而起,有形可纪。

〔庚〕更也,秋收而待来春。

〔辛〕金味辛,物成而后有味。又有认为,辛者新也,万物肃然更改,秀实新成。

〔壬〕妊也,阳气潜伏地中,万物怀妊。

〔癸〕揆也,万物闭栽,怀妊地下,揆然萌芽。

2.地支

〔子〕莩也,草木种子,吸土中水分而出,为一阳萌生的开始。

〔丑〕草木在土中出芽,屈曲着将要冒出地面。

〔寅〕演也,津也,寒土中屈曲的草木,迎着春阳从地面伸展。

〔卯〕茂也,日照东方,万物滋茂。

〔辰〕震也,万物震起而长,阳气生发已经过半。

〔巳〕起也,万物盛长而起,阴气消尽,纯阳无阴。

〔午〕万物丰满长大,阳气充盛阴气开始萌生。

〔未〕味也,果实成熟而有滋味。

〔申〕身也,物体都已长成。

〔酉〕靖也,万物到这时都緧缩收敛。

〔戌〕灭也,草木凋零,生气灭绝。

〔亥〕劾也,阴气劾杀万物,到此已达极点。

据说,对于这种有趣的天干地支,发明者是四五千年前上古轩辕时期的大挠氏。起先,天干仅是用来记日,因为每个月的天数都是以日进位的;地支用来记月,因为一年十二个月,正好用十二地支来相配。可是随之不久,人们感到单用天干记日,每个月里仍然会有三天同一天干,所以便用一个天干和一个地支分别依次搭配起来的办法来记日期,如《尚书•顾命》就有“惟四月哉生魄,王不怿。甲子,王乃洮颒水,相被冕服,凭玉几”的记载,这用现在的话来说,就是四月初,王的身体很不舒服。甲子这一天,王才沐发洗脸,太仆为王穿上礼服,王依在玉几上坐着。后来,干支记日的办法又被渐渐引进到了记年、记月和记时。这样,干支记年、记月、记日、记时的一整套体系就在实践过程中,渐次地形成了。

十天干和十二地支的最小公倍数是六十,所以它们依次从头结合到底一个循环,通称“六十甲子”。这“六十甲子”的次第是:

\begin{enumerate*}[label=\circled{\arabic*}, afterlabel={}, itemjoin= \qquad]
\item 甲子 \item 乙丑 \item 丙寅 \item 丁卯 \item 戊辰\\
\item 己巳 \item 庚午 \item 辛未 \item 壬申 \item 癸酉\\
\item 甲戌 \item 乙亥 \item 丙子 \item 丁丑 \item 戊寅\\
\item 己卯 \item 庚辰 \item 辛巳 \item 壬午 \item 癸未\\
\item 甲申 \item 乙酉 \item 丙戌 \item 丁亥 \item 戊子\\
\item 己丑 \item 庚寅 \item 辛卯 \item 壬辰 \item 癸巳\\
\item 甲午 \item 乙未 \item 丙申 \item 丁酉 \item 戊戌\\
\item 己亥 \item 庚子 \item 辛丑 \item 壬寅 \item 癸卯\\
\item 甲辰 \item 乙巳 \item 丙午 \item 丁未 \item 戊申\\
\item 己酉 \item 庚戌 \item 辛亥 \item 壬子 \item 癸丑\\
\item 甲寅 \item 乙卯 \item 丙辰 \item 丁巳 \item 戊午\\
\item 己未 \item 庚申 \item 辛酉 \item 壬戌 \item 癸亥
\end{enumerate*}
    
以上“六十甲子”,每个单位都可按照先后顺序分别代表不同的年、月、日、时。比如以日为例,清代道光二十二年壬寅(1842)农历四月十一日是己亥,那末四月十二日、十三日就可顺次推知为庚子、辛丑……,四月十日、九日就可逆次推知为戊戌、丁酉……。这样六十甲子循环往复,周而复始,以至无穷。

早在春秋战国时期,历法混乱,夏历、殷历和周历同时并存。三种历法之间的主要区别是每年开头的月建不同。秦始皇统一中国后,曾改用夏历建亥之月的十月为一年的开头。此后直到公元前104年汉武帝改用太阳历,才正式确定以夏历建寅之月的正月,作为一年的开头。打这以后的二千年间,除了王莽、魏明帝时一度改用殷历,唐代武则天和唐肃宗一度改用周历,一般都以夏历的寅月作为一年的起始。现将夏历建寅之月作为岁首的月份和地支对照名称列表如下:

\begin{table}[H]
\setlength{\tabcolsep}{0.2em} % 表格内容水平padding
\centering\footnotesize
\begin{tabular}{*{12}{c|}c} % 等同于 \begin{tabular}{c|c|c|c|c|c|c|c|c|c|c|c|c}
\hline
月份&正月&二月&三月&四月&五月&六月&七月&八月&九月&十月&十一月&十二月\\
\hline
地支&寅&卯&辰&巳&午&未&申&酉&戌&亥&子&丑\\
\hline
\end{tabular}
\end{table}

记月之外,古人还用十二地支记日,这样一个昼夜下来,就是十二个时辰。这用现在的时间概念来说,每个时辰恰好等于两个小时。所谓“小时”,就是“小时辰”,也就是“半个时辰”的意思。时辰和小时的对照情况,仍用清晰的表格形式反映如下:

\begin{table}[H]
\setlength{\tabcolsep}{0.2em} % 表格内容水平padding
\centering\footnotesize
\begin{tabular}{*{12}{c|}c}
\hline
十二时辰&子&丑&寅&卯&辰&巳&午&未&申&酉&戌&亥\\
\hline
俗称&夜半&鸡鸣&平旦&日出&食时&隅中&日中&日昳&晡时&日入&黄昏&人定\\
\hline
现代钟点\footnotemark&23-24&1-2&3-4&5-6&7-8&9-10&11-12&13-14&15-16&17-18&19-20&21-22\\
\hline
\end{tabular}
\end{table}
\footnotetext{现代钟点表中的24、2、4、6、8、10、12、14、16、18、20、22等逢双钟点中,都包含着直至每一钟点终结的59分59秒等整个钟点在内。}

表里的俗称 ,是指十二时辰在古代的一种通俗叫法。这种俗称,主要是借助一些自然特征和生物特征来表示的。“鸡鸣”、“人定”,借助于半夜鸡叫和人入夜睡觉的特征。“食时”、“晡时”,借助吃饭时刻表示时间。古人一日两餐,早饭在日出以后,隅中以前,所以称这段时间为“食时”;晚饭在日昳(太阳偏西)以后,日入以前,所以称这段时间为“晡时”。除此之外,其余留下的八个时间俗称,多半以太阳位置为主要特征来表示的。这里要注意的是对于子时的划分,因为上半时在夜半前,所以属上一天,下半时在夜半后,所以属下一天。

在十二时辰化为现代时间上,如果碰上夏时制,我们可以作相应的调整,如原来的寅时是凌晨3点到4点59分59秒,夏时制则可顺推为4点到5点59分59秒。其他时辰也按照这个规律类推。

值得一提的是,用十二地支所记月、时是固定不变的,如子月必定是十一月,子时必定是夜半23到24点\footnote{23:00:00—24:59:59},然而与之相配的天干却不是一成不变的,它们循环往复的顺序是如前所说的“六十甲子”。

在年份上,满六十甲子称为一个花甲。人们平时常说“年逾花甲”,就是超过六十岁的意思。记年时,满一个甲子后再从头算起,也和记日一样,周而复始,如环无端。如咸丰十年(1860)是庚申年,那末咸丰十一年顺推就是辛酉年,咸丰九年倒推就是己未年。隔六十年后又从庚申开始,再依次记叙下去。

干支记年的办法和现在记年的办法比较起来,虽然笨拙得多,但它在我国历史上,却差不多一直沿用到清朝灭亡以前的整个历史时期。

\section{阴阳五行}
我国古代,阴阳五行是个哲学概念。用这个概念,可以概括天地自然和人类社会的一切。

《易经》说:“无极生太极,太极生两仪。”这两仪就有阴阳的含义在内。早先,阴阳只是作为太阳日光向背的意义而出现的,向日的叫阳,背日的叫阴。不久,又引申解释为气候的寒与暖。后来随着人们认识的不断提高,就把阴阳作为一种哲学概念,用来广泛解释自然界和人类社会两种互相对立消长,矛盾而又统一着的动态平衡势力。比如日月、昼夜、明暗、动静、内外、寒热、雌雄、男女、刚柔、迟速等等,都可分成阴阳两个方面,然而这两个方面又是协调统一,相反相成的。因为有着这个原因,所以《易传》有“一阴一阳之谓道”的说法。这个道,指的就是天地自然变化发展的基本规律。

也正因为阴阳广泛包涵着事物对立统一的两个方面,因此反过来说,世界上的任何事物也都可分阴阳两个方面,这说明明阳这种现象是无所不在的。就举一本书为例吧。书本的封面是阳,背面是阴,书本的表面是阳,里面是阴。如果打开书本,暴露在光下的内页是阳,翻到背面的封面又变成了阴。这又进一步说明,阴阳也并不是一成不变的,它可以随着外界条件的转化而转化,所以《老子》说:“万物负阴而抱阳。”

当然,如果再深入一层,天地万物的阴面可以包涵着阳,天地万物的阳面同样也可包涵着阴。《黄帝内经》有言:“平旦至日中,天之阳,阳中之阳也;日中至黄昏,天之阳,阳中之阴也;黄昏至合夜,天之阴,阴中之阴也;合夜至鸡鸣,天之阴,阴中之阳也。”又说:“阳在外,阴之使也;阴在内,阳之守也。”

这种阴阳的概念早先原本是才朴素的,唯物的。到了战国末期,由于以邹衍为代表的阴阳家“乃深观阴阳消息,而作怪迂之变”,也就从这以后,本来质朴无华的“阴阳”,多少地给抹上了一重神秘的油彩。

同祥如此,五行的早期也是朴素而又唯物的。《尚书·洪范》曾这样记载说:“五行,一曰水,二曰火,三曰木,四曰金,五曰土,水曰润下,火曰炎上,木曰曲直,金曰从革(顺从人的要求变革形状),土爰稼穑(指庄稼)。润下作咸,炎上作苦,曲直作酸,从革作辛,稼穑作甘。”古人认为,天地万物都是由金、木、水、火、土等五种基本物质组成的。由于这五种基本物质的运动变化,从而构成了丰富多彩的物质世界。

战国时期,“五行”学说很是风行一时,并且还进一步总结摸索出了一套“五行相生相胜”的原理。所谓“相生'就是一种物质对另一物质有着生发促进的作用,如木能生火便是;所谓“相胜”,也就是“相克”的意思,就是一种物质对另一种物质有着克制约束的作用,如水能克火便是。正因为广泛存在在自然界的五行有着这种相生相胜的相互作用,所以天地万物才得到了动态的平衡。否则只生不克,或者只克不生,想要这天地万物维持下去,简直是不可想象的事。

对于这种五行生克的规律,古人的说法是:

1.相生:木生火,火生土,土生金,金生水,水生木。

2.相克:木克土,土克水,水克火,火克金,金克木。



\begin{center}
\begin{tikzpicture}
[post/.style={->,shorten <=2pt,shorten >=2pt,>={Stealth[round]},thick},
postxu/.style={->,shorten <=113pt,shorten >=2pt,>={Stealth[round]},thick},
xuxian/.style={dashed,thick,shorten <=2pt,shorten >=5pt},
place/.style={circle,draw=black!100,thick,inner sep=0pt,minimum size=8mm}]
\node [place] (a) at (2.5,4.755){木};
\node [place] (b) at (5,2.939){火};
\node [place] (c) at (4.045,0){土};
\node [place] (d) at (0.955,0){金};
\node [place] (e) at (0,2.939){水};
\draw [post] (a) to (b);
\draw [post] (b) to (c);
\draw [post] (c) to (d);
\draw [post] (d) to (e);
\draw [post] (e) to (a);
\draw [xuxian] (a) to (c);
\draw [postxu] (a) to (c);
\draw [xuxian] (b) to (d);
\draw [postxu] (b) to (d);
\draw [xuxian] (c) to (e);
\draw [postxu] (c) to (e);
\draw [xuxian] (d) to (a);
\draw [postxu] (d) to (a);
\draw [xuxian] (e) to (b);
\draw [postxu] (e) to (b);

\node (f) at (2,-1){相生};
\draw [->,>={Stealth},thick] (0.5,-1) -- (f);
\node (g) at (4.5,-1){相克};
\draw [dashed,thick] (3,-1) -- (3.8,-1);
\draw [->,>={Stealth},thick] (3.8,-1) -- (g);
\end{tikzpicture}
\end{center}

上面这张附图,就是根据这木火土金水的五行生克关系绘制出来的。口诀是,顺次相生,隔一相克。

为什么顺次相生呢?《命理探原》的解释是:“木生火者,木性温暖,火伏其中,钻灼而生,故木生火;火生土者,火热故能焚木,木焚而成灰,灰即土也,故火生土;土生金者,金居石依山,津润而生,聚土成山,土必生石,故土生金;金生水者,少阴之气温润流泽,销金亦为水,故金生水;水生木者,因水润而能出,故水生木也。”

为什么又隔一相克呢?《白虎通义》的认识是:“五行所以相害(相克)者,天地之性,众胜寡,故水胜(克)火也;精胜坚,故火胜金,刚胜柔,故金胜木;专胜散,故木胜土;实胜虚,故土胜水也。”

后来,随着唯心主义思想家的出现,尤其是命理学家的出现,五行也和阴阳一样,被披上了一件眩人眼目的神秘外衣,变得难以捉摸了。

\section{天干地支和阴阳五行的配合}
命理学家认为,天地万物的发展变化既然和阴阳五行的变化生克有着密不可分的联系,那么“人身一小天地”,通过对一个人出生年月日时干支所涵阴阳五行不同变化的推测,不就可以推知他一生的吉凶祸福了吗?

对此,王充《论衡•初禀》篇说:“人生性命当窗贵者,初禀自然之气,养育长大,富贵之命效矣。”“命谓初所禀得而生也,人生受性则受命矣,性命俱禀,同时并得,非先禀性,后乃受命也。”接若他还举例说:“文王在母身之中已受命也。”认为一个人的富贵贫贱,早在父母交合之时就已决定了的,不管将来长大后操行如何,都没有办法改变。然而,王充的这种说法,给后世的命理学家推命带来了莫大的困难,虽然元代孔齐《至正直记》曾有过这样的记载:“前辈多言推人五行定休咎,今以受胎日时为准,但以所生时甲子合,得十月数某甲子是也。如甲子则推己丑(原注:甲与己合,子与丑合),乙丑则庚子之类(原注:乙与庚合,子与丑合)也。又云,唐宫如此。未详。”可是这是在十月怀胎正常分娩情况下所作的一种推算受胎时间法,如若逢上早产、晚产,那就只好束手无策了。

《古今名人命鉴》是民国命理学家东河乐吾三十年代的著作,我们再来看看他的说法。在自序中,他是这样分析的:“佛言四大(地水风火),儒言五行,人之一身由四大和合而成,亦即五行秉赋而成也。光热为火,润泽为水,流动为风,质实为地。而儒家五行之分类,除水火相同外,金属为金,纤维质为木,不属于金木之质为土,故土又名杂气,此与今之科学家人体物质之分析,固有不谋而合者也。”接着他笔锋一转,拦入正题:“人之秉受不同,其原因固安在乎?曰由于感受太阳之光线、星球之吸力随时有不同也。春之气和煦,秋之气肃杀,夏热而冬寒,此显而易见者。”所以“凡人脱离母体之时”,“得气之厚,神守气足则寿;得气之强,体大用宏则贵。反是则不永其年,或所为辄阻,贫贱夭折,必居其一。”显然,这里作者认为,只有五代徐子平根据出生年月日时所立的“四柱”推荈法,才是符合科学而合理可行的。

那么,“凡人脱离母体之时”的年、月、日、时的天干地支又是怎样和阴阳五行配合挂钓的呢?

先说阴阳。干支和阴阳的配合比较简单,不比干支和五行配合那样来得复杂多变。具体划分是,根据阳数奇(单)而阴数偶(双)的原则,十个天干和十二地支里,凡是逢单的都属阳,逢双的都属阴。这可用下面的图表加以表示:
\begin{table}[H]
\setlength{\tabcolsep}{2em} % 表格内容水平padding
\centering\footnotesize
\begin{tabular}{c|c|c}
\hline
&天干&地支\\
\hline
阳&甲丙戊庚壬&子寅辰午申戌\\
\hline
阴&乙丁己辛癸&丑卯巳未酉亥\\
\hline
\end{tabular}
\end{table}

再说五行。在十天干中,五行的分配是甲、乙属木,丙、丁属火,戊、己属土,庚、辛属金,壬、癸属水。在十二地支中,五行的分配是寅、卯、辰属木,巳、午、未属火,辰、戌、丑、未属土,申、酉、戌属金,亥、子、丑属水。现列表如下,看起来可方便清楚些。
\begin{table}[H]
\setlength{\tabcolsep}{2em} % 表格内容水平padding
\centering\footnotesize
\begin{tabular}{c|c|c}
\hline
&天干&地支\\
\hline
木&甲\quad{}乙&寅卯辰\\
火&丙\quad{}丁&巳午未\\
土&戊\quad{}己&辰戌丑未\\
金&庚\quad{}辛&申酉戌\\
水&壬\quad{}癸&亥子丑\\
\hline
\end{tabular}
\end{table}

这里,天干的五行要比地支的强些,再加上干支的阴阳不同,因此同样是木,可又是不全相同的。比如天干的甲乙木和地支的寅卯木不同,而同是天干的甲乙木,因为甲是阳木,属于森林之木,乙是阴木,属于花草灌木,所以也有着一定的区别。对于这种五行在天干上的阴阳大小不同,古人大致有这样一种说法:

甲木——森林之木。\par
乙木——花草之木。\par
丙火——太阳之火。\par
丁火——灯盏之火。\par
戊土——大地之土。\par
己土——田园之土。\par
庚金——斧钺之金。\par
辛金——首饰之金。\par
壬水——大海之水。\par
癸水——雨露之水。

从地支来说,寅、卯、辰虽说同样是木,但寅是初生之木,卯是极盛之木,辰是渐衰之木。同样,从火来说,巳是初生之火,午是极盛之火,未是渐衰之火;从金来说,申是初生之金,酉是极盛之金,戌是渐衰之金;从水来说,亥是初生之水,子是极盛之水,丑是渐衰之水。至于辰、戌、丑、未四季,非但有属土之称,并且另外还有着四库的说法。其中“丑为金库,生亥子而克寅卯;辰为水库,生寅卯而克巳午;未为木库,生巳午而受金克;戌为火库,克申金而受水制”(《三命通会》卷五)。正因为这样,辰、戌、丑、未又称杂气。关于“寄旺于四季”,是指土寄旺于季春、季夏、季秋、季冬一年四季的最后一个月说的,也就是说,土寄旺于春天的三月,夏天的六月,秋天的九月,冬天的十二月。

比较复杂的是,地支的五行不象天干那样,甲木就是甲木,丙火就是丙火,而是在一定程度上除了本气外,还包含着一个或几个天干的五行成分在内。比如寅支,里面除了含有本气天干甲木之外,还兼有着丙火和戊土的成分在内。所谓本气,就是每一地支中所藏足以代表自己性质的一个天干。在十二地支的本气中,寅的本气是甲木,卯的本气是乙木,辰的本气是戊土,巳的本气是丙火,午的本气是丁火,未的本气是己土,申的本气是庚金,酉的本气是辛金,戌的本气是戊土,亥的本气是壬水,子的本气是癸水,丑的本气是己土。对于地支所栽本气和其他天干,另有古歌一首道:
\begin{center}
子宫癸水在其中,丑癸辛金己土同。\par
寅宫甲木秉丙戊,卯宫乙木独相逢。\par
辰藏乙戊三分癸,巳中庚金丙戊丛。\par
午宫丁火并己土,未宫乙己丁共宗。\par
申位庚金壬水戊,酉宫辛字独丰隆。\par
戌宫辛金及丁戊,亥藏壬甲是真踪。\par  
\end{center}


为了清晰起见,现将十二地支所含天干五行列成表格:
\begin{table}[H]
\setlength{\tabcolsep}{0.2em} % 表格内容水平padding
\centering\footnotesize
\begin{tabular}{*{12}{m{2em}<{\centering}|}m{2em}<{\centering}}
\hline
地支&子&丑&寅&卯&辰&巳&午&未&申&酉&戌&亥\\
\hline
所含天干五行&癸水&癸水辛金己土&甲木丙火戊土&乙木&乙木戊土癸水&庚金丙火戊土&丁火己土&乙木己土丁火&庚金壬水戊土&辛金&辛金丁火戊土&壬水
甲木\\
\hline
\end{tabular}
\end{table}

除了天干地支和五行的这种正规配合之外,还有一种把六十甲子和五音十二律结合起来,其中一律含五音,总数共为六十的“纳音五行”对此,古歌有云:
\begin{center}
甲子乙丑海中金,丙寅丁卯炉中火,\par
戊辰己巳大林木,庚午辛未路旁土,\par
壬申癸酉剑锋金,甲戌乙亥山头火,\par
丙子丁丑涧下水,戊寅己巳城头土,\par
庚辰辛巳白腊金,壬午癸未杨柳木,\par
甲申乙酉泉中水,丙戌丁亥屋上土,\par
戊子己丑霹雳火,庚寅辛卯松柏木,\par
壬辰癸巳长流水,甲午乙未沙中金,\par
丙申丁酉山下火,戊戌己亥平地木,\par
庚子辛丑壁上土,壬寅癸卯金箔金,\par
甲辰乙巳复灯火,丙午丁未天河水,\par
戊申己酉大驿土,庚戌辛亥钗钏金,\par
壬子癸丑桑柘木,甲寅乙卯大溪水,\par
丙辰丁巳沙中土,戊午己未天上火,\par
庚申辛酉石榴木,壬戌癸亥大海水。\par
\end{center}

对于这种“纳音五行”的宜忌,《三命通会》有着较为详细的分析。这里可以明显看出的是五行的花头比起阴阳来,真是复杂难弄得多了。但是过去学算命的为了混口饭吃,只要咬紧牙关硬挺一下,也就闯过去了。

关于正五行和纳音之间的关系,徐子平专用正五行,后来因为用正五行算常常与实际有所出入,于是便用纳音五行作为补充。因为这个原因,所以两者之间的关系可以把它看作是五行为经,纳音为纬。正如《命理探原》所说的那样:“大概看日元之强弱,定用神之得失,皆以正五行为主。若欲补偏补弊,酌盈济虚,又当参看年、月、日、时之纳音。”

\section{五行和四时五方}
五行观念,是中国算命术中最为关键核心的观念,但它们在使用时,却并不是孤立的,在很大程度上要和一年的春、复、秋、冬四时,以及方位的东、南、西、北、中五方紧密结合起来,通盘考虑,才能过细入微。

这原因主要是因为,五行在一年四季和方位的定向当中,各自有着它们所旺的季节和所主的方向,现在我们且看下表:
\begin{table}[H]
\setlength{\tabcolsep}{0.2em} % 表格内容水平padding
\centering\footnotesize
\begin{tabular}{c|c|c|c|c}
\hline
五行&所旺的四季&所主的方位&天干&地支\\
\hline
木&春&东&甲\quad{}乙&寅卯辰\\
火&夏&南&丙\quad{}丁&巳午未\\
金&秋&酉&庚\quad{}辛&申酉戌\\
水&冬&北&壬\quad{}癸&亥子丑\\
土&旺于四季&中&戊\quad{}己&辰戌丑未\\
\hline
\end{tabular}
\end{table}

表里土旺于四季的“四季”,和春夏秋冬的四季解释不同。这里的“季”,原是“末”的意思。所谓“四季”,指的就是四个季度的最后一个月。比如春天三个月,可依次分别称作孟春、仲春、季春;夏天三个月,可分别称作孟夏、仲夏、季夏,秋天三个月,可分别称作孟秋、仲秋、季秋;冬天三个月,可分别称作孟冬、仲冬、季冬。对于“土旺于四季”,前人也有说成是“土寄旺于四季”的。

那末,命理学家又是怎样在算命时根据一个人出生时间的五行,和四季结合起来推算吉凶的呢?这里自有他们的口诀。现把《穷通宝鉴》所说的那一套,条陈如下:

1.\quad{}论四时之木

\mbox{\circled{1}}春月之木,犹有余寒。得火温之,始无盘屈之患,得水润之,乃有舒畅之美。然水多则木湿,水缺则木枯,必须水火既济方佳。至于土多则损力堪虞,土薄则财丰可许。如逢金重,见火无伤;假使木强,得金乃发。

\mbox{\circled{2}}夏月之木,根干叶燥。由曲而直,由屈而伸。喜水盛以润之,忌火炎以焚之。宜薄土不宜厚土,厚则为灾,恶多金不恶少金,多则受制。若夫重重见木,徒自成林;叠叠逢华,终无结果。

\mbox{\circled{3}}秋月之木,形渐凋零。初秋则火气犹在,喜水土以资生,中秋则果实已成,爱刚金以砍削;霜降后不宜水盛,水盛则木漂,寒露前又宜火炎,火炎则木实。木多有多材之美,土厚无自立之能。

\mbox{\circled{4}}冬月之木,盘屈在地。欲土多以培养,恐水盛则亡形。金纵多,克伐无害;火重见,温暖有功。归根复命之时,木病安能辅助。惟忌死绝,只宜生旺。

2.\quad{}论四时之火宜忌

\mbox{\circled{1}}春月之火,母旺子相,势力并行。喜木生扶,不宜过旺,旺则火炎;欲水既济,不宜太多,多则火灭。土多则晦,火旺则亢。金可以施功,纵叠见富余可望。

\mbox{\circled{2}}夏月之火,势力当权。逢水制,则免自焚之咎;见木助,必遭夭折之忧。遇金必发,得土皆良。然金土虽为美利,无水则金燥土焦。若再火盛,太过必致倾危。

\mbox{\circled{3}}秋月之火,性息体休。得木生,则有复明之庆;遇水克,难逃熄灭之灾。土重掩光,金多夺势,火见火以光辉,虽叠见亦有利。

\mbox{\circled{4}}冬月之火,体绝形亡。喜木生而有救,遇水克以为殃。欲土制为荣,爱火比为利。见金则难任为财,无金则不遭磨折。

3.\quad{}论四时之土宜忌

\mbox{\circled{1}}春月之土,其势最孤,喜火生扶,忌木克削。甚比助力,忌水扬波。得金制木为强,金重又盗土气。

\mbox{\circled{2}}夏月之土,其性最燥。得盛水滋润成功,见旺火亢燥为害。木助火炎,生克不取。金生水足,财禄有余。见比肩蹇滞不通,如太过又宜木袭。

\mbox{\circled{3}}秋月之土,子旺母衰。金多则盗泄其气,木盛则制伏纯良。火重不厌,水泛非祥。得比肩则能助力,至霜降不比无妨。

\mbox{\circled{4}}冬月之土,外寒内温。水旺财丰,金多身贵。火盛有荣,木多无咎。再逢土助尤佳,惟喜身强益寿。

4.\quad{}论四时之金宜忌

\mbox{\circled{1}}春月之金,寒未尽,贵乎火气为荣,体弱性柔,欲得土生乃妙。水盛则金寒,有用等于无用。木盛则金折,至刚转为不刚。金来比助,扶持最喜。比而无火,失类非良。

\mbox{\circled{2}}夏月之金,尤为柔弱。形质未备,更忌身衰。水盛呈祥,火多不妙。遇金则扶持精壮,见木则助鬼伤身。土厚埋没无光,土薄资生有益。

\mbox{\circled{3}}秋月之金,当权得令。火来锻炼,遂成钟鼎之材;土复资生,反有顽浊之气。见水则精神越秀,逢木则琢削施威。金助愈刚,过刚则折。

\mbox{\circled{4}}冬月之金,形寒性冷。木多则难施斧凿之功,水盛则不免沉潜之患。土能制水,金体不寒。火来生土,子母成功。喜比肩类聚相扶,欲官印温养为妙。

5.论四时之水宜忌
\mbox{\circled{1}}春月之水,性滥滔淫。若逢土制,则无横流之害;再逢水助,必有崩堤之忧。喜金生扶,不宜金盛;欲火既济,不宜火炎。见木施功,无土散漫。

\mbox{\circled{2}}夏月之水,外实内虚。时当涸际,欲得比肩。喜金助体,忌火旺太炎。木盛则耗泄其气,土盛则克制其源。

\mbox{\circled{3}}秋月之水,母旺子相。得金助则清澄,逢土旺则混浊。火多而财盛,太过不宜;木重而身荣,中和为贵。重重见水,增其泛滥之忧;叠叠逢土,始得清平之象。

\mbox{\circled{4}}冬月之水,正应司权。遇火除寒,见土归宿。金多反致无义,木盛是为有情。水太微则喜比为助,水太盛则喜土为堤。

现在再说五行与五方。按照前表所列,我们可以把五行五方和天干地支捏在一起,凑成几句便于记忆的话。那就是:

东方甲乙寅卯水,\par
南方丙丁巳午火,\par
西方庚辛申酉金,\par
北方壬癸亥子水,\par
中央戊己辰戌丑未土。

为什么五行和五方要作这样的联系呢?原因是木的禀性温和向阳,而东方正是太阳初升的地方,所以木便和东方结了缘份;火的禀性是炎热盛长的,而南方气候炎热,有利于万物生长,所以火便和南方结了缘份;金的禀性是清凉肃杀的,而西方正是太阳落山,草木不生的地方,所以金便和西方结了缘份;水的禀性是澄澈寒冷的,而北方水冰地寒,所以水便和北方结了缘份;土的禀性厚实适中,有利于万物生长,中央地处东南西北的中间,所以土便和中央结了缘份。

说到五行和四时五方的关系,我们还可结合八卦方位,画成下面这样一个图表,以便观览。

\begin{center}
\begin{tikzpicture}
\draw [thick] (7em,0) rectangle (14em,21em);
\draw [thick] (0,7em) rectangle (21em,14em);
\draw [thick] (0,7em) -- (7em,0);
\draw [thick] (0,14em) -- (7em,21em);
\draw [thick] (14em,0) -- (21em,7em);
\draw [thick] (14em,21em) -- (21em,14em);

\node at (10.5em,21.8em) {坎};
\node at (10.5em,20.2em) {北方壬癸水};
\node at (10.5em,18.6em) {子};
\node at (10.5em,16em) {十一月};

\node [rotate=45] at (2.935em,18.065em) {乾};
\node [rotate=45] at (4.065em,16.935em) {戌亥};
\node [rotate=45] at (5em,16em) {九月十月};

\node [rotate=90] at (-0.8em,10.5em) {兑};
\node [rotate=90] at (0.8em,10.5em) {西方庚辛金};
\node [rotate=90] at (2.4em,10.5em) {酉};
\node [rotate=90] at (5em,10.5em) {八月};

\node [rotate=135] at (2.935em,2.935em) {坤};
\node [rotate=135] at (4.065em,4.065em) {未申};
\node [rotate=135] at (5em,5em) {六月七月};

\node [rotate=180] at (10.5em,-0.8em) {离};
\node [rotate=180] at (10.5em,0.8em) {南方丙丁火};
\node [rotate=180] at (10.5em,2.4em) {酉};
\node [rotate=180] at (10.5em,5em) {五月};

\node [rotate=225] at (18.065em,2.935em) {巽};
\node [rotate=225] at (16.935em,4.065em) {辰巳};
\node [rotate=225] at (16em,5em) {三月四月};

\node [rotate=270] at (21.8em,10.5em) {震};
\node [rotate=270] at (20.2em,10.5em) {东方甲乙木};
\node [rotate=270] at (18.6em,10.5em) {卯};
\node [rotate=270] at (16em,10.5em) {二月};

\node [rotate=315] at (18.065em,18.065em) {艮};
\node [rotate=315] at (16.935em,16.935em) {丑寅};
\node [rotate=315] at (16em,16em) {十二月正月};

\node at (10.5em,12.3em) {中央戊己};
\node at (10.5em,10.5em) {辰戌丑未};
\node at (10.5em,8.7em) {土};

\end{tikzpicture}
\end{center}

命理学家把五方观念引进命里,主要是因为通过对一个人生辰八字阴阳五行的推算,可以看出他行运的方向,以及该在什么方位生活活动,才最为有利。比如有的人利于行东方木运,不利于西方金运,如果行运一碰上西方金运,就会倒运。又如在外出上,有利于行东方木运的,最好往东方跑,如果硬要去西方,往往不利。至于其他利于南方,不利于北方等等,可以按照这个原则类推。

\section{五行的旺相休囚死和寄生十二宫}
五行的“旺相休囚死”也是和四时密切相关,并被命理学家谈得较多的一个问题。这里面总的精神,就是在春夏秋冬四个季节里,每个季节都有一个五行处于“旺”,一个五行处于“相”,一个五行处于“休”,一个五行处于“囚”,一个五行处于“死”的状态。

那末,什么叫做旺、相、休、囚、死呢?解释是:

〔旺〕处于旺盛状\par
(相〕处于次旺状态。\par
〔休〕休然无事,亦即退休。\par
〔囚〕衰落被囚。\par
〔死〕被克制而生气全无。

现在我们把五行在四时中的旺、相、休、囚、死,简括如下:
\begin{table}[H]
\centering
\setlength{\tabcolsep}{0.5em} % 表格内容水平padding
\begin{tabular}{cccccc}
〔春〕&木旺&火相&水休&金囚&土死\\
〔夏〕&火旺&土相&木休&水囚&金死\\
〔秋〕&金旺&水相&土休&火囚&木死\\
〔冬〕&水旺&木相&金休&土囚&火死\\
〔四季〕&土旺&金相&火休&木囚&水死
\end{tabular}
\end{table}


从上简括,我们可以明显看出这样的规律,就是当令的旺,我生的相,生我的休,克我的囚,我克的死。比如用木举例,春天是木当令的季节,所以木旺;火是木生出来的,所以火相;水是生木的母亲,现在木已长成旺盛之势,母亲便可退居一旁,所以水休;春木旺盛,金已无力克伐,所以靠边站而金囚;土是木所克的,现在木既当令,气势强旺,所以土死。在具体应用中,一个人如果春天出生,八字中以木为主的,就是当令得时,八字中以金为主的,就是被囚而不得时了。其他依次类推。为了便于记忆,我们现在再反过来,以五行为主线,分别把它们处在四季旺相休囚死状态括要如下:
\begin{table}[H]
\centering
\setlength{\tabcolsep}{0.5em} % 表格内容水padding
\begin{tabular}{cccccc}
〔木〕&春旺&冬相&夏休&四季囚&秋死\\
〔火〕&复旺&春相&四季休&秋囚&冬死\\
〔土〕&四季旺&夏相&秋休&冬囚&春死\\
〔金〕&秋旺&春囚&夏死&四季相&冬休\\
〔水〕&冬旺&四季死&春休&夏囚&秋相\\
\end{tabular}
\end{table}

我们在拖得五行的旺、相、休、囚、死后,还不要忘了五行寄生十二宫的原理,因为这更是搞命理深入下去必不可少的一种理论指导。

五行寄生十二宫的原理,也就是每一个具体五行在十二个月中从生长到死亡过程的原理。按照《三命通会》的说法,十二宫的名称和解释是:

〔绝〕又叫“受气”,或”、“胞”,“以万物在地中,未有其象,如母腹空,未有物也”。

〔胎〕就是“受胎”,“天地气交,氤氳造物,其物在地中萌芽,始有其气,如人受父母之气也”。

〔养〕就是“成形”,“万物在地中成形,如人在母腹成形也”。

〔长生〕“万物发生向荣,如人始生而向长也”。

〔沐浴〕又叫“败”,“以万物始生,形体柔脆,易为所损,如人生后三日,以沐浴之,几至困绝也”。

〔冠带〕“万物渐荣秀,如人具衣冠也”。

〔临官〕“如人之临官也”。

〔帝旺〕“万物成熟,如人之兴旺也”。

〔衰〕“万物形衰,如人之气衰也”。

〔病〕“万物病,如人之病也”。

〔死〕“万物死,如人之死也”。

〔墓〕又叫“库”,“以万物成功而藏之库,如人之终而归墓也”。

为了明白起见,现将五行寄生十二宫“长生”、“沐浴”、“冠带”,临官”、“帝旺”等情况列表如下:
\begin{table}[H]
\centering
\setlength{\tabcolsep}{0.1em} % 表格内容水padding
\begin{tabular}{*{10}{c|}c}
\hline
&\multicolumn{5}{c|}{五阳干顺行}&\multicolumn{5}{c}{五阴干逆行}\\
\hline
&甲木&丙火&戊土&庚金&壬水&乙木&丁火&己土&辛金&癸水\\
\hline
长生&亥&寅&寅&巳&申&午&酉&酉&子&卯\\
沐浴&子&卯&卯&午&酉&巳&申&申&亥&寅\\
冠带&丑&辰&辰&未&戌&辰&未&未&戌&丑\\
临官&寅&巳&巳&申&亥&卯&午&午&酉&子\\
帝旺&卯&午&午&酉&子&寅&巳&巳&申&亥\\
衰&辰&未&未&戌&丑&丑&辰&辰&未&戌\\
病&巳&申&申&亥&寅&子&卯&卯&午&酉\\
死&午&酉&酉&子&卯&亥&寅&寅&巳&申\\
塞&未&戌&戌&丑&辰&戌&丑&丑&辰&未\\
绝&申&亥&亥&寅&巳&酉&子&子&卯&午\\
胎&酉&子&子&卯&午&申&亥&亥&寅&巳\\
养&戌&丑&丑&辰&未&未&戌&戌&丑&辰\\
\hline
\end{tabular}
\end{table}

表里,五阳干和五阴干中的甲木、乙木等,指的是出生那一天天干的五行,而亥、子、丑、寅、卯等十二地支,则又分别指的是出生的月份。通过这表,我们可以得知,出生一天的天干如果是甲木,出生的月份是十月亥月,那么这甲木就处于万物发生向荣的“长生”状态。如果出生一天的天干换成乙木,而月份则仍是十月亥月不变,那末这乙木就处于万物临终的“死”的状态。这就告诉我们,出生一天的天干同样是木,但属于阳木的甲木是生,属于阴木的乙木却是死,彼此互换。原理是:“阳之所生,即阴之所死。”反过来,甲木死于午,则乙木必生于午。这个规律用另外一句话来说,就是“阳干顺行,阴干逆行”,这样彼此交叉下来,就是“阳临官则阴帝旺”,如阳木甲木临官是寅,那末对于乙木来说,寅就变成乙木的帝旺了。

用出生一天干支对照出生月份,找出寄生十二宫的各种状态,从而象征一个人的命运好运,虽然未免简单,但算命家们却用得很是普遍。可话也得说回来,这种办法也不是绝对的,还要参考其他的众多因素才能决定,所以《三命通会》在《论五行旺相休囚死并寄生十二宫》结束语中说道:“凡推造化,见生旺者未必便作吉论,见休囚死绝未必便作凶言。如生旺太过,宜乎制伏;死绝不及,宜乎生扶。妙在识其通变。古以胎、生、旺、库为‘四贵’,死、绝、病、败为‘四忌’,馀为‘四平’,亦大概而言之。”可见还是比较活泛的。


\section{天干地支的刑冲害化合}
在天干地支中,彼此之间的刑、冲、害、化、合,是看命的重要依据之一,所以不得不谈。

先说刑。刑就是彼此刑妨,互不相和的意思。按照命书的说法,十二地支共有三刑,就是:

子卯,一刑也;\par
寅巳申,二刑也;\par
丑未戌,三刑也。

这就是说,一个人八字的地支中如果有子、卯两个地支碰在一起,或者寅、巳、申三个地支碰在一起,或者八字中虽然没有这种现象,可是在流年或大运中碰上的,都可认为是刑。其中子卯相刑,有“无礼”的说法,所以“女命见之,尤为不良。”

然而,对于刑,也要根据一个人八字的具体情况进行分析,不可一见有刑就认为是不吉之兆,所以“鬼谷遗文”说:
\begin{tightcenter}
    君子不刑定不发,若居士途多腾达。\\
    小人到此必为灾,不然也被宫鞭挞。
\end{tightcenter}
可见对于“三刑”,在封建社会是君子得之则吉,小人得之则凶的。

接着说冲。冲有天干相冲和地支相冲的不同。天干相冲的有甲庚、乙辛、壬丙、癸丁四对关系。因为东甲西庚,东乙西辛,北壬南丙,北癸南丁,方向两两相对,性质截然相反,所以就冲了起来。对于天干中的丙庚、丁辛,固然彼此间有着相克的关系,可是丙南庚西,丁南辛酉,方向对不起来,所以只克不冲。此外,甲庚、乙辛、壬丙、癸丁相冲的理由还有,甲庚都属于阳,乙辛都属于阴,壬丙都属于阳,癸丁都属于阴,阳阳与阴阴同性相拆,配不起来,也是一个原因,不比甲己、乙庚、丙辛、丁壬、戊癸,即使相克,然而克而不拆,所以还是合化起来,成了夫妻。至于戊己居中,没有方向上的相对,所以也就没有冲。

在十二地支相冲中,因为每隔六位数就要彼此冲激起来,所以叫做“六冲”。比如子午相冲,子代表水,午代表火,并且方向相对,故而就冲了起来。六冲的具体情况是:
\begin{enumerate}[label=\circled{\arabic*},parsep=0pt,topsep=0pt,itemsep=0pt,itemindent=2em]
    \item 子午相冲;
    \item 丑未相冲;
    \item 寅申相冲; 
    \item 卯酉相冲; 
    \item 辰戌相冲; 
    \item 巳亥相冲。
\end{enumerate}
这组概念,从方位来说,都是对的;就五行来说,都是克的;就阴阳而言,都是阳克阳,阴克阴,阴阳不能配合,所以就冲了起来。

在命理学中,六冲是一个很重要的概念。就一般情况言,六冲给人留下的印象似乎并不太好,可是在命书里,也常作具体分析。比如辰戌丑未彼此两两相冲,但在十二地支中,辰戌丑未都被解作是库,也就是仓库的意思。“四库所藏,为十干财官印绶等物,尤喜冲激”(《三命通会》卷二),因为仓库平时总是锁着,里面虽有库藏,可就是只好望断秋水,到不了手。现在经过一冲,库里的财官印绶都被冲了出来,这对于命的主人来说,不就成了好事?此外如命里“寅申巳亥全,子午卯酉全”,也是一种大的格局,不作破败定论。按照《三命通会》的说法,六冲中怕就怕冲得不全,或者同类相冲。所谓同类相冲,就是八字干支的天干相同,地支相冲,比如甲子见甲午,己卯见己酉之类,其中甲和甲同,但子和午却冲了起来,己和己相同,但卯和酉却冲了起来。如果是犯着这类相冲的,怕就不太妙了,即使一时禄高名重,将来也难免终有一失。

在旧社会,男女缔结婚姻,人们一般都力求避开六冲。这也是命理学家带给民俗上的一种迷信影响,不足为据。话虽这么说,可做起来却并不那么容易,可见封建迷信一旦和民俗结合起来,其势力就是那么的根深蒂固,难以铲除。

再说害。害又名穿,就是彼此损害的意思。命书记载,害也有六种情况,就是:
\begin{enumerate}[label=\circled{\arabic*},parsep=0pt,topsep=0pt,itemsep=0pt,itemindent=2em]
    \item 子未相害;
    \item 丑午相害;
    \item 寅巳相害; 
    \item 卯辰相寄; 
    \item 申亥相害; 
    \item 酉戌相害。
\end{enumerate}
不过六冲和六害比较起来,六冲是每隔六个地支就要两两相冲,这样十二地支就有六对相冲。而六害的数目虽然也是六对,但《三命通会》又说:“六,六亲。害,损也。犯之,主六亲上有损克,故谓六害。”可知,六冲、六害虽然都不是好事,但具体情况却不一样。

至于化,这是就十个天干而说的。命书上说,十个天干两两相化,共有五种情况:
\begin{enumerate}[label=\circled{\arabic*},parsep=0pt,topsep=0pt,itemsep=0pt,itemindent=2em]
    \item 甲己化土;
    \item 乙庚化金;
    \item 丙辛化水;
    \item 丁壬化木;
    \item 戊癸化火;
\end{enumerate}
因为化的条件是合,只有合起来才能化,所以化又称“合”或“合化”《三命通会》卷二有《论十干合》一篇,就是说的这种情況。书中认为,所谓“合”,就是“和谐”的意思,为什么合化以后就和谓了呢?书中的说法是,东方甲乙木最怕西方庚辛金来克,甲是阳木为兄,乙是阴木为妹,于是甲木就想方设法把妹妹乙木嫁给阳金庚做老婆,这不就阴阳和合了吗?古来女子嫁鸡随鸡,嫁犬随犬,所以乙木嫁庚金后,就从一而终地跟了庚金了。

末了说合。在十二地支中,又有六合和三合的不同。其中六合是:
\begin{enumerate}[label=\circled{\arabic*},parsep=0pt,topsep=0pt,itemsep=0pt,itemindent=2em]
    \item 子丑合土;
    \item 寅亥合木;
    \item 卯戌合火;
    \item 辰酉合金;
    \item 巳申合水;
    \item 午未为太阳太阴。
\end{enumerate}
《三命通会》在《论支元六合篇》里说夫合者,和也。乃阴阳相和,其气自合。子、寅、辰、午、申、戌六者为阳,丑、卯、巳、未、酉、亥六者为阴,是以一阴一阳和而谓之合。”至于为什么一定要子丑合土,寅亥合木,这除了这些地支中含有相应五行外,还有一个“气数中要占阳气为尊”的原理。比如子为一阳,丑为二阴,一和二合起来是阳数的单数三;寅为三阳,亥为六阴,三和六合起来是阳数的单数九等等。其实这种说法,也是很牵强的。此外结合看命,六合中还有合禄、合贵、合马,以及男子忌合绝,女子忌合贵的说法,也是很有趣的。

三合和六合的字面含义不同。六合是十二地支两两相合,总起来的数目是六,三合则不是这样,主要是说十二地支中三个三个合起来的意思。三合的内容是:
\begin{enumerate}[label=\circled{\arabic*},parsep=0pt,topsep=0pt,itemsep=0pt,itemindent=2em]
    \item 申子辰合水;
    \item 亥卯未合木;
    \item 寅午戌合火;
    \item 巳酉丑合金。
\end{enumerate}
三合在五行中没有土的原因是,水、木、火、金四行都要依赖土才能形成格局,这就是万物都归藏于土的原理。至于辰戌丑未凑在一起,那就自然合成土局了。

据说一个人生辰八字和三合配合得好的,还可以出现一种“三会禄格”的格局。假如有幸得了这种格局,那末“折月中之仙桂”,就肯定没有疑问了。

天干地支的刑、冲、害、化、合看来内容较多,但旧时算命先生为了终身受用,也是不得不掌提的。


\section{十二生肖和地支}
生肖,这一人类社会的奇特现象,不仅中国有,就是外国也有,不过只是具体的动物有所不同罢了。费尔巴哈在《费尔巴哈哲学著作选集》下卷指出:“人之所以为人要依靠动物,而人的生命和存在所依靠的东西,对于人来说就是神。”由此可见,生肖起源于古代人们对动物的崇拜,大概不会有多大疑问。

所谓“生肖”,就是生下来的那年类属于什么动物的意思。因为从字面上解释,肖就是象,比如子年生的肖鼠,就是说,逢子年出生的,不管是甲子、丙子、戊子,还是庚子、壬子,都可能生性多疑好动,胆小警觉觉,类似于鼠的性格。按照古代术数家的说法,地支有十二,生肖也有十二,彼此一一相配,哪一年出生的人肖什么,都有一定的规定。现将十二地支配合十二种生肖的规定对照如下:
\begin{table}[H]
\centering
\begin{tabular}{*{12}{c|}c}
\hline
生年&子&丑&寅&卯&辰&巳&午&朱&中&酉&戌&亥\\
\hline
生肖&鼠&牛&虎&兔&龙&铊&马&羊&猴&鸡&狗&猪\\
\hline
\end{tabular}
\end{table}

按照这个对照表,我们便可知道哪一地支年份出生的人肖什么了。至于子年生的为什么肖鼠,丑年生的为什么肖牛,寅年生的为什么肖虎,卯年生的为什么肖兔,它们的排列顺序又为什么要这样,术数家们虽有解释,可却牵强附会,生硬得很。明代李诩《戒庵老人漫笔》卷七引王文恪公的话认为,十二生肖的排列顺序和天上二十八宿星象的位序相合。他的解释是二十八宿分布周天,以值十二时辰。每个时辰二宿,子午卯酉三宿,而各有所象。把二十八宿的禽名简化,按天空次序排列,由北而东,而南,而西,这样周转下来,正好是十二生肖绕天一周。

危(燕)、虚(鼠)、女(编)都值子时,取鼠代表子。\par
牛(牛)、斗(邂)都值丑,取牛代表丑。\par
箕(豹)、尾(虎)都值寅,取虎代表寅。\par
心(狐)、房(兔)、氐(貉)都值卯,取兔代表卯。\par
亢(龙)、角(蛟)都值辰,取龙代表辰。\par
轸(蝴)、翼(蛇)都值巳,取蛇代表巳。\par
张(鹿)、星(马)、柳(獐)都值午,取马代表午。\par
鬼(羊)、井(犴)都值未,取羊代表未。\par
参(猿)、觜(猴)都值申,取猴代表申。\par
毕(乌)、昴(鸡)、胃(雉)都值酉,取鸡代表酉。\par
娄(狗)、奎(狼)都值戌,取狗代表戌。\par
壁(貐)、室(猪)都值亥,取猪代表亥。

这种把十二生肖和天上二十八宿联系起来,从而与十二时辰地支发生关系的解释,早在南宋朱熹《十二辰诗》中已经有了隐约的苗子。诗中,朱熹从半夜听老鼠咬席子写起,写一天亮牛就去耕地,一直写到“客来犬吠催煮茶,不用东家卖猪肉”,从而把十二个动物一天的活动,按时间顺序叙述了一遍。这里,朱熹也认为十二生肖是以活动时间来排列顺序的。

我国古代“肖”又叫“属”。《周书•宇文护传》说: “生汝兄弟,大者属鼠,次者属兔,汝身属蛇。”这里所说的“属”,就是“肖”的意思。因此古代“十二生肖”,又常称为“十二相属”,或者称为“十二属相”。

从文献看,东汉王充《论衡•物势》,蔡邕《月令问答》,晋代葛洪《抱朴子•登涉》等篇都有零星记载,根据这些记栽,清朝的赵翼在《陔余丛考》卷三十四中断定,十二相属正式的说法起于东汉,因为在这以前并没有什么人系统提起过这玩意儿。然而却也有人认为,十二生肖早在西周或春秋战国时候就已有了,比如《诗经•小雅•吉日》的“吉日庚午,既差我马”,把午和马对应联系起来;又如《左传•僖公五年》有“龙尾伏辰”这样的话,把辰和龙对应起来。然而由于这种对应只是偶然零星的记载,形不成完全的系统,所以只好暂且悬在那里再说。

自从算命术盛行以后,和十二地支有联系的生肖,自然逃不了被涂上一重迷信色彩的命运。在算命先生的影响下,人们一遇休戚祸福,总会情不自禁把自己和周围人的属相牵扯联系起来,甚至在婚配里也要注意避开男女双方生肖的冲撞,什么“鸡狗断头婚”,“龙虎不相容”等等,就是对婚姻而说的。再如在社会上,属虎的人常为人们所畏忌,尤其是属虎的女人,简直没人敢要她,否则发起雌威,谁受得了。

由于十二生肖是和十二地支联系在一起的,所以十二地支的冲害学说,就自然和生肖的冲宵捆到了一起我们且看下列两组对照:
1.十二地支相冲和生肖相冲
\begin{enumerate}[label=\circled{\arabic*},parsep=0pt,topsep=0pt,itemsep=0pt,itemindent=2em]
    \item 地支子牛相冲,生肖鼠马相冲,
    \item 地支丑未相冲,生肖牛羊相冲;
    \item 地支寅申相冲,生肖虎猴相冲;
    \item 地支卯酉相冲,生肖兔鸡相冲;
    \item 地支辰戌相冲,生肖龙狗相冲;
    \item 地支巳亥相冲,生肖蛇猪相冲。
\end{enumerate}
2.十二地支相害和生肖相害
\begin{enumerate}[label=\circled{\arabic*},parsep=0pt,topsep=0pt,itemsep=0pt,itemindent=2em]
    \item 地支子未相害,生肖鼠羊相害;
    \item 地支丑午相害,生肖牛马相害;
    \item 地支寅巳相害,生肖虎蛇相害;
    \item 地支卯辰相害,生肖兔龙相害;
    \item 地支申亥相害,生肖猴猪相害;
    \item 地支酉戌相害,生肖鸡狗相害。
\end{enumerate}

在这种观念支配下,属龙的人怕和属狗的结婚,属蛇的人怕和属猪的结婚,属鸡的人怕和属兔的结婚,认为这样双方会冲起来,对男方和女方都没好处。至于鼠羊相害,牛马相寄,虎蛇相寄等等,情况虽没相冲严重,可是为了趋吉避凶,结婚时对于大多数人来说,也是不大肯冒这个险的。当然这种无稽之谈的迷信说法,到了今天,市场已经显得愈来愈小了。

\chapter{四柱算命的具体方法}
\section{怎样排八字}

命理学家看命,先要排出一个人出生年、月、日、时的干支。年、月、日、时加起来共四项,徐子平称为“四柱”,每柱一个天干一个地支,共八个字,所以叫做“八字”。算命排出八字,是最事关重要的。八字排出以后,再根据八字之间五行生克等千变万化的关系,从而推论一个人一生的吉凶祸福。如果排不出八字,那就一切都落空了。

那末,怎样排出年、月、日、时四柱的八字呢?

1.推年法\quad{}算命根据的是农历,农历哪一年出生的,那么这一年的干支就是本人年柱上的干支。比如庚辰龙年、出生的,他年柱的干支就是庚辰,辛巳蛇年出生的,他年柱的干支就是辛巳,其他类推。

推算年柱的方法大致有三种:
\begin{enumerate}[label=\circled{\arabic*},parsep=0pt,topsep=0pt,itemsep=0pt,itemindent=2em]
    \item 査看万年历,比如你只知道是公元1940年生,而不知道这一年农历叫什么年,那末打开《新编万年历》一稽,就知道这一年是庚辰年了。
    \item 没有万年历的,可以自己排一张六十花甲表,然后根据当年的干支和自己的虚龄反推上去,也可得出结论。
    \item 三是用手指推算,顺推反推,这在瞎子中用得最为纯熟。
\end{enumerate}

推年不管用什么方法,都必须严格划定以农历的立春作为一年的界限。比如正月立春后生的,用本年干支;虽然生在正月,可时间却在立春之前(还没到立春),那末就得算到上年出生而用上一年的干支作为年柱了。同样道理,虽然同是出生在农历十二月的,可是立春前生的用本年干支,立春后生的就要划到下一年去了。

2.推月法\quad{}推月的办法虽然每月的地支是固定的,如前文《天干地支》所说正月寅月,二月卯月,三月辰月,四月巳月便是,可是月份的天干却不固定,要经过一定的推算才能排出。推算的歌诀是:
\begin{tightcenter}
甲己之年丙作首,乙庚之年戊为头,\\
丙辛必定寻庚起,丁壬壬位顺行流,\\
更有戊癸何方觅,甲寅之上好追求。
\end{tightcenter}
具择办法是,如农历甲午年四月生的,那末先根据歌诀“甲己之年丙作首”推出作首的正月丙寅月,然后再顺次推出二月丁卯,三月戊辰,四月己巳,可知这一年四月的干支就是己巳了。现把歌诀化成简表,以便査阅(见下表)。

\begin{table}[H]
\setlength{\tabcolsep}{0.2em} % 表格内容水padding
\centering\small
\begin{tabular}{m{4em}<{\centering}|*{11}{m{2em}<{\centering}|}m{2em}<{\centering}}
\hline
生年天干&正月&二月&三月&四月&五月&六月&七月&八月&九月&十月&十一月&十二月\\
\hline
甲己&丙寅&丁卯&戊辰&己巳&庚午&辛未&壬申&癸酉&甲戌&乙亥&丙子&丁丑\\
乙庚&戊寅&己卯&庚辰&辛巳&壬午&癸未&甲申&乙酉&丙戌&丁亥&戊子&己丑\\
丙辛&庚寅&辛卯&壬辰&癸巳&甲午&乙未&丙申&丁酉&戊戌&己亥&庚子&辛丑\\
丁壬&壬寅&癸卯&甲辰&乙巳&丙午&丁未&戊申&己酉&庚戌&辛亥&壬子&癸丑\\
戊癸&甲寅&乙卯&丙辰&丁巳&戊午&己未&庚申&辛酉&壬戌&癸亥&甲子&乙丑\\
\hline
\end{tabular}
\end{table}

从表中我们可以看出月份的干支也和年份一样,从丙寅月起六十甲子周转下来,重新又回到丙寅月,这时已经是五年过去了。因为按照五年十二个月算,六十甲子周转卞来,不就正好是五年吗?

这里必须注意的是,一定要注意结合节气来推算月份。在一年二十四个节气里,立春、惊蛰、清明、立夏、芒种、小暑、立秋、白露、寒露、立冬、大雪、小寒是节,雨水、春分、谷雨、小满、夏至、大暑、处暑、秋分、箱降、小雪、冬至、大寒是气。而推月则严格以节作为界。如在本月节前生的,就用上个月的干支,本月下一个节后生的,也就是下一个月的节提前来到本月,就得用下个月的干支。因为在通常情况下,一个月只有一个节和一个气,然而有时又有打乱了的。比如公元1986年丙寅年正月廿六出生,査《新编万年历》,这一天正好是下个月的惊蛰节提前来到,那就算不得是庚寅正月出生,而要算到下月辛卯二月出生了。现将二十四节气和月份的分配情况,列表如下:
\begin{table}[H]
\setlength{\tabcolsep}{0.2em} % 表格内容水padding
\centering\small
\begin{tabular}{m{2em}<{\centering}|*{11}{m{2em}<{\centering}|}m{2em}<{\centering}}
\hline
月份&正月&二月&三月&四月&五月&六月&七月&八月&九月&十月&十一月&十二月\\
\hline
节&立春&惊蛰&清明&立夏&芒种&小暑&立秋&白露&寒露&立冬&大雪&小寒\\
气&雨水&春分&谷雨&小满&夏至&大暑&处暑&秋分&霜降&小雪&冬至&大寒\\
\hline
\end{tabular}
\end{table}

这就是说,命书中十二个月的划定应该是这样的:
\begin{enumerate}[label=\circled{\arabic*},parsep=0pt,topsep=0pt,itemsep=0pt,itemindent=2em]
\item 〔正月寅月〕立春经雨水到交惊蛰为止。
\item 〔二月卯月〕惊蛰经春分到交清明为止。
\item 〔三月辰月〕清明经谷雨到交立夏为止。
\item 〔四月巳月〕立复经小满到交芒种为止。
\item 〔五月午月〕芒种经夏至到交小暑为止。
\item 〔六月未月〕小暑经大暑到交立秋为止。
\item 〔七月申月〕立秋经处暑到交白露为止。
\item 〔八月酉月〕白露经秋分到交寒露为止。
\item 〔九月戌月〕寒露经霜降到交立冬为止。
\item 〔十月亥月〕立冬经小雪到交大雪为止。
\item 〔十一月子月〕大雪经冬至到交小寒为止。
\item 〔十二月丑月〕小寒经大寒到交立春为止。
\end{enumerate}

为了便于记忆,有二十四节气歌一首道:
\begin{tightcenter}
正月立春雨水节,二月惊蛰及春分,\\
三月清明并谷雨,四月立夏小满方,\\
五月芒种与夏至,六月小暑大暑当,\\
七月立秋兼处暑,八月白露秋分忙,\\
九月寒露还霜降,十月立冬小雪张,\\
子月大雪共冬至,腊月小寒大寒昌。\\
\end{tightcenter}

3.推日法\quad{}如果手头有本万年历,推日的办法比较简单,只要一査一推,就可知道每天的具体干支了。比如科学普及出版社出版的《新编万年历》,就把公元1840年庚子到公元2000年庚辰一百六十年间农历每月初一、十一、二十一的干支写得明明白白,使用起来只要根据天干地支的顺序一推便知。如以民国二十九年庚辰年九月初十为例,现在査知这一年的九月十一日是丁亥日,那末倒推一天便可得知初十那天就是丙戌日了。

4.推时法\quad{}推时法和推月法一样,要有个周折。因为时柱下面的一个地支是已知的,即由要求算命的人自报,即使他报的是现代的钟点,也无关紧要。但时柱上面的天干,就要费一番踌躇了。好在推得的办法,也就是遁得的办法,并不复杂,只要知道出生一天的干支,就可根据口诀求得了。歌曰:
\begin{tightcenter}
甲己还生甲,乙庚丙作初,\\
丙辛从戊起,丁壬庚子居,\\
戊癸何方发?壬子是其途。\\
\end{tightcenter}
这就是说,天干是甲、己天出生的,那末他出生时间便从甲子(半夜23点-1点钟)开始推算。如仍以刚才九月初十生人为例,现知道此人是辰时生,而九月初十是丙戌,那末根据歌诀“丙辛从戊起”,也就是从戊子起开始推算,依次是己丑、庚寅、辛卯、壬辰。这样推算下来,可以得知那人生时的干支,就是壬辰了。
为了便于査阅,现将根据出生日天干推时的口诀,化成表格(见下页)。

根据表格,如乙、庚等日辰时(7—9时以前)生的,我们便可査得时柱的干支是庚辰;丁、壬等日亥时生的,同样可以査得时柱的干支是辛亥。
    
\begin{table}[H]
\centering\small
\setlength{\tabcolsep}{0.1em} % 表格内容水padding
\begin{tabular}{c|*{11}{m{2em}<{\centering}|}m{2em}<{\centering}}
\hline
\diagbox{\scriptsize{生日天干}}{\scriptsize{生时干支}}{\scriptsize{生时地支}}&子时&丑时&寅时&卯时&辰时&巳时&午时&未时&申时&酉时&戌时&亥时\\
\hline
甲己&甲子&乙丑&丙寅&丁卯&戊辰&己巳&庚午&辛未&壬申&癸酉&甲戌&乙亥\\
乙庚&丙子&丁丑&戊寅&己卯&庚辰&辛巳&壬午&癸未&甲申&乙酉&丙戌&丁亥\\
丙辛&戊子&己丑&庚寅&辛卯&壬辰&癸巳&甲午&乙未&丙申&丁酉&戊戌&己亥\\
丁壬&庚子&辛丑&壬寅&癸卯&甲辰&乙巳&丙午&丁未&戊申&己酉&庚戌&辛亥\\
戊癸&壬子&癸丑&甲寅&乙卯&丙辰&丁巳&戊午&己未&庚申&辛酉&壬戌&癸亥\\
\hline
\end{tabular}   
\end{table}

现在根据八字中年柱、月柱、日柱、时柱的推排方法,试综合推算一下民国二十九年九月初十辰时出生的“四柱”。首先我们可以按照万年历査得,民国二十九年就是夏历的庚辰年,可见这人的年柱是“庚辰”两字。接着根据推月法的口诀或表格,査得九月戌月的干支属于“丙戌”。因为九月初十的这一天处在寒露和霜降之间,所以属于的的确确的九月“丙戌”无疑。再接下来是排日柱,翻开《新编万年历》公元1940年(民国二十九年)庚辰年九月十一日的干支是丁亥,现在初十倒推一位,便可査知属于丙戌。末了再根据出生“丙戌”日的天干,推出辰时的干支属于“壬辰”。这样,庚辰年、丙戌月、丙戌日、壬辰时的四柱八字就被推排出来了。过去,命理学家写“四柱”的八字,总是从$\begin{matrix}\mbox{壬}&\mbox{丙}&\mbox{丙}&\mbox{庚}\\\mbox{辰}&\mbox{戌}&\mbox{戌}&\mbox{辰}\\\end{matrix}$右到左竖着写过来的,现在我们不妨写成这样:

(年柱)庚辰\par
(月柱)丙戌\par
(日柱)丙戌\par
(时柱)壬辰\\
八字排出来后,我们便可走下一步棋了。

\section{推算大运、小运、流年和命宫}
了解八字的推算方法以后,接下来还得交待一下大运、小运、流年和命宫推法。

所谓大运,就是一生中哪一阶段走运,哪一阶段不走运的意思。因此按照命理学家的说法,“命运”两个字除了经常合在一起解释外,还可拆开来解释。其中“命”管人的一生,主要体现在八字里,“运”管人一生中的各个阶段,主要体现在由八字基础上推算出来的大运里。按照前人说法,对于一个人来说,运是很重要的:有的人八字虽生得好,可就是一直不走运,在衣食无愁中碌碌无为地过一辈子;有的人八字虽然生得一般,甚至还有破缺,经常处于逆境之中,可就是碰上那么一二次大运,从而干出一番出人头地的事业。当然此外还有命好运好,命坏运坏,或先好后坏,先坏后好等种种不同,情况千变万化,难以悉举。

那末,怎样排算大运的起运岁数呢?唯一的依据就是,假如是天干逢甲、丙、戊、庚、壬等阳年生的男性,或者天干逢乙、丁、己、辛、癸等阴年生的女性,从本人生日的那天起顺数,到下一个节止,以三天为一岁;反过来,假如是阳年生的女性或阴年生的男性,就要反其道而行之,从本人生日的那天起,逆数到上一个节止,也是三天为一岁。多下来的一天抵四个月,一个时辰抵十天。这里注意的是,在农历二十四个节气里,只有立春、惊蛰、清明、立夏、芒种、小暑、立秋、白露、寒露、立冬、大雪、小寒等十二个才能算作节,其余的只能称为气。这就是说,每个月中只有一个节和一个气,一年十二个月合起来,就成了二十四个节气。

举个例说,假如是公元1943年农历癸未年八月二十日生的女命,按照规定起运的岁数应该从生日那天起顺数到下一个节。査《新编万年历》,癸未年的八月初九是白露,九月十一日是寒露,现在八月二十日处在白露和寒露二个节之间,倒数是白露节,顺数是寒露节,可知顺数的下一个节应该是寒露节。因为那一年的八月是小月,从八月二十日顺数到九月十一寒露节的天数是二十天。这时,起运的岁数便可用三去除,得出的结果是六岁又八个月。也就是说,如果要给这位女性看命,她的起运岁数就得从六岁零八个月看起。

除了算出起运的岁数外,我们还必须排一排大运的天干地支。大运的干支是根据生月的干支推排出来的。起运岁数如果是顺数的,就由生月干支的下一个干支依次顺排下去;逆数的,就由生月干支的上一个干支依次倒排上去。比如生月是丁卯,那末顺数的大运干支就依次是戊辰、己巳、庚午、辛未……,逆数的大运干支就依次是丙寅、乙丑、甲子、癸亥……。命书规定,大运的每个干和支各管五年吉凶,看天干时可结合地支一起看,看地支时因为天干的五年已经过去,所以就撇开不管了。这里假设一个人三岁起运,顺数的大运干支是戊辰、己巳、庚午、辛未等等,那末这就是说,他三到十二岁的大运是戊辰,十三到二十二岁的大运是己巳……。看他三到八岁的大运好坏,需结合戊辰二个字的五行一起分析,接下来看八到十二岁,就要撇去戊宇的五行,单轮一个辰宇所含的五行就够了。

大运之外,还有一种小运。因为孩子如果还没交上大运,小运可以补大运的不足。比如有人八岁起运,那八岁以前的吉凶,除了配合流年太岁,还可参看小运。

关于小运的起法,有的认为男起丙寅顺行,女起壬申逆行,有的认为“甲子旬,男起丙寅,女起壬申;甲申旬,男起丙戌,女起壬辰;甲午旬,男起丙申,女起壬寅;甲辰旬,男起丙午,女起壬子;甲寅旬,男起丙辰,女起壬戌”。然而这些办法不是刻板,就是有背圣人原立起运的深义,所以都不及醉醒子的小运推法那样容易被人接受。醉醒子的方法是,以时辰的干支作为出发基点,男命阳年生的顺推,男命阴年生的倒推,女命阴年生的顺推,阳年生的倒推。庚辰年甲子时生的男命,按照上而阳年生男命从时柱出发顺推的原则,依次为一岁小运乙丑,二岁小运丙寅,三岁小运丁卯,一直接到大运。也就是说,这命从降生到世上的第一天起,就行乙丑的小运了。	

在叫法上,小运又叫行年。好多命理学家认为,当孩子行运进入大运后,也要结合小运一起观察,如果大运虽吉,小运不通,不可一下就认为是吉;反过来说,如果大运虽凶,小运吉利,也不可一下就认定是凶。然而从习惯上来说,古人对小运是并不十分重视的。

所谓流年,比较简单,就是根据农历甲子、乙丑、丙寅、丁卯……的年份,哪一年去算命哪一年就算是流年。比如有一命,去算那年的干支是丙寅,那末流年就是戸寅,去算那年的干支是丁卯,那末流年就是丁卯。有时人们为了问问当年的吉凶,这时看命的人就常会先看他出生那天的天干,然后结合流年,在命书上注明“流年丙寅,××主事”等字样。还是举例能说明问题:有人丙寅年去算命,而他出生那天的天干逢庚,那末看命人就会批上“流年丙寅,偏官主事”。因为丙火克庚金,命理学家称之为偏官,这在下面命理学家关于五行生克术语的章节里,我们还要细谈。

末了再说命宫推法。推命宫一般可以利用自己的指掌。如以左手为例,无名指末节近掌侧横纹是子位,中指末节近掌侧横纹是丑位,食指末节近掌侧横纹是寅位,食指末节上端横纹是卯位等等,这样顺时针一周下来,正好是地支的十二位数。关于十二地支位数和月份的位置,命书上另有如下图的逆向分配法。

\begin{table}[H]
\centering
\begin{tabular}{cccc}
\large{巳}\footnotesize{八月}&\large{午}\footnotesize{七月}&\large{未}\footnotesize{六月}&\large{申}\footnotesize{五月}\\
\large{辰}\footnotesize{九月}&&&\large{酉}\footnotesize{四月}\\
\large{卯}\footnotesize{十月}&&&\large{戌}\footnotesize{三月}\\
\large{寅}\footnotesize{十一月}&\large{丑}\footnotesize{十二月}&\large{子}\footnotesize{正月}&\large{亥}\footnotesize{二月}\\
\end{tabular}   
\end{table}


有了这基础后,我们就可以分两步走来推算命宫的干支了。比如庚辰年十月辰时生,第一步先算命宫地支,算法是先由图上子位作正月逆推上去,或对照附图,可知十月的地支处在卯位;接着再将卯位作为出生的辰时,开始作顺时针计数,数到卯时,正好停在十二地支的寅位上面。这样推出来的“寅”字,就算是命宫的地支了。又如甲子年三月酉时生,第一步是先由图上子位作正月逆推上去,或对照附图,可知三月的地支处在戌位;接着再将戌位作为出生的酉时,顺时针计数,一直数到卯时为止,这样酉、戌、亥、子、丑、寅、卯依次数来,卯正好停在了十二地支的辰位上面,于是这“辰”字就是算命宫的地支了。这里所要注意的关键问题是,当逆推得到月支位置后,即将此支改作出生的时辰,并顺时针序依次数到“卯”字为止。

命宫地支推出以后,第二步是再推命宫的天干。推法是根据前面所说“甲己之年丙作首”的歌诀或图表(参见《怎样排八字》篇),这样我们就可分别找出庚辰年十月辰时生的命宫天干,是和地支“寅”字相配的“戊”字,以及甲子年三月酉时生的命宫天干,是和地支“辰”字相配的“戊”字了。

命宫找出来后,算命时便可在命书后面批上“安命戊寅”、“安命戊辰”或“安命寅宫”、“安命辰宫”等字样了。

在大多数情况下,为了简便起见,命理学家有时常将命宫略去不推,这又存乎其人了。

此外,有的命书上还有什么推胎元等法的。所谓胎元,就是受胎月份的意思。推胎元的目的主要是推算命主怀孕月份的干支五行禀赋,从而作为算命时的一种参考依据;关于胎元的推法,从出生月份起,天干向前顺推一位,地支向前顺推三位,这样推出来的干支就是受胎月份的干支了。例如甲子月出生的人,天干甲向前顺推一位是乙,地支向前顺推三位是卯,那末这人受胎的月份是乙卯月了。从六十花甲周转一周来看,从乙卯月到甲子月出生,正好是十个月。古人说十月怀胎,可见这种推法是以十月怀胎为基数的。可是在事实上,除了十月怀胎外,既有怀胎七月、八月、九月不足月的,又有怀胎十一月、十二月等过月而产的,因为有着这种因素在内,所以在一般情况下,算命先生也是大多略去不推的。

\section{关于五行生克的术语和用神}
命理学家论命,很有一套理论,其中最为重要的,就是五行论命。由于五行论命有着它一整套完整的体系,所以在封建社会中,较易为士大夫阶层或知识分子所信仰。

五行论命过去有以年柱为主,结合其他三柱进行推论的;也有以日柱为主,结合其他三柱进行推论的;但以日柱为主,结合其他三柱五行进行推论的算法最具权威。

所谓以日柱为主,结合其他三柱五行进行论命,就是在算命时,先把一个人出生的年、月、日、时四柱的八字排出,并以日柱天干作为我自身论命的出发点,把四柱碎八字都化成五行;然后再根据日柱天干和周围其他干支五行之间错综复杂的关系,来进行具体的分析推论。

具体说来,比如一个人出生在公元1940年农历十月十四日辰时,我们可以先按照《怎样排八字》篇里所说的方法,依次排出他年、月、日、时的生辰八字:

(年)庚辰\par
(月)丁亥\par
(日)庚申\par
(时)庚辰\par
然后再根据日柱中自身天干和周围干支所含的五行的关系用笔注出:
\[
\text{\footnotesize{比肩}\large{庚辰}}
\left\{\begin{tabular}{@{}l@{}}
    乙木\quad{}正财\\
    戊土\quad{}偏印\\
    癸水\quad{}伤官\\
\end{tabular}\right.
\]

\[
\text{\footnotesize{正官}\large{丁亥}}
\left\{\begin{tabular}{@{}l@{}}
    壬水\quad{}食神\\
    \\
    甲木\quad{}偏财\\
\end{tabular}\right.
\]

\[
\text{\footnotesize{\qquad}\large{庚申}}
\left\{\begin{tabular}{@{}l@{}}
    庚金\quad{}比肩\\
    壬水\quad{}食神\\
    戊土\quad{}偏印\\
\end{tabular}\right.
\]

\[
\text{\footnotesize{比肩}\large{庚辰}}
\left\{\begin{tabular}{@{}l@{}}
    乙木\quad{}正财\\
    戊土\quad{}偏印\\
    癸水\quad{}伤官\\
\end{tabular}\right.
\]
这里,不管是年柱、时柱天干上所注的比肩,月柱天干上所注的正官,还是年、月、日、时地支下所注的正财、偏财、偏印、比肩、伤官、食神,都是算命术中最常见的术语。这些术语,有的书中叫做六神。不懂这些术语,就较难与之论命了。

从前文所论五行关系来看,都有一个与我同类,还是生我,我生,克我,我克的问题。上面我们宥到的一些如正财、偏财、伤官、食神之类的有关术语,就是以日柱天干作为肉身出发点,与周围其他宿关干支发生生克关系的结果现把八字中有关五行生克的术语列举如下:
\begin{enumerate}[label=\circled{\arabic*},parsep=0pt,topsep=0pt,itemsep=0pt,itemindent=2em]
\item 〔生我者为正印、偏印〕其中以阳母生阴我,阴母生阳我为正印,如戊土生辛金,辛金生壬水,戊土就是辛金的正印,辛金便是壬水的正印;阳母生阳我,阴母生阴我为偏印,如戊土生庚金,辛金生癸水,戊土就是庚金的偏印,辛金便是癸水的偏印。
\item 〔我生者为伤官、食神〕其中阳我生阴子,阴我生阳子为伤官,如甲木生丁火,丁火生戊土,丁火就是甲木的伤官,戊土便是丁火的伤官,阳我生阳子,阴我生阴子为食神,如戊土生庚金,庚金生壬水,庚金就是戊土的食神,主水便是庚金的食神。
\item 〔克我者为正官、偏官〕其中阳干克阴我,阴干克阳我为正官,如壬水克丁火,癸水克丙火,壬水就是丁火的正官,癸水便是丙火的正官;阳干克阳我,阴干克阴我为偏官,又称“七杀”或“七煞”,如壬水克丙火,癸水克丁火,壬水就是丙火的偏官,癸水就是丁火的偏官。
\item 〔我克者为正财、偏财〕其中阳我克阴干,阴我克阳干为正财,如庚金克乙木,辛金克甲木,乙木就是庚金的正财,甲木便是辛金的正财;阳我克阳干,阴我克阴干为偏财,如庚金克甲木,辛金克乙木,甲木就是庚金的偏财,乙木便是辛金的偏财。
\item 〔与我同类者为劫财、比肩〕其中阳与阴,阴与阳同类为劫财,如甲木逢乙木,丁火遇丙火,乙木就是甲木的劫财,丙火便是丁火的劫财;阳与阳,阴与阴同类为比肩,如庚金逢庚金,癸水逢癸水,庚金就是庚金的比肩,癸水便是癸水的比肩。 	
\end{enumerate}
从以上列举五项可以看出,一切术语都是从日干的自我和周围干支的关系生发出来的,其中阳与阴,阴与阳发生关系为正,也就是异性的为正;阳与阳,阴与阴发生关系为偏,也就是同性的为偏。这些术语,因为在取用神中常常提到,所以也有直接称之为用神的。

为什么会有这些印绶、食神等古怪的名字出现呢?原来命理学家认为造化流行在天地间,不过阴阳五行而已,而阴阳五行的交相为用,又不过生克制化而已。所以对于这些古怪名字来说,也就是阴阳五行交相为用、生克制化的直接产物了。

先说生我的印绶。因为生我的好比父母,所以便取了个印绶的名称。所谓“印”,就是荫庇的意思,所谓绶,就是授受的意思。好比父母有恩德荫庇子孙,子孙就借了光一样。

次说我生的食神。因为我生的是孩子,孩子长大后报答恩典,致养父母,所以说是食神。至于我生的伤官,因为能制约官星,所以一般认为并不吉利,有“伤官见官,为祸百端”的说法。再说女命见伤官,多主克夫,那就更怕人了。

接说克我的官煞。所谓“官者棺也,煞者害也”。朝廷一旦封人做官,此身就自然属于公家,政绩是好是坏,直到最后盖棺才能定论,可谓被官拖害苦了。平时人家梦棺得官,就是这个道理。

再说我克的妻财。因为妻子是事奉我而终身无违的,财产是被我自然享用的,两者都被我掌管,所以我克的就是妻财了。

末说与我同类的比劫。正因为彼此都是同类,大家肩比着肩,髙低差不了多少,所以便把比肩称为兄弟。在命书中,除了把阳干见同类的阳干、阴干见同类的阴干称为比肩外,还把阳干同类的阴干、阴干同类的阳干称为劫财。命书认为,命中见劫财多克妻害子,多破耗,并要提防小人。

在这些彼此错综的关系中,生我的正印是吉利的,可生我的偏印却不吉利,因为偏印又有枭印的叫法。我生的食神虽偏,但是吉利,可我生的伤官虽正,却不很吉利,因为据说它能破坏自己成才而做不了宫。克我的正官是吉利的,偏官却打了折头,因为官喜欢正,不喜欢偏。同样情况,我克的妻也就自然贵正不贵偏了。至于比肩、劫财,比肩要看具体情况,劫财就不那么的受欢迎了。归纳起来,“官系福身之物,财是养命之源,印乃资生之本,在人最为切要”,然而命中亦要活看,不能刻板。此外,这些关系彼此之间还有一些避忌,就是,官怕伤,被伤则祸;财怕劫,劫则被分;印怕财,贪财则坏;食怕枭,逢枭则夺。”这些话看来简单,可用意却还不小哩。

对于这些由五行之间的生克而造成的彼此间的错综关系,还派生出了好些有关术语。比如:

〔杀重身轻〕如自身天干是乙木,没有生在当令的春月,现在周围又布满重重克我的辛金,也就是克我的七杀(偏官)太重,所以叫做“杀重身轻”。

〔身强杀浅〕如自身天干是甲木,又生于春月当令之时,势必自身强旺,而周围克我的七杀庚金却少得可怜,所以叫做“身强杀浅”。

〔财多身弱〕如自身天干是甲木,没有生在当令的春月,而周围却是一片我克的戊土、己土,因为我克的是财,所以叫做“财多身弱”。反之则叫做“财弱身强”或“身强财弱”。

〔食神生财〕如身天干甲木,柱中有我生的食神丙火,火能生土,土对甲木来说,不是正财就是偏财。所以八字中如果缺财的,碰上食神也好。

〔比府重重〕如自身天干是甲木,周围的干支里义密布着重重甲木,因为与我同类而又同性的叫做比肩,所以便就有了这种说法。

〔比劫夺财〕如自身天干是甲木,而周围干支又布满了与我同类同性的比肩甲木和同类异性的劫财乙木,而我所克的正财、偏财戊己土却少得可怜,这样自己本已不多的财,就被比肩和劫财分夺掉了。

〔伤官损印〕如日干自身甲木,逢柱中丁火就是我生的伤官,而甲木的印绶则是生我的癸水。如果局中伤官丁火太旺,不利自身,这时癸水虽能前来克制,可娃在水火力欺相比悬殊的情况下,作为印绞的癸水就受损了。

〔印绶护身〕如自身天干为不当令的甲木,而又没有类相扶,这时如果周围干支中遇生我的水,因为水是木的印,所以有“印授护身”的叫法。

〔官印双全〕如自身天干为甲木,遇克我的金为官,生我的水为印,并且扶抑相当,没有太过不及,这就叫做“官印双全”。

〔财官相生〕如日干自身甲木,这时财为我克的戊己土,土能生金,其中辛金,就是克我的正官,所以说财官相生。

象以上这样的术语还有好多,但总的精神是好命要五行生克扶抑得当,如果五行生克太过或不及的,都不是好命。比如财多身弱,财多原是好事,只是自己身弱掌管克制不住,没有这个福份享用,因此命理学如果算上这种命的,反可断定他的一生没有什么大的财产,或有财也不属于他。再比如“印绶护身”自然属于好事,但是如果自身太强,周围又多与自身同类的比肩劫财,这时如再碰上生我的印,就会物极必反,走向反面,反而弄出祸患来了。

在旧时的算命书中,这种术语充斥纸面,随时可见。但是由于多少有点莫测高深的味儿,使人难以望见项背,所以遭到民国时命理学家巨擘袁树珊先生的竭力反对。平时,袁氏论命详于五行,并在他的著述《命理探原》中得到相当的体现,所以在旧中国知识界中有着一定的影响。

再说用神。顾名思义,所谓用神,就是八字或大运五行中对于自身的日干来说,具有补弊救偏或促进助成作用,为我所用的一种五行代称。其中用神出现在八字命局中的,叫做原局用神,出现在大运中的,叫做行运用神。

这种补弊救偏或促进助成作用,包涵很广,凡是四柱八字中对于日干能起扶其过弱、抑其过强作用的,都可取作用神。比如日干乙木,生不逢春,又少比肩、劫财同类的扶持,这时如果碰上八字或大运中有生我的水,就可“印绶护身”,逢凶化吉了。再如日干不论乙木或甲木,生于春月,而周围又多比劫,不仅自身强旺,并且扶持太多,有物极必反的忧虑。这时算命的就往往取制木的金,也就是官煞作为用神,从而抑其太过,达到平衡。如果八字中不见官煞金的,在大运或流年中碰上也好。如果八字或大运、流年中都碰不上官煞,或碰上也力量不够的,那这人的用神就不得力,一辈子都别想交好运了。

直接扶抑之外,间接的扶抑也可采作用神。比如自身日干木弱,八字或大运中逢上适量的官,就可官印相生,生水扶木了。因此这官对于木来说,也是用神。当然这种作为用神的官也不能太强旺了,否则强金克木,就能轻而易举地把木置于死地。同样道理,比如自身日干木强,八字或大运中偏又逢上较多的水来生木,也就是印生自身,这就使人担忧木太强了反会走向反面,这时看命的如果看到八字或大运中有制水的土,也就是自身的财,就可认定它能抑水生木,把它取作用神,大概不致大错。

在命理分析中,看准用神,被认为是算命准与不准的关键一着。在大多数情况下,算命先生都把五行中对自身天干起最重要扶抑作用的五行看作用神,但有时也把一个人的八字总起来作通盘的考虑,最缺什么就把什么取作为用神。

用前面这种办法能找到用神的,一般以出现在月份干支中的为最有力,其次是出现在时辰中,最末才是出现在年份中。比如一个人日干是秋月出生的辛金,自身较强,须要适量的火来加以炼制,然后方才能够冶铸成器。这时如果月份的干支出现丙寅火,那末算命家就可认定丙寅正官作为这个人的用神,并且这用神还非常的有力。如果月份的干支不出现丙火,而是壬子、癸亥等一片水地,那末由于金能生水,水泄金气,也可看作用神。这里需要注意的是,这人的月份中如果碰上丙子(丙火、癸水)、癸巳(癸水和巳中丙火)等水、火同时出现的现象,就又要比较一下整个八字中是火多还是水多,是更迫切需要火还是更迫切需要水,然后把这更少更迫切要的看作用神。当然,在八字中取用神时还不要顾此失彼,忘了天干地支间彼此的刑、冲、化、合等等因素,否则用神取错肴错,便就通盘都错了。比如这样个命造:

(年)壬戌\par
(月)己酉\par
(日)丁丑\par
(时)甲辰

本命丁火,理该夏月生旺,然而却生于八月酉月火囚之时,所以没能得时,而年干壬水,月干己土,又都克我、泄我,伤我元气,加之地支戌、酉、丑、辰,一片金土,又属我克我泄之神,现在亏得丁火通根年支戌库(火库),又得时柱甲木坐辰生我为印,然而从全局来看,自身仍属偏弱。弱者宜扶宜生,这时如果取年柱正官壬水作为用神,以期官印相生,有利印绶甲木,可是却有食神己土损官为病,所以权衡下来,不如直接取正印甲木作为用神,既可生扶丁火,又可克制己土,祛除壬水被制之病,这样水来生木,木来生火,岂不美善?但是,甲木在命局地支一片金土克我我克的情况下,也毕竟能力有限,所以又要结合大运来看了。大运如果行到甲寅、乙卯木运,用神得比肩相助,必定富贵优游。反之大运如入金土,用神受损,那就困苦不堪了。

关于在大运中看用神,主要有三种情况。一是八字中不乏用神,而在大运中又重新碰上的;二是八字中缺乏对自身天干弱扶强抑的用神,而偏偏出现在大运中的;三是八字和大运都没碰上对自己强有力的用神。对于末一种,一般认为都是一身偃蹇,不好的命。对于第二种,命书中有大缺大补、大偏大纠的说法,生了这种命的,倒起霉来倒煞,可是一行到大运,不是大补就是大纠,来个彻底的翻身。对于第一种,不用说就更好了。

这里很重要的一点是,既然用神对于一个人一生命运的荣枯有着这样重要的作用,那末用神不得逢冲,就被提上议事日程来了,这就是说,在一个人的八字或大运中,用神被冲被克是不吉利的事,反之,用神得到生扶或同类相助,也就转吉有望了。

对于用神,算命先生总是爱用比、食、财、官、印等术语来加以分析,用神正印、用神食神、用神偏财等等便是。

在封建社会中,算命先生大都认为八字中五行俱全,主人一身衣禄不愁,当然八字中有缺在运中补上也好。由于这种思想经过日积月累的渗透,早已普遍地蔓延到了广大的平民百姓中间,所以民间父母为孩子取名,又常会根据算命先生推算结果,把所缺的五行在名字中补上,以讨吉利。今天我们如果看到老一辈中名字有叫森、焱、垚、鑫、淼的,便可大致推断,他命里是缺了什么五行的。
\section{八字中有关的星宿神煞}
我国早期的看命法中,有一种流行很久的星宿照命和神煞入命的观念。把天上星宿神煞和人的命运结合起来,出于古代人们对于星和神的一种崇拜心理。

在一个人的四柱八字中,看星宿神煞大多以代表自身的日柱干支为出发点,再联系年、月、时或大运、流年等其他干支进行观察比照。翻开命书,自身干支中的什么字碰到年、月、时,或大运、流年干支中的什么字便算遇上了什神煞,命书都有一定的规定。譬如自身日干庚金,碰上年、月、时中地支的亥,就被认为是“文昌入命”了。这种文昌,是个吉星,假如读书人碰到了它,一定事业出人头地春风得意。然而对于有的神煞,也尽有不从日柱天干出发的。

在古往今来的命书里,有关神煞极多,这里择要介绍如下:

1.吉星照命或吉神入命

〔天德〕也称“天德贵人”,这是以出生月份的地支,结合出生日期、时辰的天干所反映出来的一种吉星。古歌说:
\begin{tightcenter}
    正丁二坤(申)中,三壬四辛同,\\
    五乾(亥)六甲上,七癸八艮(寅)逢,\\
    九丙十居乙,子巽(巳)丑庚中。\\
\end{tightcenter}
然而根据近人研究成果,也有列成下表的:
\begin{table}[H]
\centering
\setlength{\tabcolsep}{0.1em}
\begin{tabular}{m{5em}<{\centering}|*{11}{m{2em}<{\centering}|}m{2em}<{\centering}}
\hline
出生月份&正月&二月&三月&四月&五月&六月&七月&八月&九月&十月&十一&十二\\
地支&寅&卯&辰&巳&午&未&申&酉&戌&亥&子&丑\\
\hline
    出生日、时&壬&巳&壬&丙&亥&甲&戊&亥&甲&庚&巳&庚\\
    干支     &丁&申&丁&辛&寅&己&癸&寅&己&乙&申&乙\\
\hline
\end{tabular}
\end{table}
如表,寅月出生的人在日或时的天干上碰上壬或丁,卯月出生的人在日或时的地支上碰上巳或申等,就被认为是有天德贵人星了。其他如表类推。

命里有天德贵人星的人,主一生吉利,荣华富贵。现在港台等地的命书中,还曾广泛记我了世界各国名人的命。比如举天德贵人星为例,书载:“命运中有天德贵人星的人,尤其以日命为主,在社会上能出人头地的非常多。例如许多出名的歌星、影美空云雀、吉永小百合等,甚至英国前首相邱吉尔、日本美智子妃殿下,这些人都日命中有天德贵人星相助。”

〔月德〕这是一种以出生刀份地支,结合出生日期天干反映出来的吉星,规律是:
\begin{tightcenter}
寅午戌月在丙,\\
申子辰月在壬,\\
亥卯未月在甲,\\
巳酉丑月在庚。\\
\end{tightcenter}
这个规律看起来象不好记,其实好记。在前面《天干地支的刑冲害化合》篇中,我们在谈地支的“三合”时,曾有过“申子辰合水,亥卯未合木,寅午戌合火,巳酉丑合金”的说法,这里“寅午戌月在丙”,就是寅午戌月见丙火日,“申子辰月在壬”,就是申子辰月见壬水日,“亥卯未月在甲”,就是亥卯未月见甲木日,“巳酉丑月在庚”,就是巳酉丑月见庚金日,并且见的日干都是阳干,还不好记?命中有月德的人,也和天德一样,一生无险无虑。

〔三奇〕三奇有“天上三奇”、“地下三奇”、“人中三奇”三种情况。但不论哪一奇,都要以年、月、日,或月、日、时的天干挨次顺排下来才是,如果位置逆乱,就不是了。歌曰:
\begin{tightcenter}
    天上三奇甲戊庚,\\
    地下三奇壬癸辛,\\
    人中三奇乙丙丁。\\  
\end{tightcenter}
这就是说甲年生的人,月干、日干中同时挨次出现戊、庚,或甲月生的人,日干、时干中挨次出现戊、庚,就算是甲应了“天上三奇”。其他类推。

按照命书说法,八字中逢“天上三奇”、“地下三奇”、“人中三奇”的都是襟怀卓越,博学多能,大富大贵,不属凡类的人。

〔天乙贵人〕天乙贵人星的看法以日柱的天干为主,结合其他三柱地支观察。方法是:
\begin{tightcenter}
    甲戊庚见丑未,\\
    乙己见子申,\\
    丙丁见亥酉,\\
    壬癸见巴卯,\\
    辛见寅午。\\
\end{tightcenter}
这就是说,甲日戊日或庚日出生的人,见八字地支中有丑或未的,就可认定是有天乙贵人星了。其他类推。

命书认为,天乙贵人星也是一种很有用的吉星。命里有这种星的人,可以逢凶化吉,因为有贵人相助。

〔天赦〕天赦也是个吉星,这种吉星出现不多,看法是:“春戊寅,夏甲午,秋戊申,冬甲子。”解释是春月逢戊寅日出生,夏月逢甲午日出生,秋月逢戊申日出生,冬月逢甲子出圭昀,都是逢上了天赦星。

“命中若逢天赦,一生处世无忧”,这就是命书对天赦星的说项。

〔十干禄〕《渊海子平》说:“甲禄在寅,乙禄在卯,丙戊禄在巳,丁己禄在午,庚禄在申,辛禄在酉,壬禄在亥,癸禄在子。”看法以日干五行为主,结合年、月、日、时地支。如果从自身天干出发,禄在年支的叫做岁禄,禄在月支的叫建禄,禄在日支的叫坐禄,禄在时支的叫归禄。例如庚日出生的人,年支逢申,就是岁祿,月支逢申,就是建禄,日支逢申,就是坐禄,时支逢申,就是归禄。其他日干见禄依此类推。

禄为养命之源,命中逢上,一生衣禄不愁,然而最怕犯冲或入空亡(见下〔六甲空亡〕),如果这样,反而衣禄不足了。

〔文昌〕命里出现这种吉星的,对于知识分子来说,尤其有用。看法以日柱天干为主,结合其他有关地文进行观察。具体情况是:
\begin{tightcenter}
    甲乙见巳午,\\
    丙戊见申,\\
    丁己见酉,\\
    庚见亥,\\
    辛见子,\\
    壬见寅,\\
    癸见卯。\\    
\end{tightcenter}
对此,古歌有云:
\begin{tightcenter}
    甲乙巳午报君知,丙戊申宫丁己鸡,\\
    庚猪辛鼠壬逢虎,癸人见兔入云梯。\\    
\end{tightcenter}
命书认为,八字中见文昌星的,非但聪明过人,才华出众,并且另外还有着逢凶化吉的妙处。

〔将星〕这名字听上去很威严。八字中出现将星的情况是:
\begin{tightcenter}
    寅午戌见午,\\
    巳酉丑见酉,\\
    申子辰见子,\\
    亥卯未见卯。\\
\end{tightcenter}
意思是寅、午、戌日出生的人,碰上年、月、时地支中有午字的,便是有了将星。其他类推。

据认为,命中出现将星的人,有掌权之能,众人皆服。又云:“将星文武两相宜,禄重权商足可知。”

2.偏于中性,有吉有凶的星煞

〔魁罡〕魁罡是一种天冲地击之煞,凡是日干碰上戊戌、庚戌的叫天罡,逢上庚辰、壬辰的叫地罡。命中有魁罡星的,主人性格聪明,文章振发,临事果断,秉权好杀。如果运行身旺,必定发福百端;假若一见财官,那就祸患立至了。又如日柱魁罡,碰上刑冲,非但不吉,反而是个贫寒的穷人。此外,女性以阴柔为美,命中如果碰上魁罡,也是犯忌的。

〔华盖〕鲁迅诗中曾说:“运交华盖欲何求?未敢翻身已碰头。”把交华盖看成是交了坏运。其实,华盖在命柱中出现,按照命书的说法,也不一定全是坏事。看法是:
\begin{tightcenter}
寅午戌见戌,\\
巳酉丑见丑,\\
申子辰见辰,\\
亥卯未见未。\\
\end{tightcenter}
比如寅、午、戌日出生的人,在年、月、时的地支中碰上戌字,便被认为是有了华盖星。

命书说,华盖是艺术文章之人,主人必定读书刻苦,做事勤恳,但性格却不免孤僻。如果华盖多逢印绶,并且处在旺相之地,可能在政界有一定地位。又如“华盖逢空,偏宜僧道”,那就孤而不吉了。

〔驿马〕在古代,命中出现驿马有两种情况,就是贵人驿马多升跃,常人驿马多奔波。《身命赋》说:“马奔财乡,发如猛虎。”《造微论》说:“马头带剑(壬申、癸酉,地支含金),威镇边疆。”这说明驿马可吉可凶。如果四柱五行阴阳配合得宜,驿马和财、官、禄处在同一地支上,并且居于生旺之地,不逢克伐,那就不贵也富。反之,如果驿马处在死绝之地,并且又逢克伐,或竟处在空亡之乡,那就难免奔波流浪了。观察命里有没有驿马,口诀是:
\begin{tightcenter}
寅午戌见申,\\
巳酉丑见亥,\\
申子辰见寅,\\
亥卯未见巳。\\
\end{tightcenter}
意思、是寅、午、戌日出生的人,逢年、月、时地支中有申字的,就可认为是有了驿马。

3.凶神恶煞

〔羊刃〕《星平会海》说:“甲禄到寅,卯为羊刃;乙禄到卯,辰为羊刃;丙戊禄在巳,午为羊刃;丁己禄在午,未为羊刃;庚禄居申,西为羊刃;辛禄到酉,戌为羊刃;壬禄到亥,子为羊刃;癸禄到子,丑为羊刃。”这就是说,禄过则刃生,判定的方法是,从自身日柱天干出犮,凡阳干甲见卯,丙戊见午,庚见酉,壬见子,其地支必定处在禄后一位;同样,阴干乙见辰,丁己见未,辛见戌,癸见丑,其地支处在禄后一位的,虽然不叫羊刃而叫劫财,然而性质却是一样的。

命书认为,羊刃是个性子急躁,刚强凶狠的神。命中男多羊刃,妻宫有损;女带羊刃,刑夫克子。然而有时也要具体分析,如果身弱,那就不一定是凶,因为羊刃有卫禄帮身的职能;反过来,如果身强则就难免招灾惹祸了。关于身弱身强的看法,一看日干五行和降生月份之间的旺相休囚死关系,二看日干得四柱干支生助的多少,三看日干和年、月、日、时地支在寄生十二宫中处于什么状态。以上这三种看法,结合有关章节一査就知。

〔桃花煞〕在命书中,桃花煞也叫“咸池”。所谓煞,就是凶神恶煞的意思。命中出现桃花煞,多与酒色有关。看法是:
\begin{tightcenter}
寅午戌见卯,\\
巳酉丑见午,\\
申子辰见酉,\\
亥卯未见子。\\    
\end{tightcenter}
意即寅、午、戌日出生的,碰上卯年、卯月或卯时,就是犯了桃花煞。论命诗说:
\begin{tightcenter}
    风淫淫冶号咸池,并集来临祸应期,\\
    酒色相刑三二位,更加神煞血光随。\\
\end{tightcenter}

又有命书说:“酒色猖狂,只为桃花带煞。”认为有这种命的人,多为男女淫欲之征”。其中又分墙里桃花和墙外桃花:如煞的位置处在年支或月支的,叫做墙里桃花,主夫妻恩爱,可不为害,但如果冲破就不好了;煞的位置在时支的,叫做墙外桃花,如果女人逢上,那就人人可采,最为不吉。在封建社会中,如果女人家八字带上桃花,是男人家所最犯忌的。假如一个人的八字里没有桃花煞,而在大运或流年中碰上的,也可被认为是交了桃花运。

〔孤辰、孤宿〕这是一种孤寡伶仃的神煞,如果命里逢上,大多孤柄独宿,伶仃清苦。按照《三命通会》的说法,孤辰和孤宿的看法以年柱地支为主,结合月、日、时的地支来作判断:
\begin{enumerate}[label=\circled{\arabic*},topsep=0pt,parsep=0pt,itemsep=0pt,leftmargin=3.5em,labelsep=0em]
\item 年支逢东方一气寅、卯、辰时,月、日、时支中出现已的叫孤辰,出现丑的叫孤宿;
\item 年支逢南方一气巳、午、未时,月、日、时支中出现申的叫孤辰,出现辰的叫孤宿;
\item 年支逢西方一气申、酉、戌时,月、日、时支中出现亥的叫孤辰,出现未的叫孤宿;
\item 年支逢北方一气亥、子、丑时,月、日、时支中出现寅的叫孤辰,出现戌的叫孤宿。
\end{enumerate}

这里面我们不难看出这样一个规律,就是在年干地支东、南、西、北的合局中,顺数下去一位的是孤辰,倒数一位的是孤宿。如以北方一气亥、子、丑为例,亥、子、丑顺数的下一位是寅,倒数的上一位是戌,那末寅就是亥、子、丑年生人的孤辰,戌就是亥、子、丑年生人的孤宿了。

《烛神经》说:“凡人命犯孤宿,主形孤骨露,面无和气,不利六亲。生旺稍可,死绝尤甚。驿马并,放荡他乡;空亡并,幼少无倚;丧吊并,父母相继而亡。一生多逢重丧叠祸,骨肉伶仃,单寒不利。入贵格,赘婿妇家;入贱格,移流未免。”

\textbf{〔亡神〕}这也是个凶神恶煞,命中出现是倒霉的事。命书的说法是:
\begin{tightcenter}
申子辰见亥,\\
寅午戌见已,\\
巳酉丑见申,\\
亥卯未见寅。\\
\end{tightcenter}
如日支为申、子或辰的人,在年支、月支或时支中出现亥字,就算是亡神入命了。当然,由于月支和时支更贴近日柱的自身,所以情况要比年柱出现亡神更糟。

“亡神入命祸非轻,用尽机关心不宁”,可见出现这种命的人,刑妻克子,多么不幸。然而,亡神如果和吉神处在一个柱上,也主谋略深算。

\textbf{〔六甲空亡〕}《渊海予平》说:“甲子旬中无戌亥,甲戌旬中无申酉,甲申旬中无午未,甲午旬中无辰巳,甲辰旬中无寅卯,甲寅旬中无子丑。”这是说十天干和十二地支互相配合,成就六十甲子。如果以甲为基准,就有甲子、甲戌、甲申、甲午、甲辰、中寅六旬。旬是十,从甲子到甲戌,正好要过乙丑、丙寅、丁卯、戊辰、己巳、庚午、辛未、壬申、癸西十天,这说明天干从甲到癸,地支从子到酉一旬终了,天干正好用完,而地支还多着两个没给排进去。于是,这戌、亥便就成了甲子旬中的空亡。比如日柱干支在甲子旬中的,不管是乙丑、丙寅、丁卯、戊辰、己巳,还是庚午、辛未、壬申、癸酉,只要年、月、时支中出现亥、戌的,就算是逢上空亡了。同样道理,在甲戌旬中见申、酉,在甲申旬中见午、未,在甲午旬中见辰、巳,在甲辰旬中见寅、卯,在甲寅旬中见子、丑的,也都叫作空亡。大凡看命,如果喜神落在空亡位置上的,那就虚而不实,空欢喜了一场。相反,如果忌神落在空亡位置上的,那就又可转忧为喜了。

六甲空亡一览表:
\begin{table}[H]
\centering\setlength{\tabcolsep}{0.2em}
\begin{tabular}{m{3em}<{\centering}|*{10}{m{1.5em}<{\centering}|}m{3em}<{\centering}}
\hline
&1&2&3&4&5&6&7&8&9&10&空亡\\
\hline
1&甲子&乙丑&丙寅&丁卯&戊辰&己巳&庚午&辛未&壬申&癸酉&戌、亥\\
\hline
2&甲戌&乙亥&丙子&丁丑&戊寅&己卯&庚辰&辛巳&壬午&癸未&申、酉\\
\hline
3&甲申&乙酉&丙戌&丁亥&戊子&己丑&庚寅&辛卯&壬辰&癸巳&午、未\\
\hline
4&甲午&乙未&丙申&丁酉&戊戌&己亥&庚子&辛丑&壬寅&癸卯&辰、巳\\
\hline
5&甲辰&乙巳&丙午&丁未&戊申&己酉&庚戌&辛亥&壬子&癸丑&寅、卯\\
\hline
6&甲寅&乙卯&丙辰&丁巳&戊午&己未&庚申&辛酉&壬戌&癸亥&子、丑\\
\hline
\end{tabular}
\end{table}


\textbf{〔十恶大败〕}《三命通会》说:“十恶者,犯十恶重罪,在所不赦;大败者,譬兵法中,与敌交战,大败无一生还,喻极凶也。”看法是以出生年份,结合生日来定。其体情况是:
\begin{tightcenter}
庚戌年见甲辰日,\\
辛亥年见乙巳日,\\
壬寅年见丙申日,\\
癸巳年见丁亥日,\\
甲戌年见庚辰日,\\
甲辰年见戊戌日,\\
乙亥年见辛巳日,\\
乙未年见己丑日,\\
丙寅年见壬申日,\\
丁巳年见癸亥日。\\
\end{tightcenter}

以上十种,不管月、时怎样,都是不好的命。原因是这十天出生的人,不但年支和日支彼此相冲,并旦禄地碰上空亡。譬如甲辰、乙巳日出生的人,甲禄在寅,乙禄在卯,而从“六甲空亡”表中我们可以査知,寅、卯正是甲辰旬的空亡。其他类推。

除了以上神煞之外,还有红艳煞、元辰,以及六厄、勾绞、天罗地网等等一些恶煞,如唐朝韩愈就曾自叹命宫磨蝎,让磨蝎宫的凶星闯到命宫里来,哪还会有好的结果?

从以上所列星宿神煞看,多半有点硬性规定的味道,因为它不从五行生克宜忌着手分析,所以后来大多数的命理学家,都不赞成单用神煞来断定一个人一生的吉凶祸福。这用命理学家陈素庵的话来说,就逛“凡人命吉凶,皆由格局运气,安可以偶合神煞而信之”呢?

\section{命理中五行和干支的分析}

我们有了前面各篇所说的有关基础后,就可从所排四柱的八字入手看命了。

因为在年、月、日、时四柱中,日柱的干支是管自己的,所以命理中一切的一切,都得从日柱干支所含五行出发,进行分析推论。

从总的方面着眼,有关日干五行荣枯的大致情况,我们可以作这样的分析:

凡是日主属木的,必先看一看木势的盛衰。木重水多的是盛,这样就宜金砍木,如果金不足的逢土也好,因为一方面土能生金制水,另一方面木又能克土,使木自身的盛势有了宣泄的处所。木微金强的是衰,这就宜火制金,火少的逢木也好,因为木与木为同类,既可壮大自己声势,又可生火制金。至于水太盛则木漂,取土制水为上,没有土的有火也好,因为火能生土。土太重则木弱,难以制约,取木帮助克土为上,没有木的有水也好,因为水能生木制土。火太多则木焚,取水制火为上,没有水的有金也好,因为金能生水。

凡是日主属火的,必先辨别一下火力的有余与不足。火炎木多的是有余,这样就宜水济火,如果水不足的逢金也好,因为金能生水制木,另一方面火又能克金,使金自身的有余有了发挥的处所。火弱水旺的是不足,这就宜土制水,水不足的逢火也好,因为火与火是同类,既可壮大自己声势,又可生土制水。至于木太多则火炽,取水制火为上,没有水的有金也好,因为金能生水。金太多则火弱难以制约,取火帮助克金为上,没有火的有木也好,因为木能生火。土太多则火晦,取木生火制土为上,没有木的有水也好,因为水能生木。

凡是日主属土的,必先辨别一下土质的厚薄。土重水少的是厚,这就宜木疏土,假如木弱的逢水也好,因为水能生木制土,另一方面土又能够克水,使土自身壅滞的情况得以改善。土轻木盛的是薄,这就宜金制木,金不足的逢土也好,因为土与土是同类,既可壮大自己声势,又可生金制木。至于火多则土焦,取水制火为上,没有水的有金也好,因为金能生水。水太多则土流而难以保持,取土帮助克水为上,没有土的有火也好,因为火能生土。金太多则土弱,取火生土制金为上,没有火的有木也好,因为木能生火。

凡是日主属金的,必先辨别一下金的老嫩。金多土厚的是老,这就宜火炼金,假如火衰的逢木也好,因为木能生火制金,另一方面木又能克土,从而使土不能生金太过,同时,金又能克木,使重金有所宣泄。木重金轻的是嫩,这就宜土生金,假如土不足的逢金也好。至于土多则金埋,取木制土为上,没有木的有水也好,因为水能生木。水太多则金沉而难以出头,取土帮助克水为上,土少的有火也好,因为火能生土。火太烈则金销,取水制火为上,水不足的有金也好,可以壮大自身力量。

凡是日主属水的,必先辨一下水势的大小。水多金重的是大,这就宜土御水,假如土弱的逢火也好,因为火能生土以制水,另一方面火又能克金,从而使金不能生水太多,同时,水又能克火,使水有所发挥。水少土多的是小,这就宜木克土,如果木弱的逢水也好。至于金多则水浊,取火克金为上,火少的有木也好,因为木能生火。火太炎则水灼而易于消耗,取水制火壮大自身为上,水少的有金也好,因为金能生水。木太多则水缩,取金生水为上,金不足的有土也好,因为这样可以引木克土,分散木对水的贪汲。

对于算命先生来说,算命时分析八字中五行盛衰,是断定一个人一生荣枯,走运不走运的必不可少的重要前提。对此,《三命通会》引徐大升的话总结道:

\begin{adjustwidth}{2em}{2em}
\setlength{\parindent}{2em}
\setlength{\leftmargin}{3em}
\kaishu
\hspace{2em}金赖土生,土多金埋;土赖火生,火多土焦;火赖木生,木多火炽;木赖水生,水多木漂;水赖金生,金多水浊。\par
金能生水,水多金沉;水能生木,木盛水缩;木能生火,火多木焚;火能生土,土多火晦;土能生金,金多土变。\par
金能克木,木坚金缺;木能克土,土重木折;土能克水,水多土流;水能克火,火炎水热;火能克金,金多火熄。\par
金衰遇火,必见销熔;火弱逢水,必为熄灭;水弱逢土,必为淤塞;土衰遇木,必遭倾陷;木弱逢金,必为砍折。\par
强金得水,方剉其锋;强水得木,方泄其势;强木得火,方化其顽;强火得土,方止其焰;强土得金,方制其害。\par
\end{adjustwidth}

以上这种分析,都建立在对自身五行太过不及的扶抑上面,可见对于一个人来说,最好是能求得五行的动态平衡,否则太过不及,没有相应的作为用神的五行来作恰到好处的补救,那就糟了。

这种分析,如再结合具体天干地支的性质作为补充,那就更加全面,更加深细,更加灵活了。现在先说天干:

\textbf{〔甲〕}甲木属于雷木,这是因为雷是阳气嘘出来的,所以甲木属阳,取象于雷。《礼记•月令》:“仲春之月,雷乃发声。”可以作为甲木旺于春天的征验。邵子说:“地逢雷处见天根。”可见这种阳木的生长,不正是天根萌动的产物?至于甲木到申而绝,是因为雷声到申(农历七月)而渐渐收起的缘故.为此,大凡命中日柱天干属甲,喜欢碰上春天出生,行运不喜西方庚金来克。经云:“木在春生,处世安然必寿。”

\textbf{〔乙〕}乙木属于风木,这是一种山林活木,到夏天而畅茂,诗云:“千章夏木青。”所以说,午是乙木长生所在之地。在八字中,日柱天干乙木,生在秋天大吉。这是因为秋令金旺,乙木能化能从,所谓“乙庚化金”。再如乙木本身盘根错节,如不逢上金制利器,又怎么能够斫削成材呢?平生最怕的是逢上初冬亥月,因为这时叶落归根,生机已经没有了。

\textbf{〔丙〕}丙火属太阳离火(离在八卦中属火),有文明之象。太阳早出晚落,所以阳火寅(农历春正月)生而酉(农历秋八月)死。明代兵部尚书万骐《真宝赋》把丙日丑时(凌晨1—3时以前),作为“日出地上之格”是深有旨趣的。大凡日干逢六丙日的人,出生在冬夏的不及春秋好,这是因为春阳有暖万物之功,秋阳有燥万物之用,冬则阴晦,夏则炎蒸的缘故。

\textbf{〔丁〕}丁火属于星火。当入夜太阳的丙火消逝,于是星火就在天空中显现出来了。《真宝赋》说:“阴火时亥,富贵悠悠。”原来十二地支亥居北方,称为天门,有星象拱北的说法。为此,凡是出生日干逢上丁火的,喜欢碰上夜里和秋天,这是因为夜里和秋天,正是星光当令的时候。此外,日逢丁火的还喜欢身行弱地,这样才符合丁火的性质。其中丁巳一天生的不好,因为巳中丙火,为丁火的劫财,财忌比劫,兄屈弟下,加之巳中戊土又属伤官不吉,所以对父兄妻子多所克制。

\textbf{〔戊〕}戊土属于于霞土。本来土无专气,仗火而生,霞无定体,借日而现,知道了丙火为日的道理,就知道为什么戊土属于霞了:这是因为“霞者日之余,日尽而霞将灭没,火熄则土无生意”,所以把它称之为霞。当年大挠氏演纳音五行之象,把戊午当作天上之火,原意就是这样。假如八字中日柱属于戊土的,最好四柱带水,这样就霞水相映,文采斐然,成为上格。如果年干月干出现癸水的,那就更加好了,原来癸则为雨,雨后霞见,岂不文明。

\textbf{〔己〕}己土性质类属于天上的云。云为山泽之气,所以甲己化土,其气上升而云施,然后云雷交作而雨降土润,这就是造化至妙的地方。八字中如果日干属于己土的,最喜坐在地支的酉位上,因为酉对己土来说,属于十二宫中的长生吉地。此外,己日生的人还喜欢生在春月,喜欢碰上甲木。假如天干己坐在地支亥上的,则忌见乙木,因为乙木属于风木,云升天如果碰上风,那就“狼藉而不禁”了。

\textbf{〔庚〕}庚金性质类属于天上的月。庚金原是西方的阳金,为什么会和月挂上钩呢?原来五行中的有庚,正好比四时的有月一样。庚金虽然不待秋天而生,然而必定到秋天才开始旺盛;月亮不待秋而后有,但要到秋天才更加明亮。加之金白月明,金能生水.潮能应月,无论是色是气,彼此都有相通的地方。一个人的八字,如果日柱有庚,四柱中有乙、己等字透出的,叫做“月白风清”。其人出生的月份最好是秋月,其次是冬月,假如逢上春夏,那就不好了。

\textbf{〔辛〕}辛金的性质类属于秋天的霜。八月酉月为辛金建禄(临官)之地,这时白露为霜,天气清肃,草木黄落,五行阴木到此而绝,所以辛金有在霜的肃杀秉性。有人说,霜最怕见日出,为什么天干中辛金偏要与丙合气呢?答案是这符合五行相克相化的原理。原来火能克金,所以丙火一旦和霜相合,便化而成水了。八字中日干辛金,日支卯,四柱乙木不透,这是一种大富的命。如果日干辛金,坐亥透丙的,那也贵不可言。出生的月份,则一般以冬月为好。

\textbf{〔壬〕}壬水的性质类属于秋露。那末,为什么不把壬水说成是春露呢?那是因为春露是濡润万物,促进生长的雨露,秋露是挟寒而降,肃杀万物的霜露。壬水原生于申金,而欣欣向荣的木碰上申,就身逢绝地了,所以命理学家常把壬水说成是清肃的秋露。在八字中,壬日出生的人如果碰上秋月,那就最好能见丁火,因为丁为星河,壬为秋露,一彼此相逢,那就一冼炎蒸,天象昭然了。

\textbf{〔癸〕}癸水的性质类属于春霖。癸水生于卯月,叫做春霖,这是因为阴木得雨,就蓬勃而生发了。然而卯木到申而死,总是由于农历七八月已是肃杀的秋天,而天上降水量又大为减少的缘故。假如癸卯日生的人,其他三柱中最好有象征云的己土透出,这样云行雨施,那人必经事济世之才。月份是春夏吉,秋冬不吉。诗曰:“癸日生逢己巳乡,煞星须要木来降。虽然名利升高显,争奈平生寿不长”。

说过十天干后,我们再来看看十二地支,又究竟有怎么个说法。

\textbf{〔子〕}子为墨池。子的位置在正北方,五行属于癸水,色黑象墨,所以有墨池之象。凡是子年出生的人,时柱喜见癸亥,叫做“水归大海”,又叫“双鱼游墨”,这种人必定是个文章高手。

\textbf{〔午〕}午为烽堠(烽火台)。午的位置在正南方,五行属于火土,颜色赤黄,所以有烽堠之称。同时午在生肖为马,而烽堠正是戎马兵火所处的地方。大凡午年出生的人,时支最好能够见辰,因为真龙一出,就“一扫人间凡马空”了。对于这种人,行家称之为“马化龙驹”。

\textbf{〔卯〕}卯为琼林。卯的位置在正东方,五行属于乙木,在时属于仲春(春三月中的第二个月)。因为这时万物生长,树木青的好比琅玕(一种青玉)一样,所以有琼林之称。卯年出生的人,最好逢上己未的时柱,这样就“兔入月宫”,属于大贵一路的人。	

\textbf{〔酉〕}西为时钟。酉的位置在正西方,五行属于辛金。从位置来说,酉最靠近于戌亥。所谓“戌亥者,天门也”,“钟,金属也”,寺钟一鼓,不就声音响彻天门了吗?在八字的地支中,酉见寅被认为吉利的,叫做“钟鸣谷应”。

\textbf{〔寅〕}寅为广谷。在方位上,寅属于东北方的艮方,艮在八卦中为山,原是戊土生长的方,这样寅作为广谷的意义就显示出来了。因为寅年属虎,虎年生的人碰上戊辰的时柱,那就“虎啸而谷风生”,可以威震万里了。

\textbf{〔申〕}申为名都。在八卦中,坤属于地,地的体势广大无疆,不用名都作譬,不足以说明它的大。“申,坤也;都者,帝王所居”。命理学家认为,申宫生出壬水,又和艮山相对,所以有“水绕山环”的美称。这种现象应在命里的,就是申年生的人逢上亥时的,被称为有天地交泰之象。

\textbf{〔巳〕}巳为大驿。所谓大驿,就是人烟凑集,道路通达的地方。因为巳里藏有丙火戊土,加之巳的下一个地支又属于午马,所以就用大驿作为巳的象征。大凡巳年出生的人喜欢逢上辰时,巳属蛇,辰应龙,这样就“蛇化青龙”,在格局中贵为千里龙驹了。

\textbf{〔亥〕}亥为悬河。天河之水,奔流不回,所以称为悬河。古云“亥即天门”,同时亥又包藏癸水,不就明摆着是悬河之象?正因为有这种原因,所以亥年出生的人,如果碰上日支时支有寅、辰两字的,就称得上是“水拱雷门”了。

\textbf{〔辰〕}辰为草泽。“深山大泽,龙蛇生焉”,这是《左传》上说过的一句话。大泽是水聚会的地方,而辰的位置处在东方稍偏一点的地方,在四库中正属于水库,所以为草为泽,人们就习惯地称它为草泽了。八字中如果辰逢壬戌、癸亥,就贵为“龙归大海格”了。

\textbf{〔戌〕}戌为烧原。戌的月份是深秋九月,那时草木萎尽,田家烧草而耕,加之戌又属土,所以便就有了烧原的说法。因为有着这个缘故,所以“戌与辰地,皆贵人所不临”。戌年生的人如果逢上卯支,人称“春入烧痕”。

\textbf{〔丑〕}丑为柳岸。丑中有水有土有金,其中岸是土堆成的,可以用来范围流水,所以丑有柳岸之称。诗曰:“柳色黄金嫩。”这说明柳与金也有瓜葛。命书说,丑人时见己未,叫做“月照柳梢”,这是一种极为上格的命。

\textbf{〔未〕}未为花园。有人问,花园为什么要属未而不属卯呢?这是因为卯是木旺之象,所以自成林麓;未是木的墓库,好比墙垣里的花木,并且丛丛杂杂,所以叫做“花园”。辛未年生的人如果碰上戊戌时,这样两干不杂,有“双飞格”之称,这是最好不过的。


\section{十干坐支、兼得月时和行运吉凶}
旧时做八股文先要破题,这里也来个东施效颦,把题先破一番,然后引入正文。

所谓“十干坐支”,就是说,出生那天的日柱天干,它坐下的日支是哪一个地支,而四周年柱、月柱、时柱的地支又是些什么样的地支。换句话说,也就是出生那天日柱的天干和日柱的地支,和年、月、时的地支,彼此之间是个什么样的配合关系。为什么要研究“十干坐支”的问题呢?因为根据徐子平的命理原则,日柱在八字中是有关自身的,其他年、月、时三柱的宜什么、忌什么,都要由日柱出发,进行分析。

所谓“兼得月时”,就是得月、得时的意思。这主要是因为日柱在八字中,不能丢开降生的月份和时辰单独分析,而是要把日柱放进年、月、时三柱中去作综合的观察,然后才能论定吉凶祸福。得月、得时对于日柱来说即意味着得到了好的月份和好的时辰。自然,这里所说的得月和得时,同时又是囊括了反面的不得月和不得时的。

所谓“行运凶吉”,这是结合大运方向的宜忌说的。宜的是吉,忌的是凶。什么是大运方向呢?前面我们在地支分析中曾经提到,寅卯辰属木,木的方位是东;巳午未属火,火的方位是南;申酉戌属金,金的方位是西;亥子丑属水,水的方位是北。由此得知,一个人的大运如果进入寅卯辰的木地,就是运行东方。倘若,东方木地正是他所需要适宜的,那末他行运便吉。反之,他需要适宜的是东方木地,却偏偏行入了他所禁忌的西方金地,可想而知,等待他的是何等的凶险了。

明白以上原理后,我们便可具体讨论“十干坐支、兼得月时和行运吉凶”这个综合性的问题了。因为这种讨论对于命理分析来说,是至关重要而不是可有可无。

先论东方甲乙木的坐支、兼得月时和行运吉凶。

甲木属阳,有栋梁之材的美称。一个人的日干如果是甲木的,最好降生在申月(秋七月)和子月(冬十一月)为吉,假如柱中天干透出庚辛金,就好比良木遇上斧凿,哪有不名利双收的?其中尤其以辛金正官为更加佳妙。在行运过程中,碰上申酉金地或辰戌丑未土乡,也主大有发挥的好运。平生最忌寅午戌合成的火局,以及柱中透出丁火伤官,因为这将是个辛苦劳力,作事无成的命,如果以上这些不在自身八字中碰上,而在大运中碰上的,也主不顺。假如合局中丁火透出,四柱中有天干戊己土和地支辰戌丑未,再行财运(对甲木来说是土运),这就成了伤官生财的格局,不但没有不利,反而要发大福。

乙木属阴,是一种充满生气的木,逢春生而花叶茂盛。假若八字地支逢亥卯未会成木局或申子辰会成水局,再行北方水运,那末即使天干中有丙丁火或庚辛金透出,却也无关紧要。如果单逢寅午戌火或巳酉丑金,就会多所伤残。伤残而再行南方火运,那就难保不夭折了。

甲乙木日出生的人,如果身坐巳、酉、丑、申、戌金乡,行运宜土金分野。假如坐寅、卯、辰而不结木局,宜在时辰里见土见金。若果出生在巳、酉、丑、申月,要时辰引归亥、卯、未、寅也好,否则可能是个穷读书的命。

甲子、甲戌、甲申、甲午、甲辰、甲寅六甲日出生的人,以辛金为正官,庚金为偏官,戊己土为财。假如年、月、时三柱透出戊、己、辛等财官,要降生在三秋季节或辰、未、戌、丑等四季末一个月的土月中,以及碰上地支金土等局,这财官才能有用,如果年、月、时三柱中透不出戊、己、辛等财官,可是出生的月份却是三秋季节或辰、未、戌、丑等土月,月支中有财官的,也作财官论定。再如六甲日出生的人,见甲乙则夺财,丁火则伤官,名利十分艰难。如果出生春夏,八字地支中有火木局,这时虽有财官,也泄了气。在行运上,六甲日出生的人喜行西方金运和四季末一个月的土运。因为金土之地,属于甲木向官临财之运。忌行东南木火之运,因为木火之地,属于甲木伤官败财之地。至于四柱中庚、辛都出现的,叫做“官煞混杂”,这时如果官煞中不能制伏一个,留下一个,反而是个贫贱的命。如果单见庚金,四柱中又没有可以制伏庚金的,那末这个庚金当作鬼论。所谓鬼,就足偏官无制的意思。在命书中,偏官有制才能作为有用的偏官,偏官如果没有制约,就变成害人的鬼了。如果六甲日生人遇鬼的,又要具体看看身强还是鬼强,身弱还是鬼弱,才能最后断定吉凶寿夭。在一般情况下,喜身旺鬼衰运,忌身衰鬼旺运。如果庚金制伏得中,那末按理又要作偏官论,不作鬼论了。

乙丑、乙亥、乙酉、乙未、乙巳、乙卯六乙日出生的人,以戊土为正财,己土为偏财,庚金为正官,辛金为偏官。假如这时年、月、时等柱上透出戊、己、庚等财官,又出生在三秋和辰、未、戌、丑等四季末一个的土月中,或者碰上金土等局,那末这财官才算有用。如果柱中不见戊、己、庚等财官透出,但却出生在三秋季节或辰、未、戌、丑等土月,月支中有财官的,也作财官看。再如六乙日出生的人,同样怕见木火,见甲乙则夺财,丙火则伤官,这样对于名利,自然就十分艰难了。又如六乙日人假如出生在春夏季节,以及碰上木火局的,即使有财有官,也因泄气而顶不了用。在行运上,六乙日生的人喜行向官临财的西方金运和辰、未、戌、丑土运,忌行伤官败财的火木之运。末了,六乙日人也怕官煞混杂。如果有煞无制,以鬼论断。有煞有制,也要贵在制得适中,制得太过或不及的,都不是福。

次论南方丙丁火的坐支、兼得月时和行运吉凶。

丙火属阳,这是太阳的正气,能生万物。逢丙日出生的人喜欢出生在春夏季节,这样就自然成就,精神百倍,这时如再碰上天德、月德,行运东方,那就妙不可言了。对这神人来说,由于木火通明,即使碰上壬癸水也不要紧,只有在戊土透泄丙火威力的情况下,方才减了一定的分数。在大运中,如果岁君相犯,恐有官府刑狱、破财丧服的灾难。又如丙火日生的人,如果不生在春夏而偏生在秋冬,不生在白天而偏生在子夜,并且地支又合水局,那就只能是仆从的命,一生离别孤独,而贫夭残疾伴身了。

丁火属阴,为社会上千家万户所用的人间凡火。金银铜铁,如果不得丁火炼制,那就不能成器。凡逢丁日出生的人,喜欢降生夜间,月份以巳、酉、丑金月为妙。正月逢寅,出生丁日,这是天德星照命,主人大吉,如果再兼卯支,那就更好了。行西北运贵,如运行南方,剥官退职。

丙丁日出生的人,自坐申、子、辰、亥水位,又引归金时,这时如果生月逢上寅、午、巳等月份,这就叫做“水火既济”,是大贵的命。如出生在夏天午月五月的,忌三合火局,因为火炎水干;如出生在冬天子月十一月的,忌三合水局,因为水盛火灭——两者都成不了水火既济之象。在行运上,丙丁日出生的人宜西北的金水分野,但忌太过不及,偏阴偏阳,总要水火相停,彼此相济才好。此外如逢申、子、辰、亥等月出生的,则时柱地支要见寅、午、戌、巳等火地才好,如果不是这时生的,那末在大运中交上木运也好,因为木能生火,否则就虚名不贵了。

丙寅、丙子、丙戌、丙申、丙午、丙辰六丙日出生的人,以庚、辛金为财,癸水为正官,壬水为偏官。如果年、月、时柱中透出庚、辛、癸等财官,要生在秋冬金水局中,这财官方为有用。假如柱中透不出庚、辛、癸等财官,而出生月份地支却逢秋冬金水局而有财官的,也以财官论。再如六丙日人,如见丙丁为夺财,己土为伤官,如此就名利艰难了。又如生在夏月和辰、未、戌、丑等四季末一个月的火土局中,即使逢上财官,也已没有气了。在行运上,六丙日人喜行向官临财的西北金水分野。至于柱中壬、癸一起出现,官煞混杂,要有制伏,否则反而低贱。假如有壬无癸,不见制伏,那就以鬼论断,当然这时也要区分一下身、鬼彼此的强弱,才能判定吉凶寿夭。喜行的是身旺鬼衰运,忌行的是身衰鬼旺之乡。假如壬水制伏得中,则可作偏官取用,但制伏太过,就反而没了福分。

丁卯、丁丑、丁亥、丁酉、丁未、丁巳六丁日出生的人,以庚、辛金为财,壬水为正官,癸水为偏官。若年、月、时柱中透出庚、辛、壬等财官,要出生在秋冬金水局中,这财官方才有用。如果年、月、时柱中透不出庚、辛、壬等财官的,只要出生在秋冬金水局而月支中有财官的,也作财官论定。六丁日出生的人,见丙、丁夺财,戊土伤官,名利自然十分艰辛。如果出生在夏月和辰、未、戌、丑等四时末一个月的火土局中,即使逢上财官也泄了气。在行运上,六丁日人喜行西北金水分野,忌伤官、败财。此外,官煞混杂,有煞无制,也都是命中所忌的。

再论中央戊己土的坐支、兼得月时和行运吉凶。

戊土属阳,属于堤岸城墙之土,只能拒水,不能种养万物。大凡城堤不遭刑冲破害,人民方能安居乐业。平生喜甲乙木,因为煞能化印。替如甲木原是克制戊土的七煞,可是甲木能够生火,火又能够生土,所以说是煞能化印。行运忌西方金地,要火生扶,嫌水克制。此外“戊己重犯,名利两失;辛庚叠逢,作事进退”。

己土属阴,属于田地山园之土,可以种养万物。和戊土不同的是,己土可以刑冲破害,因为田地园土总是被人耕破的。平生喜生春夏和辰、巳等木火官印之乡,忌伤官损印。行运以东北方为佳,这时如果更兼亥、卯、未木,决主富贵无疑。反之,如果值逢辰、戌、丑、未的土,就成了一种背禄逐马、劫财刑伤的架势,主人“破耗讼服不一”,处世艰难。

戊己日出生的人,坐支为亥、卯、寅位,叫做勾陈得位。行运宜东方北方水木分野。如亥月子月出生的,以逢辰、戌、丑、未、巳、午等时为贵;如辰、戌、丑、未、巳、午月出生的,以亥时子时为贵。巳日丑年丑月,西方不吉,南方大显。

戊辰、戊寅、戊子、戊戌、戊申、戊午六戊日出生的人,除戊戌为魁罡,财官喜忌,论于日下外,其余五日都以壬癸水为财,乙木为正官,甲木为偏官。如若年、月、时等柱中透出壬、癸、乙等财官的,要生在春冬水木局中,这财官才能有用。如若天干中透不出壬、癸、乙等财官,只要生在春冬水木局而月支有财官的,也作财官论。再说戊日出生的人,见戊、己土为夺财,见辛金为伤官,这种人名利很是艰难。如果出生在三秋或辰、戌、丑、未等四时末一个月的土月中,以及碰上金土等局的,那就财官无气了。在行运上,喜行向官临财的东方北方水木分野,忌行败财伤官的西方和辰、戌、丑、未所居的中央之地。又如六戊日人柱中透出甲乙,官煞混杂,要有制服才好,如果单有甲木透出无制,这就要以鬼来论了。论时要分身鬼强弱,以定吉凶寿夭。喜的是身旺鬼衰运,忌的是身衰鬼旺运。假如甲木制服中和,则可作偏官论。

己巳、己卯、己丑、己亥、己酉、己未六己日出生的人,以壬、癸水为财,甲木为正官,乙木为偏官。假如年、月、时柱中透出壬、癸、甲等财官,又出生在春冬水木局中的,那就财官有气。假如年、月、时柱中透不出壬、癸、甲等财官,只要生在春冬水木局而月支有财官的,也作财官论。再如六己日人,见戊、己土为夺财,见庚金为伤官,名利艰难。如果出生在三秋和辰、戌、丑、未四时末一个月的土月,以及金土局中,即使有财官也泄了气儿。在行运上,喜行东方北方水木分野,忌伤官败财运。和六戊日人一样的是,六己日人也怕官煞相混,如若有煞无制,就以鬼论,七煞被制太过,贫困不发。

接论西方庚辛金的坐支、兼得月时和行运吉凶。

庚金属阳,属于金银铜铁一类,秉受太阳精华,要见丁火炼铸,才能成器。如见太阳丙火,那就等于不见。平生喜行东方南方木火通明之运,金得铸冶,如果碰上寅、卯临于甲乙,以及巳、午、未官星印元得气之乡,也都能够有所发挥。只是碰上西方北方,难免金沉水底,那就不能成器了。

辛金属阴,属于水银朱砂赤碧珍珠一类,秉受日精月华,最宜处于金清水秀,土气丰厚的地方。平生喜行西方北方运,如行辰、戌、巳等东南方运,要以五行四柱不见丁火为妙,如果丁火一见,好比珍珠落进炉里,那就秀而不实,不能成器了。对于宵、午、戌火局煞旺的,要身强才能抵挡得了。假如柱中有亥、卯、未木局,更见丙丁透出,这时又要行午、未运才能发福。此外如巳、酉、丑合成金局,以行东方运为大吉,不宜南方。

庚辛日出生的人,坐下如逢寅、午、戌、巳火,而出生却又偏在寅、午、戌等火月的,为了取得中和,那就要以金、土时辰出生为贵了。反之,如果秋三月或十一、十二月出生的,那就又要以木、火时辰出生的为贵了。行运喜东方南方木火分野,忌太过不及,偏阴偏阳。总的来说,金如没有火炼,就成不了器,火如没有金被炼,就显示不出作用,总要金火相停,才能乘轩服冕,成其气候。若果火太炎而无土,则金必败,只有土来生金,才能成就铸印之象。

庚午、庚辰、庚寅、庚子、庚戌、庚申六庚日出生的人,除庚戌、庚辰为魁罡,财官喜忌,见于日下外,其余庚午、庚寅、庚子、庚申四日,以甲、乙木为财,以丁火为正官,丙火为偏官。如果在年、月、时干支透出甲、乙、丁等财官的,要生在春夏火木局中,这财官方才有用,如果年、月、时柱中透不出甲、乙、丁等财官的,只要出生在春夏木火局中而月支有财官的,也作为财官论处。再如四庚日人,见庚、辛金夺财,癸水伤官,名利艰难可以想见。又如生在秋冬金水之中,虽有财官,也没了生气。在行运上,大致喜行东方南方木火分野,向官临财的运,不喜行西方北方金水分野,伤官败财的运。庚日生人如果柱中透出丙火丁火,官煞混杂,这时煞如没有制约,就会贫贱一生。如单有丙火无制,则以鬼论,如果有制,则以偏官别论。

辛未、辛巳、辛卯、辛丑、辛亥、辛酉六辛日出生的人,以甲、乙木为财,丙火为正官,丁火为偏官。如果年、月、时柱中透出甲、乙、丙等财官,要生在春夏和木火局中,这财官方才有用。如果年、月、时柱中透不出甲、乙、丙等财官,只要生在春夏木火局中而月支有财官的,也以财官论定。六辛日人以见庚、辛金为夺财,壬水为伤官,名利艰难。如果生在秋冬和碰上金水局,那就财官无气了。在行运中,六辛日人喜东方南方木火分野,向官临财的运,不喜西方北方金水分野,伤官败财的运。此外,六辛日人也怕官煞混杂。如果有煞无制,就以鬼论,制得太过,也不见有福。

末论北方壬癸水的坐支、兼得月时和行运吉凶。

壬水属阳,原是甘泽长流之水,有滋生草木,长养万物之功。逢壬日出生的人,最好生在春夏月份,如果生在值令的秋冬,就是没了生意。又如地支中见到寅、午、戌、火,火能生土,土是壬的官星,这样官星有了生助之气,也就名誉自彰了。此外如逢金局,又出生在八月酉月,金水相生,名利两遂;如逢水局,又出生在三月辰月,由于辰月壬日生的,为天德星入命,所以主贵。偎如壬日出生而地支有亥、卯、未的,则运行南方可以发财。

癸水属阴,原是大海无涯之水,不能生长万物,但也有认为癸水属于雨露滋润之水,可以滋助万物的。癸日出生的人喜欢降生在春秋季节,行巳、午、戌地的火运,这就发福非常了。最忌的是辰、戌、丑、未土运。如见己土及丑、未月,更带三刑,一生平平而过。

壬癸日出生的人,坐下如见辰、戌、丑、未、巳、午的,为玄武(北方水神)当权,宜行南方中央火土分野的运,假如太过不及,偏阴偏阳,就贵命不实。又如生在辰、戌、丑、未四时末一个月的土月,以及巳、午等火月的,最好在时支上出现代表水的亥、子等字;反之,如果出生在冬月的,则时支上宜出现代表火土的辰、戌、丑、未、巳、午等字。平生行运喜欢西方北方的金水分野,因为得金生助,可以有荣。忌比肩劫财。如果冬十一月出生,地支又三合成水,那就水涨横泛,土岸崩溃了。所以水无土则滥,土无水则干,土得水而受润通气,水得土而成堤为河,两者相互为用,不可偏废。

壬申、壬午、壬辰、壬寅、壬子、壬戌六壬日出生的人,除壬辰为魁罡,财官喜忌,论于日下外,其余五日,用丙、丁火为财,己土为正官,戊土为偏官。如果年、月、时柱中透出丙、丁、己等财官的,要生在夏月和辰、戌、丑、未四时末一个月或火土局中,这则官才能有用。如果年、月、时中透不出丙、丁、己等财官,只要出生夏天火土局中而月支有财官的,也以财官论定。再如壬日生人,见壬、癸水夺财,乙木伤官,名利艰难可知。如果生在春冬或水木局中,那就财官无气,顶不了用。在行运上,喜行南方中央火土分野,向官临财的运。此外柱中如戊土己土杂出,官煞瑕杂,这时如煞不受制约,反而成了贱命。如果柱中单单出现七煞的戊土而没有制服,作鬼论处。然而这时又要分分身鬼强弱,以定吉凶。再如七煞制伏得中,私偏官论,制伏过头,也要损了福分。

癸酉、癸未、癸巳、癸卯、癸丑、癸亥六癸日出生的人,以丙、丁火为财,戊土为正官,己土为偏官。如果年、月、时柱中透出丙、丁、戊等财官,要出生在夏月或辰、戌、丑、未四时末一个个的火土局中,这财官才顶得上用。如来年、月、时柱中透不出丙、丁、戊等财官,只要出生在夏月和辰、戌、丑、未等月的火土局中,而月支带上财官的,也作财官论定。六癸日人,见壬、癸夺财,甲木伤官,都不吉利。如果降生春冬水木局中,那也财官无气,顶不了用。在行运上,喜行南方中央财官之运。平时怕官煞混杂,假如有煞无制,作鬼论处,煞被制伏太过,也不吉利。

\section{日干、格局和干支合化刑冲的看法}

入手看命,按照命理学家的常规是先看日干,因为这是代表自身的一个天干,凡是年、月、日、时四柱中的干支,都要围着这个天干来论定吉凶宜忌。日干有得时和失时的不同,如日干碰上旺、相的月支,就是得时;碰上休、囚、死的月支,就是失时。比如日干是甲木,木生于春,水能生木,所以月支如果碰上春月,就属于旺,碰上冬月,就算是相,都属得时;如果日干甲木不生于冬春之月,而偏偏生在木能生火,火生木休的夏月,木能克土,土旺木囚的四时末一个月,也就是三、六、九、十二月的土月,甚至生在金能克木、金盛木死的秋月,就都属于失时。得时的自身强旺,不得时的自身衰弱。关于五行和一年四季旺相休囚死的关系,前面我们已有专篇谈过,一阅便知。此外,观察日干和月支的关系,还有利于我们对一个人八字格局的认定。看过日干和月支的关系后,再看日干下坐的是哪一个地支,这地支对日干来说,在寄生十二宫中处于什么样的状态?是长生、沐浴、冠带、帝旺,还是衰、病、死、基、绝、胎、养?此外,还不要忘了看看与日柱干支紧贴的左邻右舍月柱和时柱的干支;以至于年柱的干支,这些干支所代表的阴阳五行,对于自身日干来说,生克扶抑的情况如何?

这种看法说实在点,就是在日干为主的基础上,以年柱为根,可以知世脉的盛衰;以月柱为苗,可以知父母亲荫庇的有无,兄弟的得力不得力;以日柱为己身,日支为妻子,可以知妻子的贤患不贤滋;以时柱为花实,可以知小辈的兴旺不兴旺。

这里,要紧的是,我们还千万不要忘记根据日干五行所需要的生克扶抑取一下用神,然后再看看这用神喜的是什么,忌的是什么。这样才能通盘考虑,以下论断。现将命中日干、格局和干支的合化刑冲的看法具体解析如下:

1.先看日干强弱

日干的名称很多,分别有日主、命主、身主、日元、日神等叫法。在一个人的八字里,日干的地位是最为举足轻重的,因为日干代表的是一个人的本身。因此从这点出发,首先判定一个人自身日干的衰旺强弱,就成了看命的首要条件了。

大凡判定一个人日干强弱的方法,主要有以下三点。第一,看日干在所生的月份得令还是不得令。比如日干甲乙见月支寅卯,丙丁见月支巳午,戊己见月支巳午或辰戌丑未,庚辛见月支申酉,壬癸见月支亥子,都属于最佳的得令生旺状态,所以这日干就强。反之,日干在出生的月令中,如果处于一种或休、或囚、或死的状态,那就是弱。第二,日干在四柱中得到的生助是多还是少。比如日干属甲乙木,如匹柱中得水木之助多的,就是旺得势。反之,日干甲乙木得不到四柱中水木之助,甚至反遭金制火泄,那就弱而失势了。第三,把自身日干对照四柱地支,如果碰上长生、沐浴、冠带、临官(禄)、帝旺或墓库的,就是得地得气,自身自然强旺。反之,就是失地失气,强旺不起来。以上得令、得势、得地三者全都集于一身的,日干处于极旺状态,因此宜克宜泄;反过来,失令、失势、失地三者全都集于一身的,日干处于极弱状态,因此宜生宜扶。如果旺极没有克泄,弱极没有生扶,都是凶险不好的命。除了极旺、极弱,留下来的都属中和或偏旺偏弱的命,大致不会差到什么地方。且看举例:

〔日主极旺的命〕\par
\setlength{\tabcolsep}{0em} % 表格内容水平padding
\begin{tabular}{cm{3em}<{\raggedleft}cl} %
(年)&\scriptsize{比肩}&甲寅&\scriptsize{禄}\\
(月)&\scriptsize{伤官}&丁卯&\scriptsize{乙木帝旺}\\
(日)&&甲子&\scriptsize{癸水沐浴}\\
(时)&\scriptsize{比肩}&甲子&\scriptsize{癸水沐浴}\\
\end{tabular}

这一命造,日干甲木生于卯月仲春,处于帝旺状,所以得令。甲木在四柱中,生它的有日支和时支两个癸水作为印绶;和它同类的有年干、时干两个甲木作为比肩,以及月支卯中乙木作为劫财,所以得势。甲禄到寅,年支寅为甲的禄,月支卯对甲说,使甲处在帝旺的状态,所以得地。这命甲木得令、得势、得地,三者兼得,所以日主极旺。

〔日主极弱的命〕\par
(年)\quad{\scriptsize{偏财}}戊申{\scriptsize{绝}}\par
(月)\quad{\scriptsize{七杀}}庚申{\scriptsize{绝}}\par
(日)\quad{\scriptsize{\qquad{}}}甲午{\scriptsize{绝}}\par
(时)\quad{\scriptsize{七杀}}庚午{\scriptsize{死}}

这一命造,日主甲木,生于木绝的初秋申月,所以不得时令。甲木在四柱中,月柱庚申和年支、月支申金,都是克它的七杀,而日支、时支两个丁火又拼命的泄它,加上没有比劫为助,所以失势。甲木在年、月、日、时四个地支中,都处于死绝的状态,所以失地。失令、失势、失地,三者都丧失尽净,所以是个日主极弱的命。

〔日主中和的命〕\par
(年)\quad{\scriptsize{劫财}}甲寅{\scriptsize{帝旺}}\par
(月)\quad{\scriptsize{偏印}}癸酉{\scriptsize{绝}}\par
(日)\quad{\scriptsize{\qquad{}}}乙亥{\scriptsize{死}}\par
(时)\quad{\scriptsize{伤官}}丙子{\scriptsize{病}}\par

这一命造,日主乙木,生在木绝的仲秋酉月,所以不得时令。乙木在四柱中,得月干、日支、时支和年柱水木的生助,所以得势。乙木在月支和日支中虽处于绝病之地,可是I年支帝旺得气》所以中和。综合以上失时,得势,地气得失偈于中和的分析,所以是个日主中和的命。

2.次看命中格局

在四柱命理学中,看取格局也是不可忽视的一个重要环节。虽然对于这个环节各有各的看法,有的命理学家认为丢掉格局同样可以看命,然而在大多数情况下,能够看取格局,总比完全丢掉格局要强得多。按照命书说法,格局有正格和变格的不同,正格有正官、七杀、正财、偏财、正印、偏印、食神、伤官八种,如果并掉财、印两格的正偏,也有六种,至于变格,那就千变万化,较难捉摸了。

那末,怎样来具体看格局呢?

首先,可采用“月支藏干”的原则来看取格局。所谓“月支藏干”,就是月柱的地支隐藏着什么天干的意思。在采用这一原则时,先要看一看月支所藏的天干本气有没有透到月干或年干、时干上去。如果有的,臂如寅月天干透甲,卯月天干透乙,辰月天干透戊,巳月天干透丙,午月天干透丁,未月天干透己,申月天干透庚,酉月天干透辛,戌月天干透戊,亥月天干透壬,子月天干透癸,丑月天干透己,都可根据这一透出天干和日柱天干之间的生克关系,取为格局。如月支透出是正财的,就是正财格;月支透出是偏财的,就是偏财格;月支透出正官的,就是正官格;月支透出偏官的,就是偏官格;月支透出印绶的,就是印绶格;月支透出偏印的,就是偏印格;月支透出伤官的,就是伤官格;月支透出食神的,就是食神格等。

其次,对于子月、卯月、酉月等月支中只含一个本气天干的,如果这种本气不在年、月、时等柱上透发出来,也可根据月支和日干之间的关系取为格局。

第三,如果月支所藏属于本气的天干没有在月、时、年等三柱上透发出来,那末再看月支所藏的其他天干有没有透出来的,比如月支寅的本气是甲木,然而甲木如果没有上透天干,而其中所藏的丙火或戊土倒有透出来的,那就也可根据丙火或戊土与日柱天干之间的关系,取为格局,至于到底取丙取戊,这就又要看两者的力量大小了。

第四,如果月支本气和所藏的其他五行一个也没有透出天干,那就根据月支所藏各干,比较它们之间的强弱盛衰,挑选其中一个较为得力的,然后再根据这一天干和日干之间的关系,取为格局。此外,如果月支藏干和日柱之间的关系属于比、劫、禄、刃的,则一般不取为正式格局,而要特别取为变格了。比如甲日寅月,乙日卯月,丙日巳月,丁日午月,戊日巳月,己日午月,庚日申月,辛日酉月,壬日亥月,癸日子月,由于甲禄在寅,乙禄在卯,丙禄在巳,丁禄在午,戊禄在巳,己禄在午,庚禄在申,辛禄在酉,壬禄在亥,癸禄在子,所以都可撇开其他正格,取为建禄的变格。对于以上看取格局的办法,也少不得举例说明,以求刨根究底。

〔命造举例〕\par
(年)\quad{\scriptsize{\qquad{}}}辛丑\par
(月)\quad{\scriptsize{正官}}戊戌{\scriptsize{戊土\quad{}辛金\quad{}丁火}}\par
(日)\quad{\scriptsize{\qquad{}}}癸未\par
(时)\quad{\scriptsize{\qquad{}}}壬子

这一命造,生于癸日,而月支戌中藏有戊土、辛金、丁火,其中戊土透出月干,辛金透出年干,由于戌的本气是戊土,所以得取戊土来定格局。对于癸水来说,戊土是克它的正官,所以这命的格局是正官格。

〔命造举例〕\par
(年)\quad{}甲辰\par
(月)\quad{}丙子{\scriptsize{癸水}}\par
(日)\quad{}丙申\par
(时)\quad{}己亥

这一命造,生于丙日,而月支子中藏有癸水,因为子、卯、酉三支所藏只有本气,所以根据上述取格原则的第二条,按照癸水和丙火之间形成的正官关系,可径取为正官格。

〔命造举例〕\par
(年)\quad{}乙丑\par
(月)\quad{}壬申{\scriptsize{庚金\quad{}壬水\quad{}戊土}}\par
(日)\quad{}丙辰\par
(时)\quad{}己巳\par

这一命造,生于丙日,而月支申中藏有庚金、壬水、戊土,其中申的本气庚金没能透出年、月、时三柱,而只有壬水透出月干,所以根据丙火和壬水之间的阳彼克阳我者为偏官的关系,取格局为偏官格。

〔命造举例〕\par
(年)\quad{}乙卯\par
(月)\quad{}壬申{\scriptsize{庚金\quad{}壬水\quad{}戊土}}\par
(日)\quad{}壬申\par
(时)\quad{}癸巳\par

这一命造,生于壬日,而月支申中藏有庚金、壬水、戊土,其中壬水虽然透出月干,可是因为月干和日干之间属于比肩关系,所以不取为格。再看申中庚金、戊土,由于庚金属于申支的本气,力量显然超戊土,所以便取庚金和壬水
之间的偏印关系,定格局为偏印格。

至于命中尚有其他种种名目繁多的格局,我们另立专篇再谈。

3.三看干支的刑冲化合

八字中天干和天干,地支和地支之间的刑冲化合,因为对于命局的阴阳五行,有若不可忽视的影响,所以也深为命理学家所重视。其看法大致是:

〔两干相合,贵乎得中〕比如甲己合土,彼此地支都乘生旺,这就中而不偏。如果甲太旺,己太柔,这样一者太过,一者不及,就不中和了。

〔阳得阴合,阴得阳合〕命书说:“天干合,阳得阴合,福慢;阴得阳合,福紧。”比如阳干甲得阴干己合为财,阴干己得阳干甲合为官,虽都是福,可是又有前者福慢后者福紧的不同。又有认为,命中合多,性喜淫乐,所以女命最忌合多,然而对于甲己和乙庚的彼此相合,又为女命所不忌。

〔两干争合,阴阳偏枯〕如果碰上两个天干和一个天干相合,这在命书中称为阴阳偏枯。比如两甲合一己,或者两己合一甲,就好比夫多妇少,妇多夫少一样,难免相争相妒,用情不专,所以不是好事。

〔日干合化,通根乘旺〕这是说日干与年、月、时等天干的相合,要生在本干五行生旺的月份,这样就旺而有根了。比如甲己合而化土,必须生在辰、戌、丑、未土旺的月份;乙庚合而化金,必须生在巳、酉、丑月或者申月金旺的月份;丙辛合而化水,必须生在申、子、辰月或亥月水旺的月份;丁壬合而化木,必须生在亥、卯、未月或者寅月木旺的月份;戊癸合而化火,必须生在寅、午、戌月或者巳月火旺的月份。否则就不可论化。

〔间隔遥远,虽合难化〕天干的化合除了必须结合出生月份,还要看看地位远近。如果年干属乙,时干属庚,彼此路途间隔遥远,合力单薄,就不一定论化了。

〔天干相合,有吉有凶〕天干合掉之后,大多自身还有五六或六七分力量,比如乙庚合金,金虽被合,然而自身的性质却还多半存在。对于天干相合后的是吉是凶,要根据具体情况而定。在一般情况下,合并不是件坏事,可是一旦如果日干的喜神或用神被合,那就凶神乱意,情况不妙了。

〔地支六合,区别对待〕这就是说,命局所喜的地支被六合合去之后,就要减吉;所忌的地支被合掉后,就会减凶。此外地支合局还可解除刑冲不吉。具体情况要作具体分析。例如命局喜子,地支中有丑字合而化土,就减了吉的分数;反之命局忌子,地支中有丑字合而化土,则又可以减去凶的程度。又如命局喜子,但逢午冲,这时如果又有未去合午,那就解了子午之间的彼此相冲。这里需要注意的是,地支六合要彼此紧贴,如日支和月支紧贴,日支和时支紧贴,否则彼此隔位,就不合了。此外,地支如是二卯合一戌,或二戌合一卯;二寅合一亥,或二亥合一寅的,叫做妒合。

〔地支三合,论吉论凶〕在地支的申子辰合水,亥卯未合木,寅午戌合火,巳酉丑合金三合局中,如果合局为命中所苕则吉,所忌则凶。比如命局喜水的,而地支中出现申子辰三合水局,就以吿论;命局忌水的,地支中如果出现申子辰三合水局,那就要以凶论了。此外,如果地支出现申子或子辰合水,亥卯或卯未合木,寅午或午戌合火,已西或酉丑合金通常叫做半合局。半合局以紧贴为妙。然而,无论是三合局的还是半合局,都怕逢冲,造成破局。

〔地支三会,活看吉凶〕在地支的寅卯辰会东方木,巳午未会南方火,申酉戌会西方金,亥子丑会北方水等三会方向中,也和地支的三合局一样,如果会局为命中所喜则吉,所忌则凶。比如命局喜水的,地支中出现亥子丑会成北方水的,就以吉论;反之命局忌水,地支中却偏偏出现了亥子丑会成北方水,那就要以凶论了。在力量上,地支三会方向的威力大于三合局,而三合局的威力又明显大于六合。为此,如果四支中三合局或三会方向同时并见,一般都弃合论会。

〔地支六冲,本气为重〕命中地支相冲,以本气为重。比如寅申相冲,寅的本气是甲木,申的本气是庚金,所以两者相冲,首先体现在甲木和庚金的冲克上。在通常的情况下,总是申金胜而寅木败,可是如果时令碰上火旺金衰,或水旺火衰,则彼此相冲,又可分别造成寅火胜而申金败或申水胜而寅火败的局面。在吉凶上,如果命局所喜的地支冲败则凶,所忌的地支冲败则吉。这里要补充的是,相冲的地支必须彼此紧贴,才能算冲,否则彼此隔开,就只好作稍有波动看了。六冲和三合局一起出现,由于三合局的力量大于六冲,所以以合局论。不过,如果是半合的话,有时逢冲,也可把合解掉。例如酉年酉月亥日巳时,月支酉和时支巳半合,但日支亥和时支巳相冲,就解掉了月支酉和时支已的半合。

〔地支刑害,略微动摇〕地支子刑卯,卯刑子,原是水木相生;巳刑申,巳申本合;丑刑戌,戌刑未,都是同类的土;至于申刑寅,未刑丑,无非彼此相冲而已。同样,地支相害,也和地支相刑一样,影响不大,只是略微动摇而已。

\section{富贵贫贱和寿夭疾病的推算}

儒家先师,生前累累若丧家之狗的孔老夫子曾无可奈何地感叹说:“死生有命,富贵在天。”把孔子称为“累累若丧家之狗”,语出《史记•孔子世家》,原话是郑人姑布子卿在暗中观察了孔子相后,对他弟子子贡所说的。后来子贡把实话告诉了孔子,孔子非但不以为怒,还欣然笑曰:“形状(人的形相),末也。而谓似丧家之狗(指狼狈的神态),然哉!然哉!”

由于奋斗了一辈子也没能在政治上施展抱负,最后在不得已中才做起了教书先生的孔子,终于不得不在碰得头破血流后低头认命了。“不知命,无以为君子”,这就是他在追求上失败后,心情渐次趋向淡泊的自我写照。那末,怎样才能预先“知命”呢?这在孔子生活的当时,除了一些零星的相术外,显然是件不可能的事,因为当时只知有命却不知道命的推算方法。

自从算命术发明以后,由于在很大程度上遵循着这位儒家先师的遗训,所以推究预算一个人一生的富贵贫贱,穷通寿夭,自然便成了算命术的首要目的。

为什么同样一个人,生出来后的处境竟然会这么不同呢?按照命理学家的解释,就逛当人在受胎之初,阴阳二气交流,真精妙合之际,如果禀受的是清气,那就为智为能,禀受的是浊气,那就为愚为不肖。为智为能的在社会上必定多所获益,所以或富或寿;为愚为不肖的不能自我奋发,所以贫贱而夭。这反映到命里,就自然会在每个人自己的生辰八字里显现出来。

说到这种对命主本人未来的推法,命理学家也自有他的一套办法,这就是推富货贫贱先看命中自身日干得令不得令,次看用神得力不得力,末看行运顺利不顺利。如果日干得令,用神得力,运遇财官,往往富贵发福,大吉大利,反之则贫困潦倒,苦不堪言。推生死寿夭要细论岁运和原局用神的喜忌,如果岁运碰上忌神冒头,喜神无救,那就轻则凶,重则死了。但也有一种“以生月定之”的说法,《玉门关集》说:“凡寿以生月定之,生月居支干纳音旺处,及五行相生不逆,日时并胎,皆得数,不相刑克者,主上寿。”此外,《三命通会》有(壶中子》说:“颜回夭折,只因四大空亡。”注曰:‘甲子、甲午旬命无水,甲申、甲寅旬命无金。若只见两重,流年大运遇一重圆之,亦是。”就又聊备一格了。至于富贵贫贱在死生寿夭上,则一律一视同仁,可见在势利社会中,死神并不势利,

现在结合具体命造八字,试析如T:

〔命造八字〕\hspace{4.4em}大运\par
(年){\scriptsize{\qquad{}}}丁丑{\scriptsize{辛金七杀\hspace{2em}}}辛丑\par
(月){\scriptsize{\qquad{}}}壬寅\hspace{4em}庚子\par
(日){\scriptsize{\qquad{}}}乙亥\hspace{4em}己亥\par
(时){\scriptsize{用神}}己酉{\scriptsize{辛金七杀\qquad{}}}戊戌\par
\hspace{9.75em}丁酉\par
\hspace{9.75em}丙申

这一命造,日主乙木,生于初春寅月,得令生旺,又得月干壬水贴身扶持,所以自身旺盛。再看年支丑中辛金和时支酉的本气辛金,虽为克制乙木的七煞,然而年干丁火,月支寅中丙火,又把这两个七煞双双制服,只存时上己土偏财作为用神。结合行运,早年运逢辛丑、庚子,用神己土泄气,不为大佳。此后转入戊戌十年土运,用神偏财得正财之助,所以财源滚滚。此后行运申中戊土虽然也有正财,可威力比起戊戌一步运来,则显然不及了。

〔命造八字〕\hspace{6.4em}大运\par
(年)丁酉\hspace{7.3em}乙巳\par
(月)丙午\hspace{7.3em}甲辰\par
(日)戊寅{\scriptsize{寅中甲木为杀,用神\qquad{}}}癸卯   东方木地\par
(时)丁巳\hspace{7.3em}壬寅\par
\hspace{11.75em}辛丑 北方水地\par
\hspace{11.75em}庚子\par

这一命造,日主戊土,生于仲夏午月,火气炎盛,又遇年、月、时三柱干支丙丁之火生扶,戊己之土助身,可谓身旺之极。旺者宜制宜泄,所以取日支寅中制我的甲木七煞,或年支中泄我的酉中辛金作为用神。再看行运,早中年寅卯辰合木,运行东方,得木制克;中晚年又转入北方子丑水运,水旺生杀,所以是个贵过于富的命造。

〔命造八字〕 大运\par
(年)	丁丑己土伤官	壬贫\par
(月)	癸卯乙木印绶,用神	辛丑\par
(日)	丙戌戊土食神	庚子		•北方水地\par
(时)	食神戊戌戊土食神	己亥\par
		戊戌		、西方金地\par
		了酉\par

这一命造,日主丙火,生于仲春卯月,乙木生火,本属好事,可惜年支丑中己土,日支戌中戊土,时柱干支两重戊土,食伤重重,致使自身泄气太过。综览全局,当取乙木印绶作为用神,既可生我,又可制服太过的土。再看大运,壬寅以后,亥子丑一片水地。水虽能够制火,可是水又能够生木,这步运纵然比不上直接行东方木运来得更好,可足却还勉强说得过去。然而一旦行运进入戌酉金地,虽说金为财运,可是金能克木,财星破印,用神被制,这就难保活命了。

〔命造八字〕	大运\par
(年)印绶乙丑辛金	甲申\par
(月)印绶乙酉辛金,死	癸未\par
(日)丙辰	壬午\par
(时)正财辛卯	辛巳\par

这一命造,丙火生于酉月死地,根气全无,加之时上透出正财辛金,年支月支丑酉也伏藏正财辛金,可谓财多身弱。对于财多身弱而又没有比劫助身的命来说,用神最好取印,因为印能生身,所以这里的用神就压在生我的乙木上了。然而年干、月干的乙木,虽然和日支时支的辰、卯通根相连,然而从这两个乙木自身的坐支来看,却落在无情相克的财星辛金上面,可谓财星破印,上下无情。在这种情况下,表面看去用神虽多,可却不顶真用,加之没有命中所客的比、劫、禄、刃相辅,不禁举步艰难。好在大运癸未、壬午,火来助身,日干得地,也可娶妻生子。可是一经交入辛巳运,运中天干辛削用神乙木,财能坏印,运中地支已与命中年支月支丑、酉合成金局,又去大力克削命中日支时支所藏印星,一时用神彻底被伤,夭亡就在劫难逃了。

此外,《三命通会》还在卷八《六丁日乙巳时断》篇中说:“丁亥日,乙巳时,时日并冲,忧伤妻子。已酉丑、申子辰金水二局,财官得用,以富贵论。”接着还举这样两个八字命造为例,一是壬辰、甲辰、丁亥、乙巳,说是王擴侍郎的命。一是丁亥、甲辰、丁亥、乙巳,竟是个乞丐的命。

除了推命主贫富寿夭,有的命书还不忘同时推人疾病。对于疾病的推法,先要把五行和五藏六府联系起来,然后再根据五行生克的原理来加以分析。按照中医理论,五行和五藏六府的相互搭配是:

〔甲〕胆	〔乙〕肝
〔丙〕小肠	〔丁〕心
〔戊〕胃 〔己〕脾
〔庚〕大肠	〔辛〕肺
〔壬〕膀胱、三焦	〔癸〕肾、心包络

其中胆、胃、大肠、小肠、三焦、膀胱为六府,性质属阳,所以都配阳干;肝、心、脾、肺、肾为五藏,心包络则附于心系,性质属阴,所以都配阴干。歌曰:
\begin{tightcenter}
甲胆乙肝丙小肠,丁心戊胃己脾乡,\\
庚是大肠辛属肺,壬系膀胱癸肾藏,\\
三焦亦向壬中寄,包络同归入癸方。
\end{tightcenter}

又曰:
\begin{tightcenter}
甲头乙项丙肩求,丁心戊胁己属腹,\\
庚是脐轮辛为股,壬胫癸足一身覆。
\end{tightcenter}

与此同时,古人又有把十二地支和身体各部分联系起来的,因为不及与五藏联系来得重要,所以一般并不受人重视。现把十二地支和身体各部分联系的歌括照录如下:
\begin{tightcenter}
子属膀胱水道耳,丑为胞肚及脾乡,\\
寅胆发脉并两手,卯本十指内肝方,\\
辰土为脾肩胸类,巳面齿咽下尻肛,\\
午火精神司眼目,未土胃脘膈脊梁,\\
申金大肠经络肺,酉中精血小肠藏,\\
戌土命门腿踝足,亥水为头及肾囊,\\
若依此法推人命,岐伯雷公也播扬。
\end{tightcenter}

在具体看法上,以日柱干支为主,结合五行生克太过不及,以作定论。甓如日干是甲乙木,四柱八字中出现庚辛申酉等金多的,木就遭克,可能有肝胆惊悸劳瘵,手足顽麻,筋骨疼痛,头目昏晕,或口眼歪斜,左瘫右痪,以及跌扑损伤等症。假如日柱天干仍是甲乙木,四柱八字中出现丙丁巳午火多的,并且没有水来相济,这时木气被泄太过,又可出现内热口干,痰喘咯血,中风不语,以及女人血气不调,怀孕胎落,小儿急慢风,夜啼咳嗽,面色青黯等症。至于为什么木被金制或火泄过头会出现这些症状,因为这牵涉到祖国传统医学的理论,这里就不作探讨了。

为了便于记忆诵读,以少少许胜多多许,现采摘部分古赋,聊备一格于下。赋云:
\begin{yinyong}
筋骨疼痛,盖因木被金伤;眼目昏暗,必是火遭水克;土虚逢木旺之乡,脾伤定论;金弱遇火炎之地,血疾无疑。
\end{yinyong}
又云:
\begin{yinyong}
木逢金克,定主腰胁之灾;火被水伤,必主眼目之疾;心肺喘满,亦干金火相刑;脾胃损伤,盖因土水战克。支水干头有火遭,必腹病心朦;支火干头有水遇,则内障晴盲。炎上(火)烦焦蒸土曜,头禿眼昏;润下(水)纯湿无土制,肾虚耳闭;荧惑(火星)乘旺临离巽(火风),风中(中风)失音;太白(金星)坚利合兑坤(金土),兵前落魄。
\end{yinyong}

\section{从八字五行看人的性情相貌}

算命先生在算命时,有时口里总常挂着这人的性情脾气怎么样怎么样,兴致来时,还会发一番有关其人相貌的髙论。这是怎么回事呢?原来五行推人性情,只在日上时上,而以自身的日柱五行为主,并且不论纳音。对此,命书自有一番有趣的说法。

\textbf{〔木〕}东方震位,木号青龙,名曰曲直。五常主仁,其色青,其味酸,其性直,其情和,旺相(参见《五行的旺相休囚死和寄你坐十二宫》),主有博爱恻隐之心,慈样恺悌之意,济物利人,恤孤念寡,直朴清高,行藏慷慨,丰姿秀丽,骨格修长,手足纤腻,口尖发美,面色青白,语句轩昂,此则木盛多仁之义。休囚(参见篇目同前)主瘦长发少,拗性偏心,嫉妒不仁,此则木衰情寡之义。死绝(即死,参见篇目同前)则眉眼不正,悭吝鄙啬,肌肉干燥,项长喉结,行坐不稳,身多欹侧。遇火色带赤,见土色带黄,逢金色带白,见水色带黑。其余四行例见。

\textbf{〔火〕}火属南方,名曰炎上,五常主礼。其色赤,其味苦,其性急,其情恭。旺相,主有辞让端谨之风,恭敬谦和之义,威仪凛冽,淳朴尊崇。面貌上尖下阔,形体头小脚长,印堂窄而眉浓,鼻准露而耳小。精神闪烁,语言急速,性躁无毒,聪明有为。太过,则声焦面赤,摇膝好动。不及,则黄瘦尖楞,诡诈妒毒品,言语妄诞,有始无终。

\textbf{〔土〕}土属中央,名曰稼穑。五常主信。其色黄,其味甘,其性重,其情厚。旺相,主言行相顾,忠孝至诚,好敬神佛,不爽期信,背圆腰阔,鼻大口方,眉清目秀,面肥色黄,度量宽厚,处事有方。太过,则执一古朴,愚拙不明。不及,则颜色忧滞,面偏鼻低,声音重浊,事理不通,狠毒乖戾,不得众情,颠倒失信,悭啬妄为。

\textbf{〔金〕}金属西方,名曰从革。五常主义。其色白,其味辛,其性刚,其情烈。旺相,刚英勇豪杰,仗义疏财,知廉耻,识羞恶,骨肉相应,体健神清,面方白净,眉高眼深,鼻直耳红,声音清亮,刚毅果决。太过,则好勇无谋,贪欲不仁。不及,则悭吝贪酷,事多挫志,有三思,少决断,克薄内毒,贪淫好杀,身材瘦小。

\textbf{〔水〕}水属北方,名曰润下。五常主智。其色黑,其味咸,其性聪明,其情良善。旺相,则机关深远,足智多谋,学识过人,诡诈无极,面黑光彩,语言清和。太过,则是非好动,飘荡贪淫。不及;则人物矮小,行事反覆,性情不常,胆小无略。

对于这种五行配人性情相貌,大致逢生旺的好,逢死绝的差,此外如有太过或不及的,也都因为失掉了中和之美而流入偏执一路,成不了上品的人格。

由于五行配人性情相貌,内容较杂,为了钩玄提要,便于记忆,前人也有把这概括起来,编成小赋的形式。《宰公要诀》说:“智髙量远,盖因水处深源,笃信守仁,祇因土成山岳;仁慈敏厚,木成甲乙之方;性速辨明,火应丙丁之位;誉髙义重,因金归合庚辛。处于中者,正性不移。或盛或衰,性情变易。水乘衰败,性昏无赖;土力太微,蔽执寡用;木归蹇地,太柔而治事无规;火数未兴,小辨而太伤无决;金当浅薄,虽义而有始无终。”《子平赋》说:“美姿貌者,木生于春夏之时;无智识者,水困于丑未之日。性质聪明,盖为水象之秀;临事果决,皆因金气之刚。五行气足,体必丰肥;四柱无情,性多顽鄙。”《指迷赋》也说:“文章明敏兮,定须火盛;威武刚烈兮,乃是金多。木盛则怀恻隐之心,水多则抱机巧之智。至土之性,最为贵重。”这些口诀,易读易记,并且角度不同,可以彼此补充,所以很受欢迎。

然而,在正式算命中,一个人四柱八字所秉受的五行,又总常常和这里描绘的一些性情相貌对不起号来,有的甚至还出入很大,来个一百八十度的截然大相反。

后来,人们更多的则是根据日常接触,把仁而有博爱恻隐之心,直朴清高,骨格修长的人说成是秉有木性气质,把礼而有辞让端谨之风,精神闪烁,聪明性急的人说成是秉有火性气质,把信而言行相顾,忠孝至诚,背圆腰阔,面肥色黄的人说成是秉有土性气质,把义而英勇豪杰,仗义疏财,体健神清,面方白净的人说成是秉有金性气质,把智而机关深远,诡诈无极,面黑光彩,语言清和的人说成是秉有水性气质。这样,就把八字五行硬性相配的说法,给反因为果,反果为因地颠倒过来了。

有趣的是,早在我国第一部医学典籍《黄帝内经》中,也有与这相类似的通过阴阳五行原理,把整个人群划为二十五种人的做法,并不厌其繁地详细记述了每一种人性情形貌的大致情况。不过由于这是医学上用以研讨各种不同类型的人的性情脾气,从而为治疗服务的一种学术体系,所以不可和这里的五行划分划上等号。然而无论如何,如果从根本上作进一步考察,则又由于我国古代阴阳五行哲学原理广泛深入于每一学术领域,所以两者在貌似不同之中,又有着本质上惊人的相同一面。

\section{看八字,论六亲}

在算命中,算命先生除了给本人论命之外,还常少不了从一个人的八字,来推论他的六亲。对于算命先生这一套看八字,论六亲的办法,命书中也自有它的说法。按照惯例,他们的看法是:

〔祖上〕看祖上位在年宫(柱),一般以偏印为祖父,伤官为祖母。\par
〔父母〕看父母位在月宫,一般以偏财为父,正印为正母,偏印为庶母。但也有不分正偏的。\par
〔兄弟〕兄弟位置附于月宫,命书以比肩为兄弟。至于姐妹,有的命书不论,有的认为和兄弟的看法一样。\par
〔妻妾〕看妻妾位在日支,命书以正、偏财为妻妾。\par
〔子息〕看子息位在时宫,又以偏官(七杀)为男,正官为女。

为什么要以偏财和印为父母,比肩和劫财为兄弟,正财偏财为妻妾,偏官正官为子女呢?这里先从夫妻说起。比如说东方甲乙木,假设甲是阳木为兄,乙是阴木为妹,现在甲木把乙木配给庚金为妻,因为古人认为,女子理应柔顺而听命于丈夫,好跟着丈夫过日子,所以克乙木的庚金便自然成了她丈夫。同样道理,庚是阳金为兄,辛是阴金为妹,庚把妹妹辛金配给克她的丙火为妻;丙是阳火为兄,丁是阴火为妹,丙把妹妹丁火嫁给克她的壬水为妻;壬是阳水为兄,癸是阴水为妹,壬把妹妹癸水嫁给克她的戊土为妻;戊是阳土为兄,己是阴土为妹,戊把妹妹己土嫁给克她的甲木为妻。这样一个阳干,娶一个被克的阴干为妻。在命理学术语中,被克的称为正财和偏财,所以正、偏财在八字中就成了算命先生看妻子的一个重要手段。然而,阳干娶阴干为妾,在实践中却也并不这么绝对,假如一个人八字中日柱的天干恰是阴干的话,那末他也理所当然地可娶阳干或阴干为妻,如乙木娶戊、己土为妻妾便是。

再说何以偏财和印为父母?在通常情况下,庚金是由己土生出来的,其中己土属于阴性,所以为母。命书以生我者为正印、偏印,所以印就成了母亲。这里,生庚金的己土在天干中与甲相合,这在前面的篇章中已经有所提及。阳木甲克阴土己,自然甲就成了己的丈夫。可是对于庚金来说,我克者为正财、偏财,现在庚金克甲木,阳与阳相克,岂不就是偏财?前面我们说过,我克的正财、偏财都是妻妾,现在忽然又把偏财说成了父亲,岂不荒谬?为此,命书又有结合年宫看父母的说法,进行解围。

那末,克我的偏官为儿子,正官为女儿又何从说起呢?原来乙庚结为夫妇后,乙木生出丙丁的火来。丙火克庚金阳克阳,所以丙火便成了庚金的儿子;丁火克庚金阴克阳,所以丁火就成了庚金的女儿。

末了再说兄弟。因为兄弟是同类,所以庚金的比肩庚金,便就成了兄弟。

以上这些用用神术语名称来看六亲的理由,《子平真诠》曾简要总括道:
\begin{yinyong}
正印为母,身所自出,取其生我也。若偏财受我克制,何反为父?偏财者母之正夫也,正印为母,则偏财为父矣.正财为妻,受我克制,夫为妻纲,妻则从夫。若官杀则克制乎我,何以反为子女者?官杀者财所生也,财为妻妾,则官杀为子女矣。至于比肩为兄弟之类,又理之显然者。   
\end{yinyong}
现结合具体命造,举例分析如下:
〔母寿父夭例〕
(年)丁酉偏财
(月)壬子
(日)丁卯
(时)印绶甲辰

命造日主丁卯,生于冬月,月干壬水为正官,月支癸水为七煞,官煞气旺,自身衰弱,所以当取生我的印绶甲木为用神。现在地支卯辰东方会木,印旺有气,而我克的偏财酉中辛金却偏处一偶,没有根攀。正印为母,偏财为父,所以
命主母长寿而父早丧。

〔妻遭刑克例〕
(年)偏印癸卯乙木比肩
(月)比肩乙卯乙木比肩
(日)乙未丁火食神乙木比肩己土偏财
(时)印绶壬午丁火食神己土偏财

这一命造,日主乙木,生于春月,得令身旺。日支未和时支午合,其中己土为妻为财。可惜四柱中乙木重叠,比肩太过,所以不必运行比劫,妻子也会遭克。

对于这种看六亲祸福吉凶,穷通寿夭的办法,林惠祥《算命的研究和批判》曾这样概括道:“亲属自己身强者佳,遇克他的,即遭克死,但如能逢生他的,略有希望。或逢有能克克他的,也有救。偏财旺者父长寿,比劫多者父早死,正印有力者母寿,财多破印即主克母。例如己身日主为甲木,其财(父)为戊己,印(母)为壬癸,戊己土克壬癸水,水(母)被克死。比肩财多者兄弟多。见比肩、劫财、败财都会克妻妾及父,例如己身为甲木,比肩、败财即甲乙木,妻妾为正财、偏财即戊己土,甲乙木能克戊己土,故妻妾被克死。又父也是偏财,故也被克。坐下妻宫主妻好,妻即用神也主有贤妻。妻星多主克妻,妻星两透,偏正杂出,则有多妻。日干坐下的地支遇刑冲会克妻。己身强妻弱者,应配能补救这种弱势的女人,这称为‘硬配’。官杀多,伤兄弟姐妹,例如正官偏官的庚辛金太多,则伤及比肩败财的甲乙木,即伤兄弟姐妹。伤官食神多会伤子息,因为丙丁火克庚辛金。枭印多克祖父母(壬癸水克丙丁火)。看子息的方法,应先寻子星,再对照时的地支,照生旺死绝(参看本书《五行的旺相休囚死和寄生十二宫》,笔者注)来推,推法依下面的歌:
\begin{tightcenter}
长生四子中旬半,沐浴一双保吉祥。\\
冠带临官三子位,旺中五子自成行。\\
衰中二子病中一,死中至老没儿郎。\\
除非养取他人子,入墓之时命夭亡。\\
受气为绝一个子,胎中头产有姑娘。\\
养中三子只留一,男女宫中仔细详。  
\end{tightcenter}

歌中意思说,假如己身是甲,子星便是庚(庚是甲的偏官——笔者注)。庚如逢时支己,便是在长生的状态,可以在中年时得四子;如逢午便是在沐浴的状态,可以有二子;在冠带、临官都有三子;在帝旺有五子;在衰中有二子;病中有一子;在死的状态,无子;逢墓的状态子会夭亡。在绝中有一子;在胎中会有长女;在养中生三子留一子。”

现综括全文,似乎可作这样的理解,就是四柱中如果年柱上有吉神或用神的,说明主人祖基必丰;月柱上有喜神或用神的,说明主人有父母荫庇,并且兄弟和睦;日支有喜神用神的,说明夫妻协力,爱情甜美;时柱上有喜神用神的,说明主人子女得力。反过来说,如果年柱有忌神的,说明主人祖上破败寒微;月柱上有忌神的;说明父母刑伤,兄弟不和;日支上有忌神的,说明夫妻爱情生活不谐;时柱上有忌神的,说明子女难育或子女不肖。当然,这些出现在年、月、日、时上的忌神如果有制,则又可以逢凶化吉,另作别论。

再如从用神结合十二宫等来看,八字中如果以印绶作为自身的喜神,或印绶遇上长生之地,说明主人有着很厚的福荫,并且父母双双长寿;月柱上的印绶遭逢死绝之地,或者作为用神的印绶被破,说明父母不全,或难以得到父母的荫庇。八字中如果比肩、劫财为自身的喜神用神,或比肩坐在禄地的,说明兄弟得力;反之,比肩如为忌神的,说明不是兄弟不睦,就是兄弟有凋难。八字中如果以财为喜神用神,并且生化有力的,说明妻子贤惠而又得力;反过来如果财为忌神,或冲合争分,说明妻子不服丈夫,夫妻感情不好。女命以官、煞为丈夫,八字中如果官星得用,丈夫髙贵而自身也贵;以食、伤为子女,八字中如果食、伤为喜神用神的,说明子女贤孝,有享子女之福的可能,如果食、伤逢冲,或坐孤辰孤宿,说明子女稀少,或者命中克子。

此外,在看八字论六亲时,还可结合行运来看。其中父母结合幼运,夫妻兄弟结合中运,子女结合老运。比如命中幼年有鸿运的,说明有着双亲的福荫;中年有鸿运的,说明夫妻协力同心,或者得兄弟的力;晚年有鸿运的,说明得子女之力。

\section{怎样看大运和流年的吉凶荣枯}

从前面《八字中大运、流年和命宫的推算》中,我们得知了大运的推算方法和起运岁数。譬如公元1940年庚辰农历十月十四日出生的男命,可以对照《新编万年历》査出,那一年的农历十月十四日上一个节是十月初八的立冬,下一个节是十一月初九的大雪。由于庚辰是阳年,按照规定,阳年生的男命起运岁数是顺数到下一个节止,然后以三天为一岁去除。那年庚辰年的十月是小月,所以从十月十四顺数到十一月初九的大雪是二十四天,再用二十四去除三,正好是整数八。这就是说,这位先生的起运岁数该是八岁。起运岁数算出以后,接下来是排大运的干支。同样我们知道,大运的干支是根据生月的干支推排出来的,起运岁数如果是顺数的,就由生月干支的下一个干支顺排下去,逆数的,就由生月干支的上一个干支倒排上去。现在已知生月是丁亥,起运的岁数是顺数的,所以这命的大运干支就应该从丁亥起依次顺数为戊子、己丑、庚寅、辛卯、壬辰、癸巳、甲午、乙未等等。

由于命书规定,大运的天干地支每字都管五年,所以一个天干和一个地支加起来是十年。看前五年的虽以天干为主,但要结合地支一起推看,看后五年的一般丢掉天干只看地支,这就是命书所说的大运中地支重于天干的原则。

现在仍以上述八岁起运的男命为例,可以得知他的大运八到十七岁是戊子,十八到二十七岁是己丑,二十八到三十七岁是庚寅,三十八到四十七岁是辛卯,因十八到五十七岁是壬辰,五十八到六十七岁是癸巳,六十八岁到七十七岁是甲午,七十八岁到八十八岁是乙未。

推论大运的吉凶荣枯,先要从本命日柱天干出发,分析本命五行的宜和忌,再结合大运干支所代表的五行对本命日柱天干的生克扶抑,是宜是忌,以及有没有刑冲化合等等,才能作出最后的判断。对此,《命理探原》曾引陈素庵的话:
\begin{yinyong}
宜与不宜,全凭格局;利与不利,但问天干。破格者(破坏自身格局的)值之为戚(忌),助格者遇之为欢(宜)。日弱者扶之而气盛,日强者抑之而全美。旺日复到旺乡(大运的五行对于自身日干来说显得太旺),必罹悔吝(凶);衰日再临衰地(大运的五行对于自身的日干来说显得太衰),定主摧残(凶)。吉若财官印辰,喜于相见;凶如刑冲枭劫,多主不安。  
\end{yinyong}
比如日干是金,命中金强,最理想的就是能行木火的财官运,因为火能制金,不致使金太旺而走向反面,而金又能够克木,使强金有了疏泄的余地,如果遇上生金的土运和比肩、劫财的金运,对于本人来说,无疑造成了一种“旺日复到旺乡”的架势,因此极为不吉利。反之如果日干是金,命中金弱,那就又要来个一百八十度的大相反了,宜行生我扶我的印绶、比劫运,否则身弱再遇财官,岂不等于“衰日再临衰地”,又哪有不“摧残”的?

以上推大运吉凶荣枯的办法,如果结合用神来作判断,就是八字四柱配合较好,原局中有用神的,那末一生行运一般多为水流花开,得意得很。然而对于一些八字原局四柱配合得并不理想,原局中没有用神,或者用神较弱的,就要看他行运时碰上碰不上用神了。一个人行运一辈子,不可能在方向上都遇木遇金,如果原局中所缺的用神,在行运中一旦补上,纠正命中五行的偏差或不足,也可发福或有所作为。对于这两种原局和行运中的用神,行家们分别有“原局用神”和“行运用神”的叫法。从总体上说,凡是日主旺的,宜行财、官运。日主旺而财、官弱的,行到财、官运时必定一下大发。如果日主旺过头的,则宜行食、伤运,以泄自身过旺之气。反过来,凡是日主弱的,宜行比、劫或印绶运。日主弱而财、官旺的,行比、劫运比印绶运更好。假如日干不强不弱,叫做中和,中和的人也适宜于行财、官运。

还是举例说明来得容易理解,现把刚才所举男命的八字和大运列举如下,用算命先生的原话进行分析:

(年)比肩庚辰正财偏印伤官
(月)正官丁亥偏财食神
(日)庚申比肩偏印食神(禄)
(时)比肩庚辰正财偏印伤官
八岁起运,大运如下
八	戊子 地支会水
十八己丑
二十八庚寅
三十八辛卯地支会木
四十八壬辰
五十八癸巳
六十八甲午地支会火
七十八乙未

断曰:天下没有穷戊子,世间没有舌庚申。庚申称占禄,所谓禄,就是寄生十二宫里的临官。男子占禄,杖地造屋,有权。命里四金,二土,一水,一火,少木。五行少木,亥里藏甲木,辰里藏乙木,生日木日(在纳音五行中,庚申属木)。

此命生于冬天,金寒而重,要出门,这条龙要游四海好。年份比肩,月份正官、食神,日支比肩,时干比肩。比肩太重,与父亲相克,这是因为比肩的金克作为父亲的偏财木。

八字中比肩多命硬,爱人年龄相差要多,否则要重婚。同岁、兔、狗不配,与猴、鸡、鼠相合。兔属卯,龙属辰,卯辰相害;狗属戌,龙属辰,辰戌相冲,所以都不配。猴属申,鼠属子,龙属辰,申子辰合水,所以相合。此外辰酉合金,鸡属酉,所以和属辰的龙也相合。

八学中用神取财、官为好,土运不好。

八岁起运。八到十二岁偏印,十三到十七岁伤官,十八到二十二岁正印,二十三到二十七岁墓库,因为辰戌丑未都属于库,二十八到三十二岁比肩,三十三到三十七岁偏财,三十八到四十二岁劫财,四十三到四十七岁正财,四十八到五十二岁食神,五十三到五十七岁墓库,五十八到六十二岁伤官,过了六十二岁步步顺利,寿有八十四。其中十三到十七不要和父亲住在一起,有相克。二十三到二十七有墓库,多损失。二十八到三十二岁有比肩,因为命中已有四金,比肩多,故而不好,自己真心待人,人家却要暗算。三十三到三十七岁偏财,寅申相冲驿马运,动荡而有财运。三十八到四十二辛金劫财运,八字缺木,金多克财,有损失,但却亦有后福。四十三到五十二正财、食神,有十年大运。五十三到五十七岁辰土当心身体,做事收心。逃过五十八到六十I二岁伤官,稳过六十三到六十八的地支已长生运,因为已处在庚金的长生状态。此后六十八到七十二岁偏财这步运好。1总之六十三岁以后大运干支一片木火,用神得力,所以老来喜乐无忧。

注意,五十三到五十七防财、防身体,三十二以内吃亏,三十三以后偏财。四十三到五十二岁发大财。

此命幼年福气不错,但最好与父亲分开些,如能过房移居更好。青年时有挫折,中年开始转好运,直至晚年。一生有偏财,但也经常破费。此命东西南北尽皆通,有名望,从政从文更好。如出生正八点有一个儿子,八点后有两个儿子。此命要多注意身体,因为命中金太多。又因为时支属辰,算是日主庚金的养地,所以子息好,晚年幸福,一生衣禄无尽。

此人命大,四十三岁以后步步升高,不是一般的人。

从上所述可以看出,这是一张推排得非常详细的命书,其中尤其对于大运的计算,更是不厌其烦。

又如命书有光绪二十一年乙未(公元1895)年农历二月十三日酉时出生的这样一个女命:
(年)比肩乙未比肩偏财食神
(月)偏财己卯比肩
(日)	乙亥劫财正印
(时)比肩乙酉七杀

九岁八个月起运,大运如下:
九岁	庚辰
八个月
十九	辛巳
八个月
二十九	壬午
八个月
三十九	癸未
八个月

本造乙木,生于卯月,年干时干透出乙木,地支亥卯未三合木局,自身可谓强旺矣。月柱天干己土虽为偏可是被四周上下一片乙木克贼,所以难以取作用神。再看时支西中辛金属于克我的七煞,虽可取作用神,然而酉在对乙亥出生那天的甲戌旬来说,正巧碰上空亡,加之周围木多,非但无力制木,反而有被木反侮的可能,所以用神极不得力。从五行生克贵在中和平衡的观点看,乙木过于旺盛,旺者宜克宜泄,现在克既无力,泄也只有年柱未中丁火一角,因算而反为虚有其表,不幸坠入青楼为妓。

此后二十二岁流年丙辰(1916),大运辛巳,岁、运天干丙辛相合,流年地支辰与命中夫星七煞(偏官)相合。合多宜婚,有从良嫁人的迹象。迨到这一年的十二月辛丑,大运已与命中时支酉,流年月支丑三合金局,用神得力,所以跳出青搂,做了人家的妾。

这样平平过到三十岁流年甲子(1924),这时大运已经进入壬午地。原命四柱八字卯、酉本属相冲,这时流年和太岁子午又彼此相冲,这样四个方面一齐冲了起来,用神酉金非但得不到大运午中己土的生助,并且在一片交战中精疲力竭。用神一倒,便就难逃香消玉殒的厄运了。

此外,看大运除了结合五行宜忌,还有一种“年管少年,月日管中年,时管晚年”的说法,这种说法在《三命通会》卷二中,还具体有“以生月为初限,管二十五年,生日为中限,管二十五年,生时为末限,管五十年”的三限划定。看法大致以日干为出发点,其中年柱干支为客神用神的则少年发达,为忌神的则少年困苦;日月干支为喜神的则中年亨通,为忌神的则中年蹇滞;时辰干支为春神的则晚年安荣,为忌神的则晚年零落。但是一般认为,这种看法比起大运的推算来,就未免显得简单了点。

大运之外,流年和命宫的好坏,也都由日主的天干出发进行五行宜忌的详细推断,宜的为吉为荣,忌的为凶为枯。不要忘记的是,看流年时,还必须把流年放到大运里去进行观察分析。大运吉而流年吉的,其年大吉;大运吉而流年凶的,不致大凶;大运凶而流年凶的,难逃其凶;大运凶而流年吉的,难保大吉。这里大运的力置,足以左右流年。打个比方,大运好比大河,流年好比小河,太河水满,小河也满,大河水浅,小河也浅,大河的水势足以影响小河,而小河的水势却难以影响大河。

再有一种流年和命宫结合的看法。流年法以轮流值年的“太岁”为首,“命宫如值流年吉神,其年则福,值凶煞,其年则祸”。由于那些神煞分布在子丑寅卯等十二年中,每年各不相同,所以对照命宫来看,每年的吉凶也就各各不同了。然而又由于这些神煞凶多吉少,并且方法粗糙简单,所以袁树珊《命理探原》指责这种看法说:“凶煞有十之九,吉神尽十之一,其不适用可知。舍干枝五行生克之至理,而惟务此虚文,宜其毫无效验,贻讥大雅。”连命理学家本人都不相信,那荒谬的程度,也就由此可知了。

说到“太岁”,大致有两种情况,一种是四柱中的年柱,叫做当生太岁;另一种是一年年轮流过来的,叫做游行太岁。当生太岁是主终身的,游行太岁则每年游行十二宫,以定一年四时的吉凶祸福。对于后面的游行太岁,《三命通会》卷二《论太岁》说道:“经云:‘岁伤日干,有祸必轻;日犯岁君,灾殃必重。’”“岁君伤日者,如庚年克甲日为偏官,譬君治臣,父治子,虽有灾晦,不为大害。何则?上治其下,顺也,其情尚未尽绝。如甲日克戊年的偏财,譬如臣犯其君,子忤其父,深为不利。何则?下凌其上,逆也,其凶决不能免。若五行有救,四柱有情,如甲日克戊年,四柱原有庚申金,或大运中,亦有将木制伏纯粹,不能克戊土为有救。经云,‘戊己愁逢甲乙,干头须要庚辛’是也。”

\section{入格八字举例}

在命理学家眼里,虽然人们出生时辰的八字千变万化,错综复杂,可是总得有个格局统帅全局,否则不就乱了套?这就是入格八字的由来。

关于八字的格,从来就为命书所重视。如《三命通会》卷六的《杂取各格》,以及《星平会海》卷十的取格析例,都不惜以整卷的篇幅,对八字的各种格局,作出了详尽的分析。

关于入格八字的看取办法,专以代表自身的日干为主,然后配合月令、年时,而以月令为重,其中逢官看财(财能生官);逢财看煞(财能生煞),逢煞看印(印能化煞),逢印看官(官印相生)。歌云:
\begin{tightcenter}
一官二印三财位,四煞五食六伤官。\\
立法先详生与死,次分贵贱吉凶看。  
\end{tightcenter}
在命书中,对于命的格局,有正格和变格的不同。凡是以官、煞、印、财、食伤等入局的,叫做正格。正格以外,叫做变格。现把命书所载有关格局,择要举例如下。

\subsubsection{正格}

\circledNum{正官格}在用神中,正官是天地的正气,忠信的尊名,虽然治国齐家,劳苦功髙,可是八字中出现正官,只要一位就足够了,并且出现的部位,以月柱为正,又怕刑冲。如果官星太多,或官煞(偏官)相混,或部位偏离月柱,或官星逢冲,就难以入格了。这就是命书所说的:“正气官星,切忌刑冲,多则论煞,一位名真。”如果时柱上兼有财星的,更是贵不可言。

〔入格八字〕金状元\par
\setlength{\tabcolsep}{0em} % 表格内容水平padding
\begin{tabular}{cm{2.5em}<{\raggedleft}cc} %
(年)&&乙卯&\\
(月)&&丁亥&\scriptsize{正官}\\
(日)&&丁未&\\
(时)&\scriptsize{正财}&庚戌\\
\end{tabular}

八字月柱中亥宫壬水,克自身天干丁火为正官;并且四柱中只有一位官星,而时柱上又出现丁火的正财庚金,所以便把它取作正官格局了。诗曰:
\begin{tightcenter}
正官须在月中求,无破无伤贵不休。\\
玉勒金鞍真赋态,两行旌节上星州。    
\end{tightcenter}

\circledNum{偏官格}所谓偏官,就是七煞有制的称谓。如果八字中同时出现偏印、偏财,身煞平衡,就是大富大贵的命。如果七煞被制过头,或者八字中官煞混杂,那就退职离官,多致凶死。又如行运进入煞乡,也主不死而穷。此外,日柱天干无根而遇煞制,或煞重藏根,主人都有被煞制死的可能性。所说煞重藏根,就是七煞直接藏在自身日柱的地支中,比如乙酉日生的人,酉是辛金,克乙木为煞,这时如果年柱、时柱中又不见制煞的干支,就是很不吉利的命。

〔入格八字〕 沈郎中\par
\setlength{\tabcolsep}{0em} % 表格内容水平padding
\begin{tabular}{cm{2em}<{\raggedleft}cl} %
(年)&\scriptsize{}&丙子&\scriptsize{子中癸水制服偏官}\\
(月)&\scriptsize{}&甲午&\scriptsize{丁火偏官}\\
(日)&\scriptsize{}&辛亥&\scriptsize{}\\
(时)&\scriptsize{}&辛卯&\scriptsize{}\\
\end{tabular}

月柱午中丁火克自身辛金既可称为偏官,也可称为七煞,然而这丁火因为被年柱子中癸水制服,根据有制为偏官,无制为七煞,以及取格看重月令的原则,所以这是一种偏官的格局。至于年干丙火克辛金为正官,因为不在月令之上,所以不取为格。此外,月干透出甲木作为辛金的正财,也在一定裎度上为这格局添了分数。诗曰:
\begin{tightcenter}
偏官有制化为权,唾手登云发少年。\\
岁运若行身旺地,功名大用福双全。    
\end{tightcenter}

\circledNum{七煞格}

在命局中,七煞是克我的神,需要有制为福,好比恶宿小人,须要制伏,也可为我所用。在命书中,虽然有“‘七煞’有制,谓之偏官”的说法,可在举格局中,却也并不分得那么清楚。我们且看下面一个七煞格局:

〔入格八字〕 李寺丞\par
\setlength{\tabcolsep}{0em} % 表格内容水平padding
\begin{tabular}{cm{2em}<{\raggedleft}cl} %
(年)&\scriptsize{}&己巳&\scriptsize{}\\
(月)&\scriptsize{}&丁卯&\scriptsize{}\\
(日)&\scriptsize{}&丙午&\scriptsize{自旺}\\
(时)&\scriptsize{煞}&壬辰&\scriptsize{}\\
\end{tabular}\\
这里,时上壬水克自身丙火为煞,然而周围却也不乏制水的土,可见“七煞”、“偏官”等格,原,也并不区分限制得那么明显,所以有的命书,索性把“偏官”、“七煞”统称为偏官格或七煞格,倒也来得干净。

按照“时上一位为贵”的原则,凡是上好的七煞格局,七煞的位置一定要出现在时柱上,并且只要一位,不可多见。假如时柱出现七煞,而年、月、日柱上又重复出现的,那就非但不贵,反而成了辛苦劳碌的命。对于“时上一位为贵”的七煞格,只要本身自旺而有制伏,行运进入七煞旺乡,必定发迹。反之,如果命中七煞没有制伏,成了的的确确的七煞格,那末只要行到对七煞有所制伏的运里,也可发迹;只怕命里七煞没有制伏,而运又行到煞旺无制的境地,只就难免生出祸患了。诗说:
\begin{tightcenter}
    时上七煞是偏官,有制身强好命看。\\
    制伏喜逢煞旺运,三方得地发何难?\\
    元无制伏运须看,不怕刑冲多煞攒。\\
    若是身衰官煞旺,定知此命是贫寒。\\
\end{tightcenter}

\circledNum{印绶格}

在用神的名称中,印绶是生我的。符合这种格局的人,身旺为福,四柱中最喜透出官星七煞,以及行官煞的运,因为官煞能够生印。大忌柱中出现太多的财,这也因为财能伤克印绶。至于四柱纯都是印,由于印绶太过,反而走向事物的反面,所以注定主人是孤独的命。

〔入格八字〕 陈都宪\par
\setlength{\tabcolsep}{0em} % 表格内容水平padding
\begin{tabular}{cm{2em}<{\raggedleft}cl} %
(年)&\scriptsize{官}&癸未&\scriptsize{}\\
(月)&\scriptsize{正印}&乙卯&\scriptsize{正印}\\
(日)&\scriptsize{}&丙子&\scriptsize{胞胎逢印}\\
(时)&\scriptsize{官}&癸巳&\scriptsize{}\\
\end{tabular}\\
八字月柱中乙卯两个乙木,都是自身日干丙火的印绶,而日支子对于日干丙来说,在寄生十二宫中又正好处在天地气交,氤氳造物的胎的状态,这就更加需要印绶来促成了。妙在年柱、时柱中透出的两颗官星癸水,也为这正印的格局增添了分数。诗曰:
\begin{tightcenter}
月逢印绶喜官星,运入官乡福必清。\\
死绝运临身不利,后行财运百无成。   
\end{tightcenter}

\circledNum{正财格}

在格局中,正财最喜身旺印续,忌官星、忌倒印(偏印),忌身弱比肩、劫财。忌见官星的原因是怕盗财气,然而正财格中带有官星,又行上财旺生官的大运,则反而可以更加发迹。反过来说,如果柱中财多身弱,则怕行财旺生官的运,否则反而祸患临头。又如财神宜藏,藏则丰厚,蔣则浮荡,行运如果碰上比扃、劫财,非但分去财产,弄得不好,恐怕连命都要丢掉了。此外,也有些情况,比如身强财旺的逢财看煞,见官更好,所以命书乂有“财藏谣官者,当作贵推”的说法。

〔入格八字〕 孛罗丞相\par
\setlength{\tabcolsep}{0em} % 表格内容水平padding
\begin{tabular}{cm{2em}<{\raggedleft}cl} %
(年)&\scriptsize{}&壬申&\scriptsize{}\\
(月)&\scriptsize{}&丙午&\scriptsize{午中己土为财}\\
(日)&\scriptsize{}&甲午&\scriptsize{}\\
(时)&\scriptsize{}&壬申&\scriptsize{}\\
\end{tabular}\\
这八字月支午中己土,为自身甲木的正财,而自身日支又坐财地,所以在取格时把它看作是正财格。加之年柱、时柱的壬水申金,不是生我甲木的印绶就是制我甲木的七煞,所谓“逢财看煞”,对于印旺生财米说,可以说是致中和的最佳方案了。诗曰:
\begin{tightcenter}
财星忌透只宜藏,身旺逢官大吉昌。\\
怕逢比劫来相会,一生名利被分张。  
\end{tightcenter}

\circledNum{偏财格}

如果偏财出现在时上的,与时上七煞格局一样,只要一位,其他三柱不要重复出现。而这位时上的偏财,又怕逢冲,如果一旦行运进入财旺之乡,那就发福百端了。

〔入格八字〕 李参政\par
\setlength{\tabcolsep}{0em} % 表格内容水平padding
\begin{tabular}{cm{2em}<{\raggedleft}cl} %
(年)&\scriptsize{}&庚寅&\scriptsize{}\\
(月)&\scriptsize{}&乙酉&\scriptsize{正官}\\
(日)&\scriptsize{}&甲子&\scriptsize{}\\
(时)&\scriptsize{}&戊辰&\scriptsize{戊土偏财}\\
\end{tabular}\\
这一命造,月支正官不透,时柱戊土下坐辰支透气通根,所以考虑取戊土偏财为格局。入偏财格的,除了喜行财运,最怕逢冲外,还大忌行到羊刃败财和劫财的运,因为这样偏财被分被劫,就全完了。诗说:
\begin{tightcenter}
时上偏财一位佳,不逢冲破享荣华。\\
败财劫刃还无遇,富贵双全比石家。
\end{tightcenter}

\circledNum{食神格}

食神如在月令提纲中出现的只要一位,并且要身旺,因为食神能够生财,如逢身弱,那就难以克财了。对于入食神格的人来说,四柱忌印绶、官煞,以及比肩、羊刃(劫财)为祸。如果大运一旦进入食神财旺运,便可发福。

〔入格八字〕 蜀王\par
\setlength{\tabcolsep}{0em} % 表格内容水平padding
\begin{tabular}{cm{2em}<{\raggedleft}cl} %
(年)&\scriptsize{}&己未&\scriptsize{}\\
(月)&\scriptsize{}&戊辰&\scriptsize{身旺}\\
(日)&\scriptsize{}&戊辰&\scriptsize{}\\
(时)&\scriptsize{}&庚申&\scriptsize{}\\
\end{tabular}\\
这里蜀王八字的食神虽然出现在时干上,可是因为得力,所以便把它取作食神格局。由于自身戊土,生在春天末一个月的三月辰月,土令得时,所以身旺。诗曰:
\begin{tightcenter}
食神身旺喜生财,日主刚强福禄来。\\
身弱食多反为害,或逢枭食主凶灾。
\end{tightcenter}

\circledNum{伤官格}

“伤官见官,为祸百辟”,因为在用神中伤官是正官的克星,如果官来乘旺,那就祸不可言了。所以入伤官格的,伤官一定要彻底伤尽才好。所谓伤尽,就是四柱中一点也不出现官星。八字中如伤官多,有财星,或行身旺运,或行财旺运,都是富贵发福的命。命理学家认为,“伤官乃小人之情,喜财而妒官,又行财运,反生富贵”。此外伤官身旺无财的凶,这种人如果一旦碰上官运,就会大祸临头,理应尽快退身避职。大凡伤官只喜财旺身旺,如果行运进入财衰和死绝等地,那就脱财无禄,不是官司打败,就是死期临头了。

〔入格八字〕 通参政\par
\setlength{\tabcolsep}{0em} % 表格内容水平padding
\begin{tabular}{cm{2em}<{\raggedleft}cl} %
(年)&\scriptsize{}&甲寅&\scriptsize{}\\
(月)&\scriptsize{}&庚午&\scriptsize{己土伤官}\\
(日)&\scriptsize{}&丙午&\scriptsize{}\\
(时)&\scriptsize{}&甲午&\scriptsize{}\\
\end{tabular}\\
字月支午中己土对丙火来说,是我生的伤官。由于格中一点也没有丙火的官星癸水,所以伤官伤尽;加之伤官多,月干透出庚金财星,自身丙午,午又是丙的帝旺之乡,所以是个发福富贵的命。诗曰:
\begin{tightcenter}
火土伤官伤宜尽,金水伤官要见官,\\
木火见官官有旺,土金官去返成官,\\
惟有水木伤官格,财官两见始为欢。
\end{tightcenter}

以上正官、偏官(包括七煞)、财、印、食、伤八种格局,命书称为正格。下入变格。

\subsubsection{变格}

\circledNum{杂气印绶格}

在月份中,辰、未、戌、丑等月,也就是三、六、九、十二月,月支辰中有乙木、癸水、戊土,未中有丁火、乙木、己土;戌中有辛金、丁火、戊土,丑中有癸水、辛金、己土,这里面包涵了天地驳杂不纯之气。比如以东方的甲乙木举例,甲则坐镇寅位阳木,乙则坐镇卯位阴木,两者司管春令,面夺东方之气,可辰虽属于暮春三月,然而这时已处于春夏交接之处,方位已经偏向东南,所以受气不纯,禀命不一,有杂气之称。其他未、戌、丑三月,也照此原理类比。

在杂气印绶格中,如果自身日干是甲的,要出生在十二二月丑月的才称得上贵,因为丑中辛金是甲木的正官,丑中癸水是甲木的正印,丑中己土为甲木的正财。如果这时把握不住财、官、印中取哪一样来定格,可以观察月干中透出的是什么用神,然后再决定取舍。然而辰、戌、丑、未都是库藏,要有钥匙打开,才能发福,才能为我所用,而这种打开库藏的钥匙,就是刑冲破害。但这种刑冲破害,要恰到好处,否则冲破过头,反而伤了福份。大抵杂气霜要财多,便可为贵。假如在年、时等柱中有符合其他格局的,则当以其他格局来论。

〔入格八字〕 葛待诏\par
\setlength{\tabcolsep}{0em} % 表格内容水平padding
\begin{tabular}{cm{2em}<{\raggedleft}cl} %
(年)&\scriptsize{}&庚寅&\scriptsize{}\\
(月)&\scriptsize{}&丙戌&\scriptsize{丁火为印}\\
(日)&\scriptsize{}&戊子&\scriptsize{}\\
(时)&\scriptsize{}&癸丑&\scriptsize{}\\
\end{tabular}\\
这种格局,忌行财运官运。八字的主人葛待诏,早先原是卖玳瑁梳子的,只因杂气中月令透出丙火,月支藏有丁火为印,所以行运一旦戌库冲破,就发迹了。可是毕竟由于日支子为癸水,属于戊土的财,而时支丑里又含有一定量的癸水为财,这样财能破印,水去克火,平时尚可维持过去,然而一旦行入子运,运中癸水和命中癸水一起呼应起来,那就泛滥成灾,火光灭没了。后来果然当行到子运时,这位葛待诏就寿终正寝了。这用算命的术语来说,就是“贪财坏印”。诗曰:
\begin{tightcenter}
辰戌丑未为四季,印绶财官居杂气。\\
干头透出格为真,只问财多为尊贵。
\end{tightcenter}

\circledNum{杂气财官格}

命书逢辰、戌、丑、未月出生的,有杂气之称。大抵杂气要财多透露为贵,逢官也好。因为辰、戌、丑、未属于墓库,需要冲开,这样库中的财官印绶才能为我所用,否则官墓不显其名,财库不用于世,印墓不得为信,不就形同虚设?

〔入格八字〕 王尚书\par
\setlength{\tabcolsep}{0em} % 表格内容水平padding
\begin{tabular}{cm{2em}<{\raggedleft}cl} %
(年)&\scriptsize{正财}&戊子&\scriptsize{}\\
(月)&\scriptsize{}&壬戌&\scriptsize{辛金为官 戊土为财}\\
(日)&\scriptsize{}&乙亥&\scriptsize{}\\
(时)&\scriptsize{}&丁丑&\scriptsize{}\\
\end{tabular}\\
自身乙亥,生于戌月,戌中辛金为官,戊土为财,而其中戊土又透出年干,所以就成了杂气财官的格局。诗曰:
\begin{tightcenter}
杂气财官四库中,还须破害与刑冲。\\
天干透出财源格,财多身旺禄相同。
\end{tightcenter}

\circledNum{羊刃比肩格}

所谓比肩,就是同类中阳见阳,阴见阴的称谓,好比兄弟姐妹的同类一样。同类中阳见阴则不称比肩而称败财,又称羊刃,阴见阳则不称败财而称劫财。八字中如果“印财身强见者,能夺伤官七煞,身弱见者,劫财分官见剥。”


〔入格八字〕 髙太尉\par
\setlength{\tabcolsep}{0em} % 表格内容水平padding
\begin{tabular}{cm{2em}<{\raggedleft}cl} %
(年)&\scriptsize{}&庚午&\scriptsize{丁火}\\
(月)&\scriptsize{}&乙酉&\scriptsize{正官}\\
(日)&\scriptsize{}&甲寅&\scriptsize{}\\
(时)&\scriptsize{}&乙亥&\scriptsize{长生}\\
\end{tabular}\\
本命天干甲木,生于八月,以西中辛金为正官。然而年干出现庚金为七煞,这种官煞相混,就不妙了。好在乙庚合金,甲木把妹妹乙木嫁给庚金为妻,命书中有“贪合忘煞”的说法,况且年支中又有丁火制服庚金,不致为灾。再看时支透出乙木,作为甲木的羊刃,而时支亥中癸水,又使甲木处于长生状态,所以行运一旦进入丑运,丑中辛金即抑乙木,又使自身甲木官运亨通,所以官至二品。诗曰:
\begin{tightcenter}
春木夏火两相逢,秋金冬水一般同。\\
不宜羊刃天干透,运至重逢又反凶。
\end{tightcenter}

\circledNum{七煞羊刃格}

所谓七煞,就是偏官,喜制伏,喜羊刃。如命局中七煞、羊刃同时出现的,往往可以把它看作这种格局,但忌财多,否则便不成格局了。对于七煞羊刃格的人来说,最怕羊刃逢冲,替如丙日、戊日生人羊刃在午,因为午中丁火、己土分别屑于日干丙、戊的羊刃,这时如果行运进入正财子地,子午相冲,破了羊刃,就不妙了。同样,壬日生人羊刃在子,忌行午地正财的运,庚日生人羊刃在酉,忌行卯地正财的运,甲日生人羊刃在卯,忌行酉地正财的运。如果格局中,羊刃不被冲破,那末碰上财运,问题不大。

〔入格八字〕 不花平章\par
\setlength{\tabcolsep}{0em} % 表格内容水平padding
\begin{tabular}{cm{2em}<{\raggedleft}cl} %
(年)&\scriptsize{}&乙卯&\scriptsize{}\\
(月)&\scriptsize{}&戊子&\scriptsize{羊刃}\\
(日)&\scriptsize{}&壬戌&\scriptsize{七煞}\\
(时)&\scriptsize{}&壬寅&\scriptsize{}\\
\end{tabular}\\
局中命主生于壬日,月柱日柱分坐子、戌两支,子中癸水为壬水的羊刃,戌中戊土为壬水的七煞。壬水生于仲冬子月,得令身旺,七煞被时支寅中甲木所制,有制为吉。这样身强煞浅,七煞羊刃成格,所以是极贵的命。

\circledNum{金神格}

金神只有三时,就是癸酉、己巳、乙丑,凡是四柱中时柱上出现这三个时辰的,就被认为是金神格。但也有认为,要逢六甲日出生的,才可入这格局,其中甲子、甲辰更好。金神原是破败之神,凡入这格局的,四柱中要火制伏为贵,或行运进入火乡也好。如果运入水乡,水泄金气,大祸就临头。

〔入格八字〕 岳武穆\par
\setlength{\tabcolsep}{0em} % 表格内容水平padding
\begin{tabular}{cm{2em}<{\raggedleft}cl} %
(年)&\scriptsize{}&癸未&\scriptsize{}\\
(月)&\scriptsize{}&己卯&\scriptsize{}\\
(日)&\scriptsize{}&甲子&\scriptsize{}\\
(时)&\scriptsize{}&己巳&\scriptsize{金神}\\
\end{tabular}\\
《星平会海》说:“甲、己为平头煞,生逢春月,身旺财弱,主骨肉参商,平生做事,弄巧成拙。己巳金神有火制伏,巳酉丑合局,运行南方,名重禄高。柱不见火,残害化气,主凶恶暴亡。”“甲子日,己巳时,先贫后富,祖业轻微,妻勤子拗,诗说:
\begin{tightcenter}
癸酉己巳并乙丑,时上逢之是福神。\\
傲物恃才宜制伏,交逢刃煞贵人真。\\
性多狠暴才明敏,遇水相生立穷困。\\
制伏运行逢火局,超迁贵显富无伦。
\end{tightcenter}

\circledNum{魁罡格}

魁罡有四,就是庚辰、壬辰、戊戌、庚戌,其中辰是水库,属天罡,戌是火库,属地罡,辰戌相见,所以成了一种天冲地击之煞。凡是命造中日柱逢庚辰、壬辰、戊戌、庚戌的,就属于这一魁罡格局。《三命通会》说:“经云:魁罡聚众(四柱中出现魁罡的不只日柱一处),发福非常。主为人性格聪明,文章振发,临事果断,秉权好杀。”如果“运行身旺,发福百端,一见财官,祸患立至”。《子平总论》说:“身值天罡地魁,衰则彻骨贫寒,强则绝伦贵显。”然而对于这种格局,也需活看,比如《三命通会》举例张时佥事的八字是庚午、丁亥、戊戌、丙辰,刘大受少卿的八字是丁亥、癸丑、庚戌、戊寅,虽说出生的一天都是魁罡日,可是却不忌财官印,就是明证。诗曰:
\begin{tightcenter}
壬辰庚戌与庚辰,戊戌魁罡四座神。\\
不见财官刑煞併,身行旺地贵天伦。 
\end{tightcenter}

\circledNum{日德格}

入这一格局的只有阳干五天,就是甲寅、丙辰、戊辰、庚辰、壬戌。其中甲坐寅得禄,丙坐辰官库,庚坐辰财印两全,壬坐戌财官印具备,并且地支寅为三阳之首,辰、戌为魁罡之地,所以这五天的干支,就很有点和其他日子的干支不同了。八字中出现日德的,不厌其多,如果只有日柱一位日德的,那末取格时就要按照月柱中财官印食,另作别论了。平时日德除庚辰自兼魁罡二职外,不管命中还是大运,最忌和魁罡同时出现,否则便认为是很不好的命运。

〔入格八字〕 张学官\par
\setlength{\tabcolsep}{0em} % 表格内容水平padding
\begin{tabular}{cm{2em}<{\raggedleft}cl} %
(年)&\scriptsize{}&甲申&\scriptsize{}\\
(月)&\scriptsize{}&戊辰&\scriptsize{}\\
(日)&\scriptsize{}&戊辰&\scriptsize{}\\
(时)&\scriptsize{}&壬戌&\scriptsize{}\\
\end{tabular}\\
命造中有三位日德,由学官而“腰金衣紫”,得五品官诰,很是不错。又如有庚辰、己卯、戊辰、甲寅这样一命,按理有三位日德,该是好命。然而甲寅忌见身兼魁罡的庚辰,后来运行壬午财乡之地,午中己土为日干戊土的羊刃,犯了日德的忌讳,到丁巳年时,寅巳相刑,四月死,寿只三十八岁。这是《三命通会》所记载的。诗说:
\begin{tightcenter}
日德喜煞喜身强,不喜财星官旺乡,\\
为性温柔更慈善,一生福寿乐非常。\\
日德不喜见魁罡,化成煞曜最难当。\\
局中重见还须疾,运限逢之必定亡。
\end{tightcenter}

\circledNum{日贵格}
命造中出生在丁酉、丁亥、癸巳、癸卯四天的人,因为日干坐在天乙贵人星上,所以便把格局称为“日贵”。其中又有日贵、夜贵之分:丁亥、癸卯日生的,生时要在白天,叫做“日贵”,又叫“昼格”,丁酉、癸巳日生的,生时要在黑夜,叫做“夜贵”,又叫“夜格”。《三命通会》说:“经云:贵人者,慈祥恺悌之号,德星尊重之命。遇财官印食则吉,值煞刃冲刑则凶。运遇魁罡,为害不浅。”对于日贵格的命,八字中如果聚上两三位贵人的,主人存粹仁义,贵不可言,怕就怕地支贵人逢冲受损,又怕日上空亡和魁罡加临,这样非但不贵,反而贫贱而夭了。诗说:
\begin{tightcenter}
丁遇猪鸡癸兔蛇,刑冲破害漫咨嗟。\\
财临会合方成贵,昼夜分之最为佳。
\end{tightcenter}

\circledNum{建禄格}

八字中自身日柱天干五行,与月建配合起来正好处于临官禄地,比如甲乙春生,丙丁夏旺,庚辛秋锐,壬癸冬长,以及己生于巳午月等就是。按照五行寄生十二宫的说法,甲禄(临官)在寅,乙禄在卯,丙禄在已,丁禄在午,氏禄在已,己禄在午,庚禄在申,辛禄在西,壬禄在亥,癸禄在子。因此如果日干甲木,月支在寅,年支如果没有其他大的破坏,就可认为入了建禄的格。

〔入格八字〕 贺丞相\par
\setlength{\tabcolsep}{0em} % 表格内容水平padding
\begin{tabular}{cm{2em}<{\raggedleft}cl} %
(年)&\scriptsize{}&辛丑&\scriptsize{}\\
(月)&\scriptsize{}&庚寅&\scriptsize{甲禄在寅}\\
(日)&\scriptsize{}&甲辰&\scriptsize{}\\
(时)&\scriptsize{}&乙亥&\scriptsize{}\\
\end{tabular}\\
八字中日干甲,月建在寅,寅为甲禄。月干和年干庚、辛金虽然分别为甲木的七煞和正官,官煞混见,可是当大运进入丙戌、丁亥制煞之乡时,那就大富大贵了。诗曰:
\begin{tightcenter}
建禄生提月,财官喜透天。\\
不宜身再旺,官地是良缘。
\end{tightcenter}

\circledNum{归禄格}

这种格和建禄格不同的是,建禄格看日干和月支的配合是有禄地,而归禄格则把日干的禄归到时支中去找,如果时支中有禄,而四柱又没有官星七煞的,可认为入了归禄的格。经云:“日禄归时没官星,号曰青云得路。”

〔入格八字〕 林枢密\par
\setlength{\tabcolsep}{0em} % 表格内容水平padding
\begin{tabular}{cm{2em}<{\raggedleft}cl} %
(年)&\scriptsize{财}&戊子&\scriptsize{}\\
(月)&\scriptsize{}&甲寅&\scriptsize{}\\
(日)&\scriptsize{}&乙亥&\scriptsize{}\\
(时)&\scriptsize{}&己卯&\scriptsize{归禄}\\
\end{tabular}\\
此格八字中四柱没有一点官星,所以富贵入格。但如果在行运中遇到官星,就凶而不吉了。诗曰:
\begin{tightcenter}
日禄归时格最良,怕官嫌煞喜自强。\\
若见比肩分劫禄,刑冲破害更难当。
\end{tightcenter}

\circledNum{壬骑龙背格}

这一格局以壬辰日出生为主,四柱又多见壬辰、壬寅,其中辰字多的贵,寅字多的富,如果纯见寅、辰两支,而没有别的地支掺杂进来,那就宫资双全了。经云阳水4逢辰位,是壬骑龙背之乡。”这种格大忌官星盛旺,若见戌和辰冲,也不为福。

〔入格八字〕 王枢密\par
\setlength{\tabcolsep}{0em} % 表格内容水平padding
\begin{tabular}{cm{2em}<{\raggedleft}cl} %
(年)&\scriptsize{}&壬辰&\scriptsize{}\\
(月)&\scriptsize{}&甲辰&\scriptsize{}\\
(日)&\scriptsize{}&壬辰&\scriptsize{}\\
(时)&\scriptsize{}&壬寅&\scriptsize{为壬财官}\\
\end{tabular}\par
王巨富\par
\setlength{\tabcolsep}{0em} % 表格内容水平padding
\begin{tabular}{cm{2em}<{\raggedleft}cl} %
(年)&\scriptsize{}&壬寅&\scriptsize{}\\
(月)&\scriptsize{}&壬寅&\scriptsize{}\\
(日)&\scriptsize{}&壬辰&\scriptsize{}\\
(时)&\scriptsize{}&壬寅&\scriptsize{遍地是财,以致巨富}\\
\end{tabular}\\
以上两个八字,前一个辰多,所以贵过于富,后一个寅多,所以富过于贵,都是很典型的。诗曰:
\begin{tightcenter}
壬骑龙背怕官居,重叠逢辰贵有余。\\
假若寅多辰字少,须应豪富比陶朱。
\end{tightcenter}

\circledNum{六乙鼠贵格}

此格以六乙日生,时间逢子,而四柱中又没有官星,方才官髙名显。在神煞中,乙见子为贵人,而十二生肖又以子属鼠,所以有“六乙鼠贵”之名。由于此格子为乙的贵人,所以大忌逢冲。若果八字或大运中子午相冲,那就通盘都坏了。

\setlength{\tabcolsep}{0em} % 表格内容水平padding
\begin{tabular}{lm{2em}<{\raggedleft}cl} %
\multicolumn{3}{l}{〔入格八字〕}&曹尚书\\
(年)&\scriptsize{}&丁巳&\scriptsize{}\\
(月)&\scriptsize{}&壬寅&\scriptsize{}\\
(日)&\scriptsize{}&己卯&\scriptsize{}\\
(时)&\scriptsize{}&丙子&\scriptsize{贵人}\\
\end{tabular}\\
《喜忌篇》说:“阴木独遇子时,为六乙鼠贵之地。”诗曰:
\begin{tightcenter}
乙日生人得子时,名为鼠贵最为奇。\\
切嫌午宇来冲破,辛酉庚申总不宜。
\end{tightcenter}

\circledNum{六甲趋乾格}

此格为六甲日生,时间逢亥。亥宫属乾,为甲木的长生之地,所以有“六甲趋乾”的名称。入此格的,四柱和岁运中不喜财星,同时又忌寅、巳两字。因为甲财属土,能制亥水,寅与亥合,巳与亥冲,都不佳妙。

\setlength{\tabcolsep}{0em} % 表格内容水平padding
\begin{tabular}{lm{2em}<{\raggedleft}cl} %
\multicolumn{3}{l}{〔入格八字〕}&新安伯\\
(年)&\scriptsize{}&戊辰&\scriptsize{}\\
(月)&\scriptsize{}&癸亥&\scriptsize{}\\
(日)&\scriptsize{}&甲子&\scriptsize{}\\
(时)&\scriptsize{}&乙亥&\scriptsize{亥为乾}\\
\end{tabular}\\
大凡逢甲日出生的,亥支不厌其多,又无巳来相冲,这样就自然富贵了。诗曰:
\begin{tightcenter}
趋乾六甲最为奇,甲日生人得亥时。\\
岁运若逢财旺地,官灾患难祸相随。
\end{tightcenter}

\circledNum{六壬趋艮格}

入这个格的,要以六壬日生寅时为准,格中寅字多的,又叫做合禄格。在八字和岁运中,最怕逢申相冲,又忌财官。

\setlength{\tabcolsep}{0em} % 表格内容水平padding
\begin{tabular}{lm{2em}<{\raggedleft}cl} %
\multicolumn{3}{l}{〔入格八字〕}&\\
(年)&\scriptsize{}&壬寅&\scriptsize{}\\
(月)&\scriptsize{}&壬寅&\scriptsize{}\\
(日)&\scriptsize{}&壬寅&\scriptsize{}\\
(时)&\scriptsize{}&壬寅&\scriptsize{}\\
\end{tabular}\\
寅为艮宫,所以有“趋艮”的叫法。寅宫甲木能合己土,丙火能合辛金,这样就暗遨己土为壬水的正官,辛金为壬水的印绶了。行运不厌身旺之地,如遇申字相冲,那就大打折扣了。诗曰:
\begin{tightcenter}
六寅趋艮喜非常,壬日寅时是贵乡。\\
大怕刑冲并克制,逢申岁运有灾殃。
\end{tightcenter}

\circledNum{勾陈得位格}

勾陈在五行中属戊己土。在六戊日和六己日中,遇上财官的有戊寅、戊子、戊申、己卯、己亥、己未等六日其中以申子辰水局为财,亥卯未木局为官。入这种格局的,最怕刑冲煞旺,反生灾害。

\setlength{\tabcolsep}{0em} % 表格内容水平padding
\begin{tabular}{lm{2em}<{\raggedleft}cl} %
\multicolumn{3}{l}{〔入格八字〕}&丁都督\\
(年)&\scriptsize{}&丁亥&\scriptsize{}\\
(月)&\scriptsize{}&丁未&\scriptsize{}\\
(日)&\scriptsize{}&己卯&\scriptsize{}\\
(时)&\scriptsize{}&戊辰&\scriptsize{}\\
\end{tabular}\\
八字中以己卯日为勾陈,遇亥卯未木局为官星得地,所以是个贵命。诗曰:
\begin{tightcenter}
日干戊己坐财官,号曰勾陈得位看,\\
知有大才分瑞气,命中值此列朝班。
\end{tightcenter}

\circledNum{玄武当权格}

玄武在五行中属壬癸水。在六壬日和六癸日中,遇上财官的有壬寅、壬午、壬戌、癸未、癸丑、癸巳六日,其中以寅午戌火局为财,辰戌丑未土局为官。入这种格局的,在八字和岁运中最忌身弱冲破。

\setlength{\tabcolsep}{0em} % 表格内容水平padding
\begin{tabular}{lm{2em}<{\raggedleft}cl} %
\multicolumn{3}{l}{〔入格八字〕}&季都司\\
(年)&\scriptsize{}&庚戌&\scriptsize{}\\
(月)&\scriptsize{}&壬午&\scriptsize{}\\
(日)&\scriptsize{}&壬寅&\scriptsize{}\\
(时)&\scriptsize{}&辛亥&\scriptsize{}\\
\end{tabular}\\
格中地支寅午戌火局为财,水火既济,所以富贵荣华。诗曰:
\begin{tightcenter}
壬癸名为玄武神,财官两见始成真。\\
局无冲破当清贵,辅佐皇家一老臣。
\end{tightcenter}

\circledNum{稼穑格}

稼穑在五行中属中央戊己土。凡是入此格的,不仅日干要逢戊己土,并且还要地支辰戌丑未全是土局,加上无木克削,有水为用,自然便就福禄绵绵了。

\setlength{\tabcolsep}{0em} % 表格内容水平padding
\begin{tabular}{lm{2em}<{\raggedleft}cl} %
\multicolumn{3}{l}{〔入格八字〕}&张真人\\
(年)&\scriptsize{}&戊戌&\scriptsize{}\\
(月)&\scriptsize{}&己未&\scriptsize{}\\
(日)&\scriptsize{}&戊辱&\scriptsize{水}\\
(时)&\scriptsize{}&癸丑&\scriptsize{水}\\
\end{tabular}\\
此命地支辰戌丑未俱全,得水为财,又无木克,所以有福。诗曰:
\begin{tightcenter}
戊己重逢杂气天,土多只论木居全。\\
财星得遇堪为福,官煞如临有祸缠。
\end{tightcenter}

\circledNum{曲直格}

曲直在五行中属于东方甲乙木。凡是入此格的,不仅日干要逢甲乙木,并且地支还要会成亥卯未木局,或寅卯辰全,加上无金克削,有水为印,命主仁而福寿。

\setlength{\tabcolsep}{0em} % 表格内容水平padding
\begin{tabular}{lm{2em}<{\raggedleft}cl} %
\multicolumn{3}{l}{〔入格八字〕}&李总兵\\
(年)&\scriptsize{}&甲寅&\scriptsize{}\\
(月)&\scriptsize{}&丁卯&\scriptsize{}\\
(日)&\scriptsize{}&乙未&\scriptsize{}\\
(时)&\scriptsize{}&丙子&\scriptsize{}\\
\end{tabular}\\
此格日干乙木与地支寅卯未会成木局,加上时支癸水生木,又无官煞相侵,所以盛而为官。诗曰:
\begin{tightcenter}
甲乙生人寅卯辰,又名仁寿两堪评。\\
亥卯未全嫌白帝,若逢坎地必荣身。
\end{tightcenter}
诗中“嫌白帝”是说嫌弃庚辛金气,因为金属西方白帝。坎地是指水地,在八卦中坎属于水,所以古人常用坎指代水。

\circledNum{炎上格}

炎上在五行中属于南方丙丁火。凡是入这格局的,不但日干要遇丙丁火,并且还要地支会成寅午戌火局,或巳午未全,加上身旺,运行东南,便就浑成文明,夫富大贵了。

\setlength{\tabcolsep}{0em} % 表格内容水平padding
\begin{tabular}{lm{2em}<{\raggedleft}cl} %
\multicolumn{3}{l}{〔入格八字〕}&张太保\\
(年)&\scriptsize{}&乙未&\scriptsize{火}\\
(月)&\scriptsize{}&辛巳&\scriptsize{火}\\
(日)&\scriptsize{}&丙午&\scriptsize{火}\\
(时)&\scriptsize{}&甲午&\scriptsize{火}\\
\end{tabular}\\
局中自身丙火逢地支巳午未火局,一片炎上之性,故为朝中朱紫之贵。诗曰:
\begin{tightcenter}
火多炎上气冲天,玄武无侵富贵全。\\
一路东方行运好,簪缨头顶带腰悬。
\end{tightcenter}

\circledNum{润下格}

润下在五行中属于北方壬癸水。凡是入这格局的,不但日干要遇壬癸水,并且还要地支会成申子辰水局,或亥子丑全。平生忌辰戌丑未官乡,喜西方印地,不宜东南,怕冲克。

\setlength{\tabcolsep}{0em} % 表格内容水平padding
\begin{tabular}{lm{2em}<{\raggedleft}cl} %
\multicolumn{3}{l}{〔入格八字〕}&万宗人\\
(年)&\scriptsize{}&庚子&\scriptsize{水}\\
(月)&\scriptsize{}&庚辰&\scriptsize{水}\\
(日)&\scriptsize{}&壬申&\scriptsize{水}\\
(时)&\scriptsize{}&辛亥&\scriptsize{水}\\
\end{tabular}\\
这命非但地支申子辰全,子亥水局浑然,并且年、月、时柱又得庚辛生水,所以一片湛然,福量广阔,为富贵之人。诗曰:
\begin{tightcenter}
天干壬癸喜冬生,更值申辰会成局,\\
或是全归亥子丑,等闲平步上青云。
\end{tightcenter}

\circledNum{从革格}

从革在五行中属于西方庚辛金。凡是入这格局的,不但日干要遇庚辛金,并且还要地支会成已酉丑全局,或申酉戌全。平生忌南方火运,冲刑克破,喜庚辛旺运。

\setlength{\tabcolsep}{0em} % 表格内容水平padding
\begin{tabular}{lm{2em}<{\raggedleft}cl} %
\multicolumn{3}{l}{〔入格八字〕}&杨太尉\\
(年)&\scriptsize{}&辛酉&\scriptsize{金}\\
(月)&\scriptsize{}&戊戌&\scriptsize{金}\\
(日)&\scriptsize{}&庚申&\scriptsize{金}\\
(时)&\scriptsize{}&辛巳&\scriptsize{金}\\
\end{tabular}\\
命中自身庚金,地支申酉戌全,月干又透出戊土生金,所以福高禄深。诗曰:
\begin{tightcenter}
秋月金居一类看,名为从革便相欢,\\
如无炎帝来临害,定作当朝宰辅官。
\end{tightcenter}

\circledNum{弃命从财格}

命中日干身弱,四柱中又全无印绶相生,比肩扶助,而天干地支却透财会财,造成财旺身弱的局势,这时就索性丢弃自身专以财论,所以叫弃命从财格。此格喜行财旺运,怕入煞印之乡。比如天干乙木而地支辰戌丑未土全,财神极旺,这时如果四柱无依,就只有以弃命从财格论。碰上这种格的,主人平生不是怕老婆,就是做人家的招女婿。因为财者妻也,自身既无依靠,托的全是妻子的福,所以作这样的分析。

\setlength{\tabcolsep}{0em} % 表格内容水平padding
\begin{tabular}{lm{2em}<{\raggedleft}cl} %
\multicolumn{3}{l}{〔入格八字〕}&王十万\\
(年)&\scriptsize{}&庚申&\scriptsize{财}\\
(月)&\scriptsize{}&乙酉&\scriptsize{财}\\
(日)&\scriptsize{}&丙申&\scriptsize{财}\\
(时)&\scriptsize{}&己丑&\scriptsize{财}\\
\end{tabular}\\
命中财多身弱,自身少有所助,所以只有舍命从之,才能有搞。诗曰:
\begin{tightcenter}
日主无根财犯重,全凭时印旺身官。\\
逢生必主兴家业,破印纷纷总是空。   
\end{tightcenter}

\circledNum{贪合忘煞格}

四柱中财官两旺,这时如果柱中透出的煞被合去,就叫贪合忘煞。入这种格的,虽然衣禄丰厚,可却无官而好酒色。比如甲日柱中逢庚,庚就是甲的七煞,这时桂中如又透出乙木和庚相合,便就称作贪合忘煞。又如甲日柱中逢辛,辛是甲的正官,这时柱中如又透出丙火和辛相合,便也称作贪合忘官。其中煞要合,官不要合,所以有“合煞不为凶,合官真不美”的说法,又说:“煞无刃不威,刃无煞不显。”这里的刃,就是羊刃,也就是败财,可见刃煞一起出现,也并不是坏事。这又有点岔到别处去了。


\setlength{\tabcolsep}{0em} % 表格内容水平padding
\begin{tabular}{lm{2em}<{\raggedleft}cl} %
\multicolumn{3}{l}{〔入格八字〕}&王指挥\\
(年)&\scriptsize{}&丙申&\scriptsize{}\\
(月)&\scriptsize{合官}&辛丑&\scriptsize{}\\
(日)&\scriptsize{}&甲辰&\scriptsize{}\\
(时)&\scriptsize{财}&戊辰&\scriptsize{}\\
\end{tabular}\\
这命生于甲日,月柱中有官星照临,而戊辰时又财旺生官,本该大贵,可是偏偏年干丙火与官星辛金相合,命书说:“贪合忘官为颠邪。”现在官星既被合掉,只好作贱命看了。加之四十五岁入丙午运,火势太炎,申木干燥,木被火焚,则不禄矣。诗曰:
\begin{tightcenter}
贪合忘官合不足,合煞不伤为己福。\\
堪叹身弱怕逢败,更历官乡祸自逐。
\end{tightcenter}

\circledNum{干辰一字}

入此格的,年、月、日、时四柱,天干清一色而不杂,所以清贵。

\setlength{\tabcolsep}{0.5em} % 表格内容水平padding
\begin{tabular}{lccccc} %
\multicolumn{3}{l}{〔入格八字〕}&\\
(年)&壬子&戊辰&甲子&庚申&丙寅\\
(月)&壬子&戊午&甲戌&庚辰&丙申\\
(日)&壬子&戊申&甲寅&庚戌&丙午\\
(时)&壬寅&戊午&甲子&庚辰&丙申\\
\end{tabular}

\circledNum{支辰一字}

这种格局,年、月、日、时地支一字不杂,一色纯清,也是一种贵命。

〔入格八字〕\par		
甲寅\qquad{}戊辰\qquad{}乙亥\par
丙寅\qquad{}丙辰\qquad{}丁亥\par
庚寅\qquad{}甲辰\qquad{}己亥\par
戊寅\qquad{}戊辰\qquad{}乙亥\par

根据宋代吴曾《能改斋漫录》记载,宋代宰相曾布的八字为乙亥、丁亥、辛亥、己亥,此外宰相萧注的八字则为癸丑、乙丑、乙丑、丁丑,这两人显然都是属于这一格局的贵人。

\circledNum{两干不杂}

这种格局,四柱天干只有两个天干,纯一不杂,所以取名为“两干不杂”。

〔入格八字〕\par
甲子\par
乙亥\par
甲戌\par
乙丑

这一命禄,局中天干只有甲、乙两字,不杂不乱。此外如丙寅、丁酉、丙辰、丁寅,局中天干也只有丙、丁两字,纯一不杂,也明显地属于这一格局。赋云:“干头相类,铜臭官卑。”因为甲日生人逢乙,乙日生人逢甲,命书叫做“偏禄”,多半没有科名可言。

\circledNum{天元一气}
这种格局天干一色纯清,地支也一色纯清,所以除了行运地支被冲,大抵多贵人命。

〔入格八字〕 张贵妃\par
乙酉\par
乙酉\par
乙酉\par
乙酉

此外,庚辰或壬寅逢天元一气,位至三公,大富大贵。然而天元如果都是辛卯的话,那就财多身弱,反而是个身轻福浅的命了。

以上入格八字举例,命书尚有子遥巳禄、卯未遥巳、刑冲带合、刑合得禄、拱禄拱贵、冲合禄马、虎午奔巳、羊击猪蛇、财官双美、福德秀气、青龙伏形、白虎持势、朱雀乘风、还魂借气、金白水清、木火交辉等格,名目不下百余种之多。

由于八字的格,少说也有百种之多,所以纵横开合,千变万化,有的甚至还玄妙百出,使人眼花缴乱,莫测高深。然而在大多数情况下,命理学家论定八字吉凶,还是以八种正格为主,从自身日干五行出发,结合八字和岁运中五行盛衰喜忌进行总体分析,只是在一些较为特殊情况下,对于少数几种明显入变格的,才以格论。

\section{关于女命的看法}

中国哲学中的阴阳学说,认为女人宠天地的阴柔之气,男人釆天地的阳刚之气,所以把女人说成属阴,男人说成属阳。阴和阳是世间一切事物矛盾对立统一中两个截然不同的面,这种思想反映在命理中是,非但在起运的岁数和大运的推排上,女性和男性有着截然相反的不同,并且在具体的算法上,两者也有着它明显相异的地方。

在书本前面的有关章节里我们得知,男性八字取我克的正财或偏财为妻子,可是女性八字中的丈夫,就要来个彻底的相反,取克我的官(正官)煞(偏官)为丈夫了。同样,在子女的看法上,男命取克我的偏官(七煞)为儿子,正官为女儿,而女命则取我生的食神为儿子,伤官为女儿。

由于封建社会妇人一切都要依赖于丈夫,“夫利其妇必利,夫困其妇必闲”,所以看女命的好坏,先要看夫星官煞的位置和盛衰,以定贵贱,接着再看子星,因为养儿防老,做妇人家的本身没有收入,因此晚年的荣辱,就全押在子星的好坏上了。

在通常情况下,官、煞、财得地,有利于丈夫,食神得地,有利于孩子。丈夫利则出身富贵,一生享福;孩子利则晚年养厚,褒宠诰封。由于食神能够生财,财又能够生官。比如有位夫人八字日干乙木,乙木所生的食神是丁火,然后再由食神丁火生土,木能克土,所以土是乙木的财,接着又由土生出金来,金是克乙木的官,为了这个缘故,所以女命多取食神、财官作为八字的用神。如果八字中官煞财食生得既不得地,又不生旺,或者竟告缺如,行运时又没能够补上的,那就一生困苦潦倒,不用说了。

又如封建礼教崇尚妇人贞洁,从一而终,所以八字中如果见官的就不要见煞,如果见煞的也不要见官,总以一位为好。如果一旦八字中有两位官星,只要没有煞混在里面的,或者四柱里纯是煞,却没有混进官星的,也都是值得称羡的良家妇女。否则官煞相混,就被认为是个喜欢偷情而有外遇的淫荡女子。
在明代育吾山人所著《三命通会》里,曾详细论述了女命的八法。现不妨阐释如下:

\circledNum{纯}所谓纯,就是纯一的意思。比如官星纯一,煞星纯一,有财(财能生宫)有印(印绶护身),又没遇上刑冲,这就是纯。我们且看下面的一个女命八字:\par
(年)癸巳\par
(月)戊午\par
(日)辛酉\par
(时)丙申\par
八字中辛酉为自身,而酉对辛来说,由于正处在临官的禄地,所以自身生旺。古话说:“旺不从化。”按理天干合局,丙辛应该化水,现在本身专禄,也就化而不化了。这里,辛金的夫星是克我的正官丙火,联系该命出生的戊午月,正值农历五月火旺之时,所以夫星健旺。再联系年干癸水,恰恰与夫星丙火形成正官关系。在用神中,正官是个吉星,所以也对丈夫十分有利。若再联系月干戊土,又是夫星丙火的吉神食神,并且丙火和戊土,又一起归禄(临官)到年柱的地支巳里,可谓难得。夫星看后再看子星。辛金生壬水为子,而位居时支子息宫的申里,恰好涵有壬水,而这壬水和申的关系,在十二宫中又正好处在万物发生向荣的长生之地。加之天干癸戊合火,丙辛合水,水火有既济之象;地支巳午酉申,巳里庚金,申里庚金,酉里辛金,都是夫星丙火和月支午里丁火的财库,所以也就自然嫁夫为官而食天禄,属于夫荣子贵的命了。

\circledNum{和}所谓和,就是恬静的意思。比如八字中自身柔弱,只有一位克我的夫星,而四柱又没有攻破冲击的神,这就是禀其中和之气的“和”了。让我们看这样一个女命八字:\par
(年)壬辰\par
(月)辛亥\par
(日)己卯\par
(时)己巳\par
命中日柱天干己土为自身,月柱亥中甲木为夫星。亥对甲木来说,处在万物发生向荣的长生之地,既得天时,又得地利。甲木以辛为正官,现在月逢辛亥,对甲正属有利。己以金为子,时支巳中庚金虽为伤官,但是也可活看,况且巳对于庚金来说,也处在万物发生向荣的长生之地。以上这些,就叫做夫得官星,子得长生,所以主旺夫益子。至于日柱下卯中乙木,虽为自身己土的七煞,然而有时支巳中庚金为制,所以“去煞留官”,为女命中的贵象。

\circledNum{清}所谓清,就是洁净的意思。女命中或者只有一官,或者只有一煞,不相混杂,这就算是清了。总要夫星得时,柱中有财有官,有印助身,没有一点混浊之气,方才来得清贵。比如这样一个女命:\par
(年)己未\par
(月)壬申\par
(日)乙未\par
(时)甲申\par
自身日柱乙木,以月支时支申中所含庚金为夫星。申对于庚来说,处于临官的禄地,所以夫星得时,又乙木以我生的食神丁火为子星,而自身日支的未中正好含有丁火,而未对于丁来说,也正好处在临官的旺地,所以子星得地。乙木以壬水为正印,而月柱壬水又生于坐下的申金,水源不乏。加之日支未中己土,又为乙木送来偏财。这样财旺生官,四柱又没有刑冲破败。诗云:“财官印绶三般物,女命逢之必旺夫。”所以这妇人有两国之封,夫人之命。

\circledNum{贵}所谓贵,这是尊荣之号。命中有官星,并且得到财气的资生,而四柱中又没有刑冲破败,这就贵为女命中的尧舜了。经云:“无煞(偏官)女人之命,一贵可作夫人。”又说:“女命无煞逢二德,可二国之封。”所谓二德,不单只指天德,月德(见本书《八字中的神煞》篇),对于女命来说,财也是德,官也是德,如果再有印绶、食神,那就更加尊贵了。这里有这样一个女命:\par
(年)乙亥\par
(月)丙戌\par
(日)辛卯\par
(时)癸巳\par
自身日柱天干辛金,不仅以我克的年干乙木为偏财,先获一德,并且又以月干中克我辛金的丙火为官人,而这官人又坐在以万物成功而藏之的戌的墓库上面,并且还把时支中的巳支拉来作为临官禄地,所以又得一德。二德之外,时干癸水贵为夫星丙火的官,自身辛佥滋生癸水为子,而这为水的癸水恰又偏偏坐在临官的巳上,可谐与“夫禄同位”。加之时干癸见日支卯,有天乙贵人之称(见前《八字中的神煞》篇)。这样又是贵人,又是财官双美,所以丈夫和儿子都贵,两遇褒封。

\circledNum{浊}所谓“浊”,就是混而不清。女命八字如出现五行失位,水土互伤,自身太旺,代表丈夫的官星显示不出来,而偏官则一片丛杂,四柱里又没有财官印食,这就都是些下贱村浊,或娼妓婢妾,淫巧的妇人。这里且看这样一个女命:\par
(年)己亥\par
(月)乙亥\par
(日)癸丑\par
(时)己未\par
自身癸水,生于十月亥月,太泛。癸水以戊土为正官,现在正官不显,而引时干己土为偏夫,可是日支丑和时支未中都有作为偏夫的己土混杂在里,加之日柱中没有财,乙木本为癸水的食神,然而乙木生在有力的月干上面,己土受克,这就是五行失位,难免鬼败临身,而主先清后浊,不能享福了。

\circledNum{滥}所谓滥,就是婪的意思。这是说四柱天干里明的有多个夫星,地支里暗的又财旺带煞,这就难免酒色猖狂,私暗得财。碰土此等命的,不是克夫再嫁,就是身为奴婢,因为太过或是不及,从而走向反面。比如这样一命:\par
(年)庚寅\par
(月)丙戌\par
(日)庚申\par
(时)丁亥\par
自身庚金,生于秋月,日支又逢临官禄地,本身自旺。其中月柱重于时柱,理应丙火为夫,可是年支和月支寅戌会成火局,时干之上又透出丁火,难免爱火重重。再如自身庚申金暗克年支月支的寅亥木为财,而亥中壬水又为庚金的吉神食神,食神能够生财。因此这个女人虽说长得美貌有福,然而却又少不了属于滥淫而得财的一路。

\circledNum{娼}所谓娼,就是娼妓。八字中如出现身旺夫绝,官衰食盛,或四柱里不见官煞,或有而被作为凶神的伤官伤尽,或官煞混杂而食神盛旺,这些如果不是娼妓的命,就是尼姑婢妾,克夫淫奔的命,两者必居其一,且看这样一命:\par
(年)丁亥\par
(月)庚戌\par
(日)戊辰\par
(时)庚申\par
自身日干戊土,本应以年支亥中甲木为克我的夫星,可是由于其木正处在九秋的戌月,在旺相休囚死中属于失时无气的死,现在又逢月干庚金监临,所以必定克绝无疑。再看时支申中庚金,理应属于戊土的食神,而申对庚来说又是临官禄地,所以食神有力,加之戊辰原属魁罡星辰,有利男子,不利女子,现在魁罡照临,又能生食,如再结合月干和时干的庚金,就未免使得食神旺过了头。虽说辰里乙木也为克我的夫星,但因位于戊土座下,不能透出,故不能为用。此外年支亥里的壬水,日支辰里的癸水,时支申里的壬水都是自身戊土的财。戊辰本属魁罡,自身强旺,现在克我的夫星既已死绝,而四周又充满一片我生的食神,所以叫做身旺逢生,贪食贪财,是个没有丈夫的秀丽娼妇。

\circledNum{淫}所谓淫,就是淫泆。这种人的八字,自身虽然得地,可是四柱夫星太过,明暗交集,人称日干身旺,四柱中都是官煞的便是。夫星出现在天干中的叫明,出现在地支里的叫暗。比如一丁三壬,或丁火同时碰上天干壬水,地支辰中癸水,子中癸水的,就都是四柱太过或明暗交集的典型。这样的女性对于男人,真是无所不纳的了。比如这样一个坤造:\par
(年)戊辰\par
(月)壬辰\par
(日)壬戌\par
(时)癸亥\par
命中壬戌、癸亥一个处在临官禄地,一个处在万物成熟的帝旺状态,可谓自身得地。可是在夫星上,明有年柱戊土为正夫,暗有二辰一戌所含三个戊土为暗夫,这就属于命书所说的夫星交集,淫不可言了。

除了八法,女命的花样还有旺夫伤子,旺子伤夫,伤夫克子,安静守分,横夭少年,福寿两备,正偏自处,招嫁不定等八格,可谓名目繁多,应接不暇。但是归纳起来,不外古赋所说的:
\begin{yinyong}
若观女命,则异乎男。富贵者一生官旺,纯粹者四柱休囚,浊滥者五行冲旺,娼淫者官煞交差。无官多合,此为不良。满柱煞多,不为克制。印绶多而老无子,伤官旺而幼伤夫。四柱不见夫星,未为贞洁;五行多遇子曜(指食神多),难免荒淫。食神一位逢生旺,招子须当拜圣明;官煞不杂遇印扶,嫁夫定知登云路。守寒房而清洁,金猪木虎(指辛亥、甲寅日)相逢(此二日虽克夫而守正);对空帐而孤眠,土猴火蛇(指戊申、丁巳日)相遇(此二日克夫不正)。财旺生官,辅食无伤,而夫荣子贵;官食禄旺,一印有助,而后宠妃褒。伤官叠见无财印,败室刑夫,官煞重逢遇三合(见前《天干地支的刑、冲、害、化、合》篇),荒淫无耻。合多官重,贪淫好色之人,官杂气衰,嗜欲刑夫之妾。身旺官凶,非师尼而为娼婢;食神变德,先贫贱而后荣华。  
\end{yinyong}


此外,在看女命时还盛行着一种克夫的说法。首先,“凡女命,生日在官、鬼、死、墓、绝上,主克夫”。比如丙戌、庚子等日出生的女命,査前《五行的旺相休囚死和寄生十二宫》篇,丙遇戌正处在人之终而归墓的状态,而庚遇子又处在万物死的状态,因此都主克夫:然而也有认为,辛卯日生的女命,虽然逢上绝地,可却“大美小疵”,这就难以一概而论了。再之,“凡女命,生年生日同一位者,克夫”,“生年生日带六甲者,名曰带甲,主克夫,月共日俱带者亦然”。举例说,如果中午年生的女命,再碰上甲午日的生日,那就少不了十有九个要克丈夫。

有趣的是,命书中还多附有一首推算妇女怀孕,生男还是生女的歌诀。对此,《三命通会》记载说:
\begin{tightcenter}
    七七四十九,问娘何月有,\\
    除却母生年,单奇双是偶,{\scriptsize 奇男偶女}\\
    奇偶若不常,寿命不长久。
\end{tightcenter}

根据歌诀,以49为基数,如果母亲岁数是31岁(虚岁),怀孕的月份是农历正月,那末算时49加1(正月)等于50,50再减掉母亲年龄31等于19,19属于单数,所以生男。如果算出来单数应生男,双数应生女的,而生下来的结果却单数生女,双数生男,和这截然相反,那就寿命不长而夭了。然而使人大惑不解的是,有的命书算法还要在末了再加上十九,这就把《三命通会》所载的推算怀孕男女法给彻底颠倒过来了。	

\section{合婚宜忌}
公元1988年7月17日,《新民晚报》所载周柏春《法国公园相亲》一文的文末,作者写道:“隔天,有人送来了吴小姐的生辰八字,在我家灶上搁置三天。三天中,家里碗未打碎一只,人未跌过一跤。据说这预示着吴小姐将为这个家庭带来好运。”

这里,作者虽然没有进一步谈到男女双方八字的合还是不合,然而这种婚前看女方八字的做法,却是我国民间、娶妻的一个内容。合婚在古代也叫合姓,就是合二姓为婚姻的意思。由于古代结婚娶妻,双方多半没有机会看	到对方本人,更不要说是了解对方的品德操行和性格脾气了。所以,在合婚过程中,除了周柏春文中提到的一环外,更多的是男方必先要请人看看女方八字是旺夫益子,还是伤夫克子?如者是旺夫益子,则男方兴高彩烈,阖第称庆,若是伤夫克子,则男方必定改辕易辙,另起炉灶。在封建社会或旧社会结婚幸与不幸纯凭运气的情况下,这种社会心理是可以理解的。且按下这种做法本身的荒诞无稽不说,举个财说,现在男女相亲,女方送来这样一个八字:\par
(年)丁丑\par
(月)壬寅\par
(日)丁酉\par
(时)己酉\par	
命中日干丁火为女方自身,用月干克我的丰水为丁火夫星,而月支寅中甲木,既为丁火自身的印,又为夫星壬水的吉神食神。再如儿子寄居的时宫,一是丁火生出己土为子,二是夫星壬水得己土为官,三是子星己土得寅中甲木为官,四是丁火克时支酉为财。综合以上分析,必定是个荣夫益子的命,所以男方高兴还来不及,哪有不满口应承下来的?再如女方如果送来这样一个八字:\par
(年)甲辰\par
(月)癸酉\par
(日)丙子\par
(时)辛卯\par
其中以日柱丙子为自身,按照命书说法,女儿家丙子日出生是犯了阴阳煞,秉有挑诱男子的杨花水性。撇开这不说,而自身丙火既有月干癸水克我为夫,可地支辰子会水,又为暗夫。再如日时干支丙辛相合,子卯相刑,地支刑而天干合,命书认为是荒淫滚浪,酒色昏迷的命。加上丙火克酉中辛金财旺,而这财又正处在夫星座下,所以为卖奸得财无疑。这种淫败的女命,对于合婚的男方来说,又怎么接受得了呢?

由于我国封建社会,是个以男子为中心的社会,故而表现在合婚方面,更多的是男方对女方的挑剔。对此,有首古歌说:
\begin{tightcenter}
择妇须沉静,细说与君听。\\
夫星要强健,日干当柔顺。\\
二德坐正财,富贵自然来。\\
四柱带休囚,增命又增寿。\\
贵人一位正,两三做宠娉。\\
金水若相逢,必遭美丽容。\\
四贵一位煞,权家富贵说。\\
财官若藏库,冲开无不富。\\
寅申巳亥全,孤淫腹便便。\\
子午并卯酉,定是随人走。\\
辰戌兼丑未,妇道必大忌。\\
有辰怕见戌,有戌怕见辰。\\
辰戌若相见,多是淫破人。\\
有煞不怕合,无煞却怕合。\\
合神若是多,非妓亦讴歌。\\
羊刃带伤官,驳杂事多端。\\
满盘却是印,损子必须定。\\
天干一字连,孤破祸绵绵。\\
地支连一字,两度成婚事。\\
此是妇命诀,千金莫轻视。\\ 
\end{tightcenter}

话虽这样,然而反过来说,女方挑选丈夫,对于男方送来八字的研究分析,也是从来不肯马虎的。因为嫁鸡随鸡,嫁犬随犬,毕竟是件牵涉到女孩儿家终身幸福的大事,又怎能草率从事呢?在多数情况下,女孩儿家里对于男方八字的要求是五行中和,不偏不倚,认为这样的男人不但一生丰衣足食,并且性格中和,寿命绵长。由于古代提倡女子嫁人,从一而终,如果只考虑到男方的荣华富贵,而不考虑到男方的性情脾气和生命寿夭,那往后的日子又怎么过下去呢?

所以,出于以上种种对嫁娶问题的看和忧虑,命书综括男女合婚的要领是,男家择妇,八字贵看夫子二星,盖夫兴子益,其福必优也。女家择夫,八字贵得中和之气,盖不偏不倚,其寿必长也。”

然而,世间男女的八字毕竟千变万化,数目繁多,而又哪来这么多夫荣子货,八字中和的命呢?对此,男女间八字如果有偏的,合婚时互相补偏救弊,转劣为优的学说便就出笼了。

譬如男命自身的日干木,而八字中比肩、劫财的甲乙木较重,但女方送来八字的自身偏偏是戊己土,按理木克土,丈夫制约妻子,在封建伦理中最天经地义的事班,可是到底因为男方木势太重,难免中途克妻,所以这时就得看看女方的食神庚辛金如何了。如果食神重的,由于金能制木,因此招架得住,可以合婚;如果女方食神不足,只要戊己土多,能够生金的,也无伤大雅,同样可以合婚;只自身衰弱而又无食神庚金可以抵敌的,那就只得彼此说声再见,重新物色对象了。

同样道理,反过来说,如果女命中庚辛金太多,那末找丈夫时最好对方火多,能够克制得住,否则找个木多的也好,因为木多金缺,女方砍伐就费力了。据说,如果按照这种原则配对起来的夫妻,虽然自身八字都有偏胜偏衰,这样那样的不足,可是由于彼此取长补短,取得了动态的平衡,所以还是能够“琴瑟和谐,子嗣蕃衍”的。把这一男女间彼此救弊补偏的合婚原则归纳起来,就是:“男命木盛宜金者,得女命之刚金补之,则为尽美,得土生金者亦佳,得火者较次,得水木则无取矣。如女命刚金喜火者,得男命之烈火助之,则为尽美,得木生火者亦佳,得水者较次,得金土则无取矣。”其他五行偏盛偏衰的,可以照此类推。

此外,在合婚中,还有一种用“骨破牌”、“铁扫带”等凶煞来作为避忌的。判定这些凶煞的办法是根据出生年份的地支,结合‘历出生月份的地支来定。例如子年出生的人生在五月(午月),如果是位女命,就被认为是犯了“再嫁”的神煞。在合婚时男方家里如果看到这种女命,就会退避三舍,敬而远之。现把这种合婚避忌列表如下:

\begin{center}    
\setlength{\tabcolsep}{0em} % 表格内容水平padding
\begin{tabular}[H]{c*{12}{|m{2em}<{\centering}}}
\hline
流年&子&丑&寅&卯&辰&巳&午&未&申&酉&戌&亥\\
\hline
神煞&太岁剑锋伏尸&太阳天空&丧门地丧&勾绞贯索&官符五鬼&死符小耗&岁破大耗&暴败天厄&飞廉白虎&卷舌福星&天狗吊客&病符\\
\hline
\end{tabular}
\end{center}

此外还有一些其他避忌,听来煞是怕人,由于这种对于凶神避忌的办法不仅太简单了,而且还由于碰上的机会太频繁而屡屡失误,所以《命理探原》引西溪逸叟的话驳斥道:“但以人之所生年支硬配月支一字,尤为谬妄。夫以年月日时干枝八字及五行生克,论人吉凶,犹虞不足,岂可弃日时等六字,只论年月二字,即可妄断灾祥乎?”

\section{《金瓶梅》和《红楼梦》里的两次算命}
我国算命术自从五代的徐子平奠定基础以后,经过两宋元朝的氤氳作气,浸润茲延,到了明清之时,早已风靡了整个民间。社会上找人算命的,已经蔚成一种风气。百姓中间,不管是举士应考,商人经商,还是结婚生子,生老病死,都要找人问问算算,是吉是凶。到了这时,算命问卜,实际上已成了民俗中密不可分的一部分了。

对于我国这种算命术的土特产,由于它自始至终打着阴阳五行哲理的旗号,所以广泛地为知识分子所接受。社会上除了那些骗饭吃的专业算命先生外,文人学士会算命的也比比都是。正因为算命术在文人学士中有着这样的基础,所以又常反映到他们的作品中去。这不仅反映在他们的一些集子或笔记中,并且还亳不例外地反映到一些优秀的小说中去。

《金瓶梅》明人小说中熠熠发光的佼佼者。由于作者学问浩瀚,兼通命相,所以小说里涉及算命看相,占卜问卦的竟有好几处之多。除了给西门庆算命,书中第六十一回黄先生为西门庆娇妾、身患重病的李瓶儿算的那个命,就是个很典型的例子。

连日来,李瓶儿的病愈来愈重,精彩消磨,月水淋漓,六脉沉细,一灵缥渺。一连请了好几个医生,有的说是重情伤肝,肺火太旺,以致木旺土虚,血热妄行,犹如山崩而不能节制;有的说是精冲了血管而起,然后着了气恼,气与血相搏,则血如崩。这样药石乱投,你治你的,他治他的,早已乱了套儿。一天晚上,西门庆娘子吴月娘对西门庆道:“你也省可与她药吃,她饮食先阻住了,肚腹中什么儿,只是拿药淘碌他。前者那吴神仙,算她三九上有血光之灾,今年却不整二十七岁了?你还使人寻这吴神仙去,却替他打算,算那禄马数上如何?只怕犯着什么星辰,替他襄保襄保。”西门庆听了,旋差人拿帖儿往周守备府里问去。那里回答:“吴神仙云游之人,来去不定,但来,只在城南土地庙下。今岁从四月里,往武当山去了。要打数算命,真武庙外有个黄先生,打的好数,一数只要三钱银子,不上人家门。”西门庆随即使陈敬济佘三钱银子,径到北边真武庙门首黄先生家门上贴着:“妙算先天易数,每命卦金三钱。”陈敬济向前作揖,奉上卦金,说道,有一命,烦先生推算。”写与他八字,女命,二十七岁,正月十五日午时。这黄先生把算子一打,就说:“这个命,辛未年,庚寅月,辛卯日,甲午时,理取印绶之格。借四岁行运。四岁己未,十四岁戊午,二十四岁丁巳,三十四岁丙辰。今年流年丁酉,比肩用事。岁伤日干,计都星照命,又犯丧门五鬼,灾杀作吵。夫计都星者,阴晦之星也,其象犹如乱丝而无头,变异无常。大运逢之,多主暗昧之事;引惹疾病,主正二三七九月,病灾有损,小口凶殃,小人所算,口舌是非,主失财物。或是阴人(女人),大为不利。”抄毕数,敬济拿来家,西门庆正和应伯爵、温秀才坐的,见抄了数来,拿到后边,解说与月娘听。见命中凶多吉少,不觉:眉间带上三黄锁,腹内包藏一肚愁。

这里,李瓶儿八字和行运的情况是:

(年)辛未\par
(月)庚寅\par
(日)辛卯\par
(时)甲午

黄先生认为这八字理应取印绶之格,他虽没有说清原因,想来月支寅中戊土,原是生她自身辛金的印绶,所以就取了这格。

再说流年丁酉,比肩用事。酉属辛金,和自身辛金都是同类的阴干,所以说是比肩用事。至于岁伤日干,就是流年丁火太岁,克伤了日干的辛金。按照命书的说法,岁伤日干,未必就会大祸临头。这里,黄先生把李瓶儿流年说得大不吉利的主要原因是计都照命,又犯丧门五鬼,灾杀作吵。其中重点发挥了一通对于计都星照命不利的种种理由。原来命书认为,计都是星命家十一星中的一星,和罗睺星相对,十八天行一度,十八年行一周天。平时经常隐而不见,碰上日月行次即蚀,所以黄先生才有“夫计都者,阴晦之星也,其象犹如乱丝而无头,变异无常”等不吉的说法。在阴阳中,女人属阴,阴人再碰上这倒霉的阴星,也就难怪李瓶儿最终要一命呜呼了。

清朝人算命,不像明朝那样,把八字和神煞连得紧紧的,因为神煞一般都是硬套的,并且凶多吉少,从而为算命的准确性和灵活性设下了重重障碍。因此,单从本身八字出发结合岁运,论定吉凶,就成了清代命理学家的一大特色。当然,论神煞的也不是说绝对没有,如《红楼梦》曾用薛蝌的话说过:“既有这个神仙算命的,我想哥哥今年什么恶星照命,遭这么横祸?快开八字儿,我给他算去,看有妨害么?”然而和明朝人比较起来,比重要减轻多了。

有趣的是,在《红楼梦》这本封建社会百科全书中,普作者学识的光华,不仅表现在社会伦理、诗词歌赋、政治经济、琴棋书画、文物掌故、饮食烹调、儒学佛道等等多方面,并旦还深刻地表现在对医卜星相等三教九流的无所不通上。他在第八十六回中对于元妃八字的分析,就是很见算命功力的。在那个时代,知识分子探究命理,原是件十分普通的事,我们今天大可不必怕坏了他的声名而故作矜持。相反,触及一下他在这一领域里的探究,反而会使我们感到作者的形象更加丰富饱满,更加有血有肉,更加是个生活中的活生生的一员。

书中宝钗说道:“不但是外头的讹言舛错,便在家里的,一听见‘娘娘’两个宇,也就都忙了,过后才明白。这两天那府里头这些丫头婆子来说,他们早知道不是咱们家的娘娘。我说:‘你们那里拿得定呢?’他说逍:‘前几年正月,外省荐了一个算命的,说是很准的。老太太叫人将元妃八字夹在丫头们八字里头,送出去叫他推算,他独说:‘这正月初一生日的那位姑娘,只怕时辰错了,不然真是个贵人,也不能在这府中。’老爷和众人说:‘不管他错不错,照八字算去。’那先生便说:‘甲申年,正月丙寅,这四个字内,有伤官、败财,唯申宇内有正宫、禄马,这就是家里养不住的,也不见什么好。这日子是乙卯,初春木旺,虽是比肩,哪里知道愈比愈好,就象那个好木料,愈经斫削,才成大器。’独喜得时上什么辛金为贵,什么巳中正官、禄马旺地:这叫作‘飞天禄马格’。又说什么日逢专禄,贵重的很。‘天月二德’坐本命,贵受椒房之宠。这位姑娘,若时辰准了,定是一位主子娘娘。这不是算准了么?我们还记得说:‘可惜荣华不久,只怕遇着寅年卯月,这就是比而又比,劫而又劫,譬如好木,太要做玲珑剔透,木质就不坚了。’他们把这些话都忘了,只管瞎忙。我才想起来,告诉我们大奶奶,今年那里是寅年卯月呢?”

可知,作者在书中给元妃安排的八字是:

(年)甲申\par
(月)丙寅\par
(日)乙卯\par
(时)辛巳

日柱乙卯是元妃的自身。在寄生十二宫中,卯是乙木的临官禄地,以说“日逢专禄”,这是一种很好的命。再如“辛金为贵”,命书指出,辛见寅为天乙贵人,贵重得很,现在时干和月支配合,就应了这命,如果这种配合在日柱和年、月、时之间,那就更好了。“巳中正官、禄马独旺”,是说巳中庚金,为日干乙木的正官,巳支本身又处在巳中丙火的临官禄地状态,加之时支巳和日支卯相逢,应了驿马启动的命,所以算命的说元妃的命“真是个贵人,也不能在这府中”。

那末不在这府中,又怎么料定非要受宫中椒房之宠呢?这是因为“天月二德坐本命”的缘故。这里,宝钗口中所说“天月二德坐本命”和命书里排定的天德、月德有所出入,看来当是指的“归禄逢二德”了。

至于所说的“飞天禄马格”,《喜忌篇》有云:“若逢伤官月建,如凶处未必为凶,内有倒禄飞冲。”元春生于乙卯日,乙为阳木,以庚金为官星,而月上丙火能克庚金,这就成了“伤官月建”。乙日既得丙火,又生在初寅木之月,日支上的卯木便可冲出辛时所含的申金,“倒禄飞冲”便成了“飞天禄马”格。

且丢开这些不说,无论如何,作者在这里用了一定童的篇幅,借宝钗的口转述了算命先生对命理的一番分析,说明他对命书有过兴趣,有过研究,则是肯定无疑的。更不要说他在书中所说:“可惜荣华不久,只怕遇着寅年卯月,这就比而又比,劫而又劫,譬如好木,太要做玲珑剔透,木质就不坚了”的这一段话,还又是十分在行的呢。

《金瓶梅》、《红楼梦》之外,明清小说中有关算命的比比都是,如极为著名的,就有吴敬梓《儒林外史》第五十四回《病佳人青楼算命,呆名士妓馆献诗》的有关算命的描述。文中,作者通过弹三弦瞎子为青楼女聘娘算命和陈木南和瞎子之间的谈话,从一个侧面反映了当时社会算命风气之盛和作者对算命术的了解。

\section{古代名人八字举要}

这是一个很有趣的课题,可供批判性研究。在袁树珊《命谱》里,他曾为诸葛亮算命道:
诸葛武侯相后汉灵奈光和四年七月二十三日已时生
〔命造八字〕	大运
(年)	辛酉辛金偏印,庚金正印		三岁乙未
(月)	丙申	壬水劫财戊土正官癸水比肩	十三甲午
(日)	癸丑	{辛金偏印己土偏官庚金正印	二十三癸巳
(时)	丁巳	{内火正财k土正官	三十三壬辰
乙木食神	四十三辛卯
(命宫)壬辰1戊土正官
、癸水比府	五十三庚寅
日元癸水,诞生立秋节后,白帝司权.金正当令,水得金生,正气充足,再逢年干辛金,年枝西金,及月枝申藏庚金,又藏壬水,日枝丑藏辛金,又藏癸水,卺香生之助之,其为金白水济,显而易见。仅恃月干单独丙火,不独不能制金,且亦不敷济水之用,况丙与幸合,同化为水,其火之成分,又复若有若无/没有生时丁巳之二火,决不能制当令之瓯金,济有余之相水。今既得此为正式之用神,其为雨旸时,若天地颐成可知。
接着作者笔锋一转,太致认为诸葛亮大运二十三岁后金水连环,和用神火背道而驰,虽说鞠躬尽力,也只能够事倍功半。五十四岁大运庚,流年甲寅,岁支寅和命中月支申相冲,与时支已相刑,所以一旦当生命进入当年八月癸西,二十日庚辰,金水汹浦,助纣为虐之时,也就难逃厄运,卒于军中了。
按下诸葛亮八字不表,我们这里再罗列古人八字一束•
稍作提示性分析,举要而已。

〔孔子〕
(年)庚戌(月)戊子(日)庚子(时)甲申
自身庚金,归禄在时支中中,并且时干甲木为庚金偏财,本算属难得,借月文T/1,寒水当令,虽然金白水清,然而未免寒俭,况年支戌中官星丁火偏处一隅/旁受月支癸水制约,难以发挥。纵观孔子一生奔波,劳而无功,政界失意,直到晚年杏坛设教,弟子三千,说明金水流通,从次却好。孔子生于公元前551年庚戌,关于他的八字原有多种说法,录此以备一格。

〔关羽〕
(年)戊午(月)戊午(日)戊午(时)戊午

《三命通》说:“戊午日,戊午时,先刑后发,多不善终。”又说:“纯午,武职威权,名重藩镇。”基本和关羽的一生吻合C此外在格局中,这是一种a天元一气”的格局,又名凤凰池。据说,张飞的八字是“癸亥、癸亥、癸亥、癸亥”四个癸亥,一片铺天盖地的癸水,和关羽的一片火土正好来个一百八十度的大相反。火土红黄,癸水纯黑,小说中关羽面如重枣,张飞面如黝漆,大概和这不无关系。査史籍记载,有关关羽、张飞的生年,今已难以考知,但是对于他们的八字,命书却一直这样记载,这就使人费解了。

〔吕洞宾〕
(年)丙子(月)癸巳(日)辛巳(时)癸巳

算这一八字,自身日干辛金,四柱中火多刑金,好在已中戊土生我,庚金助我,而年支、月干、时干又复一片癸水,有金白水清,水火既济之象。加之时柱癸巳,天乙贵人贴身而居,所以并不一般。今査吕洞宾的生平,生于公元798宇戊寅,卒年不知,传为唐代京兆人,后来被道教全真道尊为北五祖之一。可见命书所载这一命造,可靠性是很成问题的。

(邵雍〕
(年)辛亥(月)辛丑(日)甲子(时)甲戌
日元甲木,生于季冬丑月,时支逢戌,这是移根换叶,甲木逢养的迹象,属于学界一流的命备。邵雍是北宋著名哲学家和道学家,著作有《皇极经世》、《伊川击壤集》等。

〔蔡京〕
(年)丁亥(月)壬寅(日)壬辰(时)辛亥
从这命造的日柱看,既属于壬骑龙背的格局,又可属于魁罡的格局,所以命主生平潑好。然而,《三命通会》却又认为,“壬辰日,辛亥时,秀贵,恶死”。蔡京本是北宋权臣,后来金兵攻宋,他带着全家仓皇南被钦宗下令放逐岭南,结果在途中死于潭州(今湖南长沙)。据说那时京城里有个孩子的八字,也和蔡京生得一模一样,可他却只在十岁就淹死。

〔贾似道〕虼
(年)癸酉
(月)庚申心
(日)丙子(时)丙申
《三命通会》说:“丙子日,丙申时,若通火气及寅、卯月,再行身旺运,吉。年月纯金,弃命就财,亦以吉论贾似道为南宋权相,命书说他“奸臣”。南宋末年,元军沿江东下,他率兵抵抗,兵败革职,在放逐途中,被监送人郑虎臣所杀。

〔元世祖〕
(年)乙亥
(月)乙酉	
(日)乙酉(时)乙酉
这一格局年、月、日、时,天干全是乙木,纯一不杂。按照命书说法,这是一种“干辰一字”的格局,属于大贵的命。‘元世祖名忽必烈,为元代的开貝皇帝,一生龙振虎威,功业十分显赫。

〔赵孟颊〕
(年)甲寅0!)甲戌(PO己酉(时)己巳
这一命造,时逢己巳,属于金神格局。金神原为破败之神,
算
要制伐入火乡为胜”,现在月支戌中; 火,年支寅中丙火一起制伐,加上日干己土遇印生我,比助我,所以一旦甲木制我,就格局平衡了。赵孟颊本是宋代宗室,入元后元世祖忽必烈搜访遗逸,经程钜夫荐举,官刑部主事,后又累官至翰林学士承旨,封魏国公,谥文敏。在艺术上,他的书画几乎一手笼軍了整个元代的天下,腐人学他的很多
c明太祖
(年)戊辰
(月)壬戌
(日)丁丑
(时)丁未

这一命造,如果年、月地支不见辰、戌,单是日、时地支丑、未刑冲,大有不得善终的忧虑,妙在现在年、月、日、时的地支,辰、戌、丑、未四库一应俱全,这就非但无忧,并且贵为天子了。据说明太祖朱元璋登基以后,听到天下也有一个人和他的八字相同,这就使他大为忧虑,动了杀心。后来召来一瀞,原来是洛阳地区一个姓李的穷老头儿。朱元璋问他干什么活,,他说:“老民养蜂十三窠,以之度日•”朱元璋听后宽了口气•“这和我国家享有十三省布政司的税收正好一样,把十三省税收和十三窠蜂相比,除了表面数字相同,可•实质上却有着筲壤之别

〔张居正〕
(年)乙西
(月)辛巳
(日)辛酉
(时)辛卯
这一命造,按照《三命通会》说法:“辛酉日、辛酉时,出身孤苦,中年获福,末年封妻荫子,贵。”张居正是明朝有名的政治家,在他入阁当国的十年间,推行一条鞭法,颇有政绩。公元1582年壬午张居正虚龄五十七岁,这一年,大运丙子,流年壬午,岁和运子午相冲,张居正死。

〔戚继光〕
(年)戊子
(月)癸亥
(日)己巳
(时)乙亥
这一命造,日主己巳,虽然金神位®不在时上,未能作金神格局看,可是月干偏财,时干偏官,自身又得戊土之助,所以扶抑得宜。加之地支子亥年月,所以《三命通会》认为,“以财党煞,作弃哲,兵权

〔明神宗皇后〕	
(年)甲子	
(月)乙亥	
(日)癸酉
(时)壬子
命造中亥遇乙为天德,亥遇甲为月德,这天、月二德即使没有聚在身上,可菇托父祖荫庇,也贵为皇后。

\section{无师自通的秤骨算命法}

出生年、月、日、时的秤骨份量

在旧时算命法中,有一种托名为唐代命相学家袁天罡先师的秤骨法。这种方法,只要对照一个人农历出,生年、月、日、时,分别査得年、月、日、时的称骨份量,然后再把这些份量汇总起来,便可在秤骨歌上找到有关自己一生荣枯的断语了。因为这种方法简便易行,所以有“无师自通”或“算命不求人”的说法。

\subsubsection{出生年份六十花甲秤骨份量}

\circledNum{〔甲子(鼠)〕}一两二钱\par
\circledNum{〔乙丑(牛)〕}九钱\par
\circledNum{〔丙寅(虎)〕}六钱\par
\circledNum{〔丁卯(兔)〕}七钱\par
\circledNum{〔戊辰(龙)〕}一两二钱\par
\circledNum{〔己巳(蛇)〕}五钱\par
\circledNum{〔庚午(马)〕}九钱\par
\circledNum{〔辛未(羊)〕}八钱\par
\circledNum{〔壬申(猴)〕}七钱\par
\circledNum{〔癸酉(鸡)〕}八钱\par
\circledNum{〔甲戌(犬)〕}一两五钱\par
\circledNum{〔乙亥(猪)〕}九钱\par
\circledNum{〔丙子(鼠)〕}一两六钱\par
\circledNum{〔丁丑(牛)〕}八钱\par
\circledNum{〔戊寅(虎)〕}八钱\par
\circledNum{〔己卯(兔)〕}一两九钱\par
\circledNum{〔庚辰(龙)〕}一两二钱\par
\circledNum{〔辛巳(蛇)〕}六钱\par
\circledNum{〔壬午(马)〕}八钱\par
\circledNum{〔癸未(羊)〕}七钱\par
\circledNum{〔甲申(猴)〕}五钱\par
\circledNum{〔乙酉(鸡)〕}一两五钱\par
\circledNum{〔丙戌(犬)〕}六钱\par
\circledNum{〔丁亥(猪)〕}一两六钱\par
\circledNum{〔戊子(鼠)〕}一两五钱\par
\circledNum{〔己丑(牛)〕}七钱\par
\circledNum{〔庚寅(虎)〕}九钱\par
\circledNum{〔辛卯(兔)〕}一两二钱\par
\circledNum{〔壬辰(龙)〕}一两\par
\circledNum{〔癸巳(蛇)〕}七钱\par
\circledNum{〔甲午(马)〕}一两五钱\par
\circledNum{〔乙未(羊)〕}六钱\par
\circledNum{〔丙中(猴)〕}五钱\par
\circledNum{〔丁酉(鸡)〕}一两四钱\par
\circledNum{〔戊戌(犬)〕}一两四钱\par
\circledNum{〔己亥(猪)〕}九钱\par
\circledNum{〔庚子(鼠)〕}七钱\par
\circledNum{〔辛丑(牛)〕}七钱\par
\circledNum{〔壬寅(虎)〕}九钱\par
\circledNum{〔癸卯(兔)〕}一两二钱\par
\circledNum{〔甲辰(龙)〕}八钱\par
\circledNum{〔乙巳(蛇)〕}七钱\par
\circledNum{〔丙午(马)〕}一两三钱\par
\circledNum{〔丁未(羊)〕}五钱\par
\circledNum{〔戊申(猴)〕}一两四钱\par
\circledNum{〔己酉(鸡)〕}五钱\par
\circledNum{〔庚戌(犬)〕}九钱\par
\circledNum{〔辛亥(猪)〕}一两七钱\par
\circledNum{〔壬子(鼠)〕}五钱\par
\circledNum{〔癸丑(牛)〕}七钱\par
\circledNum{〔甲寅(虎)〕}一两二钱\par
\circledNum{〔乙卯(兔)〕}八钱\par
\circledNum{〔丙辰(龙)〕}八钱\par
\circledNum{〔丁巳(蛇)〕}六钱\par
\circledNum{〔戊午(马)〕}一两九钱\par
\circledNum{〔己考(羊)〕}六钱\par
\circledNum{〔庚申(猴)〕}八钱\par
\circledNum{〔辛酉(鸡)〕}一两六钱\par
\circledNum{〔壬戌(犬)〕}一两\par
\circledNum{〔癸亥(猪)〕}六钱

\subsubsection{出生月份秤份量}

\circledNum{〔正月〕}六钱\par
\circledNum{〔二月〕}七钱\par
\circledNum{〔三月〕}一两八钱\par
\circledNum{〔四月〕}九钱\par
\circledNum{〔五月〕}五钱\par
\circledNum{〔六月〕}一两六钱\par
\circledNum{〔七月〕}九钱\par
\circledNum{〔八月〕}一两五钱\par
\circledNum{〔九月〕}一两八钱\par
\circledNum{〔十月〕}八钱\par
\circledNum{〔十一月〕}九钱\par
\circledNum{〔十二月〕}五钱

\subsubsection{出生日期秤骨份}

\circledNum{〔初一〕}五钱\par
\circledNum{〔初二〕}一两\par
\circledNum{〔初三〕}八钱\par
\circledNum{〔初四〕}一两五钱\par
\circledNum{〔初五〕}一两六钱\par
\circledNum{〔初六〕}一两五钱\par
\circledNum{〔初七〕}八钱\par
\circledNum{〔初八〕}一两六钱\par
\circledNum{〔初九〕}八钱\par
\circledNum{〔初十〕}一两六钱\par
\circledNum{〔十一〕}九钱\par
\circledNum{〔十二〕}一两七钱\par
\circledNum{〔十三〕}八钱\par
\circledNum{〔十四〕}一两七钱\par
\circledNum{〔十五〕}一两\par
\circledNum{〔十六〕}八钱\par
\circledNum{〔十七〕}九钱\par
\circledNum{〔十八〕}一两八钱\par
\circledNum{〔十九〕}五钱\par
\circledNum{〔二十〕}一两五钱\par
\circledNum{〔二十一〕}一两\par
\circledNum{〔二十二〕}九钱\par
\circledNum{〔二十三〕}八钱\par
\circledNum{〔二十四〕}九钱\par
\circledNum{〔二十五〕}一两五钱\par
\circledNum{〔二十六〕}一两八钱\par
\circledNum{〔二十七〕}七钱\par
\circledNum{〔二十八〕}八钱\par
\circledNum{〔二十九〕}一两六钱\par
\circledNum{〔三十〕}六钱

\subsection{出生时辰秤骨份量}
\circledNum{〔子时(23-1时以前)〕}一两六钱\par
\circledNum{〔丑时(1-3时以前)〕}六钱\par
\circledNum{〔寅时(3-5时以前)〕}七钱\par
\circledNum{〔卯时(5-7时以前)〕}一两\par
\circledNum{〔辰时(7-9时以前)〕}九钱\par
\circledNum{〔已时(9-11时以前)〕}一两六钱\par
\circledNum{〔午时(11-13时以前)〕}一两\par
\circledNum{〔杀时(13-15时以前)〕}八钱\par
\circledNum{〔申时(15-17时以前)〕}八钱\par
\circledNum{〔西时(17-19时以前)〕}九钱\par
\circledNum{〔戌时(19-21时以前)〕}六钱\par
\circledNum{〔亥时(21-23时以前)〕}六钱

这里要注意的是,时辰如果逢正1点的,就算下一个时辰的丑时,正3点的,就算下一个时辰的寅时,正五点的,就算下一个时辰卯时,其他类推。若逢日时制出生的,就提前一小时算。

通过自已出生的年、月、日、时,把从上表中分别奄出的秤骨份量总数加起来,然后冉把加起来份量的总数,对照下面列举的秤骨歌,就可轻而易举地揭晓每个人一生命运的穷通了。

秤骨歌

在对照:一个人农历出生的年、月、日、时的秤骨份量,并把它们一一汇总之后,我们就可根据下列份量轻重,按图索骧地找出与每个人相关的秤骨歌诀。

〔二两二钱〕
\begin{tightcenter}
身寒骨冷苦仃伶,此命推来行乞人。\\
碌碌巴巴无度日,终年打拱过平生。
\end{tightcenter}

〔二两三钱〕
\begin{tightcenter}
此命推来骨自轻,求谋作事争难成。\\
妻儿兄弟应难许,别处他乡作散人。
\end{tightcenter}

〔二两四钱〕
\begin{tightcenter}
此命推来福禄无,门庭困苦总难荣。\\
六亲骨肉皆无靠,流到他乡作老翁。
\end{tightcenter}

〔二两五钱〕
\begin{tightcenter}
此命推详祖业微,门庭营度似稀奇。\\
六亲骨肉如冰炭,一世助劳自把持。
\end{tightcenter}

〔二两六钱〕
\begin{tightcenter}
平生衣禄苦中求,独自营谋事不休。\\
离祖出门宜早计,晚年衣禄庶无忧。
\end{tightcenter}

〔二两七钱〕
\begin{tightcenter}
一生作事少商量,难靠祖宗作主张。\\
匹马单枪空做去,早年晚岁总无长。
\end{tightcenter}

〔二两八钱〕
\begin{tightcenter}
一生行事似飘蓬,祖宗产业在梦中。\\
若不过房改名姓,也当移徙二三通。
\end{tightcenter}

〔二两九钱〕
\begin{tightcenter}
初年运限未曾亨,纵有功名在后成。\\
须到四旬方可立,移居改姓始为良。
\end{tightcenter}

〔三两〕
\begin{tightcenter}
劳劳碌碌苦中求,东奔西走何日休?\\
若系终身勤与俭,老来稍可免忧愁。
\end{tightcenter}

〔三两一钱〕
\begin{tightcenter}
忙忙碌碌苦中求,何日云开见日头?\\
难得祖基家可立,中年衣食渐能周。
\end{tightcenter}

〔三两二钱〕
\begin{tightcenter}
初年运蹇事难谋,渐有财源如水流。\\
到得中年衣食旺,那时名利一齐收。
\end{tightcenter}

〔三两三钱〕
\begin{tightcenter}
早年作事事难成,百计勤劳枉用心。\\
半世自如流水去,后来运至得黄金。
\end{tightcenter}

〔三两四钱〕
\begin{tightcenter}
此命福气果如何?僧道门中衣禄多。\\
离祖出家方为妙,终年拜佛念弥陀。
\end{tightcenter}

〔三两五钱〕
\begin{tightcenter}
生平福量不周全,祖业根基觉少传。\\
营事生涯宜守旧,时来衣食胜从前。
\end{tightcenter}

〔三两六钱〕
\begin{tightcenter}
不须劳碌过平生,独自成家福不轻。\\
早有福星常照命,任君行去百般成。
\end{tightcenter}

〔三两七钱〕
\begin{tightcenter}
此命般般事不成,弟兄少力自菰行。\\
虽然祖业须微有,来得明时去不明。
\end{tightcenter}

〔三两八钱〕
\begin{tightcenter}
一身骨肉最清高,早入黉门姓氏标。\\
待到年将三十六,蓝衫脱去换红袍。
\end{tightcenter}

〔三两九钱〕
\begin{tightcenter}
此命终身运不通,劳劳作事尽皆空。\\
苦心竭力成家计,到得那时在梦中。
\end{tightcenter}

〔四两〕
\begin{tightcenter}
平生衣禄是绵长,件件心中自主张。\\
前面风霜多受过,后来必定享安康。
\end{tightcenter}

〔四两一钱〕
\begin{tightcenter}
此命推来自不同,为人能干异凡庸。\\
中年还有逍遥福,不比前时运未通。
\end{tightcenter}

〔四两二钱〕
\begin{tightcenter}
得宽怀处且宽怀,何用双眉皱不开。\\
若使中年命运济,那时名利一齐来。
\end{tightcenter}

〔四两三钱〕
\begin{tightcenter}
为人心性最聪明,作事轩昂近贵人。\\
衣禄一生天数定,不须劳碌是丰亨。
\end{tightcenter}

〔四两四钱〕
\begin{tightcenter}
万事由天莫苦求,须知福禄赖人修。\\
当年财帛难如意,晚景欣然便不忧。
\end{tightcenter}

〔四两五钱〕
\begin{tightcenter}
名利推求竟若何?前番辛苦后奔波。\\
命中难养男和女,骨肉扶持也不多。
\end{tightcenter}

〔四两六钱〕
\begin{tightcenter}
东西南北尽皆通,出姓移居更觉隆。\\
衣禄无穷天数定,中年晚景一般同。
\end{tightcenter}

〔四两七钱〕
\begin{tightcenter}
此命推求旺末年,妻荣子贵自怡然。\\
平生源有滔滔福,可卜财源若水泉。
\end{tightcenter}

〔四两八钱〕
\begin{tightcenter}
初年运道未普通,几许蹉跎命亦穷。\\
兄弟六亲无依靠,一生事业晚来隆。
\end{tightcenter}

〔四两九钱〕
\begin{tightcenter}
此命推来福不轻,目成自立显门庭。\\
从来富贵人钦敬,使婢差奴过一生。
\end{tightcenter}

〔五两〕
\begin{tightcenter}
为利为名终日劳,中年福禄也多遭。\\
老来自有财星照,不比前番目下高。
\end{tightcenter}

〔五两一钱〕
\begin{tightcenter}
一世荣华事事通,不须劳碌自亨通。\\
弟兄叔侄皆如意,家业成时福禄宏。
\end{tightcenter}

〔五两二钱
\begin{tightcenter}
一世亨通事事能,不须劳苦自然宁。\\
宗族有光欣喜甚,家产丰盈自称心。
\end{tightcenter}

〔五两三钱〕
\begin{tightcenter}
此格推来福泽宏,兴家立业在其中。\\
一生衣食安排定,却是人间一富翁。
\end{tightcenter}

〔五两四钱〕
\begin{tightcenter}
此格详来福泽宏,诗书满腹看功成。\\
丰衣足食多安稳,正是人间有福人。
\end{tightcenter}

〔五两五钱〕
\begin{tightcenter}
走马扬鞭争利名,少年作事费评论。\\
一朝福禄源源至,富贵荣华显六亲。
\end{tightcenter}

〔五两六钱〕
\begin{tightcenter}
此格推来礼义通,一身福禄用无穷。\\
甜酸苦辣皆尝过,滚滚财源稳而丰。
\end{tightcenter}

〔五两七钱〕
\begin{tightcenter}
福禄丰盈万事全,一身荣耀乐天年。\\
名扬威震人争羡,处世逍遥宛似仙。
\end{tightcenter}

〔五两八钱〕
\begin{tightcenter}
平生衣食自然来,名利双全富贵偕。\\
金榜题名登甲第,紫袍玉带走金阶。
\end{tightcenter}

〔五两九钱〕
\begin{tightcenter}
细推此格秀而清,必定才髙学业成。\\
甲第之中应有分,扬鞭走马显威荣。
\end{tightcenter}

〔六两〕
\begin{tightcenter}
一朝金榜快题名,显祖荣宗大器成。\\
衣禄定然无欠缺,田园财帛更丰盈。
\end{tightcenter}

〔六两一钱〕
\begin{tightcenter}
不作朝中金榜客,定为世上大财翁。\\
聪明天付经书熟,名显高科自是荣。
\end{tightcenter}

〔六两二钱〕
\begin{tightcenter}
此命生来福不穷,读书必定显亲宗。\\
紫衣金带为卿相,窗贵荣华孰与同?
\end{tightcenter}

〔六两三钱〕
\begin{tightcenter}
命主为官福禄长,得来富贵实非常。\\
名题雁塔传金榜,大显门庭天下扬。
\end{tightcenter}

〔六两四钱〕
\begin{tightcenter}
此格威权不可当,紫袍金带坐高堂。\\
荣华富贵谁能及?万古留名姓氏扬。
\end{tightcenter}

〔六两五钱〕
\begin{tightcenter}
细推此命福非轻,富贵荣华孰与争?\\
定国安邦人极品,威声显赫震寰瀛。
\end{tightcenter}

〔六两六钱〕
\begin{tightcenter}
此格人间一福人,堆金积玉满堂春。\\
从来富贵由天定,金榜题名更显亲。
\end{tightcenter}

〔六两七钱〕
\begin{tightcenter}
此命生来福自宏,田园家业最高隆。\\
平生衣禄盈丰足,一路荣华万事通。
\end{tightcenter}

〔六两八钱〕
\begin{tightcenter}
窗贵由天莫苦求,万金家计不须谋。\\
十年不比前番事,祖业根基千古留。
\end{tightcenter}

〔六两九钱〕
\begin{tightcenter}
君是人间衣禄星,一生富贵众人钦。\\
总然福禄由天定,安享荣华过一生。
\end{tightcenter}

〔七两〕
\begin{tightcenter}
此命推来福不轻,何须愁虑苦劳心。\\
荣华富贵已天定,正笏垂绅拜紫宸。
\end{tightcenter}

〔七两一钱〕
\begin{tightcenter}
此命生成大不同,公侯卿相在其中。\\
一生自有逍遥福,富贵荣华极品隆。
\end{tightcenter}

从以上秤骨算命法看,难免简单粗糙。然而有时为了多一重参考依据,算命先生常常喜欢把八字算命和秤骨算命合在一起推断。据说这样相互参照,可以取长补短。而民间则因为这种算法简便而容易掌握,所以一直流传不歇。

\chapter{算命术的批判}

\section{墨子的“非命”观	}
所调“非命”,就是否定、反对世界上有所谓天命的一种观点。作为学说的一种,“非命”观在先秦诸子的墨家学派中,也和“兼爱”、“非攻”一样,同样占着十分重要的地位。可以这样说,在当时一片弥搜着天命观的混浊空气中,墨家学派打出的这一旗帜鲜明的观点,无疑为当时的学坛和意识形态领域,吹进了一阵醒人耳目的清风。

《墨子》一书,《非命》共有上、中、下三篇,文中集中体现了墨家代表人物墨翟对于“天命”观的批判,是我国古代反对天命论的辉煌篇章。

在《非命》上篇中,墨子说道,古代治理国家的王公大人,都希望国家富足,人民众多,政局安定,然而他们所得到的,不是富足而是贫困,不是众多而是稀少,不是安定而是祸乱。这就是说,实际上他们没有得到原先所希望得到的,而是得到了他们原先所不希望得到的,这是什么原因呢?回答是“执(主张)有命(命运)者”混在民间的太多了。这些“执有命者”认为:“命里注定富足就富足,命里注走贫困就贫困;命里注定人多就人多,命里注定人少就人少;命里注定安定就安定,命里注定祸乱就祸乱;命里注定长寿就长寿,命里注定短命就短命。即使你使出多大的力气,又有什么用处呢?”他们既把这一套向上兜售给王公大人,又向下影响了百姓干活的积极性。所以,“执有命者”是不仁的。对于他们这些惑乱视听的言论,不可不辨个彻底的水落石出。

那末,怎样才能辨个彻底的水落石出呢?墨子的说法是,立论必定先要有个标准。立论如果没有标准,就好象在运转着的制陶转轮上去辨别方向,是怎么也辨不清楚的。正因为这样,所以墨子提出了立论一定要有“三表”(三项标准)的原则。什么叫“三表”?一是指推究本源,二是指弄清过程,三是指检验实践。怎样才能推究本源呢?这就要上推古代圣王的事迹了。怎样才能弄清过程呢?这就要下考百姓耳闻目睹的实情。怎样才能检验实践呢?这就要在刑政实施中检验是不是符合国家和人民的利益。

如今天下士君子中,认为有命运的,何不往上观察一下圣王的事迹呢?古代夏桀搞乱了天下,商汤把它接过来治理好了;商纣搞乱了天下,周武王把它接过来治理好了。这期间社会没有改变,老百姓没有变换,由桀、纣统治则天下大乱,由商汤、周武王统治却天下大治,这难道可以说是归之于命运吗?

如今天下士君子中,认为有命运的,何不往上翻看一下先王的典籍?先王的典籍,原是国家定出来,公布施行到百姓中去的一种有关宪制。先王的宪制,何曾说过“幸福不可求得,而灾祸不可避免,善良没有好处,而凶残没有害处”的话?先王用来审判案件制裁犯罪的,是国家的刑律。先王的刑律,又何曾说过“幸福不可求得,而灾祸不可避免,善良没有好处,而凶残没有害处”的话?先王用来整治军队,指挥军队进退的,是先王的军令。先王的军令,又何曾有过“幸福不可求得,而灾祸不可避免,善良没有好处,而凶残没有害处”的话?所以墨子说道:我们还没有完全统计过天底下的好书,即使统计起来也统计不完,可是从大的方面来看,基本就数宪制、刑律、军令这三个大类了。现在只要一看那些“执天命者”的言论,都是些古代先王典籍里找不到的,这不明摆着可以丢弃了吗?二看那些“执天命者”的言论,是违背天下道义的。而又正是那些违背天下道义的言论,使得百姓困苦不堪而不能自拔。把百姓弄得困苦不堪而不能自拔当作自己快乐的,就是残害天下的人。

再看,人们为什么要坚守正义的人治理国家呢?回答是义人在上,天下必治,上帝山川鬼神有了正统的继承人,万民就会得到莫大的好处。怎么才能证实这一点呢?古代商汤封在亳邑(今河南省商丘县),这地方长短大小总计起来,不过百里见方的土地,可是商汤却能和百姓“兼相爱,交相利,侈(多余)则分”,率领百姓尊敬上天,事奉鬼神,这样上天鬼神就使商汤的天下富裕起来,结果诸侯归附,百姓亲近,贤士投奔,没有终结他的一生就称王天下,做了诸侯的头头。又如古代周文王封在岐周(今陕西省岐山县),这地方让短大小总计起来.不过百里见的土地,可是周文王却和百姓“兼相爱,交相利,侈则分”,所以住在近处的百姓乐意受他的统治,住在远处的百姓也听说他的德政而竟相归附。当时只要听到周文王名字的,不仅人们都会立即起身投奔到他那里,就是连体弱病或的,也都会守候在自己的住处热切盼望说:“要是文王的土地扩展到我们这里,那末我们岂不也成了文王的百姓?”也就因为这个原因,上天鬼神就使文王的天下富裕起来,结果诸侯归附,百姓亲近,贤士投奔,没有终结他的一生就称王天下,做了诸侯的头头。刚才我们不是说过:“义人在上,天下必治,上帝山川鬼神有了正统的继承人,万民就会得到莫大的好处。”这结论就是根据这些事实推论出来的。

所以,古代圣王制定法律颁布政令,设立奖惩条例,原是用来鼓励好人,制约坏人的。刑政赏罚明白,百姓们在家就孝顺父母,出门就尊敬师长,进进出出都有一定的规矩礼节,男男女女都不杂处。国家如果派这样的人治理官府就不会偷盗,守护城池就不会背叛,国君有难就誓死保卫,国君流亡就跟随护送。而这些美德,正是国君赞赏,百姓称誉的。可是对于这些,“执天命者”却说:“君王要赞赏,是这些人命里本来就该得到的,并不是因为做了好事才被赏的。”在这种思想支配下,有些人可以在家不孝父母,出门不敬师长,进进出出都不讲规矩礼节,男男女女都杂在一起。国家如果派这样的人去治理官府就会偷盗,守护城池就会背叛,国君有难就不肯死节,国君流亡就不肯护送。而这些劣迹,却正是国君惩戒,百姓责备的。可是对于这种劣迹,“执天命者”又会说:“国君要惩罚,是这些人命里本该招致的,并不是因为劣迹斑斑才被惩罚的。”在这种思想支配下,为君的就可以不守正义,为臣的就可以不忠国君,为父的就可以不爱护子女,为子女的就可以不孝顺父母,为兄的就可以不关心弟弟,为弟的就可以不尊敬兄长。所以那些“执天命者”,简直就是一切谬论和劣迹的制造者。

那么,怎么说明“执天命者”是一切谬论和劣迹的制造者呢?且看上古时候那些不开化的百姓,吃起来贪心不足,干起活来却偷懒得很,所以吃的和穿的都告匮缺,这就难免担心自己受饿挨冻。可是对于这种受饿挨冻的简单原理,他们非但不从“我这个人太懒惰不中用,干活不得力”去考虑,反而坚持认为“我的命本来就是个受饿挨冻的命”。再看上古时候那些暴虐的君王,既不克制他们耳目声色的欲望和心里头的邪念,又不孝顺他们的父母,这就难免最终导致国破家亡。可是对于这种国破家亡的简单原理,他们非但不从“我这个人太懒惰不中用,不善于处理国政家事”去考虑,反而坚持认为“我的命本来就是个要国破家亡的命。”《书经•仲虺之告》中说:“我听说夏朝人假托天命,发布命令于天下,于是上帝便就怒而讨伐他们的罪行,因此夏朝就失掉了他们的军队。”这是商汤在否定夏桀天命观时所说的话。又如《书经•太誓》中说:“商纣在平时不肯事奉上帝鬼神,丢开他们的祖先神祇不去祭祀,竟说:‘我有好命,不必尽力做事。’这样上帝也就放弃了商纣而不去保佑他了。”这些,又是周武王在否定商纣王天命观时所说的话。

眼下,由于这些“执天命者”的言论,弄得国君不治理国家,百姓不好好干活。国君不治理国家就政局混乱,百姓不好好干活就财用不足,结果弄得上对上帝鬼神没有米饭甜酒可供祭祀,下对天下贤人达士无法收容安抚,外对诸侯宾客不能应接招待,内对贫民百姓无以充饥御寒,更不要说是将息调养老弱病残了。所以天命观“上不利于天,中不利于鬼,下不利于人”。这样“执天命者”不简直就是一切谬论和劣迹的制造者?

所以最后墨子在篇末总结道:“今天下之士君子,忠实(真心实意)欲天下之富而恶其贫,欲天下之治而恶其乱,执有命者之言不可不非,此天下之大害也。”

文中,墨翟通过推究本源,弄清过程,检验实践的“三表”原则,采用层层列举事实,步步讲清道理的方法,有力地驳斥了“执有命者之言”对于天下的严重危害性,可谓入木三分。这里面虽然也有由于时代对作者所造成的局限,使得墨翟在驳斥反对万事命定的同时,又搬出了天帝鬼神的一套,可是无论如何,这在当时一片烟障雾隔,信命如狂的社会风气中,其文章思想的光辉瑰奇,却是无可置疑的。

\section{古人都相信算命吗?}

自从唐李虚中发明用年、月、日三柱和徐子平奠定用年、月、日、时四柱推算命理以来,一时算命术大行天下,学者风从。就以那位为李虚中作墓志的大文豪韩愈来说,就是个十分信命的人。此后宋元明清,学者君子,信命的代不乏人,更不要说是一般民间的平民百姓了。

明代张瀚曾在《松窗梦语》卷六中说,有一年,作者有个名叫孙季泉的朋友,遨乡人同来一起饮酒。席间,孙季泉一个个地询问各人出生的年月日时,心里暗暗推算,可就是不出声。后来洒酣席散,孙季泉暗里拉作者到一旁说:“我和你为同年友,现在只有我们两人。我看你中年运限不利,然而不知到底怎样,现在再仔细为你推算一下。”算后,孙季泉很有把握地说:“中年虽然运行西方,只是宦途淹滞不利,对身家性命却没有多大影响。行过西方金运,进入南方火运,那就豁然通达了。”当年孙季泉高中一甲,作者中二甲。后来孙季泉官至宗伯的髙位,过了十几年后,作者也爬上了冢宰的髙位。至此,作者不禁深深感叹:“夫以数十年之迟速显晦,决于八字之间,公之精于术数如此!”

《玉堂丛语》是明朝学者焦竑的一本笔记杂著。书中卷七,作者说了这样一件事,提学萧鸣凤精于子平之术,正德丁丑年间廷试,有人拿了好多考生八字前来求教说:“请你算算这次廷试,谁是状元?”萧鸣风把各人八字一一看过后说:“这位舒梓溪可以尚中廷试第一,状元爷的桂冠他摘定了。”结果廷试下来,果然应验了他的话。

类似于以上记载的,在明清人的著述中,真是难以一一枚举,可以作为古人信命的铁证。可是,是不是古人都信命呢?答案自然是否定的。

一个人生活中贫富寿夭的遭遇,原要受着历史、社会、政治、文化、境遇等等多种因素的制约,而命理学家撇开这些不谈,大谈其命,自然是荒谬而站不住脚的。对此,在不相信命的古人中,除了春秋战国时期《墨子》一书打出《非命》三篇,对儒家的天命观念作无情批驳的那位墨家学派创始人墨翟外,唐太宗时著名哲学家吕才也是其中一个。吕才在他所著的《算命篇》中说:“汉宋忠、贾谊讥司马季主(占卜术士)曰:‘卜筮者高人禄命(说好别人的命),以悦人心,矫言祸福,以规(图谋)人财。’王充曰:‘见骨体,知命禄,见命禄,知骨体。’此则言禄命尚矣。推索本原,固其不然。积善之家,必有余庆,岂建禄(命中有禄)而后吉乎?积恶之家,必有余殃,岂劫杀(命中有劫财、七杀)而后灾乎?”接着他又举例说:“文王忧勤损寿,非初值空亡(一种凶煞);长平坑降卒,非俱犯三刑(被秦坑死的四十万赵国降卒,并不都是命里犯了三刑);南阳(汉光武帝)多近亲,非俱当六合(命中地支),与此同时,他还详论鲁庄公的命说,如果按照他出生的年、月、日去算,应该是个穷贱的命,可实际上,他却当上了一国之君。有意味的是,偏偏这个说命不准的人,却是个历史上对阴阳、舆地有着极深研究的专家。在《旧唐书》本传中,至今还保存着他的《叙宅经》、《叙禄命》、《叙葬书》等多篇。看来,大概正因为他曾深入这个营垒,所以才能反戈一击,致强敌于死命。

赵宋之时,有费衮著《梁溪溲志》十卷。在书的第九卷中,有《谈命》一则说:“近世士大夫多喜谈命,往往自能推步,有精绝者。予尝见人言:‘日者阅人命,盖未始见年月日时同者,纵有一二,必唱言于人以为异。尝略计之,若时无同者,则一时(时辰)生一人(一种),一日当生十二人;以岁计之,四千三百二十人;以一甲子计之,止有二十五万九千二百人而已。今只以一大郡计,其户口之数尚不减数十万,况举天下之大,自王公大人以至小民,何啻亿兆?虽明于数者,有不能历算,则生时同者,必不为少矣。其间王公大人始生之时,则必有庶民同时而生者,又何贵贱贫當之不同也’”末文作者虽然自谦“予不晓命术,姑记之,以俟深于五行者折衷焉”,可是却又认为“此说似有理”。可见这里,作者的天平是倾向于不信命的。

《鸡肋编》南北宋之际的著述,作者庄绰在书中卷上说道,世上以五行星历论命者多矣,这里抄录先贵而后凶的命几则:“张邦昌,元丰四年辛酉七月十六日亥时。王黻,元丰二年己未十一月初二日卯时。燕瑛,熙宁十年丁巳五月二十六日寅时。聂山,元丰元年戊午八月初十日卯时。赵野,元丰七年甲子正月十九日丑时。朱勔,熙宁八年乙卯十月二十六日申时。王寀,元丰元年戊午正月初六日子时。蔡攸,熙宁十年丁巳三月三十日寅时。邓绍密,熙宁六年癸丑九月二十三日戌时。童贯,皇祐六年三月初五卯时。”对于这些人的命造,作者又说,当他们处在全盛期时,算命的都没能够说出他们未来的灾祸,由此可见,阴阳家的话是不可太听信的,只有端正身心,好好做人,才是立身处世的唯一办法。

清代大诗人王士禛,对于算命这种玩意,他在《池北偶谈》卷二十一中引陆象的话批判说:“五行书以人始生年月日时所值辰,推贵贱夭寿祸福甚详,乃独略于智愚贤不肖,曰纯粹清明,则归之窗贵福寿,曰驳杂浊晦,则归之贱贫夭祸。《易》有否泰,君子小人之道,迭相消长,各有盛衰。纯驳清浊明晦之辨,不在盛衰,而在君子小人。今顾略于智愚贤不肖,而必归之富贵贫贱寿夭祸福,何耶?”这种从君子小人,智愚贤不肖角度入手,对算命术只推人贵贱贫宫夭寿祸福所做的批判,却也别具一格。

清代名士袁枚,也是不信算命的一位。在《随园随笔》中,他说,当初大挠作甲子,原来不过为了记数而已,就好比数一二三四一样,并没有什么多大意义,更谈不上什么五行生克配合了。听到袁枚的这种说法,当时曾有人反对道:“人在社会上本来也没姓名,可是一旦取名以后,人家一叫他就应了。天干地支这玩意,既然古人早就给它派上了阴阳五行,不也和人应答一样道理吗?”对于这种暗中偷换概念的狡辩,袁枚也用同样的狡辩术驳斥道:“人是天地万物之灵,所以一叫就应,如果派给草木禽兽什么姓名,就叫不应了,又何况天干地支这种本来就是子虚乌有的东西呢?”

清代道光年间的文人笔记中,吴炽昌的《客窗闲话》,以其流畅的文笔,多彩的内容而为读者所熟悉。书中“续集”卷七有《禄命》一则道:近来有个姓赵的算命先生,精于子平之术,自己推算下来应该得个四品的官,可是因为读书不多,难求功名。后来赵姓来到京师,看到做官的都下级承奉上级,以谄誉获利,心里感到很不是味儿,于是出都回到扬州推牌算命。平时他住在楼上,前来算命的先要用钱挂号登记次序,然后再用箩筐把人家的八字从楼下吊到楼上,除了大富贵人,一般人很难和他见而,所以名噪一时。一次,本郡太守派仆人去赵姓那里算命,赵姓漪到他的八字和自己一模一样,心甩很是诧异,就用纸条放下楼去询问来人说:“如果生在南方,和我的命差不多,如果生在北方,就有四品的官职。”后来仆人回答:“我家老爷是北方旗籍。”采然被他算中。

可是在实际中,八字完全一样而命运不一样的到处都有。那时浙江巡抚的儿子和镇江一个卖豆腐人家的儿子出生在同年同月同日同时,后来巡抚的儿子因为荫袭得官,当老子逝世之后,这做儿子的也当上了浙江巡抚的官,可是那卖豆腐人家昨儿子,却仍接替他的先人,做着卖豆腐的勾当。又如《消夏录》载纪晓岚学士的侄子,和家里奴仆的儿子刘云鹏一起降生人间,其侄十六岁而夭,刘云鹏却依然健在。当时出生,只隔着一扇窗子,两个孩子同时产出,连分秒都一样,可就是一尊一卑,一夭一寿,这又怎么解释呢?可见唐朝太常博士吕才驳论命理,千古不移。

末了,吴炽昌总结:“天下之大,每日万生万死。帝皇夭寿之日,岂无同者?昔明太祖密谕各布政,确搜与同八字之人(和明太祖朱元璋八字一样的人)。乃进三人:一僧、一丐、一市侩。帝以问刘青田,亦无以对。故曰命之理微,圣人罕见之。”这里吴炽昌的结论虽有保守成分,可是对于命理所持的怀疑态度,则是显而易见的。

对于不信命或对命理持怀疑态度的学者,除了以上几位,当然还有很多的人。可见在信命和不信命之间,从古以来就有着针锋相对的斗争。

\section{算命术中海亩蜃楼的象征律和风雨飘摇的演绎法}

中国算命术尽管博大精深,万化千变,搞得神乎其神,好象果真泄了什么天地造化之秘似的,可是剖开来看,它也有一个最基本、最原则的运算法,自始至终地贯穿在整个算命术的中间,那就是象征律和在象征律基础上推进演化出来的演绎法。

尽管阴阳五行在我国古代哲学体系中,有着它朴素的、唯物的一面,但是它也不可避免地存在着种种不足,尤其是五行:用木、火、土、金、水五种物质包罗自然界的万物,难免粗糙生硬;五行相生用金生水,只是就金属在高温下的液化状态来说的,金属的液化状态又怎能和水划上等号呢?再如土为万物之母,能生木、生金、生火、生水,却把五种原先本属各自独立并列的元素,说成了母子关系。其实,举其一点,土就是土,水就是水,土里所含的水,本是水的一种存在形式,又怎么能认为是水生出来的呢?

当然,我们不能用现代的科学水平来要求古人,否则社会就没有前进了。不过,这至少可以提供给我们这样的思考,就是建立在这种纯看生辰八字五行理论基础上的算命术,完全弃后天人为和社会因素于不顾,它的科学性又到底有多强?它的揭示人生命历程的预测又到底有多可靠?

现在暂时先不管这些,言归象征律本身。

〔人身一小天地〕把人象征天地宇宙或自然界,是古人的一种普遍认识。天地自然运周不休,人也运周不休;天地自然有阴阳五行,人也有阴阳五行。人既有阴阳五行,那末,推究每个人出生年月日时干支所秉受的阴阳五行之气的不同,从而推测他们一生的人生历程,就被引进到算命术中来了。

〔人与四时合序〕这是由人身一小天地生发出来的一种象征律。阴阳家把十天干和十二地支分为阴阳五行,按照日与天会的原理而记年,月与日会的原理而记月。这样一年有四季十二个月,算命的就把一个人降生时碰上的天干地支,分为年、月、日、时四柱,从而推定他一生的吉凶。此外,结合出生季节看五行的旺相休囚死,用的也是一种人与四时合序的象征方法。

〔五行寄生十二宫〕这是一种完全模拟自然界生物在一年十二个月中生长衰绝的象征律。在五行寄生十二宫的理论中,分别有长生、沐浴、冠带、临官、帝旺、衰、病、死、墓、绝、胎、养等状态。这些状态,循环无端,周而复始,而秉有五行之气的人,在生理上同样也有着这种相似。其中首先是绝,绝又叫受气或胞,好比万物处在地里,还没有成形,又象母亲腹空,没有怀孕;二是受胎,这时天地气交,氤氳造物,物在地下萌芽,好比人受父母之气;三是养而成形,万物在地里成形,就象孩子在母腹成形一样;四是长生,万物发生向荣,象人初成形而生长;五是沐浴,沐浴又叫作败,因为万物始生,形体柔脆,容易损伤,似婴儿刚出母腹三天,给他洗浴,容易困绝;六是冠带,这时万物渐渐秀荣,有如人开始穿起了衣冠;七是临官,万物既已渐趋秀实,就象人临官似的;八是帝旺,天地万物至此成熟,人亦至此精力健旺;九是衰,万物由成熟开始转向形衰,好比人由盛壮转向衰老;十是病,万物有病,如同人有病一样;十一是死,万物死亡,和人的死亡没有什么两样;十二是墓,墓又叫库,万物成功藏进仓库,就象人死进入坟墓。归墓以后,万物又受气胞胎而生,就这样周而复始,直到无穷。

〔五行秉性和相貌性情〕人既秉受天地五行之气而生,那末按照命理学家的说法,不同五行乘性的人,也就自然有着不同的相貌性情了。命主以木为主的人瘦长清朴,因为木形修长,木质清朴;命主以火为主的面赤聪明,因为火色红赤,火性闪烁;命主以土为主的人面黄忠淳,因为土属黄色,土性敦厚;命主以金为主的人面白刚毅,因为金属色白,性质坚硬;命主以水为主的人面黑机灵,因为水色沉黑,水性流动。

〔象征社会伦理纲常的用神〕命理学家论命看重用神,而用神名称的由来,则多半象征着封建社会的伦理纲常。《三命通会》探索古人立印、食、官、财名义时说:“生我者有父母之义,故立名印绶。印,荫也;绶,授也。譬父母有恩德,荫庇子孙,子孙得受其福,朝廷设官分职,畀以印绶,使之掌管。官而无印,何所凭据?人无父母,何所怙恃?其理通一无二,故曰印绶。我生者有子孙之义,故立名食神。食者如虫吃物,盖伤之也。虫得食物则饱,人得食物则益。被食则损,造化以子成而致养,即人子致养父母之道也,故曰食神。克我者,我受制于人之义,故立名官、煞。官者棺也,煞者害也。朝廷以官与人,此身属于公家,任其驱使,赴汤蹈火,不敢有违,至于盖棺而已,是官害之也。凡人梦棺则得官,亦是此义,故曰官、煞。我克者是人受制于我之义,故立名妻财。如人娶妻,而妻有妆奁田土,赍以事我,终身无违,我得自然享用,不致困乏,况人成家立产,须得妻室内助,故曰妻财。是四者,术家立名之大义。”奄无疑问,这些算命家立名的大义,是深深打上了封建社会伦理纲常烙印的。

以上所举五种,只是就算命术象征律中一些主要方面而说的。其实,算命术中的象征律还远远不止这些。说实在的,几乎在所有的命理观念中,都是或多或少地浸透着这种象征律的。而这种象征律,在某些局部还有点类似于机械类比推理的味迫。虽然这种类比推理的办法比较粗糙。

在这种天地阴阳五行象征基础上建立起来的算命术,在具体的推算过程中,采用设广的趟一种演绎的推理方法。在这种演绎推理中,命理学家先把一个人的出生年月日时的干支五行和他的大运推排出来,然后又把这作为推理的前提,一步一步地演绎推算下去,如命中五行自身属金,并且金多水多土多火少,就可演算出这人秉有金的属性,生性刚正不阿,由于金水相生,汩汩流通,又应聪明过人,技术超群,在运行中,不喜比劫的金、食伤的水和印绶的土,因为这在八字中已经够多的了,倒是正财偏财的木和官星的火,可以为我所用,所以一行到木地火运,就必当发迹无疑。

在实际过程中,算命先生对一个人八字的演算要远比上面所说的复杂得多。比如一个命的日干是金,在推演时不仅要考虑到整个八字五行对日主金的影响关系,并且还要综合考虑到出生的月份和寄生十二宫对自身的利弊,八字与八字之间的刑冲化合,以及干支彼此交通往来所形成的种种吉神和凶煞等等。所以这种独特的演绎法,虽然有些类似于我们今天逻辑学上的演绎推理法,然而又不能完全等同起来。因为演绎推理要求,推理的前提必须是正确的,否则就推不出正确的结论来。然而对于命理学家来说,他们所信奉或假设的前提有多大的可靠性,那就得打上几个大大的问号了。

通过说理分析,我们已不难想见,算命术象征律所象征的,不过是通过“人身一小天地”、“天人感应”等观念而作出的对自然现象的一些表面象征。然后,又通过出生年、月、日、时所得五行和这些表面象征挂起钩来进行千变万化,无穷无尽的演绎推算。这种推算,既不顾个人对自己前程的努力如何,又不顾整个社会发展对人类所造成的种种重大影晌,单从海市蜃楼的徒有堂皇表面的象征出发,那末建立在这基础上的演绎法,岂非是不攻自破?算准是偶然的,算不准是必然的,现在我们到了为算命术下这种结论的对候了。

\section{学术乎?迷信乎?}
中国算命术比起世界上任何算命术来,因为有着一个表面看去似乎十分完整严密的学术体系,所以远比其他国家的种种算命术复杂得多,难学得多。也就因为这个原因,所以一千多年来,非但一直盛行民间不衰,并且还傅得了一些学者大儒的普遍青睐。撇开算命术发明以前已深信天命的孔子、列子等名满天下的巨子不说,单就算命术发明以后,南宋的大儒朱熹、明代的闻人刘基、清代的学者俞曲园等,都是信命而且自己又会算命的一些代表人物。

由于这些学者巨儒的介入,使算命术这种玩意,更为社会上一些学问不髙的平民百姓所深信不疑。因为这些学者大儒的崇高声望,确实影响了一大批人。

学者大儒的介入,主要是因为算命术所依据的,是我国天人感应和阴阳五行的哲学理论。在这种貌似科学的理论支配下,东汉大学者王充尽管可以不信鬼神,但却坚信命运。他认为:“人禀气而生,含气而长,得贵则贵,得贱则贱。”“或贵或贱,或贫或富,富或累金,贫或乞食,贵至封侯,贱至奴仆,非天禀施有左右也,人物受性有厚薄也。”并且断言:“富贵贫贱皆在初禀之时,不在长大之后随操行而至也。”为什么他要坚持提出这种“人受命在父母施气之时,以(已)得吉凶矣”的说法呢?他说得很清楚,就是“天施气而众星布精,天所施气,众星之气在其中矣”。原来自然界中充满了一种天气和众星的精气,人在结胎之初受了这种气的或厚或薄的影响,就会影响到他今后的一生。邊种宇宙之间的气,自然也包涵了金、木、水、火、土布施出来的五行之气。

差不多和王充同时,《白虎通•五行篇》中还把自古以来的阴阳五行思想,和社会人事作了种种紧密的联系。书中煞有介琳地说:“父母生子养长子,何法?法水生木,木长大也。”“男不离父母,何法?法不离木也。女离父母,何法?法水流去金也。”“不娶同姓,何法?法五行离类乃相生也。”“子丧父母,何法?法木不见水则憔悴也。”“父死子继,何法?法木终火旺也。”“臣谏君,何法?法金正木也。子谏父,何法?法火揉直木也。”真还被说得头头是道。

后来,唐朝的李虚中和五代的徐子平接过东汉学者论命论五行的学说,在论命中大加发挥,并从而形成了一套完整的学术体系。

从上所述,由于算命术采用了阴阳五行哲学和天文星象中的一些现象作为立命的理论基础,加上通人达士学者大儒的加入和肯定,就使得神秘的中国算命术在无意中披上了一层学术的堂皇外衣。说实在的,如果撇开算命目的,单就这种算命术本身着眼,它确实有着一整套完整的体系,若要深入探究下去,也够你一辈子研究的,可是算命术的发明,毕竟是有着它的目的的,这目的就是探究每个人一生未来的吉凶荣枯、寿夭贫富等等,属于一种多少年来人们一直向往着的预知术。

预知术在我国有史以来,一直是一门人们前赴后继,悉心探索着的学问,它的内容除了四柱算命外,还广泛地包括着占卜、星相、拆字、起课详梦、扶乩等术,这里面尽管掺杂着种种江湖骗子和浓重的迷信色彩,可从另一方面来说,毕竟也注进了一些通儒学者历时绵久的研究探索,因为这种预知术对于整个人类来说,确实是太神秘,太具吸引力了,尽管到头来,这种探索努力只能是失败的。

那末说到这里,算命术是不是就不迷信了呢?答案是,就算命术本身的一整套完整体系来说,里面多少蕴涵着一种学术思想,但就预知术的目的而言,由于算命术作为演算基本的大前提,绝口不谈个人因素和社会因素,却空谈什么秉自先天的五行生克之类的哲理,其本身的合理程度由此可见。加上结合天文星象,又混进了好多宇宙中并不存在的凶神恶煞等等,所以非但推断不准,并且还在它长期的存在过程中,不可避免地给江湖骗子以可乘之机,从而使算命术更加笼上了一层浓重迷信的神秘外衣。

在目前,社会上迷信算命术的仍然不乏其人,这说明它的存在,有着极其复杂的社会因素和历史根源。要彻底根除人们对算命术的迷信,单纯靠国家禁令和行政的堵,是无济于事的。看来,最好的办法还是静下心来,对它进行无情的解剖,把它的原原本本暴露在光天化日之下,让人们自己来作一番科学而又深入的评判,这样,不仅这种预知术究竟灵验不灵验可以尽人皆知,并且还使那些江湖骗子失去了混饭吃的资本,岂不美哉?

李虚中发明算命术之初,原是根据一个人出生年、月、日所碰上的干支进行推算的,由于推算的结果,相同的人实在太多,于是五代的徐子平才又加上时间的干支,从而奠定了年、月、日、时四柱八字推命的基础。可是这样下来,据说也只有五十一万几千种命,因此社会上八字相同的人还是不少。并且在一些相同的八字中,有的还命运截然不同,判若泾渭。按照宋人有关资料记载,蔡京的八字是丁亥、壬寅、壬辰、辛亥,按理在八字的格中,这是属于壬骑龙背的格局,该是大富大贵的,可是偏偏在京城里郑粉儿子的八字,也和蔡京一模一样,但在人生的道路上,这小子却潦倒不烟,没能混好。这就给我们的命理学家出了个不小的难题。尽管前面我们说过,算命先生在碰上问题棘手时,自会举出种种遁辞,自圆其说,然而这种局面的出现,毕竟是很严峻的。

对于这种难以解释的现象,就是命书本身,也不得不作彻底的承认。《三命通会》是明末以来最为权威的一木命书,内中卷六曾载《十干十二年生大贵人例》一篇,说是只要在六甲年丁卯月乙未日戊寅时,六乙年己卯月甲戌日乙亥时,六丙年庚寅月丁巳日丙午时,六丁年丙午月壬辰日丁未时,六戊年壬戌月己丑日戊寅时,六己年辛未月己未日丙寅时,六庚年甲申月庚申日辛巳时,六辛年丙申月庚午日辛巳时,六壬年辛亥月壬辰日丁未时,六癸年丙辰月丙辰日戊子时,这六十个时辰出生的人,必定是建功立业大贵的人,不然至少也得是个出尘的神仙。以上这六十个时辰出生的人分配到六十花甲中去,每年只有一日一时,才有大贵人应世。可是对于这种说法,《三命通会》作者育吾山人无可奈何地感叹道:“大贵人莫过帝王。考历代创业之君,及明朝诸帝,无一合者。余尝谓天下之大,兆民之众,如此年、月、日、时生者,岂无其人,然未必皆大贵人。要之天生大贵人,必有冥数气运以主之,年、月、日、时多不足凭。”

好一句“必有冥数气运以主之,年、月、日、时多不足凭”,其中“冥数气运以主之”是虚晃一枪的遁辞,只有“年、月、日、时多不足凭”一句,才是书主人多年来为人卷命的甘苦之言。讲实上,作者在一生命理研究的生涯中,看到的缙绅人家和凡夫俗子同命的,多得数也数不过来。就是在缙绅和缙绅之中,八字相同而命运不同的也大有人在。因此作者接着说道:“如黄懋官侍郎,与申价副使同命,黄死于兵祸,申死牖下。申先黄死,官之大小,又不论也。朱衡与李庭龙同命,朱发科壬辰,李发科癸丑(两人没同一年登科)。朱官至尚书,李止大参,寿又不永。其子孙之多寡贤否,又不论也。万寀与饶才同命,万举进士,官至卿贰,饶止举人,官至太守。然饶多子而万则少,又万以谪戍死,而饶则否,其寿夭得丧又难论也。三河黄且斋兄弟同产,而功名先后,亦自不同。”为之,作者不由感叹:“况天下之大,九州之广,兆民之众,其八字同者何限,又乌以例论耶?”

罗大经的《鹤林玉藤》,堪称宋人笔记中的佼佼者。书中记有“大算数”一则道,一天有人拜访黄直卿,说是善算星数,能够预知吉凶祸福。对此,黄则卿回答道:“我也有个大算数,《书》说:‘惠迪吉,从逆凶。作善,降之百祥,作不善,降之百殃。’《大学》也说:‘言悖而出者,亦悖而入。货悖而入者,亦悖而出。’这个数,从古到今没有差错,难道不比你的算数强吗?”这里,黄直卿引《尚书》和《大学》的话,大意是说一个人做善事就吉,做恶事就凶,做善事,上天就会赐福给你,做坏事,上天就会降灾给你。说话背理伤人的,也会被人家所伤。用不正当手段弄进货物的,也会被人家用不正当的手段弄去。

这里,黄直卿把《尚书》、《大学》的这段话当作为人处世的大算数,从而风趣生动地批判了客人的星数,可谓笔力扛鼎。

的确,社会上立身处世,最要紧的还是“大算数”,因为这是自己给自己算命的最佳方案。种瓜得瓜,种豆得豆,佛家的因果报应论在这里和中国的传统逍德,在某种程度上该说是一线吻合的。平时佛家反对算命,这也主要是他们信奉“众善奉行,诸恶莫作”的信条,好事归我做,至于上天如何安排处置,不是我应该过问的事。

事实上,偏信命运安排,忽视“大算数”而栽跟斗的也大有人在。据说明清之际有个染坊儿子,八字算下来是个大富大贵,高官厚禄的命。家里人听说孩子生了这么个高贵的命,都大喜过望,从小开始就什么都听他的。后来孩子长大酗酒游荡,不务正业,结果酒醉落水而死,死时才十九岁。这难道不是偏信算命,从小失于教育所招致的祸患?

再如从前文推子息的歌诀来看,也是很荒谬的。歌中谈到最多的是五子,而封建社会里达官贵人广蓄姬妾,生子在五个以上的比比皆是。又如目前社会上实行计划生育,提倡一对夫妻只生一个孩子,而歌里却三个四个五个迭出,又作如何解释?这不使命理学家犯愁了么?如果硬说现在人工流产也应包括在内,可到底又有点硬解的味道了。

因此结论是:对于中国文化中的算命术,我们先要了解它,剖析它,因为了解、剖析的全过程,就是批判的全过程。堵不如导,这一条早已被证实了的历史规律。



\chapter{后记}

无论在我国文化史或学术史上,中国算命术都该占有一席之地,然而多少年来,算命术却以其自身的推测天命和其他种种原因,一直被深深地打进了冷宫。

作为一种文化现象和学术现象,我们完全可以这样认为:在我国所有学术文化中,从影响的深广来说,几乎没有什么可以和它分庭抗礼,并肩起坐的。这不仅表现在学者大儒身上,多半对这种神秘预知的方术倾注了莫大的兴趣,并且在广大平民百姓中,也是往来风靡,大有尽人皆醉的味道。虽说这种现象的产生,有着它的历史原因和社会原因,可是由于历史的巨大惯性作用和社会意识形态的根深蒂固,即使到了历史唯物主义和辩证唯物主义高度发展的今天,要把这种充满着唯心主义神秘色彩的算命术从人民群众的心目中完全抹去,显然也为时过早。堵不如导,更何况,作为一种文化现象和学术现象,也自有它存在和用作研究的必要。

当然,建立在天命观上的迷信算命术,早晚非彻底破除不可。但是我们认为,对于这种根深蒂固迷信术的破除办法,说理的导总要比生硬的堵来得更彻底,更有效。

又因为在人的心理上,对于越是神秘而弄不清楚的东西,越是密不透风不让他们知道的东西,他们就越是千方百计地想弄个水落石出,一清二楚。这里,我们索性揭开算命术的神秘面纱,把它从密不透风的保险柜里搬到光天化日,众目睽睽之下。相信这样一来,人民群众在了解它的庐山真面目后,原来对它的神秘感就会消解,原来对它的迷信就会驱散。彻底的唯物主义是无所畏惧的,因为唯物主义是科学的,而今天我们采取的,正是这种科学的态度。

值得欣慰的是,在上海人民出版社文化编辑室负责人张志国先生和责任编辑高忠先生的热情赞同和鼓励支持下,经过一个暑期来和拙荆汗流浃背、笔不停挥的共同努力,这本研究算命术和批判算命术的学术著作总算初步告一段落了。学术面前人人平等,笔者深知自己功底不深,学识有限,因此凡是有关本书不足的地方,不管是谁,只要竭诚提出,笔者无不表示由衷的欢迎。“他山之石,可以攻玉”,古训不早就这样说过了吗?

{\raggedleft 公元1988.8,洪丕谟于沪西汲绠书屋\par}

责任编辑 尚忠\\
封面装祯 王建纲\\
中国古代算命术\\
洪丕谟 姜玉珍\\
上海人民出版社出版、发行\\
(上海绍兴路54号)\\
新华书店上海发行所经销	常熟市新华印刷厂印刷\\
开本787×1092 1/32 印张7 插页2 字数138,000\\
1989年5月第1版 1990年3月第4次印刷\\
印数 140,001-190,000\\
ISBN7-208-00531-1/G79\\
定价2.85元\\

\end{document}


