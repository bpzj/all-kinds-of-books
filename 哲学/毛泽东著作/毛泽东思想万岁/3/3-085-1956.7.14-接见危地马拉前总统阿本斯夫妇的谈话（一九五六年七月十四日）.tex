\section[接见危地马拉前总统阿本斯夫妇的谈话(一九五六年七月十四日)]{接见危地马拉前总统阿本斯夫妇的谈话}
\datesubtitle{(一九五六年七月十四日)}


主席:欢迎你们,中国人民欢迎你们。全世界都知道你们的斗争,很同情你们。恐怕只有帝国主义不高兴,也只有美帝国主义,英国可能还好一些。怎么样?在美国干涉你们的时候,英国采取什么态度?

阿本斯(下简称阿):英国看起来没有表示态度。在安理会上,对危地马拉要求制止帝国主义干涉的提案,英国投了弃权票,既不表示支持,也不表示反对。

在事件发生时,有一艘英国商船正在危地马拉港口运货,被美国大使馆操纵的空军飞机打沉了。

主席:英国在危地马拉有什么利益没有?

阿:英国在危地马拉没有任何利益。在上世纪时,英国取得了危地马拉的一块领土的让与权,这一块领土成了英国的殖民地。(主席看地图)。

阿本斯夫人:就是地图上的英属洪都拉斯。

阿:我们一直在要求收回它。

主席:收回这块土地还不是主要的问题吧?主要的还是美国对危地马拉全国的侵略。在事情发生时,你们的邻国表示什么态度?墨西哥采取什么态度?

阿:墨西哥同危地马拉之间的关系一直都很友好,墨西哥政府尽力表示中立。

主席:墨西哥属北美洲吗?

阿:是的。

主席:中美洲有那些国家?

阿:中美洲包括危地马拉、洪都拉斯、萨尔瓦多、尼加拉瓜和哥斯达黎加。

主席:不包括巴拿马吗?

阿:不。巴拿马原来是哥伦比亚共和国的一部分。由于巴拿马运河区的关系,它才成了一国。另外,中美洲五国在独立后成立过统一的中美联邦。

主席:你们南边的国家态度怎么样?

阿:南美各国人民对危地马拉都很同情。

主席:干涉是来自海上的吗?

阿:不。侵略是直接从洪都拉斯和尼加拉瓜来的,显然是美国国务院直接挑起来的。事情发生前,在加拉斯加召开了美洲国家的外长会议。会上,杜勒斯提出了“反对国际共产主义干涉”的提案。这一提案是要一切美洲国家迫害共产党人和所有保卫本国利益、反对帝国主义侵略的爱国人士。危地马拉反对这一提案,因为这是美国干涉他国内政。并使垄断组织增加对其他国家的渗入的提案。

主席:这个会是几时开的?

阿:一九五四年三月开的。

主席:美国到处以反共的招牌为名,而达到侵略别人的目的。美洲国家共产党人数很少,苏联离的很远;为什么美国那样急于反共呢?

阿:美国所以急于反共,是因为共产党坚决为本民族的主权,独立以及从垄断控制下争取解放而斗争,并且揭露本国政府的卖国企图。

危地马拉代表团在加拉斯加会议上指出,美国是利用反共的旗帜掩盖着进一步渗入其他国家和加深殖民控制的企图。

美国提案的投票情况是这样的:墨西哥和阿根廷弃权,危地马拉代表团投票反对,其他国家投票赞成。

危地马拉的态度受到各国进步的革命力量的尊敬,因为它不仅表达了危地马拉人民的意愿,也表达了拉丁美洲各国人民的意愿。

主席:我们知道的。

你们的朋友多些,美国的朋友少些,你们的朋友比美国的多。

美国的胜利是暂时的,危地马拉终究是你们的,是危地马拉人民的。一切民族都要独立。

美国到处欠账。欠中南美国家、亚非国家,还有欧洲国家的账。全世界都不喜欢美国,包括英国在内。广大人民都不喜欢美国。日本不喜欢美国,它压迫日本。东方各国,没有一国不受到美国的侵略。日本、南朝鲜、中国的台湾、菲律宾、南越、巴基斯坦,这些国家都受到美国的侵略。有些还是美国的盟国。人民不高兴,有些国家的当局也不高兴。

阿:国际形势最近两年的发展,有利于各国人民争取独立。亚洲国家就是这种例子。亚洲形势对其他大陆的斗争有着决定性的意义。现在听到了亚洲的声音,以后一定会听到拉丁美洲的声音。

主席:一切会有变化。力量大的要让位给力量小的。力量小的要变成大的,因为大多数人思想上要求变,小的要求变大。美国力量大要变小,因为美国本国人也不高兴本国的政府。

我这一辈子就经历了这种变化。(主席问在座的人的年龄)我们这里有三个清朝的人、四个民国的人——四个国民党的人。(笑声)

满清,要推翻。什么人推?孙中山领导的党和人民一起推。孙中山力量很小,清朝的官员看不起他。打仗总是失败。最后,还是孙中山推翻了满清。大,也不可怕。大的要被小的推翻,小的要变大。几年以后,孙中山失败了。因为他没有满足人民的要求,比如人民对土地的要求。不知道他晓得不晓得镇压反革命,当时反革命到处跑,后来就失败于一个军阀首领袁世凯。

袁世凯的力量比孙中山的大。但还是照老规律,力量小的,同人民联系的,强;力量大的,反人民的,弱。

而后,孙中山(后来是蒋介石)同我们共产党合作,把军阀都消灭了。

蒋介石统一中国,得到全世界各国政府的承认,统治了二十年,力量最大。我们力量小,当时五万党员,经过镇压只剩下几千党员。敌人到处捣乱,还是按照老规律,强大的失败,因为他脱离人民;弱小的胜利,因为它同人民联系在一起,为人民工作。结果,也就是这样。

抗日战争,日本很强大,把国民党和共产党领导的武装力量都赶到了偏僻的地区,占领了中国的大城市北京、天津、上海、汉口、广州。日本军国主义和希特勒也是按照这个规律,没几年就倒了台。

我们经过了很多困难,从南方赶到了北方,从几十万人到只剩下几万人。长征二万五千里,剩下二万五千人。

经过抗战时期,打日本,我们发展到了九十万游击队,枪炮不如国民党。国民党军队四百万。(阿本斯向夫人惊奇地说:四百万。)打了三年,国民党打不赢我们。强大的打不赢,弱小的总是胜利。

现在美国很强,不是真的强。美国政治很弱因为它脱离广大人民,大家都不喜欢它,本国人民也不喜欢。外表很强,实际上不可怕,纸老虎。外表是个老虎,但是是纸做的,经不起风吹雨打。我看美国就是个纸老虎。(笑声)

整个历史都证明这一点。人类有阶级社会的历史已有几千年,可以证明这一点。强的要让位给弱的。美洲也是这样。在美洲,美国要让位给你们。

阿:感谢主席先生对我们的亲切关怀和帮助。

主席:我们的经验,供参考,不能照抄。

阿:我们一定会使它适合我们的情况。我们感到很有信心。再过五、六年,我们愿意再到中国来看中国人民的成就。这种成就不只是为了中国人民,也是为了各国的人民,包括危地马拉人民。

主席:中国人民是你们的朋友。中国人民的成就中,有你们的一分;你们的成就中,也有中国人民的一分。

只有帝国主义被消灭了,才会有太平。终有一天,纸老虎会被消灭的。但是它不会自己消灭掉,需要风吹雨打。

阿:亲爱的主席先生,能亲自向你致意,使我们感到非常荣幸。我们认为这是对我们的极大的友谊,对危地马拉人民的极大的友谊。在我们的人民的心目中,主席先生受到极大的尊敬。(主席:谢谢。)我们祝贺主席先生在国家工作中的成就,祝主席先生的个人幸福。

主席:我们是处在同一种地位,做同样的工作,为人民办点事,减少帝国主义对人民的压迫,搞得好了,可以根本取消帝国主义的压迫。

要帝国主义干什么?我不懂。中国人民不要帝国主义。帝国主义无存在之必要。

在这一点上,我们是同志。

我们性质上相同,只是所住地区、民族、语言不同,没有性质上的差别。我们同帝国主义却有性质上的分别。看到帝国主义就不舒服。乔冠华在三十八度线上的板门店同美国谈判了两年多,也是看到帝国主义就不舒服。

纸老虎,是从战略上说。从整体上来说,要轻视它。从每一局部来说,要重视它。它有爪有牙。要解决它,就要一个一个地来。它有十个牙齿,第一次敲掉一个,它还有九个。再敲掉一个,它还有八个。牙齿敲完了,它还有爪子。一步一步地认真做,最后总能成功。

从战略上说,完全轻视它。从战术上来说,重视它,一仗一仗的、一件一件的,要重视。现在美国强大,从广大范围、从全体、从长远考虑,它不得人心。它的政策,人家不喜欢,它压迫剥削人民。由于这一点,老虎一定要死。因此不怕,可以轻视它。但是它现在还有力量,每年产一亿吨钢,到处打人。因此还要跟它斗争,用力斗,一个阵地一个阵地地争夺,因此需要时间。在板门店谈了两年多才搞出名堂,搞出个朝鲜停战。现在又在日内瓦谈了快一年了,还没有谈出名堂来,也可能谈到二十一世纪去。只要美国拖到二十一世纪,你就得准备拖。

不要悲观,不会拖那么久的。

同美国打交道,像我们中国所说的吃牛皮糖一样。看样子美洲国家、亚洲国家只有一直同美国吵下去,吵到底,直到风吹雨打把纸老虎打破。

阿:现在各国人民的风雨正在打美国。

主席:对。

刚才你说五、六年后再来中国,是说五、六年后有把握取得胜利吗?

阿:我不是那个意思。我们不知道什么时候能取得胜利。但我们相信最后总是要胜利的。

主席:你(指阿本斯)是从欧洲移去的?

阿:我父亲是从瑞士来的。母亲是……

主席:你呢?(指阿本斯夫人)

阿本斯夫人:我相信我有点西班牙血统。

主席:西班牙血统的占人口的多少?印第安人占多少?

阿:在危地马拉,印第安人占百分之七十。

主席:其他中南美国家的印第安人多吗?

阿:墨西哥、危地马拉、秘鲁、玻利维亚、巴西印第安人多。阿根廷、智利差下多全是欧洲移民的后裔。

主席:有人说,美洲人是从中国去的?

阿:危地马拉人很像中国人。

主席:现在还没有什么证据,也可能是从中国去的,也可能不是。南美、北美一共有多少印第安人?

阿:(犹疑)几千万吧?

主席:这是不是就出现了欧洲移民同本地印第安人的合作问题?

阿:工人和农民都受到剥削。尤其印第安人受到歧视,身上标着号码,像牲畜一样。

主席:你们从欧洲移入的人,是不是可以分为两部分,一部分人统治另外一部分人,那么你们这一部分被压迫的人就容易同本地人接近了。所处地位相同。

你们的政府实行了土改的政策,把帝国主义和地主的土地分给农民。外地来的人当总统,把土地分给本地人,一定会得到他们的拥护。

阿:我们确实把反动党派搞出来的压迫取消了一些。我们那里没有种族歧视,黑人不受歧视。

主席:危地马拉有黑人吗?

阿:很少。

主席:不知道是不是你们的政策太进步了,也是一个缺点。是不是一切地主的土地都没收?有民族资产阶级吗?地主中间有没有不跟美国走的?人民不喜欢美国,对美国关系小些的地主,可以团结。这样就能组成强大的统一战线。包括一些虽对本地人歧视,但又对美国不高兴的人。

阿:地主阶级一般是亲美的。民族资产阶级是同帝国主义的利益对立的。我们听了危地马拉共产党的建议,改正了前任政府的一个错误。前任政府为吸引人民的支持,采取了一般有利于穷人不利于富人的政策,造成贫富对立。我们改变了它,支持民族资产阶级的合理要求。

但是,我们犯了一个错误,民族资产阶级、工业资本家和中产阶级支持政府,我们却远离了他们。

主席:我们这里也有地主中间的左派,从地主阶级分化出来的,反对蒋介石,和我们在一起。

除开逃往台湾的那些以外,我们接收了整个资产阶级,接收了所有的工程技术人员,大、中、小学的教授、教师。教育他们,要他们为人民服务。

我们的工作中犯了一些“左”的错误,使他们感觉到不舒服,引起他们的反感。

大体上可以说,经过五、六年的合作,团结得更好了。

国民党的文、武官员,只要没有跑到台湾,我们也接收了下来。但我们处理得不好,有些有职无权,挂个名。这些非切实改正不可。水利部有个参事郑洞国,挂个名,没有事做。长期下去如何得了。文史馆不做事情。有些地方政治协商会议也不做事情,这是缺点。

你在中国对我们的错误缺点要注意,要进行比较。(主席问陪见人:还要为阿本斯组织些什么座谈。吴茂洁答:已组织了关于私营工商业改造问题的座谈,还要组识土地改革、农业合作化和统一战线问题的座谈。)尽讲好的,你就不要相信他。他不讲坏的,你就问他,有什么坏的。

还能在中国留多久?

阿:到八月十四曰。我们因签证问题,必须在下月底前回到法国。

主席:那么还可以到各地看看。多找些人谈谈,找共产党人,也找其他党派的人,找起义将领、资本家、宗教界人士。你们如有少数民族问题,也可谈谈,这方面我们有点经验,也有错误。

张奚若:他们已参观了民族学院。

阿:很感谢主席先生所讲的话。我们将利用在中国逗留的时间来学习。我们已经学到了你们的政策的灵活性,你们给每一个不同美国在一起的人以机会,使他们为人民服务,并在服务中承认自己的错误。

美洲的一些政治领袖,往往不是这样。他们对他们自己,也从来不承认有错误。

主席:你很老实,想到的就说了出来。

我们曾经犯过错误。在根据地的头几年,采取可以叫做“一切打倒”的口号。除工人、农民外,什么人都不要,通通赶掉。这样,蒋介石倒感到舒服了。蒋介石力量大了,把我们从南方的很多根据地赶到北方的一个根据地。这样,我们就“想”,批评、改正了这个错误。

我们的历史上犯过三次“左”倾、两次右倾——陈独秀、王明。此外,局部的、个别的也犯过右的错误,如张国焘、高岗。

犯错误也有好处,可以教育人民,教育党。日本是我们的老师。我们有很多老师:日本、美国、蒋介石、陈独秀、李立三、王明、张国焘、高岗。这些学习,付出了很大的代价。在历史上,英国同我们打过很多仗。英帝国主义、法国、德国、意大利,都很喜欢我们这块地方,都是我们的教授,我们是他们的学生。

我不知道你们是不是有错误。如果你们有错误,你们也可以取得教训。美国就是你们的教授。现任总统阿马斯可以说是你们的第二个教授。

阿:在告别之前,我首先要表示感谢。我们占了你很多时间。我们知道,这些时间并没有白费掉。我们要把主席先生的话带到我们的政治力量中去,来改进我们的工作。非常感谢。

主席:谢谢你们来看我。


