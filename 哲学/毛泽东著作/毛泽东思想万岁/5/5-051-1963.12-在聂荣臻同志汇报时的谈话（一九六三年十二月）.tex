\section[在聂荣臻同志汇报时的谈话(一九六三年十二月)]{在聂荣臻同志汇报时的谈话}
\datesubtitle{(一九六三年十二月)}


〔谈到松辽平原的经验时〕石油部是第一个运用解放军的一套办法,工业部门都要学习解放军,设立政治工作部门。用政治工作来保证建设任务的完成。石油部是学习解放军的经验,像连队的政治工作一样,不脱离业务。

石油部比较单纯,一机部复杂(指产品),要调些人到工业部门作政治工作,解放军是出人材的学校。

〔汇报科学技术十年规划任务时〕要打这一仗,科学技术是生产力,过去打上层建筑,也是为了发展生产力,不打这一仗,生产力无法提高。要以革命的精神来搞科学技术工作。

〔汇报基础理论时〕不搞理论是不行的,要搞一批理论队伍,也包括社会科学。

〔汇报留学生工作时〕只派留学生,国内却固步自封,不向好的单位学习。

〔汇报向国外进口书刊时〕多少外汇?(答:××万美金。)不多嘛。社会科学的买吗?(答:也买。)影印外文杂志,广告不要弄掉。

〔汇报十年基建投资××亿时〕十年××亿,每年×亿,不多嘛!

〔汇报到受激光发射时〕要有些人专门搞这事,长远来搞。从数量上看,人家比我们多,我们搞不过人家。但是从历史上看,攻防两手,防我们要考虑。比如城墙,筑起来是为了防御。

〔汇报治理黄淮海问题时〕这个研究工作,要几万人来搞。

〔汇报医疗卫生问题时〕感冒药要认真解决。

〔汇报探索性工作时〕允许公开犯错误,但是发现错误要批评,又要鼓励,允许人家公开改正错误。

〔谈到朝鲜的战备工作,搞了××万里山洞作地下工厂时〕我们也要作蠢事情。

〔最后谈到三大革命时,问科学实验的含义是什么〕我讲的科学实验,主要是讲自然科学。社会科学、哲学、政治经济学、军事科学是不能搞科学实验的。商品,价值法则不能搞科学实验。战争不能搞科学实验。辩证法不能搞科学实验。理论法则是概括出来的。军事演习不能搞实验室。社会科学的一部分在一定意义上也可说科学实验。在中南海元旦联欢会上的讲话(一九六四年一月一日)

马列主义原来是外国的,和中国革命结合了,我们就创造性地发展了它,也就成为我们自己的了。


