\section[在中央会议上的谈话——陈毅同志传达(一九四七年十二月)]{在中央会议上的谈话——陈毅同志传达}
\datesubtitle{(一九四七年十二月)}


第一部分一十二月会议的记录及毛主席的指示。

日本投降后特别是一九四七年这一年,发生了根本变化,可以说是个伟大的事变,原因是敌我双方的形势有了根本的改变,政治、经济、土改、整党各方面都有了根本的转变,这变化形成了这个伟大事变的标志。政治方面的变化是蒋介石孤立了,我们获得全国各阶层拥护,是过去不曾有过的现象。经济上蒋介石陷于危机的漩涡,无法脱离。我们也有困难,但能渡过。军事上,人民解放军获得了伟大的胜利,这是主要标志。我们有土改,蒋介石没有,我们有办法,我们土改后也有不同情况。整党方面,发现了党内不纯,发现了地主富农的思想倾向和路线,我们从思想上组织上来一个整顿。这五条造成了变化的主要原因,使敌我形势根本变化过来。这个根本变化,一九四七年在各方面都表现出来。

政治方面:人心动向完全改变,人人向我。全国人心希望寄托在共产党身上,对蒋介石深恶痛绝。孤立蒋介石是长期斗争,而且长期没有解决的问题,现在解决了。内战时期,我们孤立,我们只有苏区工农群众,其他各阶层我们都脱离了,那时蒋介石的基础较大,内战时期,这个问题未解决,故内战失败了。抗日时期也竭力要解决这个问题,情况有所改变,我们的朋友多了,因为我们采取了适当政策。为坚决打日本,减租减息,我们拿住了抗日旗帜,中立了中小地主,改善了与地主的关系,发扬了民主,争取了资产阶级及其他各党派。军事政策是分散的游击战争,适当的进行反磨擦,防止冒险主义。抗日时期的各种政策一般是正确的。反对抗日时期的政策,否定一切,说这也是机会主义,那也是机会主义,大革命时期后期,由于机会主义搞错了,以致蒙受了失败。抗日时期,一般政策是对的,孤立了蒋介石,取得了朋友,王明路线未占统治地位,避免大革命时期的右与内战时期的“左”,使党获得了领导地位。要很好研究抗日时期的政策,不能粗粗糙糙的说那时错了。蒋介石的主要错误是消极抗战,坐山观虎斗,部队不打仗,以致使他丧失了民族领袖的地位。其次是垄断经济,政治上则是搞特务一党专政,这三条使他送掉了国民党的江山,人心向我,解决了孤立蒋介石的问题。这主要是由于我们的政策适当所致,这是四七年发生根本原因。

军事上,去年七月我们转入反攻以来,蒋介石转入防御地位,于是军事上完全改观,十年二十年来我们长期处于防御,被“围剿”地位,没有进剿敌人,进攻敌人。抗战时期我们还是处于分散防御地位,去年(四七年)七月,我们历史上第一次转入进攻,这是革命的进攻,不要说“反攻”。反攻带有防御滋味,不能完全概括这一形势的内容。战争在初期是自卫性质,我们那时的方针,是迟滞内战,现在要消灭蒋介石,已不是自卫性质。自从蒋介石开伪国大,制定伪宪法,人心愈失,同时全国实行革命进攻,不是自卫防御,把蒋介石的进攻打垮,造成大革命叫“进攻”内容更适合。蒋介石无论如何要返回过去形势,已不可能。黄鹤一去不复返。今后是我们如何转入江南、四川、两广的问题,蒋介石再想过黄河、陇海已不可能。

去年七月我转入进攻,美蒋讨论好几次,美国苦心焦虑,甚至华盛顿主张放弃纬线40°以北(东面安东经过北宁、平绥、大同,西至西安)就是说放弃东北,争取华北,巩固长江,经营华南。

美国人主张,一定要放弃一些地方,因士气不明,兵力分散,不如把队伍拉下来,整调装备,才能再行进攻共产党,魏德迈说“攻不动,守不住,不如放弃。”蒋不同意说“有放不得之苦,放弃一地,共产党就会全部推下来。”

美主张三种地区:放弃区,即纬线40°以北,安东,大同,嘉峪,西安以西这一线;力争区,纬线35°,陇海以北;坚守区,纬线30°以南。蒋介石则主张东北坚守,华北力争,中南进剿,江南防御,华南经营(包括两广,台湾,福建由美来搞),美主张撤出东北,把军队拿到华东训练,将来反攻。蒋说放不得,撤不得,撤了共产党即推下来了。

蒋介石要尽量拖到第三次世界大战,他把全部希望都寄托在第三次世界大战上,他说:“要三次世界大战,三四年内爆发就有希望。”

自从去年七月,我们转入进攻,蒋就永远不能翻身了,局部进攻还可能有,战略进攻是不可能有了。

经济方面,蒋的经济四七年比四六年更严重,美国帮助也不能解决问题,过去对西北战场担心,内战作战又遭荒,人少兵多,晋西北也有灾荒,山东敌人压境,我们经济也成问题。但自转入进攻主动移出,负担减轻,收复了大块土地,办法更多,我们的经济问题解决了。因蒋没有土改,四八年蒋介石将更困难。

土改整党:使党员上轨道,土地法颁布与未颁布大不相同。五四指示是党内指示,人家不说究竟,有些怀疑,公布了索性平分,人家反而赞成了。去年九月土地会议后,全国各区土地会议开得好,土地会议就是整党会议,彻底揭发了党内严重现象,使党为之改观。晋绥西北政治方向是对的,因为是在搞土改整党,东北工作作得好,晋冀鲁豫也作的好。估计一九四八年还有更大的变化,优势属于我,形势发展有利于我,例如,就人心而论,人心大变,群众靠我。蒋总动员,但人心反他,学运未停,三青团过去有劲,现在无劲,学生都是地主富农子弟,我们搞土改,学生们造蒋介石的反,值得注意。五四指示未公布,怕人家反对,公开宣传反而好了。

军事、经济、土改、整党这些情形,一九四八年再搞一年,有根据说,更大的胜利一定要来的。现在革命高潮已来,过去对于高潮是从城市大暴动、军事胜利两个方面来看。现在不要这样看,不必希望大城市,如果仍是那样看,则日本投降后,是看能否进平津大城市,“现在不这样看大城市。(1)四七年下半年群众运动并未低落,四八年还要高涨。(2)军队未去前,大城市暴动很困难(巴黎亦然,北伐军到了龙华,上海工人才暴动),军事是主要的,故该七月反攻即是高潮开始,主要是在军事胜利,将来胜利还会更大,不说是从城市独立暴动,大城市是野战胜利的结果。

但战争仍是长期的,要准备对付敌人最大限度的抵抗,“七大”设想了十七条困难,有些讲中了,有些未讲中,那时估计重点,没有害处,未讲中的只有好处,没有坏处,但不要设想敌人最大限度的抵抗,无力攻,但有力守,抵抗他还是要抵抗的。第二线兵力美国给他们训练装备,(美国出兵可能不大)特别美派兵守城占领南京、上海,不让我们进去,麻烦,美打与不打没分别,主要是蒋要垮台,蒋垮台以后,我们才来改变,是个困难,但不要怕。

地主阶级消灭了,全国土地分好还得十年。准备四年到五年,大概一九四九年到一九五○年可能在全国把蒋介石打垮(五年胜利,从四六年算起),到我战争结束,双方需要五百万人伤亡(每年各五十万),今年还要设想几条困难,要打不要忙,不是轻松的事。

战争不应使之间断,要一直进行到底,不使敌人有休息机会,但不要说死,主要决定南方大城市群众和中产阶级态度,例如,蒋介石大势已去,说要下野推出李济深之类的人出来,这时和与不和取决于南方群众,如果群众要和,你如不和,他会说你是好战分子,和吧!明明是蒋的欺骗,是他的金蝉脱壳、移花接木的诡计,想借以得到休息的机会,企图卷土重来。为了预见这种事情,宣传上要予以揭露,要向群众说清楚,不是消灭蒋介石个人,而是要消灭蒋介石的集团及其阶层,群众是否觉悟到,非蒋灭亡无出路。如群众对大资产阶级的幻想一扫而光,则无此问题,但如蒋下野,中间派出头,如此是否能将群众对大资产阶级的反对派的幻想,一扫而光?两种了解:(1)消灭蒋介石本人,(2)消灭蒋介石集团及其阶层。现在却要作宣传解释工作,向群众说清楚,以揭露之,否则,战争间断会有的。

靠自己,不靠外援,不靠不是不要,要还是要,要而不靠,任何时候都要,任何时候都不要靠,就是到了全国胜利,有了中央政府,还是不靠。蒋介石的战略基本上是靠外援,这是危险的,故失败。外援对我们不是决定条件,苏区时未靠外援,但靠自己未靠好,如靠自己靠得好,可以不长征,东江纵队可以北来,浙东可否多,皖南皖西鄂东则未留下,故一切靠自己政策好(军事的、群众的)完全可以靠自己。

统一战线:

总的目标划清敌我界线,达到孤立敌人,勿孤立自己,先分开,后磋合,把资产阶级孤立起来,无产阶级对资产阶级,他孤立我,我孤立他,最后把他孤立起来。

北伐后期右,脱离群众(农民),脱离军队(自己不搞武装),只剩了无产阶级,孤立自己,故失败。北伐时本不孤立,但群众要土地不给,又不搞武装,故农民失望,造成了自己孤立,以致失败。

内战时过左,城市完全孤立,孤立了党了,强迫示威,强迫罢工,白区损失百分之百。农村中战争与土改“左”的政策,进攻大城市,地主不分田,富农分坏田。赤白对立,与此有关。对中小资产阶级知识分子,农村没收一切,劳动条件过高,最高工资等片面工人利益,搞坏了经济,搞垮了工商业,赤白对立。

农村中不是完全孤立,但因损害了中农,特别对商人,结果剩下工人,贫雇农,焉得不败。

内战时期我们自己孤立了自己。

抗日战争时期学乖了,一向反对只团结而不斗争,一向又反对只斗争而不团结,反对相信国民党超过相信群众的程度,不是不完全相信国民党,他在抗日,为什么不相信,问题在于相信他不相信群众,只联不斗,反对了这个就胜利了。我们发展进步势力,争取中间势力,孤立顽固势力,早釆取了这个政策,打了八年就把蒋介石孤立了,如不在八年中采取孤立蒋介石的政策方针,今天就来不及了。

两个原则,孤立敌人要反右,争取群众要反“左”。

三三制就是对地主让步,又联合又斗争的策略,要他参加政府,又要他服从减租减息。土地政策在城市可以迷惑中小资产阶级,不要以为高潮来了就四面开枪。现在城市中争取中小资产阶级有必要,当时如不国共合作是不对的,国共合作实际国共斗争,合作是形式,斗争是内容,对蒋有合有斗。对民族资产阶级也是如此,但斗的少些,与蒋有区别,是慢慢的来。开始报上只讲合不讲斗是对的,但有坏处,就是造成对蒋介石幻想。减租减息是完全必要的,不是可有可无的,我力小,地少兵少,且减租减息发动群众,先霸到地方才能过渡到土改,开明士绅到政府是要的,地主剥削阶级反过共,还是要的。今天这样还是有的。现在有人讲是机会主义是不对的,李鼎铭死了要公布,要追悼。大家回忆双十宣言,双十宣言包括四种人,(1)未作过坏事的,(2)未沾过人血但是反共的,(3)有功的士绅,(4)追到天涯海角等四种人。只有第四种人要追到天涯海角的,但追到了也不一定搞,这是从工农着想,可少死些人。宽大政策不是替敌人宽,而是替群众宽,是群众问题,不是对地主富农同情。

故反右倾不要反到不要统一战线,说三三制错了,减租减息错了,吴满有运动也错了,不对的。三三制现在不提倡,也不反对,现在反右又反“左”,划清界线,否则还要失败。

反右:(1)过高估计敌人,怕蒋怕美,不敢到国民党区去,怕美国人来头大,刘邓等三路大军到国民党区,一个月解决了棉衣问题,完全证明了在蒋区是能作战的。(2)对中间派动摇分子的观念上模糊不清,对中间党派,民族资产阶级,中间阶级,搞不清楚,如新华社对张澜、罗隆基与国民党不妥协,大加赞扬,对冯、李反蒋就作义务宣传员,对黄炎培也宣传,对其动摇估计不足。黄炎培是中等资产阶级右翼派,动摇得很,对冯玉群、李济深、黄炎培不要宣传,这些人没有军权未上台前,可以表示革命,有了军队,就撑着我们死打,到底是地主军队,民族资产阶级是保皇党,对民族民主革命是动摇的,虽为数不多,但要批评揭露他,对其政治影响要打击,经济上不是消灭,要反右也要反“左”。

地主穿衣吃饭容易认识,就是挨边上的,不容易分清楚,两头小中间大,中间阶级很大,也是挨在边边上的不好搞,资产阶级末尾的不好搞,具体的如富裕中农与富农不容易区别,黄炎培反蒋,但又对蒋磕头,时而赞成土改,时而反对土改,这种人很难分清,接合部地方不好搞。(3)土改中有的表现消灭地主无决心,有些人打日本很有劲,消灭封建没有劲,有的对混在党内的地主富农不肯坚决肃清,整党无决心,也是右倾机会主义。

反右胜利后,最容易犯的是“左”倾,力量大了就无所顾忌了,速胜论,还要长期小心,准备对付敌人的最大抵抗。

(1)不要过高的估计敌人,是否轻敌,又是又不是,有轻敌的成分又不是轻敌。轻敌不应放在具体布置上,但在全体上要藐视敌人(如对美蒋地主却是鄙视)对这敌人不要怕,对当前敌人要谨慎。

战路上轻视敌人鄙视美蒋,战役上重视敌人,具体战斗上要注意,要慎重对待敌人,有一些人对美国出兵,原子弹,第三次世界大战怕的要命,对当前敌人忽视得很。

搞土改时封建势力要轻视它,但对你面前的地主富农具体的敌人,要谨慎对待他,不要乱打乱杀,要用细致的策略斗争把它压服。

我们同志恰恰相反,敌人愈远愈害怕,愈近愈不注意,敢于下手革命,但具体来革则不易,我们同志在此也是相反的,许多人在具体革命上栽跟斗,战略士它是落后的,我是新生的,但对当前敌人要重视,因为它是有毒的。

“左”的错误就是对当前敌人,不加分析轻效从事。

(2)不要性急,全国土改十年完成,解放后三年完成,即一九五○年完成。战争五年胜利,不要速胜论,速胜论是“左”的根源。

(3)土改的“左”是侵犯中农问题,主要是违犯共产主义的基本原则的。对中农要体贴入微,所谓体贴入微是要问寒问暖。中农是我们永久的同盟者,侵犯中农利益来满足贫雇农要求,是挖肉补疮。中农是自家人。

争取百分之九十的农村舆论,许多同志无此观念,地主富农没有多少,不超过百分之八,高了要不得,要在同志中间确定此观念,机械一点好,才不致在战略上犯错误。超过了,打击面宽,危害革命基本利益。两个数目百分之九十与百分之八要确定。剥削者只有这样多,这是事实,不要怕“人家升官发财”,中农一般只有百分之二十,贫雇农百分之七十,不要乱了我们阵线。

对地主乱打乱杀,把一个好好的解放区搞乱。地主富农是个社会问题,弄得乞丐遍地,白骨如雪,还能领导中国。

城市是中小资产阶级问题,对中小资产阶级要有策略,在农村中是中农问题,在城市是中小资产阶级问题,两个问题弄好了,一定胜利。抗战时期是打大地主,中立中小地主,现在大中小地主当作阶级都要消灭,作为阶级消灭之,作为个人保存之。资产阶级也有分大中小,一定要争取中等资产阶级及小资本家,打大资产阶级,对中小资产阶级要放在保护之列。

今天对大资产阶级要打,对中小资产阶级要保护,这与抗日时期对地主一样,对中等资产阶级右派,政治上要打击,但经济上仍然要爱护(这里所说的小资产阶级是小资本家,与平常所说的资产阶级与无产阶级之间的小资产阶级不同,资产阶级与无产阶级间的小资产阶级,是革命的进步力量,中农是劳动者)。

城市对中小资产阶级,农村对中农,搞好了能胜利,这是革命的战略问题,对地主富农要加以区别,这种区别是我们领导思想上的事,在农民看来都是剥削者,没有区别,从消灭封建上说没有区别,扫地出门与征收财产有区别,在平分时农民恨地主,少分一点给他可以,农民少分一点就不行。农民赞成少给地主分一点可以,因它有底财等于平分,原则是地主要分土地与财产,绝对平分办不到,原则上也不妥当。中农不同意平分,无论如何不强迫平分,不要因浮财耽误时间太多,彻底平分,不是绝对平均主义(毛主席对人数户数很重视)。

(4)城市中对小资产阶级的冒险政策,必须避免,不要片面强调工人利益,搞起工商业对群众不利,无论公营、私营,共同生产与营业,计划,原料,成本,推销,工资都要注意。

职工会专管工人利益,不管整个利益是不对的,职工会在解放区管工人利益,还要管整个解放区利益,因为工人阶级是领导阶级。

城市应强调公私两利,劳资两利,发展生产,繁荣经济。

对学生知识分子不能有过“左”的冒险政策,过去以为有问题的,大都没有问题,以后到大城市不要乱来,延安过去枪杀的四十个,以为有问题,现在太行,大都是好的。

民主同盟一类在国民党区,不能说没有作用,这类团体对我们有很大作用与帮助,将来还准备帮助它。

双十宣言中,“其他爱国分子”即指开明士绅(其中也有毛病),何绍南那样的人,也要反对,那些人民对何绍南,曾起过作用,同我们共过患难的,现在不影响我们土地改革的,为何要一脚踢开。如此绝情寡恩,谁还给你共事?下层(村地富)可以不要,上层则留几个,有何不好?

三三制现在可以不提了,“二.一”指示还提了,因那时还感孤立,优势问题没解决,现在不提,但也不宣传取消,消灭了封建势力,官僚资本,粮食很多,让一点给人家吃,他们会来成了。

(5)打与杀的问题,不可不杀,不可多杀,一个不杀不是“永远”,群众不赞成一个不杀,有些要追到天涯海角的,凡可以不杀的皆不杀,延安审干一个不杀,大部不捉有好处。

(6)对外国九月二十七日的文件,反刘航深指示中说,现在反对杜鲁门,将来反对华来士这个作法不对,华来士反我联苏,代表一部分工人,代表大资产阶级不多,他不要战争,美国不喜欢,看这点还要联合他。

领导权问题,领导权问题有两个条件:

第一,坚决率领被领导者向敌人作斗争,并取得胜利。如在战争中不能胜利,则他们是动摇的,领导地位就要丧失。

第二,必须给被领导者以经济利益,物资福利与政治教育,没有这条不行,两者缺一不可,例如;领导中农,一方面领导贫雇农与封建势力作斗争,但还要给他们物质福利,这样中农才会跟着我们走。打美蒋要打得坚决,打胜仗,同时还要给人家物资利益,一切有利于国民经济部门,予以发展,两条缺一不可,缺一条即无领导权。

对中小资产阶级要联合他们,在我们领导下,反蒋反美我们不坚决打不行,但必要时,还要给他们利益,假如不照顾他们而破坏他们的利益,他是不会来的。

领导权不公开讲,坏处多于好处,易于模糊了群众与党员干部思想,要公开讲,现在九国宣言已公开讲了,今后我们也要公开讲,不讲思想上模糊不清。九国宣言讲了工人阶级的领导权(如冯玉祥声明,不反对共产党,永远跟共产党走),你承认不承认工人阶级领导权,不承认要说服,批评,是有理说服。

武装斗争是争取领导权的条件,武装斗争与争取领导权是分不开的,美国调处时,我们曾不缴一支枪,但曾经相信三国、三党以为能解决问题,但实际解决不了问题。四六年六月以后的谈判,只是教育群众的作用,并无解决问题之作用,那时我们确有拖几个月之诚意,退出几个地方,战略上有需要,但相信三国三党不对。

国际关系(美苏英美系)

三个命题,一般提法是破裂或者妥协。破裂意即;英美苏三国之三次大战;妥协意即,一切妥协。正确提法应该是这样。

(1)不是破裂与妥协的问题,而是早妥协与迟妥协的问题。这就是说最近期间无三次大战危险。三次大战是远景,三次大战危险充分存在,但不是马上打,不要以为他们不想打,但战争也充分可能使之没有,因此是迟早妥协问题。


很多人意见,以为苏联与美国一定破裂,三次大战一定发生,美苏妥协一点希望也没有。这种看法不对,是迟早妥协,美没有战争意思。(宣传战争是吓人的),苏联也没有战争意思,所以打不起来,潘友新到美国谈偿付一百三十亿美元借款问题时,美问潘,苏联是否要借十亿美元,潘答复,我们不主动借此十亿美元。潘又说现在的两国政府对立形势并不妨害两国合作。所以迟早妥协,战争是这一远景,民主势力强大,可以克服战争,迟早妥协,迟早妥协是民主力量强大的结果。

(2)(略)(3)(略)

国际关系补充(根据一九四七年十一月八日毛主席在政治小组发言)

(1)战争危险的可能性(略)

(2)消灭战争危险的可能性

……马歇尔对我们是:把我们塞进政府去加以溶解,可是蒋介石害怕。对我们是一面大胆谈判,一面死守不缴枪的原则,你打我也打,马歇尔计穷无法欺骗,就让蒋介石打起来。

故各大国妥协,不等于其他国革命与反革命妥协,各国反动派之所以能上台,主要是因为工人无武装,统一战线,未动员广大群众,订立条约束缚了自己的手,不发动群众拿住枪独立自主的干,则又不一样,大国条约不能束缚自己手脚,苏联同中国战略任务相同,具体作法不同。

当时法意两国是有取得政权的革命条件的,我们不能作此结论。但可以这样看问题,注意人民起来革命,英美是否敢向法意人民开枪,戴高乐在欧洲住参政会,法国共产党跑去参加,就束缚住了,如法共不去参加,就可独立自主了,法国是缴了枪的(只秘密地存留了一部),缴枪是错误的,吃了亏,希腊未缴枪,曾占过雅典,城市占不住,就下乡打游击,至今还在干,错误正在于受雅尔塔或波茨坦的迷惑,一吓倒,二迷惑,对英美力量估计太大,白劳德即是如此。

历史证明,对反动派只有坚决斗争,相信群众,放手发动群众,斗争一定能够胜利。一九二七年机会主义不敢斗争,很大力量也垮了,以后收拾了很小力量,坚持斗争,又斗了起来,于是又斗出了今天的局面,打了日本之后群众对美英蒋已经不怕,怕的是干部,希腊也如此。希腊有两个帝国主义干涉,但也斗了下去,斗出来了地方,所以只要依靠群众,依靠武装,亦能胜利,吓倒就是吃亏。

法共说,先打戴高乐,第二步打人民党,第三步打社会党,第四步再出来。法国以为有国会,主要靠国会方式和平斗争。现在注意似有转变,有革命形式或直接发展到革命,经过曲折到革命,法意罢工,训练群众,检阅了力量,可以再来。

从希腊经验来看,美英直接出兵作用不大,把两个纸老虎戳穿了,中国也要戳穿这个纸老虎。

因此,战争危险有两个必然性。

(1)帝国主义存在一天,战争不可避免,是一个必然性。

(2)人民必起革命,革命必胜利,又是一个必然性。

帝国主义要战争,但革命势力强大,可以克服战争的危险,又产生消灭战争的必然性。

帝国主义要轿夫,佛朗哥、蒋介石,希腊,土耳其皆不行,故找德、日作轿夫,但德、日已打败,也不会抬。希特勒若不先占领欧洲各国,东方若无日本作轿夫,他不敢攻苏。

要揭露战争贩子,战争只是远景,为了吓人说得如此近。不能上当,是宣传战,两年前斯大林即写丘吉尔是吓人,何以二年来并无战争,是吓人的。再过一、二年战争危险更少,就有可能消灭战争危险,不战即和,这二年是不战不和。不能设想总是不战不和,英美苏要和,长期并存是指英美,法意两个是革掉反动派的。

革命国家与帝国主义国家的关系。

帝国主义国家对革命国家一定要干涉,对希腊、意大利、法国,有英美干涉,对中国有美国干涉。

干涉的形式,或出兵或如现在的干涉中国,希腊即驻军,派顾问,供给武器,但不参战。干涉方式是多种多样,即出兵参战,亦可打迟。我们不去挑战只应战,如美军只驻青岛,不去京沪驻兵,则我攻下京沪后,可以软的办法取青岛,如美军驻兵京沪,则非直接冲突不可。

总之,帝国主义五痨七伤,不过如此,不可怕。

铁托把帝国主义说得最凶(因美已想着南),日丹诺夫比较缓和,莫洛托夫则骂战争,三种不同态度,表现很大策略意义。

在农业国家危机不要紧,在美国因其庞大危机来了不得了。

统一战线要研究南斯拉夫经验。

中间没有了,民盟解散了,致于解放区素无中间派。

中国还有广泛的统一战线,民盟解散了。不等于解散统一战线。

以为“民族统一战线一定要和许多党派并存,人民的统一战线则包括的少数”是不对的。

南斯拉夫一千五百万人口,有二十四个党,民族统一战线一概搞掉,法国则上了当,与各党派成立协定,自己束缚手脚。

中国无自由,无多党,但并非无团体和个人,如我们占领大城市,就都出来了,我们仍遵守解放军宣言,凡参加斗争者皆团结,在农村中除地主富农外,要团结中农,争取农村中人口百分之九十,城市则争取一切反帝反封建的人,口号是“民族统一战线”。

我们历来未与各党派订条约,只是与蒋介石有过条约,对现在不妨碍斗争,且还有帮助的,为何不要。

民主立场他们自己表明过,非国非共,我们对民盟帮助过,民盟也超过对我们的好作用。现在蒋加以解散,一面是被迫解散,一面是自己屈服失掉威望。我们要批评它,但还要帮助它。

三三制以后不提,亦不声明作废。

中等资产阶级右翼要打掉,但并非作阶级来消灭。

现在蒋很孤立,我们圈子很大。

联合政府已取消。因工农政府是联合政府,党与非党联盟是联合政府。

过去讲统一战线,就是站在地主富农党员的立场去联合别人,而非站在工人农民党员立场去争取革命士绅,原因是许多共产党员,本身是地主富农,×××又是四大家族,都是党员,他们去讲统一战线,是党内的地主富农与非党的地主富农的联合。

统一战线应站在贫雇农立场,联合中农,抗日时期包括开明士绅,抗日时期对地主不是联合而是中立,同时消灭削弱其一部分。

对蒋也很难说是联合,而是孤立国民党,孤立顽固派,故有矛盾,又要国共合作,又到处破坏和孤立对方,蒋对我,我对蒋,都是互相孤立,互相破坏,消灭那消灭得了的,对整个国民党无可奈何,只好说是合作。而实际上我们在国民党区,所作的是对国民党打击政策,在可以消灭之处则消灭之。因为多年采取了消灭打击的政策,我们才有××,故所谓合作是:

①凡能消灭者,消灭之。

②凡不能消灭者,打击之,部分消灭,全部打击。

今后国民党没有了,只有小集团。

大国妥协,迟早妥协,是个方向,以后研究。

有些问题能妥协,有些问题不能妥协,日德和约迟早要妥协。不战不和僵局有些时间,战争是远景,美国出国商量要求妥协,潘友新说:“两国形势对立,不妨碍两国谈判”,苏不主动提出借款,可看出两国要妥协。马歇尔说“不愿订纸上协定,要看今后情形决定。”

陈问毛主席“是否欧美各国也有像中国一样恐惧美国对自己估计不足的情形?”毛主席说日丹诺夫已经分开讲了,东北野战军,也有怕原子弹,怕美国出兵的。粟裕的报告也说部队有些情形,地方党也有此种情绪。中国对此准备很早,整风时讲过,七大时讲过。因那时有人说反动势力强大,我们即讲讲此问题。

害怕美帝国主义是一种精神作用,中国年年与美帝国主义斗争,得到失败,故精神有些害怕,打垮了日本帝国主义以后,群众不怕英美了,干部还害怕,尤其是地主富农分子害怕。苏联哲学上的清算,文艺上的清算。文艺上搬外国搬死人,喜欢美国铲头,美国的纸烟,对伟大的现实主义看不起,这就是投降主义,是由于战争受了创伤,精神上也获得解放,怕纸老虎。现在应在干部中解释清楚,我们在江南大发展时,美国人还会有几手凶的吓你一跳,要吓倒你,你要被他吓倒,就上了他的当了。吓不倒他就算了,炮打×基打×县等,还可能重演。有些人谈帝国主义就好像谈虎色变,一个是思想根源,二个是精神根源。我们历史上有失败的经验,苏联是新国家,也有此情形。这是因为帝国主义有长期的历史,被帝国主义侵略的国家,传统的怕帝国主义,有传统的力量,现在我们五百人直属队中,我们的窑洞中即有此种人,害怕战争,害怕帝国主义,不相信自己的力量,不相信群众的创造性。陈伯钧对少将衔感到心得意满,看不起红军的光荣,这是个精神问题,要从这方面解放出来。

周恩来同志说:陆××同志关于国际形势文件发出时,英共正开代表大会已闭幕,收到此文章后,特休会一天,举行会议,宣读此文章,又根据此文章修改他们的决议案。

世界无产阶级消灭世界资产阶级,是件大事。

要搞东欧联邦就是“左”,胜利时即易于“左”。(季、铁均有此主张)

某种情况轻敌,某种情况不轻敌。

民主阵线发展前途可以消灭战争,部分接触可能有,印度尼西亚情况可能上当,因为共产党不占主要力量。越南保大投降法国,胡志明没有军火,主要阵地失掉,在乡村打游击,只有将来我们打到两广时,援助他们。

所有这些原因是一系列的,精神还未从旧的传统中解放出来。

对美蒋主要是干部问题,要很好准备,要轻视美国,美国惟一骄傲是有钱,但也有限,究竟拿得出多少来,不是什么了不起,要有斗争勇气,将来到江南时,不向美国挑衅,首先,还是退避三舍,然后还是狠狠的打他一下,他就走了。

东欧联邦问题,真理报说“是人为的,没有群众基础。”毛主席说“大概是环境顺利,有些胜利冲昏头脑,另一方面忽视现实。东欧联邦之提议没有别的好处,只给搞西欧联盟者以借力,主要是没有群众基础,共产党是与社会民主党合作,如搞联邦,社会民主党不赞成怎么办?策略上有危险性,英美作借口,还是次要的,无群众基础是主要的。”

结论

开会时期思想动态,有这样思想情况,西北晋绥有些同志,坚持“左”,杀人,整三三制人物,侵犯中农,破坏工商业等等。主席说:“有人说安文钦创议是地主立场坚决,我就说取他地主立场坚决,不坚决就没有了。人家是地主,为什么不代表地主立场”。有人主张把社会主义前途加进去。毛主席说“这是急性病,人家一九一七年十月革命过了十五、六年,到一九三二年才正式搞社会主义。今天我们还是消灭封建主义,还早着呢!何必提社会主义。”

还有人说,三次大战一定爆发,打起来不得了。

有一种空气似乎说抗战八年都错了,吸收知识分子错了,减租减息,吴满有运动错了,要恢复到内战时期。事出有因,尚无实据,我嗅到了。

①很高兴的一次会,二十年来解决的优势问题,今天解决了,局面开展,胜利可期,洛川会议,六中全会都未解决这个问题。

以前只能讲“有利于我”,现在可讲“胜利到手”,特别是三次反共高潮到日寇投降,一直形势严重,我们处于只有招架,没有还手的地位。现在我们强大了。国民党处于劣势地位,革命高潮到来了,我们是在进攻了。现在的局势确已改观,胜利前途不但领导机关人感觉,群众感觉,非解放区党外人士感觉,外国人都感觉到了。很高兴的一个会议。虽然工作中是有严重缺点,困难也很多,但都可以解决,这是很成功的一次会议。

北伐时期,局势也很开展,但优势问题未解决,反而失败了。土地革命时期战争频繁,党内纠纷太多,一直到长征是革命最大难关,幸而渡过。现在确是起了根本变化,今日形势对全世界都有意义。

这一时期的确是兢兢业业的,很担了一分心,特别日本投降后到重庆,那时蒋介石的事业好办,我们的事不好办,的确是兢兢业业领取得了现在形势。现在不同了,现在不是胆战心惊了。现在能作出结论,不是估计而是事实。过去总是估计“有利于我”,或者是说“可能”,现在事实确是如此,我们优势到胜利,蒋介石翻过来,无法打我们翻天印了,甚至在日本投降时,我们还是一则以喜,一则以惧,喜的是日本投降了,惧的是优势问题未解决,东西得的少,蒋介石仍强大,严重的内战临在头上,成败两个可能完全在斗争。现在好了,可以肯定的讲了,高兴了,前线不断的胜利,土改整党走上正规(过去未解决,不知道路),九月土改会议,即整党会议不是偶然的,现在确定了优势,二十年的艰苦奋斗,今天走完了,确立优势不是估计,而是事实。

②西北与晋绥两个中央局,政治上领导成熟,比过去更满意,更有希望,告农民书有小毛病,应进行教育。义和镇会议是满意的,两区大体相同,但有区别,陕北是反右,晋西北、是反“左”,不管“左”右不妨碍两区领导政治上成熟,即土改已开始解决了,路线是正确的,西北已经完成一个任务,在内线打破了胡匪,使其转入防御,把两区变成后方,支援前线,解放大西北,问题更有基础,更有根据,使我们更有希望。

⑧会议解决什么问题?要解决这篇文章(指目前形势与我们的任务)这篇文章是当作一个时期的政治纲领,打倒蒋介石建立新中国的纲领。比“新民主主义论”、“论联合政府”更进一步,致于经济纲领打倒蒋介石以后,还要照此路线执行一个时期。

这篇文章所要解决的问题:

甲、宣布美国是帝国主义是肯定的。

乙,宣布蒋介石是官僚资本,要打垮他是肯定的(有的人赞成打倒蒋介石,不赞成打倒官僚资本,不懂得蒋介石是代表,官僚资本是基础)。

丙、推翻地主阶级不错的,有人说地主是蒋根是小蒋,这样说法有毛病,我们要打倒蒋介石集团,蒋介石是这个反革命集团的代表,不能像把打倒蒋介石一样的力量去打三千六百万地主,要讲政策,推翻地主阶级是肯定了的,对三千六百万的人,不是都用此处理方法,是有策略的。

这三条不错,其余就不会错。中农问题早已解决,只是在处理问题出了毛病,注意纠正就是了。

中、小资产阶级,中小资本家现在还是不是同盟军,是不是反蒋统一战线,这个问题已经解决了,应肯定。

论文第四条,关于土改中中农问题,原则是老的,但对中农如此强调是非常必要的。

论文第五条,民族资产阶级问题也不是新的,但现在集中力量打击大资产阶级。官僚资本是个问题,故有作用。

土改会议后,反右问题解决了,现在解决了“左”的问题,此次会议不是重复,此次会议要注意的是:(1)中农,(2)中、小资产阶级,(3)党外人士。

中农问题不是要不要的问题,这个问题早已解决。大家都知道不能侵犯中农,但是实际工作仍有侵犯中农利益的,现在是具体分析阶级的问题,太行山分析阶级的文件无大害,但有毛病不清楚,有些不恰当,可见马列主义武器之少,晋西北原有一个分析阶级的文章,是正确的,但自己烧掉了,也是马列主义武器不多。

问题:是在三交冒出的是“左倾机会主义”,在义合冒出的是右倾机会主义。九十九户中弄出十多户地主富农,不要中农参加农民代表会,我看了惊心动魄。

肯定的说,只有两个朋友:中农与中、小资产阶级。

孙家沟也如此,他们强调不要中农,不是强调要中农。

中、小资产阶级问题这是此土地会议要新的:

(1)地主资本家的工商业非封建剥削的不要搞。

(2)勿使税收过高,劳动条件不要太高。

例如:过去陕北,曾津贴一个地主开的纸厂,现在不津贴了,李家渠繁荣,是地主的繁荣,要搞垮,怪不怪?李家渠的繁荣是解放区的繁荣。林老对此问题不能解决你看怪不怪?基本问题是对中、小资产阶级发生动摇。土改就是要造成这种繁荣,在这上面要潮流,把沙推掉,切勿把中农,中、小资产阶级推掉,这种潮流任其发展下去,是很危险的。我们要逆此潮流,且要反掉,把此潮流淹没是不好的。

党的经验痛是苦的,与资产阶级合作时,主要是右的危险,与资产阶级分裂时,主要是“左”的危险,我要反,宁可反到只剩下我一个人,保持光荣的孤立。

党外人士问题:

刘少白消息发表了,我简直看不下,我不要看,我未解决。刘少白是党外人士,谢老说:“两重性如何解决?刘少白有两重性,即代表恶霸地主,又要反蒋,他今天要进步,与我们一块七八年,有功,今天一脚踢开,这样下去一定会使我们孤立,遭受失败。我们与国民党统战一次,不但得到经验教训,而且得到有生力量的教训,就是国民党中一些人与我们熟了,这在打日本时期有用处,此次不要像土地革命时期,一切都不要了,把刘少白也不要了,把他的土地浮财分了都可以,为什么乱斗呢?你的政策是什么?

从地主阶级身上应当:(1)当作阶级消灭之,(2)当作个人如不反我们要收集。

可留者:

(一)过去合作而非反我的继续合作,过去反我今后愿意合作的,也不拋弃。地主富农人口三千六百万,吃了鱼肉,剩下鱼骨头,还有用处,鱼骨头有三干六百万到一千八百万收集起来有用处。三干六百万是个大劳动力,不可轻易抛弃。

(二)当作阶级压在地下,但阶级组成的国家,是压迫别的阶级,领导者要知三千六百万还是财富,应当利用,地主阶级的人还有作用,现在是生产力,将来是朋友(五年),是候补无产阶级,现在要强迫改造,将来会改造好。

要分别对待,只要赞成土地法,高级人员(1)分而不斗,(2)斗而不打,个别犯法,依法处理,如此解决,人心安定,理直气壮。

过去非全面解决问题。抗战时“三三制”是成功的,其中吸收其成员个别的不恰当,当然也有地主阶级思想,应社步,统一战线与之合作。

打人肉刑要废除,肉刑这个问题早已解决,我{门在红四军九次大会已解决,以后又打,应坚决废除之。肉刑不能许可,原则性灵活性只一例外,即真正出于群众义愤,坚决打时,我们不能制止。

财经问题:陕北、晋西北财经困难,灾荒很大问题,要努力解决此问题,条件是退敌打出去,收复临汾。

宪法:中央起草的,勿急于发表。现在发表个纲领,胜利后再搞,如不恰当会被孤立,支票不应开早了。

是否成立中央政府,平绥路收复以后再考虑。瑞金搞中央政府不需要。颁布宪法,成立中央政府,不要忙。

第二部分一一陈毅同志记述毛主席的一些谈话。

(一)今天反“左”与过去土地会议要结合起来看,一月决定迟迟未发表,要各地酝酿,以免反“左”,泼了冷水又恢复了右。

毛主席反复考虑反“左”是不是会泼冷水。

去年土地会议把右反掉了,是最伟大的一次整党,把石头一搬,增进了党与人民联系,把混入党的地主富农分子及党员的地主富农思想洗掉了,把群众发动起来,通过了一个完整的土地法,是保证胜利的一个伟大的会议。只有在此基础上来谈反“左”问题,才不会泼冷水,如孤立的反“左”,那就会泼冷水。

去年土地会议,整党会议,是黄河主流,是一个伟大的潮流,达到彼岸的潮流,一直流到大海,是成功的。但仅讲到这还不够,主流向支流时卷了三个浪花,即侵犯中农利益,破坏工商业,把党外人士一脚踢开。不把这三个浪花反掉,它会成为逆流,要把这三个浪花反掉,要把三个过左倾向纠正,是一个完整的精神,反“左”不使右的倾向复辟,不使他们有词可借,说你们搞“左”了,解决了右的问题再来纠正“左”,不致使右再来影响政策,“左”的纠正得好,更能保证右的不能复辟。

中农问题,为了满足贫雇农的要求而侵犯中农利益,是挖肉补疮的得不偿失。正确的是既照顾贫雇农又照顾中农,若照顾中农,不照顾富裕中农及满有式的新富农仍不能稳定中农。要告诉贫雇农,一定要与中农、富裕中农、新式富农搞好。你们之前途就是新式富农。土地不能解决贫困,只能解决一部分问题,某村贫雇农真正下不来台,无法生活,可说服中农给贫雇农调剂一下,但不能采取强迫的办法,维持与拥护土改主流,要对富裕中农、新富农加以照顾。“家徒四壁”的村庄,政府要设法救济,也不一定要搞新富农,富裕中农与中农。

对中小资产阶级,不要采取冒险政策。

工运问题未解决,二十年来工运方针有“左”倾传统,对蒋管区的工业,也不是一概破坏的方针,对官僚资本家的工厂,可以罢工反对,破坏他对中小资本家还是劳资两利的方一针,以便渡过难关。

解放区应调整工人领导地位,除了工人福利外,还有更大的利益,还要照顾整个解放区的利益,要说服工人忍受剥削。

如我们占了西安对裕华耖厂、六元大精盐公司(战前二千万元以上的资本)要不要斗,只例外,若与四大家族关系太深,是战犯的,才能没收。此外都不要搞,战犯也要看程度,还要打南京、北京,忙得很,那里有时间去搞他们。

对党外人士,有些十年、二十年共过患难的朋友,今天一脚踢开,觉得自己聪明,别人愚蠢。我们同志会打军事仗,军事仗大堡垒、小堡垒、内壕、鹿寨,许多外围,政治仗,不要外围,自己挺着打,蒋介石把外围搞光了,所以容易打,将来还要另开一席,只在土地问题上不让步,任何人的地都要分(他们同样也分一份),严守此条,不要追求其他。

这三个问题处理上,又发生了打人杀人的现象靠打多杀多解决问题,表示共产党无本领,有本领的把他们放在群众监督下,不怕你反,三千六百万是个本钱(劳动力)不能一下杀掉,不是同情地主,主要是群众不同意,你杀脱离中农、贫雇农、中小资产阶级、知识分子也危险了,工商兵还剩下多少。

你到北京,胡适捉不捉,到了南京,戴季陶、于右任、孙科就是战犯,你捉不捉?还是不捉,可以胡适当个图书馆馆长,革命到了南京、上海,你还捉他们干什么?

乱杀人是表示无本领,无纪律。

破坏庙宇、名胜、古物不好,共产党要表示有纪律有政策,政策有策略,何为当务之急了谁都可以当家,一个人随便杀人,想必多个领导机关委员会?这个问题要统一处理,从容处理。抢救运动中群众对特务发生义愤,要杀要打是领导艺术问题,要杀也一定要经过法律手续,群众真要打,也不一定与群众为敌,抢救中没开杀戒。杀机已动,领导不能随声附和,众怨之下动摇,一个环子抓不紧,整个链子就松了。那时天天看材料,明明是特务杀不杀,反复考虑,最后,决定不杀,好的领导艺术就在这里。

(二)十二月文章发表后的新区情况

新区一开始即搞平分土地,群众不接受,因新区军事未稳定,群众未觉悟,费很大力量,效果小。第一次分浮财,土地,我们一走,地主又起来杀工作人员,我们再去,群众不理。所以在新区机械的把土地塞给农民,农民不接受。

老区平分土地不适宜,特别是新富农,富裕中农,中农土地多,平分必然侵犯中农,故引不起兴趣。在老区工作团超过支部直接去发动群众,把过去八年斗垮了的地主富农,以新雇农的面目出现,把他们组织起来,反对老干部。由于这些情况构成一种新观念,即一月会议所决定的要反三个浪花,决定土改要分地区,需分新区、半老区,釆用不同步骤,土地要分别施用,不要普遍使用。

老区一般不要平分土地,老区一般是日本投降以前的老根据地,经过八年抗战,减租减息,反奸清算,五四指示,土地复查等等。土地问题已基本解决了,只是剩余的土地问题,基本上采取抽补、调剂的办法解决。这样地区对旧干部加以改造,主要是民主运动与生产运动问题,整党问题,撤换个别坏干部。这样地区中农占多数,贫雇农很少,地主已打垮,生产运动要照顾新富农,这种地区山东就有二十个县(共一百零八个县),晋西北四十五个县,这种地区只抽补土地,有些地主刚要改变成分(四二年起)是农民不是地主。特殊地方平分,一般地方抽补,以与群众联合的方式整党,发展经济,民主运动。

第二种地区,是日本投降大进军时才落到我手者,经过四五年八月底到现在,二年半,五四指示不彻底,基本上使用九月会议的一套,超过支部,甚至解散一些支部,石头搬掉彻底平分,基本使用土地法,山东由七百多万人口增加到三干多万,地主富农占优势。党内不纯,借党为恶。中贫农未得到土地,还未确立群众与党的优势。部分地方比较彻底者,也不机械平分。

第三种地区,是去年八月以来落入我手者,叫新区(前一种地区也叫半生半熟区),例如陇海以南,豫皖苏,豫鄂陕等区,主要是军事斗争。地方无党,敌我斗争军事穿插拉锯,土地革命条件仍未具备,不能立即实行土地法,应有步骤的进入土改,开始应集中力量打大地主,大官僚资产阶级,争取中小地主,知识分子,保护中小资产阶级,利用地主富农左派打地主富农右派,中立中小地主,打大地主。吸收地主富农子弟好的入党,套在我们圈子里慢慢改造,以孤立蒋介石,地方组织起游击队,群众觉悟及顾虑少了,有秩序的进入土地平分。

分别不同对象,以不同步骤来达到土改目的。

平分是战略任务,要以不同的方法达到。

老区与半老区准备三年完成,新区准备五年完成,全国准备十年完成,不要性急。

才到新区群众热烈欢迎,打土豪分浮财,人家就走了,不热烈了,这是硬土地法的结果。

减租减息政策很有不少作用,可以争取地主富农左派打击右派,到大城市对地主富农分子好的可大量吸收。

应使土改与反美反蒋结合起来。

有些人把内战时的一切东西都恢复,把抗战时学到的许多东西一下子去掉,真奇怪。

几个问题:

(一)领导干部要破除迷信。

教条,宗派,迷信,马列主义词句,盲目性发展到高度固定化就成为迷信。

打胜了以为从此天下无敌,是迷信武力可以解决问题。

狭隘经验主义也是迷信,迷信其经验可以解决问题,而不对当时具体情况加以分析。

“马、恩、列、斯、毛”(指马、恩、列、斯、毛论农民土地问题那本书说)也是迷信,以为合乎那儿条就可以解决问题。

要破除迷信,要对具体事物加以详尽分析,马列主义就是具体问题具体分析,自觉增加了,就能主动的解决问题。我们同志多少有点迷信,抽象条文及历史陈旧毛病,打人杀人如此多,就是迷信打人杀人可以解决问题。

(二)党内有小集团主义,本位主义,宗派倾向,领导同志的态度问题,思想意识问题,主张偏见(各同志各单位之间还不是很原则的)问题等等。

要告诉同志们,你们要批评一个人,坚持一个原则,只有这种权利,不要放松,要有政治上的勇气,反对你所认为不妥当的,但同时釆取合法手续,赞助你,拥护你与你合作,首先有此态度,有此基础又提出严厉批评。先取得批评资格,然后批评方才不致引起对立,上级对下级也是如此。是赞成的鼓励的,足够估计成绩,然后取得批评的自由与地位,否则无此资格,还会引起反抗,有些不是造成对立,就是不批评不讲,但不批评就不能进步,先要拥护,先赞成,先讲成绩然后再批评,这样就一定能达到团结,许多人发言就是放人家的炮。

光摸别人的老虎屁股,怕别人摸自己的老虎屁股,不能唯唯喏喏,要反潮流,任何时候要有明确的政治态度,紧要关头一定要表示自己的政治态度,但要注意合法手续。

(三)对人的看法,有时说此人有缺点,或此人有好处都不对,应看到其好处,也要看到其缺点,少说此人有好处但有缺点,或此人有缺点但有好处。张国焘是个大坏蛋,但他有点好处,即不性急,这样看法丢掉他,还获得某种价值。

(四)党内一种人尺度太宽,一种太狭,量的不合就不要,拉胡琴弦子太紧,调子太高,弦子要断的,过分严紧,使人无所措手脚。应当是“德小逾闲,小德出入可以也。”

有种人灵活但不严肃,不锐敏,好处是宽,坏处是在逆流时懦弱。应当尖锐,敢于反对逆流。孔子说“吾党之小小狂捐,狂者进取,狷者有所不为。”要有进取心,狂者大胆,尺度宽,放得开手,但圈子不紧,狷者馑小慎微,两者应中和起来。

(五)一定要定期作报告,执行两月报告制度,要亲自写,要求反映实际,提出问题,解决问题,将要所解决的问题,打电报同中央商量。过去情况,一种是不作报告,把总公司给忘掉了,一种是报告零碎,看后令人茫然;一种是讲好的多,不讲缺点,只报喜不报忧,看了仍然茫然,讲好的多放心,忽然出现了一个乱子,则手足无措。报告可帮中央解决问题,又取得中央的帮助。

