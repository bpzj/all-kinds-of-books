\section[关于土地政策的发言要点(一九四六年五月八日)]{关于土地政策的发言要点}
\datesubtitle{(一九四六年五月八日)}


中央关于土地问题的指示,已开始电发各地,在讨论和通过这一指示时,毛主席在发言中有最重要的几点,值得全党同志注意,兹摘要通知如下:

(一)七大时说:“寻找适当方法解决土地问题,实现‘耕者有其田’”。中央五四指示就是这种为群众所创造,为中央所批准的适当方法。现在类似大革命时期,农民伸手要土地,共产党是不批准,必须有坚定明确的态度。

(二)在政治上十分需要。目前国民党有大城市,有帝国主义的帮助,占有四分之三的地区。我们只有依靠广大人民群众的伟大力量与斗争,才能变这种他大我小的形势。如果在一万万几千万人口的解放区内解决了土地问题,即可使解放区人民长期支持斗争不觉疲倦。

(三)这是一个最基本的问题,是一切工作的基本环节,必须使全党干部认识其重要性。

(四)由于广大群众的行动推平了土地(即平均分配)的地方,不要去批评农民的平均主义。相反地农民这种彻底消灭封建势力的行动应该批准。但无休止地推平,不联合中农的推平,不照顾各色人等的推平,就要不得,群众未提出推平的地方照群众所提的方法办理,也不要推平。

(五)不要怕自由资产阶级和中间分子暂时的动摇,只有我们坚决实行土地改革,使农民得到土地,我们之力量更加强大巩固时,我们才能有力量,更可能争取团结他们。但对自由资产阶级及中间派应作正确而有力的解释,指出减租与耕者有其田都是实行政协决议,其方式又与内战时期大不相同。

(六)对工商业政策和工人运动必须与土地政策、与农民运动有原则区别,切忌工资及其他劳动条件订得太高,应该是劳资合作,要订出共同生产计划(原料足,产品多,成本低,质量好,销路广)努力生产,使生产发展,经济繁荣,劳资两利。只有如此,才能与外国和本国的垄断资本作斗争,使解放区工商业的发展立于不败之地。

