\section[答路透社记者――中国需要和平建国(一九四五年九月十二日)]{答路透社记者――中国需要和平建国}
\datesubtitle{(一九四五年九月十二日)}


九月二十七日新华日报发表路透社驻重庆记者甘贝尔书面提出的十二项问题,请中国共产党中央委员会主席毛泽东同志答复。问题与答复如下:

(一)问:是否可能不用武力而用协定的方法避免内战?

<b>答:</b>可能。因为符合于中国人民的利益,也符合于中国当权政党的利益。目前中国只需要和平建国一项方针,不需要其它方针,因此中国内战必须坚决避免。

(二)问:中共准备作何种让步,以求得协定?

<b>答:</b>在实现全国和平、民主、团结的条件下,中共准备作重要让步,包括缩减解放区的军队在内。

(三)问:中央政府方面须作何种妥协或让步,才能满足中共的要求呢?

<b>答:</b>中共的主张见于中共中央最近的宣言,这个宣言要求,国民党政府承认解放区的民选政府与人民军队,允许他们参加接受日本投降,严惩汉奸伪军。公平合理的整编军队,保障人民自由权利,及成立民主的联合政府。

(四)问:你对谈判会达成协定甚至暂时协定一事,觉得有希望吗?


<b>答:</b>我对谈判结果,有充分信心,认为在国共两党共同努力与互相让步之下,谈判将产生一个不止是暂时的而且是足以保证长期和平建设的协定。

(五)问:假若谈判破裂,国共问题可能不用流血方法而得到解决吗?

<b>答:</b>我不相信谈判会破裂,在无论什么情况下,中共都将坚持避免内战的方针。困难会有的,但是可能克服的。

(六)问:中共对中苏条约的态度如何?

<b>答:</b>我们完全同意中苏条约,并希望它的彻底实现,因为它有利于两国人民与世界和平,尤其是远东和平。

(七)问:日本投降后,你们所占领的地区,是否打算继续占领下去?

<b>答:</b>中共要求中央政府承认解放区的民选政府与人民军队,它的意义只是要求政府实行国民党早已允诺的地方自治,借以保障人民在战争中所做的政治上、军事上、经济上与教育上的地方性的民主改革,这些改革是完全符合于国民党创造者孙中山先生的理想的。

(八)问:如果联合政府成立了,你们准备和蒋介石合作到什么程度呢?

<b>答:</b>如果联合政府成立了,中共将尽心尽力和蒋介石合作,以建设独立、自由、富强的新中国,彻底实行孙中山先生的三民主义。

(九)问:(A)你们行动和决定,将影响到华北多少共产党员?(B)他们有多少是武装起来的?(C)中共党还在什么地方活动?

<b>答:</b>共产党的行动方针,决定于党的中央委员会。中共现在有一百二十余万党员,在它领导下获得民主生活的人民现在已远超过了一万万。这些人民,按照自愿的原则,组织了现在数达一百二十万人以上的军队和二百二十万以上的民兵,他们除分布于华北各省与西北的陕甘宁边区外,还分布于江苏、安徽、浙江、福建、河南、湖北、湖南、广东各省。中共的党员,则分布于全国各省。

(十)问:中共对“自由民主的中国”的概念及解说如何?

<b>答:</b>“自由民主的中国”将是这样一个国家,它的各级政府直至中央政府都由普遍平等的无记名的选举所产生,并向选举他们的人民负责,它将实现孙中山先生的三民主义提出的民众民治民意的原则与罗斯福的同志自由。它将保证国家的独立、团结、统一及各民主强国的合作。

(十一)问:在各党派的联合政府中,中共的建设方针与恢复方针如何?

<b>答:</b>除了军事与政治的民主改革外,中共将向政府提议,实行一个经济及文化建设纲领,这纲领的目的,主要是减轻人民负担,改善人民生活,实行土地改良与工业化,奖励私人企业(除了那些带有垄断性质的部门应由民主政府国营外),在平等互利的原则下欢迎外人投资与发展国际贸易,推广群众教育,消灭文盲等等,这一切都是与孙中山先生的遗教相符的。

(十二)问:你赞成军队国家化,废止私人拥有军队么?

<b>答:</b>我们完全赞成军队国家化与废止私人拥有军队,这两件事的共同前提就是国家民主化。通常所说的“共产党军队”按其实际乃是中国人民在战争中自愿组织起来而仅仅服务于保卫祖国的军队,这是一种新型的军队,与过去中国一切属于个人的旧式军队完全不同。它的民主性质为中国军队之真正国家化提供了可贵的经验,是为中国其他军队改进的参考。

<p align="right">(《解放日报》1945年10月8日)</p>

