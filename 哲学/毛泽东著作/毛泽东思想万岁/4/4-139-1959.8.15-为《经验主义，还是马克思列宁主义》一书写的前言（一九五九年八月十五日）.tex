\section[为《经验主义,还是马克思列宁主义》一书写的前言(一九五九年八月十五日)]{为《经验主义,还是马克思列宁主义》一书写的前言}
\datesubtitle{(一九五九年八月十五日)}


各位同志:

建议读这两本书。一本哲学小辞典(第三版),一本政治经济学教科书(第三版)。两本书都在两年读完。这里讲《哲学小辞典》一本书第三版。第一、二版错误颇多。第三版好得多了。照我看来,第三版也还有缺点和错误。不要紧,你们读时还可以加以分析和鉴别。同政治经济学教科书一样,基本上是一本好书。为了从理论上批判经验主义,我们必须读哲学。理论上,我们过去批判过教条主义,但是没有批判经验主义。现在主要危险是修正主义。在这里即出了《辞典》中的一部分,题为《经验主义,还是马克思列宁主义》,以期引起大家读哲学兴趣,以后可以读全书。至于读哲学史,可以放在稍后一步。我们现在必须作战,从三方面打败反党反马克思主义的思潮:思想方面、政治方面、经济方面。思想方面即理论方面。建议从哲学,经济学两门入手,连类而及其他部门。


