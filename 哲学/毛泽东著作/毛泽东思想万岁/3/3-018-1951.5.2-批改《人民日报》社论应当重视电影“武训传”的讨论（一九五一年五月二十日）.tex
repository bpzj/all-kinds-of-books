\section[批改《人民日报》社论应当重视电影“武训传”的讨论(一九五一年五月二十日)]{批改《人民日报》社论应当重视电影“武训传”的讨论}
\datesubtitle{(一九五一年五月二十日)}


[为什么]<font face="黑体">应当</font><font face="宋体">重视</font><font face="黑体">电影</font><font face="宋体">《武训传》的讨论(?)

在发表杨平同志“陶行知先生发扬‘武训精神’有积极意义吗?”一文时,我们说希望因此引起对于</font><font face="黑体">电影</font><font face="宋体">《武训传》的进一步的讨论。为什么[要]</font><font face="黑体">应当</font><font face="宋体">重视这个讨论呢?

《武训传》所提出的问题带有根本的性质。</font><font face="黑体">象武训那样的人</font><font face="宋体">,处在[清未农民反封建革命]</font><font face="黑体">清朝末年中国人民反对外国侵略者和反对国内的反动封建统治者的伟大斗</font><font face="宋体">[战]争的时代,根本不去触动封建经济基础及其上层建筑的一根毫毛,反而狂热地宣传封建文化,并为了取得自己所没有的宣传封建文化的地位,就对</font><font face="黑体">反动</font><font face="宋体">的封建[阶级]</font><font face="黑体">统治者</font><font face="宋体">竭尽奴颜婢膝[之]</font><font face="黑体">的</font><font face="宋体">能事,这种丑恶的</font><font face="黑体">行为</font><font face="宋体">[奴隶“文化”、奴隶“道德”,]难道是我们所应当歌颂的吗?向着人民群众歌颂这种丑恶的</font><font face="黑体">行为</font><font face="宋体">[奴隶“文化”、奴隶“道德”],甚至打出“为人民服务”的革命旗号来歌颂,甚至用革命的农民斗争的失败作为反衬来歌颂,这难道是我们所能容忍的吗?承认或者容忍这种歌颂,就是承认</font><font face="黑体">或者容忍</font><font face="宋体">[历史唯物论的灭亡。就是承认超阶级的“文化”“道德”的胜利,承认奴隶的愚民改革的胜利。]</font><font face="黑体">污蔑农民革命斗争,污蔑中国历史,污蔑中国民族的反动宣传为正当的宣传。

</font><font face="黑体">电影</font><font face="宋体">《武训传》的出现,特别是对于武训和电影《武训传》的歌颂竟至如此之乡,说明了我[们关于历史唯物论和新民主主义的学习还是极其肤浅的。这就给我们一个重要的任务,就是要利用《武训传》来学习,来清醒我们的头脑,来扫除我们思想中的唯心论和改良主义。]</font><font face="黑体">国文化界的思想混乱达到了何等的程度</font><font face="宋体">,试看下面自从</font><font face="黑体">电影</font><font face="宋体">《武训传》放映以来,北京、天津、上海三个城市</font><font face="黑体">中</font><font face="宋体">报纸</font><font face="黑体">和</font>刊物上所登载的歌颂《武训传》、歌颂武训,或者虽然批评武训的一个方面,仍然歌颂其他方面的论文的一个不完全的目录:

(目录说明)

[因为武训能有这样的意义和影响,所以我们认为认真深入这个讨论是必要的,我们特别希望在这个问题上发表过意见的人来参加这个有重要意义的讨论。]

在许多作者看来,历史的发展不是以新事物代替旧事物,而是以种种努力去保持旧事物使它得免于死亡;不是以阶级斗争去推翻应当推翻的反动的封建统治者,而是象武训那样否定被压迫人民的阶级斗争,向反动的封建统治者投降。我们的作者们不去研究过去历史中压迫中国人民的敌人是些什么人,向这些敌人投降并为他们服务的人是否有值得称赞的地方。我们的作者们也不去研究自从一八四○年鸦片战争以来的一百多年中,中国发生了一些什么向着旧的社会经济形态及其上层建筑(政治、文化等等)作斗争的新的社会经济形态,新的阶级力量,新的人物和新的思想而去决定什么东西是应当称赞或歌颂的,什么东西是不应当称赞或歌颂的,什么东西是应当反对的。

特别值得注意的,是一些号称学得了马克思主义的共产党员。他们学得了社会发展史一历史唯物论,但是一过到具体的历史事件,具体的历史人物(如象武训),具体的反历史的思想(如象电影《武训传》及其他关于武训的著作),就丧失了批判的能力,有些人则竟至向这种反动思想投降。资产阶级的反动思想侵入了战斗的共产党,这难道不是事实吗?一些共产党员自称已经学得的马克思主义,究竟跑到什么地方去了呢?

为了上述种种缘故,应当展开关于电影《武训传》及其他有关武训的著作和论文的讨论,求得彻底地澄清在这个问题上的混乱思想。

注:方括弧内是主席删除的,黑体字是主席加的。


