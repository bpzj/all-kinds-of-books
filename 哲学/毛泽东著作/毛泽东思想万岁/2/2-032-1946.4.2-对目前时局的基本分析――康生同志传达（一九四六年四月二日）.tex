\section[对目前时局的基本分析――康生同志传达(一九四六年四月二日)]{对目前时局的基本分析――康生同志传达}
\datesubtitle{(一九四六年四月二日)}


今天不是我作报告,而是将毛主席对时局的指示与任务作一个传达,日本投降以后,日意德三法西斯国家打倒之后,二次大战胜利之后已八个月了。这八个月中,一方面看到法西斯势力打倒了,全世界各国民主势力空前发展了。另一方面看到全世界各国民主势力和反动势力在作严重的斗争,国际局势,时好时坏。我们共产党对这种变动时局要有一个基本认识,这样才不致于迷失方向。因此毛主席在中央会上对时局有一个重要分析,是分析时局的方针。经毛主席允许,在党员干部中传达一下:

毛主席指示共有四条:(一)“在第二次世界大战这一时期,各国反法西斯国家团结起来,其主要是苏联和反法西斯国家人民,把德、意、日法西斯及其附庸打倒了,这就给全世界民主势力向前发展开辟了道路。”

所谓“二次大战”,是说从一九三九年至一九四五年这一时期,这时德、意、日法西斯发动侵略战争,使反法西斯国家团结起来,共同反对法西斯,其中主要是苏联及各反法西斯国家的共产党为主力军。由于他们领导这些国家的人民坚决奋斗,才把法西斯国家三轴心及附庸国(罗、保、匈、芬……)打倒了,同时苏联壮大了,各国人民力量,特别各国共产党强大起来。希望同志们看报留心各国共产党的发展,研究材料的同志可把各国共产党的发展统计起来,很有价值,作为各支部的学习之用。斯大林驳丘吉尔时也统计了这些材料,不但东欧而且欧洲各国共产党都强大的发展起来了,如芬、匈……。被侵略的国家内共产党也空前发展(波、捷、南等……)这和第一次大战完全不一样。斯答丘所以未讲亚洲,亚洲共产党力量也空前开展,如日本,岗野进过去在延安,现在东京了,如朝鲜共产党中央也成立了,菲共产党领导了战争,爪哇共产党至今和人民一起反对荷、英帝国主义,印共产党也大大增加了,安南共产党参加了政府,特别忘不了中共,更加发展起来了。想想这日子很好过,一次大战后,中共有多大?还没有,到了一九二一年才成立,第二次大战后,有没有呢?一百二十多万,还领导了一万万以上的解放区。总之无论整个欧洲或亚洲,各国共产党力量强大了,这样给全世界各国人民民主力量向前发展开辟了道路,只有在这条件下,才产生了停战协定、政协决议等,也只有这样,中国才能在这时开始和平民主的新阶段。

当同志们研究时局的时候,必须认识,和平民主新阶段的开始不是偶然的,不是张群、邵力子等态度好,才签了字的,也不是我党那个同志说的好才签了字的,而是在目前大局底下签了字的,是建筑在毛主席第一条所指的国际、国内的局势发展的基础之上的。

这一条毛主席在“论联合政府”中明确指出来了,现在又根据时局发展更明确的又一次的指示,这一条是一个大问题,是世界和中国必须要走的道路问题,离此,我们不能看清变动的时局。毛主席在七大作结论时说:“我们要对‘普遍’‘大量’的问题能看清楚,才能把握住时局。”这才是“普遍的”(法西斯打倒了)“大量的”问题。同志们在分析时局时常把大的问题忘掉,中国人有句俗话“看见芝麻,看不见西瓜”,也是这个意思。

(二)“德、意、日这些国家中法西斯残余和反法西斯国家中亲法西斯的反动势力(资产阶级右派、丘、戴、赫及中国法西斯派),这些反动势力已经在并且将来还要疯狂的组织反苏反共反民主运动,企图挑拨第三次世界大战,这两种势力现在结合在一起,成为今后世界和中国长期的重要的敌人,这些反动势力如不为人民势力所克服,则第三次大战不可避免。”

在第一条中毛主席说法西斯已打倒,世界走向和平,人民要求和平,苏联强大,反法西斯各国内部困难,大战不可能即爆发。但第二条告诉我们反动力量还是很强大的,即法西斯反动力量(残余势力的包括西班牙各国吉斯林分子)及资产阶级右派反动势力等(丘……等),他们过去是疯狂组织反苏反共反人民,而且现在和将来还如此。他们挑拨三次大战,可看到他们不但要一般的作,还要疯狂的作,这种势力成为长期的中国和世界的主要敌人,他们不会和永远不能和人民势力妥协的。同时我认为“论联合政府”中基本上指出来了,“反动力量还强大,历史上严重的曲折还会发生,如看不到,我们要犯错误的。”这里要引起同志们注意的,是主席有新的指示,明确告诉我们什么是目前人民的主要敌人,明确的指出法西斯残余势力和资产阶级右派在目前和今后是人民的主要敌人(当前还有次要的敌人),七大“论联合政府”因公开发表未明确写出,同时还指出如人民的势力不能克服反动的势力,第三次世界大战不可避免。目前各国进步,各国反动势力内部困难,苏联强大,目前不会有三次大战爆发,但不是说永远不能爆发。请看斯大林二月九日演说,内中说到“马列主义者不只一次指出:资本主义经济制度本身就包藏着战争和侵略的因素”,从这角度我想到中国问题。

总的是和平民主阶段,中国不能违反世界大势,同样,所谓和平民主阶段开始,不是说反动派已经放弃武装,内战的危险,我们受突然袭击,已经没有了。而是大战的因素是存在的,中国要走和平民主道路,但反动势力还未放弃内战的武器。

(三)“第一次世界大战给整个世界资本主义体系一个致命的震动和打击,资本主义危机无法克服。二次大战后,资本主义世界想建立一个稳定的局面已经很难了,因此恐慌即将到来,战争中表现空前因畸形繁荣的美国,也将遭到空前的危机。”(毛主席说到致命的打击。十月革命爆发,开辟世界历史的新阶段。“新民主主义论”中见到资本主义走下坡路,社会主义、新民主主义走上坡路,二次大战给资本主义一个更大的“致命的”打击。三个资本主义国受伤,苏联空前壮大了。欧洲出现了几个新民主主义国家(南、波等,亚洲有一万万人口的中国共产党。毛主席在七大说:资本主义这野兽在世界一次大战时被击了一个脚,二次大战又击了一足。所以毛主席说:资本主义残废了,现在的资本主义是爬行的资本主义。这是一年以前的话,现在证明更加正确的,资本主义的危机更不可克服。十月革命后,资本主义稳定破坏,其体系已产生不可克服的危机,各国革命失败,经20—21年恐慌后,资本主义体系还有一个相对的暂时稳定(21一29年),虽然这稳定是很脆弱,是总危机下的稳定,故称之谓相对的稳定,今天则很难有此稳定。所以严重的危机很快就会到来,毛主席“不会很久”,也不会超过十年。最近一个有名的经济学家瓦尔家说,“不出五年”,美国也一样,美国在大战中有一个回光返照,特种繁荣。毛主席说美资本主义在二次大战中有很大繁荣,但不能说资本主义万岁。是暂时回光返照的繁荣。我们不能作这样的了解,美国明天就有恐慌。相反的,我们要了解美国有很大的生产力,战争中未满足消费部门,有很大可能战后一个时期有短期的繁荣,国内市场有,国外市场扩大着,但美国的经济恐慌是要到来的,虽然美国市场扩大了,但整个资本主义市场缩小了(苏联大,欧洲新民主主义国家出现),当美国国内市场有需要,但美国生产力的发展,很快使国内市场容纳不下,发生严重的矛盾,因此可以在不久的将来看到无论国内和国际市场和美国生产力的矛盾即将到来,会遇到空前的灾难的。瓦尔加有一文章在“群众”三、四合刊上说:“总起来说,生产工具未受到破坏,且已有改进的国家(美),今后二年到四年中,生产力还可向上升涨,但即将因生产周期恐慌的到来,而呈现低落,这一危机将较21一29年的恐慌更加要长了。

在毛主席分析经济后说,“在这种情况下,一方面推动了反动势力疯狂的斗争,在国内企图造成反动的统治,在国际企图挑起世界大战,但这一情况又推动世界各国人民团结起来,并将使世界资产阶级内部分裂,这样有利于世界人民便于去组织反对反动势力的统一战线。”(这种反苏反共的企图不是证明反动势力的巩固稳定,而是证明其没落和恐慌。丘吉尔的反动演说证明了这一点。同样看到国民党二中全会法西斯嚣张叫喊不是证明了国民党统治的稳定,而是证明它统治不下去,所以要求“革新”。但另外推动人民更加团结,资产阶级的动荡恐慌,使其内部分裂了,便于人民组织统一战线。资产阶级本来内部是不统一的,如二次大战,分为法西斯与非法西斯二个营垒,同样在中国资产阶级也是分裂左、中、右三派的,现在的情况还是要继续分裂的。本来资产阶级内部不统一是马克思学说对资产阶级的主要看法的一个基本点,但这基本点常常为有些共产党员所忽视的。有些人认为世界资产阶级和中国资产阶级是统一的,“像铁板一样的”。因此,做出两个结论:“中国资产阶级和世界资产阶级一起结合起来进攻中共和人民。”(打倒一切),另一种,都是好(联合一切),前一种犯“左”倾机会主义,后者犯右倾投降主义。

我们看毛主席所有的文件,他对中国资产阶级的看法掌握了资产阶级内部分裂的观点,是毛主席思想中很重要的精华部分。如不掌握这种观点,就不是马克思主义者,只有掌握毛主席这一观点,我们的策略才能做到:“利用矛盾,联合多数,打击少数,各个击破。”四句。

这样就有利于人民组织统一战线反对反动势力。上面这种情况如何“利用”呢?毛主席说:“任何国家都有下列几种力量存在:(1)工农小资产阶级力量,他们在战争中大大提高了觉悟程度,他们是民主的主力军;(2)资产阶级内部的左派(华来士,中国民主同盟),他们主张“和苏和共”;(3)资产阶级内部的中央派(杜、阿、蒋)他们是这些国家的当权者,他们有可能和民主势力妥协,以便在有利的时节等待时机,消灭人民力量。

毛主席把蒋摆在中央派,有些人又要说:“蒋今天又有了些革命性了吧!”如果那样了解,那是完全错误的。我要大家不要去争论,蒋有无革命性,他是不消灭共产党“死不瞑目”的。毛主席所以把他放在中央则是他有两面性,一面要消灭人民力量,但不能消灭时暂时妥协,以便等待时机来消灭人民力量。蒋是半殖民地半封建国家内的买办资产阶级,他是听从国际资产阶级的命令的,以前他曾听从德国、意大利、日本法西斯的命令,今天则听杜鲁门的命令行事的。他是一个封建买办法西斯蒂,他是一个法西斯,但又是一个封建买办,可说半法西斯。这是他的特点,我们应有这样的特殊的了解。

毛主席又说:“各国资产阶级现在和过去一样,内部继续分裂成左、中、右三派,这样使共产党不但有可能联合广大人民,而且有可能去联合资产阶级内部左派,并有原则的去和资产阶级中派妥协夕去打击右派使民主势力大大发展,克服反动势力,避免世界大战,在有些国家内实行社会主义,有些国家内实行新民主主义,在人民力量发展,其力量足以克服反动力量的时候,这种可能就大大增加,在不能克服反动力量时,世界大战即爆发,这时各国人民要以革命的战争来反对反革命战争,来建立世界的和平民主。”

这一条说明,资产阶级内部分裂,使人民有利,有两种可能,取决于人民力量的发展程度和反动力量的对比。总之,能否克服反动力量,世界要走光明的道路。

(四)“党的路线,主要的联合广大人民群众,并联合资产阶级左派,并与资产阶级中间派有原则的妥协。去打击资产阶级反动派,实现国内和平的局面,在实现这主张中不是主要采取武装斗争的方法,而是采取民主主义的方法,只有在反动派破坏协议时,才用武力来制止它,这是我们的工作方法。当权的资产阶级中央派要破坏协议,是因为他们要全力反对共产党的。如不能则釆取暂时的妥协,他们的政策有二条:一条是凡能消灭者,必然消灭之;二、凡不能消灭者,则予以保留,等到将来消灭之。所以资产阶级中派有可能将来走到右派的,我们必须要警惕到的。”

毛主席所指的是什么路线,什么政策?

我认为是马列主义的路线和马列主义的政策,即不打倒一切,又不联合一切,而是利用矛盾打击一点的路线。

我们的工作方法,斗争方法,不是一成不变的。在停战后我们是釆取民主主义的方式,但为了防止反动派破坏,才被迫拿起武装。××同志报告后,延安同志们思想上有一个问题,认为武装在中国革命中是不重要,武装同志有些牢骚,说:“找错了职业。”我告诉同志们:“中国革命失掉武装,就失掉了一切。”没有革命的武装,就没有民主,没有八路军、新四军,就没有停战令、政协会、整军方案以及改组政府的事情。讲一件小事,飞机今天在延安、北平飞来飞去,也是有武装的原因。党校有人发问:改组后政府是什么性质?有人认为:“独裁加若干民主”,有人认为是“联合政府”。我说即使照协议办事,这改组后的政府依然是独裁加若干民主。道理很简单,国民党区域枪杆子还不是属于人民的,如像政协会议,一切都是想像那样好,那就不是马列主义者,中国革命失掉军队,就失掉一切,这是真理,过去如此,现在如此,将来也还是如此。今天在国民党区域你能否去进行土地改革,减租减息,在我们区域进行减租减息还不容易的事哩。由此我们可以说,当前主要斗争不是武装斗争,但武装斗争在中国革命中还是一个重要的问题。

毛主席还告诉我们,资产阶级中间派的本质是什么?毛主席指出中间派是坚决要消灭共产党的,和右派是一样的。必须要了解,在这点上,蒋和何、陈、杜和蒋,杜和赫是一样的。但既然如此,为何又有停战令,政协会呢?简单说是暂时消灭不了,予以保留,以便将来消灭之。在消灭共产党的时间上、政策上、方式上是有区别的。所谓右派则是一面,消灭不了也要消灭,蒋和杜是要走一个弯路来消灭你,这也给我们一个可利用的矛盾。我们对比如何办?釆取什么方法?自然不能叫它干脆消灭我们,他要妥协即使暂时的也好,我们是欢迎的,但思想上要认清其本质。所谓停战令、政协会等等,马歇尔、杜鲁门、蒋介石都是要等待一下,将来消灭共产党的一种办法,在形势演变中他们会变成右派的。丘吉尔是一个证明。但他们那种要消灭共产党和人民的想法,往往是做不到的,其结果常是相反的,希特勒是一个例子,人民压迫下是会警觉起来的,停战令后,一部分人有疏忽的思想。但政协令后,较场口打了一顿,把新华日报打了稀烂,就使我们要警觉起来的。蒋介石反动派是我们这样的一个先生。

综合起来:

第一条讲世界要走民主道路。

第二条讲反动力量还是很强大的。

第三条讲人民的力量是有利的,不但有广大的主力军,又有广大的后备军,不管二种可能,最后的胜利是人民的。

第四条党的路线是团结广大人民联合资产阶级左、中派,去打击反动派,完成和平民主建设事业。

这是毛主席七大路线的新发展,希望和毛主席的“论联合政府”中的基本思想联合去研究,只有把总的问题解决,才能分析时局的变动的问题。但毛主席再三告诉我们,还要最后考虑下和同志们交换意见,目前还不作公开的发表。

