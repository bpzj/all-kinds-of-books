\section[在中央工作座谈会上关于四清问题的讲话(一九六四年十二月二十日)]{在中央工作座谈会上关于四清问题的讲话}
\datesubtitle{(一九六四年十二月二十日)}

\begin{list}{}{
    \setlength{\topsep}{0pt}        % 列表与正文的垂直距离
    \setlength{\partopsep}{0pt}     % 
    \setlength{\parsep}{\parskip}   % 一个 item 内有多段,段落间距
    \setlength{\itemsep}{\lineskip}       % 两个 item 之间,减去 \parsep 的距离
    \setlength{\labelsep}{0pt}%
    \setlength{\labelwidth}{3em}%
    \setlength{\itemindent}{0pt}%
    \setlength\listparindent{\parindent}
    \setlength{\leftmargin}{3em}
    \setlength{\rightmargin}{0pt}
    }

\item[\textbf{主席:}] 总理报告,你们连‘赶上”都不敢提,我给你们加了个“赶上而且超过”。加了一段“孙中山一九〇五年就说可以超过”。这样讲了可以不登报。近代史也得看看。《孙中山全集》没有包括汪精卫、胡汉民、章太炎的他们的文章。你得看《新民丛报》,你得看梁启超的《饮冰室文集》,特别要看孙中山的《三民主义》。《三民主义》骨头很少,水分很多。孙中山晚年没有知识了。他是个讲演家,煽动家,讲得慷慨激昂,博得给他鼓掌。我听过他的讲演,也跟他谈过话。他是不准人驳的,提不得意见的。实际上他的话水很多,油很少,很不民主。我说,他可以做六十年前的好皇帝,没有民主。他一进场,全场都要站起来的,叫孙先生。没有民主,亦无知识,他的无知识达到此等程度:他给右派解释共产主义时,画了个太极图,里面画了个小圈,写上共产主义;外面又画了个圈,写上社会主义,最后外面又画了个大圈,写上民生主义。他说,社会主义、共产主义,都包括在我的三民主义里头,总司令你是最不佩服他的。

\item[\textbf{总理:}] 苏加诺讲“五基”也是把社会主义包括在他的“五基”里头。

\item[\textbf{主席:}] 湖南有个缅云山,你认得吗?他开始说,孙文没有学问,叫孙大炮,不如黄克强有学问,黄先生好,因为黄是秀才,能写一手苏东坡的字,后来他一到广东,见了孙中山,回来后一下大变了,说:“可了不起,孙先生!”

余秋里做计委副主任不行吗?他只是一个猛将、闯将吗?石油部也有计划工作嘛!是要他带个新作风去。

\item[\textbf{总理:}] 去冲破一潭死水。\marginpar{\footnotesize 189}

\item[\textbf{主席:}] 你(指贺)赞成吗?我们现在有些人从来不做总结,只管小事,不管大事。今天《人民日报》发表了四封来信,(按:指第二版“正确的设计从哪里来?”这一栏里的四封信)是谷牧组织的吧?

\item[\textbf{××:}] 不是,是胡绩伟他们。

\item[\textbf{主席:}] 我全部读光了。按语谁写的?

\item[\textbf{××:}] 按语是胡绩伟写的。

\item[\textbf{主席:}] 不是谷牧写的?过去《人民日报》我从来不看的,学蒋介石办法,不看《中央日报》。

现在好了,《人民日报》上白菜是怎样长的之类的东西少了,有些议论了。让胡绩伟参考《中国青年》《解放军报》。它里面有不少思想性的东西。有些小孩子专看《人民日报》,我说,你们又不看《中国青年报》,不看《解放军报》。

(××来了。)

\item[\textbf{主席:}] 你开讲,你挂帅。你不讲,我们散会。

\item[\textbf{××:}] 开了几天会,几个同志发了言,讲了不少问题。提出了问题,基本观点是一致的。大家蹲点了,是件大好事。讨论一下嘛!

\item[\textbf{主席:}] 讨论一下有什么矛盾。

\item[\textbf{××:}] 大家下去蹲点,认识一致了。

\item[\textbf{主席:}] 时间短了。

\item[\textbf{××:}] 还是初期,还是第一次,还没有看到发动群众的成熟经验。还看不到群众发动之后是什么样子,要看到群众发动之后才行。

\item[\textbf{主席:}] 要省、县、社、大队、小队发动了群众,都成立了贫协,才行。

\item[\textbf{××:}] 农村我知道些,对城市了解很少。农村材料我看得多一些,这几天我对城市工厂的材料努力看,也还是初期的经验。

\item[\textbf{主席:}] 白银厂的经验是比较成熟的。

\item[\textbf{××:}] 白银厂搞了两年,高扬文这次的报告,我也看了,看来还要深入。可以再写个总结性的东西。总之,大家蹲点还是第一次的初期的经验,所以很多问题还讲不出来,有了第二次、第三次的经验,就有头绪了,要有比较,你们懂得点农村以后,再做个比较,就会懂得的。现在看到农村问题的严重性了。有的单位要搞两年,具体作法不一样,等到将来再搞,你们就懂了。一个县可以搞两年,城市大厂恐怕也得两年。

\item[\textbf{主席:}] 要两年呀!打长一点,搞三年也好,横直是要把问题彻底解决。也可能缩短。

\item[\textbf{××:}] 湘潭、山东就是。陈正人提出洛阳拖拉机厂也要搞两年,不要赶时间。

\item[\textbf{主席:}] 把问题解决。

\item[\textbf{××:}] 熟练了,不用这么长的时间。有个问题,农村方面主要矛盾是什么?

××讲,农村已经形成富裕阶层、特殊阶层。他讲主要矛盾是广大贫下中农与富裕阶层、特殊阶层的矛盾。

×××说,还是地富反坏、坏干部结合起来与群众的矛盾,是吗?(××:是。)

\item[\textbf{主席:}] 地富是后台老板,台上是四不清干部,四不清干部是当权派,你只搞地富,贫下中农还是通不过的,迫切的是干部。地富反坏还没有当权,过去又斗争过他们,群众对他们不怎么样,主要是这些坏干部顶在他们头上,他们穷得很,受不了。那些地富,已经搞过一次分土地,他们臭了。至于当权派,没有搞过,没有搞臭。他是当权派,上边又听他的,他又给定工分,他又是共产党员。\marginpar{\footnotesize 190}

\item[\textbf{××:}] 这是头一同。当权派后头有地富反坏,或者是混进来的四类分子。有些坏干部与地富关系不很密切。地富反坏混进组织,包括划漏的地富变成贫农、共产党员。那也是当权派,不属于过去的地富,地富臭了,这一部分就不同了。

\item[\textbf{主席:}] 如×××讲的湟中县,是马步芳的参谋长。

\item[\textbf{××:}] 这在西北也是少数。

\item[\textbf{主席:}] 在西北是少数,在全国也是少数。

\item[\textbf{××:}] 怎么划要讨论一下,统一语言。怎么讲主要矛盾?

\item[\textbf{主席:}] 还是讲当权派。他们要多记工分嘛!“五大领袖”嘛?你“五大领袖”不是当权派!

\item[\textbf{××:}] 陶×提出这个问题,各方面有反映。有人赞成,有人不赞成,中央机关就有人不赞成,我就听到过。有三种人:漏划的地主、新生的资产阶级、烂掉了的……。多数情况是劳动人民出身。在立场、经济、思想、组织上四不清,他们同地富反坏分子勾结在一起,有的被地富反坏操纵,也有漏划的地富当了权的;还有已经摘了帽子的地富反坏分子当了权的。

\item[\textbf{主席:}] 后两种哪种多?

\item[\textbf{××:}] 还是漏划的多。

\item[\textbf{主席:}] 不要管什么阶级、阶层,只管这些当权派,共产党当权派,“五大领袖”跟当权派走的。不管你过去是国民党、共产党,反正你现在是当权派,发动群众就是整我们这个党。

\item[\textbf{××:}] 有些小队干部也坏了。

\item[\textbf{主席:}] 小队干部多数不是党员,岂有此理。一个大队只有几个、十几个、二十几个党员,党员太少了。他死也不发展。他搞久了,搞出味道来了。中心问题是整党,不然无法。不整党没有希望。

\item[\textbf{总理:}] 机关也是这样,你建工部刘秀峯,你统战部李维汉,你政协张执一,还不都是党员,非搬开不可。我们在民主人士中宣布了,他们很震动。

\item[\textbf{主席:}] 共产党是有威信的。也不要提阶层,那人就太多了,吓倒了人,得罪的人多。就提党委!地委也是个党委,县委也是个党委,公社也是党委,大队党委会、支部委员会,无非左、中、右,我相信右的是少数,特别右的只占一部分,左的也是少数,中间派油水多,要争取,你要把这部分人分化出来。×××说的,利用矛盾,争取多数,反对少数,各个击破;有拉有打,拉中有打,打中有拉。发展进步势力,争取中间势力,孤立顽固势力。我们多年没有讲过这一套了。

\item[\textbf{××:}] 这是统一战线的策略。主席:我看现在还用得着,现在这个党内都是国共合作嘛!也有统一战线。

\item[\textbf{××:}] 实际如此,不要出去讲。

\item[\textbf{主席:}] 还有少数烂掉了的,省委也有烂掉了的,你安徽不是烂掉了!你青海不是烂掉了!贵州不是烂掉了!甘肃不是烂掉了!(有人说还有云南。)云南还是“个别”的,不够。河南吴芝圃“左”得很嘛!

\item[\textbf{××:}] 不提富裕阶层,叫新的的剥削压迫分子,或者只提什么贪污盗窃分子、投机倒把分子。如果他们结成一体也可以叫集团。

\item[\textbf{主席:}] 不提阶层,叫分子或集团就全了。你们研究一下。分子嘛!分子也有团,有分子团哩!\marginpar{\footnotesize 191}

\item[\textbf{××:}] 他们跟广大群众的矛盾,这些少数人压迫剥削多数人,被压迫的总是多数,总要革命,全世界压迫者少数,压迫厉害,孤立了。信心就在这里。

\item[\textbf{主席:}] 被剥削、被压迫、不满的人多,因此就要革命。

\item[\textbf{××:}] 有几种情况要区别清楚。一种是地富站在前头的,应打倒,一种是漏划的地富分子,是阶级敌人,这种人不会干好事,把材料弄清后,处理也容易,凡是过去的地富反坏分子,混进党内来的,照四类分子处理;还有一种贫下中农,过去土改,革过命,后来被地富拉过去了,站在群众头上压制群众,对这些人要严肃斗争,彻底退赔。

\item[\textbf{主席:}] 第三部分是主要的,是多数。

\item[\textbf{××:}] 漏划的不少,和平土改区多。

\item[\textbf{××:}] 地主劳动五年改变成份,富农劳动三年改变成份,有些摘了帽子的又坏了,这个规定不行了,要改。

\item[\textbf{××:}] 那好办,再戴上。

\item[\textbf{××:}] 搞土改时我们提过中立富农的政策。

\item[\textbf{主席:}] 我们犯了错误了,认识不足,当时为了稳定中农,对富农只搞掉他们的封建剥削的那一部分,这次没有反映过去侵犯中农的材料,贫下中农发动起来,就要侵犯中农。有没有把中农划成富农的?你晋西北就搞中农嘛!

\item[\textbf{××:}] 晋西北有一个查三代的错误,没收粮食是为了度灾荒。

\item[\textbf{××:}] 一部分戴帽子的,一部分漏划的,还有一部分原贫下中农,现在当了权变坏的。原贫下中农(主席:以至中农)多数可以争取,提高阶级觉悟,但你不要把他们的财产、手表、自行车、新房子搞掉,群众不满意。退赔要搞。

\item[\textbf{主席:}] 你讲第三部分。

\item[\textbf{××:}] 不退赔,也不利于教育干部。

\item[\textbf{××:}] 搞掉,教育了新干部。干部不能当了,多数要争取,少数要戴帽子,恐怕这个政策要定下来。

\item[\textbf{主席:}] 少数恶劣的,要戴新资产阶级分子的帽子。

\item[\textbf{××:}] 我看,这些人总而言之不是共产党。但主要是整共产党。管你劳动人民出身的,漏划的地富……。总而言之,搞的结果,戴帽子的户数不能超过7%——8%,人数不能超过10%。

\item[\textbf{雪峰:}] 是包括现在的在内?

\item[\textbf{主席:}] 你们看?否则,得罪的人太多了。要知道,他们也不是一块铁板,是有变化的,有富有贫,有升有降,有好有坏,有当权有不当权。我现在在这个问题上有些右。那么多地主富农、国民党、反革命,和平演变划成20%,七亿人口,划20%多少人?恐怕要发生“左”的潮流。

\item[\textbf{雪峰:}] 能争取的干部还是要耐心地争取,不然贫下中农的比例就要减少很多。

\item[\textbf{主席:}] 群众起来划,影响你们走群众路线,群众要求多划,干部也要多划。结果不利于人民,不利于贫下中农。四不清干部,贪污四、五十元的、一百元的是多数,先解放这一批,我们就是多数嘛!犯了错误的,对他们讲清楚道理,还是要革命的。那个报告中讲的车间主任、工段长、小组长,都是老工人,犯了错误讲清楚,让他们做工作嘛!\marginpar{\footnotesize 192}

\item[\textbf{××:}] 有个富裕阶层“三大件”之类。

\item[\textbf{主席:}] 他们先富裕,用扣工分等办法,自行车、毛衣、还有后富裕的贫下中农。

\item[\textbf{××:}] 现在还是×××讲的“四清”好,可适用于机关。

\item[\textbf{主席:}] 原“四清”叫作“一清”,就是经济。挂谁的账,是河北的发明权?\\
(大家议论:“四清”这个含义,在第一个十条中已经从正面提出,是主席加的,以后又经×××的报告,河北省委从反面也提出。)

\item[\textbf{××:}] 华北把社教统统叫“四清”。

\item[\textbf{雪峰:}] 我们先讲经济上四不清,政治上四不清,后来加上组织上,×××同志报告,连思想四个四不清。

\item[\textbf{主席:}] 我没有这个印象。你们把×××压低我不赞成。我最早看到的是×××的。

\item[\textbf{康生:}] ××对四不清的提法很好。他的那个报告我很欣偿。

\item[\textbf{××:}] 你账挂到河北账下,×××也是河北人嘛。地富反坏当了权都坏,不会有什么好的。问题是贫下中农当权。

\item[\textbf{主席:}] 只要把一百到一百五十元的解放出来。

\item[\textbf{××:}] 那不一定,几百元的不少,几千元、一千斤粮食的也相当多。恐怕要把一千元的解放出来,退赔还要退赔。

\item[\textbf{主席:}] 挤牙膏,挤不净,那有什么办法?留一点也可以。挤得那么干净?宽大处理嘛!

\item[\textbf{××:}] 能挤多少就一定挤多少。一个剥削群众,一个剥削国家,还是退赔,退赔从严,要彻底,特别恶劣的,一直抵抗到底,没收。

\item[\textbf{主席:}] 国家也是人民的,我们自己没有东西,退赔从严,对!要合情合理好。不必讲“彻底”。

\item[\textbf{××:}] 打击面究竟多大好?定百分之几恐怕有利。地富分子有些老实的也可以摘帽子,那是极少数,地富子女情况不同,有分家的,有没有分家的,有表现好的,有表现一般的,有表很坏的。

\item[\textbf{××:}] 打击面控制到百分之几有利?一开始,分化四不清干部就需要。地富有一部分表现好的,也不能戴了。贫下中农和中农中极少数的戴帽子,也有好处。比如有的给戴上新恶霸分子的帽子。但多数应分化、争取。不能当干部、党员。不是打击对象,还是争取对象。

\item[\textbf{××:}] 现在还不到这个时候,将来新剥削分子多吃多占。

\item[\textbf{主席:}] 多吃多占,复杂得很哪!主要是我们这些人,汽车、房子有暖气,司机,我只四百三十元,雇不起,又要雇请秘书……。

\item[\textbf{××:}] 要退赔多少?

\item[\textbf{××:}] 退赔搞得差不多了,就行了。

\item[\textbf{主席:}] 群众知道的,搞到一定程度就行了。牙膏不可挤得太净。有的地方只有十八户,没有一个虱子,一定要捉虱子?

\item[\textbf{××:}] 一个大队定一两个这样的分子,可以不可以?有些要戴帽子,戴了什么分子的帽子,就好办了,戴帽,以后可以摘。

\item[\textbf{主席:}] 叫分子,留点出路,好嘛!不涉及家庭,还可以摘嘛!其他劳动好的,不戴贪污分子。\marginpar{\footnotesize 193}

\item[\textbf{××:}] 搞得好的,主动的,不戴什么分子的帽子。

\item[\textbf{主席:}] 陈平宰肉甚均。他做宰相时贪污,周勃等人就告他贪污,说给钱多的做大官,给钱少的做小官……,刘邦就找他谈,说人家告你贪污。他说,我养的人多,我是没有钱呀!刘邦说,给你四万两黄金,搞统一战线。有了四万两黄金,就不贪污了。《鸿门宴》这出戏现在不唱了。马××演得激昂慷慨。贪污历来举他(陈平),特别是曹操。目前正在火头上,我又怕泼冷水呀!

\item[\textbf{××:}] 只要群众充分发动起来了,群众是懂理的。

\item[\textbf{主席:}] 有时也不然,群众起来了,有盲目性,我们也有盲目性。过去武汉时代群众发动起来了工厂关门,减工资,失业了,盲目性。

\item[\textbf{××:}] 我当时就怀疑。

\item[\textbf{主席:}] 现在怕泼冷水,你们掌握气候。现在还是反右。十二月不算,明年一月、二月、三月……至少再搞五个月。一不可太宽,不可打击面太宽,二不可泼冷水。不要下边宣布!喔,讲了牙膏不可挤得太净,贪污分子也可以做宰相了。

\item[\textbf{××、雪峰:}] 对敌应包括严重四不清干部,新生资产阶级分子,社会上的老资产阶级分子和地富。(前者)叫贪污分子、投机倒把分子。

\item[\textbf{××:}] 可以。对四不清干部就是要退赔,没有搞清楚……

\item[\textbf{主席:}] 没有搞四清的地方,可以先借一些出来,借给国家,救济贫穷的;然后再搞,搞出贪污来,就不要还了。

\item[\textbf{××:}] 大体上能退赔到多少?能不能退赔到百分之七、八十?只退赔到百分之五十,大概过不了关。

\item[\textbf{主席:}] 问题是现在还有实物存在没有,如果没有那个东西,就挤不出来,有就挤。无非“四大件”,金银、房子、地下藏的什么。

\item[\textbf{雪峰:}] 严重四不清,投机倒把跟……

\item[\textbf{××:}] 城市更不同,“三原政策”合营,统战部历来不搞资产阶级,每次运动都要下一个保护资方人员的通知。新老在一起,严重的在上边、工厂、公司。因此第一步目标要鲜明,要集中力量整部、整厂、整党。例如一个部先整党组成员,一个厂先整党委书记、厂长等。要明确规定这一条,否则当权的干部会滑掉。

\item[\textbf{主席:}] 先搞豺狼,后搞狐狸,这就找到了问题。不从当权派着手不行。

\item[\textbf{先念:}] 不整当权派,最后就整到贫下中农的头上了。

\item[\textbf{主席:}] 根本问题就在这里。

\item[\textbf{××:}] 先搞豺狼,后抓狐狸,不讲阶层。不然你强调资产阶级工程技术人员,或强调下面的小偷小摸,或强调不当权的资本家出身的学生,那干部们精神就很大,斗呀!后果干部很容易滑掉。就搞不成了干部。例如:白银厂的根子在省委、冶金部,不把根子搞清,白银厂好不了的。

\item[\textbf{主席:}] 冶金部根子是谁?

\item[\textbf{××:}] 我没听说冶金部根子是谁。(××:王鹤寿嘛)

\item[\textbf{××:}] 现阶段的主要矛盾,资产阶级与无产阶级的矛盾,目前主要是四不清……以四不清的干部、当权派为主。

\item[\textbf{××:}] 一次搞不清,以后还会发生。\marginpar{\footnotesize 194}

\item[\textbf{主席:}] 只要隔两三年不搞就又来了。这是不以人的意志为转移的。一个漏划,一个新生,一个烂掉,那是当权派。要搞主要的。杜甫《前出塞》九首诗,人们只记得“挽弓先挽强,用箭先用长,射人先射马,擒贱先擒王’,这四句,其他记不得了。大的搞了,其他狐狸你慢慢地清嘛!我们对冶金部也是擒贼先擒王,擒王鹤寿嘛!不要他当部长,下去当经理,擒马下来,然后改造。

\item[\textbf{××:}] 重点是党。

\item[\textbf{主席:}] 重点在党。冶金部是党委,白银厂是党委,省委也是党委,地委、县委、公社党委、支委。抓住这些就有办法。你高扬文开始到白银厂也是庇护的,一蹲点变了。你王鹤寿庇护,变了吗?

\item[\textbf{××:}] 北大整陆平,资产阶级教授出来保护陆平。×××同志在延安不是说右吗?清华搞得好,发动了群众。

\item[\textbf{主席:}] 你姓陆的,×××整××,我是站在你这方面的。××还可不可以当校长?不能,×××××嘛!看来清华比较好。(有人问:四清与四不清是农村主要矛盾,这样提行不行?不行!)(又有人问:这些富裕阶层……是什么性质?)

\item[\textbf{主席:}] 什么性质?反社会主义的资本主义性质。还加个封建主义、帝国主义?!因为我们搞了民主革命,给资本主义开辟了道路,给社会主义也开辟了道路,你们蹲点就是开辟……横直我们搞不完,留给下一代,不要拿我们这些人的年纪做标准。

\item[\textbf{××:}] 两类矛盾交织在一起,问题的复杂性就在这里。

\item[\textbf{主席:}] 人家贪污盗窃,还社会主义?

\item[\textbf{××:}] 有的没有虱子,有的虱子很小。有个策略问题,坏干部布置了。

\item[\textbf{主席:}] 你整他,他不布置?

\item[\textbf{××:}] 四不清干部造了很多谣,说什么“先整群众,后整干部。”应明白地讲是干部。

\item[\textbf{主席:}] 这有什么,先整干部嘛!

\item[\textbf{××:}] 干部多吃多占的要退,所有社员的不退。不只是贫下中农不要搞。这样,群众的顾虑解除了,其次再解放多占些的干部,干部同社员一起分了的,只退干部多占了的部分。

\item[\textbf{主席:}] 一分为二嘛!一个群众,一个干部。

\item[\textbf{××:}] 然后再集中搞少数严重的。

\item[\textbf{主席:}] 有那么多步骤,我就不赞成你安源开始联系小职员么!你那安源,官志远、朱锦堂、朱少兼两个老婆我们联系他,一直联系他。粤汉铁路要成立工会,一个人不认识,找到一个工头,也是两个老婆,后来也枪毙了。

\item[\textbf{××:}] 争取多数.孤立少数,不要上当。扎根串连,雪峰同志讲的,扎在真正老贫农身上,这是对的。但开始扎的不一定是好的,勇敢分子也可以利用一下。

\item[\textbf{主席:}] 勇敢分子也利用一下嘛!我们开始打仗,靠那些流氓分子,他们不怕死。有一时期军队要清洗流氓分子,我就不赞成。

\item[\textbf{××:}] 老实根子,不一定工作队开始就能找出来,找出来的不一定是好的,要到一定的火候才出来。根子不要告诉他是根子。

\item[\textbf{主席:}] 什么根子不根子,横直搞社会主义。

\item[\textbf{××:}] 积极分子一批一批出来,经过斗争,到那时他是老资格了你说他不是老资格?

\item[\textbf{主席:}] 李立三不是老资格?到紧急关头不干了,才请我们国家主席去。\marginpar{\footnotesize 195}

\item[\textbf{××:}] 不只李立三,蒋先云也跑了,李立三认识他的人多,因为宣布胜利是他宣布的,那时我们活动不准杀人,你如果杀人,我们就停工。

\item[\textbf{主席:}] 那个矿一停,三天水就满了。

\item[\textbf{××:}] 凡搞剥削历史,有人不赞成的。找贫下中农积极分子,一开始是不会扎准的,×××那里就换了百分之三十多嘛!恐怕还是在斗争中逐步发现。

\item[\textbf{主席:}] 你专搞老实人,不会办事。……

\item[\textbf{××:}] 干部与贫下中农还是同时搞。背靠背,他不知道什么,干部揭发干部,群众另外揭发,消息也会走露。

\item[\textbf{主席:}] 有消息灵通人士嘛!为什么赵紫阳住的那个老贫农家喂条狗?怕人听。

\item[\textbf{××:}] 先背靠背,后坐主席团台上,让贫农先参加干部“洗澡”会,不能一下子就当主席。

\item[\textbf{主席:}] 他不没读过孙中山的《民权初步》。搞个勇敢分子当主席行不行?总而言之,把那个流氓无产阶级说的那样坏,不行。军队中有个时期要洗刷他,我就不赞成。

\item[\textbf{××:}] 五反的经验还是少。工厂核心烂掉的恐怕不是少数。基层、中层都有问题。要整顿领导核心,中层干部也要整,基层干部也要整。

\item[\textbf{主席:}] 王鹤寿有没有转变?

\item[\textbf{××:}] 有进步。

\item[\textbf{主席:}] 已变好,我很高兴。此人跟我有些关系,学解放军,学大庆,没有他,我不知道。

\item[\textbf{××:}] 总之,一脱离体力劳动,方向就错了,要参加劳动。

\item[\textbf{××:}] 能不能实现“三同”是能否蹲下点,能否联系群众的主要关键。特别要参加劳动。一参加劳动,问题就解决了,重庆钢铁厂,任白戈在那里蹲点,他们“三定一顶”实行的好,有些干部只学到了炼钢的本事。

\item[\textbf{××:}] 那批人有技才,不应该脱离生产,作工作,给点时间就行了。

\item[\textbf{主席:}] 每天要几小时?

\item[\textbf{××:}] 小组长有半小时、一小时就够了,车间主任有一小时、两小时也就够了。

\item[\textbf{主席:}] 科室人员统统下去,大庆几万人,各种舆论,一个死命令都去劳动了。这次××骂呀!要大家蹲点。我骂娘不灵,××一骂,还是下去了。

\item[\textbf{××:}] 干部中大部分是老工人,应该批评、争取。主席:所以要下死命令。要有秦始皇。中国的秦始皇是谁?就是×××。我当他的助手。

\item[\textbf{富治:}] 有个人多如何处理的问题,还有个奖金,本是工人的工资组成部分,应该如何处理的问题。

\item[\textbf{××:}] 工厂里好人多,干部不比我们下去的弱。抽出训练,他们的任务吃不饱嘛!你抽出20%,你们搞出经验来,我们有办法了。工厂搞五反的干部,就从工厂中抽。工厂人多,拿出训练骨干分子。谢富治就是这样搞的,陈正人也训练了四百人。

\item[\textbf{主席:}] 全国都要搞,你(指谢)那个厂抽出一半人来,另搞一个厂子。一个厂办两个工厂。

\item[\textbf{××:}] 工厂技术员,工程师也要参加阶级斗争,注意参加运动,才能又红又专。

\item[\textbf{主席:}] 也不那么专,他不联系群众,不参加劳动,听意见听不到,或者望一望,不下苦功夫……下一个死命令,余秋里办法,六万人里有七千人,各种各样议论。……

\item[\textbf{××:}] 有各种各样的议论,“参加劳动耽误了研究工作”,“我刚升起来,又让我去劳动”……

\item[\textbf{主席:}] 还是下个死命令,统统下去。\marginpar{\footnotesize 190}

(有人提:要组织革命委员会,现在工会贪污的很多,不行了。)

\item[\textbf{××:}] 洛阳拖拉机厂搞五反代表会。

\item[\textbf{××:}] 工会系统恐怕不行了,重新组织,从扎根串联,发现好的,重新组织,用什么名义都可以,就是要革命,开始组织20%—80%的积极分子。

\item[\textbf{主席:}] 有30%就了不起了。

\item[\textbf{××:}] 此外,工厂、机关多出来的人,如何处理,都交上来,怎么办?

\item[\textbf{××:}] 不能上交,还是雪峰讲的,三勤夹一懒,自己处理。

\item[\textbf{主席:}] 还是在工厂中三勤夹一懒好,李雪峰同志讲的嘛!不是我讲的嘛!以邻为壑,不是办法。就在这个工厂,一分为二,三勤夹一懒嘛!怕什么!适当分散。

\item[\textbf{××:}] 把那些坏人戴上帽子,放在乡下,劳动好了。

\item[\textbf{××:}] 有家可归的,是否可以回家些去?

\item[\textbf{主席:}] 有家可归的,你们去江西多少万人,不是又跑回去了?工厂搬就好了。集中几千,只几十个干部管,你说那40%就没有办法?都要交,我看交哪里去,交到到他(指总理)那里去了。

\item[\textbf{××:}] 我还想这么一种意思,前途无限光明,一个乡几万人,一个厂……昨天李雪峰同志讲的认识论,人的正确思想从哪里来,如果领导得好,真正搞好,马列主义、提高文化,认识论,毛泽东思想,可以出现又有集中又有民主,又有纪律又有自由,又有统一意志,又有个人心情舒畅,生动活泼的政治局面。在一个大工厂,一个县,一个大城市,思想方法,工作方法搞不好了,不得了,就会变颜色。江西兴国、上杭出那么多干部。

\item[\textbf{主席:}] 还有永新。

\item[\textbf{××:}] 苏联的基洛夫厂,以前叫镰刀与锤头工厂,十月革命后,全国都有他的干部。搞好了一个大工厂、一个大县、一个大市,就可以出办法,出干部,可以改造全国、全世界大的精神面貌变了。一个大工厂影响全市、全国、全世界。现在工作队慢慢搞下去,一直搞下去,对我们的新型人物……

\item[\textbf{主席:}] 列宁很重视农民,提工农联盟。《共产党宣言》就怕小资产阶级,把小资产阶级的缺点、消极面提得过多了,小资产阶级有两面性,看你强调哪一面。中国有多少小资产阶级。流氓无产者更多。对流氓无产者更不客气。就强调消极面?他有积极的一面嘛!根据我们的经验,可以改造的。

\item[\textbf{××:}] 机关也一样,无限光明。基本问题要有强的领导核心,有马列主义,无产阶级思想体系,三八作风、四个第一……搞好了坚持下去,人的、自然的面貌大大改变。再多少年一直这么搞下去,世界要改变。就对世界无产阶级革命有了大贡献。十月革命生动活泼,斯大林建设了社会主义,后来死气沉沉,赫鲁晓夫又这么一搞……世界上还没有在社会主义下放手发动群众搞革命斗争的经验。冰岛共产党记者问我,如何资本主义才能不复辟?

\item[\textbf{主席:}] 两种可能。一种复辟,一种不复辟。

\item[\textbf{××:}] 我答复他们的办法:发动群众搞四清、五反、工资不要太高,半工半读,逐步消灭脑力劳动与体力劳动的差别,三种事。开始这样做。毛主席讲三项伟大革命,阶级斗争、生产斗争、科学实验,避免修正主义,保证建设社会主义强国。\marginpar{\footnotesize 197}我们去参加,形成那么一种作风。现在开始,中国人口占全世界三分之一。三分之一搞好了,那三分之二就会过来。

\item[\textbf{主席:}] 我们希望搞好,搞成一个像样子的国家。这是一种可能性,还有一种可能性搞不好,那怎么办?也没有什么。不要性急,不要希望我们在时都能搞好。大概一个省的三分之一搞好,那三分之二不搞也可以。那三分之一动起来,那三分之二也动了。你湖北七十一县,三分之一就是二十四个县,也就好了。

\item[\textbf{××:}] 但要搞好一个县、一个厂……不付出劳动不行,没有马列主义,毛主席的认识论不行……

\item[\textbf{主席:}] 历来讲认识论不联系具体工作,离开具体工作讲认识论,那讲哲学干什么,有什么用处!

\item[\textbf{××:}] 有了可以造成……

\item[\textbf{主席:}] 不是一切人心情舒畅,总有一部分人心情不舒畅,地富反坏不舒畅,四不清干部一定时期也不舒畅,不然他们为什么封锁?

\item[\textbf{××:}] 是否杀人?我看还是个别杀……点上一般杀人不利。一杀,面上非要恐慌不可。但不是一个人不杀,什么时候杀也要考虑。

\item[\textbf{主席:}] 就是要震动。杀多了哩!多了有什么害处?一、以后找他使用,活材料没有了;二、得罪了他家人,杀父之仇。要杀的可以先关起来。不可不杀,不可多杀,杀一点,震动,怕震动干什么,就是要震动。还有一条,杀错了死者不可复生。

\item[\textbf{××:}] 像天津厉慧良要杀,也没有材料可用,家庭……。不杀,得罪了广大群众。

\item[\textbf{主席:}] 京剧界就发生问题。

\item[\textbf{××:}] 地富儿子劳动算什么成分?

\item[\textbf{主席:}] 是社员,当然是农民嘛!你社会主义不让人家参加,一家独占?

\item[\textbf{雪峰:}] 贫农、中农也叫社员,这个称呼不能解决成分问题。

\item[\textbf{总理:}] 一般农民嘛!就叫农民。

\item[\textbf{主席:}] 你们再争论争论嘛!(完)
\end{list}

