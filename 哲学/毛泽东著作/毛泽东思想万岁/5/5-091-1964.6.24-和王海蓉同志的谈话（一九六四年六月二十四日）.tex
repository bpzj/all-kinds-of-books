\section[和王海蓉同志的谈话(一九六四年六月二十四日)]{和王海蓉同志的谈话}
\datesubtitle{(一九六四年六月二十四日)}


(王海蓉是外国语学院英语专修科学生)

王:我们学校的阶级斗争很尖锐,听说发现了反动标语,都有用英语的。就在我们英语系的黑板上。

主席:他写的是什么反动标语?

王:我就知道这一条,蒋万岁。

主席:英语怎么讲?

王:long live

蒋。主席:还写了什么?

王:别的不晓得,我就知道这一条,章会娴告诉我的。

主席:好吗!让他多写一些贴在外面,让大家看一看,他杀人不杀人?

王:不知道杀人不杀人,如果查出来,我看要开除他,让他去劳动改造。

主席:只要他不杀人,不要开除他,也不要让他去劳动改造,让他留在学校里,继续学习,你们可以开一个会,让他讲一讲,蒋介石为什么好?蒋介石做了哪些好事?你们也可以讲一讲蒋介石为什么不好?你们学校有多少人?

王:大概有三千多人,其中包括教职员。

主席:你们三千多人中间最好有七、八个蒋介石分子。

王:出一个就不得了,还要有七、八个,那还了得?

主席:我看你这个人啊!看到一张反动标语就紧张了。

王:为什么要七、八个呢?主席:多几个就可以树立对立面,可以作反面教员,只要他不杀人。

王:我们学校贯彻了阶级路线,这次招生,70%都是工人和贫下中农子弟。其它就是干部子弟,烈属子弟等。

主席:你们这个班有多少工农子弟?

王:除了我以外还有两个干部子弟,其他都是工人、贫下中农子弟,他们表现很好,我向他们学到很多东西。

主席:他们和你的关系好不好?他们喜欢不喜欢和你接近?

王:我认为我们关系还不错,我跟他们合得来,他们也跟我合得来。

主席:这样就好。

王:我们班有个干部子弟,表现可不好了,上课不用心听讲,下课也不练习,专看小说,有时在宿舍睡觉,星期六下午开会有时也不参加,星期天也不按时返校,有时星期天晚上,我们班或团员开会,他也不到,大家都对他有意见。

主席:你们教员允许你们上课打瞌睡,看小说吗?

王:不允许。

主席:要允许学生上课看小说,要允许学生上课打瞌睡,要爱护学生身体,教员要少讲,要让学生多看,我看你讲的这个学生,将来可能有所作为。他就敢星期六不参加会,也敢星期日不按时返校。回去以后,你就告诉这学生,八、九点钟回校还太早,可以十一点,十二点再回去,谁让你们星期日晚上开会哪。

王:原来我在师范学院时,星期天晚上一般不能用来开会的。星期天晚上的时间一般都归同学自己利用。有一次我们开支委会,几个干部商量好,准备在一个星期天晚上过组织生活,结果很多团员反对。有的团员还去和政治辅导员提出来,星期天晚上是我们自己利用的时间,晚上我们回不来。后来政治辅导员接受了团员的意见要我们改期开会。主席:这个政治辅导员作得对。王:我们这里尽占星期日的晚上开会,不是班会就是支委会,要不就是级里开会,要不就是党课学习小组。这学期从开学到我出来为止,我计算一下没有一个星期天晚上不开会的。

主席:回去以后,你带头造反。星期天你不要回去,开会就是不去。

王:我不敢,这是学校的制度规定,星期日一定要回校,否则别人会说我破坏学校制度。

主席:什么制度不制度,管他那一套,就是不回去,你说:我就是破坏学校制度。

王:这样做不行,会挨批评的。

主席:我看你这个人将来没有什么大作为。你怕人家说你破坏制度,又怕挨批评,又怕记过,又怕开除,又怕入不了党。有什么好怕的,最多就是开除。学校就应该允许学生造反。回去带头造反。

王:人家会说我,主席的亲戚还不听主席的话,带头破坏学校制度。人家会说我骄傲自满,无组织无纪律。

主席:你这个人哪?又怕人家批评你骄傲自满,又怕人家说你无组织无纪律,你怕什么呢?你说就是听了主席的话,我才造反的。我看你说的那个学生,将来可能比你有所作为,他就敢不服从你们学校的制度。我看你们这些人有些形而上学。

(有一次谈到了学习问题)王:现在都不准看古典作品。我们班上那个干部子弟他尽看古典作品,大家忙着练习英语,他却看《红楼梦》,我们同学对他看《红楼梦》都有意见。

主席:你读过《红楼梦》没有?

王:读过。

主席:你喜欢《红楼梦》中哪个人物?

王:谁也不喜欢。

主席:《红楼梦》可以读,是一部好书。读《红楼梦》不是读故事,而是读历史,这是一部历史小说,作者的语言是古典小说中最好的一部。你看曹雪芹把那个凤姐写活了。凤姐这个人物写得好,要你就写不出来。你要不读一点《红楼梦》,你怎么知道什么叫封建社会?读《红楼梦》要懂四句话:“贾不假,白玉为堂金作马(贾家)。阿房宫,三百里,住不下金陵一个史(史家)。东海缺少白玉床,龙王请来金陵王(王家)。丰年好大“雪”,珍珠如土金如铁(薛家)。”这四句是读《红楼梦》的一个提纲。杜甫有一首长诗叫《北征》,你读过没有?

王:没读过。《唐诗三百首》中没有这首诗。

主席:在《唐诗别裁》上。(当时主席把书拿出来,把《北征》这首诗翻出家要我阅读)。

王:读这首诗要注意什么问题?要先打点予防针才不会受影响。

主席:你这个人尽是形而上学,要打什么予防针啰?不要打!要受点影响才好,要钻进去,深入角色,然后再爬出来,这首诗熟读就行了,不一定要背下来。你们学校要不要你们读圣经或佛经?王:不读,要读这些东西干什么?主席:要做翻译又不读圣经、佛经,这怎么行呢?你读过《聊斋》吗?

王:没有主席:《聊斋》可以读。《聊斋》写的那些狐狸精可善良啦!帮助人可主动啦!“知识分子”英语怎么讲?

王:不知道。

主席:我看你这个人,学习半天英文,自己又是知识分子,又不会讲“知识分子”这个词。

王:让我翻一下“汉英词典”。

主席:你翻翻看,有没有这个词。

王:糟糕,你这本“汉英字典”上没这个字,只有“知识”这个词,没有“知识分子”。

主席:等我看一看。(王把字典送给主席)只有“知识”没有“知识分子”这本“汉英字典”没有用,很多字都没有。回去后要你们学校编一部质量好的“汉英词典”,把新的政治词汇都编进去,最好举例说明每个字的用法。

王:我们学校怎么能编字典呢?又没时间又没人,怎么编呢?

主席:你们学校那么多教员和学生,还怕编不出一本字典来?这个字典应该由你们来编。

王:好,回去后我把这个意见向学校领导反映一下,我想我们可以完成这个任务。

(有一次主席接见外宾之后与王海蓉的谈话)王:外宾跟你讲英语,你能不能听懂?主席:我听不懂,他们讲得太快。

王:那你接见时讲不讲英语呢?

主席:我不讲。

王:你又不讲又不听,那你学英语做什么?

主席:我学英语是为了研究语言,用英文和汉文做比较,如果有机会还准备学点日文。

(有一次主席让海蓉读文天祥的诗)主席:假如敌人把你活捉去了,你怎么办?

王:人生自古谁无死,留取丹心照汗青。

主席:对了。你回去读一、二十本马列主义经典著作,读点唯物主义的东西。看来你这个人理论水平不高。在学习上不要什么五分,也不要搞什么二分,搞个三分四分就行了。

王:为什么不搞五分呢?

主席:五分累死人了。不要那么多东西,学多了害死人。譬如说汉高祖的《大风哥》:“大风起兮云(龙)飞扬,威加四海兮归故乡,安得猛士兮守四方。”这首诗写得好,很有气魄。写诗的汉高祖就没读过什么书,但是能写出这样好的诗来。我们的干部子弟很令人担心,他没有什么生活经验和社会经验,可是架子很大,有很大的优越感。要教育他们不要靠父母,不要靠先辈,而完全靠自己。


