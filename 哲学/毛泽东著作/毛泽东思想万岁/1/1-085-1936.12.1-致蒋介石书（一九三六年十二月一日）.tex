\section[致蒋介石书(一九三六年十二月一日)]{致蒋介石书}
\datesubtitle{(一九三六年十二月一日)}


介石先生台鉴:

去年八月以来,共产党苏维埃与红军曾屡次向先生要求,停止内战,一致抗日,自此主张发表后,全国各界不分党派,一致响应。而先生始终孤行己意,先则下令围剿,是以有去冬直罗镇之役。今春红军东渡黄河,欲赴冀察前钱,先生则又阻之于汾河流域,吾人因不愿国防力量之无谓牺牲,率领西渡,别求抗日途经,一面发表宣言促先生之觉悟。数月来绥东情势益危,吾人方谓先生将翻然复计,派遣大军实行抗战,孰意先生仅派出汤恩伯之八个团向缓赴援,聊资点缀,而集胡宗南,关麟征,毛炳文,王均,何柱国、王以哲、董关斌、孙震、万耀煌、杨虎城、马鸿达、马鸿宾、马步芳、高维滋、高双成、李仙洲等二百六十个团,其势汹汹,大有非消灭抗日红军、荡平抗日苏区不可之势。吾人虽命令红军停止向先生之部队进攻,步步退让,竟不能回先生之积恨之心,吾人为自卫计,为保存抗日军队与抗日根据地计不得已而有十一月二十一日定边山城堡之役。全国人民对日寇进攻何等愤恨,对绥远抗日将士之援助,何等热烈,而先生则集全力于自相残杀之内战。然而西北各军官佐士兵之心理如何?吾人身在战阵,知之甚悉,彼等之心与吾人之心并无二致,亟欲停止自杀之内战,早上抗日之战场。即如先生之嫡系可称劲旅者,亦难免山城堡之惨败,所以者何?非该军果不能战,特不愿中国人打中国人,宁愿缴枪于红军耳。人心与军心之背如此,先生何不清夜扪心一思其故耶?今者绥远形势,日趋恶化,前线之守卫士军队之甚微,长城抗战与上海“一二.八”之役前车可鉴。天下汹汹,为公一人,当前大计,只须先生一言而决,今日停止内战,明日红军与先生之西北剿共大军营可立即从自相残杀之内战战场,开赴抗日前栈,绥远之国防力量,骤增数十倍。是则先生一念之转,一心之发,而国仇可报,国土可保,失地可复,先生亦得为光荣之抗日英雄,图诸凌烟,馨香百世,先生果何故而不出此耶?吾人敢以至诚再一次的请求先生,当机立断,允许吾人之救国要求,化敌为友,共同抗日,则不特吾人之幸,实全国全民族唯一之出路也.今日之事,抗日降日,二者择一,绯网歧途,将国为之毁,身为主奴,失通国之人心,遭千秋之辱骂,吾人诚不愿见天下后世之人聚而称曰:“亡中国者也非他人,蒋介石也。”而愿天下后世之人,祝先生为能及时改过救国救民之豪杰,语曰:“过则勿惮改”,又曰,“放下屠刀,立地成佛”,何去何从?愿先生熟察之,寇深祸亟,语重心危,立马陈词,伫候明教。

<p align="right">毛泽东……(共署各十九人,余略——编者注)

 率中国人民红军二十万人同上

<p align="right">一九三六年十二月一日




