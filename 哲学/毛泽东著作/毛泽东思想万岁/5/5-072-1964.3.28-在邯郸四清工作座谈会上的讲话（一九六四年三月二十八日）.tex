\section[在邯郸四清工作座谈会上的讲话(一九六四年三月二十八日)]{在邯郸四清工作座谈会上的讲话}
\datesubtitle{(一九六四年三月二十八日)}


一、我四、五十年前看过一本《香山记》,开头两句是“不唱天来不唱地,单唱一本《香山记》”,唱这个就不能唱别的。

二、我们有十年没有搞阶级斗争了,五二年搞了一次,五七年搞了一次,那只是在机关、学校,这一次要把农村社会主义教育运动搞好,至少用三、四年时间。我说至少三、四年,不然五、六年。有些地方,打算今年完成百分之六十。不要急,欲速则不达。当然,这不是说可以慢吞吞的,问题是运动已经起来了。河南太急。说这是第二次土改,有道理。

三、(有人汇报,工作组以包青天自居)

包公还不是帮助土豪劣绅?

(有人汇报有的工作组打人)

包公就是打人。

四、试点失败了,不奇怪,失败了还要干。要特别注意总结失败的教训。

五、(有人说,有人主张用学大庆,学解放军代替“四清”)那是代表不搞阶级斗争的那一派。大庆难道就不搞反贪污,反浪费?就不反盗窃?

六、中央五反指示没有谈阶级斗争。

七、牛鬼蛇神要让它出来,出来一半还不行,出来一半还会缩回去。

八、关于四权下放,证明山东省委农村工作部副部长的意见是对的。周兴不同意他的意见,说不能下放到队。实际上是少数人的意见代表多数人的意见。

九、(有人说,大学教授下乡四清,说自己什么也不懂)知识分子其实是最没有知识的,现在他们认输了。教授不如学生,学生不如农民。

十、人家把机关枪都交出来了,就不要再逮捕他了。逮捕是把矛盾上交,上面又不了解情况,还是放到群众中监督的好。

十一、除了年老有病的,文化很低读不懂文件的,以及政治威信很低,像彭德怀那样的,都要宣读文件。

十二、一九四七年,《论目前形势和我们的任务》,就是我口讲,有人记下,又经过我修改的,那时我得了一种不能写东西的病。现在写东西都是由秘书写,自己不动手。当然,有些东西是可以由别人代笔的。例如总理出国讲话,就是黄镇、乔冠华他们搞的。有了病,自己口讲,叫人家写,也是可以的。自己总是不动手,靠秘书,不如叫秘书去担任领导工作好了。

十三、一九三三年我在古田调查,是反映农民的意见,是农民的意见,从我的嘴说出来的。

北京是不出意见的。工厂没有原料,出不来成品,我们就是靠你们的原料来加工。


