\section[给政治局常委及有关同志的信]{给政治局常委及有关同志的信(一九六一年九月二十九日)}
\datesubtitle{(一九六一年九月二十九日)}

我们对农业方面的严重平均主义的问题,至今还没有完全解决,还留下一个问题。农民说,六十条就是缺了这一条。这一条是什么呢?就是生产权在小队,分配权却在大队,即所谓“三包一奖”的问题。这个问题不解决,农、林、牧、副、渔的大发展即仍然受束缚,群众的生产积极性仍然要受影响。如果要使一九六二年的农业比较一九六一年有一个较大的增长,我们就应在今年十二月工作会议上解决这个问题。我的意见是:“三级所有,队为基础”。即基本核算单位是队而不是大队。所谓大队“统一领导”要规定界限,河北同志规定了九条。如果不作这种规定,队的九种有许多是空的,还是被大队抓去了。此问题,我在今年三月广州会议上,曾印发山东一个暴露这个严重矛盾的材料,又印了广东一个什么公社包死任务的材料,并在这个材料上面批了几句话:可否在全国各地推行。结果,没有通过。待你们看了湖北、河北这两批材料,并且我们一起讨论过了之后,我建议。把这些材料,并附中央一信发下去,请各中央局、省、市、区党委、地委及若干县委亲身下去,并派有力的调查研究组下去,作两星期调查工作,同县、社、大队、队、社员代表开几次座谈会。看究竟那样办好。由大队实行“三包一奖”好,还是队为基础(河北人叫做分配大包干)好?要调动群众对集体生产的积极性,要在明年一年及以后几年,大量增产粮、棉、油、麻、丝、茶、糖、菜,烟、果、药、杂以及猪、马、牛、羊、鸡、鸭、鹅等类产品,我以为非走此路不可。在这个问题上,我们过去过了六年之久的糊涂日子(一九五六年高级社成立时起),第七年应该醒过来了吧!


