\section[在省市委书记会上的讲话(一九五九年二月二日)]{在省市委书记会上的讲话}
\datesubtitle{(一九五九年二月二日)}


对事情,在高潮中,有人会怀疑,这是不足为奇的,怀疑年年会有的。有两部分人,一部分是好心、关心的人,一部分是敌对分子,像罗隆基、地富反坏。要加以区别。人家怀疑或者讲坏话,不要以为是坏事,要注意加以分析。总会有些缺点,有人怀疑不足为奇,而且有好处。

压缩空气已有两个月,现在二月一日,还要鼓足干劲。总路线不能改,还是鼓足干劲,力争上游,多快好省。干劲要鼓足,如果鼓得不足,应该鼓足。……干劲还要鼓足,上游还要争,不要中游,不要下游。十一月、十二月压缩空气,群众也要休息休息,松一点不足为奇。再鼓干劲。

……

现在搞了一年,已经展开了一个大跃进的局面。是不是暂时现象置今后若干年是否会年年有跃进?像我们这样的国家,人多,大国,资源,苏联经验,应该是可以的。美国也可以说是个大跃进,一百多年世界第一。是资本主义的,现在不进了。不论大跃进,中跃进,小跃进,总之,是可以跃进的。不大跃进,会小跃进。恐怕也会年年大跃进的。是(否)展开大跃进局面。请各位想一想。以后是大跃进、中跃进、小跃进?我是倾向跃进的。

所谓工作方法,就是辩证法。有计划按比例发展。还要有个主观能动性,有些人一讲去年的缺点时,尽是缺点,脑筋里记了几十条缺点,把成绩那方面挤得没有了。这是九个指头与一个指头的问题。是形而上学还是辩证法?形而上学有几个特点:第一,就是孤立地、片面地看问题,不把世界看成统一的、互相联系的,而看作是互不相干、互不联系的,像沙子一样。第二,从表面现象看问题。不从本质看问题,从形式看问题,不从内容看问题。第三,静止地看问题。不从发展看问题,不是透过形式看到内容,透过表面看到本质。新华社的“内部参考”,不可不看,看多了也不好。如一九五七年报导北京大学问题。说是右派猖狂进攻,闹得很厉害。陈伯达去看了。不是那么了不起。又如林希翎的讲演头一天很神气,第二天驳的人多起来了,第三天驳倒了。“内部参考”就说不得了。“内部参考”写的是历史。不可不看,不可多看。尽信书,则不如无书。武主伐纣,血流漂杵,孟夫子就不相信。现在我们讲的书就是报纸和刊物,其中有个“内部参考”。不可尽信。听话要兼听。不管我们有多少缺点的量,归根到底,不过是九个指头与一个指头的问题。几亿劳动人民,几百万,几千万的干部不会尽做坏事,我相信。我们这些在座的领导干部每天吃了饭尽做坏事,不可想象。在武昌讲过,县、公社,队做坏事的,顶多不过百分之一、二、三、四、五。我们在座的和不在座的高、中级干部,都是想做好事的,想做坏事的总不会多。至于想做好事而做坏事的,要加以区别。斯大林的悲剧,是想做好事,结果做了坏事。主观的东西要在客观实践中才能见效。我们要称赞这个计划。大进一步。宣传这个方法:有重点,又是两条腿走路。比如原材料工业,目前是重点。要加一些,加工工业减少一点,增加××亿投资,××万美元。×吨钢材搞轻、化工业是对的。要宣传、讨沦、发展这个办法,经济工作很复杂,互为因果,搞不好有连锁反应。要钻进去,调查研究,发现问题,揭露问题,解决问题。不钻下去,只能打皮下,不能打血管。索性不怕它,钻进去,揭露它。不充分揭露这个矛盾,就不可能解决这个矛盾。问题就是矛盾。许多所谓没有问题,其实是有问题。要发现问题,认识问题,解决问题。《水游传》“三打祝家庄”,就是探庄,石秀探庄。这个问题解决了,再解决另一个问题。打了败仗,瓦解三个庄,孤立祝家庄。第三个问题是祝家庄内部情况不了解。于是派人假投降,内应外合,这是很好的戏,为什么不唱?过去我们打仗都调查情况,每次打胜都是条件成熟了。现在搞建设,向自然作战,也要调查研究。搞建设我们没有经验。我和各省第一书记都是去年下半年才开始抓,以前主要是抓农业,没有抓工业。农业究竟落实不落实?××斤粮食,××担棉花,麻、油料、大牲口、小家禽,这几个指标是否落实?是不是夸大?能不能完成?不要采取假报过关,其办法是超额。粮食要搞××斤,只报××斤。不搞这么多不行,不然明年不好办,前年搞得早,去年刚好,今年动手晚一些,深耕没有?搞肥料,看报上还不错。河南肥料怎么样?水、肥、土搞得怎么样?水利争三百亿土方。今年我耽心肥料这一关。人患浮肿病,就是没有肉和青菜。庄稼不吃肥料,也是患浮肿病。所以要大搞土化肥、菌肥、沤肥、绿肥、熏肥、人粪尿、牲口粪尿,以这些为主,切实搞一下。麦子要追肥,追水、多锄。多锄就是暂时割断毛细管,减少水分的蒸发,今年搞××××亿斤粮食。应该是按土、肥、水,种、密、保、管、工的序列,中心是土,有土就有粮。水有了,应该把改良土壤的“土”字放在前面。其次是肥。第三是水,但现在暂时不修改。听说搞肥料所需的人工要(占)一半。工具改革很重要。每个人民公社都要搞一个农具工厂。因地制宜,不要三天风一过,就不行了。要单独搞一个农具研究所(浙江有研究气象、土壤的,就是没有研究农具的)。收集、研究、设计,试制农具的学校。要挖这么多土方。运这么多肥料,都用人挑.没有机械是不行的。“收割”应为“割收”。割、运、打、收,没有机械.要人去割,那怎么得了!

……

还有两个问题谈一下。有些人批评我们没有大跃进。富裕中农当中有百分之三十,论调与地、富、反、坏、右接近了。民主人士肚里有意见,口上不讲就是了。对这个问题.武昌会议我也讲过,我们有百分之五的入违法乱纪。至于有些人,衷心耿耿为党为国的人,不能算进这百分之一、二、三、四、五之内。对于干部和劳动人民的劳动积极性要保护。就是百分之五之内的人,也要区别对待。分别情节,进行教育,改正错误。如果把这个问题夸大化了是不好的。这个经验年年念一下。和尚念经,天天念。这是个别与一般,大部分与小部分、部分与全体的关系问题。我们党有几十年的经验,对于本来是好人的人,犯了一点错误就夸大起来,就会变成黑暗一片。列宁说,这种话本来说得对,只要略为说过了。就变了质。现在有些好心的人,就是方法不对,分不清部分和全体的关系,缺点一列几十条。就天昏地暗,一无是处。这一点必须警惕。在整社过程中,要让群众把缺点说出来,首先要自我批评,一定要改正,然后讲清楚缺点是一个指头与九个指头的关系。在分析问题、处理问题中,一定要搞清一个或二、三个指头的问题。当然这是指大多数而言,也有少数个别搞得很坏,一塌糊涂,但大多数一定要改正缺点。要保护积极性,否则就有“曹营之事不好办”之咸。个别真正犯了路线错误的人,不是一个指头,而是烂了九个指头,例外。结论一定要做得恰当,不然要犯错误。对民愤很大的要处罚,当然不一定每个人都枪毙。农村中有些人打人成百,不给以处罚是不好的,会影响群众。但对百分之九十五以上的干部,要坚决保护过关。这个问题,我们党有几十年的经验,如罗章龙,此人现在在武汉当教授,我很熟。罗当时反对中央很厉害,否定中央,一无是处,就是他正确,自立中央。结果搬了石头打了自己的脚。还有立三路线,也讲只有他对,别人都不对,也是否定一切。王明路线也是一样,都吹自己是百分之百的布尔什维克,把别人说成是一贯的右倾机会主义,是狭隘的经验论。还有张国焘路线,也是自立中央,编了剧本歌谣,打倒毛、周、张、博,自称是列宁主义,国际路线,结果毁坏了自己,跑到香港,把儿子放在中山大学读书,证明不是列宁主义。第二次王明路线也是如此,提出六大纲领,声势浩大,根本否定中央的一切,迷惑了许多人。这不是个别人的问题,他代表好多小资产阶级中的不稳定分子。王明告洋状,说毛有三大罪状,反国际路线;整风中强迫百分之八十的人检讨;搞宗派。武昌会议时,王明写来一封信,比过去好些,讲辞职了。高饶反党集团,他们做绝了,太过分了,反对×、周,重点在×,说有两个中心,两个摊摊,有他们的纲领,迷惑了一些人,否定一切,攻其一点,不及其余,把一点夸大成全部,结果毁灭了自己。没有提过的次要事情是很多的,这些事情还不算在内。历史上有陈独秀路线、罗章龙路线、两次王明路线、张国焘路线、高饶反党集团,……这些大事件与一九五五年、一九五六年反冒进程度不同,陆质不同。……不论中国外国,不能否定一切,凡是否定一切的人,其结果是否定了自己,毁灭了自己。对蒋介石可以否定一切,但是台湾是蒋介石当总统好,还是胡适好?还是陈诚好?还是蒋介石好。但是国际活动场合,有他我们就不去,至于当总统,还是他好。最后,美国也可能不要台湾,把它当个毒瘤沾在他们身上,我们将计就计,只要他这个葫芦挂在我们腰上,总是有办法的,十年、二十年会起变化。给他饭吃,可以给他一点兵,让他去搞特务,搞三民主义。历史上凡是不应当否定的,都要做恰当的估计,不能否定一切。否定一切的结果,那是毁了自己。在目前批评缺点的时候,讲到这段历史,就是拿历史教育我们的同志。

在南宁会议,提出九个指头与一个指头的关系问题,把问题形象化,最能说服人,就是教育干部顾大局与不顾大局的问题,就是大局与小局、部分与全体的关系问题。

关于国民经济有计划按比例发展的问题,我不甚了解,要研究。究竟如何使主观符合客观法则?列宁说,俄国的革命热情与美国的求实精神统一。理论与实践的统一,理论是精神,精神反映了物质,是接近实际的东西。马列主义的普遍真理与中国革命的具体实践相结合,普遍与具体是对立的统一。客观规律在每一个国家因历史条件不同就有不同的反映。客观法则要研究它,认识它,掌握它,熟练它。斯大林对这个问题讲了很多,但不照着去做,不按照比例,工大农小,重大轻小,大大中小小。我们现在作翻案文章,从一九五六年开始创造工业(包括交通运输)、农业两方面的高涨、跃进,开始找到了有计划按比例发展法则的门路。一九五五年的合作化以后,人民的热情起来了,开始看到经济发展有希望,反保守,凡是经过努力可以办到的事情就要努力办到,如果不去努力就叫保守,不能办到的就不办,一定要他办到就是主观主义。主观反映了客观,就成了主观能动性,不是主观主义。主观能动性有两种。一种是脱离实际的,就是主观主义,一种是符合客观规律的,是符合实践的主观主义。凡是违反客观规律的就要受挫折。比如,副食品、日用百货脱节了一部分,如不抓,很危险。……

日本人说我们不是人口论,是人手论,我们有这么多的人可以做事情。一九五八年的大跃进,可能是基本适合的,至于具体数字多一点少一点,那是另外一回事,但证明是可以大跃进的,每年都可以大跃进,无非是多于一千万吨钢或少于一千万吨钢。苏联一九五八年增产四百万吨钢,是历史上所没有的,前年只增产了三百万吨,一九二一年至一九四○年二十年中只增产一千四百万吨。战后十三年增产了三千七百万吨,以后每年增加五百万吨。我们与他们不同,我们搞大中小,几个并举,有群众路线,两参一改三结合,党与群众相结合,同时地理气候条件好,人口六亿八千万,所以可能在一九五八年开展了这样一个大跃进的局面。是否能这样说,像养猪一样,前四个月是搭架子,一九五八年是克朗猪。有了架子。没有多少肉,还不肥,以后养猪。现在我们大跃进就是搭架子。

从一九五五年提出“十大关系”起,一九五八年元旦社论搞了一个“鼓足干劲,力争上游”。这两句话很好。成都会议发展成为总路线。现在看这是对的。要不要干劲?要不要鼓足?要不要争上游?还是中游下游?要不要多快?要不要好省(质量),前两句是人的精神状态,是主观能动性。后一句是物质。

当然我们有缺点错误。抓了一面,忽视了一面,引起了劳动力浪费,副食品紧张,轻工业原料未解决(多种经营),运输失调,基本建设上马太多,这些都是我们的缺点和错误。像小孩抓火一样,没有经验,摸了以后才知疼。我们搞经济建设还是小孩,无经验,向地球开战,战略战术我们还不熟。要正面承认这些缺点错误,有人宽慰我,成都会议不是提出劳逸结合,生产波浪式前进?但是没有提出具体的时间表,还是不行。还有抓了生产没有抓生活,一定要×万人(得浮)肿病,北京一人一两蔬菜才引起注意。实践中间、斗争中间才认识了客观实际、计划、比例。在一九五八年展开了一条克朗猪,但无一条肥猪。在实践过程中,找到了门路(大跃进)。可能武昌会议的四大指示是接近实际的,但只是写在纸上。不是现实的,粮食还没有拿到手。钢铁、煤只拿到一月份(生产不大好)。要经过努力,可能转化为现实性。经过这次会议,经过努力,可能各方面的问题解决得更好。有了经验,比一九五八年要好一些,各项工作和人民生活都会好一些,事后诸葛亮变成了事前诸葛亮。劳动力有浪费,大城市副食品不足,没有注意,一部分轻工业注意不足,还有多种经营、运输问题注意不够。一种是没注意,一种是注意不足,以至引起供应不足和部分失调。这几个问题不作结论,当作一个问题,请省委常委研究一下。……

从总的看,我们的计划、指标、社论适合与否,总是从实践中找经验。即使还没有完成,只是经验不足,牛皮吹得大。报纸写诗,我赞成这个空气。完不成也是乐观的,因为可以从完不成中得到教训。……无经验,明年再搞一年。苦战三年,我们的经验就多了。不适合,我们就改,让全世界骂一顿。我们总路线也不能改,“降低干劲,力争下游,少慢差费”地建设社会主义总不行。永远要鼓足干劲,力争上游,多快好省。什么叫多,什么叫快,要从实践中去看。现在我们提出十五年建成具有现代工业、现代农业、现代科学文化的伟大的社会主义国家。不行,就更多一点时间嘛!究竟什么叫有计划按比例发展。这个问题才开始接触,请同志们研究。


