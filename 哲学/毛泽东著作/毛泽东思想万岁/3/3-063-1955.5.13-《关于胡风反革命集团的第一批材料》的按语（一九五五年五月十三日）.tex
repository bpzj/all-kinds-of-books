\section[《关于胡风反革命集团的第一批材料》的按语(一九五五年五月十三日)]{《关于胡风反革命集团的第一批材料》的按语}
\datesubtitle{(一九五五年五月十三日)}


胡风的一篇在今年一月写好,二月作了修改,三月又写了“附记”的“我的自我批判”,我们到现在才把它和舒芜的一篇“关于胡风反革命集团的一些材料”一同发表,是有这样一个理由的,就是不让胡风利用我们的报纸继续欺骗读者。从舒芜文章所揭露的材料,读者可以看出,胡风和他所领导的反共反人民反革命集团是怎样老早就敌对、仇视和痛恨中国共产党的和非党的进步作家。读者从胡风写给舒芜的那封信上,难道可以嗅得出一丝一毫的革命气味来吗?从这些信上发散出来的气味,难道不是同我们曾经从国民党特务机关出版的“社会新闻”、“新闻天地”一类刊物上嗅到的一模一样吗?什么“小资产阶级的革命性和立场”,什么“在民主要求的观点上,和封建传统反抗各种倾向的现实主义文艺”,什么“和人民共命运的立场”,什么“革命的人道主义精神”,什么“反帝反封建的人民解放的革命思想”,什么“符合党的政治纲领”,什么“如果不是革命和中国共产党,我个人二十多年来是找不到安身立命之地的”,这种种话,能够使人相信吗?如果不是打着假招牌,是一个真正有“小资产阶级的革命性和立场”的知识分子(这种人在中国成千成万,他们是和中国共产党合作并愿意接受党的领导的),会对党和进步作家采取那样敌对、仇视和痛恨的态度吗?假的就是假的,伪装应当剥去。胡风反革命集团中像舒芜那样被欺骗而不愿永远跟着胡风跑的人可能还有,他们应当向政府提供更多的揭露胡风的材料。隐瞒是不能持久的,总有一天会暴露出来。从进攻转变为退却(即检讨)的策略,也是骗不过人的。检讨要像舒芜那样的检讨,假检讨是不行的。路翎应得到胡风更多的密信,我们希望他交出来。一切和胡风混在一起而得到密信的人也应当交出来,交出比保存或销毁更好些。胡风应当做剥去假面的工作,而不是骗人的检讨。剥去假面,揭露真相,帮助政府彻底弄清胡风及其反革命集团的全部情况,从此做个真正的人,是胡风及胡风派每一个人的唯一出路。


