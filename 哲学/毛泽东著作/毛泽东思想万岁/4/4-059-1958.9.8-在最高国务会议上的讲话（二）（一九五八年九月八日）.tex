\section[在最高国务会议上的讲话(二)(一九五八年九月八日)]{在最高国务会议上的讲话(二)}
\datesubtitle{(一九五八年九月八日)}


还是谈一谈老话,关于绞索,上一次不是谈过吗?现在我要讲对杜勒斯、艾森豪威尔,对那些战争贩子使用绞刑。对他们使用绞刑的地方很多。据我看,凡是搞了军事基地的,就是一条绞索绞住了。东方,南朝鲜、日本、菲律宾、台湾;西方,西德、意大利、英国;中东,土耳其、伊朗;非洲,摩洛哥等等。每一个地方美国有许多军事基地,比如土耳其,有二十几个基地,日本听说有八百个基地。还有些地方没有基地,但是有军队占领,比如美国在黎巴嫩,英国在约旦。

现在不讲别的,单讲两条绞索:一个黎巴嫩,一个台湾。台湾是古老的绞索,他已经占领几年了。他被什么人绞住了呢?被中华人民共和国绞住了。六亿人民手里拿着一根索子,这根索子是钢绳,把美国人的脖子套住了。中东是最近套住的。谁人让他套住的呢?是他自己造的索子,自己套住的,然后把绞索的一头丢到中国大陆上,让我们抓到。黎巴嫩也是他自己造的一条绞索,自己套上去,绞索的一端就丢在阿拉伯民族手里。不但如此,而且是全世界大多数人手里,大家都骂他,不同情他,大多数国家的人民、政府手里拿着这个绞索。比如中东,联合国开了会。但是主要是在阿拉伯人民手里套住了,不得脱身。他现在进退两难,早退好,还是迟退好?早退,那么为何来呢,迟退,越套越紧,可能成为死结,那怎么得了呀?至于台湾,他是订了条约的,和黎巴嫩还不同。黎巴嫩还比较活,没有什么条约,说是一个请,一个就来了,于是乎套上了。至于台湾,就订了个条约,这是一个死结。这里不分民主党、共和党,订条约是艾森豪威尔,派第七舰队是杜鲁门。杜鲁门那个时候可去可来,没有订条约,艾森豪威尔订了个条约。这边国民党一恐慌,一要求,美国人一愿意,就套上了。

金门、马祖套上了没有?金门、马祖据我们看也套上了。为什么呢?他不是讲现在还没有定,要共产党打上去,看情况,那时候再决定吗?问题是十一万国民党军队(金门九万五,马祖一万五)。只要有这两堆在这个地方,他得关心。这是他们的阶级利益,阶级感情。为什么英国人和美国人对约旦的侯赛因和黎巴嫩的夏蒙那样好?他们不能见死不救,昨天第七舰队的总司令比克利亲自指挥,还有那个斯摩特,不是放大炮吗?引得国务院也不高兴,国防部也不高兴的那位先生,他也在那里跟比克利一道指挥。

总而言之,你是被套住了。要解脱也可以,你得采取主动,慢慢脱身。不是有脱身政策吗?在朝鲜有脱身政策,现在我看形成了金、马的脱身政策。他们那一班子实在想脱身,而且舆论上也要求脱身。脱身者,是从绞索里面脱出去。怎么脱法呢?就是这十一万人走路。台湾是我们的,那是无论如何不能让步的,是内政问题。跟你的交涉是国际问题,这是两件事,你美国跟蒋介石搞在一起,这个化合物是可以分解的。比如电解铝,电解铜,用电一解,不就分离了吗?蒋介石这一边是内政问题,你那一边是外交问题,不能混为一谈。现在五大洲,除了澳洲,四大洲美国都想霸占。首先是北美洲,那主要是它自己的地方,它有军队,然后是中南美洲,虽然没有驻军,但是他要“保护”的。再加上欧洲、非洲、亚洲,主要是欧亚非,主力在欧亚两洲。这么几个兵,分得这么散,我们不晓得它这个仗怎么打法。所以,我总是觉得,它是霸占中间地带为主。至于我们这些地方,除非是社会主义阵营出了大乱子,确有把握。一来,我们苏联、中国就全部崩溃,否则我看他不敢来的。除了我们这个阵营以外,它都要霸占。一个拉丁美洲,一个欧洲,一个非洲,一个亚洲,还有个澳洲,澳洲也在军事条约上跟它连起来了,听它的命令。它用“反共”的旗帜取得这些地方好些,还是真正的反共好些?所谓真正反共,就是拿军队来打我们,打苏联。我说,没有那么蠢的人。它只有几个兵调来调去。黎巴嫩事件发生,从太平洋调去,到了红海地方,形势不对,赶快回头.到马来亚登陆,名为休息几天,十七天不吭声,后头他一个新闻记者自己宣布是管印度洋的,这一来,印度洋大家都反对。我们这里一打炮,这里兵不够,它又来了。台湾这些地方早一点解脱,对美国比较有利,它赖着不走,就让它套到那里,无损于大局,我们还是搞大跃进。

今年要争取钢一千一百万吨,比去年翻一番。明年增加××万,争取××万吨。后年再搞××万吨,不是××万吨吗?苦战三年,××万吨钢。那么全世界除了苏联同美国,我们就是第三位。苏联去年就是五千万吨,加三年,他可以搞六千万吨。我们苦战三年。有可能超过××万吨,接近苏联,再加×年,到××年,可能出八千万到一亿吨,接近美国(美国因为它经济恐慌,那个时候也许只有一亿吨)。第二个五年计划就要接近或赶上美国。再加×年,×年,搞一亿五千万吨,超过美国,变成天下第一。老子天下第一不好,钢铁天下第一,有什么不好?粮食,苦战三年,今年可能是××到××亿斤,明年翻一番,就可能是××亿斤。后年就要放低步调了。因为粮食还是要找出路。粮食主要是吃,此外也要找工业方面的出路,例如:搞酒精作燃料,经过酒精搞橡胶,搞纤维,搞塑料,等等。

至于紧张局势。也许还可以讲几句,你搞紧张局势,你以为对你有利呀,不一定.紧张局势调动世界人心,都骂美国人。中东紧张局势,大家骂美国人,台湾紧张局势,只是大家骂美国人,骂我们的比较少。美国人骂我们,蒋介石骂我们,李承晚骂我们,也许还有一点人骂我们,主要就是这三家。英国是动摇派。军事不参加,政治上听说他相当同情。因为他有个约旦问题,你不同情一下,美国人如果在黎巴嫩撤退,英国在约旦怎么办呀?尼赫鲁发表了声明,基本上跟我们一致的,赞成台湾这些东西归我们,不过希望和平解决。这同中东各国可是欢迎啦,特别是一个阿联,一个伊拉克,每天吹,说我们这个事情好。因为我们这一搞,美国人对他们那里的压力就轻了。

我想可以公开告诉美国人民,紧张局势比较对西方国家不利,对于美国不利。利在什么地方呢?中东紧张局势对于美国有什么利?对于英国有什么利?还是对于阿拉伯有利些,对于亚洲、非洲、拉丁美洲以及其他各洲爱好和平的人民有利些。台湾的紧张局势究竟对谁有利些呢?比如对于我们国家,我们国家现在全体动员,如果说中东事件有三、四千万人游行示威,开会,这一次大概搞个三亿人口,使他们得到教育,得到锻炼。这个事情对于各民主党派的团结也好吧,各党派有一个共同奋斗目标,这样一来,过去心里有些疙瘩的有些气的,受了批评的,也就消散一点吧。就慢慢这样搞下去,七搞八搞,我们大家还不就是工人阶级了。所以帝国主义自己制造出来的紧张局势,结果反而对于反帝国主义的我们几亿人口有利,对于全世界爱好和平的人民、各阶级、各阶层、政府,我看都有利。他们得想一想,美国总是不好,张牙舞爪。十三个航空母舰,就来了六个,其中有大到那么大的,有什么六万五千吨的,说是要凑一百二十个船,第一个最强的舰队。你再强一点也好。你把你那四个舰队统统集中到这个地方我都欢迎。你那个东西横直没用了的,统统集中起来,你也上来不得。船的特点就在水里头,不能上岸。你不过是在这个地方摆一摆,你越打,越使全世界的人都知道你无理。


