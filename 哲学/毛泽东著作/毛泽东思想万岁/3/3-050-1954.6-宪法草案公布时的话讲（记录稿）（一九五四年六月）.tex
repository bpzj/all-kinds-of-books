\section[宪法草案公布时的话讲(记录稿)(一九五四年六月)]{宪法草案公布时的话讲(记录稿)}
\datesubtitle{(一九五四年六月)}


一、恰当地正确地总结了经验,反帝反封建革命的经验;建国以后各方面建设的经验;同时又总结了反面的经验(历史上各种宪法),并且总结了外国的经验,但以自己民族的经验为主,资产阶级有革命的东西,也不能全部否定。

二、恰当地正确地结合了原则性与灵活性。民主主义与社会主义原则贯彻在各方面。灵活性,统一战线,逐步实行。不逐步就“左”,不实行,就右,社会主义的原则性与逐步实行的灵活性相结合。少数民族有他的共同性和特殊性,根据特殊情况制定单行条例。

宪法可以实行,而且必须实行,尤其是国家工作人员,不实行,要弹劾(共产党员犯罪,要开除党籍)。有了宪法,全国人民就有了清楚而正确的道路,这是全国人民的总章程。这在国际上还发生很大的影响,宪法有民族性,又有国际性。

究竟要多少年进入社会主义,这话不要讲死了,十五年只能打一个基础。团结全国人民,团结一切可以团结的力量。

国家主席的权好像太小了。有些条文没有写进去,并不是谦逊,而是不科学。科学,多了也不行,少了也不行,该多少就多少。对古人今人都不能迷信,对了就信,不对就不信,我们不提倡崇拜偶像。


