\section[中央关于调查研究的决定(一九四一年八月一日)]{中央关于调查研究的决定}
\datesubtitle{(一九四一年八月一日)}


二十年来,我党对于中国历史,中国社会与国际情况的研究,虽然是逐渐进步的,但仍然是非常不足,粗枝大叶,不求甚解,自以为是,主观主义,形式主义的作风,仍然在党内严重地存在着。抗战以来,我党在了解日本,了解国民党,了解社会情况诸方面是大进一步了,主观主义,形式主义作风也减少了。但所了解者仍然多属粗枝大叶的,漫画式的,缺乏系统的周密的了解,主观主义与形式主义作风并未彻底消灭。对于二十年来由于主观主义与形式主义,由于幼稚无知识,使革命工作遭受损失的严重性,尚未被全党领导机关及一切同志所彻底认识。到延安来做报告工作的同志,其中的多数,对于他们自己从事工作区域的内外环境,不论在社会阶级关系方面,在敌伪方面,在友党友军方面,在自己工作方面,均缺乏系统的周密了解。党内许多同志,还不了解没有调查就没有发言权这一真理。还不了解系统的周密的社会调查,是决定政策的基础。还不知道领导机关的基本任务,就在于了解情况和掌握政策,而情况如不了解,则政策势必错误,还不知道,不但日本帝国主义对于中国的调查研究,是如何的无微不到就是国民党对于国内外情况,亦此我党所了解的丰富的多。还不知道,粗枝大叶,自以为是的主观主义作风,就是党性不纯的第一个表现,而实事求是,理论与实际密切联系,贝以是一个党性坚强的党上的起码态度。我党现在已是一个担负着伟大革命任务的大政党,必须力戒空疏,力戒肤浅,扫除主观主义作风,采取具体办法,加重对于历史,对于环境,对于国内外,省内外县内外具体情况的调查研究,方能有效的组织革命力量,推翻日本帝国主义及其走狗的统治,为此目的,特决定办法如下:

(一)中央设置调查研究机关,收集国内外政治、军事、经济、文化及社会阶级关系各方面材料,加以研究,以为中央工作的直接助手。

(二)各中央局、中央分局、独立区域的区党委或省委、八路军、新四军之高级机关,各根据地各级政府,均须设置调查研究机关,收集有关该地区敌我友政治、军事、经济、文化及社会阶级关系各方面的材料,加以研究,以为各该地工作的直接助手,同时供给中央以材料。

(三)关于收集材料的办法,举例如下:第一,收集敌友我三方面关于政治、军事、经济、文化以及社会阶级关系的各种报纸……

<p align="right">(毛泽东同志为中共中央起草的关于调查研究的决定,抄自中国革命历史搏/p)


