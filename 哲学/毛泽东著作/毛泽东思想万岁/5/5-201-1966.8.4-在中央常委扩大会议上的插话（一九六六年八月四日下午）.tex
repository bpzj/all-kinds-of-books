\section[在中央常委扩大会议上的插话(一九六六年八月四日下午)]{在中央常委扩大会议上的插话(一九六六年八月四日下午)}
\datesubtitle{(一九六六年八月四日)}


在治安时代以后的北洋军阀,后来的国民党,都是镇压学生的。现在的共产党也镇压学生运动,这与陆平、蒋南翔有何区别?中央下命令停课半年,专门搞文化大革命,等学生起来了又镇压他们。不是没有人提过不同意见,就是听不进。对另一种意见却是津津有味,说的轻一些是方向性的问题,实际上方向问题就是中心问题,是路线问题,违反马克思主义的,这一直是马克思主义要解决的问题,感到危险。自己下命令要学生起来革命,大家起来又加以镇压。所谓方向路线,所谓相信群众,所谓马克思主义等等都是假的,已经是多年如此,如果碰上这类的事情就要爆发出来,明明白白就站在资产阶级方面,反对无产阶级。说反对新市委就是反党,新市委镇压学生运动,为什么不能反对。

我是没有下去蹲点的,有人越蹲点越站在资产阶级方面,反对无产阶级。规定班与班,系与系,校与校之间一概不准来往,这是镇压学生,是恐怖,来自中央,有人对中央六月十八日的批语有意见,说不好讲。北大聂元梓等七人大字报是二十世纪六十年代的巴黎公社宣言——北京公社。贴大字报是很好的事,应该给全世界人民知道嘛!而雪峰报告中,却说党有党纪,国有国法,要内外有别。大字报不要贴在大门外,别让外国人知道,其实除了机密地方,例如国防部,公安部等地不让外国人看外,其他地方有什么要紧?在无产阶级专政条件下也允许群众请愿,示威游行和告状。而且言论,集会结社,出版自由都写在宪法里。从这次镇压学生的文化大革命行动看来,我不相信有真正民主,真正马克思主义,而是站在资产阶级方面反对无产阶级文化大革命。团中央不仅不支持青年学生运动,反而镇压学生运动,我看该处理。

