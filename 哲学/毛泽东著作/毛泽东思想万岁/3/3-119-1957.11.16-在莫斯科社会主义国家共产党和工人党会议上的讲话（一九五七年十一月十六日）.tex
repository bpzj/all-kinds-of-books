\section[在莫斯科社会主义国家共产党和工人党会议上的讲话(一九五七年十一月十六日)]{在莫斯科社会主义国家共产党和工人党会议上的讲话}
\datesubtitle{(一九五七年十一月十六日)}


我认为我们的宣言是好的。我们用了一个很好的方法达到目的,这就是协商的方法。坚持了原则性,又有灵活性,是原则性、灵活性的统一。这么一种进行协商的气氛现在形成了。在斯大林的后期不可能。我们没有强加于人。在人民内部,尤其是在同志内部,采取强加于人的态度是不好的。我们现在用说服的方法代替了压服的方法。费的时间不算少,但是这点时间是需要的。我们采取协商的方法并不是主张无政府主义,我们不是辩论的俱乐部。我们的方法是又有中心,又有大家,中心与大家的统一。没有中心,比如没有苏联共产党,那么就会变成无政府主义;没有大家提意见,只是一家提意见,那么就总不会完全。现在是又有中心,又有我们大家;在一种意义上,也可以说又有集中,又有民主。不能说我们这一次会议没有民主。我认为有充分的民主。

这个宣言是正确的。它没有修正主义或者机会主义的因素。将来我们见马克思的时候,他问我们,你们搞了一个什么样的宣言?他会怎样评价这个宣言呢?有两种可能性:一种可能是:他老先生发一顿脾气,说你们搞坏了,有机会主义的因素,违背了我的主义。第二种可能是:他说不坏,不是机会主义的,是正确的。也许列宁会出来为我们讲话。他说马克思、恩格斯呀,你们两位死得早,我死得迟,我熟悉他们,他们现在会作工作了,他们成熟了。你看,苏联共产党,我的后代,他们召集这样一次会议,是召集得很好的。不仅各国要感谢他们,我也要感谢他们,感谢我的后代。他也会说,起草委员会作了辛勤的劳动。这个宣言有没有冒险主义?马克思他们会这样讲?无非也是两种可能:一说有,一说没有。但是我估计,他们会说没有。研究一下,这里头有甚么冒险主义呀?我们力求和平,力求团结,看不见冒险主义。由此观之。冒险主义的性质也没有,机会主义的性质也没有。那么是一个什么宣言呢?是一篇马克思列宁主义性质的宣言,这个宣言总结了几十年的经验。有些经验是从痛苦中得来的。这些痛苦教育了我们。我们不要对于那些痛苦生气。相反,我们要感谢这些痛苦。因为它使我们开动脑筋,想一想,努力去避免那些痛苦。果然,我们就避免了那些痛苦。是不是呢?


