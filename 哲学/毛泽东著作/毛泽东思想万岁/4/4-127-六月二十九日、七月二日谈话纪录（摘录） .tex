\section[六月二十九日、七月二日谈话纪录(摘录) ]{六月二十九日、七月二日谈话纪录(摘录) }


表扬好人好事,批评坏人坏事

有鉴于去年许多领导同志,县、社干部,对于社会主义经济问题这不大了解,不懂得经济发展规律,有鉴于现在工作中还有事务主义,所以应该好好读书。中央、省市、地委各级的委员、包括县委书记,要读政治经济学的书,要给县、社干部编三本书:一本是关于“好人好事”的书,收集在去年大跃进中,敢于坚持其理,不随风倒,工作做很好,不谎报,不浮夸,实事求是的例子。一本是关于“坏人坏事”的书,收集专门说假话,违法乱纪,或者工作中犯了严重错误的例子。第三本是中央从去年到现在的各种指示文件,有系统的编一本书。成绩伟大,问题不少,前途光明

国内形势如何?总的说来,成绩伟大,问题不少,前途光明。基本问题是:①综合平衡;②群众路线;③统一领导;④注意质量。其中最主要的是综合平衡和群众路线问题。宁肯少些,但要好些、全些,各种各样都要有,农业中粮、棉、油、麻、丝、茶、糖、菜、烟、果、药、杂都要有。工业中要有轻工业、重工业,其中又要各样都有,去年集中力量搞小高炉,其他都丢了,这样搞法不行。大跃进的重要教训之一,主要缺点是没有平衡,说了两条腿走路,并举,实际上没有兼顾,整个经济中,综合平衡是个根本问题,有了综合平衡,才能有群众路线。

三种平衡:农业本身的农林牧副渔;工业内部的各部门,各个环节;工业和农业。做好这三种平衡工作,才可能正确处理整个国民经济的比例关系。把农业放在首要地位

过去安排经济计划的秩序是重轻农,今后恐怕要倒过来。现在是否是农轻重呢?也就是说,要强调把农业搞好,要把重、轻、农、商、交的秩序改为农、轻、重、交、商。这样提,还是首先发展生产资料,并不违反马克思主义。现在看来,陈云同志的意见是对的。要先把衣、食、住、用、行五个字安排好了之后,使大家过得舒服,就不会有人说闲话,骂我们,这样有利于建设,同时国家也可以多积累。

关于农村的几项具体政策

三定政策:定产、定购、定销,群众要求恢复,看来非恢复不可,可以三年不变,如果可以定,究竟定多少,增产部分是否可以征四留六,有灾减,自留地不征税,这次会议要议一下。

要恢复农村初级市场。

要使生产队成为半核算单位。加强中央统一领导,反对半无政府主义

积极性有两种,一种是实事求是的积极性,一种是盲目的积极性,红军的三大纪律,有两条可以普遍用:“一切行动听指挥”就是要统一领导,反对无政府主义;“不拿群众一针一线”,就是不搞一平,二调。体制问题:现在有些半无政府主义。“四权”过去下放多了一些,快一些,造成混乱。应该强调一下统一领导,中央集权。下放的权力,要适当收回,对下要适当控制,反对半无政府主义。过死不好,过活也不好,现在看来,不可过活。


