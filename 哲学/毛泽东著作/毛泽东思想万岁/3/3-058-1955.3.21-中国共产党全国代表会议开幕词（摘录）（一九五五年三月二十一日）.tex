\section[中国共产党全国代表会议开幕词(摘录)(一九五五年三月二十一日)]{中国共产党全国代表会议开幕词(摘录)}
\datesubtitle{(一九五五年三月二十一日)}


所谓困难,无非是社会的敌人和自然界给予我们的。我们知道,帝国主义,国内反革命分子以及他们在我们党内的代理人等等,都不过是垂死的力量,而我们则是新生的力量,真理是在我们方面。对于他们,我们从来就是不可战胜的。只要想一想我们的历史就会懂得这个道理。我们在一九二一年刚刚建党的时候,只有几十个人,那样渺小,后来发展起来,居然把国内强大的敌人给打倒了。自然界这个敌人也是有办法制服它的。

<p align="center">×××</p>

不论在自然界和在社会上,一切新生力量,就其性质来说,从来就是不可战胜的。而一切旧势力,不管他们的数量如何庞大,总是要被消灭的。因此我们可以藐视而且必须藐视人世遭逢的任何巨大的困难,把它放在“不在话下”的位置。这就是我们的乐观主义。这种乐观主义是有科学根据的。只要我们更多地懂得马克思列宁主义,更多地懂得自然科学,一句话,更多地懂得客观世界的规律,少犯主观主义的错误,我们的革命工作和建设工作,是一定能够达到目的的。

<p align="center">×××</p>

帝国主义势力还是在包围着我们,我们必须准备应付可能的突然事变。今后帝国主义如果发动战争,很可能像第二次世界大战时期那样,进行突然袭击。因此,我们在精神上和物质上都要有所准备,当着突然事变发生的时候,才不至于措手不及。这是一方面。另一方面,国内反革命残余势力的活动还很猖獗,我们必须有计划地、有分析地、实事求是地再给他们几个打击,使暗藏的反革命力量更大地削弱下来,借以保证我国社会主义建设事业的安全。如果我们在上述两方面都做了适当的措施,就可以避免敌人给我们的重大危害,否则,我们可能要犯错误。


