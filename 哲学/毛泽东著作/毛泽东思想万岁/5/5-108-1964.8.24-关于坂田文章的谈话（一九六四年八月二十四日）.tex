\section[关于坂田文章的谈话(一九六四年八月二十四日)]{关于坂田文章的谈话}
\datesubtitle{(一九六四年八月二十四日)}

主席:今天我找你们来就是想研究一下坂田的文章。坂田说基本粒子不是不可分的,电子是可分的。他们这样说,是站在辩证唯物主义立场上的。

世界是无限的。世界在时间上、在空间上都是无穷无尽的,在太阳系外有无数个恒星,它们组成银河系。银河系以外,又有无数个银河系。宇宙从大的方面来看,是无限的。宇宙从小的方面来看,也是无限的。不但原子可分,原子核也可分,而且可以无限的分割下去。庄子讲:“一尺之棰,日取其半,万世不竭。”这是对的。因此,我们对世界的认识也是无穷无尽的。要不然,物理学这门科学就不再发展了。如果我们的认识是有穷尽的,我们已经把一切都认识到了,还要我们这些人干什么?

主席:人对事物的认识,总要经过多次的反复,要有一个积累的过程。要积累大量的感性材料,才会引起从感性认识到理性认识的飞跃。关于从实践到感性,再从感性到理性的飞跃的道理,马克思和恩格斯都没有讲清楚。列宁也没有讲清楚,列宁的《唯物主义与经验批判主义》只讲清楚了唯物论,但是没有完全讲清楚认识论。最近艾思奇在高级党校讲话说到这一点是对的。这个道理,中国古人也没有讲清楚。老子、庄子没有讲清楚。墨子讲了些认识论方面的问题,也没有讲清楚。张载、李卓吾、王船山、谭嗣同都没有讲清楚。什么叫哲学?哲学就是认识论,别的没有。双十条第一个十条前面那一段是我写的。我讲了物质变精神,精神变物质。我还讲了哲学,一次不要讲得太长,最多一小时就够了。\marginpar{\footnotesize 158} 多讲,越讲越糊涂。我还讲哲学要从课堂书斋里解放出来。我这些话触到了有些人的痛处,他们出来搞“合二而一”反对我。

主席:现在我们对许多事情还认识不清楚。认识总是在发展,有了大望远镜,我们看到的星星就更加多了。说到太阳系和地球,一直到现在还没有推翻康德的星云假说——地球、太阳都是由很热很热的气体冷凝而成的。我们的地球大概还在青年时期,我们的地球已变得愈来愈大。因为每天都有不少东西投到地球上来,如陨石、阳光等。太阳大概已经到中年,现在的太阳已经不那么热了。如果地面上的阳光那么强,有一百度,人怎么受得了?太阳表面温度有五、六千度,在太阳表面上面还有一层温度有一、二千度。如果说对太阳我们搞不十分清楚,从太阳到地球中间这一大块地方也还搞不清楚。现在有了人造卫星,这方面的认识就渐渐多起来了。地球上气候变化也不清楚,这也要研究。关于冰川问题,科学家还在争论。李四光是主张每隔百万年左右有一个冰川时期。到那时候,生物界又会起一个很大的变化。古时候的恐龙就经受不了冰川时期的寒冷而灭绝了。人是产生在最近两次冰川之间的,以后来到冰川时期,对人说来是一个问题,人要准备对付下一个冰川的来临。

×××:主席方才说到望远镜,使我想起一个问题,我们能不能把望远镜、人造卫星等等概括成“认识工具”这个概念?

主席:你说的这个“认识工具”的概念有些道理。认识工具当中要包括镬头、机器等等。人的认识来源于实践。我们用镬头、机器等等改造世界,我们的认识就深入了。工具是人器官的延长。镬头就是手臂的延长,望远镜是眼睛的延长,身体、五官都可以延长。富兰克林说人是制造工具的动物。中国人说,人为万物之灵。动物也有灵长类,但是猴子不知道制造棍子打果子。在动物的头脑里,就没有概念。

×××:哲学书里通常只以个人作为认识的主体,但是在实际生活中认识的主体不是一个一个的人,而常常是一个集体,如我们党就是一个认识的主体,这样的看法对不对?

主席:阶级就是一个认识的主体。最初工人阶级是一个自在阶级,那时他对资本主义没有认识。以后就从自在阶级发展到自为阶级。这时候它对资本主义就有了认识。这就是以阶级为主体的认识的发展。

主席:地球上的水,也不是一开始就有的。最早的时候,地球温度那么高,水是不能存在的,那时候水就会爆炸成氢和氧。《光明日报》上前两天有一篇文章讲氢和氧化合成水要经过几百万年。傅鹰讲要几千万年。不知道那篇文章的作者同傅鹰讨论过没有?有了水,生物才从水里产生出来。人就是从鱼变的。人胎有一个发展阶段就像鱼。

主席:一切个别的、特殊的东西都有它的产生、发展与死亡。每个人都要死,因为它是产生出来的。人要有死,张三是人张三要死。他们见不到两千年前的孔夫子,因为他一定要死。人类是产生出来的,因为人类也会灭亡。地球是产生出来的,地球也会灭亡。不过我们说的人类灭亡,地球灭亡,和基督教徒的世界末日不一样。我们说人类灭亡,地球灭亡,是有比人类更进步的东西来代替人类,是事物发展过程更高阶段。我说马克思主义也有它的发生、发展与灭亡。这好像是怪话。但既然马克思主义说一切发生的东西有它的灭亡,难道这话对马克思主义本身就不灵?说它不会灭亡是形而上学。当然,马克思主义的灭亡,是有比马克思主义更高的东西来代替它。

主席:事物是在运动中,关于地球绕着太阳转,自转成日,公转成年,在哥白尼时代,欧洲只有三个人相信,哥白尼,伽利略,刻卜勒。在中国一个人也没有。不过宋朝有个辛弃疾,他写了一首诗里面说,当月亮从我们这里下去的时候,他照亮着别的地方*。晋朝的张华(号张茂先)在他的一首诗里写到“太仪斡运,天回地游”。这首诗收在《古诗源》里。

主席:什么东西都是既守恒又不守恒。宇宙守恒,后来在美国的中国科学家李政道和扬振宁说它不守恒。质量守恒,能量守恒,是不是也是这样?世界上没有绝对不变的东西。变,不变,又变,又不变,组成了宇宙。守恒又不守恒,这就是既平衡又不平衡。也还有平衡完全破裂的情形。发电机是一个说明运动转化的很好的例子,在煤燃烧时运动形态是什么?

×××:是化合物中原子外层电子改变运动轨道时放出来的能。

主席:这种形态转化为使水蒸汽体积膨胀的运动。

×××:这是分子运动而产生功能。

主席:然而又使发电机的转子旋转,这是机械运动,最后在铜线、铅线发出电来。

世界上一切都在变,物理学也在变,牛顿力学也在变。世界上从没有牛顿力学到有牛顿力学,以后又从牛顿力学到相对论。这本身就是辩证法。

事情往往出在冷门。孙中山是学医的,后来搞政治。郭沬若最初也是学医的,后来成为历史学家。鲁迅也是学医的,后来成为大文学家。我搞政治也是一步一步来的。我读了六年孔夫子的书,上了七年学堂,以后当小学教员,又当了中学教员。当时我根本不知道什么是马克思主义。马克思、恩格斯的名字就没有听说过。只知道拿破仑、华盛顿。我搞军事更是这样,我当过国民革命军的政治部的宣传部长,在农民讲习所也讲过打仗的重要,可就是没想到自己去搞军事,要去打仗。后来自己带人打起仗来,上了井冈山。在井冈山先打了个小胜仗,接着又打了两次大败仗。于是总结经验。总结了十六个字的打游击的经验:“敌进我退,敌驻我扰,敌疲我打,敌退我追”。谢谢蒋委员长给我们上课,也要谢谢党内的一些人,他们说我一点马克思主义也没有,而他们是百分之百的布尔什维克,可是这些百分之百的布尔什维克却使白区的党损失百分之百,苏区损失百分之九十。

主席:我们这些人不生产粮食,也不生产机器,生产的是路线、政策。路线、政策不是凭空产生出来的,比方说“四清”、“五反”就不是我们发明的,而是老百姓告诉我们的。“四清”、“五反”这个政策产生出来,还要谢谢广东的一个反革命,他写信给××和××,要我交出政权、军队。

科学家要同群众联盟,要同青年工人、老工人密切联系。我们的脑子是个加工厂。工厂设备要更新,我们的脑子也要更新。我们身体的各种细胞都不断更新,我们身体的皮肤上的细胞早就不是我们生下来的时候的皮肤上的细胞了,中间不知道换了多少次。

中国知识分子有几种。工程技术人员接受社会主义要好一些。学理科的其次。学文科的最差。你们那里的冯定,我看就是修正主义者,他写的书里讲的是赫鲁晓夫那一套。

主席:曹雪芹写《红楼梦》还是想补天,想补封建制的天。但是《红楼梦》里写的却是封建家族的衰落。可以说是曹雪芹的世界观和他的创作发生矛盾,曹雪芹的家是在雍正年里衰落的。康熙有许多儿子,其中一个是雍正。雍正搞特务机关压迫他的对手,把康熙的另外两—个儿子,好像是第九、十个儿子,一个改姓猪,一个改姓狗。

主席:分解很重要。庖丁解牛。恩格斯在谈到医学的时候,也非常重视解剖学。医学是建筑在解剖学的基础上面的。

细胞的起源问题要研究一下。细胞有细胞核,细胞质和细胞膜。细胞是有结构的,在细胞以前一定有非细胞。细胞之前究竟是什么?究竟怎样从非细胞变为细胞?苏联有个女科学家研究这个问题没有结果。

×××:我国在罗马举行的国际外科会议上报告了断手再植后,美国人说他们摸不清中国科学技术的底,有点害怕我们。

主席:有点怕,是好事,不怕倒不好了。我们有点怕美国,因为美国是我们的敌人。美国有点怕我们,说明我们是美国的敌人,而且是有力量的敌人。在科学技术上应该注意保密。不让他们把我们的底摸去。”

*木兰花慢中秋饮酒将旦,客谓前人诗词有赋待月,无送月者,因用天向体赋。

可怜今夕月,向何处,去悠悠?是别有人间,那边才见,光影东头?

是天外,在汗漫,但长风浩浩送中秋?飞镜无根谁系,妲娥不嫁谁留?