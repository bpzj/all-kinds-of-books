\section[复××的信(一九四八年二月十七日)]{复××的信}
\datesubtitle{(一九四八年二月十七日)}


看到你的报告以后,对你所提出的问题,提出如下意见:

在老区凡是已实行平分土地,地主富农经济基础已经消灭,只是还不十分彻底,须酌量调剂土地的地方,即不应再去平分土地,而是应该采取合理的抽补办法来满足部分农民土地不足的要求,此种地区如再去平分土地,那是错误的。在这种地区,只在农民内部组织贫农小组来保障贫雇农的利益,在农民委员会中,在农民代表会议及政府中,能使贫雇农与新中农共占三分之二,保障他们在农村中的领导权,并同时又使老中农能占三分之一的地位。这样做法是很好的。在贫雇农占少数,新老中农占多数的地区,也去组织贫农团,并硬要贫农团去指挥一切,这是违反全国土地会议路线的,是极端冒险的命令主义,如果你们那里的工作团或工作组还是有这样做的应立即停止,并将他们调回训练训好后再回去按照具体情况重新做起来,此事希很好注意。

其次,在老区与新区之处,还有介于二者之间的半老区。在这种半老区里,土地问题没有解决得像老区那样彻底,但它较之完全没有或大体上没有解决的新区,又更好些。在这种半老区的工作方针,应同老区有区别,也同新区有区别的,望你们加以研究,并将研究的结果函告我。

