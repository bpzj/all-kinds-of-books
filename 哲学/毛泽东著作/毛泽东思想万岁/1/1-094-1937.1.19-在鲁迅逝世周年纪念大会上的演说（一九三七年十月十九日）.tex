\section[在鲁迅逝世周年纪念大会上的演说(一九三七年十月十九日)]{在鲁迅逝世周年纪念大会上的演说}
\datesubtitle{(一九三七年十月十九日)}


同志们:

今天我们主要的任务,是先锋队的任务。当着这伟大的民族自卫战争迅速地向前发展的时候,我们需要大批的知识分子来领导,需要大批精练的先锋队来闯道路。这种先锋分子是胸怀坦白的、忠诚的、积极的与正直的,他们并不谋私利的,唯一的为着民族与社会的解放,他们不怕困难,在困难面前总是坚定的,勇往直前,他们不是狂妄分子不是风头主义者,而是脚踏实地富于实际精神的人们,他们在革命道路上起着响导的作用。目前的战局,如果只是单纯政府与军队的抗战,没有广大的人民参加,这是绝对没有最后胜利的保障的。我们现在需要造就一大批为民族解放而斗争到底的先锋队,同时又是最彻底的民族解放的先锋队,我们要为完成这一任务而奋战到底。

今天我们纪念鲁迅先生,首先要认识鲁迅先生,要懂得他在中国革命历史中所占的地位。我们纪念他,不仅是因为他是位优秀的作家而且是因为他站在民族解放的前列,他把全部力量都献给了革命斗争。他并不是共产党的组织上的一个人,然而他的思想,行动等著作都是马克思主义化的。尤其在他的晚年,表现了更年轻的力量。他一贯地不屈不挠地与封建势力和帝国主义作坚定的斗争。在敌人压迫他摧残他的恶劣的环境里,他挣扎着,反抗着。正如陕北公学的同志们,能够在这样坏的物质生活里勤谨地学习革命理论一样,充满了艰苦奋斗的精神。陕北公学的一切物质设备都不好,但这儿有真理,讲自由,是创造革命先锋的场所。

鲁迅是从溃退的封建社会中出来,但他会杀回马枪。朝着他经历过来的腐败社会进攻。朝着帝国主义的恶劣势力进攻。他用他那一支又泼辣,又幽默、又锋利的笔。去画出了黑暗势力的鬼脸。去画出了丑恶的帝国主义的鬼脸,他简直是一个高等的画家。他近年来站在无产阶级与民族解放的立场上,为真理和自由而斗争。

鲁迅先生的第一个特点,是他的政治远见。他用显微镜和望远镜视察社会,所以看得远,看得真。他在一九三六年就大胆地指出托派匪徒的危险倾向。现在的事实完全证明,他的见解是那样的稳定,那样的清楚。托派成为汉奸组织而直接拿日本特务机关的津贴,已经是很明显的事情了。

鲁迅在中国的价值,据我看要算是中国的第一等圣人。孔夫子是封建社会的圣人,鲁迅则是中国的圣人,我们为了永久纪念他,在延安成立了“鲁迅图书馆’,在延安成立了“鲁迅师范学校”,使后来的人们可以想见他的伟大。

鲁迅的第二个特点,孰是他的斗争精神。刚才已经提到,他在黑暗与暴力的进袭中,是一株独立支持的大树,不是向两旁偏倒的小草。他看清了政治的方向,就向着一个目标奋勇地斗争下去,决不中途投降妥协。有些不彻底革命者例如考茨基和普列汉诺夫,就是很好的例子。在中国,此等人也不少。正如鲁迅先生所说,最后大家都是左的,革命的;反到压迫来了,马上有人变节,并把同志拿出来献给敌人作为见面礼(我记得大意如此)。鲁迅痛恨这种人,同这种人做斗争、随时教育着训练着他所领导的文学青年。叫他们坚决斗争,打先锋开辟自己的路。

鲁迅的第三个特点,是他的牺牲精神。他一点也不畏敌惧人对他的威胁,利诱与残害,他一点也不避锋芒地把钢刀一样的笔剌向他所憎恨的一切。他往往是站在战士的血泪中坚韧地反抗着,呼啸着前进,鲁迅是一个彻底的现实主义者,他丝毫不妥协,他具备了坚决心,他在一篇文章里,主张打落水狗。他说,如果不打落水狗,他一定跳起来,不仅要咬你,而且最低限度要溅你一身污泥。所以他主张打到底,他一点也没有假慈悲的伪君子的色彩,现在日本帝国主义这条疯狗,还没有被我们打下水,我们要一直打到他不能翻身退出中国国境为止。我们要学习鲁迅的这种精神,运用到全中国去。

综合了上述的这几个条件,形成了一种伟大的“鲁迅精神”。鲁迅的一生就完全贯穿了这种精神,所以他在艺术上成为了一个了不起的作家,在革命队伍中是一个很优秀,很老练的先锋分子。我们纪念鲁迅,就要学习鲁迅的精神。把他带到全国各地的统战队伍中去使用,为中华民族的解放而斗争。

(注):本文系自1938年3月重庆出版的半月刊《七月》

