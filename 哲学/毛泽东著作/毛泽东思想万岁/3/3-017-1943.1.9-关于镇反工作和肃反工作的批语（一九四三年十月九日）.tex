\section[关于镇反工作和肃反工作的批语(一九四三年十月九日)]{关于镇反工作和肃反工作的批语}
\datesubtitle{(一九四三年十月九日)}


一个不杀大部不抓是此次反特务斗争中必须坚持的政策。一个不杀则特务敢于坦白,大部不抓(不捉),则保卫机关只处理小部,各机关学校自己处理大多数。须使各地党委坚持此种政策。

<p align="right">(一九五○年九月二十七日)</p>

保卫工作必须特别强调党的领导作用,并在实际上受党委直接领导,否则是危险的。

<p align="right">(一九五○年十二月十九日)</p>

对镇压反革命分手,请注意打得稳、打得准、打得狠,使社会各界没有话说。

<p align="right">(一九五一年一月十七日)</p>

所谓打得稳,就是要注意策略。打得准,就是不要杀错。打得狠,就是要坚决地杀掉一切应杀的反动分子(不应杀者,当然不杀)。

<p align="right">(一九五一年一月二十四日)</p>

为了不致弄错,使自己陷于被动,对尚无证据的特务及会道门头子,应当进行侦察,取得确证,而不要随便捕人杀人。

<p align="right">(一九五一年二月二十八日)</p>

所谓胁从不问,是指被迫参加而未作坏事,或未作较大坏事者。至于助恶有据,即是从犯,应当判罪,如主犯判死刑,从犯至少判徒刑,有些罪大的从犯应判死刑,不在胁从不问之列。

<p align="right">(一九五一年三月九日)</p>

对那些已经实现了彻底镇压方针的地方,则要停一下,不要多捉多杀了。无论什么地方,都要有计划.讲策略,作宣传,不杀错,这些也是当然的。

<p align="right">(一九五一年三月二十四日)</p>

镇反是一场伟大的斗争,这件事做好了,政权才能巩固。

<p align="right">(一九五一年三月二十四日)</p>

镇反包括(一)社会上的反革命;(二)隐藏在军政系统旧人员和新知识分子中的反革命;(三)隐藏在党内的反革命。镇压这三方面的反革命,当然要有步骤,不能同时并举,但是对于党政军的某些最重要部门。特别是公安部门则须及时清理,将可疑分子须作处置,使这些机关掌握在可靠人员手里,则是完全必要的。

<p align="right">(一九五一年三月三十日)</p>

请你们对镇反工作,实行严格控制,务必谨慎从事,务必纠正一切草率从事的偏向。我们一定要镇压一切反革命,但是一定不可捕错杀错。

<p align="right">(一九五一年三月三十日)</p>

镇压反革命无论何时都应当是准确的精细的有计划的有步骤的,并且完全应由上面控制。

<p align="right">(一九五一年三月三十日)</p>

凡工作好坏,应以群众反映如何为断。

<p align="right">(一九五一年四月二日)</p>

镇压反革命必须严格限制在匪首、惯匪、恶霸、特务、反动会道门头子等项范围之内,不能已将小偷、吸毒犯、普通地主、普通国民党党团员、普通国民党军官也包括在内。判处死刑者,必须是罪重者,重罪轻判是错误的,轻罪重判也是错误的。

<p align="right">(一九五一年六月十五日)</p>

“缓期二年执行”的政策,决不应解释为对于负有血债或有其它重大罪行人民要求处死的罪犯而不处死,如果这样做,那就是错误的。我们必须向区村干部和人民群众解释清楚。对于罪大恶极民愤甚深不杀不足以平民愤者必须处死,以平民愤。只对那些民愤不深,人民并不要求处死,但又犯有死罪者,方可判处死刑缓期二年执行强迫劳动以观后效。

<p align="right">(一九五一年十月一日)</p>

除非我们采取步骤去肃清反革命分子的活动,否则人民的国家就会受到危害。

<p align="right">(一九五九年九月十八日)</p>

看法妥当,让他们活动,注意观察,大有可为。他们是在如来佛手掌中,跳不出去的。你们应当当作一件大事去办。积极而又艺术地去做观察和侦察工作。

对《秘书室关于处理群众来信的报告》的批示

<p align="right">(一九五一年五月十六日)</p>

必须重视人民的来信,要给人民以恰当的处理,满足群众的要求,要把这件事看成是共产党和人民政府加强和人民联系的一种方法,不要采取掉以轻心置之不理的官僚主义的态度。如果人民来信很多,本人处理困难,应设立适当人数的专门机关或专门的人处理这些信件。如果来信不多,本人或秘书能够处理,则不要另设专人。


