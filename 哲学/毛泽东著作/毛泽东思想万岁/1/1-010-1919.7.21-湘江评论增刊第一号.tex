\section{湘江评论增刊第一号}
\datesubtitle{(一九一九年七月二十一日)}
\subsection{健学会之成立及进行}

健学会以前的湖南思想界湖南的思想界,二十年以来,黯淡已极。二十年前,谭嗣同等在湖南倡南学会,召集梁启超,麦梦华诸名流,在长沙设时务学堂,发刊湘报,“时务报”。一时风起云涌,颇有登高一呼之概。原其所以,则彼时因几千年的大帝国,屡受打击于列强,怨幅惋悔,敬而奋发。知道徒然长城渤海,挡不住别人的铁骑和无畏兵船。中国的老法,实在有些不够用。变法自强的呼声,一时透彻衡云,云梦的大倡。中国时机的转变,在那时候为一个大枢纽。湖南也跟着转变,在那时候为一个大枢纽。

思想变了,那时候的思想是怎样一种思想?那时候思想的中心是在怎样的一点?此问不可不先答于下——

(一)那时候的思想是自大的思想。什么“讲求西学’,什么“虚心考察”,都不外“学他到手还以奉敬”的方法。人人心目中都存想十年二十年后,便可学到外国的新法。学到新法便可自强。一达到自强的目的,便可和洋鬼子背城借一,或竟打他个片甲不回。正如一个小孩受了隔壁小孩的晦气,夜里偷着取出他的棍棒,打算明早跑出大门,老实的还他一个小礼。什么“西学”“新法”相当于小孩的棍棒罢了。

(二)那时候的思想,是空虚的思想,我们试一取看那时候鼓吹变法的出版物,便可晓得一味的“耗矣哀哉”。激刺他人感情作用。内酌是空空洞洞,很少踏着人生社会的实际说话。那时有一种“办学室”,“办自治”,“请开议会”的风气,寻其根柢,多半凑热闹而已。凑热闹成了风,从思想界便不容易引入实际去研究实事和真理了。

(三)那时候的思想,是一种“中学为体,西学为用”的思想,“中国是一个声名文物之邦,中国的礼教甲于万国,西洋只有格致炮厉害,学来这一点便得。”设若议论稍不如此,便被人看作“心醉欧风者泳”,要受一世人的唾骂了。

(四)那时候的思想是以孔子为中心的思想。那时候于政治上有排满的运动,有要求代议政治的运动。于学卫上有废除科举兴办学校,采取科学的运动。却于孔老爹,仍不敢说出半个“非”字。甚至盛倡其“学问要新,道德要旧”的谬说,“道德要旧”就是“道德要从孔子’的变语。

上面所举,全中国都有此就行,湖南在此情形的中间占一位置。所以思想虽然变化,却非透彻的变化,任何说是笼统的变化,盲目的变化,过渡的变化,从戊戌以致今日湖南的思想界,全为这笼统的,盲目的,过渡的变化所支配。

湖南辨求新学二十余年,而后有崭然的学风。湖南的旧学界,宋学、汉学两支流,二十年前,颇能成为风气。二十年来,风气尚未尽歇,不过书院为学校占去,学生为科学吸去,他们便必淹没在社会的底面了。推原新学之所以没有风气,全在新学不曾有确立的中心思想。中心思想之所以不曾确立,则有以下的数个原因:

1、没有性质纯粹的学会。

2、没有大学。

3、在西洋留学的很少。有亦为着吃饭问题和虚条心理,竟趋于“学非所用’的一途,不能持续研究其专门之学。在东洋留学的,被黄兴吸去做政治运动。

4、政治纷乱,没有研究的宁日。

这是湖南新学界中心思想不能确立的原故,即是没有学风原故,辛亥以来,滥竽教育的,大部市侩一流,逞其一知半解的见解造成非驴非马的局势。中心思想,新学风气,可是更不能谈及了。

近数年来,中国的大势陡转,蔡元培。江亢虎,吴敬恒,刘师复,陈独秀等,首倡革新,革新之说,不止一端,自思想,文学,以致政治,宗教,艺术,皆有一改旧观之概。甚至国家要不要,家庭要不要,婚姻要不要,财产应私有应公有,都成了亟待研究的问题。更加以欧洲的大战,激起了俄国的革命,潮流浸卷,自西向东,国立北京大学的学者首欢迎之,全国各埠各学校的青年大响应之,怒涛澎湃,到了湖南而健学会遂以成立。

\textbf{健学会之成立}

六月五日,省教育会会长陈润霖君邀集省城各学校职教员徐特立,朱剑凡,汤松,蔡湘,钟国陶,杨树达,李云杭,向绍轩,彭国君,方克刚,欧阳鼐,何炳麟,李景桥,赵翌等发起健学会。在楚怡学校开会。今录某报所载陈润霖君报告组织学会的意旨于下一一

兄弟前次到京,偶有感能,深抱乐观。象四年前,北京大学学生以做官为唯一目的。非独大学为然,即大学以外之学生,亦莫不皆然。前次居京,所见迥然不同。大学学生思潮大变,皆知注意人生应为之事,其思潮已多表露于各种杂志月刊中。因之各校学生,亦顿改旧观,发生此次救国大运动。其致此之故,则因蔡孑民先生自为大学校长以来,注入哲学思想,人生观念,使旧思想完全变掉。或该认学生救国运动为政客所勾引,而不知突出学生之自动,及新旧思潮之冲突也。盖自俄国政体改变以后,社会主义渐渐输入于远东。虽派别甚多,而潮流则不可遏抑。即如日本政府,从来对于提倡社会党人,苛待残杀,不遗余力,而近日竟许社会党人活动。如吉野博士等,则主张实行国家社会主义,以和缓过激主义,顺应世界之趋势,从看将日本政体改变为英国式虚君制。

于此可知,世界思潮改变之速,势力之大矣!我国新思潮亦甚发展,终难久事遏抑,国人当及时研究,导之正轨,国人等组织学会,在釆用正确健学之学说而为彻底之研究……

这日开会,听说街有朱剑凡君主张“各除成见,研究世界新思想,服从真理”的演说,向绍轩君主张“采用国家社会主义”的演说。在湖南思想界不可不谓空前的创闻。今录出该会所发表的会则如下一一

(一)本会同志组合,以输入世界新思潮,共同研究,择要传播为宗旨。

(二)本会定名为健学会。

(三)会所暂定长沙储英源楚怡小学校。

(四)入会者须确有研究学术之志愿,经本会会友一人以之介绍,得为本会会员。

(五)关于输入新思潮之方法一一

(1)凡最近出版之图书杂志,由本会随时搜集,以供会员阅览。会员所藏书报,得借给本会会员阅览。其有愿捐入本会者,本会尤为欢迎。

(2)延请海内外同志,随时调查,通信报告。

(3)介绍名人谈话。

(六)关于研究之方法一一

(1)研究范围,大半为哲学,教育学,心理学,论理学,文学,美学,社会学,政治学,经济学诸问题,会友必分认一门研究。

(2)重要之问题,由会友共同研究。

(3)会员有愿习外国语者,由本会会友传授。

(七)关于传播之方法一一

(1)讲演。分定期,临时两种。定期讲演,每周日曜日午前八时至十时。由会友轮流担任。讲员及讲题均于前周日曜日决定。讲友须预备讲稿,交由本会汇刊。临时讲演,凡有主要演题,或由会友,或请名人讲演,另觅地点,择期举行。

(2)出版。

 (八)本会设会计,管理图书各一人。其他会务由会友共同负责。每次开会推会友一人协时主席。 
(九)会友应守之公约如左一一

(1)确守时间.。

(2)富于研究的精神。

(3)学问上之互助。

(4)自由讨论学术.

(5)不尚虚文客气,以诚实为主。

(十)会员年纳两元以上之会金,有能特别筹助经费者,本会极为欢迎。

(十一)本会遇有重要事项,必须讨论时,得于定期讲演后,临时通告全体,举行会议。

会则中的(五)(六)(七)(九)极为重要,(九)之富于研究的精神,所以破除自是自满的成见,立意很好。尚生于研究的精神之后,继之以“批评的”精神。现代学术的发展,大半为各人的独到所创获。最重的是“我”是“个性”,和中国的习惯,非死人不加议论,著述不引入今人的言论恰成一反比例。我们当以一己的心思,居中活动。如日光之普天照耀,如探海灯之向外扫射,不管他到底是不是(以今所是的为是)合人意不合人意。只求合心所安合乎箕理才罢。老先生最不喜欢的是狂妄。岂不知古今最确的原理,伟大的事业邵是系一些被人加上狂妄名号的狂妄人所发明创造来的。我们住在这复杂的社会,诡诈的世界没有批评的精神就容易会做他人的奴隶。其君谓中国人大半是奴隶,这话殊觉不错。(九)之自由讨论学术,很合思想自由,言论自由的原则。人类最可宝贵最堪自乐的一点却在于此,学术的研究最忌演释式的独断态度。中国什么“师严而后道尊”。师说“道统”,“宗派”都是害了独断态度的大病。都是思想界的强权,不可不竭力打破。象我们反对孔子,有很多别的理由,单就这独霸中国,使我们思想界不能自由,郁郁做两千年偶象的奴隶,也是不能不反对的。

健学会之进行健学会进行事项,会则所定大要系研究及传播最新学术。现在注重于研究一面,闻已派人到京沪各处,釆买书籍,新闻纸和杂志。在省城设一英语学习班,使会员学习英语,为直接研究四方学术的预备。有年在四五十的会员都喜欢学习。又设一演讲会由会员轮流发表意见,实行知识的交换。官气十足的先生们,忽然屈尊降贵,虚心研究起来。虽然旁人尚有不满意的处所,以为官气还有十分五、六,讲演要多采用命令式和训话式。更有谓他们是青叶上青虫的体合作用。象这样的求全责备.我们为何以下坏。在这么女性纤纤,暮气重重的湖南有此一举,颇足山幽囚而破烦闷。东方的曙光,空谷的足音,我们正应拍掌欢迎,希望他可作“改造湖南的张本”看他们四次讲演的问题,如“国人误谬的生死观”。怎样做人”,“教育和白话文”,“釆用杜威教育主义”,都可谓能得其要。倘能尽脱习气采用公开讲演,尽人都可以听,则传播之外,得益之大,当有不可计量的了。

\begin{flushright}(《湘江评论》临时增刊第一号)
\end{flushright}

