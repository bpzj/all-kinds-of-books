\section[批评《人民日报》不应“反冒进”(传达纪要)(一九五八年一月)]{批评《人民日报》不应“反冒进”(传达纪要)}
\datesubtitle{(一九五八年一月)}


大家对报纸的反映比较好,报纸有进步,新闻,评论都有进步,不要满足。

新闻近来搞得比较活泼。

评论大家写,各版包干的办法是好办法。总编辑是统帅,要组织大家写,少数人写不行。

组织形式,这种生产关系有没有妨碍生产力的发展,要研究。

各部门、各版可以竞赛。

写评论,要结合情况和政治气候。转变要快。

写的不要刻版,形象要多样化。,

政论要像政论,但并不排斥抒情。

报社的人要经常到下边去,呼吸新鲜空气,同省委关系要搞好。

下去的人,又作工作,又当记者,不要长住北京,要活动一些,要经常到外边跑一跑。

人民日报是中央一个部门,同中央组织部,宣传部一样,都应该向地方学习.报纸一个很重要的任务是转载地方报纸的新东西,把这件事当作一个政治任务来作。这样,对地方报纸是鼓励,使他们非看人民日报不可。

思想评论可以搞。

红专还是大问题,但要结合当前情况谈。

标题要吸引别人看,这很重要。

除四害还要搞一个专刊。

<p align="center">×××</p>

一九五六年六月以后宣传反冒进的情况如何,要检查一下。检查那时的评论与消息。

六月二十日社论有原则性错误。说什么既要反对保守主义,又要反对急躁情绪。当时反右不到半年,这篇社论就以为反右成绩很大,估高了,是不对的。不能说这社论一点马克思主义没有,但“但是’以后是反马列主义的。社论的提法同魏忠贤的办法一样。东林党里有君子也有小人,朝廷里有小人也有君子。他的意思其实是说东林者小人。引我的话,掐头去尾,只引反“左”的两句,不引全段话,这不对。像秦琼卖马,减头去尾,只要中间一段。方法是片面的。前面讲少数如何,后面讲多数如何,形式上两面反,实际上是反“左”反“冒进”。

批评双轮双铧犁“冒进’了,说南方不能用,不好.实际是用了,很好.应翻案。为它恢复名誉。

“打破常规”的问题,社论说是“不适当地打破常规“,这话就不对。革命,就要打破常规。作为方针来讲,不能讲“反冒进”,只能讲“调整”。作为倾向来反,就把多快好省反掉了。

社论说,特别在中央提出“多快好省”,“四十条”后,产生了冒进。这说法是片面的,这样说就认为缺点来自中央。

你们反了右没有?这种说法,形式上是辩证的,实际是把辩证法庸俗化了。反“左”,反右同时并举,四平八稳。

“多快好省”方针本身是全面的,是一个整体,不能说这个适当,那个不适当。

根据经验,工作中有缺点是正常的,不是反常的。革命就是要跳跃。个别缺点不可免.要分清是九个指头,还是一个指头。工作中左一点,右一点,是正常现象。问题在于方针、方法如何.以后不要提“反冒进”,决不要提。


