\section[在接见西藏国庆观礼团时的讲话(一九五三年十月十八日)]{在接见西藏国庆观礼团时的讲话}
\datesubtitle{(一九五三年十月十八日)}


我们要和各民族讲团结,不论大的民族小的民族都要团结。例如鄂伦春族还不到两千人,我们也要和他们团结。只要是中国人不分民族,凡是反对帝国主义,主张爱国和团结的,我们都要和他们团结。团结起来,按照各民族不同地区的不同情况进行工作。有些地方可以做得快一点,有些地方可以做得慢一点,不论做快做慢都要先商量好了再做。没有商量好就不勉强做。商量好了,大多数人赞成了,就慢慢地去做,做好事也要商量去做。商量办事,这是共产党和国民党不同的地方。国民党势力大就压人。他们不仅压迫少数民族,还压迫大多数汉人。国民党是做坏事的。坏事是不应当做的,我们的干部有了错误就应当批评。我们在西藏的工作有什么缺点和错误,请你们讲,你们不同意的和你们认为不利于人民的都可以讲,便于我们纠正。有了缺点就马上纠正,这是我们和国民党不同的地方。

整个中国现在还很落后,需要发展。这是因为过去有帝国主义和国民党压迫的关系。他们现在已被我们赶走了,这四年来我们就有很大进步。在国家经济恢复以后,今年开始了第一个国家建设的五年计划。预计在三个五年计划以后,我们的大工业就建立起来了。苏联在三十六年前赶走了帝国主义和国内的坏人,建设了三十六年才有现在的成绩。我们建设起来比苏联还要快些。因为我们有苏联的帮助。中国人口很多,有五万万几千万,是全世界人口最多的国家。土地也很广大,有三个五年计划就能建设得很好了,对西藏就会有更大的帮助了。

中央有什么东西可以帮助你们的一定会帮助你们。帮助各少数民族,让各少数民族得到发展和进步,是整个国家的利益。各少数民族的发展和进步都是有希望的。

西藏政治、经济、文化,宗教的发展,主要靠西藏的领袖和人民自己商量去做,中央只是帮助。这点是在和平解决西藏办法的协议里写了的。但是要做还得一个时间;而且要根据你们的志愿逐步地做。可做就做,不可做就等一等,能做的,大多数人同意了的,不做也不好,可以做得慢一些,让大家都高兴,这样反而就快了。总之,我们的方针是团结进步、更加发展。


