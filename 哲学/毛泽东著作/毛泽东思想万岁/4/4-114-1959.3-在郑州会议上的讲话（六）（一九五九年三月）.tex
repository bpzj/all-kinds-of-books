\section[在郑州会议上的讲话(六)(一九五九年三月)]{在郑州会议上的讲话(六)}
\datesubtitle{(一九五九年三月)}


什么叫建设社会主义,集中表现。

集体所有制和全民所有制要不要一条线?还是要一条线的。斯大林同志划了一条线,指出三个条件,先决条件,这是对的,缺点讲的不太具体,四十条中提的就比较具体了。许多问题斯大林没提到:并举、全党全民办工业、群众运动、政治挂帅、整风。

城市人民公社问题很复杂,不要怕慢,但要采取积极态度。

对农民问题:大跃进时对农民积极性估计不足,大跃进以来仍是农民问题,过高的估计了农民,究竟是鞍钢是老大哥呢?还是徐水是老大哥呢?还是工人阶级,鞍钢是老大哥。有些“理论家”一遇到实际问题就打折扣,他们就回避资本主义留下来的东西:商品生产和价值规律,问题在于怎么认识,看对我们的社会主义建设有没有好处?有好处就利用,为我们服务。要利用商品生产、价值规律。


过早宣传全民所有,国家就要调拨,是实质上剥夺了农民,农民会不高兴的,谁高兴这样作呢?台湾,唯恐天下不乱。

公社也可以办赢利较多的工业。(斯大林不敢把拖拉机交给农社)。

有人把农民当成工人,这不对。

不要怕商品生产,问题要看同什么样的经济相联系;

不能把人与人的关系看成是父子关系,而是平等关系。破除不平等,但仍然要有差别。

钢铁、炼、机械、电很重要,林业很重要,也要成为根本问题之一。

价值法则不起调节作用,只是计算工具。

人民公社实行了全民所有制不算共产主义。

苦战三年,再过十二年就可过渡到共产主义。


