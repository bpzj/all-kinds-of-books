\section[在汉口会议上的插话(一九五八年四月一日至六日)]{在汉口会议上的插话(一九五八年四月一日至六日)}
\datesubtitle{(一九五八年四月一日)}


对于学习马克思主义也要破除迷信,以为只有外国人才学得好,洋人都是了不起的。中国人算不算洋人?我们叫不叫神仙呢?我问过好多人,都说不算。神仙是住在别的星球上的,我们叫凡人。别的星球上的人看我们,是不是也是神仙?这是一种迷信。中国人当奴隶习惯了,看不起自己,什么东西都是别人行,自己不行。

△为什么农村不能办大学?十五年普及,十五年提高,三十年后大家都是大学生,每乡一个大学。第一书记要当大学教授。

△每省都要釆取检查的办法,组织检查团下去,检查措施是否可靠。(安徽检查结果,有百分之二十是虚的),省委第一书记做团长,省长做付团长。民主党派也要派人下去。

△“搞水利,吃大米”,一下子讲到本质问题。人民看到前途,这么多人搞,总有希望。

不要过早宣传水利化,否则明年不好办,要留点余地。苦干三年,基本上改变面貌。以后不是不战了,我在人民内部矛盾中提出,不是三年,而是艰苦奋斗几十年才有希望。

△世界上无论什么事情都有真有假,都是真那不可设想。

(谈到领导干部参加劳动问题时)中央、省做样子好,还是不做样子好?省委先做个样子。

对科学家要破除迷信,对其科学技术要又信又不信。从古以来,都是儿子此父亲厉害,学生比先生好,青年比老年强。当然也有儿子不如父亲、学生不如先生的,一般是好。看戏的比唱戏的厉害。一般说来,戏剧的改进,主要靠观众。

△什么叫改变面貌?要粮、油、棉三者翻身。今后要大搞油料,用各种办法,千方百计搞种花生、芝麻、黄豆、养猪、养鸡。我们几年来主要注意粮食,现在要把油料提高到粮食一样的位置。回去要做出计划,雷厉风行搞,搞点油水给大师傅做菜。

各省搞民歌。下次会上每省至少要交一百首。大、中、小学生发动他们写。每人发三张纸,没有任务。军队也要写,从士兵中搜集。

△真正绿化,要在飞机上看一片绿。树种下去就叫做绿化?好多地方还是黄的,只能叫黄化。

《人民日报》不要轻易宣布完成什么化,人们以后要问,你们化了几年,为什么还要化?树种下去,稀稀拉拉的还没有活,倒宣布绿化。“化”搞得很滥,动不动就宣布“化”了。

△报纸宣传不要尽规划,要宣传深入细致、踏实。现在宣传多注意了多快,好省注意不够。不好不省如何基本改变面貌?大话不必讲,好大喜功需要,但华而不实不好,喜功变为无功,不是喜大,而是喜小,结果无功而还。

孙行者无法无天,大家为什么不学?猴子反教条主义,戴了金箍咒,就剩了一半。猪八戒一辈子都自由主义,有点修正主义,动不动就想退党,不过那个党不是一个好党,是第二国际,应该退党。唐僧是伯恩斯坦。

△高潮为什么会来,这是有历史的:(一)从前有过高潮(一九五五——一九五六下半年),有了经验。(二)一九五六年下半年——一九五七年,来个“反冒进”,搞得人不舒服。这个挫折很有益处,教育了人们。有比较,有反面教育,因为受了损失。是个马鞍形——两个高潮,一个“反冒进”。(三)为什么又高起来呢?鉴于“反冒进”不好。

△现在躭心又会不会“反冒进”,这么大的劲头,如果今年,得不到丰收。群众会泄气,势必影响上层建筑,那时议论又会出来(“还是我的对”)。民主人士、富裕中农、党内民主人士,就有不少在那里等着看我们垮台。又要刮风。党内中间偏右、观潮派,“楼观沧海月,门对浙江潮”。此时要和地、县书记讲清楚,如果收成不好,几化完不成怎么办?

△一曰好大喜功。打蒋、反右、灭资、五年计划,都是好大喜功,难道还是好小喜败?二曰急功近利。大禹惜寸阴,我辈惜分阳。刘琨、祖逖闻鸡起舞,诸葛亮“不靖中原,誓不回师”。这不是急功近利吗?古人多得很。现在三包、定额、计件工资,这不是急功近利吗?三曰鄙视既往。就是要轻视过去。难道过去帝国主义、封建主义、官僚资本主义、西藏的奴隶制度不应轻视吗?伯达说:厚今薄古。四曰迷信将来。苦战多少年,没有将来有什么意义。

△要注意储蓄粮食。今年如丰收,还是维持去年口粮。南方五百斤,北方三百六十斤,国家只买这么多(八百七十五亿斤)。多余的存在合作社,使农民看得见粮。一不上天,二不入地,三不到外国。苦战三年,还是五百斤,三百六十斤。

△做一段,休息一天,劳逸结合,有节奏,波浪式前进,很必要。指挥劳动大军,两个战役之间要休息一下,连续作战是由战役组成的。

干一个时期,专门休息一下,悠哉游哉。成都会议解决了这个问题。

技术革命是被逼出来的。世界上的东西都是逼出来的。整风,打倒帝国主义,不是逼出来的?孔明的木牛流马也是逼出来的。一个对立物,把你一逼就逼出来了。

△大鸣大放,干部我压服你,我打通你,世界观基本改变了。过去人是两只脚的猪,是奴隶主与奴隶的关系,说是人民,讲得好听,事实上多多少少是这样。做了官都有那么点官架子。

从古以来,不听个人的话,只听空气的话。斯大林在世好像什么都好,死了什么都坏。

凡是乱得厉害(的地方),问题就接近解决。让闹,闹够,你们总是不通,一不让闹,二不让闹够。

△特大灾害要向群众讲一讲,马克思主义现在还没有办法。

对群众是说服还是压服?我们从红军开始,几十年来总是要说服。不要压服。为什么解放以来,忽然来了一股风,只要压服,不要说服。只说压服地主,没有说压服群众。国民党是压服,我们也压服,与国民党还有什么区别呢?蒋介石反共还讲七分政治,三分军事,为什么我们不讲政治?

△农民瞒产可以原谅,他是没有看清前途,但不能提倡。如果像现在这样搞法,增产七百亿到一千亿斤,我们国家一年征购只八百多亿,这就等于不要征购了。他们何必再瞒产。到那时,全国粮食总产量就有四千多亿,即使多购一点,他们也不伤心。瞒产的原因主要是干部带头和粮食不足。今后要把底告诉农民,把全国总账告诉他,你再增产,国家也只要这么多。今后征购以后的余粮也保存在乡社。

世界上的事,有真必有假,有利必有弊,不可不信,不可全信,百分之百相信就会上当;不相信,就会丧失信心。我们对各项工作、各种典型,都要好好检查,校对清楚(假博士、假教授、假交心、假高产、假跃进、假报告)。

要有观点指挥材料,不要材料把观点淹没了。要学会用政治带业务,先讲政治面貌(观点、思想),然后谈工作面貌,不能倾盆大雨,而是要毛毛雨(有些人一讲两三天,少则三个钟头)。不要企图把所有的观点都拿出来,这样人们接受不了。一个时候给人家几个观点叫宣传,一个时候给人家一个观点叫鼓动。又说政治水平很高,谈起来就是数目字。不谈政治,政治都没有,哪里有水平,政治与数字是官兵关系,政治是元帅。

干群关系,大鸣大放,是全世界社会主义国家都不敢承认的问题,只有我国实行。不怕发动群众是真正的列宁主义的态度。列宁专门下乡,下厂,接近群众,发动群众,特别是对官僚主义者骂得很凶。整风没有内外夹攻是整不好的。

经过大鸣大放后。看起来政治上是扎稳了根。如这次“双反”、大鸣大放,干部和群众不仅敢放,而且放得健康。干部、人民都有了经验,知道什么应该反对,什么应该拥护,什么是人民内部矛盾,什么是敌我矛盾,对什么人应采取什么态度。

所谓稳妥可靠,结果是又不稳妥,又不可靠。我们这样大的国家,这样稳,会出大祸。对稳妥派,有个办法,到了一定时候就提出新口号,使他无法稳,这一派人数可能比较多,想看一看,如果来一个灾荒,他们还是要喊的:“看你跃吧!”“冒进”是稳妥派反对跃进的口号。

△毛主席指示:整风是纲,整风挂帅,生产是中心,带动其他工作。

△(在湖南×××汇报《群众的变化》问题时,谈到社会风气大变,农民安心在农村了)毛主席说:这是非常重要的,我们这样大的国家,如果许多人长期不安心农村要亡国的。这是非常好的新气象,应该非常注意。

△(在湖南汇报到干部普遍种试验田时)毛主席说:“这是全国普遍现象,在大跃进的面前,不仅要有干劲,而且要增加措施,空气要压得很紧。组织大检查的工作措施。这是很好的工作方法,各省、地、县都要组织检查团去基层检查。今年是个非常年,今年好好看一年,以后胆子就大了。因此要很好组织大检查。各地要检查几次,检查很好。没有检查的要补上这一课。

△(在各地汇报抓水、肥、土的措施时)毛主席说:“这个问题还要研究。水利各省搞的也很多,特别是安徽搞了水利规划,搞了水网。

究竟什么肥(人畜、土肥、堆肥、绿肥),什么肥各占多少,如果是土,那就有问题。虽然如此,比过去多得多了,这也是好的。

翻地是很重要的,值得各省注意,把土大大翻一遍就能增产很多,这个经验值得很好推广。

△(在汇报到生产高潮中,相当多的干部强调稳,不前不后走中间时)毛主席说:“这是党内的稳妥派,实际上是落后,要把这种人抓起来,办法是不断提出新任务、新口号,使他们永远赶不上,这就推动了他们,他们就不落后了。


