\section{问题研究会章程}
\datesubtitle{(一九一九年十月廿三日)}



第一条、凡事或理之为现代人生所必需,或不必需,而均尚未得适当之解决,致影响于现代人生之进步者,成为问题。同人今设一会,注重解决如斯之问题,先从研究入手,定名问题研究会。

第二条、下列各种问题,及其他认为有研究价值续行加入之问题为本会研究之问题。

(一)教育问题一一

1。教育普及问题(强迫教育问题)2.中等教育问题3.专门教育问题4.大学教育问题5.社会教育问题6.口语教科书编纂问题7.中等学校国文科教授问题8。不惩罚问题9。废止考试问题30.各级教授法改良问题7.小学教师知识健康及薪金问题12.公共体育场建设问题 13.公共娱乐场建设问题14.公共图书馆建设问题 15.学制改订问题16.大派留学生问题 17.杜威教育说如何实施问题 
(二)女子问题

1.女子参政问题2.女子教育问题3.女子职业问题4.女子交际问题5.贞操问题6.恋爱自由及恋爱神圣问题7.男女同校问题8.女子修饰问题9。家庭教育问题10.姑媳同居问题 11.废娼问题 12。废妾问题13。放足问题14.公共育儿院设置问题15.公共蒙养院设置问题 16.私生儿待遇问题 17.避妊问题 
(三)国话问题(一白话文问题)

(四)孔子问题

(五)东西文明会合问题

(六)婚姻制度改良及婚姻制度应否废弃问题

(七)家族制度改良及家族制度应否废弃问题

(八)国家制度改良及国家制度应否废弃问题

(九)宗教改良及宗教应否废弃问题。

(十)劳动问题一一

1.劳动时间问题2.劳工教育问题3.劳工住屋及娱乐问题4.劳动失职处置问题5.工值问题6.小儿劳作问题7.男女工值平等问题8.劳工组合问题9.国际劳动同盟问题10。劳农干政问题 11.强制劳动问题 12.余剩均分问题 13.生产机关公有问题  14.工人退职年金问题15.遗产归公问题(附)
(十一)民族自决问题

(十二)经济自由问题

(十三)海洋自由问题

(十四)军备限制问题

(十五)国际联盟问题

(十六)自由移民问题

(十七)人种平等问题

(十八)社会主义能否实施问题

(十九)民众的联合如何进行问题

(二十)动工俭学主义如何普及问题

(二一)俄国问题

(二二)德国问题

(二三)奥国问题

(二四)印度自治问题

(二五)爱尔兰独立问题

(二六)土耳其分裂问题

(二七)埃及骚乱问题

(二八)处置德皇问题

(二九)重建比利时问题

(三十)重建东部法国问题

(三一)德殖民地处置问题

(三二)港湾公有问题

(三三)飞渡大西洋问题

(三四)飞渡太平洋问题

(三五)飞渡天山问题

(三六)白令英吉利直布罗陀三峡凿遂通车问题一

(三七)西伯利亚问题

(三八)菲律宾问题

(三九)日本粮食问题

(四十)日本问题

(四一)朝鲜问题

(四二)山东问题

(四三)湖南问题

(四四)废督问题

(四五)裁兵问题

(四六)国防军问题

(四七)新旧国会问题

(四八)铁路统一问题(撤消势力范围问题)

(四九)满州问题

(五十)蒙古问题

(五一)西藏问题

(五二)退回庚子赔款问题

(五三)华工问题一一

1.华工教育问题2.华工储蓄问题3.华工归国后安置问题

(五四)地方自治问题

(五五)中央地方集权分权问题

(五六)两院制一院制问题

(五七)普通选举问题

(五八)大总统权限问题

(五九)文法官考试问题

(六十)澄清贿赂问题

(六一)合议制的内阁问题

(六二)实业问题一一

1.蚕丝改良问题2。茶产改良问题3.种棉改良问题4.造林问题

5.开矿问题6.纱厂及布厂多设问题7.海外贸易经营问题8.国民工厂设立问题

(六三)交通问题一一

1.钦路改良问题2.铁路大借外款厂行添第问题3。无线电台建设问题

4.海陆电线添设问题5.航业扩张问题6。商埠马路建筑问题7。乡村汽车路建筑问题

(六四)财政问题…

1.外债偿还问题2.外债添借问题3.内债偿还及加募问题4.裁厘加税问题5。盐务整顿问题6。京省财权划分问题 7.税制整顿问题 8.清丈田亩问题9。田赋均一及加征问题

(六五)经济问题一一

1.币制本位问题2.中央银行确立问题3.收还纸币问题4.国民银行设立问题5。国民储蓄问题

(六六)司法独立问题

(六七)领事裁判权取消问题

(六八)商市公园建设问题

(六九)模范村问题

(七十)西南自治问题

(七一)联邦制应否施行问题

第三条、问题之研究须以学理为根据。因此在各种问题研究之先,须为各种主义之研究,下列各种主义为特须注重研究之主义。一一

(一)哲学上之主义

(二)伦理上之主义

(三)教育上之主义

(四)宗教上之主义

(五)文学上之主义

(六)美学上之主义

(七)政治上之主义

(八)经济上之主义

(九)法律上之主义

(十)科学上之规律

第四条、问题不论发生之大小,只须含有较广之普遍性,即可提出研究,如日本问题之类。

第五条、问题之研究,有须实地调查者,须实地调查之,如华工问题之类,无须实地调查,及一时不能实地调查者,则从书册杂志、新闻纸三项着手研究。如孔子问题,及三海峡凿隧通车问题之类。

第六条、问题之研究,注重有关系于现代人生者之问题。在古代与现代及未来毫无关系者,则不注意。

第七条、问题研究之方式分为三种。一一

(一)一人独自之研究

(二)二人以上开研究会之研究

(三)二人以上不在一地用通函之研究

第八条、问题研究会,只限于“以学理解决问题”会以外。然在未来而可以预测之问题,亦注意,以“实行解决问题”属于问题研究会以外。

第九条、不论何人有心研究一个以上之问题,而愿与问题研究会生交涉者,即为问题研究会会员。

第十条、会与会员间会员与会员间,只限于“问题研究”之一点,有关此外之关系,属于问题研究会以外。

第十一条、问题研究会,设书记一人,办理会中事务。

第十二条、问题研究会,于中华民国八年西历一千九百十九年九月一日成立。问题研究会章程,即于是日订定,且发布。

\begin{flushright}(抄自《北大日刊》一九一九年十月二十三日第二版)\end{flushright}

