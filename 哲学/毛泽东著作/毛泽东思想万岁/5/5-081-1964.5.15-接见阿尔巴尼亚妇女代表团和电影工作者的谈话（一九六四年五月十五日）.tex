\section[接见阿尔巴尼亚妇女代表团和电影工作者的谈话(一九六四年五月十五日)]{接见阿尔巴尼亚妇女代表团和电影工作者的谈话}
\datesubtitle{(一九六四年五月十五日)}

\begin{list}{}{
    \setlength{\topsep}{0pt}        % 列表与正文的垂直距离
    \setlength{\partopsep}{0pt}     % 
    \setlength{\parsep}{\parskip}   % 一个 item 内有多段,段落间距
    \setlength{\itemsep}{\lineskip}       % 两个 item 之间,减去 \parsep 的距离
    \setlength{\labelsep}{0pt}%
    \setlength{\labelwidth}{3em}%
    \setlength{\itemindent}{0pt}%
    \setlength\listparindent{\parindent}
    \setlength{\leftmargin}{3em}
    \setlength{\rightmargin}{0pt}
    }

\item[\textbf{维托·卡博(以下筒称维托):}]{受到主席接见,感到幸福,这不仅是我们,而且是全体阿尔巴尼亚妇女的光荣。}
\item[\textbf{主席:}]{什么时候到的?} 
\item[\textbf{维托:}]{四月二十七日。}
\item[\textbf{主席:}]{大使什么时候来的?面孔很熟。}
\item[\textbf{大使:}]{十年前,当你作中华人民共和国主席时,向你呈递过国书。}
\item[\textbf{主席:}]{你第二次来,欢迎你。他们几位(指阿电影工作者)是干什么呢?}
\item[\textbf{维托:}]{电影工作者。}
\item[\textbf{主席:}]{我们两国团结起来。我们两党团结起来。很多马列主义政党(不是挂名的,是真的,就是修正主义所说的,“教条主义”的)团结起来。我们挨骂,说我们是教条主义,有人骂就好。}
\item[\textbf{维托:}]{敌人咒骂,我们就感到舒服。}
\item[\textbf{主席:}]没有人骂就是不舒服。许多事情别人不知道,这一骂就骂出来了,现在要辩论,要公开论战,许多人开始注意阅读马克思、恩格斯、列宁,包括斯大林的著作。开始注意起研究谁是谁非。是修正主义对,还是“教条主义”对。你们、我们被称为教条主义,挨了很多骂:{又是假革命、新托洛斯基主义、民族主义、分裂主义者……,头上帽子很多。}
\item[\textbf{维托:}]{他们除了谩骂,别无他法。}
\item[\textbf{主席:}]{可是他们不敢在报上发表我们的文章。他们说他们很有理,可是不敢把我们、你们的文章发表在报刊上。我们要和他竞赛:我们说,我们发表你们多少,你们发表我们多少,好不好,他们不干。}
\item[\textbf{维托:}]{他没有理。}
\item[\textbf{主席:}]{就是。他无理,他不干。他们说我们无理。说我们是“教条主义”、“托洛斯基主义”、“小资产阶级社会主义”,那你们把我们的东西发表啊,发表了,你们可逐条地批驳。可是他们就是不敢,胆小得很,说他们是纸老虎有道理。}
\item[\textbf{维托:}]{对!对!对!}
\item[\textbf{主席:}]{帝国主义是纸老虎,修正主义也是。}
\item[\textbf{维托:}]{修正主义是他们的同谋。}
\item[\textbf{主席:}]{中国人都欢迎你们。}
\item[\textbf{维托:}]{我们到处都受到非常热情的欢迎。每到一处,欢迎都变成了友谊、热情的示威,告别时含着眼泪。这使我们感到不是小国小党,而是与中华人民共和国一样。这是你们党进行工作的结果。\marginpar{\footnotesize 112}}
\item[\textbf{主席:}]{你们站住了,未被压下去,是一个伟大胜利。阿尔巴尼亚四面受敌人包围住,压不下去。他们提出无论如何要停止公开争论,你们赞成吗?}
\item[\textbf{维托、大使:}]{我们从来不同意。}
\item[\textbf{主席:}]{他们要停,提出那怕是三个月也好。}
\item[\textbf{维托:}]{赫鲁晓夫要争取时间。}
\item[\textbf{主席:}]{他们通过罗马尼亚向我们提出要停止公开争论。(此时团结报记者进来,主席站起来说:“欢迎你,你们都是年轻人。”)我们告诉他们:你们一走,我们就发表文章。我们的文章还没有写出来,就发表了其他兄弟党的文章。对他们的公开信还没有评完,只写了八篇文章,还要几篇才能评完。现在他们有新的提出来,又要时间。他们就想要开会,要五月中、苏两党开会。对不起,我想大概明年五月差不多。他们要对我们采取“集体措施”,我们说,把你们的“集体措施”拿出来我们看看呀。}
\item[\textbf{维托:}]{他们还会继续走下去。}
\item[\textbf{主席:}]{也许会搞什么“集体措施”,我们准备着。什么措施,你去搞吧!}
\item[\textbf{维托:}]{中国是巨人,在东北我们看到好多东西啊!}
\item[\textbf{主席:}]{中国是个落后国家,开始有一点工业,东北的工业比较发达;开始积累了一些农业的经验再有几个五年计划要好些。越整我们,也许越好点。\\(问×××同志)你这位同志是谁?
}
\item[\textbf{陈:}]{×××。}
\item[\textbf{主席:}]{现在搞文艺,去过延安吗?}
\item[\textbf{陈:}]{去过,见过主席几次。现在是搞电影工作。}
\item[\textbf{主席:}]{电影、戏剧、文学,不反映现代工农是不好的。我们社会还有许多意识形态未改造,现在正在做此工作。\\(问陈)你什么时候入党的?}
\item[\textbf{陈:}]{一九三二年,在上海。}
\item[\textbf{主席:}]{那时是土地革命时代,长征是一九三四年。今天在座的有的是搞意识形态工作的。}
\item[\textbf{维托:}]{契布里耶、齐乌是文教部副部长。}
\item[\textbf{陈:}]{还有作曲家、导演、摄影工作者。}
\item[\textbf{主席:}]{你们都是搞意识形态工作的。}
\item[\textbf{维托:}]{群众组织也做意识形态的工作,也是执行党的路线。}
\item[\textbf{主席:}]{我们党是工人农民的党,政权是工农的政权,军队是工农的军队,作为上层建筑的一部分的意识形态,应反映工农。旧的意识形态可顽固了,旧东西撵不走,不肯让位,死也不肯。就要用赶的办法,但也不能太粗暴,粗暴了,人不舒畅,要用细致的方法,战胜老的,老的还有其市场。主要的是我们要以新的东西代替它。你提倡的,不会一下子实现的,没那么回事,你提倡你的,他实行他的。文艺为工农兵服务已经提了几十年了,可是我们的一些工作同志,嘴里赞成,实际反对。现在是不是好些了?}
\item[\textbf{陈:}]{好多了。}
\item[\textbf{主席:}]{包括一些党员,党外人士,爱好那些死人,除了死人就是外国人,外国的也是死人,反映死人,又反映活人。你们国家可能也有这些问题?\marginpar{\footnotesize 113}}
\item[\textbf{维托:}]{也有。有过去的残余,有外来影响。}
\item[\textbf{主席:}]{很顽固。}
\item[\textbf{维托:}]{特别是妇女。}
\item[\textbf{主席:}]{妇女要分青年、老年、中年,老人信迷信,因为她们一辈子受过许多苫,把希望寄托在神上。她们的下一代比她们好点。你们去杭州参观一下灵隐寺,每天有许多人去烧香,有老头子,老太太,跟着他们的,还有她的儿子、女儿、孙子。他们是去真正烧香的,但他们的儿子、孙子、孙女是去玩的,不是真正烧香,是去逛逛杭州的。
    
情况也要看条件。在武汉、上海、杭州可看到木船、轮船工人不信神了,旧社会中轮船工人就不信神了,木船还信,敬龙王,因为没有保障,有风浪会翻船。妇女生孩子也是如此,医院有保证,她不信神了。没有医生,没有保证,还得信神。(问×××同志)你们注意了吗?如果没有医生,叫她不信,就不行。}
\item[\textbf{维托:}]{没有科学知识,不可能消灭迷信,因此要培养干部,逐步消除迷信。糟糕的是资本主义修正主义通过电台、音乐、文学影响我们。}
\item[\textbf{主席:}]{这种影响要逐步地加以抵制。我们、你们的国家可以不进口修正主义的文艺作品、资本主义的文艺作品,问题还是我们自己用什么代替它。}
\item[\textbf{维托:}] 特别我国小,人少更重要。
\item[\textbf{主席:}] 你们国小,可是在世界上表现出不小,很厉害,你们高举马列主义旗帜。苏联为修正主义所控制。什么叫修正主义?即资产阶级的思想、政治、经济、文化。苏联已是资产阶级掌握政权,当然不能说整个社会结构都变成了资本主义,他还来不及,还有抵抗力。修正主义既是资本主义的东西,就不能代表马列主义真理,只能代表少数人。苏联国内阶级斗争是存在的,而且是严重地存在着。反对赫鲁晓夫的,就要被关进疯人院。这是法西斯专政,很值得注意。我们这样国家要掌握好,我不能保证中国一百年后不出修正主义。
\item[\textbf{维托:}] 个别的会出,但以你们现在领导,不会发生的。
\item[\textbf{主席:}] 我们争取。
\item[\textbf{齐乌:}] 你们不是正在以革命精神教育青年吗?
\item[\textbf{主席:}] 以阶级斗争进行教育,承认阶级存在,阶级斗争存在,适当进行阶级斗争,不流血的,不是开战,但少数破坏分子也得整,反革命、破坏分子、帝国主义的走狗、蒋介石的走狗,还有赫鲁晓夫的走狗。我国社会情况相当复杂,不要简单地看问题,我常劝外国同志不要简单地看中国。有光明的一面,是主要的,也有黑暗的一面,虽然是次要,但要注意,社会是由这两面组成的。要复辟地主、资产阶级的有二、三千万人,比你们国家人口还多。
\item[\textbf{维托:}] 可组成一个一个的师。
\item[\textbf{主席:}] 可是他们是分散的,其中程度不同,我们有办法使他们守规矩,少数不守的,有办法制服。
\item[\textbf{维托:}] 专政。
\item[\textbf{主席:}] 专政总有一个对象,不要信赫鲁晓夫的话,他说专政无对象了,全民党、全民国家是骗人的,没那么回事。他提出这些口号,是掩盖他进行资产阶级复辟,骗人的。要揭露全民党、全民国家的欺骗,你们大概赞成吧?\marginpar{\footnotesize 114}
\item[\textbf{维托:}] 完全同意。
\item[\textbf{主席:}] 有阶级才有党,不是代表这个,就是代表那个。可能两个不同阶级由一个党代表。国家就是专政的工具,不然就不应叫国家。专政还要多少年?现在说不定,可能几百年。要搞共产主义各取所需,帝国主义不打倒不可能。
\item[\textbf{维托:}] 还有修正主义。
\item[\textbf{主席:}] 还有修正主义,各国反动派。还要生产达到一定水平,文化教育达到一定水平,条件具备才有可能。这次谈的太多,太久了。
\item[\textbf{维托:}] 谢谢,给我们上了一堂课。
\item[\textbf{主席:}] 问题扯的太宽了。向霍查、谢胡、卡博同志和其他同志们问好。
\end{list}

