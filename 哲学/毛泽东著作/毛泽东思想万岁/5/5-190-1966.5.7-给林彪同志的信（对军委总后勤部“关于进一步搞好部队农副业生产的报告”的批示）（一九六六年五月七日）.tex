\section[给林彪同志的信(对军委总后勤部“关于进一步搞好部队农副业生产的报告”的批示)(一九六六年五月七日)]{给林彪同志的信(对军委总后勤部“关于进一步搞好部队农副业生产的报告”的批示)}
\datesubtitle{(一九六六年五月七日)}


林彪同志:

你在五月六日寄来的总后勤部的报告,收到了,我看这个计划是很好的。是否可以将这个报告发到各军区,请他们马上召集军、师两级干部在一起讨论一下,以其意见上告军委,然后上报中央取得同意,再向全军作出适当的指示。请你酌定。只要在没有发生世界大战的条件下,军队应该是一个大学校,即使在第三次世界大战的条件下,很可能也成为一个这样的大学校,除打仗以外,还可做各种工作。第二次世界大战八年中,各个抗日根据地,我们不是这样做了吗?这个大学校,学政治、学军事、学文化。又能从事农副业生产。又能办一些中小工厂,生产自己需要的若干产品和与国家等价交换的产品。又能从事群众工作,参加工厂农村的社教四清运动;四清完了,随时都有群众工作可做,使军民永远打成一片;又要随时参加批判资产阶级的文化革命斗争。这样,军学、军农、军工、军民这几项都可以兼起来。但要调配适当,要有主有从,农、工、民三项,一个部队只能兼一项或两项,不能同时都兼起来。这样,几百万军队所起的作用就是很大的了。

同样,工人也是这样,以工为主,也要兼学军事、政治、文化。也要搞四清,也要参加批判资产阶级。在有条件的地方,也要从事农副业生产,例如大庆油田那样。

农民以农为主(包括林、牧、副、渔),也要兼学军事、政治、文化,在有条件时候也要由集体办些小工厂,也要批判资产阶级。

学生也是这样,以学为主,兼学别样,即不但学文,也要学工、学农、学军,也要批判资产阶级。学制要缩短,教育要革命,资产阶级知识分子统治我们学校的现象,再也不能继续下去了。

商业、服务行业、党政机关工作人员,凡有条件的,也要这样做。

以上所说,已经不是什么新鲜意见、创造发明,多年以来,很多人已经这样做了,不过还没有普及。至于军队,已经这样做了几十年,不过现在更要有所发展罢了。
<p align="right">毛泽东
一九六六年五月七日</p>


