\section[对中医工作的指示(一九五四年七月三十日)]{对中医工作的指示}
\datesubtitle{(一九五四年七月三十日)}


中医对我国人民的贡献是很大的。中国有六万万人口,是世界上人口最多的国家。我们人民所以能够生衍繁殖,日益兴盛,当然有许多原因,但卫生保健事业所起的作用必须是其中重要原因之一。这方面首先应归功于中医。

中西医比较起来,中医有几千年历史,而西医传入中国不过几十年。直到今天,全国人民疾病的诊疗依靠中医的仍占五万万以上,依靠西医的则仅数千万(而且多半在大城市里)。因此,若就中国从有史以来的卫生保健事业来说,中医的贡献与功劳是很大的。

祖国医学遗产,若干年来,不仅未被发扬,反而受到轻视与排斥(如对中医举行考验,内容有生理、病理等课程,考不及格就不发给证书。另外,还有中医条例,中医进不得医院等)。对中央关于团结中西医的指示未贯彻;中西医的真正团结也还未解决。这是错误的。这个问题一定要解决,错误一定要纠正。首先就要各级卫生行政部门的思想要改变。

今后最重要的是首先要西医学中医,而不是中医学西医。一、要抽调一百名到二百名医科大学或医学院毕业生交给有名的中医,去学习他们的临床经验,而且学习应当抱着很虚心的态度。西医学中医是光荣的,因为经过学习与提高,就可以把中西医界限取消,成为中国统一的医学,以贡献于世界。二、各医院要有计划地请中医来院看病和会诊,允许住院病人用中药,并提出尊重中医的各种制度,从制度上加以保证,使中医到医院里作诊工作不感到困难和顾虑。三、中药应当很好地保护和发展。我国的中药有几千年历史,是祖国宝贵的财产。如果任其衰落下去,那就是我们的罪过。所以对各省生产药材应加以调查保护,鼓励生产,便利运输,改进推销。譬如,有些药材因培植时间较长,由种植到培植收获需二、三年以上,如白芍为四年生植物,黄连为六年植物,个体农民往往没有力量种植,又如有利产药地区,如甘肃、青海交通不便,生产的药材不能及时运出,农民往往把药材当作燃料了。过去一些中药因加工设制的技术不良,浪费很大,包装和贮藏方法不好,霉烂损坏的现象很严重,亦应加以改进。从事这些工作的机构,今后应采取公私合营,制药人员应按技术水平分别给予技术干部看待。至于对中药研究,光做化学分析是不够的,应进而做药理实验和临床实验,特别是对中药的配合作用更应注意。

四、中医书籍应进行整理,过去由于难懂,再加不重视,无人整理。中医医书如不整理,就将绝版,应组织有学问的中医,有计划有重点地先将那些有用的,从古文译成现代文:时机成熟并组织他们(总结)自己的经验,编出一套系统的中医书来。

为了实现以上种种工作,首先在于纠正那种资产阶级个人主义、宗派主义思想。只要思想上有改变,上述各种工作才能贯彻。

今后那一级卫生行政部门如作不好这个工作,就将被撤职。


