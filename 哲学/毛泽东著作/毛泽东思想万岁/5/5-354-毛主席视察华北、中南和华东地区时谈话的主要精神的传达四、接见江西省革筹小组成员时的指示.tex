\section[毛主席视察华北、中南和华东地区时谈话的主要精神的传达四、接见江西省革筹小组成员时的指示]{毛主席视察华北、中南和华东地区时谈话的主要精神的传达四、接见江西省革筹小组成员时的指示}


九月十七日清晨,我们最最敬爱的伟大领袖毛主席来到了南昌。当天上午,毛主席接见了省革命委员会筹备小组程世清、杨栋梁、黄先、刘瑞森、郭光洲、陈昌奉等同志,作了极其重要极为英明的最高指示。

随主席一起参加接见的,有×××,张春桥同志,汪东兴同志,和×××等。

当程世清等同志走到我们伟大领袖毛主席的面前的时候,我们心中最红最红的红太阳、全世界人民的伟大领袖毛主席满面笑容站了起来,他老人家伸出伟大、温暖的手亲切地和程世清等同志等一一握手,并问了每个同志的情况。

我们伟大的领袖毛主席,身穿一身普通的布的便服,穿一双我们解放军战士穿的布鞋。这里我们报告同志们一个最最振奋人心的好消息,我们伟大的领袖毛主席他老人家满面红光,神采奕奕。身体非常健康,非常健康!脸上看不见皱纹,白头发很少。我们当时又想多看看他老人家,又想多听听他老人家的亲切教导,又想多记些他老人家的最新指示,又想多汇报些江西的情况。我们感到浑身充满无限的力量。这是全国人民最大的幸福,是全世界人民最大的幸福。

我们伟大领袖毛主席他老人家记忆力非常好,洞悉一切,高瞻远瞩,是全世界最伟大的天才。对我们教育最大,感受最深。主席对江西省文化大革命情况非常熟悉,非常关心,在谈话中,当谈到抚州问题时,毛主席亲自指出了抚州九个县的名字:“临川、金溪、资溪、南城、南丰、黎川,宜黄、崇仁、乐安。”在谈话中,主席曾三次谈到大联筹、大中红司办的《火线战报》。对江西过去在文化大革命中发生的重大事件,比如“6.29”事件、赣州事件,抚州叛乱事件,以及当前的情况,都很清楚。主席对当前江西文化大革命是满意的。我们向主席汇报当中,主席和蔼可亲,平易近人,谈笑风生,主席向我们问了很多问题,都是江西省文化大革命的重大问题,给我们作了许多最高最新的指示。

下边分五个问题向全省无产阶级革命派、红卫兵革命小将和广大人民群众传达。

(一)毛主席对形势问题的指示

当我们汇报到主席的思想、政策已经在江西广大人民的心中深深扎根,对江西形势起了决定的作用时,毛主席教导我们:“五月底,我写了几句话,给林彪同志、总理,说江西军区同群众为什么这样对立,值得研究,我没有下结论。我指的是江西、湖南、湖北、河南。”

毛主席还说:“六、七、八月间最紧张,紧张的时候,我就看出问题揭开了,事情好解决了,不紧张怎么解决呀!”

当我们汇报到抚州问题时,主席听了汇报后教导我们:“抚州的问题值得研究一下,为什么他们会这样大胆?他们总要开会研究形势,认为江西、全国和世界形势对他们有利,才这样干。他们对形势估计不正确,我看是。抚州问题实际是叛乱,是典型之一。说中国没有内战,我看这就是内战不是外战,是武斗,不是文斗。在赣州、吉安、宜春等地,还搞农管,一个生产队抽一个人,一个大队抽十几个人,采取强迫的办法,记工分,一天六毛钱。现在农村包围城市,我看不行。”

当我们汇报《文汇报》写了一些好文章,影响很大时,张春侨同志插话说:“人家还反我们右倾,说我们变右了。”

接着,主席教导我们:“那有那么多复辟呀,他们已经垮了,不能再复辟了。有一种说法,索性垮就垮了。其实,天下是不会乱的,天也不会塌下来。”

(二)毛主席对干部问题的指示

汇报中间,毛主席指示:“干部垮掉这样多,是好事还是坏事,你们研究了这个问题没有?总要给他们时间,来认识和改正错误,要批评打倒一切的思想。”

毛主席教导我们说:“有些人犯了错误,要给他改正错误的机会。”

当我们汇报到造反派里有的同志有报复思想时,主席教导我们:“不能不教而诛,诛就是杀,诛就是杀人。不能不教而处罚人,过去就是吃了这个亏嘛!“

“我看还应该从教育人手,坏人总是少数。“

“我对现在的右派不那样看死,有坏人是少数,多数是认识问题。有的把认识问题说成是立场问题,一提到立场问题就上了纲,一辈子不得翻身。难道立场问题就不能变吗?对大多数人来说立场是能变的,对极少数坏人是不能变的。总而言之,打击面要缩小,教育面要扩大,教育要包括左、中、右。“

主席询问了过去省委的一些人的情况,然后指示我们:“我还是倾向于多保一些人,能挽救的还是挽救,只要我们争取了多数,极少数人顽固下去也可以嘛,我们给他饭吃算了。“当主席问了过去省委几个犯了严重错误的干部时说:“如果能改,能多争取几个人也好嘛!“

当我们汇报到正在集训军队一些干部时,主席教导我们:“开训练班中央应该开,主要是各省开,不仅军队开,地方党政文教也要集训。训比不训好,时间顶多二个月,久了不行。过去黄埔五个月入伍期,四个月训练。林彪同志只住了五个月的黄埔嘛。有些军事学校,学的时间越长,学得越糊涂。“

当我们汇报到有些同志还要到外边去抓什么人时,主席教导我们:“要保护,不要使人下不了台,要使人有机会搞正错误。“

主席还教导我们:“江西还站出来一批干部嘛。你们是省一级的,省级要多站出一些人,还有市的,能站出多少于部?“

(三)毛主席对造反派教育问题的指示

当我们汇报到大联筹准备召开政治工作会议时,主席教导我们:“这个好,造反派也要训练。他们坐不下来,心野了。造反派人很多,一批不行可以训练二批三批。“

主席教导我们:“左派不教育变成极左。“

当我们汇报到有些人反右倾时,主席教导我们:“是教育左派的问题,不是右倾的问题。比如,过去有多少山头呀,江西有中央苏区,湘赣苏区,湘鄂赣苏区,闽赣苏区,还有鄂豫皖苏区,通南巴,陕北,抗战的时候根据地就更多了,我们用一个纲领团结起来。“

(四)毛主席对参加保守组织群众政策问题的指示

当我们汇报到因为前一阶段造反派受迫害、压抑,现在有些同志有报复思想时,主席教导说:“要很好说服,不打击报复,下跪、高帽子、挂牌子、还有什么喷气式啰,这不好。“

毛主席还教导我们说:“杀人总不好,人家杀你不好,你杀他也不好。“

当我们汇报到联络总站一个负责人自己回来,写了检讨,现在还在造反派那里检讨时,主席指示:“把他收回来好了,不要搞得太苦了。“

(五)毛主席对军队问题的指示

当我们汇报到人武部、军分区的情况时,主席很关心地听了汇报后,教导我们说:“人武部总是好人多,军分区有很多人是受蒙蔽的。“

“到处抓赵永夫、谭震林,哪有那么多赵永夫、谭震林?“

主席还教导我们说:“你们先把武装部干部训练一下。我看训练的办法好,内蒙古一个独立营八百多人,是支保的,反对中央对内蒙间题的决定,他们到了北京,气可大了,大闹,都不听总理的话,打破家具,会开不下去,向中央提出五条要求。以后到北京新城高碑店训练了四十天,都转了,回去还支左不支右,独立营、独立师训一下就转过来了。“

主席还特别指出:“现在有人挑拨战士反对官长,说你们每月只有六块钱,当官的钱多,还坐汽车。农民是愿意当解放军的,解放军很光荣,他们每月还有六块钱,家里还有优待,农民是愿意当兵的,我看是挑拨不起来的。“

对江西军区搞四大,主席教导我们:“江西军区搞四大,不要搞得太苦啦,战士一起来火就很大,浙江现在每天都斗,一斗就是戴高帽、挂黑牌、下跪、搞喷气式,人家受不了,也不雅致嘛。“

   当我们汇报到6011部队四排的事迹时,主席听了很满意,主席说“李文忠排的事迹,我看了火线战报,有他们三个人的照片,他们三个人都很年青。”

{\raggedleft          (原稿由江西省无产阶级革命造反派大联合筹委会翻印)\par}


