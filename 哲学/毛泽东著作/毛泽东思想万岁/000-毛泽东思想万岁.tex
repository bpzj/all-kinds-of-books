<<<<<<< HEAD
\documentclass[b5paper,oneside,12pt]{ctexbook}
\usepackage[hmargin=0.25in,vmargin=0.5in]{geometry} 
\usepackage[]{hyperref}
\pagestyle{plain} %整书页眉页脚设置
\ctexset{chapter/numbering=false}
\ctexset{section/numbering=false}
\ctexset{subsection/numbering=false}
\newcommand\datesubtitle[1]{{\begin{center} \small #1 \end{center}}}  %自定义日期副标题格式,为了保险,最好使用两层大括号

\title{毛泽东思想万岁}
\author{毛泽东}
\date{}

\begin{document}

\frontmatter
\maketitle
\tableofcontents

\mainmatter
% \chapter{}

% \section*{第一师范讲堂录}\addcontentsline{toc}{section}{第一师范讲堂录}
% 使用上面的写法或者在导言区配置: \ctexset{section/numbering=false}
\section{第一师范讲堂录}\datesubtitle{(一九一三年十一月一日)}

\textbf{修身}。人情多耽安佚而惮苦,懒惰为万恶之渊薮。人而懒惰,农则废其田畴,工则废其规矩,商贾则废其所鬻,士则废其所学,业即废矣,无以为生,而杀身亡家乃随之矣。因而懒惰,始则不进,继则退行,继则衰弱,终则灭亡,可畏哉!故曰懒惰万恶之渊薮也。

\textbf{奋斗}。夫以五千之卒,敌十万之军。策罢乏之兵,当新霸之马。如此欲图存而不亡,非奋斗不可。

\textbf{朝气}。少年须有朝气,否则暮气中之。暮气之来,乘疏懈之隙也。故曰怠隋者生之坟墓。

% \section*{第一师范讲堂录}\addcontentsline{toc}{section}{第一师范讲堂录}
% 使用上面的写法或者在导言区配置: \ctexset{section/numbering=false}
\section{读《明耻篇》后题词}\datesubtitle{(一九一五年)}
毛主席在第一师范求学期间,读了当时很多进步书刊。当他看了《明耻篇》以后,义愤填胸,来自挥笔在书上写道:
\begin{center}
“五月七日,\par{}国民奇耻;\par{}何以报仇,\par{}  在我学子!”
\end{center}

这十六个字充分表现了主席青少年时就立下了以天下为己任的革命的雄心壮志。

\section{《新青年》第七卷第一号}\datesubtitle{(一九一七年)} 
长沙受了“五四”运动的鼓动,一般人向新潮方面定的,实在不少,又因近年来屡受军阀派的摧残,弄得长沙这锦绣的地方,常常没有一点生气,于是一般稍有知识的人—各校教职员及学生—莫不深深痛恨。更觉得这《社会改造》、《思想革新》、《妇女解放》、《民族自决》种种问题万不容缓。
\section{工人夜校招生广告}
\noindent{列位工人来听我们说几句白话:}

列位最不便利的是什么?就是俗话说的:讲了写不得,写了认不得,有数算不得。列位做工的人,又要劳动又无人教授,如何才能写得几个字,算得几笔数呢?现今有个最好的法子,就是我们第一师范办了一个夜校,今年上半年学生很多,列位中想有听到过的。这个学校专为列位工人设的,从礼拜一起到礼拜五止每夜上课两点钟,教的是写信,算账,都是列位时刻需要的。讲义归我们发给,并不要钱,夜间上课又于列位工作并无妨碍。若是要求求学的,请赶快于一礼拜内到师范的号房报名。

有说时势不好,恐怕犯了戒严的命令,此事我们可以担保,上学以后,每人发讲牌一块,遇有军警查问,说是师范夜校学生就无妨碍了。若有为难之处,我们替你作保,此层只管放心,快快来报名,莫再耽搁。

\begin{flushright}第一师范学友会教学研究部启\end{flushright}
\section{夜学日记}
\datesubtitle{(一九一七年十一月十四日)片断}

甲班上课算术罗宗翰出席教以数之种类加法大略及亚拉伯数字码,历史常识毛泽东出席教历朝大势及上古事迹。学生有四人未带算盘,从小学暂借,为戒严早半时下课,管理者李端\{输fix\}、萧珍元。

实验三日矣,觉国文似太多太深,太多宜减其分量,太深宜改用通俗语(介乎白话与文言之间),常识分量亦嫌太多(指文字),宜少用文字,其讲义宜用白话,简单几句表明,初不发给,单用精神讲演,终取讲义略读一遍足矣,本日历史即改用此法,觉活泼得多。

本日算术却过浅,学生学过归除者令其举手,有十几人之多,此则宜逐渐加深。


\section{新民学会的方针}
\datesubtitle{(一九一八年四月)}



1938年4月毛泽东同志团结湖南进步青年,组织了革命团体——新民学会,新民学会后来在传播马克思列宁主义和推动革命运动起了重要作用。

明确地提出了学会的宗旨是“改造中国与世界”。

学会的方针问题。我们学会到底拿一种甚么方针做我们共同的目标呢?信里述蒙达尔尼会议,对于学会进行之方针说:“大家决定今后进行之方针在改造中国与世界。”以“改造中国与世界”为学会方针,正与我平日的主张相合,并且我料到是与多数的会友的主张相合的,以我的接触和观察,我们多数的会友都倾向于世界主义,试看多数人鄙弃爱国,多数人鄙弃某一部分,一国家的私利,而忘却人类全体的幸福的事,多数人都觉得自己是人类的一员,而不愿意更复杂地隶属于无意义之某一国家、某一家庭、或某一宗教,而为其奴隶,就可以知道了。


\section{在上海送别第一批去法国勤工俭学同学时讲话}
\datesubtitle{(一九一九年初)}



我觉得我们要有人到外国去,看些新东西,学些新道理,研究有用的学问拿回来,改造我们的国家。同时也要有人留在本国,研究本国问题。我觉得我对于自己的国家我所知道的还太少,假若我把时间花费在本国,则对本国更为有利。


\section{湘江评论创刊宣言}
\datesubtitle{(一九一九年七月十四日)}



自“世界革命”的呼声大倡。“人类解放”的运动猛进。从前吾人所不置疑的问题。所不递取的方法。多说畏缩的说话。于今都要一改旧观。不疑者疑。不取者取。多畏缩者不畏缩了。这种潮流。任是什么力量。不能阻住。任是什么人物。不能不受他的软化。

世界什么问题最大?吃饭问题最大。什么力量最强?民众联合的力量最强。什么不要怕?天不要怕。鬼不要怕。死人不要怕。官僚不要怕。军阀不要怕。资本家不要怕。

自文艺复兴。思想解放。“人类应如何生活”。成了一个绝大的问题。从这个问题。加以研究。我深了“应该那样生活”“不应该这样生活”的结论。一些学者倡之。大多民众和之。就成功或将要成功许多方面的改革。

见于宗教方面为“宗教改革”。结果得了信教自由。见于文学方面。由贵族的文学。古典的文学。死形的文学。变为学民的文学。现代的文学。有生命的文学。见于政治方面。由独裁政治。变为代议政治。由有限制的选举。变为没限制的选举。见于社会方面。由少数阶级专制的黑暗社会。变为全体人民自由发展的光明社会。见于教育方面。为平民教育主义。见于经济方面。为劳获平均主义。见于欲想方面。为实验主义。见于国际方面。为国际同盟。

各种改革。一言蔽之。“由强权得自由”而已。各种对抗强权的根本主义。为平民主义。(兑莫克拉西。——作民本主义。民主主义。庶民主义。)宗教的强权。文学的强权。政治的强权。社会的强权。教育的强权。经济的强权。思想的强权。国际的强权。丝毫没有存在的余地。都要借平民主义的高呼。将他打倒。

如何打倒的方法。则有两说.一急烈的。一温和的。两种方法。我们应有一番选择。

(一)我们承认强权者都是人。都是我们的同类。滥用强权。是他们不自觉的误谬与不幸。

(二)用强权打倒强权。结果仍然得到强权。不但自相矛盾。并且毫无效力。欧洲的“同盟”。“协约”战争。我国的“南”“北”战争。都是这一类。

所以我们的见解。在学术方面。主张彻底研究。不受一切传说和迷信的束缚。要寻着什么真理?在对人的方面。主张群众联合。向强权者为持续的“忠告运动”。实行“呼声革命”——面包的呼声。自由的呼声。平等的呼声。——“无血革命”。不至张起大扰乱。行那没效果的“炸弹革命”“有血革命”。

国际的强权。迫上了我们的眉睫。就是日本。罢课。罢市。罢工。排货。种种运动。就是直接间接对付强权日本有效的方法。

至于湘江。乃地球上东半球东方的一条江。他的水很清。他的流很长。住在这江上和他邻近的民族。浑浑噩噩。世界上事情。很少懂得。他们没有有组织的社会。人人自营散处。只知有最狭的一己。和最短的一时。共同生活。久远观念。多半未曾梦见。他们的政治没有和意和彻底的解决。只知道私争。他们被外界的大潮卷急了。也办了些教育。却无甚效力。一般官僚式教育家。死死盘踞。把学校当监狱。待学生如囚徒。他们的产业没有开发。他们中也有一些有用人材。在各国各地学好了学问和艺术。但没有给他们用武的余地。闭锁一个洞庭湖。将他们轻轻挡住。他们的部落思想又很厉害。实行湖南饭湖南人吃的主义。教育实业界不能多多容纳异材。他们的脑子贫弱而又腐败。有增益改良的必要。没有提倡。他们正在求学的青年。很多。很有为。没人用有效的方法。将种种有益的新知识新艺术启导他们。咳,湘江湘江!你真枉存在于地球上。

时机到了!世界的大潮卷得更急了!洞庭湖的闸门动了。且开了!浩浩荡荡的新思潮业已奔腾澎湃于湘江两岸了!顺他的生。逆他的死。如何承受他?如何传播他?如何研究他?如何施行他?是我们全体湘人最切最要的大问题。即是“湘江”出世最切最要的大任务。

\subsection{西方大事述评—各国的罢工风潮}

法英美三国的官阀和财阀,倾注全力于巴黎和会,用高压手段对付败北的德奥,正在兴高釆烈时候,他们的国内,忽然发生了罢工风潮。罢工在他们国里,原是一件常事,政府和财阀,虽然不敢十分轻视劳动者。每当劳动者拿着劳获不均,工时太久,住屋不适,失职无归,种种怨愤不平的问题,联合同类,愤起罢工的时候,也不得不小小给他们一点恩惠。正如小儿哭饿,看着十分伤心,大人也不得不笑着给他一个饼子。但终是杯水车薪,济得甚事。所以广义派人,都笑英法的工人是小见识。从老虎口里讨碎肉,是不能够的。

此回各国的罢工风潮,英国因为在大战初了时候,(去年十二月)英伦加苏格兰各埠交通机关,燃料业,矿山业,造船业等已演了一大罢工。故此次罢工,未发生于英伦本土。法国罢工情形,初颇严重。终亦以小惠收场,没闹出什么好结果。广义派人有乘机在巴黎实行政治运动之说,亦未见诸事实。美国一部分电报电话人员的罢工,乃在附和议员多数派反对加入国际联盟。与英法罢工,异其目的。意大利之罢工乃社会觉嫉恶其政府所为的一种运动。德国自去冬少数社会党大失败,各处大罢工,亦随之而没得好结果。多数社会党掌握政权以来,早已噤若寒蝉,不敢出声。此次为和约签字问题,有激起罢工的形势。但施特满内阁倒了,继任巴安内阁,仍是前内阁的同调。抵御外侮不足,防备家贼有余的武力,紧握在手,谁敢予侮。广义一派的猛断政略,暂时决没有发动的机会。罢工不能成为事实,亦无足怪。匈牙利所受罢工影响不大,其原因则全在缺粮没饭吃。今将一月以来各国罢工情形,分述于下——

法国\quad{}六月三日,罢工风潮发生后,蔓延甚速。巴黎一区,男女工人赋闲者,二十万人。所要求各业不同,而一致主张每日工作八小时。四日蔓延更广,推算当有五十万人罢工。五日,洗衣工人罢工。自后罢工的人数更多。地底铁道,电车,街车的用人决议继续罢工。地底铁道工人要求工值每月至少四百五十佛朗。(以每佛朗当四角合我一百八十元)满五十岁须给养老金。服役十五年后亦须给养老金若干。七日巴黎罢工现象有转机,五金业与机器业,工人与雇主,已商妥数事。十一日,五金业及地底铁道工人上工。当道已取必要万法对付铁道罢工。煤矿工人有全体罢工的形势。十二日,国会通过矿工每日工作八时议案。但矿工会议,仍不满意,决定从十六日,全体罢工。水夫联合会,也决计于十六日罢工。工人联合会,言及生活代价奇昂,(记者按,近有从巴黎回者,举一物贵实例,一个旧牙刷,价二佛朗。一双皮鞋,价六十佛胡。)谓非洲各口岸,堆集麦粮千百吨,任其朽腐。各埠存货如山,轮船火车,宁闲置不远载。这样的政府,可要快快废止他的消耗,欺骗,和垄断!十四日,风潮渐平。极端派有乘机推翻克勒满沙强权政府的运动。路工联合会拒绝之。但矿工因解释政府每日工作八小时议案,未能满意,定十六日全体罢工。恐怕路矿运输联合会工人,也会罢工,表示同情。克勒满沙老头子急了,和运输公司及运输工人代表会商,恳请彼等在国家危机时候,发出些爱国热忱。工人吃了他的浓米汤,已老老实实决议上工了。

英国\quad{}伦敦五月三十日电,全国警察拟于三日罢工的气象,正在酝酿中,政府已允增给薪资,优加待遇。但不承认警察联合会及收用已革除的警察。英国的属地澳洲,坎拿大,苏依士,昔有罢工风潮。六月四日,坎拿大维克斯兵工厂工人罢工,要求每星期工作四十小时。六月五日,苏依士运河工人罢工,局势狠恶,六月九日,澳港航务罢工,势头狠烈,各项工业,都受窒碍。十×日,风潮仍严重,他业工人,因此赋闲的逐日增多。

美国\quad{}六月七日,芝加哥电报司员联合会一时罢工,共约六万人。内有二万五千人系属电报司员联合会.该会会长康能堪氏正计划全国罢工办法。同日,全国电话司员奉命于十六日起罢工,表同情于电报司员。八日,电报人员联合会干事,向全体电报人员宣布,连收发电报生在内,全体罢工。目的在停止威尔逊总统每日在巴黎往来的电报,使他注意国民不赞成他在和会的主张。十二日,各电报公司报告,电报司员罢工没成。

意国\quad{}六月十三日,意大利斯贝齐亚地方,因粮食昂贵,发生暴动,捣毁商店。十四日,热那亚工界示威,被捕者数百人,银行商店闭门,电车不走。杜林工人此日多停工,纪念德国斯巴达团领袖卢森堡氏。米兰工人罢工,抗议热那亚与斯贝齐亚当道的行动。

德国\quad{}六月十三日,大柏林公民会议秘密会,决议罢工。各职业及军界中人,均赞助停止各项实业工作的计划。有人料此举将促成国内战争。中等社会将得政权。

匈国\quad{}五月三十一日,匈京饥饿的工人,发生暴动.红旗军奉共产政府命令到各工厂制乱。匈京几无粮食。

\subsection{东方人事述评—陈独秀之被捕及营救}

前北京大学文科学长陈独秀,于六月十一日,在北京新世界被捕。被捕的原因,据警厅方而的布告,系因这日晚上,有人在新世界散布市民宣言的传单,被密探拘去。到警厅诘问,方知是陈氏。今录中美通讯社所述什么北京市民宣言的传单于下——

一、取消欧战期内一切中日秘约。

二、免除徐树铮曹汝霖章宗群陆宗舆段芝贵王怀庆职。并即驱逐出京。

三、取消步军统领衙门,及警备总司令。

四、北京保安队,由商民组织。

五、促进南北和议。

六、人民有绝对的言论出版集会和自由权。

以上六条,乃人民对于政府最低之要求,乃希望以和平方法达此目的。倘政府不俯顺民意,则北京市民,惟有直接行动,图根本之改造。

上文是北京市民宣言传单,我们看了,也没有什么大不了处。政府将陈氏捉了,各报所载,很受虐待。北京学生全体有一个公函呈到警厅。请求释放。下面是公函的原文――

警察总监钧鉴,敬启者,近闻军警逮捕北京大学前文科学长陈独秀,拟加重究,学生等期期以为不可。特举出两要点于下:(一)陈先生夙负学界重望,其言论思想,皆见称于国内外。倘此次以嫌疑遽加之罪,恐激动全国学界再起波澜。当此学潮紧急之时,殊非息事宁人之计。(二)陈先生向以提倡新文学现代思想见异于一般守旧者。此次忽被逮捕,诚恐国内外人士,疑军警当局,有意罗织,以为摧残近代思想之步。现今各种问题,已极复杂,岂可再生枝节,以滋纠纷?基此二种理由,学生等特请贵厅,将陈独秀早予保释。

北京学生又有致上海各报各学校各界一电——

陈独秀氏为提倡近代思想最力之人,实学界重镇。忽于真日被逮。住宅亦披抄查。群清无任惶骇。除设法援救外,并希国人注意。

上海工业学会也有请求释放陈氏的电。有“以北京学潮,迁怒陈氏一人,大乱之机,将从此开始”的话。政府尚未昏聩到全不知外间大事,可料不久就会放出。若说硬要兴一文字狱,与举世披靡的近代思潮,拚一死战,吾恐政府也没有这么大胆子。章行严与陈君为多年旧交,陈在大学任文科学长时,章亦在大学任图书馆长及研究所逻辑教授。于陈君被捕,即有一电给京里的王克敏,要他转达别厅,立予释放。大要说——

……陈君向以讲学为务,平生不含政治党派的臭味。此次虽因文字失当,亦何至遽兴大狱,视若囚犯,至断绝家常往来。且值学潮甫息之秋,訑可忽兴文绵,重激众怒。甚为诸公所不取。……

章氏又致代总理龚心湛一函。说得更加激切——

仙舟先生执事,久违矩教,结念为劳。兹有恳者,前北京大学文利学长陈独秀,闻因牵涉传单之嫌,致被逮捕,迄今末释。其事实如何,远道未能详悉。惟念陈君平日,专以讲学为务。虽其提倡新思潮,著书立论,或不无过甚之词,然范围实仅及于文字方面,决不舍有政治臭味,则固皎然可征。方今国家多事,且值学潮甫息之后,讵可蹈腹诽之殊,师监谤之策,而愈激动人之心理耶。窃为诸公所不取。故就历史论,执政因文字小故而专与文人为难,致兴文字之狱。幸而胜之,是为不武。不胜人心瓦解,政纽摧崩,虽有善者。莫之能挽。试观古今中外,每当文网最甚之秋,正其国运衰歇之候。以明末为殷鉴,可为寒心。今日谣琢萦兴,清流危惧。乃遽有此罪及文人之举,是露国家不祥之象,天下大乱之基也。杜渐防微,用敢望诸当事。且陈君英姿挺秀,学贯中西。皖省地绾南北,每产材武之士,如斯学者,诚叹难能。执事平视同乡诸贤,谅有同感。远而一国,近而一省,育一人才,至为不易。又焉忍遽而残之耶。特专函奉达,请即饬警厅速将陈君释放。钊与陈君总角旧交,同衿大学。于其人品行谊,知之甚深,敢保无他,愿为左证。……

\begin{flushright}章士钊拜启六月二十二日\end{flushright}

我们对于陈君,认他为思想界的明星。陈君所说的话,头脑稍微清楚的听得,莫不人人各如其意中所欲出。现在的中国,可谓危险极了。不是兵力不强财用不足的危险,也不是内乱相寻四分五裂的危险。危险在全国人民思想界空虚腐败到十二分。中国的四万万人,差不多有三万万九千万是迷信家。迷信鬼神,迷信物象,迷信运命,迷信强权。全然不认有个人,不认有自己,不认有真理。这是科学思想不发达的结果。中国名为共和,实则专制。愈弄愈糟,甲仆乙代,这是群众心里没有民主的影子,不晓得民主究竟是甚么的结果。陈君平日所标揭的,就是这两样。他曾说,我们所以得罪于社会,无非是为着“赛因斯”(科学)和“兑莫克拉西”(民主)。陈君为这两件东西得罪于社会,社会居然就把逮捕和禁锢报给他,也可算是罪罚相敌了,凡思想是没有畛域的,去年十二月德国的广义派社会党首领卢森堡被民主派政府杀了,上月中旬,德国仇敌的意大利一个都林地方的人员,举行了一个大示威以纪念他。瑞士的苏里克,也有个同样的示威给他做纪念。仇敌尚且如此,况在非仇敌。异国尚且如此,况在本国。陈君之被逮,决不能损及陈君的毫末。并且是留着大大的一个纪念于新思潮,使他越发光辉远大。政府决没有胆子将陈君处死,就是死了,也不能损及陈君至坚至高精神的毫末。陈君原自说过,出实验室,即入监狱。出监狱,即入实验室。又说,死是不怕的。陈君可以夺验其言了。我祝陈君万岁!我祝陈君至坚至高的精神万岁!

\subsection{世界杂评}

强叫化前月的初间,日本米价顶贵时候,每石超四十元。日当局有狼狈之状。报纸证言粮食的危机已迫。可怜的日本!你肠将饥断,还要向施主逞强。天下那有强叫化续得多施的理。

研究过激党阿富汗侵印度,俄过激党为之主谋,过激党到了南亚洲。高丽的“呼声革命”正盛吋,亦有过激党参与之说,则已到了东亚。过激党这么厉害!各位也要研究研究,到底是个什么东西?切不可闭着眼睛,只管瞎说,“等于洪水猛兽”“抵制”“拒绝”等等的空话。一光眼,过激觉布备了全国,相惊而走,已没得走处了!

实行封锁前月巴黎高等经济会议议决,实行封锁匈牙利,说理直到匈政府宣言遵从民意时为止。这要分两层观察,一、协约国看错了匈政府与匈国民志愿不合。匈政府与匈国民之少数有产阶级,绅士阶级,志愿不合是有的,若与大多数无产阶级,平民阶级,没有志愿不合的理。因为匈政府,原是他们所组织的。二、实行封锁,这是帮助过激主义的传播。吾恐怕协约国也会要卷入这个漩涡。果然,则这实行封锁,真是“罪莫大焉”了。

证明协约国的平等正义德国复文和会,要求德国陆军减少之后,协约国也须同减。这话谁人敢说错了?协约国满嘴的平等主义,我们且看协约国以后的军备如何?就可求个证明。

阿富汗执戈而起一个很小的阿富汗,同一个很大的海上王英国开战,其中必有重大原因。但据英国一方面的电传,是靠不住的。土耳其要被一些虎狼分吞了。印度舍死助英,赚得一个红巾照烂给人出丑的议和代表。印民的要求是没得允许。印民的政治运动,是要平兵力平压。阿富汗是个回教国,狐死冤悲,那得不执戈而起?

来因共和国是丑国协约国要划来因流域为自己挡敌的长城,必先使之脱离德国的关系,别成一国。听说已在威萨登成立临时政府,一位这登博士做总统。这位道登阵士不知果然高兴到甚么样?金人立了刘豫,契丹立了石敬塘,我们中国也曾有几个这样的国呢。

好个民族自决波兰捷克复国,都所以制德国的死命,协约因尽力援助之,称之为“民族自决”。亚刺伯有分裂士耳其的好处,故许他半自立。犹太欲在巴力斯坦复国,因为于协约国没大关系,故不能成功。西伯利亚政府有攻击过激党的功绩,故加以正式承认.日本欲伸足西伯利亚,不得不有所示好,故首先提议承认。朝鲜呼号独立,死了多少人民,乱了多少地方,和会只是不理。好个民族自决!我们认为直是不要脸!

可怜的威尔逊威尔逊在巴黎,好象热锅上的蚂蚁,不知怎样才好?四围包满了克勒满沙,路易乔治,牧野伸显,欧兰杜一类的强盗。所听的,不外得到若干土地,收赔若干金钱。所做的,不外不能伸出己见的种种会议。有一天的路透电说:“威尔逊总统卒已赞成克勒满沙不使德国加入国际同盟的意见”。我看了“卒已赞成”四字,为他气闷了大半天。可怜的威尔逊!

炸弹暴举人人知道很文明很富足的美国,有“炸弹暴举”同时在八城发生。无政府党蔓延甚广。炸弹爆炸的附近,有匿名揭贴说,“阶级战争”业已发生,必得国际劳动界完全胜利,始能停止,炸弹往往埋藏在一些官员的住宅,屋顶上发现人头。可怕可怕!我只挂牵官员人家的一些小姐小孩子,他们晚上如何睡得着?议院里一些钱多因而票多票多因而当选的议员,还在那里痛诋暴动者,通过严惩案。我正式告诉诸位,诸位的“末日审判”将要到了!诸位要想留着生命,并想相当的吃一点饭,穿一点衣,除非大大的将脑子洗洗,将高帽子除下,将大礼服收起,和你们国里的平民,一同进工厂做工,到乡下种田。

不许实业专制美国工党首领戈泊斯演说曰,“工党决计于善后事业中有发言权,不许实业专制。”美国为地球上第一实业专制国,托辣斯的恶制,即起于此。几个人享福,千万人要哭。实业越发达,要哭的人越多。戈泊斯的“不许”,办法怎样?还不知道。但既有人倡言“不许”,即是好现象。由一人口说“不许”,推而至于千万人都说“不许”,由低声的“不许”,推而至于高声的很高声的狂呼的“不许”,这才是人类真得解放的一日。

割地赔偿不两全德国答复协约国,说,如失去西里细亚及萨尔煤矿,则无力行赔偿。我料协约国听了一定很烦脑。何以故?地可割,赔偿也可得,最为两全。据德国的说,两样便成了反比例,如之何不烦脑?虽然,奉劝协约国的衮衮诸公,天下那有两全的好事!

为社会党造成流血之地奥总代表任纳博士答复和会,说,“奥国今已坐食其较前大减之资本,若再加以摧残,必为社会党造成流血之地”,蠢哉任纳博士,你还不知道协约国一年以来之真目的,你专为造成社会党流血之地吗?

彭斯坦德博士彭斯坦演说曰,“媾和条件”乃野蛮战争的结果,德国最宜负责,和约条件十九为必要的”。我们固然反对协约国的强迫和约,但博士这话,系专对野蛮战争而发,听了倒很爽快。

各国没有明伦堂康有为因为广州修马路,要拆毁明伦堂,发了肝火。打电给岭伍,斥为“侮圣灭伦。”说,“遍游各国,未之前闻”。康先生的话真不错,遍游各国,那里寻得出什么孔子,更寻不出什么明伦堂。

什么是民国所宜?康先生又说,“强要拆毁,非民国所宜”。这才是怪!难道定要留看那“君为臣纲”,“君君臣臣”的事,才算是“民国所宜”吗?

大略不是人邓镕在新国会云,“尊孔不必设专官,节省经费”。张元奇云,“内务部祀孔,由茶房录事办理.次长司长不理,要设专官。”内务部的茶房录事,大略不是人。要说是人,怎么连祀孔都不行呢?我想孔老爹的官气到了这么久的年载,谅也减少了一点。

走昆仑山到欧洲张元奇又说,“什么讲求新学,顺应潮流,本席以为应尊孔逆挽潮流。”不错不错!张先生果然有此力量,那么,扬子江里的潮流。会从昆仑山翻过去,我们到欧洲的,就坐船走昆仑山罢。

\subsection{湘江杂评}

好计策一个学校的同学对我说,我们学校里办事人和教习,怕我们学到了他们还未学到的新学说,将图书室看×了。外面送来的杂志新闻纸和书籍,凡是稍新一点的,都没得见。我听了为之点首叹服。他们的计策真妙!岂仅某学校,通湖南的学校,千篇一律都象联了盟似的。

摇身一变一些官僚式教育家,为世界的大潮卷急了,不提防就会将他们的饭碗冲破。摇身一变,把前日的烂调官腔,轻轻收拾。一些其有所感而改变的,很可佩服。一些则是假变,容易露出他们的马脚。这类人我很为他羞!很为他危!

我们饿极了我们关在洞庭湖大门里的青年,实在是饿极了!我们的肚子固然是饿,我们的脑筋尤饿!替我们办理食物的厨师们,太没本钱。我们无法!我们惟有起而自办!这是我们饿极了哀声!千不要看错!

难道走路是男子专有的一个女学校里的办事人,把学生看做文契似的收藏起来,怕他们出外见识了甚么邪样。新青年一类的邪书,尤不准他们寓目。此次惊天动地的学生潮,北京的女学生聚诉新华门。贫儿院的小女孩子,愿到监狱替男学生抵罪。这个女学校的学生独深闭固拒,一步也不出外,好象走路是男子专有似的。

哈哈!青岛问题发生,湖南学生大激动,新剧演说,一时风行。有一位朋友对我说,一位老先生,因为他的儿子化装演剧,气得了不得。走到学校问先生,开口便说,“我的命运如何这么乖?养大的儿子竟做出那么下流事”?我听了这话,忍不住卧的一声,哈哈!

女子革命军或问女子的头和男子的头,实在是一样。女子的腰和男子的腰实在是一样。为什么女子头上偏要高竖那招摇畏风的髻?女子腰间偏要紧缚娜拖泥带水的裙?我道,女子本来是罪人,高髻长裙,是男子加于他们的刑具。还有那脸上的脂粉,就是黔文。手上的饰物,就是桎梏。穿耳包脚为肉刑。学校家庭为牢狱,痛之不敢声,闭之不敢出。或问如何脱离这弊?我道,惟有起女子革命军。


\section{湘江评论第二号}
\datesubtitle{(一九一九年七月二十一)}



\subsection{西方人事述评}

德意志人沉痛的签约

签约之前:败而不屈的德意志的代表兰超等于(五月初旬)到巴黎。(五月七日)在凡尔赛宫举行很庄严的和约交付礼。德代表的态度很倨傲。克勒满沙站起声述其开会词。德总代表兰超,则坐诵其如下之演说词——

德国军事破裂,德国失败的程度,自己明白。但这回欧战,负责的不仅德国,全欧都与有罪。因五十年以来,欧洲各国的帝国主义,实贻毒于国际局势。德国战中的罪行,固不可讳,战事的时候,人民的天良,为感情所蔽,故有罪行。然自去年十一月十一日以后,德国没与战事的人,多死于封锁的影响,协约国亦冷淡视之。威总统十四条大纲,为全世界所赞助,协约国业已声明依照此项大纲而立和约,那么,德国当不致于全没救护。国际同盟,各国都让加入,不能将德国丢在外边。德国愿以好意的精神,研究和约。……

和约为灰黄色封面一大册,和会秘书长杜斯玛,捧了交到兰超手中。兰超回到寓舍,晚餐的时候,默无一言。晚餐毕,即使人翻译和约,于晨间三点译成,送到兰超寝室。兰超看到天明方毕。另外录出几份,派专差送到柏林。八日德内阁会议许见。

内阁总理施特满,向考虑协约委员会演说——

和约条件,简直是宣告德国死刑。政府必以政治的沉静态度,讨论这可厌而狂妄的公文,……

随将和约条件,电告各联邦政府,请他们表示意见。因为感受很深的痛苦,特命公众停止行乐一星期。仅许剧院演唱和这痛苦极没相同的悲刷。股票交易所,因感受痛苦的印象,停闭三日。各界人士听得和约会要签字,皆为作怒,群相讨论拒绝签字的后患,甚至没有一人想劫或可受纳此项条件的。柏林各报一致(言尔)诋,有的说,“和约的苛刻远过最消极的预料,这系狂暴无知识的制品,若不能修改,只有用‘否’字答他。”有的说“我们如签定这约,实是屈于武力。我们的心中,应坚决拒绝。”惟独立社会党的机关报,则主张签约说,“从经验看来,拒绝徒增后患。”这时候最可注意的,是德国政党的态度。多数时会党的政府派,是不主张签字的。民治党和中央党也是这样。只有独立社会党不然。(十二日)独立社会党通过决议案,主张接受和约,并说“德现政府恢复显武主义的行为,使别人坚其对德的疑惧。德国舍屈于强迫签字,没有办法。俄德和约,及德罗和约,均没多久的寿命就取消了。凡尔赛和约,也未尝不可以革命的发展取消他。”我们为德国计,要想不受和约,惟有步俄国和匈牙利的后尘,实行社会的大革命。协约国最怕的就在这一点。俄罗斯,匈牙利,不派代表,不提和议,明目张胆地对抗协约国,协约国至今未如之何。向使去冬德国广义派社会党的社会革命成了功,则东联俄而南结奥,更联合匈牙利和捷克,广播其世界革命主义,或竟使英法美久郁的社会党,起而响应,协约园国府还食得下咽吗?独立社会党和广义派社会党,本是一党而分为二,他的议论如此,本不足怪。用革命的发展取消和约这话正不要轻看呢。

同日德因会讨论媾和条件。施特满演说――

今日为德国人生死关头!我们必须团结一致!我们除谋使国家生存,无旁的责任!德国不图进行其国家主义的梦想,并没争权的问题。于今人人的喉咙中间都觉有手塞住他的呼吸!人类的尊严,现付于诸君的手中,以保持他!

(十四日)兰超致克勒满沙一个履文,内容的大要如下——

和约中关于领土的条款,是使德国失去其×关重要的生产土地。各地薯芋的收成,将减百分之二十一。煤减三分之一。铁减四分之三。锌减五分之三。德国既因失去殖民地和商船,使经济成了麻木不仁。今又不能得充分的原料,势将被毁到极大的程度。同时输入的粮食必将大减。依赖航务和商业为生的数百万人,德国政府不能将工事和粮食供给他们则势不得不到国外求生,而重要的国家。多禁止德国移民入境。故签定和约,不啻向数百万德人宣告死刑……

兰超于上述的牒文之外,更以牒文两迈致克勒满沙。第一通的大要说,“协约国占德土地,和威总统宣布的主义不合。”第二通的乃关于赔偿条款提出抗议。谓“德国愿赔偿,但不是因为负了战争的原故。”我们看德国的抗议,大家注重(一)不独负战争责任。(二)不愿失去原料所从出的土地。其他各项,虽有抗议,但不是最重要的处所。

(十三日)晚上,柏林有大举示威。多数社会于示威时,起坛演说谓,“和约条件,较罗马施于加萨臣的,尤为刻毒而可羞。”群众游行各街,止于协约委员团所住的亚特伦旅馆前面。有人向众演说,其势汹汹,欲攻旅馆,为警察所阻。到内阁总理屋前,施特满氏临窗演说。又有人民一大队,于薄暮时候,唱歌到亚特伦旅馆,大呼“推翻强暴的和局。”“克勒满沙皆亡”。“与英伦皆亡。”众又到施特满处,请他演说,施氏讲到威尔逊总统十四公纲时,众忽大呼“与威尔逊皆亡。”这日柏林和乡间独立社会党开会,有四十处。

(十九日)柏林某报载有社会党领袖彭斯坦博士的演说,谓“非常苛刻的和约条件,非完全出于激怒与仇念。实德国政策既不能见信于人,则当然受此待遇。一切破坏咎在德国。德国之履行各项要求,不过补偿他们前所寄予人的而已。我很不以一般人士所发激烈的演说为然。告诉他们!不可再具一九一四年八月四日的气焰!”这于德国热烈的反对声中,算是一瓢凉水。

(二十日)兰超致书和会,要求改正审查及讨论和约的期限,(二十二日)克勒满沙复书,允许展长期限,至五月二十九日为止。(二十三日)晚德全权大使起程往斯巴,将和来自柏林的阁员数人会晤,决定一切。(二十四日)斯希特芒,欧士白格,从柏林乘车到斯巴,兰超及委员十六人也到,即开一极长的会议,斯希特芒主席,通过德国的反提案。会毕,政府委员回柏林,兰超等回凡尔赛。

(五月二十七日)。德国有答案交付和会,答案的第一部分,要点如下一一

(一)德国承认减少军队到十万人。

(二)交出巨大的军舰,而保留商船。

(三)反对关于东边土地的决定。要求于东普鲁士区中,举行庶民大会。

(四)承认丹齐为自由口岸。

(五)要求协约国,在签字四个月后,撤退军队。

(六)要求加入国际同盟。

(七)坚欲取得代替殖民地的权利。

(八)赔偿总数,不得过十万兆马克。

(九)拒绝引渡凯撒及其他人物。

(十)德国须有重新经商海外的权利。

德国答案的第二部分,亦有如下的要点――

(一)过渡时代,须维持大军,以保治安。

(二)须许德人开公民大会,讨论土地割让问题。并许奥人以加入德国的便利。


(三)拒绝割让西里细亚上部。

(四)不承认俄国有享取赔偿的权利。

(五)无赔偿意、门、罗、波兰等国的义务。

右之德国答案,四大国代表为长久讨论后,提出答复文,将德国所约议的,逐件驳复。全文很长。不外说德国战争责任万难推诿,德国必须尽其能力赔偿损失,必须交出戎首,和战时行暴的人,用法惩治。必须于数年内受特别的约束。凡协约国所持以构成和约的根本主义,万难更易。惟对于德国的实际建议,可以让步云云。

(五月二十九日以后)又因德代表的请求,屡次展缓签约日期。最后允展至六月二十八日为止。六月前半月的光阴,全为着往复反议事占去。

至(六月十八日)德代表团,乃由法京回德,一致告诫德内阁,拒绝签约。德内阁乃准备在韦马召集国会,议此次重大问题。此时协约方面,早做军事准备。一俟德国有不签字的表示,即行进军。德国已处于不能不签字的情势了。

(六月二十二日),协约国对德的“最后复文”于这日送达德代表,限德国以五日承受和约。“最后复文”内,述可以让步的条件,如下一一

(一)西里细亚上部,实行民众投票。

(二)西普鲁士边界,重行划定。

(三)德军暂增至二十万人。

(四)德国宣布愿于一个月内将被控破坏战时法律的人名单开出。

(五)修改关于财政问题的细则。

(六)以德国履行义务为条件,保证德国为将来国际同盟的一员。

施特满内阁知和约不能再有挽回,遂决计引退。

(二十二日)德新内阁组织成立。国务总理巴安氏,外交穆勒氏,财政欧士白格氏,内务达维特氏,陆军拿斯奇氏,殖民贝尔氏,邮电格莱斯勃资氏,劳动森士南氏,工程斯利奇氏,公业经济惠塞尔氏,国库夏勒氏,粕食斯密氏。巴安氏及诸阁员,多属多数社会党,本系前内阁的同调,在这回外交紧急声中,出当此签定和约的难局。新内阁既成,已可决其是预备签约的了。施特满退职。施特满内阁所委任的媾和代表,当然随着退职,于是德国讲和代表团易人。新代表团即以新内阁中的外交总长穆勒,邮电总长格莱斯勒资等组织而成。

此时国会业已在韦玛召集,巴安氏即赴国会,作很沉痛的演说,极言加入新政府的痛苦。恳请国会确立主张,否则战事将屡发作。巴安氏曰,“我特于自由的日耳曼最后一次,起抗此强暴破坏的和约!起抗此自决权利的假面具!起抗此奴隶德人的手段!起抗此妨害世界和平的新器!”国会乃于“反对”“赞成”的喧哗声中通过签约动议。

签约的动议通过,二十三日巴安再赴国会,申述无条件签约的必要,其演词谓,“战败的国家,身魂受世界的凌辱!吾人姑且签定和约。吾人一息尚存,终望损害吾人荣誉的人,有一日身受报应,”这时候右党提出抗议。乃付表决。结果证实准许签约。议长贵里巴贽氏起立发短节的演说,“以不幸的祖国,委托于慈悲的上帝!”且谓“各党领导,允宣告军界,全国希望海陆军树先己牺牲的模范,辅助劳工,重造祖国!”关系全世界安危的德国签字,在一场非常惨痛的演说声中,完全决定。德意志人的大纪念,有史以来,当没有过于这日了:

签字案既经国会通过,德新代表团乃到巴黎,致“允可签约的牒文”于和会。牒文的大要说,一一日耳曼民国政府,知协约国决计以武力强迫日耳曼承受和约条件。此项条件,虽没有重大的意味,然实老在剥夺日耳曼人民的荣誉。日耳曼政府虽屈服于作势的武力,但关于从古未闻离背公道的和约条件。”

右文既布,各国的欢忭,自不可言。至(二十八日)而最后展缓的满期已到。于是凡尔赛宫中,仍有亘古未闻的大签约一举。

签约之际一千九百十九年六月二十八日午后三点五分,凡尔赛宫中开会。在宫中设高坛,甚为庄严。协约国全权代表首先会集。次为德国全权代表,只到外总长穆劝和交通总长裴尔,其余均不愿到。克勒满沙主席,首发短简宣言,谓:“协约国和共同作战国政府,均赞成媾和条件。今加签字,表示彼等忠诚依守庄严的了解。”继乃“请日耳曼民国代表首先签字。”德代表所坐席次忽发大声,“德意志!”“德意志!”克勒满沙于是乃改称“德意志”。德代表即起立在约上签名,裴尔氏首先签之。时为午后三时十二分,园中喷泉四射,炮声大作,当德代表回到坐处的时候,会场皆露喜容。次为美国签字。次为英国签字。次为法国签字。次为意国签字。次为日本签字。最后签字的为捷克斯拉夫民国。三时三十五分签字完毕。克勒满沙氏宣布散会。

签约之后,当德国国会允许签约的消息传布,德国全国自即有爱国的示威运动。群众列队唱战歌,国歌,欢呼致敬于年老的统兵员,各报对于裁判德皇问题,表示极大的忿怒。有一报恳请一九一四年的陆军军官,表示如德皇受裁判,也愿受协约国的裁判。并请组织团体,或须入荷兰,保护德皇。各地暴动罢工事情,接续而起。及(二十八日)和约签字的消息传到柏林,柏林某报即载出一文,谓“德人终必报一九一九年的耻辱!”为政府禁止发行。(二十九日)各报皆有“黑线”,表示哀痛。各报皆载有极悲观的评论。柏林及各地铁路工人及电车工人罢工。柏林城里的运输机关全停。亨堡等处出了乱子。全国的罢工,有扩张形势。

评论我叙签约。我争叙感国的签约。我叙德国签约,单注重其国民精神上所感痛苦的一点。是什么意思?原来这回和约,除却国际同盟,全是对付德国的。德国为日耳曼民族,在历史上早蜚声誉,有一种崛强的。一朝决裂,新剑发硎,几乎要使全地球的人类挡他不住。我们莫将德国穷兵黩武,看作是德皇一个人的发动。德皇乃德国民族的结晶。有德国民族,乃有德皇。德国民族,晚近为尼釆,菲希特,颉德,泡尔生等“向上的”“活动的”哲学所陶铸。声宏实大,待机而发。至于今日,他们还说是没有打败,“非战之弊”。德国的民族,为世界最富于“高”的精神的民族。惟“高”的精神,最能排倒一切困苦,而惟我实现其所谓“高”。我们对于德皇,一面恨他的穷兵黩武,滥用强权。一而仍不免要向他洒一掬同情的热泪,就是为着他“高”的精神的感动。德国的民族,他们败了就止了。象这样的屈辱条件,他们也忍苦承受。他们第一次翻转面目,已从帝国变成了民国。他们的第二次翻转,或竟将民国都不要了。这话我殊敢下一个粗疏的断定。我们且看挡在西方的英法,不是他们的仇敌吗?英法是他们的仇敌,他们的好友,不就是屏障东方和南方的俄、奥、匈、捷和波兰吗?他们不向俄奥匈捷等国连络,还向何处?他们要向俄奥匈捷连络,必要改从和俄奥匈捷相同的制度。俄匈的社会革命成了功,不用说。奥捷也有此趋势,前日电传说捷克已经成了劳兵民国了。德国广义派斯巴达团,去年冬天的猛断举动,和成功仅仅相差一间。爱倍尔政府成立,多数社会党握权,所恃以制服广义派的,全是几个兵,几杆枪。和约成功,兵是要解散了。枪是会要缴出了。那时候政府还恃着什么?德国工商业的大毁败,要重造起来,不得不仰赖出力的劳动者。以后政府所应做的大事,就是向劳动者多多的磕头。而广义派的武器,不是别的,就是这些劳动者。故我从外交方面的趋势去考虑,断定德国必和俄奥匈连合,而变为共产主义的共和国。又从内治方面的趋势去考虑,也可以做同样的断定。

一千九百一十九年以前,世界最高的强权在德国。一千九百一十九年以后,世界最高的强权在法国,英国和美国。德国的强权,为政治的强权,国际的强权。这回大战的结果,是用协约国政治和国际的强权,打倒德奥政治和国际的强权。一千九百一十九年以后,×国英国美国的强权,为社会的强权,经济的强权。一千九百一十九年以后设有战争,就是阶级战争,阶级战争的结果,就是东欧诸国主义的成功。即是社会党人的成功。我们不要轻看了以后的德人。我们不要重看了现在和会高视阔步的伟人先生们,他们不能旰食的日子快要到哩!他们总有一天会要头痛!然则这回的和约“其能稳”尚靠不定。若真以“德俄和约”,“德罗和约”的例来推测,恐咱就是早晚的问题。无知的克勒满沙老头子,还抱着那灰黄色的厚册,以为签了字在上面,就可当作阿尔卑斯山一样的稳固,可怜的很啊!

\subsection{世界杂评}

高兴和沉痛克勒满沙在办公室接得德国接受和约的电话,高兴了不得。起身来,和在办公室的阁员及同僚握手。说,“诸君!我之静候这一分钟,已有十九年了!”这话何等高兴。虽然,不第高兴,又含多少沉痛的意思。一千八百七十一年,威廉第一和俾士马克,高踞凡尔赛,接受法国屈服牒文的时候,何等高兴。结果遂酿成此次的战争。虽然威廉第一,俾士马克,不第高兴,又含有多少沉痛的意思。一千八百年至一千八百一十五年,拿破伦蹂躏德意志,分裂他的国,占据他的地,解散他的兵。普王屈服,称藩纳聘。拿破伦何等高兴。结果遂酿成一千八百七十一年的战争。虽然,拿破伦不第高兴,又含多少沉痛的意思。一千七百八十九年至一干七百九十年,德奥为巨擘的神圣同盟军,深恶德国的民主自由,几度蹂法境,围巴黎,结果遂崛起拿破伦,而有蹂躏德国,令德人头痛的事。我们执因果看历史,高兴和沉痛,常相关系,不可分开。一方的高兴到了极点,热一方的沉痛也必到极点。我们看这番和约所载,和拿破伦对待德同的办法,有什么不同?分裂德国的国,占据德国的地,解散德国的兵,有什么不同?克勒满沙高兴之极,即德国人沉痛之极。包管十年二十年后,你们法国人,又有一番大大的头痛,愿你们记取此言。

卡尔和溥仪奥前皇卡尔避居瑞士,某报通讯记者求见,见其侍者。侍眶说:“皇帝的退位,本非得已,故愿望恢复帝制。惟目下暂时隐居,不问政治。”凡做过皇帝的,没有不再想做皇帝。凡做过官的,没有不再想做官。心理上观念的习惯性,本来如此。西洋人做事,喜欢彻底,历史上处死国王的事颇多。英人之处死沙尔一世(一千六百四十八年),法人之处死路易十六(一千七百九十三年),俄人之处死尼哥拉斯第二(一千九百一十八年),都以为不这样不足以绝祸根。拿破伦被囚于圣赫利拿,今威廉第二拟请他做拿破伦的后身将受协约国的裁判,总算很便宜的。避居瑞士的卡尔,和伏处北京的溥仪,国民不加意防备,早晚还是一个祸根。


\section{湘江评论增刊第一号}
\datesubtitle{(一九一九年七月二十一日)}
\subsection{健学会之成立及进行}

健学会以前的湖南思想界湖南的思想界,二十年以来,黯淡已极。二十年前,谭嗣同等在湖南倡南学会,召集梁启超,麦梦华诸名流,在长沙设时务学堂,发刊湘报,“时务报”。一时风起云涌,颇有登高一呼之概。原其所以,则彼时因几千年的大帝国,屡受打击于列强,怨幅惋悔,敬而奋发。知道徒然长城渤海,挡不住别人的铁骑和无畏兵船。中国的老法,实在有些不够用。变法自强的呼声,一时透彻衡云,云梦的大倡。中国时机的转变,在那时候为一个大枢纽。湖南也跟着转变,在那时候为一个大枢纽。

思想变了,那时候的思想是怎样一种思想?那时候思想的中心是在怎样的一点?此问不可不先答于下——

(一)那时候的思想是自大的思想。什么“讲求西学’,什么“虚心考察”,都不外“学他到手还以奉敬”的方法。人人心目中都存想十年二十年后,便可学到外国的新法。学到新法便可自强。一达到自强的目的,便可和洋鬼子背城借一,或竟打他个片甲不回。正如一个小孩受了隔壁小孩的晦气,夜里偷着取出他的棍棒,打算明早跑出大门,老实的还他一个小礼。什么“西学”“新法”相当于小孩的棍棒罢了。

(二)那时候的思想,是空虚的思想,我们试一取看那时候鼓吹变法的出版物,便可晓得一味的“耗矣哀哉”。激刺他人感情作用。内酌是空空洞洞,很少踏着人生社会的实际说话。那时有一种“办学室”,“办自治”,“请开议会”的风气,寻其根柢,多半凑热闹而已。凑热闹成了风,从思想界便不容易引入实际去研究实事和真理了。

(三)那时候的思想,是一种“中学为体,西学为用”的思想,“中国是一个声名文物之邦,中国的礼教甲于万国,西洋只有格致炮厉害,学来这一点便得。”设若议论稍不如此,便被人看作“心醉欧风者泳”,要受一世人的唾骂了。

(四)那时候的思想是以孔子为中心的思想。那时候于政治上有排满的运动,有要求代议政治的运动。于学卫上有废除科举兴办学校,采取科学的运动。却于孔老爹,仍不敢说出半个“非”字。甚至盛倡其“学问要新,道德要旧”的谬说,“道德要旧”就是“道德要从孔子’的变语。

上面所举,全中国都有此就行,湖南在此情形的中间占一位置。所以思想虽然变化,却非透彻的变化,任何说是笼统的变化,盲目的变化,过渡的变化,从戊戌以致今日湖南的思想界,全为这笼统的,盲目的,过渡的变化所支配。

湖南辨求新学二十余年,而后有崭然的学风。湖南的旧学界,宋学、汉学两支流,二十年前,颇能成为风气。二十年来,风气尚未尽歇,不过书院为学校占去,学生为科学吸去,他们便必淹没在社会的底面了。推原新学之所以没有风气,全在新学不曾有确立的中心思想。中心思想之所以不曾确立,则有以下的数个原因:

1、没有性质纯粹的学会。

2、没有大学。

3、在西洋留学的很少。有亦为着吃饭问题和虚条心理,竟趋于“学非所用’的一途,不能持续研究其专门之学。在东洋留学的,被黄兴吸去做政治运动。

4、政治纷乱,没有研究的宁日。

这是湖南新学界中心思想不能确立的原故,即是没有学风原故,辛亥以来,滥竽教育的,大部市侩一流,逞其一知半解的见解造成非驴非马的局势。中心思想,新学风气,可是更不能谈及了。

近数年来,中国的大势陡转,蔡元培。江亢虎,吴敬恒,刘师复,陈独秀等,首倡革新,革新之说,不止一端,自思想,文学,以致政治,宗教,艺术,皆有一改旧观之概。甚至国家要不要,家庭要不要,婚姻要不要,财产应私有应公有,都成了亟待研究的问题。更加以欧洲的大战,激起了俄国的革命,潮流浸卷,自西向东,国立北京大学的学者首欢迎之,全国各埠各学校的青年大响应之,怒涛澎湃,到了湖南而健学会遂以成立。

\textbf{健学会之成立}

六月五日,省教育会会长陈润霖君邀集省城各学校职教员徐特立,朱剑凡,汤松,蔡湘,钟国陶,杨树达,李云杭,向绍轩,彭国君,方克刚,欧阳鼐,何炳麟,李景桥,赵翌等发起健学会。在楚怡学校开会。今录某报所载陈润霖君报告组织学会的意旨于下一一

兄弟前次到京,偶有感能,深抱乐观。象四年前,北京大学学生以做官为唯一目的。非独大学为然,即大学以外之学生,亦莫不皆然。前次居京,所见迥然不同。大学学生思潮大变,皆知注意人生应为之事,其思潮已多表露于各种杂志月刊中。因之各校学生,亦顿改旧观,发生此次救国大运动。其致此之故,则因蔡孑民先生自为大学校长以来,注入哲学思想,人生观念,使旧思想完全变掉。或该认学生救国运动为政客所勾引,而不知突出学生之自动,及新旧思潮之冲突也。盖自俄国政体改变以后,社会主义渐渐输入于远东。虽派别甚多,而潮流则不可遏抑。即如日本政府,从来对于提倡社会党人,苛待残杀,不遗余力,而近日竟许社会党人活动。如吉野博士等,则主张实行国家社会主义,以和缓过激主义,顺应世界之趋势,从看将日本政体改变为英国式虚君制。

于此可知,世界思潮改变之速,势力之大矣!我国新思潮亦甚发展,终难久事遏抑,国人当及时研究,导之正轨,国人等组织学会,在釆用正确健学之学说而为彻底之研究……

这日开会,听说街有朱剑凡君主张“各除成见,研究世界新思想,服从真理”的演说,向绍轩君主张“采用国家社会主义”的演说。在湖南思想界不可不谓空前的创闻。今录出该会所发表的会则如下一一

(一)本会同志组合,以输入世界新思潮,共同研究,择要传播为宗旨。

(二)本会定名为健学会。

(三)会所暂定长沙储英源楚怡小学校。

(四)入会者须确有研究学术之志愿,经本会会友一人以之介绍,得为本会会员。

(五)关于输入新思潮之方法一一

(1)凡最近出版之图书杂志,由本会随时搜集,以供会员阅览。会员所藏书报,得借给本会会员阅览。其有愿捐入本会者,本会尤为欢迎。

(2)延请海内外同志,随时调查,通信报告。

(3)介绍名人谈话。

(六)关于研究之方法一一

(1)研究范围,大半为哲学,教育学,心理学,论理学,文学,美学,社会学,政治学,经济学诸问题,会友必分认一门研究。

(2)重要之问题,由会友共同研究。

(3)会员有愿习外国语者,由本会会友传授。

(七)关于传播之方法一一

(1)讲演。分定期,临时两种。定期讲演,每周日曜日午前八时至十时。由会友轮流担任。讲员及讲题均于前周日曜日决定。讲友须预备讲稿,交由本会汇刊。临时讲演,凡有主要演题,或由会友,或请名人讲演,另觅地点,择期举行。

(2)出版。

 (八)本会设会计,管理图书各一人。其他会务由会友共同负责。每次开会推会友一人协时主席。 
(九)会友应守之公约如左一一

(1)确守时间.。

(2)富于研究的精神。

(3)学问上之互助。

(4)自由讨论学术.

(5)不尚虚文客气,以诚实为主。

(十)会员年纳两元以上之会金,有能特别筹助经费者,本会极为欢迎。

(十一)本会遇有重要事项,必须讨论时,得于定期讲演后,临时通告全体,举行会议。

会则中的(五)(六)(七)(九)极为重要,(九)之富于研究的精神,所以破除自是自满的成见,立意很好。尚生于研究的精神之后,继之以“批评的”精神。现代学术的发展,大半为各人的独到所创获。最重的是“我”是“个性”,和中国的习惯,非死人不加议论,著述不引入今人的言论恰成一反比例。我们当以一己的心思,居中活动。如日光之普天照耀,如探海灯之向外扫射,不管他到底是不是(以今所是的为是)合人意不合人意。只求合心所安合乎箕理才罢。老先生最不喜欢的是狂妄。岂不知古今最确的原理,伟大的事业邵是系一些被人加上狂妄名号的狂妄人所发明创造来的。我们住在这复杂的社会,诡诈的世界没有批评的精神就容易会做他人的奴隶。其君谓中国人大半是奴隶,这话殊觉不错。(九)之自由讨论学术,很合思想自由,言论自由的原则。人类最可宝贵最堪自乐的一点却在于此,学术的研究最忌演释式的独断态度。中国什么“师严而后道尊”。师说“道统”,“宗派”都是害了独断态度的大病。都是思想界的强权,不可不竭力打破。象我们反对孔子,有很多别的理由,单就这独霸中国,使我们思想界不能自由,郁郁做两千年偶象的奴隶,也是不能不反对的。

健学会之进行健学会进行事项,会则所定大要系研究及传播最新学术。现在注重于研究一面,闻已派人到京沪各处,釆买书籍,新闻纸和杂志。在省城设一英语学习班,使会员学习英语,为直接研究四方学术的预备。有年在四五十的会员都喜欢学习。又设一演讲会由会员轮流发表意见,实行知识的交换。官气十足的先生们,忽然屈尊降贵,虚心研究起来。虽然旁人尚有不满意的处所,以为官气还有十分五、六,讲演要多采用命令式和训话式。更有谓他们是青叶上青虫的体合作用。象这样的求全责备.我们为何以下坏。在这么女性纤纤,暮气重重的湖南有此一举,颇足山幽囚而破烦闷。东方的曙光,空谷的足音,我们正应拍掌欢迎,希望他可作“改造湖南的张本”看他们四次讲演的问题,如“国人误谬的生死观”。怎样做人”,“教育和白话文”,“釆用杜威教育主义”,都可谓能得其要。倘能尽脱习气采用公开讲演,尽人都可以听,则传播之外,得益之大,当有不可计量的了。

\begin{flushright}(《湘江评论》临时增刊第一号)
\end{flushright}


\section{湘江评论第三号}
\datesubtitle{(一九一九年七月二十八日)}

\subsection{世界杂评}

畏德如虎的法兰西法国于德国畏惧他如虎狼。德国这么大败,法国尚畏惧得很。割萨尔煤矿,划来因左岸独立,毁希里哥伦炮台,力波兰独立以蹙其东陲,助捷克独立以阻其南出。日耳曼奥地利也并归德国,则不惜破坏民族自决主义,多方以妨之。殖民地,陆海天空军备,则多方以消之。商船亦须交出大部,以阻其海外贸易之恢复。这样也算够了,还不止此。又向英美两国,请求保卫。前日电传,威尔逊于离法以前,签订一约。系证将来法国一受攻击,美国当起而援助。劳合乔治亦以英国名义,签定一同一性质的条约。此意何等深刻!何等惨淡!籍非法国有不可告之大缺点,何至有这样的畏惧。法兰西民族素负豪气,何至竟象妇人孺子,斤斤乞人保护。我觉得这不是法兰西的好现象!

和约的内容斯末资将军说:“我之签订和约,非因和约乃满意的文件。为结束战争起见,不得不签订他。”又说:“新生活,人类大主义的胜利,人民趋向于新国际制度和优善世界,所抱如此希望之践行,象这样的约言,均没有截上和约。于今只有国民心腔里抒发义侠和人道的新真意,乃能解决和会里政治家困难而止的问题。”又说:“我很以和约里取消黩武主义,仅限于敌国为憾。”斯末资是英国一个武人,是手签和约的一个人,他于签约后所发议论是这样,我们就可想见那和约的内容。

日德密约巴黎的路透电说:“近今外间又有日德密约的谣传。密约是什么东西?还有什么妄人想发现于今后的国际间么?日德密约更是什么东西?日俄密约,为列宁政府所宣布了,不但没成,反丢了脸一大抉。日英法密约成了,我们的山东就要危险。什么日德密约,前年也谣传了多次。据说一千九百一十七年,德国允许日本自由处置荷兰的殖民地,爪哇苏门答腊在内,为英政府听至,告诉了荷兰,阴谋方止。我们应知道日本和德国,是屡次寻求未遂的狗男女,他们虽未遂,那寻奸的念头,是永远不×打断的。日本的强权政府军阀浪人不割除,德国的爱贝尔政府不革命,娼夫和淫妇,还未拆开,危险正多呢。

政治家斯末资云:“惟人民的新真意,乃能解决和会里政治家困难而止的问题。”人民的真意,和政治家的见解,何以这么不相同?政治家何以这么畏难?人民何以这么不畏难?这里面果有一层解释呢?我自来疑惑所谓“政治家”,怕英不是一种好东西?我如今获得了证据。巴黎和约签订后,路易乔治回到英国,在下院演说道:“我们英国,得到许多成功是我们伟大国民团结兴奋的动力。我们于今欢欣鼓舞,但不要存着祸患业已过去的妄念。已使我们获胜的精神,仍要保持,以应付将来事件。”我们不要消费精力于彼此相争。这就是政治家的大本领,这就是政治家的大魔力。不要浪费精力于彼此相争。就是说道,你们人民不要拿着生活痛苦,国民真意,种种无聊问题,和我们政府为难。那些问题都小,都不关痛痒。将来寻着事端,我们还要和别国打仗。爱国,兴奋,团结,对外,是最重要没有的。我正式告诉路易乔治这一类的政治家,你们所说的一大篇,我们都清白是“鬼话”,是“胡说”。我们已经醒了。我们不是从前了。你们且收着,不要再来罢。

\subsection{湘江杂评}

不信科学便死两星期前,长沙城里的大雷,电触死了数人,岳麓山的老树下一个屋子里面,也被雷触死了数人。城里街渠污秽,雷气独多,应建高塔。设避雷针数处。老树电多,不宜在他的下面第屋,这点科学常识,谁也应该晓得,长沙城里的警察,长沙城里三十余万的住民,没一人有闲工夫注意他。有些还说是“五百蛮雷,上天降罚”。死了还不知死因。可怜!

死鼠鼠是瘟疫发生的一个原因,长沙城里到处看见死鼠,张眼望警察,警察却站在死鼠的旁边,早几年的长沙城,都没有这个样子,警察先生们,还是请你们注意点。

\section{民众的大联合(一)}
\datesubtitle{(一九一九年七月二十一日)}



国家坏到了极处,人类苦到了极处,社会黑暗到了极处。补救的方法,改造的方法,教育,兴业、努力、猛进。破坏,建设,固然是不错,有为这样根本的一个方法,就是民众的大联合。

我们竖看历史,历史上的运动不论是那一种,无不是出于一些人的联合。较大的运动,必须有较大的联合。最大的运动,必有最大的联合。凡这种联合,遇有一种改革或一种反抗的时候,最为显著。历来宗教的改革和反抗,学术的改革和反抗,政治的改革和反抗,社会的改革和反抗,两者必都有其大联合,胜负所分,则看他们联合的坚脆,和为这种联合基础主义的新旧或真妄为断。然都要取联合的手段,则相同。

古来各种联合,以强权者的联合,贵族的联合,资本家的联合为主。如外交上各种“同盟”条约,为国际强权者的“联合”。如我国的什么“北洋派”、“西南派”,日本的什么“萨藩”“长藩”,为国内强权者的联合。如各国的政党和议院,为贵族和资本家的联合。(上院至元老院,故为贵族聚集的穴巢,下院因选举法有财产的限制,亦大半为资本家所盘踞)至若什么托辣斯(钢铁托辣斯,煤油托辣斯……)什么会社(日本邮船会社,满铁会社……)则纯然资本家的联合。到了近世,强权者、贵族、资本家的联合到了极点,因之国家也坏到了极点,人类也苦到了极点,社会也黑暗到了极点。于是乎起了改革,起了反抗,于是乎有民众的大联合。

自法兰西以民众的大联合,和王党的大联合相抗,收了“政治改革”的胜利以来,各国随之而起了许多的“政治改革”。自去年俄罗斯以民众的大联合,和贵族的大联合,资本家的大联合相抗,收了“社会的改革”的胜利以来,各国如匈、如奥、如捷,如德,亦随之而起了许多的社会改革。虽其胜利尚未至于完满的程度,要必可以完满,并且可以普及于世界,是想得到的。

民众的大联合,何以这么厉害呢?因为一国的民众,总比一国的贵族资本家及其它强权者要多。贵族资本家及其他强权者人数既少,所赖以维持自己的特殊利益,剥削多数平民的公共利益者,第一是知识,第二是金钱,第三是武力。从前的教育,是贵族资本家的专利,一般平民,绝没有机会去受得。他们既独有知识,于是生出了智愚的阶级。金钱是生活的谋借,本来人人可以取得,但那些有知识的贵族和资本家,整出什么“资本集中”的种种法子,金钱就渐渐流入田主和老板的手中。他们既将土地和机器,房屋,收归他们自己,叫作“不动的财产”。又将叫作“动的财产”的金钱,收入他们的府库(银行),于是替他们作工的千万平民,仅只有一佛朗一辨士的零星给与。做工的既然没有金钱,于是生出了贫富的阶级。贵族资本家有了金钱和知识,他们即便设了军营练兵,设了工厂造枪。借着“外侮”的招牌,使几十师团,几百联队地招募起来。甚者更仿照抽丁的办法,发招牌,明什么“征兵制度”o于是强壮的儿子当了兵,遇着问题就抬出了机关枪,去打他们懦弱的老子。我们目看去年南军在湖南败退时。不就打死了他们自己多少老子吗?贵族和资本家利用这样的妙法,平民就不敢做声,于是生出了强弱的阶级。

可巧他们的三种法子,渐渐替平民偷着学得了多少。他们当作“枕中秘”的教科书,平民也偷着念了一点,便渐渐有了知识。金钱所以出的田地和工厂,平民早已窟宅其中,眼红资本家的舒服,他们也要染一染指。至若军营里的兵士,就是他们的儿子,或是他们的哥哥,或者是他们的丈夫。当拿着机关枪对着他们射击的时候,他们便大声地唤。这一阵唤声,早使他们的枪弹,化成软泥。不觉得携手同归,反一齐化成了抵抗贵族和资本家的健将。我们且看俄罗斯的貌貅十万,忽然将惊旗易成了红旗,就可以晓得这中间有很深的道理了。

平民既已将贵族资本家的三种方法窥破,并窥破他们实行这三种是用联合的手段。又觉悟到他们的人数是那么少,我们的人数是这么多。便大大地联合起来。联合以后,有一派很激烈的,就用“以其人之道,还治其人之身”的办法,同他拚命的捣蛋。这一派的首领,是一个生在德国的,叫作马克思。一派是较为温和的,不想急于见效,先以平民的了解入手。人人要有点互助的道德和自愿的工作。贵族资本家,只要他回心向善能够工作,能够助人而不箐人,也不必杀他;这一派人的意思,更广、更深远,他们要联合地球的一周,联合人类作一家,和乐亲善一一不是日本的亲善一一共臻盛世。这派的首领为一个生于俄国的,斗作克鲁泡特金。

我们要知道世界上的事情,本极易为。有不易为的,便是因子历史的势力一一习惯一一我们倘能齐声一呼,将这个历史的势力冲破,更大大的联合,遇着我们所不以为然的,我们就列起队伍,向对抗的方面大呼。我们已经得了实验。陆荣廷的子弹,永世打不到曹汝霖等一班奸人,我们起而一呼,奸人就要站起身来发抖,就要拚命的飞跑。我们要知道别国的同胞们,是乃常用这种方法,求到他们的利益。我们应该起而仿效,我们应该进行我们的大联合!

\begin{flushright}——原载《湘江评论》第二期\end{flushright}


\section{民众的大联合(二)}
\datesubtitle{(一九一九年七月廿八日)}



以小联合作基础

上一回本报,已说完了“民众的大联合”的可能及必要。今回且说怎样是进行大联合的办法?就是“民众的小联合”。

原来我们想要有一种大联合,以与立在我们对面的强权者害人者相对抗,而求到我们的利益。就不可不有种种做他基础的小联合,我们人类本有联合的天才,就是能群的天才,能够组织社会的天才。群和“社会”就是我所说的“联合”。有大群,有小群,有大社会,有小社会,有小联合,有大联合,是一样的东西换却名称。所以要有群,要有社会,要有联合,是因为想要求到我们的共同利益,共同的利益因为我们的境遇和职业不同,其范围也就有大小的不同。共同利益有大小的不同,于是求到共同利益的方法,(联合)也就有大小的不同。

诸君!我们是农夫。我们就要和我们种田的同类,结成一个联合,以谋我们种田人的种种利益。我们种田人的利益,是要我们种田人自己去求。别人不种田的,他和我们利益不同,决不会帮我们去求。种田的诸君!田主怎样待遇我们?租税是重是轻?我们的房子适不适?肚子饱不饱?田不少吗?村里没有没田作的人吗?这许多问题,我们应该时时去求解答。应该和我们的同类结成一个联合,切切实实彰明较著的去求解答。

诸君!我们是工人。我们要和我们做工的同类结成一个联合,谋我们工人的种种利益。关于我们做工的各种问题,工值的多少?工时的长短?红利的均分与否?娱乐的增进与否?……均不可不求一个解答。不可不和我们的同类结成一个联合,切切实实彰明较著的去求一个解答。

诸君!我们是学生,我们好苦,教我们的先生们,待我们做寇仇,欺我们做奴隶,闲镇我们做囚犯。我们教室的窗子那么矮小光线照不到黑板,使我们成了“近视”,桌子太不合式,坐久了便成“脊柱弯曲症”,先生们只顾要我们多看书,我们看的真多,但我们都不懂,白费了记忆。我们眼睛花了,脑筋昏了,精血亏了,面血灰白的使我们成了“贫血症”’成了“神经衰弱症”。我们何以这么呆板?这么不活泼?这么萎缩?呵!都是先生们迫着我们不许动,不许声的原故。我们便成了“僵死症”。身体上的痛苦还次,诸君!你看我们的实验室呵!那么窄小!那么贫乏--几件坏仪器,使我们试验不得。我们的国文先生那么顽固,满嘴里“诗云”“子曰”,清底却是一字不通。他们不知道现今已到了二十世纪,还迫着我们行“古礼”守“古法”,一大堆古典式死尸式的臭文章,迫着向我们脑子里灌,我们板书室是空的,我们游戏场是秽的。国家要亡了,他们还贴着布告,禁止我们爱国,象这一次救国运动,受到他们的恩赐其多呢!唉!谁使我们的身体,精神,受摧折,不愉快?我们不联合起来,讲究我们的“自教育”,还待何时!我们已经陷在苦海,我们要求讲自救:卢梭所发明的“自教育”,正用得着。我们尽可结合同志,自己研究。咬人的先生们,不要靠他。遇着事情发生一一象这回日本强权者和国内强权者的跋扈一一我们就列起队伍向他们作有力的大呼。

诸君!我们是女子。我们更沉沦在苦海!我们都是人,为什么不许我们参政?我们都是人,为什么不许我交际?我们一窟一窟的聚着,连大门都不能跨出。无耻的男子,无赖的男子,拿着我们做玩具,教我们对他长期卖淫,破坏恋爱自由的恶魔!破坏恋爱神圣的恶糜,整天的对我们围着,什么“贞操’却限于我女子,“烈女嗣”遍天下,“贞童庙’又在那里?我们中有些一窟的聚重在一女子学校,教我们的又是一些无耻无赖的男子,整天说什么“贤妻良母”,无非是教我们长期卖淫专一卖淫。怕我们不受约束,更好好的加以教练,苦!苦!自由之神,你在那里,快救我们!我们于今醒了!我们要进行我们女子的联合!要扫荡一般强奸我们破坏我们精神自由的恶魔!

诸君,我们是小学教师,我们整天的教课,忙的真很!整天的吃粉条屑,没处可以游散舒吐。这么一个大城里的小学教师,总不下几千几百,却没有一个专为我们而设的娱乐场。我们教课,要随时长进学问,却没有一个为我们而设的研究机关。死板板的上课钟点,那么多,并没有余时,没有余力,一一精神来不及!一一去研究学问。于是乎我们变成了留声器,整天演唱的不外昔日先生们教给我们的真传讲义。我们肚子是饿的。月薪十元八元,还要折扣,有些校长先生,更仿照“克减军粮”的办法,将政府发下的钱,上到他们的腰包去了。我们为着没钱,我们便做了有妇的鳏夫。我和我的亲爱的妇人隔过几百里几十里的孤住着,相望着,教育学上讲的小学教师是终身事业,难道便要我们做终身的鳏夫和寡妇?教育学上原说学校应该有教员的家庭住着,才能做学生的模范,于今却是不能。我们为着没钱,便不能买书,便不能游历考察。不要说了!小学教师直是奴隶罢了,我们想要不做奴隶,除非联结我们的同类,成功一个小学教师的联合。

诸君!我们是警察。我们也要结合我们同类,成功一个有益我们身心的联合。日本人说,最苦的是乞丐,小学教员和警察,我们也有点感觉。

诸君!我们是车夫,整天的拉得汗如雨下!车主的赁钱那么多!得到的车费这么少!何能过活,我们也有什么联合的方法么?

上面是农夫、工人、学生、女子、小学教师、车夫、各色人等的一片哀声。他们受苦不过,就想成功于他们利害的各种小联合。

上面所说的小联合,象那工人的联合,还是一个很大很笼统的名目,过细说来,象下列的:

铁路工人的联合,

矿工的联合,

电报司员的联合,

电话司员的联合,

造船业工人的联合,航业工人的联合,

五金业工人的联合,

纺织业工人的联合,

电车夫的联合,

街车夫的联合,

建筑业工人的联合,

方是最下一级联合,西洋各国的工人,都有各行各业的小联合会,如运输工人的联合会,电车工人联合会之类,到处都有,由许多小的联合,进为一个大联合,由许多大的联合,进为一个最大的联合。于是什么“协会”,什么“同盟”,接踵而起,因为共同利益只限于一部分人,故所成立的为小联合,许多的小联合彼此间利益有共同之点,故可以立为大联合,象研究学问是我们学生分内的事,就组成我们研究学问的联合厶象要求解放要求自由,是无论何人都有分的事,就应联合各种各色的人,组成一个大联合。

所以大联合必要从小联合入手,我们应当起而仿效别国的同胞们,我们应该多多进行我们的小联合。

\begin{flushright}《湘江评论》第三期一九一九年七月二八日出版\end{flushright}


\section{民众的大联合(三)}
\datesubtitle{(一九一九年八月四日)}



中华“民众的大联合”的形势

上两回的本报己说完了(一)民众大联合的可能及必要,(二)民众的大联合,以民众的小联合为始基,于今进说吾国民众的大联合我们到底有此觉悟么?有此动机么?有此能力么?可得成功么?

(一)我们对于吾国“民众的大联合”到底有此觉悟么?辛亥革命,似乎是一种民众的联合,其实不然.辛亥革命乃留学生的发纵指示.哥老会的摇旗唤吶,新军和巡防营一些丘八的张弩拔剑所造成的,与我们民众的大多数毫无关系。我们虽赞成他们的主义,却不曾活动。他们也用不着我们活动。然而我们却有一层觉悟。如道圣文神武的皇帝,也是可以倒去的。大逆不道的民主,也是可以建设的。我们有话要说,有事要做,是无论何时可直说可以做的。辛亥而后,到了丙辰,我们又打倒了一次洪宪皇帝,原也是可以打得倒的,及到近年,发生南北战争,和世界战争,可就更不同了,南北战争结果,官僚、武人、政客,是害我们,毒我们,剥削我们,越发得了铁证。世界战争的结果,各国的民众,为着生活痛苦问题,突然起了许多活动。俄罗斯打倒贵族,驱逐富人,劳农两界合立了委办政府,红旗军东驰西突,扫荡了多少敌人,协约国为之改容,全世界为之震动。匈牙利崛起,布达佩斯又出现了崭新的劳农政府。德人奥人捷克人和之,出死力以与其国内的敌党搏战。怒涛西迈,转而东行,英法意美既演了多少的大罢工,印度朝鲜又起了若干的大革命,异军特起,更有中华长城渤海之间,发生了“五四”运动。旌旗南向,过黄河而到长江、黄浦汉皋,屡演话剧,洞庭闽水,更起高潮。天地为之昭苏,奸邪为之辟易。咳!我们知道了!我们醒觉了!天下者我们的天下。国家者我们的国家。社会者我们的社会。我们不说,谁说?我们不干,谁干?刻不容缓的万众大联合,我们应该积极进行!

(二)吾国民众的大联合业已有此动机么?此间我直答之日“有”。诸君不信,听我道来一一

溯源吾国民众的联合,应推清末谘议局的设立,和革命党一一同盟会一一的组成。有谘议局乃有各省谘议局联盟请愿早开国会一举。有革命者乃有号召海内外起兵排满的一举。辛亥革命,及革命党和谐议局合演的一出“痛饮黄龙”。其后革命党化成了国民党,谘议局化成了进步党,是为吾中国民族有政党之始。自此以后,民国建立,中央召集了国会,各省亦召集省议会,此时各省更成立三种团体,一为省教育会,一为省商会,一为省农会(有数省有省工会。数省则合于农会,象湖南)。同时各县也设立县教育会,县商会,县农会(有些县无)。此为很固定很有力的一种团结。其余各方面依其情势地位而组设的各种团体,象

各学校里的校友会,

族居外埠的同乡会,

在外国的留学生总会,分会,

上海日报公会,

环球中国学生会,

北京及上海欧美同学会,

北京华法教育会。

各种学会(象强学会,广学会,南学会,尚志学会,中华职业教育社,中华科学社,亚洲文明协会……),各种同业会(工商界各行各业,象银行公会,米业公会……各学校里的研究会,象北京大学的画法研究会,哲学研究会……有几十种),各种俱乐部……

都是近来因政治开放,思想开放的产物,独夫政治时代所决不准有不能有的,上列各种,都是单纯,相当于上回本报所说的“小联合”。最近因政治的纷乱,外患的压迫,更加增了觉悟,于是竟有了大联合的动机。象什么

全国教育会联合会,

广州的七十二行公会,上海的五十公团联合会,

商学工报联合会,

全国报界联合会,

全国和平期成会,

全国和平联合会,

北京中法协会,

国民外交协会,

湖南善后协会(在上海),

山东协会(在上海),

北京上海及各省各埠的学生联合会,

各界联合会,全国学生联合会……

都是,各种的会,社,部,协会,联合会,固然不免有许多非民众的“绅士”“政客”在里面(象国会,省议会,省教育会,省农会,全国和平期成会,全国和平联合会等,乃完全的绅士会,或政客会),然而各行各业的公会,各种学会,研究会等,则纯粹平民及学者的会集。至最近产生的学生联合会,各界联合会等,则更纯然为对付国内外强权者而起的一种民众大联合,我以为中华民族的大联合的动机,实伏于此。

(三)我们对于进行吾国“民众大联合”果有此能力吗?果可得成功么?谈到能力,可就要发生疑问了。原来我国人口只知道各营最大合算最没有出息的私利,做商的不知设立公司,做工的不知设立工党,做学问的只知闭门造车的老办法,不知共同的研究。大规模有组织的事业,我国人简直不能过问,政治的办不好,不消说,邮政和盐务有点成绩,就是依靠了洋人。海禁开了这久,还没一头走欧洲的小船,全国唯一的“招商局”和“汉冶萍’,还是每年亏本,亏本不了,就招入外股。凡是被外人管理的铁路,清洁,设备,用人都要好些。铁路一被交通部管理,便要糟糕。坐京汉,津浦,武长,过身的人,没有不嗤着鼻子咬着牙齿的!其余象学校搞不好,自治办不好,乃至一个家庭也办不好,一个身子也办不好。“一丘之貉”“千篇一律”的是如此,好容易谈到民众的大联合?好容易和根深蒂固的强权者相抗?

虽然如此,却不是我们根本的没能力,我们没能力,有其原因,就是“我们没练习”。

原来中华民族,几万万人从几千年来,都是干着奴隶的生活,没有一个非奴隶的是“皇帝”(或曰皇帝也是“天”的奴隶,皇帝当家的时候,是不准我们练习能力的)。政治,学术,社会,等等,都是不准我们有思想,有组织.有练习的。

于今却不同了,种种方面都要解放了,思想的解放,政治的解放,经济的解放,男女的解放,教育的解放,都要从九重冤狱,求见青天。我们中华民族原有伟大的能力!压迫逾深,反动愈大,蓄之既久,其发必远,我敢说一句怪话,他日中华民族的改革,将较任何民族为彻底,中华民族的社会,将较任何民族为光明。中华民族的大联合,将被任何地域任何民族而先告成功。诸君!诸君!我们总要努力!我们总要拚命向前!我们黄金的世界,光荣灿烂的世界,就在面前!

\begin{flushright}《湘江评论》第四期1919.8.4出版\end{flushright}


\section{问题研究会章程}
\datesubtitle{(一九一九年十月廿三日)}



第一条、凡事或理之为现代人生所必需,或不必需,而均尚未得适当之解决,致影响于现代人生之进步者,成为问题。同人今设一会,注重解决如斯之问题,先从研究入手,定名问题研究会。

第二条、下列各种问题,及其他认为有研究价值续行加入之问题为本会研究之问题。

(一)教育问题一一

1。教育普及问题(强迫教育问题)2.中等教育问题3.专门教育问题4.大学教育问题5.社会教育问题6.口语教科书编纂问题7.中等学校国文科教授问题8。不惩罚问题9。废止考试问题30.各级教授法改良问题7.小学教师知识健康及薪金问题12.公共体育场建设问题 13.公共娱乐场建设问题14.公共图书馆建设问题 15.学制改订问题16.大派留学生问题 17.杜威教育说如何实施问题 
(二)女子问题

1.女子参政问题2.女子教育问题3.女子职业问题4.女子交际问题5.贞操问题6.恋爱自由及恋爱神圣问题7.男女同校问题8.女子修饰问题9。家庭教育问题10.姑媳同居问题 11.废娼问题 12。废妾问题13。放足问题14.公共育儿院设置问题15.公共蒙养院设置问题 16.私生儿待遇问题 17.避妊问题 
(三)国话问题(一白话文问题)

(四)孔子问题

(五)东西文明会合问题

(六)婚姻制度改良及婚姻制度应否废弃问题

(七)家族制度改良及家族制度应否废弃问题

(八)国家制度改良及国家制度应否废弃问题

(九)宗教改良及宗教应否废弃问题。

(十)劳动问题一一

1.劳动时间问题2.劳工教育问题3.劳工住屋及娱乐问题4.劳动失职处置问题5.工值问题6.小儿劳作问题7.男女工值平等问题8.劳工组合问题9.国际劳动同盟问题10。劳农干政问题 11.强制劳动问题 12.余剩均分问题 13.生产机关公有问题  14.工人退职年金问题15.遗产归公问题(附)
(十一)民族自决问题

(十二)经济自由问题

(十三)海洋自由问题

(十四)军备限制问题

(十五)国际联盟问题

(十六)自由移民问题

(十七)人种平等问题

(十八)社会主义能否实施问题

(十九)民众的联合如何进行问题

(二十)动工俭学主义如何普及问题

(二一)俄国问题

(二二)德国问题

(二三)奥国问题

(二四)印度自治问题

(二五)爱尔兰独立问题

(二六)土耳其分裂问题

(二七)埃及骚乱问题

(二八)处置德皇问题

(二九)重建比利时问题

(三十)重建东部法国问题

(三一)德殖民地处置问题

(三二)港湾公有问题

(三三)飞渡大西洋问题

(三四)飞渡太平洋问题

(三五)飞渡天山问题

(三六)白令英吉利直布罗陀三峡凿遂通车问题一

(三七)西伯利亚问题

(三八)菲律宾问题

(三九)日本粮食问题

(四十)日本问题

(四一)朝鲜问题

(四二)山东问题

(四三)湖南问题

(四四)废督问题

(四五)裁兵问题

(四六)国防军问题

(四七)新旧国会问题

(四八)铁路统一问题(撤消势力范围问题)

(四九)满州问题

(五十)蒙古问题

(五一)西藏问题

(五二)退回庚子赔款问题

(五三)华工问题一一

1.华工教育问题2.华工储蓄问题3.华工归国后安置问题

(五四)地方自治问题

(五五)中央地方集权分权问题

(五六)两院制一院制问题

(五七)普通选举问题

(五八)大总统权限问题

(五九)文法官考试问题

(六十)澄清贿赂问题

(六一)合议制的内阁问题

(六二)实业问题一一

1.蚕丝改良问题2。茶产改良问题3.种棉改良问题4.造林问题

5.开矿问题6.纱厂及布厂多设问题7.海外贸易经营问题8.国民工厂设立问题

(六三)交通问题一一

1.钦路改良问题2.铁路大借外款厂行添第问题3。无线电台建设问题

4.海陆电线添设问题5.航业扩张问题6。商埠马路建筑问题7。乡村汽车路建筑问题

(六四)财政问题…

1.外债偿还问题2.外债添借问题3.内债偿还及加募问题4.裁厘加税问题5。盐务整顿问题6。京省财权划分问题 7.税制整顿问题 8.清丈田亩问题9。田赋均一及加征问题

(六五)经济问题一一

1.币制本位问题2.中央银行确立问题3.收还纸币问题4.国民银行设立问题5。国民储蓄问题

(六六)司法独立问题

(六七)领事裁判权取消问题

(六八)商市公园建设问题

(六九)模范村问题

(七十)西南自治问题

(七一)联邦制应否施行问题

第三条、问题之研究须以学理为根据。因此在各种问题研究之先,须为各种主义之研究,下列各种主义为特须注重研究之主义。一一

(一)哲学上之主义

(二)伦理上之主义

(三)教育上之主义

(四)宗教上之主义

(五)文学上之主义

(六)美学上之主义

(七)政治上之主义

(八)经济上之主义

(九)法律上之主义

(十)科学上之规律

第四条、问题不论发生之大小,只须含有较广之普遍性,即可提出研究,如日本问题之类。

第五条、问题之研究,有须实地调查者,须实地调查之,如华工问题之类,无须实地调查,及一时不能实地调查者,则从书册杂志、新闻纸三项着手研究。如孔子问题,及三海峡凿隧通车问题之类。

第六条、问题之研究,注重有关系于现代人生者之问题。在古代与现代及未来毫无关系者,则不注意。

第七条、问题研究之方式分为三种。一一

(一)一人独自之研究

(二)二人以上开研究会之研究

(三)二人以上不在一地用通函之研究

第八条、问题研究会,只限于“以学理解决问题”会以外。然在未来而可以预测之问题,亦注意,以“实行解决问题”属于问题研究会以外。

第九条、不论何人有心研究一个以上之问题,而愿与问题研究会生交涉者,即为问题研究会会员。

第十条、会与会员间会员与会员间,只限于“问题研究”之一点,有关此外之关系,属于问题研究会以外。

第十一条、问题研究会,设书记一人,办理会中事务。

第十二条、问题研究会,于中华民国八年西历一千九百十九年九月一日成立。问题研究会章程,即于是日订定,且发布。

\begin{flushright}(抄自《北大日刊》一九一九年十月二十三日第二版)\end{flushright}


\section{学生之工作}
\datesubtitle{(一九一九年十二月廿八日)

(1919年12月28日湖南教育月刊)}



我数年来梦想新社会生活,而没有办法。七年春季,想邀数朋友在省城对岸岳麓山设工读同志会,从事半工半读,因他们多不能久在湖南,我亦有北京之游,事无他议。今春回湘,再发此这种想象,乃有在岳麓山建设新村的计议,而先从办一实行社会说本位教育说的学校入手,此新村以新家庭学校及旁的新社会连成一块为根本理想。对于学校的办法,曾草一计划,今抄上计划书中“学生之工作”一章于此,以求同志的敦诲。我觉得在岳麓山建设新村,似可成为一问题,倘有同志对于此问题有详细规划,或有何种实际的进行,实在欢迎希望的很。

(一)

学校之教授之时间,宜力求减少,使学生多自动研究及工作。应划分每日之时间为六分。其分配如左:

睡眠二分。

游息一分。

读书二分。

工作一分。

读书二分之中,自习一分,教授占一分。此时间实数分配,即

睡眠八小时

游息四小时

自习四小时

教授四小时

工作四小时

上例之工作四小时,乃实行工读主义所必具之一个要素。

(二)工作之事项,全然农村的。列举如左:(甲)种园。(1)花木(2)蔬菜

(乙)种田。(1)棉(2)稻及他种

(丙)种林。

(丁)畜牧。

(戊)种桑。

(己)鸡鱼。

(三)

工作须为生产的,与实际生活的。现时各学校之手工,其功用在练习手眼敏活,陶冶心思精细,启发守秩序之心,及审美之情,如为手工课之优点。然多非生产的(如纸、豆,泥、石膏各细工),作成之物,可玩而不可用。又非实际生活的,学生在学校所习,与社会之实际不相一致,结果则学生不熟谙社会内情,社会亦嫌恶学生。

在吾国现时,又有一弊,即学生毕业之后,多骛都市而不乐田园。农村的生活,非其所习,从而不为所乐。(不乐农村生活,尚有其他原因,今不具论)。此讫地方自治之举行有关系。学生多散布于农村之中,则或为发议之人,或为执行之人,即地方自治得学生之中坚而得举行。农村无学生,则地方自治缺乏中坚之人,有不能美满推行之患。又于政治亦有关系,现代政治,为代议政治,而代议政治之基础第于选举之上。民国成立以来,两次选举,殊非真正民意。而地方初选,劣绅恶棍,无选举投票乡民之多数,竟不知选举是甚么一回事,尤无民意可言。推其原因,则在缺乏有政治常识之人参与之故。有学生指导监督,则放弃选举权一事,可逐渐减少矣。

欲除去上所说之弊,(非生产的非实际生活的,骛于都市而不乐农村。)第一,须有一种经济的工作,可使之直接生产,其能力之使用,不论大小多寡,曾有成效可观。第二,此种工作之成品,必为现今社会普遍之需要。第三,此种工作之场所,必在农村之中,此种之工作,:逸为农村之工作。

上述之第一,所以使之直接生产。第二,所以使之合于实际生活。第三,所以养成乐于农村生活之习惯。

(四)

讫上文所举之外,尚有一要项,今述之于下。言世界改良进步者,皆知须自教育普及使人民咸有知识始。欲教育普及,又自兴办学校始。其言固为不错,然兴办学校,不过施行教育之一端。而教育之全体,不仅学校而止。其一端只。有家庭,一端则有社会。……

(第四部分的摘记:

学校家庭和社会的关系。

而教育之全体,……

学校与家庭的矛盾,斗争有两前提,故言改良学校教育,而不同时改良家庭与社会,所谓举中而遗其上下,得其一而失其二也。)

(五)

第二节所举田园树畜各项,普旧日农圃所为,不为新生活。以新精神经营,真则为新生活矣。旧日读书人不予农圃事,今一边读书,一边工作,以神圣视工作焉,则为新生活矣。号称士大夫有知识一流,多营逐于市场与官场,而农村新鲜之空气不之吸,优美之景色不之尝,吾人改而吸尝此新鲜之空气与优美之景色,则为新生活矣。

种园有,一种花木,为花园,一种蔬菜,为菜园,二者相当于今人所称之学校园。再扩充之,则为植物园。种田以棉与稻为主,大小麦、高梁、蜀黍等亦可间种,粗工学生所难为者,雇工助之。

种林须得山地,学生一朝手植,虽出校而仍留所造之材,可增其回念旧游爱重母校之心。

畜牧如牛.羊、猪等,在可能营养之范围内,皆可分别营养。

育蚕须先种桑,桑成饲蚕,男女生皆可为。养鸡鱼,亦生产之一项,学生所喜为者也。

(六)

各项工作,非欲一人做遍,乃使众人分工,一人只做一项,或一项以上。

学生认学校如其家庭,认所作田园林木等如其私物。由学生各个所有私物之联合,为一公共团体,此团体可名之曰“工读同志会”。会设生产、消费、储蓄诸部,学生出学校,在某期间内不取出会中所存的利益,在某期间外,可取去其利益之一部而留存其一部,用此方法,可使学生长久与学校有关系。

(七)

依第三节所述,现时各学校之手工种,为不生产的,所施之能力掷诸虚牝,是谓“能力不经济”。手工科以外,又有体操科亦然,各种之体操大抵智属于能力不经济类,今有各项工作,此两科目可废弃之。两科目之利,各项工作之中,亦可获得。


\section{新民学会会员通信集第一集}
\datesubtitle{新民学会致各会发的信}



各会友均鉴:

本会出版物,分“会员通信集”与“会务报告”之二,除会务报告叙述公务状况年出二册外,会员通信集,为会员发抒所见相与杨榷讨论的场所。凡会友与会友间往来信稿,不论新旧长短,凡是可以公开的,均望将原稿和腾正稿寄来本会,以便采登第四期以后的通信集。不登之稿不退还。已登之稿声明要退还的也可以退还。稿寄长沙潮宗街文化书社毛泽东君为荷

\begin{flushright}新民学会启\end{flushright}
\subsection{发刊的意思及条例}
第一集所收多前一、二年旧信,然于学会颇关重要,因多属于团体事业之进行与发展的。留法活动一事此集只能载蔡君给各会友的信,各会友给蔡君的信,其重要者尚望蔡君付来选印,通信集之发刊,所以联聚同人精神,商榷修学,立身,与改造世界诸方法。发刊不定期,大约每两月可有一本。同人个人人格及会务固宜取绝对公开态度,俱不宜标榜,故通信集以会友人得一本为主,此外多印了几十本,以便会外同志之爱看者取去。因学会极穷,不论会友非会友,都要纳一点印刷费,集内凡关讨论问题的信,每集出后,总望各会友对之再有批评及讨论,使通信集成为一个会友的论坛,一集比一集丰富,深刻,进步,就好极了。
\subsection{给陶毅的信}
※“联军”

※“同志的分配”

※“自修大学”

※“女子留俄勤工勤学”
<p><font face="宋体">斯咏先生:

(上略)

我觉得我们要结合一个高尚纯粹勇猛精进的同志团体。我们同志,在准备时代,都要存一个“向外发展”的志。我于这个问题,颇有好些感想。我觉得好多人讲改造,却只是空泛的一个目标。究竟要改造到那一步田地(即终极目的)?用什么方法达到?自己或同志从那一个地方下手?这些问题,有详细研究的却很少。在一个人,或者还有;团体的,共同的,那就少了。个人虽有一种计划,象“我要怎样研究”,“怎样准备”“怎样破坏”,“怎样建设”,然多有陷于错误的。错误之故,因为系成立于一个人的冥想,这样的冥想,一个人虽觉得好,然拿到社会上,多行不通。这是一个弊病。一个人所想的办法,尽管好,然知道的限于一个人,研究准备进行的限于一个人。这种现象,是“人自为战”,是,“浪战”,是“用力而多成功少”,是“最不经济”。要治这种弊,有一个法子,就是“共同的讨论”。共同的讨论有两点:一、讨论共同的目的;二、讨论共同的方法。目的同方法讨论好了,再讨论方法怎样实践,要这样的共同讨论,将来才有共同的研究(此指学问),共同的准备,共同的破坏和共同的建设。要这样才有具体的效果可观。“浪战”,是招致失败的,是最没有效果的。共同讨论,共同进行是“联军”,是“同盟军”,是可以橾战胜攻取的左券的。我们非得力戒浪战不可,我们非得组织联军共同作战不可。

上述之问题,是一个大问题。至今尚有一个问题,也很重大,就是“留学或作事的分配”。我们想要达到一种目的(改造),非讲究适当的方法不可,这方法中间,有一种是人怎样分配。原来在现在这样“才难”的时候,人才最要讲究经济。不然,便重迭了,推积了,废置了。有几位在巴黎同志,发狠的招人到巴黎去。要扯一般人到巴黎去是好事;多扯同志去,不免错了一些。我们同志,应该散于世界各处去考察,天涯海角都要去人,不应该堆积在一处。最好是一个人或几个人担任开辟一个方面。各方面的“阵”,都要打开。各方面都应该去打先锋的人。

我们几十个人,结识的很晚,结识以来,为期又浅(新民学会是七年四月发生的),未能将这些问题,彻底研究(或并未曾研究)。即我,历来狠懵懂狠不成材,也很少研究。这一次出游,观察多方而情形,会晤得一些人,思索得一些事,觉得这几种问题,很有研究的价值。外边各处的人,好多也和我一样未曾研究。一样的睡在鼓里,很是可叹!你是很明达很有运志的人,不知对我所陈述的这一层话,有甚么感想?我料得或者比我先见到了好久了。

以上的话还空,我们可再实际一些讲:

新民学会会友,或旭旦学会会友,应该常时开谈话会,讨论吾济共目的目的,及达到目的之方法,一会友的留学及做事,应该受一种合宜的分配,担当一部分的责任,为有意识的有组织的活动。在目的地方面,宜有一种预计,怎样在彼地别开新局面?怎样可以引来或取得新同志?怎样可以创造自己的新生命?你是如此,魏国劳诸君也是如此,其他在长沙的同志及已出外的同志也应该如此,我自己将来,也很想照办。

以上所写是一些大意,以下再胡乱写些琐碎。

会友张田基君安顿赴南洋,我很赞成他去。在上海的肖子璋君等十余人准备赴法,也很好!彭璜君等数人在上海组织工读互助团,也是一件好事。彭璜君和我,都不想往法。安顿往俄。何叔衡想留法,我劝他不必留法,不如留俄。我一己的计划,一星期外将赴上海。湘事平了,回长沙。想和同志成一“自由研究社”(或经名自修大学),预计一年或二年,必将古今中外学术的大纲,弄个清楚,好作出洋考察的工具(不然,不能考察)。然后组一留俄队,赴俄勤工俭学。至于女子赴俄,并无障碍,逆料俄罗斯的女同志,必会特别欢迎。“女子留俄勤工俭学会”,继“女子留法勤工俭学会”而起,也并不是不可能的事。这庄事(留俄),我正和李大钊君等为商量。所说上海复旦教授汤寿军君(即前商专校长)也有意去。我为这件事,脑子里装满引俞快和希望,所以我特地告诉你!好象你曾说过杨润余君入了我们的学会,近日翻阅旧的大公报,见他的著作,真好!不杨君近日作何重活?有暇可以告诉我吗?今日的女子工读团,稻田新来了四人,该团建前共八人,湖南占六人,其属一韩人一苏人,觉得很有趣味!但将来的成绩怎样?还要看他们的能力和道德力如何,也许终究失败(男子组大概可以说已经失败了)。北京女高师,学生方面很有自动的活泼的精神,教职方面不免黑暗。接李一纯君函,说将在周南教一课,不知已来了否?再谈。

 毛泽东
<blockquote>
九月二日在北京
</blockquote>
\subsection{给周世钊的信}
※国内研究出国围研究的先后问题

※团体事业准备工夫

※自修大学
<p><font face="宋体">谆元吾兄:

接张君文亮的信,惊悉兄的母亲病故,这是人生一个痛苦之关。象吾等长日在外未能略尽奉养之力的人,尤其发生“欲报之德吴罔极”之痛,这一点我和你的境遇,算是一个样的!


早前承你寄我一个长信。很对不住,我没有看完,便失掉了,但你信的大意,已大体明白。我想你现时在家,必正纲缪将来进行的计划,我很希望我的计划和你的计划能够完全一致,因此你我的行动已能够一致。我现在觉得你是一个真能爱我,又真能于我有益的人,倘然你我的计划和行动能够一致,那便是很好的了。

我现在极愿将我的感想和你讨论,随便将他写在下面,有些也许是从前和你谈过来的。

我觉得求学实在没有“必要在什么地方”的理,“出洋”两字,在好些人只是一种“迷”。中国出过洋的总不下几万乃至几十万,好的实在很少,多数呢?仍旧是“糊涂’,仍旧是“莫名其妙”,这便是一个具体的证据。我曾以此问过胡适之和黎邵西两位,他们都以我的意见为然,胡适之并且作过一篇“非留学篇”。

因此,我想暂不出国去,暂时在国内研究各种学问的纲要。我觉得暂时在国内研究,有下列几种好处:

1、看译本比较原本快迅得多,可于较短的时间求得较多的知识。

3、世界文明分东西两流,东方文明在世界文明内,要占个半壁的地位。然东方文明可以说就是中国文明。吾人拟从先研究过吾国古今学说制度的大要,再到西洋留学才有可资比较的东西。

3、吾人如果要在现今的世界稍微尽一点力,当然脱不开“中国”这个地盘。关于这地盘内的情形,拟不可不加以实地的调查及研究。这层工夫,如果留在在出洋回来的时候做,因人事及生活的关系,恐怕有些困难。不如在现在做了,一来无方才所说的困难,二来又可携带些经验到西洋去,考察时可以借资比致。

老实说,现在我于种种主义,种种学说,都还没有得到一个比较明了的概念,想从译本及时贤所作的报章杂志,将中外古今的学说刺取精华,使他们各构成一个明了的概念。有工夫能将所刺取的编成一本书,更好。所以我对于上列三条的第一条,认为更属紧要。

以上是就“个人”的方面和“知”的方面说。以下再就“团体”的方面和“行”的方面说:

我们是脱不了社会的生活的,都是预备将来要稍微有所作为的。那么,我们现在便应该和同志的人合力来做一点准备工夫。我看这一层好些人不大注意,我则以为很是一个问题,不但是随便无意的放任的去准备,实在要有意的有组织的头准备,必如此才算经济,才能于较短的时间(人生百年)发生较大的效果。我想(一)结合同志,(二)在经济的可能的范围内成立为他日所必要的基础事业,我觉得这两样是我们现在十分要注意的。

上述二层(个人的方面和团体的方面),应以第一为主,第二为辅。第一应占时间的大部分;第二占一小部分。总时间定三年(至多),地点长沙。

因此我于你所说的巴黎南洋北京各节,都不赞成,而大大赞成你“在长沙”的那个主张。

我想我们在长沙要创造一种新的生活,可以邀合同志,租一所房子,办一个自修大学(这个名字是×××先生造的)。我们在这个大学里实行共产的生活,关于生活费用取得的方法,约可定为下列几种:

1、教课。(每人每周六小时及至十小时)

2、投稿。(论文稿或新闻稿)

3、编书。(编一种或数种可以卖稿的书)

4、劳动的工作,(此项以不消费为主,如自炊自濯等)

所得收入,完全公共,多得的人补助少得的人,以够消费为止。我想我们两个如果决行,何叔衡和邵泮清或者也会加入。这种组织,也可以叫做“工读互相团”。这组织里最要紧的是要成立一个“学术谈话会”,每周至少要为学术的谈话两头或三次。

以上是说暂不出洋在国内研究的话。但我不是绝对反对留学的人,而且是一个主张大留学政策的人,我觉得我们一些人都要过一回“出洋”的瘾才对。我觉得俄国是世界第一个文明国,我想两三年后,我们要组织一个游俄队。这是后话,暂时尚可不提及他。

出杂志一项,我觉得很不容易。如果自修大学成了,自修有了成绩,可以看情形出一本杂志(此间的人,多以恢复湘江评论为言)其余会务进行,留待面谈,暂不多说,有暇请简幅一信。

 弟泽东

 一九二○年三月十四日

 北京北长街九十九号
\subsection{给罗学瓒的信}<p><font face="宋体">荣熙兄:

兄此信我自接到,先后看了多次。今天再看一次,尤有感动。你的话我没有不以为然的。我已经决定了一种求学的方法,暂时也不必说,只是你的话我一定要行就是。你奋勉的志气很可敬。你现处环境很好,可以从事周将的观察和深湛的思考。听说你已离学校在工厂做工,西洋工厂里的情况,可由此明了,并且可以得到托尔斯泰所谓“由劳动{得来的生活是真快乐”。我现在很想作工,在上海,李声(水鲜)辩君劝我入工厂,我颇心动,我现在颇感觉专门用口用脑的生活是苦极了的生活,我想总要有一个时期专用体力去做工就好。李君声鲜以一师范学生在江南造船厂打铁,居然一两个月后,打铁的工作样样如意。由没有工钱以渐得到每月工钱十二元。他现寓上海法界渔阳里二号,帮助陈仲甫先生等组织机器工会,你可以和他通信。启民在太安里周南女校。惇元在理问街道俗报。湘潭教育腐败已极,旅省诸人组织“湘潭教育促进会”从事促进,尚无大效。一师湘潭学发会亦将有所兴作,兄信尚未转去,稍迟当转去。兄沿途寄稿均登湘南日报,无转阅必要了。此信曾经复了一次,今再附识近来感想于此。

 弟泽东

 九年十一月廿六日


\section{新民学会会员通信集第二集}
\datesubtitle{新民学会致各会友的信}



各会友均鉴:

本会出版物。分“会员通信集”与“会务报告”之二,除会务报告叙述会务状况年出二册外,会员通信集,为会员发抒所见相与杨权讨论的场所。凡会友与会友间往来信稿,不论新、旧,长、短,凡是可以公开的,均望将原稿或誊正稿寄来本会,舅便釆登第四期以后的通信集,不登之稿可退还。已登之稿声明要退还的也可退还。稿寄长沙文书社毛泽东君为荷。

\begin{flushright}新民学会启\end{flushright}
\subsection{序}

这是新民学会会员通信集“第二集”,釆集会员通信计二十八封,重要者如下:向警予讨论女子发展的计划一封;讨论女子留法勤工俭学问题及女同志联系问题一封;欧阳洋讨论新民学会共同的精神一封;毛泽东主张新民学会应取潜在的态度一封;肖子障报告男女公友在法生活状况一封;易礼容主张吾人进行要有准备一封;罗璈阶希望学会反抗靡俗估量最高价值一封;毛泽东讨论湖南自治并主张学会同人应为主义的结会一封;张国基讨论会务进行一封;报告湘人在南泽情形一封;毛泽东讨论湘人往南洋应取的态度希望会友往南洋为文化动动和南洋建国运动一封;张国基述南洋的奇闻一封;罗学瓒希望会友注意锻炼身外祛除四种迷惑解决家庭问题一封;毛泽东主张组织拒婚同盟一封。以上各信,或于身心之修养有益,或学术之讨论问题之研究有益,或于会务之进行有益,并且都是在能引起会员团体生活的兴味的。其余的信,或商量进止,或讨论事宜,或报告个人状况,具载于此,以见一班。

\begin{flushright}一千九百二十年一月二十号编者\end{flushright}
\subsection{给向警予的信}

※湖南问题

来信久到,未能即复,幸谅!湘事去冬在沪,姐首慷慨论之。一年以来,弟和萌柏等也曾间接为力,但无大效者,教育未行,民智未启,多数之湘人,犹在睡梦,号称有知识之人,又绝无理论计划。弟和萌柏主张湖南自主为国,务与不进化之北方各省及情势不同之南方各省离异,打破空洞无组织的大中国,直接与世界有觉悟之民族携手,而知者绝少。自治问题发生,空气至为黯谈自由湖南革命政府召集湖南人民宪法会议制定湖南宪法以建设新湖南之说出,声势稍振,而多数人莫明其妙,甚或大惊小怪,诧为奇离。湖南人脑筋不清晰,无理想,无远计,几个月来,已看透了。政治界暮气已深,腐败已甚,政治改良一涂,可谓绝无希望。君人惟有不理一切,另辟道路,另造环境一法。教育系我职业,顿湘两年,世已决计。惟办事则不能求学,于自身牺牲太大耳。湘省女子教育绝少进步(男子教育亦然),希望你能引大批女同志出外,多引一人,即多救一人。此颂

进步!

健豪伯母及戍熙姐同此问好

 弟泽东

 九年十一月二十五日
\subsection{给欧阳泽的信}

※学会应取潜在的态度
<p><font face="宋体">玉先生:

共同的精神四项,弟样样赞成。会员加入不限省界,也是极端赞成的。岂但省界,国界也不要限。弟在京所以有那么一说,是因为新民学会现在尚没有深固的基础,在这个时候,宜注意于固有同志之联络,以道义为中心,互相劝勉谅解,使人人如亲生的兄弟姐妹一样。然后进而联络全中国的同志,进而联络全世界的同志,以共谋解决人类各种问题。弟意凡事不可不注重基础,弟见好些团体,象没有经验的商店,货还没有办好,招牌早已高挂了,广告早四出了,结果离不开失败,离不开一个“倒”。半松园会议,都主张本会进行应取“潜在的态度”,弟是十二分赞成,兄也是赞成的一个,长沙同人,亦同此意。弟以为这是新民学会一个好现象,可大可久的事业,其基础即筑在这种“潜在的态度”之上。

你的信我在上海接到。彭、周、劳、魏都转给他们看了,我七月回湘,一回多忆,未能作答,幸谅!你现状谅好,我忘记你在芬丹白露,抑在蒙达尔尼?来信幸告。近因极倦,游觉到萍,旋中作书,言不尽意。

 弟泽东

 九年十一月二十五日萍乡旋中
\subsection{对易礼容来信的评论}

礼容这一封信,讨论吾人进行办法,主张要有预备,极忠极切。我的意见,于致陶斯咏姐及周惇之兄函中已具体表现,于归湘途中和礼容也当面说过几次。我觉得去年的驱张运动和今年的自治运动,在我们一班人看来,实在不是由我们去实行做一种政治运动。我们做这两种运动的意义,驱张运动只是简单的反抗张敬尧这个太令人过意不下去的强权者。自治运动是简单的希望在湖南能够特别定出一个办法(湖南宪法),将湖南造成一个较好的环境,我们好于这种环境之内,实现我们的具体准备工夫,彻底言之,这两种运动,都只是应付目前环境的一种权宜之计,决不是我们根本的主张,我们的主张远在这些运动之外,说到这里,诚哉如礼容所书,‘准备’要紧,不过准备的‘方法’怎样?又待研究。去年在京,陈赞周即对于‘驱张’怀疑,他说我们既相信世界主义和根本改造,就不要顾及目前的小问题,小事实,就不要‘驱张’。他的话当然也有理,但我意稍有不同,‘驱张’运动和自治运动等,也是达到根本改造的一种手段,是对付‘目前环境’最经济最有效的一种手段。但有一条件,即我们自始至终(从这种运动之发起至结局),只宜立于‘促进’的地位,明言之,即我们决不跳上政治舞台去做当局。我意我们新民学会会友于以后进行方法应分几种:一种是出国的,可分为二,一是专门从事学术研究,多造成有根底的学者,如罗荣熙肖子升之主张。一是从事于根本改造之计划和组织,确立一个改造的基础,如蔡和森所主张的共产党。一种是未出国,亦分为二,一是在省内及国内学校求学的,当然以求学储能做本位。一是从事社会运动的,可以从各而发起并实行各种有价值之社会运动及社会事业。其政治运动之认为最经济最有效者,如‘自治运动’,‘普选运动’等,亦可从旁尽一点促进之力,惟千万不要沾染旧社会习气’尢其不要忘记我们根本的共同的理想和计划。至礼容所说的结合同志,自然十分要紧,惟我们的结合,是一种互相的结合,人格要公开,目的要共同,我们总不要使我们意识中有一个不得其所的真同志就好。

 泽东
\subsection{复肖子嶂的信}
<p><font face="宋体">子嶂兄:

你沿途给我的信和照片,都收到,很感谢你的厚意。我竟没有一个信给你,很对你不住,‘半松园会议’的结果,既由他和赞周等带到了欧洲,‘蒙达尔尼会议’的情形,又由你和子升递回了亚洲,手升并且自己回国,不日就可见而了,真是乐阿!你现在谅好,谅还在学校。我意你在法宜研究一门学问,择你性之所宜者至少一门,这一门使要将他研究透彻。我近觉得仅仅常识是靠不住的,深慨自己学问无专精,两年来为事所扰,学问来能用功,实深抱恨,望你有以教我。学会出版分‘通信集’与‘会务报告’为二,通信集本年至少可出三集,请设法收集在法诸友间新旧信稿邮递‘长沙文化书社’交弟,作第四集第五集材料,至感。(请告诺友以后通信均写长沙文化书社)

 弟泽东
\subsection{给罗璈阶的信}

※叛湖南问题

※叛主义的结合

※湖南问题

※主义的结合

※学生自决
<p><font face="宋体">章龙兄:

昨信接到。重翻你七月二十五日的信,我昨信竟没有一句答复你信内的话,真对不住。今再奉复大意如下。我虽然不反对零碎解决,但我不赞成没有主义头痛医头脚痛医脚的解决。我主张湖南人不与闻外事,专把湖南一省弄好,有两个意思:一是中国太大了,各省的感情利害和民智程度又至不齐,要弄好他也无从着手,从康梁维新至孙黄革命,(两者亦自有他们相当的价值当别论)都只在这组织上用功,结果均归失败。急应改涂易辙,从各省小组织下手。湖南人便应以湖南一省为全国倡,各省小组织好了,全国总组织怕他不好,一是湖南的地理民性,均很有为。杂在全国的总组织者中,既消磨特长,复阻碍进步。独立自治,可以定出一种较进步的办法(湖南宪法),内之自庄严璀睽其河山,处之与世界有觉悟的民族直接携手,共为世界的大改造。全国各省也可因此而激厉进化。所以弟直主张湖南应自立为国,湖南完全自治,丝毫不受外力干涉,不要再为不中用的‘中国’所累,这实是进于总解决的一个紧要手段,而非和有些人所谓零碎解决实则是不痛不痒的解决相同,此意前函未尽,今再补陈于此。

兄所谓善良的有势力的士气,确是要紧。中国坏空气太深太厚,吾们戒哉要造成一种有势力的新空气,才可以将他划换过来。我想这种空气,固然要有一班刻苦励志的‘人’,尤其要有一种为大家共同信守的‘主义’,没有主义,是造不成空气的。我想我们学会,不可徒然做人的聚集,感情的结合,要变为主义的结合才好。主譬如一面旗子,旗子立起了,大家才有所指望,才知道趋赴,兄以为何如?

“会务报告”专纪会务,不载论述文字,尚未着手编写,大略每季一册尽够了。此外会友通信,发刊通信专集,为会友相互讨论商讨的场所,兄处有无会友们往还信稿,不论新旧,请检查寄弟。

“湘江”尚未出版,固因事忙,亦怡出而不好,到底出否,尚待斟酌。

弟本期在城南附小办一点事,杂以他务,自修时间很少,读‘岁月易逝无法挽回’,‘思想学术节节僵化’诸语,使我不寒而栗,我回湘时,原想无论如何每天要有一点钟看报,两点钟看书,竟不能实践。我想忙过今冬,从明年起,一定要实践这个条件才好。求学程序计,略有一点,迟后当可奉告。

讲到湖南教育,真是欲哭无泪。我于湖南教育共有两个希望:一个是希望至今还存在的一班造孽的教育家死尽,这个希望是做不到的。一是希望学生自决,我唯一的希望在此。怪不得人家说‘湖南学生的思想幼稚’,(沈仲九的话)从没有人供给过他们以思想,也没有自决的想将来自己的思想开发过,思想怎么会不幼稚呢?望时赐信为感!

 弟泽东

 (九年十一月二十五日)
\subsection{给钦文(李思安)的信(一九二〇年十一月二十五)}
<p><font face="宋体">钦文姐:

你这信我八月里就收到了。后来还接到你在星加坡寄来的一封长信,并一些印刷物。很感谢你的厚意。我因事忙,没有即答,想能原谅我吧。湘江尚未出版。湖南虽有一些志士从事实际的改造,你莫以为是几篇文章所能弄得好的。大伟人虽没有十分巩固,小伟人(政客)却很巩固了。我想对付他们的法子,最好是不理他们,由我们另想办法,另造环境,长期的预备,精密的计划。实力养成了,效果自然会见,倒不必和他们争一日的长短,你以为然么?你事务谅是忙的,我劝你总要有时间看一点新书报。并且希望你能够继续省察自己,能够知道自己的短处。你前信嘱转集虚,已转他看了。有暇望告我以近状。

 弟泽东

 十一月二十五日
\subsection{给张国基的信(一九二〇年十一月二十五)}

※会务问题

※湘人往南洋应取之态度

※南洋文化运动和南洋建国运动
<p><font face="宋体">颐生兄:

两信先后本悉,久未作复,甚歉!所言会务六项,弟大体均赞成。第一项发行会报,现已决发刊会员通讯集和会务报告两种。第二项会友加入宜郑重。第三项会友加入不要有男女老幼等区别。弟忙夏间在上海与焜甫,赞周,子障,荫柏、望成、韫广、敦洋.冀儒,玉生,等在半淞园会商,及回长沙再和长沙会友商酌,多主会友加入,要准备下列三个条件,(一)纯洁,(二)诚恳,(三)向上,并须有五人介绍,经评议部通过,然后再郑重通告全体会员,正与你的主张相合。(南洋方面同志,当然应该联络),第四项,会所的确定,也是要事,不久总要在长沙觅到一个相当的会所。书报的设置与会友研究有关,长沙巴黎南洋应分别设备,其经费可由各地会员“分”任,第五项,经费的筹措,我意只要会员常年费交齐,普通用费已够。此外只有印通信集和会务报告须款,但也不多,可由会员临时分任。会友录即印在会务报告之内,本年总可以印出一本。南洋通讯社组织极要。惟弟对于湘人往南洋有一意见,即湘人往南洋要学李石曾先生等介绍学生往法国之用意,取世界主义,而不釆殖民政策。世界主义,愿自己好,也愿别人好,质言之,即愿大家好的主义。殖民政策,只愿自己好,不愿别人好,质言之,即损人利己的政策,苟是世界主义,无地不可自容,李石曾等便是一个例。苟是殖民政策,则无地可以自容,日本人便是一个例,南洋文化闭塞,湘人往南洋者,宜以发达文化己任。兄等苟能在南洋为新文化运动,使国内发生之新文化,汇往南洋,南洋人(不必单言华侨)将受赐不浅。又南洋建国运动,亟须发起,苟有志士从事于此种运动,拯救千万无告之人民出水火而登袵席,其为大业,何以加兹。弟意我们会员宜有多人往南做教育运动,和文化运动,俟有成效,即进而联络华侨士著各地各界,鼓吹建国,世界大同,必以各地民族自决为基,南洋民族而能自决,即是促进大同的一个条件。有暇望时通信。

 弟泽东

 九年十一月二十五日给
\subsection{罗学瓒的信}

※养成读书和游戏并行的习惯

※理论上的错误

※拒婚同盟
<p><font face="宋体">荣熙兄:

兄七月十四日的信,所论各节,透彻之到身体诚哉是一个大问题。你谓中国读书人,以身殉学,是由于家庭、社会和学校的环境太坏造成的,这是客观方而的原因,诚哉不错。尚有主观方面的原因,就是心理上的惰性。如读书成了习惯,便一直谈下去不知休息。照卫生的法则,用脑一点钟,应休息一分钟,弟则常常接连三四点钟不休息,甚或夜以继日,并非乐此不渡,实是疲而不舍。我看中国下力人身体并不弱,身体弱只有读书人。要矫正这弊病,社会方面,须设造成好的环境。个人方面,须养成工读并行的习惯;至少也要养成读书和游戏并行的习惯。我的生活实在太劳了,怀中先生在时,曾屡劝我要节势,要多休息,但我总不能信他的话,现在我决定在城市住两个月,必要到乡村住一个星期。这次便是因休息到萍乡,以后拟每两个月要出游一次。

四种迷,说得最透彻,安得将你的话印刷四万万张遍中国人每人给一张就好。感情的生活,在人身原是很要紧,但不可令感情来论事。以部份概全体,是空间的误认。以一时概永久,是时间的误认。以主观概客观,是感情和空间合同的误认。四者通是犯了理论的错误。我近来常和朋友发生激然的争辩,均不出四者范围。我自信我于后三者的错误尚少。惟感情一项,颇不现能免。惟我的感情不是你所指的那些例,乃是对人的问题。我常觉得有站在言论界上的人就不佩服他,或发现他人格上有缺点。他发出来的议论,我便有些不大信用。以人废言,我自知这是我一个短处,日后要是矫正。我于后之者于说话高兴时或激烈的也常错误,不过自己却知道这是错误,所谓明知故犯罢了,(作文时也有)。

“工学励志会”,听说改成了“工学世界社”,详情我小知,请你将组织,进行,事务等,告我一信。通信尚未到。交换报一节弟可办到。请陆续将稿寄来(寄长沙文化书社交弟)。

以资本主义做基础的婚姻制度,是一种绝对要不得的事,在理论上是以法律保护最不合理的强奸,而禁止最合理的自由恋爱;在事实上,天下无数男女的怨声,乃均发现于这种婚姻制度的下面。我想现在反对婚姻制度已经有好多人说了,就只没有人实行。所以不实行,就只是“怕”,我听得“向蔡同盟”的事,为之一喜,向蔡已经打破了“怕”实行不要婚姻,我想我们正好奉向蔡做首领,组织一个“拒婚同盟’,已有婚约的,解除婚约(我反对人道主义)。没有婚约的,实行不要婚约。同盟组成了,同盟的各员立刻组成同盟军。开初只取消极的态度。对外‘防御’反对我们的敌人,对内好生正理内部的秩序,务使同盟内的各员,都实践“废婚姻”这条盟约。稍后,就可取积极的态度开始向世界“宣传”,开始“攻击”反对我们的敌人,务使全人类对于婚姻制度都得解放,都纳入同盟做同盟的一员。我这些话好象是笑话,实则兄所愈感的那些“家庭之吉”,非用这种好笑的办法,无可避免。假如没有人赞成我的办法,我“一个人的同盟”是已经结起了的。我觉得凡在婚姻制度底下的男女,只是一个“强奸团”,我是早已宣言不愿加入这个强奸团的。你如不赞成我的意见,便请你将反对的意见写出。此祝进步。

 弟泽东

 九年十一月二十六日


\section{新民学会会员通信集第三集}
\datesubtitle{新民学会致各会友的信}



各会友均鉴:

投寄通信集的信稿,请寄长沙文化书社转交为荷。
<p><font face="宋体"> 
新民学会启
\subsection{新民学会紧要启事}

本会同人结合。以互助互勉为鹄,自七年夏初成立,至今将及三年,虽形式未周,而精神一贯。惟会友个人对于会之精神,间或未能了解。有牵于他种事势不能分其注意力于本会者;有在他种团体感情甚洽因而对于本会无感情者;有缺乏团体生活之兴趣者;有行为不为会友之多数满意者;本会对于有上述形情之人,认为虽曾列为会友,实无互助互勉之可能。为保持会的精神起见,惟有不再认其为会员。并希望以后介绍新会员入会,务求无上列情形者。本会前途幸甚。


新民学会启

一千九百二十一年一月二日
\subsection{“新民学会会员通信集”第三集重点及付印出版日期}

这一集以讨论“共产主义”和“会务”为两个重要点。信的封数不多,而颇有精义。千九百二十一年一月上旬付印,下旬出版。
\subsection{给肖旭东、蔡林彬并在法诸会友的信(一九二○年十二月一日)}

※赞成“改造中国与世界”为学会方针。

※赞成马克思式的革命。

※学会的态度:(一)互助互勉。(二)诚恳(三)光明。(四)向上。

※共同研究及分门研究。

※学会的四种运动。

※联络同志之重要。
<p><font face="宋体">和森兄子升兄并转在法诸会友:

接到二兄各函,欣慰无量!学会有具体的计划,算从蒙达尔尼会议及二兄这几封信始。弟于学会前途,抱有极大希望,因之也略有一点计划,久思草具计划书提出于会友之前,以资商榷,今得二兄各信,我的计划书可以不作了。我只希望我们七十几个会友,对于二兄信上的计划,人人下一个祥密的考虑。随而下一个深切的批评,以决定或赞成,或反对,或于二兄信上所有计划和意见之外,再有别的计划和意见。我常觉得我们个人的发展或学会的发展,总要有一条明确的路数,没有一条明确的路数,各个人只是盲进,结果糟踏了各人自己外之,又糟踏了这个有希望的学会,岂不可惜?原来我们在没有这个学会之先,也就有一些计划,这个学会之所以成立,就是两年前一些人互相讨论研究的结果。学会建立以后,顿成功了一种共同的意识,于个人思想的改造,生活的向上,很有影响。同对于共同生活,共同进取,也颇有研究。但因为没有提出具体方案;又没有出版物可作公共讨论的机关;并且两年来会友分赴各方;在长沙的会员又因为政治上的障碍不能聚会讨论,所以虽然有些计划和意见,依然只藏之于各人的心里,或几人相会出之于各人的口里,或彼此通函见之于各人之信里;总之只存于一部分的会友间而己。现在诸君既有蒙达尔尼的大集会,商决了一个共同的主张;二兄又本乎自己的理想和观察,发表了个人的意见;我们不在法国的会员,对于诸君所提出当然要有一种研究,批评,和决定。除开在长沙方面会员,即将开会为共同的研究,批评,和决定外,先达我个人对于二兄来信的意见如左。

现在分条说来:

(一)学会方针问题。我们学会到底拿一种什么方针做我们共同的目标呢?子升信里述蒙达尔尼会议,对于学会进行之方针,说:“大家决定会务进行之方针在改造中国与世界”。以“改造中国与世界”为学会方针,正与我平日的主张相合,并且我料到是与多数会友的主张相合的。以我们接洽和视察,我们多数的会友,都顷向于世界主义,试看多数人鄙弃爱国,多数人鄙弃谋一部分一国家的私利,而忘却人类全体的幸福的事,多数人都觉得自己是人类的一员,而不愿意更繁复的隶属于无意义之某一国家,某一家庭,或某一宗教,而为其奴隶;就可以知道。这种世界主义,就是四海同胞主义,就是愿意自己好也愿意别人好的主义,也就是所谓社会主义。凡是社会主义,都是国际的,都是不应该带有爱国的色彩的。和森在八月十三日的信里说:“我将拟一种明确的提议书,注重无产阶级专政与国际色彩两点。因我所见高明一点的青年,多带一点中产阶级的眼光和国际的色彩,于此两点,非严正主张不可”。除无产阶级专政一点置于下条讨论外,国际色彩一点,现在确有将他郑重标揭出来的必要。虽然我们生在中国地方的人。为做事便利起见,又因为中国比较世界各地为更幼稚更腐败应先从着手改造起见。当然应在中国这一块地方做事;但是感情总要是普遍的,不要只爱这一块地方而不爱别的地方。这是一层,做事又并不限定在中国,我以为固应该有人在中国做事,更应该有人在世界做事,如帮助俄国完成他的社会革命;帮助朝鲜独立;帮助南洋独立;帮助蒙古、新疆、西藏、青海、自治自决;都是很要紧的。

以下说方法问题。

(二)方法问题。目的一一改造中国与世界一一定好了,接着发生的是方法问题。我们到底用什么方法去达到“改造中国与世界”的目的呢?和森信里说:“我现认清社会主义为资本主义的反映其重要使命在打破资本经济制度,其方法在无产阶级专政”。和森又说:“我以为现世界不能实行无政府主义,因在现世界显然有两个对立的阶级存在。打倒有产阶级的迪克推多,非以无产阶级的迪克推多压不住反动,俄国就是个明证。所以我对于中国将来的改造,以为完全适用社会主义的原理与方法。……我以后先要组织共产党,因为他是革命运动的发动者,宣传者,先锋队,作战部”。据和森的意见,以为应用俄国式的方位去达到改造中国与世界,是赞成马克思的方法的。而子升则说:“世界进化是无穷期的,革命也是又穷期,我们不认可以一部分的牺牲,换多数人的福利。主张温和的革命。以教育为工具的革命,为人民谋全体福利的革命。以工会合社为实行改造之方法。颇不认俄式――马克思式一一革命为正当,而倾向无政府――蒲鲁东式一-之新式革命,比较和而缓。虽缓然和,同时李和笙兄来信,主张与子升相同。李说:“社会改进,我不赞成笼统的改造,用分工协助的方法,从社会内面改造出来,我觉得很好。一个社会的病,自有他的特别的背影,一剂学方可以医天下人的病,我很怀疑。俄国式的革命,我根本上有未敢赞同之处”。我对子升和笙两人的意见。(用平和的手段,谋全体的幸福)在真理上是赞成的,但在事实上认为做不到。罗素在长沙演说,意与子升及和笙同,主张共产主义,但反对劳农专政,谓宜用教育的方法使有产阶级觉悟,可不至要妨碍自由,兴起战争,革命流血。但我于罗素讲演后,曾和殷柏,礼容等有极详细之辩论。我对于罗素的主张,有两句评语:就是:“理论上说得通,事实上做不到”。罗素和子升和笙主张的要点,是“用教育的方法”但教育一要有钱,二要有人,三要有机关。现在世界,钱尽在资本家的手,主持教育的人尽是一些鸯本家,或资本家的奴隶。总言之,现在世界的学校及报馆两种最主要的教育机关,又尽在资本家的掌握中,现在世界的教育,是一种资本主义的教育。以资本主义教育儿童,这些儿童大了又转而用资本主义教第二代的儿童。教育所以落在资本家手里,则因为资本家有“议会”以制定保护资本家并防制无产阶级的法律,有“政府”执行这些法律,以积极的实行其所保护与所禁止。有“军队”与“警察”,以消极的保障资本家的安乐与禁止无产者的要求。有“银行”以为其财货流通的府库。有工厂以为其生产品垄断的机关。如此,共产党人非取政权,且不能安息于其守下,更要能握得其教育权;如此,资本家久握教育权,大鼓吹其资本主义,使共产党人的共产主义宣传,信者日见其微。所以我觉得教育的方法是不行的。我看俄国式的革命,是无可如何的山穷水尽诸路皆走不通了的一个变计。并不是有更好的方法弃而不釆,单要釆这个恐怖的方法。以上是第一层理由。第二层,依心理上习惯性的原理,及人类历史上的观察,觉得要资本家信共产主义,是不可能的事。人生有一种习惯性,是心理上的一种力,正与物在斜方必倾向下之物理上的一,种力一样。要物下倾向下,依力学原理,要有与他相等的一力去抵抗他才行。要人心改变,也要有一种与这心力强度相等的力去反抗他才行。用教育之力去改变他,既不能拿到学校与报馆两种教育机关的全部或一大部到手,或有口舌印刷物或一二学校报馆为宣传之具。正如朱子所谓“教学如扶醉人,扶得东来西又倒”,“直不足以动资本主义者心理的毫未,那有同心向善之望?以上从心理上说。再从历史上说,人类生活全是一种现实欲望的扩张。这种现实欲望,只向扩张的方面走,决不向减缩的方面走。小资本家必想做大资本家,大资本家必想做最大的资本家,是一定的心理。历史上凡是专制主义者或帝国主义者,或军阀主义者,非等到人家来推倒,决没有自己肯收场的。有拿破伦第一称帝失败了,又有拿破伦第三称帝。有袁世凯失败了,偏又有段祺瑞。章太炎在长沙演说,劝大家读历史,谓袁段等失败均为不读历史之故。我谓读历史是智慧的事,求逐所欲是冲动的事,智慧指导冲动,只能于相当范围有效力,一出范围,冲动使将智慧压倒,勇敢前进,必要回到了比冲动前进之力更大的力,然后才可以将他打回。有几句俗话:“人不到黄河心不死”,“这山望见那山高”,“人心不知足,得陇又望蜀”,均可以证明这个道理。以上从心理上及历史上看.可见资本主义是不能以些小教育之力推翻的,是第二层的理由。再说第三层理由,理想固要紧,现实尤其要紧,用和平方法去达到共产目的,要向何日才能成功?假如要一百年,这一百年中宛转呻吟的无产阶级,我们对之,如何处置,(就是我们)。无产阶级比有产阶级实在要多得若干倍,假定无产者占三分之二,则十五万万人类中有十万万无产者(恐怕还不只此数)这一百年中,任其为三分之一之资本家鱼肉,其何能忍?且无产者既已觉悟到自己应该有产,而现在受无产的痛苦是不应该;因无产的不安,而发生共产的要求;已经成了一种事实。事实是当前的,是不能消灭的,是知了就要行的。因此我觉得俄国的革命,和各国急进派共产党人数日见其多,组织日见其密,只是自然的结果.以上是第三层理由。再有一层,是我对于无政府主义的怀疑。我的理由却不仅无强权无组织的社会状态之不可能。我只忧一到这种社会状态实现了之难以终其局。

因为这种社会状态是定要造成人类死率减少而生率加多的,其结局必至于人满为患。如果不能做到(一)不吃饭;(二)不穿衣;(三)不住屋;(四)地球上各处气候寒暖,和土地肥瘦均一;或是(五)更发明无量可以住人的新地,是终于免不掉人满为患一个难关的。因上各层理由,所以我对于绝对的自由主义,无政府主义,及德谟克西拉的主义,依我现在的看法,部只认为于理论上说得好听,事实上是做不到的。因此我于子升和笙二兄的主张,不表同意。而于和森的主张,表示深切的赞同。

(三)态度问题。分学会的态度与会友的态度两种:学会的态度,我以为第一是“潜在”,这在上海半松园曾讨论过,今又为在法会友所赞成,总要算可以确定了。第二是“不依赖旧势力”,我们运学会是新的,是创造的,决不宜许旧势力混入,这一点要请大家注意。至于会友相互及会友个人的态度,我以为第一是“互助互勉”,(互助如急难上的互助,学问上的互助,事业上的互助。互勉如积极的勉为善,消极的勉去恶)。第二是诚恳(不滑)。第三是光明(人格的光明)。第四是向上(能变化气质有向上心)第一是“相互间”应该具有的,第二第三第四是“个人”应该具有的。以上学会的态度二项,会友的态度四项,是会友精神所寄,非常重要。

(四)求学问题。极端赞成诸君共同研究及分门研究之两法。诸君感于散处不便,谋合居一处,一面作工,一面有集会机缘,时常可以开共同的研究会,极善。长沙方面会友本在一起,诸君办法此间必要仿行。至分门研究之法,以主义为纲,以书报为目,分别阅读,互相交换,办法最好没有。我意凡有会员两人之处,即应照此组织。子升举力学之必要,谓我们常识尚不充足,我们同志中一尚无专门研究学术者,中国现在尚无可数的学者,诚哉不错!思想进步是生活事业进步之基。使思想进步的唯一方法,是研究学术。弟为荒学,甚为不安,以后必要照诸君的办法,发奋求学。

(五)会务进行问题。此节子升及和森意见最多。子升之“学会我见”十八项,弟皆赞成。其中“根本计划”之“确定会务进行方针”,“准备人才”,“准备经济”三条尤有卓见。以在民国二十五年前为纯粹预备时期,我以为尚要延长五年,以至民国三十年为纯粹预备期。子升所列以长沙方面诸条,以“综挈会务大纲,稳立基础”,“筹办小学”,“物色基本会员”三项,为最要紧,此外尚应加入“创立有价值之新事业数种”一项,子升所列之海外部,以法国、、俄国、南洋三方而为最重。弟意学会的运动,暂时可总括为四:1、湖南运动;2、南洋运动;3`留法运动;4留俄运动。暂时不必务广,以发展此四种,而使之确见成效为鹄。较为明切有着,诸君以何如?甚和森要我进行之“小学教育”,“劳动教育”,“合作运动”,“小册子”,“亲属聚居”,“帮助各团体”讲端,我都愿意进行。惟“贴邮花”一项我不懂意,请再见示。现在文化书社成立,基础可望稳固,营业亦可望发展。现有每县设一分社的计划,拟两年内办成,果办成,效自不小。

(六)同志联络问题。这项极为要紧,我以为我们七十几个会员,要以至诚恳切的心,分在各方面随时联络各人接近的同志,以携手共上于世界改造的道路。不分男、女、老、少、士、农、王、商、只要他心意诚恳,人格光明,思想向上,能得互助互勉之益,无不可与之联络,结为同心。此节和森信中详言,子升亦有提及。我觉得创造特别环境,改造中国与世界的大业,断不是少数人可以包办的,希望我们七十几个人,人人注意及此。

我的意见大略说完了。闻子升已回国到北京,不久可以面谈。请在法诸友:阵将我的意见加以批评,以期求得一个共同的决定。个人幸甚,学会幸甚。

\begin{flushright}
弟毛泽东

九年十二月一日
    
文化书社夜十二时\end{flushright}
\subsection{给和森的信(一九二一年一月二十一日)}
<p><font face="宋体">和森兄:

来信于年底始由子升转到。唯物史观是吾党哲学的根据,这是事实。不象唯理观之不能证实而容易被人摇动。我固无研究,但我现在不承认无政府的原理是可以证实的原理,有很强固的理由。一个工厂的政治组织(工厂分配管理等)与一个国的政治组织,与世界的政治组织,只有大小不同,没有性质不同。工团主义以国的政治组织与工厂的政治组织异性,认为另一回事而举以属之另一种人,不是因为曲说以冀苟且偷安,就是愚陋不明事理之正。况乎尚有非得政权则不能发动革命不能保护革命不能完成革命在手段上又有十分必要的理由呢。你这一封信见地极当,我没有一个字不赞成。党一层陈仲甫先生等已在进行组织,出版物一层上海出的“共产党”,你处谅可得到,颇不愧“旗帜鲜明”四字,(宣言即仲甫所为)。详情后报。

 弟泽东

 十年一月廿一日在城南

\section{非自杀(节录)}
\datesubtitle{(一九一九年十一月一日)}



吾人应主张与社会奋斗。…………反抗旧社全!与其自杀而死,宁奋斗被杀而亡。为目的达不到,截肠决战,玉碎而亡,则其天下之至刚至勇,而最足以印人脑府的了!

\begin{flushright}(原载长沙《大公报》,转抄自《光辉的五四》中国青年出版社一九五九年)\end{flushright}


\section{驱逐张敬尧文电}
\datesubtitle{(一九二〇年一月)}



……吾湘不幸,叠受兵凶,连亘数年,疮痍满目。上岁张敬尧入湘以后,纵饿狼之兵,奸焚劫杀,聘猛虎之政,铲刮诈捐。卖公地,卖湖田,卖矿厂,卖纱厂,公家之财产已罄;加米捐,加盐税,加纸捐,加田税,人民之膏脂全干。洎乎今日,富者贫,贫者死,困苦流离之况,令人不忍卒闻。被张贼兄弟累资各数千百,尚不自厌,连此仅存之米盐公款,竟思攫入私囊以甘心。去年张贼曾嗾使湘痞李鸣九等电京查问此款,至去长沙设立米盐公股清查处;闻近复贿令郭人漳等以旅京湘事维持会名义向熊秉三先生诘问该款,无理取闹;推其用意,无非欲攫尽湖南财产,吃尽湖南人民,以饱其欲壑。窃吾湘遭此亘创之际,哀哀子遗,非有亘资,何以善后,米盐公款为我三千万人民忍饥食淡所储蓄,秉必不易,利用尤殷,倘被鲸吞,此劫难复。被张贼之暴戾酷害,毒我湘人,已成惯技,独不能我同此食毛践士之败类,自杀父母之邦,甘与仇敌作狗。人之无良,至于此极!公民等对于郭人漳等此种恶劣行为,誓不承认。总之此款在张贼未去湘乱未宁以前,只可暂归湘绅保管,不得变动,俟湘事保定后,再由全省民意公决用途,此际倘来无耻之徒,希图破坏者,即视为公敌。凡我湘人,应知自卫,稍纵即逝,祈毋忽焉。湖南公民代表毛泽东等五十五人(下略)

\begin{flushright}(抄自中国革命历史博物馆)\end{flushright}


\section{发起“文化书社”缘起(摘要)}
\datesubtitle{(一九二○年七月三十一日)}



没有新文化,由于没有新思想;没有新思想,由于没有新研究;没有新研究,由于没有新材料。湖南人现在脑子饥荒,实在过于肚子饥荒,青年人尤其嗷嗷待哺。文化书社愿以最迅速、最简便的方法,介绍中外各种新杂志,以充青年及前进的全体湖南人新研究的材料。也许因而有新思想、新文化的产生,那真是我们心向祷之,希望不尽的。

文化书社由我们一些互相了解,完全信得过的人发起,不论谁投的本,永远不得收回,亦永远不要利息,此书社但永远为投本的人所共有。书社发达了,本身到了几百元,彼此不因为利;失败至于只剩一元,彼此无怨;大家共认地球之上,长沙城中,有此“共有”的一个书社罢了。

\begin{flushright}(摘自《光辉的五四》第91页,中国青年出版社1959版。)\end{flushright}


\section{湖南建设问题的根本问题一一湖南共和国}
\datesubtitle{(一九二○年九月三日)}



乡居寂静,一卧兼旬,九月一号到省,翻阅大公报,封面打了红色,中间有许多我们最喜欢的议论,引起我的高兴,很愿意继续将我的一些意思写出。

我是反对“大中华民国”的,我是主张“湖南共和国”的。有什么理由呢?

大概从前有一种谬论,就是“在今后世界能够争存的国家,必定是大国家。”这种议论的流毒,扩充帝国主义,压抑自国的弱小民族,在争海外殖民地,使半开化未开化之民族,变成完全奴隶,窒其生存向上,而惟使恭顺驯屈于己。最著的例,是英美德法俄奥,他们幸都收了其实没有成功的成功。还有一个,就是中国,连“其实没有成功的成功”,都没收得。收得的是满洲人消灭,蒙人,回人,藏人,奄奄欲死,十八省乱七八糟,造成三个政府,三个国会,二十个以上督军王,巡按使王,总司令王,老百姓天天被人杀死奸死,财产荡空,外债如山,号称共和民国,没有几个懂得“什么是共和”的国民。四万万人不少,有三万九千万,不晓得写信看报。全国没有一条自主的铁路。不能办邮政,不能驾“洋船”,不能经理食盐。十八省中象湖南四川广东福建浙江湖北一类的省,通变成被征服省,屡践他人的马蹄,受害无极。这些果都是谁之罪呢?我敢说,是帝国之罪,是大国之罪,是“在世界能够争有的国家必定是大国家”一种谬论的罪。根本的说,是人民的罪。

现在我们知道,世界的大国多半瓦解了,俄国的旗子变成了红色,完全是世界主义的平民天下。德国也染成了半红。波兰独立,捷克独立,匈牙利独立。犹太,阿拉伯,亚美尼亚,都重新建国。爱尔兰狂欲脱离英吉利,朝鲜狂欲脱离日本。在我们东北的西北利亚远东片土,亦越了三个政府。全世界风起云涌,“民族自决”,高唱入云,打破大国迷梦,知道是野心家欺人的鬼话,推翻帝国主义,不许他再亲作祟。全世界盖有好些人民,业已醒觉了。

中国呢?也醒觉了(除开政客官僚,军阀)。九年假共和大战乱的经验,迫人不得不醒觉,知道全国的总建设在一个期内完全无望,最好办法,是索性不谋总建设,索性分裂去谋各省的分建设,实行“各省人民自决主义”’二十二行省,三特区,两藩地,合共二十七个地方,最好分为二十七国。

湖南呢?至于我们湖南,尤其三千万人个个应该醒觉了。湖南人没有别的法子,唯一的法子,是湖南自决自治,是湖南人在湖南地域建设一个“湖南共和国”。我曾着实想过,救湖南救中国,图与全世界解放的民族携手,均非这样不行。湖南人没有把湖南自建为国的决心和勇气,湖南终究是没办法。

说湖南建设问题,我觉得这是一个根本问题,我颇有点意思要发表出来,乞吾三千万同胞的聪听,希望共起讨论这一个顶有意思的大问题。今天是个发端,余俟明日以后继该讨论。

{\raggedleft(原载长沙《大公报》,抄自王无为编《湖南自治运动史》上编,太东图书局发行,一九二○年十二月二十日初版)}



=======
\documentclass[b5paper,oneside,12pt]{ctexbook}
\usepackage[hmargin=0.25in,vmargin=0.5in]{geometry} 
\usepackage[]{hyperref}
\pagestyle{plain} %整书页眉页脚设置
\ctexset{chapter/numbering=false}
\ctexset{section/numbering=false}
\ctexset{subsection/numbering=false}
\newcommand\datesubtitle[1]{{\begin{center} \small #1 \end{center}}}  %自定义日期副标题格式,为了保险,最好使用两层大括号

\title{毛泽东思想万岁}
\author{毛泽东}
\date{}

\begin{document}

\frontmatter
\maketitle
\tableofcontents

\mainmatter
% \chapter{}

% \section*{第一师范讲堂录}\addcontentsline{toc}{section}{第一师范讲堂录}
% 使用上面的写法或者在导言区配置: \ctexset{section/numbering=false}
\section{第一师范讲堂录}\datesubtitle{(一九一三年十一月一日)}

\textbf{修身}。人情多耽安佚而惮苦,懒惰为万恶之渊薮。人而懒惰,农则废其田畴,工则废其规矩,商贾则废其所鬻,士则废其所学,业即废矣,无以为生,而杀身亡家乃随之矣。因而懒惰,始则不进,继则退行,继则衰弱,终则灭亡,可畏哉!故曰懒惰万恶之渊薮也。

\textbf{奋斗}。夫以五千之卒,敌十万之军。策罢乏之兵,当新霸之马。如此欲图存而不亡,非奋斗不可。

\textbf{朝气}。少年须有朝气,否则暮气中之。暮气之来,乘疏懈之隙也。故曰怠隋者生之坟墓。

% \section*{第一师范讲堂录}\addcontentsline{toc}{section}{第一师范讲堂录}
% 使用上面的写法或者在导言区配置: \ctexset{section/numbering=false}
\section{读《明耻篇》后题词}\datesubtitle{(一九一五年)}
毛主席在第一师范求学期间,读了当时很多进步书刊。当他看了《明耻篇》以后,义愤填胸,来自挥笔在书上写道:
\begin{center}
“五月七日,\par{}国民奇耻;\par{}何以报仇,\par{}  在我学子!”
\end{center}

这十六个字充分表现了主席青少年时就立下了以天下为己任的革命的雄心壮志。

\section{《新青年》第七卷第一号}\datesubtitle{(一九一七年)} 
长沙受了“五四”运动的鼓动,一般人向新潮方面定的,实在不少,又因近年来屡受军阀派的摧残,弄得长沙这锦绣的地方,常常没有一点生气,于是一般稍有知识的人—各校教职员及学生—莫不深深痛恨。更觉得这《社会改造》、《思想革新》、《妇女解放》、《民族自决》种种问题万不容缓。
\section{工人夜校招生广告}
\noindent{列位工人来听我们说几句白话:}

列位最不便利的是什么?就是俗话说的:讲了写不得,写了认不得,有数算不得。列位做工的人,又要劳动又无人教授,如何才能写得几个字,算得几笔数呢?现今有个最好的法子,就是我们第一师范办了一个夜校,今年上半年学生很多,列位中想有听到过的。这个学校专为列位工人设的,从礼拜一起到礼拜五止每夜上课两点钟,教的是写信,算账,都是列位时刻需要的。讲义归我们发给,并不要钱,夜间上课又于列位工作并无妨碍。若是要求求学的,请赶快于一礼拜内到师范的号房报名。

有说时势不好,恐怕犯了戒严的命令,此事我们可以担保,上学以后,每人发讲牌一块,遇有军警查问,说是师范夜校学生就无妨碍了。若有为难之处,我们替你作保,此层只管放心,快快来报名,莫再耽搁。

\begin{flushright}第一师范学友会教学研究部启\end{flushright}
\section{夜学日记}
\datesubtitle{(一九一七年十一月十四日)片断}

甲班上课算术罗宗翰出席教以数之种类加法大略及亚拉伯数字码,历史常识毛泽东出席教历朝大势及上古事迹。学生有四人未带算盘,从小学暂借,为戒严早半时下课,管理者李端\{输fix\}、萧珍元。

实验三日矣,觉国文似太多太深,太多宜减其分量,太深宜改用通俗语(介乎白话与文言之间),常识分量亦嫌太多(指文字),宜少用文字,其讲义宜用白话,简单几句表明,初不发给,单用精神讲演,终取讲义略读一遍足矣,本日历史即改用此法,觉活泼得多。

本日算术却过浅,学生学过归除者令其举手,有十几人之多,此则宜逐渐加深。


\section{新民学会的方针}
\datesubtitle{(一九一八年四月)}



1938年4月毛泽东同志团结湖南进步青年,组织了革命团体——新民学会,新民学会后来在传播马克思列宁主义和推动革命运动起了重要作用。

明确地提出了学会的宗旨是“改造中国与世界”。

学会的方针问题。我们学会到底拿一种甚么方针做我们共同的目标呢?信里述蒙达尔尼会议,对于学会进行之方针说:“大家决定今后进行之方针在改造中国与世界。”以“改造中国与世界”为学会方针,正与我平日的主张相合,并且我料到是与多数的会友的主张相合的,以我的接触和观察,我们多数的会友都倾向于世界主义,试看多数人鄙弃爱国,多数人鄙弃某一部分,一国家的私利,而忘却人类全体的幸福的事,多数人都觉得自己是人类的一员,而不愿意更复杂地隶属于无意义之某一国家、某一家庭、或某一宗教,而为其奴隶,就可以知道了。


\section{在上海送别第一批去法国勤工俭学同学时讲话}
\datesubtitle{(一九一九年初)}



我觉得我们要有人到外国去,看些新东西,学些新道理,研究有用的学问拿回来,改造我们的国家。同时也要有人留在本国,研究本国问题。我觉得我对于自己的国家我所知道的还太少,假若我把时间花费在本国,则对本国更为有利。


\section{湘江评论创刊宣言}
\datesubtitle{(一九一九年七月十四日)}



自“世界革命”的呼声大倡。“人类解放”的运动猛进。从前吾人所不置疑的问题。所不递取的方法。多说畏缩的说话。于今都要一改旧观。不疑者疑。不取者取。多畏缩者不畏缩了。这种潮流。任是什么力量。不能阻住。任是什么人物。不能不受他的软化。

世界什么问题最大?吃饭问题最大。什么力量最强?民众联合的力量最强。什么不要怕?天不要怕。鬼不要怕。死人不要怕。官僚不要怕。军阀不要怕。资本家不要怕。

自文艺复兴。思想解放。“人类应如何生活”。成了一个绝大的问题。从这个问题。加以研究。我深了“应该那样生活”“不应该这样生活”的结论。一些学者倡之。大多民众和之。就成功或将要成功许多方面的改革。

见于宗教方面为“宗教改革”。结果得了信教自由。见于文学方面。由贵族的文学。古典的文学。死形的文学。变为学民的文学。现代的文学。有生命的文学。见于政治方面。由独裁政治。变为代议政治。由有限制的选举。变为没限制的选举。见于社会方面。由少数阶级专制的黑暗社会。变为全体人民自由发展的光明社会。见于教育方面。为平民教育主义。见于经济方面。为劳获平均主义。见于欲想方面。为实验主义。见于国际方面。为国际同盟。

各种改革。一言蔽之。“由强权得自由”而已。各种对抗强权的根本主义。为平民主义。(兑莫克拉西。——作民本主义。民主主义。庶民主义。)宗教的强权。文学的强权。政治的强权。社会的强权。教育的强权。经济的强权。思想的强权。国际的强权。丝毫没有存在的余地。都要借平民主义的高呼。将他打倒。

如何打倒的方法。则有两说.一急烈的。一温和的。两种方法。我们应有一番选择。

(一)我们承认强权者都是人。都是我们的同类。滥用强权。是他们不自觉的误谬与不幸。

(二)用强权打倒强权。结果仍然得到强权。不但自相矛盾。并且毫无效力。欧洲的“同盟”。“协约”战争。我国的“南”“北”战争。都是这一类。

所以我们的见解。在学术方面。主张彻底研究。不受一切传说和迷信的束缚。要寻着什么真理?在对人的方面。主张群众联合。向强权者为持续的“忠告运动”。实行“呼声革命”——面包的呼声。自由的呼声。平等的呼声。——“无血革命”。不至张起大扰乱。行那没效果的“炸弹革命”“有血革命”。

国际的强权。迫上了我们的眉睫。就是日本。罢课。罢市。罢工。排货。种种运动。就是直接间接对付强权日本有效的方法。

至于湘江。乃地球上东半球东方的一条江。他的水很清。他的流很长。住在这江上和他邻近的民族。浑浑噩噩。世界上事情。很少懂得。他们没有有组织的社会。人人自营散处。只知有最狭的一己。和最短的一时。共同生活。久远观念。多半未曾梦见。他们的政治没有和意和彻底的解决。只知道私争。他们被外界的大潮卷急了。也办了些教育。却无甚效力。一般官僚式教育家。死死盘踞。把学校当监狱。待学生如囚徒。他们的产业没有开发。他们中也有一些有用人材。在各国各地学好了学问和艺术。但没有给他们用武的余地。闭锁一个洞庭湖。将他们轻轻挡住。他们的部落思想又很厉害。实行湖南饭湖南人吃的主义。教育实业界不能多多容纳异材。他们的脑子贫弱而又腐败。有增益改良的必要。没有提倡。他们正在求学的青年。很多。很有为。没人用有效的方法。将种种有益的新知识新艺术启导他们。咳,湘江湘江!你真枉存在于地球上。

时机到了!世界的大潮卷得更急了!洞庭湖的闸门动了。且开了!浩浩荡荡的新思潮业已奔腾澎湃于湘江两岸了!顺他的生。逆他的死。如何承受他?如何传播他?如何研究他?如何施行他?是我们全体湘人最切最要的大问题。即是“湘江”出世最切最要的大任务。

\subsection{西方大事述评—各国的罢工风潮}

法英美三国的官阀和财阀,倾注全力于巴黎和会,用高压手段对付败北的德奥,正在兴高釆烈时候,他们的国内,忽然发生了罢工风潮。罢工在他们国里,原是一件常事,政府和财阀,虽然不敢十分轻视劳动者。每当劳动者拿着劳获不均,工时太久,住屋不适,失职无归,种种怨愤不平的问题,联合同类,愤起罢工的时候,也不得不小小给他们一点恩惠。正如小儿哭饿,看着十分伤心,大人也不得不笑着给他一个饼子。但终是杯水车薪,济得甚事。所以广义派人,都笑英法的工人是小见识。从老虎口里讨碎肉,是不能够的。

此回各国的罢工风潮,英国因为在大战初了时候,(去年十二月)英伦加苏格兰各埠交通机关,燃料业,矿山业,造船业等已演了一大罢工。故此次罢工,未发生于英伦本土。法国罢工情形,初颇严重。终亦以小惠收场,没闹出什么好结果。广义派人有乘机在巴黎实行政治运动之说,亦未见诸事实。美国一部分电报电话人员的罢工,乃在附和议员多数派反对加入国际联盟。与英法罢工,异其目的。意大利之罢工乃社会觉嫉恶其政府所为的一种运动。德国自去冬少数社会党大失败,各处大罢工,亦随之而没得好结果。多数社会党掌握政权以来,早已噤若寒蝉,不敢出声。此次为和约签字问题,有激起罢工的形势。但施特满内阁倒了,继任巴安内阁,仍是前内阁的同调。抵御外侮不足,防备家贼有余的武力,紧握在手,谁敢予侮。广义一派的猛断政略,暂时决没有发动的机会。罢工不能成为事实,亦无足怪。匈牙利所受罢工影响不大,其原因则全在缺粮没饭吃。今将一月以来各国罢工情形,分述于下——

法国\quad{}六月三日,罢工风潮发生后,蔓延甚速。巴黎一区,男女工人赋闲者,二十万人。所要求各业不同,而一致主张每日工作八小时。四日蔓延更广,推算当有五十万人罢工。五日,洗衣工人罢工。自后罢工的人数更多。地底铁道,电车,街车的用人决议继续罢工。地底铁道工人要求工值每月至少四百五十佛朗。(以每佛朗当四角合我一百八十元)满五十岁须给养老金。服役十五年后亦须给养老金若干。七日巴黎罢工现象有转机,五金业与机器业,工人与雇主,已商妥数事。十一日,五金业及地底铁道工人上工。当道已取必要万法对付铁道罢工。煤矿工人有全体罢工的形势。十二日,国会通过矿工每日工作八时议案。但矿工会议,仍不满意,决定从十六日,全体罢工。水夫联合会,也决计于十六日罢工。工人联合会,言及生活代价奇昂,(记者按,近有从巴黎回者,举一物贵实例,一个旧牙刷,价二佛朗。一双皮鞋,价六十佛胡。)谓非洲各口岸,堆集麦粮千百吨,任其朽腐。各埠存货如山,轮船火车,宁闲置不远载。这样的政府,可要快快废止他的消耗,欺骗,和垄断!十四日,风潮渐平。极端派有乘机推翻克勒满沙强权政府的运动。路工联合会拒绝之。但矿工因解释政府每日工作八小时议案,未能满意,定十六日全体罢工。恐怕路矿运输联合会工人,也会罢工,表示同情。克勒满沙老头子急了,和运输公司及运输工人代表会商,恳请彼等在国家危机时候,发出些爱国热忱。工人吃了他的浓米汤,已老老实实决议上工了。

英国\quad{}伦敦五月三十日电,全国警察拟于三日罢工的气象,正在酝酿中,政府已允增给薪资,优加待遇。但不承认警察联合会及收用已革除的警察。英国的属地澳洲,坎拿大,苏依士,昔有罢工风潮。六月四日,坎拿大维克斯兵工厂工人罢工,要求每星期工作四十小时。六月五日,苏依士运河工人罢工,局势狠恶,六月九日,澳港航务罢工,势头狠烈,各项工业,都受窒碍。十×日,风潮仍严重,他业工人,因此赋闲的逐日增多。

美国\quad{}六月七日,芝加哥电报司员联合会一时罢工,共约六万人。内有二万五千人系属电报司员联合会.该会会长康能堪氏正计划全国罢工办法。同日,全国电话司员奉命于十六日起罢工,表同情于电报司员。八日,电报人员联合会干事,向全体电报人员宣布,连收发电报生在内,全体罢工。目的在停止威尔逊总统每日在巴黎往来的电报,使他注意国民不赞成他在和会的主张。十二日,各电报公司报告,电报司员罢工没成。

意国\quad{}六月十三日,意大利斯贝齐亚地方,因粮食昂贵,发生暴动,捣毁商店。十四日,热那亚工界示威,被捕者数百人,银行商店闭门,电车不走。杜林工人此日多停工,纪念德国斯巴达团领袖卢森堡氏。米兰工人罢工,抗议热那亚与斯贝齐亚当道的行动。

德国\quad{}六月十三日,大柏林公民会议秘密会,决议罢工。各职业及军界中人,均赞助停止各项实业工作的计划。有人料此举将促成国内战争。中等社会将得政权。

匈国\quad{}五月三十一日,匈京饥饿的工人,发生暴动.红旗军奉共产政府命令到各工厂制乱。匈京几无粮食。

\subsection{东方人事述评—陈独秀之被捕及营救}

前北京大学文科学长陈独秀,于六月十一日,在北京新世界被捕。被捕的原因,据警厅方而的布告,系因这日晚上,有人在新世界散布市民宣言的传单,被密探拘去。到警厅诘问,方知是陈氏。今录中美通讯社所述什么北京市民宣言的传单于下——

一、取消欧战期内一切中日秘约。

二、免除徐树铮曹汝霖章宗群陆宗舆段芝贵王怀庆职。并即驱逐出京。

三、取消步军统领衙门,及警备总司令。

四、北京保安队,由商民组织。

五、促进南北和议。

六、人民有绝对的言论出版集会和自由权。

以上六条,乃人民对于政府最低之要求,乃希望以和平方法达此目的。倘政府不俯顺民意,则北京市民,惟有直接行动,图根本之改造。

上文是北京市民宣言传单,我们看了,也没有什么大不了处。政府将陈氏捉了,各报所载,很受虐待。北京学生全体有一个公函呈到警厅。请求释放。下面是公函的原文――

警察总监钧鉴,敬启者,近闻军警逮捕北京大学前文科学长陈独秀,拟加重究,学生等期期以为不可。特举出两要点于下:(一)陈先生夙负学界重望,其言论思想,皆见称于国内外。倘此次以嫌疑遽加之罪,恐激动全国学界再起波澜。当此学潮紧急之时,殊非息事宁人之计。(二)陈先生向以提倡新文学现代思想见异于一般守旧者。此次忽被逮捕,诚恐国内外人士,疑军警当局,有意罗织,以为摧残近代思想之步。现今各种问题,已极复杂,岂可再生枝节,以滋纠纷?基此二种理由,学生等特请贵厅,将陈独秀早予保释。

北京学生又有致上海各报各学校各界一电——

陈独秀氏为提倡近代思想最力之人,实学界重镇。忽于真日被逮。住宅亦披抄查。群清无任惶骇。除设法援救外,并希国人注意。

上海工业学会也有请求释放陈氏的电。有“以北京学潮,迁怒陈氏一人,大乱之机,将从此开始”的话。政府尚未昏聩到全不知外间大事,可料不久就会放出。若说硬要兴一文字狱,与举世披靡的近代思潮,拚一死战,吾恐政府也没有这么大胆子。章行严与陈君为多年旧交,陈在大学任文科学长时,章亦在大学任图书馆长及研究所逻辑教授。于陈君被捕,即有一电给京里的王克敏,要他转达别厅,立予释放。大要说——

……陈君向以讲学为务,平生不含政治党派的臭味。此次虽因文字失当,亦何至遽兴大狱,视若囚犯,至断绝家常往来。且值学潮甫息之秋,訑可忽兴文绵,重激众怒。甚为诸公所不取。……

章氏又致代总理龚心湛一函。说得更加激切——

仙舟先生执事,久违矩教,结念为劳。兹有恳者,前北京大学文利学长陈独秀,闻因牵涉传单之嫌,致被逮捕,迄今末释。其事实如何,远道未能详悉。惟念陈君平日,专以讲学为务。虽其提倡新思潮,著书立论,或不无过甚之词,然范围实仅及于文字方面,决不舍有政治臭味,则固皎然可征。方今国家多事,且值学潮甫息之后,讵可蹈腹诽之殊,师监谤之策,而愈激动人之心理耶。窃为诸公所不取。故就历史论,执政因文字小故而专与文人为难,致兴文字之狱。幸而胜之,是为不武。不胜人心瓦解,政纽摧崩,虽有善者。莫之能挽。试观古今中外,每当文网最甚之秋,正其国运衰歇之候。以明末为殷鉴,可为寒心。今日谣琢萦兴,清流危惧。乃遽有此罪及文人之举,是露国家不祥之象,天下大乱之基也。杜渐防微,用敢望诸当事。且陈君英姿挺秀,学贯中西。皖省地绾南北,每产材武之士,如斯学者,诚叹难能。执事平视同乡诸贤,谅有同感。远而一国,近而一省,育一人才,至为不易。又焉忍遽而残之耶。特专函奉达,请即饬警厅速将陈君释放。钊与陈君总角旧交,同衿大学。于其人品行谊,知之甚深,敢保无他,愿为左证。……

\begin{flushright}章士钊拜启六月二十二日\end{flushright}

我们对于陈君,认他为思想界的明星。陈君所说的话,头脑稍微清楚的听得,莫不人人各如其意中所欲出。现在的中国,可谓危险极了。不是兵力不强财用不足的危险,也不是内乱相寻四分五裂的危险。危险在全国人民思想界空虚腐败到十二分。中国的四万万人,差不多有三万万九千万是迷信家。迷信鬼神,迷信物象,迷信运命,迷信强权。全然不认有个人,不认有自己,不认有真理。这是科学思想不发达的结果。中国名为共和,实则专制。愈弄愈糟,甲仆乙代,这是群众心里没有民主的影子,不晓得民主究竟是甚么的结果。陈君平日所标揭的,就是这两样。他曾说,我们所以得罪于社会,无非是为着“赛因斯”(科学)和“兑莫克拉西”(民主)。陈君为这两件东西得罪于社会,社会居然就把逮捕和禁锢报给他,也可算是罪罚相敌了,凡思想是没有畛域的,去年十二月德国的广义派社会党首领卢森堡被民主派政府杀了,上月中旬,德国仇敌的意大利一个都林地方的人员,举行了一个大示威以纪念他。瑞士的苏里克,也有个同样的示威给他做纪念。仇敌尚且如此,况在非仇敌。异国尚且如此,况在本国。陈君之被逮,决不能损及陈君的毫末。并且是留着大大的一个纪念于新思潮,使他越发光辉远大。政府决没有胆子将陈君处死,就是死了,也不能损及陈君至坚至高精神的毫末。陈君原自说过,出实验室,即入监狱。出监狱,即入实验室。又说,死是不怕的。陈君可以夺验其言了。我祝陈君万岁!我祝陈君至坚至高的精神万岁!

\subsection{世界杂评}

强叫化前月的初间,日本米价顶贵时候,每石超四十元。日当局有狼狈之状。报纸证言粮食的危机已迫。可怜的日本!你肠将饥断,还要向施主逞强。天下那有强叫化续得多施的理。

研究过激党阿富汗侵印度,俄过激党为之主谋,过激党到了南亚洲。高丽的“呼声革命”正盛吋,亦有过激党参与之说,则已到了东亚。过激党这么厉害!各位也要研究研究,到底是个什么东西?切不可闭着眼睛,只管瞎说,“等于洪水猛兽”“抵制”“拒绝”等等的空话。一光眼,过激觉布备了全国,相惊而走,已没得走处了!

实行封锁前月巴黎高等经济会议议决,实行封锁匈牙利,说理直到匈政府宣言遵从民意时为止。这要分两层观察,一、协约国看错了匈政府与匈国民志愿不合。匈政府与匈国民之少数有产阶级,绅士阶级,志愿不合是有的,若与大多数无产阶级,平民阶级,没有志愿不合的理。因为匈政府,原是他们所组织的。二、实行封锁,这是帮助过激主义的传播。吾恐怕协约国也会要卷入这个漩涡。果然,则这实行封锁,真是“罪莫大焉”了。

证明协约国的平等正义德国复文和会,要求德国陆军减少之后,协约国也须同减。这话谁人敢说错了?协约国满嘴的平等主义,我们且看协约国以后的军备如何?就可求个证明。

阿富汗执戈而起一个很小的阿富汗,同一个很大的海上王英国开战,其中必有重大原因。但据英国一方面的电传,是靠不住的。土耳其要被一些虎狼分吞了。印度舍死助英,赚得一个红巾照烂给人出丑的议和代表。印民的要求是没得允许。印民的政治运动,是要平兵力平压。阿富汗是个回教国,狐死冤悲,那得不执戈而起?

来因共和国是丑国协约国要划来因流域为自己挡敌的长城,必先使之脱离德国的关系,别成一国。听说已在威萨登成立临时政府,一位这登博士做总统。这位道登阵士不知果然高兴到甚么样?金人立了刘豫,契丹立了石敬塘,我们中国也曾有几个这样的国呢。

好个民族自决波兰捷克复国,都所以制德国的死命,协约因尽力援助之,称之为“民族自决”。亚刺伯有分裂士耳其的好处,故许他半自立。犹太欲在巴力斯坦复国,因为于协约国没大关系,故不能成功。西伯利亚政府有攻击过激党的功绩,故加以正式承认.日本欲伸足西伯利亚,不得不有所示好,故首先提议承认。朝鲜呼号独立,死了多少人民,乱了多少地方,和会只是不理。好个民族自决!我们认为直是不要脸!

可怜的威尔逊威尔逊在巴黎,好象热锅上的蚂蚁,不知怎样才好?四围包满了克勒满沙,路易乔治,牧野伸显,欧兰杜一类的强盗。所听的,不外得到若干土地,收赔若干金钱。所做的,不外不能伸出己见的种种会议。有一天的路透电说:“威尔逊总统卒已赞成克勒满沙不使德国加入国际同盟的意见”。我看了“卒已赞成”四字,为他气闷了大半天。可怜的威尔逊!

炸弹暴举人人知道很文明很富足的美国,有“炸弹暴举”同时在八城发生。无政府党蔓延甚广。炸弹爆炸的附近,有匿名揭贴说,“阶级战争”业已发生,必得国际劳动界完全胜利,始能停止,炸弹往往埋藏在一些官员的住宅,屋顶上发现人头。可怕可怕!我只挂牵官员人家的一些小姐小孩子,他们晚上如何睡得着?议院里一些钱多因而票多票多因而当选的议员,还在那里痛诋暴动者,通过严惩案。我正式告诉诸位,诸位的“末日审判”将要到了!诸位要想留着生命,并想相当的吃一点饭,穿一点衣,除非大大的将脑子洗洗,将高帽子除下,将大礼服收起,和你们国里的平民,一同进工厂做工,到乡下种田。

不许实业专制美国工党首领戈泊斯演说曰,“工党决计于善后事业中有发言权,不许实业专制。”美国为地球上第一实业专制国,托辣斯的恶制,即起于此。几个人享福,千万人要哭。实业越发达,要哭的人越多。戈泊斯的“不许”,办法怎样?还不知道。但既有人倡言“不许”,即是好现象。由一人口说“不许”,推而至于千万人都说“不许”,由低声的“不许”,推而至于高声的很高声的狂呼的“不许”,这才是人类真得解放的一日。

割地赔偿不两全德国答复协约国,说,如失去西里细亚及萨尔煤矿,则无力行赔偿。我料协约国听了一定很烦脑。何以故?地可割,赔偿也可得,最为两全。据德国的说,两样便成了反比例,如之何不烦脑?虽然,奉劝协约国的衮衮诸公,天下那有两全的好事!

为社会党造成流血之地奥总代表任纳博士答复和会,说,“奥国今已坐食其较前大减之资本,若再加以摧残,必为社会党造成流血之地”,蠢哉任纳博士,你还不知道协约国一年以来之真目的,你专为造成社会党流血之地吗?

彭斯坦德博士彭斯坦演说曰,“媾和条件”乃野蛮战争的结果,德国最宜负责,和约条件十九为必要的”。我们固然反对协约国的强迫和约,但博士这话,系专对野蛮战争而发,听了倒很爽快。

各国没有明伦堂康有为因为广州修马路,要拆毁明伦堂,发了肝火。打电给岭伍,斥为“侮圣灭伦。”说,“遍游各国,未之前闻”。康先生的话真不错,遍游各国,那里寻得出什么孔子,更寻不出什么明伦堂。

什么是民国所宜?康先生又说,“强要拆毁,非民国所宜”。这才是怪!难道定要留看那“君为臣纲”,“君君臣臣”的事,才算是“民国所宜”吗?

大略不是人邓镕在新国会云,“尊孔不必设专官,节省经费”。张元奇云,“内务部祀孔,由茶房录事办理.次长司长不理,要设专官。”内务部的茶房录事,大略不是人。要说是人,怎么连祀孔都不行呢?我想孔老爹的官气到了这么久的年载,谅也减少了一点。

走昆仑山到欧洲张元奇又说,“什么讲求新学,顺应潮流,本席以为应尊孔逆挽潮流。”不错不错!张先生果然有此力量,那么,扬子江里的潮流。会从昆仑山翻过去,我们到欧洲的,就坐船走昆仑山罢。

\subsection{湘江杂评}

好计策一个学校的同学对我说,我们学校里办事人和教习,怕我们学到了他们还未学到的新学说,将图书室看×了。外面送来的杂志新闻纸和书籍,凡是稍新一点的,都没得见。我听了为之点首叹服。他们的计策真妙!岂仅某学校,通湖南的学校,千篇一律都象联了盟似的。

摇身一变一些官僚式教育家,为世界的大潮卷急了,不提防就会将他们的饭碗冲破。摇身一变,把前日的烂调官腔,轻轻收拾。一些其有所感而改变的,很可佩服。一些则是假变,容易露出他们的马脚。这类人我很为他羞!很为他危!

我们饿极了我们关在洞庭湖大门里的青年,实在是饿极了!我们的肚子固然是饿,我们的脑筋尤饿!替我们办理食物的厨师们,太没本钱。我们无法!我们惟有起而自办!这是我们饿极了哀声!千不要看错!

难道走路是男子专有的一个女学校里的办事人,把学生看做文契似的收藏起来,怕他们出外见识了甚么邪样。新青年一类的邪书,尤不准他们寓目。此次惊天动地的学生潮,北京的女学生聚诉新华门。贫儿院的小女孩子,愿到监狱替男学生抵罪。这个女学校的学生独深闭固拒,一步也不出外,好象走路是男子专有似的。

哈哈!青岛问题发生,湖南学生大激动,新剧演说,一时风行。有一位朋友对我说,一位老先生,因为他的儿子化装演剧,气得了不得。走到学校问先生,开口便说,“我的命运如何这么乖?养大的儿子竟做出那么下流事”?我听了这话,忍不住卧的一声,哈哈!

女子革命军或问女子的头和男子的头,实在是一样。女子的腰和男子的腰实在是一样。为什么女子头上偏要高竖那招摇畏风的髻?女子腰间偏要紧缚娜拖泥带水的裙?我道,女子本来是罪人,高髻长裙,是男子加于他们的刑具。还有那脸上的脂粉,就是黔文。手上的饰物,就是桎梏。穿耳包脚为肉刑。学校家庭为牢狱,痛之不敢声,闭之不敢出。或问如何脱离这弊?我道,惟有起女子革命军。


\section{湘江评论第二号}
\datesubtitle{(一九一九年七月二十一)}



\subsection{西方人事述评}

德意志人沉痛的签约

签约之前:败而不屈的德意志的代表兰超等于(五月初旬)到巴黎。(五月七日)在凡尔赛宫举行很庄严的和约交付礼。德代表的态度很倨傲。克勒满沙站起声述其开会词。德总代表兰超,则坐诵其如下之演说词——

德国军事破裂,德国失败的程度,自己明白。但这回欧战,负责的不仅德国,全欧都与有罪。因五十年以来,欧洲各国的帝国主义,实贻毒于国际局势。德国战中的罪行,固不可讳,战事的时候,人民的天良,为感情所蔽,故有罪行。然自去年十一月十一日以后,德国没与战事的人,多死于封锁的影响,协约国亦冷淡视之。威总统十四条大纲,为全世界所赞助,协约国业已声明依照此项大纲而立和约,那么,德国当不致于全没救护。国际同盟,各国都让加入,不能将德国丢在外边。德国愿以好意的精神,研究和约。……

和约为灰黄色封面一大册,和会秘书长杜斯玛,捧了交到兰超手中。兰超回到寓舍,晚餐的时候,默无一言。晚餐毕,即使人翻译和约,于晨间三点译成,送到兰超寝室。兰超看到天明方毕。另外录出几份,派专差送到柏林。八日德内阁会议许见。

内阁总理施特满,向考虑协约委员会演说——

和约条件,简直是宣告德国死刑。政府必以政治的沉静态度,讨论这可厌而狂妄的公文,……

随将和约条件,电告各联邦政府,请他们表示意见。因为感受很深的痛苦,特命公众停止行乐一星期。仅许剧院演唱和这痛苦极没相同的悲刷。股票交易所,因感受痛苦的印象,停闭三日。各界人士听得和约会要签字,皆为作怒,群相讨论拒绝签字的后患,甚至没有一人想劫或可受纳此项条件的。柏林各报一致(言尔)诋,有的说,“和约的苛刻远过最消极的预料,这系狂暴无知识的制品,若不能修改,只有用‘否’字答他。”有的说“我们如签定这约,实是屈于武力。我们的心中,应坚决拒绝。”惟独立社会党的机关报,则主张签约说,“从经验看来,拒绝徒增后患。”这时候最可注意的,是德国政党的态度。多数时会党的政府派,是不主张签字的。民治党和中央党也是这样。只有独立社会党不然。(十二日)独立社会党通过决议案,主张接受和约,并说“德现政府恢复显武主义的行为,使别人坚其对德的疑惧。德国舍屈于强迫签字,没有办法。俄德和约,及德罗和约,均没多久的寿命就取消了。凡尔赛和约,也未尝不可以革命的发展取消他。”我们为德国计,要想不受和约,惟有步俄国和匈牙利的后尘,实行社会的大革命。协约国最怕的就在这一点。俄罗斯,匈牙利,不派代表,不提和议,明目张胆地对抗协约国,协约国至今未如之何。向使去冬德国广义派社会党的社会革命成了功,则东联俄而南结奥,更联合匈牙利和捷克,广播其世界革命主义,或竟使英法美久郁的社会党,起而响应,协约园国府还食得下咽吗?独立社会党和广义派社会党,本是一党而分为二,他的议论如此,本不足怪。用革命的发展取消和约这话正不要轻看呢。

同日德因会讨论媾和条件。施特满演说――

今日为德国人生死关头!我们必须团结一致!我们除谋使国家生存,无旁的责任!德国不图进行其国家主义的梦想,并没争权的问题。于今人人的喉咙中间都觉有手塞住他的呼吸!人类的尊严,现付于诸君的手中,以保持他!

(十四日)兰超致克勒满沙一个履文,内容的大要如下——

和约中关于领土的条款,是使德国失去其×关重要的生产土地。各地薯芋的收成,将减百分之二十一。煤减三分之一。铁减四分之三。锌减五分之三。德国既因失去殖民地和商船,使经济成了麻木不仁。今又不能得充分的原料,势将被毁到极大的程度。同时输入的粮食必将大减。依赖航务和商业为生的数百万人,德国政府不能将工事和粮食供给他们则势不得不到国外求生,而重要的国家。多禁止德国移民入境。故签定和约,不啻向数百万德人宣告死刑……

兰超于上述的牒文之外,更以牒文两迈致克勒满沙。第一通的大要说,“协约国占德土地,和威总统宣布的主义不合。”第二通的乃关于赔偿条款提出抗议。谓“德国愿赔偿,但不是因为负了战争的原故。”我们看德国的抗议,大家注重(一)不独负战争责任。(二)不愿失去原料所从出的土地。其他各项,虽有抗议,但不是最重要的处所。

(十三日)晚上,柏林有大举示威。多数社会于示威时,起坛演说谓,“和约条件,较罗马施于加萨臣的,尤为刻毒而可羞。”群众游行各街,止于协约委员团所住的亚特伦旅馆前面。有人向众演说,其势汹汹,欲攻旅馆,为警察所阻。到内阁总理屋前,施特满氏临窗演说。又有人民一大队,于薄暮时候,唱歌到亚特伦旅馆,大呼“推翻强暴的和局。”“克勒满沙皆亡”。“与英伦皆亡。”众又到施特满处,请他演说,施氏讲到威尔逊总统十四公纲时,众忽大呼“与威尔逊皆亡。”这日柏林和乡间独立社会党开会,有四十处。

(十九日)柏林某报载有社会党领袖彭斯坦博士的演说,谓“非常苛刻的和约条件,非完全出于激怒与仇念。实德国政策既不能见信于人,则当然受此待遇。一切破坏咎在德国。德国之履行各项要求,不过补偿他们前所寄予人的而已。我很不以一般人士所发激烈的演说为然。告诉他们!不可再具一九一四年八月四日的气焰!”这于德国热烈的反对声中,算是一瓢凉水。

(二十日)兰超致书和会,要求改正审查及讨论和约的期限,(二十二日)克勒满沙复书,允许展长期限,至五月二十九日为止。(二十三日)晚德全权大使起程往斯巴,将和来自柏林的阁员数人会晤,决定一切。(二十四日)斯希特芒,欧士白格,从柏林乘车到斯巴,兰超及委员十六人也到,即开一极长的会议,斯希特芒主席,通过德国的反提案。会毕,政府委员回柏林,兰超等回凡尔赛。

(五月二十七日)。德国有答案交付和会,答案的第一部分,要点如下一一

(一)德国承认减少军队到十万人。

(二)交出巨大的军舰,而保留商船。

(三)反对关于东边土地的决定。要求于东普鲁士区中,举行庶民大会。

(四)承认丹齐为自由口岸。

(五)要求协约国,在签字四个月后,撤退军队。

(六)要求加入国际同盟。

(七)坚欲取得代替殖民地的权利。

(八)赔偿总数,不得过十万兆马克。

(九)拒绝引渡凯撒及其他人物。

(十)德国须有重新经商海外的权利。

德国答案的第二部分,亦有如下的要点――

(一)过渡时代,须维持大军,以保治安。

(二)须许德人开公民大会,讨论土地割让问题。并许奥人以加入德国的便利。


(三)拒绝割让西里细亚上部。

(四)不承认俄国有享取赔偿的权利。

(五)无赔偿意、门、罗、波兰等国的义务。

右之德国答案,四大国代表为长久讨论后,提出答复文,将德国所约议的,逐件驳复。全文很长。不外说德国战争责任万难推诿,德国必须尽其能力赔偿损失,必须交出戎首,和战时行暴的人,用法惩治。必须于数年内受特别的约束。凡协约国所持以构成和约的根本主义,万难更易。惟对于德国的实际建议,可以让步云云。

(五月二十九日以后)又因德代表的请求,屡次展缓签约日期。最后允展至六月二十八日为止。六月前半月的光阴,全为着往复反议事占去。

至(六月十八日)德代表团,乃由法京回德,一致告诫德内阁,拒绝签约。德内阁乃准备在韦马召集国会,议此次重大问题。此时协约方面,早做军事准备。一俟德国有不签字的表示,即行进军。德国已处于不能不签字的情势了。

(六月二十二日),协约国对德的“最后复文”于这日送达德代表,限德国以五日承受和约。“最后复文”内,述可以让步的条件,如下一一

(一)西里细亚上部,实行民众投票。

(二)西普鲁士边界,重行划定。

(三)德军暂增至二十万人。

(四)德国宣布愿于一个月内将被控破坏战时法律的人名单开出。

(五)修改关于财政问题的细则。

(六)以德国履行义务为条件,保证德国为将来国际同盟的一员。

施特满内阁知和约不能再有挽回,遂决计引退。

(二十二日)德新内阁组织成立。国务总理巴安氏,外交穆勒氏,财政欧士白格氏,内务达维特氏,陆军拿斯奇氏,殖民贝尔氏,邮电格莱斯勃资氏,劳动森士南氏,工程斯利奇氏,公业经济惠塞尔氏,国库夏勒氏,粕食斯密氏。巴安氏及诸阁员,多属多数社会党,本系前内阁的同调,在这回外交紧急声中,出当此签定和约的难局。新内阁既成,已可决其是预备签约的了。施特满退职。施特满内阁所委任的媾和代表,当然随着退职,于是德国讲和代表团易人。新代表团即以新内阁中的外交总长穆勒,邮电总长格莱斯勒资等组织而成。

此时国会业已在韦玛召集,巴安氏即赴国会,作很沉痛的演说,极言加入新政府的痛苦。恳请国会确立主张,否则战事将屡发作。巴安氏曰,“我特于自由的日耳曼最后一次,起抗此强暴破坏的和约!起抗此自决权利的假面具!起抗此奴隶德人的手段!起抗此妨害世界和平的新器!”国会乃于“反对”“赞成”的喧哗声中通过签约动议。

签约的动议通过,二十三日巴安再赴国会,申述无条件签约的必要,其演词谓,“战败的国家,身魂受世界的凌辱!吾人姑且签定和约。吾人一息尚存,终望损害吾人荣誉的人,有一日身受报应,”这时候右党提出抗议。乃付表决。结果证实准许签约。议长贵里巴贽氏起立发短节的演说,“以不幸的祖国,委托于慈悲的上帝!”且谓“各党领导,允宣告军界,全国希望海陆军树先己牺牲的模范,辅助劳工,重造祖国!”关系全世界安危的德国签字,在一场非常惨痛的演说声中,完全决定。德意志人的大纪念,有史以来,当没有过于这日了:

签字案既经国会通过,德新代表团乃到巴黎,致“允可签约的牒文”于和会。牒文的大要说,一一日耳曼民国政府,知协约国决计以武力强迫日耳曼承受和约条件。此项条件,虽没有重大的意味,然实老在剥夺日耳曼人民的荣誉。日耳曼政府虽屈服于作势的武力,但关于从古未闻离背公道的和约条件。”

右文既布,各国的欢忭,自不可言。至(二十八日)而最后展缓的满期已到。于是凡尔赛宫中,仍有亘古未闻的大签约一举。

签约之际一千九百十九年六月二十八日午后三点五分,凡尔赛宫中开会。在宫中设高坛,甚为庄严。协约国全权代表首先会集。次为德国全权代表,只到外总长穆劝和交通总长裴尔,其余均不愿到。克勒满沙主席,首发短简宣言,谓:“协约国和共同作战国政府,均赞成媾和条件。今加签字,表示彼等忠诚依守庄严的了解。”继乃“请日耳曼民国代表首先签字。”德代表所坐席次忽发大声,“德意志!”“德意志!”克勒满沙于是乃改称“德意志”。德代表即起立在约上签名,裴尔氏首先签之。时为午后三时十二分,园中喷泉四射,炮声大作,当德代表回到坐处的时候,会场皆露喜容。次为美国签字。次为英国签字。次为法国签字。次为意国签字。次为日本签字。最后签字的为捷克斯拉夫民国。三时三十五分签字完毕。克勒满沙氏宣布散会。

签约之后,当德国国会允许签约的消息传布,德国全国自即有爱国的示威运动。群众列队唱战歌,国歌,欢呼致敬于年老的统兵员,各报对于裁判德皇问题,表示极大的忿怒。有一报恳请一九一四年的陆军军官,表示如德皇受裁判,也愿受协约国的裁判。并请组织团体,或须入荷兰,保护德皇。各地暴动罢工事情,接续而起。及(二十八日)和约签字的消息传到柏林,柏林某报即载出一文,谓“德人终必报一九一九年的耻辱!”为政府禁止发行。(二十九日)各报皆有“黑线”,表示哀痛。各报皆载有极悲观的评论。柏林及各地铁路工人及电车工人罢工。柏林城里的运输机关全停。亨堡等处出了乱子。全国的罢工,有扩张形势。

评论我叙签约。我争叙感国的签约。我叙德国签约,单注重其国民精神上所感痛苦的一点。是什么意思?原来这回和约,除却国际同盟,全是对付德国的。德国为日耳曼民族,在历史上早蜚声誉,有一种崛强的。一朝决裂,新剑发硎,几乎要使全地球的人类挡他不住。我们莫将德国穷兵黩武,看作是德皇一个人的发动。德皇乃德国民族的结晶。有德国民族,乃有德皇。德国民族,晚近为尼釆,菲希特,颉德,泡尔生等“向上的”“活动的”哲学所陶铸。声宏实大,待机而发。至于今日,他们还说是没有打败,“非战之弊”。德国的民族,为世界最富于“高”的精神的民族。惟“高”的精神,最能排倒一切困苦,而惟我实现其所谓“高”。我们对于德皇,一面恨他的穷兵黩武,滥用强权。一而仍不免要向他洒一掬同情的热泪,就是为着他“高”的精神的感动。德国的民族,他们败了就止了。象这样的屈辱条件,他们也忍苦承受。他们第一次翻转面目,已从帝国变成了民国。他们的第二次翻转,或竟将民国都不要了。这话我殊敢下一个粗疏的断定。我们且看挡在西方的英法,不是他们的仇敌吗?英法是他们的仇敌,他们的好友,不就是屏障东方和南方的俄、奥、匈、捷和波兰吗?他们不向俄奥匈捷等国连络,还向何处?他们要向俄奥匈捷连络,必要改从和俄奥匈捷相同的制度。俄匈的社会革命成了功,不用说。奥捷也有此趋势,前日电传说捷克已经成了劳兵民国了。德国广义派斯巴达团,去年冬天的猛断举动,和成功仅仅相差一间。爱倍尔政府成立,多数社会党握权,所恃以制服广义派的,全是几个兵,几杆枪。和约成功,兵是要解散了。枪是会要缴出了。那时候政府还恃着什么?德国工商业的大毁败,要重造起来,不得不仰赖出力的劳动者。以后政府所应做的大事,就是向劳动者多多的磕头。而广义派的武器,不是别的,就是这些劳动者。故我从外交方面的趋势去考虑,断定德国必和俄奥匈连合,而变为共产主义的共和国。又从内治方面的趋势去考虑,也可以做同样的断定。

一千九百一十九年以前,世界最高的强权在德国。一千九百一十九年以后,世界最高的强权在法国,英国和美国。德国的强权,为政治的强权,国际的强权。这回大战的结果,是用协约国政治和国际的强权,打倒德奥政治和国际的强权。一千九百一十九年以后,×国英国美国的强权,为社会的强权,经济的强权。一千九百一十九年以后设有战争,就是阶级战争,阶级战争的结果,就是东欧诸国主义的成功。即是社会党人的成功。我们不要轻看了以后的德人。我们不要重看了现在和会高视阔步的伟人先生们,他们不能旰食的日子快要到哩!他们总有一天会要头痛!然则这回的和约“其能稳”尚靠不定。若真以“德俄和约”,“德罗和约”的例来推测,恐咱就是早晚的问题。无知的克勒满沙老头子,还抱着那灰黄色的厚册,以为签了字在上面,就可当作阿尔卑斯山一样的稳固,可怜的很啊!

\subsection{世界杂评}

高兴和沉痛克勒满沙在办公室接得德国接受和约的电话,高兴了不得。起身来,和在办公室的阁员及同僚握手。说,“诸君!我之静候这一分钟,已有十九年了!”这话何等高兴。虽然,不第高兴,又含多少沉痛的意思。一千八百七十一年,威廉第一和俾士马克,高踞凡尔赛,接受法国屈服牒文的时候,何等高兴。结果遂酿成此次的战争。虽然威廉第一,俾士马克,不第高兴,又含有多少沉痛的意思。一千八百年至一千八百一十五年,拿破伦蹂躏德意志,分裂他的国,占据他的地,解散他的兵。普王屈服,称藩纳聘。拿破伦何等高兴。结果遂酿成一千八百七十一年的战争。虽然,拿破伦不第高兴,又含多少沉痛的意思。一千七百八十九年至一干七百九十年,德奥为巨擘的神圣同盟军,深恶德国的民主自由,几度蹂法境,围巴黎,结果遂崛起拿破伦,而有蹂躏德国,令德人头痛的事。我们执因果看历史,高兴和沉痛,常相关系,不可分开。一方的高兴到了极点,热一方的沉痛也必到极点。我们看这番和约所载,和拿破伦对待德同的办法,有什么不同?分裂德国的国,占据德国的地,解散德国的兵,有什么不同?克勒满沙高兴之极,即德国人沉痛之极。包管十年二十年后,你们法国人,又有一番大大的头痛,愿你们记取此言。

卡尔和溥仪奥前皇卡尔避居瑞士,某报通讯记者求见,见其侍者。侍眶说:“皇帝的退位,本非得已,故愿望恢复帝制。惟目下暂时隐居,不问政治。”凡做过皇帝的,没有不再想做皇帝。凡做过官的,没有不再想做官。心理上观念的习惯性,本来如此。西洋人做事,喜欢彻底,历史上处死国王的事颇多。英人之处死沙尔一世(一千六百四十八年),法人之处死路易十六(一千七百九十三年),俄人之处死尼哥拉斯第二(一千九百一十八年),都以为不这样不足以绝祸根。拿破伦被囚于圣赫利拿,今威廉第二拟请他做拿破伦的后身将受协约国的裁判,总算很便宜的。避居瑞士的卡尔,和伏处北京的溥仪,国民不加意防备,早晚还是一个祸根。


\section{湘江评论增刊第一号}
\datesubtitle{(一九一九年七月二十一日)}
\subsection{健学会之成立及进行}

健学会以前的湖南思想界湖南的思想界,二十年以来,黯淡已极。二十年前,谭嗣同等在湖南倡南学会,召集梁启超,麦梦华诸名流,在长沙设时务学堂,发刊湘报,“时务报”。一时风起云涌,颇有登高一呼之概。原其所以,则彼时因几千年的大帝国,屡受打击于列强,怨幅惋悔,敬而奋发。知道徒然长城渤海,挡不住别人的铁骑和无畏兵船。中国的老法,实在有些不够用。变法自强的呼声,一时透彻衡云,云梦的大倡。中国时机的转变,在那时候为一个大枢纽。湖南也跟着转变,在那时候为一个大枢纽。

思想变了,那时候的思想是怎样一种思想?那时候思想的中心是在怎样的一点?此问不可不先答于下——

(一)那时候的思想是自大的思想。什么“讲求西学’,什么“虚心考察”,都不外“学他到手还以奉敬”的方法。人人心目中都存想十年二十年后,便可学到外国的新法。学到新法便可自强。一达到自强的目的,便可和洋鬼子背城借一,或竟打他个片甲不回。正如一个小孩受了隔壁小孩的晦气,夜里偷着取出他的棍棒,打算明早跑出大门,老实的还他一个小礼。什么“西学”“新法”相当于小孩的棍棒罢了。

(二)那时候的思想,是空虚的思想,我们试一取看那时候鼓吹变法的出版物,便可晓得一味的“耗矣哀哉”。激刺他人感情作用。内酌是空空洞洞,很少踏着人生社会的实际说话。那时有一种“办学室”,“办自治”,“请开议会”的风气,寻其根柢,多半凑热闹而已。凑热闹成了风,从思想界便不容易引入实际去研究实事和真理了。

(三)那时候的思想,是一种“中学为体,西学为用”的思想,“中国是一个声名文物之邦,中国的礼教甲于万国,西洋只有格致炮厉害,学来这一点便得。”设若议论稍不如此,便被人看作“心醉欧风者泳”,要受一世人的唾骂了。

(四)那时候的思想是以孔子为中心的思想。那时候于政治上有排满的运动,有要求代议政治的运动。于学卫上有废除科举兴办学校,采取科学的运动。却于孔老爹,仍不敢说出半个“非”字。甚至盛倡其“学问要新,道德要旧”的谬说,“道德要旧”就是“道德要从孔子’的变语。

上面所举,全中国都有此就行,湖南在此情形的中间占一位置。所以思想虽然变化,却非透彻的变化,任何说是笼统的变化,盲目的变化,过渡的变化,从戊戌以致今日湖南的思想界,全为这笼统的,盲目的,过渡的变化所支配。

湖南辨求新学二十余年,而后有崭然的学风。湖南的旧学界,宋学、汉学两支流,二十年前,颇能成为风气。二十年来,风气尚未尽歇,不过书院为学校占去,学生为科学吸去,他们便必淹没在社会的底面了。推原新学之所以没有风气,全在新学不曾有确立的中心思想。中心思想之所以不曾确立,则有以下的数个原因:

1、没有性质纯粹的学会。

2、没有大学。

3、在西洋留学的很少。有亦为着吃饭问题和虚条心理,竟趋于“学非所用’的一途,不能持续研究其专门之学。在东洋留学的,被黄兴吸去做政治运动。

4、政治纷乱,没有研究的宁日。

这是湖南新学界中心思想不能确立的原故,即是没有学风原故,辛亥以来,滥竽教育的,大部市侩一流,逞其一知半解的见解造成非驴非马的局势。中心思想,新学风气,可是更不能谈及了。

近数年来,中国的大势陡转,蔡元培。江亢虎,吴敬恒,刘师复,陈独秀等,首倡革新,革新之说,不止一端,自思想,文学,以致政治,宗教,艺术,皆有一改旧观之概。甚至国家要不要,家庭要不要,婚姻要不要,财产应私有应公有,都成了亟待研究的问题。更加以欧洲的大战,激起了俄国的革命,潮流浸卷,自西向东,国立北京大学的学者首欢迎之,全国各埠各学校的青年大响应之,怒涛澎湃,到了湖南而健学会遂以成立。

\textbf{健学会之成立}

六月五日,省教育会会长陈润霖君邀集省城各学校职教员徐特立,朱剑凡,汤松,蔡湘,钟国陶,杨树达,李云杭,向绍轩,彭国君,方克刚,欧阳鼐,何炳麟,李景桥,赵翌等发起健学会。在楚怡学校开会。今录某报所载陈润霖君报告组织学会的意旨于下一一

兄弟前次到京,偶有感能,深抱乐观。象四年前,北京大学学生以做官为唯一目的。非独大学为然,即大学以外之学生,亦莫不皆然。前次居京,所见迥然不同。大学学生思潮大变,皆知注意人生应为之事,其思潮已多表露于各种杂志月刊中。因之各校学生,亦顿改旧观,发生此次救国大运动。其致此之故,则因蔡孑民先生自为大学校长以来,注入哲学思想,人生观念,使旧思想完全变掉。或该认学生救国运动为政客所勾引,而不知突出学生之自动,及新旧思潮之冲突也。盖自俄国政体改变以后,社会主义渐渐输入于远东。虽派别甚多,而潮流则不可遏抑。即如日本政府,从来对于提倡社会党人,苛待残杀,不遗余力,而近日竟许社会党人活动。如吉野博士等,则主张实行国家社会主义,以和缓过激主义,顺应世界之趋势,从看将日本政体改变为英国式虚君制。

于此可知,世界思潮改变之速,势力之大矣!我国新思潮亦甚发展,终难久事遏抑,国人当及时研究,导之正轨,国人等组织学会,在釆用正确健学之学说而为彻底之研究……

这日开会,听说街有朱剑凡君主张“各除成见,研究世界新思想,服从真理”的演说,向绍轩君主张“采用国家社会主义”的演说。在湖南思想界不可不谓空前的创闻。今录出该会所发表的会则如下一一

(一)本会同志组合,以输入世界新思潮,共同研究,择要传播为宗旨。

(二)本会定名为健学会。

(三)会所暂定长沙储英源楚怡小学校。

(四)入会者须确有研究学术之志愿,经本会会友一人以之介绍,得为本会会员。

(五)关于输入新思潮之方法一一

(1)凡最近出版之图书杂志,由本会随时搜集,以供会员阅览。会员所藏书报,得借给本会会员阅览。其有愿捐入本会者,本会尤为欢迎。

(2)延请海内外同志,随时调查,通信报告。

(3)介绍名人谈话。

(六)关于研究之方法一一

(1)研究范围,大半为哲学,教育学,心理学,论理学,文学,美学,社会学,政治学,经济学诸问题,会友必分认一门研究。

(2)重要之问题,由会友共同研究。

(3)会员有愿习外国语者,由本会会友传授。

(七)关于传播之方法一一

(1)讲演。分定期,临时两种。定期讲演,每周日曜日午前八时至十时。由会友轮流担任。讲员及讲题均于前周日曜日决定。讲友须预备讲稿,交由本会汇刊。临时讲演,凡有主要演题,或由会友,或请名人讲演,另觅地点,择期举行。

(2)出版。

 (八)本会设会计,管理图书各一人。其他会务由会友共同负责。每次开会推会友一人协时主席。 
(九)会友应守之公约如左一一

(1)确守时间.。

(2)富于研究的精神。

(3)学问上之互助。

(4)自由讨论学术.

(5)不尚虚文客气,以诚实为主。

(十)会员年纳两元以上之会金,有能特别筹助经费者,本会极为欢迎。

(十一)本会遇有重要事项,必须讨论时,得于定期讲演后,临时通告全体,举行会议。

会则中的(五)(六)(七)(九)极为重要,(九)之富于研究的精神,所以破除自是自满的成见,立意很好。尚生于研究的精神之后,继之以“批评的”精神。现代学术的发展,大半为各人的独到所创获。最重的是“我”是“个性”,和中国的习惯,非死人不加议论,著述不引入今人的言论恰成一反比例。我们当以一己的心思,居中活动。如日光之普天照耀,如探海灯之向外扫射,不管他到底是不是(以今所是的为是)合人意不合人意。只求合心所安合乎箕理才罢。老先生最不喜欢的是狂妄。岂不知古今最确的原理,伟大的事业邵是系一些被人加上狂妄名号的狂妄人所发明创造来的。我们住在这复杂的社会,诡诈的世界没有批评的精神就容易会做他人的奴隶。其君谓中国人大半是奴隶,这话殊觉不错。(九)之自由讨论学术,很合思想自由,言论自由的原则。人类最可宝贵最堪自乐的一点却在于此,学术的研究最忌演释式的独断态度。中国什么“师严而后道尊”。师说“道统”,“宗派”都是害了独断态度的大病。都是思想界的强权,不可不竭力打破。象我们反对孔子,有很多别的理由,单就这独霸中国,使我们思想界不能自由,郁郁做两千年偶象的奴隶,也是不能不反对的。

健学会之进行健学会进行事项,会则所定大要系研究及传播最新学术。现在注重于研究一面,闻已派人到京沪各处,釆买书籍,新闻纸和杂志。在省城设一英语学习班,使会员学习英语,为直接研究四方学术的预备。有年在四五十的会员都喜欢学习。又设一演讲会由会员轮流发表意见,实行知识的交换。官气十足的先生们,忽然屈尊降贵,虚心研究起来。虽然旁人尚有不满意的处所,以为官气还有十分五、六,讲演要多采用命令式和训话式。更有谓他们是青叶上青虫的体合作用。象这样的求全责备.我们为何以下坏。在这么女性纤纤,暮气重重的湖南有此一举,颇足山幽囚而破烦闷。东方的曙光,空谷的足音,我们正应拍掌欢迎,希望他可作“改造湖南的张本”看他们四次讲演的问题,如“国人误谬的生死观”。怎样做人”,“教育和白话文”,“釆用杜威教育主义”,都可谓能得其要。倘能尽脱习气采用公开讲演,尽人都可以听,则传播之外,得益之大,当有不可计量的了。

\begin{flushright}(《湘江评论》临时增刊第一号)
\end{flushright}


\section{湘江评论第三号}
\datesubtitle{(一九一九年七月二十八日)}

\subsection{世界杂评}

畏德如虎的法兰西法国于德国畏惧他如虎狼。德国这么大败,法国尚畏惧得很。割萨尔煤矿,划来因左岸独立,毁希里哥伦炮台,力波兰独立以蹙其东陲,助捷克独立以阻其南出。日耳曼奥地利也并归德国,则不惜破坏民族自决主义,多方以妨之。殖民地,陆海天空军备,则多方以消之。商船亦须交出大部,以阻其海外贸易之恢复。这样也算够了,还不止此。又向英美两国,请求保卫。前日电传,威尔逊于离法以前,签订一约。系证将来法国一受攻击,美国当起而援助。劳合乔治亦以英国名义,签定一同一性质的条约。此意何等深刻!何等惨淡!籍非法国有不可告之大缺点,何至有这样的畏惧。法兰西民族素负豪气,何至竟象妇人孺子,斤斤乞人保护。我觉得这不是法兰西的好现象!

和约的内容斯末资将军说:“我之签订和约,非因和约乃满意的文件。为结束战争起见,不得不签订他。”又说:“新生活,人类大主义的胜利,人民趋向于新国际制度和优善世界,所抱如此希望之践行,象这样的约言,均没有截上和约。于今只有国民心腔里抒发义侠和人道的新真意,乃能解决和会里政治家困难而止的问题。”又说:“我很以和约里取消黩武主义,仅限于敌国为憾。”斯末资是英国一个武人,是手签和约的一个人,他于签约后所发议论是这样,我们就可想见那和约的内容。

日德密约巴黎的路透电说:“近今外间又有日德密约的谣传。密约是什么东西?还有什么妄人想发现于今后的国际间么?日德密约更是什么东西?日俄密约,为列宁政府所宣布了,不但没成,反丢了脸一大抉。日英法密约成了,我们的山东就要危险。什么日德密约,前年也谣传了多次。据说一千九百一十七年,德国允许日本自由处置荷兰的殖民地,爪哇苏门答腊在内,为英政府听至,告诉了荷兰,阴谋方止。我们应知道日本和德国,是屡次寻求未遂的狗男女,他们虽未遂,那寻奸的念头,是永远不×打断的。日本的强权政府军阀浪人不割除,德国的爱贝尔政府不革命,娼夫和淫妇,还未拆开,危险正多呢。

政治家斯末资云:“惟人民的新真意,乃能解决和会里政治家困难而止的问题。”人民的真意,和政治家的见解,何以这么不相同?政治家何以这么畏难?人民何以这么不畏难?这里面果有一层解释呢?我自来疑惑所谓“政治家”,怕英不是一种好东西?我如今获得了证据。巴黎和约签订后,路易乔治回到英国,在下院演说道:“我们英国,得到许多成功是我们伟大国民团结兴奋的动力。我们于今欢欣鼓舞,但不要存着祸患业已过去的妄念。已使我们获胜的精神,仍要保持,以应付将来事件。”我们不要消费精力于彼此相争。这就是政治家的大本领,这就是政治家的大魔力。不要浪费精力于彼此相争。就是说道,你们人民不要拿着生活痛苦,国民真意,种种无聊问题,和我们政府为难。那些问题都小,都不关痛痒。将来寻着事端,我们还要和别国打仗。爱国,兴奋,团结,对外,是最重要没有的。我正式告诉路易乔治这一类的政治家,你们所说的一大篇,我们都清白是“鬼话”,是“胡说”。我们已经醒了。我们不是从前了。你们且收着,不要再来罢。

\subsection{湘江杂评}

不信科学便死两星期前,长沙城里的大雷,电触死了数人,岳麓山的老树下一个屋子里面,也被雷触死了数人。城里街渠污秽,雷气独多,应建高塔。设避雷针数处。老树电多,不宜在他的下面第屋,这点科学常识,谁也应该晓得,长沙城里的警察,长沙城里三十余万的住民,没一人有闲工夫注意他。有些还说是“五百蛮雷,上天降罚”。死了还不知死因。可怜!

死鼠鼠是瘟疫发生的一个原因,长沙城里到处看见死鼠,张眼望警察,警察却站在死鼠的旁边,早几年的长沙城,都没有这个样子,警察先生们,还是请你们注意点。

\section{民众的大联合(一)}
\datesubtitle{(一九一九年七月二十一日)}



国家坏到了极处,人类苦到了极处,社会黑暗到了极处。补救的方法,改造的方法,教育,兴业、努力、猛进。破坏,建设,固然是不错,有为这样根本的一个方法,就是民众的大联合。

我们竖看历史,历史上的运动不论是那一种,无不是出于一些人的联合。较大的运动,必须有较大的联合。最大的运动,必有最大的联合。凡这种联合,遇有一种改革或一种反抗的时候,最为显著。历来宗教的改革和反抗,学术的改革和反抗,政治的改革和反抗,社会的改革和反抗,两者必都有其大联合,胜负所分,则看他们联合的坚脆,和为这种联合基础主义的新旧或真妄为断。然都要取联合的手段,则相同。

古来各种联合,以强权者的联合,贵族的联合,资本家的联合为主。如外交上各种“同盟”条约,为国际强权者的“联合”。如我国的什么“北洋派”、“西南派”,日本的什么“萨藩”“长藩”,为国内强权者的联合。如各国的政党和议院,为贵族和资本家的联合。(上院至元老院,故为贵族聚集的穴巢,下院因选举法有财产的限制,亦大半为资本家所盘踞)至若什么托辣斯(钢铁托辣斯,煤油托辣斯……)什么会社(日本邮船会社,满铁会社……)则纯然资本家的联合。到了近世,强权者、贵族、资本家的联合到了极点,因之国家也坏到了极点,人类也苦到了极点,社会也黑暗到了极点。于是乎起了改革,起了反抗,于是乎有民众的大联合。

自法兰西以民众的大联合,和王党的大联合相抗,收了“政治改革”的胜利以来,各国随之而起了许多的“政治改革”。自去年俄罗斯以民众的大联合,和贵族的大联合,资本家的大联合相抗,收了“社会的改革”的胜利以来,各国如匈、如奥、如捷,如德,亦随之而起了许多的社会改革。虽其胜利尚未至于完满的程度,要必可以完满,并且可以普及于世界,是想得到的。

民众的大联合,何以这么厉害呢?因为一国的民众,总比一国的贵族资本家及其它强权者要多。贵族资本家及其他强权者人数既少,所赖以维持自己的特殊利益,剥削多数平民的公共利益者,第一是知识,第二是金钱,第三是武力。从前的教育,是贵族资本家的专利,一般平民,绝没有机会去受得。他们既独有知识,于是生出了智愚的阶级。金钱是生活的谋借,本来人人可以取得,但那些有知识的贵族和资本家,整出什么“资本集中”的种种法子,金钱就渐渐流入田主和老板的手中。他们既将土地和机器,房屋,收归他们自己,叫作“不动的财产”。又将叫作“动的财产”的金钱,收入他们的府库(银行),于是替他们作工的千万平民,仅只有一佛朗一辨士的零星给与。做工的既然没有金钱,于是生出了贫富的阶级。贵族资本家有了金钱和知识,他们即便设了军营练兵,设了工厂造枪。借着“外侮”的招牌,使几十师团,几百联队地招募起来。甚者更仿照抽丁的办法,发招牌,明什么“征兵制度”o于是强壮的儿子当了兵,遇着问题就抬出了机关枪,去打他们懦弱的老子。我们目看去年南军在湖南败退时。不就打死了他们自己多少老子吗?贵族和资本家利用这样的妙法,平民就不敢做声,于是生出了强弱的阶级。

可巧他们的三种法子,渐渐替平民偷着学得了多少。他们当作“枕中秘”的教科书,平民也偷着念了一点,便渐渐有了知识。金钱所以出的田地和工厂,平民早已窟宅其中,眼红资本家的舒服,他们也要染一染指。至若军营里的兵士,就是他们的儿子,或是他们的哥哥,或者是他们的丈夫。当拿着机关枪对着他们射击的时候,他们便大声地唤。这一阵唤声,早使他们的枪弹,化成软泥。不觉得携手同归,反一齐化成了抵抗贵族和资本家的健将。我们且看俄罗斯的貌貅十万,忽然将惊旗易成了红旗,就可以晓得这中间有很深的道理了。

平民既已将贵族资本家的三种方法窥破,并窥破他们实行这三种是用联合的手段。又觉悟到他们的人数是那么少,我们的人数是这么多。便大大地联合起来。联合以后,有一派很激烈的,就用“以其人之道,还治其人之身”的办法,同他拚命的捣蛋。这一派的首领,是一个生在德国的,叫作马克思。一派是较为温和的,不想急于见效,先以平民的了解入手。人人要有点互助的道德和自愿的工作。贵族资本家,只要他回心向善能够工作,能够助人而不箐人,也不必杀他;这一派人的意思,更广、更深远,他们要联合地球的一周,联合人类作一家,和乐亲善一一不是日本的亲善一一共臻盛世。这派的首领为一个生于俄国的,斗作克鲁泡特金。

我们要知道世界上的事情,本极易为。有不易为的,便是因子历史的势力一一习惯一一我们倘能齐声一呼,将这个历史的势力冲破,更大大的联合,遇着我们所不以为然的,我们就列起队伍,向对抗的方面大呼。我们已经得了实验。陆荣廷的子弹,永世打不到曹汝霖等一班奸人,我们起而一呼,奸人就要站起身来发抖,就要拚命的飞跑。我们要知道别国的同胞们,是乃常用这种方法,求到他们的利益。我们应该起而仿效,我们应该进行我们的大联合!

\begin{flushright}——原载《湘江评论》第二期\end{flushright}


\section{民众的大联合(二)}
\datesubtitle{(一九一九年七月廿八日)}



以小联合作基础

上一回本报,已说完了“民众的大联合”的可能及必要。今回且说怎样是进行大联合的办法?就是“民众的小联合”。

原来我们想要有一种大联合,以与立在我们对面的强权者害人者相对抗,而求到我们的利益。就不可不有种种做他基础的小联合,我们人类本有联合的天才,就是能群的天才,能够组织社会的天才。群和“社会”就是我所说的“联合”。有大群,有小群,有大社会,有小社会,有小联合,有大联合,是一样的东西换却名称。所以要有群,要有社会,要有联合,是因为想要求到我们的共同利益,共同的利益因为我们的境遇和职业不同,其范围也就有大小的不同。共同利益有大小的不同,于是求到共同利益的方法,(联合)也就有大小的不同。

诸君!我们是农夫。我们就要和我们种田的同类,结成一个联合,以谋我们种田人的种种利益。我们种田人的利益,是要我们种田人自己去求。别人不种田的,他和我们利益不同,决不会帮我们去求。种田的诸君!田主怎样待遇我们?租税是重是轻?我们的房子适不适?肚子饱不饱?田不少吗?村里没有没田作的人吗?这许多问题,我们应该时时去求解答。应该和我们的同类结成一个联合,切切实实彰明较著的去求解答。

诸君!我们是工人。我们要和我们做工的同类结成一个联合,谋我们工人的种种利益。关于我们做工的各种问题,工值的多少?工时的长短?红利的均分与否?娱乐的增进与否?……均不可不求一个解答。不可不和我们的同类结成一个联合,切切实实彰明较著的去求一个解答。

诸君!我们是学生,我们好苦,教我们的先生们,待我们做寇仇,欺我们做奴隶,闲镇我们做囚犯。我们教室的窗子那么矮小光线照不到黑板,使我们成了“近视”,桌子太不合式,坐久了便成“脊柱弯曲症”,先生们只顾要我们多看书,我们看的真多,但我们都不懂,白费了记忆。我们眼睛花了,脑筋昏了,精血亏了,面血灰白的使我们成了“贫血症”’成了“神经衰弱症”。我们何以这么呆板?这么不活泼?这么萎缩?呵!都是先生们迫着我们不许动,不许声的原故。我们便成了“僵死症”。身体上的痛苦还次,诸君!你看我们的实验室呵!那么窄小!那么贫乏--几件坏仪器,使我们试验不得。我们的国文先生那么顽固,满嘴里“诗云”“子曰”,清底却是一字不通。他们不知道现今已到了二十世纪,还迫着我们行“古礼”守“古法”,一大堆古典式死尸式的臭文章,迫着向我们脑子里灌,我们板书室是空的,我们游戏场是秽的。国家要亡了,他们还贴着布告,禁止我们爱国,象这一次救国运动,受到他们的恩赐其多呢!唉!谁使我们的身体,精神,受摧折,不愉快?我们不联合起来,讲究我们的“自教育”,还待何时!我们已经陷在苦海,我们要求讲自救:卢梭所发明的“自教育”,正用得着。我们尽可结合同志,自己研究。咬人的先生们,不要靠他。遇着事情发生一一象这回日本强权者和国内强权者的跋扈一一我们就列起队伍向他们作有力的大呼。

诸君!我们是女子。我们更沉沦在苦海!我们都是人,为什么不许我们参政?我们都是人,为什么不许我交际?我们一窟一窟的聚着,连大门都不能跨出。无耻的男子,无赖的男子,拿着我们做玩具,教我们对他长期卖淫,破坏恋爱自由的恶魔!破坏恋爱神圣的恶糜,整天的对我们围着,什么“贞操’却限于我女子,“烈女嗣”遍天下,“贞童庙’又在那里?我们中有些一窟的聚重在一女子学校,教我们的又是一些无耻无赖的男子,整天说什么“贤妻良母”,无非是教我们长期卖淫专一卖淫。怕我们不受约束,更好好的加以教练,苦!苦!自由之神,你在那里,快救我们!我们于今醒了!我们要进行我们女子的联合!要扫荡一般强奸我们破坏我们精神自由的恶魔!

诸君,我们是小学教师,我们整天的教课,忙的真很!整天的吃粉条屑,没处可以游散舒吐。这么一个大城里的小学教师,总不下几千几百,却没有一个专为我们而设的娱乐场。我们教课,要随时长进学问,却没有一个为我们而设的研究机关。死板板的上课钟点,那么多,并没有余时,没有余力,一一精神来不及!一一去研究学问。于是乎我们变成了留声器,整天演唱的不外昔日先生们教给我们的真传讲义。我们肚子是饿的。月薪十元八元,还要折扣,有些校长先生,更仿照“克减军粮”的办法,将政府发下的钱,上到他们的腰包去了。我们为着没钱,我们便做了有妇的鳏夫。我和我的亲爱的妇人隔过几百里几十里的孤住着,相望着,教育学上讲的小学教师是终身事业,难道便要我们做终身的鳏夫和寡妇?教育学上原说学校应该有教员的家庭住着,才能做学生的模范,于今却是不能。我们为着没钱,便不能买书,便不能游历考察。不要说了!小学教师直是奴隶罢了,我们想要不做奴隶,除非联结我们的同类,成功一个小学教师的联合。

诸君!我们是警察。我们也要结合我们同类,成功一个有益我们身心的联合。日本人说,最苦的是乞丐,小学教员和警察,我们也有点感觉。

诸君!我们是车夫,整天的拉得汗如雨下!车主的赁钱那么多!得到的车费这么少!何能过活,我们也有什么联合的方法么?

上面是农夫、工人、学生、女子、小学教师、车夫、各色人等的一片哀声。他们受苦不过,就想成功于他们利害的各种小联合。

上面所说的小联合,象那工人的联合,还是一个很大很笼统的名目,过细说来,象下列的:

铁路工人的联合,

矿工的联合,

电报司员的联合,

电话司员的联合,

造船业工人的联合,航业工人的联合,

五金业工人的联合,

纺织业工人的联合,

电车夫的联合,

街车夫的联合,

建筑业工人的联合,

方是最下一级联合,西洋各国的工人,都有各行各业的小联合会,如运输工人的联合会,电车工人联合会之类,到处都有,由许多小的联合,进为一个大联合,由许多大的联合,进为一个最大的联合。于是什么“协会”,什么“同盟”,接踵而起,因为共同利益只限于一部分人,故所成立的为小联合,许多的小联合彼此间利益有共同之点,故可以立为大联合,象研究学问是我们学生分内的事,就组成我们研究学问的联合厶象要求解放要求自由,是无论何人都有分的事,就应联合各种各色的人,组成一个大联合。

所以大联合必要从小联合入手,我们应当起而仿效别国的同胞们,我们应该多多进行我们的小联合。

\begin{flushright}《湘江评论》第三期一九一九年七月二八日出版\end{flushright}


\section{民众的大联合(三)}
\datesubtitle{(一九一九年八月四日)}



中华“民众的大联合”的形势

上两回的本报己说完了(一)民众大联合的可能及必要,(二)民众的大联合,以民众的小联合为始基,于今进说吾国民众的大联合我们到底有此觉悟么?有此动机么?有此能力么?可得成功么?

(一)我们对于吾国“民众的大联合”到底有此觉悟么?辛亥革命,似乎是一种民众的联合,其实不然.辛亥革命乃留学生的发纵指示.哥老会的摇旗唤吶,新军和巡防营一些丘八的张弩拔剑所造成的,与我们民众的大多数毫无关系。我们虽赞成他们的主义,却不曾活动。他们也用不着我们活动。然而我们却有一层觉悟。如道圣文神武的皇帝,也是可以倒去的。大逆不道的民主,也是可以建设的。我们有话要说,有事要做,是无论何时可直说可以做的。辛亥而后,到了丙辰,我们又打倒了一次洪宪皇帝,原也是可以打得倒的,及到近年,发生南北战争,和世界战争,可就更不同了,南北战争结果,官僚、武人、政客,是害我们,毒我们,剥削我们,越发得了铁证。世界战争的结果,各国的民众,为着生活痛苦问题,突然起了许多活动。俄罗斯打倒贵族,驱逐富人,劳农两界合立了委办政府,红旗军东驰西突,扫荡了多少敌人,协约国为之改容,全世界为之震动。匈牙利崛起,布达佩斯又出现了崭新的劳农政府。德人奥人捷克人和之,出死力以与其国内的敌党搏战。怒涛西迈,转而东行,英法意美既演了多少的大罢工,印度朝鲜又起了若干的大革命,异军特起,更有中华长城渤海之间,发生了“五四”运动。旌旗南向,过黄河而到长江、黄浦汉皋,屡演话剧,洞庭闽水,更起高潮。天地为之昭苏,奸邪为之辟易。咳!我们知道了!我们醒觉了!天下者我们的天下。国家者我们的国家。社会者我们的社会。我们不说,谁说?我们不干,谁干?刻不容缓的万众大联合,我们应该积极进行!

(二)吾国民众的大联合业已有此动机么?此间我直答之日“有”。诸君不信,听我道来一一

溯源吾国民众的联合,应推清末谘议局的设立,和革命党一一同盟会一一的组成。有谘议局乃有各省谘议局联盟请愿早开国会一举。有革命者乃有号召海内外起兵排满的一举。辛亥革命,及革命党和谐议局合演的一出“痛饮黄龙”。其后革命党化成了国民党,谘议局化成了进步党,是为吾中国民族有政党之始。自此以后,民国建立,中央召集了国会,各省亦召集省议会,此时各省更成立三种团体,一为省教育会,一为省商会,一为省农会(有数省有省工会。数省则合于农会,象湖南)。同时各县也设立县教育会,县商会,县农会(有些县无)。此为很固定很有力的一种团结。其余各方面依其情势地位而组设的各种团体,象

各学校里的校友会,

族居外埠的同乡会,

在外国的留学生总会,分会,

上海日报公会,

环球中国学生会,

北京及上海欧美同学会,

北京华法教育会。

各种学会(象强学会,广学会,南学会,尚志学会,中华职业教育社,中华科学社,亚洲文明协会……),各种同业会(工商界各行各业,象银行公会,米业公会……各学校里的研究会,象北京大学的画法研究会,哲学研究会……有几十种),各种俱乐部……

都是近来因政治开放,思想开放的产物,独夫政治时代所决不准有不能有的,上列各种,都是单纯,相当于上回本报所说的“小联合”。最近因政治的纷乱,外患的压迫,更加增了觉悟,于是竟有了大联合的动机。象什么

全国教育会联合会,

广州的七十二行公会,上海的五十公团联合会,

商学工报联合会,

全国报界联合会,

全国和平期成会,

全国和平联合会,

北京中法协会,

国民外交协会,

湖南善后协会(在上海),

山东协会(在上海),

北京上海及各省各埠的学生联合会,

各界联合会,全国学生联合会……

都是,各种的会,社,部,协会,联合会,固然不免有许多非民众的“绅士”“政客”在里面(象国会,省议会,省教育会,省农会,全国和平期成会,全国和平联合会等,乃完全的绅士会,或政客会),然而各行各业的公会,各种学会,研究会等,则纯粹平民及学者的会集。至最近产生的学生联合会,各界联合会等,则更纯然为对付国内外强权者而起的一种民众大联合,我以为中华民族的大联合的动机,实伏于此。

(三)我们对于进行吾国“民众大联合”果有此能力吗?果可得成功么?谈到能力,可就要发生疑问了。原来我国人口只知道各营最大合算最没有出息的私利,做商的不知设立公司,做工的不知设立工党,做学问的只知闭门造车的老办法,不知共同的研究。大规模有组织的事业,我国人简直不能过问,政治的办不好,不消说,邮政和盐务有点成绩,就是依靠了洋人。海禁开了这久,还没一头走欧洲的小船,全国唯一的“招商局”和“汉冶萍’,还是每年亏本,亏本不了,就招入外股。凡是被外人管理的铁路,清洁,设备,用人都要好些。铁路一被交通部管理,便要糟糕。坐京汉,津浦,武长,过身的人,没有不嗤着鼻子咬着牙齿的!其余象学校搞不好,自治办不好,乃至一个家庭也办不好,一个身子也办不好。“一丘之貉”“千篇一律”的是如此,好容易谈到民众的大联合?好容易和根深蒂固的强权者相抗?

虽然如此,却不是我们根本的没能力,我们没能力,有其原因,就是“我们没练习”。

原来中华民族,几万万人从几千年来,都是干着奴隶的生活,没有一个非奴隶的是“皇帝”(或曰皇帝也是“天”的奴隶,皇帝当家的时候,是不准我们练习能力的)。政治,学术,社会,等等,都是不准我们有思想,有组织.有练习的。

于今却不同了,种种方面都要解放了,思想的解放,政治的解放,经济的解放,男女的解放,教育的解放,都要从九重冤狱,求见青天。我们中华民族原有伟大的能力!压迫逾深,反动愈大,蓄之既久,其发必远,我敢说一句怪话,他日中华民族的改革,将较任何民族为彻底,中华民族的社会,将较任何民族为光明。中华民族的大联合,将被任何地域任何民族而先告成功。诸君!诸君!我们总要努力!我们总要拚命向前!我们黄金的世界,光荣灿烂的世界,就在面前!

\begin{flushright}《湘江评论》第四期1919.8.4出版\end{flushright}


\section{问题研究会章程}
\datesubtitle{(一九一九年十月廿三日)}



第一条、凡事或理之为现代人生所必需,或不必需,而均尚未得适当之解决,致影响于现代人生之进步者,成为问题。同人今设一会,注重解决如斯之问题,先从研究入手,定名问题研究会。

第二条、下列各种问题,及其他认为有研究价值续行加入之问题为本会研究之问题。

(一)教育问题一一

1。教育普及问题(强迫教育问题)2.中等教育问题3.专门教育问题4.大学教育问题5.社会教育问题6.口语教科书编纂问题7.中等学校国文科教授问题8。不惩罚问题9。废止考试问题30.各级教授法改良问题7.小学教师知识健康及薪金问题12.公共体育场建设问题 13.公共娱乐场建设问题14.公共图书馆建设问题 15.学制改订问题16.大派留学生问题 17.杜威教育说如何实施问题 
(二)女子问题

1.女子参政问题2.女子教育问题3.女子职业问题4.女子交际问题5.贞操问题6.恋爱自由及恋爱神圣问题7.男女同校问题8.女子修饰问题9。家庭教育问题10.姑媳同居问题 11.废娼问题 12。废妾问题13。放足问题14.公共育儿院设置问题15.公共蒙养院设置问题 16.私生儿待遇问题 17.避妊问题 
(三)国话问题(一白话文问题)

(四)孔子问题

(五)东西文明会合问题

(六)婚姻制度改良及婚姻制度应否废弃问题

(七)家族制度改良及家族制度应否废弃问题

(八)国家制度改良及国家制度应否废弃问题

(九)宗教改良及宗教应否废弃问题。

(十)劳动问题一一

1.劳动时间问题2.劳工教育问题3.劳工住屋及娱乐问题4.劳动失职处置问题5.工值问题6.小儿劳作问题7.男女工值平等问题8.劳工组合问题9.国际劳动同盟问题10。劳农干政问题 11.强制劳动问题 12.余剩均分问题 13.生产机关公有问题  14.工人退职年金问题15.遗产归公问题(附)
(十一)民族自决问题

(十二)经济自由问题

(十三)海洋自由问题

(十四)军备限制问题

(十五)国际联盟问题

(十六)自由移民问题

(十七)人种平等问题

(十八)社会主义能否实施问题

(十九)民众的联合如何进行问题

(二十)动工俭学主义如何普及问题

(二一)俄国问题

(二二)德国问题

(二三)奥国问题

(二四)印度自治问题

(二五)爱尔兰独立问题

(二六)土耳其分裂问题

(二七)埃及骚乱问题

(二八)处置德皇问题

(二九)重建比利时问题

(三十)重建东部法国问题

(三一)德殖民地处置问题

(三二)港湾公有问题

(三三)飞渡大西洋问题

(三四)飞渡太平洋问题

(三五)飞渡天山问题

(三六)白令英吉利直布罗陀三峡凿遂通车问题一

(三七)西伯利亚问题

(三八)菲律宾问题

(三九)日本粮食问题

(四十)日本问题

(四一)朝鲜问题

(四二)山东问题

(四三)湖南问题

(四四)废督问题

(四五)裁兵问题

(四六)国防军问题

(四七)新旧国会问题

(四八)铁路统一问题(撤消势力范围问题)

(四九)满州问题

(五十)蒙古问题

(五一)西藏问题

(五二)退回庚子赔款问题

(五三)华工问题一一

1.华工教育问题2.华工储蓄问题3.华工归国后安置问题

(五四)地方自治问题

(五五)中央地方集权分权问题

(五六)两院制一院制问题

(五七)普通选举问题

(五八)大总统权限问题

(五九)文法官考试问题

(六十)澄清贿赂问题

(六一)合议制的内阁问题

(六二)实业问题一一

1.蚕丝改良问题2。茶产改良问题3.种棉改良问题4.造林问题

5.开矿问题6.纱厂及布厂多设问题7.海外贸易经营问题8.国民工厂设立问题

(六三)交通问题一一

1.钦路改良问题2.铁路大借外款厂行添第问题3。无线电台建设问题

4.海陆电线添设问题5.航业扩张问题6。商埠马路建筑问题7。乡村汽车路建筑问题

(六四)财政问题…

1.外债偿还问题2.外债添借问题3.内债偿还及加募问题4.裁厘加税问题5。盐务整顿问题6。京省财权划分问题 7.税制整顿问题 8.清丈田亩问题9。田赋均一及加征问题

(六五)经济问题一一

1.币制本位问题2.中央银行确立问题3.收还纸币问题4.国民银行设立问题5。国民储蓄问题

(六六)司法独立问题

(六七)领事裁判权取消问题

(六八)商市公园建设问题

(六九)模范村问题

(七十)西南自治问题

(七一)联邦制应否施行问题

第三条、问题之研究须以学理为根据。因此在各种问题研究之先,须为各种主义之研究,下列各种主义为特须注重研究之主义。一一

(一)哲学上之主义

(二)伦理上之主义

(三)教育上之主义

(四)宗教上之主义

(五)文学上之主义

(六)美学上之主义

(七)政治上之主义

(八)经济上之主义

(九)法律上之主义

(十)科学上之规律

第四条、问题不论发生之大小,只须含有较广之普遍性,即可提出研究,如日本问题之类。

第五条、问题之研究,有须实地调查者,须实地调查之,如华工问题之类,无须实地调查,及一时不能实地调查者,则从书册杂志、新闻纸三项着手研究。如孔子问题,及三海峡凿隧通车问题之类。

第六条、问题之研究,注重有关系于现代人生者之问题。在古代与现代及未来毫无关系者,则不注意。

第七条、问题研究之方式分为三种。一一

(一)一人独自之研究

(二)二人以上开研究会之研究

(三)二人以上不在一地用通函之研究

第八条、问题研究会,只限于“以学理解决问题”会以外。然在未来而可以预测之问题,亦注意,以“实行解决问题”属于问题研究会以外。

第九条、不论何人有心研究一个以上之问题,而愿与问题研究会生交涉者,即为问题研究会会员。

第十条、会与会员间会员与会员间,只限于“问题研究”之一点,有关此外之关系,属于问题研究会以外。

第十一条、问题研究会,设书记一人,办理会中事务。

第十二条、问题研究会,于中华民国八年西历一千九百十九年九月一日成立。问题研究会章程,即于是日订定,且发布。

\begin{flushright}(抄自《北大日刊》一九一九年十月二十三日第二版)\end{flushright}


\section{学生之工作}
\datesubtitle{(一九一九年十二月廿八日)

(1919年12月28日湖南教育月刊)}



我数年来梦想新社会生活,而没有办法。七年春季,想邀数朋友在省城对岸岳麓山设工读同志会,从事半工半读,因他们多不能久在湖南,我亦有北京之游,事无他议。今春回湘,再发此这种想象,乃有在岳麓山建设新村的计议,而先从办一实行社会说本位教育说的学校入手,此新村以新家庭学校及旁的新社会连成一块为根本理想。对于学校的办法,曾草一计划,今抄上计划书中“学生之工作”一章于此,以求同志的敦诲。我觉得在岳麓山建设新村,似可成为一问题,倘有同志对于此问题有详细规划,或有何种实际的进行,实在欢迎希望的很。

(一)

学校之教授之时间,宜力求减少,使学生多自动研究及工作。应划分每日之时间为六分。其分配如左:

睡眠二分。

游息一分。

读书二分。

工作一分。

读书二分之中,自习一分,教授占一分。此时间实数分配,即

睡眠八小时

游息四小时

自习四小时

教授四小时

工作四小时

上例之工作四小时,乃实行工读主义所必具之一个要素。

(二)工作之事项,全然农村的。列举如左:(甲)种园。(1)花木(2)蔬菜

(乙)种田。(1)棉(2)稻及他种

(丙)种林。

(丁)畜牧。

(戊)种桑。

(己)鸡鱼。

(三)

工作须为生产的,与实际生活的。现时各学校之手工,其功用在练习手眼敏活,陶冶心思精细,启发守秩序之心,及审美之情,如为手工课之优点。然多非生产的(如纸、豆,泥、石膏各细工),作成之物,可玩而不可用。又非实际生活的,学生在学校所习,与社会之实际不相一致,结果则学生不熟谙社会内情,社会亦嫌恶学生。

在吾国现时,又有一弊,即学生毕业之后,多骛都市而不乐田园。农村的生活,非其所习,从而不为所乐。(不乐农村生活,尚有其他原因,今不具论)。此讫地方自治之举行有关系。学生多散布于农村之中,则或为发议之人,或为执行之人,即地方自治得学生之中坚而得举行。农村无学生,则地方自治缺乏中坚之人,有不能美满推行之患。又于政治亦有关系,现代政治,为代议政治,而代议政治之基础第于选举之上。民国成立以来,两次选举,殊非真正民意。而地方初选,劣绅恶棍,无选举投票乡民之多数,竟不知选举是甚么一回事,尤无民意可言。推其原因,则在缺乏有政治常识之人参与之故。有学生指导监督,则放弃选举权一事,可逐渐减少矣。

欲除去上所说之弊,(非生产的非实际生活的,骛于都市而不乐农村。)第一,须有一种经济的工作,可使之直接生产,其能力之使用,不论大小多寡,曾有成效可观。第二,此种工作之成品,必为现今社会普遍之需要。第三,此种工作之场所,必在农村之中,此种之工作,:逸为农村之工作。

上述之第一,所以使之直接生产。第二,所以使之合于实际生活。第三,所以养成乐于农村生活之习惯。

(四)

讫上文所举之外,尚有一要项,今述之于下。言世界改良进步者,皆知须自教育普及使人民咸有知识始。欲教育普及,又自兴办学校始。其言固为不错,然兴办学校,不过施行教育之一端。而教育之全体,不仅学校而止。其一端只。有家庭,一端则有社会。……

(第四部分的摘记:

学校家庭和社会的关系。

而教育之全体,……

学校与家庭的矛盾,斗争有两前提,故言改良学校教育,而不同时改良家庭与社会,所谓举中而遗其上下,得其一而失其二也。)

(五)

第二节所举田园树畜各项,普旧日农圃所为,不为新生活。以新精神经营,真则为新生活矣。旧日读书人不予农圃事,今一边读书,一边工作,以神圣视工作焉,则为新生活矣。号称士大夫有知识一流,多营逐于市场与官场,而农村新鲜之空气不之吸,优美之景色不之尝,吾人改而吸尝此新鲜之空气与优美之景色,则为新生活矣。

种园有,一种花木,为花园,一种蔬菜,为菜园,二者相当于今人所称之学校园。再扩充之,则为植物园。种田以棉与稻为主,大小麦、高梁、蜀黍等亦可间种,粗工学生所难为者,雇工助之。

种林须得山地,学生一朝手植,虽出校而仍留所造之材,可增其回念旧游爱重母校之心。

畜牧如牛.羊、猪等,在可能营养之范围内,皆可分别营养。

育蚕须先种桑,桑成饲蚕,男女生皆可为。养鸡鱼,亦生产之一项,学生所喜为者也。

(六)

各项工作,非欲一人做遍,乃使众人分工,一人只做一项,或一项以上。

学生认学校如其家庭,认所作田园林木等如其私物。由学生各个所有私物之联合,为一公共团体,此团体可名之曰“工读同志会”。会设生产、消费、储蓄诸部,学生出学校,在某期间内不取出会中所存的利益,在某期间外,可取去其利益之一部而留存其一部,用此方法,可使学生长久与学校有关系。

(七)

依第三节所述,现时各学校之手工种,为不生产的,所施之能力掷诸虚牝,是谓“能力不经济”。手工科以外,又有体操科亦然,各种之体操大抵智属于能力不经济类,今有各项工作,此两科目可废弃之。两科目之利,各项工作之中,亦可获得。


\section{新民学会会员通信集第一集}
\datesubtitle{新民学会致各会发的信}



各会友均鉴:

本会出版物,分“会员通信集”与“会务报告”之二,除会务报告叙述公务状况年出二册外,会员通信集,为会员发抒所见相与杨榷讨论的场所。凡会友与会友间往来信稿,不论新旧长短,凡是可以公开的,均望将原稿和腾正稿寄来本会,以便采登第四期以后的通信集。不登之稿不退还。已登之稿声明要退还的也可以退还。稿寄长沙潮宗街文化书社毛泽东君为荷

\begin{flushright}新民学会启\end{flushright}
\subsection{发刊的意思及条例}
第一集所收多前一、二年旧信,然于学会颇关重要,因多属于团体事业之进行与发展的。留法活动一事此集只能载蔡君给各会友的信,各会友给蔡君的信,其重要者尚望蔡君付来选印,通信集之发刊,所以联聚同人精神,商榷修学,立身,与改造世界诸方法。发刊不定期,大约每两月可有一本。同人个人人格及会务固宜取绝对公开态度,俱不宜标榜,故通信集以会友人得一本为主,此外多印了几十本,以便会外同志之爱看者取去。因学会极穷,不论会友非会友,都要纳一点印刷费,集内凡关讨论问题的信,每集出后,总望各会友对之再有批评及讨论,使通信集成为一个会友的论坛,一集比一集丰富,深刻,进步,就好极了。
\subsection{给陶毅的信}
※“联军”

※“同志的分配”

※“自修大学”

※“女子留俄勤工勤学”
<p><font face="宋体">斯咏先生:

(上略)

我觉得我们要结合一个高尚纯粹勇猛精进的同志团体。我们同志,在准备时代,都要存一个“向外发展”的志。我于这个问题,颇有好些感想。我觉得好多人讲改造,却只是空泛的一个目标。究竟要改造到那一步田地(即终极目的)?用什么方法达到?自己或同志从那一个地方下手?这些问题,有详细研究的却很少。在一个人,或者还有;团体的,共同的,那就少了。个人虽有一种计划,象“我要怎样研究”,“怎样准备”“怎样破坏”,“怎样建设”,然多有陷于错误的。错误之故,因为系成立于一个人的冥想,这样的冥想,一个人虽觉得好,然拿到社会上,多行不通。这是一个弊病。一个人所想的办法,尽管好,然知道的限于一个人,研究准备进行的限于一个人。这种现象,是“人自为战”,是,“浪战”,是“用力而多成功少”,是“最不经济”。要治这种弊,有一个法子,就是“共同的讨论”。共同的讨论有两点:一、讨论共同的目的;二、讨论共同的方法。目的同方法讨论好了,再讨论方法怎样实践,要这样的共同讨论,将来才有共同的研究(此指学问),共同的准备,共同的破坏和共同的建设。要这样才有具体的效果可观。“浪战”,是招致失败的,是最没有效果的。共同讨论,共同进行是“联军”,是“同盟军”,是可以橾战胜攻取的左券的。我们非得力戒浪战不可,我们非得组织联军共同作战不可。

上述之问题,是一个大问题。至今尚有一个问题,也很重大,就是“留学或作事的分配”。我们想要达到一种目的(改造),非讲究适当的方法不可,这方法中间,有一种是人怎样分配。原来在现在这样“才难”的时候,人才最要讲究经济。不然,便重迭了,推积了,废置了。有几位在巴黎同志,发狠的招人到巴黎去。要扯一般人到巴黎去是好事;多扯同志去,不免错了一些。我们同志,应该散于世界各处去考察,天涯海角都要去人,不应该堆积在一处。最好是一个人或几个人担任开辟一个方面。各方面的“阵”,都要打开。各方面都应该去打先锋的人。

我们几十个人,结识的很晚,结识以来,为期又浅(新民学会是七年四月发生的),未能将这些问题,彻底研究(或并未曾研究)。即我,历来狠懵懂狠不成材,也很少研究。这一次出游,观察多方而情形,会晤得一些人,思索得一些事,觉得这几种问题,很有研究的价值。外边各处的人,好多也和我一样未曾研究。一样的睡在鼓里,很是可叹!你是很明达很有运志的人,不知对我所陈述的这一层话,有甚么感想?我料得或者比我先见到了好久了。

以上的话还空,我们可再实际一些讲:

新民学会会友,或旭旦学会会友,应该常时开谈话会,讨论吾济共目的目的,及达到目的之方法,一会友的留学及做事,应该受一种合宜的分配,担当一部分的责任,为有意识的有组织的活动。在目的地方面,宜有一种预计,怎样在彼地别开新局面?怎样可以引来或取得新同志?怎样可以创造自己的新生命?你是如此,魏国劳诸君也是如此,其他在长沙的同志及已出外的同志也应该如此,我自己将来,也很想照办。

以上所写是一些大意,以下再胡乱写些琐碎。

会友张田基君安顿赴南洋,我很赞成他去。在上海的肖子璋君等十余人准备赴法,也很好!彭璜君等数人在上海组织工读互助团,也是一件好事。彭璜君和我,都不想往法。安顿往俄。何叔衡想留法,我劝他不必留法,不如留俄。我一己的计划,一星期外将赴上海。湘事平了,回长沙。想和同志成一“自由研究社”(或经名自修大学),预计一年或二年,必将古今中外学术的大纲,弄个清楚,好作出洋考察的工具(不然,不能考察)。然后组一留俄队,赴俄勤工俭学。至于女子赴俄,并无障碍,逆料俄罗斯的女同志,必会特别欢迎。“女子留俄勤工俭学会”,继“女子留法勤工俭学会”而起,也并不是不可能的事。这庄事(留俄),我正和李大钊君等为商量。所说上海复旦教授汤寿军君(即前商专校长)也有意去。我为这件事,脑子里装满引俞快和希望,所以我特地告诉你!好象你曾说过杨润余君入了我们的学会,近日翻阅旧的大公报,见他的著作,真好!不杨君近日作何重活?有暇可以告诉我吗?今日的女子工读团,稻田新来了四人,该团建前共八人,湖南占六人,其属一韩人一苏人,觉得很有趣味!但将来的成绩怎样?还要看他们的能力和道德力如何,也许终究失败(男子组大概可以说已经失败了)。北京女高师,学生方面很有自动的活泼的精神,教职方面不免黑暗。接李一纯君函,说将在周南教一课,不知已来了否?再谈。

 毛泽东
<blockquote>
九月二日在北京
</blockquote>
\subsection{给周世钊的信}
※国内研究出国围研究的先后问题

※团体事业准备工夫

※自修大学
<p><font face="宋体">谆元吾兄:

接张君文亮的信,惊悉兄的母亲病故,这是人生一个痛苦之关。象吾等长日在外未能略尽奉养之力的人,尤其发生“欲报之德吴罔极”之痛,这一点我和你的境遇,算是一个样的!


早前承你寄我一个长信。很对不住,我没有看完,便失掉了,但你信的大意,已大体明白。我想你现时在家,必正纲缪将来进行的计划,我很希望我的计划和你的计划能够完全一致,因此你我的行动已能够一致。我现在觉得你是一个真能爱我,又真能于我有益的人,倘然你我的计划和行动能够一致,那便是很好的了。

我现在极愿将我的感想和你讨论,随便将他写在下面,有些也许是从前和你谈过来的。

我觉得求学实在没有“必要在什么地方”的理,“出洋”两字,在好些人只是一种“迷”。中国出过洋的总不下几万乃至几十万,好的实在很少,多数呢?仍旧是“糊涂’,仍旧是“莫名其妙”,这便是一个具体的证据。我曾以此问过胡适之和黎邵西两位,他们都以我的意见为然,胡适之并且作过一篇“非留学篇”。

因此,我想暂不出国去,暂时在国内研究各种学问的纲要。我觉得暂时在国内研究,有下列几种好处:

1、看译本比较原本快迅得多,可于较短的时间求得较多的知识。

3、世界文明分东西两流,东方文明在世界文明内,要占个半壁的地位。然东方文明可以说就是中国文明。吾人拟从先研究过吾国古今学说制度的大要,再到西洋留学才有可资比较的东西。

3、吾人如果要在现今的世界稍微尽一点力,当然脱不开“中国”这个地盘。关于这地盘内的情形,拟不可不加以实地的调查及研究。这层工夫,如果留在在出洋回来的时候做,因人事及生活的关系,恐怕有些困难。不如在现在做了,一来无方才所说的困难,二来又可携带些经验到西洋去,考察时可以借资比致。

老实说,现在我于种种主义,种种学说,都还没有得到一个比较明了的概念,想从译本及时贤所作的报章杂志,将中外古今的学说刺取精华,使他们各构成一个明了的概念。有工夫能将所刺取的编成一本书,更好。所以我对于上列三条的第一条,认为更属紧要。

以上是就“个人”的方面和“知”的方面说。以下再就“团体”的方面和“行”的方面说:

我们是脱不了社会的生活的,都是预备将来要稍微有所作为的。那么,我们现在便应该和同志的人合力来做一点准备工夫。我看这一层好些人不大注意,我则以为很是一个问题,不但是随便无意的放任的去准备,实在要有意的有组织的头准备,必如此才算经济,才能于较短的时间(人生百年)发生较大的效果。我想(一)结合同志,(二)在经济的可能的范围内成立为他日所必要的基础事业,我觉得这两样是我们现在十分要注意的。

上述二层(个人的方面和团体的方面),应以第一为主,第二为辅。第一应占时间的大部分;第二占一小部分。总时间定三年(至多),地点长沙。

因此我于你所说的巴黎南洋北京各节,都不赞成,而大大赞成你“在长沙”的那个主张。

我想我们在长沙要创造一种新的生活,可以邀合同志,租一所房子,办一个自修大学(这个名字是×××先生造的)。我们在这个大学里实行共产的生活,关于生活费用取得的方法,约可定为下列几种:

1、教课。(每人每周六小时及至十小时)

2、投稿。(论文稿或新闻稿)

3、编书。(编一种或数种可以卖稿的书)

4、劳动的工作,(此项以不消费为主,如自炊自濯等)

所得收入,完全公共,多得的人补助少得的人,以够消费为止。我想我们两个如果决行,何叔衡和邵泮清或者也会加入。这种组织,也可以叫做“工读互相团”。这组织里最要紧的是要成立一个“学术谈话会”,每周至少要为学术的谈话两头或三次。

以上是说暂不出洋在国内研究的话。但我不是绝对反对留学的人,而且是一个主张大留学政策的人,我觉得我们一些人都要过一回“出洋”的瘾才对。我觉得俄国是世界第一个文明国,我想两三年后,我们要组织一个游俄队。这是后话,暂时尚可不提及他。

出杂志一项,我觉得很不容易。如果自修大学成了,自修有了成绩,可以看情形出一本杂志(此间的人,多以恢复湘江评论为言)其余会务进行,留待面谈,暂不多说,有暇请简幅一信。

 弟泽东

 一九二○年三月十四日

 北京北长街九十九号
\subsection{给罗学瓒的信}<p><font face="宋体">荣熙兄:

兄此信我自接到,先后看了多次。今天再看一次,尤有感动。你的话我没有不以为然的。我已经决定了一种求学的方法,暂时也不必说,只是你的话我一定要行就是。你奋勉的志气很可敬。你现处环境很好,可以从事周将的观察和深湛的思考。听说你已离学校在工厂做工,西洋工厂里的情况,可由此明了,并且可以得到托尔斯泰所谓“由劳动{得来的生活是真快乐”。我现在很想作工,在上海,李声(水鲜)辩君劝我入工厂,我颇心动,我现在颇感觉专门用口用脑的生活是苦极了的生活,我想总要有一个时期专用体力去做工就好。李君声鲜以一师范学生在江南造船厂打铁,居然一两个月后,打铁的工作样样如意。由没有工钱以渐得到每月工钱十二元。他现寓上海法界渔阳里二号,帮助陈仲甫先生等组织机器工会,你可以和他通信。启民在太安里周南女校。惇元在理问街道俗报。湘潭教育腐败已极,旅省诸人组织“湘潭教育促进会”从事促进,尚无大效。一师湘潭学发会亦将有所兴作,兄信尚未转去,稍迟当转去。兄沿途寄稿均登湘南日报,无转阅必要了。此信曾经复了一次,今再附识近来感想于此。

 弟泽东

 九年十一月廿六日


\section{新民学会会员通信集第二集}
\datesubtitle{新民学会致各会友的信}



各会友均鉴:

本会出版物。分“会员通信集”与“会务报告”之二,除会务报告叙述会务状况年出二册外,会员通信集,为会员发抒所见相与杨权讨论的场所。凡会友与会友间往来信稿,不论新、旧,长、短,凡是可以公开的,均望将原稿或誊正稿寄来本会,舅便釆登第四期以后的通信集,不登之稿可退还。已登之稿声明要退还的也可退还。稿寄长沙文书社毛泽东君为荷。

\begin{flushright}新民学会启\end{flushright}
\subsection{序}

这是新民学会会员通信集“第二集”,釆集会员通信计二十八封,重要者如下:向警予讨论女子发展的计划一封;讨论女子留法勤工俭学问题及女同志联系问题一封;欧阳洋讨论新民学会共同的精神一封;毛泽东主张新民学会应取潜在的态度一封;肖子障报告男女公友在法生活状况一封;易礼容主张吾人进行要有准备一封;罗璈阶希望学会反抗靡俗估量最高价值一封;毛泽东讨论湖南自治并主张学会同人应为主义的结会一封;张国基讨论会务进行一封;报告湘人在南泽情形一封;毛泽东讨论湘人往南洋应取的态度希望会友往南洋为文化动动和南洋建国运动一封;张国基述南洋的奇闻一封;罗学瓒希望会友注意锻炼身外祛除四种迷惑解决家庭问题一封;毛泽东主张组织拒婚同盟一封。以上各信,或于身心之修养有益,或学术之讨论问题之研究有益,或于会务之进行有益,并且都是在能引起会员团体生活的兴味的。其余的信,或商量进止,或讨论事宜,或报告个人状况,具载于此,以见一班。

\begin{flushright}一千九百二十年一月二十号编者\end{flushright}
\subsection{给向警予的信}

※湖南问题

来信久到,未能即复,幸谅!湘事去冬在沪,姐首慷慨论之。一年以来,弟和萌柏等也曾间接为力,但无大效者,教育未行,民智未启,多数之湘人,犹在睡梦,号称有知识之人,又绝无理论计划。弟和萌柏主张湖南自主为国,务与不进化之北方各省及情势不同之南方各省离异,打破空洞无组织的大中国,直接与世界有觉悟之民族携手,而知者绝少。自治问题发生,空气至为黯谈自由湖南革命政府召集湖南人民宪法会议制定湖南宪法以建设新湖南之说出,声势稍振,而多数人莫明其妙,甚或大惊小怪,诧为奇离。湖南人脑筋不清晰,无理想,无远计,几个月来,已看透了。政治界暮气已深,腐败已甚,政治改良一涂,可谓绝无希望。君人惟有不理一切,另辟道路,另造环境一法。教育系我职业,顿湘两年,世已决计。惟办事则不能求学,于自身牺牲太大耳。湘省女子教育绝少进步(男子教育亦然),希望你能引大批女同志出外,多引一人,即多救一人。此颂

进步!

健豪伯母及戍熙姐同此问好

 弟泽东

 九年十一月二十五日
\subsection{给欧阳泽的信}

※学会应取潜在的态度
<p><font face="宋体">玉先生:

共同的精神四项,弟样样赞成。会员加入不限省界,也是极端赞成的。岂但省界,国界也不要限。弟在京所以有那么一说,是因为新民学会现在尚没有深固的基础,在这个时候,宜注意于固有同志之联络,以道义为中心,互相劝勉谅解,使人人如亲生的兄弟姐妹一样。然后进而联络全中国的同志,进而联络全世界的同志,以共谋解决人类各种问题。弟意凡事不可不注重基础,弟见好些团体,象没有经验的商店,货还没有办好,招牌早已高挂了,广告早四出了,结果离不开失败,离不开一个“倒”。半松园会议,都主张本会进行应取“潜在的态度”,弟是十二分赞成,兄也是赞成的一个,长沙同人,亦同此意。弟以为这是新民学会一个好现象,可大可久的事业,其基础即筑在这种“潜在的态度”之上。

你的信我在上海接到。彭、周、劳、魏都转给他们看了,我七月回湘,一回多忆,未能作答,幸谅!你现状谅好,我忘记你在芬丹白露,抑在蒙达尔尼?来信幸告。近因极倦,游觉到萍,旋中作书,言不尽意。

 弟泽东

 九年十一月二十五日萍乡旋中
\subsection{对易礼容来信的评论}

礼容这一封信,讨论吾人进行办法,主张要有预备,极忠极切。我的意见,于致陶斯咏姐及周惇之兄函中已具体表现,于归湘途中和礼容也当面说过几次。我觉得去年的驱张运动和今年的自治运动,在我们一班人看来,实在不是由我们去实行做一种政治运动。我们做这两种运动的意义,驱张运动只是简单的反抗张敬尧这个太令人过意不下去的强权者。自治运动是简单的希望在湖南能够特别定出一个办法(湖南宪法),将湖南造成一个较好的环境,我们好于这种环境之内,实现我们的具体准备工夫,彻底言之,这两种运动,都只是应付目前环境的一种权宜之计,决不是我们根本的主张,我们的主张远在这些运动之外,说到这里,诚哉如礼容所书,‘准备’要紧,不过准备的‘方法’怎样?又待研究。去年在京,陈赞周即对于‘驱张’怀疑,他说我们既相信世界主义和根本改造,就不要顾及目前的小问题,小事实,就不要‘驱张’。他的话当然也有理,但我意稍有不同,‘驱张’运动和自治运动等,也是达到根本改造的一种手段,是对付‘目前环境’最经济最有效的一种手段。但有一条件,即我们自始至终(从这种运动之发起至结局),只宜立于‘促进’的地位,明言之,即我们决不跳上政治舞台去做当局。我意我们新民学会会友于以后进行方法应分几种:一种是出国的,可分为二,一是专门从事学术研究,多造成有根底的学者,如罗荣熙肖子升之主张。一是从事于根本改造之计划和组织,确立一个改造的基础,如蔡和森所主张的共产党。一种是未出国,亦分为二,一是在省内及国内学校求学的,当然以求学储能做本位。一是从事社会运动的,可以从各而发起并实行各种有价值之社会运动及社会事业。其政治运动之认为最经济最有效者,如‘自治运动’,‘普选运动’等,亦可从旁尽一点促进之力,惟千万不要沾染旧社会习气’尢其不要忘记我们根本的共同的理想和计划。至礼容所说的结合同志,自然十分要紧,惟我们的结合,是一种互相的结合,人格要公开,目的要共同,我们总不要使我们意识中有一个不得其所的真同志就好。

 泽东
\subsection{复肖子嶂的信}
<p><font face="宋体">子嶂兄:

你沿途给我的信和照片,都收到,很感谢你的厚意。我竟没有一个信给你,很对你不住,‘半松园会议’的结果,既由他和赞周等带到了欧洲,‘蒙达尔尼会议’的情形,又由你和子升递回了亚洲,手升并且自己回国,不日就可见而了,真是乐阿!你现在谅好,谅还在学校。我意你在法宜研究一门学问,择你性之所宜者至少一门,这一门使要将他研究透彻。我近觉得仅仅常识是靠不住的,深慨自己学问无专精,两年来为事所扰,学问来能用功,实深抱恨,望你有以教我。学会出版分‘通信集’与‘会务报告’为二,通信集本年至少可出三集,请设法收集在法诸友间新旧信稿邮递‘长沙文化书社’交弟,作第四集第五集材料,至感。(请告诺友以后通信均写长沙文化书社)

 弟泽东
\subsection{给罗璈阶的信}

※叛湖南问题

※叛主义的结合

※湖南问题

※主义的结合

※学生自决
<p><font face="宋体">章龙兄:

昨信接到。重翻你七月二十五日的信,我昨信竟没有一句答复你信内的话,真对不住。今再奉复大意如下。我虽然不反对零碎解决,但我不赞成没有主义头痛医头脚痛医脚的解决。我主张湖南人不与闻外事,专把湖南一省弄好,有两个意思:一是中国太大了,各省的感情利害和民智程度又至不齐,要弄好他也无从着手,从康梁维新至孙黄革命,(两者亦自有他们相当的价值当别论)都只在这组织上用功,结果均归失败。急应改涂易辙,从各省小组织下手。湖南人便应以湖南一省为全国倡,各省小组织好了,全国总组织怕他不好,一是湖南的地理民性,均很有为。杂在全国的总组织者中,既消磨特长,复阻碍进步。独立自治,可以定出一种较进步的办法(湖南宪法),内之自庄严璀睽其河山,处之与世界有觉悟的民族直接携手,共为世界的大改造。全国各省也可因此而激厉进化。所以弟直主张湖南应自立为国,湖南完全自治,丝毫不受外力干涉,不要再为不中用的‘中国’所累,这实是进于总解决的一个紧要手段,而非和有些人所谓零碎解决实则是不痛不痒的解决相同,此意前函未尽,今再补陈于此。

兄所谓善良的有势力的士气,确是要紧。中国坏空气太深太厚,吾们戒哉要造成一种有势力的新空气,才可以将他划换过来。我想这种空气,固然要有一班刻苦励志的‘人’,尤其要有一种为大家共同信守的‘主义’,没有主义,是造不成空气的。我想我们学会,不可徒然做人的聚集,感情的结合,要变为主义的结合才好。主譬如一面旗子,旗子立起了,大家才有所指望,才知道趋赴,兄以为何如?

“会务报告”专纪会务,不载论述文字,尚未着手编写,大略每季一册尽够了。此外会友通信,发刊通信专集,为会友相互讨论商讨的场所,兄处有无会友们往还信稿,不论新旧,请检查寄弟。

“湘江”尚未出版,固因事忙,亦怡出而不好,到底出否,尚待斟酌。

弟本期在城南附小办一点事,杂以他务,自修时间很少,读‘岁月易逝无法挽回’,‘思想学术节节僵化’诸语,使我不寒而栗,我回湘时,原想无论如何每天要有一点钟看报,两点钟看书,竟不能实践。我想忙过今冬,从明年起,一定要实践这个条件才好。求学程序计,略有一点,迟后当可奉告。

讲到湖南教育,真是欲哭无泪。我于湖南教育共有两个希望:一个是希望至今还存在的一班造孽的教育家死尽,这个希望是做不到的。一是希望学生自决,我唯一的希望在此。怪不得人家说‘湖南学生的思想幼稚’,(沈仲九的话)从没有人供给过他们以思想,也没有自决的想将来自己的思想开发过,思想怎么会不幼稚呢?望时赐信为感!

 弟泽东

 (九年十一月二十五日)
\subsection{给钦文(李思安)的信(一九二〇年十一月二十五)}
<p><font face="宋体">钦文姐:

你这信我八月里就收到了。后来还接到你在星加坡寄来的一封长信,并一些印刷物。很感谢你的厚意。我因事忙,没有即答,想能原谅我吧。湘江尚未出版。湖南虽有一些志士从事实际的改造,你莫以为是几篇文章所能弄得好的。大伟人虽没有十分巩固,小伟人(政客)却很巩固了。我想对付他们的法子,最好是不理他们,由我们另想办法,另造环境,长期的预备,精密的计划。实力养成了,效果自然会见,倒不必和他们争一日的长短,你以为然么?你事务谅是忙的,我劝你总要有时间看一点新书报。并且希望你能够继续省察自己,能够知道自己的短处。你前信嘱转集虚,已转他看了。有暇望告我以近状。

 弟泽东

 十一月二十五日
\subsection{给张国基的信(一九二〇年十一月二十五)}

※会务问题

※湘人往南洋应取之态度

※南洋文化运动和南洋建国运动
<p><font face="宋体">颐生兄:

两信先后本悉,久未作复,甚歉!所言会务六项,弟大体均赞成。第一项发行会报,现已决发刊会员通讯集和会务报告两种。第二项会友加入宜郑重。第三项会友加入不要有男女老幼等区别。弟忙夏间在上海与焜甫,赞周,子障,荫柏、望成、韫广、敦洋.冀儒,玉生,等在半淞园会商,及回长沙再和长沙会友商酌,多主会友加入,要准备下列三个条件,(一)纯洁,(二)诚恳,(三)向上,并须有五人介绍,经评议部通过,然后再郑重通告全体会员,正与你的主张相合。(南洋方面同志,当然应该联络),第四项,会所的确定,也是要事,不久总要在长沙觅到一个相当的会所。书报的设置与会友研究有关,长沙巴黎南洋应分别设备,其经费可由各地会员“分”任,第五项,经费的筹措,我意只要会员常年费交齐,普通用费已够。此外只有印通信集和会务报告须款,但也不多,可由会员临时分任。会友录即印在会务报告之内,本年总可以印出一本。南洋通讯社组织极要。惟弟对于湘人往南洋有一意见,即湘人往南洋要学李石曾先生等介绍学生往法国之用意,取世界主义,而不釆殖民政策。世界主义,愿自己好,也愿别人好,质言之,即愿大家好的主义。殖民政策,只愿自己好,不愿别人好,质言之,即损人利己的政策,苟是世界主义,无地不可自容,李石曾等便是一个例。苟是殖民政策,则无地可以自容,日本人便是一个例,南洋文化闭塞,湘人往南洋者,宜以发达文化己任。兄等苟能在南洋为新文化运动,使国内发生之新文化,汇往南洋,南洋人(不必单言华侨)将受赐不浅。又南洋建国运动,亟须发起,苟有志士从事于此种运动,拯救千万无告之人民出水火而登袵席,其为大业,何以加兹。弟意我们会员宜有多人往南做教育运动,和文化运动,俟有成效,即进而联络华侨士著各地各界,鼓吹建国,世界大同,必以各地民族自决为基,南洋民族而能自决,即是促进大同的一个条件。有暇望时通信。

 弟泽东

 九年十一月二十五日给
\subsection{罗学瓒的信}

※养成读书和游戏并行的习惯

※理论上的错误

※拒婚同盟
<p><font face="宋体">荣熙兄:

兄七月十四日的信,所论各节,透彻之到身体诚哉是一个大问题。你谓中国读书人,以身殉学,是由于家庭、社会和学校的环境太坏造成的,这是客观方而的原因,诚哉不错。尚有主观方面的原因,就是心理上的惰性。如读书成了习惯,便一直谈下去不知休息。照卫生的法则,用脑一点钟,应休息一分钟,弟则常常接连三四点钟不休息,甚或夜以继日,并非乐此不渡,实是疲而不舍。我看中国下力人身体并不弱,身体弱只有读书人。要矫正这弊病,社会方面,须设造成好的环境。个人方面,须养成工读并行的习惯;至少也要养成读书和游戏并行的习惯。我的生活实在太劳了,怀中先生在时,曾屡劝我要节势,要多休息,但我总不能信他的话,现在我决定在城市住两个月,必要到乡村住一个星期。这次便是因休息到萍乡,以后拟每两个月要出游一次。

四种迷,说得最透彻,安得将你的话印刷四万万张遍中国人每人给一张就好。感情的生活,在人身原是很要紧,但不可令感情来论事。以部份概全体,是空间的误认。以一时概永久,是时间的误认。以主观概客观,是感情和空间合同的误认。四者通是犯了理论的错误。我近来常和朋友发生激然的争辩,均不出四者范围。我自信我于后三者的错误尚少。惟感情一项,颇不现能免。惟我的感情不是你所指的那些例,乃是对人的问题。我常觉得有站在言论界上的人就不佩服他,或发现他人格上有缺点。他发出来的议论,我便有些不大信用。以人废言,我自知这是我一个短处,日后要是矫正。我于后之者于说话高兴时或激烈的也常错误,不过自己却知道这是错误,所谓明知故犯罢了,(作文时也有)。

“工学励志会”,听说改成了“工学世界社”,详情我小知,请你将组织,进行,事务等,告我一信。通信尚未到。交换报一节弟可办到。请陆续将稿寄来(寄长沙文化书社交弟)。

以资本主义做基础的婚姻制度,是一种绝对要不得的事,在理论上是以法律保护最不合理的强奸,而禁止最合理的自由恋爱;在事实上,天下无数男女的怨声,乃均发现于这种婚姻制度的下面。我想现在反对婚姻制度已经有好多人说了,就只没有人实行。所以不实行,就只是“怕”,我听得“向蔡同盟”的事,为之一喜,向蔡已经打破了“怕”实行不要婚姻,我想我们正好奉向蔡做首领,组织一个“拒婚同盟’,已有婚约的,解除婚约(我反对人道主义)。没有婚约的,实行不要婚约。同盟组成了,同盟的各员立刻组成同盟军。开初只取消极的态度。对外‘防御’反对我们的敌人,对内好生正理内部的秩序,务使同盟内的各员,都实践“废婚姻”这条盟约。稍后,就可取积极的态度开始向世界“宣传”,开始“攻击”反对我们的敌人,务使全人类对于婚姻制度都得解放,都纳入同盟做同盟的一员。我这些话好象是笑话,实则兄所愈感的那些“家庭之吉”,非用这种好笑的办法,无可避免。假如没有人赞成我的办法,我“一个人的同盟”是已经结起了的。我觉得凡在婚姻制度底下的男女,只是一个“强奸团”,我是早已宣言不愿加入这个强奸团的。你如不赞成我的意见,便请你将反对的意见写出。此祝进步。

 弟泽东

 九年十一月二十六日


\section{新民学会会员通信集第三集}
\datesubtitle{新民学会致各会友的信}



各会友均鉴:

投寄通信集的信稿,请寄长沙文化书社转交为荷。
<p><font face="宋体"> 
新民学会启
\subsection{新民学会紧要启事}

本会同人结合。以互助互勉为鹄,自七年夏初成立,至今将及三年,虽形式未周,而精神一贯。惟会友个人对于会之精神,间或未能了解。有牵于他种事势不能分其注意力于本会者;有在他种团体感情甚洽因而对于本会无感情者;有缺乏团体生活之兴趣者;有行为不为会友之多数满意者;本会对于有上述形情之人,认为虽曾列为会友,实无互助互勉之可能。为保持会的精神起见,惟有不再认其为会员。并希望以后介绍新会员入会,务求无上列情形者。本会前途幸甚。


新民学会启

一千九百二十一年一月二日
\subsection{“新民学会会员通信集”第三集重点及付印出版日期}

这一集以讨论“共产主义”和“会务”为两个重要点。信的封数不多,而颇有精义。千九百二十一年一月上旬付印,下旬出版。
\subsection{给肖旭东、蔡林彬并在法诸会友的信(一九二○年十二月一日)}

※赞成“改造中国与世界”为学会方针。

※赞成马克思式的革命。

※学会的态度:(一)互助互勉。(二)诚恳(三)光明。(四)向上。

※共同研究及分门研究。

※学会的四种运动。

※联络同志之重要。
<p><font face="宋体">和森兄子升兄并转在法诸会友:

接到二兄各函,欣慰无量!学会有具体的计划,算从蒙达尔尼会议及二兄这几封信始。弟于学会前途,抱有极大希望,因之也略有一点计划,久思草具计划书提出于会友之前,以资商榷,今得二兄各信,我的计划书可以不作了。我只希望我们七十几个会友,对于二兄信上的计划,人人下一个祥密的考虑。随而下一个深切的批评,以决定或赞成,或反对,或于二兄信上所有计划和意见之外,再有别的计划和意见。我常觉得我们个人的发展或学会的发展,总要有一条明确的路数,没有一条明确的路数,各个人只是盲进,结果糟踏了各人自己外之,又糟踏了这个有希望的学会,岂不可惜?原来我们在没有这个学会之先,也就有一些计划,这个学会之所以成立,就是两年前一些人互相讨论研究的结果。学会建立以后,顿成功了一种共同的意识,于个人思想的改造,生活的向上,很有影响。同对于共同生活,共同进取,也颇有研究。但因为没有提出具体方案;又没有出版物可作公共讨论的机关;并且两年来会友分赴各方;在长沙的会员又因为政治上的障碍不能聚会讨论,所以虽然有些计划和意见,依然只藏之于各人的心里,或几人相会出之于各人的口里,或彼此通函见之于各人之信里;总之只存于一部分的会友间而己。现在诸君既有蒙达尔尼的大集会,商决了一个共同的主张;二兄又本乎自己的理想和观察,发表了个人的意见;我们不在法国的会员,对于诸君所提出当然要有一种研究,批评,和决定。除开在长沙方面会员,即将开会为共同的研究,批评,和决定外,先达我个人对于二兄来信的意见如左。

现在分条说来:

(一)学会方针问题。我们学会到底拿一种什么方针做我们共同的目标呢?子升信里述蒙达尔尼会议,对于学会进行之方针,说:“大家决定会务进行之方针在改造中国与世界”。以“改造中国与世界”为学会方针,正与我平日的主张相合,并且我料到是与多数会友的主张相合的。以我们接洽和视察,我们多数的会友,都顷向于世界主义,试看多数人鄙弃爱国,多数人鄙弃谋一部分一国家的私利,而忘却人类全体的幸福的事,多数人都觉得自己是人类的一员,而不愿意更繁复的隶属于无意义之某一国家,某一家庭,或某一宗教,而为其奴隶;就可以知道。这种世界主义,就是四海同胞主义,就是愿意自己好也愿意别人好的主义,也就是所谓社会主义。凡是社会主义,都是国际的,都是不应该带有爱国的色彩的。和森在八月十三日的信里说:“我将拟一种明确的提议书,注重无产阶级专政与国际色彩两点。因我所见高明一点的青年,多带一点中产阶级的眼光和国际的色彩,于此两点,非严正主张不可”。除无产阶级专政一点置于下条讨论外,国际色彩一点,现在确有将他郑重标揭出来的必要。虽然我们生在中国地方的人。为做事便利起见,又因为中国比较世界各地为更幼稚更腐败应先从着手改造起见。当然应在中国这一块地方做事;但是感情总要是普遍的,不要只爱这一块地方而不爱别的地方。这是一层,做事又并不限定在中国,我以为固应该有人在中国做事,更应该有人在世界做事,如帮助俄国完成他的社会革命;帮助朝鲜独立;帮助南洋独立;帮助蒙古、新疆、西藏、青海、自治自决;都是很要紧的。

以下说方法问题。

(二)方法问题。目的一一改造中国与世界一一定好了,接着发生的是方法问题。我们到底用什么方法去达到“改造中国与世界”的目的呢?和森信里说:“我现认清社会主义为资本主义的反映其重要使命在打破资本经济制度,其方法在无产阶级专政”。和森又说:“我以为现世界不能实行无政府主义,因在现世界显然有两个对立的阶级存在。打倒有产阶级的迪克推多,非以无产阶级的迪克推多压不住反动,俄国就是个明证。所以我对于中国将来的改造,以为完全适用社会主义的原理与方法。……我以后先要组织共产党,因为他是革命运动的发动者,宣传者,先锋队,作战部”。据和森的意见,以为应用俄国式的方位去达到改造中国与世界,是赞成马克思的方法的。而子升则说:“世界进化是无穷期的,革命也是又穷期,我们不认可以一部分的牺牲,换多数人的福利。主张温和的革命。以教育为工具的革命,为人民谋全体福利的革命。以工会合社为实行改造之方法。颇不认俄式――马克思式一一革命为正当,而倾向无政府――蒲鲁东式一-之新式革命,比较和而缓。虽缓然和,同时李和笙兄来信,主张与子升相同。李说:“社会改进,我不赞成笼统的改造,用分工协助的方法,从社会内面改造出来,我觉得很好。一个社会的病,自有他的特别的背影,一剂学方可以医天下人的病,我很怀疑。俄国式的革命,我根本上有未敢赞同之处”。我对子升和笙两人的意见。(用平和的手段,谋全体的幸福)在真理上是赞成的,但在事实上认为做不到。罗素在长沙演说,意与子升及和笙同,主张共产主义,但反对劳农专政,谓宜用教育的方法使有产阶级觉悟,可不至要妨碍自由,兴起战争,革命流血。但我于罗素讲演后,曾和殷柏,礼容等有极详细之辩论。我对于罗素的主张,有两句评语:就是:“理论上说得通,事实上做不到”。罗素和子升和笙主张的要点,是“用教育的方法”但教育一要有钱,二要有人,三要有机关。现在世界,钱尽在资本家的手,主持教育的人尽是一些鸯本家,或资本家的奴隶。总言之,现在世界的学校及报馆两种最主要的教育机关,又尽在资本家的掌握中,现在世界的教育,是一种资本主义的教育。以资本主义教育儿童,这些儿童大了又转而用资本主义教第二代的儿童。教育所以落在资本家手里,则因为资本家有“议会”以制定保护资本家并防制无产阶级的法律,有“政府”执行这些法律,以积极的实行其所保护与所禁止。有“军队”与“警察”,以消极的保障资本家的安乐与禁止无产者的要求。有“银行”以为其财货流通的府库。有工厂以为其生产品垄断的机关。如此,共产党人非取政权,且不能安息于其守下,更要能握得其教育权;如此,资本家久握教育权,大鼓吹其资本主义,使共产党人的共产主义宣传,信者日见其微。所以我觉得教育的方法是不行的。我看俄国式的革命,是无可如何的山穷水尽诸路皆走不通了的一个变计。并不是有更好的方法弃而不釆,单要釆这个恐怖的方法。以上是第一层理由。第二层,依心理上习惯性的原理,及人类历史上的观察,觉得要资本家信共产主义,是不可能的事。人生有一种习惯性,是心理上的一种力,正与物在斜方必倾向下之物理上的一,种力一样。要物下倾向下,依力学原理,要有与他相等的一力去抵抗他才行。要人心改变,也要有一种与这心力强度相等的力去反抗他才行。用教育之力去改变他,既不能拿到学校与报馆两种教育机关的全部或一大部到手,或有口舌印刷物或一二学校报馆为宣传之具。正如朱子所谓“教学如扶醉人,扶得东来西又倒”,“直不足以动资本主义者心理的毫未,那有同心向善之望?以上从心理上说。再从历史上说,人类生活全是一种现实欲望的扩张。这种现实欲望,只向扩张的方面走,决不向减缩的方面走。小资本家必想做大资本家,大资本家必想做最大的资本家,是一定的心理。历史上凡是专制主义者或帝国主义者,或军阀主义者,非等到人家来推倒,决没有自己肯收场的。有拿破伦第一称帝失败了,又有拿破伦第三称帝。有袁世凯失败了,偏又有段祺瑞。章太炎在长沙演说,劝大家读历史,谓袁段等失败均为不读历史之故。我谓读历史是智慧的事,求逐所欲是冲动的事,智慧指导冲动,只能于相当范围有效力,一出范围,冲动使将智慧压倒,勇敢前进,必要回到了比冲动前进之力更大的力,然后才可以将他打回。有几句俗话:“人不到黄河心不死”,“这山望见那山高”,“人心不知足,得陇又望蜀”,均可以证明这个道理。以上从心理上及历史上看.可见资本主义是不能以些小教育之力推翻的,是第二层的理由。再说第三层理由,理想固要紧,现实尤其要紧,用和平方法去达到共产目的,要向何日才能成功?假如要一百年,这一百年中宛转呻吟的无产阶级,我们对之,如何处置,(就是我们)。无产阶级比有产阶级实在要多得若干倍,假定无产者占三分之二,则十五万万人类中有十万万无产者(恐怕还不只此数)这一百年中,任其为三分之一之资本家鱼肉,其何能忍?且无产者既已觉悟到自己应该有产,而现在受无产的痛苦是不应该;因无产的不安,而发生共产的要求;已经成了一种事实。事实是当前的,是不能消灭的,是知了就要行的。因此我觉得俄国的革命,和各国急进派共产党人数日见其多,组织日见其密,只是自然的结果.以上是第三层理由。再有一层,是我对于无政府主义的怀疑。我的理由却不仅无强权无组织的社会状态之不可能。我只忧一到这种社会状态实现了之难以终其局。

因为这种社会状态是定要造成人类死率减少而生率加多的,其结局必至于人满为患。如果不能做到(一)不吃饭;(二)不穿衣;(三)不住屋;(四)地球上各处气候寒暖,和土地肥瘦均一;或是(五)更发明无量可以住人的新地,是终于免不掉人满为患一个难关的。因上各层理由,所以我对于绝对的自由主义,无政府主义,及德谟克西拉的主义,依我现在的看法,部只认为于理论上说得好听,事实上是做不到的。因此我于子升和笙二兄的主张,不表同意。而于和森的主张,表示深切的赞同。

(三)态度问题。分学会的态度与会友的态度两种:学会的态度,我以为第一是“潜在”,这在上海半松园曾讨论过,今又为在法会友所赞成,总要算可以确定了。第二是“不依赖旧势力”,我们运学会是新的,是创造的,决不宜许旧势力混入,这一点要请大家注意。至于会友相互及会友个人的态度,我以为第一是“互助互勉”,(互助如急难上的互助,学问上的互助,事业上的互助。互勉如积极的勉为善,消极的勉去恶)。第二是诚恳(不滑)。第三是光明(人格的光明)。第四是向上(能变化气质有向上心)第一是“相互间”应该具有的,第二第三第四是“个人”应该具有的。以上学会的态度二项,会友的态度四项,是会友精神所寄,非常重要。

(四)求学问题。极端赞成诸君共同研究及分门研究之两法。诸君感于散处不便,谋合居一处,一面作工,一面有集会机缘,时常可以开共同的研究会,极善。长沙方面会友本在一起,诸君办法此间必要仿行。至分门研究之法,以主义为纲,以书报为目,分别阅读,互相交换,办法最好没有。我意凡有会员两人之处,即应照此组织。子升举力学之必要,谓我们常识尚不充足,我们同志中一尚无专门研究学术者,中国现在尚无可数的学者,诚哉不错!思想进步是生活事业进步之基。使思想进步的唯一方法,是研究学术。弟为荒学,甚为不安,以后必要照诸君的办法,发奋求学。

(五)会务进行问题。此节子升及和森意见最多。子升之“学会我见”十八项,弟皆赞成。其中“根本计划”之“确定会务进行方针”,“准备人才”,“准备经济”三条尤有卓见。以在民国二十五年前为纯粹预备时期,我以为尚要延长五年,以至民国三十年为纯粹预备期。子升所列以长沙方面诸条,以“综挈会务大纲,稳立基础”,“筹办小学”,“物色基本会员”三项,为最要紧,此外尚应加入“创立有价值之新事业数种”一项,子升所列之海外部,以法国、、俄国、南洋三方而为最重。弟意学会的运动,暂时可总括为四:1、湖南运动;2、南洋运动;3`留法运动;4留俄运动。暂时不必务广,以发展此四种,而使之确见成效为鹄。较为明切有着,诸君以何如?甚和森要我进行之“小学教育”,“劳动教育”,“合作运动”,“小册子”,“亲属聚居”,“帮助各团体”讲端,我都愿意进行。惟“贴邮花”一项我不懂意,请再见示。现在文化书社成立,基础可望稳固,营业亦可望发展。现有每县设一分社的计划,拟两年内办成,果办成,效自不小。

(六)同志联络问题。这项极为要紧,我以为我们七十几个会员,要以至诚恳切的心,分在各方面随时联络各人接近的同志,以携手共上于世界改造的道路。不分男、女、老、少、士、农、王、商、只要他心意诚恳,人格光明,思想向上,能得互助互勉之益,无不可与之联络,结为同心。此节和森信中详言,子升亦有提及。我觉得创造特别环境,改造中国与世界的大业,断不是少数人可以包办的,希望我们七十几个人,人人注意及此。

我的意见大略说完了。闻子升已回国到北京,不久可以面谈。请在法诸友:阵将我的意见加以批评,以期求得一个共同的决定。个人幸甚,学会幸甚。

\begin{flushright}
弟毛泽东

九年十二月一日
    
文化书社夜十二时\end{flushright}
\subsection{给和森的信(一九二一年一月二十一日)}
<p><font face="宋体">和森兄:

来信于年底始由子升转到。唯物史观是吾党哲学的根据,这是事实。不象唯理观之不能证实而容易被人摇动。我固无研究,但我现在不承认无政府的原理是可以证实的原理,有很强固的理由。一个工厂的政治组织(工厂分配管理等)与一个国的政治组织,与世界的政治组织,只有大小不同,没有性质不同。工团主义以国的政治组织与工厂的政治组织异性,认为另一回事而举以属之另一种人,不是因为曲说以冀苟且偷安,就是愚陋不明事理之正。况乎尚有非得政权则不能发动革命不能保护革命不能完成革命在手段上又有十分必要的理由呢。你这一封信见地极当,我没有一个字不赞成。党一层陈仲甫先生等已在进行组织,出版物一层上海出的“共产党”,你处谅可得到,颇不愧“旗帜鲜明”四字,(宣言即仲甫所为)。详情后报。

 弟泽东

 十年一月廿一日在城南

\section{非自杀(节录)}
\datesubtitle{(一九一九年十一月一日)}



吾人应主张与社会奋斗。…………反抗旧社全!与其自杀而死,宁奋斗被杀而亡。为目的达不到,截肠决战,玉碎而亡,则其天下之至刚至勇,而最足以印人脑府的了!

\begin{flushright}(原载长沙《大公报》,转抄自《光辉的五四》中国青年出版社一九五九年)\end{flushright}


\section{驱逐张敬尧文电}
\datesubtitle{(一九二〇年一月)}



……吾湘不幸,叠受兵凶,连亘数年,疮痍满目。上岁张敬尧入湘以后,纵饿狼之兵,奸焚劫杀,聘猛虎之政,铲刮诈捐。卖公地,卖湖田,卖矿厂,卖纱厂,公家之财产已罄;加米捐,加盐税,加纸捐,加田税,人民之膏脂全干。洎乎今日,富者贫,贫者死,困苦流离之况,令人不忍卒闻。被张贼兄弟累资各数千百,尚不自厌,连此仅存之米盐公款,竟思攫入私囊以甘心。去年张贼曾嗾使湘痞李鸣九等电京查问此款,至去长沙设立米盐公股清查处;闻近复贿令郭人漳等以旅京湘事维持会名义向熊秉三先生诘问该款,无理取闹;推其用意,无非欲攫尽湖南财产,吃尽湖南人民,以饱其欲壑。窃吾湘遭此亘创之际,哀哀子遗,非有亘资,何以善后,米盐公款为我三千万人民忍饥食淡所储蓄,秉必不易,利用尤殷,倘被鲸吞,此劫难复。被张贼之暴戾酷害,毒我湘人,已成惯技,独不能我同此食毛践士之败类,自杀父母之邦,甘与仇敌作狗。人之无良,至于此极!公民等对于郭人漳等此种恶劣行为,誓不承认。总之此款在张贼未去湘乱未宁以前,只可暂归湘绅保管,不得变动,俟湘事保定后,再由全省民意公决用途,此际倘来无耻之徒,希图破坏者,即视为公敌。凡我湘人,应知自卫,稍纵即逝,祈毋忽焉。湖南公民代表毛泽东等五十五人(下略)

\begin{flushright}(抄自中国革命历史博物馆)\end{flushright}


\section{发起“文化书社”缘起(摘要)}
\datesubtitle{(一九二○年七月三十一日)}



没有新文化,由于没有新思想;没有新思想,由于没有新研究;没有新研究,由于没有新材料。湖南人现在脑子饥荒,实在过于肚子饥荒,青年人尤其嗷嗷待哺。文化书社愿以最迅速、最简便的方法,介绍中外各种新杂志,以充青年及前进的全体湖南人新研究的材料。也许因而有新思想、新文化的产生,那真是我们心向祷之,希望不尽的。

文化书社由我们一些互相了解,完全信得过的人发起,不论谁投的本,永远不得收回,亦永远不要利息,此书社但永远为投本的人所共有。书社发达了,本身到了几百元,彼此不因为利;失败至于只剩一元,彼此无怨;大家共认地球之上,长沙城中,有此“共有”的一个书社罢了。

\begin{flushright}(摘自《光辉的五四》第91页,中国青年出版社1959版。)\end{flushright}


\section{湖南建设问题的根本问题一一湖南共和国}
\datesubtitle{(一九二○年九月三日)}



乡居寂静,一卧兼旬,九月一号到省,翻阅大公报,封面打了红色,中间有许多我们最喜欢的议论,引起我的高兴,很愿意继续将我的一些意思写出。

我是反对“大中华民国”的,我是主张“湖南共和国”的。有什么理由呢?

大概从前有一种谬论,就是“在今后世界能够争存的国家,必定是大国家。”这种议论的流毒,扩充帝国主义,压抑自国的弱小民族,在争海外殖民地,使半开化未开化之民族,变成完全奴隶,窒其生存向上,而惟使恭顺驯屈于己。最著的例,是英美德法俄奥,他们幸都收了其实没有成功的成功。还有一个,就是中国,连“其实没有成功的成功”,都没收得。收得的是满洲人消灭,蒙人,回人,藏人,奄奄欲死,十八省乱七八糟,造成三个政府,三个国会,二十个以上督军王,巡按使王,总司令王,老百姓天天被人杀死奸死,财产荡空,外债如山,号称共和民国,没有几个懂得“什么是共和”的国民。四万万人不少,有三万九千万,不晓得写信看报。全国没有一条自主的铁路。不能办邮政,不能驾“洋船”,不能经理食盐。十八省中象湖南四川广东福建浙江湖北一类的省,通变成被征服省,屡践他人的马蹄,受害无极。这些果都是谁之罪呢?我敢说,是帝国之罪,是大国之罪,是“在世界能够争有的国家必定是大国家”一种谬论的罪。根本的说,是人民的罪。

现在我们知道,世界的大国多半瓦解了,俄国的旗子变成了红色,完全是世界主义的平民天下。德国也染成了半红。波兰独立,捷克独立,匈牙利独立。犹太,阿拉伯,亚美尼亚,都重新建国。爱尔兰狂欲脱离英吉利,朝鲜狂欲脱离日本。在我们东北的西北利亚远东片土,亦越了三个政府。全世界风起云涌,“民族自决”,高唱入云,打破大国迷梦,知道是野心家欺人的鬼话,推翻帝国主义,不许他再亲作祟。全世界盖有好些人民,业已醒觉了。

中国呢?也醒觉了(除开政客官僚,军阀)。九年假共和大战乱的经验,迫人不得不醒觉,知道全国的总建设在一个期内完全无望,最好办法,是索性不谋总建设,索性分裂去谋各省的分建设,实行“各省人民自决主义”’二十二行省,三特区,两藩地,合共二十七个地方,最好分为二十七国。

湖南呢?至于我们湖南,尤其三千万人个个应该醒觉了。湖南人没有别的法子,唯一的法子,是湖南自决自治,是湖南人在湖南地域建设一个“湖南共和国”。我曾着实想过,救湖南救中国,图与全世界解放的民族携手,均非这样不行。湖南人没有把湖南自建为国的决心和勇气,湖南终究是没办法。

说湖南建设问题,我觉得这是一个根本问题,我颇有点意思要发表出来,乞吾三千万同胞的聪听,希望共起讨论这一个顶有意思的大问题。今天是个发端,余俟明日以后继该讨论。

{\raggedleft(原载长沙《大公报》,抄自王无为编《湖南自治运动史》上编,太东图书局发行,一九二○年十二月二十日初版)}



>>>>>>> 1ea63cda3a6acbb70ed9782cf60e3680d89eaa1e
\end{document}