\section[对《解放军报》1967年11月9日社论《抓好形势教育》的重要修改(一九六七年十一月)]{对《解放军报》1967年11月9日社论《抓好形势教育》的重要修改}
\datesubtitle{(一九六七年十一月)}


在社论的第三段,毛主席特地重新引用了1938年10月他在中国共产党六届六中全会上的报告中的如下两段话:“当前的运动的特点是什么?它有什么规律性?如何指导这个运动?这些都是实际的问题。”“运动在发展中,又有新的东西在前头,新东西是层出不穷的。研究这个运动的全面及其发展,是我们要时刻注意的大课题。”

(见《中国共产党在民族战争中的地位》,《毛泽东选集》第二卷第二版523页)

在同一段“紧跟毛主席的战略部署”一句后,毛主席添加了“要经常把运动中出现的新动向、新成就、新经验、新问题,用毛泽东思想进行分析和总结,及的.向广大指战员进行教育,使他们的思想能不断跟上发展着的新形势。”一句。

第四段,在“就是要用毛泽东思想统一人们对形势的认识”后,加了“识破阶级敌人的挑拨煽动”一句。

第六段,“无产阶级文化大革命的实践证明,只要毛泽东思想同亿万人民群众相结合”后,加了“除了叛徒、特务、顽固不化的党内走资派和社会上的牛鬼蛇神(即没有改造好的地、富、反、坏、右)以外”一句。


