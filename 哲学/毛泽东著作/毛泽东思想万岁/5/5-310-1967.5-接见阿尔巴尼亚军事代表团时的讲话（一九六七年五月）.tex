\section[接见阿尔巴尼亚军事代表团时的讲话(一九六七年五月)]{接见阿尔巴尼亚军事代表团时的讲话}
\datesubtitle{(一九六七年五月)}


我在一九六二年七千人大会上曾经讲过:“马列主义与修正主义的斗争,胜负还没有分晓,很可能修正主义胜利,我们失败。我们用可能失败去提醒大家,有利于促进大家对修正主义的警惕性,有利于防修、反修……。”实际上共产党内的两个阶级、两条路线的斗争,始终是存在着的,任何人否认都否认不了,我们是唯物主义者,我们当然不应该去否认它。自这次大会以后,两条路线、两个阶级在我们党内的斗争表现是形“左”实右与反形“左”实右,反对阶级斗争存在与强调阶级斗争存在,折中主义与突出无产阶级政治等等。在这以前适当的文件中已有了论述。

今天,贵国是以军事代表团来了解我国文化大革命的,我首先就这个问题谈谈看法。

我国的无产阶级文化大革命应该从一九六五年冬姚文元同志对“海瑞罢官”的批判开始。那个时候,我们这个国家在某些部门、某些地方被修正主义把持了。真是水泼不进,针插不进。当时我建议江青同志组织一下文章批判“海瑞罢官”,但就在这个红色城市无能为力,无奈只好到上海去组织。最后文章写好了,我看了三遍,认为基本可以,让江青同志拿回去发表。我建议再让一些中央领导同志看一下,但江青同志建议:“文章就这样发表的好,我看不用叫恩来同志、康生同志看了。”(林彪同志插话:有人说毛泽东同志就是拉一派打一派,现在中央领导同志凡是在革命群众中有威信的,全是毛主席事先将文化大革命的底交给他们了,所以他们未犯错误。我看无产阶级文化大革命倒是一个不考试的考试,谁能紧跟马列主义毛泽东思想,谁就是无产阶级革命派。所以我总说对毛泽东思想理解的要执行,暂时不理解的也要执行。)姚文元的文章发表以后,全国大多数的报纸都登了,但就是北京、湖南不登。后来我建议出小册子,也受到抵制,没有行得通。

姚文元的文章不过是无产阶级文化大革命的信号,所以我在中央特别主持制定了五月十六日的通知。因为敌人是非常敏感的,这里有一个信号。他那里就有行动。当然我们也必须行动。这个通知中已明显地提出了路线问题,也提出了两条路线的问题。\marginpar{\footnotesize 311}当时多数人不同意我的意见,暂时只剩下我自己,说我的看法过时了。我只好将我的看法带到八届十一中全会上去讨论。通过争论我才得到了半数多一点的同意。当时是有很多人仍然不通的,李井泉就不通,刘澜涛就不通。伯达同志找过他们谈,他们说:“我在北京不通,回去仍然不通。”最后我只能让实践去进一步检查吧。

八届十一中全会后,重点是在一九六六年十月、十一月、十二月三个月,对资产阶级反动路线进行了批判,这是公开地挑开了党内的矛盾。这里顺便提起一个问题,就是广大工农、党团干部,在批判反动路线过程中受了蒙蔽。我们研究,对受蒙蔽的同志怎样看?我从来认为,广大的工农兵是好的,绝大部分党团员是好的,无产阶级各个时期的革命,他们都是主力军。无产阶级文化大革命更不能例外。广大的工农是具体的劳动者,自然了解上层的情况少,加上广大党团骨干是内心对党、对党的干部无限热爱,而“走资本主义道路的当权派”又都是打着红旗反红旗,所以他们受了蒙蔽,甚至较长一段时间内转不过来,这里有历史因素的。受了蒙蔽改了就算了嘛!随着运动深入发展他们又成了主力军了。“一月风暴”就是工人搞起来的,随着全国工农起来了。这是革命发展规律,民主革命是如此,无产阶级文化大革命也是如此。“五四”运动是知识分子搞起来的,充分体现了知识分子先知先觉,但真正的北伐长征式的彻底革命就要依靠时代的主人做主力军去完成,靠工农兵去完成。工农兵实际上只不过是工农,因为兵只不过是穿军装的工农。批判资产阶级反动路线是知识分子、广大青年学生搞起来的,但“一月风暴”夺权,彻底革命就要依靠时代的主人,广大工农兵做主力军去完成。知识分子从来就是转变觉察的快,但受到本能的限制,缺乏彻底革命性,往往带有投机性。

无产阶级文化大革命,从政策策略上讲大致可分为四个阶段:从姚文元同志文章发表到八届十一中全会,这可算做第一阶段,这主要是发动阶段。八届十一中全会到“一月风暴”这算做第二阶段。第三阶段为戚本禹的《爱国主义还是卖国主义?》及《〈修养〉的要害是背叛无产阶级专政》。以后,可算做第四阶段。第三、第四阶段都是夺权问题。第四阶段是在思想上夺修正主义的权,夺资产阶级的权。所以这是两个阶级、两条道路、两条路线斗争决战的关键阶段,是主题,是正题。本来“一月风暴”以后,中央就一再着急大联合问题,但未得奏效,后来发现这个主观愿望是不符合阶级斗争客观发展规律的。因为各个阶级、各个政治势力都还要顽强表现自己。资产阶级、小资产阶级思想没有任何阻力的泛滥出来了,因此破坏了大联合。大联合,揑是揑不成一个大联合的,捏合了还是要分,所以中央现在的态度只是促,不再揑了。拔苗助长的办法是不成的。这个阶级斗争的规律,不以任何人的主观意识为转移。在这个问题上是有很多例子可以说明的。××市的工代会、红代会、农代会。除了农代会打的比较少一点外,工代会、红代会彼此打得都热闹,看来××市革命委员会也还得改组。

本来想在知识分子中培养一些接班人,现在看来很不理想。在我看来,知识分子,包括仍在学校受教育的青年知识分子,从党内到党外,世界观基本上还是资产阶级的,因为解放十几年来,文化教育界是修正主义把持了,所以资产阶级思想溶化在他们的血液中。所以要革命的知识分子必须在这两个阶级、两条道路、两条路线斗争的关键阶段很好的改造世界观,否则就走向革命的反面,在这里我问大家一个问题:你们说文化大革命的目的是什么?(当场有人答:是斗争党内走资本主义道路当权派)斗争走资本主义道路当权派这是主要任务,绝不是目的。目的是解决世界观问题,挖掉修正主义根子的问题。\marginpar{\footnotesize 312}

中央一再强调要群众自己教育自己,自己解放自己。因为世界观是不能强加的。改造思想也必须是外因通过内因去起作用,内因是主要的。世界观不改造,无产阶级文化大革命怎么能叫胜利呢?世界观不改造,这次文化大革命出了两千走资本主义道路当权派,下次可能出四千。这次文化大革命代价是很大的,虽然解决两个阶级、两条道路斗争问题不是一、二次、三、四次文化大革命所能解决的,但这次文化大革命后,起码要巩固它十年。一个世纪内至多搞上它两三次,所以必须从挖修正主义根子着眼,以增强随时防修、反修的能力。在这里我问大家另外一个问题,你们说什么叫走资本主义道路的当权派?(众不语)所谓走资木主义道路的当权派,就是这些当权派走了资本主义道路吧!就是说,这些人在民主革命时期,对反对三座大山是积极参加的,但到全国解放后,反对资产阶级了,他们就不那么赞成了;在打土豪分田地时,他们是积极赞成并参加,但到全国解放后,农村要实行集体化时,他们就不那么赞成了。他不走社会主义道路,他现在又当权,那可不就叫走资本主义道路当权派吗!就算是“老干部遇到新问题”吧!这都叫“老干部遇到新问题”。但具有无产阶级世界观的人,就坚决走社会主义道路。具有资产阶级世界观的人,就要走资本主义道路。这就叫做资产阶级要按照资产阶级世界观改造世界,无产阶级要按照无产阶级世界观改造世界。有人在无产阶级文化大革命中犯了方向路线错误时,也说这是“老干部遇到了新问题”。但既然犯了错误,这就说明你这个老干部资产阶级世界观还未得到彻底改造。今后,老干部还会遇到更多的新问题,要想保证坚决走社会主义道路,就必须在思想上来个彻底的无产阶级革命化。我问大家,你们说究竟怎样具体地由社会主义走向共产主义,这是一个国家大事,世界的大事。

我说,革命小将革命精神很强烈,这很好。但你们现在不能上台,你们现在上台明天就会被人赶下台。但这话被一位副总理从自己话里说出去了,这很不对。对革命小将是个培养问题,不能在他们犯有某种错误的时候,用这话给他们泼冷水。有人说选举很好,很民主,我看选举是个文明的字句,我就不承认有真正的选举。我是北京区选我作人民代表的,北京市有几个真正了解我?我认为周恩来当总理就是中央派的。也有人说中国是酷爱和平的,我看就不那么样达到酷爱的程度。我看中国人民还是好斗的。

对待干部必须首先建立一个相信95$\%$以上的人是好的,或比较好的信念,不能离开这个阶级观点!对革命的及要革命的领导干部,就要保,要理直气壮的保,要从错误中把他们解放出来。就是走了资本主义道路,经过长期教育,改正了错误,还是允许他们革命的。真正的坏人并不多,在群众中最多百分之五,党团内部是百分之一、二,顽固定资本主义道路的当权派只是一小撮。但这一小撮党内走资本主义道路的当权派,我们必须作为主要对象打,因为他们的影响及流毒是深远巨大的。所以也是我们这次文化大革命的主要任务。群众中的坏人,最多只是百分之五,但他们是分散的,没有力量的,如按百分之五,三千五百万算,如他们组成一支军队,有组织地反对我们,那确实是值得我们考虑的问题,但他们分散在各地没有力量,所以不能作为无产阶级文化大革命的主要对象。但要提高警惕,尤其在斗争的关键阶段,更要防止坏人钻空子。所以大联合应有两个前提:一是破私立公,一是必须经过斗争。不经过斗争的大联合是不能奏效的。

这次文化大革命的第四阶段,是两个阶级、两条道路、两条路线斗争的关键阶段,所以安排大批判的时间比较长,中央文革还在讨论,有人认为今年年底为宜,有人严为明年五月份为宜。但时间还得服从阶级斗争的规律。\marginpar{\footnotesize 313}

