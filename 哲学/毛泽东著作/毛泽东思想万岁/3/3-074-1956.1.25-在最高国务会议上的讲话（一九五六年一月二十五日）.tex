\section[在最高国务会议上的讲话(一九五六年一月二十五日)]{在最高国务会议上的讲话}
\datesubtitle{(一九五六年一月二十五日)}


目前我国正处在伟大的社会主义革命的高潮中。中华人民共和国的成立标志着中国革命由资产阶级民主革命阶段转变到社会主义革命阶段,即进入由资本主义到社会主义的过渡时期。在过去的六年中,前三年的工作主要是恢复国民经济和进行前一革命阶段中没有完成的各项社会改革,主要是土地改革。从去年夏季以来,社会主义改造,也就是社会主义革命就以极广阔的规模和极深刻的程度展开起来。大约再有三年的时间,社会主义革命就可以在全国范围内基本上完成。

社会主义革命的目的是为了解放生产力。农业和手工业由个体所有制转变成为社会主义的集体所有制,私营工商业由资本主义所有制变为社会主义所有制。必然使生产力大大地获得解放。这样就为大大的发展工业和农业的生产创造了社会条件。

我们进行社会主义革命所用的方法是和平的方法。对于这种方法,过去在共产党内和共产党外都有许多人表示怀疑。但是从去年夏季以来,由于农村中合作化运动的高潮和最近几个月以来城市中社会主义改造的高潮,他们的疑问已经大体解决了。在我国的条件下,用和平的方法,即用说服教育的方法,不但可以改变个体的所有制为社会主义的集体所有制,而且可以改变资本主义所有制为社会主义所有制。过去几个月来社会主义改造的速度大大超过了人们的意料。过去有些人怕社会主义这一关难过,现住看来,这一关也还是容易过的。

目前我们国家的政治形势已经起了根本的变化。去年夏季以前在农业方面存在的许多困难情况现在已经基本上改变了,许多曾经被认为办不到的事情现在也可以办到了。我国的第一个五年计划有可能提前完成或者超额完成。1956年到1967年全国农业发展纲要的任务,就是在这个社会主义改造和社会主义建设的高潮的基础上,给农业生产和农村工作的发展指出一个远景,作为全国农民和农业工作者的奋斗目标,农业以外的各项工作,也都必须迅速赶上,以适应社会主义革命高潮的新形势。

我国人民应该有一个远大的规划,要在几十年内,努力改变我国在经济上和科学文化上的落后状况,迅速达到世界上的先进水平。为了实现这个伟大的目标,决定一切的是要有干部,要有数量足够的、优秀的科学技术专家;同时要继续巩固和扩大人民民主统一战线,团结一切可以团结的力量。我国人民还要同全世界各国人民团结一起,为维护世界的和平而奋斗。

<p align="right">(《新华半月刊》一九五六年第四号)</p>

(注:这次会议是毛主席为讨论全国农业发展纲要草案而召集的。)


