\section{“湖南自治运动”应该发起了}
\datesubtitle{(一九二〇年九月二十六日)}



无论什么事,有一种“理论”而没有一种“运动”继起,这种“理论”的目的,是不能实现出来的。湖南自治,固然要从“自治的以必要”,“现在是湖南谋自治的最好机会”,“湖南及湖南人确有自立自治的要素与能力”等理论上,加以鼓吹推究,以引起尚未觉悟的湖南人的兴趣和勇气。但若不纵之以实际的运动,湖南自治,仍旧只在纸上好看,或在口中好听,终究不能实现出来。并且在理论上,好多人从饱受痛苦后的直感中,业已明白了。故现在所缺少的,只在实际的运动,而现在最急须的便也只在这实际的运动。

我觉得实际的运动有两种:一种是入于其中而为具体建设的运动,一种是立于外而为促进的运动。两者均属重要。两者均属重要,而后者在现在及将来尤为必须,差不多可说湖南自治的成不成好不好,都系在这种运动的身上。

我又觉得湖南的运动应该由“民”来发起的。假如这一回湖南自治,真个办成了,而成的原因,不在于“民”,乃在于“民”以外,我敢断言这种自治,是不能长久的。虽则具了外形,其内容是打开看不得的,打开看时,一定是腐败的,虚伪的,空的,或者是干的。

“湖南南自治运动”,在此时一定要发起了。我们不必去做具体的建设运动,却不可不做促进的运动。我们不必因为人数少便不做,人数尽管少,只要有真诚,效力总是有的。什么事情,都不是一起便可以成功,一起便可以得到多数的同情与帮助,都是从近及远,从少到多,从小至大的。颇有人说湖南民智末开,交通不便,自治难于办好的话,我看大家不要信这种谬脸。

(同上)

