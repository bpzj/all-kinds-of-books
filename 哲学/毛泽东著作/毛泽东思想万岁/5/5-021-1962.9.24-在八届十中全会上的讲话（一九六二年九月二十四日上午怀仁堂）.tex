\section[在八届十中全会上的讲话(一九六二年九月二十四日上午怀仁堂)]{在八届十中全会上的讲话}
\datesubtitle{(一九六二年九月二十四日上午怀仁堂)}


现在是十点,开会。

这次中央全会解决了几个重大问题:一是农业问题;二是商业问题,这是两个重要问题,还有工业问题,计划问题,这是第二位的问题,第三个是党内团结问题。有几位同志讲话,农业问题由陈伯达同志说明,商业问题由李先念同志说明,工业计划问题由李富春、薄××说明。另外,还有监察委员会扩大名额问题,干部上下左右交流问题。

会议不是今天开始的,这个会开了两个多月了,在北戴河开了一个月,到北京差不多也是一个月。实际问题在八、九两月,各个小组(在座的人都参加了)经过小组,实际上是大组,都讨论清楚了,现在开大会不需要很多时间了,大约三天就够了,二十七号不够就开到二十八号,至迟二十八号要结束。

我在北戴河提出三个问题:阶级、形势、矛盾。阶级问题,提出这个问题,因为阶级问题没有解决。国内不要讲了。国际形势,有帝国主义、民族主义、修正主义存在,那是资产阶级国家,是没有解决阶级问题的,所以我们有反帝任务,有支持民族解放运动的任务,就是要支持亚、非、拉三大洲广大的人民群众,包括工人、农民、革命的民族资产阶级和革命的知识分子。我们要团结这么多的人,但不包括反动的民族资产阶级,如尼赫鲁,也不包括反动的资产阶级知识分子,如日共叛徒春日庄次郎,主张结构改革论,有七、八个人。

那末,社会主义国家有没有阶级存在?有没有阶级斗争?现在可以肯定,社会主义国家有阶级存在,阶级斗争肯定是存在的。列宁曾经说,革命胜利后,本国被推翻的阶级,因为国际上有资产阶级存在,国内还有资产阶级残余,小资产阶级的存在,不断产生资产阶级,因此,被推翻了的阶级还是长期存在的,甚至要复辟的。欧洲资产阶级革命,如英国、法国等都曾几次反复。社会主义国家也可能出现这种反复,如南斯拉夫就变质了,是修正主义了,由工人、农民的国家变成一个反动的民族主义分子统治的国家。我们这个国家就要好好掌握,好好认识,好好研究这个问题。要承认阶级长期存在,承认阶级与阶级斗争,反动阶级可能复辟。要提高警惕,要好好教育青年人,教育于部,教育群众,教育中层和基层干部,老干部也要研究,教育。不然,我们这样的国家还会走向反面。走向反面也没有什么要紧,还要来个否定的否定,以后又会走向反面。\marginpar{\footnotesize 34}如果我们的儿子一代搞修正主义,走向反面,虽然名为社会主义,实际是资本主义,我们的孙子肯定会起来暴动的,推翻他们的老子,因为群众不满意。所以我们从现在起就必须年年讲,月月讲,天天讲,开大会讲,开党代会讲,开全会讲,开一次会就讲,使我们对这个问题有一条比较清醒的马克思列宁主义的路线。

国内形势:过去几年不大好,现在已经开始好转。一九五九年、一九六〇年,因为办错了一些事情,主要由于认识问题,多数人没有经验。主要是高征购,没有那么多粮食,硬说有。瞎指挥,农业、工业都有瞎指挥。还有几个大办的错误。一九六〇年下半年就开始纠正。说起来就早了,一九五八年十月第一次郑州会议开始了,然后十一月、十二月武昌会议,一九五九年二、三月第二次郑州会议,然后四月上海会议,就注意纠正。这中间,一九六〇年有一段时间对这个问题讲的不够,因为修正主义来了,压我们,注意反对赫鲁晓夫了。从一九五八年下半年开始,他就想封锁中国海岸,要在我们国家搞共同舰队,控制沿海,要封锁我们。赫来我国就是为了这个问题。然后是一九五九年九月中印边界问题,赫支持尼攻击我们,塔斯社发表声明。以后赫来,十月在我国国庆十周年宴会上,在我们讲坛上攻击我们。然后一九六〇年布加勒斯特会议围剿我们。然后两党会议,二十六国起草委员会,八十一国莫斯科会议,还有一个华沙会议,都是马列主义与修正主义的争论,一九六〇年一年,与赫打仗。你看,社会主义国家,马列主义中出现这样的问题,其实根子很远,事情很早就发生了,就是不许中国革命。那是一九四五年,斯大林就是阻止中国革命,说不能打内战,要与蒋介石合作,否则中华民族就要灭亡。当时我们没有执行,革命胜利了。革命胜利后,又怀疑中国是南斯拉夫,我就变成铁托。以后到莫斯科,签订中苏同盟互助条约,也是经过一场斗争的,他不愿签,经过两个月的谈判最后签了。斯大林相信我们是从什么时候起呢?是从抗美援朝起。一九五〇年冬季,相信我们不是铁托,不是南斯拉夫了。但是,现在我们又变成“左倾冒险主义”、“民族主义”、“教条主义”、“宗派主义”者了。而南斯拉夫倒变成“马列主义”者了。现在南斯拉夫可行啊,吃得开了,听说变成了“社会主义”,所以社会主义阵营内部也是复杂的,其实也是简单的。道理就是一条,就是阶级斗争问题。无产阶级与资产阶级的斗争问题,马列主义与反马列主义的斗争问题,马列主义与修正主义之间的斗争的问题。

至于形势,无论国际、国内都是好的。开国初期,包括我在内,还有刘××,曾经有这个看法,认为亚洲的党和工会、非洲党,恐怕受摧残。后来证明,这个看法是不正确的,不是我们所想的。第二次世界大战后,蓬蓬勃勃的民族解放斗争,无论亚洲、非洲、拉丁美洲都是一年比一年地发展的。出现了古巴革命,出现了阿尔及利亚独立,出现了印尼亚洲运动会、几万人示威,打烂印度领事馆,印度孤立,西伊里安荷兰交出来了,出现了越南南部的武装斗争,那是很好的武装斗争,出现了苏伊士运河事件,埃及独立,阿联偏右,出现了伊拉克,两个都是中间偏右的,但它反帝。阿尔及利亚不到一千万人口,法国八十万军队,打了七、八年之久,结果阿尔及利亚胜利了。所以,国际形势很好。陈毅同志作了一个很好的报告。

所谓矛盾,是我们同帝国主义的矛盾,全世界人民同帝国主义的矛盾,是主要的。各国人民反对反动资产阶级,各国人民反对反动的民族主义,各国人民同修正主义的矛盾,帝国主义国家之间的矛盾,民族主义国家与帝国主义国家之间的矛盾,帝国主义国家内部的矛盾,社会主义与帝国主义之间的矛盾。中国的右倾机会主义,看来改个名字好,叫做中国的修正主义。从北戴河到北京的两个月会议,是两种性质的问题,一种是工作问题,一种是阶级斗争的问题,就是马克思主义与修正主义的斗争。\marginpar{\footnotesize 35}工作问题也是与资产阶级思想斗争的问题。工作问题有几个文件,有工业的、农业的、商业的等,有几个同志讲话。

关于党如何对待国内、党内的修正主义问题,资产阶级问题,我看还是照我们原来的方针不变。不论犯了什么错误的同志,还是一九四二年到一九四五年整风时的那个路线,只要认真改变,都表示欢迎,就要团结他,要团结,治病救人,惩前毖后,团结——批评——团结。但是,是非要搞清楚,不能吞吞吐吐,敲一下吐一点,不能采取这样的态度。为什么和尚念经要敲木鱼?《西游记》里讲,取回的经被黑鱼精吃了,敲一下吐一个字,就是这么来的。不要采取这种态度,和黑鱼精一样,要好好想想。犯了错误的同志,只要认识错误,回到马克思主义的立场方面来,我们就与你团结。在座的几位同志,我欢迎,不要犯了错误见不得人。我们允许犯错误,你已经犯了嘛!也允许改正错误。不要不允许犯错误,不许改正错误。有许多同志改的好,改好了就好嘛!李××的发言就是现身说法。李××的错误改了,我们信任嘛!一看二帮,我们坚决这样做。还有很多同志,我也犯过错误,去年我就讲了,你们也要允许我犯错误,允许我改正错误,改了,你们也欢迎。去年我讲,对人是要分析的,人是不能不犯错误的。所谓圣人,说圣人没有缺点是形而上学的观点,而不是马克思主义、辩证唯物主义的观点。任何事物都是可以分析的,我劝同志们,无论是里通外国的也好,搞什么秘密反党小集团的也好,只要把那一套统统倒出来,真正实事求是讲出来,我们就欢迎,还给工作做,决不采取不理他们的态度,更不采取杀头的办法。杀戒不可开,许多反革命都没有杀,潘汉年是一个反革命嘛,胡风、饶漱石也是反革命嘛,我们都没有杀嘛。宣统皇帝是不是反革命?还有王耀武、康泽、杜聿明、杨广等战犯,也有一大批没杀。多少人改正了错误就赦免他们嘛,我们也没有杀。右派改了的摘了帽子嘛。近日平反之风不对,真正错了才平反,搞对了不能平反。真错了的平反,全错全平反,部分错了部分平反,没有错的不平反,不能一律都平反。

工作问题,还请同志们注意,阶级斗争不要影响了我们的工作。一九五九年第一次庐山会议本来是搞工作的,后来出了彭德怀,说:“你操了我四十天娘,我操你二十天娘不行?”这一操,就被扰乱了,工作受到影响。二十天还不够,我们把工作丢了。这次可不能,这次传达要注意,各地、各部们要把工作放在第一位,工作与阶级斗争要平行,阶级斗争不要放在很突出的地位(抄者按:此指对敌斗争)。现在已经组成两个专案审查委员会,把问题搞清楚,潘汉年是一个反革命嘛!胡风、饶漱石也是反革命嘛,我们也没有杀嘛。不要因阶级斗争干扰我们的工作,等下次或再下次全会再未搞清楚。把问题说清楚,要说服人。阶级斗争要搞,但要有专门人搞这个工作。公安部门是专搞阶级斗争的,它的主要任务是对付敌人的破坏。有人搞破坏工作,我们开杀戒,只是对那些破坏工厂,破坏桥梁,在广州边界搞爆破案,杀人放火的人。保卫工作要保卫我们的事业,保卫工厂、企业、公社、生产队、学校、政府、军队、党、群众团体,还有文化机关,包括报馆、刊物、新闻社。保卫上层建筑。

现在不是写小说盛行吗?利用写小说搞反党活动,是一大发明。凡是要想推翻一个政权,先要制造舆论,要搞意识形态,搞上层建筑,革命如此,反革命也如此。我们的意识形态是搞革命的,马克思的学说,列宁的学说,马列主义普遍真理和中国革命具体实践相结合。结合的好,问题就解决的好些,结合的不好就失败受挫折。讲社会主义建设时,也是普遍真理与建设相结合,现在是结合好了还是没有结合好?我们正在解决这个问题。军事建设也是如此。如前几年的军事路线与这几年的军事路线就不同。叶剑英同志搞了部著作,很尖锐,大关节是不糊涂的,我一向批评你不尖锐,这次可尖锐了。\marginpar{\footnotesize 36}我送你两句话:“诸葛一生唯谨慎,吕端大事不糊涂。”

请邓××宣布那几个人不参加全会。政治局常委决定五人不参加。

(×××宣布:政治局常委决定五个同志不参加全会:彭、习、张、黄、周,是被审查的主要分子,在审查期间,没有资格参加会议。)

因为他们的罪恶实在太大了,没有审查清楚以前,没有资格参加这次会议,也不参加重要会议,也不要他们上天安门。主要分子与非主要分子要有分析,是有区别的。非主要分子今天参加了会议。非主要分子彻底改正错误,给他们工作。主要分子如果彻底改正错误,也给工作。特别寄希望于非主要分子觉悟,当然也希望主要分子觉悟。

从现在起以后要年年讲阶级斗争,月月讲,开大会讲,党代会要讲,开一次会要讲一次,以使我们有清醒的马列主义的头脑。

一九五九年八届九中全会胜利粉碎了彭反党集团向党的进攻。十中全会又一次揭露彭反党活动一高饶反党分子成员习仲勋。


官僚主义小则误国误民,大则害国害民。

第一、不调查,不研究,脱离实际,脱离群众的官僚主义。

第二、主观瞎指挥的官僚主义。

第三、忙忙碌碌,不抓政治,迷失方向的官僚主义。

官僚主义发展严重了,一种革命意志衰退、腐化、堕落。一种是互相勾结,敌我不分,官僚主义是修正主义的温床。

治官僚主义的办法:接触群众,接触实际。

三自:固步自封,骄傲自满,夜郎自大。

一高:高官厚禄。

一爱:爱形而上学,爱好资产阶级思想方法。

爱好形而上学,缺乏两分法,这表现在爱讲成绩,不讲缺点,爱听表扬,不爱听批评,老虎屁股摸不得。

遇事不做全面分析,扶得东来西又倒。

