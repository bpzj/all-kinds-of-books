\section[在杭州会议上的讲话(二)(一九五八年一月四日)]{在杭州会议上的讲话(二)}
\datesubtitle{(一九五八年一月四日)}


《哲学研究》第五期上李×的一篇文章,第六期上冯××的一篇文章,都可以看一看。形式逻辑是量变阶段的科学,是辩证法的组成部分。量变与质变是对立的统一。事物有他的相对的固定性。定出计划,做出决议,是相对的平衡。定了以后,还是要变。平衡、巩固、一致……都是暂时的,不平衡、闹矛盾是绝对的。开会的都有散会的思想,越开得长,越想散会。

形式逻辑好比低级数学,辩证逻辑好比高级数学。这种说法值得研究。圆周割裂千万片它就方了。圆与方是对立的统一。

思维形式表现为:概念、判断、推理。形式逻辑是研究思维形式。形式逻辑有不少错误的大前提,因而不能得出正确的判断。可是按形式逻辑来说并不错。它只管数量。不管内容。内容是各种学部门的事。

昨天讲了两个十二条。下面再谈几个问题:

一、水、肥、土等十二条,要抓住,相互平衡。有水无肥,有肥无水都不行,是相互联系的。农业十年以后(也许要更多的时间)要实行电气化。电气化犁田。畜牧与肥料有关。又是动力,是肉食,是工业原料。除四害与劳动力有关,增加体力,振作精神。

二、工业、手工业、农业等十二条要抓住。

三、反浪费。上海材料,梅林罐头食品厂四年中浪费了四十五万,占资产之一半,八年可以建同样大的一个工厂。这是普遍性的问题。如果每个工厂、学校、机关、合作社都搞一下要节约多少物质。什么事都要有证明,没有证明人家都不信。只要一个材料就够了。剖一个麻雀就够了。不一定剖太多。

整风中以十天时间专搞反浪费(从放到改十天到十五天时间就可以)。可以搞几十亿。

四、消费和积累的比例要当作大问题来研究。这个比例有百分之四十五、百分之五十、百分之五十五、百分之六十等,各种分法要研究。春季即抓,不抓来不及了。这是大问题,搞得不好,工人农民不满意。一九五四年搞了九百六十亿斤,得罪了几亿农民。今年八百五十亿斤,暂时规定三年。我倾向于百分之五十。歉收、丰收有所不同。跟勤俭持家结合起米,可过日子。婚丧喜庆,一律从简。

五、要搞试验田。湖北省委关于试验田(的报告)是个好报告。(××插话:浙江省委的报告,因为有了《人民日报》的按语,大家就注意去看了。)以后翻译的书,没有序言不准出版。初版要有序言,二版修改也要有序言。《共产党宣言》有多少序言。许多十七、八世纪的东西,现在如何去看它。这也是理论与中国实际的结合,这是很大的事。

六、红与专。政治与业务是对立的统一。要做两方面的批判。专管政治,不熟悉业务不好。政治与业务就是红与专。搞政治的都要专有困难,但主要的部分要专一下。湖北省委试验田很见效,搞的时间并不长。

七、打掉官风。上海提出的。不要做官,统统把官风打掉,与老百姓平等。

八、按时交计划。

九、除四害。开展以除四害为中心的爱国卫生运动,每月大检查一次。“五年看三年,三年看头年”,这句话很好。

十、绿化。今年彻底抓一抓,做计划,大搞。听说三丈多高的树每天要吸收和散发一吨多水,会不会影响地下水?

十一、第二个五年计划。各省地方工业产值比例要超过农业生产(产值)(包括下放给地方管的工业在内)。要有全国平衡,不能无政府主义。

十二、开会办法。要大中小型。(各省)“非常会议”(如党代会)一年一、二次;中型几十人到二、三百人(如开县委书记会议),小型的如地委书记会议。要下去开会,了解他们,“非常会议”谈政治,业务会议也要谈政治。

十三、省委书记、委员轮流离开办公室,一年四个月,到处跑,可以釆取走马看花,下马看花两种办法。到一处谈三四个小时就走也好,下去一周半月也好。不一定到一个地方就是三、四个月。

十四、内部不要请客,不要要人家请吃馆子,不要迎送。不专搞舞会,不请看戏。谁去机场接客人,要处分被接的人,这是打掉官气的一个方面。

十五、关于两类矛盾问题。一个是敌我矛盾,一个是人民内部矛盾。人民内部矛盾有两类,一类是阶级斗争性质的,一类是工人阶级、劳动人民内部的,先进落后性质的。人民内部矛盾分两类,一是无产阶级同资产阶级、小资产阶级之间的矛盾,这是阶级矛盾,还有劳动人民内部的矛盾,这些劳动人民内部的矛盾,一部分属于阶级斗争性质的,如受封建思想影响,打老婆,甚至因此杀老婆,又如自由主义,个人主义(资产阶级、小资产阶级思想),绝对平均主义(小资产阶级思想)都反映着私人所有制问题,一部分是属于先进落后的,是认识上的问题,问题看不到底,譬如农业合作化高潮,在北京开会时还看不大清楚,会后南下.到山东,到其他地方一看,形势大变,才有把握写序言。这是先进落后之间的矛盾,形势看不清。有些人硬不肯增产,总认为没有条件。开展整风,右派一下子起来,几篇社论,六月到七月大局一定,许多事是无法完全预料到的。

主要的占大量的矛盾,阶级斗争是主要的,是要革一个东西嘛。宪法上规定三大改造,实际上是两大改造,改造资产阶级,改造小资产阶级。阶级矛盾是过渡时期的(主要矛盾),搞得好再有××年就行。××年加八年,共××年,不用××年。我们每年这样搞一次整风,就把资产阶级思想搞掉了。大量的是先进与落后的矛盾。同大量中间派的矛盾是阶级矛盾。富裕中农中百分之四十赞成合作化,百分之四十不那么热心,百分之二十想退社,未真心退社。坚决退的是少数,有百分之五可能是右派,他们还是劳动人民,不划右派,要七擒七纵。

产生人民内部矛盾的原因:

1.资产阶级、小资产阶级思想影响劳动人民。个人主义、自由主义、绝对平均主义、官僚主义(现在要挂在资产阶级账上)都是资产阶级、小资产阶级思想影响。

2.主观主义的原因。看不准,估计不足,右了。要经常提醒跟着形势前进。

3.有领导的原因。领导好一些,先进占多数,落后占少数。可以这样领导,可以那样领导(如浙江平阳和黄岩两县就不同),只要解剖一个麻雀,就可以了解整个气候。如何领导,很值得研究。山东寿张县有个刘传友是深入领导的。鲁西无喂猪习惯,现在每户养两头,他还改造土壤。浙江桐庐油厂、酒厂在同样条件下比出品率高低,使落后赶上先进的材料很好。《人民日报》王朴写的短评也写得好,里面有辩证法(均见一月三日《人民日报》)。要宣传理论,讲辩证法,讲唯物论,如上层建筑与经济基础,生产关系与生产力,是历史唯物论的基本内容。

经常把问题放在心上想一想,和个别同志,少数同志吹一吹,和自己的秘书平等地吹吹,看看他们的看法如何,找几个党委书记谈一谈,不作为决定,作为酝酿。在一个时期内把几个问题想一想,吹一吹,作为一个重要方法。不要事先不动脑筋,不想一想就开会。有些东西是慢慢成熟的。

十六、谈谈不断革命论。我说的不断革命论和托洛茨基讲的不同,是两种不断革命论。我们革命的步骤是:

1.夺取政权,把敌人打倒,这在一九四九年就完成了。

2.土地革命,一九五○年——一九五二年三月基本完成了。

3.再一次土地革命,社会主义的,现在讲主要生产资料集体所有,一九五五年也基本完成了,一九五六年有些尾巴。这三件事是紧跟着的,两个三年当中解决了。趁热打铁,这是策略性的,不能隔得太久,不能断气,不能去“建立新民主主义秩序”,如果建立了,就得再花力气去破坏。波匈“断”了这么多时候的“气”,资产阶级思想扎了根,再搞就不大好搞了,中农以上的就不想搞合作化了。保加利亚好些,百分之三十合作化。

4.思想战线上政治战线上的社会主义革命——整风运动,这一次今年上半年就可以完成。明年上半年还要搞。

5.还有技术革命。

1——4项都是属于经济基础和上层建筑性质的。土改是封建所有制的破坏,是属于生产关系的。技术革命是属于生产力、管理方法、操作方法的问题,第二个五年计划要搞。1、2,3项今后没有了,思想战线上政治战线上的革命仍旧有的,一个人过一两年又会生霉了。但重点放在技术革命。要大量发展技术专家,发动向技术好的人学习。在工厂、农村中有初级的技术家。红安县领导干部原来是空头政治专家,后来又红又专,工业找[照]桐庐县的方向,搞试验与技术革命联系起来,政治家与技术家结合起来。

从一九五八年起,在继续完成思想政治革命的同时,着重在技术革命方面,着重搞好技术革命。斯大林在提干部决定一切的口号时,也提出了技术革命。

着重搞好技术革命,不是说不要搞政治了,政治与技术不能脱离。思想政治是统帅,政治又是业务的保证。

消灭阶级再有××年就好了。以后,人与人之间思想政治斗争(或者叫革命)还会有,但性质不同,不是阶级关系,都是劳动人民先进与落后之间的矛盾。这时的斗争也是分两部分,一是劳动人民受资产阶级思想影响;一是属于主观主义——认识上的原因,或者还有领导的原因。懂得马克思主义领导艺术的好一些,否则差一些。“无冲突论”是形而上学的。为什么莫斯科宣言中加上辩论法一段呢,因为它适应过去,现在和将来。将来全世界都统一了,两派人争权还会有的,因为意见不一致,出各种报,演各种戏,各人争取群众。有思想交锋。那时上层建筑意识形态总是有的,有生产关系与生产力,有上层建筑与经济基础的矛盾,就会有左中右三种人。上层建筑在那些顽固落后的人手中,又不许大鸣大放了。不会不改正错误,也会有冲突。没有军队了,也可能用拳头,用木棒。那时没有阶级,处理得好,对抗,处理的不好,也会对抗。一个进步路线,一个落后路线,是互相排斥的,是对抗的。××年后,国家权力对内职能逐步地不存在,都是劳动人民了。现在对劳动人民来说,权力也基本不存在了。对劳动人民只能说服,不能压服,对劳动人民不能使用国家权力,用权力就是压服。好像很“左”,其实很右,是国民党作风。打倒官气,十分必要。对敌人威风凛凛是对的,对人民就不行了。

十七、政治家一定要懂些业务,农业搞试验田,工业搞试验品。要比较,比较是对立的统一。企业与企业之间,企业内部车间与车间、小组与小组、个人与个人之间是不平衡的。不平衡不仅是社会法则,也是宇宙法则。刚刚平衡,立即突破;刚刚平衡,又不平衡。要讲评比,在大体相同的条件下,先进与落后比较,是可比的,不是不可此的。

政治家总要懂得一些业务。技术上要比,政治上也要比。技术与政治相结合。看哪一个搞得好。

大家把几篇文章看一看,《解放日报》登的上海梅林厂开展反浪费专题鸣放,一月三日《人民日报》关于桐庐厂比出品率的报导和王朴写的短评)。一件事反映了全国性的事,大家一定要把人家的好事当作自己的好事。搞社会主义,不论问题出在哪里,都要当作自己的事。

学生中的右派,百分之八十可以留校继续读书,要加强对他们的工作,学生要和他们往来,逐步把他们化过来,他们做了好事也要表扬,当然也有假积极的。

不要以为经过这次整风,一切都是黄河为界,界线划得那么清楚。

