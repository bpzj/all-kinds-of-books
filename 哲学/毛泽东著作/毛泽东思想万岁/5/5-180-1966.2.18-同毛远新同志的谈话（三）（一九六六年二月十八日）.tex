\section[同毛远新同志的谈话(三)(一九六六年二月十八日)]{同毛远新同志的谈话(三)}
\datesubtitle{(一九六六年二月十八日)}


在谈到军事工程学院先搞二、三年,然后搞二年半工半读并结合预分时,毛主席说:

理工科还要有自己的语言,六年中先搞三年试试看,不一定急于搞二年。尖端科学搞三年,要有针对性也许行。三年不够,将来再补一点。有针对性才能少而精,有针对性才能一般和特殊相结合。六年搞成三年,这样做以后,步骤稳妥,方向对头。

新事物,干它几年,不断总结经验才行。

理工科有它的特殊性,有它自己的语言,要读一点书。但是也有共性,光读书不行。黄埔军校就读半年,毕业后当一年兵,出了不少人材,改成陆军大学以后(没有记下读几年),结果出来尽打败仗,做我们的俘虏。

理工科我是不懂的,医科我多少懂一点。你要听眼科大夫说话,神乎其神,但人总有一个整体。

科学的发展,由低级到高级,由简单到复杂,但讲课不能都按照发展顺序来讲,学历史主要学习近代史,现在有文字记载的历史才三千多年,要是到一万年该怎么讲呢?

尖端理论包括通过实践证明了的,有的基础理论中要去掉通过实践证明没有用的和不合理的部分。

讲原子物理只讲坂田模型就可以了,不必要从丹麦学派波尔理论开始。你们这样学十年也毕不了业。坂田都用辩证法,你们为什么不用?人认识事物总是从具体到抽象。医学讲心理学,讲神经系统那些抽象的东西,我看不对,应该先讲解剖学。数学本来是从物理模型中抽出来的,现在就不会把数学联系到物理模型来讲,反而把它进一步抽象化了。

