\section[在中央政治局常委扩大会议上的讲话(摘录)(一九六六年三月十七日至二十日)]{在中央政治局常委扩大会议上的讲话(摘录)}
\datesubtitle{(一九六六年三月十七日至二十日)}


我们在解放以后,对知识分子实行包下来的政策,有利也有弊。现在学术界和教育界是资产阶级知识分子掌握实权。社会主义革命越深入,他们就越抵抗,就越暴露出他们的反党反社会主义的面目。吴晗和翦伯赞等人是共产党员,也反共,实际上是国民党。\marginpar{\footnotesize 257}现在许多地方对于这个问题认识还很差,学术批判还没有开展起来。各地都要注意学校、报纸、刊物、出版社掌握在什么人手里,要对资产阶级的学术权威进行切实的批判。我们要培养自己的年青的学术权威。不要怕青年人犯“王法”,不要扣压他们的稿件。中宣部不要成为农村工作部。(注:中央农村工作部一九六二年被解散。)

《前线》也是吴晗、廖沬沙、邓拓的,是反党反社会主义的。

文、史、哲、法、经要搞文化大革命,要坚决批判,到底有多少马克思主义?

