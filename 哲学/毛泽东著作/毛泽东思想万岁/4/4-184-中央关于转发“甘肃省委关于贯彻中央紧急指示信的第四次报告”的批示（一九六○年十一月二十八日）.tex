\section[中央关于转发“甘肃省委关于贯彻中央紧急指示信的第四次报告”的批示(一九六○年十一月二十八日)]{中央关于转发“甘肃省委关于贯彻中央紧急指示信的第四次报告”的批示(一九六○年十一月二十八日)}


从现在起,至少(同志们注意.说的是至少)七年时间公社现行所有制不变。即使将来变的时候,也是队共社的产。而不是社共队的产,又规定从现在起至少二十年内社会主义制度(各尽所能,按劳分配)坚决不变.二十年后是否能变。要看那时情况才能决定。所以说“至少”二十年不变。至于人民公社队为基础的三级所有制规定至少七年不变,也是这样。一九六七年以后是否能变,要看那时情况决定,也许再加七年,成为十四年后才能改变。总之,无论何时,队的产业永远归队所有或使用。永远不许一平二调。公共积累一定不能多。公共工程也一定不能过多。不是死规定几年改变农村面貌,而是依情况一步一步地改变农村面貌。甘肃省委这个报告,没有提到生活安排,也没有提一、二、三类县、社、队的摸底和分析.这是缺点.这两个,关系甚大。请大家注意。


