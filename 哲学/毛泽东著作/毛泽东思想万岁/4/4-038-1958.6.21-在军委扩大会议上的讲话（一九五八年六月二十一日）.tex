\section[在军委扩大会议上的讲话(一九五八年六月二十一日)]{在军委扩大会议上的讲话}
\datesubtitle{(一九五八年六月二十一日)}


这次讲话不是指示。先讲几点意见供参考,是否对,供讨论。只当做问题提出,几年来习惯于这种形式,你们什么时候起来的?今天?我是昨天起来的。大家比我高明。经常接触实际。对军事没有抓。这几年,有同志批评了军队。但他也有责任,我有点单打一,这方式曾提倡过,但有时也有缺点。军事工作基本上作得好,有成绩,也有缺点。也怪我去抓别的了,有本书叫《香山记》讲观音菩萨。头一页上讲不唱天来不唱地,开头先唱香山记。事情总是这样的。我这几天,不唱天,不唱地,开始先唱别的戏。这事也难免,唱《打渔杀家》不能同时唱《西厢记》,如红线女每年有多少工作日?据说有二百个,其他的时间不搞工作。虽有理由。但也不大好,有个空军政委说,他这次参加会议很荣幸,沾一点边,是否中央间接有点责任,军委有责任,中央也有,这是提出了问题。这次检查首先检查军委和各部。不要针对各军区,他们可以同去检查。你们越压迫我,我越舒服,中国人不受帝国主义压迫不起来革命。人不逼不革命。帝国主义压迫劳动人民,实际作了自己掘墓人。当然你们的压迫不是帝国主义的.是督促的意思,否则我们不会动了。事情关系到六亿人民,关系到世界持久和平问题。

军委几年来工作基本对的,但有很大缺点,责任首先在领导。

国内形势起了很多变化,几亿人民起来奋斗,这次会议批发了富春、先念、薄××、冶金部、一机部几个文件,希望大家很好地阅读(主要讲了第二个五年计划问题)。值得仔细看看。现在我们国家的问题是:一曰粮,二曰钢,三曰机械。第一是粮,最重要,如若不相信,让厨房停伙三天行不行?现在的口号是:“苦战三年,争取每人千斤”(六千五百亿斤)今年可能增产一千亿斤,达到四千七百亿斤到四千八百亿斤。“苦战五年,争取达到每人一千五百斤”。到那时,我们的腰杆硬起来了。二曰钢,今年计划:五年之内争取六千万吨,要接近苏联(七千万吨)明年可能超过英国。从工业方面想办法,干部一年下去四次,抓四次。军队也如此,你们要我管少了,是要我多抓,驴驹子喜欢人骑,中央这些驴驹子也要有人骑。如果没有钢,不能造机械,炼钢本身也要机械(平炉、轧钢等)的发展,电力、石油、炼油、采矿都需要机械,还有交通运输机械(汽车、火车、轮船、飞机),农业也要机械(拖拉机、排灌机械等)。

当前是粮、钢、机械三个重要问题。

军委会开的很长,但还要准备很长的开,多则一年,少则三个月,最少一个半月。反正共产党会多,国民党税多。六十条中有一条抓军事,同样每年要搞几次军事,过去中央抓得不够,现在愿意抓了。现在各方面进步很快,首先是地方,特别是农业方面。这是说中央各部门是有缺点的,多了一点东西,少了一点东西,多了一点官气,少了一点政治。南宁、成都会议,八届二次会议后,政治扩大会议后,有很大进步,不能说他们“一多二少”了。官气少了,政治多了。军队有的同志讲,工、农、兵、学、商,兵在老三,现在落后了,如果有些落后,是领导上有缺点,像今天这样的会议开的很少,这几年主要抓农业,搞出一套办法。解决了,以后搞工商也都弄出来了一套,还有一“兵”和学都往那里开会(康生:前几天主席讲:工农兵学商,农搞出一套思想解放了,工业也基本上搞出一套,商(财贸)也如此,只是兵与学未搞出一套来,意即“兵”“学”落后、思想落后。)开始动了。(康生:前几天主席的话有道理,教育的确还未搞出一套中国形式的来。)民族的、大众的、科学的社会主义教育。民族的——马克思列宁主义普遍真理与中国实际特点相结合的;科学的——马克思主义辩证唯物主义,不是形而上学唯心主义;大众的——走群众路线的。教育与广大群众生产劳动相结合的。留同志们三天讨论一下这个问题。我们到底有何迷信?思想没解放?这次会议务实太多,务虚不够,讨论中的时间不够,也可以看出这问题。大跃进中,群众思想跃进,教育工作跃进,我们领导思想仍落后。(主席把军事与教育会议并列,军事会议比我们务实多,时间多,他们也没有那么多实的问题)我们这会有点事务主义,(他们的会还要开一个半月至两个月)这两方面都在开会。军队的落后,不是全部落后,中间脱节,可能上层热情少点,但也不是上层都如此,军事干部对我说:“各方面是大跃进,军事进步不够,心不安。”这是好的,经过这次会议后,会起变化的,军事会议要开得好,要讲清道理,达到一个目的。中国有这么多的人,六亿五千万,地方比苏联小一半,但是比较好是温带,海岸线长,达一万二千公里,全国面积九百五十万平方公里,这国家有山有平原,山是西边,平原在东边,这有好处,使水位落差比较多,可建水坝利用水,资产阶级民主革命比苏联早一些,从一九一一年辛亥革命算起,已有四十七年;俄国是一九一七年二月至十月,时间很短,中国从辛亥革命到一九四九年有三十八年,我们落后,但革命是自己搞成的,自己找马克思主义,并把马克思主义普遍真理与具体实践相结合,这是一种唯物主义的原理,也是一种辩证主义的原理,理论是从实践中来又作用于实践的。马克思主义普遍真理是什么?在莫斯科会议前,我们有五条,后变成九条,这种普遍真理与中国革命的具体实践相结合,不结合不行,因中国有其历史的特点,照搬不行,有过照搬,党的历史上曾犯过教条主义的错误,结果受到很大损失,但也有很多好处,给了我们教育,得了很多经验,同样的,凡是我们的敌人也给我们很多教育,也有好处。蒋介石、日本帝国主义、美国帝国主义等,再如陈独秀、罗章龙、张国焘,有些是党内性质的,王明现仍在党内,但他走到何处?将来再看,这人一开会就有病就是了。政治上犯错误,军事上也一定会犯错误。写过游击战争小册子,古田会议决议案是总结那时经验,划分资产阶级与无产阶级军事的界限,那时这种界限不清楚,四军九次代表大会军队中有两种路线,有些同志对资产阶级军事学,一套资产阶级管理制度很有兴趣,认为外行不能领导内行,所谓外行指当时军代表政委不能领导军队,这种现象过去存在,现在当然不存。有些同志认为要领导军队非学过军事不可,你们大家是否进过学校,没有的。这不是说过去在军队呆过的同志没用,他们起了很大作用,没有他们革命要迟好几年,俘虏兵也有作用。(过去内战中俘虏兵也是内行)。

如果是缺点,大家都有,四军九次代表大会有那么一股空气,对资产阶级军事学特别有兴趣。林彪虽进过旧学校,但他反对那一套,另搞了新花样。实际上资产阶级军事管理制度无非是打骂制度。这一时期,喜欢旧的资产阶级军事学,管理制度那一套是一种什么思想?可叫作资产阶级教条主义,后来又来了个无产阶级教条主义王明路线时代,有个外国人(李德)反对诱敌深入,反对游击战争思想,说是上山主义,他们要正规化,短促突击。御敌于国门之外。此时旧的资产阶级教条主义归降洋教条主义,无产阶级教条主义,实际也不是归降,而是两种教条主义合作,结果把我们的工作送掉了,来了个大游击,二万五千里长征,二万五千里穿过地球的中心。其他根据地也犯过错误,中央苏区还来了一个大扭秧歌,一直到陕北。

我们党有三次“左”路线,两次右倾路线,结果军队搞少了,受损失。但也有好处,取得经验,经过遵义会议,延安会议,延安整风,釆取“惩前毖后,治病救人”方针,把绝大多数同志说服过来,纠正过来。因此,不能说军事没犯过错误,正因为犯过错娱,蒋介石请我们走,我们就走了。长征中还有一些,一、四方面军会合,张国焘反中央。以后转入抗日战争时期,有些人犯过二次王明右倾机会主义错误。若按王明右倾机会主义,今天没办法在此怀仁堂开会和看梅兰芳的戏。

抗战后,解放战争中全党全军没有基本分歧。一九三八年六中扩大全会上讲述游击战争的好处,可建党建军建政,由九十万游击队变成四个野战军,主要说抗战,解放战争不是按洋教条,也不是按资产阶级教条办事。而是按照自己的一套,当然也参照外国经验,不是按着两种教条主义取得的胜利。

一九四九年胜利后,我们又打败了美国帝国主义,抗美援朝战争不是小战争。苏联同志讲过,在朝鲜战场上,敌空军的威力在第二次世界大战战场上未遇到过,当时我空军不能到第一线(敌后)。第二线(三八线)只能在后方顶一下,在当时情况下,打败美帝国主义很不容易。有人说我没参加世界大战,不对,我们参加过比第二次世界大战更厉害的战争。美国陆军怎么样?不清楚,总也不是轻而易举的。因此,我在战略上把美帝国主义看成是纸老虎,但是在战役和战术上要看成是铁老虎。

胜利后办了许多学校,产生了教条主义,请那么多苏联专家来,教条主义自然有了。到底有教条主义没有?对此有四种看法:(1)没有;(2)有,不多;(3)有,不少;(4)有,很多。那种对请大家讨论。

党的历史有两种教条主义,一种是资产阶级军事学和管理制度;一种是无产阶级接受外国的军事学和管理制度。对这两种都作过斗争,克服过。

解放后又出现了教条主义,是否有?我看有点,分量问题可研究,说完全没有,是不妥当的。不加分析的搬外国是妄自菲薄,不相信自己。有人说要学我的军事学,我没有。只写过几篇文章,那时有一肚子气,没有气也写不出来。王明由苏联回来,王明速胜论。(写文章说四年抗战胜利)当时有速胜论和亡国论两种。党内主要批评速胜论,对国民党则主要批评亡国论。那是应时文章。现在的一套我不懂了,小米加步枪,我是看过的,辛亥革命时也背过。后来南、北议和,成了外行。

我军有两种传统,一是优良传统,一是错误传统。一是马克思主义传统;一是非马克思主义传统。

这次会议要作点决定,大的一个,小的可几十个,成都会议就是这样(三十多个)。实事求是讲道理,大家说,大家要讲心里话,我们也提倡直道心事(坦白、交心)。直道工作,有人怕讲了穿小鞋,穿有什么要紧?共产党怕穿小鞋?中国过去女人都穿小鞋。三寸金莲不妨碍她们生四万万五千万人,共产党人怕什么?穿上几十双有何要紧?我在成都会议讲过五条,杀头当然不会,我们提倡交心,我向你们交心,得罪人有的,总之要插旗子,不插红的就插白的或灰的,不要讲出话来过多斟酌,讲话可以,今天讲话也是乱吹,写出稿子就困难了。目的是为了团结,全党全军团结,要团结起来,必须把问题搞清楚,否则不会真正团结,军人要爽快,交心才能达到团结。

为何要团结?要争取时间,最好十年不打仗,一九一八年至一九三九年,有二十一年不打仗,如果第三次世界大战照二十一年算,还有八年。那么我超过英国只要三年,或者明年就差不多,再有七年超过美国,现在蒋介石台湾很小,但还神气,常搞出飞机来。如果我有一亿五千万吨钢,那时只要吹口气,他就要走路了,现在打雷他不理,再有七年,有强大工业,苏联有七千万吨,我有六千万吨,一九六七年可以超过苏联,接近美国,十年可以超过美国(有把握超过),到那时导弹工业,原子弹许有可能。

部队一年要开一次会。心里有一个方向,不要糊里糊涂,对肖劲光要讲明白,海军重要,但现在还不行,可是大有希望。明年可能搞到两千五百吨钢,事情也难说。谁想一亩麦子产四千五百斤,粮食今年可增产一千亿斤,也有同志估计为一千三百亿厅,我估计增它百亿斤有把握,一千三百也不反对。总之,国家强大,军队也必然强大,有七年可能接近美国,十年超过美国,全党全军团结起来,为此而奋斗。


