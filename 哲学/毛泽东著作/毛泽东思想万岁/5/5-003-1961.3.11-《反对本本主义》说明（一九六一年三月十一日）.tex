\section[《反对本本主义》说明(一九六一年三月十一日)]{《反对本本主义》说明\\{\large(原名:“关于调查研究”)}}
\datesubtitle{(一九六一年三月十一日)\footnote{据《毛泽东年谱》,3月10日—13日 在广州小岛招待所主持召开三南会议,主要讨论人民公社体制和工作条例问题。同时,刘少奇、周恩来、陈云、邓小平于三月十一日至十三日在北京主持召开有中共中央西北局、东北局、华北局及所属各省、市、自治区负责人参加的工作会议(即三北会议),讨论的问题与三南会议相同。后来,北京方面向毛泽东建议,两边合起来开会,得到毛泽东同意。十四日,参加三北会议的同志到达广州。3月11日 同胡乔木、田家英谈《调查工作》一文修改问题。同日 为印发《调查工作》一文给三南会议写如下批语:“这是一篇老文章,是为了反对当时红军中的教条主义思想而写的。那时没有用‘教条主义’这个名称,我们叫它做‘本本主义’。写作时间大约在一九三〇年春季,已经三十年不见了。一九六一年一月,忽然从中央革命博物馆里找到,而中央革命博物馆是从福建龙岩地委找到的。看来还有些用处,印若干份供同志们参考。”并批注:“送林彪同志阅,一九三〇年的,从闽西找出来的。阅后退毛。”毛泽东在印发这篇文章时,对正文作了一些文字修改,将标题改为《关于调查工作》。}}

这是一篇老文章,是为了反对当时红军中的教条主义思想而写的。那时没有用“教条主义”这个名称,我们叫它做“本本主义”。\marginpar{6}写作时间大约在一九三〇年春季,已经三十年不见了。一九六一年一月,忽然从中央革命博物馆里找到。而中央革命博物馆是从福建龙岩地找到的。看来还有些用处。印若干分供同志们参考。

\kaitiqianming{毛泽东}
\kaoyouriqi{一九六一年三月十一日}

