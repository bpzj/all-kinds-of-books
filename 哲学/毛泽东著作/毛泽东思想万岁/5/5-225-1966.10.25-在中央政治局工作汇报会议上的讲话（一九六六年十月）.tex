\section[在中央政治局工作汇报会议上的讲话(一九六六年十月)]{在中央政治局工作汇报会议上的讲话}
\datesubtitle{(一九六六年十月)}


邓小平耳朵聋,一开会就在我很远的地方坐着。一九五九年以来,六年不向我汇报工作,书记处的工作他就抓彭真。你们不说他有能力吗?(聂荣臻说:这个人很懒。)

对形势的看法,两头小、中间大。“敢”字当头的只有河南,“怕”字当头的是多数。真正“反”字的还是少数。反党反社会主义分子有薄一波、何长工、汪锋,还有一个李范五。

真正四类干部(右派)也就是百分之一、二、三。(总理说:现在已经大大超过了。)多了不怕,将来平反嘛!有的不能在本地工作,可以调到别的地方工作。

河南一个书记搞生产,其余五个书记搞接待,全国只有刘建勋写了一张大字报,支持少数派,这是好的。

聂元梓现在怎么样?(康生说:还是要保。李先念说:所有写第一张大字报的人都要保护。)对!

(谈到大串连问题时总理说:需要有准备地进行。)要什么准备,走到哪里没饭吃?

对形势有不同看法,天津万晓棠死了以后,开了五十万人的追掉会,他们以为这是大好形势,实际上是向党示威,这是用死人压活人。

李富春休息一年,计委谁主持工作我都不知道。富春是守纪律的,有些事对书记处讲了,书记处没有向我讲。邓小平对我是敬而远之。

