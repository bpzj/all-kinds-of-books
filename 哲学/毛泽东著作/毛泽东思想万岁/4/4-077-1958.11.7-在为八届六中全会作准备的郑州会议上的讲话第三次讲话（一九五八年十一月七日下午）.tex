\section[在为八届六中全会作准备的郑州会议上的讲话第三次讲话(一九五八年十一月七日下午)]{在为八届六中全会作准备的郑州会议上的讲话第三次讲话(一九五八年十一月七日下午)}
\datesubtitle{(一九五八年十一月七日)}


徐水的全民所有制,不能算是建成社会主义。小全民,大集体,人力、财力、物力都不能调拨。这一点需要讲清楚。两者混同起来不利。现在不少干部模糊,如果说不是就是“右倾”。

有两种所有制,即全民与集体,但有一种起决定作用,即能调拨,不能服从全国的调拨,不能算是全民所有制。全民所有制的调拨,就不是政治经济学上的“商品”了。不完成“两化”产品不可能丰富,不可能直接交换,不可能废除商品交换。

三个差别(工农差别、城乡差别、体力与脑力差别),还要加上一个熟练与非熟练。

现在的生产关系还是小集体到大集体,互助组带有社会主义萌芽,由初级社到高级社,由高级社到人民公社,取消了自留地,经营范围扩大,教育、公共食堂、带小孩都由社会负担,废除家长制,吃饭不要钱,这是大变化。但所有这些变化,都是一个社、一个县范围内的变化,与社外无关。对全国来讲,还不是根本的变化。这还不如鞍钢,吃饭不要钱的问题,比鞍钢进步了。现在有些地方,废除了定额,按劳动强度、技术高低、态度好坏三者来作定额了,定额和计酬都有变化。

人民公社的性质是工、农、兵、学、商相结合的社会结构的基本单位,它的作用,主要是生产与生活组织者,同时又体现政权需保留的部分作用,现在有人不懂得政权的作用是对付地、富、反、坏的监督改造与对外保护社会主义建设,而不是用来对付解决人民内部问题,现在有人误用政权对待人民内部,如营长打连长,这是强迫命令。公社是大跃进的产物,不是偶然的,是实现两个过渡的最好的形式,又大又公,有利于过渡,也是将来共产主义社会的基层单位。

价值法则是一个工具,只起计算作用,但不起调节生产的作用。斯大林著作有许多好的东西。

全民所有制就是要产品调拨。

回去开会,征求意见,不要说郑州开了会。是不是全民所有制,是不是已经到了共产主义?共产主义因素算不算?不要把决议一下子推广出去。家庭要废除,言不由衷,口里很左,斩头去尾,父离子散。

县联社与一县一社,各县有差别,一县一社容易出秦始皇,联社不容易出秦始皇,秦始皇不好当,徐水县是独立王国,许多事情没有和省委、地委商量,省委、地委对它没办法。

徐水不如安国,以后要宣传安国,不要宣传徐水,徐水把好猪集中起来给人家看,不实事求是,有些地方放钢铁“卫星”的数目也不实在,这种作法不好,要克服,反对浮夸,要实事求是,不要虚假。大的方针政策要有个商量,领导机关要清醒。

田间管理,无人负责不好,搞责任制,不要混乱,统筹安排,将来产品分配,采取“三三”制。

城市公社,要搞,有先有后,搞的好的,北京、上海搞慢一点,搞快了,黄炎培怎么办?一个城市,一个公社恐怕也是联社性质,一定要有自己的经济基础。罗果夫赞成我们的公社,说苏联过去公社只是搞消费,我们的人民公社是以生产为中心的组织形式。城市分两种,一个大工厂的城市,市民想打主意,福利方面要开门。大工厂、大学属于国家,成员加入公社,但干部、产品不能调动,可以拨一点福利,帮助建立卫星厂,享受公社的权利,也尽义务,这是国营工厂与公社的关系,军队也是如此。

斯大林最后一封信,几乎全部错误,认为机器交给集体农庄,是倒退。

公社不能派工厂各种任务,为全国、全省服务的企业、学校一般放不下为公社所有。不这样,可能发生毛病,如石家庄的制药厂几乎停产,把人员拉去搞深翻地、炼钢。

工人制造价值大,应当略高于农民,明年应当考虑,把取消计件工资的损失补起来,并做到略略有所增加。

大集体、小自由。每家搞点锅灶.家庭作为历史作用的家庭是破坏了,消费还有部分存在,抚养还有一大部分,生育还存在,家长制和金钱系关破坏了,普遍的社会保险。中国的旧家庭是家庭的共产主义,每个家庭都是吃饭不要钱,但是不平等。

休息时间,公社社员一个月至少两天(女子要五天,月经期间要强调休息)。将来要做到六小时工作制、四小时学习,资本家、教授、民主人士、演员、运动员,要有所不同。

资本家定息不取消(不要可以,但不宣布),加入公社是自愿原则。干部待遇问题,要慎重。先试验,不要宣布太早(党员降低工资,也不必宣布。)。


