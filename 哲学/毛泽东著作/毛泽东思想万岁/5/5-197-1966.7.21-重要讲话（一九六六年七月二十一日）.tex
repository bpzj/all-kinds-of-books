\section[重要讲话(一九六六年七月二十一日)]{重要讲话}
\datesubtitle{(一九六六年七月二十一日)}


开两个会,讲了一些大学革命工作,主要讲工作组要撤,要改变派工作组的政策。前天讲工作组不行。前市委烂了,中宣部烂了,文化部烂了,高教部坏了,人民日报也不行。六一公布大字报,就考虑到非如此不可。文化革命就得靠他们去做,不靠他们靠谁?你去不了解情况,两个月也不了解,半年也不了解,一年也不行。如翦伯赞,写那么多书,你能看?能批判?只有他们能了解情况,我去也不行。只有依靠革命师生。现在总是怕字当头。总是怕乱。现在停课又管吃饭,吃了饭要发热。要闹事,不叫闹事干什么?只有依靠他们搞。照目前办法搞下去,两个月冷冷清清,搞到何年何月?昨天说,你们要改变派工作组的政策。现在工作组起了什么作用呢?一起阻碍作用,二不会:一不会斗,二不会改。我也不行,现在无非是搞革命,一是斗坏人,一是革思想。文化大革命,批判资产阶级的思想、权威,陆平有多大斗头?李达有多大斗头?翦伯赞出那么多书,你能斗了他?群众出对联,讲他是“庙小神灵大,池浅王八多”,搞他,你们行?我也不行,各省也不行。什么教学改革,我也不懂,只有依靠群众,然后集中起来。

工作组搞成联络员或是叫顾问。你们讲顾问权大,那还叫联络员。工作一个多月,起阻碍革命的作用,实际上是帮了反革命。有的工作组是坐山观虎斗,看着学生斗学生。西安交大限制人家打电话、打电报,限制人家上北京。要在文件上写上,可打电话,可打电报,可派人到中央,党章早就有了嘛!南京新华日报被包围,我看可以包围,三天不出报,没有什么了不起。你不革命就牵涉到你头上来。为什么不准包围省、市委、报馆、国务院?好人来了你们不见。你们不出去,我去见。你们又派小干部,自己不出去,我出去。总之,是怕字当头,怕

反革命,怕动刀动枪,都不下去,不到有乱子的地方去看看,李雪峰、吴德,你们不去看,天天忙具体事务。没有感性知识,如何指导?北京大学三次辩论,我看不错。所有到会的人都要到出乱子的地方去。有人怕讲话,叫讲就讲几句,我们是来学习的,是来支持你们革命的,召之即来,随叫随到,以后再来。

你叫革命师生一点毛病都没有,搞一、二个月一点感性知识也没有?你去就是叫围吗?广播学院、北师大打人问题,有人怕挨打,叫工作组保护自己。没有死人嘛!左派挨打是锻炼。总之,工作组是一不能斗,二不会改,半年不行,一年也不行,只有本单位的人才能斗,才会改。斗就是破,改就是立。教科书半年编不出来,我看可以去繁就简,错误的去掉,加来不及了,加要加中央社论和通知(有人提:加主席著作)那个是方向、指南,不能当了教条,如处理广播学院打人,哪本书上有?哪个将军打仗还翻书?现在这个阶段要把方向转过来。

文化革命委员会,要包括左中右,右派也要有几个。如翦伯赞可被右派用,也可被左派用,是个活字典,但不能集中,像中华书局那样,可搞个训练班,活字典,只要不是民愤极大的。代表会,革命委员会都要有个对立面,常委就不能要了。

除你(指李雪峰)那个市委,人不要多,多了他们就要革命、打电话、出报表,我这里就一个人嘛,很好嘛。现在部长很多都要秘书,统统去掉。我到延安前就没有。市委机关可搞个收发,你夫人(不知指谁)不要当秘书了,下去劳动嘛!国务院的部有的可改为科,庞大机关,历来无用。

有些人不想想,一不上课,二管饭吃,三要闹事,闹事就是要革命。工作组出来后,有些要复辟,复辟也不要紧,我们有的部长就那样可靠?有些部、报馆是谁掌握呀!我回北京四天后还倾向保现成的。许多工作组就是阻碍运动,如清华、北大。文件上要写上,行凶、杀人、放火、放毒的才叫反革命,写大字报,写反动标语的不能抓。有人写拥护党中央,反对毛泽东,你抓他干什么呀!他还拥护党中央嘛!不打历史反革命,留下用,表现不好的斗争嘛!不准打人,叫他们放嘛!贴几张大字报,几条反动标语怕什么?


