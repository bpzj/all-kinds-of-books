\section{学生之工作}
\datesubtitle{(一九一九年十二月廿八日)

(1919年12月28日湖南教育月刊)}



我数年来梦想新社会生活,而没有办法。七年春季,想邀数朋友在省城对岸岳麓山设工读同志会,从事半工半读,因他们多不能久在湖南,我亦有北京之游,事无他议。今春回湘,再发此这种想象,乃有在岳麓山建设新村的计议,而先从办一实行社会说本位教育说的学校入手,此新村以新家庭学校及旁的新社会连成一块为根本理想。对于学校的办法,曾草一计划,今抄上计划书中“学生之工作”一章于此,以求同志的敦诲。我觉得在岳麓山建设新村,似可成为一问题,倘有同志对于此问题有详细规划,或有何种实际的进行,实在欢迎希望的很。

(一)

学校之教授之时间,宜力求减少,使学生多自动研究及工作。应划分每日之时间为六分。其分配如左:

睡眠二分。

游息一分。

读书二分。

工作一分。

读书二分之中,自习一分,教授占一分。此时间实数分配,即

睡眠八小时

游息四小时

自习四小时

教授四小时

工作四小时

上例之工作四小时,乃实行工读主义所必具之一个要素。

(二)工作之事项,全然农村的。列举如左:(甲)种园。(1)花木(2)蔬菜

(乙)种田。(1)棉(2)稻及他种

(丙)种林。

(丁)畜牧。

(戊)种桑。

(己)鸡鱼。

(三)

工作须为生产的,与实际生活的。现时各学校之手工,其功用在练习手眼敏活,陶冶心思精细,启发守秩序之心,及审美之情,如为手工课之优点。然多非生产的(如纸、豆,泥、石膏各细工),作成之物,可玩而不可用。又非实际生活的,学生在学校所习,与社会之实际不相一致,结果则学生不熟谙社会内情,社会亦嫌恶学生。

在吾国现时,又有一弊,即学生毕业之后,多骛都市而不乐田园。农村的生活,非其所习,从而不为所乐。(不乐农村生活,尚有其他原因,今不具论)。此讫地方自治之举行有关系。学生多散布于农村之中,则或为发议之人,或为执行之人,即地方自治得学生之中坚而得举行。农村无学生,则地方自治缺乏中坚之人,有不能美满推行之患。又于政治亦有关系,现代政治,为代议政治,而代议政治之基础第于选举之上。民国成立以来,两次选举,殊非真正民意。而地方初选,劣绅恶棍,无选举投票乡民之多数,竟不知选举是甚么一回事,尤无民意可言。推其原因,则在缺乏有政治常识之人参与之故。有学生指导监督,则放弃选举权一事,可逐渐减少矣。

欲除去上所说之弊,(非生产的非实际生活的,骛于都市而不乐农村。)第一,须有一种经济的工作,可使之直接生产,其能力之使用,不论大小多寡,曾有成效可观。第二,此种工作之成品,必为现今社会普遍之需要。第三,此种工作之场所,必在农村之中,此种之工作,:逸为农村之工作。

上述之第一,所以使之直接生产。第二,所以使之合于实际生活。第三,所以养成乐于农村生活之习惯。

(四)

讫上文所举之外,尚有一要项,今述之于下。言世界改良进步者,皆知须自教育普及使人民咸有知识始。欲教育普及,又自兴办学校始。其言固为不错,然兴办学校,不过施行教育之一端。而教育之全体,不仅学校而止。其一端只。有家庭,一端则有社会。……

(第四部分的摘记:

学校家庭和社会的关系。

而教育之全体,……

学校与家庭的矛盾,斗争有两前提,故言改良学校教育,而不同时改良家庭与社会,所谓举中而遗其上下,得其一而失其二也。)

(五)

第二节所举田园树畜各项,普旧日农圃所为,不为新生活。以新精神经营,真则为新生活矣。旧日读书人不予农圃事,今一边读书,一边工作,以神圣视工作焉,则为新生活矣。号称士大夫有知识一流,多营逐于市场与官场,而农村新鲜之空气不之吸,优美之景色不之尝,吾人改而吸尝此新鲜之空气与优美之景色,则为新生活矣。

种园有,一种花木,为花园,一种蔬菜,为菜园,二者相当于今人所称之学校园。再扩充之,则为植物园。种田以棉与稻为主,大小麦、高梁、蜀黍等亦可间种,粗工学生所难为者,雇工助之。

种林须得山地,学生一朝手植,虽出校而仍留所造之材,可增其回念旧游爱重母校之心。

畜牧如牛、羊、猪等,在可能营养之范围内,皆可分别营养。

育蚕须先种桑,桑成饲蚕,男女生皆可为。养鸡鱼,亦生产之一项,学生所喜为者也。

(六)

各项工作,非欲一人做遍,乃使众人分工,一人只做一项,或一项以上。

学生认学校如其家庭,认所作田园林木等如其私物。由学生各个所有私物之联合,为一公共团体,此团体可名之曰“工读同志会”。会设生产、消费、储蓄诸部,学生出学校,在某期间内不取出会中所存的利益,在某期间外,可取去其利益之一部而留存其一部,用此方法,可使学生长久与学校有关系。

(七)

依第三节所述,现时各学校之手工种,为不生产的,所施之能力掷诸虚牝,是谓“能力不经济”。手工科以外,又有体操科亦然,各种之体操大抵智属于能力不经济类,今有各项工作,此两科目可废弃之。两科目之利,各项工作之中,亦可获得。

