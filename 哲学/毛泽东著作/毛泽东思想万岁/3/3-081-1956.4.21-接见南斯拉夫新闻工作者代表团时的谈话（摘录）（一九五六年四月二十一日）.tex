\section[接见南斯拉夫新闻工作者代表团时的谈话(摘录)(一九五六年四月二十一日)]{接见南斯拉夫新闻工作者代表团时的谈话(摘录)}
\datesubtitle{(一九五六年四月二十一日)}


……原子弹并不比刀枪厉害,你们相信不相信?古人用刀枪打了几百年。汉代中国人口有五千万,一打只剩一千万。唐代又发展到五千万,安禄山造反,一打只剩下一千多万。用冷武器打仗拖延很久,死人很多。大片土地无人,欧洲历史上可能也有这种情况。后来拿破仑时代开始发展到步枪,可打八百米远,以后又发展到大炮,用热武器打仗。我们没有试过原子弹,但可以算一算:全世界有二十五亿,原子弹就算杀掉一半还有十多亿,也比过去用冷武器时死的人少。我们中国有六亿人口,原子弹杀死掉一半,还有三亿人口,有二、三十年又恢复过来了。

……每一个民族里总有坏人的,不可能设想只有一种好人。一万年以后也是这样。因此,一万年以后戏台上还要演好人与坏人。假如没有坏人,没有好人与坏人的冲突和矛盾,那么就没有戏可看了。


