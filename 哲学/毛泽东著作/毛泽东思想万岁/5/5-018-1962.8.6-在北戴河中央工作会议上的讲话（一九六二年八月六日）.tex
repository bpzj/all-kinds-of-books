\section[在北戴河中央工作会议上的讲话(一九六二年八月六日)]{在北戴河中央工作会议上的讲话}
\datesubtitle{(一九六二年八月六日)}


先开工作会议,为中央全会准备文件。从明天起,开始讨论,刘××建议成立核心小组,还有许多小组,解决六个大组不能畅所欲言的问题。核心小组有常委、书记处,再加大区第一书记,中央各口负责同志,共二十三人:毛、刘××、周、朱×、邓××、彭×、富春、先念、谭××、伯达、陆××、富治、谷牧、罗××、陈毅、杨××,加上各大区第一书记。

一、社会主义国家,究竟存在不存在阶级?在外国有人讲,没有阶级了,因此党是全民的党,不是阶级的工具,无产阶级的党了。无产阶级专政不存在了,全民专政没有对象了,只有对外矛盾了。像我们这样的国家是否也适应?这个问题是否谈一下。我同几个大区的同志都谈了话,了解到有的人听说,国内还有阶级存在,大吃一惊。资产阶级右派从来不承认有阶级存在,认为没有阶级了,不要改造,不承认阶级斗争,说阶级斗争是马克思捏造出来的。资产阶级不承认阶级斗争,孙中山就不讲阶级,说只有大贫小贫之分。有没有阶级,这是个基本问题。

二、形势问题,也要谈一下,国际问题要找几个人准备一下,究竟是什么情况?帝国主义、修正主义、反动的民族主义、广大人民群众各阶层、民族资产阶级、农民、城市小资产阶级……

国内形势谈一谈。究竟这两年如何?有什么经验?过去几年有许多工作没搞好,有许多还是搞好了,如工业建设、农业建设、水利等等。有人说,农村去年比前年好,今年比去年好,这个说法对不对?工业上半年不那样好,有主客观原因,下半年怎样,还要看一看。有些同志过去曾经认为是一片光明,现在是一片黑暗,没有光明了。是不是一片黑暗?两种看法那种对?如果都不对,是不是应有第三种看法?不是一片黑暗,基本光明,有黑暗,问题不少,确实很大。回到一九五九年庐山会议的三句话。“成绩很大,问题不少,前途光明”。两年调整,彻底调整、巩固、充实、提高的方针做得不那么好。以农业为基础,讲了三年,一九五九年至一九六二年,四个年头,实际上没有实行。中央的东西,有些没有下去,有些成了废品。所谓没有实行,就是没有认真做,个别做了,或者做得很不好。形势问题,我倾向于不那么悲观,不是一片黑暗。现在一片光明的看法没有了,不存在。有些人思想混乱,没有前途,丧失信心,不对。

三、矛盾问题。有些什么矛盾?一类是敌我矛盾,一类是人民内部矛盾。人民内部矛盾有两类。有一种矛盾,对资产阶级的矛盾,实质上敌对的,是社会主义与资本主义的矛盾,我们当作人民内部矛盾处理,如果承认国内阶级还存在,就应该承认社会主义与资本主义的矛盾是存在的。阶级的残余是长期的,矛盾也是长期存在的,不是几十年,我想是几百年,究竟那一年进入社会主义,进入了社会主义是不是就没有矛盾了?没有阶级,就没有马克思主义了,就成了无矛盾论,无冲突论了。现在有一部分农民闹单干,究竟有百分之几十,有说百分之二十,安徽更多。就全国来说,这时期比较突出。究竞走社会主义还是走资本主义道路?农村合作化要不要?“包产到户”还是集体化?已经“包产到户”的,不要强迫纠正,要做工作。为什么要搞这么多文件?为了巩固集体经济。现在就有闹单干之风,越到上层越大,有阶级就有阶层,地、富残余还存在着,闹单干的是富裕阶层、中农阶层、地富残余,资产阶级争夺小资产阶级闹单干,如果无产阶级不注意领导,不做工作,就无法巩固集体经济,就可能搞资本主义。有些人也是要闹单干的。

再有,生产和分配的矛盾,积累与消费的矛盾。积累过多,消费就少了。

再有,集中与分散的矛盾,七千人大会之后,我看没有解决,还要继续做工作,民主与集中的矛盾,要用民主的方法达到集中的目的。要让人家讲话,不民主,集中不起来,还要做工作。

社会主义与资本主义,本质上是敌对矛盾,我们当作人民内部矛盾来处理。积累过多,民主与集中还要做工作。阶级存在不存在?国内形势如何?矛盾,一个是敌我矛盾,一个是人民内部矛盾。敌我矛盾有个肃反问题,还有反革命存在,要看到,看不到不好,看得太严重也不合乎事实。


