\section{给杨怀中先生的信(摘录)}


……怀中先生言,日本某君以东方思想均不切于实际生活,诚哉其言,吾意即西方思想亦未必尽是,几多之部分亦应与东方思想同时改造也。今入动教子弟宜立志,又曰某君有志愚,此最不通志者。吾有见宇宙之真理,照此以定,吾人之心之所之之谓也,今人所谓立志,如有志为军事家,有志为教育家,乃见前辈之行事,及近舍施为羡其成功,盲目以为己志。乃出于一种模仿性,真能欲立志不能如是容易,必先研究哲学伦理学.以其所得真理奉以为己身言动之准,立之为前途之鹄。

再择其合于此鹄之事,尽力为之,以为达到之方始谓之有志也,如此之志,方为真志,而非盲从之志,其始所谓立志只可谓之有求善之顷向或求真求美之顷向,不过一种冲动耳,非真正之志也,显然此志也容易立哉、十年未得真理即十年无志,终身未得则终身无志,此又学,之所以贵乎幼也。……

