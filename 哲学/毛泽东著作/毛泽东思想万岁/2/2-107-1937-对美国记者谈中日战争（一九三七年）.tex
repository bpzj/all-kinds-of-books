\section[对美国记者谈中日战争(一九三七年)]{对美国记者谈中日战争}
\datesubtitle{(一九三七年)}


<B>记者问:</B>你们主张全国联合起来,以与日本帝国主义作殊死战斗而将其驱逐出中国。这是不是说中国现在已无须外援而能单独战胜日本?

<B>毛答:</B>首先我要提醒你,中国和日本都不是一个孤立的国家,东方的治乱问题是世界的问题。日本暗中有其同盟国――如德意等,一一同时中国要想抗日成功也必须在其他强国中找到帮助。但这并不是说没有外援中国就不能和日本决战,这并不是说我们在抗日以前一定要找到外援。

中国蓄有巨大的潜力,在伟大的斗争中一定能够组织而成为有力的抵抗线。中国人民在过去的斗争中已经深切地了解这种力量,并且找到了运用她的良方。中国的大众靠了长久的政治经验已能熟练地运用他们有效的武器去反抗他们的敌人。

我们坚信中国人民决不投降日本帝国主义,我们要动员我们的伟大潜力以抵抗日本,以我们最大的精力和侵略者的挑战周旋,在这个斗争中,最后的胜利一定是中国的。假如中国单独作战,自然牺牲较大,战期较长,因为日军有优美的军备,此外它也有她的同盟国。所以要想用最短的时间,最少的牺牲去战胜日本帝国主义者,中国首先要在自己的境内组成联合战线,其次还要将它扩展到关心太平洋和平的列强。

<b>问:</b>在何等条件之下,中国人民能战胜日本的军力?

<b>答:</b>三个条件能保证我们的胜利:第一,中国民日本帝国主义联合战线之成立。二、世界反日战线之组成。三、日本帝国主义压迫下的人民革命之发动。但其中最主要的自然还是中国人民本身的联合。

<b>问:</b>请问这样的战争要拖延到多久?

<b>答:</b>这要看中国政治动员的力量。中国反日本的许多有利条件的因素,和国际援助中国的程度以及日本革命发展的迅速而言。假如中国的政治动员是强有力的,假如在纵的和横的方面都是有效的组织起来,假如明了自身利益受日本帝国主义威胁的国际方面大量地予中国以援助,假如日本的革命很快地爆发的话,那末这次战争是很快地就能得到胜利。假如这些条件还不能具备的话,那末这次战争是非常长久,但结局日本还是要败的,只不过牺牲重大,且成为全世界的一个痛苦时期。所以中国政府以及中国人民都已准备和任何国家联合以缩短战期。但假如无人与我们联合,我们还是要单独干的。

<b>问:</b>你以为苏联及外蒙是否要参加这次战争以助中国?并发生在何种形势之下?

<b>答:</b>当然苏联也不是一个孤独的国家,她不能漠视远东的事变。它是不得不动的。还是过(?)日本征服全中国后之以为攻击苏联的根据地?还是帮助中国人民反对日本以获得独立,而与苏联的人民建立友好的关系呢?我想苏联一定是釆取最后一条路途的。

我们相信只要中国人民开始抵抗并需要和苏联以及其他友邦建立友好的同盟时,那末苏联一定会与我握手做先锋的,反对日本帝国主义是一个世界的事业,苏联既是世界的一部分,她自然不会较英美更守中立。

翻印者说明:此文抄自《抗战文选》,文中有该记者一段自述,声称:他是七月初旬到陕北,逗留了四个月,到达陕甘宁,并在前线待了一个月,在延安访问过党政高级干部,也访问过毛主席“评论许多事情”,此文是他把关于中日战争问题的谈话用问答的方式写出来的。

