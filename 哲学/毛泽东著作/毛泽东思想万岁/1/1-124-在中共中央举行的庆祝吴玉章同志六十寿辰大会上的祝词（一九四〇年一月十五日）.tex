\section[在中共中央举行的庆祝吴玉章同志六十寿辰大会上的祝词(一九四〇年一月十五日)]{在中共中央举行的庆祝吴玉章同志六十寿辰大会上的祝词(一九四〇年一月十五日)}


今天大家欢聚一堂,为吴老祝寿,想在我在两年前为徐老祝寿时的感想,我那时就说过,我们替他祝寿,不是无原因的。记得我在小的时候,很不喜欢老人,因为他们是会欺负青年人的,青年人谁没点错误呢?但是你,错不得,他们对你是很凶的。一切事情,小孩子和青年人是没有发言权的。中国的青年人受封建家庭封建社会的苦太大了。但是现在世界改变了,青年人喜欢老年人。就象我们的吴老、林老、徐老、董老、谢老都是很受青年们欢迎的。为什么有这个转变呢?因为这些老同志不但不欺负青年人,而且非常热心地帮助青年,他们的行为是为青年模范,所以青年都十分敬爱他们。党外也有许多受青年尊敬的老人,例如马相伯就是一个,他做寿时我们共产党还打了贺电去,因为他是主张抗日与民主政治。

人总是要老的,老人为什么可贵呢?如果老就可贵,那么可贵的人就太多了,因此我们要有一个标准:就是说可贵的是他一辈子总是做好事,不做坏事,做有利于人民的事,不做害人的事,如果开头做点好事,后来又做坏事,这就叫做没有坚持性。一个人做点好事并不难,难的是一辈子做好事,不做坏事,一贯的有益于广大群众,一贯的有益于青年,一贯的有益于革命,艰苦奋斗几十年如一日,这才是最难最难的啊,我们的吴玉章同志就是这样一个几十年如一日的人,他今年六十岁了,他从同盟会到今天干了四十年革命。中间颠沛流离终于不变,这是不容易的啊!从同盟会中留下到今天的人已经不多了,而终为革命奋斗,无论如何不变其革命节操的人更没有几个人了,要这样做不但需要坚定正确的政治方向,而且需要艰苦奋斗的精神,不然就不能抵抗各种恶势力、恶风浪,例如:死的威胁,饥饿的威胁,革命失败的威胁,等等。我们的吴玉章同志就是经过这样无数的风浪而来的。因此我们要学习他的各方面的好处。但特别要学习他对于革命的坚定性。

<p align="right">(摘自1940年工月24日《新中华报》)</p>

