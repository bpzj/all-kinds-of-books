\section[论十大关系(一九五六年四月二十五日)]{论十大关系}
\datesubtitle{(一九五六年四月二十五日)}


最近两个月来,政治局分别听取了中央的经济、财政三十四个部门的工作汇报,交换了一些意见,政治局又讨论了几次,综合起来,有十个问题,十个矛盾。

提出这十个问题,都是为着一个目的,为着调动一切积极因素,动员一切可用的力量,来多、快、好、省地建设社会主义。

调动一切积极因素,动员一切可用的力量,是我们历来的方针。过去实行这个方针,是为了人民民主革命的胜利,为了结束帝国主义、封建主义和官僚资本主义的统治。现在为了新的革命,就是社会主义革命,建设社会主义的国家。不论在革命中间,或者建设中间,同样应当实行这个方针。这是大家都清楚的。但是,有一些问题还是值得谈,其中有些新东西。我们的工作也还有缺点,有不够的地方。谈一谈,考虑这些问题,正确处理这些矛盾,可以少走些弯路。我先把十个问题念一念:

第一个问题,工业和农业,重工业和轻工业的关系;

第二个问题,沿海工业和内地工业的关系;

第三个问题,经济建设和国防建设的关系;

第四个问题,国家、生产单位和生产者个人的关系;

第五个问题,中央和地方的关系;

第六个问题,汉民族和少数民族的关系;

第七个问题,党和非党的关系;

第八个问题,革命和反革命的关系;

第九个问题,是非关系;

第十个问题,中国和外国的关系;

这些关系,都是一些矛盾。世界到处都是矛盾,没有矛盾,就没有世界。

现在我来讲上面提的十个矛盾。

第一,工业和农业、重工业和轻工业的关系。

重工业是重点,要优先发展,大家没有异议。在处理重工业和轻工业、工业和农业的关系问题上,我们并没有犯过原则性的错误,我们没有犯过有些社会主义国家那种错误。他们片面地着重重工业,而忽视轻工业和农业,因而市场上的货物不够,生活品不够,货币不稳定。我们对于轻工业和农业都是比较注重的。我们市场上的货物比较充足,同有的国家在革命以后的市场情况不同。我们的生活品,说十分够也不那么够,但是我们有相当丰富的民生日用商品,并且价格是很稳定的,人民币是稳定的。这并不是说,现在没有问题了。也还是有问题的,就是对于轻工业,对于农业,比过去要更加注意,就是适当地调整一下重工业和轻工业、工业和农业的投资比例,要在工农业总投资中适当地增加轻工业和农业的投资比重。

这样,是不是重工业不是为主了呢?还是为主。是不是对于重工业不注重了呢?现在这样提,投资的重点也还是重工业。

今后需要在轻工业和农业方面多投一点资,让这方面的比重加重一些。加重一些,是不是要改重点?重点没有变,重工业还是重点,但是轻工业和农业这方面要加重一些。

加重的结果会怎么样?结果就是会更多更好地发展重工业,就是会更多更好地发展生产资料的生产。

发展重工业,需要有资金的积累。积累从那里来?重工业可以积累,轻工业和农业也可以积累。但是,轻工业和农业能积累得更多些,更快些。

这里就发生了一个问题:你究竟想发展重工业还是不想?或者想得厉害一些,还是想得差一点?你如果是不想,那就打击轻工业,打击农业;你如果想得差一点,也可以对轻工业少投一些资,对农业少投一些资;你如果想得厉害,那你就要注意发展轻工业,就要注意发展农业,使得生活品更多些,积累就会更多些,几年之后,投到重工业方面的资金也就会更多些。所以这是一个真想、假想的问题。

当然,对于发展重工业,真想假想,在我们这里来说,不适当。谁不真想?就是想得厉害不厉害。你真正想重工业想得厉害,你对轻工业就应当多投一些资,不然你想得就不十分真,只有九分真,那就不厉害,那就是你对于重工业不十分注重。你如果十分注重,你就要注意发展轻工业,因为:第一是它能满足人民的生活,第二是它能更多更快地提供积累。

在农业问题上,有的社会主义国家的经验证明,农业集体化了,搞得不好也还是不能增产,农业机械化了,搞得不好,也同样不能增产。有的国家的农业,不能增产的根本原因,是由于国家对农民的政策有问题,在税收上使农民的负担很重,在价格上农产品很便宜,工业品很贵。我们在发展工业特别是重工业的同时,得把农业摆到一定的位置上,实行正确的农业税收政策和正确的工农业品价格政策。

农业对整个国民经济的重要性,从我们的经验来看,是很清楚的。解放几年来的事实证明,那一年的农业丰收了,我们那一年的日子就好过。这是一个规律性的问题。

我们的结论是这样:用少发展一些轻工业和农业的办法,来发展重工业,这是一种办法。用多发展一些轻工业和农业的办法,来发展重工业,这又是一种办法。前一种办法,片面发展重工业,不照顾人民生活,后果是,人民不满意,重工业也不会真正搞好。从长远观点来看,这种作法,反而会使重工业发展得慢些和差些。几十年后算总账,那是划不来的。后一种办法,把重工业的发展建立在满足人民生活需要的基础上,使重工业发展的基础更加稳固,结果是,会使重工业发展得多些和好些。

第二,沿海工业和内地工业的关系。

发展内地工业是对的,是主要的,但必须注意照顾沿海。

在这个问题上,我们没有根本的、大的错误。但有一些缺点。最近几年,对于沿海工业,有一点不那么十分注重了,恐怕要改变一下。

原有工业,无论是重工业,轻工业,多少在沿海?

所谓沿海,就是辽宁、河北、北京、河南的东部、山东、安徽、江苏、上海、浙江、福建、广东、广西。我国全部工业的百分之七十在这些沿海的地方,重工业的百分之七十也在这些沿海地方,只有百分之三十在内地。

如果我们不重视这个事实,对沿海工业估计不足,如果还不充分地利用沿海工业的生产能力,那就很不对了。

我们应当尽量利用可能的时间,使沿海工业有所发展。我不是讲新的工厂都建在沿海。新的工厂百分之九十以上应当建在内地。但是沿海也可以建立一些,比如鞍钢、抚顺就在沿海;比如大连有造船工业;唐山有钢铁工业,有建筑材料工业;塘沽有化学工业;天津有钢铁工业,有机械工业;上海有机器工业,有造船工业:南京有化学工业,还有其他许多其他工业;现在我们准备在广东的茂名(那地方有油页岩)搞人造石油,那也是重工业。

今后,大部分的重工业,百分之九十或者还多一点的重工业,应当摆在内地,使全国工业布署逐步平衡起来,使全国工业有个合理的布局。这毫无疑义。但是,部分重工业还要在沿海新建和扩建。

过去工业的老底子主要在沿海,我们如果不注重沿海工业,就要吃亏。充分的利用沿海工业的设备能力和技术力量,好好地发展沿海的工业,可以使我们更有力量发展内地工业,支持内地工业。对沿海工业采取消极的态度是不对的。这种消极态度,不但妨害沿海工业的充分利用,而且也阻碍内地工业的迅速发展。

我们都想发展内地的工业,问题在于你是真想还是假想,如果是真想而不是假想,就必须多利用沿海的工业,努力搞好一些沿海的工业,特别是轻工业。

从现有材料看来,有些轻工业工厂建设很快,投入生产并全部发挥生产能力以后,一年就可以收回全部投资。这样,五年之内,除本厂以外,就可以增加三四个厂,有的五年可以增加两三个,有的可以增加一个,至少可以增加半个。这同样说明利用沿海工业的重要。

我们的远景规划缺少四十万名技术干部,可以从沿海的工人和技术人员中培养出来。技术干部不一定要科班出身。高尔基只读了两年小学,鲁迅,大学没有毕业。在旧社会,他只能当讲师,不能当教授。肖楚女同志更是没有上过学校。应当相信,技术工人,他们在实践中学习,可以成为很好的技术干部。

沿海工业,技术高,产品质量好,成本低,新产品出的多。它的发展对全国工业的技术水平和产品质量的提高,有带动作用。我们必须充分重视这个问题。

总之,不发展轻工业就不能发展重工业,不利用沿海工业就不能建设内地工业,对沿海工业不能只是维持,而是要适当的发展。

第三,经济建设和国防建设的关系。

国防不可不有。把兵统统都裁掉了好不好呢?那不好,因为还有敌人,敌人在“整”我们,我们还受敌人包围么!

我们已经有了一个相当可观的国防力量。在抗美援朝这一仗以后,我们的军队更强大了。自己的国防工业正在建立起来。自从盘古开天辟地以来,我们不晓得制造汽车,不晓得制造飞机,现在我们开始能制造汽车了,也开始能制造飞机了。我们的汽车工业,先搞卡车,不搞轿车,所以我们每次只好坐外国的车子来开会,想要爱国,爱不那么快,哪一天我们开会的时候,坐自己的汽车那就好了。

我们现在还没有原子弹,但是我们过去也没有飞机和大炮,我们是用小米加步枪打败了日本侵略者和蒋介石的。我们已经相当强,以后还要更强,可靠的办法就是把军政费用摆在一个适当的比例上。使军政费用支出的比重,分几个步骤,降到国家预算的百分之二十左右。增加经济建设费用,使经济建设有更大和更快的发展。在这个基础上国防建设也就能够得到更大的进步。这样,在一个不长的时期内,我们就不但会有很多飞机,很多大炮,而且还可能有自己的原子弹。

你真想原子弹吗?你就要降低军政费用的比重,就要多搞经济建设。你假想要原子弹吗?你就不降低军政费用的比重,就少搞经济建设。究竟怎样才好,请大家研究一下。这是战略方针问题。

在一九五○年,我们在党的七届三中全会上,就已经提出精简国家机关,减少军政费用的问题,并且认为这是争取我国财政经济情况根本好转的三个条件之一。但是,第一个五年计划期间,军政费用占国家预算全部支出的百分之三十二,即有三分之一的支出用于不生产的方面,这个比重太大了。第二个五年计划应当想办法把这个比重降低下来,以便抽出更多的资金,投入经济建设和文化建设。

第四,国家、生产单位和生产者个人的关系。

最近,我们跟各省的同志谈,他们对这个问题谈的比较多。

讲工人。工人的劳动生产率提高了,每个工作日的产值增加了,工资也需要适当调整,不注意这点是不妥的。

解放以来,工人的生活有很大改善,这是大家知道的。有些过去家里根本没有职业的,现在有人就业了。有些只有一个人就业的,现在有两个人或者三个人就业了。我就碰到过这样的家庭,过去他们没有就业的,后来夫妇两人还有一个女儿都有了职业,合起来生活当然就不错了。我们的工资,一般的说,还不算高,但是因为就业的人多,又因为物价低而稳,生活安定,工人的生活水平同解放以前是根本不可比的。工人群众的积极性一直是高的。

上面讲的,是要注意发挥工人的主动性和积极性,工厂,整个生产单位也有一个主动性和积极性的问题。

任何事物都有统一性和独立性,都有统一性和差别性。不能光有统一性,没有独立性,没有差别性。比如,现在开会是统一性,散会以后是独立性。有的人去散步,有的人去读书,有的人去吃饭,各人都有各人的独立性。如果一直把会开下去,无休止的开下去,那怎么行呢?那不是会把人开死吗!所以,每个生产单位,每个人都要有主动性,都要有一定的独立性,都要有同统一性相联系的独立性。

给生产者个人以必要的利益,给生产单位以一定的主动性,这对整个国家工业化好不好?应当是更好一些,如果更差一些,那当然不要。把什么东西统统都集中起来,把工厂的折旧费也都统统拿走,使得生产单位没有一点主动性,那是不利的。在这个问题上,我们的经验不多,在座的同志们的经验恐怕也不多,我们正在研究。那么多工厂,将来还要多,使得它们的积极性能够充分地调动起来,这对我国的工业化一定会有很大的好处。

讲到农民。我们和农民的关系历来都是好的。但是,在粮食问题上也曾经犯过一个错误。一九五四年全国因水灾减产,我们多购了七十亿斤粮食,这样一减一多,农民就有意见了。不能认为我们一点缺点也没有。没有经验,摸不清底,多购了七十亿斤,这就是缺

点。由于我们发现了这个缺点,一九五五年就少购了七十亿斤,又搞了一个“三定”,加上丰收,一增一减,使农民手里多了一百多到二百亿斤粮食,所有过去对我们有意见的农民都没有意见了,都说:“共产党真是好”。这个教训全党必须记住。

农民集体经济组织同工厂一样,也是生产单位。在集体经济组织里面,集体同个人的关系必须搞好,必须处理得恰当。搞得不好,不注意农民的福利,集体经济就不会办好。在这个问题上,有的社会主义国家可能是犯了错误。在那里,集体经济组识,有些大概办得好,有些办得并不是那么好的。办得不好的,农业生产就不那么发展。集体要积累,但必须注意,不能够向农民要得太多,不能够把农民挖得太苦,除了碰到不可抗拒的灾害以外,必须在增加农业的基础上,使农民每年的收人比前一年有所增加。

我们跟各省的同志们谈了夏收、秋收的分配问题。所谓分配问题就是:①国家拿多少;②集体拿多少;③农民得多少。以及怎样拿法的问题。国家是税,集体经济组织是积累和经营管理费,个人就是分粮食分钱。

集体经济所有的东西,都是为农民服务的,生产费不必说了。管理费也是必要的,公积金是为了再扩大生产,公益金是为了农民的福利。对生产费、管理费、公积金、公益金这几项,我们应当同农民在一起研究出一个确当的比例。

国家要有积累,集体也要有积累,但是都不能过多。国家的积累,我们主要是经过税收,而不是经过价格。工农业品的交换,在我们这里是采取缩小剪刀差,等价交换或者近乎等价交换的政策,工业品是采取薄利多销的政策和稳定物价政策。

总之,国家和工厂,国家和工人,工厂和工人,国家和集体经济组织,国家和农民,集体经济组织和农民,都必须兼顾,都不能只顾一头,这一条有一些新的东西。这是一个大问题,是关系到六亿人民的大问题,必须引起全党的重视。

第五,中央和地方的关系。

中央和地方的关系,也是一个矛盾。解决这个矛盾,目前要注意的是:应当更多的发挥地方的积极性,在中央的统一计划下,让地方办更多的事。

现在看起来,恐怕要扩大一点地方的权利,地方的权利过小,对建设社会主义是不利的。我们的宪法上规定,地方没有立法权,立法权集中在全国人民代表大会。但是,只要不违背中央的政策,在法律规定的范围内。而情况需要,工作需要,地方也可以订些章程,订些条例。在这方面,宪法并没有约束。

重工业要发展,轻工业要发展,就要有市场和原料,而要达到这个目的,就必须发挥地方的积极性,要巩固中央的领导,就要注意地方的利益。

现在几十只手插到地方,使地方的事情不好办。各部天天给省、市的厅、局下命令,这些命令虽然中央不知道,国务院不知道,但是都说是中央来的,给地方压力很大。表报之多,闹得泛滥成灾,这些都应当改变,都要商量出调整的办法。

中央的部可以分成两类。有一类,它们的领导可以一直管到企业,它们设在地方的管理机构和企业由地方进行监督;有一类,它们的任务是提出指导方针,制定工作规划,事情要靠地方办,要由地方作主。

我们要提倡同地方商量办事的作风。党中央办事,总是同地方商量,不同地方商量从来不盲目下命令。在这方面,我们希望中央各部好好注意。凡是同地方有关的事情,都要

先同地方商量,在商量好了以后再下命令。

我们要统一,也要有特殊。为了充分发挥地方的积极性,各地都必须有适合当地情况的特殊,这种特殊,不是为高岗那种闹独立王国的特殊。而是为了整体利益,为了加强全国统一所必要的特殊。

省、市对中央部门有不少意见,要提出来。地、县、区、乡对省市也会有不少意见,省市也要注意听,发挥他们的积极性。正当的积极性,正当的独立性应当有,省、市、地、县、区、乡都应当有。中央对省市,省市对地、县、区、乡都不能够也不应当框得太死。

当然,也要告诉下面的同志,不要乱来,必须谨慎。可以统一的,应当统一的,必须统一;不可以统一的,不应当统一的,不能强求统一。

有两个积极性,比只有一个积极性好得多。不是从地方主义出发,不是从本位利益出发,而是从国家整体利益出发。要为国家利益争“地”,闹其可闹者。

中央准许的独立性,是正当的独立性,不能叫作“闹独立性”。

总之,地方要有适当的权力,这对我们建设强大的社会主义国家反而有利,把地方的权力缩的很小,恐怕是不那么有利。

在解决中央和地方的关系问题上,我们的经验也还不多,还不成熟,希望大家好好研究讨论。

第六,汉民族和少数民族的关系。

这个问题,我们的政策是稳当的。得到少数民族的赞成。我们着重反对大汉族主义。地方民族主义是有的,但那不是重点。重点是要反对大汉族主义。按人口,汉人占大多数,如果汉人搞大汉族主义,排挤少数民族,那就很不好。所以,要在汉族中间广泛地进行无产阶级的民族政策教育。对汉族和少数民族的关系要来一次检查。早两年有一次检查,现在应当在来一次检查。如果有关系不正常的,应当加以调正,不要只口里讲。现在有许多人讲不要大汉族主义,口里讲得好,实际上没有做。

在少数民族地区,经济管理体制,财政体制,究竟怎么样才适合,也要好好研究一下。少数民族地区是地大物博,汉民族是人口众多。少数民族地区的地下宝藏不少,是建设社会主义所需要的。汉民族必须积极帮助少数民族进行社会主义的经济建设和文化建设,经过民族关系的改善,把一切有利于社会主义建设的因素,包括人的因素和物的因素,统统调动起来。

第七,党和非党的关系。

这是说中国共产党和民主党派、无党派民主人士的关系,这一条不是什么新的,但是因为说到这里,应当把这个关系说一说。

究竟是一个党好,有几个党好?现在看来,还是有几个党好,不但过去如此,而且将来也可以如此,一直到一切党派都自然消失了的时候为止。共产党和各民主党派,长期共存,互相监督有好处。

党派是历史上产生的东西。世界上的东西,没有什么不是历史上产生的,这是第一条。第二条,凡是历史上产生的,也要在历史上消灭。共产党是历史上产生的,因此它总有一天要消灭,民主党派也是这个命运。

无产阶级政党和无产阶级专政在将来都是要消灭的。但是,现在非有不可,否则不能

镇压反革命,不能抵抗帝国主义,不能建设社会主义,为了实现这些任务,无产阶级专政必须有很大的强制性。但是,必须反对官僚主义,不要机构庞大,我建议党政机构进行大精简,砍掉它三分之二。

话又说回来了。党政机构要精简,不是说不要民主党派。

现在,我们国内是民主党派林立,其中有些人对我们还有很多意见。对这些人,我们采取又团结又斗争的方针,要把他们调动起来为社会主义服务。

中国在形式上没有反对派,所有民主党派都接受中国共产党的领导,但是实际上,这些民主党派中的一些人就是反对派。在“把革命进行到底”,外交政策“一边倒”,抗美援朝,士地改革等等问题上,他们都是又反对又不反对。对于镇压反革命,他们也还有意见。他们说共同纲领好得不得了,不想搞宪法,但是宪法起草出来了,他们又全都举手赞成。事物常常走到自己的反面,民主党派中的一些人对许多问题的态度也是一样。他们是反对派,又不是反对派,因为他们要爱国,常常由反对走到不反对。

共产党和民主党派的关系要有所改善。我们要让民主党派人士说出自己的意见,只要说得有理,不管谁说的,我们都接受,这对党、对国家、对人民、对社会主义都有利。

因此我希望我们的同志抓一下统一战线工作,省委书记要抽出一定时间检查一下,部署一下,把这个工作推动起来。

第八,革命和反革命的关系。

反革命是什么因素呢?它是消极因素,它是破坏因素,它不是积极因素,它是积极因素的反对力量。

那么,消极因素可以不可以转变为积极因素?破坏因素可以不可以转变为有利因素?反革命分子可以不可以转变?这要看什么社会条件。死顽固,死心塌地的反革命,必然有。但是,在我们的社会条件下,就他们大多数人来说,将来有一天是会转变的,当然,有些人或许没有来得及转变,阎王就请去了,有些人谁晓得他们那一年会转变?

由于人民力量的强大,由于我们对待反革命分子采取了正确的政策,让他们在劳动中改造自己成为新人。这样,有不少反革命分子变成不反革命了,他们参加了农业的劳动,参加了工业的劳动,有一些人还很积极,做了有益的工作。

关于镇压反革命的工作,有几点是应当肯定的。比如讲,一九五一年和一九五二年那一次镇压反革命是不是应该的?似乎有这么一种意见:那一次镇压反革命也可以不要。这么看是不对的。应该承认,那一次镇压反革命是必须的。

对待反革命分子的办法是:杀、关、管、放。杀,大家都知道是什么一回事。关,就是关起来劳动改造。管,就是放在社会上让群众监督改造。放,就是可捉可不捉的就不捉,或者捉起来以后,表现好的,把他放掉。按照不同情况,给反革命分子各种不同的处理,是应当的。这些办法,都需要给老百姓讲清楚。

杀了的那些,是什么人呢?那些是老百姓非常仇恨的,血债重的分子。六亿人民的大革命,不杀掉一批“东霸天”、“西霸天”,对他们讲宽大,老百姓不赞成。肯定过去杀这批人杀的对,在目前有实际意义。不肯定这一点就不好。这是第一点。

第二点,应当肯定的,在社会上还有反革命分子,但是已大为减少。我们的社会秩序很不错,也还不能放松警惕。说一个反革命分子也没有了,高枕无忧,那就不对。有少数

反革命分子,还在进行破坏活动,例如,把牛弄死,把粮食烧掉,破坏工厂,盗窃情报,贴反动标语等等。

今后社会上的镇反,要少捉少杀,对多数反革命分子,要把他们交给农业合作社来管制生产,劳动改造;但是,我们还不能宣布一个不杀,还不能废除死刑,假定有一个反革命分子杀了人或者炸了工厂,你说杀不杀?那就一定要杀。

第三点,应当肯定的,在机关、学校、部队里面进行镇反工作,我们要坚持在延安开始的一条,就是一个不杀,大部不抓。有些人不杀,不是他没有可杀之罪,而是杀掉了没有什么好处,不杀掉却有用处。一个不杀,有什么害处呢?能劳动改造的,就让他去劳动改造,把废物变为有用之物。再说,人的脑袋不像韭菜那样,割了一次还可以长起来,如果割错了,想改正错误也没有办法。

机关镇反,实行一个不杀的方针,不妨碍我们对反革命分子采取严肃态度,但是可以保证不犯错误,可以保证犯了错误还能够改正错误,可以稳定很多人。不杀头就要给饭吃。所以对一切反革命分子都应当给以生活出路,使他们都有奔头。这样做,对人民事业,对国际影响都有好处。

镇压反革命,还要作长期的艰苦工作,大家不能松懈。

第九,是非关系。

党内党外都要分清是非。如何对待犯了错误的人,这是一个重要的问题,正确的态度,应当是允许人革命,人家犯了错误,必须采取“治病救人,惩前毖后”的方针,帮助他们改正。阿Q正传是一篇好文章,我劝看过这篇文章的同志再看一遍,没有看过的同志好好地看一看。鲁迅在这篇文章里面,主要是写一个落后而不觉悟的农民,写他最怕人家批评,一批评就和人家打架。他头皮上长了几处癞疮疤,自己不愿说,也怕人家说,愈是这样,人家说的愈厉害,结果闹得愈被动。鲁迅在这篇文章里专门写了“不准革命”一章,说假洋鬼子不准阿Q革命,其实阿Q的所谓革命,不过是想抢点东西而已,可是这样的革命也还是不准。

过去,我们党内在这个问题上犯过错误,那是以王明为首的教条主义者当权的时候,对不合他们胃口的人,他们总是随便给人先安上犯过什么错误的罪名,不许人家革命,打击了很多人,使党受了很大损失,我们要记住这个教训。

如果我们在社会上不准人家革命,那是不好的,已经进了党了,人家犯了错误,不准他改正错误,也是不好的。

我们应当容许人家革命,有人说对犯了错误的人,要看他是否改正。这样说是对的,但是只说对了一半,还有另外一半,那就是要向他们做些工作,帮助他们改正错误,给他们改正错误的机会。

对犯错误的人应当一是“看”,二是“帮”。对犯错误的人,要给工作,要给帮助,不要幸灾乐祸,不给帮助,不给工作,是宗派主义的做法。

对于革命来说,总是多一点人好,犯错误的人,其中除了极少数坚持错误,屡犯不改的以外,大多数是可以改正的。正如得过伤寒病的可以免疫一样,犯过错误的人,只要善于从错误中取得教训,谨慎了,可以少犯错误。我们希望所有犯过错误的人都有免疫力,倒是没有犯过错误的人有危险,更要警惕,因为没有这种免疫力,容易翘尾巴。

我们要注意,对犯错误的人整得过分,常常会整到自己身上,搬起石头打自己的脚,

结果跌倒了爬不起来。好意对待犯错误的人可以得人心。对待犯错误的同志,究竟是采取敌视态度还是采取帮助态度,这是区别一个人好心还是坏心的一个标准。

是非要搞清楚,分清是非关系,可以教育人,可以团结全党,党内有争论,有批评,有斗争,这是必要的。按照情况,恰如其分的合乎实际的批评,甚至采取一点斗争,这是为了帮助他改正错误,是为了帮助人家。

第十,中国和外国的关系。

我们提出向外国学习的口号。这个口号,我想是提得对的。有一种国家领导人不敢提这个口号,也不愿提。要有一点勇气,就要把戏台上的那个架子放下来。

我们愿意学习世界上一切国家的长处,一切民族的长处。每一个民族都有他的长处,不然为什么能存在,为什么能发展。承认每一个民族都有长处,不是说就没有缺点,没有短处。优点和缺点,长处和短处,这两点都会有。我们的支部书记、军队的连长、排长,他们都晓得,在小本本上写着,今天开会不为别的,总结经验有两点:一个是优点,一个是缺点,他们都晓得有两点,为什么我们只提一点,只有优点没有缺点?那有这个事?一万年都有两点,那个时候有那个时候的两点,现在有现在的两点。个人有个人的两点。总而言之,是两点,而不是一点,说只有一点,叫知其一不知其二。

我们提出学习外国的长处,当然不是学习他的短处。过去我们这里有些人闹不清楚,人家的短处也去学,当着学到以为了不起的时候,人家那里已经不要了,结果栽了一个跟斗,像孙悟空一样,翻过来了。

有些人对任何事物都不加分析,完全以“风”为准,今天刮北风,他是北风派,明天刮西风,他是西风派,后来又刮北风,他又是北风派,自己毫无主见,绝对主义,往往当这个极端走到另一个极端,我们不要这样,不可盲目地学,要有分析,要有批判地学,不可以搞成一种偏向,对外国的东西一概照抄,机械搬用。

我们这里曾经有一个时期搞过教条主义,对这种教条主义,我们进行过长期的斗争,但是现在学术界也好,经济界也好,还是有一些教条主义,应当继续做批判工作。

我们是这样提问题的,学习普遍真理和中国实际相结合。我们的理论,是马克思列宁主义的普遍真理同中国的具体实践相结合,我们要能够独立思考。

我们公开的提出向外国学习的口号,学习外国的一切先进的优良的东西,而且永远地学下去,我们公开的承认本民族的缺点,别民族的优点。

要向外国学习,就要认真地学习外国文字,有可能,最好多懂得几国文字。

我认为,我们中国有两条缺点,同时又有两条优点。

第一,我们过去是殖民地、半殖民地,受帝国主义压迫,工业不发达,科学技术水平低,除了地大物博、人口众多,历史悠久等等以外,很多地方不如人家,翘不起尾巴,骄傲不起来。但是做奴隶做久了,感觉事事不如人,有点过分,在外国人面前伸不直腰,像法门寺的贾桂一样。人家让他坐,他说站惯了,不想坐。在这方面要鼓点劲,要把我国人民的自信心提高起来。要像孟子所说的“说大人则藐之”,把抗美援朝中所提倡的“藐视美帝国主义”的精神发展起来。我们的方针是:一切外国人的长处都学,政治、经济、科学、技术、文学、文艺的一切好东西都要学。

第二,我们的革命是后进的,虽然辛亥革命打倒皇帝比俄国早,但是,那时没有无产

阶级政党,革命也失败了。人民革命的胜利是在一九四九年,比苏联的十月革命晚了三十几年,在这点上也轮不到我们来骄傲。当然我们比起其他一些殖民地国家来说,革命先胜利一步,也要防止骄傲。

前面这两点,是缺点,也是好处。我曾经说过,我们穷得很,又是知识不多。一为“穷”,一为“白”。“穷”,就是没有多少工业,农业也不算那么发达。“白”,就是一张白纸,文化水平、科学水平不高。穷则思变,才要革命,才要发奋图强。一张白纸正好写字。当然,我是就大概而言,我国的劳动人民有丰富的智慧,而且已经有一批不错的科学家,不是说都没有知识。

一穷二白,使我们的尾巴翘不起来。即使将来工农业很大发展了,科学文化水平大为提高了,我们也还是要把谦虚谨慎的态度保持下去,不要把尾巴翘起来,还是要向人家学习。一万年都学习嘛。这有什么不好呢?

一共讲了十点。总之,我们要调动一切积极的因素,直接的因素,间接的因素,直接的积极因素,间接的积极因素,为建设伟大的社会主义国家而奋斗,为进一步加强和巩固社会主义阵营,为争取国际共产主义运动的胜利而奋斗!

附:

各中央局、各省、市、自治区党委,中央各部委,国家机关和人民团体各党组、党委,总政治部:

毛泽东同志在一九五六年四月作的《论十大关系》,是一篇极为重要的文件,对社会主义革命和建设的基本问题作了很好的论述,对现在和今后的工作具有很重要的指导作用。为此,特印发县、团级以上党委学习。

这个文件是当时讲话的一篇纪录稿,毛泽东同志最近看了后,觉得还不大满意,同意下发征求意见。请各级党委对文件的内容提出意见,汇总报中央,以为将来修改时参考。

<p align="right">中央

一九五六年十二月二十七日</p>


