\section[关于大批制问题的指示(一九六七年五月二日)]{关于大批制问题的指示}
\datesubtitle{(一九六七年五月二日)}


批判文章不要像九评那样长,每篇一、二千字,不要超过三千字。一篇一个中心,一个概念,明明白白。长了就没有人看,记不住。\marginpar{\footnotesize 306}

同意。团结和服从都是有条件的,不是无条件的。

年纪大的、年纪中的、年纪小的也要“三结合”,不要看不起年轻人。二十几岁、三十几岁的人都能做工作。搞“三结合”,要老中小,老的都会去见上帝的。

今年形势好,布粮还是要抓。社论要搞快一些。批判文章文字要短。彭、罗、陆杨可以提为反革命修正主义,其他人都称修正主义分子。中国的赫鲁晓夫在文章中提,在标题上要提。


大批判要慎重,要确实,要调查清楚。调查清楚,批判才有力量,否则就会一风吹。引他(指刘少奇)的话不能只顾头不顾尾。批判要站得住。“托拉斯”这个名词,不能一概驳,主要驳他走资本主义道路。有些旧名词要赋予新的意义。

“驯服工具论”要批判,但也要有无产阶级纪律。服从、团结,那是有条件的。

今年形势好,布、粮还是要抓。社论要搞快一些。

彭、罗、陆、杨可以称为“反革命修正主义分子”,其他人称“修正主义分子”。“中国的赫鲁晓夫,”文章中提,在标题上不要提。


写文章要琢磨,要准确,要尖锐明确。

\kaoyouerziju{(1967年9月)}

