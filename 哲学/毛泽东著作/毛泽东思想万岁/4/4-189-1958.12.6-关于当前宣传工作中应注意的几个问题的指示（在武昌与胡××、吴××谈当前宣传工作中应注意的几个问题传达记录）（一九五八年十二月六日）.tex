\section[关于当前宣传工作中应注意的几个问题的指示(在武昌与胡××、吴××谈当前宣传工作中应注意的几个问题传达记录)(一九五八年十二月六日)]{关于当前宣传工作中应注意的几个问题的指示(在武昌与胡××、吴××谈当前宣传工作中应注意的几个问题传达记录)}
\datesubtitle{(一九五八年十二月六日)}


一、反对浮夸作风。不是追求责任,而是接受教训。自大跃进以来.在通讯社、报纸,广播的宣传工作是取得一定成绩的,对推动:欠跃进起了很大作用.但是宣传也有些问题。使思想上发生了一些混乱现象.这是由于大跃进形势来得很猛.我们又缺乏经验。在这种情况下也是很难免的。但新华社和人民日报要始终保持冷静的头脑,要实事求是,反对浮夸作风。不追究责任.而是要取得教训。一定要作些检查,要实事求是地宣传我们的大跃进。过去强调鼓足干劲.力争上游,多快好省,苦战三年。放“卫星”,“三化“,起了很大作用。总的方面是对头的,就是在宣传的分寸上有些问题,界限掌握得不很准。

二、放“卫星’对不对?在开始是好的,但老是放下去就值得研究了,长期下去会造成人为的紧张,不见得好。特别是有些单位放的钢铁卫星。人民日报登。“这个省、县能做得到,那个省、县为什么做不到”。对各地压力很大.运动搞起来了,不要这样干,人民日报在宣传广西鹿寨县炼钢时,连续写了两篇社论,大家都去参观,但事实上数量没有那么多,质量也很差,很多是烧结铁,影响不好。宣传工作中光讲成绩,不讲缺点。如河南今年成绩很大。缺点也不少,只讲成绩,会助长下边的浮夸作风,急躁情绪,这是不对的。文艺界放‘卫星’时,不要这样宣传,因为放出了“卫星”后,还要群众承认,还是先搞出来以后。群众说“卫星”才算是“卫星”。

三,宁可少说,不可多说。对于生产成绩的数字。各省都公布出来了,但很难对起总数.今后发表统一的数目就行了,今年粮食产量翻了一番多。但究竟多少,说法不一,有说一万亿斤,有说九千亿斤,也有说七千五百亿斤的,我说就七千五百亿斤的好。报道中究竟说多好呢?还是说少好呢?说少好,说少了,顶多落个干劲不足,说少了结果多,粮食还在。说多了,事情就不好办.粮食翻一番,在全世界是少有的,现在有些人就不相信,如果说多了,拿不出东西来,不单对你那个数字不相信。而且会把我们整个成绩都推翻了,对党的威望有损失。

四、郑州会议后,宣传工作中要特别注意三个问题。

(1)防止夸大消极面,不要认为过去我们都是搞左了,现在又要纠偏,不要只讲坏的不讲好的。不要说紧张得不得了,不要说过去糟得很,不要造成一种纠偏空气,要实事求是。实际上成绩是主要的。

(2)要保护群众的积极性.在宣传中不是要泼冷水。而是在水缸里放些明矾,使水澄清一些。

(3)要考虑国际影响.现在国外许多人怀疑我们大跃进的成绩。我们不能说这是浮夸,那也是浮夸,你说了。人家就更不相信了,为了防止副作用.要从正面宣传。如说关心群众生活,不要说过去生活搞得糟,要说一面抓生产,一面抓生活,民主与集中要结合讲。不要说过去不民主.要说管理民主化。

人民日报在讨论张春桥同志的文章时,他的文章基本上是正确的,但不全面,本应将片面的补充补充,但越论越片面了,把按劳分配也否定了。本来准备结束这一讨论,现在要继续讨论三个月,有教育意义。郑州会议有些东西不宜宣传,有些内容在党内宣传。有些内容在党外宣传.搞不清楚的不要急于宣传,应请示。

五、破除迷信。不是要否定科学,现实主义与浪漫主义要很好结合。现在宣传工作中提倡敢想、敢说,敢干,起了很大的作用.现在报纸的标题比较生动.能吸引人,这是浪漫主义与现实主义的结合。但有些不很正确,不管什么都加以诗化。经济工作和写诗不一样,要切实。浪漫主义有一定限度。反对浮夸。运用群众语言有进步。但有些滥用,群众语言变成文字时,一定要求正确性,确切性,科学性。

六,对中央各领导同志到各地视察时的报导的问题。中央报纸很注意这些报道,对各地工作起了很大作用,密切了党和群众的联系,把领导同志深入实际,深人群众的作风传播出去了,但也有些缺点,把负责同志的每句话,衣服的颜色,笑了一笑,什么时候点了一下头等小动作,写了又写,不必要,有的写群众等领导同志,等了很长时间.给人以官僚主义的印象。有些将领袖写神化了,这样报道不好。有些领导同志在视察中对一些问题,随便点点头,并没有肯定或者考虑要推广的,但报纸上就加以肯定了。有些事就造成被动。报道××同志在河南视察,周总理在清华大学视察时,都有类似现象,今后对中央负责同志视察的报道,中央规定。

(1)只限于政治局委员以上的同志;

(2)写成后必须交本人审阅,不能随便发表;

(3)中央报纸不得随便转载地方报纸关于这方面的报道。

七、教育与生产劳动相结合的宣传问题.好的方面是主要的,但在宣传中也有片面性,违反中央精神的也有,有些问题只讲搞劳动,不讲搞学习。现在有的中学与大学把许多基础课程都放松了,这是一种危险。学校在基础课上打不好基础,要赶上世界先进水平是不可能的。另外,我们现在的学校基本上有两种。一种是一般的大中学校,以学习为主,另一种是红专学校,以劳动为主。但现在人们中有一种趋势,想将两种学校搞成一种红专学校,要纠正.两种学校都要,否则要赶上世界先进水平是不行的。在学校中过分强调社会化,强调学生住学,这作为方向是可以的,有的作的过急,会造成不好的结果。

八、对家庭问题的宣传有很多缺点,强调了斩头去尾,男女分居,各地对此反映非常反感,有些人认为共产主义社会,对“家庭”两个字都不可以提,可成立父母兄弟姐妹“小组”。这些东西现在不要宣传,将来的家庭,到底是什么样式,还不清楚,不宣传它,对当前社会主义建设也没有什么妨碍,宣传了实际意义不大,反而造成混乱。

九、对城市公社化的宣传问题。城市公社是个很复杂的问题,因农村中地主富农都被消灭了,但城市中还有资产阶级,要集中起来搞生产还没有把握,究竟怎么搞还没有经验,现在可积极试办,但不宜在人民日报宣传.阳泉市大量调整房子的事,这在小城市当然可以,因为房子质量差不多,但在上海、北京就不行,在城市中还有一部分人生活悬殊很大,住宅不一样,还有一部分人还不从事生产,要调整房子会天下大乱,所以暂不要宣传。

十、对农业问题的宣传.农村大跃进以来,农业跑到工业前面,这是一种实际情况,农业过了关,这是一种实际情况,但也有一些人造成错觉,认为农村先进,城市落后,农民老大,工人老二。究竟农民先进还是工人先进呢?这要从三方面看:(一)从所有制看,工人是全民所有制,农民是集体所有制。(二)对国家的贡献来说。鞍钢一个工人一年生产一万八千元。所得工资八百元,贡献很大,而农民是无法和工人比较的,农民有三种情况.一种是丰收地区,对国家可拿出贡献来,一种是刚刚可以供给自己生活的;还有一种是国家要补贴的。如灾区。(三)农民组织纪律性不如工人阶级.所以今天农民还是在工人阶级的领导下,通过共产党来领导。

十一、对《苏联社会主义经济问题》一书,不要公开对外宣传。该书基本上是正确的,但有些缺点,苏联把这本书批判得很差,如我们公开讨论,那样就不好。

由社会主义全民所有制过渡到共产主义所有制的标准。不要宣传,最好要慎重.我们和苏联的标准不一样.苏联是一九三六年宣布建成社会主义,它有两个条件。一条是消灭阶级,一条是工业占国民经济的百分之七十,如果宣传了我们的标准,那么等于说苏联现在还未建成社会主义。


