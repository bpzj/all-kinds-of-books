\section[在上海会议上的讲话和插话(一九五九年三,四月)]{在上海会议上的讲话和插话(一九五九年三,四月)}
\datesubtitle{(一九五九年三)}


人民公社有几个问题.

组织章程问题。社,队,组都应有章程。来一次全民讨论。

小队部分所有制。

吃饭不要钱如何?要研究。供给制要和按劳分配相结合.采取按劳分配的原则。来处理吃饭不要钱的问题。

算账问题。过去说一般不算账,现在要改过来,一般要算账,我是站在算账派这一边的.

公社委员的选举问题。委任制是危险的,应该由有生产经验的人当委员。要规定那些人不到场不能作决定。

征收要搞绝对数。多少议一下。多了不利。

县要不要积累?我看不要好。有些已经抓了一些。手很长。办工业赔了钱.两头抓.抓了财政部和人民,后者不要抓好不好?

银行贷款.贷款一律收回,其目的就是破坏公社,违反专款专用的方针。

投资,×亿元投资如何分配?投在谁手里,如何用?有些公社干部,以共产主义之名。行本位主义之实。要有利于人民。有利于人民就有利于国家。

农业指标,要实事求是,不能把指标定的太高,要留有余地,使下边能够超过,潜力让群众去发挥。

听取汇报时的插话

计划是不是也有主观呢?

搞了十年的经验,才知道准备要一套一套地抓!

据说你们很忙,睡不成觉,忙什么呢?越搞越不落实.

老干部办工业。我看还如×××,官这样作下去,怎么得了!去年××搞××××万千瓦的发电设备,还不知道要合金钢管!

群众路线讲一万年.为什么不交给群众讨论?你们开会就找厂长,工业书记。不找车间主任。没有对立面。

我们党有一条群众路线。现在看来工业还没有,老是说不够。什么叫不够?秀才打官腔。

对事务的观察又深入了,重中有重,急中有急,这个提法很好。

明年马鞍形,积蓄力量,后年来个大跃进!“天增岁月人增寿”,岁月过去了。经验还没有总结起来,余致力于工业凡九年,才知发电设备要无缝钢管。

搞工业,××××项。搞不成何必多搞!分散力量是破坏大跃进的办法。

运输力量不足。什么叫短途运输?短多少?

以物易物有两种,一种是正常的互通有无,一种是不好的。

来个狠,实事求是.你这个人不狠,听以不实。

一面搞钢铁。一面搞粮食,双管齐下,抓得狠,抓得紧,抓得实。

世界上的东西没有废品,像打麻将一样,就是靠、“吃’和‘碰’。上家的废品。是下家的粮食。

什么“可否”!犹豫不决。这两个字改为“一定”。

国外市场,极为重要。不可轻视。节衣缩食,不可出口。自己少吃一点。

要搞增产节约。

全民办工业如何解释?全民办工业讲得太多了,全民办农业差不多。

工人阶级老底子是四百万人,大革命时(一九二六年)两百万人,经过二十三年增加两百万人。到一九四九年四百万人,解放八年,一九四九年到一九五七年.每年平均增加一百万人。增加了八百万人。鸦片战争到一九四九年,一百零九年只有四百万人,一九四九年到一九五七年。八年中间就增加了八百万人,去年一年增加××××万人,这是一种特殊现象。照这样搞下去,一年××××万。十年×万万,有人没有机器,一个人的事可能七、八人做。美国一亿七千万人,就业六百万人,包括“美国之音”等,只有六千万人。我们这样搞下去,十年×万万人,怎么得了!学徒一年二百一十六元,壮丁在城市赚钱,老弱在农村吃饭,壮者散在四方,流在城市,结果劳动生产率下降了,这同马克思主义不合,同社会主义不合。

接见十一个兄弟国家代表团和驻华使者谈话纪要(节录){一九五九年五月九日)

世界上有人怕鬼,也有人不怕鬼。鬼是怕好呢?还是不怕好呢?中国的小说里有一些不怕鬼的故事。我想你们的小说里也会有的。我想把不怕鬼的故事、小说编成一本册子。经验证明鬼是伯不得的,越怕鬼就越有鬼,不怕鬼,就没有鬼了。有一个狂生夜坐的故事。有一天晚上,狂生坐在屋子里,有一个鬼站在窗外,把头伸进窗内来,很难看。把舌头伸出来。这么大,头伸得这么长。狂生怎么办呢?他把墨涂在睑上。涂得像鬼一样,也伸出舌头。面向鬼望,一小时?二小吋,三小时望着鬼。后来鬼就跑了。

今天世界上鬼不少。西方世界有一大群鬼。就是帝国主义.在亚洲、非洲、拉丁美洲也有一大群鬼,就是帝国主义的走狗。反动派。

尼赫鲁是个什么呢?他是半个鬼,半个人,不完全是鬼,我们要把他的脸洗一洗,尼赫鲁是半个绅士,半个流氓,他是印度资产阶级的中间派,同右派有区别。整个印度的局势,我估计是好的。

那里有四亿人民,尼赫鲁不能不反映四亿人民的意志。西藏问题成为世界问题,这是很大的事,要大闹一场,要闹久些,闹半年也好,闹一年也好,可惜印度不敢干了,我们的策略是使亚洲,非洲,拉丁美洲的劳动人民得到一次教育,使这些国家的共产党也学会不怕鬼,每逢大闹一次,都要引起反苏、反共的风暴,对我们有没有好处,是帝国主义巩固了,还是社会主义巩固了?匈牙利的同志们也在座,整个社会主义阵营比一九五六年十月以前更巩固了。那时骂苏联,苏联现在怎么样?骂倒了没有?究竟是……整个社会主义阵营巩固而生气勃勃呢?还是帝国主义阵营有朝气?事件以前的匈牙利有朝气呢?还是事件以后的匈牙利有朝气?那就是因为不怕鬼,把鬼打下去了。现在西藏问题闹出许多鬼。这是好事.让鬼出来,我是十分欢迎的。


