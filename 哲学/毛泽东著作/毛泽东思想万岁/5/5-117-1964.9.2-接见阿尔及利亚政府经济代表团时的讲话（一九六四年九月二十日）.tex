\section[接见阿尔及利亚政府经济代表团时的讲话(一九六四年九月二十日)]{接见阿尔及利亚政府经济代表团时的讲话}
\datesubtitle{(一九六四年九月二十日)}

\begin{list}{}{
    \setlength{\topsep}{0pt}        % 列表与正文的垂直距离
    \setlength{\partopsep}{0pt}     % 
    \setlength{\parsep}{\parskip}   % 一个 item 内有多段,段落间距
    \setlength{\itemsep}{\lineskip}       % 两个 item 之间,减去 \parsep 的距离
    \setlength{\labelsep}{0pt}%
    \setlength{\labelwidth}{3em}%
    \setlength{\itemindent}{0pt}%
    \setlength\listparindent{\parindent}
    \setlength{\leftmargin}{3em}
    \setlength{\rightmargin}{0pt}
    }

\item[\textbf{主席:}] 欢迎你们。

\item[\textbf{布马扎(团长):}] 主席同志,我代表全体团员表示,今天是伟大的日子。因为我们有机会见到主席。\marginpar{\footnotesize 181}我们感谢您抽出宝贵时间和我们见面。我在这里向您转达民族解放阵线总书记本·贝拉和阿尔及利亚政府和阿尔及利亚人民对中国领导人,特别是毛主席和中国人民的兄弟敬意和友情,特别感谢你们在我国解放战争时期给予我们的同情和支持,在建设时期给予我们的帮助。我临来之前,本·贝拉总统委托我转达他给你一封信件(布马扎把信交给毛主席),并口头转达:阿尔及利亚人民钦佩中国人民、中国领导人,特别是毛主席在建设自己国家中所表现的胆量和智慧,你们的方针对全世界人民,对所有革命的国家都有指导意义,我们从中国人民那里获得鼓舞。本·贝拉总统还要我口头告诉毛主席,我们能有毛主席和伟大的中国人民这样的朋友,感到骄傲,我们感谢你们在解放斗争中和独立后给予的同情和帮助。同时我代表全体团员感谢中国领导人对我们的接待,我们经济代表团的任务是要加强中阿两国之间的友好合作关系,你们帮助我们完成了这一使命,让我们看到许多东西,使我们了解了你们的国家,我们感到非常满意,这次访问对我们有很大教育。法国有句成语:“凡是结束得好的事情,本身就是好事情”。在结束我们对中国的访问时,我们会见了主席,就表明了这一点。

\item[\textbf{主席:}] 感谢你们。感谢整个代表团的朋友们。我们支持你们,你们给我们的帮助更大。整个非洲人民支持你们,全世界反对帝国主义的人民(包括中国人民在内)支持你们。你们打了八年仗,牺牲很大,值得各国人民支持你们,你们在整个非洲、亚洲、拉丁美洲做出了一个榜样。这个榜样对各国有很大影响。而且你们正在支持尚未独立的国家。自己国家刚刚解放就支持别国。例如支持刚果,这是一种国际主义精神。所以我们和你们有共同的语言。我们经常注意你们的活动,注意你们国家在本·贝拉总统领导下的活动,例如镇压反革命。反革命不镇压不得了。不镇压反革命,政权就不能巩固。以后还会有反革命的,他们从国内颠复你们的政府,破坏你们的经济,甚至于暗杀你们的领导人。这些人代表外国帝国主义的利益。在国内代表封建地主阶级的利益,代表买办资产阶级的利益。他们不代表广大工人、农民、革命知识分子的利益,你们就非同那些反革命分子作斗争不可。说现在是搞建设,还不如说是搞革命,搞社会革命,我们的国家也一样,一方面在建设,一方面在进行社会革命。同你们国家一样,我们国家也有地主分子、资产阶级分子、资产阶级右派及知识分子中的右翼分子,他们同我们捣乱。你们以为我们把蒋介石赶走,我们的国家就太平了吗?并不是这样。你们看到的是表面,时间很短,没有深入到社会去调查研究。这是大使的任务,我曾向大使说过,要研究中国,研究中国社会,要做社会调查。不要只看表面。例如我国工厂增加,工人也增加,就混进了国民党的中将、少将、上校、地主、富农、警察、宪兵等。

在农村也有他们的人,他们从这个省跑到那个省的农村,说是难民、是贫雇农,其实是逃亡地主。我们估计有百分之五左右的人反对社会主义,反对共产党。百分之五就是说有三千多万人,比你们全国人口还多。他们为什么不造反呢?因为他们分散在各地方。如果都集中起来,三千万人就不得了。你们的那些反革命分子,反对你们的人,也是分散在各地,所以你们可以消灭他们,前途是光明的,只要注意这件事情,警惕他们就好了,要记住这点。

我们支持你们建立代表大多数人的党。听说你们正在开始建党。我和大使讲过,建党的原则是建立在百分之几的少数剥削者,还是建立在百分之九十的被剥削者,还是建立在不被剥削也不剥削别人的基础上。听说你们把阿巴斯软禁起来了。\marginpar{\footnotesize 182}是建设阿巴斯的党,还是建设本·贝拉总统所主张的那样的党?如果大批党员都是阿巴斯那样的人,他们不代表工人、农民,那么将来还得革命,总会有人起来革阿巴斯那样人的命,阿巴斯还算文明一点,他没有拿起枪打,如果拿起枪来打,那就更厉害一些。

关于我国经济建设经验,我们是吃过亏的,如尽搬外国经验,工厂要越大越好。其实不然,现在我们正在把工厂缩小,化一个工厂为几个小工厂。你们到过沈阳看过什么工厂?布马扎:我自己去朝鲜了,我在沈阳看了展览会,看了模型,代表团部分人参观了钢铁厂,机床厂和自行车厂。主席:鞍山这个工厂太大了,十几万人,什么东西都要自己来搞,从采矿、炼焦、炼铁、轧钢都是它。我们这条道路是错误的。我劝你们不要抄这条经验。过去有一个时期我们不重视农业和轻工业,就发生许多问题,粮食不够,轻工业品不够,吃、穿、用都不多,恰好这些是广大人民生活的必需。同时,这两个方面是积累资金的主要来源。要自力更生,非积累资金不可,靠帝国主义是不行的。有些兄弟国家把专家撤走。这一来我们不得不自己干。技术、资金从哪里来呢?任何外国不贷款给我们。我们也不愿向外借。技术从哪里来?都得靠自己,当然从外国购买一些技术如我们从英国、法国、西德、意大利、日本进口一些技术,但是要付款,不然他们不给。只好把建设拖长一些。要很快建设先进的工业、农业、国防、研究技术是困难的,把时间放长一些就可能了。过去。我们在国内发公债,现在不发了,正在还。我们欠苏联的外债,明年就可还清,明年再还一千七百万外债就没有了,至于内债,国内公债到一九六八年就还完。现余很少了,欠了一身债是不好过。这个你们考虑过没有?

\item[\textbf{布马扎:}] 考虑过。我们也这样想过,欠债就往往不那么自由。

\item[\textbf{主席:}] 说话就不那么响亮。还是请工人、农民帮忙,工人和农民会帮助你们的。工人、农民帮助你们打败了帝国主义。过去八年打胜仗,并非外国帮助你们打的胜仗,外国帮助顶多是十个指头中的一个指头。工人、农民力量很大,他们能帮助你们战胜那么强大的帝国主义,他们会帮你们镇压反革命,巩固政权。请问如果没有广大工人、农民帮助你们能不能镇压反革命?你们有一支很好的军队,这些军队主要是工人和农民,我也有一支军队,我们的军官百分之九十九是没有文化的,他们不识字,或者识字很少,而国民党的军官都是知识分子,都是在军官学校学过的,可是哪一个打胜了,哪一个打败了呢?我们同三个敌人打过仗,第一个是蒋介石,第二个是日本,和这两个敌人一共打了二十二年。第三个是美帝,在朝鲜打了三年。我们那时三个军的炮还抵不过他们一个师的炮。他们的空军白天黑夜轰炸,使我们运输很困难,可是最后还是我们和朝鲜人民军把美国军队打败了,所以美国不喜欢我们,美国说我们坏。从前法国也说我们很坏,就是因为我们支持胡志明打了奠边府。我们也支持你们,你们胜利了,法国人就和我们建立外交关系,你们不胜利,法国人是不会同我们建立外交关系的。第一个支持你们的不是我们,而是阿联,第一个承认你们临时政府的不是我们,也是阿联。谁劝说我们尽快承认你们临时政府呢?我同你们说,是你们的阿拉伯兄弟国家,那时叫埃及。没有它,有武器也运不去。现在阿联也支持刚果武装斗争。阿联同你们一道反对以色列的侵略。以色列人才有多少?以色列有一百多万人,不是以色列的问题,而是它背后的帝国主义的问题。是工人和农民帮助了古巴领导人在没有外援的情况下\marginpar{\footnotesize 183},夺取了政权。你们依靠工人、农民不仅在战争中获得了胜利,而且可以,也一定可以把你们的国家建成一个强盛的国家。这是你们的内政,外国人的话只是一种参考性质的资料,究竟怎么搞好,由你们自己决定,谁也不能干涉。请问一九五四年的革命是外国人帮助你们决定的,还是你们自己决定的呢?那时摆在你们面前的问题是革命还是不革命的问题。我听说你们看到胡志明打败了法国,就考虑到为什么你们不可以打败法国呢?你们采取了胡志明的经验。大使不是要去越南吗?阿大使:我们要先回国,十月一日以后准备去。主席:总之,只有依靠群众才有可为,脱离群众就很危险。最可靠的是本国的群众。是你们怕法国人还是法国人怕你们?事实已经证明,是法国人怕你们。是你们强大还是法国强大?看起来法国是强大的,它有四千多万人口,当时法国人在阿尔及利亚的有一百万之多,一切重要工业、农业都控制在法国人手里。法国人不但有陆军、还有海军和空军。你们没有海军,一架飞机也没有,同我们一样,陆军也不及法国的十分之一。

但是你们打胜了,这个道理很可以想一想。我也经常同外国人举你们的例子,说明依靠群众实际上比强大的帝国主义更强大。你们知道我们亚洲有战争,就是在南越,那里遇到的敌人是一个比法国更强大的帝国主义。现在美国怕南越,他们不知道如何是好。

打下去还是退出来?谈判还是继续打?是扩大战争还是缩小战争?南越人也只有一千万多一点,可是美国和他所支持的傀儡是不得人心的。不管怎么讲,我看美国总是要失败的。

我感谢本·贝拉总统给我的第二封信。第一封信我还没有回答,请你回去代我问候他,我收到第二封信很高兴,我还要回他的信。

\item[\textbf{布马扎:}] 非常感谢你的讲话,我们非常注意听你刚才的话,我要再次讲,刚才你提到的问题。不论是国内或国际问题,中国人民和阿尔及利亚人民的看法是一致的。关于你提到的许多点,我很满意的说,由于一些提法是你这样有经验的领导人讲的,就更进一步证明了我们的立场是正确的。我们的国家是一个小国,从革命角度来说是年青的,从经验上来讲,与中国革命的经验比较,我们的革命是处于初级阶段,刚才你提到,胡主席的胜利在多大程度上影响了阿尔及利亚的革命,应该说他的胜利给我们很大的鼓舞,并证明了农民和劳动者所说的一句话:不一定需要像殖民主义者那么多的武器也可以打败帝国主义。那时在阿有两派,一派是改良主义,主张用和平主义的方法,另一派是革命派,主张用暴力的方法。北越的胜利给阿尔及利亚开展革命提供了有利条件。

那一派中有阿巴斯和其他反革命分子。我们在这问题上打了赌,我们赢了。我们一开始就支持了越南的斗争。阿尔及利亚、奥兰等港口的工人拒绝装卸武器,为此,同法国的左派组织搞得很僵,因为他们对越南的支持是犹豫的。

\item[\textbf{主席:}] 那时奥兰和阿尔及利亚等地都有抵制装卸的活动吗?

\item[\textbf{布马扎:}] 每次装卸时都有这样的活动,我和代表团的同志们如果去越南时,一定会遇到不少我们认识的越南人,他们过去在法国和我们一道共同斗争。

\item[\textbf{主席:}] 他们对你们一定会很友好的。

\item[\textbf{布马扎:}] 越南战争有好处。他们的斗争帮助我们加强了我们所维护的观点。我们也利用一些别国的经验,如一九一七年的革命和中国长征的革命。我想你们一定从法国资产阶级报纸上看到,他们埋怨说阿尔及利亚在游击战这一方面运用毛泽东的著作来进行斗争。

关于建立革命的党的问题,不久以前,我们开了代表大会,通过纲领,使我们能澄清一些问题,继续前进。纲领中写着阿尔及利亚革命主要是依靠农民劳动者,革命知识分子。另外,在党员成分方面,主要从农民劳动者,革命知识分子中吸收党员,我们选择社会主义,但不是说现在就是社会主义了。我们在这个问题上没有幻想,我们知道要经过一个阶段。

\item[\textbf{主席:}] 要经过一个相当长的阶段。

\item[\textbf{布马扎:}] 现在的问题是在经济上获得解放。在国内仍然存在着外国特权分子,特别是需要高超技术的工业部门和石油部门。外国掌握这些部门的事实对我们是个威胁。他们同从内部进行颠复的活动是同样危险的。

\item[\textbf{主席:}] 在工业部门没有自己的技术人员来代替他们,这个问题不可能得到顺利的处理。你们的经济同法国的关系很大,这是应该注意到的。听说百分之七十五的对外贸易是同法国进行的,没有一段相当长的过程,要脱离法国是不可能的,那么法国就要利用这个关系来压价,暂时你们也得容纳他们。同时开辟另外的道路,使他们将来同你们平等贸易。例如石油你们还不能自己开采,那么只好让他们来开采。是不是能顺利的培养自己的石油技术人员和知识分子?

\item[\textbf{布马扎:}] 我们和苏联合办了一个石油学院,准备培养二百个工程师。我回国后还准备和罗马尼亚石油部负责人共同考虑培养石油技术人员问题。所以法国也提出要帮助我们培养干部。

\item[\textbf{主席:}] 法国过去是不培养的,如在越南过去没有多少技术人员,同日本在我们东北一样,根本不培养技术人员。帝国主义可挖苦得很。你们有苏联、罗马尼亚培养技术人员的条件下,法国才可能帮助培养技术人员。

\item[\textbf{布马扎:}] 现在阿尔及利亚只有三个石油工程师。

\item[\textbf{主席:}] 有三个人就好!

\item[\textbf{布马扎:}] 这三个人是在战争中培养出来的。

\item[\textbf{主席:}] 有三个就了不起,可以培养出三十个、三百个。听说你们有五十万人在法国,这是个有利的条件,这五十万人中可能有一部分技工或工程师。

\item[\textbf{布马扎:}] 他们都是有技术的工人,特别是在纺织方面,我们建设工厂快开工时,可从法国搞一些阿尔及利亚人回来。

主席;所以有希望。困难是有的,困难是可以克服的。不管他多么大的困难,总是可以克服的。帝国主义欺侮你们非洲人,说你们不行,帝国主义也瞧不起我们亚洲人,说我们不行。我不相信,我和你们一样不相信。你们北非人,东非人和西非的人都是很聪明、很勤劳、很勇敢的人。帝国主义也看不起我们亚洲人,看得起日本人,就是看不起中国人。中国人到外国去,人家就问是中国人还是日本人。现在还有这样情况。

\item[\textbf{布马扎:}] 但现在必须承认中国人所做的都是伟大的事业。

\item[\textbf{主席:}] 现在还不行,中国经济上还没有站起来。大概再有一百年,那个时候他们就可以区别中国人和日本人是不同的。看起来中国人和日本人都是黄脸皮但是两个民族。你们今天怎么安排?

\item[\textbf{×××:}] 下午游览参观,晚上有晚会。明天到杭州,再到广州,然后回国。

\item[\textbf{主席:}] 还有什么意见?

\item[\textbf{布马扎:}] 非常高兴有这个机会同您讨论问题。这是我们期待很久的日子。

\item[\textbf{主席:}] 很希望看到你们,我感到你们有个特点,你们比我们年青。你们是刚升起来的太阳。

\item[\textbf{布马扎:}] 但我们觉得毛主席永远年青,思想总是明智开朗,看问题很清楚。你的话我们都详细记录了下来,回去要汇报,要向党的干部传达。这次谈话很有教育意义。谁知道可能有一天在阿尔及利亚或非洲看到您。

\item[\textbf{主席:}] 我也希望。

\item[\textbf{布马扎:}] 二十年前亚洲人、非洲人都受人欺侮,过去亚洲人、非洲人都被歧视。如对中国和日本不同对待,非洲人也如此。他们也想分化我们,说有一部分是好的,另一部分是坏的。但现在我们自由了。

\item[\textbf{主席:}] 你们解放了。亚洲问题还大,美军基地在日本、南朝鲜、菲律宾、南越、泰国,在太平洋西岸包围我们,威胁着我们,也威胁着当地人民,也威胁着印度尼西亚和锡兰。帝国主义受伤了,但是还没有死,要多少年,恐怕要一百年吧!那是下一辈子的事,但是我相信帝国主义总有一天要死亡的。
\end{list}
