\section[在中共中央招待合作社代表会议上的讲话(摘录)(一九四四年)]{在中共中央招待合作社代表会议上的讲话(摘录)}
\datesubtitle{(一九四四年)}


……自四二年边区高干会后,一年半中间合作社事业有了很大的发展,整个工作走上了轨道,出现了大批模范的合作社,都是一本活的教科书。

……第一,合作社是为什么人办的?是为广大群众办的,为边区一百四十万老百姓和十万部队机关学校人员。这个方向都为前年冬天高干会上就已经确定,这是刘建章的方针。一年多以来,很多合作社都朝着这个方面走,都有很大成绩。但是还有些合作社没有解决这个问题。因此这次合作社会议要重新宣布这一条方针。第二,合作社办些什么事?合作社的业务主要有十项:工业、农业、运输、畜牧、供销、卫生、信用、教育、植树、公益,通过合作社把全边区的人民组织起来。……第三、合作社统一战线的性质,所有农民、工人、地主、资本家都可以参加合作社,它是政府领导,各阶层人民联合经营的经济文化及社会公益事业的组织。第四、所有做政府工作的同志们,都应该认清合作社事业的重要性,不应该有丝毫轻视的心理,边区的任何经济、文化事业都必须通过合作社才能完成。因此,要鼓励更多的同志们去参加合作社事业。

