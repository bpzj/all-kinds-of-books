\section[关于陕西驻军虚心听取群众意见改进工作报告的批示和批注(一九六七年四月二十三日)]{关于陕西驻军虚心听取群众意见改进工作报告的批示和批注}
\datesubtitle{(一九六七年四月二十三日)}


林彪、恩来同志:

将此件印发军委扩大会议各同志,军队这样做是对的,希望全军都采取此种做法。
<p align="right">毛泽东
四月二十三日</p>

附:陕西驻军负责同志虚心听取群众意见、改进工作

陕西省军区司令黄经耀、政委袁克服、驻陕部队××××军军长胡伟等负责同志四月中旬以来,连继召集西工大、西军电造反派、交大文革代表座谈,听取他们对“支左”问题的意见和批评。

座谈中,同学们批评了部队在前段工作中旗帜不鲜明,调查研究不够,没有支持真正的革命造反派,有时还支持保守组织,压制革命派。批评部队没有把训练内容和西安地区文化大革命联系起来,而是采取压制的与世隔绝的办法,搞“关门训练”,所以训练过程中,几次出现贴军队大字报高潮。(不要怕批评,全军在这种批评过程中将会正确地认识世界,并改造世界。——毛注)

黄经耀、胡伟等同志欢迎和感谢同学们对部队的诚恳、善意、坦率的批评。随后黄经耀、胡伟等同志因势利导,转入讨论如何掌握斗争的大方向,做好促进造反派联合的准备工作。李世英同学(交通大学学生领袖,曾经被打成反革命,并几乎被捕死亡,后被救活者——毛注)对军区提出了几条意见。即:军区“支左”必须旗帜鲜明,态度明朗。对新成立的组织要进行调查研究,区别对待,对保守组织要在承认错误和斗争大方向一致的基础上主动争取团结。

要帮助工总×部队整顿,进行调查清理,为大联合扫清障碍。确实做好各大头头的工作。

抓好活思想,相信大多数干部和群众(这是最根本的一条。——毛注)

在做好各院校工作的基础上采取互相串连的方法,广泛开展谈心活动,加强相互间了解,增强团结,促进二大革命派之间的大联合。(开展谈心活动,这个方法很好。——毛注)

黄经耀、胡伟同志认为,李世英同学提出的意见是对的,表示支持,并决定四月二十一日召集西工大、西电、冶院和交大四院校的负责人就如何掌握斗争的大方向,促进造反派的大联合,作进一步协商讨论。


