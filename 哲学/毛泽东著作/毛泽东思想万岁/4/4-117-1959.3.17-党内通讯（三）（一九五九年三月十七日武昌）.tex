\section[党内通讯(三)(一九五九年三月十七日武昌)]{党内通讯(三)(一九五九年三月十七日武昌)}
\datesubtitle{(一九五九年三月十七日)}


各省、市、区党委第一书记同志们:

关于县和公社会议问题。

各省、市、区六级干部大会即将结束,是否应开县的四级或五级干部大会呢?我的意见应当开,并且应当大张旗鼓的开,只是一律不要登报。河南各县正在开四级干部大会,开得很热闹很有益,河南省级负责同志正在直接领导几个县,以其经验,指导各县。湖北、广东、江苏,均已布置全省各县一律开会。江苏省的江阴县委,已经布置五万人大会。河南有两个县是万人大会,多数是四五千人的。我建议县应召开五级干部大会,即县委一级,公社党委一级,生产大队(或管理区)一级,生产队(即原高级社)一级。生产小队(即生产组,又称作业组)一级,每级都要有代表参加,使公社的所属的小队长,所有的支部书记和生产队长,所有管理区的总支书记和生产大队长以及公社一级的若干干部都参加会议,一定有思想不通的人,观潮派,算账派的人参加,最好占十分之一。社员中的积极分子,也可以找少数人到会。使所有这些人都听到县委第一书记的讲话。因为他的讲话比一般公社第一书记的水平要高一些。然后展开讨论,言者无罪,大大鸣放,有几天时间,将思想统一起来。要使三种对立面在会上交锋,十分之一的观潮派,算账派,(有许多被认为观潮派算账派的人,其实并不是观潮派算账派,他们被人看错了)同十分之九的正面人物之间交锋。辩论有三天至四天的时间就够了。然后,再以三天至四天的时间解决具体问题,共有七、八天时间就很够了。县的五级大会一定会比省的六级大会开得更生动、更活跃,要告诉公社党委第一书记和县委第一书记如何工作。在会中,专门召集这些同志讲一次,使他们从过去几个月中因为某些措施失当,吹共产风,一平二调三收款,暂时脱离了群众,这样一个尖锐的教训中,得到经验。以后要善于想问题,善于做工作,就可以与群众打成一片。应当讨论,除公社、管理区(即生产大队)、生产队(即原高级社)二级所有,二级管理三级核算之外,生产小队(生产小组或作业组)的部分所有制问题,这个问题是王××、陶××两位同志提出来的。我认为有理,值得讨论,县的大会在三月下旬即可完结,四月一个整月可以不开公社的代表大会了,忙一个月生产,开些小会,解决些具体问题,由各生产队在工作余暇召开党员大会,再开群众大会,形成全民讨论。因为每个公社都有几百人在县开过会了,问题已讲透了,可以直接进行全民工暇讨论,湖北已有些县在进行全民讨论,到五月间,全国各公社抽出三天时间(三天尽够了)开人民公社第一次社员代表大会,代表要有男的、女的、老的、少的、正面的、反面的(不要地富反坏,但要富裕中农)讨论一些问题,选举公社委员会,这种代表大会建议一年开四次,每次一天、二天至三天。公社第一书记要学会善于领导这种会议。我们的公社党委书记同志们,一定每日每时关心群众利益,时刻想到自己的政策,一定要适合当前群众的觉悟水平和当前群众的迫切要求,凡是违背这两条的,一定行不通,一定要失败。县委和地委都要注意加强公社的领导,要派政治上强的同志,去帮助政治上较弱的同志。地委要注意派人帮助领导较弱的县委。县和公社都要注意加强做为基本核算单位的生产队(一般指原来的高级社)的领导骨干。以上只是当做建议,究竟如何处理较为适宜,请你们考虑决定,县开会时,公社各级都要专人领导生产,或交替到会,不误农时。

<p align="right">一九五九年三月十七日上午七时于武昌</p>


