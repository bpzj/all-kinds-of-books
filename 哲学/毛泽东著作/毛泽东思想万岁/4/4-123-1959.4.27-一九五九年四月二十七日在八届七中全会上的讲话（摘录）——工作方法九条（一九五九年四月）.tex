\section[一九五九年四月二十七日在八届七中全会上的讲话(摘录)——工作方法九条(一九五九年四月)]{一九五九年四月二十七日在八届七中全会上的讲话(摘录)——工作方法九条(一九五九年四月)}
\datesubtitle{(一九五九年四月二十七日)}


一、多谋善断,这句话重点在谋字上。要多谋,少谋是不行的,要与各方面去商量,要反对少谋武断。多谋,过去往往与相同意见的谋得多,与相反意见的人谋得少,与干部谋得多,与生产队员谋得少。商量又少又武断,那事情就办不好,谋是基础,只有多谋,才能善断。多谋的方法很多,如开调查会,座谈会。谋的目的就是为了断。有些同志少谋武断是要不得的。

二、留有余地,这不仅是工作方法问题,而且是个政治问题,我们在安排工作计划时,需留有余地,给下面点积极性,不给下面留有余地,就是不给自己留有余地。留有余地上下都有好处,如农村包产问题,包产指标两千斤,这就是没给下面留余地,也是没有给上面留余地。过去我们打仗留预备队,现在搞生产就忘掉了。经济工作不能蛮打蛮攻,生产东西不能都停掉。计划工作就要留余地。保证重点是主要的,没有重点就没有政策,我们是按照政策办事情的。

三、波浪式前进,凡是运动就有波,在自然科学中有波,曲波,凡是运动就是波浪式前进,这是运动发展的规律,是客观存在,不以人的意志为转移的,我们做工作都是由点到面,由小到大,都是波浪式前进,不是直线上升。

四、要善于观察形势。要经常注意观察政治动态,经济动态。所谓政治动态,就是观察各阶级思想,观察他们立场变化,书记要观察,各委员也要观察。各委员不但要做好分管的工作,也要做好集体的工作。

五、当机立断。把握形势的变化来改变我们的计划,机不可失,时不再来,要当机立断,不要优柔寡断,基本建设摊子大了一些,就要缩短一点。对党内一些不良倾向要当机立断。

六、与人通气。上下左右,左邻右舍,上上下下都要通气。中央与地方商量通气,党委委员商量通气,与书记要通气,我们过去通气少了一些,要想办法通气。现在釆取了写信的办法,一个月写一次,这是通气的办法。不要满足于书记处办事,更不要对省委书记封锁消息。

七、一个人有时胜过多数,因为真理往往在他一个人手里,真理往往掌握在少数人手里,如马克思主义就是在他一个手里。列宁讲要有反潮流的精神,各级领导要考虑多方面的意见。各级党委要考虑多方面的意见,要听多数人的意见,也要听少数人的意见和别人的意见,在党内要造成有话讲、有缺点要改正的空气,批评缺点往往就有点痛苦的,但批评之后,改了就好了。不敢讲话无非是六怕:怕警告,怕降级,怕没有面子,怕开除党籍,怕杀头,怕离婚,杀头,岳飞就是杀头才出名的。要言者无罪,按照党章可以保留自己的意见。过去朝廷有廷杖的制度,不知打死多少人,但还有许多人死在朝廷。

八、要集中:集中在书记处、常委会,要少数服从多数,但党内一定要造成一种空气,精神要解放,批评要开展,批评就是同志式的帮助。

九、凡是看不懂的文件,一定不准拿出来,拿出来也要顶回去。写文件要通俗,要有口语,要有目的性,观点要明朗,讲话要看对象。鲁迅的《阿Q正传》写了很多通俗的话。


