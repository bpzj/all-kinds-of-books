\section[对卫生工作的指示——在中央政治局讨论中央卫生部党组报告时所作(一九五四年十一月)]{对卫生工作的指示——在中央政治局讨论中央卫生部党组报告时所作}
\datesubtitle{(一九五四年十一月)}


一、关于中央卫生部向中央的报告要点,和中医工作指示草案,毛主席指示,这个报告写得很好,特别指出这次检查卫生部工作的方法很好,开了二十多次会,从下面提出许多意见,翻了个底,这样方能解决问题。单是上面讲话还是不够的,必须靠自下而上的批评,这样才能解决得透,解决得好。单从上面解决问题,总有一定的片面性;和自上而下的批评结合起来,看法才能全面,解决才能完全正确。他说,这种办法在今年夏季的全国财经会议用过了,解决了大问题;文委也用这个办法来检查卫生部的工作,结果也很好。这种办法应该提倡。

二、毛主席指出,卫生工作的队伍很大,单是国家卫生工作人员就有二十几万,他要管的是五亿多人口的生、老、病、死,这真是一件大事业,极其重要。因而我们的责任很重。

三、毛主席说,几年来全国卫生工作的成绩很大,但是缺点也很多很大。他指出,卫生工作最大的缺点是政治少了,政治工作少了,马列主义、社会主义太少了。

技术是不是多了?也不见得。医疗、医病都有许多问题没有很好解决。正因为政治太少了,技术也管不好。

毛主席最着重指出的是:党必须领导一切,必须领导卫生工作。他说,“卫生部党委领导同志认为卫生工作是特殊的技术工作,中央不懂技术,我们所不能解决的问题,报告了中央也不一定有办法,因而许多重大问题没有请示报告;认为文委是文化机关,尽是文化人,不懂医学”,因此,“对文委领导有一种不够尊重的情绪”。这些想法是完全错误的。毛主席说,从这种逻辑出发,×总司令就不能当海陆空的总司令了,因为他不能开坦克,驾飞机,打大炮呀!但×总司令完全能当总司令,因为我们是用政治来领导军事。毛主席说,“你不懂这一门,你就不能管我”,这种思想是相当多的,也不仅卫生部门有。在我们的军队中,军事不服政治管的问题,是经过长期斗争才解决了的。卫生部门必须用最大的力量解决这个问题。

我们的党要领导一切:要领导飞机大炮,领导生旦净丑,领导生老病死,就是说,要领导军事、经济、文教卫生和其他一切工作。任何专门的东西,党都要去领导。譬如唱戏,党也要领导。毛主席说,×××唱戏唱得好,但他唱的只是旦角,不唱生、净、丑角,旦角中他只唱青衣,别的也不唱。他是很专的。他不能领导别的。如果他来当主席,就当不好。但是我们的党领导他,领导戏剧艺术的发展和提高。

我们凭什么来领导?毛主席说,我们要用政治来领导。领导首先是政治领导,领导工作就是政治工作。离开政治就说不上领导。他们当卫生部长的,自己的职位就是作领导工作,也就是作政治工作。毛主席批评过去卫生部几位副部长的分工办法,谁管政治,谁管业务的办法是错误的。在旧中国,官僚们有所谓政务官,军事官之分。在我们这里,可不是这样,也不能是这样。凡是领导人都应该政治领导。作为领导人只管技术,只记得技术,只管皮下注射肌肉注射,势必政治挤不进去了,没有政治领导了。作领导工作的人都要懂得党的工作,懂得作群众工作,决不能只管医务,不管政治工作。

卫生工作一定要受党的领导,决不能离开党的领导搞独立王国。毛主席问周总理,卫生部的重要文件经过政务院通过批准的多不多?周总理说很少。毛主席说,人家是独立王国,就不受你管!

四、毛主席也着重地批评了卫生部缺乏集体领导,并且指示了应该怎样进行集体领导。他说,××同志积极负责,就是说骨干是好的。但单干、盲干就不好了。盲干就是忽视党的领导,忽视政治对业务的领导。单干就是不要集体领导。大家知道,集体领导是我们党的领导的最高原则。××同志在自己的检讨中说在许多重要问题上,往往不经组织的认真讨论,即由个人决定,独断专行。毛主席指出,这种单干的作风就要把事情办坏,使自己孤立。这也不能说明个人的个性强,这不是真正的强,而是弱。毛主席指示,作为一个领导者,必须耐烦地和人家商量问题,遇事和群众商量,和上中下三方面商量。自己以为什么都正确,这是完全行不通的。领导者必须讲道理,有说服力,这样才算真强。党组开会要认真准备,要虚心争取大家的意见,要学习中央开会的办法。

五、毛主席着重指出,我们的医务部门领导人员中有浓厚的资产阶级思想,其具体表现就是那种不愿受党的领导,缺乏政治,群众观点薄弱,不要集体领导等思想。这种资产阶级思想在全国卫生队伍中相当普遍,我们一定要加以批判。

六、中央卫生部工作经过检查后,已开始有所转变。但同志们信心还不够,这首先领导上要有决心。毛主席指出,中央卫生部工作的缺点和错误,××同志当然要负主要责任,但也不能挂在××同志一个人账上,其他同志也应负一定责任。他希望中央卫生部的领导工作要有个彻底转变。他问××同志有这样的决心没有。××同志回答说有这个决心。……毛主席说,这主要要提高思想转变作风,这次会议一定要开好,使大家有信心。

七、关于业务方针上的几个问题,中央对报告中所提出的各项方针任务基本上都同意。

关于中医问题,毛主席说,我认为中国对世界上的大贡献,中医是其中的一项。中医是在农业、手工业的基础上发展起来的。西医是在近代工业的基础上发展起来的。中医宝贵的经验必须加以继承和发扬。对其不合理部分要去掉。西医也有不正确的地方,也有机械唯物论。将来发展只有一个医,应该是唯物辩证法作指导的一个医。看不起中医是错误的。把中医提得过高了,也是不适当的。中医医院可以试办。中医进修问题,让他们学一些基础医学知识是可以的,但应该加入中医临床经验课程。医院可以请中医看病,治好了要向他学习。中西医的名称是不妥当的。

八、城市、工矿卫生,城市医疗要改革,要革命。


