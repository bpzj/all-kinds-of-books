\section[在南京部队、江苏、安徽二省党员干部会议上的讲话(一九五七年三月二十日)]{在南京部队、江苏、安徽二省党员干部会议上的讲话}
\datesubtitle{(一九五七年三月二十日)}


我变成了一个游说先生,一路来到处讲一点话。现在这个时期,有些问题需要答复,就游说到你们这个地方来了。这个地方吗南京,从前也来过。南京这个地方,我看是个好地方,龙蟠虎踞。但有一位先生,他叫章太炎,他说“龙蟠虎踞”古人之虚言,是古人讲的假话。看起来,这在国民党是一个虚言,国民党在这里搞了二十年,就被人民赶走了。现在在人民手里,我看南京还是个好地方。

各地方的问题都差不多,现在我们处在一个转变的时期,就过去的一种斗争-阶级斗争,基本上结束,基本上完毕了。对帝国主义的斗争是阶级斗争;对官僚资本主义、封建主义、国民党的斗争,抗美援朝,镇压反革命,也是阶级斗争;后来呢?我们又搞社会主义运动,社会主义改造,它的性质也是阶级斗争的性质。

那么,合作化是不是阶级斗争呢?合作化当然不是一个阶级向一个阶级作斗争。但是合作化是由一种制度过渡到另一种制度,由个体的制度过渡到集体的制度。个体生产,它是资本主义的范畴,它是资本主义的地盘。资本主义发生在那个地方,而且经常发生着。合作化就把资本主义发生的地盘,发生的根据地去掉了。所以从总的来说,过去我们几十年就干了个阶级斗争,改变了一个上层建筑,旧的政府,蒋介石的政府,我们把它打倒了,建立了人民政府,改变了生产关系。改变社会经济制度。从经济制度和政治制度来说,我们的社会面貌改变了。你看,我们这个会场的人,不是国民党,而是共产党。从前,我们这些人,这个地方是不能来的,那一个大城市都不许我们去的。这样看来是改变了,而且改变了好几年了,这是上层建筑,政治制度;经济制度改为社会主义经济制度,就在最近几年了,现在也可以说是基本上成功了。这是过去我们几十年斗争的结果。拿共产党的历史来说,有三十几年;从鸦片战争反帝国主义算起,有一百多年,我们仅仅做了一件事-干了个阶级斗争。同志们,阶级斗争改变上层建筑跟社会经济制度,这仅仅是为改变另外事开辟道路。现在,遇到了新的问题。过去那个斗争,就我们国内来说,现在基本上完结了,就国际上来说,还没有完结.为什么我们还要解放军呢?主要是为了对付外国帝国主义,恐怕帝国主义要来侵略,它是不怀好心的,国内也还有少数没有查出来的反革命残余分子,有一些过去被镇压过的,比如像地主阶级、国民党残余,如果我们没有解放军,它又会起来的。地主、富农、资本家,现在守规矩了,资本家还不同些,我们把它当作人民内部的问题来处理。民族资产阶级接受社会主义,跟农民接受合作化不同,他们可以说是一种半强迫,就是说有些勉强,而且是在对他们相当有利的条件下接受改造的,所以,现在是处在这么一个变革时期:由阶级斗争到了向自然界作斗争。要改善生活,要搞建设,就要向自然界作斗争。由革命到建设,由过去我们反帝反封建的革命和后头的社会主义革命,到技术革命,到文化革命。关于思想工作我们这个国家要建设,就要有技术,就要有机器,就要有科学,这个方面,我们过去是用手工,用手使用的工具来做桌椅扳凳,用手去种粮食种棉花的,一切都是用手,现在要改为用机器,使用机器,那么一种技术,这就是一个很大的革命。没有这样一种革命,我们这个国家单是政治改变了,社会制度改变了,我们国家还是一个穷国,还是一个农业国,还是一个手工业、手工技术的国家。为了这个我们就要进行一个文化革命。这种情况的变化,我想大家也就懂得了,也就意识到了。但是,似乎还有一些人不大清楚,不认识这么一种变化。现在发生一些新的问题,科学技术的问题,文化的问题,人民中间向我们提出一些要求,发生一些人民闹事。这些问题在什么地方?临时工、农村的困难户,学生方面,合作化方面,都发生新的问题。有一小部分工人发生罢工,有一小部分学生发生罢课,或者游行示威,请愿,复员军人也有想闹事的,也有一部分合作社社员,富裕中农,人数不多,但是他们不满意,要脱离合作社,还有些别的事情,他们不满意,向我们闹。怎么办?人民闹事怎么处理?我们就有必要讲清楚这个问题。矛盾应该分为两类:第一类,就叫做敌我之间的矛盾;第二类就是人民内部的矛盾。我们过去几十年就解决了第一个矛盾,现在要解决第二个矛盾。这里面表现在各方面。比如现在从农业国到工业国,这也是矛盾。

我们技术不够,没有机器,没有文化,生活不好。有些人讲:“到了社会主义,大概是要过好生活了。”“不是讲社会主义已经成功了吗?去年共产党开了大会,说社会主义基本上完成了大概就要过好日子了。”这就是不懂什么叫社会主义。社会主义,它作为社会制度,就是生产关系。我们建立了一种关系,跟过去的关系不同,人民进行生产,过去是资本家和工人的那么一种关系,地主和农民的那么一种关系;现在,我们建立为一种社会主义的关系,用这样的一种相互关系去进行生产,也就是一种社会主义的关系来进行生产。因为过去旧的方法不适宜,不利于生产,不利于发展,比较差些,弄得中国人民长期地又穷又是文盲,在世界上是人们看不起的。现在这种关系刚改变,还没有生产,没有生产就是没有生活,没有多的生产就是没有好的生活。要多少年呢?我看大概要一百年吧。要一百年,我就不在这个世界上了,我就不能享福了。那当然不要那么长,分几步来走,大概有十几年要稍为好一点;有个二、三十年就更好一点;有个五十年可以勉强像个样子;有一百年,那就了不起,那就和现在大不相同了。一百年,是很短的时间,就是搞一万年,人总要生活下去的。

刚刚革命,刚刚搞社会主义,这个生活怎么会好起来?粮食怎么会多起来?粮食多了没有呢?是多了,一九四九年,人民政府成立的那一年,我们只有二千二百亿斤粮食,去年我们就有三千六百多亿斤粮食,增加了一千四百多亿斤。但是多少人吃呢?我们这个国家好处就是人多,缺点也是人多。人多就嘴巴多,嘴巴多就要粮食多。增加这一千四百亿斤粮食,就不见了,有时还觉得没有粮食。一九四九年没有粮食,现在还是没有粮食,还是不够。要过好生活,现在我们搞出十二年的生产计划,科学计划,农业计划,工业计划,一步一步来工作、生产。我们年纪大一点的人,这个问题容易懂,青年人可不容易懂。好像他来到了世界上,样样要像个样子,因此要向他们进行教育,要向广大人民群众进行教育,特别是对青年进行教育,要进行艰苦奋斗,白手起家的教育。

我们现在是白手起家,我们祖宗给我们很少。我们的祖宗是谁呢?帝国主义、封建主义、蒋介石,它是我们上一辈子,我们上一个政府。他们给我们的就是把人民身上的肉刮去了,但是,他们走了,也就好了。他们走了,空出一块地方来。我们这块地方有九百六十万平方公里,东边从海边起,西边到昆仑山、帕米尔高原,北边到黑龙江,南边到海南岛,就是这么一块地方,让我们跟全国人民一道、跟国家一道、跟青年们一道,干它个几十年。长期不说,干它个五十年。……。

要分清两类矛盾,第一类敌我矛盾不能和第二类人民内部的矛盾混淆在一块。关于社会主义社会,是有矛盾的,是存在看矛盾的。这一点列宁曾经指示过,他是承认社会主义社会有矛盾的,斯大林在开头,在列宁死了以后的一个时期,苏联的内部生活,还是比较生动活泼的,跟我们现在差不多。也有各党各派,也有一些比如托洛茨基那么样出名的人物,他有很多人,不过他大概是共产党内部的民主人士,而且是调皮角色,跟我们闹。此外,还有一些人,社会上也可以说各种话,可以批评政府,那个时候曾有那么一个时期。后来不行了,后来就搞得很专制了,就是批评不得谁要批评,百花齐放,那是很怕的。只能放一朵花,百家争鸣也怕的,风吹草动,就都说是反革命,就抓人,就杀人。这就是把两种矛盾混淆了。把人民内部的矛盾误认为敌我矛盾。你们南京许家屯同志说,很多学生向他来请愿,队伍很整齐,省长彭冲也说,纪律很好,在路上都是很好,一到他那个衙门里头,就喊“打倒官僚主义”,要解决什么问题。这样的问题,如果拿到斯大林,我看就要抓他几个,难免有几个,头要落下地。你打倒官僚主义,不是反革命?其实一个反革命都没有,是很好的青年学生。而是那个问题应该解决,确是有点官僚主义,因为,那些华侨学生闹事,这个问题没有解决好。学生这么一请愿,就帮助了我们,学生也得到了教育,很多干部也得到了教育,华侨学生也得到了教育。闹事的、打人的,也不打了,何必一定要天天打呢?因为我们过去没有对他们很好地进行教育,没有发动群众批判他们。像这类问题,还有什么复员军人闹事。曾希圣同志,他在你们附近的安徽省,他那个地方复员军人闹事,找到他,他讲了四十分钟的话,问题解决了。开头还有一股气,后来这一股气不晓得哪里去了。总之问题是没有好多了,问题就解决了。其中已经查出一个冒充革命军人的人,也很坏,他是个领袖。

对于人民闹事,就是不能采取对地主阶级、对国民党、对帝国主义的那种做法,要采取完全新的方法。除了那个犯了法的人以外,比如讲,他拿刀杀人,打伤人,跑到办公室把桌椅打得稀烂,这就应该按照法律程序处理;其他的人,那怕他犯了错误,他是领袖,率领人闹事,应该说服教育,不要开除出工厂,不要开除出学校,不要开除出机关。你开除他,他跑到哪里去呢?你开除出这个学校,他就进那个学校,那不是一样吗?你开除出这个工厂,他就跑到那个商店,他总要落一个地方。他不能跑到中山陵上去活,那个地方没有粮食,没有房子,他不能在那个地方活的,不能在野外活的,总要落一个地方。你这里开除,是以邻为壑,可以舒服一点,我说你这样太舒服了不好,一个机关,一个学校,一个工厂,有这么一点闹事的人,是不坏的。

我们再谈到第一类矛盾-敌我矛盾,敌人同人民,人民同敌人反革命的矛盾。现在存在的两种观点都不妥当,一种就是右倾观点,认为世界上太平无事了,对于一些应该依法处理的,反动分子和坏人,不依法处理,这个不好,这是右倾机会主义的观点。这方面,现在各省都有发生,各省应该注意。民主人士中间,民主党派中间,有一些朋友,他们对于这个问题,也是有右倾的观点,他们比我们右倾观点有时还右得厉害一点。因为有些反革命,就是他们的老朋友,现在有些关在班房里,有些杀掉了,他们有点伤心了,因为杀了他们的亲戚,杀了他们的朋友。有的观点,我们要讲清楚。在党内这种观点是不好的。第二,也有夸大的观点,也有“左”的观点。说现在还有很多反革命,这可不对。现在还有暗藏的反革命,要肯定。过去肃反根本上是正确的,如果不肃反,那可不得了。我们中国不会出匈牙利那样的事情,其中有一条原因,就是我们肃清了反革命,而匈牙利没有肃清。所以说,现在还有很多反革命,这个观点就是不符合情况了,夸大了。

对于第二类的矛盾,人民内部的矛盾。因为阶级斗争基本结束,现在比较显露出来了,比较暴露出来了。对于这类问题,我们同志中间的意见也不一致,有各种不同的意见。还要经过说明,经过讨论,经过研究,使得同志们在这个问题上,意见统一起来。对于人民还跟我们闹事,没有精神准备。因为我们过去跟人民一道反对敌人,现在敌人不在了看不见敌人了,那么就剩下我们跟人民。他有事情,不向你闹,向谁去闹呢?过去是向敌人闹事,那就革命,现在这个就不叫革命了。你把我革掉,怎么办呢?又请蒋介石来呵?但是事还是要闹的。因为事情你没有解决得好,凡是解决得好的,十个地方九个地方解决得好,十个问题九个解决得好,有一个地方一个问题没有解决得好,那个地方就要闹。这个闹事是正常的。你没有解决得好他为什么不闹呢?请问:他不向你闹向谁去闹呢?向“蒋委员长”去闹吗?他到台湾去了。他们就要向工厂的厂长闹,向合作社的社长闹,向乡政府闹,向市政府闹,向人民政府闹,向学校的校长闹了,因为你没解决得好。我们工作中有官僚主义、主观主义、宗派主义,我们人很多,意见也不齐,因此存在官僚主义。这里也是有“左”有右的观点,对于人民闹事,也有主张用老的办法来对付,横竖我们办法是有一套,我们是搞了几十年,你晓得老子干革命干了多久呵!是不是?总是有一套办法,就是拿我们对付敌人的办法,有时候也用一下,压一下,叫警察。就有几个地方叫警察抓人。学生罢课叫警察抓人,这是国民党的办法,国民党就用这办法。也有束手无策的,那完全是没有办法。过去对付帝国主义很神气、很威武、就是不怕。小米加步枪,你帝国主义、蒋介石、飞机大炮,不怕。不怕帝国主义,可有点怕人民闹事。我说,帝国主义你不怕,你还怕老百姓?可就是有那么怪,就是怕老百姓,帝国主义他是可以不怕,老百姓一向他闹事,他可没有办法了。因为他没有学过这一条,他还没有学好。过去学的是对付帝国主义的,对付蒋介石的那一套,要讲打土豪、分田地,他是好手;要讲对付人民闹事,他没有学好,这一课没有上过,所以,很值得研究。要跟党内党外公开地提这个问题,开展讨论,办法就出来了。

同志们!究竟是对付帝国主义容易些,还是对付老百姓容易些?对付敌人难办些,还是对付人民难办些?敌人,你撵了多久,他就是不走,赖皮得很,特务钻到机关里头、学校里头、工厂、农村里头,他就不走。老百姓无论如何不是特务,他不是帝国主义、他不是地主、资本家,他是老百姓,他是劳动人民,很容易讲清道理,所以我们许家屯同志、彭冲同志对于华侨学生打人,很大一批学生不满意,向他们请愿,他们就来了一个说服,结果是很好地解决了问题。

我说,对于全国,我们的方针是要“统筹兼顾,适当安排”,加强思想教育。这个方针可以说是战略方针。因为这是六亿人口的“统筹兼顾,适当安排”。这里包括地主、富农、民族资本家。没有杀掉的反革命分子,你总要去安排他;失业的各种人,都要做适当的安排,总要使他们能够生活,有事情做。其中,有大约五百万知识分子。中国这个国家,知识分子太少,但是也有一批,大概有五百万左右。其中不到一百万进了共产党,还有四百万在党外。他们干什么事情呢?他们在我们政府系统做工作,军队里面也有一小部分,教育系统有二百万人,大学、中学、小学都算,财政经济系统有一百万人;还有科学工作者,文学艺术工作者文学家、诗人、艺术家、画画的、唱戏的,还有新闻工作者,办报纸的,等等。上海很多。

这么一大堆知识分子,有些是工农化了的,接受了马克思主义,进了共产党,有些没有进,但是很接近我们。积极拥护马克思主义,赞成马克思主义的占一小部分,大概占百分之十或者多一点,据江苏说有万分之十七。另外一点,他们是有敌对情绪的,但是,不是特务。他同我们敌对,不赞成马克思主义,社会主义制度勉强地接受,没有办法,大势所趋。这样的人,也是的,可能占百分之几。中间的有百分之八十,或者没有这么多,有百分之七十至八十。他们是中间派,动摇分子,对马克思主义有些赞成,也读了几本书,但是,没有读进去,就是读在这个上面(指额头上面),没有进到里头去。考知识分子有个办法。人们说要分别一下,究竟是小资产阶级知识分子,还是资产阶级知识分子?意思就是说,带一顶小资产阶级的帽子,比较资产阶级的要舒服一点。可是我说不然。我就是个资产阶级知识分子。进的资产阶级学校。那个社会的空气是资产阶级的空气,搞的那一套,就是唯心论的什么东西,康德的唯心论我就信过,你说那是小资产阶级的?读的是资产阶级,信的是资产阶级,你还能说是个小资产阶级?有的时候,是可以分的,也应该分的,但是讲到世界观,可就难分了。你说,小资产阶级的世界观是什么东西呵?是半唯物主义吗?我这个人马克思主义是后来钻进去的,后头学的,而且是经过很长的时间,跟敌人作斗争逐步改造的。有一件事情,可以考知识分子,就是跟工人、农民一道,看谁跟劳动人民打成一片。知识分子有一部分人可以打成一片,大多数知识分子,他还离开得这么远,想要打成一片,打不拢来。他跟工人、农民没有感情,不是朋友,工人、农民有话也不跟他们讲,他也看不起工人、农民。知识分子有一条尾巴,要泼它一瓢冷水。狗,泼它一瓢冷水,尾巴就夹起来了,你如果是另外一种情况,它就翘得很高,它就有些神气。因为他读了几句书,确实有些神气。劳动人民看见你那个神气,看见你那个样子,他就不舒服。

我们的任务,就是争取百分之七十至八十的动摇着的中间派。他们一般的赞成社会主义制度,但是对于马克思主义作为世界观,他们还没完全接受。我说,他们不是全心全意地为人民服务,而是半心半意为人民服务。他们有半条心想为人民服务,这是好的。但是,还有半条心,不晓得放在哪个地方。你要说他拥护台湾,他也不拥护台湾,但他一讲外国,恐怕还是美国好。“你看,美国有那么多钢,美国科学很发达”。我说,外国好,资本主义国家、西方国家好,是不是好呢?是好。他们有那么多机器,那么多钢铁,我们什么都没有。但是他好是他好,不是你好,你美国有那么多钢,是美国人的钢,并不是我们中国人民的钢。我们天天吹他那个好,有什么用处?它每年生产一万吨钢,我们没有。我们每年能多几万吨钢,我们就高兴。我们现在只有四百万吨钢,按照第一个五年计划,可以完成四百一十二万吨钢,可能超过,可能多一点,可能有四百几十万吨钢。一九四九年只有十几万吨钢,最高年产量是一九三四年,主要是日本人的,那才有九十万吨。蒋介石搞了二十年,我说蒋介石该倒,是有道理的,并不是糊里糊涂我们把他赶走了。他搞了二十年,只有几万吨钢,其中还有一部分是清朝末年张之洞他们搞的。我们搞了七年,今年来讲八年,可以搞四百多万吨钢。所以我们增加一吨钢是我们的事情,我们就高兴。你美国增加几百万吨钢,我也不高兴,你越多我越不高兴。你增加那么多钢,干什么呢?你增加那么多钢就很危险,要打我们的。我们有些知识分子,还在那里吹美国钢!钢!钢!那么多!这是我们要说服他们的,我们要说服知识分子。

有些知识分子在作教员。科学家都是教员,大学教授都是教员,中学、小学教员都是教员,教育人民;新闻记者,办报的教育人民;广播员、文学家、艺术家都是人民的教员;技术员,工程师,是我们工厂所必不可少的。这几百万知识分子,我们如果看不起他们,“我们离开这几百万知识分子”如果以为可以不要他们,这种观点是不妥当的。我们离不开他们,我们离开这几百万知识分子,我们就不能活动。可以说,一步都不能走。我们学校就不能办。我们有许多报纸就不能出,我们的文学艺术,共产党就没有出梅兰芳,就没有出周信芳;现在有个袁雪芬进了共产党,总还没有梅兰芳、周信芳、程砚秋,大学教授就没有,工程师现在开始有一点,很少;技术人员开始有一些进了共产党,大批的还是党外人士。

所以,我们几百万知识分子,不管他多么动摇,但是,是有用之人,是我们人民的财产,是人民的教员。现在只有他们当教员,没有别的教员,因为他是上代遗留下来的,是社会遗下来的。论他们的出身都是地主、富农、资产阶级,但是可以教育过来,我们不要搞唯成分论,鲁迅也是地主、富农、资产阶级,马克思就是地主、富农、资产阶级,列宁也是,那怎么得了呀?这就不能讲唯成分论。因为现在,他们——地主、富农、资产阶级,没有社会根子了,他们的社会基础,他们的社会根子,我们已经挖了,现在他们落在空中,就像降落伞一样,吊在空中了,所以就便于改造,并不要怕他们。

有些工农出身的同志,工农干部,看到知识分子,有点恼火,恐怕吃不消,咬不烂。知识分子有点麻烦是真的。因为,知识分子就是知识分子,他们的麻烦也就是读几句书,我们就是少几句书,你少几句,他的尾巴就翘起来,事情也就不好办,有的就难办。所以讲难办也就是难办。但是讲好办也还是好办。这七年,这几百万知识分子还有进步,应该肯定这一点。你们江苏省可以证明,你们这里知识分子最多,是不是,总算是还有进步。

民生党派是什么呢?民主党派他们还都是知识分子。民主党派里面工农干部很少,有什么工农干部?民主党派——民革、民盟、民建、九三学社,农工民主党,都是一些知识分子。所以,我们党提出“百花齐放,百家争鸣”,“长期共存,互相监督”。江渭清同志说,要我讲讲这个问题。

提出这样的方针是有理由的。“百花齐放,百家争鸣”,还是要“放”,还是要“收”?现在党外人士就说我们“放”的不够,他们就深怕我们“收”。而我们同志呢?看那个样子似乎不对,就有一点想“收”。有一点想收兵。那么我们呢?中央的意见,也和各省的同志谈了,去年十一月开了二中全会,今年一月开了省委书记会议,我们取得了一致的意见。认为:“百花齐放,百家争鸣”这样的方针,是应该坚持下去。应该“放”不是“收”,不对的应该批评,错误的意见,错误的作品,或者一篇文章里头,一个作品里头有一部分错误,那就要批评那一部分,但是要用说服的办法。所以还是用“说”,还是用“压”?还是用“说服”,还是用“压服”?这两个办法里头,采用一个办法。还是“放”,还是“收”?这两个里头取一个。我们认为还是要“放”,不要“收”。那么“放”,就放出许多东西来了。有许多东西就不对头,怎么办?那么就“压”?还是采取另外一个方法,说服它?有些同志,手就有些痒,想去“压”,把对付阶级斗争的邢一套搞出来,军法从事,用简单的办法,或者是不调兵,用行政命令,看不顺眼的,就把它“压”一下。中央认为,这样不好,压是不会服人的。从古以来,就没有压服过人的。我们对敌人那是要压,压了之后,还是要“说”。比如俘虏一解除武装,我们就说服他。反革命,只要不杀头的,我们还是争取、教育、改造它。高压政策不能解决问题,人民内部的问题不能采取高压政策。这么一“放”,又是用“说服”,不要用行政命令来高压,会不会天下大乱?我们说不会乱。会不会从各方面进行批评,在报纸上、刊物上、会议上批评我们的缺点,会不会把我们批评得不得下地,把人民政府批倒,像那个匈牙利一样,会不会那样?我说不会。中国的情形跟匈牙利不同,共产党有很大的威信,人民政府有很大的威信。马克思主义是真理,这是批评不倒的。老干部也批评不倒,所以,老干部不要怕批评。老干部批评一下很有好处,如果我们身上有官僚主义,有缺点,首先让党内批评,然后党外批评,批评我们的缺点,把我们的官僚主义改一改,把缺点改一改,不就好了吗?会不会倒呢?不会倒的,人民政府怎么吹得倒呢?上海去年刮了龙卷风,把个什么大东西吹起了,房子还有什么装石油的,吹上天去了。起那么一股风,但是上海人民政府没有吹走,不管刮多大台风,我看人民政府,共产党,马克思主义,老干部、新干部,只要是真心真意为人民服务的,吹不倒,半心半意替人民服务的,那要吹倒一半;那一点心思都没有,跟人民政府敌对的,那么该吹倒。

毒素怎么办?百花齐放这一来,放出许多毒素,蛇口里吐出一朵花来。我说对于有毒素的东西,有篇文章叫《再论无产阶级专政的历史经验》。同志们大概都看过了,这个里头有这样几句话,说民主集中制,如果有缺点,就应该批评。毫无疑问,民主集中制的集中必须建立在广泛的民主的基础上,党的领导必须是密切联系人民群众的领导。在这些方面如果发生了缺点,就必须坚决地加以批判和克服。但是对于这些缺点的批判,只能是为着巩固民主集中制,巩固党的领导,而绝不能像敌人所企求的那样,造成无产阶级的“涣散和混乱”。这个对不对?这个对了很好。你们拿这个东西,跟民主人士讨论,要大学生去讨论。这篇文章上讲了的:批判是可以,但是批评的结果,批评的目的,就是要巩固民主集中制,巩固党的领导,绝对不能像敌人所希望的那样,造成无产阶级队伍的涣散和混乱。

这些原则性,原则必须如此。但是有一种灵活要注意,让人家说话的时候,会说出一些很不好听的话来,百花齐放,会放出一些很不好看的花。有些什么罢工、罢课、请愿、游行示威,中间可能出一些乱子,它的目的不一定跟这个相符。群众闹起事来,那些知识分子,百分之八十都是没有学会马克思主义的,还是资产阶级世界观,他怎么会了解这一条?开起会来,批评的时候,会出一些乱子。这时候,如果到处都拿这一条去压:“你看,我有一本书,你看过没有?”也不行。你是要学我们彭冲同志、许家屯同志他们解决这个问题的样。他们在这个时候,如果就搬出这一条来念,尽念。就是念这两句,别的话都不讲,保证他这个市长干不成了。因为要解决具体问题,他们有时候搞得过火一点,这是难免的。

文学作品里头,有些是不对头的。上海唱“狸猫换太子”,我是没有看过那个戏,说是各种妖魔鬼怪都上来了,我说上来一点也不要紧。妖魔鬼怪,很多人就没有看过,不要天天搞妖魔鬼怪,今天也搞,明天也搞,尽搞。搞那么一点,见见世面,见识那个封建时代遗留下来的艺术化了的意识形态。这跟神话不同,比如“大闹天宫”,是大家赞成的,没有哪个反对,还有什么“劈山救母”,“水漫金山”,“断桥”之类,那些都是神话。也没有哪个反对。就是有另外一些东西,也不必那么急,让它搞一个时期,会有人批评的。出了几篇小说,写了几首诗,演了一些什么“狸猫换太子”,心里就那么急?我们可以慢一些,让社会评论,逐步地使他们那些作品,那些戏剧,加以适当改变,而不要用行政命令来禁止。同志们!不要误会我在这里提倡妖魔鬼怪,我不是提倡这个东西,而想消灭它,而消灭的办法,要让它出现,让社会上大家公评,总有一天它会慢慢要丧失,要逐步改造的。我们过去就是用一个命令禁止,禁止了七年,现在人家又慢慢出来了。可见得,我们的那个禁止是不灵的。

最后我讲一下要在党内党外把我们党提出的这些问题开展讨论。讲一样的话,在党内讲,也跟党外讲。比如半条心这样的话,也要跟他们讲“我就是讲你半条心,你怎么样?”“那就不行!你讲我半条心,就不行!”不行,我要打架,打架也不怕。你是半条心,我说你半条心,你还不舒服?你有半条好嘛!你还有半条心,这是讲世界观,我不是讲你对社会主义制度。社会主义制度你是拥护的,但是有一些也不见得,譬如合作社这个制度,有些是怀疑的。但一般讲社会主义,五年计划,你要问他赞不赞成,他说赞成。宪法赞成不赞成,他赞成。共产党一般他拥护,但是讲到世界观,辩证唯物论,这一套就不赞成了,或者只有一部分赞成。我们讲半条心,就是这个道理,因此,你还有一个任务,还要改造。你有二重性,一重是赞成社会主义,二重是不彻底,所以是半条心。想跟人民接近,但是又不能完全接近,不能打成一片,从乡里溜一转回来,还是差不多。

听说,你们南京曾经出了一件事,在“三反”的时候闹出来的。有一位作家,是作家协会里头的秘书长什么人,下去体验生活,带了城里的饭到乡里去吃,回来的时候,预先就有报信的到南京,南京作家协会接了信之后,就排起队伍来欢迎,站立两厢,我们这位体验生活的作家,走中街而过。总而言之,很神气就是了。还有一位在“三反”中间,他这个人结婚,一定要在结婚的那一晚,睡总统府蒋介石睡的床,他要去睡一晚。总之世界上有各种各样的怪事就是了。来了“三反”、“五反”把这些事都暴露出来了。南京方面这些材料送到北京,我们也看到了,那种人心情状态,总是有他的兴趣,这是总统睡过的床呀,这个总统姓蒋,蒋介石呀,我今天结婚,一定要在那个地方睡一晚。我们要使用民主党派、民主人士。刚才讲知识分子,民主党派就是些知识分子。我们要用他们。人们说用是好,可是他们没有用处,是老废物。废物也要利用,废物也有好处。应该用他们,应该开会,这回北京开政协会议,我也跟他们讲过一回,每年不要开会就是应付一下,应付一下过了就算了,而要是利用开会,每个省一年开一回,两回,利用这些机会,给他们做工作,说服他们,使他们替我们去做工作,因为他们联系一些人,经过他们去说服那些人。这样的态度,就是一种积极的态度,而不是一种消极使用他们的态度。就是“利用”、“限制”、“改造”嘛。我们同志喜欢后面两条,一条叫“限制”,一条叫“改造”,就不喜欢头一条那个“利用”。我就“改造”你,我就“限制”你。当然那是对资本家讲的,现在对民主人士不好这么样,对民主人士不好讲“利用”、“限制”、“改造”,可是,我们的同志事实上就是一个限制,就是不去“改造”,不去“利用”他们,他们可以做一些我们不能做的工作,要同他们讲真心话,有很多事情不要用两套,不要党内一套,党外一套。像我跟同志们讲的话,我都可以跟他们讲,当然任何一个党,它有一部分事情是不跟党外讲的。民主党派他们也有一部分事不跟我们讲,我们也有一部分事情不跟他们讲,但是,关系政策的事情,都可以讲,党内是这样讲,跟党外也是这样讲。材料都可以看。有一材料,比如讲罢工、罢课、游行示威,暴露我们官僚主义的这样一些材料,可以印发给他们看,他们平时看不到,这样一来,反比较好。有时可以开两个党的共同的会议,党内党外同时参加。最近,北京开宣传会议,开的比较好,有一百五、六十个党外人士参加。占五分之一。你们如果开的时候,可以多一些,可以让他们占五分之二。我们在政策这些问题上,艰苦奋斗这一点上,做很多工作。

艰苦奋斗,江渭清同志也讲了这一点,我先也讲了,要有多少年我们人民的生活才能改善?现在要提倡艰苦奋斗,但是不等于女同志不穿花衣服的,在衣服还是可以穿,据他们研究,花衣服便宜,现在做两套,里面穿花衣服,外面穿兰布褂子,那花钱太多,我们要节省,穿花衣服是个节省的办法。各种事情,我们都要从节约这点上,艰苦奋斗这点上,做很多工作。

现在已经可以看出,有些同志精神不振作,就是做那么一点事情,没有事情就是打扑克、打麻将,听说打扑克成风,有时一打打到天亮。要养成读书的习惯。我不反对打扑克,也不反对跳舞,也不反对看戏,就是不要太多。我们的长处使用不上了。我们的长处就是阶级斗争,就是政治军事。我们现在的缺点,就是缺乏文化,缺乏科学,缺乏技术。我们要在这方面学习。这些话我在一九四九年《论人民民主专政》那一篇文章里就讲到过,说我们过去的长处,我们所会的东西,我们所了解的东西,快要不用了,而我们所不懂的东西,现在摆在面前,因此给我们一个任务,就是学习。过了七年,更加觉得要提倡学习,要养成看书的习惯,使看书占领工作之外剩余的时问,把剩余的精余的精力用在那上头,这样想打扑克的那个味道,就没有那么多了,味道钻到书里去了,钻在学习里去了。

有一些同志,表现革命意志衰退,缺乏“拼命”精神。什么叫“拼命”?水浒上有位叫“拼命三郎石秀”,就是那个“拼命”。我们从前干革命就是一种拼命精神,有那么一股劲。现在这几年就慢慢地有一些同志劲头不大了。一闹一评级,有些人闹得不成样子,像什么三天不吃饭,我说四天还可以。你三、四天不吃饭,人送扳去,我说送早了,你让他饿四天、五天,一个星期不吃饭,就相当有问题了。三天不吃饭有什么要紧?急于送牛奶、鸡蛋,何必那么急呢。有些人痛哭流涕,比级别高低,比薪水多少,比衣服漂亮程度。

工资恐怕是要加以调整,中央还没有做出决定,去年不是提出工资改革,增加了工资?工资是应该增加的,但是有些是增加多了一点,如行政系统,也许还有教育系统。这不是讲工人、讲工厂,而是讲行政系统。我们有一百七十万行政工作人员(包括乡干部,不包括合作社),教育系统有二百万人,另外有商业系统以及工厂以外的事业系统,还有三百八十万解放军,合共有一千万出头一点。这就是我们组成为国家的这么一个人数。产业工人过去几百万。现在增加到一千二百万人。因为我们国家大,工作人员也是要多一点。所以准备在有条件的时候,就是讲有出路的时候,把一部分人回到工厂里头去,回到合作社里头去。主要的生产是两个部门,一个工业,一个农业。就靠这两个部门来生产。现在看起来工资有一些不很适当,引起人们不满。我说这个问题,如果社会上讲话的多了,倒是好办,因为讲话的多了,我们就有据据来一个调整,要保持过去我们革命的时候,阶级斗争的时候,那样一种精神,那么一股劲儿,那么一股热忱。革命热情,就是要把这个工作做到底。每一个人,有一条生命,看你能活多久?或者六十岁,或者七十岁,或者八十岁,或者九十岁。有一个画家齐白石,他有九十八岁。看你有多长的生命,到那个时候,你真是不能工作,你就不工作,当你还能工作的时候,你多多少少应该工作。而工作的时候,要有一种热情。缺乏热情停滞下来,这些现象不好,应该作教育。有些人就是因为机关庞大,许多人围在一起,没有工作,打扑克,毫无出路。你堆上一大堆,只有那么几件小事办,他不打扑克?

要加强思想政治工作。军队里头怎么样?今天军队里头的同志很多,平时的政治工作和战时的政治工作是不是有些不同?……现在实行了军街制度和各种制度,一面实行这些制度,一面还要上级和下级打成一片,跟士兵打成一片,还要准许他们有批评,比如开党代表大会,给他们一个批评的机会。陈毅同志讲得好,他在“三反”的时候,在华东军区,说是我们专制了几年,现在让人家向我们专制一个星期可以不可以?我们发号施令多少年,都可以的,现在让下级向我们说一点话,批评一下,批评一个星期,只有一个星期,可以不可以?他的意思是说,应该是可以的。我是赞成这个说法。硬着头皮,让下级批评我们一个星期,在他们批评之前,先准备一下,做点报告,自己有什么缺点,无非是一、二、三、四……有那么三、四条,然后让同志们发言,补充一点,批评一下。有功劳的,人们不会把我们的历史丢掉的。战士们对连排长,也给他们一个机会,最好一年有这么一回,开这么几天的批评会。我们曾经做过,结果是有益的。军事里面的民主,军队里面的民主,这样不因为我们有军衔制度,有各种制度,而使上下、官兵以及军民、军队跟地方这个密切的关系受到损失。毫无疑问,上下应该密切,应该是一种同志的关系;官兵、军事干部跟战士们,应该密切地打成一片,军队跟人民、跟地方党政,应该密切的。

全党应该强调思想工作,我们今天讲的总题目就是思想工作,思想问题。因为这个问题近来比较突出,特别是“百花齐放,百家争鸣”,“长期共存,互相监督”,人们就说,是不是要“放?”恐怕危险。长期共存,为什么要怕长期共存?“你那个民主党派,我们革命的时候,你在哪里?”只要搬出这一条来,那他就没话说,他就遭殃,我们说这个时候还不要搬这一条,不要靠官职,靠职位高吃饭。不要靠老资格吃饭。说资格老,几十年,都是真的,可是有一天,你办了一件糊涂事,讲了一篇混账话,这个时候,人们也不谅解。你解决的事情解决得不对尽管你过去做过许多好事,职位有多大,因为你今天的事办得不好,你对人们有损害,这一点他不能原谅。因此,我们不要靠老资格吃饭,要靠解决问题正确来吃饭。靠正确,不靠老资格。你还是靠正确、还是靠资格?靠资格就吃不了饭,因为你搞的事情不正确,你解决问题不正确。你虽然有资格,但是人们不能原谅。因此我们索性都不靠资格,还是一个什么官都没有做,就是不摆老爷架子。老爷架子不要摆。收起来,跟人民见面,跟下级见面不摆官僚架子,不靠老资格吃饭。这一条,特别是老干部要注意,新干部他们没有这个包袱,他们比较自由。他讲“老并不是你们好呀,你们就是革命几十年,我就是那个时候,你们革命的时候,我还在地上爬呀!”这一条他讲我们不赢,因此他就没有这个包袱。

那些新干部,我们对他们要处在平等的地位。有很多东西,我们不如他们。比如讲知识,我们要向他们学习。现在这一代,只有他们能够在知识上教育工人阶级、教育农民。因为现在我们还有什么知识分子呢?没有嘛!现在的大学,约有百分之八十是地主、资本家、富农的儿女;中学据江苏省统计高中百分之六十是地主、富农、资本家的儿女;初中也有百分之四十是地主、富农、资本家的儿女;只有小学倒过来,大概地主、富农资本家的儿女占百分之二十、三十,工人、农民子弟可能有百分之七十、八十。这种情况,需要很长的时间,可能要一、二十年,才能改变。所以,现在的知识分子,就是资产阶级知识分子。所以我们要耐心争取他们,一面要说服他们,使他们进步,使他们接受马克思主义,就是说教育他们,他们要做教员,必须学习,另一面我们要向他们学习,要向现在的资产阶级知识分子学习,因为,除他们就没有知识分子了。

无产阶级必须有自己的知识分子。我们的国家是无产阶级领导的,只有它有前途,其它阶级都是向它过渡的。比如说农民,将来要变成农业工人,合作社几十年之后,就要变成国营农场,社员要变成农业工人,资本家现在正在变,再过若干年,他们也变成了工人。整个社会都是工人,所以只有工人才有前途,其它都是过渡的阶级。无产阶级要有他的知识分子,要有全心全意为他服务的知识分子,而不是半心半意。大概估计,三个五年计划完成,我们可能从五百万知识分子中间,由现在的百分之十几或百分之十五,或者百分之十七(这里包括一部分进了共产党而不算在里头的),因为进了共产党,他并不一定等于完全接受马克思主义世界观,不是共产党有一部分人倒是接受马克思主义世界观,比如过去鲁迅就是这样的人。鲁迅好?还是陈独秀好?还是张国焘好?还是高岗好?我说还是鲁迅好,一个是共产党,一个是非党人士,就个别人而论,进了共产党,并不等于比党外人士好(有一部分党外人士比共产党员好),争取扩大到三分之一的知识分子,他们或者是进党,或者在党外接受马克思主义世界观的,跟工人农民接近。还有三分之二,在这十几年当中,也会有些改变,由半心半意改变成再多一点,三七开,尾巴掉一点,有些进步。我们要争取这个前途。因为讲思想问题,就联系到知识分子问题。

至于合作化许多问题,我就不讲了,合作化是个好事,是确定它有优越性的。因为我们许多合作社证明这一点,是没有疑义的,但是有一部分同志,还有疑问,党外人士还有异义。这一点要向他们作解释。


