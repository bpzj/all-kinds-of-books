\section[给郭沫若回信中的几句话(一九六二年一月)]{给郭沫若回信中的几句话}
\datesubtitle{(一九六二年一月)}

一九六一年,郭沫若看了《孙悟空三打白骨精》后,写了一首诗\footnote{郭沫若诗:《七律·孙悟空三打白骨精》人妖颠倒是非淆,对敌慈悲对友刁。咒念金箍闻万遍,精逃白骨累三遭。千刀当剐唐僧肉,一拔何亏大圣毛。教育及时堪赞赏,猪犹智慧胜愚曹。},毛主席十一月十七日写了一首七律《和郭沫若同志》\footnote{一从大地起风雷,便有精生白骨堆。僧是愚氓犹可训,妖为鬼蜮必成灾。金猴奋起千钧棒,玉宇澄清万里埃。今日欢呼孙大圣,只缘妖雾又重来。},站得更高,看得更远。以后一九六二年一月六日郭沫若又作了一首和主席的诗,诗曰:

赖有晴空霹雳雷,不教白骨聚成堆。九天四海澄迷雾,八十一番弭大灾。僧受折磨知悔恨,猪期振奋报涓埃。金睛火眼无容赦,哪怕妖精亿度来。

该和诗由康生同志转交主席,主席回信郭沫若说“和诗好,不是‘千刀当剐唐僧肉’。对中间派采取了统一战线政策,这就好了。”
