\section[接见边区部队一部分负责同志的讲话(一九四七年三月十六日)]{接见边区部队一部分负责同志的讲话}
\datesubtitle{(一九四七年三月十六日)}


地点:延安王家坪毛主席会议室。

保卫边区的负责同志首先向毛主席汇报了战士保卫延安,保卫边区的决心那怕只剩下一个人也不能让蒋介石占领延安。

战士们的这种决心很好,我们在延安住了十年,挖了窑洞,吃了小米,学了马克思列宁主义,领导了全国革命。全中国全世界者都知延安,同志们对延安这种深厚感情完全是可以理解的。我们必须用坚决斗争的精神,保卫和发展陕甘宁边区和西北解放区。

现在国民党派了二十多万兵力,进攻陕甘宁边区,他们有美帝国主义的支持,有飞机坦克。而我们在边区的部队只有二万人,我们在武器方面,基本上是小米加步枪。在这种情况下,敌人在数量方面与武器方面,虽然比我们优越,可是我们还要看到边区地形险要,群众条件好,回旋余地大,只要我们有一个正确作战方针,我们一定能战胜敌人。我们的作战方针是诱敌深入牵着敌人的鼻子在山里周旋,把敌人磨得十分疲劳的时候,十分缺乏粮食的时候。然后寻找机会消灭它,这就是磨菇战术。蒋介石占领延安这决不是他们的胜利,而是搬起石头打自己的脚,他们不久就要倒霉的。

我们的作战目的是以消灭敌人有生力量为主,而不是以保守一个地点为主。只要我们能大量消灭敌人的有生力量,我们将来就可以恢复失地,并且能够夺权新的地方。因此说,有人失地,人地则存,存地失人,人地皆失。现在我们主动放弃延安,就是这个道理。

你们回去给战士们讲一讲。一年,最多两年我们还要回延安来,到那时,延安就永远是人民的了。

