\section[“整顿学风、党风、文风”的补充(一九四二年二月一日)]{“整顿学风、党风、文风”的补充}
\datesubtitle{(一九四二年二月一日)}


那几样东西呢?就是一个思想问题,一个党内党外的关系问题,还有一个语言文字问题。在这三个问题上,我们有些同志还有些不大正派的作风没有去掉。

(第八一四页第二行)

如果我们身为中国共产党员,却对于中国问题熟视无睹,天天看,看不见,带了眼镜,还是看不见,看见了只是书架上的马恩列斯的现成文献,那么,我们在理论战线上的成绩就未免太坏了。如果我们只知背诵马克思主义的经济学和哲学,从第一章到第十章都背得烂熟了。(笑声)但是完全不能应用,这样是不是就算得一个马克思主义理论家呢?大概不能算罢,这样的“理论家”实在还是少一点好。假如一个人读了一万本子马恩列斯,每本都又读了一千遍,以致于句句都背得,这还是不能算得理论家的。

(第八一六页第七行)

现在工作只用百分数计算成缉,那么,象读一万本书,每本读了一千遍,但是完全不能应用,这究竟应该算多少分数呢?我说一分也不算。(笑声)

(第八一七页第三行)

他们须知学这种知识并不那么困难,甚至可以说是最容易的。象大师傅煮饭就不容易,要把柴米油盐酱醋等件合起来创造成吃的东西,这是并不容易的事情,弄得好吃更加不容易,西北菜社和我们家的大师傅比较起来,就有很大的区别。火大了要焦,盐多了要发苦,(笑声)煮饭做菜真正是一门艺术。书本上的知识呢?如果只是读死书,那么,只要你认得三五千字,学会了翻字典,手中又有一本什么书,公家又给了你小米吃,你就可以摇头摆脑的读起来。书是不会走路的,也可以随便把它打开或者关起,这是世界上最容易办的事情。这比大师傅煮饭容易得多,比他杀猪更容易,你要捉猪,猪会跑(笑声),杀它,它会叫(笑声),一本书摆在桌上的既不会跑,又不会叫(笑声),随你怎么摆布都可以。世界上那有这样容易办的事呀!

(第八一八页第一一行)

我说:是的,马克思一不会杀猪,二不会耕田。但是他参加了革命运动,他又研究了商品。商品这个东西,几百万人,天天看它,用它,但是熟视无睹。只有马克思,偏偏研究了它,他拿了商品这样看,那样看,不象我们读联共党史这样马虎从事。

(第八一九页第一行)

否则我们为什么要去学马列主义呢?是不是因为我们吃了小米不得消化,因此要念消食经呢?我们党校确定要学马列主义是为了什么呢?这个问题不讲明白,我们党的理论水平永远不会提高。

(第八二二页第一行)

好象道士们到茅山学了法就可降妖捉怪一般。它也没有什么好看,也没有什么神秘,它只是很有用。

(第八二二页第三行)

那怕是马列主义,也可以使他变成空洞的东西。一种是偏于感性与局部的知识,没有发展成为理性的与普遍的东西。

(第八二二页第五行)

说句不客气的话,实在比屎还没有用。我们看,狗屎可以肥田?人屎可以喂狗。教条呢?既不能肥田,又不能喂狗。有什么用处呢?(笑声)同志们,你们会知道,我这样说的目的,就是故意挖苦这些把马列主义看成教条的人使他们大吃一惊,苏醒过来,好拿正确的态度对待马列主义。

(第八二二页第七行)

李立三也对共产国际闹过独立性,结果犯了立三路线的错误。现在讲的,虽然不是张国焘李立三那样极端错误的宗派主义。

(第八二三页第九行)

<p align="right">一九四二年四月二十七日《解放日报》</p>

