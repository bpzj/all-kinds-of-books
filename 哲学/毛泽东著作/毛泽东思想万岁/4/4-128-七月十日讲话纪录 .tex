\section[七月十日讲话纪录 ]{七月十日讲话纪录 }


团结问题。

对形势的认识不一致,就不能团结,要党内团结,首先要把问题搞清楚,要思想统一。有些同志对形势缺乏全面分析,要帮助他们认识,得的是什么,失的是什么。

要把问题搞清楚,有人说总路线根本不对,所谓总路线,无非是多快好省,多快好省根本不会错。

我们把道理讲清楚,把问题摆开,总可以有百分之七十的人在总路线下面。

要承认缺点错误,从一个局部来讲,从一个问题来讲,可能是十个指头,九个指头,七个指头,或者三个指头、两个指头,但是从全局来讲,只是一个指头的问题,从总的形势来讲,就是这样,九个指头和一个指头。

我总是同外国同志说,请他们隔十年时间再来看看我们是否正确。因为路线的正确与否,是实践的问题,要有时间,从实践的结果来证明,我们对建设说应该还没有经验,至少还要十年。这一年来的会议,我们总是把问题加以分析,加以解决,坚持真理,修正错误。党内有些同志不了解整个形势,要向他们说明:从某些具体事实说来,确实有得不偿失的,但总的来说,不能说得不偿失。取得经验总是要付学费的。


