\section[强调团结与进步(一九四〇年二月七日)]{强调团结与进步(一九四〇年二月七日)}


延安新中华报,自改为共产党机关报以来,已一年了。这个小型报依我看,是全国报纸中最好的一个。其主要的原因,一是共产党办的,二是在民主政治下。没有这两个同时具备的条件,要办得这样好,是不可能的,报馆诸同志的努力,亦是一个重要条件,没有他们的积极性、创造性,要办好,亦是不可能的,但“好”是没有止境的.今年是第二年,我希望这个报纸进一步的好起来。

抗战,团结,进步,这是共产党在去年七七纪念时提出的三大方针。这是三位一体的方针,三者不可缺一。如果强调抗战而不强调团结与进步,那末,这样所谓“抗战”是靠不住的,是不能持久的,缺乏团结与进步纲领的抗战,虽然现时还是抗战,但终有一日要把抗战改为投降,或者捷抗日归于失败,我们与产党认为一定要三者合一。为了抗战就要反对投降,反对汪精卫的卖国协定,反对汪的伪政府,反对一切暗藏在抗日阵线中的汉奸与投降派,为了团结就要反对分裂运动,反对内部摩擦,反对从抗日阵线后面进攻八路军,新四军及一切进步势力,反对破坏前线的抗日根据地,反对破坏八路军的后方陕甘宁边区,反对否认共产党的合法地位,反对雪片一样的“限制异党活动”文件。为了进步就要反对倒退,反对把三民主义与抗战建国纲领束之高阁,反对不实行总理遗嘱上唤起民众的指示,反对把进步青年送进集中营,反对把抗战初(期)仅有的一点言论出版自由取消干净,反对把党政运动变为少数人包办的官僚事业,反对在山西进攻新军,消灭牺盟,惨杀进步人员,反对三民主义青年团在咸榆公路陇海铁路一带拦路劫人,反对讨九个小老婆与发一万万国难财的无耻勾当,反对贪官污吏之横行与土豪劣绅之猖獗。不这样做.没有团结与进步,所谓抗战的只是空唤,抗日胜利是没有希望的。新中华报第二年的政治方向是什么?我们了解是强调团结与进步,以反对一切危害抗战的乌烟瘴气,以期抗日事业有进一步的胜利。

于新中华报新刊一周年纪念之时,书此致其希望之忱并望全国同胞共勉之。


<p align="right">(一九四○年二月七日《新中华报》)</p>

