\section{湘江评论}
\subsection[创刊宣言(一九一九年七月十四日)]{创刊宣言}
\datesubtitle{(一九一九年七月十四日)}


自“世界革命”的呼声大倡。“人类解放”的运动猛进。从前吾人所不置疑的问题。所不递取的方法。多说畏缩的说话。于今都要一改旧观。不疑者疑。不取者取。多畏缩者不畏缩了。这种潮流。任是什么力量。不能阻住。任是什么人物。不能不受他的软化。

世界什么问题最大?吃饭问题最大。什么力量最强?民众联合的力量最强。什么不要怕?天不要怕。鬼不要怕。死人不要怕。官僚不要怕。军阀不要怕。资本家不要怕。

自文艺复兴。思想解放。“人类应如何生活”。成了一个绝大的问题。从这个问题。加以研究。我深了“应该那样生活”“不应该这样生活”的结论。一些学者倡之。大多民众和之。就成功或将要成功许多方面的改革。

见于宗教方面为“宗教改革”。结果得了信教自由。见于文学方面。由贵族的文学。古典的文学。死形的文学。变为学民的文学。现代的文学。有生命的文学。见于政治方面。由独裁政治。变为代议政治。由有限制的选举。变为没限制的选举。见于社会方面。由少数阶级专制的黑暗社会。变为全体人民自由发展的光明社会。见于教育方面。为平民教育主义。见于经济方面。为劳获平均主义。见于欲想方面。为实验主义。见于国际方面。为国际同盟。

各种改革。一言蔽之。“由强权得自由”而已。各种对抗强权的根本主义。为平民主义。(兑莫克拉西。——作民本主义。民主主义。庶民主义。)宗教的强权。文学的强权。政治的强权。社会的强权。教育的强权。经济的强权。思想的强权。国际的强权。丝毫没有存在的余地。都要借平民主义的高呼。将他打倒。

如何打倒的方法。则有两说。一急烈的。一温和的。两种方法。我们应有一番选择。

(一)我们承认强权者都是人。都是我们的同类。滥用强权。是他们不自觉的误谬与不幸。

(二)用强权打倒强权。结果仍然得到强权。不但自相矛盾。并且毫无效力。欧洲的“同盟”。“协约”战争。我国的“南”“北”战争。都是这一类。

所以我们的见解。在学术方面。主张彻底研究。不受一切传说和迷信的束缚。要寻着什么真理?在对人的方面。主张群众联合。向强权者为持续的“忠告运动”。实行“呼声革命”——面包的呼声。自由的呼声。平等的呼声。——“无血革命”。不至张起大扰乱。行那没效果的“炸弹革命”“有血革命”。

国际的强权。迫上了我们的眉睫。就是日本。罢课。罢市。罢工。排货。种种运动。就是直接间接对付强权日本有效的方法。

至于湘江。乃地球上东半球东方的一条江。他的水很清。他的流很长。住在这江上和他邻近的民族。浑浑噩噩。世界上事情。很少懂得。他们没有有组织的社会。人人自营散处。只知有最狭的一己。和最短的一时。共同生活。久远观念。多半未曾梦见。他们的政治没有和意和彻底的解决。只知道私争。他们被外界的大潮卷急了。也办了些教育。却无甚效力。一般官僚式教育家。死死盘踞。把学校当监狱。待学生如囚徒。他们的产业没有开发。他们中也有一些有用人材。在各国各地学好了学问和艺术。但没有给他们用武的余地。闭锁一个洞庭湖。将他们轻轻挡住。他们的部落思想又很厉害。实行湖南饭湖南人吃的主义。教育实业界不能多多容纳异材。他们的脑子贫弱而又腐败。有增益改良的必要。没有提倡。他们正在求学的青年。很多。很有为。没人用有效的方法。将种种有益的新知识新艺术启导他们。咳,湘江湘江!你真枉存在于地球上。

时机到了!世界的大潮卷得更急了!洞庭湖的闸门动了。且开了!浩浩荡荡的新思潮业已奔腾澎湃于湘江两岸了!顺他的生。逆他的死。如何承受他?如何传播他?如何研究他?如何施行他?是我们全体湘人最切最要的大问题。即是“湘江”出世最切最要的大任务。

\subsubsection{西方大事述评—各国的罢工风潮}

法英美三国的官阀和财阀,倾注全力于巴黎和会,用高压手段对付败北的德奥,正在兴高釆烈时候,他们的国内,忽然发生了罢工风潮。罢工在他们国里,原是一件常事,政府和财阀,虽然不敢十分轻视劳动者。每当劳动者拿着劳获不均,工时太久,住屋不适,失职无归,种种怨愤不平的问题,联合同类,愤起罢工的时候,也不得不小小给他们一点恩惠。正如小儿哭饿,看着十分伤心,大人也不得不笑着给他一个饼子。但终是杯水车薪,济得甚事。所以广义派人,都笑英法的工人是小见识。从老虎口里讨碎肉,是不能够的。

此回各国的罢工风潮,英国因为在大战初了时候,(去年十二月)英伦加苏格兰各埠交通机关,燃料业,矿山业,造船业等已演了一大罢工。故此次罢工,未发生于英伦本土。法国罢工情形,初颇严重。终亦以小惠收场,没闹出什么好结果。广义派人有乘机在巴黎实行政治运动之说,亦未见诸事实。美国一部分电报电话人员的罢工,乃在附和议员多数派反对加入国际联盟。与英法罢工,异其目的。意大利之罢工乃社会觉嫉恶其政府所为的一种运动。德国自去冬少数社会党大失败,各处大罢工,亦随之而没得好结果。多数社会党掌握政权以来,早已噤若寒蝉,不敢出声。此次为和约签字问题,有激起罢工的形势。但施特满内阁倒了,继任巴安内阁,仍是前内阁的同调。抵御外侮不足,防备家贼有余的武力,紧握在手,谁敢予侮。广义一派的猛断政略,暂时决没有发动的机会。罢工不能成为事实,亦无足怪。匈牙利所受罢工影响不大,其原因则全在缺粮没饭吃。今将一月以来各国罢工情形,分述于下——

法国\quad{}六月三日,罢工风潮发生后,蔓延甚速。巴黎一区,男女工人赋闲者,二十万人。所要求各业不同,而一致主张每日工作八小时。四日蔓延更广,推算当有五十万人罢工。五日,洗衣工人罢工。自后罢工的人数更多。地底铁道,电车,街车的用人决议继续罢工。地底铁道工人要求工值每月至少四百五十佛朗。(以每佛朗当四角合我一百八十元)满五十岁须给养老金。服役十五年后亦须给养老金若干。七日巴黎罢工现象有转机,五金业与机器业,工人与雇主,已商妥数事。十一日,五金业及地底铁道工人上工。当道已取必要万法对付铁道罢工。煤矿工人有全体罢工的形势。十二日,国会通过矿工每日工作八时议案。但矿工会议,仍不满意,决定从十六日,全体罢工。水夫联合会,也决计于十六日罢工。工人联合会,言及生活代价奇昂,(记者按,近有从巴黎回者,举一物贵实例,一个旧牙刷,价二佛朗。一双皮鞋,价六十佛胡。)谓非洲各口岸,堆集麦粮千百吨,任其朽腐。各埠存货如山,轮船火车,宁闲置不远载。这样的政府,可要快快废止他的消耗,欺骗,和垄断!十四日,风潮渐平。极端派有乘机推翻克勒满沙强权政府的运动。路工联合会拒绝之。但矿工因解释政府每日工作八小时议案,未能满意,定十六日全体罢工。恐怕路矿运输联合会工人,也会罢工,表示同情。克勒满沙老头子急了,和运输公司及运输工人代表会商,恳请彼等在国家危机时候,发出些爱国热忱。工人吃了他的浓米汤,已老老实实决议上工了。

英国\quad{}伦敦五月三十日电,全国警察拟于三日罢工的气象,正在酝酿中,政府已允增给薪资,优加待遇。但不承认警察联合会及收用已革除的警察。英国的属地澳洲,坎拿大,苏依士,昔有罢工风潮。六月四日,坎拿大维克斯兵工厂工人罢工,要求每星期工作四十小时。六月五日,苏依士运河工人罢工,局势狠恶,六月九日,澳港航务罢工,势头狠烈,各项工业,都受窒碍。十×日,风潮仍严重,他业工人,因此赋闲的逐日增多。

美国\quad{}六月七日,芝加哥电报司员联合会一时罢工,共约六万人。内有二万五千人系属电报司员联合会。该会会长康能堪氏正计划全国罢工办法。同日,全国电话司员奉命于十六日起罢工,表同情于电报司员。八日,电报人员联合会干事,向全体电报人员宣布,连收发电报生在内,全体罢工。目的在停止威尔逊总统每日在巴黎往来的电报,使他注意国民不赞成他在和会的主张。十二日,各电报公司报告,电报司员罢工没成。

意国\quad{}六月十三日,意大利斯贝齐亚地方,因粮食昂贵,发生暴动,捣毁商店。十四日,热那亚工界示威,被捕者数百人,银行商店闭门,电车不走。杜林工人此日多停工,纪念德国斯巴达团领袖卢森堡氏。米兰工人罢工,抗议热那亚与斯贝齐亚当道的行动。

德国\quad{}六月十三日,大柏林公民会议秘密会,决议罢工。各职业及军界中人,均赞助停止各项实业工作的计划。有人料此举将促成国内战争。中等社会将得政权。

匈国\quad{}五月三十一日,匈京饥饿的工人,发生暴动。红旗军奉共产政府命令到各工厂制乱。匈京几无粮食。

\subsubsection{东方人事述评—陈独秀之被捕及营救}

前北京大学文科学长陈独秀,于六月十一日,在北京新世界被捕。被捕的原因,据警厅方而的布告,系因这日晚上,有人在新世界散布市民宣言的传单,被密探拘去。到警厅诘问,方知是陈氏。今录中美通讯社所述什么北京市民宣言的传单于下——

一、取消欧战期内一切中日秘约。

二、免除徐树铮曹汝霖章宗群陆宗舆段芝贵王怀庆职。并即驱逐出京。

三、取消步军统领衙门,及警备总司令。

四、北京保安队,由商民组织。

五、促进南北和议。

六、人民有绝对的言论出版集会和自由权。

以上六条,乃人民对于政府最低之要求,乃希望以和平方法达此目的。倘政府不俯顺民意,则北京市民,惟有直接行动,图根本之改造。

上文是北京市民宣言传单,我们看了,也没有什么大不了处。政府将陈氏捉了,各报所载,很受虐待。北京学生全体有一个公函呈到警厅。请求释放。下面是公函的原文――

警察总监钧鉴,敬启者,近闻军警逮捕北京大学前文科学长陈独秀,拟加重究,学生等期期以为不可。特举出两要点于下:(一)陈先生夙负学界重望,其言论思想,皆见称于国内外。倘此次以嫌疑遽加之罪,恐激动全国学界再起波澜。当此学潮紧急之时,殊非息事宁人之计。(二)陈先生向以提倡新文学现代思想见异于一般守旧者。此次忽被逮捕,诚恐国内外人士,疑军警当局,有意罗织,以为摧残近代思想之步。现今各种问题,已极复杂,岂可再生枝节,以滋纠纷?基此二种理由,学生等特请贵厅,将陈独秀早予保释。

北京学生又有致上海各报各学校各界一电——

陈独秀氏为提倡近代思想最力之人,实学界重镇。忽于真日被逮。住宅亦披抄查。群清无任惶骇。除设法援救外,并希国人注意。

上海工业学会也有请求释放陈氏的电。有“以北京学潮,迁怒陈氏一人,大乱之机,将从此开始”的话。政府尚未昏聩到全不知外间大事,可料不久就会放出。若说硬要兴一文字狱,与举世披靡的近代思潮,拚一死战,吾恐政府也没有这么大胆子。章行严与陈君为多年旧交,陈在大学任文科学长时,章亦在大学任图书馆长及研究所逻辑教授。于陈君被捕,即有一电给京里的王克敏,要他转达别厅,立予释放。大要说——

……陈君向以讲学为务,平生不含政治党派的臭味。此次虽因文字失当,亦何至遽兴大狱,视若囚犯,至断绝家常往来。且值学潮甫息之秋,訑可忽兴文绵,重激众怒。甚为诸公所不取。……

章氏又致代总理龚心湛一函。说得更加激切——

仙舟先生执事,久违矩教,结念为劳。兹有恳者,前北京大学文利学长陈独秀,闻因牵涉传单之嫌,致被逮捕,迄今末释。其事实如何,远道未能详悉。惟念陈君平日,专以讲学为务。虽其提倡新思潮,著书立论,或不无过甚之词,然范围实仅及于文字方面,决不舍有政治臭味,则固皎然可征。方今国家多事,且值学潮甫息之后,讵可蹈腹诽之殊,师监谤之策,而愈激动人之心理耶。窃为诸公所不取。故就历史论,执政因文字小故而专与文人为难,致兴文字之狱。幸而胜之,是为不武。不胜人心瓦解,政纽摧崩,虽有善者。莫之能挽。试观古今中外,每当文网最甚之秋,正其国运衰歇之候。以明末为殷鉴,可为寒心。今日谣琢萦兴,清流危惧。乃遽有此罪及文人之举,是露国家不祥之象,天下大乱之基也。杜渐防微,用敢望诸当事。且陈君英姿挺秀,学贯中西。皖省地绾南北,每产材武之士,如斯学者,诚叹难能。执事平视同乡诸贤,谅有同感。远而一国,近而一省,育一人才,至为不易。又焉忍遽而残之耶。特专函奉达,请即饬警厅速将陈君释放。钊与陈君总角旧交,同衿大学。于其人品行谊,知之甚深,敢保无他,愿为左证。……

\begin{flushright}章士钊拜启六月二十二日\end{flushright}

我们对于陈君,认他为思想界的明星。陈君所说的话,头脑稍微清楚的听得,莫不人人各如其意中所欲出。现在的中国,可谓危险极了。不是兵力不强财用不足的危险,也不是内乱相寻四分五裂的危险。危险在全国人民思想界空虚腐败到十二分。中国的四万万人,差不多有三万万九千万是迷信家。迷信鬼神,迷信物象,迷信运命,迷信强权。全然不认有个人,不认有自己,不认有真理。这是科学思想不发达的结果。中国名为共和,实则专制。愈弄愈糟,甲仆乙代,这是群众心里没有民主的影子,不晓得民主究竟是甚么的结果。陈君平日所标揭的,就是这两样。他曾说,我们所以得罪于社会,无非是为着“赛因斯”(科学)和“兑莫克拉西”(民主)。陈君为这两件东西得罪于社会,社会居然就把逮捕和禁锢报给他,也可算是罪罚相敌了,凡思想是没有畛域的,去年十二月德国的广义派社会党首领卢森堡被民主派政府杀了,上月中旬,德国仇敌的意大利一个都林地方的人员,举行了一个大示威以纪念他。瑞士的苏里克,也有个同样的示威给他做纪念。仇敌尚且如此,况在非仇敌。异国尚且如此,况在本国。陈君之被逮,决不能损及陈君的毫末。并且是留着大大的一个纪念于新思潮,使他越发光辉远大。政府决没有胆子将陈君处死,就是死了,也不能损及陈君至坚至高精神的毫末。陈君原自说过,出实验室,即入监狱。出监狱,即入实验室。又说,死是不怕的。陈君可以夺验其言了。我祝陈君万岁!我祝陈君至坚至高的精神万岁!

\subsubsection{世界杂评}

强叫化前月的初间,日本米价顶贵时候,每石超四十元。日当局有狼狈之状。报纸证言粮食的危机已迫。可怜的日本!你肠将饥断,还要向施主逞强。天下那有强叫化续得多施的理。

研究过激党阿富汗侵印度,俄过激党为之主谋,过激党到了南亚洲。高丽的“呼声革命”正盛吋,亦有过激党参与之说,则已到了东亚。过激党这么厉害!各位也要研究研究,到底是个什么东西?切不可闭着眼睛,只管瞎说,“等于洪水猛兽”“抵制”“拒绝”等等的空话。一光眼,过激觉布备了全国,相惊而走,已没得走处了!

实行封锁前月巴黎高等经济会议议决,实行封锁匈牙利,说理直到匈政府宣言遵从民意时为止。这要分两层观察,一、协约国看错了匈政府与匈国民志愿不合。匈政府与匈国民之少数有产阶级,绅士阶级,志愿不合是有的,若与大多数无产阶级,平民阶级,没有志愿不合的理。因为匈政府,原是他们所组织的。二、实行封锁,这是帮助过激主义的传播。吾恐怕协约国也会要卷入这个漩涡。果然,则这实行封锁,真是“罪莫大焉”了。

证明协约国的平等正义德国复文和会,要求德国陆军减少之后,协约国也须同减。这话谁人敢说错了?协约国满嘴的平等主义,我们且看协约国以后的军备如何?就可求个证明。

阿富汗执戈而起一个很小的阿富汗,同一个很大的海上王英国开战,其中必有重大原因。但据英国一方面的电传,是靠不住的。土耳其要被一些虎狼分吞了。印度舍死助英,赚得一个红巾照烂给人出丑的议和代表。印民的要求是没得允许。印民的政治运动,是要平兵力平压。阿富汗是个回教国,狐死冤悲,那得不执戈而起?

来因共和国是丑国协约国要划来因流域为自己挡敌的长城,必先使之脱离德国的关系,别成一国。听说已在威萨登成立临时政府,一位这登博士做总统。这位道登阵士不知果然高兴到甚么样?金人立了刘豫,契丹立了石敬塘,我们中国也曾有几个这样的国呢。

好个民族自决波兰捷克复国,都所以制德国的死命,协约因尽力援助之,称之为“民族自决”。亚刺伯有分裂士耳其的好处,故许他半自立。犹太欲在巴力斯坦复国,因为于协约国没大关系,故不能成功。西伯利亚政府有攻击过激党的功绩,故加以正式承认。日本欲伸足西伯利亚,不得不有所示好,故首先提议承认。朝鲜呼号独立,死了多少人民,乱了多少地方,和会只是不理。好个民族自决!我们认为直是不要脸!

可怜的威尔逊威尔逊在巴黎,好象热锅上的蚂蚁,不知怎样才好?四围包满了克勒满沙,路易乔治,牧野伸显,欧兰杜一类的强盗。所听的,不外得到若干土地,收赔若干金钱。所做的,不外不能伸出己见的种种会议。有一天的路透电说:“威尔逊总统卒已赞成克勒满沙不使德国加入国际同盟的意见”。我看了“卒已赞成”四字,为他气闷了大半天。可怜的威尔逊!

炸弹暴举人人知道很文明很富足的美国,有“炸弹暴举”同时在八城发生。无政府党蔓延甚广。炸弹爆炸的附近,有匿名揭贴说,“阶级战争”业已发生,必得国际劳动界完全胜利,始能停止,炸弹往往埋藏在一些官员的住宅,屋顶上发现人头。可怕可怕!我只挂牵官员人家的一些小姐小孩子,他们晚上如何睡得着?议院里一些钱多因而票多票多因而当选的议员,还在那里痛诋暴动者,通过严惩案。我正式告诉诸位,诸位的“末日审判”将要到了!诸位要想留着生命,并想相当的吃一点饭,穿一点衣,除非大大的将脑子洗洗,将高帽子除下,将大礼服收起,和你们国里的平民,一同进工厂做工,到乡下种田。

不许实业专制美国工党首领戈泊斯演说曰,“工党决计于善后事业中有发言权,不许实业专制。”美国为地球上第一实业专制国,托辣斯的恶制,即起于此。几个人享福,千万人要哭。实业越发达,要哭的人越多。戈泊斯的“不许”,办法怎样?还不知道。但既有人倡言“不许”,即是好现象。由一人口说“不许”,推而至于千万人都说“不许”,由低声的“不许”,推而至于高声的很高声的狂呼的“不许”,这才是人类真得解放的一日。

割地赔偿不两全德国答复协约国,说,如失去西里细亚及萨尔煤矿,则无力行赔偿。我料协约国听了一定很烦脑。何以故?地可割,赔偿也可得,最为两全。据德国的说,两样便成了反比例,如之何不烦脑?虽然,奉劝协约国的衮衮诸公,天下那有两全的好事!

为社会党造成流血之地奥总代表任纳博士答复和会,说,“奥国今已坐食其较前大减之资本,若再加以摧残,必为社会党造成流血之地”,蠢哉任纳博士,你还不知道协约国一年以来之真目的,你专为造成社会党流血之地吗?

彭斯坦德博士彭斯坦演说曰,“媾和条件”乃野蛮战争的结果,德国最宜负责,和约条件十九为必要的”。我们固然反对协约国的强迫和约,但博士这话,系专对野蛮战争而发,听了倒很爽快。

各国没有明伦堂康有为因为广州修马路,要拆毁明伦堂,发了肝火。打电给岭伍,斥为“侮圣灭伦。”说,“遍游各国,未之前闻”。康先生的话真不错,遍游各国,那里寻得出什么孔子,更寻不出什么明伦堂。

什么是民国所宜?康先生又说,“强要拆毁,非民国所宜”。这才是怪!难道定要留看那“君为臣纲”,“君君臣臣”的事,才算是“民国所宜”吗?

大略不是人邓镕在新国会云,“尊孔不必设专官,节省经费”。张元奇云,“内务部祀孔,由茶房录事办理。次长司长不理,要设专官。”内务部的茶房录事,大略不是人。要说是人,怎么连祀孔都不行呢?我想孔老爹的官气到了这么久的年载,谅也减少了一点。

走昆仑山到欧洲张元奇又说,“什么讲求新学,顺应潮流,本席以为应尊孔逆挽潮流。”不错不错!张先生果然有此力量,那么,扬子江里的潮流。会从昆仑山翻过去,我们到欧洲的,就坐船走昆仑山罢。

\subsubsection{湘江杂评}

好计策一个学校的同学对我说,我们学校里办事人和教习,怕我们学到了他们还未学到的新学说,将图书室看×了。外面送来的杂志新闻纸和书籍,凡是稍新一点的,都没得见。我听了为之点首叹服。他们的计策真妙!岂仅某学校,通湖南的学校,千篇一律都象联了盟似的。

摇身一变一些官僚式教育家,为世界的大潮卷急了,不提防就会将他们的饭碗冲破。摇身一变,把前日的烂调官腔,轻轻收拾。一些其有所感而改变的,很可佩服。一些则是假变,容易露出他们的马脚。这类人我很为他羞!很为他危!

我们饿极了我们关在洞庭湖大门里的青年,实在是饿极了!我们的肚子固然是饿,我们的脑筋尤饿!替我们办理食物的厨师们,太没本钱。我们无法!我们惟有起而自办!这是我们饿极了哀声!千不要看错!

难道走路是男子专有的一个女学校里的办事人,把学生看做文契似的收藏起来,怕他们出外见识了甚么邪样。新青年一类的邪书,尤不准他们寓目。此次惊天动地的学生潮,北京的女学生聚诉新华门。贫儿院的小女孩子,愿到监狱替男学生抵罪。这个女学校的学生独深闭固拒,一步也不出外,好象走路是男子专有似的。

哈哈!青岛问题发生,湖南学生大激动,新剧演说,一时风行。有一位朋友对我说,一位老先生,因为他的儿子化装演剧,气得了不得。走到学校问先生,开口便说,“我的命运如何这么乖?养大的儿子竟做出那么下流事”?我听了这话,忍不住卧的一声,哈哈!

女子革命军或问女子的头和男子的头,实在是一样。女子的腰和男子的腰实在是一样。为什么女子头上偏要高竖那招摇畏风的髻?女子腰间偏要紧缚娜拖泥带水的裙?我道,女子本来是罪人,高髻长裙,是男子加于他们的刑具。还有那脸上的脂粉,就是黔文。手上的饰物,就是桎梏。穿耳包脚为肉刑。学校家庭为牢狱,痛之不敢声,闭之不敢出。或问如何脱离这弊?我道,惟有起女子革命军。

\subsection[第二号(一九一九年七月二十一)]{第二号}
\datesubtitle{(一九一九年七月二十一)}

\subsubsection{西方人事述评}

德意志人沉痛的签约

签约之前:败而不屈的德意志的代表兰超等于(五月初旬)到巴黎。(五月七日)在凡尔赛宫举行很庄严的和约交付礼。德代表的态度很倨傲。克勒满沙站起声述其开会词。德总代表兰超,则坐诵其如下之演说词——

德国军事破裂,德国失败的程度,自己明白。但这回欧战,负责的不仅德国,全欧都与有罪。因五十年以来,欧洲各国的帝国主义,实贻毒于国际局势。德国战中的罪行,固不可讳,战事的时候,人民的天良,为感情所蔽,故有罪行。然自去年十一月十一日以后,德国没与战事的人,多死于封锁的影响,协约国亦冷淡视之。威总统十四条大纲,为全世界所赞助,协约国业已声明依照此项大纲而立和约,那么,德国当不致于全没救护。国际同盟,各国都让加入,不能将德国丢在外边。德国愿以好意的精神,研究和约。……

和约为灰黄色封面一大册,和会秘书长杜斯玛,捧了交到兰超手中。兰超回到寓舍,晚餐的时候,默无一言。晚餐毕,即使人翻译和约,于晨间三点译成,送到兰超寝室。兰超看到天明方毕。另外录出几份,派专差送到柏林。八日德内阁会议许见。

内阁总理施特满,向考虑协约委员会演说——

和约条件,简直是宣告德国死刑。政府必以政治的沉静态度,讨论这可厌而狂妄的公文,……

随将和约条件,电告各联邦政府,请他们表示意见。因为感受很深的痛苦,特命公众停止行乐一星期。仅许剧院演唱和这痛苦极没相同的悲刷。股票交易所,因感受痛苦的印象,停闭三日。各界人士听得和约会要签字,皆为作怒,群相讨论拒绝签字的后患,甚至没有一人想劫或可受纳此项条件的。柏林各报一致(言尔)诋,有的说,“和约的苛刻远过最消极的预料,这系狂暴无知识的制品,若不能修改,只有用‘否’字答他。”有的说“我们如签定这约,实是屈于武力。我们的心中,应坚决拒绝。”惟独立社会党的机关报,则主张签约说,“从经验看来,拒绝徒增后患。”这时候最可注意的,是德国政党的态度。多数时会党的政府派,是不主张签字的。民治党和中央党也是这样。只有独立社会党不然。(十二日)独立社会党通过决议案,主张接受和约,并说“德现政府恢复显武主义的行为,使别人坚其对德的疑惧。德国舍屈于强迫签字,没有办法。俄德和约,及德罗和约,均没多久的寿命就取消了。凡尔赛和约,也未尝不可以革命的发展取消他。”我们为德国计,要想不受和约,惟有步俄国和匈牙利的后尘,实行社会的大革命。协约国最怕的就在这一点。俄罗斯,匈牙利,不派代表,不提和议,明目张胆地对抗协约国,协约国至今未如之何。向使去冬德国广义派社会党的社会革命成了功,则东联俄而南结奥,更联合匈牙利和捷克,广播其世界革命主义,或竟使英法美久郁的社会党,起而响应,协约园国府还食得下咽吗?独立社会党和广义派社会党,本是一党而分为二,他的议论如此,本不足怪。用革命的发展取消和约这话正不要轻看呢。

同日德因会讨论媾和条件。施特满演说――

今日为德国人生死关头!我们必须团结一致!我们除谋使国家生存,无旁的责任!德国不图进行其国家主义的梦想,并没争权的问题。于今人人的喉咙中间都觉有手塞住他的呼吸!人类的尊严,现付于诸君的手中,以保持他!

(十四日)兰超致克勒满沙一个履文,内容的大要如下——

和约中关于领土的条款,是使德国失去其×关重要的生产土地。各地薯芋的收成,将减百分之二十一。煤减三分之一。铁减四分之三。锌减五分之三。德国既因失去殖民地和商船,使经济成了麻木不仁。今又不能得充分的原料,势将被毁到极大的程度。同时输入的粮食必将大减。依赖航务和商业为生的数百万人,德国政府不能将工事和粮食供给他们则势不得不到国外求生,而重要的国家。多禁止德国移民入境。故签定和约,不啻向数百万德人宣告死刑……

兰超于上述的牒文之外,更以牒文两迈致克勒满沙。第一通的大要说,“协约国占德土地,和威总统宣布的主义不合。”第二通的乃关于赔偿条款提出抗议。谓“德国愿赔偿,但不是因为负了战争的原故。”我们看德国的抗议,大家注重(一)不独负战争责任。(二)不愿失去原料所从出的土地。其他各项,虽有抗议,但不是最重要的处所。

(十三日)晚上,柏林有大举示威。多数社会于示威时,起坛演说谓,“和约条件,较罗马施于加萨臣的,尤为刻毒而可羞。”群众游行各街,止于协约委员团所住的亚特伦旅馆前面。有人向众演说,其势汹汹,欲攻旅馆,为警察所阻。到内阁总理屋前,施特满氏临窗演说。又有人民一大队,于薄暮时候,唱歌到亚特伦旅馆,大呼“推翻强暴的和局。”“克勒满沙皆亡”。“与英伦皆亡。”众又到施特满处,请他演说,施氏讲到威尔逊总统十四公纲时,众忽大呼“与威尔逊皆亡。”这日柏林和乡间独立社会党开会,有四十处。

(十九日)柏林某报载有社会党领袖彭斯坦博士的演说,谓“非常苛刻的和约条件,非完全出于激怒与仇念。实德国政策既不能见信于人,则当然受此待遇。一切破坏咎在德国。德国之履行各项要求,不过补偿他们前所寄予人的而已。我很不以一般人士所发激烈的演说为然。告诉他们!不可再具一九一四年八月四日的气焰!”这于德国热烈的反对声中,算是一瓢凉水。

(二十日)兰超致书和会,要求改正审查及讨论和约的期限,(二十二日)克勒满沙复书,允许展长期限,至五月二十九日为止。(二十三日)晚德全权大使起程往斯巴,将和来自柏林的阁员数人会晤,决定一切。(二十四日)斯希特芒,欧士白格,从柏林乘车到斯巴,兰超及委员十六人也到,即开一极长的会议,斯希特芒主席,通过德国的反提案。会毕,政府委员回柏林,兰超等回凡尔赛。

(五月二十七日)。德国有答案交付和会,答案的第一部分,要点如下一一

(一)德国承认减少军队到十万人。

(二)交出巨大的军舰,而保留商船。

(三)反对关于东边土地的决定。要求于东普鲁士区中,举行庶民大会。

(四)承认丹齐为自由口岸。

(五)要求协约国,在签字四个月后,撤退军队。

(六)要求加入国际同盟。

(七)坚欲取得代替殖民地的权利。

(八)赔偿总数,不得过十万兆马克。

(九)拒绝引渡凯撒及其他人物。

(十)德国须有重新经商海外的权利。

德国答案的第二部分,亦有如下的要点――

(一)过渡时代,须维持大军,以保治安。

(二)须许德人开公民大会,讨论土地割让问题。并许奥人以加入德国的便利。


(三)拒绝割让西里细亚上部。

(四)不承认俄国有享取赔偿的权利。

(五)无赔偿意、门、罗、波兰等国的义务。

右之德国答案,四大国代表为长久讨论后,提出答复文,将德国所约议的,逐件驳复。全文很长。不外说德国战争责任万难推诿,德国必须尽其能力赔偿损失,必须交出戎首,和战时行暴的人,用法惩治。必须于数年内受特别的约束。凡协约国所持以构成和约的根本主义,万难更易。惟对于德国的实际建议,可以让步云云。

(五月二十九日以后)又因德代表的请求,屡次展缓签约日期。最后允展至六月二十八日为止。六月前半月的光阴,全为着往复反议事占去。

至(六月十八日)德代表团,乃由法京回德,一致告诫德内阁,拒绝签约。德内阁乃准备在韦马召集国会,议此次重大问题。此时协约方面,早做军事准备。一俟德国有不签字的表示,即行进军。德国已处于不能不签字的情势了。

(六月二十二日),协约国对德的“最后复文”于这日送达德代表,限德国以五日承受和约。“最后复文”内,述可以让步的条件,如下一一

(一)西里细亚上部,实行民众投票。

(二)西普鲁士边界,重行划定。

(三)德军暂增至二十万人。

(四)德国宣布愿于一个月内将被控破坏战时法律的人名单开出。

(五)修改关于财政问题的细则。

(六)以德国履行义务为条件,保证德国为将来国际同盟的一员。

施特满内阁知和约不能再有挽回,遂决计引退。

(二十二日)德新内阁组织成立。国务总理巴安氏,外交穆勒氏,财政欧士白格氏,内务达维特氏,陆军拿斯奇氏,殖民贝尔氏,邮电格莱斯勃资氏,劳动森士南氏,工程斯利奇氏,公业经济惠塞尔氏,国库夏勒氏,粕食斯密氏。巴安氏及诸阁员,多属多数社会党,本系前内阁的同调,在这回外交紧急声中,出当此签定和约的难局。新内阁既成,已可决其是预备签约的了。施特满退职。施特满内阁所委任的媾和代表,当然随着退职,于是德国讲和代表团易人。新代表团即以新内阁中的外交总长穆勒,邮电总长格莱斯勒资等组织而成。

此时国会业已在韦玛召集,巴安氏即赴国会,作很沉痛的演说,极言加入新政府的痛苦。恳请国会确立主张,否则战事将屡发作。巴安氏曰,“我特于自由的日耳曼最后一次,起抗此强暴破坏的和约!起抗此自决权利的假面具!起抗此奴隶德人的手段!起抗此妨害世界和平的新器!”国会乃于“反对”“赞成”的喧哗声中通过签约动议。

签约的动议通过,二十三日巴安再赴国会,申述无条件签约的必要,其演词谓,“战败的国家,身魂受世界的凌辱!吾人姑且签定和约。吾人一息尚存,终望损害吾人荣誉的人,有一日身受报应,”这时候右党提出抗议。乃付表决。结果证实准许签约。议长贵里巴贽氏起立发短节的演说,“以不幸的祖国,委托于慈悲的上帝!”且谓“各党领导,允宣告军界,全国希望海陆军树先己牺牲的模范,辅助劳工,重造祖国!”关系全世界安危的德国签字,在一场非常惨痛的演说声中,完全决定。德意志人的大纪念,有史以来,当没有过于这日了:

签字案既经国会通过,德新代表团乃到巴黎,致“允可签约的牒文”于和会。牒文的大要说,一一日耳曼民国政府,知协约国决计以武力强迫日耳曼承受和约条件。此项条件,虽没有重大的意味,然实老在剥夺日耳曼人民的荣誉。日耳曼政府虽屈服于作势的武力,但关于从古未闻离背公道的和约条件。”

右文既布,各国的欢忭,自不可言。至(二十八日)而最后展缓的满期已到。于是凡尔赛宫中,仍有亘古未闻的大签约一举。

签约之际一千九百十九年六月二十八日午后三点五分,凡尔赛宫中开会。在宫中设高坛,甚为庄严。协约国全权代表首先会集。次为德国全权代表,只到外总长穆劝和交通总长裴尔,其余均不愿到。克勒满沙主席,首发短简宣言,谓:“协约国和共同作战国政府,均赞成媾和条件。今加签字,表示彼等忠诚依守庄严的了解。”继乃“请日耳曼民国代表首先签字。”德代表所坐席次忽发大声,“德意志!”“德意志!”克勒满沙于是乃改称“德意志”。德代表即起立在约上签名,裴尔氏首先签之。时为午后三时十二分,园中喷泉四射,炮声大作,当德代表回到坐处的时候,会场皆露喜容。次为美国签字。次为英国签字。次为法国签字。次为意国签字。次为日本签字。最后签字的为捷克斯拉夫民国。三时三十五分签字完毕。克勒满沙氏宣布散会。

签约之后,当德国国会允许签约的消息传布,德国全国自即有爱国的示威运动。群众列队唱战歌,国歌,欢呼致敬于年老的统兵员,各报对于裁判德皇问题,表示极大的忿怒。有一报恳请一九一四年的陆军军官,表示如德皇受裁判,也愿受协约国的裁判。并请组织团体,或须入荷兰,保护德皇。各地暴动罢工事情,接续而起。及(二十八日)和约签字的消息传到柏林,柏林某报即载出一文,谓“德人终必报一九一九年的耻辱!”为政府禁止发行。(二十九日)各报皆有“黑线”,表示哀痛。各报皆载有极悲观的评论。柏林及各地铁路工人及电车工人罢工。柏林城里的运输机关全停。亨堡等处出了乱子。全国的罢工,有扩张形势。

评论我叙签约。我争叙感国的签约。我叙德国签约,单注重其国民精神上所感痛苦的一点。是什么意思?原来这回和约,除却国际同盟,全是对付德国的。德国为日耳曼民族,在历史上早蜚声誉,有一种崛强的。一朝决裂,新剑发硎,几乎要使全地球的人类挡他不住。我们莫将德国穷兵黩武,看作是德皇一个人的发动。德皇乃德国民族的结晶。有德国民族,乃有德皇。德国民族,晚近为尼釆,菲希特,颉德,泡尔生等“向上的”“活动的”哲学所陶铸。声宏实大,待机而发。至于今日,他们还说是没有打败,“非战之弊”。德国的民族,为世界最富于“高”的精神的民族。惟“高”的精神,最能排倒一切困苦,而惟我实现其所谓“高”。我们对于德皇,一面恨他的穷兵黩武,滥用强权。一而仍不免要向他洒一掬同情的热泪,就是为着他“高”的精神的感动。德国的民族,他们败了就止了。象这样的屈辱条件,他们也忍苦承受。他们第一次翻转面目,已从帝国变成了民国。他们的第二次翻转,或竟将民国都不要了。这话我殊敢下一个粗疏的断定。我们且看挡在西方的英法,不是他们的仇敌吗?英法是他们的仇敌,他们的好友,不就是屏障东方和南方的俄、奥、匈、捷和波兰吗?他们不向俄奥匈捷等国连络,还向何处?他们要向俄奥匈捷连络,必要改从和俄奥匈捷相同的制度。俄匈的社会革命成了功,不用说。奥捷也有此趋势,前日电传说捷克已经成了劳兵民国了。德国广义派斯巴达团,去年冬天的猛断举动,和成功仅仅相差一间。爱倍尔政府成立,多数社会党握权,所恃以制服广义派的,全是几个兵,几杆枪。和约成功,兵是要解散了。枪是会要缴出了。那时候政府还恃着什么?德国工商业的大毁败,要重造起来,不得不仰赖出力的劳动者。以后政府所应做的大事,就是向劳动者多多的磕头。而广义派的武器,不是别的,就是这些劳动者。故我从外交方面的趋势去考虑,断定德国必和俄奥匈连合,而变为共产主义的共和国。又从内治方面的趋势去考虑,也可以做同样的断定。

一千九百一十九年以前,世界最高的强权在德国。一千九百一十九年以后,世界最高的强权在法国,英国和美国。德国的强权,为政治的强权,国际的强权。这回大战的结果,是用协约国政治和国际的强权,打倒德奥政治和国际的强权。一千九百一十九年以后,×国英国美国的强权,为社会的强权,经济的强权。一千九百一十九年以后设有战争,就是阶级战争,阶级战争的结果,就是东欧诸国主义的成功。即是社会党人的成功。我们不要轻看了以后的德人。我们不要重看了现在和会高视阔步的伟人先生们,他们不能旰食的日子快要到哩!他们总有一天会要头痛!然则这回的和约“其能稳”尚靠不定。若真以“德俄和约”,“德罗和约”的例来推测,恐咱就是早晚的问题。无知的克勒满沙老头子,还抱着那灰黄色的厚册,以为签了字在上面,就可当作阿尔卑斯山一样的稳固,可怜的很啊!

\subsubsection{世界杂评}

高兴和沉痛克勒满沙在办公室接得德国接受和约的电话,高兴了不得。起身来,和在办公室的阁员及同僚握手。说,“诸君!我之静候这一分钟,已有十九年了!”这话何等高兴。虽然,不第高兴,又含多少沉痛的意思。一千八百七十一年,威廉第一和俾士马克,高踞凡尔赛,接受法国屈服牒文的时候,何等高兴。结果遂酿成此次的战争。虽然威廉第一,俾士马克,不第高兴,又含有多少沉痛的意思。一千八百年至一千八百一十五年,拿破伦蹂躏德意志,分裂他的国,占据他的地,解散他的兵。普王屈服,称藩纳聘。拿破伦何等高兴。结果遂酿成一千八百七十一年的战争。虽然,拿破伦不第高兴,又含多少沉痛的意思。一千七百八十九年至一干七百九十年,德奥为巨擘的神圣同盟军,深恶德国的民主自由,几度蹂法境,围巴黎,结果遂崛起拿破伦,而有蹂躏德国,令德人头痛的事。我们执因果看历史,高兴和沉痛,常相关系,不可分开。一方的高兴到了极点,热一方的沉痛也必到极点。我们看这番和约所载,和拿破伦对待德同的办法,有什么不同?分裂德国的国,占据德国的地,解散德国的兵,有什么不同?克勒满沙高兴之极,即德国人沉痛之极。包管十年二十年后,你们法国人,又有一番大大的头痛,愿你们记取此言。

卡尔和溥仪奥前皇卡尔避居瑞士,某报通讯记者求见,见其侍者。侍眶说:“皇帝的退位,本非得已,故愿望恢复帝制。惟目下暂时隐居,不问政治。”凡做过皇帝的,没有不再想做皇帝。凡做过官的,没有不再想做官。心理上观念的习惯性,本来如此。西洋人做事,喜欢彻底,历史上处死国王的事颇多。英人之处死沙尔一世(一千六百四十八年),法人之处死路易十六(一千七百九十三年),俄人之处死尼哥拉斯第二(一千九百一十八年),都以为不这样不足以绝祸根。拿破伦被囚于圣赫利拿,今威廉第二拟请他做拿破伦的后身将受协约国的裁判,总算很便宜的。避居瑞士的卡尔,和伏处北京的溥仪,国民不加意防备,早晚还是一个祸根。

\subsection[增刊第一号(一九一九年七月二十一日)]{增刊第一号}
\datesubtitle{(一九一九年七月二十一日)}
\subsubsection{健学会之成立及进行}

健学会以前的湖南思想界湖南的思想界,二十年以来,黯淡已极。二十年前,谭嗣同等在湖南倡南学会,召集梁启超,麦梦华诸名流,在长沙设时务学堂,发刊湘报,“时务报”。一时风起云涌,颇有登高一呼之概。原其所以,则彼时因几千年的大帝国,屡受打击于列强,怨幅惋悔,敬而奋发。知道徒然长城渤海,挡不住别人的铁骑和无畏兵船。中国的老法,实在有些不够用。变法自强的呼声,一时透彻衡云,云梦的大倡。中国时机的转变,在那时候为一个大枢纽。湖南也跟着转变,在那时候为一个大枢纽。

思想变了,那时候的思想是怎样一种思想?那时候思想的中心是在怎样的一点?此问不可不先答于下——

(一)那时候的思想是自大的思想。什么“讲求西学’,什么“虚心考察”,都不外“学他到手还以奉敬”的方法。人人心目中都存想十年二十年后,便可学到外国的新法。学到新法便可自强。一达到自强的目的,便可和洋鬼子背城借一,或竟打他个片甲不回。正如一个小孩受了隔壁小孩的晦气,夜里偷着取出他的棍棒,打算明早跑出大门,老实的还他一个小礼。什么“西学”“新法”相当于小孩的棍棒罢了。

(二)那时候的思想,是空虚的思想,我们试一取看那时候鼓吹变法的出版物,便可晓得一味的“耗矣哀哉”。激刺他人感情作用。内酌是空空洞洞,很少踏着人生社会的实际说话。那时有一种“办学室”,“办自治”,“请开议会”的风气,寻其根柢,多半凑热闹而已。凑热闹成了风,从思想界便不容易引入实际去研究实事和真理了。

(三)那时候的思想,是一种“中学为体,西学为用”的思想,“中国是一个声名文物之邦,中国的礼教甲于万国,西洋只有格致炮厉害,学来这一点便得。”设若议论稍不如此,便被人看作“心醉欧风者泳”,要受一世人的唾骂了。

(四)那时候的思想是以孔子为中心的思想。那时候于政治上有排满的运动,有要求代议政治的运动。于学卫上有废除科举兴办学校,采取科学的运动。却于孔老爹,仍不敢说出半个“非”字。甚至盛倡其“学问要新,道德要旧”的谬说,“道德要旧”就是“道德要从孔子’的变语。

上面所举,全中国都有此就行,湖南在此情形的中间占一位置。所以思想虽然变化,却非透彻的变化,任何说是笼统的变化,盲目的变化,过渡的变化,从戊戌以致今日湖南的思想界,全为这笼统的,盲目的,过渡的变化所支配。

湖南辨求新学二十余年,而后有崭然的学风。湖南的旧学界,宋学、汉学两支流,二十年前,颇能成为风气。二十年来,风气尚未尽歇,不过书院为学校占去,学生为科学吸去,他们便必淹没在社会的底面了。推原新学之所以没有风气,全在新学不曾有确立的中心思想。中心思想之所以不曾确立,则有以下的数个原因:

1、没有性质纯粹的学会。

2、没有大学。

3、在西洋留学的很少。有亦为着吃饭问题和虚条心理,竟趋于“学非所用’的一途,不能持续研究其专门之学。在东洋留学的,被黄兴吸去做政治运动。

4、政治纷乱,没有研究的宁日。

这是湖南新学界中心思想不能确立的原故,即是没有学风原故,辛亥以来,滥竽教育的,大部市侩一流,逞其一知半解的见解造成非驴非马的局势。中心思想,新学风气,可是更不能谈及了。

近数年来,中国的大势陡转,蔡元培。江亢虎,吴敬恒,刘师复,陈独秀等,首倡革新,革新之说,不止一端,自思想,文学,以致政治,宗教,艺术,皆有一改旧观之概。甚至国家要不要,家庭要不要,婚姻要不要,财产应私有应公有,都成了亟待研究的问题。更加以欧洲的大战,激起了俄国的革命,潮流浸卷,自西向东,国立北京大学的学者首欢迎之,全国各埠各学校的青年大响应之,怒涛澎湃,到了湖南而健学会遂以成立。

\textbf{健学会之成立}

六月五日,省教育会会长陈润霖君邀集省城各学校职教员徐特立,朱剑凡,汤松,蔡湘,钟国陶,杨树达,李云杭,向绍轩,彭国君,方克刚,欧阳鼐,何炳麟,李景桥,赵翌等发起健学会。在楚怡学校开会。今录某报所载陈润霖君报告组织学会的意旨于下一一

兄弟前次到京,偶有感能,深抱乐观。象四年前,北京大学学生以做官为唯一目的。非独大学为然,即大学以外之学生,亦莫不皆然。前次居京,所见迥然不同。大学学生思潮大变,皆知注意人生应为之事,其思潮已多表露于各种杂志月刊中。因之各校学生,亦顿改旧观,发生此次救国大运动。其致此之故,则因蔡孑民先生自为大学校长以来,注入哲学思想,人生观念,使旧思想完全变掉。或该认学生救国运动为政客所勾引,而不知突出学生之自动,及新旧思潮之冲突也。盖自俄国政体改变以后,社会主义渐渐输入于远东。虽派别甚多,而潮流则不可遏抑。即如日本政府,从来对于提倡社会党人,苛待残杀,不遗余力,而近日竟许社会党人活动。如吉野博士等,则主张实行国家社会主义,以和缓过激主义,顺应世界之趋势,从看将日本政体改变为英国式虚君制。

于此可知,世界思潮改变之速,势力之大矣!我国新思潮亦甚发展,终难久事遏抑,国人当及时研究,导之正轨,国人等组织学会,在釆用正确健学之学说而为彻底之研究……

这日开会,听说街有朱剑凡君主张“各除成见,研究世界新思想,服从真理”的演说,向绍轩君主张“采用国家社会主义”的演说。在湖南思想界不可不谓空前的创闻。今录出该会所发表的会则如下一一

(一)本会同志组合,以输入世界新思潮,共同研究,择要传播为宗旨。

(二)本会定名为健学会。

(三)会所暂定长沙储英源楚怡小学校。

(四)入会者须确有研究学术之志愿,经本会会友一人以之介绍,得为本会会员。

(五)关于输入新思潮之方法一一

(1)凡最近出版之图书杂志,由本会随时搜集,以供会员阅览。会员所藏书报,得借给本会会员阅览。其有愿捐入本会者,本会尤为欢迎。

(2)延请海内外同志,随时调查,通信报告。

(3)介绍名人谈话。

(六)关于研究之方法一一

(1)研究范围,大半为哲学,教育学,心理学,论理学,文学,美学,社会学,政治学,经济学诸问题,会友必分认一门研究。

(2)重要之问题,由会友共同研究。

(3)会员有愿习外国语者,由本会会友传授。

(七)关于传播之方法一一

(1)讲演。分定期,临时两种。定期讲演,每周日曜日午前八时至十时。由会友轮流担任。讲员及讲题均于前周日曜日决定。讲友须预备讲稿,交由本会汇刊。临时讲演,凡有主要演题,或由会友,或请名人讲演,另觅地点,择期举行。

(2)出版。

 (八)本会设会计,管理图书各一人。其他会务由会友共同负责。每次开会推会友一人协时主席。 
(九)会友应守之公约如左一一

(1)确守时间。

(2)富于研究的精神。

(3)学问上之互助。

(4)自由讨论学术。

(5)不尚虚文客气,以诚实为主。

(十)会员年纳两元以上之会金,有能特别筹助经费者,本会极为欢迎。

(十一)本会遇有重要事项,必须讨论时,得于定期讲演后,临时通告全体,举行会议。

会则中的(五)(六)(七)(九)极为重要,(九)之富于研究的精神,所以破除自是自满的成见,立意很好。尚生于研究的精神之后,继之以“批评的”精神。现代学术的发展,大半为各人的独到所创获。最重的是“我”是“个性”,和中国的习惯,非死人不加议论,著述不引入今人的言论恰成一反比例。我们当以一己的心思,居中活动。如日光之普天照耀,如探海灯之向外扫射,不管他到底是不是(以今所是的为是)合人意不合人意。只求合心所安合乎箕理才罢。老先生最不喜欢的是狂妄。岂不知古今最确的原理,伟大的事业邵是系一些被人加上狂妄名号的狂妄人所发明创造来的。我们住在这复杂的社会,诡诈的世界没有批评的精神就容易会做他人的奴隶。其君谓中国人大半是奴隶,这话殊觉不错。(九)之自由讨论学术,很合思想自由,言论自由的原则。人类最可宝贵最堪自乐的一点却在于此,学术的研究最忌演释式的独断态度。中国什么“师严而后道尊”。师说“道统”,“宗派”都是害了独断态度的大病。都是思想界的强权,不可不竭力打破。象我们反对孔子,有很多别的理由,单就这独霸中国,使我们思想界不能自由,郁郁做两千年偶象的奴隶,也是不能不反对的。

健学会之进行健学会进行事项,会则所定大要系研究及传播最新学术。现在注重于研究一面,闻已派人到京沪各处,釆买书籍,新闻纸和杂志。在省城设一英语学习班,使会员学习英语,为直接研究四方学术的预备。有年在四五十的会员都喜欢学习。又设一演讲会由会员轮流发表意见,实行知识的交换。官气十足的先生们,忽然屈尊降贵,虚心研究起来。虽然旁人尚有不满意的处所,以为官气还有十分五、六,讲演要多采用命令式和训话式。更有谓他们是青叶上青虫的体合作用。象这样的求全责备,我们为何以下坏。在这么女性纤纤,暮气重重的湖南有此一举,颇足山幽囚而破烦闷。东方的曙光,空谷的足音,我们正应拍掌欢迎,希望他可作“改造湖南的张本”看他们四次讲演的问题,如“国人误谬的生死观”。怎样做人”,“教育和白话文”,“釆用杜威教育主义”,都可谓能得其要。倘能尽脱习气采用公开讲演,尽人都可以听,则传播之外,得益之大,当有不可计量的了。

\begin{flushright}(《湘江评论》临时增刊第一号)
\end{flushright}

\subsection[第三号(一九一九年七月二十八日)]{第三号}
\datesubtitle{(一九一九年七月二十八日)}

\subsubsection{世界杂评}

畏德如虎的法兰西法国于德国畏惧他如虎狼。德国这么大败,法国尚畏惧得很。割萨尔煤矿,划来因左岸独立,毁希里哥伦炮台,力波兰独立以蹙其东陲,助捷克独立以阻其南出。日耳曼奥地利也并归德国,则不惜破坏民族自决主义,多方以妨之。殖民地,陆海天空军备,则多方以消之。商船亦须交出大部,以阻其海外贸易之恢复。这样也算够了,还不止此。又向英美两国,请求保卫。前日电传,威尔逊于离法以前,签订一约。系证将来法国一受攻击,美国当起而援助。劳合乔治亦以英国名义,签定一同一性质的条约。此意何等深刻!何等惨淡!籍非法国有不可告之大缺点,何至有这样的畏惧。法兰西民族素负豪气,何至竟象妇人孺子,斤斤乞人保护。我觉得这不是法兰西的好现象!

和约的内容斯末资将军说:“我之签订和约,非因和约乃满意的文件。为结束战争起见,不得不签订他。”又说:“新生活,人类大主义的胜利,人民趋向于新国际制度和优善世界,所抱如此希望之践行,象这样的约言,均没有截上和约。于今只有国民心腔里抒发义侠和人道的新真意,乃能解决和会里政治家困难而止的问题。”又说:“我很以和约里取消黩武主义,仅限于敌国为憾。”斯末资是英国一个武人,是手签和约的一个人,他于签约后所发议论是这样,我们就可想见那和约的内容。

日德密约巴黎的路透电说:“近今外间又有日德密约的谣传。密约是什么东西?还有什么妄人想发现于今后的国际间么?日德密约更是什么东西?日俄密约,为列宁政府所宣布了,不但没成,反丢了脸一大抉。日英法密约成了,我们的山东就要危险。什么日德密约,前年也谣传了多次。据说一千九百一十七年,德国允许日本自由处置荷兰的殖民地,爪哇苏门答腊在内,为英政府听至,告诉了荷兰,阴谋方止。我们应知道日本和德国,是屡次寻求未遂的狗男女,他们虽未遂,那寻奸的念头,是永远不×打断的。日本的强权政府军阀浪人不割除,德国的爱贝尔政府不革命,娼夫和淫妇,还未拆开,危险正多呢。

政治家斯末资云:“惟人民的新真意,乃能解决和会里政治家困难而止的问题。”人民的真意,和政治家的见解,何以这么不相同?政治家何以这么畏难?人民何以这么不畏难?这里面果有一层解释呢?我自来疑惑所谓“政治家”,怕英不是一种好东西?我如今获得了证据。巴黎和约签订后,路易乔治回到英国,在下院演说道:“我们英国,得到许多成功是我们伟大国民团结兴奋的动力。我们于今欢欣鼓舞,但不要存着祸患业已过去的妄念。已使我们获胜的精神,仍要保持,以应付将来事件。”我们不要消费精力于彼此相争。这就是政治家的大本领,这就是政治家的大魔力。不要浪费精力于彼此相争。就是说道,你们人民不要拿着生活痛苦,国民真意,种种无聊问题,和我们政府为难。那些问题都小,都不关痛痒。将来寻着事端,我们还要和别国打仗。爱国,兴奋,团结,对外,是最重要没有的。我正式告诉路易乔治这一类的政治家,你们所说的一大篇,我们都清白是“鬼话”,是“胡说”。我们已经醒了。我们不是从前了。你们且收着,不要再来罢。

\subsubsection{湘江杂评}

不信科学便死两星期前,长沙城里的大雷,电触死了数人,岳麓山的老树下一个屋子里面,也被雷触死了数人。城里街渠污秽,雷气独多,应建高塔。设避雷针数处。老树电多,不宜在他的下面第屋,这点科学常识,谁也应该晓得,长沙城里的警察,长沙城里三十余万的住民,没一人有闲工夫注意他。有些还说是“五百蛮雷,上天降罚”。死了还不知死因。可怜!

死鼠鼠是瘟疫发生的一个原因,长沙城里到处看见死鼠,张眼望警察,警察却站在死鼠的旁边,早几年的长沙城,都没有这个样子,警察先生们,还是请你们注意点。
