\section[接见法国事务部长马尔罗时的谈话(一九六五年八月三日)]{接见法国事务部长马尔罗时的谈话}
\datesubtitle{(一九六五年八月三日)}


主席:马尔罗先生来了多久了?

马尔罗:十五、六天了。见了陈毅副总理,到延安等地去了一次。回来后见了周恩来总理。

主席:喔,你到了延安。

马尔罗:这次去延安,使我学到很多中国革命的历史情况,比过去知道得多,我今天能坐在除了列宁之外当代最伟大的革命家旁边,感到很激动。

主席:你说得太好了。

马尔罗:我在延安看到了过去艰苦的环境,人们都住在窑洞里,我也看到蒋介石住宅的照片,对比了一下就知道中国革命为什么会成功。

主席:这是历史发展的规律,弱者总是能战胜强者的。

马尔罗:我也是这样想的,我也曾领导过游击队,不过当时的情况不能同你们比。

主席:我听说你打过游击。

马尔罗:我是在法国中部打过游击,领导农民军队反对德国。

主席:十八世纪末法国革命推翻了封建统治,当时推翻封建统治的那些力量最初也不是强的,而是弱的。

马尔罗:这很有意思,农民都没有打过仗。不知如何打,但他们能当很好的战士。拿破仑手下就有很多这样的战士。我认为在毛主席之前,没有任何人领导农民革命获得胜利。你们是如何启发农民这么勇敢的?

主席:这个问题很简单。我们同农民吃一样的饭,穿一样的衣,使战士们感到我们不是一个特殊阶层。我们调查农村阶级关系,没收地主阶级的土地,把土地分给农民。

马尔罗:主席是否认为重要的是土地改革?

主席:土地改革,民主政治,此外还有一条,要打赢仗。如果不打赢仗,谁听你的话?打败仗总是有的,但少打一点败仗,多打一点胜仗。

马尔罗:在二千多年的历史中,农民习惯于打败仗,打胜仗的不多。

主席:我们打过胜仗,也打过败仗。甚至整个南方根据地都失掉了,跑到北方来。

马尔罗:尽管如此,但人民对红军的怀念仍然存在。

主席:以后在北方建立了很多根据地。发展了军队,发展了党,发展了群众组织,北方人民得了土地。我们解放了北京、天津、济南等大城市。队伍逐渐扩大到几百万,由北方向南方打。要讲经验还有一条,就是在中国要把民族资产阶级、资产阶级知识分子团结起来,团结民族资产阶级和资产阶级知识分子中凡是不跟敌人跑的人。我们在一个时期甚至跟大资产阶级蒋介石建立了统一战线,如果蒋介石不进攻我们,我们也不会进攻他。

马尔罗:为什么蒋介石要进攻你们呢?

主席:他想把我们吞掉,他认为他可以。

马尔罗:他是否认为中国共产党很弱?

主席:我们有许多根据地,人口占全国五分之一,军队有一百万左右。蒋介石却不同,他有四百多万军队,有美国的援助,我们的根据地很大,但根据地是分散的,也没有外部的援助。

马尔罗:是不是还有一个原因,即蒋介石只相信城市的力量。

主席:他有城市,有外国援助,同时他在农村的人口比我们多。

马尔罗:我曾去过俄国,曾同高尔基谈过这个问题,同他谈了毛泽东,那时您还不是主席,高尔基说,中国共产党最大困难是没有大城市。当时我问他,没有大城市是会失败还是成功?

主席:高尔基回答了你没有?(马尔罗摇头。)他不知道中国的情况,所以不能答复你。

马尔罗:高尔基常说各地农民都一样,但我认为每个地方的情况不一样。现在我提一个问题:中国再一次要把中国变成伟大的中国。几世纪前中国从技术上来说是强国,如丝绸,后来欧洲变成技术上先进的国家,有武器、军火,现在中国也有武器、军火,又要成为强国了。当然,中国不需要成为欧洲式的强国。中国要变为中国式的强国,不知需要什么东西。

主席:需要时间。

马尔罗:我们希望你们有你们需要的充裕时间。

主席:至少需要几十年。我们还需要朋友,例如同你们往来,建立外交关系,这就是朋友的关系。我们有各种朋友,你们就是朋友的一种,同时在北京访问的印度尼西亚共产党主席艾地,也是我们的朋友,我们还没有见到他,我们同艾地有共同点,同你们也有共同点。

马尔罗:这些共同点是不一样的。

主席:有一点是一样的,例如如何对付美帝国主义。对于英国的两面派,你们比英国明朗。

马尔罗:实际上反对美国在越南“逐步升级”的只有法国。

陈毅;英国人支持美国侵略越南,而你们反对。

马尔罗:英国有马来西亚问题。

主席:英美俩要交换。

马尔罗:从戴高乐总统恢复政权后,法国已结束了它的殖民主义立场。每天援助阿尔及利亚几亿法朗,是我亲自去非洲四国宣布他们独立的。在戴高乐总统看来,中、法有一个绝对的共同点,如果有苏、美双重世界霸权,那么中国要成为一个真正的中国和法国要成为真正的法国是办不到的。

主席:当然啰,一个是你们的盟国,一个是我们的盟国,你们的盟国是美国,对你们是不怀好意的。我们的盟国是苏联,对我们也不怀好意。

马尔罗:列宁死后,人们谈到苏联时就会想起斯大林,斯大林死后,……斯大林的制度被推翻了,至少部分被推翻了。但苏联领导人却假说苏联的制度没有改变,赫鲁晓夫就是这样亲自告诉我们的。我认为现在制度不同了,尽管用的词一样,但是内容很不同了。

主席:他还进一步说要建立共产主义,这是斯大林都没有说过的。

马尔罗:我感到赫鲁晓夫和柯西金使人想到似乎不是过去所理想的苏联了。

主席:它是代表一个阶层的利益,不是代表广大人民的利益。

马尔罗:他们甚至改变了政府行政管理方法。

主席:苏联想走资本主义复辟的道路,对这一点,美国是很欢迎的。欧洲也是欢迎的,我们是不欢迎的。

马尔罗:难道主席真正认为他们想回到资本主义道路?

主席:是的。

马尔罗:我认为他们在想办法远离共产主义,但他们要往哪里去?去找什么?连他们自己思想上也不清楚。

主席:他就是用这样一种糊里糊涂的方法迷惑群众,你们也有自己的经验,法国社会党难道真搞社会主义?法国共产党真相信马克思主义?

马尔罗:法国社会党党员中只剩下百分之七是工人,其余主要是职员,在这方面他们是强大的,因为有工会,是由职员们组成的。另外,还有些党员是南方的葡萄种植园主,是地主。至于说法国共产党,法国共产主义者,则是另外一个问题。法国农民的作用同中国的不一样,不只是绝对数少,占人口比例也不及中国。法国的共产主义者要起作用,只有两种可能,一是在知识分子中寻找力量,一是在真正无产者中间寻找力量。

法国的共产主义者可能在感情上偏向中国,但因具体情况不同于中国,因而实际上又偏向于苏联。

主席:他们是反对我们的。

马尔罗:作为一个党是反对中国的,但工人、知识分子,农民并不反对中国,目前党内冲突很严重,法共就像最懒的人一样,两面都想应付,你们可能已看到很多这种情况了。

陈毅:他们并没有应付我们。

主席:党是可以变化的,普列汉诺夫和孟什维克过去都是马克思主义者,后来就反对列宁,反对布尔什维克,脱离了人民。现在是在布尔什维克本身内部发生了变化,中国也有两个前途,一种是坚决走马列主义的道路,一种是走修正主义的道路。我们有要走修正主义道路的社会阶层,问题看我们如何处理,我们采取了一些措施,避免走修正主义道路。但谁也不能担保,几十年后会走什么道路。

马尔罗:现在中国修正主义阶层是否广泛存在?

主席:相当广泛,人数不多,但有影响。这些是旧地主、旧富农、旧资本家、知识分子、新闻记者、作家、艺术家以及他们的一部分子女。

马尔罗:为什么有作家?

主席:有一部分的作家的思想是反马克思主义的,我们把旧的摊子都接受下来了。我们原来没有艺术家、记者、作家、教授、教员,这些人都是国民党留下的。

佩耶:我有一感觉,中国青年正朝着主席所主张的方向在培养中。

主席:你来了多久了?

佩耶:十四个月了,从广州到北京一路上学了许多东西。以后参加使节旅行,去过华中华南,荣幸的访问了主席的故乡韶山,还到长沙、四川,最近又到东北去了一次,很有意思。我在工厂、公社、街上、戏院都尽量同人民接触。感到青年人没有那些要你们操劳的矛盾。

主席:你看了一个方面的现象,另一方面的现象没有注意到,一个社会不是一个单体,是个复杂的社会,存在着两种可能性。佩耶:我感到有一种力量引导青年,使他们走向你们所指出的方向,矛盾当然还会有,但是方向肯定了。

主席:一定有矛盾。

马尔罗:主席看,在反对修正主义方面,下一步的目标是什么。我指在国内方面。

主席:那就是反对修正主义,没有别的目标。我们反对贪污、盗窃、投机商人,反对修正主义的一切基础,不只是在党外,党内也有。

马尔罗:下一步的目标是什么?例如举行党代表大会就要确定一个目标。是否是农业问题,因为我感到工业问题已解决了,或起码是走上健全道路了。

主席:工业和农业问题都没有解决。

马尔罗:我在西安参观了纺织厂。在法国,纺织厂同革命有很大关系,一七八九、一八三〇、一八四〇、一八五一年都是这样,特别在里昂,因为纺织工人是最穷困的。我在西安看到该地的纺织业已达到解放前上海的水平,大部分机器是中国造的,机器多,工人少。显然中国党能在纺织工业执行它要执行的政策。但相反在农业方面,可耕地很少,使中国政府处境困难,现在农业方面是否考虑发动一次超过人民公社范围的大规模的运动。

主席:人民公社在组织机构和生产关系方面不会有什么改变,在技术方面开始有了改变。

马尔罗:你考虑可以增加些耕地面积吗?

主席:可以增加一些,主要的还是增加单位面积的产量,这里有很多文章可作,今天就不多谈了,请回去问候你们的总统。

马尔罗:关于外交政策的问题,我已同周总理,陈副总理谈过了,不再向主席重复了。今天谈的是戴高乐总统阁下最关心的基本问题。感谢今天的接见,并转达戴高乐总统阁下的问候。(出门时)

主席:我接见过法国议员。

马尔罗:我最不相信议员的话。

主席:他们对美国的态度没有你这样明朗。

马尔罗;可能是因为我更负有责任的缘故吧。

