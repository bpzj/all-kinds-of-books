\section[在省市委书记会议上的插话汇集(一九五七年一月)]{在省市委书记会议上的插话汇集}
\datesubtitle{(一九五七年一月)}


一、思想斗争问题

去年下半年以来,有一股右倾机会主义的风,在地面之上,云层之下流动。党内外有一股反社会主义的逆流。

对社会上的歪风,一定要打下去。打的办法要有说服力,不是简单骂几句。如果简单粗暴,则歪风越刮越大。

今年五月人代会,七月政协会,怪议论一定会多。一个是法制,一个是农民苦,一个是没肉吃。总的是社会主义有没有优越性。要准备打这两仗,在座的代表要出马参战,每人准备一篇,要有分析。

同小资产阶级特别是富裕中农思想作斗争,要反复斗争多少年,富裕中农的影响很大,特别厉害,应当注意。

光搞业务工作,不搞思想工作,就会闹乱子。

每个省要有一、二个马克思,一、二个鲁迅。你们应该写文章,六十岁以下的都应该写。

每个省要培养理论家。现代有培养唱戏的,画画的,没有培养搞理论的。这也是个体制问题。你们靠中央,中央并没有不让你们搞。

二、闹事问题

对闹事怕不怕,我看还是不怕,闹更多的事也不怕。年年会闹事,以后会更多,怕有何用?矛盾要揭露,问题才能解决。社会是对立的统一。教授、学生与我们总会有矛盾,总会有吵架。再过廿年,老的死了,新的还会有问题。苏联革命四十年还有闹事。工厂中有百分之三十是地主、富农、资本家家庭出身的,大学中有百分之八十是地主、富农、资本家家庭出身的,他们闹事不要怕,闹到北京来也不怕。应该采取积极态度,不要采取消极态度。格罗从南斯拉夫回到布达佩斯骂群众,一下丧失威信,被轰下台,卡达尔则不同,既不骂群众也不跟纳吉跑,要准备出大事,可能就不会出大事,出了事要采取积极态度,正确方法,争取分化。有本领的人就是要在波兹南、匈牙利那样的情况下,搞出一个局面来。

但愿天下太平。但要放在最坏的方面,准备出大事。我们从延安来,准备再回延安。过去没有看过梅兰芳的戏,现在看了七年,第八年准备回延安。无非是打原子弹,打世界大战,犯错误,出匈牙利事件。只要思想上从坏的方面作准备,一切不怕。不准备就会哭鼻子。“七大”提出十七条,“赤地千里”。现在得了七亿人口的国家,出点事算什么。天要下雨,娘要嫁人,无可奈何。

罢工、罢课、请愿,宪法上有的规定了,有的也未禁止。因此,第一,不要提倡,第二,不管合理不合理,要罢工请愿,就让他罢工请愿。提的合理的要纠正,提的不合理的要解释。教授要发议论让他发,以揭露矛盾解决矛盾为好,不要一棍子打死,如果总是压,结果一定会总爆发,就会变成拉科西,也不能一味迁就,如果一味迁就,又会变成纳吉。

小资产阶级想专政,把你搞下来他专政。想搞匈牙利的,要整一二十年。各省要开群众大会,演讲会,辩论会,展开争论,看谁胜利。但事前要作准备。小会他神气大,大会他没办法。你要大民主,我就照你的办。有屁让他放,不放对我不利,放出来大家鉴别香臭,社会发生分化,我们争取大家。大家认为臭,他就被孤立了。

不要怕闹,闹的越大越长越好。七闹八闹总会闹出名堂来的,可以看清是非。不管怎样闹,不要怕,越怕鬼越来,但也不要开枪,什么时候开枪都是不好的。

全国大乱不可能,那里有脓包,有细菌总是要爆发的,大者五万,中者三万,小者一万,准备闹事。年终结账,如果达不到这个数字,就算工作作好了。

对闹事要开县区委书记会议作准备,区委书记、区专以上要有精神准备,先太平无事,消极怕闹,结果闹起来了,被动。先思想麻痹后事事答应。

不打问题不能解决,矛盾要经过斗争解决。河南回民杀牛受限制,起来打干部,一打就解决了。打有合理的,有不合理的,就是不合理的,也是好的,有脓总得放。

对于坏人的处理,一律开除好不好?我看不要急于开除。也不要杀掉(反革命刑事犯除外)利用他作工作,把他当做政治教材,剥夺他的政治资本。简单处理就是思想工作不承认对立统一。对千肖罗、丁雪之类的人,杀、关、管都不好,要抓他许多小辫子,在社会上把他搞臭。对学校闹事的党员,讨论清楚再开除党籍,不要急于开除党籍。也不能开除出校。陕西关于工人成份摸清楚,搞的调查很好。匈牙利没有调查,乱子出来了还不知道什么原因,鉴于这个教训,必须把工人成份摸清楚,了解可靠的、中间的、有问题的各有多少,现在调查可靠的只有百分之二十五,要增加工人的比重,过三个五年计划就可能逐渐改变这种情况。大学生成份再过廿年才能改变。现在招许多工人、农民,在学校还办不到。

苏共二十次大会,把斯大林一棍子打死,痛快的很,现在斯大林又活过来了。摇过去是本心,摇过来不是本心。帝国主义一棍子,社会主义几棍子,又摇过来,开口波兹南、闭口匈牙利,不说不是本心。

三、哲学问题

懂得分析,就是懂得辩证法。列宁说:“可以把辩证法简单的确定为关于对立面的统一的学说。这样就会抓住辩证法的核心,可是这需要说明和发挥。”又说:“对立面的统一(一致、统一、均势)是有条件的,暂时的,易逝的,相对的,互相排斥的。对立面的斗争则是绝对的,正如发展、运动是绝对的一样。”平衡是暂时的,不断地在破坏,有的几天几个月就破坏了,刚平衡,马上又不平衡。同一者即是一致、同一、合一。党内有意见,开个会统一了,过两个月新的问题又出现了,又开会。实事求是和主观主义也是对立的统一的,一万年以后,还会有主观主义。

唯物主义和唯心主义是对立统一的。辩证法和形而上学是对立统一的。哲学永远有斗争。有人讲哲学只讲一面,百花齐放,只讲放香花,不讲放毒草。我们承认社会主义有对立的东西。斯大林就有形而上学、主观主义。苏联不承认有对立,法律上不允许,实际许多错误的东西,都是在社会主义的幌子下面掩盖着。

列宁认为单讲唯物主义不能解决问题,要解决问题,就要和唯心主义做斗争。要斗争就要研究唯心主义。马克思主义的三个组成部分就是研究了资本主义的东西斗争出来的。

哲学领域里两个对立的东西在斗争,世界观是唯物主义与唯心主义斗争,方法论是辩证法与形而上学斗争。我们与蒋介石是对立统一;与民主人士也是对立统一。各种事物有矛盾的两个侧面联系,斯大林在辩证法上有错误,“否定的否定”,十月革命否定了资本主义,但是他们不承认社会主义会被否定,我们认为天下是稳定的,又是不稳定的。社会主义有一天也会消亡。如果说有一个社会上层建筑不会灭亡,那就不是马克思主义,而是同宗教一样了。

四、农村问题

丰收很重要。苏联去年丰收,许多情况好办了。我们有了一年的经验,今年要来个丰收年。如果今年大家努力丰收了,对世界共产主义运动的意义很大,这是社会主义运动史上的创举,历史上社会主义合作化都减产,我一九五六年增产不大,一九五七年要大增一下。

百分之九十以上的社员增加收入是个目标,看三、四年能否达到。湖北是丰收,还有百分之十五的社员没有增加收入。应该向党政军民讲清楚,合作化只有一年到一年半的历史,我们搞了一辈子革命还犯错误,人家只有一年多的历史为什么不犯错误呢?要有三年的历史,百分之九十以上的社员增加收入是可能办到的。农民有百分之十到百分之十五的人生活苦,笼统的说农民苦是不对的,要有分析。所谓农民苦,就是富裕中农收入减少了,有一部分老干部家里富了,反映地主、富农、富裕中农的思想。

四十条在今年党代表大会上讨论过,宣传一下,亩产四、五、八百斤,争取提前一年完成。十二年计划十一年完成。

头三脚难踢,办事有三折,今年完成了才一折,要三个五年计划才好办。

自古以来,凡是先进的东西,一开始没有不挨骂的。进化论、马克思、孙中山、共产党都挨过骂。

只要有一个合作社有优越性,就可以驳倒一切胡说,以此做宣传。

五、经济问题

化学肥料,各省自己开厂。

要注意搞粮食,不搞粮食很危险。有了粮食,就有了一切,罢工、罢课也有饭吃,天下不会大乱。油料和猪肉由合作社包干。国家只要不给。收来粮食也是这样(经济作物区除外),国家只管统购,不管农村统销,社包社,队包队。

两年后,不是与民争粮,而是怕国家不买,只要统购,不要统销。湖北省孝感县,现在情况就是如此。

新工矿区,要开国营农场,解决粮、油、肉、菜等问题。

大的工厂是必要的,小的要以有原料和销路为原则,大的要以国力有多大则搞多少为原则。

六、学校问题

初中要加农业课。

各省市要派干部加强学校,把学校抓起来。整个文教工作都应如此。

省委、地委、县委同志要分区分片向学生进行教育。

各省开一次学生代表大会、教师代表大会,一年一次,先准备好。一天报告,三天讨论解决问题,对学生工作没有安排应当安排。一年开一两次会,讲一两次话。要注意发现问题,不要等问题大暴露了再解决。现在好管了,工厂、合作社、商店、学校,就是这么些单位。

七、文学艺术问题

百家争鸣有好处。让那些牛头、蛇身、鬼子、王八都出来。各省要注意,对重大成熟的问题,以一个人为主,组织些人写文章,准备一个月写一篇,把邪气压下去。

种牛痘可以产生免疫力(给你身体内部埋点对立物),为什么文学艺术不可以种牛痘?

宣传部,一、二个月召集记者谈一、二次,两个月谈一次也可以,现在是一年谈不上一次。

八、干部问题

干部尽往上提,提上来又没事做,我们在这个问题上犯了错误。现在只得下放,不准上提。领导机关,人越少越好,无非是写文章,打电话。

国家是阶级斗争的工具,只能由少数人组成国家。五亿人统统变成国家,那能行?大学教授有什么提拔?还不是白发苍苍的当一世教授?工人农民如何提拔?还不是白发苍苍当一世工人农民?不是不提拔,死了就得补。

极端民主,绝对平均,总是不行的。颜渊没有评级,是第二级圣人。

外来干部与本地干部要调整一下,县委委员中,可以外来与本地各一半。

九、领导问题

中央委员,省委书记,部长,每年都要到工厂、农村去跑一跑,了解情况,才不变成拉科西。

省委书记、县委书记对各部看到等于没有看到一样,往工厂、合作社里钻。

真正的知识,出在工厂、合作社、商店、学校,越是上面,越没有东西。要解决问题,一是要下去,二是调人上来汇报。北京这个地方不好,搞不到很多的知识,省里好一点。一定要下去,工业部长下工厂,农业部长跑农村。

中央委员、省委书记、部长,每人一年一定要有一些时间下乡,研究问题。搞一个县、一个乡,两个月可以搞清楚,两个月不看电报还能看懂。现在听说大家都不下去了,机关里事情多了。

省委书记可以兼一个县(市)委书记,把原来的县(市)委书记压成第二书记。省委书记下去了,地县委书记也会下去。市委书记可以兼厂长、校长、党委书记。这样就可以深入实际,吸收知识。

区委书记会议,一年开两次好。逼的你研究问题。

上面放的屁不全是香的,这里也有对立,有香也有臭(包括北京在内),一定要嗅一嗅。

各省要注意,城市依靠工人阶级,农村依靠贫农,学校依靠左翼,总要有个依靠。

胡志明在依靠贫农的问题上有错误,不是纠“偏”,而是纠“正”。对土改不肯定成绩,光讲错误,发了一个罪己诏——告农民书。地主一变为农民,民主党派反对一党专政,知识分子反党。有人提出要学中国百花齐放,百家争鸣。但不让学,说过去学中国学错了,结果一花独放,一家独鸣。现在改变了。

十、其它

精简各省自己搞,有出路的先搞下去,尽可能转回生产。无出路的设教养院,听候出差。


