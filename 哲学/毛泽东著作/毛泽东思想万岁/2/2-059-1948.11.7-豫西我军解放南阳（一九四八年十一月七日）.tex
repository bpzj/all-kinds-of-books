\section[豫西我军解放南阳(一九四八年十一月七日)]{豫西我军解放南阳}
\datesubtitle{(一九四八年十一月七日)}


在人民解放军伟大的胜利的攻势下,南阳守敌王凌云于四日下午弃城南逃,我军当即占领南阳。南阳为古宛县,三国时曹操与张绣曾于此城发生争夺战。后汉光武帝刘秀,曾于此地起兵,发动反对王莽王朝的战争,创立了后汉王朝。民间所传二十八宿,即刘秀的二十八个主要干部,多是出于南阳一带。在过去一年中,匪首蒋介石极重视南阳,曾于此设立所谓绥靖区,以王凌云为司令官,企图阻遏人民解放军向南发展的道路。上月,白匪崇禧使用黄维兵团三个军的力量,经营整月,企图打通信阳南阳间的运输道路,始终未能达到目的。最近蒋匪因全局败坏,被迫将整个南部战线近百个师的兵力,集中于以徐州为中心和以汉口为中心的两个地区,两星期前已放弃开封,现又放弃南阳。从此,河南全境,除豫北之新乡、安阳,豫西之灵宝,内乡,豫南之确山、信阳、潢川、光山、商城、固始等地尚有残敌外,已全部为我解放。去年七月,南线人民解放军开始向敌后实行英勇的进军以来,一年多时间内,除歼灭了大量的国民党正规部队以外,最大的成绩,就是在大别山区(鄂豫区),皖西区,豫西区,陕南区,桐柏区,江汉区,江淮区(郾皖东一带)恢复和建立了稳固的根据地,创立了七个军区,并极大地扩大了豫皖苏军区老根据地。除江淮军区属于苏北军区管辖外,其余各军区,统属于中原军区管辖。豫皖苏区豫西区,陕南区,桐柏区现已联成一片,夕没有敌人的阻隔。这四个军并已和华北联成二片成武装力量,除补上野战军和地方军一年多激烈战争的消耗以外,还增加了大约二十万人左右,今后当有更大的发展。白崇禧经常说:“不怕共产党凶,只怕共产党生根”,他是怕对了。我们在所有江淮河汉区域,不仅是树木,而且是森林了。不仅生了根,而且枝叶茂盛了。在去年下半年的一个极短时间内,我们在这一区域曾经过早地执行分配土地的政策,犯了一些政策上左的错误。但是随即纠正了,普遍地利用了抗日时期的经验,执行了减租减息的社会政策和各阶层合理负担的财政政策。这样,就将一切可能联合或中立的社会阶层,均联合或中立起来,集中力量反对国民党反动统治势力及乡村中为最广大群众所痛恨的少数恶霸分子。这一策略,是明显地成功了,敌人已经完全孤立起来。在我强大的野战军和地方军配合打击之下,困守各个孤立据点内的敌人,如像开封、南阳等处,不得不被迫弃城逃窜。南阳守敌王凌云统率的军队是第二军,第六十四军以及一些民团,现向襄阳逃窜。襄阳也是国民党的一个所谓“绥靖区”,第一任司令官康泽被俘后,接手的是从新疆调来的宋希濂。最近宋希濂升任了徐州的付总司令兼前线指挥所主任,去代替原任的杜聿阴。杜聿明则刚从徐州飞到东北,一战惨败,又逃到了葫芦岛到襄阳,大概是接替宋希濂当司令官。但是,从南阳到襄阳,并没有走得多远,襄阳还是一个孤立据点,王凌云如不再逃,康泽的命运是在等着他的。

