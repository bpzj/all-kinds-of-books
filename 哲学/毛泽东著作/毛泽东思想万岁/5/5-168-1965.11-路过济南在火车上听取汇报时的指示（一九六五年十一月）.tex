\section[路过济南在火车上听取汇报时的指示(一九六五年十一月)]{路过济南在火车上听取汇报时的指示}
\datesubtitle{(一九六五年十一月)}


(一九六五年十一月主席去华东路过济南,在火车上,有关领导向主席汇报工作)主席指示:你们有没有钢?(答,准备搞××吨,将来再发展到×××万吨)哦,搞×××吨,那好。

我对这一条比较积极,我是支持地方的。各省总要有个×万吨的钢铁厂,能制造机器,制造武器。我不怕你们造反,我自己也是造反的,造了多少次反,袁世凯当皇帝逼出了个蔡锷造反。如果中央出了军阀,出了修正主义,你们就可以造反。但是你们不能随便造反,不能造马列主义的反,否则你们就会吃亏,会成为修正主义。一个省也造不起反来。一个省搞点钢!搞×万吨左右的钢铁厂,一翻×万吨,再一翻××万吨,再一翻××万吨。但是要有条件。没有条件就不能炼钢。


