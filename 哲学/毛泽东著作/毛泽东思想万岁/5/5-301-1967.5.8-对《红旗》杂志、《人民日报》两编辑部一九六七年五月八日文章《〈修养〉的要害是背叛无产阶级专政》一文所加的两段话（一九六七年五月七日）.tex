\section[对《红旗》杂志、《人民日报》两编辑部一九六七年五月八日文章《〈修养〉的要害是背叛无产阶级专政》一文所加的两段话(一九六七年五月七日)]{对《红旗》杂志、《人民日报》两编辑部一九六七年五月八日文章《〈修养〉的要害是背叛无产阶级专政》一文所加的两段话(一九六七年五月七日)}
\datesubtitle{(一九六七年五月八日)}


这种对于共产主义社会的描绘,不是什么新的东西,是古已有之的。在中国,有《礼运·大同篇》,有陶潜的《桃花源记》,有康有为的《大同书》,在外国,有法国和英国空想社会主义者的大批著作,都是这一路货色。

照作者的意见,共产主义社会里,一切都是美好的,一点黑暗也没有,一点矛盾也没有,一切都好了,没有对立物了。社会从此停止发展,不但社会的质永远不变化,连社会的量似乎也永远不变化了,社会的发展就此终结,永远一个样子。在这里,作者把马克思主义一个基本规律抛掉了——任何事物,任何一个人类社会,都是由对立斗争,由矛盾而推动发展的。作者在这里宣扬了形而上学,抛弃了伟大的辩证唯物论和历史唯物论。

