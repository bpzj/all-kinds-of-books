\section[论中印关系问题(一九五九年五月十五日)]{论中印关系问题}
\datesubtitle{(一九五九年五月十五日)}


(注,外交部在一九五九年五月为潘××大使拟了一份对印度外交部外事秘书杜德的书面谈话稿。毛主席在审阅这篇谈话稿时,加了这段文字.标题是原编者加的。)

总的说来,印度是中国的友好国家,一千多年来是如此,今后一千年一万年,我们相信也将如此。中国人民的敌人是在东方,美帝国主义在台湾,在南朝鲜。在日本,在菲律宾,都有很多的军事基地,都是针对中国的。中国的主要注意力和斗争方针是在东方,在西太平洋地区,在凶恶的侵略的美帝国主义,而不在印度,不在东南亚及南亚的一切国家。尽管菲律宾、泰国,巴基斯坦参加了旨在对付中国的东南亚条约组织,我们还是不把这三个国家当作主要敌人对待,我们的主要敌人是美帝国主义。印度没有参加东南亚条约,印度不是我国的敌对者,而是我国的友人。中国不会这样蠢,东方树敌于美国,西方又树敌于印度。西藏叛乱的平定和进行民主改革,丝毫也不会威胁印度。你们看吧。“路遥知马力,日久见人心”(中国俗话),今年三年,五年、十年,二十年,一百年,……中国的西藏地方与印度的关系,究竟是友好的,还是敌对的,你们终究会明白。我们不能有两个重点,我们不能把友人当敌人,这是我们的国策。几年来,特别是最近三个月,我们两个之间的吵架,不过是两国千年万年友好过程中的一个插曲而己。值不得我们两国广大人民和政府当局为此而大惊小怪。我们在本文前面几段所说的那些话,那些原则立场,那些是非界线。是一定要说的。不说不能解决目前我们两国之间的分歧。但是那些话所指的范围,不过是暂时的和局部的一一即属于西藏一个地方,我们两国之间的一时分歧而已。印度朋友们,你们的心意如何呢?你们会同意我们的这种想法吗?关于中国的主要注意力只能放在中国的东方。而不能也没有必要放在中国的西南方这样一个观点,我国的领导人毛泽东主席,曾经和前任印度驻中国大使尼赫鲁先生谈过多次,尼赫鲁大使很明白和欣赏这一点。不知道前任印度大使将这些话转达给印度当局没有。朋友们,照我们看,你们也是不能有两条路线的,是不是呢?如果是这样的话。我们双方的会合点就是在这里.请你们考虑一下吧。请让我借这个机会.问候印度领袖尼赫鲁先生。


