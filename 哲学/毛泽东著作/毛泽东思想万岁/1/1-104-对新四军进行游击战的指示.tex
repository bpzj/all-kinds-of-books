\section[对新四军进行游击战的指示]{对新四军进行游击战的指示}


项英同志:在敌后进行游击战争虽有困难,敌情方面虽较严重,但只要有广大群众活动地区,充分注意指挥的机动灵活,也会能够克服这种困难,这是河北及山东方面游击战争已经证明了的。在侦察部队出去若干天之后,主力就可以跟进,在广德、苏州、镇江、南京、芜湖五区之间广大地区创造根据地,发动民众的抗日斗争,组织民众武装,发展新的游击队,是完全有希望的。在茅山根据地大体建立起来之后,还应准备分兵一部进入苏州、镇江、吴淞三角地区去,再分一部分渡江进入江北地区。在一定条件下,平时也是能发展游击战争的,条件与内战时候,很大不同。当然,无论何时,应有谨慎的态度,具体的作战行动,应在具体的情况许可之下,这是不能忽视的。(下略)

