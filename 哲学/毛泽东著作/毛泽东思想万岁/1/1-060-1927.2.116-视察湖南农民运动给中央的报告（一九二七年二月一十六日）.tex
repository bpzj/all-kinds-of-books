\section[视察湖南农民运动给中央的报告(一九二七年二月一十六日)]{视察湖南农民运动给中央的报告}
\datesubtitle{(一九二七年二月一十六日)}


弟二月十二日由长沙到武昌,兹将各事报告于次:

1、弟十二月十七日由汉口到长沙,参与全省农民代表大会,在大会的议案起草委员会中商量了各种决议案,此次决议各案大体还算切实。十二月三十日大会闭幕。在区委决之将大会代表各同志开一短期训练班,弟作了三次关于农民问题及调查方法的报告。

2、一月四日起往乡下去考察,至二月五日止,共考察了三十二天,经过湘潭、湘乡、衡山、醴陵、长沙五县,布乡下在县城邀集有经验的农民及农运同志开调查会,所得材料颇不少。在各县乡下所见所闻与在汉口在长沙所见所闻的几乎完全不同,始发现从前我们对农运政策上处置上几个颇大的错误点,在湘潭、湘乡、衡山三个县调查后回到区委向负责同志作了一次详细的报告,在党校、团校各作了一次报告,在醴陵、长沙两县调查后,又在区委做了一次报告,党从前对农运的错误,已经有所攻正,其重点如:(一)以“农民运动好得很”纠正政府和国民党以及地主阶级一致的论调“农民运动糟得很”(二)以“贫农乃是革命先锋”纠正社会“痞子运动”和“惰农运动”的论调,(三)以从来没有过联合战线,纠正“农运破坏了联合战线”的论调,今后的作法是应共同积极为建设联合战线而努力。(四)革命分为三个时期,第一个时期是组织时期;第二个时期是革命时期;第三个时期是准备建立联合战线时期;是不能越过第二个时期(革命暴动时期)而到第三时期而不经过一个猛烈打倒封建威风的时期(五)湘中、湖南各县大都经过革命时期,农民暴风雨般打倒地主阶级,乡村处于无政府状态,应立即实现民主的乡村自治制度,变无政府为有政府,具体的建立农村联合战线,以免去农民孤立的危险;农村中武装,民食,教育建设,仲裁等问题也有最后的着落;目前湘南的政治问题,莫急于完成乡村自治这一点,省民会议,县民会议非在完成乡村自治之后,快无可言。(六)第二个时期(即农民起来暴动)的时期,农民对地主的一切作法都是对的,即使过分一些也是对的,矫枉必须过正,因为不过分,不用大的力量就不会把地主几千年积累的势力打垮,就不能迅速地完成国民革命。因此,农民协会万万不能请县政府或团防来逮捕“痞子”,只能提“整顿纪律”的口号,只能自己动手来进行农民协会基层组织的整顿,清除“少数不良分子”否则其他的任何作法都会减损农民的志气,长地主的威风。(七)阻谷问题是各界的怨府,其实乃多数贫农要阻,只少数富农要放,农协只能处于劝告地位,劝告贫农扯富农让步,不能专代表富农去打击贫农,阻谷所以利害,全因乡村无政府,不能保障民食,这是政府的责任,不全是农协的责任,要谷米流通只是从速建立新的乡村自治机关,负责保障民食。(八)农村间的各种冲突,如农工冲突,农商冲突,农党冲突,农学冲突,贫农与富农冲突等,均必须抬出国民党(K。M。T。)的招牌去解决,万不可马上抬出共产党牌子去解放。因此,农民中必须普遍的发展国民党,让国民党去调和敷这些极难调和敷衍的事情。以前国民党发展的程度与农运发展的程度相差得太远,必须大大地在农民尤其在贫农中发展国民党组织。(九)农民问题只是一个贫农问题,而贫农的问题有二个:即资本问题与土地问题这两个都已经不是宣传的问题而是立即实行的问题了。(十)在湖南的许多县分的农村中,农民已经完成民主革命,要求进入一个别的革命、贫农的情绪仍旧很高,依现在的形势,干百万的贫农群众(据长沙统计,贫农占十分之七,中农占十分之二,富农占十分之一)迫切要求进入一个别的革命。据我们的考察,无论如何压抑也不能长久的压抑住,现在是群众向左,而党在许多问题上的表示,都是不能与群众的情绪相符合。K。M。T信就更是如此,这是一个非常值得注意的问题。

(十一)因此,无论(A)为应付目前环境,(B)准备不久要求的革命,我们党都需要一个大大的发展,最小数目湖南党在六个月内要发展到两万人(现在才六千),省农民协会会员二万以上的县均须成立独委,这样才有办法。(十二)洪会是一种势力,必须拉拢这种势力而不可采取打击的方法。(十三)妇女是儿童在乡村起来的形势极佳,妇女尤是一个伟大的力量,不可不加注意。

上列十三项,举其要目,详细情形当从明日起三四日内写出一个报告送兄处审核,并登报导。

