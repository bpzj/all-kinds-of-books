\section[关于哲学问题的讲话(一九六四年八月十八日)]{关于哲学问题的讲话}
\datesubtitle{(一九六四年八月十八日)}


有阶级斗争才有哲学(脱离实际谈认识论没有用)。学哲学的同志应当下乡去。今冬明春就下去,去参加阶级斗争。身体不好也去,下去也死不了人,无非是感冒,多穿几件衣服就是了。

大学文科现在的搞法不行。从书本到书本,从概念到概念。书本里怎能出哲学?马克思主义三个构成部分,基础是社会学,阶级斗争。无产阶级和资产阶级之间作斗争,马克思他们看出,空想社会主义者想劝资产阶级发善心,这个办法不行,要依靠无产阶级的阶级斗争。那肘已经有许多罢工。英国国会调查,认为与其十二小时工作制不如八小时工作制对资本家有利。从这个观点开始,才有马克思主义,基础是阶级斗争,然后才能研究哲学。什么人的哲学?资产阶级的哲学,无产阶级的哲学。无产阶级哲学是马克思主义哲学,还有无产阶级经济学,改造了古典经济学。搞哲学的人,以为第一是哲学,不对,第一是阶级斗争。压迫者压迫被压迫者,被压迫者要反抗,找出路,才去找哲学。我们都是这样过来的。别人要杀我的头,蒋介石要杀我,这才搞阶级斗争,才搞哲学。

大学生今年冬天就要开始下去,讲文科。理科的现在不动,动一两回也可以。所有学文科的:学历史的、学政治经济学的、学文学的、学法学的,统统下去。教授、助教、行政工作人员、学生统统下去。去五个月,有始有终。农村去五个月,工厂去五个月,得到点感性知识,马、牛、羊、鸡、犬、豕、稻、梁、菽、麦、黍、稷,都看一看。冬天去,看不到庄稼,至少还可以看到土地,看到人。去搞阶级斗争,那是个大学,什么北大、人大,还是那个大学好!我就是绿林大学的,在那里学了点东西。我过去读过孔夫子,四书五经,读了六年,背得,可是不懂。那时候很相信孔夫子,还写过文章。后来进资产阶级学校七年。七六十三年。尽学资产阶级那一套自然科学和社会科学。还讲了教育学。五年师范,两年中学,上图书馆也算在内。那时就是相信康德的二元论,特别是唯心论。看来我原来是个封建主义者和资产阶级民主主义者。社会推动我转入革命。我当了几年小学教员、校长,四年制的。还在六年制学校里教过历史、国文。中学还教过短时期,啥也不懂。进了共产党,革命了,只知道要革命,革什么?如何革?当然,革帝国主义,革旧社会的命,帝国主义是什么东西,不甚了解。如何革?更不懂。十三年学的东西,搞革命却用不着,只用得着工具——文字。写文章是个工具。至于那些道理,根本用不着。孔夫子讲“仁者人也”,“仁者爱人”,爱什么人?所有的人?没那回事。爱剥削者?也不完全,只是剥削者的一部分。不然为什么孔夫子不能做大官?人家不要他。他爱他们,要他们团结。可是闹到绝粮,“君子固穷”,几乎送了一命,匡人要杀他。有人批评他西行不到秦,其实,诗经中《七月流火》,是陕西的事。还有《黄鸟》,讲秦穆公死了杀三个大夫殉葬的事。司马迁对《诗经》评价很高,说《诗经》三百篇皆古圣贤发愤之所为作也。《诗经》大部分是风诗,是老百姓的民歌。老百姓也是圣贤。“发愤之所为作”,心里有气,他写诗:“不稼不穑,胡取禾三百廛兮?不狩不猎,胡瞻尔庭有悬狙兮?彼君子兮,不素餐兮!”“尸位素餐”,就是从这里来的。这是怨天,反对统治者的诗。孔夫子也相当民主,男女恋爱的诗他也收。朱焘注为淫奔之诗。其实有的是,有的不是,是借男女写君臣。五代十国的蜀,有一首诗吗《秦妇昤》,韦庄的,少年之作,是怀念君王的。

讲下去的事。今冬明春开始,分期分批下去,去参加阶级斗争。只有这样,才能学到东西,学到革命。王××作了报告,她去搞了一个大队,那里没有暖气,同吃同住,吃得不好,害了两次感冒,春节回来的时候,见了她,我问她,还去不去,她说还去,无非是发几天烧。你们知识分子,天天住在机关里,吃得好,穿得好,又不走路,所以生病。衣食住行,四大要症。从生活条件好,变到生活条件坏些,下去参加阶级斗争,到“四清”、“五反”中去,经过锻炼,你们知识分子的面貌就会改观。

不搞阶级斗争,搞什么哲学!

下去试试看,病得不行了就同来,以不死为原则,病得快死了就回来。一下去精神就出来了。

(康生:科学院哲学社会科学部的研究所,也统统要下去。现在快要成为古董研究所,快要变成不食人间烟火的神仙世界了。哲学所的人《光明日报》都不看。)

我专看《光明日报》、《文汇报》,不看《人民日报》,因为《人民日报》不登理论文章,建议以后,他们登了。《解放军报》生动,可以看。

(康生:文学研究所对周谷城问题不关心。经济所孙冶方搞利别尔曼那一套,搞资本主义。)

搞点资本主义也可以。社会很复杂,只搞社会主义,不搞资本主义,不是太单调了吗?不是没有对立统一,只有片面性了吗?让他们搞,猖狂进攻,上街游行,拿枪叛变,我都赞成。社会很复杂,没有一个公社、一个县、一个中央部不可以一分为二。你看,农村工作部就取消了。它专搞包产到户,“四大自由”,借贷、贸易、雇工、土地买卖自由,过去出过布告。邓子恢同我争论。中央开会,他提议搞四大自由,巩固新民主主义秩序。永远巩固下去,就是搞资本主义。我们说,新民主主义是无产阶级领导的资产阶级民主主义革命,只触动地主、买办资产阶级,并不触动民族资产阶级。分土地给农民,是把封建地主的所有制改变为农民个体所有制,这还是资产阶级革命范畴的。分地并不奇怪,麦克阿瑟在日本分过地。拿破仑也分过。土改不能消灭资本主义,不能到社会主义。

现在我们的国家大约有三分之一的权力掌握在敌人或敌人的同情者手里。我们搞了十五年,三分天下有其二,是可以复辟的。现在几包纸烟就能收买一个支部书记,嫁给个女儿就更不必说了,有些地区是和平土改,土改工作队很弱,现在看来问题不少。

关于哲学的材料收到了(指关于矛盾问题的材料——记录者注),提纲看了一遍(指批判“合二而一”论的文章提纲——记录者注),其它来不及看。关于分析与综合的材料也看了一下。

这样搜集材料,关于对立统一规律,资产阶级怎么讲,马恩列斯怎么讲,修正主义怎么讲,是好的。资产阶级讲,杨献珍讲,古人是黑格尔讲。古已有之,于今为烈。还有波格丹诺夫、卢那察尔斯基讲造神论。波格丹诺夫的经济学我看过。列宁看过,好像称赞过他讲原始积累那一部分。

(康生:波格丹诺夫的经济学比现代修正主义者的那一套还高明一些。考茨基的比赫鲁晓夫的高明些,南斯拉夫的也比苏联的高明一些。德热拉斯还讲了斯大林的几句好话,说他在中国问题上作了自我批评。)

斯大林感到他在中国问题上犯了错误。不是小错误。我们是几亿人口的大国,反对我们革命,夺取政权。为了夺取全国政权,我们准备了好多年,整个抗战都是准备,看那时中央的文件,包括《新民主主义论》,就清楚。就是说不能搞资产阶级专政,只能建立无产阶级领导下的新民主主义,搞无产阶级领导的人民民主专政。在我国,八十年,资产阶级领导的民主主义革命都失败了。我们领导的民主主义革命,一定要胜利。只有这条出路,没有第二条。这是第一步,第二步搞社会主义。就是《新民主主义论》那一篇,是个完整的纲领。政治经济文化都讲了,只是没讲军事。

(康生:《新民主主义论》对世界共产主义运动很有意义。我问过西班牙的同志,他们说,他们的问题就是搞资产阶级民主主义,不搞新民主主义。他们那里就是不搞这三条:军队、农村、政权。完全服从苏联外交政策的需要,什么也搞不成。)

这是陈独秀那一套!

(康生:他们说,共产党组织了军队.交给人家。)

没有用。

(康生:也不要政权。农民也不发动。那时苏联同他们讲,如果搞无产阶级领导,英法就会反对,对革命不利。)

古巴呢?古巴恰恰是又搞政权,又搞军队,又发动农民。所以就成功了。

(康生:他们打仗也是打正规仗,资产阶级的一套,死守马德里。一切服从苏联外交的一套。)

第三国际还没有解散,我们没有听第三国际的。遵义会议就没有听,长征把电话丢了,听不到。后来四二年整风,到“七大”的时候才作出决议,《关于若干历史问题的决议》。纠正“左”的都没有听。教条主义那些人根本不研究中国特点。到了农村十几年,根本不研究农村土地、生产关系和阶级关系。不是到农村就懂得农村。要研究农村各阶级、各阶层关系。我花了十几年功夫,才搞清楚。茶馆、赌场,什么人都接近、调查。一九二五年我搞农民运动讲习所,作农村调查。我在家乡找贫苦农民调查,他们生活可惨,没有饭吃。有个农民,我找他打骨牌(天、地、人、和、梅十、长三、板凳),然后请他吃一顿饭。事先事后,吃饭中间,同他谈话,了解到农村阶级斗争那么激烈。他愿意同我谈,是因为,一把他当人看,二请他吃顿饭,三可以赢几个钱。我是老输,输一、二块现洋,他就很满足了。有一位朋友,解放后还来看过我两次。那时候有一回,他实在不行了,来找我借一块钱,我给了他三块,无偿援助。那时候这种无偿援助是难得有的。我父亲就是认为,人不为己,天诛地灭。我母亲反对他。我父亲死时送葬的很少,我母亲死时送葬的很多。有一回哥老会抢了我家,我说,抢得好,人家没有嘛。我母亲也很不能接受。长沙发生过一次抢米风潮,把巡抚都打了。有些小贩,湘乡人,卖开花蚕豆的,纷纷回家,我拦着他们问情况。乡下青红帮也开会,吃大户,登了上海《申报》,是长沙开兵来才剿灭的。他们纪律不好,抢了中农,所以自己孤立了。一个领袖左躲右躲,躲到山里,还是抓去杀了。后来乡绅开会,又杀了几个贫苦农民,那时还没有共产党,是自发的阶级斗争。

社会把我们这些人推上政治舞台。以前谁想到搞马克思主义?听都没听说过。听过还看过的是孔夫子、拿破仑、华盛顿、彼得大帝、明治维新、意大利三杰,就是资本主义那一套。还看过富兰克林传,他穷苦出身,后来变成文学家,还试验过电。(陈伯达:富兰克林最先提出人是制造工具的动物这一说法。)他说过人是制造工具的动物。从前说人是有思想的动物,“心之官则思”。说“人为万物之灵”,谁开会选举的?自封的。后来马克思提出,人能制造工具,人是社会的动物。其实,人至少经过了一百万年才发展了大脑和双手,动物将来还要发展。我不相信就只有人才能有两只手,马、牛、羊就不进化了?只有猴子才进化?而且猴子中又只有一类猴子能进化,其它就不能进化?一百万年,一千万年以后还是今天的马、牛、羊?我看还是要变,马、牛、羊、昆虫都要变。动物就是从植物变来的,从海藻变来的。章太炎都知道。他的与康有为论革命书中,就写了这个道理。地球原来是个死的地球,没有植物,没有水,没有空气。不知几千万年才形成了水,不是随便一下就由氢氧变成了水,水有自己的历史。以前连氢、氧二气都没有,产生了氢和氧,然后才有可能两种原素化合成水。

要研究自然科学史,不读自然科学史不行。要读些书。为了现在斗争的需要去读书,与无目的地去读书,大不相同。傅鹰讲氢和氧经过千百次化合成水,并不是简单地合二而一,他这话,讲的倒是有道理的,我要找他谈谈。(对××讲)你们对傅鹰也不要一切都反对。

历来讲分析、综合讲得不清楚。分析比较清楚,综合没讲过几句话。我曾找艾思奇谈话,他说现在只讲概念上的分析、综合,不讲客观实际的分析、综合。

我们怎样分析、综合共产党与国民党、无产阶级与资产阶级、地主和农民、中国人民和帝国主义?拿共产党和国民党来说,我们怎样进行分析和综合?我们分析:无非是我们有多少力量,有多少地方,多少人,多少党员,多少军队,多少根据地。如延安之类。弱点是什么?没有大城市,军队只有一百二十万,没有外援,国民党有大量外援。延安同上海比,延安只有七千人,加上机关、部队一共二万人,只有手工业和农业,怎能同大城市比?我们的长处是有人民支持。国民党脱离人民。你地方多,军队多,武器多,可是你的兵是抓来的,官兵之间是对立的。当然他们也有相当大一部分很有战斗力的军队,并不是都一打就垮。他们的弱点就在这里,关键就是脱离人民,我们联系人民群众,他们脱离人民群众。

他们宣传共产党共产共妻,一直宣传到小学里。编了歌:“出了朱德毛泽东,杀人放火样样干,你们怎么办?”叫小学生唱。小学生一唱,就去问他们的父母兄弟,反倒替我们作了宣传。有个小孩听了问他爸爸,他爸爸说,你不要问,将来你长大以后,你自己看就知道了。这是个中间派。又去问他叔叔,叔叔骂了他一顿,说“什么杀人放火?你再问,我揍你!”原来他叔叔是个共青团。所有的报纸、电台,都骂我们,报纸很多,一个城市几十种,每一派办一个,无非是反共。老百姓都听他们的?没有那回事。中国的事我们经验过,中国是个麻雀,外国也无非是富人和穷人,反革命和革命,马列主义和修正主义。切不要以为反革命宣传会人人信,会一起来反共。我们不是都看报纸吗?也没有受他影响。

《红楼梦》我读了五遍,也没有受影响。我是把他当作历史读的,开头当故事读,后来当历史读。什么人看《红楼梦》都不注意第四回,其实这一回是《红楼梦》的总纲。还有冷子兴演说荣国府,“好了歌”和注。第四回“葫芦僧判断葫芦案”,讲护官符,提出四大家族:贾不假,白玉为堂金作马;阿房宫,三百里,住不下金陵一个史;东海缺少白玉床,龙王来请金陵王;丰年好大“雪”(薛),珍珠如土金如铁。”《红楼梦》里四大家族都写到了,《红楼梦》阶级斗争激烈,有好几十条人命。而统治者也不过二、三十个人(有人算了说是三十三个人),其它都是奴隶,三百多个,鸳鸯、司棋、尤二姐、尤三姐等等。讲历史不拿阶级斗争观点讲,就讲不清楚。只有用阶级分析,才能把它分析清楚。《红楼梦》写出来,有二百多年了,研究《红楼梦》的到现在还没有搞清楚,可见问题之难。有俞平伯、王昆仑,都是专家,何其芳也写了个序,又出了个吴世昌,这是新红学,老的还不算。蔡元培对《红楼梦》的观点是不对的,胡适的看法比较对一点。

怎么综合?国民党、共产党,两个对立面,在大陆上怎么综合的,你们都看到了。就是这么综合的:他的军队来,我们吃掉,一块一块地吃。不是杨献珍的合二为一,不是两方面和平共处的综合。他不要和平共处,他要吃掉你。不然,为什么他打延安?陕北除了三边三个县以外,他的军队都到了。你有你的自由,我有我的自由。你二十五万,我二万五千。几个旅,两万多人。分析了,如何综合?你要到的地方你去,我一口一口地吃你。打得赢就打,打不赢就走。整整一个军,从一九四七年三月到一九四八年三月,统统跑光,因为消灭了他好几万。宜川被我们包围,刘戡来增援,军长刘戡打死了,他的三个师长,两个打死,一个俘虏了,全军覆没,这就综合了。所有的枪炮都综合到我们这边来了,士兵也都综合了:愿留下的留下,不愿留下的发路费。消灭了刘戡,宜川城一个旅不打就投降了。三大战役,辽沈、淮海、平津战役,怎么综合法,傅作义就综合过来了。四十万人,没有打仗,全部缴枪。

一个吃掉一个,大鱼吃小鱼,就是综合。从来书上没有这样写过,我的书也没有写。因为杨献珍提出合二而一,说综合是两种东西不可分割地联系在一起。世界有什么不可分割的联系?有联系,但总要分割的,没有不可分割的事物。我们搞了二十几年,我们被敌人吃掉的也不少。红军三十万军队,到了陕甘宁只剩下两万五,其他的有被吃掉了的,逃跑了的,打散了的,伤亡了的。

要从生活中来讲对立统一。

(康生同志:只讲概念,不行。)

分析时也综合,综合时也分析。

人吃动物,吃蔬菜,也是先加以分析。为什么不吃砂子,米里有砂就不好吃。为什么不吃马、牛、羊吃的草,只吃大白菜之类?都是加以分析。神农尝百草,医药有方。经过多少万年,分析出来,那些能吃,那些不能吃才搞清楚。蚱蜢、蛇、乌龟王八可以吃,螃蟹、狗、下水能够吃。有些外国人就不吃。陕北人就不吃下水,不吃鱼。陕北猫也不吃。有一年黄河发大水,冲上来几万斤鱼,都作肥料了。

我是土哲学,你们是洋哲学。

(康生同志:主席能不能讲讲三个范畴的问题。)

恩格斯讲了三个范畴,我就不相信那两个范畴。(对立统一是最基本的规律,质量互变是质和量的对立统一,否定之否定根本没有。)质量互变,否定之否定同对立统一规律平行的并列,这是三元论,不是一元论。最基本的是一个对立统一。质量互变就是质和量的对立统一。没有什么否定之否定,肯定、否定、肯定、否定……事物发展,每一个环节,即是肯定,又是否定。奴隶社会否定原始社会,对于封建社会,它又是肯定,封建社会对奴隶社会是否定,对资本主义社会又是肯定,资本主义社会对封建社会是否定,对社会主义社会又是肯定。

怎么综合法?难道原始社会和奴隶社会并存?并存是有的,只是小部分。作为总体,是要消灭原始社会。社会发展也是有阶段的,原始社会又分好多阶段,女人殉葬那时还没有,但是服从男人。先是男人服从女人,走到反面,女人服从男人。这段历史还搞不清楚,有一百多万年。阶级社会不到五千年。什么龙山文化,仰韶文化,原始末期有了彩陶。总而言之,一个吃掉一个,一个推翻一个,一个阶级消灭,一个阶级兴起,一个社会消灭,一个社会兴起。当然在发展过程中,不是很纯的,到了封建社会里还有奴隶制,主体是封建制,还有些农奴,也有些工奴,手工业的。资本主义社会也不那么纯粹,再先进的资本主义社会,也有落后部分。如美国南部的奴隶制,林肯消灭奴隶制,现在黑人奴隶还有,斗争很激烈,二千多万人参加,不少。

一个消灭一个,发生、发展、消灭,任何东西都是如此。不是让人家消灭,就是自己灭亡,人为什么要死?贵族也死,这是自然规律。森林寿命比人长,也不过几千年。没有死,那还得了。如果今天还能看到孔夫子,地球上的人就装不下去了,赞成庄子的办法,死了老婆,敲盆而歌。死了人要开庆祝会,庆祝辩证法的胜利,庆祝旧事物的消灭。社会主义也要灭亡,不灭亡就不行,就没有共产主义,共产主义至少搞个百把万、千把万年,我就不相信共产主义就没有质变,就不分质变的阶段了?我不信。量变质,质变量。完全一种性质,几百万年不变了,我不信!按照辩证法,这是不可设想的。就一个原则,各尽所能,各取所需。就搞一百万年,就是一种经济学,你信不信?想过没有?那就不要经济学家?横直一本教科书就可以了,辩证法也死了。

辩证法的生命就是不断走向反面。人类最后也要到末日。宗教家说末日,是悲观主义,吓唬人。我们说人类灭亡,是产生比人类更进步的东西,现在人类很幼稚。恩格斯讲,要从必然的王国到自由的王国,自由是对必然的理解。这句话不完全,只讲了一半,下面的不讲了。单理解就能自由了?自由是必然的理解和必然的改造。还要做工作,吃了饭没事做,只理解一下就行?找到了规律要会用,要开天辟地,破破土,砌房子,开矿山,搞工业。将来人多了,粮食不够,要从矿物里取食品,这就是改造,才能自由,将来就能那么自由?列宁讲过,将来空中飞机像苍蝇一样多,闯来闯去,到处撞怎么得了?怎么调动?调动起来那么自由?北京现在有一万辆公共汽车,东京有十万辆(还是八十万辆)所以车祸多,我们车少再加上教育司机,教育人民,车祸少。一万年以后,北京怎么办?还是一万辆车?会发明新东西,不要这些交通工具,就是人起飞,用简单机器,一飞就飞到一个地方,随便哪里都可以落,单对有必然的理解不行,还要改造。

不相信共产主义社会不分阶段,没有质的变化。列宁讲过,凡事都可以分。举原子为例,他说不仅原子可以分,电子也可以分。可是以前认为不可分。原子核分裂,这门科学还很年轻,才二、三十年,几十年来,科学家把原子核分解,有质子、反质子、中子、反中子、介子、反介子,这是重的,还有轻的。这些发现,主要还是第二次世界大战中间和以后才发展起来。至于电子和原子核可以分裂,那早就发现了。电线里,就是用了铜、铅的外电子的分离。地球三百公里的上空还发现有电离层,那里电子和原子核也分离。电子到现在还没有分裂,总有一天能分裂。庄子说“一尺之棰,日取其半,万世不竭”(《庄子·天下篇》引公孙、龙子语)这是个真理,不信就试试看。如果有竭,就不是科学了。事物总是发展的,是无限的。时间、空间是无限的。空间方面,宏观、微观是无限的,是无限可分的。所以科学家有工作做,一百万年以后还有工作做。我很欣赏《自然科学研究通讯》上坂田昌一那篇基本粒子的文章,以前没有看到过这样的文章,是辩证唯物主义者。他引了列宁的话。

哲学界的缺点是没有搞实际的哲学,而是搞书本的哲学。

总要提出新的东西,不然要我们这些干什么?要后人干什么?新东西在实际事物里,要抓实际事物。任继愈到底是不是马克思主义者?很欣赏他讲佛学的那几篇文章,有点研究,是汤用彤的学生。他只讲到唐朝的佛学,没有触及到以后的佛学。宋朝的明理学是从唐朝禅宗发展起来的,由主观唯心论到客观唯心论。有佛、道,不出入佛、道是不对的。有佛、道,不管它,怎么行?韩愈不讲道理,“师其意,不师其词”,是他的口号,意思完全照别人的,形式、文章改一改。不讲道理,讲一点也基本上是古人的。《师说》之类有点新的。柳子厚不同,出入佛、道,唯物主义。但是,他的《天对》太短了,就那么一点。他的《天对》从屈原《天问》产生出来,几千年来,只有这个人做了《天对》。这么一看,到现在,《天问》《天对》讲些什么,没有解释清楚,读不懂,只知其大意。《天问》了不起,几千年以前,提出了各种问题,关于宇宙,关于自然,关于历史。

(关于合二而一问题的讨论)《红旗》可以转载一些比较好点的东西,写一篇报导。


