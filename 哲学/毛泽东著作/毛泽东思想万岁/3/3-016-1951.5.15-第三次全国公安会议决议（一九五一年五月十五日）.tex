\section[第三次全国公安会议决议(一九五一年五月十五日)]{第三次全国公安会议决议}
\datesubtitle{(一九五一年五月十五日)}


大批应判徒刑的犯人是一个很大的劳动力,为了改造他们,为了解决监狱的困难,为了不让判处徒刑的反革命分子坐吃闲饭,必须立即着手组织劳动改造的工作。

经过此次全国规模的镇压反革命运动以后,尚未破获的特务间谍分子的活动,必须更加隐蔽。因此公安部门必须进行更系统的侦察工作,并教育人民群众多方面的注意防奸工作。

<p align="center">×××</p>

对于有血债或其他最严重的罪行非杀不足以平民愤者和最严重地损害国家利益者,必须坚决地判处死刑,并迅即执行。对于没有血债、民愤不大和虽然严重地损害国家利益但尚未达到最严重程度、而又罪该处死者,应当采取判处死刑,缓期二年执行,强迫劳动,以观后效的政策。特别是对于在共产党内,在人民政府系统内,在人民解放军系统内,在文化教育界,在工商界,在宗教界,在民主党派和人民团体内清出来的应判死刑的反革命分子,一般处决十分之一、二为原则,少余十分之八、九均应采取判处死刑缓期执行强迫劳动以观后效的政策。如此才能获得社会的同情;才能避免我们自己在这个问题上犯错误;才能分化和瓦解敌人有利于彻底消灭反革命势力,又保存了大批的劳动力,有利于国家的生产建设。

<p align="center">×××</p>

此外应明确的规定:凡介在可捕可不捕之间的人一定不要捕,如果捕了就是犯错误;凡介在可杀可不杀之间的人一定不要杀,如果杀了就是犯错误。

<p align="center">×××</p>

目前在全国进行的镇压反革命的运动是一场伟大的激烈的和复杂的斗争。全国各地,已经实行的有效的工作路线,是党委领导,全党动员,群众动员,吸收各民主党派及各界人士参加,统一计划,统一行动,严格的审查捕人和杀人的名单,注意各个时期的斗争策略,广泛的进行宣传教育工作,打破关关门主义和神秘主义,坚决的反对草率从事的偏向。凡是完全遵照这个路线去做的,就是完全正确的。凡是没有遵照这个路线去做的,就是错误的。凡是大体上遵照这个路线,但没有完全遵照这个路线去做的,就是大体上正确但不完全正确的。我们认为这个工作路线是继续深入镇压反革命工作和取得完满胜利的保证。在今后镇反工作中必须完全遵照这个工作路线。其中最重要者为严格审查捕杀名单和广泛地做好宣传教育。做到了这两点,就可以避免犯错误。


