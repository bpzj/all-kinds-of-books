\section[在为八届六中全会作准备的郑州会议上的讲话第一次讲话(一九五八年十一月二日下午)]{在为八届六中全会作准备的郑州会议上的讲话第一次讲话(一九五八年十一月二日下午)}
\datesubtitle{(一九五八年十一月二日)}


人民公社问题。究竟扩大自然经济?还是扩大商品经济?还是两者都扩大?人民公社的经济,主要是自然的,说法不对。徐水办工厂,工人不与农民交换,如何吃饭?它应向两方面发展。它同时要扩大商品交换。不交换,就不能消费,不扩大交换,就不能发工资。京津郊区富裕,就是商品发展能交换。

管理区是专员公署性质的,徐水县都是叫公社,上有管理区,再上是县公社。搞分级管理,级太多不好。有个体制问题,作风问题(两类矛盾)、生产问题(不可单调,要尽量生产能够交换的东西,向全省、全国、全世界交换)。

一穷二白,愈穷愈赞成,现在不是夸富的时候。人民解放军、党是集中了这个意志,军队里头有饭吃,从古以来军队就吃大锅饭,就是集体化、战斗化、军事化。武王伐纣,也是如此。

[陈伯达同志插话:一个县一个公社优越性比较大:(1)统一调配劳动力;(2)统一财政收入;(3)统一产品分配;(4)统一发工资(不同的管理区),在二三年内,工资水平不一定要强求拉平。生产发展可拉平,现在保持一定差别有好处;(5)实行分级管理,发挥各级的积极性。管理还可以分四级;(6)服从国家统一的经济计划和上缴利润的计划。一县一社和一县数社相比较:①统一调配劳动力,联社也可以,但是联社不易统一供给……。]

统一是统其可统者,不过过去中央的办法,是有条件的,无条件的说法是错误的。无条件服从党的领导的说法,也是不对的,不正确的就可以不服从,要破除这个迷信。无论什么时候,都有条件的。要破条件论,不要立条件论。

[陈伯达同志接着说:②联社各分社自负盈亏,但联社也可抽一部分公社积累;③统一调拨商品,一县一社好办,一县数社就难办;④统一财政收支,一县一社在一个县的范围内,可说是有全民所有制的性质,但在一个省的范围内还不能那样说。]

鞍钢一个工人拿八百元,一个工人劳动产值一万八千元,除了七千二百元原料,生产成本,还有一万零八百元,这就是国民收入。分两部分:一部分叫消费,即八百元工资;另一部分为国家作公共积累。以县为范围的公社可说是小全民所有制或大集体所有制。

农业是两条:(1)深耕。可以少种,可以除虫,可以蓄水。深耕细作,少种多收;浅种粗作,多种少收。只要种现有耕地的三分之一,即六亿亩即可。这是第一步,下一步还可以减少。现在亩产二百斤,将来亩产万斤,种那么多干什么。(2)机械化,但不都是拖拉机。要创造新机器,解放人力出来搞工业,主要是工农学兵。

保证劳动者忙时睡六小时,闲时睡八小时,要强迫命令,要保证吃好、睡好。睡觉六小时,吃饭一小时半,工作和休息还有十四个半小时。这一点要下命令,否则就没有完成任务,吃好睡足,效果更好。八小时工作制在英国国会内争论过,有个资本家说。八小时工作制对资本家有利,因为工作效率高,然后才通过。要睡好觉,要实行午休制。

饭要吃饱,要吃好,花样要多些,要吃热饭。一个月打一次牙祭。早饭是上午的劳动力,中饭是下午的劳动力,要把这件事当一件大事办。像现在这样苦战,不良后果可能在几年后才能看出来。

还有一个,带好小孩子。小孩子占人口的百分之二十五。什么人也不愿去托儿所,食堂也不派有能力的人去办,这叫见物不见人(见钢不见娃娃)。把小孩子带好,要比家庭管的还好。吃饭、睡觉也是人的问题,搞不好就要垮台。要派有能力的人去干。

北戴河会议以来,两个月,人海战术,大搞钢铁,组织了队伍,初步学会了技术,从农业中化出几千万工人是一件大事,是新社会分工(不完全合理)。明年要定点,搞综合企业(煤钢联营),提高技术,“小土洋”、“中土洋”,结合“群”。挖矿,挖煤,冶炼,运输,要组织的比较合理。还有许多人要转到其他“小土群”方面,要以“小土群”来搞铝、煤、石油、机械。用机器再把农业劳动力解放出来。农业将来以妇女、小孩负担为主,男子为辅。钢铁以男子为主,妇女为辅。

公社要多搞商品生产,现在好像自给自足才是名誉的,而生产商品是不名誉的,这不好。要扩大商品生产,扩大社会交换。

肥料,应该主要用人粪尿、塘堤、沟泥、墙土,炕土、厩肥、绿肥、有机肥料。少搞洋化肥,多搞土化肥。亩产×千斤,根本不依靠硫酸铵。要搞一点,但不作为主要的。拖拉机要搞一点,但也不是主要的。

以公社,联社为单位,搞工农业同时并举,搞使用价值与交换价随同时发展,因此要大修交通。

按人口计算,英国五千万人,二千五百万吨钢。我国七亿人口,要三亿吨钢,要那么多钢铁干什么?修桥补路是一个出路。


