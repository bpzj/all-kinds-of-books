\section[接见非洲和拉丁美洲青年学生代表团时的谈话(一九六四年八月二十五日)]{接见非洲和拉丁美洲青年学生代表团时的谈话}
\datesubtitle{(一九六四年八月二十五日)}


\begin{list}{}{
    \setlength{\topsep}{0pt}        % 列表与正文的垂直距离
    \setlength{\partopsep}{0pt}     % 
    \setlength{\parsep}{\parskip}   % 一个 item 内有多段,段落间距
    \setlength{\itemsep}{\lineskip}       % 两个 item 之间,减去 \parsep 的距离
    \setlength{\labelsep}{0pt}%
    \setlength{\labelwidth}{3em}%
    \setlength{\itemindent}{0pt}%
    \setlength\listparindent{\parindent}
    \setlength{\leftmargin}{3em}
    \setlength{\rightmargin}{0pt}
    }
\item[\textbf{主席:}] 欢迎你们。\\
你们是哪些国家的朋友?(这时外宾起立,分别向主席作自我介绍)\\
你们是正在念书的,还是读完了?

\item[\textbf{日拉尔:}] 已经念完了。(其他外宾答:还正在念书。)

\item[\textbf{主席:}] 你们有什么问题要问的吗?

喝茶。愿抽烟的抽烟,不抽烟的不抽烟。(全体笑)

\item[\textbf{谢尔盖·巴里奥:}] 我想了解一下毛泽东同志对我们秘鲁的看法,希望我们秘鲁成为什么样的国家,对我们秘鲁有什么希望。主席:先要问你,你有什么意见。(主席笑)你们国家的情况,我不太清楚。

\item[\textbf{谢尔盖·巴里奥:}] 毛泽东同志讲对我们国家的情况了解得少,我知道是这样的情况,因为我们国家离你们的国家很远,而且我们的政府禁止人民到中国来,所以到中国来的秘鲁的人是很少的。不管怎样,我们国家在拉了美洲是有很悠久的历史的,并且在将来,毛泽东同志很快可以听到,它能够在拉丁美洲发出一个很重要的声音。

\item[\textbf{主席:}] 什么很重要的声音?

\item[\textbf{巴里奥:}] 革命的声音。

\item[\textbf{主席:}] 革什么样的命?

\item[\textbf{巴里奥:}] 是反对资产阶级的、帝国主义的、资本主义的,反对压迫阶级的,反对压迫民族的革命。

\item[\textbf{主席:}] 一切资产阶级都反对吗?你们国家有没有民族资产阶级?有没有爱国的,反对帝国主义的民族资产阶级啊?

\item[\textbf{巴里奥:}] 在我们拉丁美洲的大部分国家,特别是在西部和南部的一些国家,进步的资产阶级的力量是很弱的,大部分是跟帝国主义的利益,跟大资产阶级的利益紧密联系的。因此,它们和革命的利益结成联盟的可能性是很小的。

\item[\textbf{主席:}] 我赞成你们反对帝国主义,反对给帝国主义当走狗的那些人。其他的人,首先是工人、农民,然后是爱国的民族资产阶级,可能有少数进步的民族资产阶级,这种人是比较少的,同这些人应该结成统一战线。这样,反对的对象,革命的对象,就比较少一些,革命的力量就比较大一些。你赞成不赞成啊?因为压迫人的人在世界上总是少数,一百个人中间只有几个,这样就可以团结百分之九十以上的人。这是我们共同的问题,是整个拉丁美洲、非洲和亚洲共同的问题。在欧洲、北美大体上也是这样。譬如在美国,譬如在英国,工党、社会民主党,社会党,美国那些大工会的领导人,他们是帮助资产阶级的。工人阶级有很大一部分现在还不觉悟,但是将来他们会觉悟的。所以我们只反对帝国主义者同它在各国的走狗,其他的人,我们不反对。\marginpar{\footnotesize 164}我们并不反对整个美国人嘛。他(指谢尔盖·巴里奥)又不赞成了。\\
还有什么问题?

\item[\textbf{阿里乌:}] 我们是黑非留法学联、西非学生总会的代表,我们想借这个机会向毛泽东主席讲几句话,并且通过毛泽东主席向中国人民表示敬意。我们想请毛泽东主席知道非洲的情况,并且就您知道的一些问题给我们一些答复,正像刚才毛泽东主席给秘鲁的朋友的答复一样。

我们今天非常高兴能够得到毛泽东主席的接见,我们愿意借这个机会,对毛泽东主席和中国人民表示崇高的敬意。在地理上,非洲和中国是离得很远的,但是我们共同反对帝国主义,反对老殖民主义和新殖民主义,多年的斗争把我们同中国人民结成了深厚的友谊。而现在,我们在中国虽然只呆了几个星期,但是我们看了很多东西,我们看到中国正在热情地进行建设,中国的人民群众也在参加建设工作。非洲曾经是许多国家的殖民地,譬如法国的殖民地,法国已经给了这些国家许多独立,但是人民并没有真正地参加建设工作。我们感谢毛泽东同志和中国的其他领导人对非洲国家的一贯支持。我们非洲的学生正在进行着反对帝国主义的斗争,正因为我们进行了这个斗争,我们才同中国学生联合会建立了关系,正因为这个斗争今天才使我们来到中国。通过这次座谈能够使毛泽东同志了解非洲的情况,同时希望毛泽东同志对非洲的情况发表一些意见。

\item[\textbf{主席:}] 还有非洲朋友要问吗?

\item[\textbf{艾克洛:}] 我认为,我们非洲现在面临着另外一种危险,这是一个现实问题——毛泽东主席是不是能够同意——社会主义阵营中一个国家的问题,这个国家过去曾经是我们很好的同盟者,现在是修正主义。过去我曾经有机会在法国念书,现在又在一个社会主义国家——捷克念书,在那里,我每天都看到修正主义直接的影响。我们在捷克已经念了三年书,但是在这三年内,他们要求我们不要谈反对帝国主义的斗争,譬如在一些会议上,我们所见到的就是让大家跳舞,不帮助提高我们的政治觉悟,就是应该怎样武装起来,争取我们的自由。我可以举一个例子。委内瑞拉的同学过去每年都组织一些讨论委内瑞拉国内斗争情况的会议,自从去年以来,委内瑞拉和捷克建立了商务关系以后,就禁止再组织这样的会议了。他们对委内瑞拉的朋友说:你们在这里可以读书,回国以后不要再参加解放斗争。还有另外一个例子。喀麦隆人民联盟的留学生,自从捷克和阿希乔政府建立了外交关系以后,在那里的喀麦隆的留学生就不能再举行任何集会了。不论在布拉格,在莫斯科,都是一样。去年我曾被邀请去参加喀麦隆人民联盟的一次支部大会,他们的负责人曾同当局协商,但是并没有得到结果,后来这个集会只是在学生的一个房间里举行了。当然,现在非洲在国外的留学生,有些还没有觉悟。他们仍然认为,不谈政治,不谈反帝,是一件好事。我还可以举一个例子。我曾同捷克和平委员会的一个人谈话。他说,现在已经争取了一切行动,来反对黑非留学生联合会,但是我们认为,黑非留法学生联合会是一个先进组织,捷克当局怎么能够说这样的话呢?从这些例子可以看出,我们非洲正面临着一种修正主义的危险。现在的问题是不是要进行国际共产主义运动的问题,是不是我们大家都愿意参加这个斗争的问题,而他们有些人却把现在国际共产主义运动的这些争论说成是北京和莫斯科之争。我们认为,这不是北京和莫斯科之争,而是世界人民要不要革命的问题,\marginpar{\footnotesize 165}是不是要同帝国主义妥协的问题。这是我自己的看法,不知道毛泽东主席是不是会同意我的看法。我们认为,这不是北京和莫斯科之间的争吵,不是两个国家首都之间的争吵,而是世界革命所面临的问题。正像秘鲁的同志已经讲过的一样,这是反对帝国主义、殖民主义、新殖民主义和各国反动派斗争的问题,也就是同帝国主义作斗争的问题,而事实上,我们非洲已经遭到了帝国主义的统治,现在是不是要同帝国主义合作、同帝国主义共处的问题。

\item[\textbf{主席:}] 帝国主义是压迫各国人民的一些集团,各国被压迫人民怎么能够跟它们和平共处呢?帝国主义,新殖民主义,老殖民主义的问题是各国走狗的问题。不管那些人如何,如果不反对他们,就无所谓革命,就无所谓革命的胜利。不谈政治,单跳舞,是不能打倒帝国主义的。(众笑)修正主义要你们服从他们跟帝国主义妥协的路线,它也要我们服从,它要全世界各国革命的人民都服从,我们是不服从的。我们也不服从帝国主义,也不服从新、老殖民主义,也不服从修正主义。也不服从它们各国的走狗。譬如在中国就有那么一个走狗,顶著名的人物是蒋介石,我们能够跟蒋介石合作吗?蒋介石现在在大陆上有他自己的朋友,就是地主阶级的残余,资产阶级的残余,同这些人不能合作,要教育他们,在劳动中改造他们。如果他们要造反,譬如破坏,烧房子,破坏牲畜,搞投机倒把,杀人,暗杀革命者,我们必须进行镇压。我们的方针就是这样,比较简单明了,没有什么吞吞吐吐。无论见效的,没有见效的,只要他反对我们,我们就反对。

你们知道,譬如阿尔及利亚的革命,古巴的革命、越南南方的革命,我们都是公开支持的,刚果(利)的武装斗争,我们也是公开支持的,冲伯就是帝国主义的走狗,我们不跟他建立外交关系,我们站在刚果(利)人民的一边。譬如加纳,我们支持加纳人民的斗争。帝国主义者两次暗杀他们的总统,我们是反对那种惨无人道的暗杀行为的。

整个拉丁美洲的革命是有希望的,不仅你们秘鲁,你(指谢尔盖·巴里奥)不是问我秘鲁的前途怎么样吗?秘鲁的前途和整个拉丁美洲的前途一样,是要用革命斗争去推翻帝国主义同它的走狗。我说的是大的、忠实的走狗,而不是跟帝国主义联系较少的那些人。这样,我们的统一战线反而会扩大一些。

还有问题吗?

\item[\textbf{埃内斯加:}] 我们在这里代表拉丁美洲革命组织的海地、秘鲁、哥伦比亚、委内瑞拉的代表首先向毛泽东主席表示感谢,感谢您接见了我们,并且通过您表达我们对中国人民为反对帝国主义和走狗所进行的长期的革命斗争表示崇高的敬意。

\item[\textbf{主席:}] 谢谢。

\item[\textbf{埃内斯加:}] 我们对中国人民进行社会主义建设所作的努力也表示敬意。同时,我们代表海地、秘鲁、哥伦比亚、委内瑞拉的革命组织对中国人民和中国政府对我们这些国家的解放斗争所给予的支援表示感谢。中国人民和中国政府对亚洲、非洲、拉丁美洲其他国家的解放斗争也给予了同样的支援。我们深信,中国一定会站在我们反对美帝国主义,争取民族解放斗争一边的。

我们同意在反对帝国主义的斗争中应该团结尽可能多的各阶层的人民。我们革命运动组织的总路线,也就是要建立一个最广泛的统一战线,当然我们也不能不注意到,\marginpar{\footnotesize 166}在我们国内要建立这样一种广泛的统一战线是有困难的。古巴革命给我们各国人民的革命提供了一个榜样。古巴的人民使我们拉丁美洲广大人民认识到,他们可以进行反对帝国主义的斗争,并且也使那些资产阶级的反动阶层懂得,这些革命对他们来说意味着什么。现在我们所进行的反对帝国主义的革命运动是有深刻的政治意义的,现在这些革命运动已经在马克思列宁主义路线的指导下进行了。因此,这些资产阶级就要选择,要不就投靠工人、农民,参加反对帝国主义、争取民族解放的斗争,要不就跟帝国主义一块去平分利润。

我们相信,最后由于工人和农民强大的联盟,这些资产阶级将不再为帝国主义的利益服务,而投到民族解放斗争中去,同时,这也包括一些中间的阶层,主要是指进步的知识分子。有些国家的困难会多一点,有些国家的困难会少一点。我们也深信最后胜利将属于我们各国人民。

\item[\textbf{主席:}] 讲得对,我赞成,最后胜利总是属于全世界各国人民的。这已有许多证据。譬如古巴不是胜利了吗?当然还没有最后胜利,中国不是胜利了吗?也没有最后胜利,最后胜利要全世界帝国主义倒下去了,全世界各国人民都翻身了。

我们周围的许多国家都有美帝国主义的军事基地,我们的台湾还没有解放。我们现在这样一种状况就算最后胜利了吗?没有。这是几十年的事情。美国帝国主义不打倒,这些军事基地不撤走,日本人民不翻身,南朝鲜、南越、菲律宾、柬埔寨、老挝、泰国、马来亚这些国家不把帝国主义赶走,不把本国的垄断资本或者是亲帝国主义分子打倒,我们这个国家也不能得到最后解放。

拉丁美洲、非洲和整个亚洲,还有欧洲、北美、大洋洲如果不解放,一个国家或者少数国家解放了,最后解放是不可能的,算是暂时地解放了,譬如中国、古巴、阿尔及利亚、北越、北朝鲜。但是,我们面临着很大的敌人。所以我们跟你们要团结起来,站在一条战线上,向共同的敌人作斗争。

在座的各国朋友的思想也一致嘛?

\item[\textbf{×××:}] 不一致。有马列主义者,有民族主义者。

\item[\textbf{主席:}] 不完全一致不要紧,有些相信马克思主义的人,有些现在还不相信,甚至有些人信宗教,但是,我们在反对帝国主义及其走狗这一点上团结起来。现在帝国主义的头子是谁呢?就是美国帝国主义。当然,在非洲说来,法国、英国、比利时、葡萄牙、西班牙这些国家是有影响的。在拉丁美洲,美国的影响是主要的,我们在反对帝国主义的统一战线中可以团结起来。

修正主义自己不反对帝国主义,还妨碍我们反对帝国主义,那我们也不赞成,要批评他们。现在帝国主义和修正主义仍在欺骗人民,因此我们要做批评工作。刚才那位朋友(指艾克洛)讲得好,他就做了批评工作。

你(指艾克洛)是那个国家的?

\item[\textbf{艾克洛:}] 是多哥的。

\item[\textbf{主席:}] 多哥在哪里?

\item[\textbf{王××:}] 在西非。

\item[\textbf{主席:}] 是属于哪个国家的?

\item[\textbf{×××:}] 原来是法国的殖民地。\marginpar{\footnotesize 167}

\item[\textbf{主席:}] 你们国家独立了没有?

\item[\textbf{艾克格:}] 名义上独立了,现在所谓的独立,就是有一个国歌,有一个国旗。

\item[\textbf{主席:}] 你们认识了这一点。所谓名义上的独立和实际上的独立有区别嘛!

要做群众工作。知识分子如果不同工人、农民结合,不对工人,农民做工作,团结工人、农民,而是脱离工人、农民,那就不好了。知识分子不是一个阶级,它不属于无产阶级,就属于资产阶级,它不为无产阶级服务,就为资产阶级服务,它可以替这个阶级服务,也可以替那个阶级服务。譬如我们中国的北京有一个北京大学,你们去看了没有?

\item[\textbf{×××:}] 没有去,因为放假了。

\item[\textbf{科西·加普逊:}] 访问过人民大学。

\item[\textbf{×××:}] 在上海看了什么大学?(答:没有看。)

\item[\textbf{日拉尔:}] 在西安看了交通大学。

\item[\textbf{主席:}] 无论那个城市的大学、中学、小学,那里的教授、教员,以及行政工作人员过去都是国民党的,很少有我们的教授,很少有我们的教员,那些人都是替国民党服务的,都是亲帝国主义的,有些亲日本帝国主义,有些亲美帝国主义,有些亲法国帝国主义,有些亲德国的,有些亲英国的。解放的时候,只有很少的人跑掉了,百分之九十以上的教授、教员都留下来了,替我们服务。现在有些人已经进步了,赞成马克思主义了,有些人还处于中间状态,是中间派,此外有少数人思想很右,他们的脑筋还是旧的,大概占百分之几的人数,他们赞成修正主义,不那么公开讲就是了。有极少数的人希望蒋介石再回来,社会就是这样复杂的,但是不妨碍大局,因为左派和中间派联合起来占百分之九十以上。

你们会问,为什么中国解放十五年了,有许多人还是中间派,有一部分还是右派呢?外国人说我们“洗脑筋”,为什么这些人的脑筋还没有洗好呢?(众笑)思想工作就是这样不容易做的,需要一定的时间,不能强迫他们洗脑筋,(众笑)只能劝说他们,只能说服他们,不能压服他们,要他自己遂步了解,逐步觉悟起来。他们这些人是不跟工人、农民接近的,他们脱离群众。现在我们想些办法,使他们同工人、农民接近。

知识分子脱离了群众就没有什么用。这是我们的经验,也是列宁的经验,也是马克恩、恩格斯的经验。所以,我们希望你们不要脱离人民群众,不要脱离你们国家占最大多数人口的工人和农民。

要做群众工作,就要交朋友。如果没有工人、农民做朋友,你就不了解工人、农民的思想状况。这就是说要做调查研究工作。知识分子要接近群众,做调查研究,是不那么容易的。第一条,知识分子过惯了城市生活,他就不想到乡下去做调查研究工作,赶也赶不下去。(众笑)他们成了习惯。第二条,到乡下去做调查研究,去了并不等于真正交好了朋友。因为知识分子有知识分子的派头,摆一付老爷架子,农民看不惯,摆一付老爷架子去接近工人,工人也看不惯。开始他们弄不清楚,不知道你们是帮助他们的,还是伤害他们的。我自己就有这样的经验,要经过一个过程。譬如讲组织工会,举行罢工,经过一个过程,工人才相信,你是帮助他的。而不是伤害他的。同农民说话,绝不能摆起一付知识分子的架子,看不起他们。\marginpar{\footnotesize 168}我曾经说过这样的话,知识分子就某一点说来,是比较最有知识的人,但是不如工人、农民的知识多。

因为我们读的书,无论你读的是什么书,读的是马克思主义的书也好,资本主义的书也好,或者封建孔夫子的书也好,都是书本上的东西。这些书本都不教我们怎样革命,只有马克思主义的书教我们怎样革命,但是也不等于读了书就知道如何革命了。读革命的书是一件事情,实行革命又是一件事情。我借这个机会讲一点我的经验,也许你们不赞成,将来可能有一天你们会想起我今天讲的这些话。

解放以前,中国只有几百万工人,大约有四百万工人,有几万万农民。剥削者和压迫者全中国只有几千万,占百分之五左右,大约只有三千多万。那么我们站在那一边呢?是站在少数剥削者方面,还是站在几万万农民同几百万工人方面呢?这个问题在开头我是没有搞清楚的,因为我读的是孔夫子的书,资本主义的书,后来读了马克思主义的书,又组织了共产党,这就下决心赞成马克思主义了,世界观就改变了,由唯心主义者变成了唯物主义者,逐步地变为彻底的唯物主义者。什么叫彻底的唯物主义者呢?就是辩证唯物主义者和辩证的历史唯物主义者。

讲多了,到时间了。

\item[\textbf{科西·加普逊:}] 我的讲话我相信所有的同志都会赞同的。就是我们感到今天能够同这样一位伟大的人物会见,是一个非常难得的机会,我们对毛泽东同志所说的一切话,都很受感动,也非常感谢。我们听到他能够在这么短的时间对整个世界局势作出结论,斗争应该如何办,提供了许多建议,对这点我们的印象是非常深刻的,也很感动。毛泽东同志还给我们说,不管我们读多少革命的书籍,谈多少革命的话,如果我们不深入到农村中去,向群众学习,同群众打成一片,我们就不能成为真正的革命者。同时,我们感到毛泽东同志对非洲的事物是非常关心的。我们感到中国一贯是帮助非洲人民进行革命,帮助世界各国人民进行革命的。我们可以向毛泽东主席表示,我们将高举革命的旗帜,一直到全世界各国人民最后摆脱帝国主义和新老殖民主义的枷锁为止。谢谢主席。

\item[\textbf{主席:}] 就谈到这里吧!还有什么问题?

\item[\textbf{外宾:}] 希望毛泽东主席在《毛主席诗词》上签上您的名字。

\item[\textbf{主席:}] 可以。

(外宾请主席签字。最后外宾同主席握手告别。)
\end{list}
