\section[在军委扩大会议小组长座谈会上的插话(根据记录整理为九条)(一九五八年六月廿三日)]{在军委扩大会议小组长座谈会上的插话(根据记录整理为九条)(一九五八年六月廿三日)}
\datesubtitle{(一九五八年六月)}


一、人民解放军有没有教条主义呢?我在成都会议上说过,搬是搬了一些,但建军基本原则坚持下来了。现在有四种说法:一种说是没有,一种说是有,一种说是很多,一种说是相当多。说没有教条主义是不符合事实的。究竟有多少,这次军委会议要实事求是地加以研究,不要夸大,也不要缩小,要坚持真理,修正错误。学习苏联的方针是坚定不移的,因为它是第一个社会主义国家,但一定要有选择地学,因此就要坚决反对教条主义,打倒奴隶思想,埋葬教条主义。

二、十大军事原则,是根据十年内战,抗日战争,解放战争初期的经验,在反攻时期提出来的,是马列主义普遍真理与中国革命战争实践相结合的产物。运用了十大原则,取得了解放战争,抗美援朝战争的胜利(当然还有其他原因)。十大军事原则目前还可以用,今后有许多地方还可以用,但马列主义不是停止的,是向前发展的,十大军事原则也要根据今后战争的实际情况,加以补充和发展,有的可能要修改。

三、现在形势很好,特别是国内形势很好。现在我们要抓三样东西,粮食,钢、机械。×××同志打电话给我,说华东五省(山东、江苏、浙江、安徽、福建)今年可以增产粮食××亿斤,如果江西、湖南、湖北、河南作一个单位,两广、云、贵、川作一个单位,也各增产××亿斤,这样即使不算东北、华北、西北,今年就可以增产××亿斤。钢一九六二年可以达到××万吨到××万吨。赶上美国不要×年,×年×年就可以了。×年可以赶上苏联,×年,最多×年就可以赶上美国。有了粮食、钢、机械、十年内又不打仗的话,人民解放军大有希望,威力会大大地加强。到那时候,肖劲光同志的愿望就可以实现了。现在“小米加步枪”的经验还是主要的,新的还没有,就把小米加步枪否定了是错误的。当然停留在旧阶段也是不对的。

四、肯定讲我们要学习苏联,因为苏联是第一个社会主义国家。我们过去学了,现在要学,将来也还要学。苏联的经验有三种:一种是好的,我们用得上的,就要取“经”;第二种是不好不坏的,要取其好的一部分;第三种是坏的,也可以研究,引以为戒。所以对苏联的经验要有选择地学。苏联顾问、专家,大多数是忠心耿耿的,对我们的帮助是很大的,应当肯定。我们对他们应当做兄弟一样看待,不要把他们当做客人。他们绝大多数是好的,坏的只是少数,个别的,也有九个指头与一个指头的问题。他们就是有一个缺点:政治水平不高。有问题时要好好同他们研究,商量问题时要多问几个“为什么”。要团结诚恳对待他们,也要有批评,有斗争。这一方面第二机械工业部做得很好,向专家宣传我们的总路线,专家也说要向我们学习,批评我们有些人有依赖思想。要注意不要因为反教条主义而否定一切。

五、我们社会主义建设是三个“并举”。斯大林只强调一面,强调工业,忽视了搞农业;强调集中,忽视分权;强调大型的,忽视中小型的。我们比斯大林要完整。苏联现在有两个地方有改进,注意了农业,注意了分权,但他们还是不大注意走群众路线,不提倡搞中小型。我们还有一条,就是洋办法和土办法结合,人民解放军搞现代化,搞洋办法,也应该搞点土办法,例如民兵是土办法。土办法发展以后,也可以变成洋办法。“小米加步枪”同现代化可以结合起来。

六、军委的领导方法,工作方法一定要改进,方法就是一年去摸四次,像地方一样,有系统地去摸。住北京的人,官气多一些,政治少一些(现在有改进),因此要注意这个问题。军委领导要改进,下面有批评,我看这些是好的,要求高一些,都是为了改进军委领导。今后军委这样的会要每年开一次。今后中央开代表会议应多吸收军队同志加强,如地方上应相当地委书记以上参加者,军委师以上党委书记可以参加。你们不是说文件看不到吗?这次给你们多看一些。你们参加省委没有?(刘培基答:我们有八个人参加福建省委。)福建这个问题解决了,我知道的。

七、现在学校奇怪得很,中国革命战争自己的经验不讲。专门讲“十大打击”(指苏军在第二次世界大战后期的反攻)而我们几十个打击也有,却不讲。应该主要讲自己的,另外参考人家的。

八、王明至今还向苏联告状,告我三个东西:一是反对共产国际;二是搞个人崇拜;三是强迫百分之八十以上的干部都做检讨。×××说:共产国际有错误为什么不能反?×××就是参加过共产国际的,他说得很对,××同志的这个报告也要印给大家看看。

九、第一个五年计划建设资金是××亿,由于没有经验,加上教条主义地照搬,浪费了一半,本来可以搞××亿的事业,可以做更多的事而没有做到。好处是取得了经验教训。


