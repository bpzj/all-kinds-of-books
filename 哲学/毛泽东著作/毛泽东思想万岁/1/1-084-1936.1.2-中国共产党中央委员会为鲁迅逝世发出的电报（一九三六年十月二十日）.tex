\section[中国共产党中央委员会为鲁迅逝世发出的电报(一九三六年十月二十日)]{中国共产党中央委员会为鲁迅逝世发出的电报}
\datesubtitle{(一九三六年十月二十日)}


(一)致许广平的追悼鲁迅先生的唁电</font></p>
上海文化界救国会转许广平女士鉴:

鲁迅先生逝世,噩耗传来,全国震悼。本党与苏维埃政府及全苏区人民,尤为我中华民族失去最伟大的文学家,热忱追求光明的导师,献身于抗日救国的非凡的领袖,共产主义苏维埃运动之亲爱的战友而同声哀悼。仅以至诚电唁,深信全国人民及优秀之文学家能继续鲁迅先生之事业,与一切侵略者,压迫势力作殊死的斗争,以达中国民族及其被压迫的阶级之民族和社会的彻底解放。

<p align="right">肃以电达

 一九三六年十月二十日</font></p>
<p><font face="黑体">(二)告全国同胞和全世界人士书

噩耗传来,中国文学革命的导师,思想的权威,文坛上最伟大的巨墨,鲁迅先生损落于上海。当此德日等法西斯蒂张牙舞爪,挑拨世界大战,中华民族危急存亡之秋,鲁迅先生的死,使我们中华民族失掉了最前进最无畏的战士,使我们中华民族遭受了最巨大的不可补救的损失,中国共产党中央委员会,中华人民苏维埃中央政府对于鲁迅先生的死表示最深沉痛切的哀悼!

鲁迅先生的一生的光荣战斗事业,做了中华民族一切忠实儿女的模范,做了一个为民族解放,社会解放,为世界和平而奋斗的文人的模范,他的笔是对于帝国主义,汉奸卖国贼,军阀官僚,土豪劣绅,法西斯蒂以及一切无耻之徒的大炮和照妖镜,他没有一个时候不和被压迫的大众站在一起与那些敌人作战,他的匪利的笔尖,完美的人格,正直的言论,战斗的精神,使那些害虫和毒物无处躲避,他不但鼓励着大众的勇气向着敌人冲锋,并且他的伟大,陡他的死敌也不能不佩服他,尊敬他,惧怕他。中华民族的死敌,曾用屠杀监禁,禁止发表鲁迅的一切文学,禁止出版和贩卖鲁迅的一切著作来威吓他,但鲁迅先生没有屈服,民族的死敌想用“赤化”,“受苏联津贴”等捏造的罪状来诬陷他,但一切诬陷都归失败;民族的死敌,特别是托洛斯基派想用甜言密语来离间大众的阵线,但鲁迅先生给了他以迎头痛击。鲁迅先生在无论如何艰巨的环境中,永远与人民大众一起与人民的敌人作战,他永远站在前进的一边,永远站在革命的一边。他唤起了无数的人们走上革命的大道,他扶助着青年们成为象他一样的革命战士,他在中国革命运动中立下了超人一等的功绩。

