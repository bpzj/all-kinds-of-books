\section[在最高国务会议上的讲话(三)(一九五八年九月九日)]{在最高国务会议上的讲话(三)}
\datesubtitle{(一九五八年九月九日)}


教育这个东西比较带原则性,牵涉广大的知识界,是一个革命。几千年来都是教育脱离劳动,现在要教育劳动相结合,这是一条基本原则。大体上有这样几条:一条是教育劳动相结合,一条是党的领导,还有一条是群众路线。群众路线大家懂得,没有问题了,党的领导现在可能问题也不多了,中心问题是教育劳动相结合。现在苏联对这个问题也想改革,他们正在搞一个文件,在那里酝酿,我们社会主义国家,马克思讲了的,教育必须与劳动相结合。我在天津看了两个大学,有几个大工厂,那些学生在那里作工。老读书实在不是一种办法。书是什么东西呢?书就是一个观念形态,人家写的,让这些没有经验的娃娃来读,净搞意识形态,别的东西看不到。如果是学校办工厂,工厂办学校,学校办农场,人民公社办学校,勤工俭学,或者半工半读,学习和劳动就结合起来了。这是一大改革。

在财政方面,我找了一个材料:一九五零年到一九五七年这八年全部财政收入是一千七百亿,今年起,第二个五年计划预计大概可以收××亿。你看,八年一千七百亿,五年可以搞××亿。这个事情很可以注意。那八年的头一年,一九五零年,只有六十五亿,可怜得很。第二年,一九五一年,一百三十三亿,增加了。第三年,一九五二年,一百四十八亿。这两年都是一百亿以上。到五三年,就是五年计划的第一年,跃到二百二十三亿,五四年二百六十五亿,五五年二百七十二亿,五六年二百八十七亿。总而言之,这四年相当停滞,有所发展,都没有突破二百亿以上到三百亿。三百亿是去年,去年是三百一十亿。你看,以前搞了四年,都是二百几,去年一年就是三百一。今年可以搞到××亿。你看。由三百一,一跃进到了××。明年应该是××几或者××几亿吧?也不要××几,也不要××几,一下跃到××亿。我这说的是第三本账,××亿有可能。去年三百一只有一年,今年××亿也只有一年,××没有,××没有,明年一跳可以跳到××亿。是不是能够搞到,还要看,这是一种预计,或者还会更多一点。后年就会更多。五年××亿,平均每年××亿。这个数目值得注意。还有一个数目也是值得注意的;基本建设投资,五零年可怜得很,只有十一亿,五一年二十三亿,五二年四十四亿,五三年八十三亿,五四年九十一亿,五五年九十三亿,可怜。五六年不是搞“冒进”吗?由九十三亿一跃跃到一百四十八亿。说是搞“冒进”了,不是犯了错误吗?五七年就减少了一点,由五六年的一百四十八亿减少到一百三十八亿,减少了十亿,所以成为“马鞍形”。今年是二百六十八亿,明年总应该就可以搞××亿。前面这八年,五零、五一、五二那三年合起来是八十亿的基本建设投资,第一个五年计划不到五百亿,只有四百九十二亿,第二个五年计划,今年只有××多亿,明年就可以搞到××亿,就是一年等于那五年,而那五年的五百亿办的那么多工厂,就浪费差不多一半,本来可以办两个,只办一个,时间本来只要一年的,要两年,那么明年这××亿,就可以当作××亿(等于一倍)来用。因为现在有了经验了。双反,破除迷信,打破了一些规章制度。这是两笔大账。

此外,人民公社是一件大事。人民公社大概九月就差不多搭架子搞起来了。看样子。来势很猛,没有办法阻挡,你叫他慢,那不行。至于把一些问题搞清楚,充实这个架子,那就要冬春。这件事要好好领导,要积极领导,要采取欢迎的态度。人民公社的特点是大公社,这是最近几个月出来的新事物。

还有一点是抓工业,搞了八九年了,实际上我们这些人没有抓工业,重点不放在这里,放在革命上去了。搞土地改革,镇压反革命,抗美援朝,三反五反,整风反右,公私合营,合作化,这都是属于革命范畴。忙那些事情忙得要死。但是地方,他们除了这些之外,还抓了农业。认真抓农业,搞试验田,是从去年冬季起。这一抓,就抓起来了。现在我们要转过方向,人有只两手,一手抓农业,一手抓工业。我讲了个抓紧,什么叫抓?什么叫紧?抓而不紧,没有抓起来,等于不抓,你拿烟也拿不到,拿饼干也拿不到,拿洋火也拿不到。抓工业要抓紧。主要是抓一个钢铁,一个机械,有了这两门,万事大吉。钢铁是原材料,机械就是各种设备,包括挖煤炭和挖矿山的机械,开油田的钻探机械,电力机械,建筑机械,/t学机械,交通运输机械(无非是汽车、轮船、飞机,我们坐飞机就是坐机器),还有农业机械,如拖拉机,耕种机械,收割机械,农村用的运输机械,农村电气化用的电力机械。苦战三年,农村机械化还不会那么多,在第×个五年计划的后两年可能基本上用机械武装起来。所以一是抓钢铁,二是抓机械。没有钢铁,机械就没有材料,就不能造机械。机械里头有个工作母机,什么矿山,什么炼油,什么电力,什么化学。什么建筑,什么农业,什么交通运输,这些机器都要有个工作母机,无非是车、铣、磨、刨、钻之类,这些东西是根本的。今年不是二万多台,一发展就是五万多台,现在又搞到八万多台。这就是指工作母机。明年不是搞××万台吗?实际上明年争取搞××万台,如果明年能搞××万台,后年再搞××万台,我们这个国家第二个五年计划就要搞一百多万台机器。我们解放的时候,四九年只有八万台工作母机,还是破破烂烂都在内。今年年底有二十六万台。从张之洞起,到今年搞二十六万台工作母机。而这二十六万台里蒋介石交给我们的遗产是八万台。二十六万台减八万台,我们这九年搞了十八万台,但是同志们,明年这一年就不是八万了,也不是二十六万台了,而是××万台,一年××万台,后年搞××万台。苦战三年,明年××万,后年××万,××万台,连前头的二十六万台,是××万台。那个时候,我们跟美国人谈判就神气一点了。


