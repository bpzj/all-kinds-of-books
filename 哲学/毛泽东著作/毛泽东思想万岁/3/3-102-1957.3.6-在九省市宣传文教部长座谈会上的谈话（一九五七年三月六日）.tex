\section[在九省市宣传文教部长座谈会上的谈话(一九五七年三月六日)]{在九省市宣传文教部长座谈会上的谈话}
\datesubtitle{(一九五七年三月六日)}


(关于会议开法,听了主席录音报告,内容很丰富,要好好消化,最好多开小会,少开大会。)

好,开一星期,不够,再多开二、三天,开到解决问题为止,多开些小会。

(各地到会有党外人士,我们来的是清一色。)

党内也不会是清一色。

(对陈其通等四人文章,听报告录音与前次省市委书记会议所传达精神,不同。)

他们是以卫道君子的面目出现的,教条主义的方法,宗派主义的情绪,是达不到卫道的。过去讲的是插话,可能听漏了什么了。

(关于党委各部门应分管政治思想工作及政府各业务部门不抓政治思想工作问题。)

第一书记要抓思想,回去告诉他们,希望第一书记把思想工作抓起来。第一书记才行,第二也不行。光是宣传部,孤单单的不好办。管业务管得很好,不管思想工作,结果来了大民主,就会把你搞掉。各部门、各党组一定要管思想工作。省委要抓好思想工作,特别是第一书记,各司局都要管思想工作。

(“百花齐放、百家争鸣”方针提出后,问题很多。)

问题是多的,要放手,有好处。

(批评工作没做好。)

打大仗往往不可免先要打几次败仗,有了经验就会胜利。

(康生:团结,批评,团结,但往往一批之后,没达到新的团结,是批回去了,急得很,主要是仓促应战。)

领导要对问题进行研究分析,才能解决问题。

(上海反映,党员与党员在报上斗。)

党内思想也很混乱,不要害怕,要放手,怕什么,难道地球会炸了!

(康生:党外也怕,怕宗派主义,不让人争鸣。)

(有人反映老教授对“争鸣”还有顾虑,写文章不容易,第一天文章一登,第二天给人一批,学生对自己也不尊敬了。)

那第三天再写一篇批回去就是了。

(有人提出,陈其通文章问题是否省市委书记会议强调一面,国务会议又强调一面,因国务会议有党外人士?)

党内党外要一起谈,关起门来,对外又一套,或单独对党外谈也不行.要党内外一起谈。

(闹事问题)

特殊的、个别的犯法分子当然要按法律处理。一般闹事的不要开除,总开除不出中华人民共和国以外去,他总要有个立脚点,还不是到学校、机关、工厂、合作社去。西安抓了一百多个流氓,社会很高兴,但学生中“流氓”跟社会流氓不同,知识分子不是社会那样“流氓”法,如不犯法,仍不要开除。……

(希望讲讲上层建筑与基础关系问题。有图画家说花草没有阶级性,还有自然科学问题。)

自然科学不是上层建筑,但它要靠人去搞,就可能渗进一些人的阶级意识进去。

(当前思想斗争,是否可概括为无产阶级思想与资产阶级思想斗争?)

马寒冰文章是教条的,钟惦棐则是右的,两派我们都要批评,……“电影的锣鼓”文章基本方向是不对。

(周×:所提的一些缺点,的确是有的,但“中申”一些同志情绪不对头。部里是不同意该文的。)

无产阶级思想要同资产阶级思想作斗争,是不错的,但必须采取说服教育的方法。百花齐放中,资产阶级思想出现会多起来,但并非都是资产阶级思想,并非无一可取,扣帽子会把人家吓倒,不必每篇文章一出现就马上驳倒。

(陆××:大将不要先出马,不要打冲锋,先让党外人士发言。)

马克思主义是无产阶级思想,中国知识分子少,要他们信马克思主义不容易的,如三个五年计划,有三分之一知识分子信了马列主义,就是很大的胜利。知识分子可以接受社会主义,因他没办法,不能不接受,但思想上不是很服。(××:北大哲学系系主任公开说相信唯心论。)他们政治上可以跟我们走,但要信马列主义不容易,有的搞康德,黑格尔几十年,可以逐步改变;有些一辈子也不变的。

(××说:百家争鸣还有顾虑,怕把是非问题弄成敌我问题。)

怕升级。

(有的人看到百花齐放、百家争鸣之后,暴露出许多资产阶级思想,好像过去工作白费了。)

不暴露好像没有,暴露就很多,是否过去工作白费,解放才六、七年,资产阶级思想就没有了?他们说出来是好的,我们得到机会去教育。问题是我们现在有些文章还没有说服力,企图压服。压是压不服的。无产阶级思想与资产阶级思想斗争要几十年,批评人家一定要有研究,想打几棍子不是办法,不能解决问题。过去思想改造是有成绩的,那是大风暴,是粗枝大叶,基本解决分清敌我,分清拿刀子杀人的,这是有数的。现在是分清是非,就要具体地讲,仔细一件一件的讲,如数学、物理等等…,还有一派派,必须具体讲是非。(有人问:这个是非怎么分?如新闻工作等就有争论,请主席讲一下。)我不是新闻家,你们是内行,××就是专家,我们要讲的是大是大非,要分清敌我。不是特务,有选举权的,宪法就规定他有言论自由,我们就得让人家讲话,我可以批评他,他也可以批评我,这就是言论自由。有人问马克思主义可不可以批评?如果马克思主义被批评得倒,也该倒。证明那样的马克思主义是没有用的。

(湖北反映初级党校科级干部酝酿上街游行有五条理由,反对官僚主义。)

官僚主义改了,人家就不游行了,因此要加强党员教育。

(现在思想混乱,究竟原因何在?有的说是官僚主义,有的说是要民主、自由、个性,不要集中、专政、共性。领导要纪律、共性、集中,被领导的干要另一面,有的说是资产阶级思想等等。)

如果动不动就说“反对社会主义”,说是资产阶级思想,是不妥当的,因为学校存在问题没解决,所以闹事,如湖北党校是什么问题呀?要分析。

(思想工作的主要内容是什么?提倡什么?反对什么?反对资产阶级思想还是小资产阶级思想?还是先进思想与落后思想的斗争?官僚主义?……)

都有,不能用一个简单口号去套一切。现在与过去反对帝国主义不同了。合作化要搞好几年,个人与集体是矛盾的,资产阶级与小资产阶级的知识分子还要改造,这就叫过渡时期。(……)要对具体问题进行具体分析。

(兰州原要开除几十个学生,现在不开除了,又有点草率收兵。)

开除几十个学生是国民党办法,事情结束如不解决问题,将来还是要闹事的,如讲到兰州林业学校、护士学校招生骗人家,学生闹事,我是站在学生这方面的。你欺骗人家么?(康生等:欺骗就是犯法,可以告到法院,有罪的。)像这样学校,你说是什么问题?这两校都是官僚主义,欺骗,又有官僚主义。……共产党员工作这样,要整不罢课怎能整掉官僚主义?学生中很多成分不好,但他要说“对,我是地主成分,但你为什么骗了我?”我们要向党内外宣布,在人民内部无所谓专政,在人民内部讲专政是错误的!马克思什么时候说过人民自己专自己的政?个别犯法是例外,不犯刑法就不能法律制裁!

(党校讲课都用大纲,党内关于马列主义教育能否争鸣?)

马克思主义就是一家,如果有不同解释,就是曾经有第二国际,列宁的第三国际,斯大林有马列主义,也有教条主义。

(教学大纲以前叫“法律”,后改为“参考”了。)

以前叫“法律”,现在叫“参考”,可见这“法律”就不太严肃了吧。

(四川曾想封《星星》杂志社。)

《星星》还是不要封。这次会议一开,资产阶级与小资产阶级思想又会冒出来,不要急,我们不忙于理它,它大有劲头了。你们不是反映有些教授说:“百花齐放,百家争鸣”是“诱敌深入”吗?我们对资产阶级与小资产阶级思想有两条:(一)必须批评,(二)必须批评得好,因此必须要有准备,要有说服力,毒草在中国长了几千年,再长七、八年、也不要紧。而且我们还是要做事情的。他们一肚子气,可以让他们讲,毒草不可怕,如用压下来的办法,还是要翻的。(有人说,《星星》所谓七君子中二个有杀父之仇的。)这样,《星星》出现那些东西是有历史原因的。我们如何对付不正确思想:要有办法,不要急躁,不要简单,应该研究方法。中国有几千万地主、富农、资产阶级与知识分子,高等学校80%学生是他们子弟,那些有杀父之仇的,能不恨,不骂我们。但应估计到,剥削阶级出身的知识分子大多数是可以争取的,现在高等学校工农子弟还不多,在中小学。匈牙利高等学校60%学生是工农子弟,照样闹事,反苏反共,我们80%是那样子弟还没闹事。专业学校官僚主义为主,还有欺骗。在工厂里也有官僚主义,百花齐放,百家争鸣中,毒草是有的,但不是多数,占百分之几?农民、小手工业者都是小资产阶级,……现在是转变时期,大变化时期,在观念形态上一定有所表现的,但99%还要多一点是能够教育过来的,问题在方法,在于有说服力的文章,学校的教员,工厂的干部能讲得清道理,能说服学生、工人。光压,压得服人?

(百家争鸣中党内外有人动不动就说庸俗社会学。)

简单口号压不了人的,我们应当去研究马克思主义美学,要有说服力的文章。如果大家看来都对,只有一人说是教条主义,问题就解决了,如果多数人都说你是教条主义,那就不成了。地、富、资产阶级及其子女、知识分子是可以教育过来的,问题在于方法。极少数的人会教育不过来的,大约只有千分之几,但如果他们不拿起刀子来也不要紧。《星星》的“草木篇”是应该批评的,如不批评真是让毒草长,起来了。钟惦棐的文章也是毒草,是机会主义之花;马寒冰的文章是教条主义之花,马寒冰文章中有这么一种气味——你是什么人?江山是老子打来的,你就乱放?我们要取得经验,要学会如何掌握。我们对付蒋介石、帝国主义是有经验的,掌握得好的,匈牙利的问题我们也掌握了。因此对问题必须研究,要用脑子,要学习,重要的是在斗争中取得经验。有两点:(一)对毒草估计过分了,几千年来,人民就这样容易毒倒啦?人民是有分辨能力的,不怕!让他放一下吧!党员(领导干部)不要先讲话,先让民主人士写文章,让党外充分讨论。(二)过去搞阶级斗争,我们是有办法的,现在是思想斗争,不能再用老办法了,打倒蒋介石,抗美援朝,三反五反,土地改革等的办法,现在是思想斗争,是不同了。思想斗争是动口不动手,而且动口要恰当,不是采取专政的办法.不要将敌人夸大而小估自己,没有什么可怕的。去年一些专业学校采取欺骗办法,有个学校7000人罢课,全国有500万中学生,中学校长和党委书记要好好研究如何办好学校,500万学生闹起事来,也不好办啊。

领导思想斗争的方法要研究,过去是对敌斗争,有“三大纪律、八项注意”等等。现在问题是复杂的,有科学、文艺、高等学校,还有“草木篇”,他能写,我就不能写,诗经、楚辞是什么呀!大部是草木篇。他们和我们作对,不怕,四川是大国,有几千万人口。陈其通四人文章,老干部十之八九是同情的,但党外不赞成,党就孤立了。要发动知识分子讨论“再放”。

(航空学院付支书张云风写反动标语问题。)

他写“苏军滚出匈牙利”,我们要他“滚进”,他要“滚出”,学校里同意这意见究竟有多少?千分之一大概会有的。(有人提糊涂思想还很多)这是显明的敌对问题,大概就是千分之一,糊涂思想不算,世界上多得很。(××:支部表决开除张党籍问题,5人赞成,4人主张留党察看,是在负责同志讲了要开除以后表决的。)九个党员,五个赞成四个不赞成,说明时机未成熟,何必急,张有意见可在支部中公开讲,为什么要秘密写标语?这是恶劣!我们还可以研究经验,是否要开除?如要开除,什么时机开除?

资产阶级有资产阶级思想,资产阶级的儿子也有资产阶级的思想,说来中国的资产阶级还是少了些,像张云凤这样的少。(××:像张东荪那样的人,几年了,现在还放在那里教书,对我们来说也没有什么损害。许多教授公开说:“共产党会胜利是无天理!”我们也还是采取严肃批评与耐心教育相结合的方针,最近他自己公开检讨了,这说明还是有效的,对党无损害。去年还参加了知识分子会议。)还加了薪呢。思想斗争是“文”的,要惩前毖后,治病救人,统一战线,团结批评团结,我们是当医生,开刀是为了把人救出。(……)

(有些教师认为不需要继续改造了.)

不要改造?我6O岁了,还要改造,一万年还要改造。人要前进,就要改造,这是合乎规律的。如果还是老一套,像过去骑在马上打蒋介石那样,就不行,就要改。共产党还是有点人情,是做事的。说到学术,人家说我们“不学无术”,我们这方面的确也是不多,所以很需要学习研究。

(有高级知识分子说你们政治是为业务服务的。)

对,政治是为业务服务的。(××:在一定范围内是应当这样说的。)政治是上层建筑,为基础服务的,我们提出“向科学进军”、“十二年规划”,至于半导体、原子能等就应该由你们去搞,我就不懂,过去我们是阶级斗争,是种攻势,他们没办法,但不舒服。现在要建设,知识分子就以改造之身出来讲话,批评我们的官僚主义。他们批评是好的,为什么不可以把尾巴夹起来呢?要学习,要研究。(……)

(知识分子中的新老问题,老的怕新的批评。)

规律就是如此,后一代的人学过前人就批评前人。马克思就是如此,如果后代跟我们一样,什么好处?

(他们不公开宣称唯心,而是把唯心唯物搞在一起,搞乱了,借反教条主义之名,以抬高唯心,是他们百家争鸣之目的。)

百花齐放,百家争鸣,各有各的目的,但结果是我们的目的能够达到。(××:有人提出,“百家争鸣”没有“百花齐放”、“推陈出新”那么明确,是否加上“去伪存真”或“实事求是”。)实事求是是马克思()方法,唯心主义者是虚中求是。

(有的科学家认为党不能领导科学。)

党能不能够领导科学?能够,办办看,如领导不起,还是不行,我们是从政治上来领导科学,搞十二年规划,向科学进军,这个我还领导不得?

(××:去年我们搞照顾等等,做了总务长的工作。)

我们,总务长加李富春(计划),第一给饭吃,第二计划。现在科学研究,我们就是没那么多党员去充当所长,难怪人家说,这点不要去争,自然科学我就不懂,要请他们当先生,李富春加总务长,我一点都不领导?自然科学,先生还是你。

(苏联是否过渡到共产主义?)

还没有。

(陈伯达同志在知识分子会议上讲话所提知识分子通过两条道路达到马克思主义问题。)

自然科学家是通过自己业务实践道路来达到马克思主义的,这是指他个人通过的道路,并不是说不要领导。有些知识分子口头说党不能领导,但实际上我们领导了。列宁懂导半导体?但还是领导了。自然科学部门那么多,科学家自己也是懂得这个不懂得那个的。梅兰芳能领导京戏,还能领导话剧?他是旦角,难道能领导丑角,他领导得了程砚秋?结果还是外行领导内行,政治就是领导。他们事实上是说这么一个问题-“共产党还没有科学家”。苏联情况和我们不同,他已有大批党员科学家。(教授说:政治为教学服务。)

政治为教学服务,整个人民政府就是为工农服务,也为科学技术服务的。也是的,搞得好一点就发展得快一点,就是为其服务,并且全心全意地服务。

(对闹事学生、工人中共党员该怎么办?有困难。)

有困难,应当看有理无理。如果有理,党员应该站在群众方面,对十分坏的官僚主义,要反对。当然首先要极力争取和平解决,不罢工罢课,对于无理的,就不卷入。有理的,党员、工会、学生会应该站到群众方面。

(报纸要不要干预生活?)

报上的宣传要看对人民有利无利。没有抽象的言论自由,只有什么阶级的自由,阶级中集团的自由,有阶级的时候,是阶级办报,如果把报纸当作没有阶级性的,将来是对的,现在可不。

(有人提出省里传达说,陈其通等人的文章是对的。)

我是先后插话谈过“这几个同志对党是忠心耿耿的,为了党的事业的,但文章则不堪领教”,下面这句话没了。

(康生:可能反映到会同志的一些思想,上一截容易接受,后一句就容易忽略。)

他们的文章反映出对敌对思想的仇恨情绪,没有这也不得了,也要保护。问题是他们是教条主义,方法是错误的。百花齐放,百家争鸣,现在不是放多了,是少了,应该再放。当然在放之中任何错误的东西都应该批评。现在放够了吗?鸣够了吗?不够的。人家还在猜我们的意图。认为我们是“诱敌深入”,因此必须再放。现在开宣传会议,大家同意这方针,要很好研究这方法。(一)党员不忙于写文章,让党外先写,当然要领导。(二)党员也应该写,但必须是要有说服力的,有研究的,有分析的,而不是形而上学的、教条主义的方法,有分析就有说服力,我们应采取帮助人家改正错误的态度,而不是一棍打死的态度。

(关于八大决议,制度与生产力的矛盾问题所引起的争论。)

矛盾有,但不大。但保护生产力发展的,不能拿将来与今天比,也不能拿美国与中国比,这是不妥的,5O年后的中国会有不同,生产关系在50年内变动不多,苏联已经过40年么,现在还是适应的。矛盾吗,现在有,将来更大,现在小一些。这一句话在决议中本来已经删去,但给照原样印发了。临开会才知道。

在七个省市教育厅局长座谈会上对当前教育工作的指示

全国宣传工作会议期间,主席于三月七日晚间在颐仁堂和天津、山东、江苏、河南、湖南、四川、陕西七个省市教育厅局长座谈当前普通教育问题。参加座谈的有:康生……等。

首先由主席询问学生闹事情况。经河南、山东两省同志汇报后,主席指示,学生闹事。应当重视,也要具体分析。如许昌学生向视察的人民代表提了一百多条意见,不是完全没有理由,你们应当很好地解决问题,不应当拖。你们都看过“困难重重的昆明航空工业学校为什么没有闹事”的参考资料(中宣部办公室印发)吗?学校负责同志如能像该校校长和教师那样加强政治思想工作,主动地给学生讲清真实情况,师生同甘共苦,战胜困难,不是解决了问题吗?

主席又问去年由于合作化,办了很多学校,但出了很多问题,戴帽子的学校怎样,好不好?经各省同志反映情况,一般群众都欢迎,同时也提出困难问题,如校舍、设备、师资、领导等。然后主席指示,戴帽子学校是先进经验,群众是欢迎的、拥护的,不愿搞掉,这就是好的。要继续办下去。你们专家要求过高,并要正规化,这就不对。有困难,可以补充一些教师,发一些经费,要求放低一些;先使普及,便于生产,课程可简单一些。这时主席又问×××需要多少钱。经初步估计约八千万。主席说,一亿行吗?只要贯彻增产节约的方针,节约一亿并不很难。

大家又请示主席,这样的办法,将来学生可以升学吗?主席指示,这种学校是为了满足人民对文化的要求,学生将来主要的是参加农业生产,课程简单一些才容易办,学生学得好的,同样可以升学。

主席又问学校就学有什么困难。大家都说很多学生交不起费,要依靠助学金来解决困难,特别是农村学生,但是助学金又不够分配。主席指示,合作化不过一年,合作社还没有巩固,增加生产,增加收入,还有困难。三、五年后,合作社都像遵化县建明农林牧生产合作社(中宣部办公室印发参考资料)一样,就没有什么困难了。因此,在三五年内,农民子女就学,要帮助解决困难,助学金不宜降低,还可增加一些。这时主席又问需要增加多少。经大家反映情况,×××提出高中增加到百分之二十八,初中增加到百分之十八,可以解决问题。

主席又问每年初中高小毕业生升学是不是很紧张呢?同志们一致反映紧张的情况。主席指示,全部解决是不可能的,但可以略增加一些任务,适当地解决群众迫切的要求。戴帽子学校可以继续办下去,民办小学也可以提倡。群众需要而又能办的可以办,这时×××也提出厂矿企业机关办学的问题,主席表示同意。

大家又请示主席,对教育事业的发展,如多少、快慢应该如何掌握?对初中、高小毕业生不能升学,参加生产劳动,应该如何宣传教育?主席指示,对群众宣传,可就国家经济情况来说明。李富春同志已在全国政协讲过,你们可以听听他的说明。(三月十日听取李富春关于第二个五年计划建设的若干问题的说明录音)。

同志们又提出若干关于教学计划和教材方面的问题,如科目多,分量重,内容深,教师水平又不高,师生负担都很重,教学质量都不高,特别是文学课学习苏联办法更感困难。主席指示,学习苏联要结合中国实际,不能有教条主义。中国是六亿人口的国家,情况很复杂,不能采取划一的办法。科目要减少,七、八门,九、十门就可以,教材要适合各地方情况,要允许各地方自编地方教材,如地理(如湖南省)、文学(当地文学家的作品)、动物、植物。现在这样的文学课本不好,要改编。这时候×××说:大家意见还不一致,要再研究。同志们又反映,政治课取消后,对学生进行政治思想教育困难。这时×××说明,政治课教学效果不好,因而学习苏联,不教政治课,只教宪法。主席指示,高初中要增加政治课,教材要另编生动的,包括基本知识和做人的道理,隔二、三年要修改一次。宪法课这样教不行。当时就指定×××、×××、××负责编。并指示,要教育学生艰苦奋斗,招生时就要说清楚艰苦奋斗,不要讲得太好。师生要同甘共苦,共同办学,发挥创造精神。要劝导青年迟婚。各省市宣传部长、教育厅局长,要有一人专管政治思想教育。当时又指定×××拟发指示。

大家又反映,基层教育工作繁重,干部力量不足,又常常抽调搞其他工作。因此,要做工作,完成任务,必须充实人力,健全机构,如县市教育科局,乡文教委员。教师忙乱,侮辱教师,占用校舍等问题也要解决。主席指示,要加强地方教育行政领导,并指定×××在《人民日报》发表社论。

最后主席指示,教育工作是政治工作,很重要,省地市委第一书记要管政治思想教育。我就是第一书记,一要管政治思想教育。如果忙,每年开几次会,就可以解决问题。

<p align="right">(一九五七年)</p>


