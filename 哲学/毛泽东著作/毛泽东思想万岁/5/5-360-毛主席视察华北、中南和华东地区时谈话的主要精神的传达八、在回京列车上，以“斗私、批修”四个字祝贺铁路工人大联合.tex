\section[毛主席视察华北、中南和华东地区时谈话的主要精神的传达八、在回京列车上,以“斗私、批修”四个字祝贺铁路工人大联合]{毛主席视察华北、中南和华东地区时谈话的主要精神的传达八、在回京列车上,以“斗私、批修”四个字祝贺铁路工人大联合}


这次我们跟随毛主席出去视察,坐了一段火车。车上有三派(都是铁路系统的),都说自己是造反派,人家是保守派。起初我和××同志去做工作,他们说我们“合稀泥”。后来毛主席把三派的代表找来,听他辩论,他们辩论了两个小时。

主席说:“我看你们没有根本的利害冲突,应该联合嘛!”

三派还有些不想大联合,都指责对方,不做自我批评。

主席说:“不要尽看别人的缺点错误,你就讲你自己的,人家的问题他自己会讲的。人家的缺点应该人家讲,你讲什么?”

这样就没吵起来。等到他们从主席车厢走出来,还是说对方是“老保”。我就和××同志一直和三派做工作,和他们一起学习《毛主席语录》135页那一段话,我们说,主席和一个省谈话也不过两个小时,和你们却谈了两个小时,你们还不联合,回去怎么交账? 结果三派均照主席指示检查了自己,彼此越听越感动,他们联合起来。当然,也不那么简单,有一派还提条件,可尽是一些鸡毛蒜皮的事,离北京只剩三个小时,经过做工作,也通了。

这样,三派达成了联合的协议,到毛主席那里去报喜。

主席说:“祝贺你们,送你们四个字:‘斗私,批修'”。“斗私,批修”就是这么来的。

的确,如果私字不去掉,无法达成大联合的协议,达成了,也会分裂。就是把黑手斩断了,不把私心杂念去掉,特别是不去掉造反派领导人头脑里的私心杂念,就不能实现革命大联合、三结合。即使勉强联合也是不巩固的。

斗私、批修是相互联系的。

姚文元同志九月二十九日晚接见华东四个铁路分局革命派时传达说:最近铁路工人群众组织,实现了革命大联合,我们向毛主席汇报,革命大联合后搞什么?

毛主席讲:斗私,批修。

