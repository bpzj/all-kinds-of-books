\section[在郑州会议上的讲话(二)(一九五九年二月二十七日)]{在郑州会议上的讲话(二)}
\datesubtitle{(一九五九年二月二十七日)}


人民公社决议只提一句“按劳分配”。究竟如何按劳分配,没有完全解决。什么是生产责任制?马克思讲过“生产责任制”,怎样的责任制也未讲。现在要谈的问题是公社所有制问题,所有制问题即公社所有制要不要一个过渡来建立,是不是公社建立的时候就是集体所有制。我在山东看了一个公社——济南东郊人民公社,二万一千户,十二万人,一个生产队,距公社所有制很远,实际上是公社党委所有,这还得了!问题就在这里。现在很多人不通,就是要统多统死,就是过去地方讲我们的,现在不是下放了吗?公社有三级,生产队一级有七、八百户,有一千多户的。所谓统多,就是多搞积累。所谓所有制,一曰土地等生产资料,二曰劳动力,三曰劳动产品。这些究竟归谁所有。现在公社党委、县委、地委、省委包括中央恐怕还急于进入共产主义,因此要统多统死。现在工人和农民的情况不同,以鞍钢为例,一个工人的总产值一万八千元,除七千二百元,剩一万○八百元,工人收入八百元,个人消费占他的产值不是十二分之一,为国家积累很少,我们想积累,河南公社积累、国家税收、管理费用,公益金共占50%,生产费用占20%,农民实际所得30%(,但农民要活,因此要瞒产15%,方法几十种,这是合法权力,而我们批评他们为本位主义其实是违犯按劳付酬的原则。现在所有制实际是队所有制,生产资料,生产者归队所有,产品所有制也归队,农民现在站岗放哨,保卫产品所有制。为他的劳动成果而斗争,你分给他30%,他就加15%,实际上是45%。现在公社与生产队激烈斗争是两个问题,一是人力,二是产品,农民不怕把土地搬走,但怕把产品运走,农民往城里跑。现在财政部门把全部贷款收回,因此使人民公社无法维持,这是一种破坏生产、反人民公社的倾向,贷款全部收回的还要退还。卖猪卖白菜的钱不给公社,白菜大批烂;拚命的吃,城里吃不到菜,原因就在这里,不完全是运输问题。现在顶牛,一方面生产队批评上边是平均主义,另一方面上边批评下边是本位主义。两种主义可能都有,但是我们在党内主要锋芒还要反左。生产费与积累占70%,消费只占30%。积累太多,猪卖了,各种物资卖了,都是归社,这种公社所有制是破坏生产的,是危险的政策。不应该批评他瞒产是本位主义,东西本来是他的,你不给他分,他只好瞒产私分。所有制的改变,少者四年,多者五、六、七年。富队帮穷队提高,穷队逐步向富队看齐。不要把富队的头砍下来补给穷队,这种性质是无偿占有别人的劳动。我们对民族资产阶级还用赎买的办法。苏联五千五百万吨钢,我们一千一百万吨钢,砍苏联二千万吨钢补我们不合理。一部分农民无偿的占另一部分农民的产品,不叫抢劫,而叫共产主义风格。这与救济穷的不同。工业办多了,为什么积累这么多,财贸部门为什么把一切贷款都收回,就是办多了工业。中央、省、地、县、公社都想大办工业,看来各级的积极性过多了些。这点情有可原,情者合乎实际,因为土地、劳力、产品均属他的,中央、省、地、县、公社、管理区六级对付生产队和队,六级有权,但是农民人多。什么是一盘棋。现在不是一盘棋,是半盘棋,分配太少了。不承认生产队的所有制,只分给人家30%,要拉平分配,这叫作半盘棋,大批人马调动,大批积累,这种权利是冒险主义的权利,只要共产主义,不要本位主义很危险,要正当的提积累,要正当的办工业,而不是为疯狂的提积累办工业,还是要共产主义,还是要本位主义,光要共产主义不行。农民瞒产情有可原,他们的劳动产品应该归他们所有,积累,无代价的修铁路、修公路,修和他们不相干的水库,这一部分无偿劳动很大,提积累、收贷款,购买东西不给钱,组织运输力也不给钱,这就是农民想尽办法保卫他的劳动果实的原因。

六中全会对积累问题分配问题还没有完全解决,而不能决此问题,大跃进就无积极性。现在要出安民告示,现在人民公社基本上是生产队的集体所有制,公社只部分所有制,公社和管理区实际是联系介乎全民所有制和集体所有制两者之间,不出告示危险,今年库存减少,没有增产。反本位主义越反越收购不到。这种情况普遍存在。为什么去年秋收那样粗糙,东西收不到,就是没有解决分配制度问题。河南新乡地委说,收柿子宣布谁收谁得,一夜完。要承认农民瞒产合法,中央与省应说服地、县、社三级党委,社说服管理区总支,我们站在一边首先支持农民的合法权利,也说明我们无非是想搞工业化,工资级别死级活评,一个月评一次,多劳多得,一月变一次,工资总额不变,又叫上死下活。究竟公社要统多少?统三大项。国家税收、公积金、公益金。还有统购、计划、物价、教育,教育办的过多了也不好。

此外工业办多了。社办、县办、地办,省未纳入国家计划的工业也多了,要规定。不可不办,不可过多,中央、省、地、县、社五级工业都要有所调整,五级工业都不可太多。现在要继续把冷水泼下去。要把所有制问题讲清楚,要把斯大林的政策和我们的政策加以比较,斯大林的积极性太多,对农民竭泽而渔,现在即此病,理由是,反保守主义,反本位主义,我就支持这些主义。不是反本位主义情有可原,改为合法权利。我相当支持瞒产私分,除了贪污破坏以外,是正当权利。分配中的消费部分要增加,要发展生产,把穷队逐步提高到富队水平,不要拉平,工业不要办得太多了,釆取这种办法积累。搞大型工业、大型水利和公路等要有限制。分配给个人的要增加,超额分成。十条猪完成十一条任务那一个分成。田家英的警卫员是河北人回去看一次,家中杀一条猪60斤,为什么要杀?等不起,等了要拿走。我看要写个决议案。以所有制为中心,积累问题,分配问题。每个生产队生产的粮食多寡不一,每个生产队的吃粮标准也应该有差别,有的可能少于380斤只得如此。粮食多产多吃,工资也是多产多分,死级活评。基本原则是按劳分配。这是消费。积累是建设费用。公社不能办脱离生产的文工团,各级干部太多,要大大精简,节约办社要在决议里写一条,办工业的积极性,第一要称赞,第二要约束。中央、省、地、县都要约束。要有所不为而后才有所为,那些工业归县归社要有所调整。

我写了几句话:一、所有制的问题:几年内,譬如说四、五年内逐步完成由基本上生产队所有制过渡到基本上公社所有制。而目前在公社说来,只有部分的所有制,积累公益金等,产品所有也是这样。即“承认生产队的保守主义或本位主义”,这样我们的六级干部就可以和六亿人民打成一片。一方面批评我们的平均主义,一方面批评他们的本位主义。去年秋前好像农民跑向工人之前,但秋后即瞒产私分,这就是农民的两面性,农民还是农民。是不是向农民让步的问题?这不是向农民让步的问题,这等于按劳分配逐步到按需分配一样。在某一点上说,即对于过于积极办工业是一个让步。公社所有制,只能经过几年的时间一步一步地引导农民去完成,而不可能在目前一下子完成,把它当成一个过程去看待。由互助到高级社经过了四年,(从1953年到工956年)经过了几个步骤才完成,由高级合作社的集体所有制到公社的集体所有制的完成,可能也要经过四年或者更多一点时间,譬如五年、六年、七年时间,要整过急的思想。问题是把穷队提高到富队的生产水平,要经过这样一个过程。由于社大队多所以要有较长的过程,这个过程也即是农业机械化、电气化、公社工业化、国家工业化、人民社会主义共产主义觉悟的提高与道德品质的提高,人民文化教育和技术水平的提高的过程(我们计算四年钢可到五千万吨,明年拨一百万吨,后年拨二百万吨,大后年拨三百万吨。即六百万吨钢材装备农业机械化就差不多了。公社工业化有四、五、六、七年就差不多了)当然,这还是第一个阶段,以后还有第二、第三个提高的阶段,才能完成社会主义建设(即十五年、二十年或者更多一点时间。)在这个整个过程中,其性质是社会主义按劳分配的过程。但是,在这个过程的第一阶段,即是说,从1958年算起的三、四、五、六、七年内人民公社的集体所有制完成了,并且可能有一部分人民公社或大部分人民公社转到全民所有制。

1958年粮、棉、油、麻等大丰收,但是在最近四月内(从去年十一月到今年二月)有大闹粮、油不足的风潮,你说怪不怪,出乎意料之外,世界上天有不测风云。一方面中央、省、地,县、社、管理区六级党委,大批评生产队和生产小队的本位主义,即所谓瞒产私分。帽子一顶叫本位主义。另一方面,生产队、生产小队普遍一致瞒产私分,深藏密窖,站岗放哨,保卫他们自己的产品,翻过来批评公社和上级平均主义,抢产共产,写条一点,普遍过斗拿走。我以为生产队、生产小队的作法基本上不是所谓不合法的本位主义,而是基本上是合法的正当权利。

(他产的吗,马克思百年前讲过多劳多得吗,他懂得点马克思主义。他们就是按照这个原则来办事的。)这里有两个问题:一、穷富队拉平,平均主义的分配方法。是无偿的占用别人的一部分劳动成果。是违犯按劳分配的原则。二、国家农村税收只占农村总产值的7%左右(如河南)不算多,农民是同意的,但是公社和县从生产队的总收入中抽出的积累太多,例如河南竟占26%连同税收7%为33%再扣除1959年的生产费20%,再加上公益金、公社管理费共计占53%以上,社员个人所得只有47%以下,我认为个人所有太少了,不合物质刺激的原则,政治不可少,七分政治,三分物质太少了。管理费包括很大的浪费,用人太多,一个公社竟有三几千人不劳而食或半劳而食,其中有脱产文工团180人之多,晋南的例子。此外还有扎牌楼、导具等浪费。

公社是1958年秋成立的,刮起一股共产风。内容有几条。一是穷富拉平(已纠正,还有余波)。二是积累太多,三是猪、鸡、鸭(有的部分,有的全部)无偿归社,还有部分桌、椅、板凳、刀、锅、碗、筷等无偿归公共食堂,还有大部分自留地归公社(有些是正当的归公社)有些是不得不借用,有些是不应当归社而归社的,有的没作价,这样以来,共产风就刮遍全国,无偿占有别人的劳动成果,这是不允许的。我们看我们的历史,我们只是无偿剥夺帝国主义的、封建主义的、官僚资本主义的生产资料。此外,我们曾经侵犯了地主一部分多余的生活资料。所有这些,都是劳动人民的劳动成果,并非侵占帝、官、封的劳动成果,而把自己的劳动成果收回来,对民族资产阶级,采取赎买政策,因为他们过去是同盟者,又拥护改造,还要利用他们工作等等。既然如此,我们为何可以无偿占有农民的劳动成果呢?过去没有对基层干部讲清楚,动不动就要共产。当然,共同积累不是当作消费资料,也不是无偿占有,而是为了扩大再生产的建设资金,国家也是如此,这是对的。不从所有制问题讲道理讲不清楚,他们实际上是把公社当作全民所有制,只设想大集体所有制,不没想生产队所有制。

二、劳动分配问题。现在农民同我们的矛盾,一个是抢产品,一个是抡劳动力。现在土地、人力、产品三者名义是归公社所有,实际上基本上仍归生产队所有,目前阶段只有部分的东西归公社所有,即社的积累,社办工业,社办工业的固定工人和半固定工人,此外还有点公益金。所谓社有,如此而已。虽然如此,希望也就在这里。年年增加积累,年年扩大社办工业,社有大型、中型农业机械,社办电站社办学校等等,有三、五、七年就可以把现在的这种所有制状况翻过来,即由基本队有,部分社有变为基本社有部分队有的所有制。当然还会拖一个一部分个人所有制的尾巴,例如宅旁林木、家禽、家畜、小农具、小工具等,房屋在大规模建筑公共住宅以前,因为是消费性的,当然是私人的。现在农民不怕拉走土地,怕的是拉走人力和产品。要人要钱的积极性大,一压下去五亿农民没有出路,设所抵抗。去年农民拚命抵抗,把产品让它烂掉,甚至破坏,这抵抗的好,使我们想一想这个问题。

劳动分配,现在极为不合理,农业(包括农、林、牧、副、渔)分配太少,而工业,行政人员和服务行业的人员太多(有的多到30%到40%),必须坚决的减下来。过去八年只增八百万工人,去年全国所增的工人一千万未算在内,

(实际上是二千六百万人)中国从张之洞办工业以来产业工人只有四百万,解放以来平均每年增长一百万,即八百万,共一千二百万,而去年一年增了二千六百万,再加上各行各业转过来转过去的四百万,共为三千万,突然增加三千万,一则一喜,一则一忧。上面这三部分人,都有大批浪费,必须坚决减下来,从事农林牧副渔,否则有危险。据说工业浪费20%,要回农村,服务行业要大减,行政人员只许有千分之几。公社不允许有脱产的文工团。生产队与社、县、国家争人力是项严重的问题。

分配问题:分配是讲消费部分本身的分配,生产队人体上有穷、中、富三等,吃粮、工资标准都应有差别,吃粮也要有差别,和工资一样,队队不同,除征购外,多得多吃,少得少吃,工资实行死级活评,上死下活制度,要严格规定一个收粮、管粮(有国家、社、队的仓库),用粮(要有定量)制度,用粮要精打细算。去年大丰收,使我们麻痹了,粮食问题十年也不要说解决了,每人每年有三千斤粮食也不要说解决,要大反浪费,生产永远也不能满足浪费的需要,旧的需要解决了,新的需要又发生了。1958年积累多了一点,也是好心肠,有鉴于此,应当向群众公开宣布。1959年公社积累不超18%,连同国税7%,总共不超过25%,以安定人心,提高生产积极性,有利于春耕。

最后讲讲下放当社员的问题。各级干部分级分批下放当社员,每年至少三十天。多者四十五天。一部分下厂、下矿当工人,这样我们可以和群众打成一片,就不会有现在这样紧张局势了。过去历来第一是国家,第二是公社,第三是个人,现在我们倒过来,第一是安排人民的生活,第二是公社的积累,第三是国家的税收。


