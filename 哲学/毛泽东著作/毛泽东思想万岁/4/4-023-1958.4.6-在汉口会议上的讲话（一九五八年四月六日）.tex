\section[在汉口会议上的讲话(一九五八年四月六日)]{在汉口会议上的讲话}
\datesubtitle{(一九五八年四月六日)}


生产高潮形成的原因。现在生产高潮是怎么来的?

(1)以前有过高潮,有了领导高潮的经验,一九五五年冬一九五六年春曾有过高潮……。

(2)反冒进的错误使许多人不舒服,使干部抬不起头来。但挫折对我们很有益处。一种搞快些,一种搞慢些,这样就有了两种工作方法的比较。反冒进就慢。这几年来两高潮形成了马鞍形,一九五五年至一九五六年春和一九五七年冬目前就高,中间反冒进就低。这种形势对我们很有利。

(3)中央根据实际工作经验,在三中全会和青岛会议上及时恢复了四十条,多快好省的和做促进派的口号。

(4)经过整风反右派斗争,群众干劲起来了,干劲足了。

两个战役之间休整问题。目前的生产高潮,动员群众很广。动员这么多群众是从古以来没有过的,过去只有在战争时期,在参军上搞过大规模动员的。群众是个劳动大军,各级干部是指挥者,指挥者应当懂得在两个战役之间需要有休整,不要老紧下去,紧张之间要有调节,不要使群众总紧张下去。

同时,生产高潮中要务实,不要搞空气,不实在。苦战三年基本改变面貌。是提基本改变面貌,还是提初步改变面貌,这个口号起了动员作用,不要改,中央还要再看一个时候,改变面貌,光挖沟植树不能算,粮食、油料和棉花增产,因为我们不是布置花园,做到棉、油、粮增产才算改变面貌,挖沟只是手段,不能算目的。去年我们注意粮食、肉,从明年起要大搞油料,各省要规划,雷厉风行。四十条要增加油料增产指标和措施,为了帮助兄弟国家也要提高,还可提口号(陕西种核桃,各地还可以搞什么),如为了支援东欧国家等,这样号召力更好些,对农民也进行了国际主义的教育。

关于“化”的问题。今后《人民日报》不轻易宣传某某地方什么“化”了,有些地方稀稀拉拉种了几棵树就算绿化了,怎么行。这产生了一个缺点,这种宣传只能促进睡觉,要回去研究一下,怎样才算“化”了。种上是“化”还是长出来是“化”?除四害也不要轻易宣布“四无”。除四害今年只搞一下,取得经验,再看一看。今年要把许多事情搞完,我不相信。

报纸不要简单宣传指标,要多宣传措施,多宣传先进经验。要搞水利“好多吃大米”,这样就到问题本质了。口号新鲜,人民就看到前途了,很高兴。不要过早宣传水利化,“化”了明年怎么办?还有什么干头呢?做事情应留余地。

苦战三年以后,还需要再战。我在《正确处理人民内部矛盾》的报告中提的不是三年,而是还要战斗十几年,不要把事情简单化了。宣传要给自己留余地,讲问题要看远一点,以后不要说什么“化”了,“四无”了,将来要变成“四有”城了怎么办?宣传多了,以后就不好动员了。世界上很多事情都有真有假,没有假也没有真,在运动中的事,打了折扣才可靠。干什么都要踏实些。说苦战三年水利化,我怀疑如果如此,将来我们子孙干什么呢?今后十年内会不会遇见几次大水灾,大旱灾?三个大灾两个小灾应考虑。问题不由你我决定,三年内只能逐步改变面貌。在若干年内只能管地,还管不了天。如果来了灾这个账怎么算?三年基本改变面貌,如果发生灾,就不是三年基本改变面貌,计划中的大灾除外写上去,要留有余地。

怎样才叫绿化,种上树不算绿化,真正绿化是从飞机上看一片绿才算。现在坐在飞机上看还是一片黄色。年年有工作做,不是没事情。五年能搞掉四害,就算好。

总之,做事情要留有余地,要务实。粮食到手,树木到眼才算数。要比措施,比实际。现在很多指标还不是群众的东西,是领导在头脑中的,好多还是会议上的,到秋后看一看再说。今年是历史上大跃进的一年,把经验总结一下。

宣传工作要务实。报纸宣传要实际。要深入、细致、踏实。不要光宣传指标。现在我们的宣传只注意宣传多快,对好省宣传的不够,不好、不省怎么会改变面貌?

好大喜功是需要的,但大话是不需要的,华而不实是不好的,如果华而不实,喜功便会无功。不是喜大而是喜小,结果会轰轰烈烈之后无功而返,这就不好了。

生产高潮方面的指标,现在我担心会不会再来一个反冒进,今年干劲这样大,如果得不到丰收,群众情绪受挫就会反映到上层建筑上来,社会上就会有人说话了,喊“冒进”了。民主人士、富裕中农,党内有右倾思想的人,就会出来刮台风;观潮派“楼看沧海月,门对浙江潮”就会说话,群众有怨言,就会从上而下的反映意见,影响上层建筑。要在党内讲清楚,党内要有精神准备,给地县委讲清楚,如果收成不好,计划完不成怎么办。

现在我们的劲头很大,不要到秋天泄气,要搞措施,到十二月比实际,要看结果,吹牛不算数。实际上九月便会看出,比输了,活该。不要浮而不深,粗而不细,华而不实。今年像平津战役、淮海战役那样子,增产有希望,办法是放手发动群众,一切通过试验。湖北省有这样的话:“鼓起眼睛看丰收,干部带头,革新先试验,干劲加办法,跃进会实现。”向农民讲清楚,可能某些地区有天灾,要鼓起眼睛看丰收,也要准备无丰收。要特别注意深翻地换土,大有味道,宁可一亩地花几百个工也花得来。

调节生产节奏。做一段,休息一段,劳逸结合,要有节奏地,要波浪式前进。连续作战是由各个战役组成的,做了一个时间休息几天,悠哉悠哉,很必要。

压缩空气,河南一年要实现几个化,当然现在我们不要说他们过火了,但某些口号要调整一下,登报时要小心些。压缩空气,空气还是那么多,氧气并没有减少,只是压缩,变成液体、固体。反冒进是将氧气砍掉一半。我们压缩还要加氧气。

