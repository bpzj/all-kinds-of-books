\section[机关枪迫击炮的来历及其他(一九五九年八月十六日)]{机关枪迫击炮的来历及其他}
\datesubtitle{(一九五九年八月十六日)}


昨天上午我说,以《马克思主义者应当如何对待革命的群众运动》为题的那个文件,“不知道是那一位秀才同志办的,他算是找到了几挺机关枪、几尊迫击炮,向着庐山会议中的右派朋友们,乒乒乓乓地发射了一大堆连珠炮弹”。这个疑问,昨天晚上就弄清楚了。不是庐山的秀才同志,而是北京的×××同志和他的两位助手,发大热心,起大志愿,弄出来的。庐山出现的这一场斗争,是一场阶级斗争,是过去十年社会主义革命过程中资产阶级与无产阶级两大对抗阶级的生死斗争的继续。在中国,在我党,这一类斗争,看来还得斗下去,至少还要斗二十年,可能要斗半个世纪,总之要到阶级完全消灭,斗争才会止息。旧的社会斗争止息了,新的社会斗争又起来。总之,按照唯物辩证法,矛盾和斗争是永远的,否则不成其为世界。资产阶级政治家说,共产党的哲学就是斗争的哲学,一点也不错。不过,斗争形式依时代不同有所不同罢了。就现在说,社会经济制度变了,旧社会遗留下来残存于相当大的一部分人们头脑里的反动思想,亦即资产阶级思想和上层小资产阶级思想,一下子变不过来,要变需要时间,而且需要很长的时间。这是社会上的阶级斗争。

党内斗争反映了社会上的阶级斗争,这是毫不足怪的,没有这种斗争才是不可思议。这个道理过去没有讲透,很多同志不明白,一旦出了问题,例如一九五三年高饶问题,现在的彭、黄、周、张问题,就有许多人感到惊奇,这种惊奇,是可以理解的。因为社会矛盾是由隐到显的。人们对于社会主义时代的阶级斗争的理解,是要通过自己的斗争和实践才会逐步深入的。特别是一些党内斗争,例如高、饶、彭、黄这一类斗争具有曲折复杂的性质。昨日还是“功臣”,今日变成祸首,“怎么搞的,是不是弄错了?”人们不知道他们的历史变化,不知道他们历史的复杂和曲折,这不是很自然的吗?应当逐步地、正确地向同志们说清楚这种复杂和曲折的性质。再则,处理这类事件,不可用简单的方法,不可以把它当作敌我矛盾去处理,而必须把它当作人民内部矛盾去处理。必须采取“团结——批评——团结”,“惩前毖后,治病救人”,“批判从严,处理从宽”,“一曰看,二曰帮”的政策。不但要把他们留在党内,而且要把他们留在省委员会内、中央委员会内,个别同志还应当留在中央政治局内。这样,是否有危险呢?可能有。只要我们采取正确的政策,可能避免。他们的错误,无非是两个可能:第一,改过来;第二,改不过来。改过来的条件是充分的。首先,他们有两面性,一面,革命性,一面,反革命性。直到现在,他们与叛徒陈独秀、罗章龙、张国焘、高岗是有区别的,一是人民内部矛盾,一是敌我矛盾。人民内部矛盾可能转化为敌我矛盾,如果双方采取的态度和政策不适当的话;可能不转化为敌我矛盾,而始终当作人民内部矛盾,予以彻底解决,如果我们能够把这种矛盾及时适当地加以解决的话。下面的这些条件是重要的。全党全民的监督,中央和地方的大多数干部的政治水平,比较一九五三年高、饶事件时期大为提高了,懂事了。庐山会议上这一场成功的斗争,不就是证据吗?还有,我们对待他们的态度和政策,一定要是符合情况的马克思主义的态度和政策,而我们已经有了这样的态度和政策。改不过来的可能性也是存在的。无非是继续捣乱,自取灭亡。那也没有什么不得了。向陈独秀、罗章龙、张国焘、高岗队伍里增加几个成员,何损于我们伟大的党和我们伟大的民族呢?但是,我们相信,一切犯错误的同志,除陈、罗、张、高一类极少数人以外,在一定的条件下,积以时日,总是可以改变的。这一点,我们必须有坚定的信心。我党三十八年的历史提供了充分的证据,这是大家所知道的。为了帮助犯错误的同志改正错误,就要仍然把他们当同志看待,当作兄弟一样看待,给以热忱的帮助,给他们以改正错误的时间和继续从事革命工作的出路。必须留有余地。必须有温暖,必须有春天,不能老留在冬天过日子。我认为,这些都是极为重要的。


