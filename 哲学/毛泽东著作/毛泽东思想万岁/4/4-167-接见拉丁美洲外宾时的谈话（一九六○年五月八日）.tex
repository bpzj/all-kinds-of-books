\section[接见拉丁美洲外宾时的谈话(一九六○年五月八日)]{接见拉丁美洲外宾时的谈话(一九六○年五月八日)}


欢迎各位朋友。

今天都是拉丁美洲的朋友,你们都是我们的朋友。我们和你们有许多共同点,中国和拉丁美洲各国地位大体是相同的,我们要站在一条战线上。当然各国的具体情况不同,但是我们有共同点,都处在一个不很发达的地位。我们中国在政治上虽然获得了独立,经济发展仍然是很差的,过去有很长时间处于停滞的局面。你们大体上知道,中国过去是处于半殖民地的地位,名义上是独立的,实际上是外国人控制的,这样的时间有一百多年。一百多年当中,帝国主义就在中国人民身上刮了许多油水走了。我们没有工业。中国民族工业是受压迫的。中国是受帝国主义、封建主义和买办资本主义统治的国家;人民很穷,每年要饿死很多人;没有文化,百分之八十的人是文盲,因此我们是一个“一穷二白”的国家。譬如钢铁,掌握在中国政府手上的钢铁,也就是蒋介石所管的钢铁,年产量只有十万吨。在座的有巴西朋友,听说巴西每年有一百六十万吨钢,而蒋介石灭亡的一九四九年只产钢十万吨。全中国只有四百万产业工人,但有广大的贫农,占几亿人口。所以工人阶级依靠贫农,结成联盟;也就是说,无产阶级同半无产阶级首先结成联盟,然后再联合中农,然后再联合民族资本家,再联合爱国的知识分子。这样一来,一百人中间我们就联合了九十个人,反对我们的人就只有百分之十了。这样一团结,革命就能胜利了。还有国外的帮助,国际的帮助,这种帮助也是一种声援的性质。我们长期被敌人围困,不可能从外国取得物资帮助,但是我们取得道义上的帮助。譬如苏联以及其他社会主义国家,亚洲、非洲和拉丁美洲民族独立运动的力量,都对我们有帮助。今天在座的是拉丁美洲的朋友,你们的斗争,就是对我们的帮助。因此我们要感谢你们。你们牵制了帝国主义的力量,譬如古巴,就在美国的旁边,你们的斗争帮助了我们。其他拉丁美洲的国家,如智利、哥斯达黎加、厄瓜多尔、哥伦比亚、巴西、阿根廷、秘鲁,我们并没有建立国家与国家之间的外交关系,但是你们的工作帮助了我们。我们很感谢,我们是这样看的。就是说,你们是我们的朋友,不是我们的敌人,在座的没有我们的敌人,是不是?(全场大笑)我们没有利害冲突,我们有一种友谊关系,我们希望你们胜利。

拉丁美洲有差不多两亿人口,美国只有一亿多人口。美国人并不都是反对我们的,要把美国的人民和资本家区别开来。美国的垄断资产阶级也就是对你们不利的那个阶级。比如古巴的糖,就被那个资产阶级所垄断。听说智利的铜,哥伦比亚的石油也是受美国控制的。美国在巴西有没有投资?(巴西朋友说:很多。)阿根廷有没有?(阿根廷朋友答:也很多。)美国人就是钱多,(全场活跃)他们的钱不仅是剥削本国工人阶级和农民得来的,也是从各国剥削得来的,他们的财富建立在对我们的剥削上。昨天我会见了非洲十二个国家和地区的朋友,非洲人口和拉丁美洲差不多,他们是两亿一千万,拉丁美洲是一亿九千万,只比你们稍多一点,差不多相等。合起来,这就是四亿人口。亚洲人口有十四亿五千万,苏联两亿人口,其他社会主义国家有一亿。全世界共有二十七亿人口,西方世界只有五亿人口,但也还

要加以区别,大多数是好的,殖民主义者只有少数人。全世界的殖民主义者同他们在各国的同盟者——例如中国的蒋介石,朝鲜的李承晚,土耳其的曼德列斯,古巴的巴蒂斯塔。加起来顶多是一亿。就算占人口的十分之一,也只有二亿七千万,人相当的少。十个指头只占一个指头。有一些人现在还不觉悟,譬如美国工人中也有同资产阶级合作的,他们总有一天会觉悟起来。事情还是靠人民决定。究竟是巴蒂斯塔的力量大,还是“七·二六”运动的力量大?在三年以前,即在一九五七年七月二十六日以前,你们从墨西哥登陆时只八十二个人,而巴蒂斯塔那时有很大的力量,他曾经杀死两万古巴人。我们过去也是手无寸铁的,权力是在蒋介石手里,后来又有美国人帮助他。但是蒋介石还是打败了。巴蒂斯塔也是打败了。由此得出一条结论;反动派,就是帝国主义分子和他们的走狗,他们的力量形式上很大,实际上有的已经被推翻了,有的将要被推翻。手上没有武器的人,推翻那些手上有武器的人;被剥削阶级推翻剥削阶级;穷人把富人推翻了。是我们推翻敌人,不是敌人推翻我们。还是人民决定。团结百分之九十的人就有办法。要组织统一战线,团结一切可以团结的人。我们不但团结了工人、农民,而且团结了民族资本家,我们到现在还是跟他们一道。团结民族资产阶级和他们的知识分子,教授、教员、艺术家、工程技术人员、新闻记者,凡是赞成民族独立民主革命的这些人都团结起来。

我们解放十年了。过去只有十万吨钢,去年我们就有××××多万吨,×年可能达到××××万吨左右,×年可以超过日本和法国,×年再看吧,如果能有××万吨的话,就可以超过英国。所以事情是由人办起来的。我看我们的事情好办,包括拉丁美洲、非洲、整个亚洲和社会主义各国在内。当然困难是有的,有些国家受外国控制,有帝国主义的军事基地。但是人民要斗争。南朝鲜、土耳其的人民已经起来斗争,这对你们都有帮助。中国的工作对于你们也许有些帮助。虽然我们的信仰不同,社会制度不同,但是可以团结起来。

还有一个问题,对西方国家怎么办呢?几百年来,世界是由他们决定的。听说他们的力量还相当大。要防止战争,争取和平,要不要跟他们打交道?我们认为应该同他们打交道。因此,我们支持大国会议,能够裁军,不打原子大战就好。但是专依靠大国会议行不行呢?不知道你们意见怎么样?专依靠,我们看来恐怕不行。(外宾中有人说:要两条腿走路。)两条腿走路,我赞成,就是同时要依靠各国人民的斗争。而且第一是依靠各国人民的斗争。第二才依靠会议。譬如我想和美国人打交道,可是他不干,并且还影响许多国家,譬如法国、西德、日本都不同我们打交道,那有什么办法?我们是不是睡不着觉呀?不,睡不着觉的不是我们,他们想把中国分为两个,承认台湾,不承认我们,在联合国让蒋介石代表中国。蒋介石是中国的巴蒂斯塔。巴蒂斯塔跑到葡萄牙,蒋介石跑到台湾,都是在美国保护之下。我们认为美国人有一种不好的习惯,喜欢帮助坏人。(全场大笑)喜欢帮助巴蒂斯塔、蒋介石、李承晚、阿登纳、岸信介、佛朗哥。他们喜欢那些人,脾气怪得很。(全场大笑)各位朋友有什么意见?

(古巴军队总督察:我想大家都同意你的意见。)

很好,我们能够取得共同一致的观点,所以我们是朋友。我们的朋友很多,全世界顶多只有十分之一的坏人。

(古巴军队总督察:十分之一还不到,有些人受欺骗,所有的人民都是高贵的。)

可能不到十分之十。但是人民不能包括蒋介石、李承晚、巴蒂斯塔,他们是人民的敌人。总之,只要我们团结起来,事情就好办。什么都要依靠人民。我们今年可能取得××多

万吨钢,但是我们有这么多人口,按人口比例来说,还不多。我们人很多,粮食不多,棉花不多,钢铁不多,但是我们相信可以发展,因此需要时间,需要和平,需要朋友。

你们准备在中国呆多久?(一位外宾:时间由中国工会决定。中国很大,需要很多时间才能参观到。)

由你们决定吧。想呆多久就呆多久吧。还有朋友发表意见没有?

〔在一些外宾发言以后〕再说几句,有些朋友说到中国来学习的。我们要互相学习,互相交换经验,尤其是古巴的经验。古巴经过两年半的斗争,我们经过二十多年。条件不一样,古巴离美国很近,取得了胜利,我们离美国很远,也取得了胜利。中国的经验是在中国的土地产生的,中国有中国的条件,希望朋友们作分析,哪些是优点,哪些是缺点,有哪些是经验,有哪些是错误。现在我们工作中还有一些错误,我们有个整风运动,每年一次两次。中国犯的错误,你们研究也有意义,也就可以避免再犯类似错误。经验不能照搬,只能作参考。

有的朋友提到建立人民公社问题,我们的人民公社就是把合作社扩大发展起来的,把二十个、三十个合作社并成一个公社。人多了,平均五千户一社,少的二千户,多的一万户。人多,土地多,力量大,所以它比合作社更好。有的朋友也想办人民公社,如果要办,我建议不要采取人民公社这个名称。一九五八年,一九五九年,因为这个名称,我们挨骂。假如不采用这个名称,把合作社扩大也可以。但是不是后悔?我们不后悔。这个名称是群众取的,就是这个省——河南省开始办起来的。一九五八年夏季成立了人民公社。杜勒斯骂我们,说人民公社是不好的,不能长久的,进行奴隶劳动,没有自由,又拆散家庭,丈夫见不到老婆,妈妈见不到孩子。还有一个大跃进,也是骂的名称。我们如果改个名称叫高速度发展,可能就不挨骂了。总之,只要稍微改变一点习惯,人家就驾。但是,我们已经改变了,他们也没有办法。我们离美国很远,离上帝很远。(全场大笑)罗马教皇不喜欢我们,这是没有办法的。人民公社是不是奴隶劳动?是不是自由?在座的朋友已作了答复。中国人过去是奴隶劳动,就是为帝国主义及其走狗工作,替他们当奴隶。现在他们不当奴隶了,自己掌握了自己的命运。敌人说只有台湾有自由,大陆上是奴隶劳动。现在又轮到古巴身上了。美国说巴蒂斯塔领导有自由,××××领导下是奴隶。他们的逻辑就是这样。西方国家和美国的逻辑和我们是两套。朋友们,哪个对,将来看吧!他们帝国主义制度那么好,我看不出。所以请他走路,回老家去。(笑声)总有一天,美国人民不喜欢帝国主义制度。

艾森豪威尔因为没准备好,又看到我们力量大,不敢打世界大战。但是很难说,有两个可能。一是有争取持久和平的可能,要为此而奋斗;另有一个可能还有大战。阿根廷朋友讲的好:“要保持弹药是干燥的”。东方有个日本,一亿人口,是军国主义,和美国签订侵略性的军事同盟条约,它和西德一样,正在复活军国主义。西德和日本,都是美国喜欢的,支持的。但是另外一种情况是,日本人民正起来斗争,他们不赞成岸信介政府,不赞成日美侵略性的军事条约,明天将有几百万日本人民进行示威。我们要支持日本人民。南朝鲜、土耳其人民的斗争,对我们对你们都有帮助,日本人民的斗争对你们也有帮助。

祝你们斗争胜利!大家团结起来!


