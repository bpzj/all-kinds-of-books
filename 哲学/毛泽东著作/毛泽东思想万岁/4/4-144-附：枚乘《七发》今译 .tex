\section[附:枚乘《七发》今译 ]{附:枚乘《七发》今译 }


楚国的太子有了病,一个吴国来的客人去问候他。客人道:“听说太子贵体不安,是不是好一点了呢?”太子道:“还疲乏得很,谢谢你关心。”客人趁机进言道:“现今天下安宁,四方和平。太子正在少壮之年。或许你长期贪恋安乐,日日夜夜没有节制。邪气侵犯,在体内凝结阻塞,以至于心里烦乱郁闷,情绪恶劣好像醉酒似的。常常心惊,睡不安宁。中气不足,听觉不灵,厌恶人声。精神散漫,好像百病齐生。耳目昏昏,喜怒无常。这样下去日子长了,性命就要不保。太子你是不是这样呢?”太子道:“谢谢你。我倚靠国君中力量,常常有些享受,但是还不到你所说的程度。”客人道:“一般贵家子弟,住在深宫内宅之中,内有媬母照料,外有师傅陪伴,要想交结朋友是不可能的。饮食一定是淳美香嫩,肉肥酒浓。衣服一定是温暖轻细而且多,永远热得像过夏天。这样,虽有金石的坚强,也是要销损而解体的,何况筋骨之间呢?所以说,放纵耳目的嗜欲,贪图肢体的安逸,就要妨碍血脉的和畅。出入都坐车子就是瘫痪之兆。幽深和清凉的宫室就是寒热病的媒介。妖姬美女就是斫伤生命的斧子。美味的酒肉就是烂肠子的毒药。如今太子皮肤太细嫩,四肢不灵活,筋骨松弛,血脉阻滞,手脚无力。听使唤的,前有越国的美女,后有齐国的佳人。往来游玩吃喝,在曲折幽深的房子里纵情取乐,这简直是把毒药当着美餐,和猛兽的爪牙戏耍呀。过去的影响已经深远,而又拖延不改,即使叫扁鹊来治疗,巫咸来祷祝也来不及啦。现在对付你这样的病,需要世上的君子,知识广博的人,有机会时给你出出主意,改变改变旧习和成见,常在您的身边,做您的辅佐。那么沉沦的享乐,荒唐的心思,放纵的欲望,就无从产生了。”太子道:“等我病好了一定照你的话行事。”

客人道:“现在太子的病,可以不用药石针炙而用精深的理论来治疗。您不想听听吗?”太子道:“我愿意听。”客人道:“龙门山的桐树,高达十丈还没有分枝。中心纹理盘曲,树根分布很广,上有千仞的高峰,下有百丈的深涧。急流逆波摇荡它。它的根半死半生。冬天的烈风霜雪刺激它,夏天的雷电打击它。早愿有鸟鸣晚上有鸟宿。孤鸿在上面呼号,鹍鸡在下面哀叫。于是在秋冬之间,叫最精于弹琴的师挚砍下它来做成琴。用野茧的丝做弦,用孤子的带钩做隐,用九子寡母的耳珠做琴徽。叫师堂弹奏那名叫《畅》的曲子。叫伯牙来唱歌词。那歌词是这样:“麦芒尖尖啊野鸡晨飞,面向废墟啊背倚枯槐,道路穷绝啊溪流遇回。”鸟儿听了,拢起翅膀不再飞。野兽听了,垂下耳朵不再走。虫蚁听了,支起咀巴不再前进。这是天下最感动人的歌声,太子您够勉强起来听听吗?”太子道:“我的精神不好,怕不能够。”

客人道:“煮熟小牛腹部的肥肉,加上笋蒲。用肥狗的肉来加羹,盖上一层石耳菜。煮饭用楚地的粳稻,或是雕胡的米粒。粘的饭成团,松的到口就散。教尹伊来煎熬,易牙来调味。熊掌炖得烂烂地,加些芍药酱。将薄切的兽脊烤来吃,新鲜的鲤鱼做鱼片。配上秋天变黄的紫苏,白露时节的蔬菜。兰花泡的酒,舀来漱口。还有野鸡肉,豹子胎。少吃些干的,多喝点稀的,就像热汤浇在雪上似的容易消化。这也是天下最可口的了,太子您能够勉强起来赏赏吗?”太子道:“我的精神不好,怕不能够。”

客人又说:“钟、岱等地的良马,年龄到了就用来驾车,跑在前头的像一飞鸟,跑在后面的像距虚。用爵麦喂马,马都成了急性子的,配上结实的马缰,赶上平坦的大道。让伯乐前后视察,让王良、造父来赶车,秦缺,楼季做车右,他们能制止乱跑的马,能扶起翻了的车子。然后和人家下千镒黄金的大睹注,在千里的长途上赛跑。这是天下最快的车子,太子您能够勉强起来乘坐吗?”太子道:“我的精神不好,怕不能够。”

客人道:“登上景夷台,南望荆山,北望汝水,左江右湖,那乐趣是天下少有的,这时可以叫博学有辩才的人,陈说山川的本原,尽举草木的名称,排比事物,编成文辞以类相连。在一番徘徊游览之后,下台到虞怀宫中摆酒。游廊连接四面有檐的建筑,有台的城,结构重迭,采画纷纭。车道交错,池水曲折,养的鸟几有混章、白鹭、孔崔、鹨鸡、天鹅、鹓鶵、鵁鶄。有些鸟头上有翠毛,颈上有紫缨。螭龙鸟、德牧岛,鸣声相和。鱼类腾跃,鳞翅振奋。清净的水边长着蕏草和水蓼,还有蔓生的草和芳香的莲。女桑和河柳,素叶而紫茎,苗松、枕木和樟木、枝条上达青天。梧桐和棕树,一片林子望不到头。各种浓郁的芳香,和种种的声音相杂。树头舒缓地随风摇动,树叶翻复忽明忽暗。大家入座畅饮,乐声振荡快人心意。这时让景春来助兴,让田连来调音。各种美味陈列在面前,各色肴饭都具备。娱目的是精选的美色,悦耳的是流转的好音。这时发出旋风般的《激楚》之曲,飘起郑卫的悠扬乐声。叫先施(西施)、征舒(夏姬)、阳文、段干、吴娃、闾娶、傅予等美人,拖着各色的裙裾,垂着燕尾。眼光对人挑逗,表示心里暗许。她们引流水洗濯,身上带了杜若的香气。头发上好像笼罩着尘雾,涂上了兰膏。换上便服来待奉。这也算是天下美妙盛大的娱乐了,太子您能够勉强起来玩耍吗?”太子道:“我的精神不好,怕不能够。”

客人道:“那么我要为您训练骐骥,驾有窗的轻车,您坐在这样壮马拉的车子上,右边带着夏后氏箭囊里的劲箭,左边带着黄帝的鸟号之弓,去到云梦的林中,围绕生长兰草的洼地奔驰,奔到江边然后缓缓地行进。在青苹之间休息,迎着清风。陶醉在春天的空气里,满怀春意的心为之动荡,然后追赶狡兽,许多支箭射中了轻捷的飞鸟。这时犬马的本领发挥尽致,野兽的脚困乏万分,看马和御车的人使尽了他们的智慧和技巧。威慑着虎豹,吓坏了鸷鸟。奔马的铃当当乱响。跨越游鱼,抓着麋角,踢倒麕兔,践踏麋鹿。这些动物都流着汗珠,窘迫屈伏。无伤而吓死的,足足装满随从的车子。这是规模最壮大的田猎,太子您能够勉强起来干一场吗?”太子道:“我的精神不好,怕也不能够。”但是他这时眉间呈现了喜色,渐渐地这喜色几乎布满了整个面部。

客人见太子有高兴的样子,就更进一步道:“打猎时夜火烧灭,兵车雷滚。旌旗高举,装饰着鸟羽和牛尾,整齐而众多。放开马蹄追逐,因为爱得野味,个个争先。为拦捕野兽而焚烧过的地面非常广阔,远望去可以看见边缘。然后把毛色纯一躯体完整的牺牲献到诸侯之门。”太子道:“妙啊!我愿意再听下去。”


客人道:“这还不曾完呢。那时丛林之间和沼泽深处,烟雾弥漫,野牛和老虎都跑了出来。勇壮的人非常强悍袒露着身体空手上去搏斗。白刃闪闪,矛戟纷纷。收取猎获物的人同时掌管记功,赏赐银绢铺下杜若,盖上苹草,为牧人布置了筵席。有美酒和佳肴,有细切的烧烤肉,款待宾客。盛满的酒杯并举,言语入耳动心,诚实无悔,说一不二。忠信的表情就像刻在金石上一样。高声歌唱,长久不厌倦。这真是太子所喜爱的,能够勉强起来去玩玩吗?”太子道:“我很愿意跟你们去,就怕我这病人成为大夫们的累赘。”看起来太子的病已经有起色了。


客人道:“我们将在八月十五日,同诸侯和远方来的朋友们兄弟们齐到广陵的曲江观涛。初到时还不会见到涛的形状,不过看水力所到之处,已经令人大大吃惊。看那水力所驾陵的,所拨起的,所激乱的,所结聚的,所洗荡的,纵然是有才能辩的人,也不能详加描述。既而目迷神乱,心惊胆战。浪涛滚滚而来。初时迷芒一片,少时奇峰突起,忽然声势浩大,超越旷远。似乎要陵驾南山,以望东海。那浪头几乎上冲苍天,而它的边际煞费想象。这时观赏奇景无穷无尽。然后注意东方日出之处,只见浪头迅速地乘流而下,不知要奔到何处才停。有时奔乱曲折地奔泻,忽然纠结着流去再不回头。浪涛冲到南岸然后远逝,看的人心神紧张更加疲惫,涛的形象在心中久久不散,直到天明,然后才心安神定。这时胸中受到荡涤,五脏受到冲刷。洗净手足,又洗颜面、发、齿。驱逐了倦意,去净了尘垢,困惑消失,耳目开朗。在这时候纵然是久病的人也要伸直驼背,招起跛脚,张开瞽目,打通聋耳来观看它,何况小小烦闷病酒的呢?所谓启发昏蒙、解除迷惑,都不消说了”。太子道:“妙得很!那么涛究竟是一种什么气呢?”

客人道:“那是不见于记载的,不过我曾听师付说过,涛有三点似神非神:雷声轰隆传达百里是其一;江水倒流,海水上潮是其二;山中云气吞吐,日夜不息是其三。初时江水平满而流得极快,然而波涌涛起。开始的时候,水淋淋而下,像许多白鹭在降落,再进一步,就滚滚翻翻,像白车白马,张着白的帷盖。当浪头像云堆似的,纷纷扰扰,就如三军奋起。当两旁的奔流忽然涌起,飘飘地,就如将领乘着轻车在统御军队。驾车的是六条蛟龙,跟着太白帅旗。忽而但见白色的虹蜺在奔驰,前后相连不断。高高低低,前前后后,挨挨挤挤。又见壁垒重重,人多马众有如军队。大声轰轰,漫天沸腾,势不可当。看那两边岸旁,汹涌激怒,盛气冲突,向上击刺,向下投石。正如勇壮的战士,猛扑无畏,冲营抢渡。小湾小港无所不到,跨出崖岸越过沙滩。遭遇者死亡,阻挡者崩溃。开始的时候从或围津口出发,逢山陇而回转,遇川谷而分流。到青篾时盘桓回旋,到檀桓时无声急进,到伍子山速度减低,远奔胥母之战场。上赤岸,扫紫桑。横奔像雷滚,显然威武,如发狂怒水声混混,犹如擂鼓。涛在岸合处被阻而又发怒,上升远跳,大波扬起,在名叫借借的隘口大战起来。这时鸟来不及飞起,鱼来不及转身。兽来不及逃走。纷纷忙忙,也像波涌云飞一般的混乱。向前扫荡南山,回头冲击北岸,丘陵颠复,然而又荡平两岸。浪头高峻,破坏堤防,直到全胜方才罢休。然后急速地奔泻,任意泛滥,十分横暴。鱼鳖都不能自主,颠倒翻复,稀里花拉,起伏不绝,连神物也觉得骇异。这种景象难以尽述,简直令人吓倒,或者吓得昏头昏脑。这是天下最奇的奇观,太子你能勉强去看看吗?”太子道:“我的精神不好,不能去啊!”

客人道:“我要为太子推荐有道术的,有才智的人才,如庄周、魏牟、扬朱、墨翟,便蜎、詹何之类,让他们谈谈天下最精深的道理,讨论万事万物的是非,让孔丘和老聃来总结,孟子来稽核,这样,一万个问题也错不了一个。这是天下最切要最美妙的理论了,太子您可以听听吗?”于是太子扶着几案站起来,说道:“我现在一切疑虑都消散了,好像已经听到圣人辩士的言论。”这时他出了一身透汗,忽然之间,老病全消。

《七发》简介

《七发》是西汉著名赋家枚乘的作品。枚乘主要的活动时间是汉文帝和汉景帝二代。(公无前179——143年)

《七发》的结构是用七段文字描写七件事,开头另加一段序曲。叙述缘起,并借一主一客的问答把各段联系起来。


