\section[《前十条》和《六十条》为什么能调动人的力量?(一九六四年)]{《前十条》和《六十条》为什么能调动人的力量?}
\datesubtitle{(一九六四年)}


因为它解决了人民内部的矛盾,领导和被领导的关系,把力量组织起来。人是生产力诸要素的首要因素。人,劳动手段(包括畜力、农具、肥料等)劳动对象,这是生产力的三大要素嘛,实行了《六十条》《双十条》,还是原来的那些人、畜、农具、土地等等,但是结果却大不相同了。

当讲到农村社会主义教育后“贫下中农心正,地富心服”,那也不一定。贫下中农心正,以后有可能正,也有可能不正。地富心服,也不一定都服。

当讲到“小站有个假劳模不劳动,一年得一万五千多工分、二千元。”那是剥削阶级了,撤职,一定要撤。账目要算清楚,一定要退赔。

还有小霸王之类,也要整。

贪赃枉法,有资产阶级,也有无产阶级,事情就是这样复杂,没有才怪哩。有个对立面也好。

当讲到曲阜陈家庄陈玉梅被打下去,亩产从五百斤降到三百斤,去年再上来把亩产从三百斤翻到六百斤。还是靠自力更生。事情总是会起变化的,把好的打下去,比如把莫洛托夫打下去,从五百斤降到三百斤。莫洛托夫上来,又从三百斤翻到六百斤。陈玉梅这些人,小学没上过,大学也没上过,可是能把事情办好。

修正主义上台,也就是资产阶级上台,就是这么惨。像陈家庄那样,砍树铲葡萄,把房子里的桌椅都搬掉了,好人上台,又都变了。赫鲁晓夫也要把苏联变成这样砍树铲葡萄,只要有利,向魔鬼借钱也愿意。我们不走这条路,魔鬼不给我们贷款。贷款给我们也不要。我们要靠陈家庄陈玉梅,大寨陈永贵。

不要只看到阴暗面,凡是事情总是一分为二的,百分之十的模范,带动大多数,整百分之十到二十的坏人。

有些支部,被不好的老党员霸着。他们有一条顶县委的办法,你知道有多少中央委员,姓甚名谁,答不上来,他就说你问题解决不了。上面的人根本下不去,不了解情况。\marginpar{\footnotesize 205}

