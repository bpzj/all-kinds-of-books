\section[在青岛会议上的讲话(一九五七年七月十八日)]{在青岛会议上的讲话}
\datesubtitle{(一九五七年七月十八日)}


(谈到工商界)现在是过第三关,过第三关顶少还要十年。历史上包下来一批王八蛋、一路来敲锣打鼓,拥护了七年是假的,只是到了现在,他们就翘尾巴了。每年召开人民代表大会,政协会议,总是要对付他们一场。通过法案,他们都举手,下去视察,他们就找岔子,并且搞组织活动。估计到他们有一部分人随时会叛变,但历来都没有暴露他们的办法。现在提出处理人民内部矛盾,加上整风,右派分子的头子就翘尾巴,反共反社会主义了。把孙行者的尾巴打下去,花果山的猴子就不神气了。抓住改造,逐步把领导权转到左派和中间派手里。各省市基层把民主党派,教育界的领导权拿过来。学校实行党委制可以考虑,但我们必须领导学校,领导权过去是不巩固的,有一部分领导权是在人家手里,缺乏战斗力,一时搞得天昏地暗,例如清华大学党团员占百分之八十,但党的领导并不巩固。科学技术界,新闻,艺术,出版,教育,卫生界,都有左中右,我们不占领导阵地,就搞不成社会主义。总之,要搞意识形态,要搞知识分子。过去搞五大运动,三大改造,没有好好搞教育界。过去省委书记怕见大学教授,现在见了一下,见出名堂来了。再摸一个时期,就更清楚了。

敌我双方对政治形势的估计都不大对头。敌人方面对形势估计错了,我们对自己的估计也不是那么正确,在鸣放期间,觉得我们的力量比较小,对工农兵——我们的基础估计不足。现在看来,在工农党团员中只有少数,大部分是好的,我们也为假象所迷惑。

多次人民代表大会,政协都要斗一场。名义上在台上参加政权,大体上顺着我们,实际上是反对派,拥护是假的。到去年苏共二十次代表大会以后。我们提出的“两百”方针,加上匈牙利事件,他们开始翘尾巴了。首先看出这个问题的是陈××等几个同志。说百花齐放可以,但放的不妙,放出牛鬼蛇神了。周恩来二月回来以后,罗隆基公开批评总理的外交报告,这是从来没有过的。还有几个学校闹事,工厂闹事,农村闹事,以为形势不妙了,其实就那么几个人,就是大学也是九个指头是好的,只有一个指头不好。反右只有一个多月,就看出来了。民主党派打他百分之二十,也还有百分之八十,经过六月,七月半个月,我们就完全清楚了。凡事皆要加以分析。分析了就有底了。这一下我们的心就放下来了。

敌人是把自己的力量估计过高,对工农估计不足,对左派也估计不足,这一点右派同我们一样。在中国搞匈牙利是搞不起来的。放,还应当有决心,不然就放不起来。

还有,广播,电话,电报,邮电要抓住,不让民主党派去发展。民主党派要抓住其中的右派,狠狠的打,在打的当中,建立领导权。抓住这个时期搞他一年。如果搞到明年六月,全国人大改选,我们假如再提他们的名,他们自己都不好意思的。经过反右派,一定要换掉一大批。地方,明年年初就要开始政治改组,三月各省把改选搞完。我们这样估计:打算打一年。现在报纸上这样积极地吹,总吹也不行,还要搞点别的。右派分子虽然多,但头儿已经打了,打右派主要是打头子,把头子打了。就不好作怪了,以后采取剥笋政策。其他的人,那怕也有些错误×他是领头××,×××,×××怎么打法?×××反人民,不公开反共不闹事,但死也不改。右派分子人数虽少,但不可小看,对右派分子要剥笋。

知识分子多数人不是马克思主义者,对社会主义、工农事业不那么有感情。但他们有这样种种倾向,只要他们今天倾一点,明天倾一点,再过十天功夫就改造了。山东提议:要反右派的事情向工人农民作宣传。地主、资产阶级好像都在替工人,农民说话。我们要讲右派怎样进攻,我们怎样捉住了一批右派(如广东罗翼祥)。对工人农民手工业者,一般不要扣右派帽子。

合作化没有希望?有这样一些富裕中农,本想叫他们慢慢进社,也挡不住,进来了又想退出去。是否让他们退出去?从阶级路线来看,少数不愿在社的,可以让其退出。山东临沂地区一百多万户,退了一万户,过去闹事最多,现在是全省比较太平的地方了。退出去的不要苦他,也不扣右派帽子,将来愿意来的就来,要退的大体上有百分之一。愿意退的就允许退。今年秋天下这个决心看如何。

农村、工厂整风如何整法,要把解决的问题收集一下,准备意见。中央拟于八月十五日到二十五日开一次中央全会,地委书记也参加,研究整风与体制问题。党代表大会可在一、二月底到春节前召开。

整风问题,头一步以政治为主,结合思想(阶级敌对的、内部的);下一步以思想为主,结合政治。这里讲的政治,主要指反右派斗争。其实,同敌对阶级的斗争,多数人向我们提的意见,如傅鹰,和我们不是敌对阶级的关系,而是人民内部的矛盾,也是政治关系。如权限下放,也是政治关系。但像张轸那样的家伙,写信批评周,×,就要狠狠批评,要登报。这些人要告诉他好好重新做人,留一碗饭吃。民主人士的职务要重新安排,代表,委员,部长,厅,局长,在这次运动中有功劳的要上台,右派要下台。他们由长期共存,变为短期共存,由互相监督,变为无权监督,自己否定自己。

有职无权,人事制度,评级评薪都是政治性的问题,调整关系的问题。

每天都有高潮低潮,上报的人数要增加,并且要各行各业都反映到报上去,这样就不单调。下一步整风,思想要搞得深一些,我们自己的问题也要费点时间,认真改正自己的缺点。

今后反右派斗争就是两个字:一叫深,一叫改。对内和,对敌狠。徐州地委有百分之五十的基层干部是好的。另外百分之五十分三类:轻微错误的有百分之二十,比较严重错误,手头不干净的有百分之二十九,严重违法乱纪的只有百分之一。前两类可以主动下楼。办法是开开干部会,肯定成绩,检查错误。

什么叫正确处理人民内部矛盾?就是群众路线四个字。总是不要脱离群众,要倾听群众意见,加以分析。好像游水一样,要顺水性,或者说是鱼水关系。干部脱离群众,就不能活。过去军队在战争中都是搞三大民主,为什么现在工厂、农村里搞三大民主呢?军队是拿着枪的,都不损害干部威信,为什么县、区、乡干部不能这样作呢?对干部一个是撑腰,一个是帮助。毛病要改,又不要伤害干部。大家批评一下,一篇检讨就可以过关,要教会基层干部过关。农村,工厂不能采取市机关,学校的办法。而采取三级干部会,社员代表会,支部大会,职工代表会等,就是陈伯达同志的办法。陈伯达在福建的办法,开始区、乡干部都不赞成,后来大部分的干部选上了,选举不要党团员保证,而是充分酝酿,群众提名。各省可搞点经验。

工厂干部要下车间,还要到宿舍。

干部和群众的关系,应该是讲道理的关系,不是我打你通,我压你服。共产党不怕群众。实行民主的结果是增加威信,而不是降低威信。过去军队实行民主,就打胜仗。

反革命搞得厉害的地方,要镇压。肃反不彻底的,要杀一些人。少杀不是不杀,杀少数人是完全必要的。浙江仙居,临海两县,合作社解散百分之八十。每年准备一百万人闹事,或者还要准备多一点。

河南长葛县修飞机场赶走农民,打伤四个人,这是国民党作风残余。有些国民党作风残余的是合乎实际的,一定要整好,虽然是要经过一个过程才能彻底整好。这次整风,要想一次整好,一劳永逸,我才不信。现在要把闹事的数目扩大一点,六亿人民每年准备有三百万闹事,百分之零点五的人闹事,经过几年右派一闹,我们摸了底,就不怕了。过渡时期有个不稳定的时间,再有五年半能把社会秩序大体稳定下来,就是好的。总要相信群众的绝大多数,陈伯达讲的不是狭隘经验。

(以上是7月17日上午会议上讲话)

(谈到打右派)北京打击了一千五、全国要打出十个北京来。大树是有根的。

大地主的儿子。这是新式肃反,要好好交待地委书记,县委书记的斗争策略。

大字报是好东西。为什么我们不怕大字报?因为我们总是多数。

工厂一定不要妨碍生产,那个地方烂,崩溃越多,那个地方就越好办事。省要派有经验的干部下去,硬着头皮,大放大鸣两个星期,只听不说。

(谈到区、乡、社整风问题)每个五年计划搞两次整风,一大一小。

不要搬石头,现在我们还靠这些石头。粮食年年增产,但是越来越少。去年动用库存七十亿斤,今年一定要搞八百五十亿斤。穿也是大事,现在折半了,越搞越少。怎么得了?

有反必肃,抓起来再说。不能像胡老头一样、罪已诏千万不下。

手工业社脱产人员太多,浙江兴登县七万人有三千五百多官,真是太多了,应该减三千,留五百。工厂非生产人员也太多,占百分之三十。苏联百分之二十五,美国只百分之三至五,现在要考虑。多的人要养起来,几年内转到生产中去。用节约下来的钱扩大经济事业。

整风范围扩大,党、团、民主党派,人民团体、工商联,学校,艺术界,经济事业机关,合作社都要整。

除了少数知名人士之外,把一些右派都搞去劳动教养。

共产主义者协会要大于共产党,中央领导同志去当会长。

地、富、反革命摘了帽子的,要调皮再给戴上。搞个劳动教养条例。死刑不要轻易废除。地、富、反革命的选举权,不要给得太早了。不搞大,只搞小,这些关口要卡住。唐僧这个集团,猪八戒较简单,可以原谅;孙悟空没有紧箍咒不行。反右倾思想,右派,极右派,合乎人情,顺乎天理。各省市对民主党派,不管其中央,予以腰斩。

(以上是7月18日会议上的讲话)


