\section[土地法(一九二九年四月兴国县土地法)]{土地法(一九二九年四月兴国县土地法)}
\datesubtitle{(一九二九年四月)}


(一)没收一切公共土地及地主阶级的土地归兴国工农兵代表会议政府所有,分给无田地及少田地的农民耕种使用。

(二)一切公共土地及地主阶级的土地,经工农兵政府没收并分配后,禁止买卖。

(三)分配土地的数量标准:

(1)以人口为标准,男女老幼平均分配。(2)以劳动力为标准,能劳动的此不能劳动的多分土地一倍。以上两个标准,以第一个为主体,有特别情形的地方得适用第二个标准。釆取第一个标准的理由:

(甲)在养老育婴的设备未完毕以前,老幼如分田过少,必致不能维持生活。

(乙)以人口为标准计算分田,比较简单方便。

(丙)没有老小的人家很少。同时老小虽无耕种能力,但在分得田地后,政府亦得分配以相当之公众勤务如任交通等。

(四)分配土地的区域标准:

(1)以乡为单位分配。(2)以几乡为单位分配(如永新之小江区)。(8)以区为单位分配。

以上三种标准,以第一种为主体。迁特别情形时,得适用第二第三两种标准。

(五)山林分配法:

(1)茶山柴山照分田的办法,以乡为单位平均分配耕种使用。

(2)竹木山归苏维埃政府所有。但农民经苏维埃政府所许可后得享用竹木。竹木在五十根以下须得乡苏维埃政府许可,百根以下须区苏维埃政府许可,百根以上须得县苏维埃政府许可。

(8)竹木概由县苏维埃政府出卖,所得文钱由高级苏维埃政府支配之。

(六)土地税之征收:

(1)土地税依照生产情形分为三种:(一)百分之十五,(二)百分之十,(三)百分之五。以上三种方法以第一种为主体。遇特别情形,经高级苏维埃批准,得分别适用二三两种。

(2)如遇天灾或其他特殊情形时,得呈明高级苏维埃政府核准,免纳土地税。

(3)土地税由县苏维埃政府征收,交高级苏维埃政府支配之。

(七)乡村手工业工人,如果自己愿意分田者,得分每个农民所得田的数量之一半。

(八)红军及赤卫队的官兵,政府及其他一切公共机关服务的人,均得分配土地,如农民所得之数,由苏维埃政府雇人代替耕种。

按:这是前一个土地法制定后第四个月,红军从井岗山到赣南之兴国发布的,内容有一点重要的变更,就是把“没收土地”改为“没收公共土地及地主阶级土地”,这是一个原则的改正。但其余点均未改变,这些是到了一九三○年才改变的。这两个土地法,有之以见我们对于土地斗争认识之发展。

