\section[接见张春桥姚文元同志对上海文化大革命的指示(一九六七年二月十二日——十八日)]{接见张春桥姚文元同志对上海文化大革命的指示(一九六七年二月十二日——十八日)}
\datesubtitle{(一九六七年二月十二日)}


二月至四月是无产阶级文化大革命的关键时刻。这三个月中,文化大革命要看见眉目。

上海的工作总的方面是很好的。上海的工人在安亭事件的时候,第一次去的时候不是只有一、两千人吗?现在已经到了一百多万人啦!说明上海的工人发动得比较成功。

我们现在这个革命,无产阶级文化大革命,这是无产阶级专政下的革命,是我们自己搞起来的。这是因为,我们的无产阶级专政的机构中间有一部分被篡夺了,这一部分不是无产阶级的,而是资产阶级的,所以要革命。要中央文化革命小组考虑一下,写篇文章,就叫作“无产阶级专政下的革命”。这是一个很重要的理论问题。

一定要(三)结合。福建的问题不大,贵州问题也不大,内蒙古问题也不大,乱就乱一些。现在山西省有百分之五十三是革命群众,百分之二十七是部队,百分之二十是机关干部。上海应向他们学习。一月革命胜利了,但二、三、四月更关键、更重要。

“怀疑一切、打倒一切”的口号是反动的。怀疑一切、打倒一切的人一定走向反面。一定被人家打倒,干不了几天。我们这儿还有个单位,连副科长都不要。副科长都不要的人,这种人是搞不了几天的。

应该相信百分之九十五以上的群众,百分之九十五以上的干部是会跟着我们的,中国的小资产阶级相当多,中农占的数量很大。城市里小资产阶级、小手工业者,包括以至于小业主,这个数量相当大。只要我们善于领导,他们也是会跟着我们走的。我们要相信大多数。

一个大学生,领导一个市,刚刚毕业,有的大学生还没有毕业,就管一个上海市是很难的。我看当个大学校长也不行。当个大学校长,学校很复杂,你是一个刚刚毕业或还没有毕业的人,学校情况很复杂。照我看,当一个系主任也不行。系主任总要有一点学问吧!你这个学问还没有学完,大学刚刚毕业,学问还不多,而且没有教书的经验,没有管理一个系的经验。搞个系主任,我们已经培养了一批助教,讲师,原来的领导干部,总要选些人出来。这些老的人,也不能够都不要。恐怕周谷城不行了吧!周谷城再教书不行了吧!

巴黎公社,不是我们都讲搞巴黎公社是个新政权吗?巴黎公社是一八七一年成立的,到现在九十六年了,如果巴黎公社不是失败了,而是胜利了,那么,据我看呢,现在也已经变成资产阶级的公社了,因为法国的资产阶级不可能允许法国的工人阶级掌握政权这么大。这是巴黎公社。苏维埃的政权形式。苏维埃政权一出来,列宁当时很高兴,认为这是工农兵的一个伟大的创造,是无产阶级专政的新形式。但是列宁当时没有料到这种形式工农兵可以用,资产阶级也可以用,赫鲁晓夫也可以用。那么,现在苏维埃,从列宁的苏维埃变成了赫鲁晓夫的苏维埃。

英国是君主制,它不是有国王吗?美国是总统制,它本质上还是一样,都是资产阶级专政。南越伪政权是总统制,它旁边的柬埔寨西哈努克是王国,哪一个比较好一点?恐怕还是西哈努克比较好一点。印度是总统制,他旁边的尼泊尔是王国,这样的哪一个国家好一点呢?看起来还是王国比印度的好一点。这是从现在的表现来看啰。旧中国的三皇五帝,周朝是叫王,秦朝是叫皇帝,秦始皇他把三皇五帝都叫了。太平天国就叫天王,唐太宗也是天皇。你看,名称变来变去。我们不是只看名称变了,问题不在名称,而在实际,不在形式,而在内容。

名称不宜搞得太多,我们不在名词,而在实际,不在形式,而在内容。汉朝王莽这个人是最喜欢搞名字的啰,他一当了皇帝就把所有的官职都像现在我们很多人不喜欢“长”啊,都改了,他统统改了,把全国的县名也统统改了,有些像我们红卫兵把北京街道名字改的差不多,改了大家都记不得,还是记老名字。王莽下诏书,下命令都困难了,老百姓也不知道是改成什么了,这样使得下公文就麻烦了。话剧这个形式,中国可以用,外国可以用,无产阶级可以用,资产阶级也可以用。

主要经验就是巴黎公社和苏维埃,我们也可以设想中华人民共和国,两个阶级都可以用,如果我们被推翻,资产阶级上了台,他们也可以不改名字,还叫中华人民共和国。主要是哪一个阶级掌握政权,谁掌握这是根本问题,不在于名字。

我们是否还是稳当一点好,不要都改名字了。因为这样就发生了改变政体的问题,国家的体制问题,国号问题,是不是要改成中华人民公社呢?中华人民共和国主席就叫什么主任、社长呢?不但出了这个问题,还出了一个问题,如果改就紧跟着有个外国承认不承认的问题。改变国号,外国大使就作废了,重新换大使,重新承认。我估计苏联就不承认,他不敢承认,因为承认会给苏维埃造成麻烦,怎么出了个中华人民公社?他不好办。资产阶级国家可能承认。

如果都改公社,党怎么办呢?党放在哪里呢?公社里的委员有党员和非党员,党委放在哪里呢?总该有个党嘛!要有一个核心,不管叫什么,叫共产党也好,叫社会民主党也好,叫社会民主工党也好,叫国民党也好,叫一贯道也好,它总得有个党。公社总要有个党,公社能代替党吗?

我看还是不要改名字吧,不要叫公社吧,还是按照老的办法,将来还是人民代表大会,还是选举人民委员会。这些名字改来改去都是形式的改变,不解决内容问题。现在建立临时权力机构,是不是还叫革命委员会,大学是否还是叫文革委员会,十六条规定了。

上海的人民很喜欢人民公社,很喜欢这个名字,怎么办?是不是回去商量一下,无非是几种办法,一个办法就是不改,还叫上海人民公社,这个办法的好处是可以保护上海人民的热情,大家喜欢这个公社。缺点是全国只有你们一家,你们不很孤立吗?现在不能登《人民日报》,大家都要叫人民公社,中央如果承认人民公社,一登《人民日报》,那样全国都要叫,为什么只准上海叫,不准我们叫?这样不好办。不改有优点也有缺点。第二个办法就是全国都改,就得发生改变政体,改变国号,有人不承认,很多麻烦事,也没什么意思,没什么实际意义。第三个办法,就是改一下,这样就和全国一致了。当然也可以早一点改,也可以晚一点改,不一定马上改,如果大家说还是不想改,那你们就叫一个时候。你们看怎么样啊,能说得通吗?

刘少奇的《论共产党员修养》我看过几遍,这是反马列主义的。现在我们的斗争方法要高明一些,不要老是“砸烂狗头”,“打倒×××”,我看大学生应该更好地研究一下,选几段,写文章批判。

以后不要提“打倒坚持资产阶级反动路线的顽固分子”,还是提“打倒走资本主义道路的当权派”。

上海的工作总的方面是很好的,你上一次去的时候不是只有一、二百人的吗?那么现在已经到了一百多万的人了。工人组织起一百万人了,这就说明上海的工人群众发动得比较充分。

关于中央文革小组处理上海红革会问题的《紧急指示》我看过,写的很好,有造反派的气魄,最后一点说:“将采取必要措施”,这一次炮轰张春桥大会如果开的话,一定要采取必要措施抓人。

(上海人民)公社在镇压反革命的问题上手软了一些,有人向我告状,公安局抓人前门进,后门出。

一、二、三兵团怎样?他们上这里来告你(们)的状。

你们那个时候学生都不是到了码头吗?现在那些学生是否还在码头上啊?(张春桥回答说:“还在”)很好,以前学生和工人结合没有真正结合好,现在才是真正结合。

《文汇报》搞得好,很同意他们对里弄干部的观点,我支持他们。有几笔账以后算。

※此文是根据张春桥同志二月二十四日在上海人民广场的讲活录音稿和有关传单整理的。是否每句都是毛主席的原话,很难断定,只供参考。


