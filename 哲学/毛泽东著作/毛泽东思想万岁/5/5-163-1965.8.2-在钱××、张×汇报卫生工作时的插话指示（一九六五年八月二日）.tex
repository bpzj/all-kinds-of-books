\section[在钱××、张×汇报卫生工作时的插话指示(一九六五年八月二日)]{在钱××、张×汇报卫生工作时的插话指示}
\datesubtitle{(一九六五年八月二日)}


八月二日,钱××和张×向主席汇报:根据主席六月二十六日指示,卫生部作了检查,请主席再给予指示。

主席:我要讲的都已讲了。你们打算怎么办?

答。(略,汇报到打算为农村生产队培训不脱产卫生员)

主席:训练多长时间?

答:半个月左右。

主席:半个月太短了吧?

答:他们主要学会些针灸,常见病的治疗和一些预防工作,训练后还要带他们做。

主席:带很重要,带多长时间?

答:城市医疗队下去可一直带,共带三、四个月。

主席:这还可以,带三、四个月,学会十几种病。

(当汇报到搞一些半农半读训练班时)

主席:半农半读怎么搞的?

答:采取农忙不学,农闲多学的方式。

主席:多长时间?

答:二年,三年。

主席:二年就是读一年书,三年就是读一年半书,这个办法好。

(我们又汇报了根据主席指示作的检查)

主席:现在有多少卫生技术人员?

答:(略)

主席:这个队伍不小!

答:(略)

主席:你们用了这么多人力,这么多钱,为那么少的人服务。

(我们又汇报:现在正在召开医学教学会议,讨论主席、总理指示、要办三年制卫生学校,为农村培养医生)

主席:高等教育要学五年,读那么长时间书,值得研究。

(我们说:现在想两条腿走路,办三年制,多在农村办,在农村招生)主席:招什么生?答:大部分招高中生。主席:初中还不行吗?答:目前农村高中生还不少。主席:招中学生还不行吗?你们不在城市招了吗?答:城市也招一些,主要在农村招。

(当谈到医生的政治思想和技术的关系时)主席:这一点很重要。医生一定要政治好,招些政治好的中学生就可以了。

(我们又汇报了城市问题)主席:你们对城市是不是服务的好?

(我们当时对这句话体会不够,答了:也没有服务好,要求愈来愈高)主席:怎么愈来愈高?答:以北京来讲,××万干部公费医疗,××万工人劳保,还有家属。主席:工人不是老爷,工人还是要为他们服务的。

(我们说:工厂医务工作没有搞好,工人也都到医院去看病)主席:你们为什么不在工厂设不脱产卫生员呢?小厂可以设卫生员,大厂设医务所。答:非正式医生开不了假条,要请假还要到医院去,经医生开证明才行。主席:这涉及劳保问题,要好好考虑。

(我们又汇报城市组织医务人员下农村的问题,每年去三分之一)主席:怎么下去?多久?答:轮流下去。下去后城市医疗卫生工作仍较重。我们打算高等医学院校学生三年后就一边工作一边上课。护士也一面工作一面学习。主席:光会念书是不行的。主席:你们对打仗考虑了没有?怎么考虑的?答:(略)主席:打仗了还能都到北京医院看病?要考虑到打仗。你们分科那么细,打起仗来怎么办?

打起仗来还不是什么都要看,只会内科不会外科怎么行?

(当汇报到要有一小部分力量搞尖端时)主席:搞科研的人看不看病?答:一部分人看病,一部分人不看病。主席:什么人不看病?答:搞基础学科的,如搞生理、药理、生化的,研究理论不看病。主席:理论还是要结合实际呵!主席又说:科学尖端还是要搞的。

(当谈到各级党委应当多抓卫生工作时)主席:这话很对,党委是集体领导,什么都要管。卫生部门是为人民服务的,当然要抓,党委也要研究卫生工作。(我们又请示农村卫生员工分问题。人民公社六十条规定,补贴工分不超过百分之二。卫生员的补贴工分社员同意给,但六十条没有规定)

主席:给群众办好事,没有问题,农民会同意的。现在干部参加了劳动,百分之二已用不了,有的地方只用百分之一,工分不成问题,农民会同意的。

(我们又请示主席,建议中央今年召开一次农村卫生工作会议,中央召开,可以找省、市领导同志参加)

主席:可以。

(当谈到药品问题时)

主席:药品应当降价。

主席:天津计划生育不要钱。看来国家出了钱,实际是划得来的。国家出点钱保护生产力是合算的。药钱拿不起也可以不拿。

又说:你们开展农村卫生工作后,要搞节制生育。

(最后,我们请示主席还有什么指示)

主席:没有什么了,就按你们讲的办吧。

