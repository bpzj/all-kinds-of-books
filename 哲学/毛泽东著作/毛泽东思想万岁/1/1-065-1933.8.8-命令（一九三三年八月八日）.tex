\section[命令(一九三三年八月八日)]{命令}
\datesubtitle{(一九三三年八月八日)}


于约溪总司令部

(一)敌之整个布置见蒋介石八月二号命令,卫另录。龙冈有敌第五师四团五十四师两团(昨日未与我军作战的),加上昨天与我军作战的四十七师五十四师败兵约两团,共尚有敌八团。崇贤方太之蒋蔡部,兴国江背洞之赵欢涛部,青塘古龙冈之孙建仲部均在向我军前进,但明日(九号)不能到达龙冈。

(二)我军决心以全力消灭龙冈之敌。

(三)攻击龙冈布署如左:


三军及七军任左翼,今夜宿营于兰石,担任攻击龙冈之西北端。三军任右翼,今夜宿营于小别,担任攻击龙冈之东北端。四军任正面,今夜宿营于表湖,担任攻击龙冈之南端。十二军位置于四军之后作总预备队,今夜应宿营于表湖附近。

各军明晨(九号)四时一律总攻,务于上午十二时以前解决战斗。

(四)解决战斗后,各军宿营地区分于左;

三军团及七军在上固、下固、缺家坪一带。

三军在小别。

四军在水西、高车、大蕉坑、下车、中塘陂一带。

龙冈圩、表湖、张家车、樊铺一带不驻兵。

(五)明日上午作战时总部在表湖。作战后至夜间,在龙冈圩区政府(圩场的南端),后晨移石路坑。

(六)明日下午七时,各军军长、政治委员、军团总指挥、政治委员均到区政府总部开会。

<p align="right"> 总司令政治委员

 毛泽东

(抄自中国革命军事博物馆)</p>

