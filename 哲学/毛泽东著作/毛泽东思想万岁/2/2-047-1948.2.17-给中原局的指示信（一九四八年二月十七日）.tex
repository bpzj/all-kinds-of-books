\section[给中原局的指示信(一九四八年二月十七日)]{给中原局的指示信}
\datesubtitle{(一九四八年二月十七日)}


(上略)你们主张在新解放区先建立贫农团,以提高贫雇农的觉悟性与组织性,树立贫雇农的威信,在与国民党和地主的斗争中,吸收中农参加;贫农团建立了几个月之后,再建立联合全体农民的农民协会。这一点我完全同意,在中央对新解放区土改要点指示中已规定了。贫农团和农民协会的组织,均须严防地主富农民其他投机分子混入。农民协会的会员及其委员会的委员中,必须有三分之二是贫雇农,以确实掌握贫雇农的领导权。这一点也是完全必要的。

新解放区的斗争策略阶段,必须分为先斗地主,后斗富农的二大阶段。在前一阶段中又必须分几步骤,先从斗大地主恶霸反动分子开始,根据农民群众的觉悟程度,逐步把打击面推广,总的打击面,一般不要超过人口的百分十,如果成分是富农地主(甚至个别的中农)的反动保甲长恶霸分子,为群众所痛恨,群众要求先斗他们时,我们应当允许农民这样做。特别是因为有许多乡村没有大地主,只有中、小地主与富农,更应允许农民这样做。我们允许打击富农和中小地主的反动保甲长恶霸分子,和我们在第一阶段的策略(缩小打击面,对于富农中小地主的多数暂时不去惊动他们)并不矛盾。

