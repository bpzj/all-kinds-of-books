\section[论新阶段(节选)(一九三八年十一月)]{论新阶段(节选)}
\datesubtitle{(一九三八年十一月)}


在一切为着战争的原则下,一切文化教育事业均为使之适合战争的需要,因此,全民族的第一个任务,在于实现如下各项的文化教育政策。第一,改订学制,废除不急需与不必要的课程,改变管理制度,以教授战争所必须的课程及发扬学生的学习积极性为原则。第二,创造并扩大增强各种干部学校,培养大批的抗日干部。第三,广泛发展民众教育,组织各种补习学校,识字运动,戏剧运动,歌咏起动,体育运动,创办敌前敌后各种地方通俗报纸,提高人民的民族文化与民族觉悟。第四,办理义务的小学教育,以民族精神教育新后代。一切这些,也必须拿政治上动员民力与政府的法令相配合,主要的在于发动人民自己教育自己,而政府结合恰当的指导与调整,给以可能的物质帮助,单靠政府用有限的财力办几个学校、报纸等等,是不足完成提高民族文化与民族觉悟之伟大任务的。抗战以来,教育制度已在变化中,在抗战政府有了显着的改进,但至今还没有整个制度适应抗战需要的变化,这种情形是不好的。伟大的抗战必须有伟大的抗战教育运动与之相配合,二者间不配合的现象应免除。

