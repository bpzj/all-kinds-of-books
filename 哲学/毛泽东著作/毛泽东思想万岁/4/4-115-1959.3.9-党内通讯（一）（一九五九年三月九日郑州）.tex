\section[党内通讯(一)(一九五九年三月九日郑州)]{党内通讯(一)(一九五九年三月九日郑州)}
\datesubtitle{(一九五九年三月九日)}


各省、市、区党委第一书记同志们:

中央决定在三月二十五日在上海开政治局扩大会议,你们都要到会。各省、市、区党委根据此次郑州会议决议精神,以讨论人民公社为主题而召开的下级干部大会,大会约需要开十天左右,因此应当立即召开。例如湖北省委定于三月十一日开会是适当的,开迟了,不利。时间太短,问题的分析、揭露和讨论,势必不充分,解决得不会很适当,很彻底,就是说,不深不透。各省、市、区大会应当通过一个关于人民公社管理体制和若干具体政策问题的决议,第一书记要做一个总结性的讲话,以便又深又透地解释人民公社当前遇到的主要矛盾和诸项政策问题。将这两个文件立即发下去,使下面获得明确的根据。而这样两个文件的思想形成和文字起草,需要时间。假如三月十一日大会开幕,可能要开到三月二十日或者二十二日才能结束,各第一书记才有可能抽出空来,于三月二十五日到上海开会,这样就从容些,不至于太迫促。河南的六级干部大会,三月十日可以结束。他们的一个决议,一个总结性讲话,三月九日可以最后定稿。这两个文件,中央将在三月十四日以前用飞机送到你们手里以供参考。河南的下一步,是开县的四级干部大会,传达和讨论省的六级干部会议的出席人是:(一)县级若干人;(二)公社级若干人;(三)生产大队每队一人至二人;(四)生产队每队一人。外加着干观潮派,算账派。共计少者千余人,多者二千人。会期七天至十天。河南各县定于三月十三日或十四日同时开会,三月二十日或二十四日以前结束,三月份还剩下一星期,各级公社、大队、生产队去开会。总之,三月份可以基本上澄清和解决人民公社问题中一大堆糊涂思想和矛盾抵触问题。四月起,全党全民就可以一个方向地展开今天的大跃进了。我希望各省、市、区也这样办。各省、市、区的六级干部大会,如果能像湖北那样在三月十一日召开,三月二十日或二十二日以前可以结束,三月底可以结束县的四级干部会议,四月十日以前可以结束社和队的讨论,比河南的也只迟十天左右。有些同志或者认为仓促,无准备,大会召开的时间应当推迟。我认为不宜如此。我们已经有了明确的方针。把六级干部迅速找来,到地方即刻放出去,三四天内就会将大小矛盾轰开,就会获得多数人的拥护。我们已经有了明确的方针取得主动。观潮派算账派无话可讲。当然会有一部分人想不通,骂我们开倒车。这些人会有几天睡不好觉,吃不好饭,但不几天而已。三天后。就会通的。总之,慢了不好,要快,可以做些准备工作,首先稍稍打通地县两级思想,不必全通,有三天时间也就可以了。拖长反而不好。

以上是我的建议。是否可行,还是由你们根据你们自己的情况去决定。

<p align="right">一九五九年三月九日上午四时于郑州</p>


