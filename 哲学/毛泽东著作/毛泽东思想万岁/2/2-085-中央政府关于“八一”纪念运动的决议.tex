\section[中央政府关于“八一”纪念运动的决议]{中央政府关于“八一”纪念运动的决议}


八月一日是全世界反帝国主义战争纪念日,同时是中国南昌暴动纪念日,中国工农红军即由南昌暴动开始,逐渐在斗争中成长起来。今年的“八一”正是帝国主义新的强盗战争及反苏联战争的危险极度紧张的时候,正是日本帝国主义大规模侵略中国,中国国民党公开出卖东三省热河与华北的时候,同时是全国反帝反国民党运动极大的高潮,苏维埃运动与革命战争得到空前伟大胜利的时候,因此今年的“八一”有着非常伟大的革命斗争的意义,中央执行委员会为了纪念中国工农红军的成立及奖励与优待红军战士起见,特决议如下:

(一)批准中央革命军事委员会的建议,规定以每年“八一”为中国工农红军纪念日。并于今年“八一”纪念并授战旗于红军各团,同时授与奖章与领导南昌运动的负责同志及红军中有特殊功勋的指挥员和战斗员。(二)责成内务部人民委员部制定红军家属优待证,发给一切红军战士的家属以执。(三)在区苏土地部与乡苏之下,组织红军公开管理委员会,管理红军公田的生产收获,及收获品保管等事宜。在区苏土地部与内务部共同管辖下及在乡苏下组织优待红军家属委员会,管理优待红军家属的一切事宜。

<p align="right">中央政府主席毛泽东</p>

