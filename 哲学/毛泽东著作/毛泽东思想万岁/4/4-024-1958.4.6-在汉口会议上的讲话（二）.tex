\section[在汉口会议上的讲话(二)(一九五八年四月六日)]{在汉口会议上的讲话(二)}
\datesubtitle{(一九五八年四月六日)}

过渡时期阶级斗争的形势怎样?

两条道路斗争。恐怕还有几个回合。我们要有策略。要冷一冷,然后再放一放,不冷不放,他不会出来的。在成都会议上说过的,两个剥削阶级,两个劳动阶级。第一个剥削阶级为帝国主义、封建主义、官僚资本主义、国民党残余,三十万右派也包括进去。地主现在分化了,有改造过来的,有没有改造过来的。没有改造过来的地、富、反、坏和右派分子,这一些人反共,就是现在的蒋介石、国民党,是敌对阶级,如章伯钧等。党内的右派分子也是一样的。包括一些现在划为中间偏右还没有触动过的右派。人数大约是百分之五,就是三千万,比较恰当。这是敌对阶级。尚待改造。一要斗、二要拉,要把十分之七分化出来,就是大胜利。要调动他们,化消极力量为积极力量。几年之后,他们把心交出来,真正改变,可以摘掉帽子。右可能转左,或转成中间;左也可能转为右,如考茨基。第二个剥削阶级,是民族资产阶级及其知识分子,加上一部分上层小资产阶级(如刘绍棠、陈伯华;农村富裕中农也包括在内)。民族资产阶级及其知识分子,大多数是中间分子,他们是剥削者,与前一个剥削阶级不同。又反共又不反共,是个动摇的阶级。他们反共,但不坚决,与蒋介石不同;看谁力量大,就跟谁走。汉口有个资本家从汉口到北京就靠“拥护共产党,拥护公方代表”这句话吃饭,多一句也不讲。实际上思想没有多大改变。去年右派进攻,如果我们不坚决打下去,中国出了纳吉,右派登台,这些人一股风都上来了,打倒共产党,他们都干。这些人对共产党是两条心的,是半心半意的。右派是无心无意的。经过去年一年到现在的斗争,这些人政治上正在发生着变化。去年这些人多数是迷失方向的。但是经过大鸣大放,农村、城市整风一胜利,一年生产的大跃进,形势逼人,他们就不能不有所改变。形势是人造成的。人成堆,多数人逼少数人。长江大桥、工业化等可放在形势里面。这个剥削阶级比较文明一点,我们也用文明的办法对待,采取批评方式,与反右斗争的方法不同。对右派釆取带点武的性质,无非是把他们搞臭。这是两个剥削阶级,我们的方针也不同。我们是团结后一个剥削阶级,孤立打倒前一个剥削阶级,即团结中间,孤立右派。他们虽有三千万之多,但分散全国,在包围之中,处于孤立地位。开右派大会,他们料不到有这样的事情,就等于皇恩大赦。各大城市(三十万人口以上的大城市)都要开,要主要负责同志讲话,讲透一些。首先一训,然后一拉。训则凄凄惨惨,冷冷清清,拉则全身热,通身舒畅,指明前途,使他们有希望,像刘姥姥进大观园借钱一样,开始凤姐表示冷淡,后来很热情,搞得刘姥姥很高兴。凤姐这个人很厉害,有人说他为治世之能臣,乱世之奸雄。

两个劳动阶级:工人和农民。过去心不齐,意识形态、相互关系没有搞清楚。工人农民在我们党的领导下,做工、种田,我们在相互关系问题上,过去处理不恰当。我们干部的作风一般说来,是同国民党有原则的区别,但有一部分差不多。如老爷对小民,奴隶主对奴隶一样,只压服不说服。上海某大学一个女职员霸占一个厕所,不许别人进去。有些干部的坏作风同国民党差不多,个别的甚至超过国民党。因此,工人、农民就把他们看作是国民党。所以,过去工人、农民的世界观未变,为“五大件”而奋斗。工人、农民不敢说话。怕挨整,怕“穿小鞋”,怕不好乱,谁敢贴大字报、大鸣大放、大整大改一来,这种关系有了很大变化。工人自己批评自己为“五大件”而奋斗不对,工作态度改变了。理发、洗澡工人说自己不应该增加工资。武汉有个商店工人一当干部,对店员就扳起面孔,这就是国民党作风。红安县干部,老爷气一经改变,与群众就打成一片,关系就大改变了。所有制、相互关系、分配关系是生产关系中的三个问题,我们抓中间,也就是抓住相互关系,我们的整风,就是解决相互关系问题。共产党员中某些人是在社会上、学校里学了一些奴隶主的神气。刘介梅是向社会上学来的。把相互关系整一整,工厂里的党政工团和工人的关系,合作社干部与社员的关系,各级党政人员与下级的关系,干部和群众的关系,校长教师与学生的关系,一句话,是人民内部矛盾,用说服的方法,不用压服的方法去解决。这一来揭开盖子,人民舒服,精神解放,敢写大字报,这是列宁主义,不是机会主义。列宁死早了,他的作品,特别是在革命时期的著作,生动活泼。他说理,把心交给人民,讲真话,不吞吞吐吐,即使和敌人作斗争也是如此。斯大林这位同志有点老爷味道,在教会学校读书,辩证法不甚通,唯物论也不甚通。脱离实际,相互关系没有搞好,相当僵硬。过去苏联与我们是父子、猫鼠关系,现在好一些了。我们的民主传统有悠久的历史,根据地搞民主,无钱、无粮、无枪,孤立无援,必须依靠群众,党必须与人民一致,军队必须与人民打成一片,官必须与兵一致。要搞好这些关系,非搞三大纪律八项注意不可。以平等待人民,军队内废除肉刑,不枪毙逃兵,经常教育,经常做斗争,打一仗,新兵来,又要做教育。所以,老爷态度虽有点,但民主作风还是学了一些。这是因为斗争艰苦,时间长,在斗争中锻炼出来的。可是至今还有一部分人不赞成说服方法。如济南有人说,(五七年)春季右倾了,只赞成夏季形势,不赞成春季形势。其实夏季形势也是不赞成的。夏季形势一文就说过,军队可用民主,对人民为什么不可用民主?可见这问题还没有解决。经过去年一年,特别是今年丰收,苦战三年(基本或是初步)改变落后面貌,那时候人们就通了,真相信了,但还要写文章,用理论说服这些人。

我看过渡时期阶级斗争的形势,百分之五的细菌还是会有的,中间派也可能变坏,他们肚子里是有意见的,不过嘴巴暂时不说,将来还要说的。斗争是长期的反复的。几亿人民蓬篷勃勃起来了。右派孤立了,三十万右派搞臭了,没有资本,资产阶级也臭了,三反五反就臭了。对知识分子戴上两个帽子,封了他们资产阶级知识分子,又封了他们迷失方向。出英雄是左派,是我们这些人。将来犯错误的人,也出在左派,因左派有资本,一不小心就会犯错误。如××××是四十年政治局委员,脱离群众,一个工厂不去,一个农村不去。××××的好处,就是下去到处跑,人家说他是旅行家。当旅行家也有好处,过去我们打游击,是旅行家,旅行了几十年,现在还是南方旅行到北方,还要当旅行家。中央和省两级规定,四个月当旅行家,地县更多。这是赶出大门。

过渡时期阶级斗争究竟如何?一定要估计反复。要估计是否还要出什么大问题,如国际上出什么问题,世界大战,大灾荒,右派可能会作乱,中间派还会出乱子的。

