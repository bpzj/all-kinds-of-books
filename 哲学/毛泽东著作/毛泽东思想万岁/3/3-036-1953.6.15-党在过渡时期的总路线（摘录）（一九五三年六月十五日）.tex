\section[党在过渡时期的总路线(摘录)(一九五三年六月十五日)]{党在过渡时期的总路线(摘录)}
\datesubtitle{(一九五三年六月十五日)}


一九四九年十月一日中华人民共和国的成立,……标志了新民主主义革命阶段的基本结束和社会主义革命阶段的开始。

……

从中华人民共和国成立,到社会主义改造基本完成,这是一个过渡时期。党在这个过渡时期的总路线和总任务,是要在一个相当长的时期内,逐步实现国家的社会主义工业化,并逐步实现国家对农业、对手工业和对资本主义工商业的社会主义改造。这条总路线是照耀我们各项工作的灯塔,各项工作离开了它,就要犯右倾或“左”倾的错误。

我们的各项工作,应当遵循总路线,为总路线服务。离开了总路线就要犯错误。我们必须过渡,把社会主义看作“遥遥无期”是不对的。过渡只能是逐步的,企图“一步登天”也是不对的。

……

有的同志在革命成功以后,仍然停留在原来的地方。他们没有懂得革命性质的转变,还在继续搞他们的“新民主主义”,不去搞社会主义改造。这就要犯右倾的错误。就农业来说,社会主义的道路是我国农业唯一的道路。发展互助合作运动,不断地提高农业生产力,这是党在农村中工作的中心。通过农业合作化,逐步建立农业中的社会主义生产关系,限制和消灭农村中的资本主义,在这个基础上增加农业生产,这是我们的主要方法。只有在农业合作化的基础上,才能巩固和扩大我国的工农联盟。农业支援工业,促进工业化;工业支援农业,帮助农业合作化。这就是过渡时期工农联盟的新的经济基础。手工业的情况也是这样,我们必须对个体手工业进行社会主义改造,实行集体化,引导手工业劳动者走社会主义的道路。但是有的同志却不是这样看问题,他们有几种错误观点,是必须批驳的:

第一,有同志提出,确立新民主主义的社会秩序。这种提法是有害的。过渡时期的一切事物,一切社会秩序,时时刻刻处于发展变动之中。例如,我们的农村互助合作运动,年年在变,月月在变,天天在变,不断有新的东西代替旧的东西。过渡时期充满着矛盾和斗争。我们现在的革命斗争甚至比过去的武装革命斗争还要深刻。这是要把资本主义制度和一切剥削制度彻底埋葬的一场革命。“确立新民主主义社会秩序”的想法,是不符合实际斗争情况的,是妨碍社会主义事业的发展的。

第二,有同志提出,“确保私有财产”。这种提法,当然更是错误的。农村的互助合作运动在不断前进,农民的个体经济必然要被集体经济所代替,这是事物发展的客观规律。就生产资料方面说,我们不能“确保私有财产”。“确保私有财产”的口号实际上是代表资产阶级和富农利益的口号,这个口号只能束缚我们自己的手脚.束缚广大群众的手脚。

第三,有同志提出,“由新民主主义走向社会主义”。这种提法看来没有多大错误,但是不明确,也不符合实际的情况。“走向”社会主义就是还没有走到社会主义,只是在向着社会主义的目标走。我们不能在整个过渡时期老是“走向”社会主义。事实上,我们的国营经济早已就走到完全的社会主义了。“由新民主主义走向社会主义”的提法,可能使人认为我们现在并不是处于社会主义革命的阶段.不能明确认识我们的社会主义革命的责任。所以,这种提法也是不恰当的。

上面所说的是右倾性质的错误。

……


