\section{长沙铅印活版工会致“大公报”记者盾(李抱一)书(一九二二年十二月十四日)}


对站在群众运动后指手画脚的人指出:工人,农民并非不愿意接受别人的教训,但是“教训人者”必须做到以下三点:

一、但愿教训我们的人能站在我们的地位来教训我们,能够不为我们的师长,而降格为我们的朋友。再不要开口就“你们做工人的缺乏常识”、“不守秩序”、“品类日杂”、“忠告你们工人”、“助长工人习气”,应当要说:“我们大家……”才好啊!先生,你真个能诚恳帮助我们,忠告我们吗?那末,我们是很愿意和你握手,请你赶快拿出你的手来,切莫再“你们、我们”,恍然象你们是“官”,你们是“小的该死”一样。

二、但愿教训我们的人,能将事实调查清楚,不要含沙射影,更不要蔑视人家的人格。譬如我们说你贵报馆,每月多受某私人多少金钱,你能承认吗?我们且问你,你之所谓“不再受人驱策”,而云然,请你早答复我们,我们更愿新闻记者要翔实些才好啊……

三、但愿教训我们的人,能够下得身段,真真实实地教训我们。我们工人所需要的知识,这是很不错的;我们工人很愿意有知识的人们能挺身而出,做我们的真实朋友。先生说我们受人驱策,为人牺牲,先生‘可怜’我们,先生便应该做我们的真实指导者。我们很愿先生真个能脱去长农,辞去大编辑职务,帮助我们干劳动运动;至少,也应当做一个真实的劳工教育者,切莫再站在旁观地位,什么我因此更进而告从事劳动运动者你们一一啊!先生,我们只承认能牺牲自己的地位,忍饥吃苦,而为我们大多数工人谋利益的人,是我们的好朋友。先生能惠然肯来吧?请快快脱去长衣!

\begin{flushright}1922年工2月14日\end{flushright}

