\section[在七届六中全会扩大会议上的总结(一九五五年)]{在七届六中全会扩大会议上的总结}
\datesubtitle{(一九五五年)}


一、这次会议牵涉面很广,是一次大的争辩

这次会议,解决了很多问题,是一场很大的辩论。这是党在过渡时期的总路线是不是完全正确的一次大辩论。这次全党的大辩论,是从农业合作化的方针问题引起的。牵涉面很大——重工业、轻工业、财政、金融、贸易、交通、文教、卫生、科学、镇反、手工业改造、资本主义工商业改造、党、团、工会、青年、妇女、内务、军队等。应当有这次大辩论,总路线公布后没有展开过这样的辩论。应当使这次辩论到农村支部的同志中去展开。城市支部党员中也要展开。使各方面工作的速度和质量都能和总路线规定的任务相适应。各项工作都要有全面规划。大约三个五年计划左右完成农业的社会主义改造,在同一时间内完成资本主义工商业的社会主义改造。那个时候,工业和农业,工人和农民的联盟,就能在新的社会主义的基础上巩固起来,才能彻底割断了农民与资产阶级的关系,才能彻底孤立了资产阶级,才能在农村的广阔土地上最后根绝了资本主义的来源。

现在,我们还没有完成农业合作化,工人和农民还没有结成巩固的联盟。过去在土地改革基础上的工农联盟,正在动荡不定。土地改革的利益,农民已经不满足了,有的已经忘了;我们还没有给农民新的利益(新的利益就是社会主义),农民还没有共同富裕起来,粮食和工业原料还不富裕,资产阶级还会找到岔子向我们进攻。

几年之后,我们将看到新的情况:工人和农民在新的基础上结成更巩固的联盟。过去土地改革基础上的工农联盟是暂时的,不巩固的。不改变农村生产关系,就不能提高农业生产力,农民就不能共同富裕起来,阶级就会向两级分化,穷的、富的都不会相信我们,工农联盟也就不能巩固。农村合作化以后,穷的变富,富的更富,农民就会相信我们。过了若干年以后,地主、富农也会相信我们。全体农民一年一年富裕起来了,商品粮食增加了,就会出现完全新的情况,资产阶级的咀也就堵起来了。

我们现在有两个同盟:一个是和农民的同盟,一个是和资产阶级的同盟。这两个同盟在中国经济落后的情况下都是必要的。我们和资产阶级同盟,对他们实行利用、限制,暂不没收,是为了搞到他们的工业品,来满足农民的希望。农民和我们一样,爱吃饭,爱穿衣,很爱东西(工业品),光给票子还不行,还会藏粮不卖。我们利用和资产阶级的同盟,来对付农民的惜售思想。我们又利用和农民的联盟,取得粮食与工业原料,去制服资产阶级。要他们拿出工业品,接受改造。你不拿出工业品,就不给你粮食和工业原料。这样,就在经济上制止了资产阶级搞自由市场,在政治上孤立了资产阶级。没有这一条,他们不服。这就是两个改造的互相关系。

两个同盟,和农民的同盟是基本的,永久的;和资产阶级的同盟是暂时的。因为资产阶级是要被消灭的。将来无产阶级队伍中,要有几百万资产阶级里的人加入进来。但是,反资产阶级思想的斗争则是长期的。土地改革是资产阶级性质的民主革命,它破坏封建所有制,不破坏资本家所有制和个体所有制。它使农民分得了土地,使工人和农民在民主革命的基础上结成同盟,使资产阶级第一次感到孤立,一九五零年三中全会曾经指出,没有完成土地改革,不能四面出击。土地改革完成之后,农民靠拢到我们这边,我们才能搞“三反”和“五反”。农业合作化以后,是在社会主义的基础上,最后巩固了工农联盟。这个新的工农联盟,将使资产阶级最后被孤立起来,使资本主义在六亿人口的中国绝种。有人说我们太没有良心了,我们说马克思主义者对资产阶级的良心是不多的,在这方面良心少一些才好。我们有些同志太仁慈了。我们要使资本主义在地球上绝种,使资产阶级成为历史上的东西,这是很有意义的,是一件好事情。凡是历史上发生的东西,都是要消灭的,资本主义也是一定要消灭的。

现在的国际环境,有利于我们完成过渡时期的总任务。再有十二年,就可以基本上建成社会主义。每年可以产钢一千八百万吨到两千万吨,发电量七百三十亿度左右,原煤二亿八千万吨左右,原油一千几百万吨左右,金属切削机床六万台左右,拖拉机(折合每台十五匹马力)十八万三千台,汽车十二万八千辆,水泥一千六百八十万吨左右,化学肥料七百五十万吨左右(这个水平相当于苏联一九四○年的水平,单以拖拉机的出产来说,和苏联一九五四年的产量差不多),粮食达到六千亿斤,增产一倍,棉花六百多万吨,超过了一倍(均和一九五二年比较),拖拉机共达六十万台,机耕地达到百分之六十一。再有两个五年计划,完成农业技术改革,机耕地达到百分之百。我们实现这个任务,需要一个和平建设的时间。这个时间能不能取得?我们外交部的同志,对外联络部的同志,军队上的同志,要努力争取,才有可能取得。

在这个五年内,国内国外的阶级斗争是很尖锐的。在这个斗争中,我们已经取得了很多很大的胜利,将要取得更大的胜利。过去几年,我们取得了四个方面的胜利:反对唯心主义,宣传唯物主义;镇压反革命;粮食统购统销;农业合作化。这四个方面的胜利,都带有反资产阶级的性质,给了资产阶级很大的打击;以后还要给以粉碎性的打击。反对唯心论,要继续长期的搞,三个五年计划把唯心论彻底反下去,建立起马克思主义的唯物论队伍。今明两年,在一千一百万到一千二百万人中进行肃反。反革命分子看不见,但一查就查出来了。中央组织部,中央公安部都查出了反革命分子。全国已经在二百二十万人中间查出了十一万个反革命分子,还有五万个重大嫌疑分子。搞反革命,要合乎规格标准,要搞真的。漏一点是难免的(如延安整风,漏了潘汉年、刘雪苇),但要少漏。在粮食问题上,我们打了一个大胜仗。农业合作化问题上,也打了胜仗。在这四个问题上,我们对资产阶级展开了巨大的斗争,打击了资产阶级,使它抬不起头,取得了主动,否则,我们被动。

二、这次会议打破了许多错误观点,破除了很多迷信

会议中的许多发言,破除了许多迷信,打破了许多错误的观点。

首先是:大发展呢还是小发展?这一次解决了这个问题。群众要求大发展,农业适应工业也要求大发展,主张小发展是错误的。

新区能不能发展?新区也能发展。

山区能不能发展?山区也能发展。

灾区能不能发展?灾区也能发展。

落后乡村谁不能发展?落后乡村也能发展。

少数民族地区能不能发展?除西藏、大小凉山外,少数民族地区也能发展。

没有资金能不能办社?也能办。

没有文化能不能办社?也能办。

干部少能不能办社?也能办。

过去有人说:“建社容易巩固难。”建社并不十分容易,巩固也不见得难,要说难,都有点难,要说容易,都容易。为什么偏要说“巩固难”呢?就是不要办社。

没有机器不能办社的空气也不大了。

坏社、三等、四等、五等社,除解散外,还有什么路可走?除开个别不解散不行的及富农搞的假社可以解散外,其余可以不解散。经过整顿是可以办好的。

中央农村工作部不光出谣风,还出道理。有人说:“如不赶决下马,就要破坏工农联盟。”我们是一字之争。我们说:“如不赶快上马,就要破坏工农联盟。”

耕牛死亡,一部分是合作社的“罪过”,但主要原因不在合作社。死牛的原因是粮食问题,牛皮价格问题、牛的年老问题、水旱问题等。

今年春季农村紧张,有人说是合作社办得多了引起的。根本不能这样讲。主要是地主、富农的叫嚣,富裕农民也叫嚣,不缺粮的也争着买粮。部分是粮食问题引起的,部分则是虚假现象。叫嚣一下也好,我们把粮食问题摸了一下底:去年遭灾减产,我们多购了六十亿斤,今年增产二百亿斤,又少购了六十亿斤,在农民方面,就增加了二百六十亿斤粮食。

“合作社只有三年优越性”的论调,也是悲观主义。苏联搞社会主义几十年了,还有优越性,我看社会主义的优越性总有几十年,几十年以后,社会主义没有优越性了,我们又搞共产主义。

最近几年,应该不应该办一批高级社?应该办。办多少,各地斟酌。

木帆船兽力车也可以办合作社,几百万劳动者应该组织起来。

三、全面规划,加强领导

要有全面规划。我的文章上讲合作社规划,除此以外,对农业生产,全部经济规划(副业、手工业、多种经营、短距离开荒、移民、绿化山庄——特别是北方的光山希望绿化,北戴河、香山都是石山,也可以绿化——供销社、信用社、技术推广站),文化教育(时事、扫盲、小学、中医、唱戏、放映电影、收音机、出版物),整党建党,整团建团,镇反(四川江津的落后乡,镇反搞得好,江津地委书记的发言,可以看一看),妇女工作(没有妇女,就没有儿子,我们都是妇女生的,轻视妇女就是反对母亲,反对母亲就是不孝),青年工作都要规划。

每个合作社都要有自己的规划,每个乡要有全乡的规划,全国应该有二十二万个乡规划。区的规划有人说不要,我看要一个好,不然没有责任。每个县要有全县的规划。省、自治区都要有规划,着重做好县、乡两级的规划,各省应该先搞出一两个乡、县的规划,发出去叫大家照办。

发展速度:根据大家意见,分三种地区:一种是多数地区;一种是一部分少数地区;一种是另一部分少数地区。

1、多数地区:1956、1957、1958年,三个冬季,三个浪潮。一波未平,一波又起。两波之间有一伏,两山之间有一谷,应该有,一个间歇。1958年基本上完成半社会主义合作化。

2、一部分少数地区(华北、东北、城市郊区):分两个浪潮,即l956、1957年两个冬春(其中个别地区,也允许一个浪潮,一个冬春),基本上完成半社会主义合作化。

3、另一部分少数地区(西藏、大小凉山除外):希望四个、五个甚至六个浪潮,到1960年基本上完成半社会主义合作化。即使这样,也不能算机会主义。因为需要那么长的时间,是短不了的。总之,条件不成熟的不要搞。

什么叫做基本上完成半社会主义合作化?就是完成了百分之七十、七十五、八十。

太慢、太快都不好,都叫机会主义。有慢的机会主义,也有快的机会主义。

要加强领导。省、地、县三级,必须时刻掌握运动发展的情况。随时看到问题,随时解决,一有问题就去解决,不要放马后炮(半个省也可开会)。不要爱好事后批评,最好问题刚露头就批评,不要使问题成了堆再批评。情况不对,就要立刻刹车。省、地、县三级,均有剎车之权。

必须注意防左。防左也是马列主义,马列主义不是光防右。速度有前边说的就行了,以后就要比质量,比规格。质量的标准:要增产,不要死牛(我没有说不要死猪,但是猪也不能死)。如何才能达到标准?(1)执行自愿互利政策;(2)全面规划;(3)灵活指导。这三条领导上要抓好。有人拿苏联犯过左倾错误来警告我们,我们要记取苏联的经验。

关键在于今后两年,主要在今后五、六个月(今年十月至明年三月)。这五、六个月,务必不要出大问题。

死牛之风,应该算过去了,不要再重复。牛也要讲衣食住,要有人管。作价入社与否,总之不要死牛。中国共产党如果有本领,今后不要发生死牛之风了。要做到三不叫:人不叫、牛不叫、猪不叫。叫了就有死的可能。

今后五个月之内,省、地、县、区、乡主要负责人,首先是书记、副书记,都要钻进合作社,成为内行。钻不进去,就要改换工作,就不叫他钻了。

五个月之后,中央也许又召开这样的会议。每省也许请个别县委书记参加,全国增加二、三十人。都要写发言稿,要有新的东西,讲新的问题,如全面规划、经营管理、领导方法(如何作到又多又快又好)。

领导方法的几个建议:

1、一年开几次会,开大会,开小会。

2、遇到问题随时解决,不要使问题成了堆。只要搞清楚几个合作社,就可以作出结论。捉一个麻雀,解剖开;麻雀虽小,肝胆俱全。中国麻雀,外国麻崔都一样,不希望个个去解剖。

3、打电报、打电话。

4、巡视检查。坐吉普车也可以,坐马车也可以,两条腿跑也可以。

5、改善刊物。各省的刊物我都看了,有编得好的。缺点是编辑不负责。不管内容如何,见了文章就登。刊物的字不要太小,行不要太密,不要用新五号字,要用老五号字,使人家容易看。这次编“怎样办农业生产合作社”,中央农村工作部看了一千二百篇。我关起门十一天,看了一百二十篇,周游列国,写按语序言。我看每个省,每个自治区,每年要编一本书,每县要有一篇,公开出版。说是“党内秘密”,我看毫无秘密。“怎样办农业生产合作社”,就叫人民出版社出版,给每个民主人士送一本。

6、发简报。县委对地委、地委对省委十天一次简报,紧张时五天一次。内部是:发生什么问题,进度如何。省委对中央半月一次简报,紧张时十天一次,内容要简单明了,写几百字就够了。

四、关于思想斗争

这次会议,在思想上交了锋,这很好。历来经验证明:思想斗争要中肯,思想要交锋,你一刀,我一枪。不交锋就缺乏明确性、彻底性。交了锋,就会帮助大多数同志把问题搞清楚,帮助犯错误的同志改正错误。

历史上犯错误的人有两种:一种是愿意改正的,一种是不愿改正的,要争取愿意改正的。对犯错误的同志有两条:一条,本人愿意继续革命;另一条,别人也要准许人家革命。凡是不准许人家革命的就很危险。白衣秀士王伦,赵太爷不准许人家革命,陈独秀、张国焘、高岗、饶漱石,也是不准许人家革命,结果自己的命被人革了。历史上犯过经验主义、教条主义错误的人,大多数都能改正过来。一条是自己改,一条是要人家帮助,要有欢迎批评的态度。除极少数人如陈光、戴季英等人外,都是可以改正的。

中央农村工作部一部分同志犯了错误,首先是邓子恢同志。错误的性质是经验主义的右倾错误。邓子恢同志已经作了检讨,有人说检讨不彻底,中央政治局认为基本上是好的。邓子恢同志过去在长期斗争中做了许多工作,是有成绩的。但不要把成绩当成包袱。只要虚心一点,不摆老资格,就可以改正错误。

“四大自由”、“巩固新民主主义秩序”是资产阶级的纲领,是反二中全会决议的。提出这样的口号,是纲领性的错误。“四大自由”应有限制,改为“四小自由”。经过利用、限制和改造,最后把它搞掉。不能既不限制,又不改造。要搞掉,需要有准备,要有代替的东西,没有找到代替的东西,就要搞掉,就会犯“左”倾错误。

有些同志“言不及义,好行小惠”。这些同志不好好学习,两耳不闻窗外事,根本不理党的决议。自己不翻书本,也不叫秘书查一查。

有些同这很喜欢分散主义,爱闹独立王国,爱独裁,不爱和别人商量,或者口头上拥护集体领导,实际上爱个人独裁。总觉得独裁舒服,受了批评不舒服,好像不独裁就不像个领导的样子。资产阶级还有他们的民主,难道我们还要独裁吗?

有些同志只办公事,不研究问题,不接触干部,不接近群众,不与人商量,不交换意见,老是教训人。这些同志嗅不到政治气候,虽然事物已经大量普遍存在,但感觉不到。这是得了政治上的感冒。

五、关于一些个别性质的问题

1.改变富裕中农在合作社内的领导地位,要讲步骤、方法。要使本人和群众都知道不宜作领导,并且要有较好的代替人。处理方法要分别不同情况,不要一阵风都拉下来。有些人可以拉下来;有些人可以降为副职,或者作委员。个别干得好的还要继续存在,不要动。不要把富裕中农当富农看待,他们对合作化的态度和富农不同,他们是动摇的,富农是反对的,他们是劳动者,不是剥削者。

2.在支部和群众中说明:我们分阶层,分上、下中农,并不是划成分。是因为上下中农的经济地位和政治态度不同,对合作社有热心与不热心的区别。就是在一个阶层内,态度也有不同。所以要分期分批吸收入社。几年之后,所有农民都入了社,也就没有阶层的分别了。

3.地主、富农入社,在基本合作化的地区,可以按其具体表现,分批吸收入社。过去说全国达到百分之五十以上才能吸收,现在有一个县基本合作化以后就可以吸收了。对他们中间老实的,可以给社员称号。对不老实的,有的不给社员称号,叫“候补社员”,劳动改造,经济上同样给报酬;有的不让入社。所有入社的地主、富农,三年、五年内,一律不准担任领导职务。地主、富农出身的青年知识分子,可不可以当文化教员?我看可以,但不作结论,你们再研究。不要叫当会计,当会计是有点危险的。

4.高级社的条件,请各地按实际情况研究。条件成熟了就可以办。

5.合作化与粮食“三定”要两不误。粮食“三定”与合作化的时间如何安排,由各地自己决定。

6.征兵,个别地方可以推迟到明年四月,但一般不要推迟。

7.夏秋之间可以建社,不要完全集中到冬春。但在两个浪潮之间要休整。人每天都要休整一次,不休整,换不过气来。

8.“勤俭办社”的口号很好。城市反浪费,乡村也要反浪费。勤俭起家,勤俭办社,勤俭建国。不要懒惰,不要豪华。提高劳动生产率,降低成本,厉行节约,反对浪费,经济核算等等,都要这样办,但经济核算要逐步推行。

9.明年四月以前,不要在县、区、乡三级搞肃反。去年结合建社搞粮食和征兵,工作就粗糙。今年要细法,不要搞得哇哇叫。

10.国营农场问题,这次没有讲,是个缺点。下次要有人讲。中央农村工作部下去研究一下。将来国营农场的比重会一天一天大起来。

11.继续地、严格地反对大汉族主义。大汉族主义就是资产阶级思想。当然狭隘民族主义也要反对,但首先是反对大汉族主义。对少数民族要真心真意地帮助他们。没有少数民族不行,他们占土地百分之七十至八十,财富多,没有赛福鼎、乌兰夫不行。

12.在合作社中,要把文盲扫掉,不要把扫盲扫掉。

13.什么叫“左”倾?什么叫右倾?事物在时间、空间(特别是时间)中运动,人们观察事物如果不合实际情况,看过了叫“左”,看不到叫右。例如合作化大发展的条件成熟了,不去大发展,就右了;明年一年要全国达到百分之八十,就“左”了。中国有句老话:“瓜熟蒂落”,“水到渠成”就是说要自然地而不是勉强地达到目的。好像妇女生娃娃,七个月就压出来,是“左”了;过了九个月不准出来,是右了。

14、是否有发生“左”倾的可能?完全有此可能。如果领导上不注意发展情况,不注意群众觉悟,没有全面规划,没有分期分批,只喜欢数量,不注意质量,没有控制,就一定会发生“左”倾错误,一定会弄得人叫、牛叫、猪叫;叫了就会死,死人、死牛、死猪。是不是会要“倒宣传”,必须设想到一切可能发生的不利情况和各种困难,如可能减产,死牛等。要公开向群众讲,使群众有准备,当然也不要把群众吓倒了。

在适当时机,在人们的脑筋膨胀起来的时候,压缩一下,不使它过分膨胀,是必要的。

是不是根本不要忧虑?必要的忧虑和清规戒律仍然要,不是完全不要。猪八戒还有三规五戒。必要的间歇、刹车、关闸是应该有的。当人们的尾巴刚翘起来的时候,就给他新的任务(例如明年要比质量),使他来不及骄傲。

15.允许县一级有百分之十的机动数可不可以呢?我看可以,不作结论,建议你们考虑。

16.煤油太贵,是否减价?陈云同志说可以解决。

17.有人怀疑将来会不会翻案,我看大势所趋,翻不了案。

18.有人问趋势如何?大约十五年左右即三个五年计划左右基本上建成社会主义。还要加一点:大约五十年到七十五年左右即十个到十五个五年计划左右,可以争取赶上或超过美国。在这个时期内,国际国内,党内党外,一定会发生许多严重复杂的程度不同的冲突和斗争,一定会有许多困难,例如世界大战,耍原子弹,出贝利亚、高岗、饶漱石等。有许多事,现在还不能预料到。但我们是马克思主义者,一切困难都是可以克服的,一定会出现一个强大的社会主义中国,五十年以后会出现一个共产主义的中国。

19.决议和章程:决议会议通过,政治局修改后,很快公布。章程等于第二个宪法,还要国务院公布,征求意见,然后提交全国人民代表大会批准。

20.所有发言稿都以可带回去,但不要印。每个人留下一份修正稿,要换的稿子,十月二十五日前交中央办公厅。“怎样办农业生产合作社”的样本可以带回。现在的文章,古文多,半文半白,“应该”只写“应”,“并且”只写“并”,“时候”只写“时”,“贯彻执行”只写“贯彻”。写文章,要讲逻辑(文章结构有内部联系,前后不冲突),要讲文法修辞(文字紧密,语言生动),应该请文章专家帮忙。

21.如何传达?讨论工作中,把记得的用自己的语言说一说。决议和章程要学习。

22.注意作好“八大”代表的选举工作。

23.周总理给大家作一次时事报告。

24.省、大区为单位开一次小会。


