\section[党内通讯(一九五九年四月二十九日)]{党内通讯}
\datesubtitle{(一九五九年四月二十九日)}


省级、地圾、县级、社级、队级、小队级的同志们:

我想和同志们商量几个问题,都是关于农业的。

第一个问题,包产问题。南方正在插秧,北方也正在春种,包产一定要落实。根本不要管上级规定的那一套指示。不要管那些,只管现实可能性。例如去年亩产只有三百斤的,今年增产一百斤,也就很好了。吹上八百斤,一千斤、一千二百斤,甚至更多,吹牛而已,实际办不到,有何益处呢?又例如去年亩产五百斤的,今年增产二百斤、三百斤的也就算成成绩很大了。再增上去一般总不可能的。

第二个问题,密植问题。不可太稀,不可太密。许多青年干部和某些上级机关缺少经验,一个劲要密。有些人觉说愈密愈好。不对。老农怀疑,中年人也有怀疑的,这三种人开一个会,得出一个适当的密度,那就好了。既然要包产,密植问题就得由生产队、生产小队商量决定,上面死硬的密植命令,不但无用而且害人不浅。因此,根本不要下这种死硬的命令。省委可以规定一个密植幅度,不当做命令下达,只给下面参考。此外上面要精心研究到底,密植程度如何为好,积累经验,根据因气候不同,因地点不同,因土、肥、水、种等条件不同,因各种作物的情况不同,因田间管理水平高底不同,作出一个比较科学的密植程度的规定,几年内达到一个实际可行的标准,那就好了。

第三个问题,节约粮食问题。要十分抓紧,按人定量,忙时多吃,闲时少吃,忙时吃干,闲时半干、半稀,杂些番薯、青菜、萝卜、瓜豆、芋头之类,此事一定要十分抓紧。每年一定要把收割、保管、吃用三件事(收、管、吃)抓得很紧很紧,而且要抓得及时,机不可失,时不再来。一定要有储备粮,年年储一点,逐年增多,经过十年、八年的奋斗,粮食问题可能解决,在十年内一切大话、高调切不可讲,讲就是十分危险的,须知我国是一个六亿五千万人口的大国,吃饭是第一件大事。

第四个问题,扩种面积要多的问题。少种、高产、多收的计划,是一个远景计划,是可能的,但在十年内不能全部实行,也不能大都实行,十年以内只看情况逐步实行。三年以内大都不可行。三年以内要力争多种。目前几年的方针是:广种薄收与少种、多收的高额丰产田,同时实行。

第五个问题,机械化问题。农业的根本出路在于机械化,要有十年时间。四年以内小解决,七年以内中解决,十年以内大解决。今年、明年、后年、大后年这四年内,主要依靠改良农具、半机械化农具。每省、每地、每县都要设一个农具研究所。集中一批科学技术人员和农村有经验的铁匠、木匠,搜集全省、全地、全县各种比较进步的农具,加以比较,加以改进,试制新式农具。试制成功,在田里实验,确实有效,然后才能成批制造。加以推广。提到机械化,用机械制造化学肥料这件事,必须包括在内。逐年增加化学肥料,是一件十分重要的事。第六个问题,讲真话问题。包产能包多少,就讲能包多少,不讲经过努力实在做不到,而又勉强讲做得到的假话。收获多少就讲多少。不可以讲不合实际情况的假话。各项增产措施,实行八字宪法,每项都不可讲假话。老实人,敢讲真话的人,归根到底于人民事业有利,于自己且不吃亏。爱讲假话的人一害人民,二害自己,总是吃亏。应当说,有许多假话是上面压出来的。上面“一吹、二压、三许愿”使下面很难办。因此,干劲一定要有,假话一定不可讲。

以上六件事,请同志们研究,可以提出不同的意见,以求得真理为目的。我们办农业、工业的经验还很不足,一年、二年积累经验,再过十年,客观必然性可能逐步被我们认识,在某种程度上,我们就自由了。什么叫自由?自由是必然的认识。

同现在流行的一些高调比较起来,我在这里唱的是低调,目的在于真正调动积极性,达到增产的目的,如果事实不是我讲的那样低,而达到了较高的目的,我变为保守主义者,那就谢天谢地,不胜光荣之至。


