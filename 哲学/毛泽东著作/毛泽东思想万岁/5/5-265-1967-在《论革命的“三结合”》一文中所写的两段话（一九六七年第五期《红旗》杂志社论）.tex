\section[在《论革命的“三结合”》一文中所写的两段话(一九六七年第五期《红旗》杂志社论)]{在《论革命的“三结合”》一文中所写的两段话(一九六七年第五期《红旗》杂志社论)}
\datesubtitle{(一九六七年)}


在需要夺权的那些地方和单位,必须实行革命的“三结合”的方针,建立一个革命的、有代表性的、有无产阶级权威的临时权力机构。这个权力机构的名称,叫革命委员会好。

从上至下,凡要夺权的单位,都要有军队代表或民兵代表参加,组成“三结合”,不论工厂、农村、财贸、文教(大、中、小学)、党政机关及民众团体都要这样做。县以上都派军队代表,公社以下都派民兵代表,这是非常之好的。军队代表不足,可以暂缺,将来再派。

