\section[中国共产党第八届全国代表大会开幕词(一九五六年九月十五日)]{中国共产党第八届全国代表大会开幕词}
\datesubtitle{(一九五六年九月十五日)}


同志们:

中国共产党第八次全国代表大会,现在开幕了。(全体起立,长时间的热烈鼓掌)

从我们党的第七次全国代表大会以来的十一年间,在全中国和全世界,为了共产主义和人类解放事业而英勇奋斗和辛勤工作,因而付出了自己生命的同志和朋友,是很多的,我们应当永远纪念他们。(全体起立,默哀)

我们这次大会的任务是:总结从七次大会以来的经验,团结全党,团结国内外一切可能团结的力量,为了建设一个伟大的社会主义的中国而奋斗。(热烈鼓掌)

在七次大会以来的十一年中,我们在一个地广人多、情况复杂的大国内,彻底地完成了资产阶级民主革命,又取得了社会主义革命的决定性的胜利。在两个革命的实践中,证明了从七次大会到现在,党中央委员会的路线是正确的,我们的党是一个政治上成熟的伟大的马克思列宁主义的政党。(热烈鼓掌)我们的党现在比过去任何时期都更加团结,更加巩固了。(热烈鼓掌)我们的党已经成了团结全国人民进行社会主义建设的核心力量。(热烈鼓掌)我们各方面的工作都有很大的成绩。我们的工作是做得正确的,但是也犯过一些错误。在这次大会上,需要把我们工作中的主要经验,包括成功的经验和错误的经验,加以总结,使那些有益的经验得到推广,而从那些错误的经验中取得教训。

就国内的条件来说,我们胜利的获得,是依靠了工人阶级领导的工农联盟并且广泛地团结了一切可能团结的力量。为了进行伟大的建设工作,在我们的面前,摆着极为繁重的任务。虽然我们有一千多万党员,但是在全国人口中仍然只占极少数。在我们的各个国家机关和各项社会事业中,大量的工作要依靠党外的人员来作。如果我们不善于依靠人民群众,不善于同党外的人员合作,那就无法把工作做好。在我们继续加强全党团结的时候,我们还必须继续加强各民族、各民主阶级、各民主党派、各人民团体的团结,继续巩固和扩大我们的人民民主统一战线,必须认真地纠正在任何工作环节上的任何一种妨害党同人民团结的不良现象。

在国际范围内,我们胜利的获得,是依靠了以苏联为首的和平民主社会主义阵营的支持,(热烈鼓掌)以及全世界爱好和平的人民的深厚同情,(热烈鼓掌)现在,国际形势的发展对于我国的建设事业是更加有利了。我国和各社会主义国家都希望和平,世界各国的人民也都需要和平,渴望战争、不要和平的,仅仅是少数帝国主义国家中的某些依靠侵略发财的垄断资本集团。由于爱好和平的国家和人民的不断努力,国际的局势已经趋向和缓。(鼓掌)为了争取世界的持久和平,我们必须进一步地发展同社会主义阵营中各兄弟国家的友好合作,(热烈鼓掌)并且同一切爱好和平的国家加强团结。(热烈鼓掌)我们必须争取同一切愿意和我们和平相处的国家,在互相尊重领土主权和平等互利的基础上,建立正常的外交关系。亚洲、非洲和拉丁美洲各国的民族独立解放运动,以及世界上一切国家的和平运动和正义斗争,我们都必须给以积极的支持。(热烈鼓掌)我们坚决支持埃及政府收回苏伊士运河公司的完全合法的行动,坚决反对任何侵犯埃及主权和对于埃及实行武装干涉的企图。(热烈鼓掌)我们必须使帝国主义的制造紧张局势和准备战争的阴谋彻底破产。(长时间的热烈鼓掌)

我国的革命和建设的胜利,都是马克思列宁主义的胜利,把马克思列宁主义的理论和中国革命的实践密切地联系起来,这是我们党的一贯的思想原则。许多年来,特别是从一九四二年整风运动以来,我们在加强党内的马克思列宁主义的教育方面,做了许多工作。现在,比起整风运动以前,我们党的马克思列宁主义的思想水平,已经提高了一步。但是我们还有严重的缺点。在我们的许多同志中间,仍然存在着违反马克思列宁主义的观点和作风,这就是:思想上的主观主义、工作上的官僚主义和组织上的宗派主义。这些观点和作风都是脱离群众、脱离实际的,是不利于党内和党外的团结的,是阻障我们事业进步、阻碍我们同志进步的。必须用加强党内的思想教育的方法,大力克服我们队伍中的这些严重的缺点。(鼓掌)

十月革命以后,列宁给苏联共产党提出了这样的任务:学习,再学习。苏联的同志们,苏联的人民,按照列宁的指示做了。他们在不长的时间内,取得了极其灿烂的成就。(长时间的热烈鼓掌)苏联共产党在不久以前召开的第二十次代表大会上,又制定了许多正确的方针,批判了党内存在的缺点。可以断定他们的工作,在今后将有极其伟大的发展。(长时间热烈的鼓掌)

我们现在也面临着和苏联建国初期大体相同的任务。要把一个落后的农业的中国改变成为一个先进的工业化的中国,我们面前的工作是很艰苦的,我们的经验是很不够的。因此,必须善于学习。要善于向我们的先进者苏联学习,(鼓掌)要善于向各人民民主国家学习,(鼓掌)要善于向世界各兄弟党学习,(鼓掌)要善于向世界各国人民学习,(鼓掌)我们决不可有傲慢的大国主义的态度,决不应当由于革命的胜利或在建设上有了一些成绩而自高自大。国无论大小,都各有长处和短处。即使我们的工作得到了极其伟大的成绩,也没有任何值得骄傲自大的理由。虚心使人进步,骄傲使人落后,我们应当永远记住这个真理。(热烈鼓掌)

同志们,我和大家都相信:已经得到解放的中国人民的力量是无穷无尽的,我们又有伟大的盟国苏联和其他兄弟国家的援助,(鼓掌)我们又有世界上一切兄弟党的支持,(鼓掌)又有世界上一切同情者的支持,(鼓掌)我们并没有孤立的感觉,这样,我们就一定能够一步一步地把我国建设成为一个伟大的社会主义工业化的国家。(热烈鼓掌)我们这次大会,对于我国的建设事业的前进,将要起很大的推动作用。(鼓掌)

今天在座的有五十几个国家的共产党、工人党、劳动党和人民革命党的代表。(长时间热烈的鼓掌)他们都是马克思列宁主义者,他们和我们有一种共同的语言。(鼓掌)他们走了很长的路程来到我国,以崇高的友谊参加我们党的这次代表大会。这对于我们是一个很大的鼓舞和支持。(热烈鼓掌)我们对他们表示热烈的欢迎。(全体起立,长时间的热烈的鼓掌)

今天在座的还有我们国内各民主党派和无党派民主人士的代表。(热烈鼓掌)他们是和我们一道工作的亲密的朋友。(鼓掌)他们一向给了我们很多的帮助。(鼓掌)我们对他们表示热烈的欢迎。(全体起立,长时间的热烈的鼓掌)

<p align="right">(《新华半月刊》一九五六年第二十号)</p>


