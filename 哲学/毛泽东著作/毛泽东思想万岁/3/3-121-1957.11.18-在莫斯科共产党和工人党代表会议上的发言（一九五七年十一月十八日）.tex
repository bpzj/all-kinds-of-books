\section[在莫斯科共产党和工人党代表会议上的发言(一九五七年十一月十八日)]{在莫斯科共产党和工人党代表会议上的发言}
\datesubtitle{(一九五七年十一月十八日)}


同志们:

我讲几句话。请同志们允许我即席讲话。因为我在几年前害过一次脑贫血症,最近两年好一些,站起来讲话还有些不方便。

我想讲两个问题:形势问题,团结问题。

现在我感觉到国际形势了一个新的转折点。世界上现在有两股风:东风,西风。中国有句成语:不是东风压倒西风,就是西风压倒东风。我认为目前形势的特点是东风压倒西风,也就是说,社会主义的力量对于帝国主义的力量占了压倒的优势。

四十年前的十月革命是整个人类历史转折点,怎么现在又有转折点呢?还是有的。打希特勒,在一个时期,有一、二年时间,希特勒占了上风。那时,希特勒不但占领大半个欧洲,而且打进苏联,苏联让出了一大块土地,可见希特勒那时占了上风。斯大林格勒一战成为转折点,从此希特勒就走下坡路,苏联就势如破竹一直打到柏林。这不是一个转折点吗?据我看来,斯大林格勒一仗,是整个第二次世界大战的转折点。

去年,最近这几年,西方世界非常猖狂,利用我们阵营中的一些问题,特别是匈牙利事件,在我们阵营的脸上擦黑,我们的天上飞起许多乌云。但是匈牙利反革命被镇压下去了。在苏伊士运河中,苏联的警告也起了制止侵略战争的作用。西方擦黑我们的脸的目的,依我看,主要是想“整”各国共产党。在这一方面,他们也达到了一部分目的。例如美国的法斯特,共产主义的可耻的叛徒,就跑出党去了。还有一些共产党也跑出去了一些人。帝国主义对此大为高兴。我想我们也应该高兴,叛徒跑出去有什么不好?

今年,一九五七年,形势大为不同了,我们的天上是一片光明,西方的天上是一片乌云。我们很乐观,而他们呢,却是惶惶不安。两个卫星上天,使他们睡不着觉。六十几国共产党在莫斯科开会是从来没有过的事,从来也没有这样大的规模。但是在社会主义阵营各国中,在各国共产党中,特别是在各国人民中,还有相当多的人总是相信美国了不起。你看,它还有那么多纲,有那么多飞机大炮。我们的比他们的少。西方国家无数的报纸、广播电台天天吹,美国之音、自由欧洲电台等等吹得神乎其神,于是乎造成一种假象,欺骗了相当多的一那分人。我们就要揭穿这种欺骗。我有十件证据来说明这个问题:究竟是他们行还是我们行,究竟是东风压倒西风,还是西风压倒东风?

第一件,打希特勒的时候,罗斯福和邱吉尔的手里有多少钢呢?大约有七千万吨。可是吃不下希特勒,毫无办法。总要想个办法吧,于是采用了旅行的办法,一走就走到雅尔达,请求苏联帮助。那时,斯大林手中有多少钢呢?在战前有一千八百万吨。因为在战争中损失了许多地方,据××××告诉我,钢产打了个对折,剩下九百万吨。有七千万吨钢的人,来请求有九百万吨钢的人。条件是什么呢?易北河以东划为红军的进攻区,就是说,他们忍痛下决心让这一大块区域脱离他们的体系,让这一大块区域有可能为社会主义体系。这件事很有说服力,说明物资力量多少不完全决定问题,人是主要的,制度是主要的。雅尔达又谈到打日本。又是美国人吃不下日本,又是要请共产主义帮助。中国的满洲,朝鲜的一部分,作为红军的攻击区。并且决定让日本退还半个库页岛、一个千岛群岛。这也是忍痛让步呵!为了吃掉他们的同伴——日本帝国主义。

第二件,中国革命。一九四九年初国民党被我们打得呜呼哀哉的时候,向杜鲁门人喊救命说:美国老爷呀,你出几个兵吧!杜鲁门说:我一个兵也不能出!于是国民党又说:你可以不可以讲几句话呢?说:长江以南这块地方,如果共产党到了那里的时候。美国就不能坐视。杜鲁门说:这个不行,讲不得的,共产党很厉害。于是乎蒋介石只好开跑。他现在在台湾。

第三件,朝鲜战争。在开始的时候,美国一个师有八百门炮,中国志愿军三个师才有五十多门炮。但是一打就像赶鸭子一样,几个星期就把美国人赶了几百公里,从鸭绿江赶到三八线以南去了。后来美国人集中了力量进行反攻,我们和金日成同志退到了三八线相持,构筑阵地。一打,整个朝鲜战争差不多打了三年,美国的飞机就像黄蜂一样,我们在第一线一架飞机也没有。双方同意讲和。在什么地点?他们说在一条丹麦的船上。我们说在开城,在我们的地方。他们说:好。因为在我们地方,他们每天开会得打着白旗子来,开完会打着白旗子回去。后来,他们感到不好意思了——天天打白旗子。说改一个地方吧,改到双方战线的中间,地点叫板门店,我们说也可以。但是又谈了年把,美国总是不甘心签字,拖。最后,在一九五三年,我们在三八线上突破了二十一公里的防线,美国人吓倒了,马上签字。那么厉害,有那么多钢的美国人也只得如此。这个战争,实际上是三国打的,朝鲜、中国、苏联。苏联出了武器。但是敌人方面呢,有十六个国家。

第四件,越南战争。法国人被胡志明打得呜呼哀哉,屁滚尿流。有人可以作证,胡志明同志在座。法国人不想干了,美国人一定要干,因为他的钢多。但是美国人也只是出武器,维持紧张局势,出兵就不来。于是乎有日内瓦会议,把大半个越南划给越南民主共和国。

第五件,苏伊士运河事件。两个帝国主义进攻,打了几天,苏联人讲了几句话.就缩回去了。当然还有第二个因素,就是全世界在讲活,反对英法侵略。

第六条叙利亚。美国作好了计划要打,又是苏联人讲了之几句话,还任命了一个将军,叫作罗科索夫斯基。作了这两件事情,他们说不好打了。这件事情还没有完结,还要警惕,可能将来还闹乱子。但是现在总算没有打。

第七件是苏联抛上了两个卫星。抛卫星的国家有多少钢?五千一百万吨。不是讲美国非常厉害吗?你为什么到现在连一个山药旦还没有抛上去?你有一万万吨钢,牛皮吹得那么大呀,做出了先锋计划。先锋计划要改名了,得改成落后计划。

由这七件事,我想可以得出一个概念:西方世界被抛到我们后面去了。抛得很近还是抛得很远?照我讲一一也许我这个人有些冒险主义,我说,永远地抛下去了。在苏联发射人造卫星以前,社会主义国家在人心归向、人口众多方面已经对于帝国主义国家占了压倒的优势;而在苏联发射人造卫星以后,就在最重要的科学技术部门方面也占了压倒的优势。人们说,美国也会赶上来的,他也会抛卫星的。这是真的。××××的告诉了美国会抛了卫星的。但是他们现在正在争论究竟是一年、二年还是五年才能赶上苏联。我不管你是一年、二年还是五年,你总是被抛到后面去了。我们的苏联同志,大概只是晚上睡觉,白天不睡觉。所有苏联人不会白天晚上,一年、二年、五年总是睡觉吧。你一年、二年、五年赶上苏联,但是苏联又前进了。

……归根结底,我们要争取十五年和平。到那个时候,我们就无敌子天下了,没有人敢同我们打了,世界也就可以得到持久和平了。

现在还要估计一种情况,就是想发动战争的疯子,他们可能把原子弹、氢弹到处摔。他们摔,我们也摔,这就打得一塌糊涂,这就要损失人。问题要放在最坏的基点上来考虑。我和一位外国政治家辩论过这个问题。他认为如果打原子战争,人会死绝的。我说,极而言之,死掉一半人,还有一半人,帝国主义打平了,全世界社会主义化了,再过多少年,又会有二十七亿,一定还要多。我们中国还没有建设好,我们希望和平。但是如果帝国主义硬要打仗,我们也只好横下一条心,打了仗再建设,每天怕战争,战争来了你有什么办法呢?我先是说东风压倒西风,战争打不起来,现在再就如果发生了战争的情况,作了这些补充的说明,这样两种可能性都估计到了。

我讲十件证据,刚才讲了七件,下面再讲三件。

第八件是英国退出亚洲、非洲很大一片土地。

第九件是荷兰退出印尼。

第十件是法国退出叙利亚、黎巴嫩、摩洛哥、突尼斯,在阿尔及利并没有办法。

落后国家强些,还是先进国家强些?印度强些.还是英国强些?印尼强些,还是荷兰强些?阿尔及利亚强些,还是法国强些?我看所有帝国主义就是下午六点钟的太阳,而我们呢,是早上六点钟的太阳。于是乎转折点就来了。就是说,西方国家抛到后边了,我们大大占上风了。一定不是西风压倒东风,因为西风是那么微弱。一定是东风压倒西风,因为我们强大。

问题是不能用钢铁数量多少来做决定,而是首先由人心的向背来做决定的。历史上从来就是如此。历史上从来就是弱者战胜强者,没有枪的人战胜全副武装的人。布尔什维克曾经一支枪也没有。苏联同志告诉我,二月革命的时候,只有四万党员;十月革命的时候,也只有二十四万党员。联共党史简明教程那本书上的第一页第一节写了一个辩证法:从小组到全国,开头是稀稀拉拉几十几个的小组,后来变成整个国家的领导者。苏联同志,你们修改“联共党史”的时候,我希望不要把这几句话修改掉了。我们中国也是如此,开头是稀稀拉拉几十个人的共产主义者的小组,现在发展成一千二百万党员。我这活是特别想同资本主义国家共产党同志们交换意见的,因为他们现在还在国难中;有些党很小,有些党成批党员退出党。我说这不足为怪,也许是好事。我们的道路是曲折的,是按照螺旋形上升的。

还要讲个纸老虎问题。一九四六年蒋介石开始向我们进攻的时候,我们许多同志,全国人民,都很忧虑,战争是不是能够打赢?我本人也忧虑过这件事。但是我们有一条信心。那时有一个美国记者到了延安,名字叫安娜.路易斯.斯特朗。这个人就是在苏联住了二、三十年,后来被斯大林赶走,以后又被××××同志恢复了名誉的那位女作家。我同她谈话的时候谈了许多问题,蒋介石、希特勒、日本、美国、原子弹等等。我说一切所有号称强大的反动派统统不过是纸老虎(Papertiger)。原因是他们脱离人民。你看,希特勒是不是纸老虎?希特勒不是被打倒了吗?我也谈到沙皇是纸老虎,中国皇帝是纸老虎,日本帝国主义是纸老虎,你看,都倒了。美帝国主义没有倒,还有原子弹,我看也是要倒的,也是纸老虎。蒋介石很强大,有四百多万正规军。那时我们在延安。延安这个地方有多少人?有七千人。我们有多少军队呢?我们有九十万游击队,统统被蒋介石分割成几十个根据地。但是我们说,蒋介石不过是一个纸老虎,我们一定会打赢他。为了同敌人作斗争,我们在一个长时间内形成了一个概念,就是说,在战略上我们要藐视一切敌人,在战术上我们要重视一切敌人。也就是说在整体上我们一定要藐视它,在一个一个的具体问题上我们一定要重视它,如果不是在整体上藐视敌人,我们就要犯机会主义的错误。马克思、思格斯只有两个人,那时他们就说全世界资本主义要被打倒。但是在具体问题上,在一个一个敌人的问题上,如果我们不重视它,我们就要犯冒险主义的错误。打仗只能一仗一仗地打,敌人只能一部分一部分地消灭。工厂只能一个一个地盖,农民犁田只能一块一块地犁。就是吃饭也是如此。我们在战略一上藐视吃饭:这顿饭我们能够吃下去。但是具体地吃,却是一口口地吃的,你不可能把一桌酒席一口吞下去。这叫做各个解决,军事书上叫作各个击破。


