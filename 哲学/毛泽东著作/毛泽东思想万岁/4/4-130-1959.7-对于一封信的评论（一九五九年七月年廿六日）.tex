\section[对于一封信的评论(一九五九年七月年廿六日)]{对于一封信的评论(一九五九年七月年廿六日)}
\datesubtitle{(一九五九年七月)}


收到一封信,是一个有代表性的文件。信的作者在我们的经济工作搜集了一些材料,这些材料专门属于缺点方面的。作者只对这一方面的材料有兴趣;而对另一方面的材料,成绩方面的材料,可以说根本不发生兴趣。他认为,从一九五八年第四季度以来,党的工作中缺点错误是主流。因此做出结论说,党犯了“左倾冒险主义”,“机会主义”的错误,而其根源在于一九五七年整风反右斗争没有“同时”反对左倾冒险主义的危险。作者李云仲同志(他是国家计委一个付局长,不久前调任东北协作区委员会办公厅综合组组长)的基本观点是错误的。他几乎否定一切。他认为几千万人上阵大炼钢铁损失极大而毫无效益,人民公社也是错误的,对基本建设极为悲观,对农业他提到水利,认为党的“左倾冒险主义,机会主义”错误是办水利引起的,他对前冬去春几亿农民在党的领导下大办永利没有好评。他是一个“得不偿失”论者,某些地方简直是“有失无得”论者。作者的这些结论性的观点放在第一段,篇幅不多。这个同志的好处是把自己的思想和盘托出。他跟我们看见的另一些同志,他们对党和人民的工作基本上不是高兴,而是不满,对成绩估计很不足,对缺点估计过高,为现在的困难所吓倒、对干部不是鼓劲而是泄气,对前途信心不足,甚至丧失信心,但是不愿意讲出自己的想法和看法,或者讲一点,留一点,而采取“足将进而趔趄,口将开而嗫嚅”,躲躲闪闪的态度大不相同。李云仲同志和这些人不同,他不隐瞒自己的政治观点,他满腔热情地写信给中央同志,希望中央采取步骤克服现在的困难。他认为困难是可以克服的,不过时间要长些,这些看法是正确的。信的作者对计划工作中缺点的批评占了大部分篇幅,我认为很中肯。十年来还没有一个愿意和敢于向中央中肯地有分析地系统地揭露我们计划工作的缺点,因而求得改正的同志。我就没有看见这样一个人。我知道这种人是有的,他们就是不敢越衙上告。因此我建议,将此信在中央一级和地方一级(省、市、自治区)共两级的党组织中,特别是计划机关中,予以传阅并且展开辩论,将一九五八年、一九五九年两年自己所做的工作的长短,利害得失,加以正确的分析,以利统一认识,团结同志,改善工作,鼓足干劲,奋勇前进,争取经济工作及其他工作(政治工作、军事工作、文教卫生工作、党的各级组织的领导工作,工、青,妇工作)的新的伟大胜利。党中央从去年十一月第一次郑州会议以来至此次庐山会议,对于在自己领导下的各项重大工作中的错误缺点在足够地估计成绩(成绩是主要的,缺点是第二位的)的条件下,进行了严肃的批判,这次批判工作已经有九个月了。必须看到,这种批判是完全必要的,而且是迅速地见效和逐步地见效的,又必须看到,这种严肃的认真的批判,必定而且已经带来了一定的付作用,就是对于某些同志有泄气。错误必须批判,泄气必须防止。气可鼓而不可泄。人而无气,不知其可也。我们必须坚持今年三月第二次郑州会议纪录上所说的,在满腔热情地保护干部的精神下,引导那些在工作中犯有错误者,存在缺点者,批判和改正自己的缺点错误。错误并不可怕,就怕不肯批评,不肯改正,就怕因批评而泄了气,必须顾到改错和鼓劲两方面,必须看到批评整改虽然已经几个月了、一切未完工作还必须坚持做完,不可留尾巴。但是现在党内外出现了一种新的事物,就是右倾情绪、右倾思潮、右倾活动已经增加,大有猖狂进攻之势,这表现在此次会议印发各同志的许多材料上。这种情况还没有达到一九五七年党内外右派猖狂进攻那种程度,但是苗头和趋势已经很清楚,已经出现在地平线上了,这种情况是资产阶级性质的,另一种情况是无产阶级内部的思想性质的,他们和我们一样都要社会主义,不要资本主义,这是我们和这些同志的基本相同点。但是这些同志的观点和我们的观点是有分歧的。他们的情绪有些不正常,他们把党的错误估计得过大了一些,而对几亿人民在党的领导下所创造出来的伟大成绩估计得过小了些,他们做出了不适当的结论,他们对于克服当前的困难信心很不足。他们把他们的位置不自觉地摆得不恰当,摆在左派和右派的中间。他们是典型的中间派。他们是“得失相当”论者。他们在紧要关头不坚定,摇摇摆摆。我们不怕右派猖狂进攻,却怕这些同志的摇摆。因为这种摇摆不利党和人民的团结,不利于全党一致鼓足干劲,克服困难,争取胜利。我们相信,这些同志的态度是可能改变的。我们的任务是团结他们,争取他们改变态度。为要达到此目的,必须对此种党内的动态作必要的估计。不可估计太高,认为他们有力量可以把党和人民的大船在风浪中摇翻。他们没有这样大的力量。他们占相对的少数,而我们则占大多数。我们和人民中的大多数(工人、贫农、下中农、一部分上中农和知识分子)是团结一致的。党的总路线和体现总路线的方针、政策、工作方法,是受到广大党员、广大干部和广大人民群众的欢迎的。但也不可把他们的力量估计过低,他们有相当一些人。他们的错误观点在受到批判、接受批判、端正态度以前,是不会轻易放弃自己的观点的,这一点必须看到。党内遇到大问题有争论表现不同的观点,有些人暂时摇摆,站在中间,有些人站到右边去,是正常的现象,无须大惊小怪。归根结底,错误观点乃至错误路线一定会被克服,大多数的人,包括暂时摇摆,甚至犯路线错误的人,一定会在新的基础上团结起来。我们党三十八年的历史就是这样走过来来的。反右必出“左”,反“左”必出右,这是必然的。时然若言。现在是讲这一点的时候了,不讲对团结不利,于党于个人都不利。现在这一次争论,可能会被证明是一次意义重大的争论,如同我们在革命时期,各次重大争论一样,在新的历史时期——社会主义建设时期,不可能没有争论的,风平浪静的。庐山会议可能被证明是一次意义重大的会议。“团结——批评——团结”,“惩前毖后,治病救人”,是我们解决党内矛盾的正确的已被历史证明的有效方法,我们一定要坚持这种方法。

我的这些意见,大体已在七月廿三日的全体会议上讲了,但有些未讲完,作为那次讲话的补充,又写了这些话。


