\section[《农民问题丛刊》序(一九二六年九月一日)]{《农民问题丛刊》序}
\datesubtitle{(一九二六年九月一日)}


农民问题乃国民革命的中心问题;农民不起来参加并拥护国民革命,国民革命不会成功;农民运动不迅速的做起来农民问题不会解决;农民问题不在现在的革命运动中得到相当的解决农民不会拥护这个革命。一一这些道理,一直到现在,即使在革命党里面,还有许多人不明白。他们不明白经济落后半殖民地革命最大的对象是乡村宗法封建阶级(地主阶级)。经济落后之半殖民地外,而帝国主义及其统治阶级,对于其他压迫榨取的对象主要是农民,求所以实现其压迫与榨取则全靠那封建地主阶级给他们以死力拥护,否则无法行其压榨。所以经济落后半殖民地的农村封建阶级,乃其国内统治阶级国外帝国主义之唯一坚实的基础,不动摇这个基础,便万万不能动摇这个基础的上层建筑物。中国的军阀只是这里乡村封建阶级的首领,说要打倒军阀而不要打倒乡村的封建阶级,岂非不知轻重本末,明显的例摆在广东:那一个土豪劣绅贪官污吏比较敛迹的县分,必定是农民运动已经做起来有了大的农民群众加入了农民协会的县分。换句话说,即是那一个陈炯明势力削减的县分,必是农民起来的县分。我们无庸讳言:一年以前是陈炯明有广东,革命政府可以说并没有广东;一年以来到现在是在革命政府与陈炯明平分广东天下,虽然陈炯明自己不在广东境内;往后须得农民从广东各县逐渐的起来,才可以确实证明陈炯明的势力从广东各县逐渐的削减下去。陈炯明的故乡历来土豪劣绅贪官污吏集中的海丰县,自从有了五万户二十五万人之县农民协会,便比广东任何县都要清明一县知事不敢为恶,征收官吏不敢额外括钱,全县没有土匪,土豪劣绅鱼肉人民的事几乎绝迹。因此,乃知中国革命的形势只是帝国主义军阀的基础一一土豪劣绅贪宫污吏镇压住农民,便是革命势力的基础一一农民起来镇压住土豪劣绅贪官污吏。中国的革命,只有这一种形式,没有第二种形式。全中国各地都必须办到海丰这个样子,才可以算得革命的胜利,不然任便怎样都算不得。全中国各地必须都办到海丰这个样子,才可以算得帝国主义军阀的基础确实起了动摇,不然,也算不得。因此,乃知所谓国民革命运动,其大部分即是农民运动。因此,乃知凡属不重视甚至厌恶农民运动之人,他实际上即是同情土豪劣绅贪官污吏,实际上即是不要打倒军阀,不要反对帝国主义。

有人以为买办阶级之猖獗于都市,完全相同于地主阶级之猖獗于乡村,二者应相提而并论。这话说猖獗对,说完全相同不对。买办阶级集中的区域全国不过香港、广州、上海、汉口、天津、大连等沿海沿江数处,不若地主阶级之领域在整个的中国各省各乡。政治上全国大小军阀都是地主阶级(破产的小地主不在内)挑选出来的首领,这班封建地主首颔即封建军阀利用城市买办阶级以拉拢帝国主义,名义上实际上都是军阀做主体,而买办阶级为其从属。财政上军阀政府每年几万万元的消耗,百分之九十都是直接间接从地主阶级驯制下之农民身上挖得来的,买办阶级如银行工会等对北京政府有条件的借债,究竟比较甚少。故我总觉得都市的工人学生中小商人应该起来猛击买办阶级,并直接对付帝国主义,进步的工人阶级尤其是一切革命阶级的领导;然若无农民从乡村中奋起打倒宗法地主阶级之特权,则军阀与帝国主义势力总不会根本倒塌。

基此理由,我们的同志于组织工人组织学生组织中小商人许多工作以外,要有大批的同志,立刻下了决心,去做那组织农民的浩大工作。要立刻下了决心,把农民问题开始研究起来。要立刻下了决心,向党里要到命令,跑到你那熟悉的或不熟悉的乡村中间去,夏天晒着酷热的太阳,冬天冒着寒冷的风雪,搀着农民的手,问他们痛苦些什么,问他们要些什么。从他们的痛苦与需要中,引导他们组织起来;引导他们向土豪劣绅斗争;引导他们与城市的工人学生中小商人合作,建立起联合战线,引导他们参与反帝国主义反军阀的国民革命运动。我们预计:全国三万万以上农民群众当中,以十分之一加入农民协会计算,可以得到三千万以上有组织的农民。尤其是南方的粤赣,北方的直鲁豫,中部的鄂皖,几个政治上特殊重要的省分,应该下大力从事组织。有了这几个重要省分的农民起来,其余省分的农民便都容易跟着起来。必须到这时候,帝国主义军阀的基础才能确实动摇,国民革命才能得着确实的胜利。

说到研究农民问题,便感觉太缺乏材料。这种材料的搜集自然要随农民起动的发展才能日即于丰富,目前除广东外各地农运都方在开始,所以材料是异常贫乏。这回尽可能搜集了这一点,印成这一部分丛刊,作为各地农运同志的参考。其中各省农村状况调查一部分,乃农民运动讲习所第六届学生三百余人所做,在学生们分别组织的各该省农民问题研究会内提出讨论,又经过相当的审查才付印的。他们以前多没有农民状况的详细的调查,故所述只属大略。然从前连大略都没有,今有了一点,便也觉得可贵。我们应该拿了这一点大略在不久的时期内从各地的实际工作实际考察中引出一个详细的具体的全国的调查来。关于农业生产问题的材料,本书只收得五种(第二十二种至二十六种)关于此问题的材料并不是很缺乏,因为出版仓卒收集不及,他日尚当另外编印。农民问题本来包括两个方面的问题:即帝国主义军阀地主阶级等人为的压迫与水旱天灾病虫害技术掘劣生产减缩等天然的压迫问题。前一问题固然是目前的紧急问题,同志们的注意力自然都集中在这里。但后一问题也是非常之严重,我们不能不积极的注意。要解决后一个问题,需要着全国的革命的政权与科学的方法,不是即刻能办之事,但时期也就快要到来了,我们应得预先准备。这部书内关于广东的材料,占了八种,乃本书最精华部分,他给了我们做农民违动的方法,许多人不懂得农民运动怎样去做,就请过细看这一部分。他又使我们懂得中国农民运动的性质,使我们知道中国的农民运动乃政治斗争经济斗争这两者汇合一起的一种阶级斗争的远动。内中表现得最特别的尤在政治斗争这一点,这一点与都市工人远动的性质颇有点不同。都市工人阶级目前所争政治上只是求得集会结社之完全自由,尚不欲即时破坏资产阶级之政治地位!乡村的农民,则一起便碰着那土豪劣绅中大地主几千年来持以压榨农民的政权(这个地主政权即军阀政权的真正基础),非推翻这个压榨的政权,便不能有农民的地位,这是现时中国农民运动的一个最大的特色。我们五年来从各地的农民运动经过看来,我们读了这部书的广东农民大会决议案,海丰农民运动报告及广宁普宁两个农民反抗地主始末记,不由得不有些感觉。本书对于外国的材料也搜集了一点(第十五种至第十八种),但是太少,各国尤其是俄国的农民运动农业经济的材料很多,可惜没有人详细的翻过来。本书内惟俄国农民与革命一篇算得比较的详细,我们亦很可以拿来与中国的情形比较一番。

<p align="right">(根据江西省博物馆复制品排印)</p>

