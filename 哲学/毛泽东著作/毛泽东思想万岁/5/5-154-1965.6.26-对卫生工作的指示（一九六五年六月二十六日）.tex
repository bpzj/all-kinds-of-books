\section[对卫生工作的指示(一九六五年六月二十六日)]{对卫生工作的指示}
\datesubtitle{(一九六五年六月二十六日)}


告诉卫生部,卫生部的工作只给全国人口的15%工作,而且这15%中主要还是老爷。广大农民得不到医疗。一无医生,二无药。卫生部不是人民的卫生部,改成城市卫生部或城市老爷卫生部好了。

医学教育要改革,根本用不着读那么多书。华陀读的是几年制?明朝李时珍读的是几年制,医学教育用不着收什么高中生、初中生,高小毕业生学三年就够了。主要在实践中学习提高,这样的医生放到农村去,就算本事不大,总比骗人的医生与巫医的要好,而且农村也养得起。书读得越多越蠢。现在那套检查治疗方法根本不符合农村,培养医生的方法,也是为了城市,可是中国有五亿多农民。

脱离群众,工作把大量人力、物力放在研究高、深、难的疾病上,所谓尖端,对于一些常见病,多发病,普通存在的病,怎样预防,怎样改进治疗,不管或放的力量很少。尖端的问题不是不要,只是应该放少量的人力、物力,大量的人力、物力应该放在群众最需要解决的问题上去。

还有一件怪事,医生检查一定要戴口罩,不管什么病都戴,是怕自己有病传给别人?我看主要是怕别人传染自己。要分别对待嘛!什么都戴,这肯定造成医生与病人间的隔阂。

城市里的医院应该留下一些毕业一、二年本事不大的医生,其余的都到农村去。四清到××年就扫尾基本结束了。可是四清结束,农村的医疗、卫生工作是还没有结束的!把医疗卫生的重点放到农村去嘛!


