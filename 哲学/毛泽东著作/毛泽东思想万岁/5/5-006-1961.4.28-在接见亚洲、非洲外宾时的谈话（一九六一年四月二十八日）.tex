\section[在接见亚洲、非洲外宾时的谈话(一九六一年四月二十八日)]{在接见亚洲、非洲外宾时的谈话}
\datesubtitle{(一九六一年四月二十八日)}

毛主席对非洲、阿拉伯各国人民反对帝国主义的斗争表示深切的同情和支持。毛主席指出当前的国际形势对非洲、亚洲、拉丁美洲人民的反帝斗争是非常有利的。毛主席说,对帝国主义进行斗争当中,采取正确的路线,依靠工人、农民,团结广大的革命知识分子,小资产阶级和反对帝国主义的民族资产阶级以及一切爱国反帝力量,紧紧地联系群众,就有可能取得胜利。毛主席指出,革命政党和力量,在开始时都是处于少数地位的,但最有前途的就是他们。

毛主席严厉地谴责美帝国主义对古巴的侵略,并指出,美国帝国主义迫不及待地进攻古巴,再一次在全世界面前揭露了它的真面目,说明了肯尼迪政府只能比艾森豪威尔政府更坏些,而不是更好些。美国帝国主义利用联合国作为工具侵略刚果和杀害芦蒙巴的罪行,将非洲人民对美国帝国主义的认识进一步提高了。

毛主席表示,最近万隆亚非团结会议上表达的亚非人民同拉丁美洲人民加强团结的愿望,是好的。对世界人民反对帝国主义斗争的共同事业是有益的。

毛主席说,中国人民把亚洲、非洲、拉丁美洲人民的反帝国主义斗争的胜利看作是自己的胜利,并对他们的一切反帝国主义;反殖民主义的斗争给以热烈的同情和支持。

\kaoyouriqi{(《人民日报》一九六一年四月二十九日)}


