\section[湘赣边界各县党第二次代表大会决议案(一九二八年十月二十五日于宁岗步云山)]{湘赣边界各县党第二次代表大会决议案(一九二八年十月二十五日于宁岗步云山)}
\datesubtitle{(一九二八年十月二十五日)}


一、政治问题及边界党的任务

(一)国内国际政治状况(与毛选一第一部分同,略)49一一50页

(二)中国此时各地苏维埃政权发生及存在的原因。

(与毛选一50—52的第二部分同,略)

(三)边界割据及八月失败(与毛选一53—54页第三部分同,略)

(四)边界割据在湘鄂赣三省工农暴动中的地位。(与54页第四部分同)

(五)经济问题(与55页第五部分同,略)

(六)军事根据地问题(与56—57页第六部分同,略)

二、边界各县党的改选与建议

边界特委是最近四个月才产生的,在此短短的期间内,虽然曾领导工农红军作过了“土地革命”“建立苏维埃”及给旧社会以伟大的破坏等工作。但因为边界是个农村经济的环境,加之党的历史很短,独立斗争很少,(因为有红军可靠)故机会主义的遗毒农民党的倾向,在边界各级党部中表现出特别可怕的色彩。指出过去一切党内的错误,洗刷党内机会主义的遗毒,改造各级党部,使之走上真正无产阶级领导的道路,是今后各级党部重要的责任。

攸县的党到现在还未建立。其他各县党的影响亦只及各该县之大部或一部,并未建立坚实的中心区域及普遍发展到四处的广大群众中击。故建立攸县的党和各县的坚实中心区域,及普遍建立各县党的基础,同样是边界党部今后的重要责任。

(一)过去党的错误

1.过去各县党中机会主义遗毒太深,依靠军队而不领导群众做独立斗争,这是一个很大的错误。

2.过去各县的党,很有农民党的色彩,有走上非无产阶级领导的倾向,永新的党要公开脱离特委,成立“独立国”。其他各县,如宁、酃、莲等县亦不注意向特委作报告,以与特委发生关系。这都是组织上很严重的错误。

3.过去党的机关,在上半期是个人专政,书记独裁,完全没有集体领导及民主的精神。如特委只有毛泽东一人,永新县委只有刘珍一人,以致客观上群众发生只认识个人不执识党的误解。这完全不是布尔什维克的党。紧急会议中虽然很严重的指出了这种错误,特委经常有三个常委驻会指示工作,但各科及秘书等机关,因人材缺乏之故,仍然还是无法建立,加之特委工作人员病者极多。政治时生变化,各县委与特委的关系不能密切,故特委本身反对各级总部的指示,仍然还是没有尽量执行紧急会议之决定。

4.指导机关,多数小资产阶级知识分子领导,对于提拔工农分子参加指导机关,缺乏注意。

5.过去党的组织扩大,完全只注意数量的发展,没有注意质量上的加强。党与阶级没有弄清楚,而只是拉夫式的吸收办法。这样将使党的组织破底,其结果必变成不能斗争的党。

6.过去党没有注意基本组织一一支部。

7.过去党的工作方式不对。党的一切问题应该集权于常委会,组织科宣传科……等不过是常委会的技术机关。但是边界的党,不但没有集权到常委会,甚至连组织科都没有,而仅是书记一人的独裁,紧急会议后,特委本身及永新县委比较好点(由常委会决定各项工作),但特委以下之各级党部,仍然还是犯同样的毛病,一点都没有改正。

8.过去边界各县的党太没有注意秘密工作,以致养成极大部分党员不懂秘密工作。在夺取政权时,就全部公开出来;在失掉政权时,就去“打埋伏”。

9.过去工作,上级与下级隔开,下级对上级也没有很好的巡视和指导。党只偏重机关工作,犯了与群众隔离的错误。

10.过去党太少注意城市工作和工人运动。

11.过去特委只注意军事范围的布置,如宁岗、永新两县工作。没有顾及全部,以致成为军队附属品。如军队打到某县,就注意某县的工作,军队没有到某县,某县的工作就不注意了。

12.过去党太不注意团的工作,甚至有取消团的倾向。

(二)今后党的改造与建设

l。党必须彻底改造,从支部改造起,肃清组织上和政策上机会主义的领导。

2.特委县委都须有四个以上的巡视员,经常指导下级工作,帮助各级党部改造。

3.尽量提拔工人同志到指导机关,各级执委会、常委会都须有过半数的工农同志参加,提拔工农分子,特别要注意教育的意义。

4.各级党的机关,必须组织健全,反对个人领导,一切权力集中常委会,各科都是技术机关。

5.在党的改造当中,应完全站在无产阶级的观点上,极力注意讨论和执行党的新政策,坚决与过去党的小资产阶级、自由独立、浪漫的分子,严密防止“独立国’的倾向。

6.党应该扩大民主化到最高限度,一切政策都要党员热烈讨论,深切了解,使党员群众能根据政策定出工作计划。各级党部委员及书记应尽量用选举法办产生。

7.党员成分必须是先进的觉悟的忠实的勇敢的贫苦工农分子,对于小资产阶级、知识分子、富农、须严格限制。

8.党的发展,特别注意质量,在介绍党员当中,介绍者应对被介绍者做许多宣传和考查工作。凡介绍一个新同志应在一个支部会议上通过,经过区委批准,反对拉夫式的吸收党员,必须使每个党员成为无产阶级的战斗员,党的组织不必求其普遍,应特别注意造成坚实中心区域党的基础。

9.党要注意党的基本组织……支部,实现“一切工作归支部”的口号。同时要特别注意城市支部工作,并提拔很好的工人同志尽量当农村支部书记及委员,增加工人领导力量,严格防止农民党的倾向,在乡村中的党员,须选出先进分子,加以特别训练,以培养成党的中坚。

10.党的组织应绝对秘密,各级党部每个党员,应当极力注意秘密工作,反对依靠军事政治势力去组织党,应在敌人范围内秘密组织党,反对逃跑和“打埋伏”。

11.特委应极力注意使本身和各县委健全,攸县县委应马上建设,对于边界各县党的工作应有整个的布置。

12.“铁的纪律”为布尔什维克党的主要精神,只有如此,才能仰止党走向非无产阶级的道路,消灭机会主义分子,洗刷不斗争的腐化分子,只有如此,才能集中革命先进分子的力量团集在党的周围,使党壁堡森严、步伐整齐的成为强健的斗争组织。只有如此,才能增加无产阶级的领导力量。所以严格的执行纪律为改造与建设党中央的重要工作。

(三)各县工作问题

各县工作详细计划由特委负责讨论。

(四)农村斗争问题

1.过去农村斗争,并没有坚决执行“土地革命”。所谓分田,完全不是适应贫苦雇农的彻底要求,而只是站在富农中农贫农的调和观点上去平均分配的,这是过去一种重大的错误。

2.过去我们在“土地革命”过程中,并没有施行严厉的赤色恐怖,杀戮地主豪绅及其走狗(上花茶陵比较好点)。

3.过去在乡村赤色政权之下,太忽略农村中富农、中农、贫农之间的阶级斗争,以致在白色恐怖之下,贫农无团结无力量,富农反水,中农动摇。

4.我们今后农村斗争整个的策略是:团结贫农,抓住中农,深入土地革命,厉行赤色恐怖,毫不顾惜的杀戮地主豪绅及其走狗,用赤色恐怖手段威胁富农,使不敢帮助地主阶级。

5.根据这个策略,我们应即组织:(1)雇农工会(贫苦的佃农加入此组织),以此团结雇农,加强雇农的力量,以为乡村中的主干。(2)赤杀队或暴动队,在白色恐怖之下,以最勇敢的工农分子组织之。赤杀队以五人或七人为一队,实行黑夜游击,造成乡村中的赤色恐怖,在夺取政权时,赤杀队可改变为赤卫队。(3)在工农群众中选举勇敢分子组织暴动队,发展农村暴动,夺取乡村政权。

(五)工人运动问题

1.工人是各种劳动群众之先锋,是各种劳动群众的领导者,过去没有注意工人运动,工人领导更是说不上,以致造成农民党的倾向,这是党的一种重大危机。

2.农村手工业工人及城市工人,我们党须用大的力最去组织工会,领导工人由零碎经济斗争进而武装暴动,纠正过去忽视工人运动的错误。

3.各级党部各级苏维埃,应极力提拔工人,使之能站在领导地位,领导斗争。

(六)兵士运动问题

中国共产党所主张的“武装暴动,夺取政权”的政策,在执行这个政策中必须将工农兵三种力量很好配合起来,然后暴动才有胜利的可能,因为中国的“民权革命”尚未实现,故军阀豪绅买办阶级得以运用封建关系愚弄工农而雇用之,以为保护他们的工具,故现在的士兵有大多数尚安心在敌对阶级指挥之下过饥寒交迫(敌军有几年未发过饷的)的生活。但在国民党未叛变之前,国民革命军相当的都受了一点“阶级斗争”的宣传。至于事变以后,投入军营的(农灾协或工会的办事人)当然更有觉悟。这些觉悟分子在反动军官高压防范的严密下,而又苦于找不着领导者,故不敢轻易在反动军营中作宣传组织工作,实则他们的革命情绪,反水志愿是很丰富的,这可以证明在客观上的士兵运动的可能性,实在已经具备了。而且此运动的成功,在湖南、广东等省,已有事实可据。现在反动统治的屏障,完全是靠几百万尚未觉悟的国民革命军,我们士兵运动做的好,则反动统治很迅速的便要倒台,反之,我们不注意士兵运动而只专门做农民工作,则中国的革命,永久也不会成功的。

边界各级党部向来少注意士兵工作而只顾工民运动(有些党部还仅只有农民工作)。须知我们过去许多失败的经验,主要原因是没有士兵运动参加暴动,(如去年广东的年关暴动,湖南屡次的暴动失败……。)现在如再不注意,则将来的失败便可预言。很明显的现在单靠边界主观的方量,妄想打出一个天下,或造成更大的割据是不可能的,所以各级党部必须努力进行士兵运动。莫忘记“武装暴动,夺取政权”的政策,是要将工农兵三种势力配合起来才有可能的。

1.士兵运动此时与工农运动同样重要,各县应有计划有组织的选派大批工农同志去反动军队内当兵、当夫、当伙头……在敌兵内部起作用。尤其是永新、茶陵、辽川等大部敌兵汇集的县分,要特别注意此项工作。

2.加紧对敌兵的宣传工作。 
3.派人到敌人内部去组织党,不要组织士兵委员会,以免组织复杂而易于惹起敌军军官之发觉。

d。利用灰色同志及农村妇女,向敌兵作口头宣传和煽动工作。

5.造谣、恐怖煽动,以动摇敌军人心,由动摇而瓦解。

(七)宣传问题

1.过去边界各县的党,太没有注意宣传工作,妄以为只要几支枪就可以打出一个天下,不知道共产党是要左手拿宣传单,右手拿枪弹,才可以打倒敌人的。同时在各项工作中(如组织苏维埃、暴动队、分田、组织党等工作),完全不宣传其方法和意义,只是利用军事政治势力去逼着做“不做就杀”。这是一种最严重的错误。

2.特委县委宣传科,应设法使之健全,每周标语宣传大纲,都须按期发出,每日壁报亦应发给各级党部缮写张贴。各游击队出发游击,都须有很好的宣传(群众大会一一化装讲演、宣传队、个别宣传)。

3.以后下级党部对上级党部工作报告,须有宣传工作的报告。上级巡视和检查下级工作时,亦须注意宣传工作的检阅。特委县委每周应有宣传大纲发给各下级党部。

4.目前我们在政治上应极力分析军阀内部的冲突,注意反对军阀混战的宣传工作,同时。应极力宣传工农及共产党的力量之伟大,说明军阀混战其结果必被工农暴动所消灭。

5.我们目前对一般工农群众的宣传,须极力揭破军阀及豪绅欺骗工农政策,多发表本党的主张。

6.苏维埃,土地革命,共产主义,红军,暴动队,都须制定专门宣传纲要,加紧宣传,深入到群众脑海中。

7.此时我们应对同志和群众,详细分析统治阶级与政治上经济上之矛盾与冲突,极力宣传工人农民本身力量及各地暴动的势力,打破灰心没有希望的失败观念,同时要打破同志及群众专门依靠军队的等待观点(自然我们不否认军队发动暴动与帮助工农暴动的力量)。

(八)训练问题

1.过去各地党之所以没有力量,就是因为党员没有训练,甚至入党式都没有过,现在每个党员都须加以党的基本理论的训练。

2.特委组织教宣委员会,制定训练材料,计划每周训练工作。

3.特委要办经常训练班,各县亦应尽量多办短期训练班,造成干部人材。

4.各级党部的会议及实际工作当中,应极力提拔工农分子,训练工农分子及干部人材。

5.目前基本训练工作,应竭力铲除一般同志的机会主义思想和封建小资产阶级思想,确定无产阶级革命的人生观。

6.增高同志的文化程度政治教育,同时要做识字运动,以提高工农同志的“写”“看”能力。

(九)苏维埃问题

1.过去苏维埃政府差不多完全是农民协会的变形,所以它的工作,是秘书长和主席包办,甚至有些政府还是富农当权,或为知事衙门。这些所谓苏维埃政府,应一律改组。

2.由特委制定苏维埃组织法,各县各区各乡苏维埃,应一律照组织法组织之。

3.苏维埃必须以工人贫农革命兵士为主要力量。反对富农秘书长把持,实现一切政权归苏维埃。

4.湘赣边界政府重新彻底改造。

5.各级党部与各级苏维埃的关系要弄清楚,免除党即政府的弊病。关于党与政府不同,特委须发一通告,各级党部要作一普遍宣传。

(十)土地问题。接受中央对土地问题的通告,交特委讨论,作最后决定。

(十一)Cy问题

1.青年团是党在工农群众中的政治组织,过去边界各级团部多不明了青年团的政治任务,团在边界各县仅只作了些扩大影响的文化运动。事实上成了党的附属机关,因此造成边界各级党更严重的错误一一主张取消团。

2.目前边界各县应当是建设团的工作,但是团的本身力量薄弱,很难单独担负起这个责任,各级党部必须划出一部分力量,经常注意团的工作,帮助团在边界各县普遍的建设团的支部,扩大团的组织,健全团的指导机关。

3.团的经费应该独立,以便团在工作上能够措置如意,避免团在经费上依赖党的观念。

4.注意团的工作,是各级党部应有的责任。今后各级党部向上级报告中,必须有“团工作”的一项。上级巡视员在检查各级工作中,亦须留心团的工作。

5.各级党部、团部多不明了党与团的关系,以致发生各干各的倾向。

