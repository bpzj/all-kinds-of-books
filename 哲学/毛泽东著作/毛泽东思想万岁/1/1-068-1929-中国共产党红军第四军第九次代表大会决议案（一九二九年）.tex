\section[中国共产党红军第四军第九次代表大会决议案(一九二九年)]{中国共产党红军第四军第九次代表大会决议案}
\datesubtitle{(一九二九年)}


一、关于纠正党内的错误思想(见毛选第一卷87~99页)

二、党的组织问题红军的组织问题,现在到了非常严重之时期,特别是党员质量之差和组织之松懈,影响到红军的领导与政策之执行非常之大。大会对于这个问题特作详细的分析与决定。同志们应站在大会的精神上,努力去改造党的组织,务使党的组织确实能担负党的政治任务,才算得到成功。

(一)党的组织路线

甲、党员发展的路线,以战斗兵为主要对象,同时,对非战斗兵如夫子勤务兵等,亦不应忽视。

乙、每连建设一个支部,每班建设一个小组,这是军中党的组织的重要原则之一,在党员数量过少的部队,事实上不能每班建立一个小组时,则暂以排为单位建设小组,而把组员有计划的分配到各班,但须明白这是过渡方法。

丙、红军原有的小组编制法,即干部分子与一般分子,知识分子与劳动分子的混合编制法,是很对的。但还缺乏有计划的将各种工作不同能力不同分子很好的混合编制起来,以后此点要多加注意。至于单纯地把干部分子编成小组,那是要不得的。

(二)党的组织松懈问题

甲、四军党的组织现状:

(一)党员加入太随便,许多不够党员资格的也拉了进来,特别是官长,不要任何条件,一概拉进。因此党的质量就弄得很差了。

(二)各级党部的工作做到了解决问题,但完全忘却了教育同志一个任务。关于训练性质的会议,如活动分子大会,书宣组联席会,委组联席会,支部大会,纵队或支队党员大会等非常开得少。

(三)一股的纪律松懈,特别在负担重要工作没有人代替他的情况下,对他所犯错误,往往敷衍下去,不执行纪律。对一人如此,对他人亦不得不如此,因此把纪律一般的放松了。

(四)所有官长既都成了党员,结果所有军事政治机关的工作人员,便很少注意他们的社会职业的工作了,都以为社会职业的工作就是党的工作,两者分不开来。如各级军事工作会议,各级政治工作会议几乎完全没有,一切都以为党决定了就完事了,党员在社会职业中起核心作用一件事,简直不谈起了。

(五)上下级关系不密切。下级的报告,上级少有批答或指示。下级会议上级少有派人出席。这固然是因为上级机关组织不健全,但上级机关工作态度不积极,也是对下级指导缺乏或指导不详尽的原因之一。特别是对于指导实际工作,如一个部队出发游击的工作指导,一般的缺乏详尽。还有某几个部队,几乎连粗略的指导都没有。

(六)支部大会与小组会,有许多没有按时开。

乙、纠正的路线:

(一)旧的基础厉行洗除。如政治观念错误,吃食鸦片,发洋财及赌博等,屡戒不改的,不论干部及非干部,一律清洗出党。

(二)以后新分子入党条件:

1.政治观点没有错误(包括阶级觉悟);

2、忠实;

3、有牺牲精神,能积极工作;

4、没有发洋财的观念;

5、不吃鸦片,不赌博。

以上五个条件完备的人,才能够介绍他入党。介绍人事先要审查被介绍人是否确实具备上列条件。经过必须的介绍手续。介绍入党后,要详细告诉新党员以支部生活(包括秘密工作),及党员应遵守的要点。介绍人对被介绍人应负相当责任。支委要派人对将入党的人谈话,考查是否具备入党条件。

(3)各级党部不单是解决问题和指导实际工作的,他还有教育同志的重大任务,各种训练同志的会议,及其他训练办法如训练班、讨论会等,都要有计划的举行起来。

(4)严格的执行纪律,废止对纪律的敷衍现象。

(5)分清党员社会职业与党的工作的性质,每个党员(除开党内负了重要或专门任务的人是为革命职业家外),必须担负一件社会职业,同时即在他的社会职业中担负党给他的工作。

(6)各级党部的工作态度,应该比较从前更积极起来,下级对上级要有详尽的报告,上级对于这些报告要有详尽的讨论和答复,并尽可能派人出席下级会议,不能借口工作人员少工作能力薄弱和工作时间不够,来掩盖自己的不积极,而把这些工作疏忽起来。

(7)支委会及支委以上各级党部,应该有计划的每月规定支部大会及小组会讨论的材料,并规定会期,严密的督促开会。

(三)怎样使党员到会有兴趣

甲、党员到会少兴趣的原因:

(1)不明白会议的意义。支部会议的重要意义,第一是解决问题,内部的问题都要在会议上集中讨论去解决它,若不到会或到会不积极发表意见,就是他不了解会议的政治意义,就是他对斗争没有兴趣,凡是对斗争积极的人,他一定是热心到会,热心发言的。第二是教育同志,会议不仅解决了问题,而且在解决问题的过程当中,要考察问题的环境,要参考上级的指示,这样就发动了同志们的心思才力,由会议的政治化与实际化,同志们每个人的头脑也都政治化实际化了。每个同志都政治化实际化了,党的战斗力就强大起来。这就是会议的教育意义。红军党员因为不明这些意义,所以成为不爱到会或到会少兴趣的第一个原因。

(2)决议案决议了不执行,或对上级请求事项很久得不做答复,因此减少讨论兴趣。

(3)负责人事前没有好的准备。不准备议事日程,对问题的内容及环境不明了,问题应怎么解决也没有准备一点意见。

(4)主席轻易停止党员发言,发言偶出题外,便马上禁止他,他便挫兴不做声了。如发言有错误,除停止外还讥笑他。

(5)封建式的会场秩序,死板无活气,到会如坐狱,

乙,纠正的方法:

第一,会议要政治化实际化。第二,要把会议的重大意义时常对同志提醒,尤其是对新党员及工作不积极的党员。第三,决议不要轻意,一成决议,就要坚决执行。第四,上级机关要勤快答复下级的问题,不要拖延太长,失了热气。第五,负责人事先要准备议事日程,议事日程要具体化;对问题的内容和环境先要调查清楚,对于怎样解决先要想一想。第六,主席指导会议要采用很好的技术,要引导群众的讨论潮流奔赴到某一问题,但有重要意义的超出题外的发展,不仅不要大杀风景地去制止他,而且要珍重地捉住这一发展的地点,介绍给大众,成立新的议题。这样会议才有兴趣,问题才能得到真正的解决,同时会议也才能真正实现了教育的意义。第七,废止封建的会场秩序。共产党的会场是要反映无产阶级之积极活动的爽快的精神,把这些做成秩序。

(四)红军党内青年组织及其工作

(1)部队中青年利益与成年利益不能划分,团没有特殊的工作对象。又党的小组以班为单位建设方有利斗争。因此党的支部里头无设立团的小组的必要。

(2)党员中青年部分因其有比成年不同的情绪,除一般的接受党的训练外,还有特别受一种青年教育之必要,又因争取青年工农群众是党的重要任务之一,须有专门组织去担任这种工作。因此,应在支部中划出年在二十岁以下的青年党员(有特别情形如担任党的重要工作的除外)成立青年工作会议。这种会议,除以大队为单位经常有计划的召集开会外,支队及纵队亦各应酌定时间,召集开会。

(3)为计划对青年党员的教育,计划争取青年工农群众的方法,并指导青年工作会议,前委及纵委里头各设立五人之青年工作委员会,支队委及支委则各设一个青年委员,在各级党部的指导之下从事工作。

(五)政治委员与党内工作关系

大队支队两级党部的书记,以不兼政治委员为原则。但在工作人员缺乏的部队,仍可暂时兼充。没有担任党的书记的政治委员,上级党部应察看情形,在条件适合的环境下,得委为党的特派员,有指导该级党的工作的任务。

(六)直属部最高党部问题

军及各级队的直属队,均组织直属队委为最高党部,委员人数五人至七人。

(七)士兵会党团问题

大队士兵会不设党团,工作由支委指导。纵队士兵会要设党团,此党团受纵委指导。

三、党内教育问题

(一)意义

红军党内最迫切的问题,要算是教育的问题。为了红军的健全与扩大,为了斗争之任务之能够负荷,都要从党内教育做起,不提高党内政治水平,不肃清党内各种偏向,便决然不能健全并扩大红军,更不能担负重大的斗争任务。因此,有计划的进行党内教育,纠正过去之无计划的听其自然的状态,是党的重要任务之一。大会规定用下列的材料和方法去教育党员,党的指导机关要有更详细的讨论,去执行这一任务。

(二)材料

(1)政治分析;

(2)上级指导机关的通告讨论;

(3)组织常识;

(4)红军党内八个错误思想的纠正;

(5)反机会主义及托洛茨基主义反对派问题的讨论;

(6)群众工作的策略和技术;

(7)游击区域社会经济的调查研究;

(8)马克思列宁主义的研究;

(9)社会经济科学的研究;

(10)革命的目前阶段和它的前途问题。

以上十项除一部分(如社会经济科学的研究)事实上限于适用在干部分子外,其余都适用于一般党员。


(三)方法

(1)党报;

(2)政治简报; 
(3)编辑各种教育同志的小册子;

(4)训练班;

(5)有组织的分配看书;

(6)对不识字党员读书报;

(7)个别谈话;

(8)批评;

(9)小组会;

(10)支部大会;

(11)支部委组联席会;

(12)纵队为单位组长以上活动分子大会;

(13)全军文书以上活动分子大会;

(14)纵队为单位党员大会;

(15)纵队为单位各级书、宣、组联席会议,

(16)全军支队以上书、宣、组联席会议;

(17)政治讨论会;

(18)适当地分配党员参加实际工作。

四、红军宣传工作问题

(一)红军宣传工作的意义

红军宣传工作的任务,就是扩大政治影响争取广大群众。由这个宣传任务之实现,才可以达到组织群众,武装群众,建立群众,消灭反动势力,促进革命高潮等红军的总任务。所以红军的宣传工作,是红军第一个重大工作。若忽视了这个工作,就是放弃了红军的主要任务,就等于帮助统治阶级剥削红军的势力。

(二)红军宣传工作的现状

甲、宣传内容的缺点:

(1)没有发布具体的政纲(从前发布的政纲如四言布告等,均不具体);

(2)忽略群众日常斗争的宣传与鼓动;

(3)忽略城市贫民之取得;

(4)忽略对妇女群众的宣传;

(5)对青年群众的宣传不充分;

(6)对流氓无产阶级的宣传不充分;

(7)极少破坏地主阶级武装组织(民团靖卫团等)的宣传;

(8)宣传没有时间性地方性。

乙、宣传技术的缺点:

(1)宣传队不健全,

1、宣传员由每大队五个缩小到三个,有些只有一两个,有些只有一个,有少数队连一个都没有了。

2、宣传员成分太差,俘虏兵也有,伙夫马夫也有,吃鸦片的也有,有逃跑嫌疑便把他解除武装塞进宣传队去的也有,当司书当不成器便送入宣传队的也有,因残废了别的机关不用塞进宣传队去的也有,现在的宣传队简直成了收容所,完全不能执行它的任务了。

3、差不多官兵一致地排斥宣传队(同时也是因为宣传队成分太差,工作成绩少,引起一般人对它的不满)。“闲杂人”“卖假膏药的”,就是一般人送给宣传队员们的称号。

4、宣传队没有够用的宣传费。

5、对宣传员的训练没有计划,同时对他们工作的督促也不好,因此宣传队的工作简直随随便便,做一点不做一点都没有人理它。

(2)传单、布告、宣言等,陈旧不新鲜,同时散发和邮寄都不得法。

(3)壁报出的很少,政治简报内容太简略,又出得少,字又太小看不清。

(4)革命歌谣简直没有。

(5)画报只出了几张。

(6)化装宣传完全没有。

(7)含有士兵娱乐和接近工农群众两个意义的俱乐部,没有办起来。

(8)口头宣传,又少又糟。

(9)红军纪律是一种对群众的实际宣传,现在的纪律比较松懈了,因此给了群众以不好的影响。

(10)上门板、捆禾草、谈话和气、买卖公平、借东西照还、赔尝损失,这些都是红军宣传工作的一种,现在也做得不充分。

(11)群众大会很少开,又开得不好。

(12)对白军士兵的宣传方法不好。

(三)纠正的路线

甲、宣传内容方面:

(1)发布一个具体的政纲,名曰红军政纲。

(2)宣传要切合群众的斗争情绪,除一般的发布暴动口号外,还要有适合群众斗争情绪尚低地方的日常生活口号,以发动日常斗争,去联系着那些暴动口号。

(3)城市贫农(中小商人与学生)是民权革命过程中的一个相当的力量,忽视了这个力量之争取,就无异把这个力量还给豪绅资产阶级。以后对城市中小商人及学生群众,要有深入的宣传工作去取得他们。

(4)妇女占人口的半数,劳动妇女在经济上的地位利她们特别受压迫的状况,不但证明对革命的迫切要求,而且是决定革命胜败的一个力量,以后对妇女要有切实的口号,做普遍的宣传。

(5)劳动青年群众占人口百分之三十以上,在斗争中他们又是最勇敢最坚决的。所以做好对青年群众的宣传,是整个宣传任务中的一个重要任务。

(6)中国广大的游民群众,若站在革命阶级方面,就成了革命的工具,若站在反动阶级方面,就成了反革命的工具,因此从反动阶级影响之下夺取游民群众,是党的宣传任务之一。执行宣传工作时,须注意各部分游民生活与性质之不同,分别的对他们宣传。

(7)地主阶级武装组织之破坏及其群众之争取,是农村中土地革命胜利条件之一,以后关于对民团靖卫团等团丁群众的宣传工作,特别要注意。

(8)到一个地方,要有适合那个地方的宣传口号和鼓动口号。又有依照不同的时间(如秋收与年关,蒋桂战争时期与汪蒋战争时期)制出不同的宣传口号和鼓动口号。

乙、宣传技术方面,

(1)宣传队问题。

1、意义:

红军的宣传队,是红军宣传工作重要的工具,宣传队若不弄好,红军的宣传任务就荒废了一个大的部分,因此关于宣传队的整理训练问题,是目前党要加紧努力的工作之一。这个工作的第一步,就是要从理论上纠正官兵中一般对宣传工作及宣传队轻视的观点,“闲杂人”“卖假膏药的”等等奇怪的称呼,应该从此取消掉。

2、组织。

子、以支队为单位,军及纵队直属队均各成一单位,每单位组织一个中队,队长队副各一人,宣传员十六人,挑夫一人(挑宣传品),公差二人,每个中队的宣传员分为若干分队(按照大队或其他部队与机关的数目,定出分队的多少),每分队有分队长一人,宣传员三人。

丑、各支队宣传队,受支队政治委员指挥。各大队分开游击时,每大队应派去一个宣传分队随同工作,受大队政治委员指挥。直属队宣传队,受政治部宣传科长指挥,全纵队各宣传队,受纵队政治部宣传科指挥,全军宣传队,受军政治部宣传科指挥。

寅、宣传队用费,由政治部发给,须使之够用。

卯、改造宣传员成分的方法,除请地方政府选进步分子参加红军宣传队之外,从各部队士兵中挑选优秀分子(尽可能不调班长)为宣传员。政治部应经常地作出训练宣传队的计划,规定订”综的材料、方法、时间、教授人等,积极地改进宣传员的质量。

(2)传单、布告、宣言等宣传文件,旧的应加以审查,新的应从速起草。

宣传品分布的适当有效,应是宣传队技术问题的重要一项,邮寄宣传品从邮件中夹带宣传品,或在邮件上印上宣传鼓动口号,政治工作机关应当注意去做,且要做得好。

(3)壁报为对群众宣传的重要方法之一,军及纵队各为一单位办一壁报,由政治部宣传科负责,名字均叫做“时事简报”。内容是:(一)国际国内政治消息:(二)游击地区群众斗争情形;(三)红军工作情形。每星期至少出一张,一概用大张纸写,不用油印。每次尽量多写几张。政治简报的编印,应当注意下列各项:l、要快;2、内容要丰富一点;3、字要稍大一点,要清楚点。

(4)各政治部负责,征集并编制表现各种群众情绪的革命歌谣,军政治部编制委员会负督促及调查之责。

(5)军政治部宣传科的艺术股,应该充实起来,出版石印的或油印的画报,为了充实军艺股,应该把全军绘画人材集中工作。

(6)化装宣传是一种最具体最有效的宣传方法,各支队各直属队的宣传队均设化装宣传股。

(7)以大队为单位在士兵会内建设俱乐部。

(8)宣传队中设口头宣传股及文字宣传股。

(9)严格地执行三条纪律。

(10)政治部及宣代队,要有计划有组织的召集各种群众会议,要预先规定开会秩序、演说人、演说题目及时间。

(11)对白军士兵及下级官长的宣传非常之重要,以后要注意下列的方法:

1、宣传文字要简单,使他们顷刻间能看完;要精警,使他们一看起一个印象。

2、除有计划的在敌人经过道路两旁多写切实某现实部队的标语之外,还要将宣传品存储于沿途党部及群众机关中,俟敌军经过时,巧妙地传达给他们。

3、从俘虏官兵中及邮件检查中,调查敌方官兵的姓名及所属部队番号,邮寄宣传品去,或写信给他们。

4、优待敌方俘虏兵,是对敌宣传的极有效的方法。优待俘虏兵的方法:第一是不搜他们身上的钱和一切物件,过去红军搜检俘虏兵财物的行为,要坚决的废掉。第二是要以极大的热情欢迎俘虏兵,仗他们感到精神上的欢乐,反对给俘虏兵以任何言语上的或行动上的侮辱。给俘虏兵以和老兵一样的物资上的平等待遇。第四是不愿留的,在经过宣传之后,发给路费,放他们回去,使他们在白军中散布红军的影响,反对只贪兵多,把不愿留的分子勉强割下来。以上各项,对于俘虏过来的官长,除特殊情况外,完全适用。

5、医治敌方伤兵,亦是对敌军宣传的极有效方法。对于敌方伤兵的医治和发钱,要完全和红军伤兵一样。并且要利用一切可能的条件,把上好了药发给了钱的伤兵,送返敌军。对待敌人受伤官长亦然。

五、士兵政治训练问题

(一)材料问题下列各项,应很艺术的编制课本,作对士兵训练的材料:

1、目前政治分析及红军之任务与计划;

2、土地革命各方面;

3、武装组织及其战术;

4、三条纪律建设的理由;

5、早晚点名口号;

6、识字运动:

7、怎样做群众工作;

8、红军标语之逐个解释;

9、各种偏向之纠正;、

10、苏俄红军;

11、革命的目前阶段和它的前途;

12、红军白军比较;

13、共产党国民党比较;

14、革命故事;

15、社会进化故事;

16、卫生;

17,游击区域的地理及政治经济常识;

18、革命歌;

19、图报

(二)方法

甲、上政治课:

(一)分普通、特别、干部三班。普通班分两种形式:一个支队在一起时,以支队为单位上课,教授以支队政治委员为主任,大队政治委员分担各课,除战斗兵须到课外,传令兵、勤务兵`长夫、马夫、伙夫,均须到课。普通班之目的,在使一般士兵得到初步政治常识。

(二)特别班以支队为单位,从各大队士兵中挑选稍识文字及略有政治常识者五十五编成。支队政治委员为主任教授,大队政治委员分担各课。特别班之目的,在造成较普通班高一组的政治常识人材,以备将来升当下级干部之用。

(三)干部班以纵队为单位,军直属队另成一单位,以大队长、大队副、中队长、中队副、各级军佐,及其他指定人员编成。目的在提高现有下级干部的政治水平线。使能领导群众,以备将来能充当中级干部。由纵队政治委员政治部主任纵队司令官及其他有适当能力的人担任教授。

(四)以支队为单位组织政治训练委员会,以支委各政治委员会并军事长官中能任政治训练者组织之,以大队政治委员为主任。其任务为讨论在一个支队内关于执行士兵政治训练的种种问题。

(五)军及各纵队直属部队的政治训练,由军及各纵队政治宣传科负责组织,政治训练委员会执行之。

(六)教授法,

1、启发式(废止注入式);

2、由近及远,

3、由浅入深,

4、说话通俗化(新名词要解俗),

5、说话要明白;

6、说话要有趣味;

7、以姿势助说话;

8、后次复习前次的概念;

9、要提纲

10、干部班要用讨论式。

乙、早晚点名说话;

(一)说话时间每次最多不过半小时;

(二)材料:

1、报告政治消息;

2、批评日常生活;

3、解释每周政治口号。

丙、集合讲话:

(一)支队每星期一次,纵队每半月一次,全军不定期。

(二)每次讲话,须要有计划的由政治工作机关(支队由政治委员商同军事工作机关)规定讲话内容,指定讲话人,并分配讲话时间。

(三)每次讲话时间,勤务的以外,不准不到。

(四)每次讲话内容及对于群众的影响,下级政治机关须报告上级政治机关。

丁、个别谈话

(一)对下列各种人必须和他做个别谈话;

1、有偏向的;

2、受了处罚的;

3、伤兵;

4、病兵;

5、新兵;

6、俘虏兵;

7、对工作不安的;

8、思想动摇的。

(二)谈话前,须调查谈话对象的心理及环境。

(三)谈话时,须站在同志的地位,用诚恳的态度和他说话。

(四)谈话后,须记录谈话的要点及其影响。

戊、游戏:

(一)以大队为单位,充实士兵会娱乐部的工作,做下列各种游艺:

1、捉迷藏等;

2、打足球;

3、音乐;

4,武术;

5、花鼓调;

6、旧剧。

(二)于每个宣传队下,设化装宣传团。

(三)游艺设备费用公家支给(大队的由纵队政治部批准)。

己、改良待遇:

(一)坚决废止肉刑;

(二)废止辱骂;

(三)优待伤病兵;

(四)恢复每月发草鞋大洋四角的制度。

庚、怎样做新兵及俘虏兵的特别教育:

(一)把红军生活惯如①官兵生活平等(官兵之间只有职务的分别,没有阶级的分别,官长不是剥削阶级,士兵不是被剥削阶级);②三大纪律及其理由!⑧士兵会的意义和作用;④红军中的经济制度(经济的来源、管理经济的组织,经济公开主义及士兵审查制度);⑤经济委员会管理大队伙食和分伙食尾数;⑥废止肉刑辱骂;⑦优待俘虏等项,讲给新兵们俘虏们听。

(二)讲述红军斗争略史:

(三)红军的宗旨:1、红军与白军所以不同,此点对俘虏兵要详细讲;2、红军与土匪所以不同;3、红军三大任务。

(四)讲述红军组织系统

(五)普通政治常识。如l、国民党与共产党;2、英日美三大帝国主义侵略中国;3、各派军阀受帝国主义指挥,到处混战;4、分田;5、苏维埃;6、赤卫队等。

六、青年士兵的特种教育

(一)各纵队政治委员负责编制青年识字课本(以商务馆小学教材,平民千字课,龙岩文化社教本作参考)。

(二)每个纵队内设青年士兵学校一所,分为三班至四班,每个支队一班,直属队一班,每班学生不得超过二十五人。以政治部主任为校长,以宣传科长为教务主任。每班设一主任教员,每班以教授九十小时为一学期。

(三)由公家出钱备置纸笔墨等用具发给学生。

七、废止肉刑问题

(一)红军中用的肉刑的效果

各部队中凡打人最厉害,士兵怨恨和逃跑的就越多。最显着的例子,如三纵队第八支队某官长爱打人,结果不仅传令兵伙夫差不多跑完了,军需上士及副官都跑了。九支队第二十五大队,曾经有一时期,来了一个最喜欢打人的大队长,群众还给他的名字叫做铁匠,结果,士兵感觉没有出路,充满了怨恨空气,这个大队长调走了,士兵才得到解放。特务支队第三大队打人的结果,跑了四个伙夫,一个特务长,两个斗争好久的班长。其中一个名萧文成,临走留下一封信,申明他不是反革命,因受不起压迫才逃跑。四纵队初成立时,一二三纵队调去的队长,一味的蛮打士兵,结果,士兵纷纷逃跑,最后这班官长自己也立不住脚,都不得不离开四纵队。二纵队逃兵比任何纵队多,原因虽不止一个,然二纵队下级官长的大多数打人的习惯最厉害,乃是最重要原因之一。二纵队曾发生过三次自杀事件(排长一,士兵二),这是红军最大的污点,意义是非常之严重的。这也不能不说是二纵队打人的风气特别浓厚的一种结果。现在红军中一般士兵的呼声是:“官长不打士兵,打得要死!”这种群众的不平和怨愤的表露,实在值得我们严重注意。

(二)肉刑的来源和废止它的理由

封建阶级为了维持它的封建的剥削,不得不用最残酷的刑罚做工具,以镇压被剥削者的反抗和叛乱,这是肉刑,所以为封建时代的产物的理由。经济的发展,进到资本主义制度它便需要提出自由主义,以发展工农士兵的个性,增强他们的劳动能力和打仗能力,以造成资本主义发展之条件。因此,凡资产阶级的国家,一般的废止肉刑,在军队中亦早就没有什么打人的怪事了。至于经济发展到社会主义的诞生,阶级斗争的激进,工农阶级有推翻统治阶级的权利和依于这个权利的剥削,便要发展自己阶级群众的广大力量,才能取得斗争的胜利。苏维埃政权,是最进步的阶级的政权,它的下面,不应有一切封建制度的残余存在。因此,苏联不但红军中老早没有肉刑在一般法律上,亦通通严禁肉刑的使用。红军第四军产生于封建制度尚未肃清的中国,它的主要成份,又多是从封建军阀军队里头转变过来的,一般封建的制度思想和习惯;依然很浓厚的存在于一般官长士兵之中,由是打人的习惯和非打不怕的习惯,还是与封建军阀军队里头的习惯一样。虽然老早就提出了官长不打士兵的口号和规定士兵会有申诉他们的痛苦的权利,但简直没有什么效力,结果造成官兵间的悬隔,低落了士兵以至官长情绪,逃跑的数目日多,军中充满了怨恨空气,甚至发生自杀事件,这是与红军斗争任务完全背驰的现象,如不赶快纠正,危险不可胜言。

(三)纠正的方法

(1)坚决的废止肉刑。

(2)举行废止肉刑运动。这个运动,要从官兵两方面工作,使“废止肉刑才利于斗争”的意义,普及于官兵群众之中,这样才能使官长方面,不但不至因废止肉刑觉得兵带不住了,而且了解废止肉刑之后,将要更利于管理和训练,士兵方面,不但不至因废止肉刑更顽皮了,而且因废止肉刑增加了斗争的情绪,撤去了官兵的隔阂,将要自觉的接受管理训练和一般的纪律。

(3)肉刑废止之后,因为历史的习惯的原因,发生一些临时不良现象是会有的,这应该加紧我们的责任,努力于说服精神和自觉遵守纪律精神的提倡,去克服这个违反斗争任务的最恶劣的封建制度。决不能借口有封建现象,便掩护了它的封建制度打人习惯。凡那些借口临时不良现象的对废止肉刑或对废止肉刑运动怠工的,客观上便是妨碍革命斗争的发展,也就是帮助了统治阶级。

(四)红军废止肉刑的法律程序


(1)修改红军惩罚条例;(2)由最高军事机关会衔发布废止肉刑的通令,并颁布新的红军惩罚条例;(3)通令发布后,一方面由军政机关召集官长会议,详细说明废止肉刑的理由,使全体官长拥护这个通令的重大改革,良好地努力地在部队中执行起来;(4)一面由士兵全召集士兵代表会议,除拥护这个改革,以后要自党的遵守纪律外,并要森严群众的纪律制裁,以达到肉刑废止后的良好收获。

八、优待伤兵问题

(一)伤病兵痛苦的现象及影响 
(1)全军各部队卫生机关不健全,医官少,薪少,担架设备不充足,办事人少与不健全,以致有许多伤病兵,不但得不到充分治疗,即大概的初步的治疗有时都得不到。

(2)全军军事政治机关对伤病兵注意不充分。如1、对于卫生机关的健全,不但没有尽得最大的努力,而且简直不加注意,各种会议对卫生问题讨论很少。2、官长对于伤病兵没有尽其可能去随时安慰他们,如官长替伤病兵送茶水、盖被窝,随时慰问等习惯在红军中简直没有。官长对伤病兵釆一种不理问态度,甚至表示讨厌他们的态度。3、行军时,官长以至士兵对在沿途落伍的伤兵,完全不表示一点同情,不但不为他们想法子,反而一味的怒骂,或无情的驱逐他们。

(3)重伤重病兵给养和用费不够,伤兵伤后七八天还没有衣服换,调养费,病官有,病兵没有。

(4)蛟洋医院的缺点:①无组织状态;⑦医官和药太少;⑧医官卖私药;④不清洁;⑤御寒衣被不够;⑧看护兵太少;⑦饭食恶劣;⑧房子窄;⑨与当地群众关系不良。以致伤病兵看医院如牢狱,不愿在后方。

上例各种对伤病兵的待遇不良,便发生了下列影响:①使士兵不满红军,“红军好是好,就只不要带花,不要病”,这种舆论简直普遍全体士兵以及下级官长之中;②士兵和官长不满意,越发增加官长间悬隔;⑧士兵及下级官长都怕带花,因此,减少红军战斗力;④逃兵多;⑤影响工农群众,减少他们加入红军的勇气。

(二)解决的办法

(1)军政机关对于卫生问题,再不能象从前一样不注意,以后各种会议,应该充分讨论卫生问题。

(2)卫生机关的组织应特别使之健全,办事人要找有能力的,不要把别地方用不着的人塞进卫生队去。并要增加办事人,达到照料完备之目的,医生少和薪少的问题,要尽可能设法解决。对于医生,应注意督促他们看病详细一点,不要马马虎虎。

(3)官长,特别是和士兵接近的连上官长,应当随时看视伤病兵,送茶水给他们吃,晚上替他们盖被窝。他们觉得冷,要替他们想办法,如向别人借,增加衣服。以上这些照护伤病兵的方法,要定为一种制度,大家实行起来,因为这是最能取得群众的方法。

(4)对行军时,沿途落伍的伤兵:1,禁止任何人对他们的怒骂和讥笑;2、伤病兵让路的时候,要好好对他说,不要一把推开他;3、无论那一个部队和机关,凡有因病因伤落伍下来的,不论是战斗兵、非战斗兵,均要立即派一个人去照顾,他如系重伤重病,并要尽量设法雇夫抬来;4、每次行军,后卫要耐烦带上落伍的伤病兵,必需时还要替他们背回枪弹。

(5)发给伤病兵零用钱,要酌量伤病兵的轻重,重伤重病的要此轻伤轻病的多给一点。调养费一项,对于特别重伤重病的,应该不分官兵夫,酌量发给。

(6)伤病兵衣服被窝问题,公家尽力置备外,应该在名部队官兵中发起募捐,这不仅为了增加伤病兵的零用钱,而且是唤起全军互相济难精神的好方法。

(7)蛟洋后方医院许多缺点,应该有计划去纠正。此外还应该在闽西工农群众中发起募捐(衣被现款粮食)。以密切联系工农群众与红军。

九、红军军事系统与政治系统关系问题

(一)在高级地方政权没有建设以前,红军的政治机关与军事机关,在前委指导下,平行的执行工作。

(二)红军与群众的关系:

(1)凡有全军意义的事项,如发布政纲等,军事政治两机关会衔发布。

(2)群众工作,如宣传群众、组织群众、建设政权等以及没收、审判、处罚、募捐、筹款、济难等事之指挥监督,在地方政权没有建设以前,均系政治部职权。

(3)凡没有建设政权机关的地方,红军政治部即代替当地政权机关,至地方政权建设时为止。凡地方政权机关已经建设的地方,应以使地方政机独立处理一切事情,在群众中巩固其信仰为原则。只有在地方政权机关还不健全,及红军与地方有关系的事项,得用地方政权机关与红军政治部会衔的方法处理之。

(4)帮助地方武装之建立与发展,这个责任是政治部的,帮助地方武装之平时的军事训练及战时的作战指挥,这个责任是司令部的,但均需尽可能的经过地方政权机关的路线,极力避免直接处理。

(三)红军里面用人行政,军事政治两个系统,各有独立的路线,彼此有关系时,如人员等之互相调动,消息之互相传达时,则用公函平行通报。

(四)礼节及军风纪之执行,军事政治两个系统,相互间均用阶级服从原则,不得借口系统不同,有所怠慢或不服指挥。

(五)凡给养、卫生、行军、作战、宿营等项,政治系统应接受军事系统之指挥。凡政治训练及群众工作事项,军事系统应接受政治系统之指挥。但指挥的形式,只能直达对方机关里头的从属机关(总务科或副官处)。

(六)凡红军筹款的指挥及政治工作用费的决定与支出,均属于政治部,军事机关不得干涉(取款手续,政治部直接军需处)。党部用费由政治部支给。

(七)军事机关一切命令,除政治委员须副署外,政治部主任无须署名。政治机关一切命令,政治机关单独行使,政治委员无须副署。
