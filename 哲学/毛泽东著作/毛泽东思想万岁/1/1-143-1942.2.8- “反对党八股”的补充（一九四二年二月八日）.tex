\section[ “反对党八股”的补充(一九四二年二月八日)]{ “反对党八股”的补充}
\datesubtitle{(一九四二年二月八日)}


第二,要学外国语言,外国人间的语言并不是洋八股,中国人抄来的时候,把它的样子硬搬过去,就变成要死不括的洋八股了。

(第八三八页第十行)

比如,《党与非党的联盟》,这是斯大林在关于苏联新宪法的演说中讲到的。我们吸收在陕甘宁边区的施政纲领里面,讲到《共产党员与党外人士实行民主合作》如此等类,总之,我们非多多吸收外国的好东西不可。

(第八三八页倒数第三行)

斯大林在联共十八次大会上说:“有一部分同志对于新鲜事物失去了感觉”,我们有些同志也是这样,很多的新鲜事物都看不见,这个毛病必须医治。

(第八三九页第三行)

在中国生活的共产党员,离开中国的实际需要来读马克思主义,纵令你把马克思主义读一万本一千遍,也还是一个假马克思主义,这样的“马克思主义理论家”,也还是一个“老鼠上称钩,自己称自己”的假理论家。

(第八四五页倒数第四行)

<p align="right">(一九四二年六月十八日《解放日报》)</p>

