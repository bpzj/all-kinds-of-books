\section[同毛远新同志的第二次谈话(一九六四年八月)]{同毛远新同志的第二次谈话}
\datesubtitle{(一九六四年八月)}


主席:这半年有没有进步?有没有提高?

远新:我自己也糊里糊涂,说不上有进步,有,也只是表面的。

主席:我看还有进步。你现在对问题的看法不是那样简单了,你看过“九评”没有?接班人的五个条件看了没有?

远新:看过了。(接着把“九评”上所讲的接班人五条件的主要内容讲了一下)

主席:讲是讲到了,懂不懂?这五条是互相联系的,第一条是理论,也是方向;第二条是目的,到底为谁服务,这是主要的,这一条学好了什么都好办;三、四、五条是方法问题。要团结多数人,要搞民主集中制,不能一个人说了算,要有自我批评,要谦虚谨慎,这不都是方法吗?

(主席在件到接班人的第一条时说,你要学马列主义,还是修正主义)

远新:我当然要学马列主义。

主席:那可不一定,谁知道你学什么,什么是马列主义,你知道吗?

远新:马列主义就是要搞阶级斗争,搞革命。

主席:马列主义的基本思想就是要革命,什么是革命?革命就是无产阶级打倒资本家,农民推翻地主,然后建立工农联合政权,并且把它巩固下去。现在革命任务还没有完成,到底谁打倒谁还不一定,苏联还不是赫鲁晓夫当政,资产阶级当政。我们也有资产阶级把持政权,有的生产队、工厂、县委、地委、省委都有他们的人。有的公安厅付厅长也是他们的人。文化部是谁领导的?电影、戏剧都是为他们服务的,不是为多数人服务的!你说是谁领导的?学习马列主义就是学习阶级斗争,阶级斗争到处都有,你们学院就有。你们学院出了个反革命知道不知道?他写了十几本反动日记,天天在骂我们,这还不是反革命分子?你们不是感觉不到阶级斗争吗?你们旁边不是就有吗?没有反革命还要什么革命?(远新汇报说,在工厂实习听到一些工厂五反运动情况,受到教育很大)哪里都有反革命,工厂怎么没有?国民党的中将,少将,县党部书记长都混进去了,不管他改变什么面貌,现在就是要把他们清查出来。什么地方都有阶级斗争,都有反革命分子。陈东平不是睡在你的身边吗?你们学校揭发的几个材料厂我都看了,你与反革命睡在一起还不知道?

(主席接着问学院的政治思想工作如何,毛远新同志讲了自己的看法,并说开会、讲课形式上轰轰烈烈的,解决实际问题不多。)

主席:全国都大学解放军,你们是解放军,为什么不学?学院有政治部吗?那是干什么的?有政治教育吗?(毛远新说明了政治教育情况)都是上课、讨论有什么用处?应当到实际中去学。你们就是思想第一没有落实。你们一点实际知识也没有,讲那些东西能听懂?

(主席特别提倡在大风大浪中游泳,并让远新天天坚持去)

主席:你敢不敢到浪里去游泳?(在北戴河游泳)

远新:敢。(立即就游出去了。)

主席:(远新回来后)还敢去吗?(远新又游出去了。)

远新:(回来后)这次差点没回来。

主席:水你已经认识它,已治服它了,这很好。你会骑马吗?(远新答:不会)当兵不会骑马不应该(主席叫远新去学骑马,主席也经常练习骑马,还叫秘书、工作人员也去学)。

主席:你会打枪吗?

远新:有四年没摸了。

主席:现在民兵打枪打得很好,你们解放军还没打过枪,哪有当兵不会打枪的。

有一次游泳天气较冷,水里比较暖和,毛远新上来后,觉得有点冷,就说:“还是水里舒服”,主席瞪了远新一眼“你就喜欢舒服,怕艰苦。”

主席在讲到接班人第二条时说:你就知道为自己着想,考虑的都是自己的问题。你父亲在敌人面前坚毅不屈,丝毫不动摇,就是因为他为多数服务。要是你还不是双膝跪下乞求饶命了。我们家很多都是让国民党、美帝国主义杀死的,你是吃蜜糖长大的,从来不知道什么叫苦。你将来不当右派,当个中间派我就满足了,你没有吃过苦嘛,怎么能当上左派?(毛远新说:我还有点希望吧?)有希望,好,超过我的标准就更好。

主席在讲到接班人的第三条时说:你们开会是怎样开的?你当班长是怎么当的?人家提意见你能接受吗?提错了受得了吗?如果受不了那怎么团结人?你就喜欢人家捧你,嘴里多吃点蜜糖,耳里听的赞歌,这是最危险的,你就喜欢这个。

主席在讲到接班人第四条时说:你是否与群众合得来,是否只和干部子女在一起,而看不起别人?要让人家说话,不要一个人说了算。

主席在讲到接班人第五条时说:你现在有了进步,有点自我批评了,但还刚刚开始,不要认为什么都行了。

以后主席又谈到学院的工作,你们学校最根本的是四个第一不落实,你不是讲要学习马列主义吗?你们是怎么个学法?只听讲课能学到多少东西?最主要的是要到实际中去学习。(毛远新说:工科学院与文科学院不一样,没有安排那么多时间去接触社会)不对,阶级斗争是你们的一门主课,你们学院应该到农村去搞四清,从干部到学员全部都去。对于你不仅要参加五个月的四清,而且要到工厂搞上半年五反,你对社会一点也不了解嘛!不搞四清你不了解农民,不搞五反你不了解工人,这样一个政治教育完成了,你才算毕业,不然军工让你毕业,我是不承认的。阶级斗争都不知道,你怎么能算大学毕业生呢?你毕业了,我还要给你安排这一课,你们学院就是思想工作不落实,这么多反革命,你没感觉?陈东平在你身边你就不知道,(毛远新说:陈东平是在家休学收听敌人广播变坏的。)听敌人广播就那么相信?你听了没有?敌人连饭吃都没有,他的话你能相信?卫立煌就是在香港作生意赔了本才回来的。卫立煌这样的人人家都看不起,难道能看得起他(指陈东平)。什么是四个第一?(远新讲了一遍)知道了为什么抓不住活思想?听说你们学院政治干部很多,就是不抓基层,当然思想也抓不住。学梡当然有成绩,出了毛病也没有什么了不起的,军工才办了十年,军队办技术学校我们也没有经验,好像二七年我们打仗一样,开始不会打,老打败仗,后来就学会了。

主席又问:你们学校的教学改革的情况怎么样?

远新:过去就是分数概念,学习搞的不主动。

主席:你能认识就好,这也不能怪你,整个教育制度就是那样,公开号召去争取那个全优,那样会把你限止死了的。你姐姐也吃过这个亏。北大有个学生,平时不记笔记,考试时也是三分半到四分,可是毕业论文水平最高,人家就把那套看透了,学习也主动了。就有那样一些人把分数看透了,大胆主动的去学。你们的教学就是灌,天天上课,有那么多可讲的?教员应该把他的讲课底稿印发给你们,怕什么,应当让教员去研究。讲稿也对学生保密?到了讲堂上才让学生抄,把学生限止死了。我过去在抗大讲课就是把讲稿发给学员,我只讲三十分钟,让学生自己去研究,然后提出问题,教员再答疑。大学生,尤其是高年级学生,主要是自己钻研问题,讲的那么多干什么?过去公开号召大家争全优,在学校是全优,工作不一定就全优,中国历史上凡是状元的都没有真才实学,反倒连举人都考不取的人有真才实学。唐朝最伟大的两个诗人连举人也没考取。不要把分数看重了,要把精力集中去培养、训练分析问题能力和解决问题能力上,不要跟在教员后面跑,受约束。教改的问题主要是教员的问题,教员就那么多本事。离开了讲稿什么也不行,为什么不把讲稿发给你们,与你们一起研究问题?高年级学生提出问题教员能回答百分之五十,其他就说不知道,和学员一起商量,就是不错的。不要装着样子去吓唬人。反对注入式教学法,就连资产阶级都提出来了,我们为什么不反,只要不把学生当打击对象就好了。教改的关键是教员。(有一次毛远新动员毛主席去看科学成就展览,主席说,现在忙,不能去看,看详细了没有时间,走马观花又没意思。接着,主席说:你怎么对这个感兴趣,对马列主义不感兴趣,不然,平时怎么很少听你问起这方面的问题来。)

主席又问毛远新平时看什么报,主席说:要看解放军报,中国青年报工人战士写的文章,实际活泼,又能说明问题。合二而一的讨论你看了吗?(毛远新说:很少看,看不懂)是嘛,你看看这份报纸,(主席递给一份中国青年报),你看工人是怎样分析的,团的干部是怎么分析的,他们分析的很好。主席又说:你们政治课主要是讲课,光讲课能学到多少东西?最主要的是到实际中去学习。你为什么对专业感兴趣,对马列主义不感兴趣?研究历史不接合现实不行,研究近代史不去搞村史、家史就等于放屁!研究古代史要结合现实,也离不开挖掘,考古,尧、舜、禹有没有?我就是不信,你没有实际材料证明嘛!商有乌龟壳证明可以相信。钻到古书堆中去学,越学越没有知识了。


