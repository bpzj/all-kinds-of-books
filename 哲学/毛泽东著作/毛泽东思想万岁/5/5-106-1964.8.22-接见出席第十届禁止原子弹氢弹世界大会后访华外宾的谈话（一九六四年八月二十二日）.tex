\section[接见出席第十届禁止原子弹氢弹世界大会后访华外宾的谈话(一九六四年八月二十二日)]{接见出席第十届禁止原子弹氢弹世界大会后访华外宾的谈话}
\datesubtitle{(一九六四年八月二十二日)}


\begin{list}{}{
    \setlength{\topsep}{0pt}        % 列表与正文的垂直距离
    \setlength{\partopsep}{0pt}     % 
    \setlength{\parsep}{\parskip}   % 一个 item 内有多段,段落间距
    \setlength{\itemsep}{\lineskip}       % 两个 item 之间,减去 \parsep 的距离
    \setlength{\labelsep}{0pt}%
    \setlength{\labelwidth}{3em}%
    \setlength{\itemindent}{0pt}%
    \setlength\listparindent{\parindent}
    \setlength{\leftmargin}{3em}
    \setlength{\rightmargin}{0pt}
    }
\item[\textbf{主席:}] 欢迎各位朋友。很感谢你们来到我们国家访问。这是头一次见到你们,但是我们的思想有共同点。民族不同,国家不同,信仰也可能不同,但我们根本的是反对帝国主义、新的同老的殖民主义。你们认为我说的对,还是不对?(大家鼓掌表示同意)我们这些国家,包括中国在内,是没有原子弹、氢弹的。这里有法国朋友么?(法国人示意有)有英国朋友么?(无人答)喔,没有。法国是有原子弹的,但是你们是反对发动原子战争的,是不是这样的?(法国人带头鼓掌)我们的国家将来可能生产少量的原子弹,但是并不准备使用。既然不准备使用,为什么要生产呢?这是我们做为防御的武器。现在一些核大国,特别是美国,拿原子弹吓唬人,美国有这么多原子弹,也只使用过两次,就是在长崎和广岛。这里有日本朋友么?(西园寺点头)他们的国家是受害的。美国在日本丢了两个原子弹,但是因此使美国的名声不好,在世界大部分人民中间,美国的名声不好,世界人民是反对用原子弹杀人的,反对发生第三次世界大战,也反对那些在越南南方的外国军队干涉别人内政,打“特种战争”。法国政府说,法国政府过去在越南打仗是打败了的,你们美国又在打,照法国人看来,美国人也要打败的。因此法国反对用战争办法解决越南问题、老挝问题,主张和平谈判。

法国已经取得了一定的资格来讲话,日本也取得了一定的资格来讲话。第二次大战中,日本政府是强迫日本人民进行侵略战争的,但后来起了变化,遭到了美国的原子弹之害,所以日本人民包括某些政府人士也反对原子战争。在座大多数代表的国家同我们一样,是没有原子弹的,我们跟法国有外交关系,在反对美国侵略这一点上,我们有共同点,不但跟在座的法国朋友在共同点,跟戴高乐也有。(大家笑)这就是说,世界已经起了变化,不再允许侵略和干涉别的国家,侵略干涉别国是要失败的。

在座的各位代表了人类大多数人口,你们的理想是要胜利的。当然我讲的还未成事实,是不是会胜利,要靠斗争。但是我可以讲讲我自己的经验。我是一个当小学教员的,读了很多孔夫子的书,他的道理是封建主义的。后来就读资本主义的理论,熟悉英国的克伦威尔,法国的大革命,华盛顿领导的大革命,民主革命的思想,也读过法国康德的二元论、唯心主义和唯物主义,就是不知道世界上有什么马克思主义,列宁主义更不知道。后来不知道什么原因使我离开我原来的工作去搞政治了。依我看来,还是帝国主义压迫中国,国内反动派压迫人民,因此,使我们这些人一下子就跑到共产党一边去了。孔夫子的道理也不信了,资产阶级的思想也不信了,相信了马克思列宁主义,搞起了工人罢工、学生运动,组织了工会,后来又组织了农民运动组织。那时,我们跟国民党合作,我还当上了国民党中央委员会的委员,在国民党的宣传部工作。那个时候国民党领袖孙中山先生还在世,我听过他演讲,跟他谈过话。孙中山夫人现在还在上海,她是我们国家的副主席,她不是共产党。\marginpar{\footnotesize 155} 这里有一位廖承志,他是个共产党员,(转向廖问)你是搞什么工作?哦,和委会副主席。他(指廖)的父亲是国民党,是中央委员,国民党的左派,被右派暗杀了。他很早就只有母亲,没有父亲了。\\(问廖)你父亲是那年死的?

\item[\textbf{廖:}] 一九二五年。

\item[\textbf{主席:}] (让大家抽烟),他的母亲在北京工作。\\(问廖)你母亲多大年纪了?

\item[\textbf{廖:}] 八十七岁。

\item[\textbf{主席:}] 她是一位进步的民主人士,是一位艺术家,能画画。我这是讲怎样使我参加到共产党当中来。我没有准备打仗,更不知道原子弹,也没有无线电。是什么原因使我到军队中去呢?还是帝国主义、蒋介石杀人。一九二七年我们有五万党员,蒋介石背叛革命,搞白色恐怖,到处杀人,这五万党员中有一批被杀牺牲了,另一批投降了蒋介石,第三批没有被杀,也没有投降,不干了,五万人只剩下几千人,这几千人就打游击,以后打了十年。那时候我们不懂打仗,谁教我们的呢?还不是蒋介石!我们也没有枪。谁给我们枪呢?还不是蒋介石!蒋介石也没有多少枪炮,是背后的帝国主义给他枪炮,主要是英国和美国。这样我们就有了枪炮,一打就打了十年,军队到了三十万人。我们叫作工农红军。这十年中间,我们犯过错误,后来我们打败了,国民党打胜了,我们就走路,一走就走了一万二千五百公里,相当于地球的直径,由南方走到北方。这不能怪蒋介石,只能怪自己犯了错误。军队由三十万剩下两万五千人,现在看起来没有什么了不起,可是那个时候,腿走那么远也是难受的。(大家笑)那时候没有扩音设备,也没有茶,也没有水果吃,仅仅每人每天三钱油、三钱盐、一升米。

可是你们不要以为我们这个时候比过去更弱了,相反,是更强了。因为我们得到了教训,以后就是日本军队占领了大半个中国。又打了八年,又跟蒋介石合作,不打仗了,一齐跟日本人打了。我们的军队打了八年以后,由两万五千人扩大到一百二十万。以后日本投降了。战争快要结束的时候,美国投了两个原子弹,损害日本人民。我们中国人民也对胜利作了贡献。日本军队走了,美国人来了。一九四五年第二次大战结束,一九四六年蒋介石就打内战,打了四年。一九四九年蒋介石说不打仗了,他跑了,(大家笑)他就到台湾去了。我跟你们讲了历史的变化。讲了多久?(看表)讲的太长了吧?(外宾中有人说:“很好”。)就是讲,世界在起变化。帝国主义是可以打败的。蒋介石几百万军队是可以打败的。人民总是要胜利的。我就不相信人民不能胜利。我看你们也不相信你们不能胜利,你们是失败主义者吗?(外宾中许多人答:“不是。”)我们要的是全世界人民的解放。(大家鼓掌)我要讲的就是这么一点,也没有什么深奥的道理。(问大家)你们有什么话要讲?

\item[\textbf{孟德斯(巴拿马):}] 不知道能不能提几个问题。你对修正主义有什么具体的意见?修正主义目前提的问题你看有什么出路?前景是什么?

\item[\textbf{主席:}] 我看修正主义没什么出路。修正主义是适应帝国主义的要求,适应国内资本主义势力的要求,而不适应广大人民的要求。他们一时好像是多数,将来会证明他们不是多数,而是少数。修正主义他们不讲革命,不讲反对帝国主义,有时候讲几句反帝,那是假的,这里有阿尔及利亚、法国的朋友,你们国家的共产党就是修正主义领导的党,我们跟他们谈不来,我们反而跟本贝拉谈得来。\marginpar{\footnotesize 156} (鼓掌)反而跟戴高乐总统谈得来,我们跟戴高乐不是一切都谈得来,是在反美这一点点谈得来。(鼓掌)还有什么问题,想一想?

\item[\textbf{易卜拉欣(约旦):}] 我们阿拉伯国家面临着很复杂的巴勒斯坦问题,巴勒斯坦的阿拉伯人如何才能回到自己的国土是个问题。我们很感谢中国人民,中国政府和中国的党对我们的支持。希望听到毛主席的意见。

\item[\textbf{主席:}] 就是站在你们那一边嘛。(鼓掌)巴勒斯坦的阿拉伯人应该回到家乡去。我们和以色列政府到现在还没有外交关系,你们是大多数嘛!整个阿拉伯世界的人民和政府都是反对把巴勒斯坦人赶出去嘛!我们不赞成你们就是犯错误,所以我们采取了赞成你们。问题他们不是一个以色列,而是谁站在以色列背后的问题,是个世界性的问题,也是个美国的问题。(问外宾)还有问题没有?

\item[\textbf{琼斯夫人(特兰尼达):}] 在美国所有的黑人,非洲血统的人都非常感谢主席支持黑人的声明,针对美国当前形势更加尖锐化,美国更失去其世界地位的情况,请就此声明再说几句话。

\item[\textbf{主席:}] 黑人是美国国内的少数民族,但是人数也不少,有两千万人。他们反歧视、反压迫的斗争正在发展,总有一天要胜利。美国无产阶级总有一天要觉醒。就是要大多数白人无产者、半无产者、进步人士和黑人团结一致,反对他们共同的敌人。一个国家里面要分阶级的。例如在我们这个国家,我们跟蒋介石就是两个不同的阶级,他不喜欢我们,我们也不喜欢他。但是都是黄种人,中国人,都讲中国话,写中国字。有人说我们是民族主义者,照他们的说法,我们就应当同蒋介石团结,而不是同你们团结。在座的这里有不同肤色、不同国家的人,但是在反帝,特别是反美这一点上,我们是一致的。不管彼此认识不认识,在今天同你们见面以前我一个也不认识。没见过的人可多哩!难道都见过么?我们国家七亿人口,我就没都见过。(问法国外宾)你们法国几千万人,你们也没有都见过,难道几千万人你们都见过么?(大家笑)不可能的。(问大家)还有问题吗?

\item[\textbf{恩朗(喀麦隆):}] 毛主席知道我们非洲人正在进行反对帝国主义、争取独立的斗争。我们一向很满意中国领导人支持非洲人民反对帝国主义和新老殖民主义的斗争。今天希望毛主席对非洲局势,特别是刚果局势讲几句。

\item[\textbf{主席:}] 非洲的局势是很好的局势,对人民有利的局势。在非洲有蓬勃的反对帝国主义的斗争,这使修正主义不高兴,帝国主义更不高兴。刚果应该是刚果人民的刚果,应该是卢蒙巴那样的人的刚果。卢蒙巴身体死了,他的思想没有死。我们支持刚果人民的斗争,不是偷偷摸摸地支持,是公开支持,公开支持他们的斗争。我们报纸上公开登载双方的情况,就是讲压迫他们的一面和反抗压迫的一面都提,但是我们的心是向着被压迫的人民一面的,就是说,我们是有偏向的,有人看来这是不公正的,你们为什么不两方面都相好呢?对帝国主义走狗也相好,对卢蒙巴、对正在斗争的人民武装力量也相好!我们不干,我们只跟一方面好。要说我们有偏向,我们就是有偏向!在巴勒斯坦问题上,我们偏在大多数阿拉伯人民一边,在东京大会上,我们偏在你们一边。东京大会我没去,他(指刘宁一)去了。他叫刘宁一,他就是个有偏心的人。(大家笑)他不是两个会都参加,只参加了一个。(大家笑,热烈鼓掌)我看你们也有偏心的,只参加了一个会,不参加另一个。(大家又笑)。因此我们大家团结起来。\marginpar{\footnotesize 157} (问外宾)差不多了吧?

\item[\textbf{瓦达达(乌干达):}] 现在南非人民受压迫,南非统治者武装到牙齿,受美帝国主义支持。非洲各国领导人是反对南非统治者的,可是他们之间也存在着不团结,依主席看来,南非非洲人取得胜利的前景如何?

\item[\textbf{主席:}] 要很快胜利可能有困难。要经过长期的、曲折的斗争。因为南非与别的地方形势不同,甚至同阿尔及利亚也有所不同。阿尔及利亚人民打了八年仗,他们胜利了,南非可能要更长的斗争,更曲折,也许要超过阿尔及利亚人民取得胜利的时间。也许不需要像中国人取得胜利的这么多时间,我们是打了二十二年仗,如果包括朝鲜战争三年,就是二十五年,所以我们这些人半辈子都耗费在打仗上了。无论如何总是要胜利,有全世界人民支持南非人民。南非有一千万非洲人,三百万外国人。三百万外国人中也有一部分同情非洲人。我问过南非朋友这个问题:是不是三百万外国人都反对你们?他们说,不是,有一些进步力量帮助他们,甚至有高级知识分子,如律师,帮助他们辩护。我就劝他们要向外国人作工作,不要以为三百万外国人都是不好的。全世界白色人种大多数是好的,顶多有百分之几是不好的。有百分之九十以上的人或者现在已经觉悟,或者现在还不觉醒,将来总会觉醒的。他们是好心的。有人说我们团结有色人种,反对白人,在座有不少白人,你们白人朋友是代表大多数白人的,也并非所有有色人都是好的,蒋介石就不好!(大家笑,鼓掌)(主席作一手势)好,完了吧。
 
\end{list}