\section[论新阶段(节录)(一九三八年十月)]{论新阶段(节录)}
\datesubtitle{(一九三八年十月)}


同志们,我代表中央政治局向扩大的六中全会作报告,我准备说些什么呢?我要说的分为下述几部分:

(一)五中全会到六中全会;

(二)抗战十五个月的总结;

(三)抗日民族战争与抗日民族统一战线发展的新阶段;

(四)全民族的当前紧急任务;

(五)长期战争与长期合作;

(六)中国民侵略战争与世界反法西斯运动;

(七)中国共产党在民族战争中的地位;

(八)党的七次全国代表大会;我要说的就是这些问题。

同志们,在全国炮火连天全世界战争危机紧迫的环境中来开我党的六中全会扩大会,我们要做些什么工作呢7我们的目的何在呢?我们一定要同全中国一切爱国党派一切爱国同胞永远的团结起来,克服新的困难,动员新的力量。在目前,停止敌之进攻,在将来,实行我之反攻,达到逐日本帝国主义建立三民主义共和国之目的,我们一定要自由,我们一定要胜利,这就是我们的目的,也就是我的报告的总方向。

(一)五中全会到六中全会

一、扩大的六中全会之召集

我们党的中央全体会议,自从一九三四年一月在江西开过第五次中央全会以来。差不多五个年头了。因为各中央委员分散工作于国内外各种不同的环境,使我们不能聚集一块。此次则除了几个同志之外,最大多数的中央委员都到了,而且到了全国各种许多领导工作的同志,使我们这次中央全会成为第六次全国代表大会以来人数到得最多的一次。本来第七次全国代表大会准备在本年召集的,因为战争紧张的原故,不得不把七中推迟到明年。而当前时局向我们提出了许多的问题,必须作明确的解决,以便争取抗战的胜利,所以召集了这次扩大的中央全会。

二、五中全会到六中全会

五年以来,我们党经历许多重大的事变。最大与最主要的是,由国内各党派各阶级互相对立的局面转到了抗日民族统一战线,由国内战争转到了抗日战争。

过去国内战争形式的原因,在于一九二七年不幸破裂了国共两党的统一战线。这是由于当时的历史环境造成的。

抗日民族统一战线的政策,又是怎样形成的呢?乃是由于新的历史环境。大家都已非常明白,自从九一八事变以来,中华民族的敌人日本帝国主义,经过侵略东四省的第一步骤,进到准备并实行向全中国侵略的第二步骤。这种空前的历史事变,使得国内国际情况都发生了变化。首先变化了与变化着国内各个阶层,各个党派,各个集团之间的相互关系,同时也变化了与变化着国际间的相互关系。因而我们的党,根据着空前的历史事变,根据新的国内国际关系,沿着远在一九三三年就已开始采取了的新的政治立场(三个条件下与国民党内任何愿意同我们合作的成份订立抗日作战协定)的道路把它提高到抗日民族统一战线的新政策,因而发表了一九三五年八月的宣言,十二月的决议,一九三六年八月的致国民党书,九月的民主共和国决议。并且根据了这些,使得我们能够在当年十二月间发生的西安事变,坚持了和平解决的方针。并于一九三七年二月敌国民党三中全会一个团结抗日的具体建议。去年五月,召集了一次临时性的代表大会(名曰苏区代表大会。有当时苏区非苏区及红军代表参加)通过了“抗日民族统一战线在目前阶段的任务”。批准了红军实行改编为国民革命军,苏维埃实行改为民主制。这样,就在事实上由国内战争的状态转到了开始建立抗日民族,统一战线的新时期。……(略)十二月,为着巩固与发展抗日民族统一战线,我们又发表了愿与国民党不但合作抗日而且合作建国的宣言。此时未久,南方的红军游击队改编为国民革命军新编第四军,开赴江南作战。从此以后,抗日团结便日益进步了。

同志们,这种由两党十年战争转到两党重新合作,并在极端困难的条件之下执行了这个转变,奠定了两党长期合作的始基,是经过了许多艰难曲折才完成的。然而由于中央与全党的努力,总算是完成了。共产国标完全同意我们党的这个新的政治路线(共产国际决议,九月八日“新华日报”。)并为了中华民族的胜利,号召世界各国共产党与无产阶级援助中国抗日战争。

同志们,假如没有国共两党为基础的抗日民族统一战线的发起,建立与坚持,如此伟大、抗的日民族革命战争之发动,持久与争取胜利是不可能的。现在全中国全世界的人都明白,中华民族是站起来了。一百来年受人欺凌、侮辱、侵略、压迫、特别是九一八事变以来那种难堪的奴隶地位,是改变过来了。全中国人手执武器走上了民族自卫战争的战场。全中国的最后胜利,即中华民族自由解放的曙光,已经发现了。

我们知道,我们今天的这一伟大的民族战争,和中国过去的一切历史时期的战争都不相同。因为这个战争是为了把中华民族从半殖民地状态中从亡国灭种危险中解放出来的战争。而且这个战争是在中华民族历史上最进步的时期进行的。同时,又是在我们的敌人日本帝国主义自寻死路走向崩溃的时期进行中的。同时,又是在全世界先进人类正承备着空前广大与空前深刻的斗争力量以便抵抗与战胜德日意法西斯魔王争取世界光明前途的时期进行的。这样三方面因素的结合一以中国进步并且继续进步为主要基础的三方面的结合。就保证了我们的抗日战争一定能够最后取得胜利,而自由解放的新中国一定要出现于东亚,并成为未来光明世界中一个极重要的组成部份。这样的一个中国,不但将造福于四万万五千万中国人,而且将造福于人类。

三、六中全会的任务

这次扩大的六中全会,是处于抗日战争将要进入一个新的发展阶段的重要关头开会的。扩大六中全会担负了重大的历史任务。

完全不错,抗日战争英勇奋斗了一年多,全国有了伟大的团结与伟大的进步,给了日本帝国主义以严重的打击。虽然失地很多,但同时就有了很多胜利,这是无可否认的事实。战争发展下去,主要的由于中国继续进步,同时配合着日本增加困难,国际助我增强,最后胜利定属于我,不属于敌,这也是可以预断的。谁要是看不见过去的伟大成绩和未来的胜利前途,谁就要陷入悲观主义的深坑而不能自拔。然而单看到这一方面,是不够的。抗日战争还有另一方面,还有它的消极方面,这就是我们面前摆着的许多困难。目前的情况告诉我们,一年多以来中国所有的奋斗,团结,进步,胜利,还未能阻止敌人的前进,还没有反攻敌人的力量。武汉现在正处于敌人的威胁中,敌人还要向广州,长沙及西北等地进攻。因此全国人民都盼望共产党发表意见,新的环境提出了许多问题。同志们,我们必须发表意见,必须解决问题。对的,我们党早已发表了意见,许多根本问题也早已解决了。但新的环境要求我们发布新的意见,解决新的问题。

什么是新的问题呢?

如何在现有基础上增加新的力量,渡过战争难关,停止敌之进攻,准备我之反攻,达到驱逐敌人之目的,这就是当前问题的关键。这个问题在全国无数人们中议论着,焦思着我们应不应该回答这个问题呢?无疑是应回答的。

这个问题展开于各方面,发生了许多的问题。

例如:十五个月抗战的经验究竟证明了什么呢?十五个月经验证明抗战是长期的,还是短期的呢?战略方针是持久胜敌的,还是速战胜敌呢?最后胜利是中国的还是敌人的了抗战有出路,还是妥协有出路呢?如果战争是长期的,又用什么方法去支持长期战争与取得最后胜利呢?所有这些,是否能在十五个月经验中找到根据给以明确的回答,并是否可依据这些过去基础而在抗战的新阶段中起其积极作用,籍以克服新的困难,争取新的胜利呢?这些都是重要的问题。这是一种的问题。

又如:整个抗日战争将怎样发展变化呢?所谓新阶段,究将是一种什么性质的阶段呢?假定武汉不守,战争趋势将怎么呢?今后全国努力的方向,即全中华民族的当前紧急任务,应是什么呢?有些什么好办法足以渡过战争的难关呢?这些更是重要的问题。这又是一类问题。

又如:国共合作的前途与远景会怎样呢?共产党有何根据来说长期合作呢?共产党有何办法来改善两党之间的关系呢?所谓不但合作抗战而且合作建国,究将建立一个什么呢?三民主义与共产主义的关系怎样呢?这也是很重要的问题。这又是一类问题。

又如:世界风云如此紧急,其趋势将怎样呢?中国抗日战争与世界反法西斯运动有何利害关系?这也是重要的问题。这又是一类问题。

还有:中国共产党在民族战争中的地位如何呢?共产党员为着其党的政治方针而奋斗时,其工作态度该怎样呢?共产党有什么更好的方法同他党合作,同人民联系,足使艰难的时局走向顺利呢?共产党的内部关系怎样7有些什么好方法团结全党使之在抗日战争中尤其在当前艰难的时局中起其大的作用呢?共产党的七次代表大会将怎样呢?这也是重要的问题。这又是一类问题。

所有这些问题,都是党内党外迫切要求解决的。近几个月来,我们经常遇到要求回答这些问题的人们。

同志们:我们的国家是一个广大而复杂的国家,而这个国家现在正处于同一个强的帝国主义作决死的斗争。这个斗争现在已接近一个新的发展阶段。目前正处于新阶段发展的过渡期间。我们的扩大的六中全会是在这个时候开会的。扩大的六中全会的责任非常重大,我们要解决许多的问题。

(二)抗战十五个月的总结

一、十五个月经验证明了什么?

让我们从十五个月的经验说起罢。

十五个月的经验给了我们以什么呢?我以为主要的有三方面。第一:说明了抗日战争是长期的不是短期的,因而抗战的战略方针是持久战而不是速决战。第二:证明了中国的抗战能够取得最后胜利,悲观论者之没有根据。第三:证明了支持长期战争与取得最后胜利之唯一正确的道路,在于统一团结全民族,力求进步与依靠民众,籍以克服困难,争取胜利,而不是其他。

二、抗日战争是长期的不是短期的。战略方针是持久战不是速决战。

抗战初起之时,许多人不从敌我力量基本上的对比出发,而从若干一时的与表面的现象出发。设想战争不久就可解决,必胜思想笼罩一时。……(略)我们则在很早的时候就指出了抗日战争是长期性不是短期可以解决的。“战争的结果,日本必败,中国必胜。只是牺牲要大,要经过一个痛苦的时期”(一九三六年七月十六日毛泽东与斯诺谈话)“应该看到这一抗战是艰苦的持久战。”(一九三七年八月十五日中共中央关于目前形势与党的任务的决定)所有这些都指出:抗日战争是长期的不是短期的,战略方针的持久战不是速决战。抗战十五个月的经验完全证明其正确。

理由何在?在于敌强我弱,敌是优势,我是劣势,敌是帝国主义国家,我是半殖民地国家。

我们从早就指出过,战胜日本帝国主义要有三个条件:第一是中国的进步,这是基本的,主要的;第二是日本的困难;第三是国际的援助。让我们来看一看这些条件在抗战十五个月中已经怎样了?一句话回答:已经有了一个基础,但距必要的程度还很远。

拿第一个条件(战胜敌人基本的主要的条件)中国的进步来说,十五个月来确已有了一个基础,但惟有继续进步,才能最后胜利,所谓中国的进步,包括国内政治,军事,党务,民运,文化教育等一切方面。这些方面的进步,十五个月来是非常显著的。然而单拿这些已有的东西,还不能停止敌之进攻,实行我之反攻。反攻必须有一个准备时期,必须经过全民族的努力,使我们民族中一切生动力量有了一个广大的与深刻的发动,才有反攻胜敌之可能,因此速胜论是没有根据的,他忘记了敌强我弱这一特点,忘记了敌是优势,敌是帝国主义国家,我是半殖民地国家,中国具有很大潜伏势力,发动起来足以使自己转败为胜,转弱为强,根本变化敌我形势,然而还待今后的努力,不是现成的事实。

拿第二条件日本困难来说,也是这样。十五个月中,敌人出兵百万,伤亡数十万,用费数十万万,军队锐气日减,财政经济日形竭镢,国际舆论纷来谴责,这些都是日本的野蛮侵略与中国的英勇抗战造成的结果。然而敌人这些已存的困难,还不足以停止他的进攻及利于我们的反攻。还须待到敌更大的困难我有更大的进步时,才是反攻敌之机会。因此急速胜论在敌情方面也没有根据,十五个月经验已经证明了。

拿第三条件国际助我一点来说,现在也还未至最大有利之时。十五个月以来,我们有了国际间广大的舆论声援,苏联和其他民主国家根据国联决议已经给了我们许多帮助,证明了我们是不孤立的。然而我们必须看到国际和平阵线各国有其不相同情况。资本主义国家,人民助我,政府则取某种程度中立态度,其资产阶级则利用战争做生意,还在大量输还军火与原料给日本,社会主义国家根本上不同于资本主义国家,在援华上已经具体的表现出来。然而国际形势目前还不容它作超过现时程度的援助。因此,我们对国际援助暂时不应希望过高。抛开自力更生的方针,而主要地寄其希望于外援,无疑是十分错误的。十五个月经验证明:只有主要依靠自力更生,同时不放松外援之争取,才是正确的道路。在这点上,过去经验也否定了速胜论。

总起来说,不论中国方面,敌人方面,国际方面,十五个月经验,都证明速胜论主张之毫无根据。相反,显露了战争的长期与残酷。因此,我们的战略方针,决不能是速决战,而应说是持久战。持久胜敌――这就是抗日战争的唯一正确方针。过去不相信这种方针的,现在事实给了明白教训,应该再没有疑问了!这就是十五个月抗战的第一个总结。

三、最后胜利是中国的,悲观论者毫无根据

抗战以前,唯武器论大张旗鼓,认为中国武器不如人,战必亡,中国必会作阿比西尼亚。抗战以后,这种议论表面没有了,他暗中流行着,抗战每至一紧张关头,这种论必兴风作浪一次,认为中国应该停战议和,不堪再战,再战必亡。我们则反,我们认为中国武器诚不如人,但武器是可以用人的努力加强的,战争胜负主要决定于人不决定于物。持久抗战的结果,依据于全民族的努力,中国必能逐渐克服自己的弱点,增加自己的力量,化被动为主动,化劣势为优势;同时敌人方面的困难必逐渐增加,国际方面对我之援助必逐渐扩大。综合这些因素,最后必能战胜日本帝国主义。…………(略)中国共产党亦早日指出:“日本在中国抗战长期消耗下,它的经济行将崩溃,它的士气行将颓靡,中国方面则抗战的潜伏力一天一天的奔腾高涨。所有这些因素和其他因素配合起来,就使我们能够对日本占领地作最后的攻击,驱逐日本的侵略者出中国。”(毛泽东与斯诺谈话)“我们相信:已经发动的抗战必将因为全国人民的努力冲破一切障碍的前进与发展。只要真的组织千百万群众进入抗日民族统一战线,抗日战争的胜利是无疑的。”(中共去年八月决定)所有这些,都被十五个月的经验证明了。悲观或亡国论者认为敌人强不可抗,中国不堪一战,妥协才是出路等等荒谬说法,已经证明是完全错的了。

理由何在?在于敌强我弱仅是一方面的事实,敌人尚有弱点存在,中国尚有优点存在。

甚么是敌人的弱点呢?第一,他是比较小的国家,他的兵力,财力不足,经不起长期的消耗。由于他的兵力不足,在中国的坚强抵抗面前又不得不分散与消耗,使他无法占领地区,亦实际只能占领大城市,大道与某些平原地带,其他仍然是中国。第二、敌人战争的性质是帝国主义的,是退步的。他们内部矛盾迫使他举行侵略战争,并迫使他釆用异常野蛮的掠夺政策。

这样,就使他的战争,一方面变为同整个中华民族绝对对立的战争,迫着中国无论甚么阶层,无论甚么党派,都不能不团结起来坚决抗战的。另一方面,变为同他本国人民大众绝对对立的战争。日本帝国主义悉率人财以应战争的结果,已经在他国内人民与前线士兵中间,逐渐酝酿了许多的不满,战争发展下去,无疑有迫使他的人民与士兵大众走上用坚决的方法反对战争本身的趋势。这些都在十五个月中已经开始表明了的。这一点,就是存在于敌人自己方面而使敌人必归失败的最主要的根据。第三,正是由于敌人战争的性质是帝国主义的;换句话说,损人利己的,就不得不把自己同一切和他利害相反的国家处于对立地位。除了两三个法西斯国家之外,一切国家,尤其是这些国家的人民大众,都不赞成日本的侵略战争。这样,就使得日本不得不益缩小其国际活动的范围,日益处于孤立地位。这也是十五个月中已开始证明了的。

这样,日本国度比较小,影响到他兵力,财力的不足;日本战争的退步性,日本国际地位的孤立;这三者同时结合在一块,成为日本战争中存在着的先天性的弱点与困难,而这些弱点与困难又正在日益发展之中。对于这些,亡国论者与悲观主义者是瞎子,他们全没有看见,而仅仅看见了敌强我弱的一点。所以亡国论与悲观主义在敌情方面并没有根据,因而他们的妥协政策只能是亡国政策,我们是最后胜利论者,我们的观点则在敌情方面有充足的根据,十五个月经验已经开始证明了。

甚么是我们的优点呢?第一,我们是大国,地大,物博,人多,兵多。虽然敌人占去了我们主要的大城市与交通线,然而我们还有大块土地作为我们长期抗战与争取最后胜利的根据地。即在敌占区域我们也还有许多游击队战争根据地。这个特点,是和小国如捷克,比利时等根本不同。这是我们的第一个优点。第二,我们今天的抗日战争不同于中国一切历史时期的战争,我们的战争是民族革命战争,是进步的战争,不但战争本身的性质是进步的,而且这个战争是在中国前所未有的进步基础之上进行的。十世纪四十年代的中国,不同于一切历史时期的中国,我们有了比之任何历史时期都不相同的进步的人民,进步的政党与进步的军队。在这基础之上进行的抗日民族革命战争,它本身包含着可能继续发展进步的伟大力量。这一点,就是存在于我们方面而使我们足以支持长期战争取得最后胜利的主要的根据。十五个月的经验证明,在原有进步基础上进行的伟大神圣的民族革命战争,已经推动了全中国的进行,旧的民族腐败传统是在破坏着,新的民族进步力量在生长着,一个全民族统一团结进步发展的伟大过程是在向前完成着。抗战以前的中国不同于抗战以后的中国,这是有眼睛的人都能看到的了。而抗战第一阶段(这个阶段目前尚未完结)的中国又将不同于抗战以后的中国,也已经可以预断。还有,第三,我们的抗日战争不是孤立的。不管资本主义国家现时还保存其许多矛盾政策,也不管国际局势可能暂时的形影到各国助我的程度,使中国抗日战争与世界反侵略反法西斯的斗争,是不分离的结合着。反对日本侵略战争的不仅是中国人,还有欧洲人,美洲人、澳洲人、以及其他亚洲人。十五个月来世界各国的同情与援助,给了我们以这种确信。主要的依靠自力更生的中国,能够同时配合着世界的援助,因为今天的世界,已不是从前的世界,整个世界先进人类已成为休戚相关的一体,敌人要使我们陷于孤立的企图,只会是徒然的。

这样,我们是一个很大的国家,我们的战争是进步的战争,我们又有国际的援助,这三者同时结合在一块,这些都是我们的有利条件,不但已经存在,并且日益发展之中。在这里,亡国论与悲观主义者同样是瞎子,他们一点也看不见,而只看我们是弱国,是劣势,是半殖民地这一点,喃喃发出其“抗战必亡”,“抗战必亡”的胡说;其中许多坏蛋就根据这种胡说暗地进行其投降妥协的阴谋。我们相反,我们要根据十五个月经验中已证明了的东西,向全党全国明确的指出我们国家于我们战争中长处与短处,有利条件与不利条件。并提出长处与有利条件在全战争中占居着优势,号召全国努力奋斗,发挥自己的长处,增强自己的有利条件,克服自己的短处与不利条件,为争取最后胜利而斗争。最后胜利将是谁的呢?我们确定地答复;中国的。在这个基础上决定我们的政策!坚决抗战还是动摇妥协呢?我们确定地答复:决不能有任何的动摇妥协,只有坚决抗战才是出路。东四省之沦亡,奥国之灭亡,捷国之瓜分,都并非因为抗战,这是有目其共见的。现在还是一样,在中国的许多优良条件下,抗战必兴;但如走妥协道路,则灭亡无可避免。因此坚决反对妥协论,反对悲观主义,唤起全民族奋战到底,乃是唯一无二的方针。

总起来说,敌强我弱这个矛盾着的对比,决定了战争的长期性,决定了持久战的战略方针,我们是持久胜敌论者,不是速胜论者。敌小,我大;敌人战争是退步的,我们战争是进步的;敌之国际地位比较孤立,我则比较能得外援:这几个矛盾着对比,又决定了战争的最后胜利定属于我,不属于敌,这就是十五个月抗战经验的第二个总结。

四、支持长期战争与争取最后胜利的唯一道路,在于统一团结全民族,力求进步,依靠民众。

抗日战争是长期的,最后胜利是中国的,这两个基本问题已从十五个月抗战经验中证明了,但支持长期战争与争取最后胜利的具体方案如何?则过去国人的意见不一致的。许多人认为照老样下去就可以了,他们不注意团结全国,不注意军事、政治、文化、党务、民运等各方面的改进,甚至加重磨擦,阻碍进步。我们从来不赞成这种意见,认为唯有全民族的统一进步,力求进步,依靠民众,才能支持长期战争与取得最后胜利,否则是不可能的。中国国民党在其抗战建国纲领中明白指出:“欲求抗战必胜,建国必成,固于转于本党同志之努力,尤须全国人民戮力同心共同负担。”中国共产党亦早已指出了“抗战时期最中心的任务,是动员一切力量,争取抗战胜利,而争取胜利抗战的中心关键,在使已经发动的抗战发展为全面的全民族的抗战,只有这种全面的全民族的抗战才能使抗战得到最后胜利。”(中共去年八月决定)这些都是完全正确的,十五个月经验已经证明了。

抗战以来,把国内各个互相对立的阶级、党派、集团都团结起来了,各个不同的区域,不同的军队,都统一于国民政府与军事委员会指挥之下,抗战十五个月的坚持,没有这种统一团结是不可能的。也只有抗战,才能统一团结各方面,这种团结一致就是抗日民族统一战线。但十五个月经验又向我们证明:敌人破坏阴谋之严重与内部团结巩固之不足。抗战为甚么遭受很多挫折,为甚么至今还不能停止敌之进攻,实行我之反攻?除了客观原因之外,统一战线力量之不足,统一战线还没有必要的扩大与巩固,是其最主要的原因,由此可知,只有更加统一团结全民族,巩固与扩大抗日民族统一战线,才能支持长期战争与争取最后胜利,这是第一。第二,十五个月抗战,不由推动了全民族的团结,同时又暴露了这个团结之不足;而且推动了军事、政治、文化、党务、民运各方面的进步同时又暴露了这种进步之不足。支持长期战争与争取最后胜利,必须发动全民族各阶层中一切生动力量,而欲达此目的,非从军事、政治、文化,党务、民运等各方面力求进步不可。没有各方面的更大的进步,就不能发动全民族一切生动力量,也就不能更进一步的统一团结全民族。第三,十五个月抗战又证明了民众援助抗战力量之伟大;同时也证明了民众力量之仅在开始发动,因而使抗战得不到民众的广大援助而遭受了许多挫折。从此得到教训,国人必须进一步的认识抗战依靠民众这个基本问题,依靠民众则一切困难能够克服。任何强敌能够战胜,离开民众则将一事无成。中国今后的进步,还必须充分表现在发动民众力量这一方面。

总之,支持长期战争与争取最后胜利的唯一正确道路,在于巩固与扩大全民族的统一团结,在于力求进步以发动全民族的生动力量,在于依靠民众以克服困难,这就是我们的第三个总结。

同志们,坚持抗战,坚持持久战,力求团结与进步一一这就是十五个月抗战的基本教训,也就是今后抗战的总方针。我们是能够战胜敌人的,只要我们与全国坚持这个总方针,并作了长期的扩大的努力。抗日战争正在向着一个新阶段发展,新阶段中有许多新的任务,但这个总方针是不变的,十五个月经验作了我们观察新的形势提出新的任务基础。

(三)抗日战争与抗日民族统一战线发展的新阶段

一、研究战争与统一战线的规律性是决定政策的基础

同志们,在我们总结了过去经验之后,重要的问题,在于看一看当前形势发展的趋向。抗日战争与抗日民族统一战线将会怎样从过去基础之上向前变化发展的?这是我们现在要答复的问题,这一点对于我们解决当前的问题有重要的意义。因为如果对于整个抗日战争变化发展的行程没有一个大概的估计,我们就只能跟着战争打圈了,让战争把自己束缚起来,而不能将其放在自己的控制之下,加之以调节整理,造出为战争所必须的条件,引导战争向我们所需要的方向走去,争取战争的胜利。因此必须懂得抗日战争的规律性,才能实现对于它的战略指导。才能决定为战争服役的一切战略,战术,政策,计划与方案。对于抗日民族统一战线也是一样,只有我们研究了与认识了它的规律性,我们才能有效的推动统一战线使之,进入巩固发展之途,而为战争的胜利起其支柱的作用。

我们现在先来说战争问题

二、特定的历史条件与主观能力的优劣决定战争的发展过程

历史上的战争有一个阶段就完结的,例如一九○五年的日俄战争,只有日军进攻,俄军败退,就结束了。又如:意亚战争,也只有意大利进攻,亚比西尼亚失败,就告结束。中国一九二六年开始反对北洋军阀的战争,也是一样。这是一种情形,这是由于出方面双方强弱不同,又一方面双方指导能力优劣不敌而造成的,这是第一类战争。第二类战争以两个阶段宣告完结。例如法俄战争,拿破仑从进攻到退却,俄国从退却到反攻,双方都有两个阶段。中国古代有名的吴魏赤壁之役,秦晋肥水之役,也是这样。虽则两军强弱不同,但弱者善于和用其他优良条件,给以正确指导,故于退却之后,接着反攻,战胜敌人,但是还有第三类战争,例如外国的七年战争,八年战争,三十年战争,百年战争,乃至二十年前四年的欧洲大战(特别表现于西战场)都有三个阶段;甲方进攻,乙方退却,为第一阶段;双方相持不决,为时甚长,为第二阶段;乙方反攻,甲方退却为第三阶段。中国历史上也有许多这类的战争。这类战争的特点,在于有一个较长的或很长的相持阶段,这也是由于特定的历史条件与战争指导集团的特性而造成的。

中日战争属于那一类战争呢?我以为属于第三类战争的。这是由于双方不同的历史条件与不同的战争指导集团之特殊情形而造成的。

三、中日战争的长期性表现于战争的三个阶段

中日战争的长期性将表现于在敌则进攻,相持,退却,在我则防御,相持,反攻,这样三个阶段之中。由于敌强我弱(敌是优势,我是劣势,敌是帝国主义国家,我是半殖民地国家),故出现了敌方进攻,我方防御的第一阶段。不说退却而说防御,是说以战略的运动防御即节节抵抗的姿态而表现其退却,不是一下子干脆退却。但又由于在敌则小国,退步,寡助,在我则大国,进步,多助这些特殊的条件,我之英勇抗战又使敌在进攻中受到分散的困难与消耗的损失,而不得不于一定时机结束其战略上的进攻,转入军事上保守其占领地而从政治上与经济封锁上向我进攻的阶段。此时敌虽消耗,但一时尚未消耗到使之转入失败的态度;我虽坚决抗战于各方面向前进攻,但一时也难进攻到足以转入反攻驱敌出国的态度。依上诸因,一个双方相持的第二个阶段,或中间阶段,就形成了。由于第二阶段中敌之困难与我之进攻俱日增,又配合着国际有利于我不利于敌之形势,就能使敌强我弱敌优我劣的原来状态逐渐发生变化,进到在全局看来日益于敌不利而有利于我之局面,先到敌我平衡,再到我优敌劣,彼时,就可转入我之反攻,敌之退却的第三阶段了。

上述三个阶段的看法,是依据敌我既存的与将来可能发生的双方相反对比之具体条件而作出的一种对于整个战争过程的估计,现在并不是事实,而是一种可能的趋势,要依我之主观努力,创造出为这种可能趋势所必要的条件,才能使可能趋势变为事实。然而依据既存条件,加上正确指导与全民族扩大而坚持的努力,是能够使这种可能趋势变为事实的。

四、速胜论者与亡国论者都反对这种估计

速胜论者反对阶段论,认为我能迅速反攻,无需乎一个中间阶段,这是不对的。抗日战争面前存在着许多困难,克服这些困难需要一定的时间,迅速反攻是不可能的,速胜论者的反对三阶段,是因为他们一方面过低估计了敌人力量,另方面又过高估计了自己力量的原故。亡国论者也反对三阶段,认为相持与反攻都是不可能的,中国只是一个亚比西尼亚。这是不对的。他们与速胜论者相反,过高估计了敌人力量,而过低估计了自己力量。在他们面前只有黑暗,承认敌人能够灭亡全中国,我之对抗与努力只是徒劳,办到敌我相持亦不可能,更不说什么反攻胜利了。因此,必须一方面反对速胜论,又一方反对亡国论,才能坚持我们的三阶段论。而在当前情况下,反对亡国论比之反对远胜论更加重要。另外一些入,口头上赞成持久战,但不赞成三阶段论。这也是不对的。所谓持久战,所谓长期战争,表现在什么地方呢?表现在战争的三个阶段之中。如果承认持久战或长期战争,又不赞成三个阶段,那末,所谓持久与长期就是完全抽象的东西,没有任何的内容与现实,因而就不能实现在任何实际的战略指导与任何实际的抗战政策了。实际上,这种意见仍属于速胜论,不过穿上“持久战”的外衣罢了。

五、三阶段论与国际形势的关系

当张鼓峰事件发生之时,国内一部分与兴高彩烈,以为日苏战争如果爆发,中国就可以转入反攻,无需乎要持久战了。在这种观点下,三阶段论当然不能成立,我们的估计是错误的了。这是主要依靠外援的思想,是速胜思想之一种。然而国际形势不是照着这些朋友们的主观志愿发展的,而是依照它自己的规律。世界的主要重心在欧洲,东方是环绕着它的重要部分。世界的主要和平阵线国家与主要法西斯国家,正在为着欧洲战争危机问题,在西方纠缠不清,无论是各大国间的战争前夜或战争爆发,西方的各大小国家都将以解决欧洲问题放在议程的第一位,东方问题则不得不暂时放在第二位。拿这种情况来看中日战争,迅速反攻的两阶段论也是没有理由的。我们必须以自力更生为主,我们不但不怕三阶段,而且正要造成三阶段。三阶段是中日战争的规律,不但在敌我力量对比上有其根据,而且也在国际形势上有其根据。

六、相持阶段战争的枢纽

三个阶段的主要特点,在于包含一个过渡的中间阶段。这就是说,第一,我之抗战必须用尽一切努力去停止敌之进攻,假如敌之进攻不能在一定时间与一定地区停止下来,就无所谓性质不同的三个阶段。第二,相持阶段出现了时,必须用尽一切努力去准备我之反攻所必需的一切条件,设若不然,就不能过渡到反攻阶段里去,而只是永远的相持,也无所谓三阶段。在这里,对于速胜论者,我们肯定地说:必须经过一个准备时期,才能团结全国,克服困难,生长新的力量,同时配合着敌人的困难,国际的援助,最后实行反攻,驱逐出国,否则是不可能的。拿主要依靠自力胜敌的观点来看问题,不可避免的要作出这个结论。对于亡国论者与悲观主义者我们肯定的说:只有这个过渡阶段,才是全战争的枢纽,中国化为殖民地还是获得解放,不决定于第一阶段中主要的大城市与交通线之丧失,而决定于第二阶段中,全民族努力的程度。大城市与交通线的丧失是可惜的,增加了敌人的力量,减少了自己的力量。然而很多没有丧失的东西尚可作为制胜敌人的资本,唉声叹气于实物的丧失是无益的。第一阶段中保存着的领土与各种新生力量,物别已经获取的军事,政治,文化,党务,民运等各方面的进步,是最可宝贵的,这是第二阶段中继续进步与准备反攻的基础,然而这仅仅是继续进步与准备反攻的基础,还不能决定反攻,决定反攻的东西是第二阶段中增加上述的力量,没有伟大的新生力量之增加,反攻只是空喊的。

七.三个阶段的特点,第一阶段

抗日战争三个阶段的特点,已经出现了的,尚未出现而可以预计得到的,有概略指明之必要,对于战导战争与决定政策有重要的关系。

第一阶段有些什么特点或重要标志呢?

有如下三方面的东西。

第一,中国方面:民族统一战线的形成,全国军队的参战,抗战的坚决性,国民党抗战建国纲领的发布,国民参政会的开会,共产党及其他党派的取得合法地位,游击战争的创造,全国军队的进步,民族运动的发展等等。这些都是中国方面表现进步的大事件。但同时,却又有许多不利事件与不良现象,例如:主要大城市,交通线和主要工商业的丧失,土地与人口的丧失,全国进步的不平等,(有些地方进步得非常之慢),政治制度之一般还仅在开始走向民主化,顽固份子与腐败现象的存在,妥协倾向的酝酿等等。

第二,敌人方面:军力财力的消耗,世界舆论的责备,军纪的败坏,军队战斗力的相对地减弱,国内人心与前线军心不满的酝酿,张鼓峰战争的失败,汉奸军队的难于组成及已经组成者的无能等等。这些都是表示其困难的大事件。但同时却又有表现其能力的东西,那就是:进攻的坚决性,军力的顽强,占领地的扩大,政治组织力的强韧,阴谋机关的有力等等。

第三,国际方面:援华运动的增长,苏联力量的壮大及其对于中国的援助,这些都是有利于中国的东西。但是还有别的东西:欧洲大战的酝酿,英日间某种程度的妥协倾向,各国军火原料的助敌,这些都于中国不利。

以上中国,日本,国际的许多东西,都是抗战第一阶段中十五个月来表现的特点,这些特点,将分别生其影响于新的阶段之中。

八、第二阶段

在假定武汉不守的情况下,战争形势又将出现许多新的东西。虽然敌占领武汉并不即等于旧阶段的完结,新阶段的开始,由现在敌人尚能继续进攻到他被迫停止进攻之时的这段时间,还是一个由旧阶段转向新阶段去的过渡期间。虽然如此,但武汉不守成为事实之后,就将发生许多新的情况。

武汉不守之后,以及新阶段的大部份时间,可以预计的基本情况,将是一方面更加困难又一方面则更加进步。这是新阶段中的基本特点。

更加困难将表现于下述各方面:(一)由于主要的大城市与交通线之丧失,国家政权与作战阵地就在地域上被敌分割了,由此将发生许多新的问题;(二)财政经济之异常困难;(三)英日种种程度的妥协倾向(或相反,在日本坚持独占与威胁南洋的条件下,英日有进一步冲突的可能);(四)如果敌攻广州,中国主要的海道交通被割断之虞,国际援助将暂时的部份减弱;(五)全国性伪政权有形成的可能及其对于抗日阵线的影响;(六)抗日阵线中部份叛变的可能,妥协空气的增长;(七)悲观情绪的生长,意见纷歧现象的增加等等,这些都是可盲目发生而将加诸抗日战争身上的困难事项。估计到这些之新困难,才便于有准备有计划地克服之。

更加进步将表现于下述各方面:(一)蒋委员长与国民党的坚持抗战方针及其在政治上的更加进步;(二)国共联系的改善,抗日民族统一战线的巩固与扩大;(三)军队改造工作的进步;(四)游击战争的广大发展;(五)国家民主化的进步;(六)民众运动的更大发展;(七)新的战时财政经济政策的实施;(入)抗战文化教育的提高;(九)苏联援助的继续与可能增加及中苏关系的更加亲密等等。

整个第二阶段即相持阶段,是中国准备反攻的阶段。其时间长短,依敌我力量变化的程度及国际环境如何而定。但我们应该准备长期战争,熬过这一段艰难路程,胜利的坦途就到来了。

第二阶段中虽然敌我在战略上是相持的,但仍有广泛的战争,主要表现于主力军在正面防御,而广大游击战争则发展于敌人的后方。那时,游击战争在许多重要战略地区将变为非常艰苦的战争,现在就应该准备对付这种艰苦。

九、第三阶段

具体情况不能预计。但彼时必是我之反攻条件业已准备完毕,同时敌之困难程度大大增加起来,国际形势又大大于我有利,彼时战争形势,不是战略防御或战略相持,而是战略反攻了;不是战略内线,而是战略外线了。彼时国内政治上必须有大的进步,军事上必须有新式技术,否则反攻是不可能的。

十、保卫武汉是争取时间问题不是死守问题。

保卫武汉斗争的目的,一方面在于消耗敌人,又一方面在于争取时间便于我全国工作之进步,而不是死守据点。到了战况确实不利于我而放弃则反为有利之时,应以放弃地方保存军力为原则,因此必须避免大的不利决战,战略决战在一二两阶段中都是不应该有的,都足以防碍抗战的坚持与反攻的准备,因此必须避免。避免战略决战而力争有利条件下的战役与战斗的决战,应是持久战的方针之一,于必要时机与一定条件下放弃某些无可再守的城市,不但是被迫的不得已的,而且是诱敌深入,分散,消耗与疲惫敌人的积极的政策。在坚持抗战而非妥协投降的大前提下,必要时机放弃某些据点,是持久战方针内所许可的,并无为之震惊的必要。

十一、由目前过渡到相持阶段。

只有停止敌之进攻,才有利于我之准备反攻。而要达此目的,还须给一个大的努力。故由目前过渡到敌人被迫停止其战略进攻,转入保守其占领地,出现整个敌我相持的阶段之时,还是一个斗争的过程,须克服许多困难才能达到。因为敌在占领武汉之后,还不会立即结束其进攻,他必定还想,西安、宜昌、长沙、衡州、梧州、北海、南昌、汕头、福州等地及其附近地区进攻。我要停止敌之进攻,还须针对着敌人企图继续执行战略的运动防御战,用极大努力进行坚持的战斗,再行大量的消耗敌人而又不为敌人所算,使敌之进攻不得不停止,把战局过渡到敌我相持的有利局面。

十二、但相持局面快要来到了

敌人占领武汉以后,他的兵力不足与兵力分散的弱点将更形暴露了。如果他再要进攻西安、宜昌、长沙、南昌、梧州、福州等地并作占领之企图,他的兵力不足与兵力分散的弱点所给予他的极大困难,必将发展到他的进攻阶段之最高度,这就是我之正面主力军的顽抗与我之敌后庞大领土内游击战争的威胁,所加给敌人兵力不足(他不能足)与兵力分散(他不能分散)现象上的极大困难。这一形势――在敌则兵力不足与兵力分散,在我则正面防御与敌后威胁,这是敌之极大劣势,我之极大优势。当然,在整个敌我力量对比上说来,敌强我弱敌优我劣的基本形势并未变化,这只有在长期相持阶段内我用全民族的极大努力,并配合国外条件,才能使之变化,然而敌在进攻武汉的战斗中,他之强的力量已经进一步发挥了。这种强的力量之进一步发挥,一方面固然给了我们以损失,然而同时就给了他自己以困难。因为敌之强的力量(同时即是其不足的与分散的力量)在其作了进一步发挥之后,气力势将衰退下去,就不得不使其总的战略进攻接近了一个顶点。我们承认敌之进攻还有一点余威,并最好与最恰当的是估计到他的这点余威还相当的大,因此还有充分可能他要攻略西安、宜昌、长沙,南昌,梧州,福州等处及其他附近地区,甚至要准备他向着整个粤汉路与西兰公路之进攻。然而这在总的敌人力量上将只是一点余威。在日本的整个国力上说来,他要北防苏联,东防美国,南对英法,内镇人民,他只有那么多的力量,可能使用于中国方面的用的差不多了。并且在其正面与占领地内必须对付广泛战争还依然存在,日苏,日美,日英,日法之间的矛盾在增长着,国内政府与人民的矛盾,前线官长与士兵的矛盾,大量支出与财政竭镢的矛盾在加深着,这些都是使得敌人大大皱眉的地方。我们及全国人民必须看到这些地方,不为主要大城市与交通缓之丧失所震惊,赞助政府调整全国之作战,有计划地部署粤汉路、陇海路、西兰公路及其他战略地区之作战,部署庞大敌后地区之游击战争,捉住敌人兵力不足与兵力分散的弱点,给以更多的消耗,促使更大的分散,使战争胜利地与确定地转入敌我相持的新阶段,这是全国当前的紧急任务。

十三、敌力在逐渐减少我力在逐增加中

敌人是否增加了力量呢?就其原有的力量来说,没有甚么增加,相反,他的力量大大的减少了,敌人原有的军力与经济力,是大大消耗了,十五个月战争中,他的军力伤亡了数十万人,消耗了大量的武器,弹药,与军用资材,毁灭了数百架飞机与百余艘军舰,支出了数万万元经费,这个消耗在日本历史上是空前。直到他被迫停止其战略进攻之时为止,他还要消耗一大批力量。在这点上,他的盟友希特勒早已大大地发起愁来了。然而敌人毫无力量增加吗?有的,这就是对于中国主要的大城市与交通线及部份乡村之占领,从各国手里及中国民族资本家手里夺取了市场,从中国手里夺取了资源与生产工具,夺取了许多人力,这些都是日本战争之唯一目的。他是暂时的增加了这些东西。然而问题是,这些东西能抵偿已有的战争消耗吗?不能,消耗了全部战争“投资”是已经消耗了,他要取偿还需付以新的生产投资。问题又是:拋开日本生产投资之无能不说,假定他能的话,他能补偿其战争消耗吗?也不能,因为往后依然存在着广大战争,依然每天要消耗。只要有广大的敌后游击战争存在,例如现在华北的游击战争那样,他的补偿是很困难的。由于不断的战争,他将不但不能补偿旧的,并且还须支付新的,只要我们的抗战坚持下去,日本的这条可怜命运是大体确定了的。我们说日本在第二战略阶段即相持阶段中将逐渐化强为弱,化优为劣,这种继续消耗是决定一方面,现在说到中国方面,中国力量究竟是减少了呢?还是增加了呢?我们的回答是减少了,又增加了。减少的是原有力量的质与量,这表现在军队人员武器的消耗,人口,工业、土地与资源等损失等方面,这是使得我们感到困难的重要的一方面。然而不是没有增加的,增加的是新的质与量,这表现在全国的团结,军队的进步,政治的进步,文化的进步,人们觉悟程度与组织程度的提高。主力军虽后退了,游击战争却前进了。一部份地方虽损失了,另一部份地方却进步了,问题是在:增加的程度今天还不够,今天还不够停止敌之进攻,今后更不够实行我之反攻,因此发生了必须用广大持久的努力去增加新的力量的问题。而这种增加,即全民族各个阶层中生动力量之更大发动与党政军民各方面之更大进步,甚于今后之广大持久的努力是完全可能的。在主要的依靠自己生长的力量。再配合之以敌人困难之加重,国际助我之增强,就能使整个敌我形势发生变化,由敌优我劣之现时形势,先走到敌我平衡,再走到我优敌劣,这就是长期相持阶段中必须解决也可能解决的根本问题。

十四、敌据城市我据乡村,所以战争是长期的,但乡村能够最后战胜城市。

于是问题在:敌人占领中国主要的大城市与交通线后,敌据城市以对我,我据乡村以对敌,乡村能够战胜城市吗?答复:有困难,但是能够的。抗日战争的长期性,不但由于敌是帝国主义国家,我是半殖民地国家,而是由于这个帝国主义又复占据我之城市,我则退至乡村以抗敌,因而造戊了长期性,速胜论在此是毫无根据的。然而今天中国的城市乡村问题,与资本主义外国的城市乡村问题有性质上的区别。在资本主义国家,城市在实质上形成上都统制了乡村,城市之头一断,乡村之四肢就不能生存,不能设想。在英、美、法、德、日、意等国,不能支持长期反城市的乡村农民战争。半殖民地小国也不可能。半殖民地大国如中国,在数十年前也困难。半殖民地大国如中国,在今天,却产生了这种可能。这里明显的是三位一体的条件。第一是半殖民地条件。在半殖民地,城市虽带着领导性质,但不能完全统制乡村,因为城市太小,乡村太大,广大的人力物力在乡村不在城市。第二是大国的条件;失去一部,还有一部;敌以少兵临大国,加以我之坚强抵抗,就迫使敌人发生了兵力不足与兵力分散的困难,这样就不但给了我一个总的抗日根据地,即大后方,例如云、贵、川等地,使敌无法占领。第三是今日的条件,如果在数十年前中国被一个强大帝国主义国家武装侵占,例如英占印度那样,那是难免亡国的。今天则不同,今天主要的是中国进步了,有了新的政党,军队与人民,这是胜敌的基本力量。其次是敌人退步了,日本帝国主义的社会经济发展过程已临到衰老的境界,日本资本主义的发展造成了与造成着把他自己送进坟墓的条件。又其次是国际形势变化了,旧的世界接近死灭,新的世界已现曙光。这些道理,我在“论持久战”中已详细说过了。总之,在今天的半殖民地大国如中国,存在着许多优良条件,利于我们组织坚持的长期的广大的战争,去反对占领城市的敌人,用犬牙交错的战争,将城市包围起来,孤立城市,从长期战争中逐渐生长自己力量,变化敌我形势,再配合之以世界的变动,就能把敌人驱逐出去而恢复城市。毫无疑义,乡村反对城市就在今天的中国也是困难的,因为城市总是集中的,乡村总是分散的,敌人占领我主要的大城市与交通线之后,我之行政区域与作战阵地就在地域上被分割,给了以我们很多困难,这就规定了抗日战争长期性与残酷性。然而我们必须说,乡村能够战胜城市,因为有上述三位一体的条件。在内战条件下,极小部分的乡村又支持了长期反对城市的战争,还当帝国主义各国一致反共的时期。谁能说在民族战争条件下,又当帝国主义阵营分裂之时,中国以极大部分的乡村,不能支持长期战争去反对城市敌人呢?毫无疑义是能够的,并且现在的所谓乡村,与内战时期的乡村有很大不同,不但地域广大;而且在云、贵、川等省大后方中,尚有许多城市与许多工业,尚可与外国联络,尚可建设。依据于大后方的保持与敌后游击战争根据地的建立,从长期中生息我之力量,削弱敌之力量,加上将来国家有利条件之配合,就能举行反抗,收回城市。……(略)战争虽困难,胜利前途是存在的。

十五、妥协危机严重存在,但是能够克服的。

我们早就说过,一部分患着恐日病的人们时刻企图动摇政府的抗战决心,主张所谓和平妥协,过去曾见之于南京失守之后,现在又在蠢蠢欲动了,这是敌人阴谋在抗日阵线内部的反映。这种危机是严重存在的,国人必须充分注意,不让亲日派得售其奸。亲日派的企图和敌人的企图是一致的,必然集中于反蒋反共。假令得售其奸,抗战的前途就成大问题。因此,全国上下憬然觉悟于敌人阴谋与内部反间之可畏,自动自觉地努力反对这种阴谋,一刻也不放松。这种危机是否能够克服呢?那是能够的。在国共两党及一切爱国志士团结一致并作了必要的努力之后,克服妥协危机,驱除助敌张目的妖魔鬼怪,而把抗战坚持下去,不但是绝对必要的,而且是完全可能的。因为亲日派究竟没有多大的势力,抗日派的势力大于亲日派。

十六、相持阶段中游击战争的新形势

新阶段中,正面防御的是主力军,敌后游击战争将暂时变为主要的形式。但敌后方游击战争在敌我相持的新阶段中,将釆取一种新形势发展着,什么是游击战争的新形势呢?即第一,在广大地区中仍然能广泛的发展。这是因为在我则土地广大,在敌则兵力不足与兵力分散,只要我能坚持发展游击战争的方针并正确的指导之。敌要根本限制我之发展是不可能的。但第二,在某些重要战略地区,例如华北长江下流一带,势将遇到敌人的残酷的进攻,平原地带将难于保存大的兵团,山地将成为主要的根据地,某些地区的游击部队可能暂时的缩小其数量,现在就应准备这一形势的到来。在现在,为了策应正面主力军的战斗,为了准备转入新阶段,应把敌后游击战争大体分为两种地区,一种是游击战争充分发展了的地区如华北,主要方针是巩固已经建立了的基础,以准备新阶段中能够战胜敌之残酷进攻,坚持根据地。又一种是游击战争尚未充分发展,或正开始发展的地区,如华中一带,主要的方针是迅速的发展游击战争,以免敌人回师时游击战争发展的困难。在将来,为了配合正面防御使主力军得到休息整配机会,为了生长力量准备战略反攻,必须用尽一切努力坚持保卫根据地的游击战争,在长期坚持中,把游击部队锻炼成为一支主力军,拖住敌人,协助正面,一般说来,新阶段中敌后游击战争比较前一阶段要困难得多的,我们必须预先见到这种困难,承认这种困难,不可因为前一阶段发展的容易而冲昏了头脑,因为敌人一定要转过去进攻游击战争。然而是能够是坚持的,一切敌后工作的领导人们必须要有这种自信心。因为民族战争中的游击战争,不论敌人是如何的强,总比内战时的条件优良得多。在这里,争取与瓦解伪军以孤立日势,是非常重要的任务。

十七、抗日战争发展的新阶段同时即是抗日民族统一战线发展的新阶段

以上说的都是抗日战争的形势问题,以下说到抗日民族统一战线的形势。

抗日战争发展到了新阶段之时,同时即是抗日民族统一战线发展到一个新的阶段之时。由于新阶段中将遇到比前更多的困难,抗日民族统一战线也就应该适应这种情况而表现其向困难斗争并将任何困难战而胜之之伟力。为了在目前过渡期间以及到了将来的新阶段,中国抗日民族统一战线不是表示其对于困难之无能,而是表示其具有克服困难之伟力:就必须认真的巩固统一战线与扩大统一战线,长期的战争必须有长期的统一战线来支持,战争的长期与统一战线的长期性,是不能分离的。

十八一一二十(略)

二十一、中国抗日民族统一战线的特点

由于中国的历史原因,使得今天中国的抗日民族统一战线,不同于任何外国的统一战线,如人民阵线等,也不同于中国历史上的统一战线,如第一次国共合作等。它有今天的中国的特点,认识这些特点,对于巩固与扩大抗日民族统一战线,是有非常严重的意义的。

这些特点是甚么呢7归结起来,共有八个,即是:全民族抗日的,长期性的,不平衡的,有军队的,有十五年经验的,大多数民众尚无组织的,三民主义的,处于新的国际环境中的。

首先是全民族抗日的。这个特点规定了我们统一战线的根本性质。一方面,我们统一战线的目的,是为了反对侵入国土的异族日本帝国主义而建立起来与发展起来的。又一方面我们统一战线的组成,是包括全民族所有不同的党派,不同阶级,不同军队,国内不同民族之一个最广大团体。由于是反对异族侵略的,所以组织成份能够如此广大。由于组织成份异常广大,所以这一个统一战线具有伟大的力量;但同时,统一战线内部又难免许多相互间的磨擦,而须恰当地调整之,才能达到团结对外之目的。我们统一战线的这种最基本的特点一一政治目的反对异族侵略与组织成份之异常广大,不同于法国与西班牙的人民阵镕,也不同于第一次大革命时期的民族阵线(当时的国共合作),使得今天的统一战线产生了许多的特殊内容与特殊结果,这是今天统一战线的第一个特点优点,虽然在其组织复杂一方面不免同时包含着缺点。

第二是长期性的。这个特点是从第一个特点产生的。由于这个统一战线是用民族战争来反对日本帝国主义的,而日本帝国主义是一个强的帝国主义,就产生了抗日战争的长期性,因而又产生了统一战线的长期性。这一点,我在报告的第五部份还要说到的,这是一切政策的出发点。这一点也和第一次国共合作不相同。

第三是不平衡的。由于历史原因,造成了各党各派各阶层政治力量的不平衡,同时在地域的分布上也表现这种不平衡。国民党是第一个具有实力的大党,共产党是第二党,其他又在其次,这一情况,产生了许多特殊的东西。

第四是有军队的。国共两党都有军队一一这个特殊历史条件造成的结果,不是缺点而是优点,由于有两党的军队,使得抗日战争中两党克尽分工合作最善责任,互相观摩激励的好处也更多了,这一点和西班牙相同,但和法国不同,和第一次国共合作也不同,这也是使得两党能够长期合作的因素之一。

第五是有十五年经验的。一九二四年至二七年第一次国共合作的四年,一九二七至一九三六年国共分裂的九年,现在国共重新合作又有了两年,这个十五年中合作一一分裂一一又合作的经验,最深刻地教育了国共两党,其他党派与全国人民,结论是:只应合作,不应分裂。这也是长期合作的基础之一。这种宝贵经验,世界各国都没有,第一次国共合作时也没有的。

第六是大多数民众尚无组织的。这是中国的特点,西洋各国与此不同,所以是一个缺点,使得统一战线缺乏现成有组织的民众基础。但同时,各党之间可以分工地去组织民众,不须挤在一块老是磨擦,因为有的是尚无组织的民众,正待组织起来以应抗战之急需。

第七是三民主义的。抗日民族统一战线以三民主义为政治基础,不但是合作抗日的基础,而且是合作建国的基础。三民主义中的民族主义将引导这个合作到争取全民解放,其民权主义将引导这个合作到彻底的建立民主国家,其民生主义则更可能引导这个合作到很长的时期,三民主义的政治纲领与政治思想保证统一战线的长期性。

第八是处在新的国际环境中的。今天的世界政治经验条件,比之第一次合作与两党内战两时期都不同。今天只有一部份帝国主义国家如日、德、意反对国共合作与抗日民族统一战线,另一部份帝国主义国家,由于他们与日本的矛盾,现时也并不反对我们的统一战线,反而采取赞助的立场。所有国家的先进人民都是赞助我们的。苏联更是诚执的赞助。这种新的国际环境,对于我们的长期合作有重大影响。

深刻地研究与认识上述这些特点,才能釆取恰当的政治上的政策与工作上的态度,不是头痛医头,脚痛医脚地应付政治问题与工作问题,而是站在科学的基础上正确地解决问题。抗日战争的胜利与抗日民族统一战线的巩固与扩大,是需要这种科学基础的。

二十二、统一战线的新形势

抗日战争的新阶段中,抗日民族统一战线必须以一种新的姿态出现,才能应付战争的新局面,这种新姿态,就是统一战线的广大的发展与高度的巩固。十五个月团结抗战的教训,将促使各党认识这种发展与巩固之必要。发展方面,是扩大各党的组织与扩大民众的组织。巩固方面,是各党采取新的政策与新的工作,减少相互间的磨擦,办到真正的精诚团结,共赴国难,抗战新阶段中存在着许多的困难,唯有统一战线各党扩大发展其组织与高度巩固各党的内部关系与各党之间的相互关系,才能有力地执行新的政治任务,战胜新的困难,达到停止敌之进攻与准备我之反攻之目的。这就是抗日战争新形势中统一战线的新形势,由于各党的共同努力与全国人民的热烈拥护,这种新形势的到来是完全可能的。

(四)全民族当前的紧急任务

根据过去抗战的总结与当前抗战之统一战线发展新阶段的估计,全民族的当前紧急任务应该是什么呢?应该和过去有些什么不同呢?

总的任务应该是:坚持抗战,坚持持久战,巩固与扩大统一战线,以便克服困难,停止敌之进攻,准备力量,实行我之反攻,达到最后驱逐敌人之目的。

分别说来,有如下各方面的具体任务。一切抗日民族统一战线的组织成份,应该赞助政府,并在政府领导之下,动员全民族实行起来,共产党员应成为执行这些任务的模范。

一、高度的发扬民族自尊心与自信心,坚持抗战到底,克服悲观情绪,反对妥协企图。

估计到新的抗战形势下,必有一部份人,因为主要大城市与交通线的丧失,财政经济的困难,国际援助的不及时,因而发生着与增长着对于抗战前途悲观失望的情绪。而日寇,汉奸,亲日派,必将利用这种情绪,大放其和平妥协空气,企图动摇我抗战的决心。因此,全民族的第一任务,在于高度发扬民族自尊心与自信心,克服一部分人的悲观情绪,坚决拥护政府继续抗战的方针,反对任何投降妥协的企图,坚持抗战到底。这一任务,此过去任何时期为重要。

为此目的,必须动员报纸,刊物,学校,宣传团体,文化艺术团体,军队政治机关,民众团体,及其他一切可能力量,向前线官兵,后方守备部队,沦陷区人民,全国民众,作广大之宣传鼓动,坚定地有计划地执行这一方针,主张抗战到底,反对投降妥协,清洗悲观情绪,反复地指明最后胜利的可能性与必然性,指明妥协路线是灭亡,抗战才有出路,号召全民族团结起来,不怕困难,不怕牺牲,我们一定要自由,我们一定要胜利,用以达到全国一致继续抗战之目的。

为此目的,一切宣传鼓动应顾到下述各方面,一方面,利用已经产生并正在继续产生的民族革命典型(英勇抗战,为国捐躯,平型关,台儿庄,八百壮士,游击战争的前途,慷慨捐输,华侨爱国等等)向前线后方国内国外,广为传播。又一方面,揭发,清洗,淘汰民族阵线中存在着与增长着的消极性(妥协倾向,悲观情绪,腐败现象等等),再一方面,将敌人一切残暴兽行的具体实例,向全国公布,向全世界控诉,用以达到提高民族觉悟,发扬民族自尊心与自信心之目的。须知没有这种觉悟与自信心,是大大妨碍着克服困难与准备反攻的基本任务的。

二、(略)

三、提高主力军的战斗力,保卫华中华南与西北,停止敌之进攻。

针对着敌人现时进攻武汉并继续进攻华南与西北之企图,全民族的第三个任务,在于提高主力军的战斗力,整理现有军队,增编新的军队,为保卫华中华南与西北而战,停止敌人进攻。为此目的,中国主力军方面:第一,必须发展高度的运动战,同时补助之以必要的与可能的阵地防御,节节抗拒敌人,消耗敌之实力。第二,必须在大后方建立可能的军事工厂,并提高制造能力,接济前线的枪械与弹药。第三,必须在军队中认真实施民族革命的政治工作,实施政治文化娱乐等教育,提高全军英勇奋斗持久苦战的精神。第四,就现有物质基础改善士兵生活,在连队中组织经济委员会,由士兵管理伙食。第五,提倡自觉纪律,废止打骂制度,提倡官兵之间的亲密团结,以改善官兵关系。第六,实行公买公卖,待人和气,不强迫征粮,不强迫拉夫,不强迫当兵,改取政治动员方式解决食粮,夫役与新兵问题,以改善军民关系。第七,在前线各军之间,前线与后方各军之间,提高友爱互助精神,作战则互相策应,工作则互相观摩,消除互相观望互相嫉妒等不良现象,以改善各军之间的关系。第八,整理现有军队,补充缺额,同时增编新的军队,加紧教育训练,以利持久作战。用这一切办法,提高主力军的战斗力,为保卫华中华南与西北而战,为停止敌之进攻与准备我之反攻而奋斗。

四、广大地发展敌后游击战争,创立和巩固我之根据地,

缩小敌之占领地,配合主力军作战。

针对着敌之目的在于还要继续向我进攻,又将于一定时机抽兵进攻游击战争,企图巩固其占领地,使中国反攻困难,全民族的第四个任务。就在于广大地发展敌人后方的游击战争,创立许多根据地,巩固已有的根据地,用以缩小敌之占领地,在目前,配合主力军为停止敌之进攻而战,在将来,配合主力军为实行反攻而战。半殖民地中国抗日民族战争的重要特点之一,在于游击战争的广大性与长期性,没有这种游击战争,便不能牵制大量敌军,有力地配合正面主力军之作战,而停止敌之进攻;便不能使敌人占领地限制于一定地带,使之无法全部占领:便不能在敌人后方建立多数的抗日堡垒,坚持游击战争,并准备将来配合主力军之战略反攻。因此,第一,必须广大地发展一切敌人后方地带的游击战争,并创立多数的游击战争根据地,巩固已经建立起来的根据地。第二,必须依照华北榜样,留置或派遣足够数量之正规军队于敌后各个战略区域,作为长期坚持游击战争的骨干。这些军队应该逐步学会游击战术,加强政治工作,发展民众运动,创立根据地,并帮助敌后民众游击队与游击战,争逐渐提高到正规军战争的道路上去。第三,一切战区与敌人后方,必须发动所有男女人民卫国保乡的热忱,除动员他们大批加入脱离生产的游击队与补充留置敌后的正规军外,把他们组织到半军事性质的抗日人民自卫队中去。抗日人民自卫队的组织,应成为一切战区与敌人后方的普遍与经常的民兵制度,他们是不脱离生产的。第四,必须协助人民组织广泛的游击队。这是脱离生产的各县各区应该都有,成为袭击敌人保卫地方的普遍的小队伍。第五,必须建立游击部队中的政治工作,加强其军事政治文化娱乐的教育,用以提高其战斗力。第六,必须建立游击部队中正确的军事政治制度,实行官兵的平等待遇,经济公开。第七,改造土匪部队,使他们走上抗日!肃清我军后方的及被敌利用的土匪。第八,游击战争的军火接济是一个极重要问题,一方面,大后方尽可能的接济他们;又一方面,每个游击战争根据地都必须尽量设法建立小的兵工厂,办制自制弹药,步枪,手榴弹等的制度;使游击战争无军火缺乏之虚。第九,依照敌情与我之战略需要,重新划分敌后各地作战区域与行政区域,使之适合新的战争情况。第十,必须依照战略需要,统一敌后各部队与行政区之领导,以便集中抗敌力量,消除内部矛盾;但应反对互相吞并的军阀行为。

五.提高军事技术,创立机械化兵团,准备反攻实力。

敌以不及我数倍之兵力而能节节深入者,除了我之政治原因外,我之技术落后是主要原因。针对着敌之长处与我之短处,全民族的第五个任务,在于提高军事技术,增加飞机大炮,战车等数量与使用人材,为着实行反攻而准备实力,为此目的,一方面须就现有的及可能继续增加的制造能力从部份制造与修理开始,认真从事这个工作。另一方面,多方设法从外国输入新式武器,用以逐渐改善军队的装备,创立真正现代化的机械兵团。毫无疑义,我们应该从实际看问题。在现在,实际上战斗着的是大数量的低级武器,因此,我们应该号召全国军队与人民武装,相信低级武器也能胜敌,提高政治精神,改善作战方法,发展游击战争,以补新式技术之不足。不在这方面着重致力,我们就忽视了当前的实际问题,无以克服目前的困难。在将来,为着准备战略反攻,非提高新式技术建设新式军队不可,须知没有现代新式技术装备的足够数量的军队,要实行反攻,收复失地是不可能的。不在这方面提高注意力,并就可能范围内认真开始去做,我们就只看见现在,忘记了将来,无以克服目前的困难。在人力物力丰富的中国,只要政治条件改善,动员方法进步,加之以外国的协助,逐渐改善技术装备,决不是不可能的。

六、实行集中领导下的民主政府,密切政府与人民的联系,发挥抗日政权的最大能力。

敌人乘我弱点之处,不但在军事,而且在政治,在我政治制度之不民主化,不能与广大人民发生密切的联系。为补救此弱点,全民族的第六个任务,在于实行集中领导下的民主制度,没有这一方面的改进,要最后战胜日寇也是不可能的。民主政治是发动全民族一切生动力量的推动机,有了这种制度,全国人民的抗日积极性将会不可计量地发动起来,成为取之不竭的深厚渊源。我全民族彻底地统一团结的伟大过程之完成,也只有依靠民主制度之建立。关于这一点,须从各方面实际有所施设。第一,国民参政会的开会已开始了国家民主化的第一步,尔后应使该会工作公开的顺利的进行,该会议决事项应全部付之实施,并依据该会已经决定的方案认真的建立各省各级地方参政会,推进民主政治。第二,保证抗战建国纲领所规定的人民言论,出版,集会,结社,信仰等自由权在全国范围之充分实施,这种自由是在抗战建国范围之内的,只有充分地保证了这种自由,才利于普遍发挥抗日建国的力量。这里问题是保证中央法令在各地方之实施而不受地方之随意的限制。应该限制的只是不利于抗日建国的那种自由,即汉奸,亲日派的自由,其他都不应在限制之例。第三,从战区与敌人后方开始实行多量的民主制。例如:民选各级地方政府再由上级加以委任。战区各级政府采取民主集权的委员制,并设立各级人民代表机关。战区政府增设某些必要的工作部门;改变公文程式;清除贪污腐化无能份子,吸收抗日积极份子;减低薪俸,提倡艰苦生活,用以适合战区艰苦,复杂与浅动的环境,战区地方政府在中央政府统一领导下,有颁布地方单行法令之权。战区男女公民,除汉奸外,均有选举被选举权,均有言论,出版,集会,结社与武装抗日之自由。战区一切抗日党派均有公开合法地位等等。在战区尤其在敌人后方,没有这些政治改革,要支持长期艰苦的抗日战争是不可能的,以上这些办法,都是为着密切政府与人民的联系,增加政府的实力,使之能在抗日战争中起其最大作用。没有问题,全国任何地方政府,应集中于中央政府领导之下,不应因行政区域在地域上之被敌分割而有任何不尊重中央领导的表现。全国必须是依照中央发令而推行民主制的,但全国必须是统一于中央的。

七、扩大统一的民众运动,全力援助战争。

全民族的第七个任务,在于扩大各种民众运动,并使之统一起来,全力援助战争。长期艰苦的抗日战争,一切须取给于民众,没有普遍发展的并全国统一的民众运动,要长期支持战争是不可能的。尤其在战区与敌人后方,亟须这样做,抗日战争正在遇到新的困难,唯有动员民众。才能有效地克服这些困难。在全国,尤其在战区与敌人后方,亟须做到下列各项:第一,保障一切抗日民众团体与抗日运动的自由,确立民众团体在法律上的地位。第二,物质上帮助民众团体,尊重民众团体的独立性。第三,认真建立有广大群众参加的工人、农民、青年、妇女、商人、自由职业者、文化人与儿童的各种救国会,并使之依照地域与职业两种原则建立联合的组织。第四,发动民众积极参加各方面的抗战工作,积极援助政府与军队,尤其在战区不可和缓。

八、改良民众生活,激发民众的抗战热忱与生产热忱。

改良民众生活问题,过去实行太微弱了,因此不能激发广大劳动人民的抗战热忱与生产热忱,对于坚持长期战争是非常不利的。因此,今后全民族的第八个任务,在于实行下列各项改良民众生活的政策:第一,优待抗日军人家属与残废的抗日军人。第二,救济战区灾民难民及失业工人。第三,在战区及敌后开始适当的减租减息,承认工人农民对雇主地主的团体契约权。第四,禁止雇主、地主,师父、工头等对工人学徒的虐待打骂。实行这些初步的生活改良办法之后,必能提高工农贫民群众拥护政府,参加战争与参加增产的积极性,不但战争需要的一切动员帮助将大大改观,而且工业农业生产的数量质量与商业的流通也会大大增加与提高起来,国家财政也就在新的农工商业基础之上而得到满意的解决。

九、实行新的战时财政经济政策,渡过战争难关。

主要的大城市与交通线丧失之后国家财政经济必大现困难,没有新的有效的办法,便无以渡过战争的难关。然而只要实行新的政策,动员人民力量,便任何困难也能够克服。因此,全民族的第九个任务,在于实行一种新的战时财政经济政策。主要事项如下:第一,新政策以保障抗日武装部队一切必要供给,满足人民必需品的要求,并和敌人的经济封锁与经济破坏作斗争为目的。第二,有计划的在内地重新建立国防工业,从小规模的急需的部门开始,逐渐发展改进,吸收政府,民间与外国三方面的资力;并从政治上动员工人,保障其最低限度的物质待遇,改良工厂管理制度,以提高生产率。这些,不但是必需的,而且是可能的。第三,用政治动员与政府法令相配合,发展全国农业与手工业生产,组织春耕秋收运动,使全国农业手工业在新的姿态下发展起来。在战区注意保护农具牲畜及手工作坊,保证被隔断区域的经济自给。第四,保护私人工商业的自由营业,同时,注意发展合作事业。第五,在有钱出钱原则下,改订各种旧税为统一的累进税,取消苛杂和摊派制度,以舒民力而利税收。第六,用政治动员与政府法令相配合,征募救国公债,救国公粮,并发动人民自动捐助经费及粮食,供给作战军队,以充实财政收入。第七,有计划的与敌人发行伪币及破坏法币的政策作斗争,允许被隔断区域设立地方银行,发行地方纸币。第八,厉行廉洁运动,改订薪饷办法,按照最低生活标准规定大体上平等的薪饷制度。第九,由国家银行办理低利借贷,协助生产事业的发展及商品的流通。第十,恢复与发展战区的邮电交通。以上所指,不过大端,必须有认真改革旧制实行新制的决心,并持之以毅力,才能消除新的困难,支持长期战争,其重心在于组织广大人民的生产积极性,使之为着战争供给而效力。中国的抗战是在一种特殊情况之下进行的,主要的大城市与交通线被敌占领,抗战的主要依靠是乡村与农民。农民是有伟大力量支持战争的,但须实行必要的政治方面与经济方面的改革。这里所说各项新政策,就是根据这种特殊情况而提出的。

十、实行抗战教育政策,使教育为长期战争服务。

在一切为着战争的原则下,一切文化教育事业均应使之适合战争的需要,因此全民族的第十个任务,在于实行如下各项的文化教育政策。第一,改订学制,废除不急需与不必要的课程,改变管理制度,以教授战争所必需之课程及发扬学生的学习积极性为原则。第二,创设并扩大增强各种干部学校,培养大批的抗日干部。第三,广泛发展民众教育,组织各种补习学校,识字运动,戏剧运动,歌咏运动,体育运动,创办敌前敌后各种地方通俗报纸,提高人民的民族文化与民族觉悟。第四,办理义务的小学教育,以民族精神教育新后代。一切这些,也必须拿政治上动员民力与政府法令相配合,主要的在于发动人民自己教育自己,而政府给以恰当的指导与调整,给以可能的物质帮助,单靠政府用有限财力办的几个学校,报纸等等,是不足完成提高民族文化与民族觉悟之伟大任务的。抗战以来。教育制度已在变动中,尤其战区有了显著的改进。但至今还没有整个制度适应抗战需要的变化,这种情形是不好的。伟大的抗战,须有伟大的抗战教育运动与之相配合,二者间的不配合现象亟应免除。

十一、力争国际援助,集中反对日本帝国主义。

从长期战争与集中反对日本帝国主义的原则出发,组织一切可能的外援,是不可忽视的。因此,当前的第十一个任务,在于:第一,坚决反对一部分人所谓走德意路线的主张,因为这实际上是一种准备对敌投降的步骤。第二,力争各民主国家与苏联对我物质援助之增加,同时尽力促成我国实行国联制裁日本之决议。第三,设立一定机关,系统的收集一些敌军暴行制成具体的文书,报告,宣扬国外,唤起全世界注意,起来惩罚日本法西斩。第四,由各党派各人民团体推出代表,组织国际宣传团体,周游列国,唤起各国人民与政府的对我同情与我国政府的外交活动相配合。第五,保护一切同情国家在中国的侨民及其和平通商传教等事业。第六,注意保护华利益,并经过华侨的努力推进各国反日援华运动。一切这些,不管各国助我之程度如何暂时的没有增加或甚至可能不分的减少,国联决议可能依然是一句好听的话,我们部应努力的做。根据抗战的长期性,外交方针也应着眼于长期,不重在眼前的利益,而重在将来的增援,这一点远见是必要。

十二、建立中国与日本兵民及朝鲜台湾等被压迫民族的反侵略统一战线,共同反对日本帝国主义。

日本帝国主义的侵略战争,不但是危害中华民族的,同时也是危害日本全体兵民与朝鲜台湾等被压迫民族的,要使日本的侵略战争失败下去,必须中日两人民族的军民大众及朝鲜,台湾等被压迫民族作广大而坚持的共同努力,建立共同的反侵略统一战线。为此目的,全民族的第十二个任务,在于:第一,向两国人民士兵大众及朝鲜、台湾民族提出这个反侵略统一战线的方针,号召他们为此而斗争。第二,由政府下令所有抗日军队抗日游击队全体官兵一律学习必要数量与恰当内容的日本语,由高级政治部准备与派出教日本语的教员到各军队中实行施敌,从学几句话起到能够同日本军官兵讲一篇道理为止,教育全体抗日官兵向全体敌军士兵与下级军官作口头宣传,同时补助之以文字图画宣传,逐渐感化他们,要求他们同意建立共同的反侵略统一战线,使百万日本侵略军变成我们的友军,退出中国,推翻日本法西斯。第三,尊重与优待敌军俘虏,给以教育,经过他们去影响其余,为建立反侵略统一战线而努力。第四,设法从日本内地组织反侵略的文化人员到中国来参加这一斗争。第五,保护在中国的诚实的日本侨民。第六,教育我国军民大众。一方面提高民族自重心,又一方面则须纠正军队与人民中的一些错误思想,区别日本帝国主义与日本人民,区别敌军官与士兵,并区别上级军官与下级军官,实行了上述的方针与办法,付以广大而坚持的努力,这个反侵略统一战线是能够建立起来的。中国的胜利,主要依靠自己力量的增加;但同时,敌人的困难与国际的援助,必须争取其配合。在敌人困难方面,除了因我之坚持长期战争给以逐渐的消耗,努力外交活动使敌日陷于孤立而外,和日本兵民大众及朝鲜、台湾等民族建立其共同反侵略战线的政策,是不可缺少的部份。日侵略战争愈延长,这一个统一战线便愈有建立的基础。


十三、团结中华各族,一致对日


我们的抗日民族统一战线,不但是国内各个党派各个阶级的,而且是国内各个民族的。针对着敌人已经进行并还加紧进行分裂我国内各少数民族的诡计,当前的十三个任务,就在于团结各民族为一体,共同对付日寇。为此目的,必须注意下述各点:第一,允许蒙、回、藏、苗、徭、夷、番各民族与汉族有平等权利,在共同对日原则之下,有自己管理自己事务之权,同时与汉族联合建立统一的国家。第二,各少数民族与汉族杂居的地方,当地政府须设置由当地少数民族的人员组织成的委员会,作为省县政府的一部门,管理和他们有关事务,调节各族间的关系,在省县政府委员中应有他们的位置。第三,尊重各少数民族文化、宗教、习惯、不但不应强迫他们学汉文汉语,而且应赞助他们发展用各族自己言语文字的文化教育。第四,纠正存在着的大汉族主义,提倡汉人用平等态度和他族接触,使日益亲善密切起来,同时禁止任何对他们带侮辱与轻视性的言语,文字,与行动,上述政策,一方面,各少数民族应自己团结起来争取实现,一方切应由政府自动实施,才能彻底改善国内各民族的相互关系,真正达到团结对外之目的,怀柔羁糜的老办法是行不通了的。

十四、厉行锄奸运动,巩固前线与后方。

新的形势下汉奸敌探,托派,亲日派必然较前更加猖獗,大肆其造谣,污蔑,分裂,破坏的阴谋,因此当前的第十四个任务,在于实现下列办法,厉行锄奸运动。第一,唤起前线与后方一切军民人等的警觉性,严密注视汉奸,敌探,托派,亲日派之活动,依照政府法令,毫不容情的镇压之;第二,注意保护国家的机密以叛国罪惩办泄漏机密之叛徒。第三,学校教科书中加进锄奸一课,实施提高警觉性的教育。第四,军队中设置各级管理锄奸工作之部门。民众团体中人民自卫队中设置锄奸小组,国家警察加重锄奸教育,使叛徒在众目集视下无法藏身。抗战以来,吃些叛徒们的亏真是不可计量了。前线的将士,惊叹汉奸多与损害作战利益之大,早已异口同声。即在后方,单是泄漏国家机密与引导敌机惨炸二事亦已人人共愤。长期抗战中如不肃清叛徒,将不能设想战争的胜利,发动广大民众之民族革命的警觉性,厉行上述锄奸办法,并使之成为广泛的运动,是争取胜利不可缺少的严重的任务。应该指出:锄奸运动应注意区别首要与胁从,自觉的与被骗的,坚决份子与动摇份子,分别处理,前者从重,后者从轻,并注意争取后者使之回心向善,决不可一律看待。还须注意确实证据,勿用刑讯,严防诬陷。锄奸目的在肃清真正奸徒,只有用正确政策与正确方法,才能达到目的。

十五(略)

(五)长期战争与长期合作

现在,我们就专抗日民族统一战线的长期性问题来讨论一番,向着异常关心国共两党关系的人们所已经发生了的许多疑问,作一个全盘的答复,这一点,对于巩固与扩大抗日民族统一战线,巩固与扩大国共合作,顺利的执行当前紧急任务,渡过战争的难关,是有重要意义的。

问题有如下各点;战争的长期性决定合作的长期性,战争中的合作决定战争后的合作,长期合作的内容与主要条件,三民主义与共产主义,长期合作的组织形式,长期合作中的互助互让政策,民主共和国问题,这些都是很多人所关心的,我们都寻明确的给以答复。

一、战争的长期性决定合作的长期性

由于抗日战争是长期的,整个抗日民族统一战线也能够且必须是长期的,其中主要的两党一一国民党与共产党的合作,也能够且必须是长期的,这是一切政策的出发点。因此,我们的政策,无论如何要一个长期的民族统一战线,要一个长期合作,无论如何要共同维持一个统一政府,反对分歧与分裂,方才有利于渡过战争难关,对抗敌人破坏,打退日本帝国主义,并于战后完成建立新中国的任务。这是和一九二四至一九二七年的国共合作根本不同的,那次是短期的,这次是长期的。

二、战争中的合作决定战争后的合作

所谓长期合作,不但在战争中的,而且是在战争后的。抗日战争是长期的战争中的合作已经算得是长期的了。但是还不够,我们希望继续合作下去,也一定要继续合作下去。这有什么保证呢?保证就在:战争中的合作决定着战争后的合作。抗日民族统一战线中主要的国共两党,必须同患难,共生死,力求进步,并经过长期的努力,才能打退日本帝国主义,否则不能。战争之后,这样长期同过患难的有了进步的两个党,就造成了继续合作的基础。那时的国内国际条件将更有利于合作,也是现在想得到的。没有疑义,战争中的合作必有其各个合作阶段的内容,战争后的合作将更有新的内容,然而战争中的合作,将决定着战争后也能够合作,这不是没有根据的预断。

三、长期合作的内容与主要条件

所谓长期合作就是长期的民族统一战线,所有阶级,从资本家到工人,所有政党,从国民党到共产党,所有民族,从汉族到苗徭弱小民族,所有军队,从中央军到八路军,所有政府,从国民政府到陕甘宁边区政府,只有民族叛徒除外,一切都在内。而且是长期在内的。民族统一战线内,有些人在长期战争中,当然熬不过艰苦斗争,个人利益超过民族利益时,会要变为民族叛徒的。因此民族统一战线是要不断地把这些民族叛徒们除外的。但这些除外。依然是民族统一战线。其理由,即是长期合作的主要条件。首先是战争的野蛮性与长期性。由于敌人战争的野蛮性,严重的危害着全民族各个阶层的存在,这样就迫使上层阶级也不得不与其他阶级一道抗日。上层阶级中一部份是难免退出抗日战线的,但其他部份和其他阶级大体一样,是受压迫的,不反抗便无出路。又由于这种野蛮性的战争是长期的,就决定合作是长期的。这些是决定长期合作的一方面。但是还有第二方面,要合作中的各党,首先是国共两党,采取正确的政策进行必要的工作。什么样的政策与工作呢?应该是从长期战争与长期合作的基点出发而规定出来与实行起来的政策与工作。应该是照顾现在又照顾将来,照顾这一阶级又照顾那一阶级,照顾这一党派又照顾那一党派,照顾这一军队又照顾那一军队,照顾这一民族又照顾那一民族的政策与工作。否则政策不对,工作不行,自乱步骤,将使合作难于持久。这样,一方面,敌人战争的野蛮性与长期性,又一方面,统一战线中的正确政策与必要工作,就使中国的民族统一战线不但应该是长期的,而且能够是长期的。是民族战线,不是人民阵线。是包括战争的与战争后的国共合作,不是企图在战争后又分裂又内战的国共合作。

四、三民主义与共产主义

三民主义是抗日民族统一战线与国共合作的政治基础。但是三民主义与共产主义的关系如何呢?共产党员对三民主义应取何种态度呢?直至现在还有一些人不清楚,因此有再一次解释的必要。

还在一九三六年五月间开的我们党的临时性的代表大会士,就通过了如下的关于“坚决实行三民主义”的提纲:

“共产党是否同意三民主义,我们的答复是同意的。三民主义有它的历史变化。孙中山先生的革命的三民主义,曾经因为同共产党合作与坚决执行而取得人民的信仰,发动了一九二五一一二七年的胜利的大革命。又曾经因为排斥共产党(清党起动),实行相反的政策而失去人民的信仰,招致革命的失败。现在则因为民族危机与社会危机极端严重,全国人民与国民党中爱国份子,因而有两党合作的迫切要求。因此重新整顿三民主义的精神,在对外独立解放的民族主义,对内民主自由的民权主义,与增进人民幸福的民生主义之下,两党重新合作,并领导人民坚决的实行起来,是完全适合于中国革命的历史要求,而应为每个共产党员所明白认识的。共产党决不抛弃其社会主义与共产主义理想,他们将经过资产阶级民主革命阶段达到社会主义与共产主义的阶段。共产党有自己的党纲与政策,其党纲是社会主义与共产主义,这是与三民主义有区别的。其民主革命政策,亦比国内任何党派为彻底,但对国民党第一次及第二次代表大会所宣布的三民主义纲领,则是基本上没有冲突的。因此我们不但不拒绝三民主义,而且愿意坚决实行三民主义,而且要求国民党同我们一道实行三民主义,而且号召全国人民实行三民主义,使国民党,共产党,全国人民,文同一致为民族独立,民权自由,民生幸福这三大目标而奋斗。”(中国抗日民族统一战线在目前阶段的任务)

去年九月二十二日,我们党的中央为公布国共合作成立的宣言中,又着重地说道:“孙中山先生的三民主义为中国今日之必需,本党愿为其彻底实现而奋斗。”

一个共产主义的政党为什么采取这种态度呢?很明显的,民族独立,民权自由民生幸福,正是共产党在民族民主革命阶段所要求实现的总目标,也是全国人民要求实现的目标,并非某一党派单独要求的东西,只要看一看从共产党诞生以来的文献,它的政治纲领,就会明白。因此,在过去,不但在一九二四至二七年国共两党第一次合作时期,我们共产党员曾经坚决实行了三民主义。就在一九二七年两党合作不幸破裂之后,我们的一切做法,也没有违背三民主义。那时,我们坚决地反对帝国主义,这是符合于民族主义的;我们实行了人民代表会议的政治制度,这是符合于民权主义的;我们又实行了耕者有其田的土地制度,这是符合于民生主义的。那时,我们的一切做法,并末超过资产阶级民主革命基本范畴的私有财产制。现在在抗战的阶段与战后彻底完全民主共和国的阶段,都是三民主义的阶段,都是资产阶级民立革命性质的阶段,为了彻底完成这个民主阶段的任务,一切共产党员,毫无疑义,应该依照自己的一贯的革命总方针,自己的决义与宣言,同中国国民党与全国其他党派,全国广大人民一道,诚心诚意实行三民主义。谁要是不忠实于三民主义的信奉与实行,谁就是口是心非,表里不一,谁就不是一个忠实的马克思主义者。在中国,任何忠实的马克思主义者,他是同时具有现时实际任务与将来远大理想两种责任的。并且应该懂得:只有现实的实际任务获得尽可能彻底的完成,才能有根据有基础地发展到将来的远大理想那个阶段去。所谓将来的远大理想,就是共产主义,这是人类最美满的社会制度。孙中山先生也曾经认为必要实行它,才解决将来的社会问题。所谓现在的实际任务,就是三民主义,这是“求国际地位平等,求政治地位平等,求经济地位平等”的现阶段的基本任务,是国共两党与全国人民的共同要求。因此,共产党应该如像他们研究共产主义一样,好好研究三民主义,用马克思主义的眼光,研究三民主义的理论,研究如何使三民主义具体地见之实施,研究如何用正确的三民主义思想教育人民大众,使之由了解而变为积极行动,为打退日本帝国主义,建设三民主义新中国而斗争。

五、长期合作的组织形式(略)

六、长期合作中的互助互让政策

长期战争需要长期的统一战线,前已说过,这是一切政策的出发点。因此,共产党员在工作中,在其同友党发生关系中,随时随地都要顾到这个长期性。凡于长期合作有利的,应该坚决的勇敢的做,不利的,则决不应做。……(略)

互助就不是互害,损人利己,在个人道德是不对的在民族道德更加不对。因此,无理的磨擦,甚至捉人杀人等事,无论如何是要不得的。共产党员决不应该以此对待友党。而如若友党以此对待我们时,我们也决不容置之不理。凡无理的事必须以严正态度对待之,才是待已待人的正道,互相规过,是朋友间的美德,也是政党间应该提倡的作风。

统一战线中有什么互让呢?有的,我们曾经在政治上作过一些让步,那就是停止没收土地,改编红军,改编苏区制度,这是一种政治上的让步,这是为了建立统一战线团结全民族共同对敌的必要步骤。我们的友军也作了让步,那就是承认共产党的合法地位等等。这种为了团结抗日为了长期合作的互让政策,是很好的,很对的。只有政治上糊涂或别有用心的人,才会说:共产党投降了国民党,或国民党投降了共产党。

现在我们又主张所有各统一战线中的党派,互不在对方内部招收党员,组织支部,进行秘密活动。我们认为这种政策是必要的。现在当然和过去不同,在过去内战时期,国共两党间除了公开的战争以外,还互相使用秘密手段,进行破坏对方的活动。合作以后,也应该停止,使彼此安心,才能适合于长期战争中长期合作之目的。我们现在正式向国民党同志申明:我们停止在你们内部作招收党员组织支部的活动,不管统一战线采取何种的共同组织形式,我们都是这样做。但同时也希望你们,这样做。双方约定之后,下级党员如有违背,由违背之一方的上级负责处理。


此外,双方同志接触,应釆谦和,尊敬,商量态度,不釆傲慢,轻视,垄断态度,以改善双方之关系,这也是必要的。

一切我们所说的,共产党员应该首先实行,不管对方某些人员或尚未用同样的政策,方法,态度对待我们,但我们仍然这样做,做出久了,对方某些一时尚未明白的人员也会明白了。

共产党员对于一切为国为民的事业,应该坚持自己的立场,始终不变地向着战胜日寇建立新中国的方向走去,谁要违背了这种立场,这个方向,谁就丧失了共产党员的资格。但共产党层又必须有互助互让的精神,必须确尊重友党及友军同志用谦和商量态度解决问题的精神,一切有友党同志的地方,都应和他们商量解决所有事项,不应独断。没有这种精神,就不能巩固统一团结,所谓为国为民事业,战胜日寇建立新中国之目的,也就达不到。因此,决不能把必要的互让政策解释为消极行为。不但互助是积极的,因为必要的让步,是巩固两党合作求得更好的团结与更大的进步之不可缺少的条件。

七、民主共和国问题

虽然我们的党远在一九三六年的九月间,就公布了关于建立民主共和国的决议案,虽然中央同志曾经多次的说明过这个问题,但外间对于我们的主张仍有许多不明白的,这是一个关于抗战前途的问题。抗战的结果将怎么样呢?所谓抗战建国,照共产党的意思,究竟建立一个什么国呢?这是存在着的问题。再一次解释这个问题,对于巩固各党各派长期合作的信心,是有利益的。

建立一个什么国呢?一句答复:建立一个三民主义共和国。

我们所谓民主共和国就是三民主义共和国,它的性质是三民主义的。按照孙中山先生的说法,就是一个“求国际地位平等求政治地位平等求经济地位平等”的国家。第一,这个国家是一个民族主义的国家。它是一个独立国,它不要任何外国干涉,同时也不去干涉任何外国。即是说,改变中国原来的半殖民地地位,它独立起来了;但同时,无论它强胜到什么程度,决不把自己变为帝国主义,而是以平等等精神同一切尊重中国独立的友邦和平往来,生存互惠。对国内各民族。给予平等权利,而在自愿原则下互相团结,建立统一的政府。第二,这个国家是一个民权主义的国家。国内人民,政府地位一律平等;各级官吏是民选的;政治制度是民主集中制;设立人民代表会议的国会与地方会议;凡十八岁以上的公民,除犯罪者外,不分阶级、男女、民族、信仰与文化程度,都有选举与被选举权。国家与予人民以言论、出版、集会、结社、信仰,居住、迁徙之自由,并在政治上物质上保护之。第三,这个国家是一个民生主义的国家。它不否认私有财产制,但须使工人有工作,并改良劳动条件。农民有土地,并废除苛捐杂税重租重利。学生有读书,并保留贫苦者入学。其他各界都有事作,能够充分发挥其天才。一句话,使人人有衣穿,有饭吃,有书读,有事作。我们所谓民主共和国,就是这种一种国家,就是真正三民立义的中华民国。不是苏维埃,也不是社会主义。

中国要变为这样一个国家,要同谁作斗争呢?要同日本帝国主义作斗争。日本布国主义剥夺我们的独立,我们就要向他要独立。日本帝国主义把我们当奴隶,我们就要向他要自由。日本帝国主义使我们陷入饥寒交迫,我们就要向他要饭吃。怎样要法?用枪口向他要。一句话:赶走日本帝国主义,就有一个独立自由幸福的三民主义新中华民国。

(六)中国的反侵略战争与世界的反法西斯运动

一、中国与世界不可分

中国已紧密地与世界联成一体,中日战争是世界战争一部分,中国抗日战争的胜利不能离开世界而孤立起来。新的抗战形势中可能暂时的减少一部份外国的援助,加重了中国自力更生的意义,中国无论何时也应以自力更生为基本立脚点,但中国不是孤立也不能孤立,中国与世界紧密联系的事实,也是我们的立脚点,而且必须成为我们的立脚点。我们不是也不能是闭关主义者,中国早已不能闭关,现在更是一个世界性的帝国主义用战争闯进全中国来,全中国人都关心世界与中国的关系,尤其关心目前欧洲时局的变动,所以我们来分析一下当前的国际形势,是有意义的。

二、重新分割世界的第二次世界大战已经开始

资本帝国主义的本性,不但是和本国人民大众矛盾的,是和殖民地半殖民地矛盾的,是和社会主义国家矛盾的,而且是帝国主义诸国之间自相矛盾的。这最后一种矛盾在历史上的最尖锐的表现,就是二十年前的世界大战。那次两组帝国主义互战的结束,产生了新的国际形势。战后世界的政治经济新的发展的结果,使世界又临到新的大战面前。在东方日寇侵略东四省,西方希特勒登台之后,新的重分世界的战争业已开始了。“法西斯主义就是战争”,一点也不错,在此情势下,一方面日德意组成了侵略阵线,实行大规模的侵略。另一方面各民主国却为保守已得利益而在和平的名义之下准备战争;但至今,不愿用实力制裁侵略者,尤其是英国的妥协政策实际上帮助了侵略者。在这种情况下,中国东四省首先被牺牲,接着亚比西尼亚亡于意大利,西班牙则助长了叛军的气焰,中国又受到日寇新的大规模的侵略,到最近,奥国与捷克又先后牺牲于希特勒。全世界已有六万万人口进入了战争,范围普及到亚、非、欧三洲,这就是新的世界战争的现时状况。

三、现在世界战争的特点

由于一方面日德意诸法西斯国家的坚决的侵略意志,又一方面各民主国家,不愿实力制裁尤其是英国的妥协政策,使得新的战争的现时状态表现了和第一次世界大战的不同特点,这就是首先侵略中间国家与釆取各种不同的战争形式。中国、亚比西尼亚、西班牙、奥大利、捷克等国,都是半独立国家或小国。日德意诸国就拣了这些肥肉先行吞蚀。在侵略这些中间国家中,侵略者采取了三种特殊的战等形式。第一种是日本对中国,意大利对亚比西尼亚的战争,这是公开的直接的大规模的战争,但是在不宣而战形式下进行的,开了战争史上的新纪元。釆取这种不宣而战政策的目的,在于侵略名利用各民主国家的无意于制裁尤其是英国妥协政策,暂时避免和它们的直接冲突,便利其先夺取中间国家的行动。第二是意德两国侵略西班牙的方式,采取了援助叛军的办法,这是历史上老办法的重演,历史上这类办法是有过的。第三是希特勒侵略奥捷两国的方式,这里没有战争的表面(没有打击),但有战争的实际,出动了强大兵力占领奥国全国与捷克一部,并使捷克余部归属其统治,这是不战而亡人国的最巧妙的办法。这三种战争形式的采用,都是由于一方面,侵略国本身力量还不充足,暂时未便和大国直接作战,因而采取了巧妙的战争方法,企图使自己先行壮大起来,同时即时使各大国削弱起来,再与各大国作战。又一方面,则是各民主国家不愿裁制侵略者,尤其是英国的怯懦妥协政策的结果,这种政策实际上援助了侵略者,便利其侵略各中间国家。

四、英国妥协政策将引导法西斯各国实行更大规模的战争

以张伯伦为首的英国保守党内阁,正在逐步进行其所谓四强合作的政策,慕尼黑协定之后,欧洲政局有暂时逆转的可能。英国大部份保守党的政策,历来是以排斥苏联妥协德意为原则的,由于他们畏惧苏联的强盛,畏惧自己过早卷入战争,畏惧本国人民远动与殖民地独立运动,早已决心牺牲西班牙、奥国、捷克等国,成就其排斥苏联妥协德意的企图。过去因为保守党内部的不统一,法国人民战线的积极政策,国内国际舆论的责备,而没有成功。现在则利用了英国及全欧人民不愿战争的心理,利用了法国佛兰亭党的右倾,在希特勒威迫之下,订立了慕尼黑协定。这个协定是英国妥协政策的结果,假如英国不改变它的政策,势将引导法西斯各国进行更大规模的冒险战争。各大国间的战争虽暂时还可能不爆发,暂时限制于侵略中间国家的过程虽还在继续着,但最后势必引导各大国卷入空前残酷的战争里去,这是没有疑义的前途。“搬起石头打自己的脚”,这就是张伯伦政策必然的结果。

五、全世界多数人类在逐渐动员中

在资本主义各国方面,由于经济的危险,资本主义已走到毫无出路的地步,六万万人口的战争牵动了全世界,新的更大的战争在威胁全人类。在社会主义国家方面,则一切都是光明的,进步的,强盛的。在这两种相反的对比之下,全世界大多数人类逐渐地找到了如何保卫自己与解放自己的方向,正在用空前的广大性与空前的深刻性逐步地团结自己并准备斗争。第一次世界大战,二十年来社会主义国家的强盛,资本主义国家的衰落,六七年来法西斯国家的侵略战争。中国的伟大抗日战争,西班牙的人民战争,乃至张伯伦的妥协政策等等,逐渐地教育了英法等国与全世界人民,使他们懂得惟有组织与斗争才是出路,惟有团结世界一切自求解放的人类为一体,惟有世界人民与被压迫民族的统一战线,才有出路。这全世界人民觉悟,组织,斗争,与统一战线的伟大过程,是在向前发展着,但须经过广大而艰苦的努力才能完成。法西斯的战争威胁与张伯伦的妥协政策,最后将遇到伟大的反抗,这也是没有疑义的前途,也是法西斯战争与张伯伦政策的必然的结果。

六、中国民侵略战争与世界反法西斯运动的配合

过去的,大家都明白,各民主国家在某种程度上都是援助中国的,主要是其人民同情中国,苏联的援助则更加积极。现在,由于日寇进攻的深入,又加深了英美法苏对日本的矛盾。虽然英国在西方的妥协政策可能搬用到东方,为了企图多少保存在日本占领区的商业,为了幻想减轻日本对南洋的威胁,英国有可能同日本进行某种程度的妥协,但根本妥协是困难的,至少暂时有困难。这是日本独占政策的结果,东方问题与西方问题在当前具体情况下有某种程度上的区别。日本的深入进攻,进一步加深了日美间的矛盾,苏联与中国的友谊是增长的,中美苏三国有进一步亲近的可能。但是我们第一不可忘记资本主义国家与社会主义国家的区别;第二不可忘记资本主义国家之政府与资本主义国家之人民的区别;第三,更加不可忘记现在与将来的区别,我们对前者不应寄以过高的希望,应该努力争取前者一切可能的援助。在一定程度上不但是可能的,而且是事实,但过高希望则不适宜。中华民族解放运动与外援的配合,主要的是和先进国家与全世界广大人民反法西斯运动之将来的配合,主要的是和先进国家与全世界广大人民反法西斯运动之将来的配合,以自力更生为主。同时不放松争取外援的方针,应该放在这种基点之上。

(七)中国共党在民族战争中的地位

(这部分的第一一一十三节见《毛泽东选集》第二卷)

十四、团结全党到团结全民族

伟大的斗争需要伟大的力量,团结全民族,发动全民族一切生动力量进入这个斗争中去,是我们确定了的方针。而要达此目的,中国共产党内部的团结,是有决定作用的,是最基本的条件,遵义会议与克服张国焘错误之后,我们的党是第六次全国代表大会以来最团结最统一的时期了。现在我们党内,无论在政治路线上战略方针上,时局估计与任务提出上,中央委员会与全党的意见都是一致的。这种政治原则的一致,是团结的基本条件。党员与党员,干部与干部,领导者与领导者之间的相互关系,习惯上所谓人事关系,我们也学会了许多正确的恰当的方法,造成了在正确的政治原则下的和衷共济的空气,有了更好的相互关系。由于地区的广大,情况的复杂,工作部门的不同,不同的意见是难免的,应该的,党内民主的实际,就是容许任何不同意见提出讨论。也正是由于民主方法。保证着交换意见,并使之概括起来作出结论,形成全党一致的方针,在这里,客现的与全国的看问题的态度,不杂主观成见与意气,不要片面的看问题,这种马克思主义的方法,我们也逐渐地学会了,这又保证着党的团结。我们是科学马克思主义者,自以为是的成见与意气用事的作风,是无用的长物。经过了十七年锻炼的中国共产党及它的领导人员,已经有了老练的作风了。所有这些,就能保证中央以至全民族形成一个团结一致的核心与重心,推动抗战进到胜利。同志们,全党团结起来,全民族团结起来,胜利一定是我们的!

(八)召集党的七次代表大会

现在我来说最后的一个问题,召集七次大会的问题。

同志们,我们党的全国代表大会,自从一九二八年开过第六次代表大会以来,由于环境的原因,已有十年没开大会了。去年十二月政治局会议决定准备召集七次代表大会,但准备工作尚未完成,因此今年尚难召集。此次全会扩大会应该讨论加紧这个准备工作的问题,并决定在不久时间实行召集大会。这次会议的政治意义是重大的,它将总结过去的经验,主要的是全国抗战与抗日民族统一战线的经验。讨论国内国际的政治形势。讨论如何进一步的团结全民族,团结国共两党及其他党派,进一步的巩固与扩大抗日民族统一战线。讨论如何在长期战争与长期合作中争取抗战最后胜利的方法与计划。讨论如何动员全国工人阶级及劳动人民更积极的参加抗战。并在讨论党在新的情况下如何进一步的团结自己,加强自己,巩固自己与其他党派及全国人民的联系,以便顺利的执行抗日民族统一战线的总方针。除了这些政治的与组织的问题之外,七次大会应该选举新的中央委员会,将全党中最有威信的许多领导同志选进中央委员会来,加强对于全党工作的领导。同志们,这次大会的意义如此重大,因此扩大的六中全会闭幕之后,诸位同志回到各地工作,便应在努力发展党与巩固党的基础之上,依照民主的方法,适时地进行选举,使那些最优秀的最为党员群众所信仰的干部与党员有机会当选为大会的代表,使七次大会能够集全党优秀代表于一堂,保证大会的成功。我们相信,这次全国代表大会一定能够成功,一定能够给日本帝国主义的侵略战争以最壮严的最有力量的回答,社日本帝国主义在我们的全国代表大会面前发起抖来,滚到东洋大海里去,中华民族是一定要胜利的。

我的报告就此完结。

×这是毛泽东同志在一九三八年十月十二至十四在中央扩大的六中全会的报告。

