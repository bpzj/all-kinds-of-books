\section[在中央工作会议上的讲话(一九六一年一月十三日)]{在中央工作会议上的讲话}\datesubtitle{(一九六一年一月十三日)}

这次工作会议,据我看比过去几次都要好,大家头脑比过去清醒了些,冷热结合得好了一些。过去总是冷得不够,热得多了些,这次比过去有了进步,对问题有了分析,对情况比较摸底了。当然,有许多情况还是不摸底。中央和省市都有这种情况,比如对一、二、三类的县、社、队比较摸底:一类是好的,执行政策,不刮共产风。二类也比较好,三类是落后的,不好的,有的领导权被地、富、反、坏分子篡夺了,实际上是打着共产党的招牌,干国民党、地主阶级的事情,是国民党、地主阶级的复辟。全国县、社、队有百分之30是好的,百分之50是一般的,百分之20是坏的。在一个具体地方,坏的有超过百分之20的,有不到百分之20的。但是究竟情况怎样,也不是完全清楚,也不完全准确,只能说大体上是这样。不要以为一、二类社、队都是好的,其中同样也有坏人,三类队中也有好人。××同志批了河南灵宝县的一个报告,指出了一、二类社里也有问题,群众发动以后,谁是好人,谁是坏人,群众是摸底的,公社是摸底的,就是我们不太摸底。总的看好的和较好的占百分之八十,还是好的多,群众知道好坏,就是领导不摸底。我们要有决心,这些地方没有强有力的领导,如果不派大批干部深入发动群众,找出贫农和下中农中的积极分子,采取两头压的办法,是不能解决问题的。

灵宝县一、二类社尚有许多问题,也还有坏人,何况三类社?现在我们虽然还不完全摸底,但已向这个方向进了一步,今后好好地进行调查研究,就可以更摸底。譬如粮食产量究竟有多少?现在比较摸底了,口粮搞低标准,瓜菜代,粮食过秤入库,比较摸了底。但也有地方不摸底,河北省还有百分之××的县、社、队不摸底。口粮标准有的不按省里规定吃,吃多了。

至于城市工业问题,比较接近实际。今年钢只定××××万吨,煤、木材、矿石、运输还得搞那么多。煤的指标要增加,不但冬季烧煤不够,而且发电用煤也不够。今年着重在搞质量、规格、品种。钢的产量已居世界第×位,数量不算少,目前是质量不够,所以今年不着重发展吨数。

省委书记、常委,包括地委第一书记,他们究竟摸不摸底?他们不摸底就成问题了。应该说现在比过去进一步,也在动了。要用试点方法去了解情况,调查问题。调查不需要很多,全国有通海口一个就行了,但现在也只有这么一个报告。三类社、队的问题,有信阳地区的整顿经验的报告,那么整三类社、队的问题就够了。还有河北保定的一个材料很有说服力,这个报告说什么时候刮共产风,如何纠正,如何整顿组织,如何改进领导,以及怎样实现大生产。现在河南出了好事,出了信阳文件,纪登奎的报告。希望大家回去后,把别的事放开,带一两个助手,调查一两个社、队,在城市也要彻底调查一两个工厂、城市人民公社。

省委第一书记只有那么一个人,怎么能又搞农村又搞城市呢?因此要有个助手,分头去调查,使自己心里有底。心中没底是不能行动的。过去打仗,心中有底,靠什么?解放战争初期,中央直接指挥的经验少,但有两个办法;一靠陕北打胡宗南的经验,到四七年四、五月间,就靠各地区前方的报告,这是阳的,还靠阴的,即各方面的情报,所以情况很清楚。现在这些情报没有了,死官僚又封锁了消息,中央就得不到更多的消息。

我们下去搞调查研究,检查工作,要用眼睛去看,用耳朵去听,用手去摸,用嘴去讲,要开座谈会。看粮食是否增了产?够不够吃?要察颜观色,看看是否面有菜色,骨瘦如柴。这是眼睛可以看得出来的。保定的办法是请老农、干部开座谈会,与总支书、支书谈,群众也发言议论,这些意见是有钱买不到的东西。

这些年来,这种调查研究工作不大作了。我们的同志不作调查研究工作,没有基础,没有底,凭感想和估计办事。劝同志们要大兴调查研究之风,一切从实际出发,没有把握就不要乱发言,不要下决心。作调查研究也并不那么困难,人不要那样多,时间也不要那么长,在农村有一两个社队,在城市有一两个工厂,一两个学校,一两个商店,合起来有七、八个,十来个,也就行了。也不必都自己亲自去搞,自己搞一两个,其他就组织班子去搞,亲自加以领导。保定的报告是农村工作部搞的,是个大功劳;通海口是省委抽人下去的,灵宝县的报告是纪登奎同志下去搞的,信阳的报告是搞造后的地委下去搞的。

调查研究这件事极为重要,要教会许多人。所有省委书记、常委、各部门负责人、地委、县委、公社党委,都要进行调查研究,不做,情况就不清楚。公社有多少部门,第一书记不一定知道,一个公社,有三十多个队,公社党委只要摸透好、中、坏三个队就行。做工作要有三条:一是情况明,二是决心大,三是方法对。这里情况明是第一条,这是一切的基础。情况不明,一切都无从谈起,这就要搞调查研究。资产阶级是讲调查研究的。美国发言人总是说胡志明的军队进入老挝,但究竟进去什么兵,什么官,什么兵种,他们不说。资产阶级比我们老实,不知道就不讲。我们有时没有底,哇里哇啦一套。但是资产阶级也有冒失鬼,资本主义国家有个杂志说:从五一年到六零年\footnote{原文为“六〇年”,下同。},就把苏联和一切社会主义国家都消灭掉。

这次会议,情况逐渐明朗,决心逐步大。但是决心还是参差不齐的。有的同志讲刮共产风要破产还债,听起来不好听,但实际上是要破产还债。县、区、社两级通通破掉就好了,破掉以后再来真正的白手起家。……我们是马克思列宁主义者,不能剥夺劳动者,只能剥夺剥夺者,这条是马克思列宁主义的基本原则。……资产阶级、地主阶级剥夺劳动人民,马克思列宁主义者不能剥夺劳动人民。资产阶级、地主阶级的方法比我们还高明,他们是逐步使劳动者破产欠债,我们是一下子平掉,用这种办法建立社有经济、国营经济。我们的国营经济赚钱太多,到农村中去收购,常常压级压价,剥夺农民,交换非常不等价,这就使工人阶级脱离他们的同盟者。这个道理,同志们也懂得,话也好讲,但实行起来决心不大,不那么容易。是不是所有的省委书记都有那么大的决心破产还债,还得看看。这也是不平衡的,各省也会是参差不齐的。可能有的省决心大,彻底一些,把群众团结在自己的周围。有些省决心不大,作的差一些。一省之内,几十个县也会是不平衡的,因为领导人的情况不同。一类县、社、队有百分之三十共产风刮了一下,停的早,五九年郑州会议后就停下来了,他们懂得不能剥夺农民,不能黑手起家,决心大,退赔的彻底,以后就不再刮了。有些搞变得不彻底,一次再一次刮共产风。去年春季,中央情况不明,以为共产风不很严重,所以搞得不彻底。其实去年春季就应该开这样的会,纠正共产风,可是没有开。我们对情况不够明,问题不集中,决心不大,方法也不大那样对头,不是像现在信阳、通海口、保定、灵宝的方法。所以这件事是个大事情,这是一场大斗争,要在实践与斗争中认识问题,解决问题。农忙过后还要再搞,一、二类社、队也还不少,还要抓紧搞,下决心搞彻底。总而言之,过去抗战时期、解放战争时期,调查研究比较认真,实事求是,从实际出发,情况明了,决心就大,方法就对头,解决问题的措施也较有利。只有正确的方针政策,但情况不明,决心不大,方法不对,还是等于零。郑州会议讲不能一平二调,方针是对的,说不算账、不退赔,这点不对。上海会议十八条讲了要退赔,紧接着我批了浙江、麻城的经验报告。五九年三、四月,我批了两万多字的东西。现在看来,光打笔墨官司,不那么顶用。他封锁你,你情况不明,有什么办法?那时省委、地委的同志也不那么认识共产风的危害性。有的同志讲郑州会议是压服,不是说服,思想还有距离,所以决心不大,搞的不够彻底。

工业开始摸了一些底,还要继续摸底。要缩短工业战线,重工业战线,特别是基本建设战线。要延长农业战线,轻工业要发展。重工业除煤炭、矿山、木材、运输之外,不搞新的基本建设,过去搞了的,有些还要搞,但有些也不搞,癞了头就让它癞头去吧。

长远计划现在搞不出来,我们要再搞十年,从六零年到六九年,这是个革命。中国的封建主义搞了那么多年,民主革命也搞了那么多年,没有民主革命的胜利,就没有社会主义。搞社会主义建设不能那么急,十分急搞不成,要波浪式前进.陈伯达同志提出,社会主义建设是否也有个周期率,若干年发展较快,有几年较低,如同行军一样,有大休息、中休息、小休息,要劳逸结合,两个战役间要休整。这次工作会议也有劳逸,决议文件也不多,譬如郑州会议就只搞了那么一个决议嘛!还是看情况明不明,决心大不大,方法对不对头。

现在看一个材料说:西德钢产量去年是三千四百万吨,英国二千四百万吨,西德六零年比五九年增加百分之十五,法国是一千七百万吨,日本是二千二百万吨。但他们的生产率是长期积累的,搞了那么多年,才那么多,我们才几年,就××××万吨。今、明、后年,搞几年慢腾腾,搞扎实一些,然后再上去。指标不要那么高,把质量搞上去,让帝国主义说我们大跃进垮台了,这样对我们比较有利,不要务虚名而得实祸。要提高质量、规格、品种,提高管理水平,提高劳动生产率。现在我们劳动生产率很低。五七年我们职工有二千四百多万人,现在有五千多万人,还要下放。不然,五六个人围着一台机器,一个人做,几个人看,这不行。解决这个问题也是要情况明,决心大,方法对。

陈伯达同志有个材料,美国一个农民劳动力养活三十个人,英国二十六个人,苏联六个人,我们只有三个半人。有人说我们也可以养四个人,那就看你怎样养了,如果一天只吃几两米,那不行。

国际形势我看也是很好的。原来我们讲要硬着头皮顶,准备顶它十年。从前年西藏闹事到现在,不过二十多个月,现在反华的空气大为稀薄了,但空气还是有,有时还有寒流。莫斯科会议以后,空气还好一些。

今年搞一个实事求是年。实事求是是汉朝的班固在汉书上说的,一直流传到现在。我党有实事求是的传统,但最近几年来不大了解情况,大概是官做大了,摸不了底了。今年要摸它一个工厂、一个学校、一个商店、一个连队、一个城市人民公社,不搞典型就不好工作。这次会议以后,我就下去搞调查研究工作。总而言之,现在摸到这个方向,大家都要进行。不要只讲人家的坏话,有的地方工作有错误,人家搞了,就要欢迎人家。


