\section[在八大二次会议上的讲话(摘要)(一)(一九五八年五月八日下午四时五十分)]{在八大二次会议上的讲话(摘要)(一)(一九五八年五月八日下午四时五十分)}
\datesubtitle{(一九五八年五月八日)}


地点:中南海怀仁堂

我讲一讲破除迷信。

我们有些同志有几“怕”。

怕教授,怕资产阶级教授。整风以后,最近几个月,慢慢地不太怕了。有些同志,如柯庆施同志,接受了复旦大学的聘书当教授,这是不怕教授的一种表现。

另外一种怕,是怕无产阶级教授,怕马克思。马克思住在很高的房子里,要搭很长的梯子才上得去。于是乎说:“我这一辈子没有希望了。”这种怕,是否需要?是否妥当?在成都会议上我谈过对马克思也不要怕。马克思也是两只眼睛,两只手,跟我们差不多,只是那里头有一大堆马克思主义。他写了很多东西给我们看,我们不一定都要看完。×××同志在不在?(答:在)你看完了没有?你看完了,你上到楼上去了,我没看完,还在楼底下。我们没有看完他的著作,都是楼下人。但不怕,马克思主义那么多东西,时间不够,不一定都要读完,读几份基本的东西也就可以了。我们实际做的,许多超过了马克思。列宁说的做的,许多地方都超过了马克思。马克思没做十月革命,列宁做了。我们的实践,超过了马克思。实践当中是要出道理的。马克思革命没有革成,我们革成了。这种革命的实践,反映在意识形态上,这就是理论。二月、十月中国革命成功了,理论上就不能没有反映。我们的理论水平不高,现在不高,但不要怕,可以努力。我们要努力。我们可以造楼梯,而且可以造升降机。不要妄自菲薄,看不起自己,中国被帝国主义压迫了一百多年。帝国主义宣传他们那一套,要服从洋人;封建主义宣传那一套,要服从孔夫子。“非圣则违法”,反对圣人,就是违犯“宪法”。对外国人说我不行,对孔夫子说我不行,这是什么道理?

我问我身边的同志:“我们住在天上,还是住在地上?”他们摇摇头说:“是住在地上。”我说:“不,我们是住在天上。如果别的星球有人,他们看我们,不也是住在天上吗?”所以我说,我们是住在地上,同时,又住在天上。

中国人喜欢神仙,我问他们(指主席身边的同志),我们算不算神仙?他们说:“不算!”我说不对,我们是住在“天上”,为什么不算是“神仙”呢?如果别的星球有人,他们不把我们看成是神仙吗?

中国人算不算洋人?大家说,外国人才算洋人,我们不算洋人。我说不对,我们叫外国人叫洋人,在外国人看来,我们不也是洋人吗?

有一种微生物叫做细菌。我看细菌虽小,但是,在某一点上,它比人厉害。它不讲迷信,它干劲十足,多快好省,力争上游,目中无人,天不怕,地不怕。它要吃人,不管你有多大,即使你有八十多公斤的体重,你有了病它也要吃掉你。它的这种天不怕地不怕的精神,不比某些人强吗?

自古以来,发明家创立新学派的,在开始时,都是年轻的,学问比较少的,被人看不起的,被压迫的。这些发明家在后来变成了壮年、老年、变成有学问的人。这是不是一普遍规律?不能肯定,还是调查研究,但是,可以说,多数是如此。

为什么?这是因为他们的方向对。学问再多,方向不对,等于无用。

“人怕出名猪怕壮。”名家是最落后的,最怕事的,最无创造性的。为什么?因为他已经成了名。当然不能全盘否定一切名家,有的也有例外。

年轻人打倒老年人,学问少的人打倒学问多的人,这种例子多得很。

战国的时候,秦国有个甘罗。甘罗十二岁为丞相,他才是个“红领巾”。他的祖父甘茂没有主意,他却有主意,他到赵国解决了一个问题。

汉朝有个贾谊,十几岁就被汉文帝找去了,一天升了三次官,后来调到长沙,写了两篇赋,《吊屈原赋》和《鹏鸟赋》。后来又回到朝廷写了一本书,叫《治安策》。他是秦汉历史学家。他写了几十篇作品,留下的是两篇文学作品(两篇赋),两篇政治作品——《治安策》和《过秦论》。他死的时候,只有三十二岁。

刘邦的年纪比较大。项羽起兵的时候只有廿四岁,三年到咸阳。霸王别姬的时候,应该还是年轻的时候,他死的时候,只有三十二岁。

韩信也是一个被人看不起的人。他在年轻的时候,曾经受过“胯下之辱”。

孔夫子当初也没有什么地位,他当吹鼓手,后来教书。他虽然做过官,在鲁国当过“司法部长”。鲁国当时只有几十万人口,和我们现在一个县官差不多,他那个“司法部长”,相当于我们现在的县政府的司法科长。他还当过“会计”,做过管钱的小官,可是他却学会了许多本领。

颜渊是孔子的徒弟,他算个“二等圣人”,他死的时候,也只有三十二岁。

释迦牟尼创立佛教的时候,他只有十几、二十岁,他是印度当时一个被压迫民族的人。

红娘是个有名的人物,她是青年人,她是奴隶,她帮助张生做那样的事情,是违犯“婚姻法”的,她被拷打,可是她不屈服,反抗一过,还把老夫人责备一顿。你们说,究竟是红娘的学问好,还是老夫人学问好?是红娘是“发明家”,还是老夫人是“发明家”?

晋朝的荀灌娘是个十三岁的女孩子,顶多不过是“初中程度”,他到襄阳去搬救兵,你看她多大的本领?

唐朝的诗人李贺,死的时候只有二十七岁。

唐太宗李世民起兵的时候只有十八岁,做皇帝的时候只有二十岁。

李贺、李世民都是贵族。

罗士信是山东人,也是二十四岁起兵,打仗很勇敢。

做《滕王阁序》的王勃,唐初四杰之一,他是一个年轻人。

宋朝的名将岳飞,死的时候才三十八岁。

范文澜同志你说对不对?你是历史学家,说的不对,你可以订正。

马克思的马克思主义,并不是壮年、老年的时候创造出来的,而是在年轻的时候创造出来的。写《共产党宣言》时才二十几岁。

列宁也是三十一岁(一九○三年)创造出的布尔什维克主义的。

周瑜、孔明都是年轻人,孔明二十七岁当军师。程普是老将,他不行,孙吴打曹操不用他,而用周瑜作都督,程普不服,但是周瑜打了胜仗。有个黄盖,是我的老乡,湖南零陵人,他也在这个战役中立了功,我们老乡也不胜光荣之至。

晋朝的王弼,做《庄子》和《易经》的注解,他十八岁就是哲学家,他的祖父是王肃。他死的时候才二十四岁。

发明安眠药的不是什么专家,据说是一个司药。我在一个小册子上看到的。他为了发明安眠药,在做实验的时候,几乎丧失生命。试验成功了,德国不赞成他,法国人把他接过去了,给他开庆祝会,给他出书。

盘尼西林——青霉素的发明是一个染匠,因为他女儿害病,无钱进医院,就在染缸边抓了一把土,用什么东西和了和,吃了就好了。后来经过化验,这里头有一种东西,就是盘尼西林。

达尔文,大发明家,他也是个青年人,研究生物学,到处跑,南北美洲、亚洲都跑到了,就是没有到过上海。

最近的那个李政道,杨振宁也是年轻人。

郝建秀,全国人民代表,她在十八岁的时候,创造了先进的纺纱的办法。

作国歌的大音乐家聂耳,也是年轻人。

哪吒——托塔李天王李靖的儿子,也是年轻人,他的本领可不小嘛!

南北朝的兰陵王也是年轻人,他很会打仗。

现在许多的优秀的乡干部,社干部都是年轻人。……举这么多例子,目的就是要说明,年轻人要胜过老年人的,学问少的可以打倒学问多的人。不要被权威、名人吓倒,不要被大学问家吓倒。要敢想、敢说、敢作,不要不敢想、不敢说、不敢作。这种束手束脚的现象不好,要从这种现象里解放出来。

劳动人民的积极性、创造性,从来是很丰富的。过去是在旧制度的压抑下,没有解放出来。现在解放了,开始爆发了。

我们现在的方法是揭盖子,破除迷信。让劳动人民的积极性和创造性都爆发出来。

过去不少的人认为工业高不可攀,神秘得很,认为搞工业“不容易”呀,总之,认为搞工业有很大的迷信。

我也不懂工业,对工业也是一窍不通,可是我不相信工业就是高不可攀。我和几个管工业的谈过,开始不懂,学过几年,也就懂了。有什么了不起!我看,大概只要十几年的功夫,我们的国家就可以变为工业国。不要把它看得那么严重。首先蔑视它,然后重视它。……

“让高山低头,要河水让路”,这句话很好。高山嘛,我们要你低头,你还敢不低头?河水嘛,我们要你让路,你还敢不让路?

这样设想,是不是狂妄?不是的,我们不是狂人,我们是实际主义者,是实事求是的马克思主义者。

……

不要大国沙文主义,大国沙文主义是丑恶的行为,是低级趣味!

法门寺这个戏里有个角色叫贾桂,他是刘瑾的手下人,刘瑾是明朝太监,实际上是“内阁总理”,掌大权的人。有一次刘瑾叫贾桂坐下,贾桂说:我站惯了,不敢坐。这就是奴隶性。中国人当帝国主义的奴隶当久了,总不免要留一点尾巴。要割掉这个奴隶尾巴,要打倒贾桂的作风。

有两种谦虚,一种谦虚是庸俗的谦虚,一种是合乎实际的谦虚。

教条主义者照抄外国,是过分谦虚。你自己干什么?你就不动脑筋。中国古诗中有一种拟古诗,就是过分谦虚。自己没有独创风格,要去模拟别人。

修正主义者也是过分的谦虚。如铁托无非是照抄伯恩施坦,从资产阶级老爷那里搬点东西来。

教条主义是一国的无产阶级照抄另一国的无产阶级。有好的就抄好的,有不好的也抄了。这就不好。抄是要抄的,要抄的是精神,是本质,而不是皮毛。比方说,莫科斯宣言的九条共同纲领(“再论”说是五条,莫斯科宣言分成为九条)是各国的共同的东西,少一条也不行。普遍真理要与中国的具体实践相结合。如果不结合,只是照抄,那就是过分的谦虚。非普遍真理,就不能照抄。就是国内的东西,也不能照抄。土地改革的时候,中央没有特别强调哪一个地方的经验,这就是怕照抄。现在工作当中,也要注意这个问题。

修正主义者是资产阶级化了的人,抄了资产阶级,铁托抄伯恩施坦就是一例。

我们要学列宁,要敢于插红旗,越红越好,要敢于标新立异。标新立异有两种:一种是应当的,一种是不应当的。列宁向第二国际标新立异,另插红旗,这是应当的。红旗,横直是要插的。你要不插红旗,资产阶级就要插白旗。与其资产阶级插,不如我们无产阶级插。要敢于插旗子,不让它有空白点。资产阶级插的旗子,我们就要拔掉,要敢插敢拔。

列宁说过:“先进的亚洲,落后的欧洲”。这是真理,到现在还是如此,我们先进,西欧落后。……

我们蔑视资产阶级,蔑视神仙,蔑视上帝。但是不能蔑视小国,蔑视自己的同志。

十五年之后,我们变成现代化,工业化,文化高的大强国。可能要翘尾巴,我们不要怕,现在就讲清楚。狗翘尾巴,不一定要打棍子,泼一瓢冷水就行了。我们有时候要浇一浇冷水的。

不正当的自信心,庸俗的自信心,虚伪的自信心,那是不允许的。不建立在科学基础上的谦虚不叫谦虚,真正的谦虚是要合乎实际。比如说,我们见了外国人说,中国现在还是农业国,工业建设刚开始,……这就是实际,但外国人说我们谦虚。一般是合乎实际的。

也有谦虚低于实际,过分谦虚。一般的是合乎实际。

这种说法,类似鲁迅对于讽刺的说法。鲁迅说:用精练的或者有些夸张的笔墨写出真实的事物,就叫讽刺。……

范文澜同志最近写的一篇文章,我看了很高兴。(这时站起来讲话了)这篇文章引了许多事实,证明了厚今薄古是我国的传统,引了司马光……可惜没有引秦始皇。秦始皇主张“以古非今者族’,秦始皇是厚今薄古的专家。当然。我也不赞成引秦始皇。(林彪同志插话:秦始皇焚书坑儒)秦始皇算什么?他只坑了四百六十个儒,我们坑了四万六干儒。我们镇反。还没有杀掉反革命的知识分子吗?我与民主人士辩论过,你骂我们是秦始皇,不对,我们超过了秦始皇一百倍。骂我们是秦始皇独裁者,我们一贯承认,可惜的是,你们说的不够,往往要我们加以补充。(大笑)


事物总是要走向自己的反面。

希腊的辩证法,中世纪的形而上学,文艺复兴。这是否定的否定。

中国也是如此。战国时期的百家争鸣,这是辩证法。封建时代的经学,这是形而上学。现在又叫辩证法。

是不是?范文澜同志,你对这些很熟悉。

我看,十五年后尾巴肯定要翘起来,要出大国沙文主义。出了大国沙文主义也不怕,难道就怕变成大国沙文主义而就不为建设社会主义而奋斗吗?即使将来出现了大国沙文主义,也会走向自己的反面的。有一种正确的东西代替大国沙文主义的,有什么可怕的,社会主义国家,不可能全部的人都变成大国沙文主义。

列宁的辩证法,斯大林的部分的形而上学,现在的辩证法,也是否定的否定。

斯大林不完全是形而上学,他懂得辩证法,但不甚懂得人民群众的创造性,这是客观存在的。设置对立面很重要。对立面是客观存在的。如我们对右派,让他放,让他讲,这是有计划地这样做,目的是要设对立面。整右派以后,有的同志忽视整改,又强调大字报搞双反,这样设置了对立面,出了一亿张大字报,逼得非改不可。

设对立面不是说客观不存在而设置。所谓对立面,是要客观存在的东西才能设置起来。客观不存在的东西,是设置不了的。


我讲完了,这个题目叫做破除迷信,不要怕教授,也不要怕马克思。


