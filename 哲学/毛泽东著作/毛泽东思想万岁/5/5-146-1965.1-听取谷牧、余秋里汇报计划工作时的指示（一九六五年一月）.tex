\section[听取谷牧、余秋里汇报计划工作时的指示(一九六五年一月)]{听取谷牧、余秋里汇报计划工作时的指示}
\datesubtitle{(一九六五年一月)}


(一九六五年一月谷牧、余秋里同志向主席汇报计划工作,当讲到敢想敢干时)主席说:

要敢想敢干,不要乱来,破除迷信,不要破除科学。不要讲了半天,又是设计,又是试制,最后什么结果也没有。

(汇报到今年钢可以搞到××万吨时),主席说。

不是有一个消息吗,英国人听说我们搞调整、巩固,就害怕了。你不搞冒进,搞质量,搞品种、规格,他就怕了。数量慢慢上去,不要急。

(讲到××建设时)主席说:

××建设要抓紧,就是同帝国主义争时间,同修正主义争时间。

(汇报到我们的技术要赶上和超过国际水平时)主席说:

是的,我们要有……管他什么国,管他什么弹,原子弹、氢弹,都要超过。我说过,原子弹响了,人类即使死一半,也还有一半。斯诺同我谈话,问我为什么不辟谣,我说不要辟谣。我只说过战争打起来人死一半,还有另外一半。有比我更厉害的,美国的一个影片形容得很厉害呀!赫鲁晓夫比我说的多得多嘛!他说有毁灭人类的武器,什么死光。我不是说中国,而是说全世界死三分之一,总而言之,死一半!××建设,我们把钢铁、国防、机械、化工、石油、铁路基地都搞起来了,那时打起来就不怕了。搞不成,打起来怎么办?\marginpar{\footnotesize 227}我们就用常规武器跟他们打。从前我们没飞机、大炮……又没有黄色炸药,还不是打赢了!打起来还可以继续建设,你打你的,我建设我的。

(讲到设计一定要采用新技术时)主席说:

设计要做比较,哪些花钱少,办事多,哪些花钱多,办事少。设计人员是在家里设计,还是到现场设计?我看了一万二千吨水压机设计的文章,有些设计经过一次、二次,甚至几百次的失败,不经过失败是不会成功的。

