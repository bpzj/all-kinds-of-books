\section[贫农团组织与工作大纲(一九三三年七月十五日)]{贫农团组织与工作大纲}
\datesubtitle{(一九三三年七月十五日)}


贫农团在土地革命中一般说来是起了伟大的作用,他不但团结了广大雇农贫农群众,联合着中农在其周围,在共产党与苏维埃领导之下,以绝大的革命力量,推翻了农村中的封建半封建势力,从地主阶级手中夺取了土地积极的推进了土地上面的生产。即对于战争动员,经济动员和文化建设等项重大的工作,也起了极大的作用。贫农团的这种伟大的革命成绩,是十分值得称赞的。但是依据看来,贫农团不是都在一切地方都有这样的成绩的,也不是有了这些成绩就可以不注意其组织上和工作的进行上还有缺点与错误。却正相反,只有看到了贫农团过去的成绩,同时还看到了贫农团的缺点与错误,还看到了有些地方的成绩不够,因此努力去改正贫农团组织与工作上的一些缺点与错误,才能争取以后更加伟大的成绩,完成贫农团在苏维埃运动中应有的责任。

根据许多地方的材料,贫农团在组织上与工作上,还有许多不健全的地方,有些地方甚至还只有一一个空名,没有实际的工作,总括这些材料看来,可以分成两种睛形:(一)有别地方还没有乡贫农团,只有村贫农团、村贫农团之下设小组。有些地方没有村贫农团,只有贫农团,乡贫农团之下设小组。在乡贫农团和村贫农团内部都没有干事会,内分主任、组织宣传三个工作部门,由干事会指挥各小组,定期开会,有的十天一次全体会,五天一次小组会,有的一星期一次全体会,五天一次小组会。加入贫农团的不论层农苦力贫农,要有三人介绍,否则不能加入,把许多雇农苦力贫农关在贫农团的门外,以致会员数量不甚发展,减少了贫农团的作用。(二)另外一种更加不好的情形,就是有些地方只简单地宣布除掉地主富农中农以外,其余一切都是贫农团会员,由中共支部或乡苏指定一个贫农团主任,这样该乡就算有了贫农团的组织了。贫农团主任不晓得有几多会员,也不晓得要做些什么工作,三四个月甚至半年不开会,只挂一个空名,这实际上等于没有贫农团。这样的贫农团的任务和他在农村中应起的作用,是更加不够完成了。第一种情形的贫农团,其中有些代替了政府的职务,如瑞金武阳区的龙冈乡,黄柏区的新庄、北村两乡,就有这种情形。而第二种情形的贫农团,则没有经常工作,散漫不起作用。这两种现象都是不好的。但是苏区之中有不少地方的贫农团在组织上工作上都是很健全地发展的,大数量的会员,紧张的斗争情绪,能够积极地讨论各种革命斗争问题,又并不侵犯乡苏的职权,他们不但和上述第二种情形的贫农团(挂空名的)大不相同,比起第一种情形的贫农团(还有错误的)来也更加进步,他们真算得苏区中贫农团的模范,值得各地贫农团的学习。为了总结过去贫农团斗争的经验,统一贫农团的组织,普遍建立贫农团的经常工作,完成贫农团在目前革命阶段中的重大任务,中央政府特根据土地斗争发展的经验,及瑞金、会昌、雩都、胜利、博生、石城、宁化、长汀八县贫农团代表大会的建议,印发这一贫农团组织与工作大纲。希望全国农村中广大的贫农群众都在这一大纲之下,一致的团结与行动起来。

(一)贫农团不是纯粹一个阶级的组织,而是在苏维埃管辖区域内的广大贫农群众的组织,同时农村工人必须参加贫农团,组成工人小组在里面起积极领导作用,同时团结广大贫农群众在无产阶级领导之下,成为苏维埃政权最可依靠的柱石。

(二)贫农团的作用是赞助政府实现政府的一切法令,而不是代替政府的工作。关系于工人贫农的利益与权利的各种问题,应该提出自己的意见向政府建议。

(三)贫农团要特别注意中农的利益和权利,使中农环绕在贫农团的周围,建立贫农团与中农的坚固联盟,成为无产阶级联合中农的坚固的一环,以便利于进行消灭地主阶级与反对富农的斗争。

(四)贫农团只有共产党与苏维埃的领导之下,才能正确的实现他的一切任务,不致受富农的影响,不致受一落后区的农民意识,如绝对平均观念和地方观念等所支配。在还没有组织贫农团的地方,可由农业工会或贫农中的积极分子来发起,农业工人及手艺工人工会要能够做到在自己的全体大会上通过整个加入贫农团,以实现无产阶级在贫农团内的经常的领导作用。

(五)贫农团在初成立的时候,是吸收贫农中积极分子参加,以后便逐渐成为全体贫农群众的组织,为了吸收全体贫农群众加入贫农团,(当然破坏土地斗争包庇地主富农的那种坏分子,即使是贫农也不能加入)贫农团的老会员应该经常负责去做扩大贫农团的宣传,积极找贫农民农村工人以加入贫农团,但不须用介绍形式,而是向工人贫农开大门,告诉并指引他们加入贫农团。加入贫农团是以自愿为原则,一切男女老少的工人贫农,均可报名加入,那种按户派人的作法是错误的,为了严厉地防止地主富农的混入,须从老会员找来的及自动报名的一切新会员中,按照分析阶级的标准加以考查,遇到有成分不对的,立即开除出去,以保障贫农团不被地主富农混入。

(六)当着开展斗争,或当着进行查田运动的时候,如果因为过去贫农团散漫无作用,或者贫农团操纵在少数地主分子与富农分子里,起了相反的作用,因此就用命令主义的办法,解散贫农团重新组织,这是脱离了群众,十分不对的。这种时候的正当办法,应该团结贫农中的积极分子加紧教育他们,在贫农团内发动激烈的斗争,揭破地主富农的欺骗,争取会员群众脱离地主富农分子的影响,而坚决把地主富农分子及个别无法再教育的极坏的分子洗刷出贫农团去,这样来坚强贫农团的战斗力量,争取土地斗争与查田运动的彻底胜利。如果贫农团中有中农加入,而便不能像对付地主富农分子一样,把他们一下洗刷就完事,一定要经过一番清楚的解释,对他们说明不必加入贫农团的理由,在他们出贫农团之后,贫农团开会时他们仍可来旁听,并且欢迎一切中农都来旁听。

(七)贫农团既不是纯粹一个阶级的组织,因此他并不须要如工会一样的严密的组织形式,不需一定的章程,也不需要缴纳会费,(需要用费时,在全体大会的同意下,可向会员举行临时的募捐),更不需要省县区的系统的组织,只是以乡为单位来组织贫农团。乡贫农团之下分小组,小组可以一个屋子(即小村子)为单位,每个屋子的会员为一小组,如果一个屋子里只有极少数人家,因之会员人数不多,可以两三个接近的屋子成立一个小组。如果一个屋子有几十家甚至更多的人家,因之贫农人数甚多,可以在一个村子内成立几个小组。

(八)为着贫农团工作进行的便利,应由大会选举三个人(最积极的分子)组织委员会,较大的乡或会员很多的贫农团,可推举五个人组织委员会,由委员会推定主任一人,主持全盘工作,此外不在设别的工作部门。贫农团应当实行很广泛的革命民主制度,凡遇到重要问题,必须召集全体会员开会讨论。只有平常的问题,才单由革委会讨论,或由委员会召集小组组长参加讨论。

(九)贫农团大会,委员会,及小组开会,不必机械的固定时间,以防止形式主义的开会,减少会员群众的兴趣。凡遇重要问题,即行召集会议。在农村阶级斗争特别激烈的时候,如像分田与查田的时候,就是三四天,五六天开一次会员大会,两三天开一次委员会及小组长会,都是应该的。

(十)贫农团的工作,在于随时能注意到工人贫农以及中农的利益,为了苏维埃政权的巩固与发展而斗争,今将贫农团的重要工作列举如下:

(甲)讨论豪绅地主的土地房屋、农具财产和富农的土地及多余的耕牛农具房屋等的没收与分配的问题。讨论这个问题时,中心是如何对待地支富农的反抗,如何使土地革命的利益完全落在工人贫农中农的身上。这里要特别注意不妨碍中农利益,并且要亲密的联合中农要注意彻底消灭农村中的封建势力,并不使地主富农冒称中农贫农偷取土地利益,要做到不使一个地主留一寸土地,不使一个富农偷取一丘好田。

(乙)讨论农业生产上的问题,如怎样进行春耕、夏耕、秋收、秋耕的运动。在各季的生产运动中,怎样增加人工,增加肥料,改良种子,开发水利,调剂耕牛,添置农具,消灭害虫,开垦荒田,种植树木与保护山林等问题。

(丙)讨论经济动员上的问题,除上述发展农业生产外,主要是讨论合作社的发展;如发展粮食合作社,消费合作社、信用合作社、生产合作社、犁牛合作社,使国民经济大规模的发展起来,以抵制商人残酷剥削,打破敌人的经济封锁,使群众生活得到进一步改良,革命战争得到极充实的力量。

(丁)讨论救济灾荒的问题,如灾荒时粮食、种子、耕牛、农具等的互相帮助,被敌人骚扰区域被难群众的设法救济等。

(戊)讨论群众卫生的问题,如发起普遍的卫生运动,讲究清洁扫除,以抵制疾病疫疠,保障群众生活的健康。

(已)讨论优待红军的问题,如红军公田的耕种收获与保存,及帮助红军家属耕田,实行优待红军条例。

(庚)讨论战争动员上的问题,这里第一是扩大红军,第二是筹借经费接济红军,第三是慰劳红军,第四是扩大赤卫队与少先队,这些都是关于战争动员上极重要的问题。

(辛)讨论参加苏维埃选举运动与检举运动的问题,如当着选举时保障工人贫农中的积极分子使他们当选进去,并吸引最好的中农分子参加苏维埃工作。当着苏维埃中混入有阶级异己分子,及发现有贪污腐化,消极怠工等分子时,参加工农检察部所号召的检举运动,发动对苏维埃工作人员的自我批评,使苏维埃的工作绝对健全起来。

(壬)讨论苏维埃一切法令决议命令使之在本乡完全实现的问题。

(癸)讨论一切临时发生的重大问题。

贫农团必须经常注意上列种种问题的讨论,并且积极的向政府建议,在政府领导下坚决的参加各种革命战线上的斗争,使各种革命任务完全实现,他才能不断的有自己的经常工作,而真正替工人贫农中农谋得利益,不致成为有名无实的团体。

(十一)为着更能巩固农村无产阶级对于广大农民群众的领导,贫农团委员会(并可选几个积极的贫农分子同去)可以建议与农业工会手艺工会的领导机关开联席会议,这种会议由工会召集,如组织拥护红军委员会,组织反帝拥苏同盟,组织革命互济会及某种庆祝纪念与示威大会等,都可举行联席会议来讨论,在一致的同意下共同努力使实现之。

<p align="right">中央政府主席毛泽东项×张××</p>

