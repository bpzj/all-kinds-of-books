\section[在为八届六中全会作准备的郑州会议上的讲话第二次讲话(一九五八年十一月六日)]{在为八届六中全会作准备的郑州会议上的讲话第二次讲话}
\datesubtitle{(一九五八年十一月六日)}


只搞两个月,就搞出了一个名堂,明年再搞一年就有办法了。明年一年。极其重要,以钢为纲,三大元帅,两个先行。

什么叫建成社会主义?什么叫过渡到共产主义,要搞个定义。

苦战三年,再搞十二年,十五年过渡到共产主义。不要发表但不搞不好。

由集体所有制过渡到全民所有制要多长时间?三、四、五、六,或更多一点时间,是不是短了?还是长了?有时觉得长了,有时又耽心短了,我耽心短的时间多。人民公社什么时候能达到像鞍钢一样?能不能把农业变成工厂?产品和积累能够调拨,积累不全部要调,但必须调动的产品,则必须无条件的调动,才算全民所有制。河南说是四年,可能短了,加一倍,八年,范县说苦战两年,过渡到共产主义。

斯大林写的东西必须看。好处是只有他讲社会主义经济,最大的缺点是把框子划死了,说集体农庄只愿意商品交换,不愿意调拨。这是因为不要不断革命,巩固社会主义秩序。俄国农民不会那么自私,不会不要不断革命,俄国建立了社会主义秩序,但这秩序是不能巩固的。我们则相反,破坏社会主义秩序一部分,供给制部分就是破坏这种秩序的。

公粮、积累、劳力,都是调拨性的,全民所有制的。百万雄师下江南,现在为什么不能调人去劳动。现在只能部分的调,全省、全国的调不行。如安国准备明年给阜平每人五百斤麦子,是世界上没有过的事。调拨须有可能与必要,不能乱调。秦始皇调七十万人替他修墓,结果垮台,隋炀帝也因乱调劳力而垮台了。

武王伐纣是否三化?自古以来就是三化,先从军事上开始的。

从集体所有制到全民所有制要多少年?四年是否可以?河南说四年,范县说两年。标准是鞍钢。鞍钢除七千二百元成本折旧,下余一万零八百元,工人所得八百元,为国家积累一万元,要这样的调拨。这种过渡,对斯大林是千难万难的,要多少年来说明期限。这是第一个过渡。第二个过渡,从“按劳取酬”到“各取所需”。现在已开始准备第二个过渡,吃饭不要钱。苏联也吹,只见楼梯响,不见人下来。我们吃饭不要钱是各取所需的萌芽。我们现在油太少,一般不到四两,有一两、二两、三两、四两、五两,一般是三、五两,营养取自粮食,故吃饭多,可以转化,不要冒险,凡是可做的必须逐步去做。这不能不说是共产主义因素。

自给、供给制,公社内部调拨与商品交换,要向两方面发展,没有商品经济的发展,发不了工资。教授参观徐水大学,一看一月发五元,每天吃不到两合前门牌,你叫优越性?河北有三个县要救济;十几个县只能吃饭,第三种发工资,从两毛起到几块。北京、上海钱多。农村水、火不算钱,要议一个标准。

母亲肚里有娃娃,社会主义有共产主义萌芽,斯大林看不到这个辩证法。

不能两三年内搞农业,现在就要搞工业,继续发展农业,我们也怕搞掉农业。要提倡每个公社生产商品,不要忌讳“商品”二字。

中国是一个极不合理的生产方法,五亿人口只搞饭吃,搞那么一点一一三千七百亿粮食。得出两条经验:提倡机械化和少种多收,节省出劳力,大办工业。明年、后年两年能达到就行了。河北明年准备搞一千万亩,亩产万斤,则一千亿斤,去年八千八百万亩,还只二百五十亿斤,今年也只有四百五十亿斤。用机械电气之力,种少量之地,得多量之粮食。不要吹那么厉害,结果像黄炎培说的,郑州只有吃素。吹了一顿,不过一两五钱油一个月,有什么优越性?低标准又有几等,照范县标准,可以叫共产主义,百分之九十五的工业。现在不能叫共产主义。水平太低,只能说共产主义的因素和萌芽,不要把共产主义的高标准降低了。

供给制是便于过渡的形式,不造成障碍。建成社会主义,为准备过渡到共产主义奠定基础。

标准各不一样。有和尚标准,恩格斯要吃油、吃肉。

苏联的集体农庄,不搞工业,只搞农业,农业又广种薄收。所以过渡不了。苏联的社会主义是集体所有制和全民所有制,斯大林的过渡到共产主义,说得难,不向全民所有制过渡、共产主义因素根本不提倡,割裂重轻工业,公开提倡不着重消费资料的生产,几个差别扩大了。

要重读斯大林《社会主义经济问题》,《资产阶级法权问题文集》要看一下。

人民公社性质,如何过渡,多少时间,四十条宪法,要议一下。城市人民公社如何搞?不同的人,是否原则上不降低标准?干部党内略微调整,不宣传,教授降薪,前门牌少抽,标准还降低,有什么优越性?是否搞点步骤,提高才算优越性。工人恐怕要增加点工资,农民已经起来了,城市供给制也是要搞的。

明年一月一日开始变化,睡足八小时。四小时吃饭,休息,二小时学习。农民,八一四一二一十制,工人最好是八一四一二搞作息时间表,否则不能持久。星期天休息。

不休息,这是共产主义精神,劳动不是为人民币,已经不只是生活的手段,而是生活的必需。我这个人不是为五百三十九元,而是为了必需。要搞个调查。操儿女之心,个人之心,变为操社会事业之心。

今年一年好事多得很,开辟了道路,许多过去不敢设想的事实现了。因此,才敢设想人休息,三分之一地休息。

林业,可了不起,不要看不起。威廉士说:“农林牧要结合,不要以为种树只是绿化而已。”南方要十五一一二十五年,北方要四十一一五十年,种树也必须密植,有伴容易长,大家长就舒服,孤木不长。种树也要有一套,养鱼如养猪,种树如种粮,挖一丈深,分层施肥。

今年大搞小土群,有人说森林、煤炭有很大浪费,也大为节省,几千万人上山,又查出资源,又得了经验,这是收入。

不能完全以生活水平来讲,否则那些腐败的皇帝和贵族早已是共产主义了。要讲需要,要说热量等于五百卡路里就够了。做皇帝、诸侯,超过了就受不了。少了也不行。食物含有氮、氢、碳、氧、镁、钾、钠、锗、磷、氯,共产主义也只是以一定量的元素作营养。不能太多。

四十条连今年十五年。十年完成。


