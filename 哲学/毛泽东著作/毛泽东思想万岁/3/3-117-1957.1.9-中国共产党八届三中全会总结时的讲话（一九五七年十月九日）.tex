\section[中国共产党八届三中全会总结时的讲话(一九五七年十月九日)]{中国共产党八届三中全会总结时的讲话}
\datesubtitle{(一九五七年十月九日)}


一、这次会议评论一下究竟开的好坏?省委书记说很好,怕有的同志,不一定这样看法,开的很长很累,有些同志害了感冒,我也感冒了。恐怕成绩是主要的吧!交换经验,明确方向,统一意志,有很大好处。这次会议有两天没有参加,发言也未看完,我打算全看一遍,有的要看两遍,也怕你们没看完,希望你们也看一遍,这次发言有很好的经验。这样的会有必要一年开一次。我们是大国,工作复杂,一个也要开一次这样的会,省、地、县、区四级参加,加一部分社主任,中央明年打算一部分县委书记参加,我建议一年一次。去年未开吃亏了,来个右倾,工作松了劲(因开八大,无时间)。一年一次大概需一个月时间,各省可开一次四级干部会,把问题扯清楚。

二、整风讲几句。湖北提“大胆的放,彻底的放,坚决的放”的口号好,广东南方日报用这样的题目写了一篇社论,很好,还说要大胆的放,彻底的放,坚决的放,认真的放,也得认真的改。反右上了轨道,不必提大胆的反(主要是县以下基层还要反右派),省、中央也还要放,但主要还是要改的问题。今年整风创造了一种革命的形式,即几大:“大鸣、大批、大争、大字报”,现找到了这种形式,过去是找不到的,不能产生的,层层辩论那时不许可,因而土改打仗不能采取这种形势。现在我们取得了全国的胜利,政权是巩固的用摆事实讲道理来处理人民内部矛盾,找到了适合阶级斗争的形式,今后事情就好办得多了。大是大非也好,小是小非也好,整风也好,建议也好,都可以用辩论形式解决。不仅与中间派一道,也与右派一道,农村与地主、富农一道,上报,骂的狗血淋头,“党天下”“下轿”等,这是好的形式,容易发挥群众的主动性,积极性,责任心。过去也有扎根串连商量,连长与战士平等友谊的谈话,这是民主的传统,无此传统是不可能有“几大”的,延安整风是写笔记,开小组会、反省,大家都感到帮助很大,搞了几个月,那次克服了主观主义,以后三查三整,三反五反,扎根串连都表现了一定民主形式。但大鸣大放和大辩论,这种更大的民主形式则是这次产生的,找到了这种形式对党的事业,克服官僚主义,命令主义有极大的好处,这是民主传统的一个很大发展,要传下去。社会主义一定要能充分发挥民主,这种民民主只有社会主义才有,别的阶级是没有的,在此基础上实行集中,就可以加强无产阶级专政,建立巩固的集中制。民主是为着加强集中制,而不是削弱它。中国无产阶级人数少,只有几千万,专政靠几亿贫下中农、革命知识分子、贫苦手工业者,城市贫民,发动他们的积极性,就可以巩固专政,加强集中。

三、农业。四十条修改后发下,在农村中希望很好组织一次辩论。提高认识,作出规划,省、地、县、区、乡、社我都问了一下。省、县、乡、社应该作出规划,专区和区也要作规划。这六级要作,抓紧搞。全面规划(即计划),加强领导,书记动手、全党办社。去年下半年不大强调了。这个老口号,现在还要强调做,何时作好规划?我问了一些同志,有的已做好了,有的还未做好。是否在明春,省、地、县三级做好,或明年“五一”前做好,不行就在明年底,这是长远规划——十年至二十年,定出后可能不完全合适,做了不合适将来修改,总是要修改的,大概是三年一小修,五年一大修,有规划比没有好。只有十年了,再不抓紧,“四、五、八”就有危险,抓紧即可完成,要靠精耕细作吃饭,我国将来要是世界上最高产国家,现在已拥有潮汕千斤县。“四、五、八”是否可以再提高到“八百、一千、一千二”,我看可以,还有二、三十年就行。二十世纪速度快,不要照旧历书行事,最多也是廿一世纪初就可以达到,我看农业生产,地也不要那么多,有人就好办事,每人有几分地就够了,三亩地有些多了,当然还有节制生育,我不是奖励生育,各级规划都要到农村去讨论,规划很多,可分级分批的去讨论,粮食已有底,大小口平均四百斤就够了,种子、饲料不在内,湖南湘潭地委发言说,有一家七口人,每人五百零九斤,只吃四百斤,每人还余四百零九斤,南方四百斤稻就可以了,河北只有三百廿斤,三百一十斤,二百八十斤,豫北辩论的结果,过去富裕中农才吃到三百六十斤,还大喊大叫不够吃,饿死人。要很好的提倡勤俭持家,节约才能积累,国家、社、家庭都要积累,否则都吃掉如何能富裕。今年未受灾和收成好的地方要坚决做到“以丰补欠”,多搞一些积累,湖南提出搞百分之二十的生产费(公积金、公益金在外),全省共六亿元,不是平常的生产费,而是买肥料修农具,内有28%的基础(山区、平原不同,有多有少)即每年一亿二千元,这样做到很好,你们看可以则行,不然扣多了,就无优越性。管理费一定收缩在l%之内,以之增加到基层上面。要教育全国每个人要有志气,有远大目标。到二十一世纪整个世界情况会大变,要看这一点,大吃大喝不算什么志气,还要勤俭持家,大约有十年就好了,你死了还有儿子、孙子、红白喜事应该少办,棺办农民大概还要的(我们中央的同志死掉烧掉,已经签了名的,要火葬否自定),但要节省,慢慢来,习惯是可以改变的,也可以通过大鸣大放大辩论解决,也搞个十二年规划狠狠改一改,要不要赌博,也只有通过辩论,改变风气。关于除“四害”我是有兴趣,但无人响应。同志讲的不多,幸喜有广东省韶关地委乐昌县搞的好,已做出榜样,消灭“四害”,把卫生搞好,这是文化,要提倡,两双筷子吃饭很好,可是费点竹子,但这样对手工业有好处,就多做生意。我提议:从明年起,两年试点、五年普及,三年扫尾,行否?或者三年试点,五年普及,二年扫尾,两头小,中间大,各地可参差不齐。我对这四样东西很感兴趣。中国要变成四无国好不好?过去无政府主义者提出三无,“即无政府、无家庭、无阶级”。将来无政府是可以实现的,无家庭几百年之后也会不存在的,家庭是生产单位、教育单位,现在生产单位已不是了,条件全部改变即没有家庭了,大概需一千年,现在首先搞“四无”,把“四害”除掉,使我国变成高度文化的国家。

人口节约,要三年试点宣传,三年推广,四年普及推行,也是十年计划,不然人口达八亿再搞就晚了,初步达到计划生育。在少数民族地区不要推广,山区人口过少的地方也不推广,也要大鸣大放大辩论一下。我主张中学也加一门节育课,人类在生育上完全无政府主义是不行的,也要有计划生育。还有各地搞规划,有个综合规划很重要,即工农商学(无兵)的计划要结合起来搞,不是各搞各的,请你们看浙江的文件,红安县的经验很有用,各级农村工作领导同志都搞点试验田是好办法,我也很想搞点试验田,县、区、乡、社干部各搞一块试验田,这样搞技术就摸底了,无业务不行,要学点,内行光政治不行,工、商、农业都要懂点业务。今天讲的是农业,也搞个十年规划,十年,技术业务都要熟悉。我们不要先红而后专,一定要又红又专,现在的干部是先红后专,有些红也不红变成右倾机会主义,政治上是白的技术上不专。有些人提出先专后红,这是资产阶级观点,我们不要,但我们必须在十年之内建立一支强大的无产阶级知识队伍,斯大林说:“干部决定一切”,就是说的技术,业务是可以学的,五十岁也可以学,不要胆怯,问题是在肯学,我们能的就是红星旗很红,政治上行,技术还是不行,右派和我们争的不可开交,说服不了人家,所以还要学。组织科学技术队伍,省、地、县都要有计划。十年内党委就是变成无产阶级知识分子,各级都要有计划变成专家,百年树人,现在要减少九十年,要十年树人,在北方十年树木不行,树人是可以的,无此一条,没有庞大的技术队伍是不行的,社会主义是建设不成的,我们已解放八年,再十年,即十八年树人,基本上造成有马克思主义思想的科学技术队伍,以后就是扩大加强科学技术队伍再有十年,即二十八年达到苏联四十年的水平,缩短二十年,行否?这是可能的,因为有苏联的经验。

农业和工业的关系,××同志讲了,优先发展重工业不动摇,在此条件下,工农业同时并举,建成现代化的工业与现代化的农业。光说现代化工业(其中包括现代化农业)还不够完善,过去宣传工业是对的,今后要着重多宣传农业。

四、两种方法。凡事至少有两种方法,才能有所比较,一种是比较差的慢的方法,一种是比较好的快的方法,我们就可以选择比较好快的一种,如修铁路,一般都要选择几条路线加以比较,最后才确定。如大鸣大放好!还是小鸣小放好,要大字报不要?大字报哪个好?这里都有比较。不要大鸣大放总是放不开。北京市有几十个大专院校没有一个顺顺当当放得开,做了许多工作,一次两次会最后逼上梁山,不能不搞,否则引火烧身。过去革命也是有两种方法两种政策的,到了抗日战争时期才摸到一套正确方法,进步比较快了,建设有两种方法两种政策的,也可以有这样那样,如苏联、南斯拉夫、波兰……等国家的方法就不一样,这些国家不同方法,我们可以选择做参考,但比较起来,苏联的建设道路是最正确的。我们革命有经验,在建设上经验甚少,苏联已经四十年经验,我们只有八年,苏联经验基本上是成功的,是较完全的,将来还会有错娱,当然成绩是主要的。苏联比我们强,我们要学习苏联,学习有便宜,不学习有损失。学习有很大好处,不学习有很大坏处。特别在我们反了教条主义后,学习苏联还不会有什么危险,学习对我们只有利而不会吃亏。因为苏联的经验最完全,比如人造卫星,现在大家都说它行,过去中国就有人(龙云)说它不行,美国也看不起它,可是他就放出了这个东西,五天了还在转。我过去也不相信什么人能到月亮去旅行之类的宣传,现在相信了,因只懂社会科学不懂自然科学,我不是专家,只红不专,现在看来,月亮可以去,金星、火星也可以。他们说五年到十年可以去月亮,我现在如还不相信,那就变成顽固派。但是,是不是可以将苏联的弯路撇开,比苏联搞得更快一点,质更好一点,苏联从一九一八年底才有四百万吨钢,再经过二十年只搞到一千八百万吨钢,我们三个五年计划或更多一点时间达到二千万吨钢,这就说明我们可以搞得更快一些。我主张今后小钢厂多开一点有好处,比如二、三十万吨,三、四十万吨很有好处。

五、去年扫掉了,把多、快、好、省扫掉了。好省没人反对,但多快不提了,实际上多、快、好、省是互相制约的,好省是限制多快的。不像卫星跑的那样快,不切实际的多快,即不可能多快,那不能搞。但反过来,好是好,省是省,就是那么一点点,慢得要死,那也不行。我高兴的是这次会有个别同志提出了多快好省这个完全的口号。我看要实事求是,要提倡合乎实际的不是主观主义的多快。去年把这个口号扫掉了,现在要恢复这个口号,你们赞成吗?还扫掉农业纲要四十条,四十条不吃香了,现在又复辟了。

共产党应该永远是促进会。我们有许多委员会,最根本的是共产党委员会,究竟你是个促进委员会,还是个促退委员会?共产党是促进委员会,国民党是促退委员会。去年二中全会组织个小组促退,就是去年多花了三十亿,就只能在这点促退一下,多促退就要犯错误,也就是说,不可过分强调反冒进。因右派是促退派,我们与右派的根子不同,我们永远是促进派,要促退只是暂时的、局部的,如农业发展纲要上的六百万部双铧犁被促退了,消灭蚊子、苍蝇、老鼠、麻雀我是始终坚持的。

六、关于资产阶级无产阶级两条道路的矛盾问题。这是主要矛盾,毫无疑问。过去是反帝反封建,已经解决了,现在是社会主义革命,社会主义革命的主要锋芒,是消灭资产阶级剥削制度,在农村中是改造小资产阶级,中心问题是以合作化解决个人主义与集体主义,社会主义与资本主义的矛盾。

七、八大决议没有否定社会主义过渡时期无产阶级与资产阶级为主要矛盾的阶级斗争,八大分析是指生产力讲的!先进的社会制度与落后生产力的矛盾,虽然这句话说的不够完善,但得到了好处,并未发生毛病。七届二中全会指出国内基本矛盾是资产阶级与无产阶级的矛盾,现在看来是非常正确的。


