\section[在北戴河政治局扩大会议上的讲话(五)(一九五八年八月三十日上午)]{在北戴河政治局扩大会议上的讲话(五)}
\datesubtitle{(一九五八年八月三十日上午)}


人民公社是群众自发搞起来的,不是我们提倡的。我们提倡不断革命,破除迷信,解放思想,敢想、敢说、敢做,群众就起来了。南宁会议、成都会议、“八大”二次会议都未料到。共产主义本来是有群众自发的因素,先有空想社会主义、古典唯物论、辩证法,然后再由马克思那些人总结出来的。我们的人民公社是在合作社的基础上发展起来的,不是没有来由的。把这个问题条理化,需要我们去搞清楚。人民公社的特点是,一曰大,二曰公。地大物博,人口众多,工农商学兵,农林牧副渔,大,了不起,人多势众。公,就是社会主义比合作社多,把资本主义的残余逐步去掉,如:自留地、私养牲畜取消,搞公共食堂、托儿以、缝纫组,全体劳动妇女可以解放。实行工资制度,搞农业工厂,每个男人,每个女人,每个老年,每个青年,都有工资,发给每一个人,和以前分配给家长不同,直接领取工资,青年、妇女非常欢迎,破除了家长制度,破除了资产阶级法权制度。还有一个公的特点,是劳动效率比合作社可以提高。

全国现有七十万个合作社,搞成万人,万户的大合作最好。河南提出二千五百户左右一个,当然也可以。这是一个新问题。只要一传播,把道理一讲,可能只要几个月,一秋一冬一春可能就差不多了。当然,离实行工资制,吃饭不要钱,还要一个过程,也许一年,也许有些人要三年。决议案上有句,一、二年或者四、五年,或者更多一点时间,由集体所有制过渡到共产主义所有制,和工厂差不多,即是吃饭,穿衣,住房都公有。苏联还鼓动私人盖房子,我们将废除私人房屋。

绿化问题:园林化,城市乡村都像中山公园、颐和园。中山公园不出粮食不好。中国刚建设,要想建设得怎样更合理,更好些。有人说,城市工厂占地更多,农村就不同,中国每人三亩地,我们两亩就够了。几年后亩产提高了。可以拿三分之一种树,三分之一种粮食,三分之一让地休息,地也轮班。假如亩产万斤,达到现在的“卫星”时,一亩等于四十亩、八十至九十亩,还种那么多于什么呢?种树要有规划,有计划地种。法国人把街道、房屋、林荫搞得很好,资本主义能搞,为什么我们不能搞?应当把它搞得有秩序一点。康有为咏西湖的一付对联:“如此园林,四洲游遍未曾见。”其实何必游四大洲,我们绿化起来,全国到处可以游。何必一定游西湖?西湖水浅,林也不好。房屋要好好安排一下,今年大搞还不行,有些今年开始,有些明年开始。如果搞××斤粮食(今年可能是××斤,明年加一番)的话,我们就可以搞规划,园林化、绿化、畜牧、住房等。河北、河南我看了一下,什么绿化?没有树怎样绿化?真正绿化,我看每人有了几千斤粮食,腾出三分之一地来种树,才能大搞绿化.农、林、牧是互相结合,互相影响的。

人民公社还有许多问题,现在不知道,还需要继续研究。已经有了一个章程,河南卫星公社:十四条,它的“宪法”一公布,全国闻风兴起的就会不少。人民公社在两、三年(明年、后年)内能不能由集体所有制过渡到全民所有制?实行土地国有,工资制,办农业工厂。有个文件写第三个五年计划向共产主义过渡,我加了个第四、五个。有个文件讲,明年是决定性的一年,这句话讲得好,粮食再翻一番,钢搞到××到××吨,争取××吨,这是一场大仗,还是没有休息的,机器不能休息。今年还有四个月,我犯了错误,早抓一个月就好了,六月十九日出了题目,但没有具体措施,大家都抓计划去了。热情是好的,但对今年的生产有所放松,我没有搞好,责任是我的,不是大家的,从八月二十一日起,还有十九个星期。一百三十三天,一天不多,一天不少,现在又过去十天,相当危险,要紧急动员,能否完成,我有怀疑。我是“观潮派”,明年一月一日能不能搞到,我总是十五个吊桶,七上八下,如果没有搞到,一是题目出错了,二是工作没有抓紧,是我的错误。冶金部汇报讲九百万吨,我说:干脆一点吧!翻一番,何必拖拖拉拉呢?摘一千一百万吨,问了许多人。都说可以,有希望。一九五六年粮食增产轰轰烈烈,有人说一九五七年的粮食生产,比一九五六年更扎实得很,实际上增产不多,只增产××斤。今年一千一百吨钢,到底扎实不扎实。我是怀疑的,拿到手才算数。“钢铁尚未成功,同志仍须努力。”明年××吨,后年再增××吨,苦战三年,达到××吨,基础就搞起来了,剩下两年,到六二年搞到××到××吨钢,就接近××。七亿人口要多少钢,我看一人一吨,搞它七亿吨。粮食比钢少一半,搞三万五千亿斤。粮食产品要多样化,不要光地瓜。

共产主义的第一个条件,是产品丰富,第二个条件是要有共产主义精神。一有命令,每个人都自觉地去工作,没有懒汉。共产主义不分高低,我们有二十二年的军事共产主义生活,不发薪水,与苏联不同,苏联叫余粮征集制,我们没有搞,我们叫供给制,军民一致,

官兵一致,三大民主。我们原来分伙食尾子,津贴,进城以后,熬了三年,到五二年搞了薪水制,说资产阶级的等级、法权那样神气,把过去的供给制说成是落后的办法,游击习气,影响积极性。其实是把供给制变成资产阶级的法权制,发展资本主义思想。难道二万五千里长征,土改革命,解放战争是靠发薪水发过来的吗?抗战时期,二三百万人,解放战争时期,四、五百万人,是军事共产主义的生活,没有星期天,在一元化领导之下,没有什么“花”,官兵一致、军民一致、拥政爱民,把日本鬼子打走了,打败了蒋介石。打美国的时候也没有“花”,现在有“花”,发薪水都要有等级,分将、校、尉,可是有的还没有打过仗。结果是脱离群众,兵不爱官,民不爱干。因为这一点和国民党差不多,衣分三色,食分五等,办公桌、椅子也分等,工人、农民不喜欢我们,说“你们是官一一党官、政官、军官、商官”,其官之多,怎么不出主义?官气多,政治少,所以出官僚主义。整风以来,就是整官气,政治挂帅。争等级、争待遇就不多了。我看要打掉这个东西,薪水制可以不要马上废除,因为有教授,但一、二年要作准备。人民公社搞起来,就逼着我们逐步废除薪水制。进城后受资本主义影响,我们搞运动,本来是马克思主义的东西,是民主作风,他们说我们是“农村作风”,“游击习气”,跟资产阶级、土豪劣绅搞在一起,正襟危坐,学资产阶级的样子,剃头,刮胡子,一天刮三次,都是从这里学来的。解放初,五零年、五一年扭秧歌时期,我们压倒资产阶级,后来秧歌吃不开了,梅兰芳出来了(宇宙锋)压倒了秧歌。本来是与马克思主义的那一些,吃不开了,现在又恢复了“农村作风”,“游击习气”,是马克思主义作风,讲平等,官兵一致,军民一致,没有星期天,老百姓说:“老八路又回来了。”

我请陈伯达同志自己编了一本书,《马、恩、列、斯论军事》,我读了一二篇,有一篇说,许多东西从古就是从军队首先执行的。我们共产主义也是从军队首先实行的。中国的党是很特别的党,打了几十年仗,都是实行共产主义的。八年抗战,四年自卫战争,群众看到我们的生活很艰苦,群众支援前线,没有工资,粮食自带,打仗要死人,还能那样作。有人说,平均主义出懒汉,过去二十二年,出了多少懒汉,我没有看见几个,这是什么原因?主要是政治挂帅,阶级斗争,有共同的目的,为多数人而辛苦。现在,对外有与帝国主义作斗争,对内主要是向自然作斗争,目标也明确。我们现在搞生产建设,全国一千多万干部,是为谁服务呢?是为了人民的幸福,不是为了少数人的幸福。现在发明一个东西,要给一百块钱,倒是会出懒汉,争吵,不积极。过去创造发明多的很,哪里是钱买来的呢?计件工资不是个好制度。我不相信,实行供给制,人就懒了,创造发明就少了,积极性就低了。因为几十年的经验,证明不是这样的。

人民公社,有的地方采用军事组织一师、团、营、连,有的地方没有,但“组织军事化,行动战斗化,生产纪律化”这三化的口号很好,这就是产业大军,可以增产,可以改善生活,可以休息,可以学文化,可以搞军事民主。似乎一讲军事就没有民主,恰好民主出在军队,即军事、政治、经济三大民主,战斗中可以互助,官长压迫士兵在我们军队中是犯纪律的,不名誉的。公社“三化”很好。这几年来,学了那一套,一从资产阶级一一本国生长的,二从无产阶级一一苏联老大哥,好在时间不长,根未扎深,命还好革。整风以来,各种规章制度革得差不多了。资产阶级那一套,去掉了不少。这回军委开会取消“花”。干部参加劳动,写了一个决议,中央委员每年一个月,其他干部还多,年老有病的除外。种试验田,何止一个月呢?云南有一个师长,到连队当了一个月的兵,我看所有的“长”一一军长师长等,都至少当一个月兵,头一年最好搞两个月,要服从班长、排长指挥。有些是当过兵的。现在有多年不当兵了,再去当一下。文职干部,每年至少参加一个月的劳动,修十三陵水库时,许多部长都参加劳动了。一年学农、二年学工,轮着学,总得学会两套本领。人民公社军事化,并不是资产阶级军事化,有纪律,有民主,相互关系是同志关系,是说服不是压服。劳动要有严格的纪律。

全民办工业,暂时出了一些混乱现象,界限未划清。这次会议工、农、商、学、兵都有,重点是工业。全党全民办工业,从今以后,第一书记要偏到工业方面来,过去我们偏到农业万面,拿农业压迫工业,将它的军,农业搞起来了。农业上了轨道,工业还没有上轨道。工业要抓,有人说,睡到土地上,睡到机器旁边去。就可以搞起来,不到机器旁边睡觉不行。东北三省过去抓工业,但农业未搞好,东北要一面抓工业,一面抓农业。其它各省、自治区重点抓工业。明年是决战的一年,主要指工业,首先是钢铁,机械。有了钢铁、机械,可以挖煤、发电,什么都好办。封为“元帅”是有理由的。要抓,还要抓紧,不要抓而不紧,以后考就是考这个东西。六条纪律:一警告,二记过,三撤职留任,四撤职,五留党查看,六开除党籍,不坐班房,坐班房损失劳动力。这几条都是神经战,不可少,是属于惩罚一类的性质,九个指头是说服,靠政治,凭“良心”办事,一个指头是纪律。马克思主义不是靠惩罚,靠惩罚办事就犯错误。我们党历靠说服教育和斗争,如××、×××、古大存、孙作宾,新疆的什么拉巴拉也夫,总之,大大小小有几十个,只有那么少数人,只是十个指头的一个,你说服不了他,就得惩罚。劝告警告,紧急的时候,一下撤职也是有的。××是军队中的右派(没有划右派),××是地方上的右派,王明也是右派。为什么又选王明当中央委员呢?因为他是老党员,搞了许多年,不能便宜了他,不当不行,你想不当,我想叫你当,不当中央委员就没事了,他的原则是一开会就害病,让他当,有益处。×××也是当一下好。或者改好,或者不改,总有一天要开除,这是说服与纪律的关系。

死与活的问题,不是死人之“死”,是统死统活的问题,世界上没有“死”是不行的。一千一百万吨钢,少一吨也不行,这是“死”的,明年××吨到××吨,争取××吨,其中××吨是“死”的,是“死钢”,另外××吨到××吨是活的,归地方支配。有些同志怕没有活命了,统统都活不行,要有死有活。统一计划,分级管理,各级办工业,全民办工业,有重点,有骨干,有枝叶.树长有干,才有枝叶。人靠一根脊椎骨,是脊椎动物,是高级动物。狗是一种很懂人性的高级动物,就是不懂马克思主义,不懂炼钢,和资本家差不多。

下次会议。两个半月以后再开,即十一月半在南方再开一次小型会议,时间不要这样长,因为,那时还不能总结。十月后。剩下个把月,还可以抓一下。

《马、恩、列、斯论共产主义》一书(斯大林办得不太好)请各省都印,广为散发,让大家看一下,很有启发,但又相当不足,因为那时受条件限制,没有经验,所论当然模糊不明确。不要以为老祖宗都放香屁,一个臭屁也不放,讲到将来,是一定有许多模糊地方的。苏联有四十一年的经验,我们有三十一年的经验,要破除迷信。

除四害。国庆、阳历年、阴历年抓一下,我希望四样东西越搞越少。因为这些东西对劳动人民有害,直接影响人民的健康,要把各种疾病大大消灭。杭州有个地方,去年只有一人生病,出勤率达到百分之九十以上。医生没有事做,可以去种地,作研究工作,哪一天中国消灭了四害,要开庆祝会的,历史要写进去。资本主义国家没有办到,所谓文明,可是苍蝇,蚊子多得很。