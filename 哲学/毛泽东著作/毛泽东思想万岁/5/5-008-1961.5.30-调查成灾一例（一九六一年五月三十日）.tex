\section[调查成灾一例(一九六一年五月三十日)]{调查成灾一例\\{\large——对“关于‘调查研究’的调查”的批示}}
\datesubtitle{(一九六一年五月三十日)}

如果还是如同去长辛店铁道机车车辆制造厂做调查的那些人们实行官僚主义的老爷式的使人厌恶得透顶的那种调查方法,党委有权教育他们。死官僚不听话的,党委有权把他们轰走。\footnote{5月28日 阅田家英报送的戚本禹五月十二日写的材料《关于“调查研究”的调查》和田家英报送这个材料的信。田家英的信中说:秘书室工作人员戚本禹,去年六月下放到长辛店机车车辆工厂劳动。最近他寄了一份材料给我,反映一些机关、学校人员到工厂作调查的情况。这个材料提出了一些在大兴调查研究之风中间值得注意的问题。戚本禹的材料说,他们利用业余时间摸了一下各级领导机关到长辛店机车车辆厂做调查研究工作的情况,认为在二十几个调查组的工作里,比较普遍地存在着“十多十少”的问题。毛泽东为戚本禹的材料拟了一个题目《调查成灾的一例》。批示:“此件印发工作会议各同志。同时印发中央及国家机关各部门各党组。派调查组下去,无论城乡,无论人多人少,都应先有训练,讲明政策、态度和方法,不使调查达不到目的,引起基层同志反感,使调查这样一件好事,反而成了灾难。”30日,毛泽东对这个材料再次批示:“此件,请中央及国家机关各部门各党组,各中央局,各省、市、区党委,一直发到县、社两级党委,城市工厂、矿山、交通运输基层党委,财贸基层党委,文教基层党委,军队团级党委,予以讨论,引起他们注意,帮助下去调查的人们,增强十少,避免十多。如果还是如同下去长辛店铁道机车车辆制造工厂做调查的那些人们,实行官僚主义的老爷式的使人厌恶得透顶的那种调查法,党委有权教育他们。死官僚不听话的,党委有权把他们轰走。同时,请将这个文件,作为训练调查组的教材之一。”}\footnote{见链接《毛泽东年谱(1949—1976)》http://dangshi.people.com.cn/n/2013/0527/c85037-21626561-7.html}
