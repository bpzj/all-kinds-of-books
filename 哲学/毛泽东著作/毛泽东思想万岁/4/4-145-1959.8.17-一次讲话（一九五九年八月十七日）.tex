\section[一次讲话(一九五九年八月十七日)]{一次讲话}
\datesubtitle{(一九五九年八月十七日)}


集体有一长,班有班长,连有连长,有三个党员是一个小组.要有个组长。没有集体不行,光有集体也不行,有集体就要有个长,不然就没有力量。鞍钢工厂和耕田不同,耕田慢一点,快一点没关系。工厂技术那样复杂要听指挥。像乐队,没有指挥,一群人就不知所措。这就是生产秩序,生产秩序所需要的。交通警察不要不行,几十年、几百年以后,纪律将会更加森严,上街可能要排队。不信,请你注意健康,再活一百年,就可以看到。开会要有人发通知,要有秩序,散会也要有人宣布,这是必然性。至于姓张、姓李的来主持,那是偶然性。必然性通过偶然性来表现。有统一指挥,是社会斗争、自然斗争所必需的。无政府主义者蒲鲁东、布朗基主义者“左”得不得了,巴黎公社是无政府主义者搞起来的,但是群众超出了他们的意志,被迫不得不搞。反对的人并非完全不知道船要有船长,组要有组长,无非他们要搞派别活动,反对敌对的派别。列宁当时的政治局只有五个人,在困难的时期有人反对他,说他不民主,不召开会议。列宁说会还没有开,可是革命胜利了。常委会的同志年纪都大了,总要办交代,总要新的人来代替,不能是无政府主义的,总要有组织,有领导的。资产阶级反对破坏,无产阶级不破坏就不能建设。你要讲破坏,我就不能谦虚了。列宁说你们是资产阶级、小资产阶级的派别,要你们来干是不行的。地方上也有这个问题。省、地、县都要建立一个领导核心,但要有一个过程,不然不行,每个县、公社都要如此。河北昌黎县委就没有一个领导核心。有意识地健全我们的领导机关,当挡的风挡回去,很有必要。彭德怀二十四日信没有通知讨论,我告诉两方面都要硬着头皮顶住,我们公道,××到昨天以前还硬着头皮顶住。

这次会开得很好,逐步发展,中期、后期解决了大问题,同时工作又没有耽误。我建议地方的会议不必等都到齐了才开,到一半人就可以开,有老兵有新兵。这次会议是胜利的会议。避免了党的大分裂,避免了大马鞍形。大跃进是客观形势决定了的,是群众的要求。革命和建设两个运动,搅在一起,是人与人的关系,又是人与自然界的关系。比如农械问题,东北是工业化地区,可以在第二个五年计划解决×%,其他地区也要力争嘛。国内形势是好的,有的同志欢迎中央的同志去看,有些同志不去看。同样一件事,可以有两种看法,有缺点可以改,并不难改。国际形势也是好的,印度、印尼要搞和平,但如果不准备夺取政权,是要犯错误的。这与和平民主新阶段不同,因为和平民主新阶段是为了争取时间、准备夺取政权。日本投降早了一点,再有一年我们就会准备的更好一些。社会主义时期也有两条路线,一条是一化三改,一条是现在的建设路线。去年五月的八大二次会议到今年七月的卢山会议,仅一年多,出了乱子,不是说路线不行了嘛?经过这次会议,又说行了。路线问题,是要经过考验的,并非太平无事。将来风波还是有的,但总的趋势是好的,不管出什么乱子,无产阶级的劳动人民总要占优势,即使不占优势也是暂时的。现在建设有了保证,中央委员会的绝大多数是团结一致的,但也要估计到不是那样风平浪静,自然界的台风年年要来的,政治上的台风什么时候来,料不到,但一定时期内,总会有的,因为有阶级存在,精神上要有准备。世界和平的可能性很大,但不是没有战争的可能,一定要有准备。


