\section[关于官僚主义严重存在的问题(对一个文件的批语)(一九六○年春)]{关于官僚主义严重存在的问题(对一个文件的批语)(一九六○年春)}


积极方面是形势大好,这是主要的.消极方面,突出的表现是五多,五少。就是说。会议多,联系群众少;文件表报多,经验总结少;事务多,学习少;一般号召多,细致的组织工作少,人们蹲在机关多,认真调查研究少。他们这个文件,现在发议论,你们看看。其中说到会议多和文件表报多,多到什么程度呢?他们说.县委及县委各部门,今年一月一日到三月十日,七十天中,开了有各公社党委书记和部门负责人参加的会议,共有一百八十四次。电话会议五十六次。印发文件一千零七十四件,表报五百九十五份。同志们,这种情况是不能继续下去的。物极必反,我们一定要创设条件,使这种官僚主义走向它的反面.历城县已经订出办法,克服五多五少。山东省委已将历城办法推广到全省施行。同志们,这种官僚主义状态只是存在于历城一县或者山东一个省吗?何见得?很可能到处都存在。请你们各自调查一个县,一个市(在大城市里调查一个区)就可以知道底细了。克服五多五少的办法,可以仿照历城的办法。

办法:

一,走出办公室。田间会师。

二,实行“三同”,“三包”。

三,采取在党委统一领导下的“条条”、“块块”、“片片”相结合的办法,既做好中心工作。又做好所分工的业务工作的经验。

四,立即精简会议。减少文件表报。</p>

