\section[夜学日记(一九一七年十一月十四日)]{夜学日记}
\datesubtitle{(一九一七年十一月十四日)片断}

甲班上课算术罗宗翰出席教以数之种类加法大略及亚拉伯数字码,历史常识毛泽东出席教历朝大势及上古事迹。学生有四人未带算盘,从小学暂借,为戒严早半时下课,管理者李端\{输fix\}、萧珍元。

实验三日矣,觉国文似太多太深,太多宜减其分量,太深宜改用通俗语(介乎白话与文言之间),常识分量亦嫌太多(指文字),宜少用文字,其讲义宜用白话,简单几句表明,初不发给,单用精神讲演,终取讲义略读一遍足矣,本日历史即改用此法,觉活泼得多。

本日算术却过浅,学生学过归除者令其举手,有十几人之多,此则宜逐渐加深。

