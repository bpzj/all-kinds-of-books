\section[关于统一抗日根据地党的领导及调整各组织间关系的决定(中华民国三十一年九月一日中共中共政治局通过)]{关于统一抗日根据地党的领导及调整各组织间关系的决定(中华民国三十一年九月一日中共中共政治局通过)}
\datesubtitle{(三十一年九月一日)}


抗战以来,各抗日根据地党的领导一般的是统一的,党政军民(民众团体)各组织间的关系基本上是团结的,因而支持了几年来艰苦斗争的局面,配合了全国的抗战。然而由于主观主义,宗派主义的遗毒,由于对某些政治观点与组织关系还缺乏明确的了解与恰当的解决,党政军民关系中(实际士是军员系统中党员干部的关系)在某些地区,还存在着一些不协调的现象。例如:统一精神不足步伐不齐,各自为政,军队尊重地方党,地方政权的精神不够,党政不分,政权中党员干部对于党的领导闹独立性,党员包办民众团体,本位主义,门户之见等等.这些不协调的现象,妨害抗日根据地的坚持与建设,妨害我党进一步的布尔塞维克化。根据地的建设与民主制度的实行,要求每个根据地的领导一元化。加以日寇“扫荡”的残酷、封锁线与根据地增强,上下的联系很困难,抗战的地区性与游击性的增大,要求各系统上下级隶属关系更加灵活,每一地区(军区、分区)活动独立性以及活动各方面的领导统一性更加扩大与增强,要求各地区的各种组织更加密切的配合,不给敌人以任何可利用的间隙。为此目的,中央特作如下之决定:

(一)党是无产阶级的先锋队和无产阶级组织的最高形式,它应该领导一切其他组织,如军队,政府与民众团体,根据地领导的统一与一元化,应当表现在每个根据地有一个统一的领导一切的党的委员会(中央局,分局,区党委,地委),因此,确定中央代表机关(中央局、分局)及各级党委(区委会、地委)为各地区的最高领导机关,统一各地区的党政军民工作的领导,取消过去各地区党政军委员会(党政军委员会的设立,在根据地创立时期是必要的正确的)。各级党委的性质与成分必须改变,各级党委不应当仅仅是领导地方工作的党委,而应当是该地区的党政军民的统一的领导机关(但不是联席会议),因此它的成分,必须包括党委、政府、军队中主要负责的党员干部(党委之常委亦应包括党务、政府及军队三方面的负责干部),而不应全部或绝大多数委员都是党务工作者。各级党委的工作应该是照顾各方面,讨论与检查党政军民各方面的工作,而不应仅仅局限于地方工作。

(二)中央代表机关及区党委地委的决议,决定或指示下级党委及同级政府党团、军队军政委员会军队政治部及民兵团体党团及党员,均须无条件的执行。政府、军队、民众团体的系统与上下级隶属关系仍旧存在。上级政府的决定,命令,上级军事领导的命令,训令,上级民众团体的决定(以上文件之重要者必须经过各该机关党员负责人及同级党委批准,或事先商得党委负责人同意,然后颁发,但不是一切都要批准),不仅下级机关、军队;民众团体必须无条件执行,下级党委也必须无条件执行,不得假借无上级党委指示而违抗或搁置。下级党委对上级政府、军队、民众团体之决定如有不同意见,可报告上级党委。在遵照各组织上级的决议解决具体问题而党委内部发生争论时,以少数服从多数的原则解决之。政府、军队、民众团体负责人即使不同意多数意见,亦必须执行同级党委之决定,但可将自己的意见报告上级有关机关。

(三)中央局与中央分局为中央代表机关,由中央指定之。区党委、地委,由军队与地方的党组织的统一的代表大会选出,经上级批准之。区党委、地委,应包含地方党的组织,军队党的干部与政府党团的负贵人,主力军是否参加县委,由各地按具体情况决定,县委(无主力军参加之县委)及区委,只包括地方党,地方军及政府的党的负责人。各级党委书记,应选择能掌握党政策及各方面工作的同志担任之。因此,党委书记不仅须懂得党务,还必须懂得战争和政权工作。区党委书记人选,由中央局分局议定,经中央批准之。地委书记人选,由区党委议定,经分局中央局批准之。为统一地方党与军队党的领导,分局、区党委,地委书记,兼任军区、分区(师或旅)政委,另设副书记,管理党务工作。如军区、分区政委被选为分局、区党委、地委书记,则可设副政委,专管军队工作。分局、区党委、地委书记应照顾各方面工作,除兼政委外,再不宜兼任其他具体工作。如有个别特殊情况党委书记不必兼政委或政委不必兼党委书记时,须得上级党委或中央批准,军队中军政委员会及政治部,成为同级党委(中央局、分局、区党委、地委)的一个部门,与其他部门(如组织部、宣传部等)有平等权利和义务,不隶属其他部门或委员会,但与其他委员会和其他部门不同,仍保持其上下级直接领导和隶属关系。军事政策(扩兵建军原则,政治工作等等)与军事行动的大政方针(如反扫荡的战略战役计划及总结等),须交党委会讨论,但具体军事行动由司令员政治委员(即党委书记)决定之(司令员与政委对军事行动之最后决定权仍照政治工作条例),无限制的民主讨论只会引导军事的失败。军队主要人的任免,仍须经过军事机关依照已定规则进行之。

(四)主力军是党领导下的武装部队,是建设根据地与支持斗争的有力柱石。主力军应以巩固和坚持各所在根据地为其第一等任务。主力军固全国性,但同时具有地方性。过去有些根据地领导不统一,主要的是由于该地主力军某些领导同志对于该根据地之建立与坚持缺乏正确的统一的认识,因而其所实行的某些政策(如军事建设中之地方武装问题、扩兵问题,财政经济政策问题中之统筹统支等)只注意了主力军,而忽视了根据地,整个工作的配合,因而与地方党地方政府发生争执。同时有些地区,党政领导机关对于一切服从战争的认识不深刻,对主力军人员之补充,粮食及物质保证,优抗及反逃亡斗争,未能尽到应有的责任,因而使军队与地方党政间的关系不协调。今后为了实现根据地领导一元化,除实现以党委为各地统一的党的领导机关外,还必须纠正某些地方主力军,某些地方党政机关领导人思想上政治上的一些错误,实现中央关于根据地的各种政策。实行军委关于军事建设的几号指示。今后主力军必须执行各级党委的决议、决定与各级政府的法令。主力军对于驻扎所在地的下级党委与下级政府(如县、区,乡)的决定,亦必须执行。如有不同意处,可报告上级党委与上级政府。并应当经常的彼此联系,彼此帮助。主力军的军事措施如军事行动,布置及戒严令等等,地方党政民机关必须遵照执行。主力军应当负有保护党政民机关的责任,凡因军队之疏忽与漠不关心,因而使军政民机关受到不应有的损失时,军队负责人应当受到处分。今后,如有争执,首先应当就地协同解决,并将争论及解决经过报告上级,反对各个组织只是向上告状,而不在本地当面商谈解决问题的办法。

(五)政权系统(参议会及政府)是权力机关,他们的法令带有强制的性质。军委及政权系统的关系,必须明确规定。党委包办政权系统工作,党政不分的现象与政权系统中党员干部不遵守党委决定,违反党纪的行为,都必须纠正。为了实行三三制,党对政权系统的领导,应该是原则的,政策的,大方向的领导,而不是事事干涉,代替包办。下级党委无权改变或不执行上级参议会及政府的决定与法令,党的机关及党员应该成为执行参议会及政府法令的模范。党应该进行政治工作以提高参议会及政府的威信,党的干部及党员违反参议会及政府法令时,党的组织应给予严厉的处分。党对参议会及政府工作的领导,只能经过自己的党员和党团,党委及党的机关无权直接命令参议会及政府机关。党团必须服从同级党委,但党团的工作作风必须刷新,不是强制党外人士服从,而是经过自己的说服与政治工作。在党团万一没有说服参议会及政府的大多数因而党团意见未被参议会及政府通过时,必须少数服从多数,不得违反民主集中制的原则。但是假如党团同志因为自己的意见与同级党委有分歧因而不坚决执行党委的决定,这是党团同志违反党纪的行为,应该受到指斥与处分。党必须派遣得力的干部到参议会及政府中工作,一切忽视政权工作,把干部堆在党的机关中的现象,必须纠正。在实行三三制时,党员在政权系统中的数量减少,但在政权系统中工作的党员质量必须大大提高,在政权系统中工作的党员和干部,必须服从党委与党团的决议、决定与纪律,不得利用自己的地位自由行动。在这里,应特别提出,党对三三制政权之领导的实现,有赖于政权系统中党员干部之言论行动的一致及其对党的决定的绝对服从,所以严整政权中党员及党员干部的党纪是有严重意义的。党委在调动政权系统中的党员时要慎重,还必须经过党员实行向政权机关辞职的手续。政府与军队的关系必须改善;军队中应进行拥护政府的教育;政府应保证军队的给养与运输;军队的首长应被选为政府委员及参议员。军队应成为尊重政权执行法令的模范,军人的违法行为,军事机关必须严格处分。军人除以公民及政府委员,参议员资格对参议会及政府发表意见外,一切军事机关无权干涉参议会及政府内部工作。但军队政治机关,必须尽可能的帮助政府工作。

(六)民众自己的自愿组织的团体,党、政府、军队不应直接干涉民众团体内部的生活,党对民众团体的领导,经过自己的党员及党团。但党民不分、包办、青一色的现象,必须纠正。民众团体的各级委员会须尽可能有半数以上的非党员。民众团体中的党团问题与政府中党团同。政府应尊重民众团体的独立工作。给民众团体以必要的帮助,要求民众团体执行政府的法令。民众团体应依法向政府请求登记,取得合法地位。如民众团体违反政府法令时,政府可加以处分,甚至解散,此外一律不干涉民众团体的生活与工作。民众团体应号召民众,拥护政府和军队,帮助抗战动员工作。但民众团体并非政权机关,不得代替政府行政及对人民执行逮捕、审讯、判决等事宜。军队与民众团体应相互帮助,但不应相互干涉。

(七)在游击区因为它的特殊性,领导的一元化不仅是在相互关系上应有所确定,而且在党政军民的机构上在必要时亦须一元化。党委、政府、民众团体的机关,可与军队指挥机关政治工作机关合并。党政民干部在军队或游击队中,担任一定的职务(如正副军事指挥员、政委及政治部各种工作),战时参加军队与游击队工作,战斗空隙时则仍执行其原来的党政民的职务(如党委书记、县长、工会主席)。

(八)党的领导的一元化,一方面表现在同级党政同级组织的相互关系上,又一方面表现在上下级关系上。在这里,下级服从上级,全党服从中央的原则之严格执行,对于党的统一领导,是有决定意义的。各根据地领导机关在实行政策及制度时,必须依照中央的指示。在决定含有全国全党全军普遍性的新问题时,必须请示中央,不得标新立异,自作决定危害全党领导的统一(凡带地方性的不违反上级及中央决定的不在此例)。下级党政军民组织对上级及中央之决议、决定、命令、指示,不坚决执行,阳奉阴违,或在解决新的原则问题及按性质不应独断的问题时,不向上级及中央请示,都是党性不纯与破坏统一的表现。

在这里,应当再一次的提醒各地党政军民领导同志的注意,各级党委及政府军队民众团体中的党员负责同志,不得中央许可,不得发表带有全国意义和全党全军意义的宣言、谈话及广播,各级领导同志的文章应经过同级党委或党团适当人员的审阅。分局委员以上,师以上负责人的文章,凡带有全国及全党意义的,应事先将主要内容报告或电告中央。各级不应再直接对外广播,应统一于延安新华社。应当深刻认识,一个党的负责高级干部,不经过同级或上级一定组织的同意,而擅自发表政见,是何等违反党的组织原则,何等妨碍党的统一的恶劣行为!

(九)为统一根据地的领导,为改进党政军民的关系,必须在党政军民各系统党员干部中进行思想教育,整顿三风,肃清主观主义宗派主义的遗毒。在干部会议上,根据中央决定与毛泽东同志报告,教育干部识大体,顾全局,号召干部实行批评与自我批评,使干部管得全局,不陷于局部与本位的偏向,而懂得全体与局部、上级与下级,这一局部与那一局部间的正确关系。要加强党政军民各组织中的教育工作,使全体同志认识领导一元化及根据地革命秩序与革命法令的重要性。在这里,应特别警惕军队干部,党政军民关系不协调,在一般情况之下,军队干部应负担较大的责任,军队手中有枪,容易独断独行,轻视党政,不守纪律。自由行动,破坏群众利益。因此,军队中军政干部必须特别约束部下,检点自己。必须号召自己的部下,拥护党的领导,拥护政府,坚决执行党的决定与政府的法令,同时,又应纠正某些党员和干部中不合事实的观点:认为只有在党务及党的机关中工作才叫党的工作,其实一切党员和党的干部不管其执行的是政府工作,军事工作,群众工作,经济工作,技术工作,文化工作等等,都是党的工作,在党的机关的工作,只是党的工作的一部分(党务工作)。党员服从党的领导,是服从党的路线、政策、决议、决定、指示与纪律。某些党员空喊或曲解服从党的领导,而对于党的路线、政策、决议、决定、指示与纪律,则不认真研究,不认真执行,这种态度是不正确的。

(十)加强各抗日根据地领导的统一,是为了更顺利的进行反对日寇的战争,“一切服从战争”是统一领导的最高原则。要在全党中说明,假若军队削弱,假若战争失败,则根据地无法存在,党政军民都会塌台,因此,党委、政府、民众团体以及全体人民,都有巩固军队,加强其战斗力的义务。军队的人员补充,粮食、服装、弹药的供给与运输,营舍的让予,伤病残废人员的输送、看护与保养,抗日军人家属的优待等等,党委及政府、民众团体,应有随时加以解决的责任,一切漠不关心的现象,都是极端错误的有害的。在军队本身,则都深深了解,没有党、政府、民众团体的配合,光杆军队是一天也不能支持抗战的。因此,必须加强部队中的教育,做到能爱惜根据地,爱惜人力物力,尊重党政,加强军纪,给党政民以必要的帮助。军队的人员物资补充、运输、优待等等,必须依照政府的法令、规章去做,乱来一顿,只会损害抗战,于军队本身是不利的。

(十一)各根据地领导机关根据本决定的原则,根据各地具体情况,制定与此有关的各种细则,以政府法令、军队条例,党团规则、民众团体章程等等方式规定之,以解决统一领导的许多具体问题。制定后须报告中央。

