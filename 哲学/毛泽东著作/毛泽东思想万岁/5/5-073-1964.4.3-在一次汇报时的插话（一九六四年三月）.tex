\section[在一次汇报时的插话(一九六四年三月)]{在一次汇报时的插话}
\datesubtitle{(一九六四年三月)}

看到你的信,你们想找我谈一谈。最近因为搞反修斗争,等等,好久没有找你们谈了。

你们看,我们跟赫鲁晓夫斗争,能否取得胜利?我们跟敌人斗争了一辈子,敢跟帝国主义斗争,也打败了帝国主义,我们就不能打胜赫鲁晓夫?

我们现在主要是对帝国主义、修正主义作斗争。至于反动派,如尼赫鲁,那不算什么!

(谈到有人在一九六○年上海会议上提出粮食产量的高指标问题时)

真理,一切真理,开始的时候,总是在少数人手里,总是要受到多数人的压力。四百年以前,波兰人哥白尼,他是个伟大的天文学家,发现了地球是动的。他一生最大的成就,是以科学的日心地动学说,推翻了在天文学上统治了一千余年的地心天动学说。当时宗教界群起而攻之,都反对他,说他是异端,他是一直受压迫的。他的《天旋论》,一直到他临死前(一五四三年)才出版,他高兴了。当时意大利的伽利略(一五六四——一六四二年),是一个卓越的物理学家和天文学家,他赞成哥白尼“太阳中心说”的意见,从一六○九年起,他自制望远镜观察天空,看星球是否动的,但是,他受到当时宗教界的迫害,受到罗马反动法庭判罪。另一个人是被火烧了。烧死一个人算什么!真理还不是在他手里头?!烧死一个人,地球还是动的。发明安眠药的是德国人,是个药店子的药剂师。他们几个人在药店试验,开始他们的目的,是想减少妇女生育的痛苦。他们经过了多少次的试验,有一次八个人都中了毒,几乎死了,但终于发明了安眠药。可是德国人不准他们制造和推销。法国人买了他们的发明的专利权,把这个药剂师请到法国,开欢迎会,这才推广了。也很奇怪,在那一个地方不灵,在别的地方就灵起来了。这种事情也很多,比如说,佛教是印度发明的,可是在印度并不那样吃得开,到中国和其他地方就灵了。又比如,现在马克思列宁主义在欧洲和苏联就不灵,到中国就又灵了。达尔文,他本人也是信仰宗教的,他的《物种由来》出来之后,受到宗教界的迫害,都反对他。

(谈到社会主义教育问题时)

最近我们讨论了一次,批了两个文件,从中央委员到县委委员,都到群众中宣读,用一两年的时间读完。我批了,凡不是年老有病的(比如徐老、吴老),凡不是不认识字的,在群众中有威信的(就是说不是右派,比如彭德怀不要去了),都去读。军队中的将军们都下去读了,说行嘛!其他的人为什么不行?

实际上,向群众宣读文件,就是向群众学习。你要到那一个地方宣读.你就要首先进行调查研究。

“四清”,“五反”,这都是群众教给我们的,我们的头脑中并不产生什么。“四清”就是保定地委提出的。河北省有八个地委,只有保定地委提出。保定地委开始也是不懂得搞“四清”,后来群众提出,非搞“四清”不行,他们接受了。干部参加劳动,是山西昔阳县教给我们的,以后又有浙江省的几个材料。

(谈到全国现在正掀起学习毛主席著作的热潮时)

“毛选”,什么是我的!这是血的著作。苏区斗争是很激烈的,由于王明路线的错误,不得不进行两万五千里的长征。“毛选”里的这些东西,是群众教给我们的,是付出了流血牺牲的代价的。

有些文章应该再写,把新的东西写进去。“两论”是几十年前写的东西,现在一切都发展了,内容更丰富了,应该重写。

(谈到一九五八——一九六○年三年大发展中间的经验教训时)

大有好处。不经过这么一次是不行的,是学不会建设的。搞全国规模的建设,我们没有经验。革命肘期,我们有些根据地搞经济建设的经验。那时,最迫切要解决的问题是三个:一要吃,二要穿,三要盐。因此,就必须发展生产。这就是我们当时搞经济建设的由来。

土地改革纲领(一九三三年文件),我在这前后费了十年功夫。不费十年功夫,是搞不出来的。在大革命时,我办了两次农民运动讲习所,广州一次,武汉一次,也做过一些调查研究,但还没有解决。还是以后在兴国和其他地方进行了八个调查,长岗乡调查,才溪乡调查,才解决了问题。这是群众教给我的,说应该这么样办。

我们学会革命,从一九二一年开始,到一九四五年七大。用了二十五年的功夫。延安整风的时候,我们知道了陈独秀的右倾机会主义,也知道了三次“左”倾路线,特别是王明路线,我们总结了这些经验。所以,我们到抗日战争结束时,能够发展到一百二十万军队,民兵还不在内。七大开得很好,统一了思想,团结了全党。当然,也还有些问题,比如高岗、彭德怀,但我们还信任他们。彭德怀以后当西北野战军司令员。一九四六年跟国民党是小打,一九四七年七月就开始反攻,每月消灭它八个旅,可灵咧!到一九四八年,逐步打下了石家庄、济南,以后就是三大战役。

学会打仗,是用了十五年的功夫。我开始不会打仗,也没有想过要打仗。大革命失败了,我们当时有五万党员,分成几部分,一部分被杀了,一部分投降了,一部分不敢干、逃跑了,只剩下一、两千人。七大统计时,还有八百人。这几年,除了老死的,只有六百人了;井冈山的人也只有三十个人了。在那个时候,逼上梁山,非拿起枪学打仗不行。也没有进过什么军事学校,住过军事学校的是少数。学会打仗,主要是蒋介石这个“老师”教给我们的。他把苏区打垮,叫我们进行两万五千里长征,三十万军队,到达陕北时只剩下两万多人。而这两万多人,还并不都是长征来的,是经过陕甘边境庆阳、关中的云阳和东征发展来的。当时我说,这两万多人是比三十万人强了,而不是弱了。走了两万五千里,腿“讲话”了,发言了。这样我们的脑子就要想一想,遵义会议就开成了,才改过来。学会打仗,什么都是逼出来的。

(谈到学习解放军、学习石油部,要学他们的革命精神、打歼灭战时)

不能急。农村社会主义教育运动要打个歼灭战,没有这么个×、×年功夫不够。至少四年,去年一年,今年一年,明年一年,后年一年。不能急。学习解放军,学习石油部,也大概需要×、×年,×、×年,才能全学到手。也不能急。有的省今年就要把社会主义教育搞完,太快了。你没有那么多好干部嘛!工业基本建设也是这样,也不能太急,×年、×年建成(指年产××万吨以上的煤井,过去一般要×年才能建成),这就算是多快了,太急了不行。你逼得厉害,他就要弄虚作假。

(谈到什么叫做拥护总路线、大跃进,人民公社时)

阶级斗争、生产斗争、科学实验三者必须结合。只搞生产斗争、科学实验,而不抓阶级斗争,人的精神面貌不能振奋,还是搞不好生产斗争、科学实验的。只搞生产斗争,不搞科

学实验,行吗?只搞阶级斗争,而不搞生产斗争、科学实验,说“拥护总路线”,结果是假的。我说,石油部作出伟大的成绩,它既振起了人们的革命精神,又搞出了××万吨石油;而且不只是××万吨石油,还有××万吨的炼油厂,质量是很高的,是国际水平。只有这样,才能说服人嘛!

(谈到某同志革命意志衰退,需要提拔青年干部时)

有些人到底是有病,还是革命意志衰退?还是一礼拜跳六次舞?!还是爱美人、不爱江山?!说是病得不得了,不能工作,能病得那样厉害?!……像某些同志,到底是爱美人,还是爱江山?!我看叫他搞××不一定能搞好,要给他配个“宰相”。

多年提倡下去调查研究,就是不下去。搞了多少年工业,并不知道什么叫工业。不懂机器,不懂设备,怎么行?!

现在必须提拔青年干部。赤壁之战,群英会,诸葛亮那时是二十七岁,孙权也是二十七岁。孙策干事时只有十七、十八岁。周瑜死时才不过三十六岁,那时也不过三十岁左右。曹操五十三岁。事实上,年青人打败了年老人。“长江后浪推前浪,世上新人赶旧人”。

(谈到大寨生产队的陈永贵时)

可不要看不起老粗。全国人代会开会时,我的一个同学×××,现任湖南省副省长,他要跟我谈一谈。他说,现在了解到了,知识分子是比较最没有知识的,历史上当皇帝的,有许多是知识分子,是没有出息的:隋炀帝,就是一个会做文章、诗词的人;陈后主、李后主,都是能诗善赋的人;宋徽宗,既能写诗又能绘画。一些老粗能办大事:成吉思汗,是不识字的老粗;刘邦,也不认识几个字,是老粗;朱元璋也不识字,是个放牛的。我们军队内,也是老粗多,知识分子少。许世友念过几天书!×××没有念过书,韩先楚、陈锡联也没有念过书,××念过高小,刘亚楼也是念过高小。当然,没有几个知识分子也不行。林彪、徐向前、×××、×××、……,我们算是中等知识分子了。结论是:老粗打败黄埔生。

(谈到现在风气不错,大家都愿意进行批评和自我批评,愿意向别人学习时)

凡事情都是一分为二的。我这个人也是一分为二的。我是个小学教员,小时候也信过神仙,跟我母亲朝过山,在十月革命以前并不知道有马克思,知道有马克思是以后的事情。

哪里有没有错误的人呢?我们有些同志就是喜欢形而上学。什么叫做形而上学?就是片面性,就是只准说好的,不准说坏的,只爱听好的,不爱听坏的。前年,一九六一年,××部门就是听不得批评。像×××同志,这是个好同志,但就是不愿意让人家看他们的坏的,只愿意让人家看好的,生怕触着痛处。

马克思也是一分为二的。马克思的哲学,是从黑格尔和费尔巴哈学来的,经济学是从英国李嘉图等学来的,又从法国学了空想社会主义。这都是资产阶级的。从这里一分为二,就产生了马克思主义。请问,马克思他小时候,是否读过马克思主义的著作?

我们这个党也是一分为二的。

在反一次“围剿”之前,有人说,搞军队非打人不可,不打人怎么能带动军队?!那时,军阀主义可厉害咧!士兵说:“爱兵爱兵,连长骑马”。这句话不对,连长应该骑马。

彭历来是闹分裂的。在中央苏区时,立三路线来了,他们可“左”咧!要打大城市,打九江、武汉、长沙。我说不行,他们说非打不行。当时有个吉安地委书记李文林,也给中央写了信,说分土地、土地革命发展和巩固并重是农民意识,说先打吉安后打九江要断送革命。这是说非打九江不可,“左”得很!十年内战,党内斗争可厉害了。

五中全会选张闻天为政治局委员,那时张并不是中央委员。现在查,张是否党员,何时入党,何人介绍,都查不出来。但是,那时却选他为政治局委员,反而不让我这个政治局委员参加会议。

长征到遵义会议,情况有些改变。王明路线,应该有个分别,遵义会议前和会议后不同。

跟四方面军会合,我们讲老实话,告诉张国焘,说我们出发时是八万人,现在只有三万人了。讲老实话嘛!那时四方面军还有八万人,张国焘就向我们要领导权,我们不给。张国焘他的错误,是路线错误。

以后,就到了陕北。抗战中间也并不是没有问题的。有王明路线,还有彭德怀的百团大战之类的东西。七大前,开了斗争彭德怀的会议。他在庐山会议不是说,你们骂了我四十天,我也骂你们二十天。延安斗争会,你们参加了嘛!他就是不分散(指百团大战),要搞集中。实际上,那时一个排分散出去,就可以发展成一个团、一个师。

解放以后,还不是一分为二?高饶反党集团,一九五三年是一个大暴露。财经会议时,他们说,××、××等是一个宗派。我谈了,中国革命就是许多山头闹成的,没有山头,那有革命?我们那时又没有共同纲领。

彭与高岗是在陕北结合到一起的。没有想到,邓华也跟他们搞到一起。邓华跟我谈过话,他觉得井冈山没有山头,很没味道,以后就找彭去了。死了的那个陈光,也感到没有山头,不满意。

一九六二年,又闹不讲阶级、不讲阶级斗争,各部门可不稳呢!邓子恢要搞“包产到户”。王稼祥过去从来是有病,那半年没有病了,就是要“三和一少”。可积极哩!我们现在就是要“三斗一多”。绕战部要把资产阶级的政党变成社会主义的政党,并且定了五年计划,软绵绵地,软下来了,就是向资产阶级投降。那时他们在国际上是要搞“三和一少”,在国内是要搞“三自一包”。彭德怀的反攻书,也是那个时候出来的。习仲勋他们的《刘志丹》一书,也是那个时候出来的。

(谈到读书时)

愚公移山,是有道理的,在一百万年或者几百万年以内,山是可以平的。愚公说得对:他死后有他的儿子,他的儿子再生儿子,孙子也再生儿子,子子孙孙一直发展下去,而山不增高,总有被铲平的一天。

哲学讲半个钟头就行了,讲久了反而讲不清楚。书也不要读得太多,读几十本就够了,越读多越不清楚。

(谈到农村粮食收购、换购的)

有些地区不搞基本口粮,我不赞成。要搞基本口粮。
