\section[国民党右派分离的原因及其对于革命前途的影响(一九二六年一月)]{国民党右派分离的原因及其对于革命前途的影响}
\datesubtitle{(一九二六年一月)}


有些人说:国民党现在又分离出去一个右派,这里党里左派分子的操切,这是中国国民党与中国国民革命的不幸。这个意见是不对的。半殖民地中国的国民革命政党,在今日应有这个分裂。这是一种必然的现象。我们虽不必以此为喜,却断不是什么不幸的事。要知道这个理由,只要一看近代的时局,只要一看从兴中会到现代的中国国民党的历史,就可以完全明白,十八世纪末期至十九世纪中期欲英日本资产阶级反抗封建贵族阶级的民主革命,与十九世纪末期至二十世纪初期殖民地半殖民地的小资产阶级半无产阶级无产阶级合作反抗帝国主义及其工具官僚军阀买办地主阶级的国民革命,性质完全不同。不但如此,辛亥年的革命,与现在的革命,性质也不相同。前代英法德美国本各国资产阶级的革命,为资产阶级一阶级的革命;其对象,是国内的封建贵族,其目的是建设国家主义的国家即资产阶级一阶级统治的国家,其所谓自由平等博爱及当时资产阶级用以笼络欺骗小资产半无产无产阶级使为已用的一种策略;其结果是达到了他们的目的建设了国家主义的国家;其终极是发展了全世界的殖民地半殖民地造成了国际资本帝国主义。现代殖民地半殖民地的革命,乃是小资产阶级半无产阶级无产阶级这三个阶级合作的革命,大资产阶级是附属于帝国主义成了反革命势力,中产阶级是介入革命为反革命之间动摇不定,实际革命的乃是小资产阶级半无产无产这三个阶级成立的一个革命的联合,其对象是国际帝国主义;其目的是建设一个革命民众合作统治的国家;其所号召的民权民主主义并不是某一阶级笼络欺骗某一阶级使为已用的一种策略,而是各革命阶级一种共同的政治经济要求,由他们的代表者(孙中山先生)列为他们政说的纲领:其结果是要达到建设各革命民众统治的国家:其终极是要消灭全世界的帝国主义建设一个真正平等自由的世界联盟(即孙中山先生所主张的各类平等世界六同)。再看辛亥年的革命与目前的革命不同之点。辛亥年的革命,虽然基本属应该是反对国际帝国主义,然因当时多数党员还没有看清此点,黄兴章炳麟宋教仁等一些右倾领袖们只知道国内满朝贵族阶级的敌人,革命的口号变成简单的“排满”,党的组织和内容是极其简单,作战的队伍是极其孤弱,这是因为当时还没有有组织的工农群众,当时国内还没有代表无产阶级利益的中国共产党;国际的局面是几个强国霸占了全世界,只有压迫阶级的反革命的联合,没有被压迫阶级的革命的联合,只有资产阶级的国家,没有无产阶级的国家,因此中国当时的革命没有国际的援助。现在的局面与辛亥年完全两样;革命的目标已转抵到国际资本主义国家;党的组织逐渐严密完备起来,因为加入了工农阶级的分子,同时工农阶级形成了一个社会的势力;已经有了共产党;在国际又实现了一个无产阶级国家的苏联和一个被压迫阶级革命联合的第三国际,做了中国革命有力的后援。以此之故,在辛亥年参加革命的人,现在只剩下了少数革命意志强固的还主张革命,大多数就因为畏惧现在的革命把革命事业放弃了,或者跑向反革命队伍里同着现在的国民党作对。因此老右派新右派以着革命的发展和国民党的进步,如笋脱壳,粉纷分裂。我们完全明白这分裂的原因,还要着本党从兴中会以来本党党员的社会阶级的属性。我们知道领袖农村无产阶级向满清贵族及地主阶级作农民革命的洪秀全,乃孙中山先生最初革命思想的源泉;兴中会的组织,完全是收集满民无产阶级的会党;同盟会的组织,一部份是海外华侨工人,一部是内地的会党,另一部份则为小地主子弟出身的留学生,小地主子弟出身的内地学生及自耕农子弟出身的内地学生,总之同盟会的成分,乃无产阶级(会党)半无产阶级(侨工)小资产阶级(一部内地学生)中产阶级(留学生及一部份内地学生)这四个阶级的集合体。此时领袖中国大地主阶级的康有为派保皇党,与领袖中国无产阶级半无产阶级小资产阶级中产阶级的孙中山派同盟会,成了对抗。辛亥革命初成,同盟中代表小地主的一派即不赞成孙先生平均地权节制资本见之于实行,结果解散革命的同盟会,改组不敢革命的“国民党”,合并了许多代表小地主阶级利益的政团,使小地主阶级在国民党中成了绝对多数的支配者。虽然此时与代表大地主阶级的进步党(进步党为清末谘议局化身,谘议局乃各省大地主机关,与现今各省有议会之为大地主机关完全一样),还是立于对抗地位,但革命性几乎没有了。孙先生因此大愤,决志改组中华革命党,毅然提出“革命”二字做一党的名称,不惜与小地主阶级领袖黄兴分裂,以保持革命的正统。黄兴一派小地主阶级们为了怕革命不肯加入中华革命党脱离了孙先生之后,另外成立了一个欧事研究会。不久加以扩张,招引许多大小地主加入,成立了政学会。我们只要看政学会中人几乎无一个不是地主阶级,即可知道他们何以必须脱离孙先生;何以必须放弃革命;何以渐渐与大地主阶级由进步脱化而成的研究系相视目迷,何以到近日竟组成了联治派(联治派乃南方各省地主阶级近四年欲组织而未成型的政党),拥护赵恒惕陈炯明唐继花熊克武把西南各省的政权以省议会县议会及团防局的武装为其工具,向农村中自耕农佃农雇农都市工人学生小商施行极大的压迫,完全站入反革命地位。中华革命党改成中国国民党时,又加入一批中产阶级的非革命派,此时,而且有一部分代表买办阶级的分子混入了进来。他们站在支配党的地位,孙中山及少数革命派的领袖拿了,伪不能革命。乃于去年一月毅然召集第一次党的全国代表大会,明白决定拥护工农阶级的利益,从工农阶级中扩张国民党的组织,并且容纳共产党分子入党。当去年一月孙先生在广州长坝亚洲酒店招宴全国大会代表时,第祖权起持异议,反对容纳共产党分子。孙先生起立作长篇之演说,谓二十年以来,党总是阻挠我革命,总是丢掉民生主义。跟随我们的很多,但总是想打自己的主意。真正跟我来革命的,如×××先生一样的人不出二十个。今日还要阻挠我容纳革命的青年!到了第一次全国大会的人,都听见孙先生的话。然而此举首先得罪了代表买办阶级的领袖们,冯自由马索等首先与帝国主义军阀勾结脱离了国民党,另外组织同志俱乐部。国民党左派两年来在广东的工作,为了拥工人阶级的团结与罢工,得罪了帝国主义工具买办阶级;为了拥护农民的团结与减租得罪了地主阶级;为了保持革命根据地用严厉的手段对付反动派,得罪了帝国主义工具买办地主阶级的代表魏邦平陈炯明熊克武一班人,于是又激动了一班新右派,他们已在北京开会,图谋脱离左派领袖的国民党另外组织右派的国民党。惟闻在北京开会议中,代表小地主及与工商资产阶级的一派与代表买办阶级的一派意见不合,前一派有会议未终即离京南下之说。我们觉得这种现象也是必然的。中国现在已到短兵相接的时候,一面是帝国主义为领袖,统率买办阶级大地主官僚军阀等大资产阶级组织反革命的联合战线,站在一边;一面是革命的国民党为领袖,统率小资产阶级(自耕农、小商、手工业主)半无产阶级(半自耕农、佃农、手工业工人、店员小贩)无产阶级(产业工人`苦力、雇农、游民无产阶级)组织革命联合战线,站在一边。那些站其中间的中产阶级(小地主、小银行家、钱庄主、国贷商、华资工厂主)其欲望本系欲达到大资产阶级的地位,为了帝国主义买办阶级大地主官僚军阀的压迫使他们不能发展,故需要革命。然因现在的革命,在国内有本国无产阶级的猛勇参加,在国外有国际无产阶级的积极援助,他们对之不免发生恐惧,又怀疑各阶级合作的革命,中国的中产阶级(除开其左翼即中产阶级中历史和环境都有特别情况的人,可与其他阶级合作可与其他阶级合作革命,但人数不多),到现在还在梦想前代西洋的民主革命,还在梦想国家主义之实现,还在梦想中产阶级一阶级的领袖不要外援欺抑工农的“独立”革命,还在梦想其自身能够于革命成功后发展成壮大的资产阶级建设一个一阶级独裁的国家.他们革命的出发点,与其余阶级革命的出发点完全不同,他们的革命是为了发财,其余阶级的革命是为了救苦;他们的革命是为了准备做新的压迫阶级,其余阶级革命是为了要得到自己的解放并且使将来永无压迫自己的人。这班中产阶级“独立”革命派(小地主出身的最多)现在还在冒了孙先生的牌,说孙先生的“主义”,“遗教’是代表了他们。其实孙先生绝非如此。孙先生的主义遗教绝对是为了“救苦”,绝对不是为了“发财”绝对是使人类从压迫阶级解放出来,绝对不是为了准备做新的压迫阶级。无论将孙先生的主义遗教如何曲解,这个意义绝对不能变动。他们介在革命派和反革命派之间,自以为可以独立革命,其实没有这回事。他们疑忌工农阶级之兴起,他们疑忌国内及国际无产阶级政党之援助,他们丢弃了群众,丢开了帮手,在二十世纪半殖民地内外强力高压的中国,决没有做成革命的道路。在人数上说,四万万人中买办大地主官僚军阀等大资产阶级至多每四百里头有一个(四百分之一)计一百万人。小地主国货工商业等中产阶级,大约每百个里头有一个(百分之一)即约四百万人。此外的数目都是其余的阶级,自耕农、小商、手工业主等小资产阶级约占一万万五千万;半自耕农、佃农,手工业工人,店员,小贩等半无产阶级人数最多约占二万万;产业工人,都市苦力雇农游民无产阶级等完全无产阶级占四千五百万。依此分析,则中国为了救苦为了自求解放的革命民众有多少呢?有三万万九千五百万,占百分之九十八、七五。其敌人有多少呢?有一百万,占百分之零零点二三。中间派有多少呢?有四百万,占百分之一。在这种情况下,我们可以毫不犹疑地断定:代表中间阶级的国民党右派之分裂不足以妨碍国民党的发展,并足以阻挠中国的国民革命。他们的分裂是基于他们的阶级性,是基于现在特殊的时局,使他们不得不分裂,并不是为了什么左派的操切,所谓左派(所谓左派,是国民党的右派,并非拥护共产党,共产党员在国民党内乃共产派,不是国民党的左派)的操切,就是扫平扬刘扫平郑黄东江南路北江给了陈炯明邓本殷熊克武以大打击,坚持省港罢工给了英国帝国主义以大打击,这些革命的工作,然而这也是基于革命派的阶级性,基于现在特殊的时局,不得不奋斗,不得不革命。奋斗与革命乃他们唯一的出路,并不是什么操切不操切。象这样紧迫的时局,不但无驰缓希望,而且将继续紧迫。我们料到在不远的将来情况之下,中间派只有两条路走,或且向右加入反革命派,或且向左加入革命派(此乃没有此可能),乃没有第二条路,而在现在,他们留在国民党内,实在是×××先生所说的“假革命派”,不但无益而且有害,为了他们的分出去,为了他们的对于革命派(左派)的反动和反击,革命派将因此成功一个更大的团结。所以我们现在到处可以听得见的口号,几乎都是这样的一句:全国革命派团结起来!

<p align="right">《政治周报》第四期</p>

