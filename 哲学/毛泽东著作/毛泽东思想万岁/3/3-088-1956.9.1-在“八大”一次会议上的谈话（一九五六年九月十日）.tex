\section[在“八大”一次会议上的谈话(一九五六年九月十日)]{在“八大”一次会议上的谈话}
\datesubtitle{(一九五六年九月十日)}


七大开了一个团结的会,团结的结果,使得中国取得了革命胜利。现在是建设的任务,也要求我们团结。

我们中国革命时间很长,已经二十八年,这中间经过很多曲折复杂的道路,犯过多少路线错误。所以这样,还是个觉悟的问题。如果全党都觉悟很高,那些错误路线就执行不通。直到错误很明显了,才清楚。可是反对了这种错误以后,又犯了错误。不过,正是因为经过的时间那样久,我们才从中搞出一套政治路线、军事路线、组织路线、和党外人士合作的统一战线。不经过那样许多挫折,是不能有这些建设的。现在我们搞建设,希望错误不那样长。那样长的时间,是牺牲了多少人才让我们学会了党内关系,党外关系与群众关系——走群众路钱,才有了认识。我写了那些文章,没有几次斗争经验是搞不出来的。问题是,在建设当中,还要犯那样长的错误,栽那样多的筋斗吗?可以不必,主要的问题在于认识。我们六年学会很多,但是科学技术还没有学会,还要作很大努力。现在我国有十五万高级知识分子,第三个五年计划,要发展到一百五十万高级知识分子,那时我们就有许多科学家、工程师。……我们应当避免走那样多弯曲的道路,就要主观主义、宗派主义更少一些。一般讲中央和高级干部中间主观主义还是比较少的,否则中国革命为什么会胜利了,我们在斗争中也没陷于孤立呢?但是我们还是有主观主义的,比如中央批转过一个地区的一长制经验,当时就是不清楚。现在清楚了,不如集体领导与个人负责制相结合好,团结全党关系很大,我们这次会议应该是一个团结的大会。

这一百七十人名单,是反映了革命发展过程的。比如上海、天津、鞍山、沈阳各处工人都没有,有的说那你还是工人阶级,还是马列主义?问题在于二十二年根据地,根据地里这批人学会了马克思主义。你说我们不是,我们可是反对了帝国主义,在建设社会主义呀!马、恩、列、斯都不是工人,他们不懂得中国农村先胜利,城市后胜利这个条件。城市胜利才几年,要等他们成长起来,我们就会变了成分。纪念孙中山先生


