\section[在北戴河政治局扩大会议上的讲话(三)(一九五八年入月二十一日上午)]{在北戴河政治局扩大会议上的讲话(三)(一九五八年入月二十一日上午)}
\datesubtitle{(一九五八年)}


保证重点,明年搞××到××万吨钢,××万台机床,完成这些就是胜利,因此,要拚命干,要一星期抓一次,还有十九个星期要抓十九次。二十四日开工业书记和厂党委书记会议,看有没有把握。三令五申,凡有不拿出来者,要执行纪律。对搞分散主义的,一警告,二记过,三撤职留任,四撤职,五留党察看,六开除党籍,不然反而不利。我看一千一百万吨钢有完不成的危险,六月间,我问×××,钢是否能翻一番了?问题是我提出的,实现不了,我要作检讨。有些人不懂得。如果不完成一千一百万吨钢,是关系全国人民利益的大事。

要拚命干,上海有十多万吨废钢废铁回炉。要大收废钢废铁,暂时没有经济价值的铁路,如宁波、胶东线,可以拆除,或者搬到重要地点去,首先保证重点设备一一高炉、平炉、轧钢机、电机和重要铁路重点工程、车床、吊车。要向干部和人民讲清楚,首先保证几件大事,才是万年幸福。寃各有头,债各有主,一省只能有一个头,看同意不同意。同意,一个人也不能乱跑。在国家计划之外,各协作区之间,省与省之间,还可以互相调剂一点。还有一百三十三天,十九个星期,每星期抓一次,一定要抓好。

我们的人民是很有纪律的,给我印象很深。我在天津参观时,几万人围着我,我把手一摆,人们都散开了。河南修武县,全县二万九千多户,十三万人,成立了一个大公社,分四级:社、联队、中队、战斗小组。大,好管,好纳入计划,劳动集中,土地集中经营,力量就不同了。秋收翻一番,群众就看出好来了,甘肃洮河引水上山,那么大的工程,就是靠党的领导和人民的共产主义精神搞起来的。人民的干劲为什么这样大呢?原因就是我们向人民取得少,我们不要义务销售制,和苏联不一样。我们是一个党,一个主义,群众拥护。我们与人民打成一片,大整风以后,一条心。红安经验,就是一个典型。

要使同志们了解,马克思、恩格斯、列宁、斯大林对生产关系包括所有制、相互关系,分配三个部分相互关系,他们接触到了,但没有展开,我看经济学上没有讲清这一条。苏联在十月革命以后也没有解决。人们在劳动中的相互关系,是生产关系中的主要部分。搞生产关系,不搞相互关系是不可能的.所有制改变以后,人们的平等关系,不会自然出现的。中国如果不解决人与人的关系,要大跃进是不可能的。

在所有制解决以后,资产阶级的法权制度还存在,如等级制度,领导与群众的关系。整风以来,资产阶级的法权制度差不多破坏完了,领导干部不靠威风,不靠官架子,而是靠为人民服务,为人民谋福利,靠说服。要考虑取消薪水制,恢复供给制问题。过去搞军队.没有薪水,没有星期天,没有八小时工作制,上下一致,官兵一致,军民打成一片,成千成万的调动起来,这种共产主义精神很好。人活着只搞点饭吃,不是和狗搞点屎吃一样吗?不搞点帮助别人,搞点共产主义,有什么意思呢?没有薪水制,一条有饭吃,不死人,一条身体健康。我在延安身体不大好,胡宗南一进攻,我和总理、胡××,江青等六人住两间窑洞,

身体好。到西柏坡也是一间小房子。一进北京后,房子一步好一步,我的身体不好.感冒多了。大跃进一来,身体又好了。三天到四天中,有一天不睡觉,空想社会主义的一些理想,我们要实行。耶苏教清教徒的生活艰苦,佛教创教,释迦牟尼也是从被压迫民族中产生的。唐朝佛教“六祖坛经”记载惠能和尚,河北人,不识字,很有学问,在广东传经,主张一切皆空,是彻底的唯心论,但他突出了主观能动性,在中国哲学史上是一个大跃进。惠能敢于否定一切,有人问他:死后是否一定升西天,他说不一定。都升西天,西方人怎么办?他是唐太宗时的人,他的学说盛行于武则天时期。唐朝末年乱世,人民思想无所寄托,大为流行。

马克思关于平等、民主、说服和人们相互关系、打成一片的思想,没有发挥。人们在劳动中的关系,是平等的关系,是打成一片的关系。劳心与劳力是分离的,教育与生产是分离的。列宁曾说,要打破常备军,实行人民武装。有帝国主义存在,常备军是要的,但苏联军队中的等级制度,官兵关系,受了沙皇时代的若干影响。苏联共产党员多数是干部子弟,普通工人农民提不起来。所以需要找寻我们自己的道路。我们是一定要把干部子弟赶到群众中去,不能有近水楼台。我们的军官,像云南的一个师长,一年当一个月兵,我看这是好办法。是否到处推广,这样,我们的军队就是永远打不败的军队。

嵖岈山公社章程,《红旗》杂志要登出来,各地方不一定都照此办,可以创造各种形式。要好好吹一下,一个省找十来个人吹。大社要与自然条件、人口、文化等各种条件结合起来。河北省×××同志,找来十来个人吹共产主义思想作风,很有劲,你们回去也这样吹一下。进城后,有人说我们有“农村作风”、“游击作风”这是资产阶级思想侵蚀我们,把我们的一些好的东西抛掉了,农村作风吃不开了,城市要求正规化,衙门大了,离人民远了。要打成一片,要说服,不要压服,多年如此,这些怎么都成了问题呢?原因在于脱离群众,在于特殊化。我们从来就讲:上下一致,官兵一致,拥政爱民,拥军优属。供给制比较平等,衣服差不多。但进城以后变了,经过整风,群众说,八路军又回来了。可见曾经离开过。城市恰恰要推行“农村作风和游击习气”。蒋介石的阴魂在城市中没有走,资产阶级的臭气熏染我们,与他们见面,要剃头,刮胡子,学绅士派头,装资产味,实在没有味道。为什么要刮胡子呢?一年剃四次头,刮四次胡子不是很好吗?湖南省委书记周惠说,在县工作时,能和群众打成一片,在地委工作时,还能接近群众,到省委三年,干部和群众就不好找了。去年整风发生了变化。过去我们成百万的人,在阶级斗争中,锻炼成为群众拥护的共产主义战士。搞供给制,过共产主义生活,这是马克思主义作风与资产阶级作风的对立。我看还是农村作风、游击习气好。二十二年的战争都打胜了,为什么建设共产主义不行呢?为什么要搞工资制?这是向资产阶级让步,是借农村作风和游击习气来贬低我们,结果发展了个人主义。讲说服不要压服就忘掉了。是不是由于干部带头恢复供给制。华东老根据地搞过地道战,北方都经过战争锻炼,那个地方生长的干部生活习惯就有些不同。经过二万五千里长征的干部也出坏蛋,如××、××。××的意识非常落后,很隐蔽,摸不透他的心思,看来监察委员会不起作用,高岗,饶漱石都没有“监察”出来,无非是检查湖南、湖北的“青森五号”(粳稻),真正起作用的是军委这次一千四百人的会。

我们已相当地破坏了资产阶级的法权制度,但还不彻底,要继续搞,不要马上提倡废除工资制度,但是将来要取消。要强调农村作风、游击习气,一年参加一个月的劳动,分批下乡参加。列宁写过一篇文章,十月革命前夕。他到过一个工人家庭作客,这个工人找不到面包,后来找到了,非常高兴,“这回到底把面包找到了!”列宁从这里才知道面包问题的重要。我们的同志一年劳动,与人民打成一片,对自己的精神状态会有很大影响。这一回要恢复军事传统一一红军、八路军、解放军的传统,恢复马克思主义的传统,要把资产阶级思想作风那一套化掉,我们“粗野”一点,是真诚的,是最文明的;资产阶级好像文明一点,实际是虚伪的,不文明的。恢复供给制好像“倒退”。“倒退”就是进步,因为我们进城后退了。现在要恢复进步,我们带头把六亿人民带成共产主义作风。

人民公社,有共产主义萌芽。产品十分丰富。粮食、棉花、油料实行共产。那时道德大为进步,劳动不要监督,要他休息不休息,建华机械厂搞“八无”。人民公社大协作,自带工具、粮食,工人敲锣打鼓,不要计件工资,这都是些共产主义萌芽,是资产阶级法权的破坏.希望大家对这些问题的看法吹一下,把实际中共产主义道德因素在增长的情况也吹一下。

过去革命打死很多人,是不要代价的,现在为什么不可以这样干呢?如果做到吃饭不要钱,这是一个变化,大概×年左右,可能产品非常丰富,道德非常高尚,我们可以从吃饭、穿衣、住房子上实行共产主义。公共食堂吃饭不要钱,就是共产主义。将来一律叫公社,不叫工厂,如鞍钢叫鞍山公社;城市乡村一律叫公社,大学、街道都办成公社。乡社合一,政社合一,暂时挂两个牌子。公社中设一个“内务部”(行政科),管生死登记、婚姻,人口、民事。

有人问,统一以后,要不要有机动了?机动还是需要的,在保证一千一百万吨钢以外,允许有机动。如果树、棉花要整枝,其它就不整枝,统一主要是钢铁、机械。准备一百亿元冲击,使合作社的冲击力有东西可冲。国家保证××万吨钢,剩下××万吨由省、地、县去安排,能超过一点更好,计划不可能搞得那样准确,不可能样样事先有计划,有些事情难以预料,盲目性是不可避免的,乱是有一点,成绩是很大的,空前的。过去我们没有管,现在全党要管这件事,第一书记右手抓工业,左手抓农业,各级党委都要设几个书记。


