\section[反攻(三)]{反攻(三)}


北京右派会议与帝国主义

帝国主义痛恶左派国民党的广州中央执行委员会做了全国反帝国运动的总指挥,使一切帝国主义栗栗危惧,北京右派会议议决停止广州中央执行委员会的职权;帝国主义痛恶国民党政治委员会能集中权力指挥省港罢工和扫除其种种有用的工具――杨希闵,刘震寰,梁鸿楷,郑润琦,莫雄,魏邦平,陈炯明,林虎,洪兆鳞,邓本殷,熊克武,北京右派会议议决取消政治委员会;帝国主义痛恶国民党左派领袖汪精卫领导国民革命与帝国主义作殊死战,北京右派会议议决开除汪精卫的党籍;帝国主义痛恶苏联帮助国民党政府雇用俄顾问增加了攻击帝国主义的力量,北京右派会议议决辞退俄颂问鲍罗亭;帝国主义痛恶国民党容纳共产分子增加一枝反帝国主义的生力军,北京右派会议议决开除李大钊、谭平山等的党籍。观此,我们可以知道北京右派会议替帝国主义做了些什么工作。

帝国主义最后的工具

北京右派会议替帝国主义做了适合其需要的种种工作,既如上述。然而这是帝国主义对付中国反帝国主义运动的最后之一方法。帝国主义的工具杨希阂刘震寰企图推翻广州政府实现今日北京右派会议之目的,无效;帝国主义的工具梁鸿楷郑润琦魏邦平莫雄朱卓文等企图于刺杀廖仲恺后实现今日右派会议之目的,又无效;帝国主义的工具熊克武图从北江袭取广州实现今日右派会议之目的,又无效;帝国主义的工具段祺瑞派遣兵舰图从虎门袭击广州实现今日右派会议之目的,又无效;帝国主义的工具陈炯明邓本殷欲从东南方扎到广州实现今日右派会议之目的,又无效。帝国主义这些工具所做的工作都没有效。国民党右派愤激起来了,于是有北京会议之召集,由“炮援袭击”的方法,改用了“议决案”的方法。这个方法的效力怎么样呢?实在难说。自然,右派会议的种种议决,只算是儿戏的议决,但是这种“窠里反”的方法,确实要此“窠外反”要进步。帝国主义于一切工具用尽之后,找到了这个最后的工具,使他于失败之余忽然得到了一点小慰。即使右派中有几位口里还在说打倒帝国主义,即使右派中还有一小部分并无诚心为帝国主义利用,即使他们怎样不承认自己做了帝国主义的工具,然而在事实上是大大帮了帝国主义的忙,事实上是做了帝国主义的工具,因为他们的工作适合了帝国主义的需要。

右派的最大本领

右派机关上海民国日报十二月三日社论说:“只有怕革命党的军阀,那有怕军阀的革命党”。他拿这个理由去反对下面汪精卫等感电的话:“中央全体会议属于公开性质,若在北京开会,外则有军阀的压迫,内则有反动分子利用军阀而从中作梗”。因此民国日报认为中央全体会议在北京开会,可以表示勇敢,可以表示不怕军阀。民国日报的错误在那里?他的错误就在于不懂得革命党的活动有公开与秘密之别。一个真正的革命党,他的党组织与会议在敌人势力之下完全是秘密的,他的主张和宣言则是公开的。在敌人势力之下要将党的组织党一的会议公开起来,那必须先得敌人的谅解,就是至少有某几点是于敌人有利才能得他的默许或者还能得到他的保护。但这还成了什么党呢?这只能是敌人的朋友,不是要革敌人的命的革命党。段祺瑞允许了右派在北京公开的开会,他能允许汪精卫谭延闿等到北京公开的开会呢?该报又谓孙总理去年曾到北京不怕段祺瑞。不知孙总理去年能到北京有两个原因:一是段祺瑞刚上台其政权尚末稳固,压迫国员党的政权尚未确定;一是当时北京的警察权尚握在同情于国民党的冯玉祥手里。没有这两个原因,孙先生是不能公开地到北京的,假如孙先生至今还在,段祺瑞一定不能容许他在北京公开地做革命运动,他一定须秘密起来,或跑到别处去。北京乃至全国各地反动军阀盘踞的地方。都有国民党的组织,各级党部到处都有机关,党员及干部同志随时都有会议,都有各种企图消灭敌人势力的勇敢奋进的工作,但这都是秘密的。在这些组织和工作中,只有左派在那里不断的奋斗,右派党员都畏惧不敢近前。右派的长处就是一张嘴,“打倒帝国主义”“打倒军阀”几个口号,他们也能不看党的决议案背得;至于实际的做法,实际的行动,他们一听见就吓落了胆。右派只有一张嘴,他们并没有手与脚。他们只有胆子在段祺瑞面前开会,没有胆子到广州开会,因为广州的革命空气把他们吓杀了。他们议决了他们所谓的第二次全国大会明年三月在上海或北京开会。依我的观察,他们的大会如真能召集得成(不管人数多少)未必敢在北京开,因为段执政的家庭已是坐不大稳了。他们大概会在上海开会。在帝国主义的老窠里,向各国领事工部局洋大人巡捕房红头阿三面前公开着国民党的全国会议,这很可以表示“勇敢”。能在军阀帝国主义面前公开的开会,这是右派最大的本领,左派分子望尘莫及!

<p align="right">(一九二五年十二月二十日《政治周报》第三期)</p>

