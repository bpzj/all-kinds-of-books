\section[中共关于目前政治形势与党的任务的决议(瓦窑堡会议)(一九三五年十二月二十五日中央政治局通过)]{中共关于目前政治形势与党的任务的决议(瓦窑堡会议)(一九三五年十二月二十五日中央政治局通过)}
\datesubtitle{(一九三五年十二月二十五日)}


一、目前形势的特点

目前政治形势起了一个基本上的变化,在中国革命史上划分了一个新时期,表现在日本帝国主义变中国为殖民地,中国革命准备进入全国性的大革命。在世界是革命与战争的前夜。

日本帝国主义并吞中国四省之后,现在又并吞了整个华北、而且并准备并吞全中国,把全中国从各帝国主义的半殖民地变成日本的殖民地,这是目前时局的基本特点。

日本帝国主义鉴于直接公开的武装占领东北四省,曾经引起中国反日的怒潮,这次他就釆取比较隐蔽的方式,即用国民党南京政府下命令委任中国某些卖国军阀政客,作为他在华北的代理人,以达到直接武装占领华北,这次方式虽然是比较“九.一八”的方式更为狡猾与凶恶,但必是过渡到直接武装占领与建立华北围攻其傀儡政府的步骤。

第二个满洲国傀儡政府是必然的归宿,作为中国汉奸卖国贼集团的主要成分的许多军阀政客土豪劣绅为劣绅卖办及银行资本家,特别是其中的亲日派,是这个傀儡政府的组成分子及其赞助者,没有这群大汉奸卖国贼,日本帝国主义变中国为殖民地是不能如此顺杨的。

日本帝国主义并吞华北并准备并吞全中国的行动,向着四万万人的中华民族送来了亡国灭种的大祸,这个大祸就把一切不愿当卖国奴才不愿充汉奸卖国贼的中国人,追到走上一条唯一的道路:向着日本帝国主义及其走狗卖国贼开展神圣的民族革命战争。一切爱国的中国人为保卫自己的国家而血战到底,这是日本帝国主义亡人亡国亡人亡种的冒险事业的前进道路上必然得到的回答。日本帝国主义于吞下了东北四省这些较小的炸弹之后,又着手吞下中国本部这个绝大的炸弹。

日本帝国主义蚕蚀并吞中国的行动,使帝国主义内部的矛盾达到了空前紧张的程度,美帝国主义完全为着他自己帝国主义的目的,是和日本帝国主义势不两立的,太平洋战争是必然的结果。而英国却想求得日本的某些让步和妥协,使它的主要力量能够拿去对付它的主要敌人:苏联、美国、意大利。

日本帝国主义并吞中国的行动,促进了中国反革命统治首先是卖国贼头子蒋介石的统治之削弱与崩溃。一贯的卖国政策,不但使蒋介石丧失某些社会的与群众的基础,而且缩小了它的地盘,同苏维埃的长期斗争,尤其是五次“围剿”。与对中央红军的“围剿”使它的精力消耗了,疲劳了,分散了,统一中国的梦想,是宣告了最后的破产。国民党五次全国代表大会,实际上是不利于蒋介石的分赃会议,蒋介石只有更加依靠出卖中国以维持自己的统治。但是它的出卖,将更促使他的统治加速度地走向灭亡。这种情形,也就更加深了以新的形势上新的性质,而出现的反蒋战争的爆发的可能性。

日本帝国主义并吞中国的行动,正当中国苏维埃红军转入了一个新局面的时期,自从中央红军退出中央苏区,长江下游的一些苏区受到一些损失之后,现在是各地红军的新胜利,新根据地的创造,老苏区的游击战争的开始转入反攻,与新的游击战争蓬勃发展的时期,困难的关头已经过去了。中央红军以十二个月的功夫,用二万五千里的长征,战胜了蒋介石的长追,宣告了帝国主义及其走狗蒋介石的追堵截的破产,突破了历史上军事远征的纪录。并且以宣传队的作用,向着他所纵横驰聘的十一省区二万万以上的民众,指出了痛苦救国救己的道路。以播种机的作用,散布了许多革命种子,中央红军与二十六、七军会合之后,对于对着陕甘苏区进攻的敌人第三次“围剿”之彻底粉碎,更加表示了苏维埃运动新时期的到来,它同目前总的革命形势的新局面相会后,成为中国革命新形势的一个重要的组成部分。它指明:在全中国人民反对日本帝国主义的强盗吞并,扰欲中国出乎亡国灭种,大祸的伟大力量中,有着苏维埃红军铁一般的中坚力量。

日本帝国主义并吞中国的行动,重新推醒了全中国人民,懂得亡国灭种大祸临头的危险形势,掀起了新的民族革命的高潮,这种民族革命高潮,是在中国革命经过历史上无数次锻炼之下(主要是一九二五年至一九二七年的大革命)产生的。是在目前中国已经有了苏维埃革命根据地与革命形势之下产生的。是在苏联已经有着一切力量足以战胜侵略国家援助被压迫民族的形势之下产生的,因此,它就将以特别广大、特别坚决、特别与世界革命因素互相影响与互相帮助的性质而出现。

无疑的,新的反日的民族革命高潮,不但提醒了中国工人阶级与农民中更落后的阶层,使他们积极参加斗争,而且广大的小资产阶级群众与知识分子,现在又转入了革命。中国国民经济的总崩溃,千千万万民众的失业失地,千千万万的灾民难民,更使得新的反日的民族革命高潮,同群众的救死救生的日常斗争密切地联系起来,大大地扩大了民族革命的群众基础,广大民众的革命义愤是在全国一切地方酝酿着,并已经在普及各大城市的学生,反日示威运动中开始表现出来了,在反革命营垒中是新的动摇分裂与突击、一部分民族资产阶级、许多的乡村富农与小地主,乃至一部分军民,对于目前开始的新的民族运动,是有釆取同情中立以致参加的可能的,民族革命战线是扩大了。

目前的世界是处在大革命与大战争的前夜的形势下,一切帝国主义国家的经济危机,以及由此产生的革命危机,使得帝国主义除了战争找不出第二条挽救死亡的出路,日本帝国主义大举进攻中国与意大利帝国主义大举进攻阿比西尼亚的民族革命战争,各帝国主义国家及其许多殖民地半殖民地革命危机的成熟,无疑的要引导到世界的大革命,在目前的革命与战争的前夜时期,己可明显地看到世界反革命力量的薄弱,与世界革命力量的增长,在将来就是大战争与大革命,葬送世界上的一切反革命。这一形势,使得中国革命脱离了过去的孤立,世界革命是中国革命的有力帮手,同时中国革命现在就已经成了世界革命的伟大因素,将来则要以全民族的雄伟阵势帮助着世界革命。

这在日本与中国的关系上也是一样的。在有力量的日本共产觉领导下的日本工农及被压迫民族(朝鲜、台湾)等,正在准备着伟大力量,为打倒日本帝国主义,建设苏维埃日本而奋斗,这就把中国革命向日本革命在共同目标打倒日本帝国主义基础上会合起来,日本革命民众是中国革命民众的有力的帮助。

中国革命是处在有利环境中,中国革命有着光明灿烂的前途,但是中国革命的主要敌人帝国主义,尤其目前凶横直接的日本帝国主义,是准备了决心和力量来对付中国革命的。在中国反革命集团方面,由于其统治力量的减弱,而不得不更加为虎作伥,投靠万恶的日本帝国主义,向着革命的民族作决望的斗争与决斗,把这一形势同目前依然存在着的中国革命不平衡发展的形势结合起来看,就知道中国革命保存了一种持久性,它向中国革命民众及其首领中国共产党指明:准备长时间同敌人奋斗吧:为着同敌人作持久战而准备自己的持久艰苦工作吧!没有几千万几万万的革命军,是不能最后解决敌人的,一切策略,一切努力,向新组织千千万万民众进入伟大的民族革命战场上去。准备了伟大的力量,就是准备了决战的捷报。

二、党的策略路线

目前的形势告诉我们,日本帝国主义并吞中国的行动,震动了全中国、全世界,中国政治生活中的各阶级、阶层,政党及其武装势力,重新改变了与正在改变着他们之间的相互关系。民族革命战线与民族反革命战线是在重新改组中。因此,党的政策略线,是在发动、团结与组织全中国全民族一切革命力量去反对当前主要的敌人――日本帝国主义与卖国贼头子蒋介石。不论什么人,什么派别,什么武装部队,什么阶级,只要是反对日本帝国主义与卖国贼头子蒋介石的,都应该联合起来,开展神圣的民族革命战争。驱逐日本帝国主义出中国,打倒日本帝国主义的走狗在中国的统治,取得中华民族的彻底解放,保持中国的独立和领土的完整,只有最广泛的反日民主统一战线(下层的与上层的)才能战胜日本帝国主义及其走狗蒋介石。

当然,不同的个人,不同的集团,不同的社会阶级与阶层,不同的武装部队,他们参加反日的民族革命,各有他们不同的动机和立场,有的是为了保持他们的原有地位,有的是为了要争取运动的领导权,使运动不得超出他们所允许的范围之外,有的真是为了中华民族的彻底解放。正因为他们的动机与立场各有不同,有的在斗争开始时就要动摇叛变的,有的会在中途消极退出战线的,有的愿意奋斗到底的。但是我们的任务,是在不但要团结一切可能的反日的基本方量,而且要团结一切可能的反日同盟者,是在中国人为有力出力,有钱出钱,有枪出枪,有知识出知识,不使一个爱国的中国人不参加到反日的战线上。这就是党的最广泛的民族统一战线策略的总路线,只有这种路线我们才能动员全国人民的力量去对付全国人民公敌:日本帝国主义与卖国贼蒋介石。

中国工人阶级与农民,仍然是中国革命的基本动力。广大的小资产阶级群众、革命的知识分子,是民族革命中最可靠的同盟者,工农小资产阶级的坚固联盟,是战胜日本帝国主义及其汉奸卖国贼的基本力量,一部分民族资产阶级与军阀,不管他们怎样不同意土地革命与苏维埃制度。在他们对于反日反汉奸卖国贼的斗争釆取同情,或善意中立,或直接参加之时,对于反日战争的开展都是有利的。因为这就离开了总的反革命力量,扩大了总的革命力量。为达到此目的,党应当采取适当的方法与方式,争取这些力量到反日战线中来,不但如此,即在地主买办阶级营垒中间,也不完全统一的,由于中国过去是许多帝国主义互相竞争的结果,产生了各帝国主义互相竞争的各卖国贼集团,他们之间的矛盾与冲突,党应使用很多的手段使某些反革命力量暂时处于不积极的反对反日战线的地位。对于日本帝国主义以外的其他帝国主义的策略也是如此。

党在发动集团与组织全国人民的力量以反对全中国人民的公敌时应该坚决不动摇的同反日统一战线内部一切动摇、妥协、投降与叛变的倾向作斗争。一切破坏中国人民反日运动者,都是汉奸卖国贼,应该群起而攻之。共产党应当让自己的正确的反日反汉奸卖国贼的言论与行动去争取自己在反日战线中的领导权。也只有在共产党的领导下,反日运动才能取得彻底的胜利。反日战争的广大民众,应该满足他们基本利益要求(农民的土地要求,工人、士兵、贫民、知识分子等改良生活待遇的要求)。只有满足了他们的要求,才能动员更广大的群众走进反日的阵地上去,才能使反日运动得到持久性,才能使运动取得彻底的胜利。也只有如此,才能取得党在反日战争中的领导权。

三、国防政府与抗日联军

……(略)

因为国防政府与抗日联军是反日反卖国贼的最广泛的与最高的民族统一战线的组织,所以他应该有最广泛的行动纲领。这个纲领如下:

一、没收日本帝国主义一切在华的财产作抗日经费。

二、没收一切卖国贼及汉奸的土地财产分给工农及灾民难民。

三、救灾治水,安定民生

四、废除一切苛捐杂税,发展工商业。

五、加薪加饷,改良工人、士兵及教职员的生活。

六、发展教育,救济失学的学生。

七、实现民主权利,释放一切政治犯。

八、发展生产技术,救济失业的知识分子。

九、联合台湾、日本、朝鲜内的工农,及一切反日力量,结成巩固的联盟。

十、对于中国的民族运动表示同情赞助或守善意中立的民族或国家,建立亲密的友谊关系。

共产党必须在抗日战争的过程中,求得这纲领的实现,并经过这些纲领以求得党的十大政纲的实现。

四、苏维埃人民共和国

五、党内的主要危险是关门主义

要战胜中国人民的公敌日本帝国主义及其走狗卖国贼,共产党员必须深入到群众中去,参加与领导一切群众的、民族的与阶级的斗争。这里,主要的关键是运用广泛的统一战线。广泛的统一战线,一方面是在靠集中最大的力量去对付最重要的敌人,另一方面是在使广大的群众根据于他们自己的政治经验,了解党的主要的正确,争取他们到党的旗帜之下。

必须更深刻地了解革命领导权的问题。共产党要在中国革命中取得领导权,单靠党的宣传鼓动是不能够的。必须使它的一切党员在实际工作中,在每日的斗争中,表现出他们是群众的领导者,只知道如何在下层群众中间进行工作(这是主要的)是不够的。必须知道如何同别党、别派和下层群众有关系的上层领袖,进行谈判、协商、妥协、让步,以期争取其中可能继续合作的分子,以期在群众前面最能揭穿那些动摇欺骗与叛变分子的面目,而以群众的力量把他们驱逐出去。党的领导权取得,单靠在工人阶级中的活动是不够的(这是要紧的),共产党必须在农村中,兵士中,贫民中,小资阶级与知识分子中,以及一切革命同盟中,进行自己的活动,为这些群众的切身利益而斗争。使他们相信共产党不但是工人阶级的利益的代表者,而且也是中国最大多数人民的利益的代表者。只要有群众的地方,不论那里的领导者是怎样地反动,共产党员应参加到里面进行革命的工作。只有共产党员表现出他们是无坚不破的最活泼有生气的中国革命的先锋队,而不是空谈的抽象的共产主义原则的圣洁的教徒,共产党才能取得中国革命的领导权。

为了更大胆地运用广泛的统一战线,以争取党的领导权,党必须对党内“左”的关门主义倾向做坚决的斗争。在目前形势下,关门主义是党内的主要危险。关门主义的来源:第一,是因为对目前新的政治形势的不了解,因此就不了解变更自己的策略以适合于新的形势的必要。第二,是不会把党的基本口号与基本的政纲同目前的行动口号与行政纲领在实际行动中联系起来。第三、基本是由于不会把马克思列宁斯大林主义活泼地运用到中国的特殊的具体环境中去,而把马克思列宁斯大林主义变为死的教条。这种关门主义倾向,实质上表现出惧怕敌人、惧怕群众与对于自己力量的不相信。因此就惧怕运用广泛的统一战线的策略。这种关门主义的倾向,实质上是与右倾机会主义相同的,因为关门主义继续的结果,必然是党脱离群众,使党放弃中国革命领导权的任务。因此,党必须反对“左”的关门主义,大胆地运用广泛的统一战线,深入到千千万万的群众中去,不怕可能发生的某种错误,从斗争中去学习领导群众的艺术。

虽然,党在反对“左”的关门主义的斗争中,丝毫也不要放松反对右倾机会主义的斗争。右倾机会主义压抑群众利益的斗争,牺牲农民夺取土地,工人兵士贫民改良待遇的要求去适合民族资产阶级与富农的利益,对同盟者惧怕使用言论批评的武器,惧怕率领群众逼迫同盟者走上革命的更高阶段。右倾机会主义接受民族资产阶级,上层小资产阶级分子,以及乡村中富农的政治影响,而把自己变成他们的尾巴。一九二七年期间陈独秀主义,在新的大革命中,在部分的党部与党员中的复活,是不可能的。毫无疑义,党应该向着这种右倾机会主义,作坚决的斗争。但在目前说来,“左”的关门主义是党的最大危险。目前的反对右倾正是为得要顺利地克服“左”倾,彻底地击破关门主义,使广泛的统一战线的策略正确地大胆地运用到一切工作中去,使党不落在群众斗争的后面,使群众从争取日常的切身利益出发,提高到反对中国人民的公敌日本帝国主义及其走狗中国卖国贼的民族革命战争的位置。

六、为扩大与巩固共产党而斗争

为了完成中国共产党在伟大历史时期所担负的神圣任务,必须在组织上去扩大去巩固党。在新的大革命中,共产党需要数十万至数百万能战斗的党员,才能率领中国革命进入彻底的胜利。

中国共产党是中国无产阶级的先锋队。因此,一切愿为着共产党的主张而奋斗的人不问他们出身如何,都可以加入共产党。一切在民族革命和土地革命中的英勇战士,都应该吸收入党,担负党的各方面的工作。由于中国是一个经济落后的殖民地半殖民地,农民分子与小资产阶级出身的分子常常在党内占大多数,但这丝毫也不减弱中国共产党的布尔什维克化的地位。事实证明,这样成分的党是能够完成世界无产阶级先锋队共产国际所给于的光荣任务的,是能够艰苦奋斗百折不回的,在世界各国共产党中,除联邦共和党外,中国共产党站在光荣的先进的地位。

必须同党内发展组织中的关门主义倾向作斗争。能否为党所提出的主张而坚决奋斗,是吸收新党员的主要标准,社会成分是应该注意的,但不是主要标准.应该使党成为一个共产主义的熔炉,把许多愿意为共产党主张而奋斗的新党员,锻炼成为有最高阶级觉悟的布尔什维克战士,党内两条路线的斗争与共产主义的教育,就是区别这一目的方法,党在思想上的布尔什维克的一致,是党的坚强的无产阶级领导之具体表现,不从积极的革命斗争需要出发,不从恐怖观念出发的组织问题上的关门主义,必须彻底的击破,民族革命与土地革命的伟大斗争,已经涌出和正在涌出无数的积极分子与群众领袖,党的组织应以热烈欢迎的态度向他们开门。党不惧怕某些投机分子的侵入,党用布尔什维克的政治路线与铁的纪律,去保证党的组织的巩固,党不惧怕非无产阶级党员政治水平的不一致,党用共产主义教育,去保证提高他们的先锋队的地位。

必须大量的培养干部。党要成千成万的新干部,一批又一批地送到各方面的战线上去,不是把领导才能每天都教好了,才给干部以工作,而是放这些干部到斗争中去,使他们从斗争中去学习,不是以如何使用机械一样的态度去使用干部和党员,而是爱护他们,信任他们,分配他们以适当的工作,充分发挥他们的天才与自动性,不是以官僚主义的态度去对待干部和党员,而是以对于任务的解释说服,对于工作的具体指示,把党的领导机关同他们之间活泼有生气地联系起来,对于干部与党员在思想上与工作上的错误,不是轻易地给以打击,加上机会主义帽子,以及严厉地处罚他们,而是给以耐心的一次又一次的说服教育,思想上与工作上的错误是难免的,错误是可以改正的,列宁主义的学习精神与从斗争中求锻炼,是改正错误的最好方法,党内斗争的火力,应该向着那些坚持错误观点的不愿学习锻炼,不受领导教育的同志,一定限度的组织上的结论,也仅仅是对于那些错误严重与无法说服的同志才是必要的,但一切必要的党内斗争与组织结论仍然是带有对于本人与全党的教育性质,只有对于那些有一贯错误路线机会主义者,党才不应当因为他们一时表现改正而轻易给它们以重要工作。

伟大斗争时期,党的干部坚固的团结于党的领导机关的周围,是有决定意义的,党要团结全党领导最广大群众走上民族革命与土地革命的战争,没有很多的与很好的干部作枢纽,是不能成功的。正确的组织路线与干部政策,是完成这个任务的前提。

中国共产党中央号召全党及其干部坚持执行党的策略路线而斗争,把统一战线运用到全国去,把国防政府与抗日联军成立起来.把苏维埃共和国变为全民族的国家,把红军变为全民族的武装部队,把党变成伟大的群众党,把土地革命与民族革命结合起来,把国内战争与民族战争结合起来。

神圣的民族革命战争万岁!

中国的独立与统一万岁!

苏维埃新中国万岁!

