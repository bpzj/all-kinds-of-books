\section[关于礼宾工作指示(一九六五年)]{关于礼宾工作指示}
\datesubtitle{(一九六五年)}


【宴会】主席说:我们招待外国人的宴会规格太高,而且不看对象,千篇一律都要上燕窝、鱼翅那些名贵的菜,花钱很多,又不实惠,有些外国人根本就不吃这些东西。这是西太后和《红楼梦》里贾宝玉、林黛玉那些人吃的。我们请外国人,热菜有四菜一汤就可以了。宴会的时间不要太长,我们又要同外国人谈话,又要同他吃饭,我们陪不起。听说外国人的宴会就比较简单。

主席请外宾就比较节俭,而且区别对待。有些非洲人不吃鱼翅,有些欧洲人不吃海参,主席请客一般不用这些东西。六三年××国王来,在勤政殿请他吃饭,陈总开的四川菜单,主席讲这次的菜好,扎实、大方,有一个菜叫“拷方”,只吃皮不吃肉,主席说好吃,就是这样吃有些浪费,他说吃了皮以后把肉再回锅。有一次在杭州,请×国人吃饭,主席说,他们看不起我们,说我们不行,今天的菜搞得简单一些就是我平时吃的菜再加一两个就可以了。听说,×国的宴会没有什么吃的。就是这样,×国人还吃的很高兴,说是很丰富。有一次接待××国客人,主席说,他们不是执政党,生产很困难,今天的菜要搞的很丰富,实惠,要让他们吃好。不要吃那些山珍海味,这些东西××比我们多。前几年国内经济困难的时候,主席还说过,大区书记,到我们这里开会,不要给他们吃得太好,吃得太多,他们饿了肚子就会想到老百姓吃不饱不好受的味道。

【礼品】主席说:我们送外国人的礼物,化钱多,规格高,吃穿用的东西多,有纪念意义的东西少,其实送礼不在多少,而要送有民族的特点,又能长期保存的东西。送礼自然要大方,但不能没有个边,大手大脚,大少爷作风,不能靠多送礼品的办法拉友谊,友谊要靠政治,外国人送我们的礼品就比较简单,艾地同志来,送我一根“金鸡毛”,约多同志送我一个小孩玩的弹弓,这种礼品就很好,礼品就是来表示意思,也不能靠礼品过日子。

主席还说过,我们给外国人送礼花的是国家的钱,外国人送给我们的礼品也要归国家。不应归个人所有,送给我的礼品要好好处理,有展览价值和纪念意义的,找个地方陈列出来,没有展览价值的一些日用品,可以内部作价处理(收回来的钱归公)或者交给国家使用,还有一些吃的东西,可以分给工作人员尝尝。送给主席的礼品,就是遵照主席以上指示处理的。有些水果之类的东西,我们曾建议给主席家里的人和小孩子吃了算啦。主席不肯,并且说,你们为老百姓做事,为我服务,有功劳,应该吃,小孩子有什么功劳,不要给他们吃。现在把一些吃的东西轮流分给中南海几个单位的同志吃,虽然东西不多,但同志们反映非常好,说主席对自己要求那么严格,而对周围的同志却那么关心。

【礼节】主席说外交礼节,不能学洋人那一套,什么黑衣服,薄底皮鞋等等,都是从外国人学来的,我们是中国人,有我们自己的习惯,外国人到中国来是看你的政治,看你的社会主义革命和社会主义建设,是看群众的情绪,人家不是来看你穿什么样的衣服,穿什么样的皮鞋。

还有过国庆节时,江青同志接见外宾,与主席商量做不做新衣服。主席说,你接见了客人,客人就会高兴的,人家不是来看你穿什么样的衣服。后来没有做,结果外宾反映很好。说主席夫人很朴素。

【接待】接待外宾时,工作人员多,秩序乱,主席每次接见或宴请外宾时,事先都亲自交待让那些人参加,主席说过,他不喜欢人多,说人少坐的靠拢,谈话方便,有一次主席在武汉宴请×××××外宾,主席交待摆两桌就行了。但××部的同志搞了四桌,事后主席批评了我,说没有按照他说的办。一九六三年,主席在中南海宴请××国王,本来都准备好了,但礼宾司一个同志说这样不好,要摆大桌子,从大会堂把桌子抬来后,因桌子大,门小进不去,正在忙乱的时候,主席和外宾说完话后,要吃饭,结果没有准备好,又把主席和外宾挡了回去。事后主席和总理都批评了我们。

(据汪东兴同志传达。)

