\section[执行优待红军条例实施办法(节录)(一九三二年二月一日)]{执行优待红军条例实施办法(节录)}
\datesubtitle{(一九三二年二月一日)}


中华苏维埃工农兵第一次全国代表大会通过了红军优待条例,这个优待条例,对于红军规定了许多优越的权利。为什么对于红军定出这许多的优待条例呢?因为红军在几年来的斗争中,是坚定实行土地革命,坚决反对帝国主义、反对国民党军阀的主要力量,是苏维埃政府的有力保卫者。他在过去中国的革命斗争历史上,做了英勇斗争的光荣事业。目前中国革命是一种残酷的革命战争环境,是积极进行革命战争的时候,是要与帝国主义国民党军阀作更残酷的大规模的战争,必须要有广大的红军,才能争取一省或数省首先胜利以及苏维埃在全中国的胜利。红军是为了解放工农民一切被压迫民众而作战,为苏维埃政权而作战的战士,是以最大牺牲精神来为工农民一切劳苦民众利益和解放而奋斗。因此,苏维埃政府和工农群众,就应当对于红军加以特别的优待,使这些红军战士心中得着安慰,对于家庭没有什么挂念,可以一心一意地去勇敢作战,因此,第一次全苏大会特别定出这个优待红军条例。

过去各地方苏维埃政府对于红军及其家属的优待,确已有些规定,但在实际上的执行是缺乏注意的,有的地方简直是忽视不执行;有的地方对待红军家属,简直是破坏红军;这与扩大红军、加强红军战斗力是有很大的妨碍的,这是很大的错误。以后各级苏维埃政府应严格地纠正过去的这些错误,绝对执行全苏大会通过的优待红军条例,以后若再忽视优待红军或对执行优待红军条例懈怠,须当作反革命一样的来处罚。兹特规定下列优待红军条例实施的具体办法,以便执行。

(具体条例略)

