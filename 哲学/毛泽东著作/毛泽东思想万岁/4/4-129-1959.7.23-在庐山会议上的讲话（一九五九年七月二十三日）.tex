\section[在庐山会议上的讲话(一九五九年七月二十三日)]{在庐山会议上的讲话}
\datesubtitle{(一九五九年七月二十三日)}


你们讲了那么多,允许我讲点吧,可以不可以?吃了三次安眠药,睡不着。

讲点这样的意见。我看了同志们的记录、发言、文件,并和一部分同志们谈了话。我感觉到有两种倾向,这里讲一讲。一种是触不得,大有一触即发之势。吴稚晖说,孙科一触即跳。因此有部分同志感到有压力,即是不愿人家讲坏话,只愿人家讲好话,不愿听坏话。我劝这些同志要听。话有三种,咀有两用。人有一个咀巴:一曰吃饭,二曰讲话之义务。长一对耳朵就要听。他要讲,你有什么办法?有一部分同志就是不爱听坏话。好话坏话都是话,都要听。话有三种,一是正确的,二是基本正确的或不甚正确的,三是基本不正确或不正确的。两头是对立的。正确与不正确是对立的。

现在党内外夹攻我们。右派讲:秦始皇为什么倒台?就是因为修长城。现在我们修天安门,要垮台了,这是右派讲的。党内一部分意见还没有讲完,集中表现在江西党校的反映,各地都有。所有右派的言论都拿出来了。江西党校是党内的代表,有些人就是右派,动摇分子。他们看不完全,做点工作可以转变过来。有些人历史上有问题,挨过批评,也认为一塌糊涂,如广州军区的材料。这些话都是会外的讲话。我们是会内外结合。可惜庐山地方太小,不能把他们都请来。像江西党校罗隆基,陈铭枢,这是江西人的责任,房子太小吆!

不分什么话,无非是讲得一塌糊涂。这很好,越讲得一塌糊涂越好,越要听。我们在整风中创造了“硬着头皮顶住”这样一个名词。我和有些同志讲过,要顶住,硬着头皮顶住。顶多久?一个月、三个月,半年、一年、三年、五年、十年、八年,我们有的同志说“持久战”,我很赞成。这种同志占多数。

在座诸公,你们都有耳朵,听嘛!难听是难听,欢迎!你这么一想就不难听了!为什么要让人家讲呢?其原因,神州不会陆沉,天不会塌下来。因为我们做了些好事。腰杆子硬。我们多数同志腰杆子要硬起来。为什么不硬?无非是一个时期蔬菜太少,头发卡子太少,没有肥皂,比例失调,市场紧张,以致搞得人心紧张。我看没有什么紧张的。我也紧张,说不紧张是假的。上半夜你紧张紧张,下半夜安眠约一吃就不紧张了。

说我们脱离了群众,其实群众还是拥护我们的。我看困难是暂时的,就是三个月。春节前后,我看群众和我们结合得很好。小资产阶级狂热性有那么一点,但是并不那么多。我同意同志们的意见。问题是公社运动,我到遂平详细地谈了两个多钟头,嵖岈山公社党委书记告诉我,七、八、九三个月,平均每天三千人参观,十天三万人才,三个月三十万人。徐水、七里营听说也有这么多人参观,除了西藏都来看了。唐僧取经嘛。这些人都是县、社、队干部,也有地专干部。他们的想法是:河南人才,河北人创造了经验,打破了罗斯福免于贫困的“自由”。搞共产主义,这股热情,怎么看法?小资产阶级狂热性吗?我看不能那么说,要想多一点,无非是想多一点。这种分析是否恰当?三个月当中,三十万朝山进香,这种广泛的群众运动,不能泼冷水,只能劝说,同志们,你们的心是好的,事情难以办到,不能性急,要有步骤。吃肉只能一口一口地吃,要一口吃成个胖子不行。林×一天吃一斤肉还不胖,十年也不行。总司令和我的胖并非一朝一夕之功。这些干部率领几亿人民,至少30%是积极分子,30%是消积分子及地、富、反、坏、官僚。中农和部分贫农,40%随大流。30%是多少人,一亿几千万人。他们要求办公社,办食堂,搞大协作,非常积极。他们愿搞,你能说这是小资产阶级狂热性吗?这不是小资产阶级,是贫农、下中农、无产阶级,半无产阶级。随大流者也可以,不愿搞的30%,总之30%,加40%为70%,三亿五千万人在一个时期有狂热性,他们要搞。到春节前后有两个多月,他们不高兴了,变了,干部下乡都不讲话了,请吃红薯、稀饭,面无笑容,这叫刮“共产风”,但也要有分析,其中有小资产阶级狂热性,这是什么人?“共产风”主要是县社两级干部,特别是公社一部分干部,刮生产队和小队的,这是不好的。群众不欢迎,坚决纠正,说服他们,用一个月的功夫,三、四月间把风压下去,该退的退,社与队的账清了。这一个月的算账教育是有好处的,极短的时间使他懂得平均主义不行,“一平二调三提款”是不行的。听说现在大多数人转过来了,只有一部分人还留恋“共产风”还舍不得。哪里找这样一个大学校,短期训练班,使几亿人几百万干部受到教育。东西要交回,不能说你的就是我的,拿起就走了。从古以来,没有这个规矩,一万年以后也不能拿起就走。有,只有青红帮,青偷红劫,明火执仗,无代价地削剥人家劳动,破坏等价交换。宋江的政府叫“忠义堂”,劫富济贫,理直气壮,可以拿起就走,拿的是土豪劣绅的,那个章程,我看可以的。宋江劫的是“生辰纲”,就是我们打土豪劫的是不义之财,“劫之无碍”,刮自农民,归到农民。我们已长期不打土豪了,打土豪,分田地归公,那也可以,因为那也是不义之财。我们刮共产风,取生产队、小队之财,肥猪、大白菜拿起就走,这样是错误的。我们对帝国主义财产还有三种办法:征购、购买、挤垮,怎么能剥削劳动人民的财产呢?为什么一个多月就熄下这股风呢?证明我们的党是伟大的,光荣的、正确的,不仅有历史材料为证。三、四月份和五月有几百万干部和几亿农民受到教育,讲清了,他们想通了。主要是干部,不懂得这个财是义财,分不清界线,没有读好政治经济学,未搞通价值法则,等价交换,按劳分配,几个月就说通了,不办了。十分搞通的,未必有,几分通,七、八分通,教科书还没懂,叫他们读,公社一级不懂点政治经济学是不行的。不识字,可以讲。通几分,可以不读书,用事实来教育。梁武帝有个宰相陈发之,一字不识,强迫他作诗,他口念叫别人写,他说有些读书人,还不如老夫的用耳学。当然我不反对扫文盲,柯老说全民进大学,我也赞成,不过十五年得延长。还有南北朝有个姓曹的将军,打了仗以后要作诗:“出师儿女悲,归来笳鼓霓,借问过路人,何为霍去病”。还有北朝斛律金敕勒歌“敕勒川,阴山下,天似穷庐,笼盖四野。天苍苍,野茫茫,风吹草低见牛羊”。这也是一字不识的人。一字不识的人可以当宰相,为什么我们的公社干部,农民不可以听政治经济学呢?我看大家可以学,讲讲,政治经济学不识字可以讲,讲讲就懂了。他们比知识分子容易懂。教科书我就没有看,略为看了一点,才有发言权,要挤出时间,全党来个学习运动。

我们不晓得作了多少次检查了。从去年郑州会议以来,大作特作,有六级会议,影响五级会议,都要检查,北京来的人哇啦哇啦,他们就听不进去,我们检讨多次,你们就没有听到,我就劝这些同志,人家有嘴巴么!要人家讲么!要听听人家的意见。我看这次会议有些问题不解决,有些人不会放弃他们的观点,无非拖着么!一年、二年、三年、五年,听不得怪话不行,要养成习惯。我说就是硬着头皮顶住呵!无非是骂祖宗三代。这也难,我少年、中学时代,也是一听到坏话就一肚子气,人不犯我,我不犯人,人若犯我,我必犯人,人先犯我,我后犯人,这个原则,现在也不放弃。现在学会了听,硬着头皮顶住,听他一个、两个星期,再反击。劝同志们要听,你们赞成不赞成是你们的事,不赞成,如我错,我作自我批评。

第二方面,我劝另外一部分同志,在这样的紧急关头,不要动摇,据我观察,有一部分同志是动摇的。他们也说大跃进、总路线、人民公社都是正确的,但要看讲话的思想方面站在那一边?向那方面讲,这部分人是第二种人,“基本正确,部分不正确”的这一类人,但有些动摇。有些人在关键时就是动摇的,在历次大风大浪中就是不坚决的。历史上有四条路线,陈独秀路线、立三路线、王明路线、高饶路线,现在又是一条总路线。站不稳,扭秧歌。(国民党说我们是秧歌王朝)。他们忧心如焚,想把国家搞好,这是好的。这叫什么阶级呢?资产阶级还是小资产阶级?我现在不讲。在南宁会议,成都会议,党代大会讲过,1956年、1957年的动摇,不戴高帽子,讲成思想方法问题。如果讲小资产阶级的狂热性,反过来讲,那时的反冒进,就是资产阶级的冷冷清清,惨惨戚戚的泄气性,悲观性。我们不戴高帽子,因为这些同志和右派不同,他们也搞社会主义,只不过是没有经验,一有风吹草动就站不住脚,就反冒进。那次反冒进的人这次站住脚了。如××同志劲很大,受过那次教训,相信陈云同志也会站住脚的,恰恰是那次批判××同志的,他们那一部分人这次取他们的地位而代之。不讲冒进了,可是有反冒进的味道。比如说:“有失有得”。“得”放往后面是经过斟酌了的,如果戴高帽子,这是资产阶级动摇性,或降一等是小资产阶级动摇性。因为右的性质,往往受资产阶级的影响,在帝国主义、资产阶级压力下右起来了。

一个生产队一条错误,七十几万个生产队七十几万条错误都登出来,一年登到头,登得完登不完?还有文章长短,我看至少要一年,这样结果如何?我们的国家就垮台了,那时候帝国主义不来,国内人民也会起来把我们统统打倒。你办那个报纸天天登坏事,无心工作,不要说一年,就是一个星期也要灭亡的。登七十万条,专登坏事,那就不是无产阶级了,那就是资产阶级国家了,资产阶级的章伯均的设计院了,当然在座没有人这样主张,我是用夸大说法。假如办十件事,九件是坏的,都登在报刊上,一定灭亡,应当灭亡,那我就走,到农村去,率领农民推翻政府,你解放军不跟我走,我就找红军去。我看解放军会跟我走的。

我劝一部分同志讲话的方向要注意。讲话的内容基本正确,部分不妥,要别人坚定,首先自己坚定,要别人不动摇,首先自己不动摇。这又是一次教训。这些同志据我看不是右派是中间派,不是左派(不加引号的左派)。我所谓方向,是因为一些人碰了一些钉子了,头破血流,忧心如焚,站不住脚,动摇了,站到中间去了,究竟中间偏左偏右,还要分析。重复56年下半年、57年犯错误同志的道路,他们不是右派,可是自己把自己抛到右派边缘去了,距右派还有30公里,因为右派很欢迎这个论凋,现在有些同志的论调,右派不欢迎才怪。这种同志采取边缘政策,相当危险,不相信,将来看。这些话是在大庭广众当中讲的,有些伤人,现在不讲,对这些同志不利。

我出的题目中加一个题目,团结问题。还是单独写一段,拿着团结的旗帜,人民的团结,民族的团结,党的团结。我不讲对这些同志是有益是有害?有害,还是要讲。我们是马克思主义政党。第一方面的人要听人家讲,第二方面的人也要听人家讲。两方面的人都要听人家讲,我说还是要讲吗?一条是要讲,一条是要听人家讲。我不忙讲,硬着头皮顶住,我为什么现在不硬着头皮顶了呢?顶了廿多天,快散会了,索性开到月底。马歇尔八上庐山,×××三上庐山,我们一上庐山,为什么不可以?有此权利。

食堂问题,食堂是个好东西,未可厚非,我赞成积极办好。自愿参加,粮食到户,节约归己。我看在全国保持1/3我就满意了,一讲,吴芝圃就紧张了,不要怕。河南等省有50%的食堂还在,那也可以试试看,不要搞掉,我是就全国来讲。不是跳舞有四个阶段吗?“一边站,试试看,拚命干,死了算”。有没有这四句话?我是个粗人,很不文明。三分之一农民,一亿五千万坚持下去就了不起了。第二个希望,一半左右,二亿五千万,多几个河南、四川、湖南、云南、上海等等。取得经验,有些散了,还得恢复,食堂并不是我们发明的,是群众创造的,河北一九五六年公社化以前就有办的,一九五八年办得很快,曾希圣说:食堂节省劳动力,我看还有一条,节省物质,如果没有后面这一条,就不能持久。可否办到?可以办到。我建议河南同志把一套机械化搞起来,比如自来水,搞个东西不用挑,这样一来可以节省劳力,可以省物质,现在散掉一半左右有好处。总司令我赞同你的说法,但又和你的说法有区别。不可不散,不可全散。我是个中间派。我是个中间派,河南、四川、湖北等是左派,可是有个右派出来了。科学院昌黎调查组说食堂没有一点好处,攻其一点,不及其余,学“登徒子好色赋”的办法。登徒子攻宋玉三条:漂亮,好色、会说话,不能到后宫去,很危险。宋玉反驳说:“漂亮是父母所生,会说话是先生所教,好色无此事。天下佳人不如楚,楚国出丽者,莫若臣里,臣里之美者,莫若陈东家之子,增一分过长,减一分过短。……”登徒子是大夫,大夫就是今天的部长,是大部。如冶金部长,煤炭部长,还有什么农业部长,科学院调查组是攻其一点,不及其余。攻其一点的办法,无非是猪肉、头发卡子。无论什么人都有缺点,孔夫子也有错误,我也看过列宁的手稿,改得一塌糊涂,没有错误,为什么要改?食堂可以多一点,再试试看,试它一年、二年,估计可以办成。人民公社会不会垮台?现在没有垮一个,准备垮一半、垮七分,还有三分,要垮就垮,办得不好,一定要垮,共产党就是要办好,办好公社,办好一切事业,办好工业、农业、商业、交通运输,文化教育。

许多事情根本料不到,不是说党不管党吗?现在计划机关不管计划。一个时期不管计划。计划机关不只是计委,还有其它各部,还有地方,一个时期不管综合平衡。地方可以原谅,计委同中央各部十年了,忽然在北戴河会议后开始不管了,名曰计划指示,等于不要计划,所谓不管计划,就是不要综合平衡,根本不去算要多少煤,要多少铁,要多少交通。煤铁不能自己走路,要车马运,这点我没料到。我和××总理根本没有管。不知可说也。我不是开脱也是开脱,因为我不是计委主任,去年八月以前,主要精力放在革命方面,对建设根本外行,对工业计划一点不懂,在西楼(中南海西楼)时曾经说过不要写英明领导,管都没管,还说什么英明。但是,同志们,一九五八,一九五九主要责任在我身上。过去责任在别人××,××,现在应该说我,实在有一大堆事没管。始作俑者,其无后乎?我无后乎(一个儿子打死了,一个儿子发了疯)大办钢铁的发明权是柯庆施还是我?我说是我,我和柯庆施谈过一次话,说六百万吨。以后我找大家谈话,有×××也觉得可行,我六月讲1070万吨,后来去做,北戴河搞在公报上,××建议觉得可行,从此闯下大祸。九千万人上阵。……搞了小土群……看了很多讨论,大家讲还可以搞,要提高质量,降低成本,降低硫的成分,出真正好铁而努力奋斗。只要抓,也有可能。共产党有个方法叫抓,共产党和蒋介石都有两只手,共产党的手是共产主义者的手,一抓就抓起来了。钢铁要抓,粮油、棉、麻、丝麻、糖、药,还有烟果盐,农、林、牧、付、渔有十二项要抓,要综合平衡,各地不同,不能每县都一个模范。湖北有九峰山,白云中长竹木。要搞粮食,把竹木不搞了。有些地方不长茶,不长甘庶,要因地制宜。苏联不是搞过回民地区养猪么,岂有此理?工业计划搞了一篇文章,写得还好。至于党不管党,计划机关不管计划,不搞综合平衡,搞什么去了?根本不着急,总理着急,他不急。人不着急,没有一股神气,没有一股热情,办不好事情。有人批评计委李富春是“足将进而趑趄,口将言而嗫嚅”也,不要像李逵,太急了也不行。列宁热情磅礴实在好,群众很欢迎,口将言而嗫嚅,无非有各种顾虑。上半月顾虑甚多,现在展开了,有话讲出来了,记录为证,口说无凭,以此为证。你们有话讲出来嘛!你们抓住,就整我么,不要怕穿小鞋,成都会议讲过不要怕坐班房,甚至于不要怕杀头,不要怕开除党籍,一个共产党员高级干部,那么多顾虑,就是怕讲得不妥受整,这叫“明哲保身”啊!病从口入,祸从口出,我今天要闯祸,两种人都不高兴我,一种是触不得,一种是方向有点问题,不赞成,你们就驳,说主席不能驳,我看不对,事实上纷纷在驳,不过不指名,江西党校,中央党校一些意见就是驳,说始作俑者,其无后乎。一个是一○七○万吨钢。一○七○万吨钢是我建议,我下的决心,其结果是九千万人上阵,×××人民币,“得不偿失”。其次是人民公社,人民公社我无发明之权,有建议之权。北戴河决议是我建议写的,当时嵖岈山章程如获至宝。我在山东,一个记者问我:“人民公社好不好”?我说:“好”!他就登了报。小资产阶级狂热性也有一点,以后新闻记者要离开。

我有两条罪状,一条叫一○七○万吨,大炼钢铁,你们赞成也可以给我分一点,但是始作俑者是我。推不掉,主要责任是我,人民公社,全世界反对,苏联也反对,还有总路线是虚的实的,你们分一点,见之于行动是工业、农业。至于其他一些大炮,别人也要分担一点,你们那大炮也相当多,放的不准心血来潮,不谨慎,共产共的快。在河南讲起江苏,浙江的记录传得快,说话不谨慎,把握不大,要谨慎一点。长处是一股干劲,肯负责任。比那凄凄惨惨戚戚要好,但放大炮在重大问题要慎重,我也放了三大炮,公社,炼钢,总路线。彭德怀说他粗中无细,我是张飞粗中有点细。人民公社我说集体所有制。我说集体所有制到共产主义全民所有制的过程,两个五年计划太短了一些,也许要二十个五年计划。

说要快,马克思也犯过不少错误,天天想看欧洲革命要来了,又没来,反反复复,一直到死了,还没有来。到列宁时才来了,那不是性急?小资产阶级狂热性(某某插话说:列宁说世界革命形势到了,以后没有来。)马克思开始反对巴黎公社,季诺维也夫反对十月革命,季诺维也夫后来被杀了。马克思是否也杀呀?巴黎公社起来了,他又赞成,估计会失败,看到这是第一个无产阶级专政,三个月也好。要讲经济核算,这划不来。我们也有广州公社,大革命失败了。我们现在的工作是否像一九二七年那样失败?像二万五千里长征大部分根据地丧失,苏区缩小到十分之一?不能这样讲。现在失败没有?到会同志都说有所得,没有完全失败。是否大部分失败?不是,是一部分失败,刮了一阵共产风,全国人民受到了教育。

斯大林(社会主义经济问题)在郑州读过两遍,就讲学。现在要深入研究,否则事业不能发展,不能巩固。如讲责任,××、×××有点责任。农业部×××有点责任,第一个责任是我。柯老,你的发明权有没有责任?(柯老:有)是否比较轻?你那是意识形态问题。我是一个一○七○万吨钢,九千万人上阵,这个乱子就闹大了,自己负责。同志们自己的责任都要分析一下,有屎拉出来,有屁放出来,肚子就舒服了。


