\section[党内通讯(四)(一九五九年三月二十九日北京)]{党内通讯(四)(一九五九年三月二十九日北京)}
\datesubtitle{(一九五九年三月二十九日)}


上海几个县的材料可阅。

城市,无论工矿企业、交通运输业,财政金融贸易事业,教育事业及其他事业,凡属大政方针的制定和执行,一定要征求基层干部(支部书记、车间主任、工段长)群众中的积极分子等人的意见,一定要有他们占压倒多数的人到会发表意见,对立面才能树立,矛盾才能揭露,真理才能找到,运动才能展开。总支书记、厂矿党委书记、城市委书记、市委市府所属各机关负责人和党组书记、中央一级的司局长同志们,我们对于这些人的话,切忌不可过分相信。他们中的很多人几乎完全脱离群众,独断专行。上面的指示不合他们胃口的,他们即阳奉阴违,或者简直置之不理。他们在许多问题上,仅仅相信他们自己,不相信群众,根本无所谓群众路线。有鉴于此,尔后每年一定要召开两次五级或者六级、或者七级的干部大会,每次会期十天,上基层,夹攻中层,中层干部的错误观点才能改正,他们的僵死头脑才能松劲,他们才有可能进步,否则是毫无办法的。听他们的话多了,我们也不同化,犯错误,情况不明,下情不能上达,上情不能下达,危险至极。每年这样的大会开两次,对于我们也极有益处,可以使我们明了情况,改正错误。这里说的是城市问题,农村问题同样如此。我在前次通讯中,已经大体说过了。

<p align="right">毛泽东

一九五九年三月二十九日于北京</p>


