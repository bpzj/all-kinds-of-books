\section[在武昌会议上的讲话第一次讲话(一九五八年十一月二十一日上午)]{在武昌会议上的讲话第一次讲话(一九五八年十一月二十一日上午)}
\datesubtitle{(一九五八年十一月二十一日)}


睡不着觉,心里有事。翻一翻,作为第一本账。出点题目,请大家研究。你们写文章,我有我的一些想法。

(一)过渡共产主义,你们看怎么样?有两种方法,我们可能搞得快一些,看起来我们的群众路线是好办法。这么多人,什么事都可以搞。赫鲁晓夫的报告提纲,登在十一月十五日的《人民日报》上,希望看一看。要详细看一下,讨论一下。文章不长,也好看。他已经四十一年了,现在想再七年加五年,共十二年,看他意思准备过渡,但只讲准备,并没有讲过渡,很谨慎。我们中国人,包括我在内,大概是冒失鬼,赫鲁晓夫很谨慎,他已有五千五百万吨钢,一亿多吨石油,他尚且那样谨慎,还要十二年准备过渡。他们有他们的困难,我们有我们的长处。他们资产阶级等级制度根深蒂固,上下级生活悬殊,像猫和老鼠。我们干部下放,从中央以下干部都参加劳动,将军当兵。他们缺乏群众路线这一条,即缺政治。所以搞得比较慢,还有几种差别,工农、城乡、脑力体力,没有去破除。但他们谨慎。我们在全世界人民面前,就整个社会主义阵营来说,我设想一定要苏联先过渡(不是命令),我们无论如何要后过渡,不管我们搞多少钢,这条大家看对不对?也许我们钢多一点,因为我们人多,还有群众路线,十年搞几亿钢。他七年翻一番,五千五百万吨翻一番,一亿一千万吨,只讲九千一百万吨,留有余地。想一想对不对?因为革命,马克思那时没有成功,列宁成功了,完成了十月革命,苏联已经搞了四十一年,再搞十二年,才过渡,落在我们后头,现在已经发慌了。他们没有人民公社,他们搞不上去,我们抢上去,苏联脸上无光,整个全世界无产阶级脸上也无光。怎么办?我看要逼他过,形势逼人,逼他快些过渡,没有这种形势是不行的。你上半年过,我下半年过,你过我也过,最多比他迟三年,但是一定要让他先过。否则,对世界无产阶级不利,对苏联不利,对我们也不利。现在国内局势,我们至少有几十万、上百万干部想抢先,都想走得越快越好,对全局顾及不够。只看到几亿人口,没有看到二十七亿,我们只是一个局部(六亿人口),全世界是全局。是不是有这样一个问题?是不是要考虑?这个问题牵涉到我们的想法,作计划,对苏联的学习和尊重,去掉隔阂等一系列的问题。他们的经济底子比我们好,我们的政治底子又比他们好。他们两亿人口,五千五百万吨铜,一亿吨石油,技术那么高,成百万的技术人员,全国人民中学程度,它的本钱大,美国比不上他。我们现在是破落户,一穷二白,还有一穷二弱。我们之穷,全国每人平均收入不到八十元,大概在六十到八十元之间,全国工人平均每月六十元(包括家属)。农民究竟有多少,河南讲七十四元,有那么多?工人是月薪,农民是年薪。五亿多人口,平均年薪不到八十元,穷得要命。我们说强大,还没有什么根据。现在我们吹得太大了,不合乎事实。我看没有反映客观事实。苏联四一年,我们只有九年。我们搞社会主义建设没有经验,我看要过渡到共产主义,一定要让苏联先过,我们后过,这是不是机会主义,他是十二年只有一亿吨钢,我们也不能先过,也有理由,我们十年四亿吨钢,一百六十万台机器,二十五亿吨煤,三亿吨石油,我国有天下第一田,到那个时候,地球上有天下第一国。搞不搞得到是另一个问题。郑州会议的东西,我又高兴又怀疑,搞四亿吨钢好不好?搞四十亿吨更好。问题是有没有需要?有没有可能?今年到现在十一月十七日统计,只搞了八百九十万吨钢,已经有六千万人上阵,你说搞四亿吨要多少人?当然条件不同,鞍钢现有十万人,搞了四百万吨。让苏联先过,比较好,免得个人突出。我担心,我们的建没有点白杨树,有一种钻天扬,长得很快,就是不结实,不像邓××,就是不“钻”的。钻的太快,不平衡,可能搞得天下大乱。我总是担心,什么路线正确不正确,到天下大乱。你还说你正确啊?

(二)有计划按比例,钢铁上去各方面都上去,六十四种稀有金属都要有比例。什么叫比例?现在我们谁也不知道什么叫比例,我是不知道,你们可能高明一点。什么是有计划按比例,要慢慢摸索。恩格斯说,要认识客观规律,掌握它,熟练地运用它。我看斯大林认识也不完全,运用也不灵活,至于熟练地运用就更差,对工、农,轻、重工业都不那么正确,重工业太重,是长腿,农业是短腿,是铁拐李。现在赫鲁晓夫人有两条腿走路之势。我们现在摸了一点比例,是两条腿走路,三个并举。重工业轻工业和农业。何长工没有来,他的腿就是没有按比例,我们按三个并举,就是两条腿走路,几个比例,大中小也是个比例,世界上的事总有大中小的。现在十二个报告,我看了,大多数写得好,有些特别好。口语与科学名词结合也是土洋结合,过去我常说经济科学文章写得不好,你自己看得懂,别人看不懂,希望大家都看一遍。我们有这么多天,一个看一个就容易看完了,似乎我们有点按比例。三个并举,有个重点,重工业为纲,但真正掌握客观规律,熟练地运用它还有问题。

总的讲,是一定要让苏联先进入,我们后进入,如果我们实际先进入了,怎么办。可以挂社会主义的招牌,行共产主义的实际。有实无名,可不可以?比方一个人,学问很高,如孔夫子、耶苏、释迦牟尼,谁也没有给他们安博士的头衔,并不妨碍他们行博士之实。孔子是后来汉朝董仲舒捧起来的,但到南北朝又不太灵了,到唐朝韩愈这些人,又写了他,特别是宋朝的朱熹,朱夫子以后,圣人就定了,到了明、清两代才封为“大成至圣文宣王之位”,到“五四”运动又下降了,圣人不圣人吃不开了,我们共产党是历史唯物主义者,承认他的历史地位,但不承认什么圣人不圣人,他们的数学不及我们,初中程度,恐怕只是高小程度。如果说数学,我们大学生是圣人,孔夫子不过是贤人。就是说,我们过渡到共产主义,不封为圣人,搞个贤人和普通人,何必急急忙忙自封圣人?封个贤人这不妨碍本质,人有三种,普通人、贤人和圣人,就搞个圣人好了。我们共产主义者本质是圣人,不封。搞个贤人,并不妨碍本质,是否好一些。

我们也有缺点。北戴河会议讲三、四年或五、六年或更多一点时间,搞成全民所有制,好在过渡到共产主义还有五个条件,(1)产品极为丰富;(2)共产主义思想觉悟道德的提高;(3)文化教育的普及和提高;(4)三种差别和资产阶级法权残余的消灭,(5)国家除对外作用外。其它作用逐渐消失。三个差别,资产阶级法权消灭没有一、二十年不行。我并不着急,还是青年人急,三个条件不完备,不过是社会主义而已,这个问题请大家想一想。这不是说我们要慢腾腾的,多快好省是客观的东西,能速则速,不能勉强。图104飞机高到一万多公尺,我们飞机只几千公尺,老柯坐火车更慢,走路更慢。速度是客观规律,今年粮食九千亿,我不信。七千四百亿翻了翻。是可能的。我很满意了。我不相信八千亿斤,九千亿斤,一万亿斤。速度有两个可能,一是相当快,一个是不那么快。我们设想十年之内搞四亿吨钢,可能搞到,可能搞不到,一个是可不可能,一个是需不需要。究竟要不要这么多,买主是谁?无非是吕正操,修铁路,无非是造船,这是交通部的事。机械电气设备还有其他,究竟需要不需要。做到做不到?大概农业方面比较有把握,工业比农业难,你们办工业的,你们说能不能?真正全党全民办工业,只有两个月谁有把握?这就涉及到四十条了,是否就这样还没有把握,四十条这次可以议一下,不作为重点。郑州会议搞了,很好,有伟大的历史意义。但继续下去,议不出什么事来,可不可能搞四亿吨钢,需要不需要?×××同志给我的说明不解决问题。只说明可能,需要不需要,他没有回答。美国一亿吨钢。出口一千万吨稍多一点(包括机器),即十分之一。至少苦战三年,明年和后年,才能搞到一点边,心血来潮。一想就出个数目字。明年是否搞三千万吨钢,需要大概是需要的。可能不可能?大家议一下,今年一千一百万吨,是搞迟了。明年是十二个月。(××插话,三千万吨是元帅,其它怎么安排?)

四十条这个问题,如果传出去,很不好。你们搞那么多,而苏联搞多少?叫做务虚名而受实祸,虚名得也不到,谁也不相信,说中国人吹牛。说受实祸,美国人可能打原子弹,把你打乱。当然也不一定。将来一不可能,二不需要。这样岂不如自己垮台?我看还是谨慎一点。有些人里通外国,到大使馆一报,苏联首先会吓一跳,如何办?粮食多一点没关系,但每人一万斤也不好。要成灾的,无非是三年不种田。吃完了再种。听说有几个姑娘说,不搞亩产八万斤不结婚,我看他们是想独身主义的,把这个作挡箭牌。据伯达调查,她们还是想结婚的,八万斤是不行的。这是第二个问题,究竟怎样好?摆他两三年再说,横竖不碍事,过去讲过不搞长远计划,没有把握,只搞年度计划,但在少数人头脑中有个数,还是必要的,四十条纲要要有两种办法,一是认真议一下,作为全会草案讨论通过。另一种方法是根本不讨论,不通过,只交待一下。说明郑州会议的数字没有把握,但有积极意义。

(三)这次会议的任务。一是人民公社。一是明年计划的安排(特别是第一季度的安排)。当然还可以搞点别的。如财贸工作的“两放、三统、一包”等等。

(四)划线问题。要不要划线?如何划法?郑州会议有五个标准。山西有意见。建成社会主义的集中表现为全民所有制,这与斯大林在一九三六年宣布的不一致。什么叫完成全民所有制?什么叫建成社会主义?斯大林在一九三六年、一九三八年两个报告(前者是宪法报告,后者是十八次代表大会报告)提出两个标志:一是消灭阶级,一是工业比重已占百分之七十。但苏联过了二十年,赫鲁晓夫又来个十二年,即经过三十二年才能过渡,到那时候集体所有制和全民所有制才能合一,在这个问题上我们不照他们的办。我们讲五个标准。不讲工业占百分之七十算建成。我们到今年是九年,再过十年共十九年。苏联从一九二一年算起,到一九三八年共十八年,只有一千八百万吨钢,我们到一九六八年也是十八年,时间差不多,肯定东西要多,我们明年就超过一千八百万吨钢,我们建成,叫会主义。是所有制合为一个标准,都是全民所有制,我们已完成全民所有制为第一标准,按此标准,苏联就没有建成社会主义。它还是两种所有制,这就发生了一个问题。全世界人民要问,苏联到现在还没有建成社会主义?(曾希圣插话。这条不公布。)不公开也会传出去。另外一个办法,是不这样讲。像北戴河会议一样,只讲几个条件,什么时候建成不说,可能主动一些,北戴河文件有个缺点,就是年限快了一点。是受到河南的影响.我以为北方少者三、四年,南方多者五、六年,但办不到。要改一下。苏联生活水平总比我们高,还未过渡,北京大学有个教授,到徐水一看,他说;“一块钱的共产主义,老子不干。”徐水发薪也不过二、三元。十年三三制,一年调拨三分之一,那就是三分之一的全民所有制.当然另有三分之一的积累,总还有农民自己消费的,所以也近乎全民所有制了,现在就是吃穷的饭,什么公共食堂,现在就是太快,少者三、四年,多者五、六年,我有点恐慌,怕犯什么冒险主义的错误,×××脑子也活动了,认为长一点也可以。还有完成“三化”;机械化、电气化、园林化。要五年到十年,占压倒优势才叫化。(×××:达到150元到200元的消费水平,就可以转一批,将来分批转,这样有利,否则,等到更高了,转起来困难多,反而不利。)(×××:就是三化不容易做到,尤其园林化。)(××:我们搞了土改,就搞大合作,又搞公社,只要到每人150元到200元就可以过渡,太多了,如罗马尼亚那样,农民比工人收入多时,就不好转了,把三化压低,趁热打铁,早转此晚转好,三、四年即可过渡。)照你的讲法,十八年建成社会主义大有希望。(×××:机械化、电气化不容易。)(柯庆施,集体所有制是否促进生产?都包下来是否有利?)(×××:按三分之一调拨的三三制,恐怕要十年,三几年是做不到的。)按照××、××的意见,是趁穷之势来过渡,趁穷过渡可能有利些,不然就难过渡。总之,线是要划的,就是如何划,请你们讨论,搞几条标准,一定要高于苏联的。

(五)消灭阶级问题。消灭阶级问题,值得考虑。按苏联的说法,是一九三六年宣布的十六年消灭,我们十六年也许可能,今年九年,还有七年,但不要说死,消灭阶级有两种,一种是作为经济剥削的阶级容易消灭,现在我们可以说已消灭了,另一种是政治思想上的阶级(地主、富农、资产阶级,包括他们的知识分子),不容易消灭,还没有消灭,这是去年整风才发现的。我在一九五六年写的批语中有一条说,“社会主义革命基本完成,所有制问题基本解决”,现在看来不妥当了。后来冒出来一个章罗联盟,农村地主喜欢看《文汇报》,《文汇报》一到,就造谣了。“地、富、反、坏”乘机而起,所以青岛会议才开捉戒,开杀戒,湖南斗十万,捉一万,杀一千,别的省也一样,问题就解决了。那些地、富、反、坏经济上不剥削,但作为政治上、思想上的这个阶级,如章伯钧一起的地主、资产阶级还存在,搞人民公社,首先知识分子、教授最关心,惶惶不可终日。北京有个女教授。睡到半夜,作了一场梦,人民公社成立,孩子进了托儿所,大哭一场,醒来后才知道是一个梦,这不简单。

斯大林在一九三六年宣布消灭阶级,为什么一九三七年还杀了那么多人,特务如麻。我看消灭阶级这个问题让他吊着,不忙宣布为好。阶级消灭究竟何时宣布才有利,如宣布消灭了,地主都是农民,资本家都是工人,有利无利?资产阶级允许入人民公社,但资产阶级帽子还要戴,不取消定息。鉴于斯大林宣布早了,宣布阶级消灭不要忙,恐怕基本上没有害了,才能宣布。苏联的知识分子里面,阶级消灭的那样干净?我看不一定。最近苏联一个作家,写了一本小说。造成世界小反苏运动,香港报纸大肆宣传,艾森豪威尔说;“这个作家来了我接见。”他们作家中还有资产阶级,大学毕业生中还有那么人信宗教,当牧师。(××:爱仑堡如果在中国,就是一个右派。)恐怕他们以前没有经验,我们有经验,谨慎一些。

(六)经济理论问题。究竟要不要商品。商品的范围包括哪些了在郑州只限于生活资料,加上一部分公社的生产资料,这是斯大林的说法,斯大林主张不把生产资料卖给集体农庄。我国还宣布土地国有。机械化的机器自己搞。农民作不了的,我们供应。现在有个消息,苏联政治经济学教科书第三版,把商品范围扩大了。不但是生活资料,而且包括生产资料,这个问题可以研究一下,斯大林有一点讲的不通,农产品是商品。工业品是非商品,一个商品,一个非商品(国营工业的产品),两者交换(布匹与农庄粮食交换)这怎么能讲的通呢?我看现在的讲法比较好,生产资料。×××的钢吃不得,穿不得。赵尔陆的机器也是这样,化工穿用得多,张霖之,你的东西也不能吃,李葆华的水可以吃,电也不能吃。归根结底,生产资料为了制造生活资料(包括衣食住行。文化娱乐,唱戏的二胡、笛子、文房四宝等等)。一个时期。仿佛认为商品越少越好,时间越短越好。甚至两三年就不要。是有问题的。我看商品时间搞久一点好,不要一百年,也要三十年,再少说得十五年。这有什么害处。问题看有什么害处,看他是否阻碍经济的发展。当然。有个时期是阻碍生产发展的。因此,四十条中商品写得不妥当,还是照斯大林的写的,而斯大林对于国营生产的生活资料和集体农庄生产的生活资料的关系没弄清楚,请大家议一下,是政治经济学第三版,其他没有大改。所以斯大林的东西只能推倒一部分,不能全部推掉。因为他是科学,全部推倒不好。谁人第一个写社会主义政治经济学?还是斯大林。当然那一本书其中有部分缺点和错误,例如第三封信。为抓农民辫子起见.机器不卖给农庄。写规定有使用之权,无所有权,这就是不信任农民,我们是给合作社,……我问过尤金同志,农庄有卡车,有小工厂,有工作机具,为什么不给拖拉机?我们这些人,包括我,过去不管什么社会主义政治经济学,不去看书。现在全国有几十万人议论纷纷,十人十说,百人百说,还要看书,没有看过的要看,看过的再看一遍,还要看政治经济学教科书,你们看了没有?教科书每人发一套。先看社会主义部分。不是要务虚吗?

(七)会不会泼冷水?要人家吃饱饭,睡好觉,特别人家正在鼓足干劲,苦战几昼夜,干出来了。除特殊外,还是要睡一点觉。现在要减轻任务。水利任务,去冬今春全国搞五百亿土石方。而今冬明春全国要搞一千九百亿土石方,多了三倍多。还要各种各样的任务,钢铁、铜、铝、煤炭、运输、加工工业、化学工业,需要人很多,这样一来,我看搞起来,中国非死一半人不可。不死一半也要死三分之一或者十分之一,死五千万人。广西死了人,陈漫远不是撤了吗,死五千万人你们的职不撤,至少我的职要撤,头也成问题。安徽要搞那么多,你搞多了也可以,但以不死人为原则。一千九百多亿土石方总是多了,你们议一下,你们一定要搞,我也没办法,但死了人不能杀我的头,要比去年再加一点,搞六、七百亿,不要太多。×××和×××的文件其中有这么一项,希望你们讨论一下。此外,还有什么别的任务,实在压得透不过气来的,也可以考虑减轻些。任务不可不加,但也不可多加。要从反面考虑一下,翻一番可以,翻几十番,就要考虑。钢三千万吨,究竟要不要这么多?搞不搞得多?要多少人上阵?会不会死人?虽然你们说要搞基点(钢、煤),但要几个月才能搞成?河北说半年,这还要包括炼铁、煤炭、运输、轧钢等等。这要议一议。(×××:明年任务各省自议。三千万吨,他们同意不同意?不同意就得改?是不是三千万吨是应该考虑的)(插话,六千万人出了一千万吨铁,实际只有七百万吨,好铁只占百分之四十,不是按高估价。定点之后把人收回来,否则菜籽也无人收,口也不能出了。一千一百万吨钢,好钢不超过九百万吨,可能是八百五十万吨,如搞三千万吨是加二点五倍。)今年有两个侧面,中国有几个六千万人。几百万吨土铁,土钢,只有四成是好。明年不是不老老实实翻一番,今年一千零七十万吨。明年二千一百四十万吨。多搞一万吨。明年要搞二千一百四十一万吨。我看还是稳一点。水利照五百亿土石方,一点也不翻。搞他十年。不就是五千亿了吗,我说还是留一点儿给儿子去做,我们还能都搞完哪?

此外,各项工作的安排,煤、电、化学、森林、建筑材料、纺织、造纸。这次会议要唱个低调,把空气压缩一下,明年搞个上半年,行有余力,情况顺利,那时还可起点野心,七月一日再加一点。不要像唱戏拉胡琴,弦拉得太紧了,有断弦的危险,这可能有一点泼冷水的味道,下面干部搞公社,有些听不进去,无非骂我们右倾,不要怕,硬着头皮让下面骂.翻一番。自从盘古开天地,全世界都没有,还有什么右倾呀!?

农业指标搞多少?(×××:对外面说搞一万亿斤差不多,每人有两千斤就差不多了)北戴河会议的东西还要议一下,你说右倾机会主义,我翻一番吆!机床八万台,明年翻四番,搞三十二万合,有那么厉害?北戴河会议那时,我们对搞工业还没有经验。经过两个月,钢铁运输到处水泄不通,这就有相当的经验了。总是要有实际可能才好,有两种实际可能性,一种是现实的可能性,另一种是非现实的可能性,如现在造卫星就是非现实的,将来可能是落实的。可能性有两种,是不是?伯达同志:可能转化为现实的是现实的可能性,另一种是不能转化为现实的可能性,如过去的教条主义,说百分之百的正确。不是地方都丢了吗?我看非亩产八万斤不结婚,也是非现实的可能性。

(八)人民公社要整顿四个月,十二、一、二、三月要搞万人检查团,主要是看每天是否睡了八小时,如只睡七小时是未完成任务,我是从未完成任务的,你们也可以检查贴大字报,食堂如何,要有个章程,人民公社要议一下,搞个指示,四个月能不能整顿好?是不是要少了。要半年。现在据湖北说,有百分之七、八的公社搞得比较好了,我是怀疑派,我看十个公社,有一个真正搞好了的就算成功。省(市)地委集中力量去帮助搞好一个公社,时间四个月,到那时候要搞万人检查团,不然就有亡国的危险。杜勒斯,蒋介石都骂我们搞人民公社。都这样说,你们不搞公社不会亡。搞会亡,我看不能说他没有一点道理。总有两种可能性。一亡,一不亡。当然亡了会搞起来,是暂时的灭亡。食堂会亡,托儿所也会亡,湖北省谷城县有个食堂,就是如此。托儿所一定要亡掉一批,只要死了几个孩子,父母一定会带回的。河南有个幸福院死了百分之三十,其余的都跑了。我也会跑的,怎么不垮呢?既然托儿所、幸福院会垮,人民公社不会垮?我看什么事都有两种可能性:垮与不垮,合作社过去就垮过的,河南、浙江都垮过,我就不相信你四川那么大的一个省,一个社也没有垮?无非是没有报告而已。人身上的细胞从三岁小孩起,就开始要有一批死亡的,脱皮、掉头发都是局部的死亡现象。死细胞是生长过程,新陈代谢,有利于生长。党内有一部分党员成了右派,从支部到中央都有垮台的。中央的陈独秀、张国焘、高岗、饶漱石还不是垮台了,王明还没有垮台,现在他的态度好转,(给中央的信,给他印发)可能是…我们这条路线硬是好像百分之百的正确。

我是提问题。把题目提出来,去讨论,那样为好,各个同志都可以提问题,这些时候,这些问题在我的脑子里,总是十五个吊桶打水七上八下,究竟那个方法好。如钢铁究竟是三千万吨还是二千一百四十万吨好?

这次会议是今年这一年的总结性会议。已十二月了嘛,安排明年,主要是第一季度。


