\section[向莫斯科的全体中国留学生、实习生、使馆机关干部的讲话(摘录)(一九五七年十一月十七日在莫斯科大学)]{向莫斯科的全体中国留学生、实习生、使馆机关干部的讲话(摘录)(一九五七年十一月十七日在莫斯科大学)}
\datesubtitle{(一九五七年十一月十七日)}


同志们:

我向你们问好。(热烈鼓掌)

世界是你们的,也是我们的,但是归根结底是你们的。(鼓掌)我们都老得这样了。但各有各的长处。我们老的有经验,你们青年人朝气蓬勃,正在兴旺时期,好像早晨八、九点钟的太阳。希望寄托在你们身上。(热烈鼓掌)

世界的风向变了。去年气候不好,今年气候好了。社会主义阵营和资本主义阵营间的斗争,“不是西风压倒东风就是东风压倒西风”。你们读过《红楼梦》没有?(有人答:读过。)这句话是《红楼梦》里的林黛玉说的。二个阵营,其中有中间地带,西方世界有四亿人,其中有很多我们的人,我们可以挖它的墙角,那里会发生地震的。我们有十亿人,我们中间也有它们的人,譬如中国的右派。这种人比较少,在中国约占百分之二左右。二方面都有对方的人。就好比宋末元初的赵孟俯妻子的一首词里所说的:“二个泥菩萨,一起都打碎,用水调和,再做成二个泥菩萨,你身上有我,我身上有你。”这个比喻虽然不完全恰当,但有点是对的,即帝国主义阵营中有我们的人,我们这里也有他们的人。但他们阵营中我们的人多,在我们阵营中他们的人少。

据联合国的统计,全世界共有二十七亿人,我们约有十亿人,帝国主义约有四亿人,还有几亿呢?(台下有人说:十三亿。)你们都是数学家,一下就算出来了。这十三亿基本上分布在三个洲:亚洲、美洲、拉丁美洲。十三亿中七亿多已取得民族独立,如印度、印尼、巴基斯坦、埃及、苏丹、突尼斯、摩洛哥、黄金海岸等。还有六亿在那边,如日本、伊朗、台湾、南朝鲜、南越、土耳其等那一套。在帝国主义阵营中德、意、日不想打仗,也打不起来。美不合作,这中间地带的十三亿人由二个阵营争夺,他们大多数是倾向我们的。(鼓掌)因为英法有老殖民主义,美国有新殖民主义,我们什么殖民主义也没有,也没有在那里搞军事基地。

我们中国是个大国,(热烈鼓掌)又是小国。在政治、人口上是大国,但是经济是小国,还比不上比利时呢!你们大概不高兴吧!(台下有人说:不高兴。)但又有什么可以不高兴的呢!比得上就比得上,比不上就比不上!

马克思、恩格斯以后一百年来这次大会是最大的一次,六十四个国家的共产党都参加了,这几天十二个社会主义国家在开会,商量了很多事情。这个会开得很好,决定了很多事情,决定社会主义阵营以苏联为首。你们不反对吧!(台下说:不反对。)(热烈鼓掌)这二天在开六十四个国家共产党的会议,今天是星期日,休息一天,估计明后天就能结束。

十月社会主义革命是人类历史一上一个大转折点。在人类历史土有很多转折点,如斯大林格勒战役是第二次世界大战的转折点。二个卫星上天,六十四个国家的共产党在一起开会,这也是大转折点。这是世界的战争,西风压不倒东风,东风一定要压倒西风。(暴风雨般的掌声)

真正的彻底的社会主义革命不是一朝一夕可以成功的,在我国真正的社会主义革命的胜利,有人认为在一九五六年,我看不对,应在一九五七年。一九五六年,所有制的改变是比较容易的。就这么一提,好比人民政府在这头,工农群众在那头,资本家在中间,二边一挟就挟过来了。

有些外国人说:我们思想改造是洗脑筋。我看也说得对,就是洗脑筋嘛,我这个脑子也是洗出来的。参加革命后,慢慢洗,洗了几十年。我从前受的都是资产阶级教育,还有一些封建教育。孔夫子的书读了不少。我们那时根本不知道马克思、恩格斯,只知道华盛顿、拿破伦。你们就好了,你们很幸福,像你们这么大的娃娃就知道了马克思、恩格斯、列宁、赫鲁晓夫、多列士、杜克洛、陶巴亚蒂等。我们那时对于中国革命是怎么搞法,有谁知道?

我们大家都要割尾巴,我也劝你们割尾巴。中国有句古话“夹起尾巴做人”,这句话很有道理。现在人都进化了,摸起来就没有尾巴了,但无形的还有。右派就是尾巴翘得太高了。(大笑)青年人应该具备二点,一是朝气勃勃,二是谦虚谨惧。(热烈鼓掌)

今年国内五月到六月是满天乌云,我们的方针是硬着头皮顶住,让右派骂。他们骂共产党是王八蛋,共产党不能领导中国革命,社会主义建设成绩少错误多。我们把它都登在人民日报上了。我们还给机关学校下了指示:硬看头皮,别开口,国内贴了许多大字报,你们这里没有贴吧?北京大学贴了几万张,人民日报是“小字报”。好,鸣放就鸣放,就到人民面前去报账,让人民去讨论。右派是打垮了,我们工作中的缺点还是有的。你们没有做过工作,不知道。你们去当当厂长、党委书记、校长、副校长、教授、工程师试试看。一做工作,就会有错误。八年来错误是有的。这次整风是件很大的事,我们要认真的改。世界上就怕“认真”二字,共产党最讲认真。干部放下去,和群众打成一片,农民都说:“老八路又来了。”基层干部中只有百分之一有严重缺点,脱离群众,绝大部分都是好的,其中一部分有错误,但可以改正。

我国人口现有六亿四千万,已经不是六亿了。(鼓掌)要六亿四千万人动手,人人振奋,移风易俗,改造世界。要做到这一点,问题很复杂。你们看过农业发展纲要四十条没)?(回答:看过)现在新的四十条出来了,老的四十条基本上是正确的,但部分有主观主义成分,在新的纲要中,化学肥料增加到一千五百万吨。要在第二个五年计划内,使全部合作社在生产和消费上超过富裕中农。我曾召集不少省委书记、地委书记谈话,问他们能否做到,他们都说完全可能,有的还说能超过。又如除四害的问题,这不是一件简单的事情。有人建议麻雀不要打,那就网开一面,城里的不打。这里有没有四川人?(回答说:有)四川的老鼠是多得很。北京的苍蝇打完了,过了两年又有了。这个问题也要有决心,大家动手,人人振奋,移风易俗。这类事情上也是:不是东风压倒西风,就是西风压倒东风。

我们现在生产力还很低,钢只有五百二十万吨。过了一个五年计划后,将有一千二百万吨,再过一个五年计划后,钢的产量就有二千二到二千四百万吨,到第四个五年计划完成时,就会有四千多万吨。我问过波立特同志,他说十五年后英国约能生产三千万吨,那么,再经过三个五年计划,我们就超过英国,苏联就会超过美国。(热烈鼓掌)那时世界面貌就会大大改变。

要完成这个任务还要十五年或者多一些,这个责任就落在你们身上了。(热烈鼓掌)我也有个五年计划,再活五年。如再活十五年我就心满意足了,(高呼:毛主席万岁!)能超额完成,当然更好。可是,天有不测风云,人有旦夕祸福,这也是自然辩证法,要是孔夫子现在还不死,二千多前的人现在还不死,那还成什么世界。(高呼:毛主席万岁,万万岁,)所以,我开始就和你们说了,世界是属于你们的。现在我再说一句,祝贺你们,世界是属于你们的。(全体起立,热烈鼓掌,欢呼)

毛主席的三项嘱咐:

一、和苏联朋友要亲密团结;

二、青年人既要勇敢,又要谦虚;

三、祝同学们身体好,学习好,将来工作好。


