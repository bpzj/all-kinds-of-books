\section[在成都会议上的讲话(六)(一九五八年三月二十六日)]{在成都会议上的讲话(六)}
\datesubtitle{(一九五八年三月二十六日)}


会议文件怎样处理,有些文件可以发给省、地、县,各省、各部选择一下,不一定都印。七届二中全会决议有必要可以印,反分散主义草案可以印发参考,人民内部矛盾报告的国内外反应可以印一本。至于这些讨论的指示,记录等,还要等候北京中央政治局发正式文件,不必全部印,也不禁止印,选印为好。总而言之,自己选择。

这次会议开得还可以,但是事先未准备虚实并举,实多了一点,虚少了一点,如果虚也有五天就好。这次实业长了一点,但也有好处.一次解决大批问题.并且是跟地方同志

一道谈的。也就比较合乎实际。虚实并举,先实后虚或先虚后实(南宁是先虚后实),各省各部可以去斟酌情况办理。也不是讲任何会议都要一虚一实,过去我们太实了,现在希望虚多一点好,以便引导各级领导同志关心思想、政治、理论的问题。红与专结合。希望各省、各部去安排一下,没有到会的省、部由协作区区长,中央同志去传达。

一年抓四次,三年看头年,是否对?如果不抓四次,改为半年一次.由于形势发展快,很多矛盾要很快反映和解决,不抓四次,许多问题不能及时解决,还是一年抓四次。省、地,县可否这样?请地方同志去斟酌。协作区会议一年六次,每两月一次(曾有规定)是否引起大家埋怨开会太多了,开一下再看看,两个月一次,一次的时间不能太长,觉得太多了,将来再减少。目的是今年抓紧一点,以便更及时地掌握群众的情绪,稳一点掌握建设的速度。

下次会议七月开,重点是工业。

现在的问题,还是不摸底,农业比较清楚,工业,商业、文教不清楚。工业方面除到会的几个部接触了一下外,其余没有摸,煤、电、油、机械、建筑、地质、交通、邮电、轻工业、商业没有接触。财贸还有文教历来没有摸过,林业没有摸过,今年,这些要摸一摸。政治局书记处都摸一摸,政治局开座谈会是个好办法。过去有一句口号:“工农兵学商团结起来,打倒帝国主义!”现在还是工农兵学商团结起来,(学是文教,兵是国防),今后这五年,还是要抓五方面。这次接触了国防,但是没有怎么谈论,过去总是搞军事。现在几年都不开会,文件都没有看。人有五官——眼、耳,口,鼻,舌。五性——色、声、香、味、触。我们工农兵学商样样有。还要加上一个“思”。南宁会议讲工农和思想,再次要讨论国防问题。地方也要讲点军事工作。从一九五三年下半年起(抗美援朝后)没有管国防。军事工作,地方也只是抽兵走,转业来而已,地方也要管军事工作。今后要回过头来搞点军事工作。

阶级分析,我们国内存在两个剥削阶级、两个劳动阶级。两个剥削阶级。第一是帝国主义、封建主义、官僚资本主义的残余。地、富、反、坏未改造好的部分。现在要加上右派。反社会主义的阶层。富农一有剥削,有选举权,但不受人尊重。右派本来是与我们合作的,现在他们反社会主义,故看到敌人。国民党所做的事、就是右派做的事。帝国主义、台湾蒋介石非常赞成、关心右派分子。地、富、反、坏、右是可以改造的,大多数可以改造好的。第二是民族资产阶级、资产阶级知识分子、民主党派的大多数(民主党派中的右派占百分之十。其余百分之九十是中间派和左派),至于资产阶级和资产阶级知识分子中右派占百分之十,老教授中的右派比例多了(大概全国约有右派分子三十万人,其中县以上和大专是十六万)。

右派这么多,所以釆取除少数外,不提它不取消选举权,而釆取分化改造的政策。中间派对我们又反对又拥护。“苏报案”是章士钊的文章。他反社会主义反共。对民族资产阶级三百万人要做好工作。他们是可以转变的。

劳动阶级是工人、农民。过去的被剥削者和不剥削人的独立劳动者也可以说是两个劳动阶级。独立劳动者,有一部分是有轻微剥削的,如富裕中农和城市的上层小资产阶级.

有些同志说,希望第一书记工作解放一点出来,从中央、省到地三级的第一书记和其他某些同志解放一部分繁重工作,这才有可能比较注意一点较大的问题。党报的总主笔也须解放一部分。不能天天工作,少搞一点事,就有可能多管些事。解放出来,做一些调查研究工作,如何解放,大家研究。


