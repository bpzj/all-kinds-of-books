\section[关于劳改工作的指示(一九六五年四月)]{关于劳改工作的指示}
\datesubtitle{(一九六五年四月)}


劳改犯办了很多事情,要用,在一定条件下,可以为我们所用。我们没有几个贪污分子不行,这样大的国家没有几个反革命分子不行。

他们可有才能!没有才能,反革命干什么?在一定条件下他能做很多事,凡有功的,可以摘掉帽子,有的还可以奖励。二十三条为什么规定这么一条:工作队的成员不一定要十分干净,有坏人不要怕,有些人政治上不好,但很有才干,改造嘛!改造不了也不要紧,只要把他们引到正路,就很有用,我们不会贪污,不懂贪污,也不懂技术,他们懂得这些,只要我们知道他们,政治挂帅,可以让他们办很多事,可以发挥他们的作用。就是要调动他们,他们中间可有才能,他发明真空电炉,是高级产品,我们不行,他行。

\kaoyouerziju{ (1965年4月,关于王灿文发明真空冶炼电炉的指示)}

改造要抓紧,不要在经济上做文章,不要想在劳改犯人身上搞多少钱,要抓改造,让他们寄点钱回家。

第一是思想改造,第二是生产,劳改工作干部不能太弱,要训练。训练两个星期就行了。

\kaoyouerziju{ (1965年4月,在十四次公安会议期间对劳改工作的指示)}


