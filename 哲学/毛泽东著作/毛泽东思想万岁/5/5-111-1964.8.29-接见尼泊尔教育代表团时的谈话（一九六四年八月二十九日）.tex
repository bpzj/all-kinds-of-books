\section[接见尼泊尔教育代表团时的谈话(一九六四年八月二十九日)]{接见尼泊尔教育代表团时的谈话}
\datesubtitle{(一九六四年八月二十九日)}

\begin{list}{}{
    \setlength{\topsep}{0pt}        % 列表与正文的垂直距离
    \setlength{\partopsep}{0pt}     % 
    \setlength{\parsep}{\parskip}   % 一个 item 内有多段,段落间距
    \setlength{\itemsep}{\lineskip}       % 两个 item 之间,减去 \parsep 的距离
    \setlength{\labelsep}{0pt}%
    \setlength{\labelwidth}{3em}%
    \setlength{\itemindent}{0pt}%
    \setlength\listparindent{\parindent}
    \setlength{\leftmargin}{3em}
    \setlength{\rightmargin}{0pt}
    }
\item[\textbf{主席:}] 欢迎你们。

\item[\textbf{潘迪:}] 能够同您这样一位伟大人物会见,我们教育代表团的全体团员都感到非常高兴和十分荣幸。我们无法用言语来表达这种快乐。

\item[\textbf{主席:}] 谢谢。我没有什么伟大,跟你们差不多,在某些方面可能比你们还差一些。

\item[\textbf{潘迪:}] 我们都把您看作是一个伟大国家的伟大领袖。

\item[\textbf{主席:}] 可能你们看得不对。你们的国王很好,是一个很好的人。\marginpar{\footnotesize 169}

\item[\textbf{潘廸:}] 马享德拉国王的确是我国最伟大的领袖,在他执政期间,我们的国家很好地向前迈进了。

\item[\textbf{主席:}] 你们的国家有进步,有发展。逐步摆脱了帝国主义势力以及某些人强加给你们的影响,独立自主地建设自己的国家。我们也逐步摆脱了帝国主义和某些人的不正确影响。

以教育制度来说,我们正在进行改革。现行的学制年限太长,课程太多,教学方法有很多是不好的,考试方法也有很多是不好的。学生读了课本还是课本,学了概念还是概念,别的什么也不知道。四体不勤,五谷不分。许多学生不知道什么是马、牛、羊、鸡、犬、豕,也分不出什么是稻、粱、菽、麦、黍、稷(粱就是小米,菽就是豆子,黍就是黍子)。学生要读到二十几岁才能读完大学,学年太长了,课程太多。采取的方法是注入式而不是启发式。考试的方法是把学生当敌人看待,举行突然袭击。(众笑)所以我劝你们千万不要迷信中国的教育制度,不要以为它是好的。现在要改革还有很大的困难,有许多人就是不赞成。目前赞成新方法的少,不赞成的多。这可能泼了你们的冷水,你们希望看好的,而我专门讲坏的。(笑声)

但是,也不是一点好的都没有,比如说,拿工业方面的地质学来讲,旧社会给我们留下来的地质方面的学者和技术工人只有二百多人,现在我们有二十多万了。

大体上可以说,搞工业的知识分子比较好一些,因为他们接近实际。搞理科的,也就是搞纯科学的就差一些,但是比文科还要好一些。最脱离实际的是文科。无论学历史的也好,学经济学的也好,都太脱离实际,他们最不懂得世界上的事情。

请喝茶。你们国家的人民是不是也喝茶?

\item[\textbf{潘迪:}] 我们也喝茶,但是要加糖和牛奶。

\item[\textbf{主席:}] 我们不加糖也不加牛奶,没有这个习惯。

大多数人民喝白水,根本不喝茶。居住在高山上的人要喝茶。

\item[\textbf{潘迪:}] 中、尼这样比较冷的地方需要喝茶,以增加热量。

\item[\textbf{主席:}] 西藏的人民以吃牛羊肉为主,他们要喝茶。以粮食为主的地方可以不喝茶。同时茶叶也没有那么多,供应不了。你们国家产茶叶吗?

\item[\textbf{潘迪:}] 产得很少。最近几十年才有此习惯,现在不仅有钱的人,一般的人也喝茶。

\item[\textbf{主席:}] 茶叶相当贵,很穷的人是喝不起的。我们的国家还是个穷国,文化也是落后的。(问×××部长)普及教育的情况怎么样?

\item[\textbf{××:}] 城市基本上普及了,农村还不行,有的地方好一些,有的地方差一些,农村学龄儿童入学的一般只达到百分之六十几。

\item[\textbf{主席:}] 比解放前的情况好,城市基本上普及了,但是乡村还没有。

\item[\textbf{×××:}] 尼泊尔在马享德拉国王执政以后,文化数育有很大的发展。克伊腊克先生是一位诗人,老教育家。

\item[\textbf{克伊腊克:}] 我对教育工作很有兴趣。我从事教育工作有五十多年,我的兴趣就是扫除文盲。

\item[\textbf{主席:}] 这很好。我也当过几年教员,当的是小学教员。后来闹革命,就当不成小学教员了。那时组织工会,搞罢工,组织农民协会,同农村的恶霸作斗争。然后蒋介石搞白色恐怖,把我们赶到山上了,一打就是十年。以后日本人打进来了,一打又是八年,十年加八年不是十八年了吗?日本人走了以后美国人又来了,\marginpar{\footnotesize 170} 支持国民党向我们进攻,又打了四年。十八年加四年,就是二十二年。如果加上朝鲜战争就是二十五年。打了这么多年的仗,但是打仗这门学问我没有学过,也没有看过什么兵法,自己也没准备去打。谁人叫我去打的呢?就是帝国主义和蒋介石。他们实行白色恐怖,到外杀人,我们这些人只好上山。当时没有枪,就从蒋介石那里夺取武器。也没有外国援助。是外国人援助蒋介石,而蒋介石再援助我们。所以我们也可以说还是有外国援助的。(笑声)

(×××向主席介绍代表团几位团员的身分。)

\item[\textbf{主席:}] 你们会受到中国人民的热诚欢迎的。我们两国是友好国家。潘迪:我们在中国受到的热情款待,使我们感到像在亲人家里一样。主席:不能搞大国沙文主义,不能看不起小国。说小国不行是错误的。另一方面,小国自己也要有信心。大国有大国的缺点,小国有小国的长处。你们这个国家在历史上对帝国主义没有屈服过。帝国主义没有能够征服你们,可是征服了我们的国家。怎样征服法?就是让中国政府听外国人的命令。清末皇帝是听外国人的命令的。孙中山建立了第一个共和国,但是几个月便垮了。然后,袁世凯作皇帝,他也听从外国人的命令。然后,就是北洋军阀专政,造成中国的分裂。他们之间打了多年的仗。然后,就是国共两党合作北伐,打胜仗。然后,就是国民党杀共产党,我们就跟他打了十年。

蒋介石统治中国,他开始是听英国人的,后来是听美国人的,因为英国人不行了。然后日本人打进来,打了八年。日本人走了,又同国民党打了四年,全国才获得解放。现在蒋介石还在台湾,美国人管着他。他“代表”全中国人民参加联合国,而我们却没有权利进入联合国。

联合国批评我们是“侵略者”,侵略者怎么能够加入联合国呢?头一个“侵略者”是我。说我们主要是“侵略”了中国,然后是“侵略”了朝鲜,然后听说是“侵略”了印度。我们跟印度打了几星期仗,为什么后来把兵撤回来呢?因为他们的兵都散了,没有兵了,打仗没有对象了!(笑声)现在我们撒回到边界以后二十公里的地方。印度人现在好一些了。比较守规矩了。一条所谓麦克马洪线,中国连袁世凯都没有承认,他要我们承认,岂不荒唐?我们事实上不越过这条线,而且从这条线后退二十公里。

我们同你们两国的边界问题解决得很好,很容易讲得通,同缅甸也很容易讲得通,很容易解决。同巴基斯坦也容易讲得通,很容易解决。同阿富汗也签订了边界条约。就是印度这个朋友很难讲得通。

\item[\textbf{马拉:}] 为什么?

\item[\textbf{主席:}] 我也不知道。印度人民是好的,印度政府中有一些坏人。印友人民的生活并不怎么好。印度人民对我们,对你们都是好的。

听说印度同你们的关系比较好一些了,这是一件好事,我们赞成。

关系老是那么紧张不好。现在印度政府同我们的关系也不是那么坏了,首先他们的军队不越过麦克马洪线了。过去他们越过这条线到这边来,越过了几十公里,把我们向这一边挤,现在比较规矩一点了。这就算好。

你们这个国家要通过外国才能进口和出口,人员往来也要通过外国,所以你们同外国的关系搞好一些是好事。所谓通过外国,就是说通过印度。\marginpar{\footnotesize 171} 在关系紧张的时候,印度就是捣乱,比如机器运到你们国家,他们进行破坏,甚至把一些另件拆掉。现在不知道情况改善了一些没有?

\item[\textbf{潘廸:}] 现在稍好了一些,例如,我曾买了一架中国收音机,拔针是坏的,许多人买的拔针也被搞坏了。

\item[\textbf{主席:}] 所以你们要开一个后门,现在后门还没有通。

\item[\textbf{潘廸:}] 中国政府帮助我们修筑的从加德满都到科达里的公路修通以后,就有一个后门了。除此以外,我们还在开辟别的道路。主席:这可能是有益处的。

\item[\textbf{巴斯尼亚特代办:}] 团长已经说过,我们正在探索各种通道,除了加德满都到西藏边界的一条以外,还有一条路,通过巴基斯坦的达卡。

\item[\textbf{巴特:}] 我们这次来中国就没经过印度,而是直接经过达卡来的。

\item[\textbf{主席:}] 等几条路都通了,那时印度可能就比较尊敬你们了。(代表团的团员们频频点头)

\item[\textbf{巴斯尼亚特代办:}] 我曾去过拉萨,这使我相信,从加德满都到西藏的路修通以后,对中尼两国的经济、贸易和文化交流将起重大作用。

\item[\textbf{克伊腊克:}] 中、印是两个大国,而我们是小国。你们两个大国打架,我们就害怕,你们两国和好了,我们就可以高枕无忧了。

\item[\textbf{主席:}] 主要问题还不是一条麦克马洪线的问题,而是西藏问题。因为我们进军西藏,后来又进行了改革,印度政府就不高兴了。因为就是那位麦克马洪先生,在几十年前背着中国政府,同西藏地方当局签订了一个协定。所以在印度政府看来西藏是他们的。

\item[\textbf{克伊腊克:}] 您刚才说中印关系正在改善,这是很好的消息。我们希望中印的争执问题能够友好解决,那处于二者之间的我们,日子便好过了。

\item[\textbf{主席:}] 现在不大吵架了,双方对骂的照会也少了。

\item[\textbf{马拉:}] 希望最近的将来,中印的边界争端能够得到双方都满意的解决。

\item[\textbf{主席:}] 这是可能的。

\item[\textbf{马拉:}] 这是个好消息,令人非常高兴。

\item[\textbf{主席:}] 两个国家应该和好。

\item[\textbf{马拉:}] 您能不能告诉我们,您所以这样伟大的秘密是什么?您怎么能够这样伟大,您力量的源泉是什么?以便让我们多少能学到一点。

\item[\textbf{主席:}] 我已经说过,我没有什么伟大。就是从老百姓那里学了一点知识而已。当然我们学了一点马克思主义,但是单学马克思主义还不行,要从中国的特点和事实出发来研究中国问题。

我们中国人,比如像我这样的人,开始时对中国的情况并不了解,知道要反对帝国主义,反对帝国主义的走狗,但就是不知道如何反法。这就要求我们研究中国的情况,同你们要研究你们国家的情况一样。我们花了很长一段时间,由中国共产党成立到全国解放,整整化了二十八年,才逐步形成了一套适合中国情况的政策。

力量的源泉就是人民群众。不反映人民群众的要求,哪一个人也不行。要在人民群众那里学得知识,制订政策,然后再去教育人民群众。所以要当先生,就得先当学生。没有一个教师不是先当过学生的。而且就是当了教师以后,也还要向人民群众学习,了解自己学生的情况。所以在教育科学中有心理学、教育学这两门科学。\marginpar{\footnotesize 172}

\item[\textbf{巴特:}] 主席阁下,中国学校,向青少年进行“五爱”的教育,教育他们爱祖国,爱邻居(原话如此),爱科学,等等,这给我以很深的印象。我个人一直这样想,只有一个伟大国家才能以这样的精神教育每一个人。

\item[\textbf{主席:}] 你的意见怎么样?

\item[\textbf{巴特:}] 刚才我已经说过,我对此印象很深。另一件给我们以很深的印象的事情是,今天上午我们参观了清华大学,这个学校的付校长谈到他们怎样把高等教育同实际结合起来。

我国也面对着这样的问题,许多人念了很多书,但是不大了解实际。

\item[\textbf{主席:}] 清华大学有工厂。它是一所理工科大学,学生如果只有书本知识而不做工,那是不行的。

但是,大学文科不好设工厂,不好设什么文学工厂,历史学工厂,经济学工厂,或者设什么小说工厂。(笑声)文科要把社会作为自己的工厂。师生应该接触农民和城市工人,接触农业和工业。不然学生毕业以后用处不大。比如学法律的,如果不到社会中去了解犯罪的情况,法律是学不好的。不可能有什么法律工厂,要以社会为工厂。

所以比较起来,我国的文科最落后,就是因为接触实际太少。无论学生也好,教授也好,都是一样。就是在教室里讲课。讲哲学,就是书本上的哲学。如果不到社会上和人民中间去学哲学,不到自然界中去学哲学,那种哲学学出来没有用处,仅仅是懂得一点概念而已。逻辑学也是如此,可以读一点课文,但是不会懂得很多,只有在运用中才能逐步理解。我读逻辑的时候就不大懂。在运用的时候才逐步懂得。这里我讲的是形式逻辑。还有,比如学文学的要学语法(Vammav),读的时候也不大懂。要在写作的过程中才能理解语法的用处。人们是按照习惯写文章,习惯讲话的,不学语法也可以。我国几千年来就是没有语法这门课的,但是古人的文章有些写得相当好。当然,我并不是反对语法。

至于修辞学,学也可以,不学也可以。伟大的文学家并不学什么修辞的(对克伊腊克),你是先学了修辞学再写文章的吗?(笑声)

\item[\textbf{克伊腊克:}] 不,思想上得到启发,或者说有了“灵感”以后,就进行写作,而不是先学修辞学。

\item[\textbf{主席:}] 我就是不理修辞学的。我看过修辞学,但是不理它。照修辞学上说的办法是写不出好文章的,清规戒律太多。

各位还有问题吗?

\item[\textbf{克伊腊克:}] 要爱一切人,而不要仇恨任何人,这是生活中的一条原则。照此原则行事,日子就会过得很好。中、印好比两条大牛,尼泊尔好比在旁边的一只小牛犊,大牛打架时,小牛犊就怕得要命,担心被粉碎了。中、印和好了,我们就很高兴。小牛犊就能平安无事了。

\item[\textbf{主席:}] 只能爱大多数的人。比如说,我们爱蒋介石,但是他不爱我们,(笑声)他要吃掉我们。过去,日本帝国主义占领大半个中国,要变中国为它的殖民地。那我们没有办法,只好打。

\item[\textbf{克伊腊克:}] 实行自尊、自学、自制(克制自己)的三条原则,就可以使自己获得独立的主权。我们看到中国每一个人都有自尊、自爱的精神,这是使中国成为伟大国家的因素。\marginpar{\footnotesize 173}

\item[\textbf{主席:}] 这个对。

\item[\textbf{克伊腊克:}] 我个人为了祖国和人民,受过许多痛苦。因此很高兴看到中国发展起来。

\item[\textbf{主席:}] 你们正在发展,我们看到了也很高兴。

\item[\textbf{克伊腊克:}] 让我们两国手携手地向前迈进。可是最重要的是和平。你们打了那么多年仗,在时间、金钱和精力上造成了很大的损失,要是把这些时间、金钱和精力节省下来,中国的进步会更快一些。主席:可惜我们的敌人不给我们以时间,请他走他也不走。没有办法,只好打。英国人自己走了,日本人就是不走,到了一九四五年,他们实在没有办法才走的。蒋介石也是不想走的。那时北京城我们是不能进来的,只有美国人,蒋介石和他的军队才能进来。

后来他打败了,我们就来了。你们也就来了。(笑声)你们的国家承认我们,但是不承认蒋介石。你们过去同蒋介石建立过外交关系没有?

\item[\textbf{巴特:}] 没有。我们的政府支持在联合国中恢复中国的合法权利。

\item[\textbf{主席:}] 做得对。蒋介石本身是帝国主义的奴才,但是他却看不起你们,甚至也看不起尼赫鲁,有一段时间,同印度的关系搞得很紧张。蒋介石这个人我是比较熟悉的。(笑声)

今天,我们是不是就谈到这里?

\item[\textbf{潘廸:}] 好,对于您的接见,请允许再一次表示感谢。对我们每一个人来说,今天都是一个难忘的日子。

\item[\textbf{主席:}] 再见。请你们转达我对国王陛下的问候。
\end{list}
