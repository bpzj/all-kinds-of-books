\section[毛泽东关于共产国际解散问题(一九四三年五月二十六日)]{毛泽东关于共产国际解散问题}
\datesubtitle{(一九四三年五月二十六日)}


毛泽东同志于二十六日晚,在延安干部大会上的讲坛上出现时,全会场报以震动屋宇的掌声,该会系由中国共产党中央书记处召集,传达共产国际执行主席团及中共中央关于解散共产国际问题的两个历史文件。任弼时同志宣布开会理由及李富春同志朗读两大文件后,毛泽东同志以中共中央政治局主席之资格,向大会作报告。

毛泽东同志首先指出共产国际的解散,正如美国通讯社所报导的是一件“划时代的大事”。现于四天以来全世界各国不论是反法西斯阵营中的和法西斯侵略者阵营中的,不论何党何派,对此问题都给以极度注意,就可证明这一点。

毛泽东同志发问道:共产国际为什么要解散呢?难道它不是为全世界工人阶级谋解放和为反法西斯战争尽力的么?

毛泽东同志说:是的,共产国际是列宁手创的。在它存在的整个历史时期中,在帮助组成真正革命的工人政党上,在组织反法西斯战争的伟大事业上,有其极端巨大的功劳。毛泽东同志特别指出共产国际在帮助中国革命中的功劳,他说:“共产国际在中国人民中的影响,是很大的。其原因就在于中国虽然是经济落后的国家,却在二十三年中连续不断地进行了三个巨大的革命运动,而共产国际对于这三个革命运动都作了很大的帮助,这就是北伐战争、土地革命与抗日战争”。毛泽东同志说到北伐前夜,共产国际如何帮助了孙中山先生及其领导下的国民党,在一九二四年实行改组,并缔结了国共两党的合作;说到当时蒋介石先生曾衔中山先生之命到过莫斯科,后当国民党的代表曾列席过共产国际的会议等历史事实时,他说:“这就足以证明共产国际对中国革命的援助和在中国人民中的影响何等巨大,再不必说以后的土地革命和近年的抗日战争了”

毛泽东同志又指出:“革命运动是不能输出也不能输入的。虽然共产国际的帮助,中国共产党的产生及其发展,乃是由于中国本国有了觉悟的工人阶级,中国工人阶级自己创造了自己的党一中国共产党。中国共产党虽然还只有二十二年的历史,但却进行了三次伟大的革命运动。”

既然共产国际对中国以及各国有了很大的功劳,那么为什么要宣布取消呢?毛泽东同志答复道:马列主义的原则,革命的组织形式应该服从于革命斗争的需要。如果组织形式已经与斗争的需要不相适合时,则应取消这个组织形式。毛泽东同志指出,现在共产国际这个革命的组织形式,已经不适合斗争需要了。如果还继续保存这个组织形式,便反而会妨碍各国革命斗争的发展。现在需要的加强各国民族共产党.即无需这个国际的领导中心的必要了。其所以如此,主要的是由于以下三个理由:第一,因为各国内部与各国之间的情况,比之过去更为复杂,其变化亦更为迅速。统一的国际组织,无法适应这样非常复杂而且迅速变化的情况,正确的领导,要从仔细研究情况出发,这就更加要各国共产党自己来做。远离各国实际斗争的共产国际,在过去情况比较单纯,变化比较还不很迅速的时候是适合的,现在就不适合了。第二,法西斯强盗在法西斯集团与反法西斯集团各民族之间划了深刻的鸿沟,反法西斯国家中有社会主义的、资本主义的、殖民地的、半殖民地的各种类型的国家,法西斯及其附庸国中也有很大的差别;此外还有各种情况的中立国。为了迅速地有效地组织一切国家的反法西斯斗争,国际性的集中组织,早己感到不大适宜,这种情况,至近来乃特别显著。第三,各国共产党的领导干部已经成长起来,他们在政治上已经成熟,毛泽东同志以中国共产党的侧面来说明这一点,中国共产党经过三次革命运动,这些革命运动是连续不断的,是非常复杂的,甚至比之俄国革命还更复杂。在这些革命运动中,中国共产党已经有了自己的身经百炼的优秀干部。自一九三七年共产国际第七次世界大会以来,共产国际即没有干涉过中国共产党的内部问题,而中国共产党在整个抗日民族的解放战争中的工作是做得很好的。

毛泽东同志总合四天来各国对于解散共产国际的舆论称,反法西斯同盟各国的一切正义人士,对此举动都交口称誉。但法西斯国家却不同,一切血腥的侵略者,过去曾经订立过“反对共产国际”协定的,现在却似乎不愿意共产国际的解散,你们看奇怪不奇怪呢,他们都异常狼狈地拚命指责共产国际的解散。在同盟国中,例如斯托哥尔姆与伦敦两地社会民主党中的顽固派分子,从前以“受共产国际指令”为理由,拒绝该国共产党加入该党,现在却又不欢迎共产国际的解散,也算一件小小的怪事。

现在全世界一切反法西斯国家的任务,在于使工人运动归于统一,以便有利地迅速地打败法西斯,此种工人运动的顽固派,因为共产国际的解散失去了他们的借口,他们就很不高兴,甚至说各国共产党也应该解散。毛泽东同志接着指出:中国也许会有这一类毫无常识的议论出现,我们且看一看罢。但是我相信:全中国大多数正义人士是不会附和这种议论的,其理由,就是因为这种议论,缺乏任何起码的常识。毛泽东同志说:共产国际的解散,不是为了减弱各国共产党,而是为了加强各国共产党,使各国共产党更加民族化,更加适应于反法西斯战争的需要。党近年的整风运动,反对主观主义、宗派主义和党八股这些不好的东西,就正是为了使中国共产党更加民族化,更加适合抗战建国的需要。

至此,毛泽东同志以极其庄重的语调指出:现在共产国际没有了,这就增加了我们的责任心,每个同志都要懂得自己负担了极大的责任。从这种责任心出发,就要发挥共产党人的创造力。我们正处在艰难的民族解放战争中,八路军新四军在敌人后方抗拒着极其强大的敌人,我们的环境很艰苦,战争的时间还很长。但是这种长期的艰苦的斗争,正好锻炼我们自己,使我们用心的想一想,绝不粗枝大叶,自以为是,使我们认真去掉主观主义、宗派主义以及老一套的党八股作风,而拿出完全的负责的态度与高度的创造力来。


毛泽东同志极其强调地指出下列两种团结的绝对必要:一种是党内的团结,一种是党与人民的团结,这些就是战胜艰难环境的无价之宝,全党同志必须珍爱这两个无价之宝。第一,全党同志必须团结在中央周围,任何破坏团结的行动都是罪恶,只要共产党人团结一致,同心同德,任何强大的敌人,任何困难的环境都会向我们投降的。第二,全党同志都要善于团结人民群众,这里我想要请同志们学习近日解放日报报导的陈宗尧、左齐两个同志的榜样。陈宗尧同志是八路军的团长,他率领全团走了几百里路去背米,他不骑马,自己背米,马也背,全团的指战员为他的精神所感动,人人精神百倍无一个开小差。左齐同志是该团政治委员,他在战争中失去了一只手,开荒时他拿不起锄头,但在营里替战士们做饭,挑上山去给战士们吃,使得战士们感动到不可名状。毛泽东同志号召全体党的干部,学习这两位同志的精神,和广大群众打成一片,克服一切脱离群众的官僚主义。毛泽东同志说:我们共产党人不是要做官,而是要革命,我们人人要有彻底革命精神,我们不要有一时一刻脱离群众。只要我们不脱离群众,我们就一定会胜利。

(一九四三年五月二十八日《解放日报》)

