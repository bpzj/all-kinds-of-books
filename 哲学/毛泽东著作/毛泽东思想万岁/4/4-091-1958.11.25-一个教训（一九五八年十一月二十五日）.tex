\section[一个教训(一九五八年十一月二十五日)]{一个教训}
\datesubtitle{(一九五八年十一月二十五日)}


这是一个有益的报告,是云南省委写的,见“宣教动态”145期。云南省委犯了一个错误,如他们在报告中所说的那样,没有及时觉察一部分地方发生肿病的问题。报告对问题作了恰当的分析,处理也是正确的。云南工作可能因为肿病这件事,取得教训,得到免疫力,他们再也不犯同类错误了。坏事变好事、祸兮福所倚。别的省份,则可能有的地方要犯像云南那样的错误。因为他们还没有犯过云南所犯的那样一种错误,没有取得深刻的教训,没有取得免疫力,因而,他们如果不善于教育干部(主要是县级,云南这个错误就是主要出于县级干部)不善于分析情况,不善于及时用鼻子嗅出干部中群众中关于人民生活方面的不良空气的话,那他们就一定要犯别人犯过的同类错误。在我们对于人民生活这样一个重大问题缺少关心。注意不足,照顾不周,(这在现在时几乎普遍存在)的时候,不能专门责怪别人,同我们对于工作任务提得太重,密切有关。千钧重担压下去,是乡干部没有办法,只好硬着头皮干,少于一点就被叫作“右倾”,把人们的心思引到片面性上去了,顾了生产,忘了生活。解决办法:(一)任务不要提得太重,不要超过群众负担的可能性,要为群众留点余地;(二)生产、生活同时抓,两条腿走路,不要片面性。


