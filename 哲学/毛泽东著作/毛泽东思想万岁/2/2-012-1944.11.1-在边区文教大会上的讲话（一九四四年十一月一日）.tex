\section[在边区文教大会上的讲话(一九四四年十一月一日)]{在边区文教大会上的讲话}
\datesubtitle{(一九四四年十一月一日)}


毛主席首称:“我们的一切工作的总目标就是打倒日本帝国主义。日本帝国主义与希特勒一样是快要灭亡了,但是它还有力量,中国人民尤其中国解放区必需继续努力,才能达到最后消灭敌人的目的。这个努力首先是战争,其次是生活,然后便是文化,没有生产的军队是饥饿的军队,没有文化的军队是愚蠢的军队,而愚蠢的军队是不能战胜日寇,解放人民和建设现代化的新中国的,所以我们必须有文化。有一部分同志曾经轻视文教工作,这是错误的,在这次会议后,大家都应该对文教工作予以应有的重视。”

毛主席指出:“它应该是新民主主义的文化,也就是人民大众反日反汉奸反黑暗反封建统治的文化,这种文化的政治经济基础,就是民主政府,就是减租减息,就是以各种规模的工厂与各种形式的合作社为领导的个体经济。新民主主义的文化一方面是这种社会形态的反映,一方面又推动这个社会形态继续前进。”

毛主席说:中国的现代工业还很弱小,解放区的工业更是弱小,但它们是有无限前途的,中国必须以此为基础克服自己的落后。解放区的经济有其进步的方面与落后的方面,解放区的文化也有其进步的方面与落后的方面,解放区有作为领导方向的人民大众的新文化,但也有广大的落后的封建遗产,如陕甘宁边区就有一百多万文盲,两千个巫神,封建迷信的还在经过文化生活的各方面影响着边区的群众,反对群众的脑子里的这个敌人甚至比反对日本帝国主义还困难,边区文教会议的任务,无论教育艺术卫生报纸那一项,就都是告诉边区一百五十万人民自己起来和自己封建、迷信、文盲、不卫生等旧习惯作斗争,为了进行这个困难的斗争,就不能不有广泛的统一战线。如在教育方面,即不但要有比较集小比较正规的中小学,而且要有普遍分散比较不正规的村学读书识字组,不但要有新内容的民办学校,而且要利用和改造旧的村塾。在艺术方面,即不但要有话剧,而且要有秦腔、秧歌,不但要有新秦腔、新秧歌,而且要利用和改造旧戏班,特别是百分之九十的旧秧歌。在医药卫生方面尤其如此,陕甘宁边区现在婴儿死亡率高至百分之六十,成人死亡率高至千分之三,去年死牛七千八百头,死驴四千头,死羊二十一万只,死骡二千三百匹,人民相当普遍的相信巫神,在这种情况下仅仅依靠少数机关部队的西医是不可能的,为机关部队服务是很重要,西医此中医更科学,但西医如在这种情况下,不关心人民,不为边区人民训练更多的两医,不联合和帮助改造边区的一千个中医和旧式的兽医,就是实际上帮助巫神,帮助边区人畜的死亡,所以新形式与旧形势的统一战线是完全必要的。统一战线的两原则于此完全适用:第一是团结;第二是批评,或教育改造。投降旧形式是错误的,排斥、鄙弃也是错误的,我们的任务是联合一切可用的旧形式、旧人,而帮助、感化与改造他们。为了改造他们,就首先要团结他们,只要我们做的恰当,他们是会欢迎我们的帮助与改造的。

<p align="right">(1944年11月1日《解放日报》)</p>

