\section[给第一届全国卫生会议的题词(一九五〇年)]{给第一届全国卫生会议的题词(一九五〇年)}


团结新老中西各部分医药卫生工作人员,组成巩固的统一战线,为开展伟大的人民卫生工作而奋斗。

×××传达毛主席的指示

一、打破封锁,改变半殖民地经济状态

力戒急躁,计划要切实可行。

帝国主义对我们各种斗争直到封锁,目的在迫我就范。打破封锁之道,最重要的是军事上迅速解放全中国,外交上及时主动一面倒,内部则强调自力更生,从长远建设着眼来提出问题。中央同意××局对城乡问题意见,应迅速进行,决心把农村工作做好,并认真动员和作出调动干部下乡计划。另外,反封锁要在政治上抓理,绝不可仅由共产党作决定,而应与各界共同决定。最好通过各界代表会讨论决定。疏散难民应立即进行,疏散旧人员应慎重,给以生活之道,做到仁至义尽。工厂内迁不是从解决上海当前困难着眼,而应从新民主主义经济建设的长远计划来提出,作出计划,分别先后、公私,对私厂内迁,必须:(1)与各阶层共同讨论决定,大家负责。(2)必须自愿,不可强制,但应宣传动员。(3)自愿迁者予以真正帮助,不论公私工厂,均应求其内迁后有出路。(4)可迁而不愿迁者,倒闭了责不在我。学校可内迁一部,国民党中央各机构,也应适当安置各地.精简节约应即实施,无作战任务之部队可分驻农村,加强农村剿匪右群众工作,减少地方武装。对精简人员应注意安插,不可送走了事。节衣缩食以不病倒病人为原则。最后,应充分认识半殖民地经济不能希望一下改变,力戒急躁,一切计划,必须切实可行。

二、工作上政策上要争取主动,避免被动

对党内应从理论上反复教育,说明必须团结党外人士的重要意义,克服怕麻烦,不愿接近民主人士,对招待民主人士较为看不惯,认为:“早革命不如迟革命,迟革命不如不革命”(毛主席说,一定还会加一句:“不革命不如反革命”)的思想倾向,使全党了解团结争取党外人士。甚至把付作义、程潜、张治中之流争取过来为我所用,虽有小麻烦,但可以少打多少仗,少死多少人,对人民非常有利。要学会算大账,要善于批制地学习敌人的斗争策略,才能消灭敌人。今天我们力量强大,应有大气魄,一切能用的人都拿来用,虚心学习其长处,不怕他们造反。对大中学处理要非常慎重,不在人事上轻易更动,对资产阶级学者要慢慢加以改造。在学术上必须采取老实态度,知之为知之,不知为不知,不要不懂充懂,否则必然大失威信,陷于被动。对党外人士要主动谈“过三关”问题。第一关是封锁,大约三个月以上;第二关是土改,要三、五年左右;第三关是社会主义。资产阶级也知社会主义迟早要实现,但仍力争时间,总要求心里有个数才放心。

三、与党外人士合作的几种形势

(1)人民代表会议――大体上接收城市二、三个月,了解了情况就可召开,主动把党的意见变成大家共同决定,共同负责执行。

(2)共产党和各党派,各界人士的干部会议——约一月召开一次。除党内秘密外,一般政策问题,要争取主动普遍传达。开会时应由主要负责人出席,才能使人家重视。这种会可减少党外人士顾虑,并可以教育广大中间分子,及时给进步分子以斗争武器,孤立右派分子。

(3)军政委员会。

(4)顾问团。

四、改进工作办法与领导方法

领导机关对必要会见的人要见,必要的会议要开,但不可过多过长,陷于被动。要学会节约时间,集中力量解决重要问题,平时要充分利用通报、电报、社论来实现领导。对下面要定章程出“安民告示”(明确指出),经常给予任务,定期检查,传达经验,表扬优点,批评缺点,使领导经常处于主动。

五、加强国际主义教育

向党内党外反复说明一面倒的真理。对苏联朋友表示热情、关心、友好的态度;苏侨来往采取护送制度;主动向其解释不同习惯;对缺点也可经相关负责人正面讲清。帝国主义国家的侨民,愿走者尽量帮他们走,多走无害,但方式要讲究。


