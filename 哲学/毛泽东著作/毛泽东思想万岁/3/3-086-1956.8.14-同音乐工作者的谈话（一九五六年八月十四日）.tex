\section[同音乐工作者的谈话(一九五六年八月十四日)]{同音乐工作者的谈话}
\datesubtitle{(一九五六年八月十四日)}


世界各民族的艺术,在基本原理方面是相同的,在形式和风格方面又是互有区别的。社会主义各国的艺术都以社会主义为内容,而又各有自己的民族特色。有同,有异,有共性,有个性,这是自然的法则。一切事物,不论自然界、社会界、思想界,都是如此。好比一棵树上的叶子,看上去大体相同,仔细一看,每片叶子都不一样,每一棵树要找出完全相同的两片叶子是不可能的。

阶级斗争、社会革命、由资本主义向共产主义过渡,在基本原理方面,各国都相同,而在基本原理指导下的一些小的原则和表现形式,各国又有不同。十月革命和中国革命就是这样。在基本原理方面,两个革命是相同的,在表现形式上,两个革命却有许多不同。例如革命的发展,在俄国是由城市到乡村,在我国是由乡村到城市,就是两个革命的许多区别之一。

世界各民族的艺术,都有自己独特的民族形式和民族风格。有些人不了解这一点,他们否认自己的民族特点,盲目崇拜西方,以为一切都是西方的好,甚至主张“全盘西化”。这是错误的。“全盘西化”是行不通的,是中国老百姓所不能接受的。艺术和自然科学不同。例如割阑尾,吃阿司匹林,这些医疗方法,就没有什么民族形式。但是艺术却不同,艺术就有民族形式问题。这是因为艺术是人民的生活、思想、感情的表现。同民族的习惯和语言有密切的关系。它的发展具有民族范围的继承性。

中国的艺术、中国的音乐、绘画、戏剧、歌舞、文学,有自己的发展历史。那些主张“全盘西化”的人,为了否认中国的东西,就说中国的东西没有自己的规律。这种说法对不对呢?不对,中国的音乐、绘画、戏剧、歌舞、文学,都有自己的规律。没有自己的规律,就不会形成中国老百姓所喜闻乐见的中国艺术的民族形式和民族风格。抱有这种错误思想的人,只是没有去研究中国艺术的规律,不愿意去研究和发展中国的东西。这是对于中国艺术的一种民族虚无主义的态度。

世界上各个民族都有自己的历史,都有自己的长处和短处。历史上的东西,有精华,有糟粕,混杂在一起,积累的时间又很长,要把它整理出来。分清精华和糟粕,是一个困难的任务。但是,不能因为困难就不要历史。把历史割断,把遗产抛弃,是不行的,老百姓不会赞成的。

当然,这决不是说,我们不希望向外国学习。外国的很多东西,我们都要学习,而且要学好。基本理论尤其要学好。有些人主张什么“中学为体,西学为用”这种主张对不对呢?不对。所谓“学”,就是基本理论,基本理论是中外一致的,不应该分中西。

马克思主义这种基本理论就是在西方产生的,这难道能分中西?我们难道能够不接受?中国革命的实践证明,不接受马克思主义,对自己是不利的,也没有不接受的道理。过去第二国际曾经企图否定和修正马克思主义的基本理论,讲了一些否定、修正的道理,都被列宁完全驳倒了。马克思主义是放之四海而皆准的普遍真理,我们必须接受。但是这个普遍真理又必须同各国革命的具体实践相结合。中国人民正是接受了马克思主义,并且把它同中国革命实践相结合,这才取得了中国革命的胜利。

学习外国的东西,是为了研究和发展中国的东西。就这一点说来,自然科学同社会科学是一样的。一切外国的好东西,我们都要学,学好了就要在运用中加以发展。在自然科学方面,我们也要作独创性的努力,并且要用近代外国的科学知识和科学方法来整理中国的科学遗产,直到形成中国自己的学派。例如,西方的医学和共有关的近代科学:生理学、病理学、生物化学、细菌学、解剖学,你说,不要学?这些近代科学都要学。但是,学了西医的人,其中一部分又要学中医,以便运用近代科学的知识和方法来整理和研究我国旧有的中医和中药,以便把中医中药的知识和西医西药的知识相结合起来,创造中国统一的新医学、新药学。社会科学和自然科学是这样,艺术当然更是这样。要向外国学习。吸收外国一切好的东西。但是,学了外国的东西,要用来研究和发展中国各民族的艺术,否则就没有研究和发展的对象了。我们学习外国的艺术,学习他的基本原理和基本技巧,其目的就是为了创造中国各民族自己的具有独特的民族形式和民族风格的社会主义新艺术。

要承认,在近代文化上,西方的水平比我们高,我们是落后了。艺术方面是不是这样的呢?在艺术上,我们有长处,也有短处。必须善于吸收外国的好东西,以收取长补短之效。故步自封,外国的文学不研究,不介绍,列国的音乐不会听,不会演奏,是不好的。不要像慈禧太后那样,盲目排外。盲目排外同盲目崇外一样,都是错误的,有害的。

在学习外国的问题上,既要反对保守主义,又要反对教条主义。在政治上,我们吃过教条主义的亏。什么都是照抄外国,照搬外国,结果是一个大失败,使白区党组织损失百分之百,使革命根据地和红军损失百分之九十,把中国革命的胜利推迟了许多年。其原因就是有些同志,不从实际出发,而从教条出发,没有把马克思列宁主义的基本原理同中国革命的具体实践相结合。这种教条主义,假使我们不反掉,就没有今天中国革命的胜利。

在艺术方面,我们也应当吸取这个教训,注意不要吃教条主义的亏。学外国的东西,不等于统统进口,硬搬外国的一套。要批判的吸收。向古人学习是为了今天,向外国人学习是为了中国人。

外国的好东西要学到,中国的好东西也要学到。半瓶醋是不行的,要使两个半瓶醋变成两个一瓶酣。中国的东西和外国的东西,两边都要学好,两边都要有机的结合起来。鲁迅就是这样,他对于外国的东西和中国的东西,两边都很熟悉。但是他的光彩,首先不在于他的翻译,而在于他的创作。他的创作既不同于外国的,也不同于中国古式的,但是他是中国的。我们应当学习鲁迅的精神,精通中外,吸收中外艺术的长处,加以溶化,创造出新的具有独特的民族形式和民族风格的艺术。

当然,要把中国的东西和外国的东西很好地结合起来,是不容易的。这要有一个过程。中国的东西里面也可以掺杂一些外国的东西。例如写小说,语言、人物、环境,必须是中国的,但是不一定是章回体。不中不西的东西也可以搞一点,非驴非马,成了骡子也并不坏。两者结合是要改变形象的,完全不改变是不可能的。中国的东西要变。无论在政治上、经济上、文化上,中国的面貌都正在大起变化。但是无论怎样变,中国的东西还是要有自己的特点。外国的东西也在变。十月革命以后,世界的面貌就发生了根本的变化。第二次世界大战以后,这种变化又有新的发展。我们要注意批判地吸收外国的东西,特别注意吸收社会主义世界的东西和资本主义世界的进步的人民的东西。

总之,艺术要有独创性,要有鲜明的时代特点和民族特点。中国的艺术,既不能越搞越复古,也不能越搞越洋化,应当越搞越带有自己的时代特点和民族特点,在这方面不要怕“标新立异”。特别是像中国这样的国家,历史悠久、人口众多,更必须有适合中国各民族需要的“标新立异”。这种为中国各民族老百姓所欢迎的“标新立异”,越多越好,不要雷同,雷同就成为八股了。土八股也好,洋八股也好,都是没有生命力的东西,都是中国老百姓所不欢迎的。

这里有一个对待受过西方教育的资产阶级知识分子的问题。这个问题,如果处理不好,不但对艺术事业不利,对整个革命事业也不利。中国民族资产阶级及其知识分子,有几百万人。他们人数不多,但是有近代文化,我们一定要团结、教育和改造他们。买办阶级有文化,那是奴隶文化。地主阶级有文化,那是封建文化。中国的工人和农民,由于长期受压迫,文化和知识还不多。比较起来,在没有完成文化革命和技术革命以前,资产阶级知识分子在近代文化技术方面有较高的知识和技能。只要我们政策正确,把他们教育和改造过来,就可以使他们为社会主义事业服务。能不能够把他们教育和改造过来呢?能够的。我们在座的许多人过去都是资产阶级知识分子,都从资产阶级那边转到无产阶级方面来了,为什么他们就不能转过来呢?事实上,已经有许多人转过来了。所以,一定要团结他们,把他们教育和改造过来。只有这样做,才有利于工人阶级的革命事业,才有利于今天的社会主义革命和社会主义建设。

在座的都是音乐家,学西洋音乐的,你们有很重要的责任。整理和发展中国的音乐,要靠你们学西洋音乐的人,好比整理和发展中医要靠西医一样。你们学的西洋东西是有用的,只是你们应当把西洋的东西和中国的东西,两边都学好,而不要“全盘西化”。你们要重视中国的东西,要努力研究和发展中国的东西,要以创造中国自己有独特民族形式和民族风格的东西为努力目标。你们掌握了这样一个基本方向,你们的工作就是前途远大的了。

<p align="center">×××</p>

我们当然提倡民族音乐,作为中国人不提倡民族音乐是不行的。但是军乐队千万不能拿唢呐、胡琴去吹奏,还是管弦乐队。我们要提倡民族音乐,但西洋的管弦乐还是应当要,因为世界上都作兴这种东西。这等于我们穿军服,还是穿现在这种样式的,总不能把那种胸前和背后写着“勇”字的褂子穿起。民族化也不能那样化,当然我们的军服同苏联的也不完全一样,有中国军队色彩。


