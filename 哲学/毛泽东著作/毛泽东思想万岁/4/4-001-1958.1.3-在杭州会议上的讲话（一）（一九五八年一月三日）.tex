\section[在杭州会议上的讲话(一)(一九五八年一月三日)]{在杭州会议上的讲话(一)}
\datesubtitle{(一九五八年一月三日)}


右派是反对派,中右也反对我们,中中是怀疑的。基本群众和资产阶级、资产阶级知识分子中的左派是赞成我们的。

关于对待资产阶级的问题,好多国家怀疑中国是右了,好像不像十月革命。因为我们不是把资本家革掉,而是把资本家化掉。其实,最后把资产阶级(化掉),如何可以说右呢?仍是十月革命。如果都照十月革命后苏联的做法,布疋没有,粮食没有,(没有布疋,就不能换得粮食)、煤矿、电力各方面都没有了。他们缺乏经验。我们根据地搞的时候多了,对官僚资本(生产秩序)来个原封不动,对民族资本更是为此。但是不动中有动。全国资本家七十万户,资产阶级知识分子几百万,没有他们就不能够办报、搞科学、开工厂。有人说“右”了。就是要“右”,慢慢化掉,正确处理人民内部矛盾,就是这个路线贯彻下来的。有的是一半敌人,一半朋友,有的三分之一或多一些是敌人。

治淮十二亿人民币搞了七年,治淮的数量,如果打七折(有些有质量上的问题),也是合算的。原来计划低了,后来超过了,批评右倾保守。就很舒服。愈批评愈高兴。甘肃一千多万人口,劲头很厉害,值得去学习。

要抓十二条,今后要评比:1.水,2.肥,3.土,4.种(优良品种),5.改制——如复种、晚改早,旱改水,6.除病虫害(浙江因虫损失一亿斤,一年夏把虫灭得差不多了,日本无马克思主义。已搞得无虫了,我们有马克思主义),7.机械化(新式农具、双轮双铧犁、抽水机等),8.畜牧,9.副业,10.绿化,11.除四害,12.治疾病讲卫生。

还要抓另一个十二条,也要评比,1.工业,2.手工业,3.农业,4.副业,5.林业,6.渔业,7.牧业,8.交通运输业,9.商业,10.科学,11.文教,12.卫生。

四十条到第二个五年计划第三、四、五年就要修改,愉快地批判右倾。一九五六年工业产值增百分之三十一,没有一九五六年的突飞猛进,就不能完成五年计划。今年三月比一次,夏季比一次,到十月开党代会再比一次。省与省比,县比县比,社与社比。如果大家同意再商量着办。大家都要到别的省参观参观,自己不出门,比输了就活该。很值得到甘肃去一次。

要全面规划,几次检查,年终评比,开几次会,开小型的会,抓地、县书记。地委书记的会,两个月开一次,每次不超过五天,县委到地委去开会(几个地委一起开,比较有兴趣)。大型的“非常会议”,一年只能开一两次,要两个月抓一抓,否则一年很快过去了。各省评比,在中央开会。

我自己每月同大家谈四次,到处跑一跑,每次两天到三天,看五、六个单位。

工业也要四十条,科学文教也要有。先有个别人准备意见,大家再七凑八凑才成。

全国搞几个经济协作的区域,有些省可以交叉。要认庙不认神。要有这么一个传统,有庙不论谁在那里挂帅,就由他做头,能凑合过去就行了。沙文汉,杨思一等是另外一回事。

湖北省委有个关于领导干部亲自搞试验田的报告,中央批转了,看到没有,很重要,要普遍搞试验。

关于积累与消费问题,究竟积累问题多大,有的提百分之四十五,有的百分之五十,有的百分之五十五。最好一半一半,要看年成,看地方,做几种规定。百分之六十分掉不是一般标准,是减产的情况。要勤俭持家,个人的消费要节约,红白喜事,大吃大喝要反对。各省要出个布告禁赌。婚丧喜庆,红白喜事,都要从简,家庭造酒,完全禁止也不好,爆竹不要禁了,有振奋精神的作用。

政治业务要结合,也就是红与专的问题。政治叫红(我们为红,在美国为白)。红与专是对立的统一。两者不同,有区别。一个是搞精神的,一个是搞物质的。有些业务部门的负责同志,说话时口边政治很少,可见平素不大谈政治。忙得很,一谈就是业务。各省管业务的恐怕更厉害些。一定要批评不问政治的倾向,同时要反对空头政治家。要懂得点业务才好,否则名红实不红。不懂农业,要指导农业就不行。搞试验田,红与专的问题就解决了。一种是不懂业务,空头政治家,一种是迷失方向的经济家或其他技术家,都是不好的。要去分析分析。但是批评人家的时候,先要检查自己,自己也有点空,不甚了了嘛。总理去年就钻了一下工资福利问题。我在北京看了工业展览会,一次看一个馆不够,还要多看看。

整风要贯彻到底,不要半途而废。上海说的第三类人,就会作官,要打掉官风。上海提要有干劲,很好。《浙江日报》社论《是促进派,还是促退派》,《人民日报》要转载。整风中要反浪费,时间不要太长,几天就行了。结合整改,专题鸣放一次,鸣放了,大家警惕。每个家庭成员都要进行教育,要勤俭持家。

什么时候交计划,省地县社都要搞,先搞粗线条的省,要半年交卷。

