\section[日本进犯中国的近因及其前途(一九三七年)]{日本进犯中国的近因及其前途}
\datesubtitle{(一九三七年)}


一、我们对敌人的面孔应该认识得更清楚些

六十年来,特别是最近六年来,中国从日寇得来的教训,本来已经够多了。但是一直到现在,还有许多人对他本身及其行动的观察,常常认识不足,甚至犯严重的错误。就拿最近日寇大规模进犯中国这一严重事件来说吧,对于这一事件,就有许多离奇怪状的看法,有些人认为这次日寇向华北以至向上海开火,完全是日本少数少壮派军人的行动,而不是日本帝国主义整个的国策(他们认为连日本金融资本家以至近卫内阁也不同意);有些入认为这次事件的暴发,是出于“偶然”的,而不是日寇预先有布置有计划的行动(他们的证据是,在日本未发动进攻中国之前,东京金融市场并没有紊乱。等到日本在华发动进攻以后,公债及各种商业股票才开始惨跌);甚至还有些人认为日寇这次行动只是“示威”性质,最少在主观上并“不愿意把事件扩大”等等。

不消说,这些看法,都是极端错误的。第一种看法的错误在于把日本的“少壮派军人”看得太神化,而没有把日本帝国主义和近卫内阁的本来面目认识清楚。固然,在华北发动战争和指挥作战的是日本的“少壮派军人”;但是,这些少壮派军人谁派他们来的呢?他们的行动代表谁的利益呢?稍微注意一点时事的人都知道,日本在中国特别是在华北的驻屯军干部,大部分部是在“二.二六”事件后,或在去年华北驻屯军强化计划实施后由东京调来,现在华北驻屯军的首脑,被称为荒木派的香月清中即则是在芦沟桥事件发生后,才由东京方面派来的;经过“肃军”后的东京当局,例如预先没有“放火”的计划,为什么要派这些“点火者”到最容易发火的华北来呢?尽管这些点火者在点火和扩大火焰的瞬间不一定要得到东京方面的同意,尽管他们在某些部分的问题上甚至会和他们的主人一一金融资本家地主等有不同意见,但是他们的行动无疑是忠实于他们主人的利益的,老实说,日本帝国主义已把“征服支那全土”的计划布置好了,驻华北的少壮派军人不过忠实地活泼地按着这计划去做罢了。如果有人认为日本的金融资本家大地主以至近卫内阁不同意华北驻屯军人这次在华北发动战争的行动,那是完全不合事实的谬误之谈。要知道东京的金融资本家大地主等,希望无事平稳地继续五年来资本收益日益增大的过程固是事实;但是到了除非用战争解除他们的危机和延长他们的生命的时候,他们也决不会同意发动战争的!只要看在芦沟桥事件发生后,东京各政党对“近卫政策”(其实就是军部的政策)完全表示支持,就可以看到政策背后的金融资本家和大地主的态度如何了。至于说,近卫内阁也不同意这次战争,那就更加荒谬,我们早已指出,近卫内阁的出现,主要就是为了调和日本统治阶级内部的矛盾,为了实现对中国的侵略战争。

第二、第三两种观察错误的根源是和第一种相同的,不过其意义来得更加严重;由于没有看得清日本统治阶级的矛盾的一致性,由于没有把握住日本帝国主义向中国发动大规模战争的必然性,而只凭一些极表面的事实或不可靠的材料不加思索地乱下断语,那只自然会在有意无意间中了敌人宣传的毒计了!但我们必须指出,敌人天天宣传这次日中“小冲突”(!)是出于“偶然”,日本政府“不愿意把事件扩大”……完全是具有烟幕作用的!它所以要散布这些烟幕弹,一方面是为要向其它帝国主义掩饰它独占中国的既定计划,使得这种计划能畅行无阻;另一方面则是为要动摇中国不坚定分子,使这些分子继续保留一线苟安的幻想!对于这些烟幕,中国的舆论界早就应供给以揭破;可惜中国的舆论界,当华北战事发生的时候,往往只注意到华北的问题,当上海战争发生的时候,又往往只注意到上海的问题,而对于敌人对中国进攻的整个计划,却不但很少有正确的把握,并且有些论调还陷于严重的错误!反问这次战争真的是出于“偶然”吗?日本政府真的“不想把事件扩大吗”?在芦沟桥事变发生之前,日本政府对言论界彻底的统制,关西几个师团到了除队时间而不下令解散等等,为的是什么啊?芦沟桥事变发生后,日本政府即下总动员令,在短时间内即动员近三十万大军(包括第一、五、六、七、八、九、十、十二、十五、十六、十九、二十等师团)到中国来,又为的是什么啊?我们根据事实,不能不强调指出:日本帝国主义者这次向中国实行大规模的进攻,不但是有计划的行动,而这种计划的严密性及其广大性是以往所没有!不健忘的读者总应记得,当日本军阀在一九三一年发动“九一八”事变的时候,日本统治阶级间的意见是非常分歧的。当时的日本原外相居然敢向日本军阀诽议“日本吞下满洲,等于吞下一颗炸弹”(大意如此)。但是这一回却不同了,根据东京的通讯:芦沟桥事变的消息传到东京后,各报的专电,各杂志的书论,各政党负责人的发言……。差不多是“句句一样,众口一词”的,由此可以看到日本政府对这次在事前事后,都有严密的布置,其严密性甚至连日本的“交易所”有可能在发动战争之前也不知情(代表资本家和地主的政府,当它执行对主人有利的计划,特别是军事计划时,不一定要在事前完全得到它的主人们的同一意,)至于说到日本帝国主义者这次侵略计划的广大性,我们可以断定,是“九一八”以来所没有的:现在已经开始了大规模的侵略战和全国性的抗战了。

二、日本为什么要在目前发动大规模的侵略战

日本帝国主义者自从完成了田中奏折里所明白记载的“征服满洲”的计划后,天天在准备着实现“征服支那全土”的计划;但是日本帝国主义者所以要选择在目前下决心执行这个计划,当然是由种种客观的原因所促成的。

首先,我们从日本经济方面观察。日本新的经济恐慌的征兆已经非常明白地表现出来了,不管日本的军需工业有怎样畸形的发展;但是由于军费无穷尽和无限制的膨胀,致使赤字公债无法消除,物价的高涨无法抑止;同时由于军需原料须从外国大量购入,致又引起对外贸易的空前入超;跟着国际收支的无法平衡,大量的黄金也不能不往外送了,据日本官厅发表的统计:日本自去年四月到今年一月间,物价总指数高涨了百分之三十一(国内商品高涨了百分之十三,输入商品高涨了百分之五十六),几个月来物价高涨之势,还是有加无已;至于入超和黄金外流之数,尤为惊人,本年上半年的入超达六亿四千万元,截至目前为止输往美国的现金共分四次,总额达二亿七千万元。照目前情形看来,日本经济的恶化,只有日益深刻,而丝毫没有“好转”的倾向,如此发展下去,新的恐慌时来临,只是时间的问题――这是无论是怎样蹩脚的经济学家都可以看得出来。惯于向中国抢劫以“打开”危机局面的日本帝国主义者,要在这样的时候,发动大规模的侵略战,不是“偶然”的了。

第二,我们从日本内政方面观察。随着经济危机的加深,日本国内阶级与阶级统治阶级间的矛盾也日益尖锐起来,这也是应有的文章了!据日本官方发表,本年一月到五月发生的罢工事件达一千三百三十二次,此去年七百次几增加了一倍;农民斗争则仅一月至三月已达一千六百五十六次之多了。由这些简单的数字,可以看到日本国民生活的不安,和日本工农大众在世界无此的残酷的警察制度压迫下面,如何英勇地要求生存了。

至于日本统治阶级间的矛盾,也始终不易彻底缓和,虽则日本统治者在面对着各种危机之前,极力设法使军部和党政间的矛盾和缓下去,然而以实现“军资和作”为任务的林内阁的“不幸短命而死”(只有一百十七日的寿命),表示调合前途的不可乐观。近卫文磨的登台,虽则也以“缓和国内之彼此对立,以收举国一致之实”为己任;但是近卫登台以后,不但“军资合作”既得的成绩,无法维持(结城不愿再当财政部长),不但已成的许多对立,无法消除,就是在内阁之内,马场广田贺屋之间也未见融洽。光阴易逝,特别议会召集之期已经不远,可是“新党”的组织,毫无把握,这不能不使近卫内阁感到前途的黯淡;经济危机与政治危机是连结在一起的,日本统治者在这样的危机的威胁下,当然要发狂起来,拚命向中国狂噬了。

第三,我们从军事方面观察。侵略主义的日本为要补救军需工业资源的缺乏,为要完成建立进攻苏联和对抗英美的有利的军事根据地,在获得了和相当巩固了满洲热河等“生命线”之后,进一步夺取华北、华中以至华南,原是必走的路径和既定的方针。现在眼看着苏联第三个五年计划行将开始。中国抵抗的实力显著地增加,同时世界和平势力和侵略势力最后搏斗的时期也日益迫近,在这样的种种刺激的下面,不能不使这位“争先天不足,后天发育不全,而又偏要硬撑好汉”的日本帝国主义,对它日夜企图要实现的既定方针,加速地“断然”执行起来!

第四,我们从日本国际环境来观察。在这一方面,从日本统治者看来,目前执行它的“既定方针”是相当有利的。他们认为欧洲的国家正忙于应付他们的盟友希特勒墨索里尼在西班牙的“硬干”。当然没有余力干涉远东的事情,他们特别高兴英国“友谊的态度”实在的,英国绅士这种态度对于日本发动大规模进攻中国,给予不少的鼓励。

苏联消息报早已指出,英日谈判是促进日本发动战争的主要原因之一;我在《中国抗战与英国》一文中,也已说明,英国在远东一贯的动摇不定的老政策对于帮助日本进攻中国是有着莫大的作用。

此外,美国爱玩的“中立”政策,和日本军阀对苏联的厉行清党肃军错误的估计,都使日本军阀误认是有利的条件。

最后,最近中国内政的动向,对于日本的发动大规模侵略战争又有怎样的关系呢?我们可以这样断定:假如中国内部的团结早已十分巩固,民主政治早已实现,那么,日本对中国的进攻是踌躇一下的;然而不幸得很,中国自西安事变后,和平统一虽已得到初步的成功,但是中央政治机构的彻底改革,国防政府的建立,亲日势力的肃清,民主政治的实现,还相去很远,因此日本帝国主义便乘着这些空隙,赶快向中国开炮了。

以上的几种因素当然是互相联系的,日本帝国主义就是在这些因素的交互促进和影响下,便决定向中国找寻出路。

三,侵略中国是不是日本的出路

在六年前,日本军阀曾在国内危机非常深刻,国际环境却相当顺利的条件下,干过一回,结果可以说相当得手;现在,它又要干第二回了。这一回也是不是像前一回那样的顺利呢?要解答这一问题,也应该从几方面来观察。

第一,日本统治者以为进攻中国可以“打开”经济的危局,我以为这完全是一种错算。在战争还没有发动以前,日本的经济危机已经如此深刻,财政方面已经如此没有办法,在战争发动以后,日本每年将需要多少战费呢?据专家估计,起码要一百五十亿到两百亿元,这是倾日本全国民的收入也不足此数三分之一或四分之一。战争除了引起财政的破产外,、将使对外贸易更加恶化,物价更加高涨,那也是意想中的事情。据最近东京通讯:“华北事变发生后,金融市场已经被搅得一塌糊涂,一般粮商更乘机操纵粮价,囤积居奇,致使粮食价格猛增不已”(申报)。将来战争愈持久,财政和经济的破产必然更加显著,所谓“未见其利,先见其害”,将是日本这次企图以战争来打开经济危机的必然归趋!

第二,日本统治者企图以进攻中国来缓和日本国内的政治危机,结果也必然适得其反,不能不使日寇企图在中国造成“既成事实”,而逼使英国承认。

日寇以其素来狡猾的手段,一方面实际上在侵略行动上,对中国进行大规模侵略战争;另一方面在表面上,在外交辞令上,则散布“不扩大事态”“地方解决”。日寇这种狡猾的手段,一方面用以掩盖他的为世界人类所攻击的侵略战争,以便在这种外交辞令的掩盖之下,进行“不宣而战”的对中国大规模的侵略战争,借以减弱对国际舆论和英美的刺激。另一方面则用这种外交辞令,以减弱中国的自卫行动,以勾引某些人对日寇的幻想,企图作和平的幻想以牺牲中国;同时以便给某些亲日派的分子,还在南京政府中保持“有用的”地位,因此这些亲日派久为全国人民所痛恨唾弃,自国内和平统一开始以来,他们逐渐的在南京政府内失去地位。

芦沟桥抗战的失败,平津的陷落,还是由于日寇的狡猾的手段所达到的。宋哲元氏上了日寇的老当,受了日寇的欺骗,由于宋哲元反对日寇和平解决的幻想,企图苟且偷安,临事犹豫动摇,不去积极地进行自卫战争,反而与日寇进行妥协谈判。因此把中国的政治经济军事文化上的重镇轻易的失去了。当然平津的陷落也有其他原因,如过去南京政府的退让政策,使日寇在平津造成优势的地位。但这早已成为错误,如果在芦沟桥事变时,釆取坚决抗战的方针,积极进行自卫战争,是能够补救的。可是冀察当局宋哲元氏没有这样做,而南京政府在芦沟桥虽有×××坚决抗战的方针,但对于冀察当局与日寇的妥协并没有表示反对,相反的却是容忍了这种妥协,在八月十四日南京政府发表的声明中说:“中国地方当局,为维持和平计,业已接受日本方面所提议之解决办法,中央政府亦以最大的容忍,对于此次之解决办法,亦未予反对”。在芦沟桥事变开始的时候,对于平津之增援,亦未采取积极的自卫战争。平津失守的教训,已经是再一次的说明,如果对日寇还保存“和平解决”的幻想,是要上日寇的老当,是要受日寇的欺骗,而使中华民族遭受莫大的损失。

芦沟桥事变以后,全国人民一致要求抗战,全国舆论全国空气都是为坚决抗战所压倒。×××七月十七日所发表的代表南京政府抗战的方针,及南京政府在军事上的措施是值得全国人民所赞许的。在平津、上海、南口的抗战中,中国军队英勇地抵抗是为全国人民所致敬的,以芦沟桥事变为起点,已经开始了全国性的抗战。

摆在中国人民面前的神圣的任务,就是一切为着争取抗日战争的胜利。中华民族的生路,只有进行全民族的神圣的自卫战争,去抵抗日本帝国主义的侵略战争,去消灭日本帝国主义的侵略战争。

现在已经到了“最后的关头”,已经由准备抗战进入了实行抗战的新阶段,中共中央在八月十五日的决定中指出了;“中国的政治形势从此开始了一个新的阶段,这就是实行抗战的阶段,抗战准备的阶段已经过去了。在这一新阶段内的最中心的任务,是动员一切力量争取抗战的胜利”。

现在已经到了须要进行全国的总动员,须要进行全面的全民族抗战,去保障抗日战争的胜利。今天的争论已经不是应否抗战的问题,而是如何争取抗日战争胜利的问题。中共中央同时指出:“在这一阶段内我们同国民党及其他抗日派别的区别和争论,已经不是应否抗战的问题,而是如何争取抗战胜利的问题。”“今天争取抗战的中心关键,是在使国民党发动的抗战发展为全面的全民族的抗战。只有这种全面的全民族的抗战,才能使抗战得到最后胜利”。

中共中央为着保障抗日战争最后的胜利,提出了争取抗战胜利的十大纲领:

(略。见《毛泽东思想万岁》第一集第二二五~二二六页)

上面十大纲领是保障中国抗日战争最后胜利唯一正确的方针,我们号召全国人民中国政府为着争取抗日战争最后胜利,只有坚决实行这十大纲领。

国民党和南京政府到今天为止,还是把抗战限制于单纯的政府抗战,到处限制人民和拒绝人民的参加抗战,阻碍政府军队和人民结合起来,不给人民以抗日救国的民主权利,不肯彻底改革政府的政治机构,这种抗战的方针,是包含极大的危险性,虽然可能取得暂时的局部的某些胜利,但决不能保障抗战的最后胜利。南口失利的经验就可证明这一点。某些前线作战的部队,不但不发动人民参加抗战,反而压制人民的救国运动,解散在前绝的人民的救国团体。

上海抗战中也同样表现这种现象,上海和南京的救亡团体遭受当局禁止活动,文化界出版的救亡刊物亦遭受查封。当八十七、八十八师在上海英勇抗战的时候,上海人民自动的参加和慰劳,都被上海某些当局所拒绝。在某些人操纵下的上海抗敌后援会,没有积极地援助八十七、八十八师的抗战,结果使两师在前线一天一夜没有饭吃没有水喝,而抗敌后援会却有堆积如山的慰劳品,让其腐烂而不送给前线的部队。因此前线抗战的部队埋怨的说这次上海的人民不如上次“一。二八”积极。其实这次上海人民积极援助上海的抗战,是大大的超过上次“一.二八”,可是因为国民党的限制政策,使得人民与军队隔离,如果这样继续下去,上海的抗战又有重复南口教训的可能。

芦沟桥抗战的失利,平津的陷落,虽有廿九军英勇的抵抗,佟麟阁赵登禹奋勇的牺牲,但是由于宋哲元氏乃没有抗战的决心,当然也就不会去实行正确的抗日战争的方针。当平津的人民和全国的人民起来援助廿九军时,宋乃不但拒绝了人民的援助,而且宣布戒严禁止人民的爱国运动。平津虽有许多不利的条件,但这并不能说明平津绝对不可保守,如果当时发动广大的人民,武装人民,平津是能够保持相当的时间在中国的手里,以待中国大批援军的赶到,可是宋氏深夜退出北平,宁愿把北平给日寇占去,而不愿北平人民起来保卫北平。

我们号召全国的人民,只有坚决的实行中共中央所提出的抗战方针,才能保障中国抗战的胜利,我们号召国民党全体党员,应以自我批评的精神,放弃统制的政策,采纳中国共产党所提出的方针,这是保障抗战胜利唯一的方针。

阿比西尼亚的亡国,并不是因为它发动了对意大利侵略主义者大规模的民族战争,而是因为他没有实行正确的民族抗战的纲领。阿比西尼亚的国皇,没有依靠在全国人民的力量上,他控制着民族战争,他个人曾成为民族英雄,但全民族则陷入亡国的命运。反之,西班牙的民族战争,则依靠人民的力量,结果能够与德意两国最大的法西斯侵略者战争到一年多,而仍保持马德里在政府军的手里,并击退了法西斯屡次的进攻,并取得胜利。我们还应该知道西班牙政府抵抗弗朗哥叛军的进攻时,西班牙原有的军队百分之八十在叛军手里,因为西班牙政府釆取了正确的抗战方针,迅速地组成了新的军队。中国全民族抗日战争,必须要从阿比西尼亚和西班牙两个民族战争中去学得教训。

救中国的关键,是在实行一个正确的抗战的方针。要把政府的抗战变为民族的抗战,要把单纯的军事抗战变为全面的抗战。中共中央的十大纲领中,就是包括全面的全民族抗战的方针。

要实行中共中央十大纲领的最主要的条件,必须是为着抗战而发展民主权利,开放民众爱国运动,只有这样才能发动人民参加抗战,方能使人民自由的参加抗战,只有人民参加抗战,才能使政府军队与人民结合起来,才能把单纯的军事抗战政府的抗战,变为全面的抗战全民族的抗战。

国民党中有些人说,要等到抗战全面胜利以后才能实现民主。这种意见的错误,显然不懂得没有民主,抗战就不能胜利。因此中共的决定中提出:“过去阶段中由于国民党的不愿意与民众的动员不够,因而没有完成争取民主的任务,这必须在今后争取抗战胜利的过程中去完成。”争取抗战胜利与实现民主权利不是互相分离的,而是互相联系的,互相依赖的。

国民党中有些人说,要抗战要牺牲自由。这种意见的错误。也正如对于民主与抗战的关系的错误了解是一样。如果没有政治自由,爱国的自由,要发动民众参加抗战就成为不可能,要争取抗战胜利也无保障。

我们赞成在民主的基础上,以争取抗战胜利为目的,进行全国民众运动的统一。但统一民众运动,并不能解释为统制民众运动。对民众运动的统制,事实上即是对民众运动的压制,在统制政策之下,自然不能动员广大的民众参加抗战。所以我们主张民众运动的统一,而反对民众运动的统制。

某些民族失败主义者,在全国抗战空气压倒的情机之下,不敢公开的揭起反对抗战的旗帜,于是在各种言词的掩盖之下,散布民族失败主义的情绪,如说“中国是弱国不能抗战”,如说“战呢,是会打败仗的”,就老实的承认打败仗是了。这些民族失败主义者,都不了解中国虽然是一个弱国,正因为他是一个被侵略的国家,进行正义的神圣自卫抗战,依靠民众力量,是能够取得抗战的最后胜利的。

要争取抗战的胜利,只有依靠民众的力量,发动全国的民族抗战,我们是一定能够战胜日本帝国主义的。如果把抗战看作只是政府的抗战,单纯的军事抗战,那么这种抗战是包含极大的危险性,有遭受重大的挫折与失败可能,为要避免这种危险性,争取抗战的最后胜利,那只有坚决实行正确的抗战的方针,即是实行全面的全民族抗战,这是争取抗战最后胜利的保证。

维护中共争取抗战胜利的十大纲领!

坚决实行中共争取抗战胜利的十大纲领!

动员一切力量为着争取抗战的胜利!

<p align="right">(原载《毛泽东言论集》,巴黎救国时报刊第四七~一七二页)</p>

