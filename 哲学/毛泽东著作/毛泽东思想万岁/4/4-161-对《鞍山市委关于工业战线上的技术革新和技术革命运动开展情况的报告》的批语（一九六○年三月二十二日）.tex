\section[对《鞍山市委关于工业战线上的技术革新和技术革命运动开展情况的报告》的批语(一九六○年三月二十二日)]{对《鞍山市委关于工业战线上的技术革新和技术革命运动开展情况的报告》的批语(一九六○年三月二十二日)}


上海局,各协作区委员会,各省委、市委、自治区党委,中央一级各部委、各党组:

鞍山市委这个报告很好,使人越看越高兴,不觉得文字长,再长一点也愿意看,因为这个报告所提出来的问题有事实,有道理,很吸引人。鞍钢是全国第一个最大的企业,职工十多万,过去他们认为这个企业是现代化的了,用不着再有所谓技术革命,更反对大搞群众运动,反对两参一改三结合的方针,反对政治挂帅,只信任少数人冷冷清清的去干,许多人主张一长制,反对党委领导下的厂长负责制。他们认为“马钢宪法”(苏联一个大钢厂的一套权威性的办法)是神圣不可侵犯的。这是一九五八年大跃进以前的情形,这是第一阶段。一九五九年为第二阶段,人们开始想问题,开始相信群众运动,开始怀疑一长制,开始怀疑“马钢宪法”。一九五九年七月庐山会议时期,中央收到他们的一个好报告,主张大跃进,主张反右倾,鼓干劲,并且提出了一个可以实行的高指标。中央看了这个报告极为高兴,曾经将此报告批发各同志看,各同志立即用电话发给各省、市、区,帮助了当时批判右倾机会主义的斗争。现在(一九六○年三月)的这个报告,更加进步,不是“马钢宪法”那一套,而是创造了一个“鞍钢宪法”。“鞍钢宪法”在远东、在中国出现了,这是第三个阶段。现在把这个报告转发你们,并请你们转发所属大企业和中等企业,转发一切大中城市的市委,当然也可以转发地委和城市,并且当作一个学习文件,让干部学习一遍,启发他们的脑筋,想一想自己的事情,在一九六○年一个整年内,有领导地,一环接一环、一浪接一浪地实行伟大的马克思列宁主义的城乡经济技术革命运动。


