\section[最高国务会议上的讲话纪要(一)(一九五八年九月五日)]{最高国务会议上的讲话纪要(一)}
\datesubtitle{(一九五八年九月五日)}


最高国务会议,二月开了一次,现在是九月,六个月没有开了。二月那次会上,我们谈了鼓足干劲,力争上游,多快好省。讲了个大有希望,不晓得同志们记得不记得?我还比较一下,不是中有希望,更不是小有希望,而是大有希望。还讲了以一个普通劳动者的姿态在人民群众中出现。我们在座诸位,以及共产党里头有许多人,要办到这一点,也是要努一番力的。我们跟国民党相反,他们是以一个贵族的姿态,老爷派头在人民中出现,我们是以一个普通劳动者的姿态在人民中出现。

那一次讲了几句不好听的话,批评了“好大害功,急功近利,鄙视过去,迷信将来。”你说是坏的,我说是好的,这不是唱对台戏吗?有些人看了那四句评语实在舒服。“共产党好大喜功。急功近利,鄙视过去,迷信将来,岂有此理?”我说,恰好是有理,不是岂有此理,而是确有此理。有些人为什么支持那种批评呢?就是因为他们对于辩证唯物主义和历史唯物主义、马克思主义政治经济学、无产阶级的阶级斗争和无产阶级专政这三门科学,或者是了解得不深不透,或者简直就不去理会,因此就缺乏分析。怎么分析呢?有资产阶级的好大喜功,有无产阶级的好大喜功,两种好大喜功。有资产阶级的急功近利,有无产阶级的急功近利。“攀攀为利者,跖之徒也”,这大概是今天的资产阶级的一类。孜孜为利者,资本家之徒也。我们呢?我们就是另外一种急功近利。至于鄙视过去,迷信将来,也是有阶级不同的。资产阶级是迷信过去,鄙视将来。过去的古董那就是宝贝。至于将来,什么共产主义,社会主义,那就是狗屁。这不是迷信过去,鄙视将来吗?

这六个月,发生很大变化。我想,在座诸位都是有变化的。我的脑筋也有变化。有许多事情料不到的。今年二月那个时候虽然讲大有希望,那个希望究竟怎么样?比今天的现实还是落后些。

国内形势,如大家所知道,就是阶级关系,阶级力量对比,起了很大变化。几亿劳动群众,工人农民,他们现在感觉得心里舒畅,搞大跃进。这就是整风反右的结果。整风以前我们许多干部有两条:一条叫官气较多,二条叫政治较少。不是要反五气吗?五气的头一气就是官气。经过整风,没有整好的还有一些,但是大多数人是以普通劳动者的姿态出现,跟工人农民打成一片,人们感觉跟国民党时期确实不同了。过去不是的。因为有一部分干部,在工人看来,他们神气不对,是在做官,跟国民党没有分别,还是压在他们头上,所以有些工人工作就不那么积极,不为社会主义和共产主义奋斗,是为手表、自行车、钢笔、收音机、缝纫机等五大件而奋斗,就是为个人奋斗。那个时期许多人不觉悟,积极分子只是一部分,落后分子相当多。因为共产党批判了三风五气,他们也就自我批判了:我们这个五大件也是为个人的,不为社会,也不对呀。工作就积极起来了。农民也是这样,因为合作社干部,县、区、乡干部搞试验田,跟他们打成一片,一股热潮就起来了。冬季很冷也要搞水利,他们知道是为谁来干事情,是为他们自己,为集体,为全国。这一干的结果,今年大概可以差不多增产××,即有可能从去年三千七百亿斤,增到××亿斤。棉花去年是三千三百万担,今年大概有××万担,可以超过××。烟叶可以超过三四倍。只有油料只超过半倍,还是不足的。麻类作物,过去没有注意,没有抓紧。钢铁可以翻一番。一九五六年中共八大第一次会议,总理在那里建议,五年计划搞钢铁一千零五十万吨到一千二百万吨,如果说一千零五十万吨,今年就有超过的可能,可能搞一千二百万吨。农业发展纲要四十条不是十二年吗?五六年开始,五六、五七、五八,基本完成。这些都还是一些预计,还要看实际的结果。今年如果搞到××多亿斤粮食,明年如果又翻一番,就是××亿斤。明年也许不能搞这么多,太搞多了,除了人吃马喂之外,现在还没有找到用途,也许会发生问题。但是明年总是可能超过××亿斤,钢铁明年可能超过××万吨。总而言之,明年是基本上赶上英国,除了造船、汽车、电力这几项之外。明年都要超过英国。十五年计划,两年基本完成。谁人料到?这就是群众的干劲的结果。

资产阶级、民主党派,也超了变化,并且还在继续起变化中。以一个普通劳动者的姿态出现的这个问题,在民主党派中间是个相当严重的问题,要逐步来,不能很性急。但是形势逼人。形势就是人,就是多数人压迫少数人。多数人造成一种形势,少数人就感到压力,就得打点主意,我历求是主张对立面的,没有对立面,谁也不干的。我有什么对立面呢?在我们民主队伍里头有很多对立面。此外还有在我们队伍以外的“地富反坏右”,这都是对立面。工人农民压迫我们,他们说,你做官,你得好好做,你做不好,我就整你。比如五五年上半年,有许多老百姓也实在不喜欢我们,人人谈统购,家家说粮食,那个时候你说粮食没有危机,我也可以讲一个危机。一个原因是因为粮食不足,还有一个原因就是富裕中农兴风作浪。其中主要是共产党员,他们是党员,但是他们当了县区乡干部,他们叫“农民苦”,所谓“农民苦”,就是他们苦,所谓他们苦,就是余粮多。据江苏省的统计,在我们县区乡干部里头,这种人有百分之三十。他们每天叫“农民苦”,说统购统销太多了,不赞成。这一来,煽起这些农民,本来够吃的,也说不够吃,用各种办法来吵。这一压迫,就打主意吧,就搞合作化。合作化的决心就是那个时候搞起来的。批判各种迷信;什么解放区合作化不行,什么没有会计,什么平地可以,山上不可以,什么汉人可以,少数民族不可以,等等,破除了这些迷信,没有几个月,合作化就搞起来了。然后又影响工商界,敲锣打鼓,全行业公私合营。到一九五六年那个时候,有些人又觉得恐怕不行了,多快好省也不灵了。同时斯大林问题也发生了,匈牙利事件出来了,帝国主义反共反苏,那么一个潮流,国内就是右派酝酿活动。经过整风反右,才把这些东西扭转过来。而现在呢,就转到一个比较有利的方向,民主党派也好,科学界的人也好,工程技术界的人也好,资产阶级(工商界者)也好,总而言之,绝大多数人,或者已经改变立场,或者正在向改变立场前进,也还有少数没有改变的,还有左中右。阶级还是存在的。说现在阶级不存在了,阶级斗争已经消灭了,这个观点恐怕是不对的,我说像吃鸦片烟一样,吃鸦片烟上了瘾,是不容易戒的,资产阶级思想,还有封建主义思想,那么容易戒我就不相信。要慢慢戒,不要形势逼人,还要看事实。一个长江大桥可以说服许多人,你没有长江大桥,他就不信。出了个长江大桥,许多人去看了,他就信了。今年一千一百万吨钢,明年××万吨钢,苦战三年,后年××万吨,粮食由三干七百亿斤到××亿斤。我还是讲个可能性,要努力。到底那个时候怎么样?有两个可能:一个可能达到,一个可能少一点,不可能达到这么多。这样一来,天天劳累,是不是人就大批死亡,或者由胖子变成瘦子,或者生病?这也有的,也有伤亡的,变成瘦子也有的,生病也有的。但这是个别的,多数人我看是相反,一不死,二不瘦,还要胖一些,也不生病。特别是把四害一除,疾病大为减少。农民劳动起来是有纪律的,军事化干劲甚大。公共食堂一来,节省时间,免得往返。节省粮食,节省柴火,节省经费,此外还节省大批时间。这是××县的经验,科学技术也有很大的进步,“将军”将得厉害,就是学生将教师,讲师、助教将教授,研究员将所长。有那么一个科学研究所,叫作药物研究所,设在上海,所长赵承暇有门本领,就是可以在中国的一种植物里头提炼出一种药来,可以治高血压。但是他老先生就是对什么人也不讲,他也不作。他那个所里的青年人就没有办法,他们呕了气,就自己干,结果苦战多少昼夜,搞出来了,能够提炼出那个药来了。这样的事情不止一个所,有相当几个所。大学教授相当有一些人落后于学生,编讲义,编教材,编教学大纲,编学生不赢,学生是苦战几昼夜,集体来搞。听说师范大学有个文学班,要编一个文学史,一个班有二十六个人,苦战四昼夜,读了二百九十部中外文学名著,编出一本文学史大纲。这是形势逼人,就是压迫。青年人不压迫老年人,老年人不会进步的。这一压,老年人就有出路了,他们不进步不行了。当然,不是青年人个个都是好的,也有坏的,青年里头,吊儿郎当的,阿飞,偷东西的,那种人也有。但是一般说来,总是“譬如积薪,后来居上”。我们要承认这一条。

国际形势,我们历来有个观点,总是乐观的。后来总结为一个“东风压倒西风”。

美国人现在在我们这里来了个“大包干”制度,索性把金门、马祖,还有什么大担岛、二担岛、东碇岛一切包括进去,我看他就舒服了。他上了我们的绞索,美国人的颈吊在我们中国人的铁的绞索上面。台湾也是一个绞索,不过要隔得远一点。他要把金门这一套包括进去,那他的头更接近我们。我哪一天踢他一脚,他走不掉,因为他被一根索子绞住了。

我现在提出若干观点,提出一些看法供给各位,并不要把它作为一个什么决定、作为一个法律。作为一个法律就死了,作为一个看法就是活的,拿这些观点去观察国际形势。

第一条,谁怕谁多一点。我看美国人是怕打仗,我们也怕打仗。问题是究竟哪一个怕得多一点。这也是个观点,也是个看法。请各位拿这个观点去看一看,观察观察,以后一年,二年,三年,四年,就这样观察下去,究竟还是西方怕东方多一点,还是我们东方人怕西方多一点?据我的看法,是杜勒斯怕我们怕得多一点,是英美德法那些西方国家怕我们怕得多一点。为什么他们怕得多一点呢?就是一个力量的问题,人心的问题。人心就是力量,我们这边的人多一点,他们那边的人少一点。共产主义、民族主义、帝国主义,这三个主义中,共产主义和民族主义比较接近。而民族主义占领的地方相当宽,有三个洲,一个亚洲,一个非洲,一个拉丁美洲,则使这些洲里头有许多统治者还是亲西方的,比如泰国、巴基斯坦、日本、土耳其、伊朗,可是人民中间亲东方的不少,可能是相当多。就是垄断资本家以及中了他们的毒最深的人是主张战争的。比如北欧几个国家,当权的也是资产阶级,他们是不愿意战争的。力量对比是如此。因为真理是抓在大多数人手里,而不抓在杜勒斯手里。他们的心比我们虚,我们的心比较实。我们依靠人民,他们是维持那些反动统治者。现在杜勒斯就干这一条,他们专扶什么“蒋委员长”、李承晚、吴庭艳这类人。我看是这样,双方都怕,但是他们比较怕我们多一点,因此战争是打不起来的。

第二条,美帝国主义他们结成军事团体,什么北大西洋、巴格达、马尼拉,这些团体的性质究竟怎么样?我们讲它们是侵略的。它们是侵略的,那是千真万确的。但是它现在的锋芒向哪一边呢?是向社会主义进攻,还是向民族主义进攻?我看现在是向民族主义进攻,就是向埃及、黎巴嫩和中东那些弱的进攻。社会主义国家,除非是比如匈牙利失败了,波兰也崩溃了,捷克、东德也崩溃了,连苏联也发生问题,我们也发生问题,摇摇欲倒,那个时候他们会进攻的。你要倒了,他们为什么不进攻?现在我们不倒,我们巩固,我们这个骨头啃不动,它们就啃那些比较可啃的地方,搞什么印尼、印度、缅甸、锡兰,想搞垮纳赛尔,想搞垮伊拉克,想征服阿尔及利亚等等。

现在拉丁美洲有个很大的进步。尼克松是个副总统,在八个国家不受欢迎,被吐口水,打石头,美国的政治代表在那些人面前,被用口水去对付,这就是藐视“尊严”,没有“礼貌”了,在他们心目中间不算数了。你是我们的对头,因此拿口水、石头去对付你。所以不要把那三个军事集团看得那么严重,要有分析。它是侵略的,但是它并不巩固。

第三条,关于紧张局势。我们每天都是要求缓和紧张局势,紧张局势缓和了,对世界人民是有利的。那么,凡是紧张局势就对我们有害,是不是?我看也不尽然。这个紧张局势,对我们并不是纯害无益,也有有利的一面。什么道理呢?因为紧张局势除去有害的一面外,还可以调动人马,调动落后阶层,调动中间派起来奋斗。怕打原子战争,就要想一想。你看金门、马祖打这样几炮,我就没有料到现在这个世界闹得这样满天飞雨,烟雾冲天。这就是因为人们怕战争,怕美国到处闯祸。全世界那么多国家,除了一个李承晚之外,现在还没有第二个国家支持美国。可能还加一个菲律宾,叫作“有件条的支持”。比如伊拉克革命,还不是紧张局势造成的?紧张局势并不取决于我们,是帝国主义自己造成的,但是归根结底对于帝国主义更不利。这个观点列宁说过的,他是讲战争,他说,战争调动人们的精神状态,使他紧张起来。现在当然没有战争,但是,这种在武装对立的情况下的紧张局势也是能够调动一切积极因索,并且使落后阶层想一想。

第四条,中东的撤兵问题。英美侵略军必须撤退。但是帝国主义现在想赖在那里不走,这对人民是不利的,可是同时它也有教育人民的作用。你要反对侵略者,如果没有个对象,没有个靶子,没有个对立面,这就不好反。他自己现在跑上来当作对立面,并且赖着不走,就起了动员全世界人民起来反对美国侵略者的作用。所以他迟迟不撤退,总起来看对人民也不见得那么纯害无利,因为这样人民每天就可以催他走,你为什么不走?

第五条,戴高乐登台好,还是不登台好?现在法国共产党和人民应该坚决反对戴高乐登台,要投票反对他的宪法,但是同时要准备反对不了时,他登台后的斗争。戴高乐登台要压迫法共和法国人民,但对内对外也有好处:对外,这个人喜欢跟英美闹别扭,很有益处。对内,为教育法国无产阶级不可少之教员,等于我们中国的“蒋委员长”一样。没有“蒋委员长”,六亿人民教育不过来的,单是共产党正面教育不行的。戴高乐现在还有威信,你这会把他打败了,他没有死,人们还是想他。让他登台。除非是顶多搞个五年、六年、七年、八年、十年,他得垮的。他一垮了,没有第二个戴高乐了。这个毒放出来了,这个毒必须得放,放出来毒就消了。

第六条,禁运。不跟我们作生意。这个东西究竟对我们的利害怎么样?资本主义国家跟我们多作生意,还是少作生意好?现在生意是作,是少作。我看,禁运对我们的利益极大,我们不感觉禁运有什么不利。禁运对我们的衣食住行以及建设(炼钢炼铁)有极大的好处。一禁运,我们得自己想办法。我历来感谢何应钦。一九三七年红军改编成国民革命军第八路军,每月有四十万法币,自从他发了法币,我们就依赖这个法币。到一九四○年反共高潮就断了,不来了。从此我们得自己想办法。我们想什么办法呢?我们就下了命令,说法币没有了,你们以团为单位自己打主意。从此。各根据地搞生产运动,产生的价值不是四十万元,可能是一亿两亿。从此就靠我们自己动手。现在的何应钦是谁呢?就是杜勒斯,改了个名字,现在他们禁运。我们就自己搞,搞大跃进,搞掉了依赖性,破除了迷信,就好了。

第七条,不承认问题。是承认比较有利,还是不承认比较有利?我说,等于禁运一样,帝国主义国家不承认我们比较承认我们是要有利一些。现在还有四十几个国家不承认我们,主要的原因就是因为美国。比如法国,想承认,但是因为美国反对就不敢。其他还有一些中南美洲、亚洲、非洲、欧洲的国家.以及加拿大,都是因为美国而不敢承认。资本主义国家现在承认我们的,合起来只有十九个,加上社会主义阵营十一个,有三十个,再加上南斯拉夫,有三十一个。我看就是这么一点过日子吧。我们搞三亿吨钢,最好搞七亿吨钢,三万五十亿斤粮食,这要多少年,我看××年就差不多。说工业了不起,可难啦,什么科学可难啦.这也是个迷信,就不要信那些。十五年赶上英国。我们是两年基本上赶上。这是讲总数,不是按人口,按人口平均赶上英国,就钢铁来说,要三亿吨,英国五千万人口,有两千二百万吨钢,我们有七亿人口,得要三亿吨钢,刚才讲七亿吨钢,要三亿吨钢翻一番还要多一点,那可能要十五年,也许还要多一点。世界上的事情有这么怪,不搞就不搞,一搞就很多,要么就没有,要么就很多。你们不信这一条?比如我们打了二十二年的仗,二十一年就是不胜利,而在二十二年这一年,就是一九四九年就全国胜利了,叫突变。粮食也是一样,搞了八年,七搞八搞,还只有那么一点。一九四九年粮食是二千一百亿斤,去年三千七百亿厅,在我们手里搞了八年,只增加一千六百亿斤,而今年一年就可以增加××亿斤,可能到××亿斤。搞八年没有摸到一条路,不会搞,也是因为制度没有改革,个体经济。初级合作化,没有整风反右。钢铁也是一样,几十年只有那么一点,蒋介石只有四万吨钢,这还是张之洞遗留下来的。两个东西最要紧:一个粮,一个钢。有了钢就能作机器,什么机器也可以作,挖煤炭的机器,开矿山的机器,发电的机器,炼石油的机器,火车、轮船、飞机等交通机器、化学工业的机器、砌房子的机器、农业机器,都要钢。所以一为粮,二为钢,加上机器,叫三大元帅。三大元帅升帐,就有胜利的希望。还有两个先行官,一个是铁路,一个是电力。扯远了,还是回到不承认的问题上来,不承认我们,我看是不坏,比较好,让我们更多摘一点钢,搞个六、七亿吨,那个时候他们总要承认。那个时候也可以不承认,他不承认有什么要紧。

最后一条,就是准备反侵略战争。头一条讲了,双方怕打,仗打不起来。但世界上的事情还是要搞一个保险系数。因为世界上有个垄断资产阶级,恐怕他们冒里冒失乱搞,所以要准备作战。这一条要在干部里头讲通。第一,我们不要打,而且反对打,苏联也是。要打就是他们先打,逼着我们不能不打。第二,但是我们不怕打,要打就打。我们现在只有手榴弹跟山药蛋。氢弹、原子弹的战争当然是可怕的,是要死人的。因此,我们反对打。但是这个决定权不操在我们手中,帝国主义一定要打,那么我们就得准备一切,要打就打。就是说,死了一半人也没有什么可怕。这是极而言之。在整个宇宙史上,我就不相信要那么悲观。我跟尼赫鲁辩论过这个问题,他说,那个时候没有政府了,统统打光了,想要讲和也找不到政府了。我说哪有那个事,你这个政府被原子弹消灭了,老百姓又起一个政府,又可以议和。世界上的事情,你不想到那个极点,你就睡不着觉。无非是打死人,无非是一个怕打。但是他一定要打,是他先打,他打原子弹,这个时候,怕,他也打,不怕,他也打。既然是怕也打,不怕也打,二者选哪一个呢?还是怕好,还是不怕好?每天总是怕,在干部和人民里头不鼓起一点劲,这是很危险的,我看,还是横了一条心,要打就打,打了再建设。因此,现在我们要搞民兵。人民公社里头都搞民兵,全民皆兵,要发枪。开头发几百万支,将来要发几千万支,各省造轻武器,造步枪、机关枪、手榴弹、小迫击炮、轻迫击炮。人民公社有军事部,到处练习。在座的有文化人,又是文化,又是武化。

有这么八个观点,当作一种看法,供各位观察国际形势的时候采用。


