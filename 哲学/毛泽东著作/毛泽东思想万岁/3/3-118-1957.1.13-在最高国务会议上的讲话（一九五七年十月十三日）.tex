\section[在最高国务会议上的讲话(一九五七年十月十三日)]{在最高国务会议上的讲话}
\datesubtitle{(一九五七年十月十三日)}


现在整风找出了一个形式。这个形式,就是大鸣,大放,大争,大辩,大字报,这一套。在群众中间创造了这么一个方式。这个方式跟历史上有所区别的。在延安那一次整风,也有大字报,但是那个时候我们不敢推行,恐怕也是有点怕。后头有一个三查三整。三查是查历史,查工作,查思想,三整是整什么我忘记了。那个时候我们新华社社长范长江就下不了台,整了他两个月,然后才翻身。因为发动了群众。三反中间有多少个部长下不了楼,后头我们摆了梯子才接下来。从前军队里头,在战争时期,依靠战士们,依靠当地人民,没有那个地方发饷,没有那个地方创造枪炮的工厂,就只好依靠群众,所以长期以来,军队,地方都形成一种民主作风。但是那个时候,整个革命时期,就没有现在这个大放,大鸣,大争,大辩,大字报。三反五反没有搞大报。三查三整没有搞大字报。再上去,延安那个时候出了一点大字报,我们也没有提倡。这是什么理由?我想或者是我们这些人那个时候蠢一点吧?恐怕还有客观原因,就是那个时候金鼓齐鸣,打仗,阶级斗争那么尖锐,内部又这么大吵,那就不好了。现在就不同了,那样的阶级斗争已经过去了,基本上结束了,全国和平了,所以才出现这样一个东西,找到了这么一个形式。这种革命的内容,它要找到这种形式的,现在的革命就是社会主义革命,就是为了建设,而找到了这种形式,就可以很快普及,很快学会,几个月就可以学会。虽然有几怕,主要是两怕:一个就是怕乱。你们怕不怕乱?你们胆子那么大?我就不那么十分相信。因为我们有许多人也是怕乱的,你说我一点也不怕乱呀?各民主党派是不是也有一点怕乱?还有一个是怕下不了台。当厂长的,当合作社主任的,当学校校长的,当党委书记的,这一放出来,火一烧,怎么下台呀?现在好说了,比如在五月间那个时候就不容易说服人。北京三十四个大专学校,开了多少会才放开。北京大学的江隆基算是愿意放的,也开了多少会。为什么可以不怕?为什么放有利?大鸣大放有利,还是小鸣小放有利?或者不鸣不放有利?不鸣不放是不利的,小鸣小放不能解决问题,还是大鸣大放。大鸣大放,一不会乱,二不会下不得台,除开个别人,比如丁玲,她就下不得台了。比如冯雪峰,他是个出版社的社长,他在那里放火,他就下不得台了。那是少数人,就是右派。他放的火,他的目的是要烧共产党,冯雪峰是个共产党。其他的人就不要怕。基础就是要相信群众的多数,要相信人民中间的多数人是好人。应该讲,工人多数人是好人,农民多数人是好人,资本家多数是可以改造的,知识分子、民主党派的成员,共产党,青年团,应该相信多数,他们不是想要把我们国家搞乱的。所谓多数,究竟什么叫多数,是不是百分之五十一就叫多数呢?不是这么个数目,是百分之九十到百分之九十八。现在全国,我跟地方的同志摸一摸底,究竟多少人不赞成社会主义?搞社会主义这个事,什么人都是生的,我们都是生手。我们过去只是搞好民主革命,那是资产阶级性质的,不破坏个体所有制,不破坏资本主义所有制,但是破坏封建阶级所有制,买办阶级(蒋、宋、孔、陈那套人)所有制。所以有许多人在民主革命阶段,他可以过来。某些人对彻底的民主革命也不热心,但是他可以过来。有些人对彻底的民生革命他肯干的,比如湖北省有那么一个雇农叫刘介梅,报上登了的,他是三代要饭,后来翻了身,发起来了,现在当了区长一级的干部,这回他非常不满意这个社会主义,非常不赞成合作社,要搞“自由”反对统购统销。现在他开了展览会,痛哭了,要求不要开除党籍,愿意改。他是湖北黄岗县的,在那个县里开个展览会,他就当指导员,分两个阶段,要饭的阶段,同后头发起来的阶段。像那样的人,我跟湖北同志商量,恐怕可以不开除党籍,因为他愿意改嘛。民主革命可过关,而社会主义这个关就有些人难过。因为这是最后一关,它是要破私人所有制,变为集体所有制,当然,这个斗争要搞很多年的,究竟多少时候叫过渡时期,现在也还很难定,大体上我看要三个五年计划或者还多一点时间。已经过去了一个五年计划,还需要十年左右。但是我看今年这一年是一个高峰。以后是不是还有这样的洪峰?像每年七、八月,八、九月间,黄河的洪峰要来,以后这十五年是不是年年要来洪峰,你们水利部要筑堤,(笑声)我看恐怕不是那样,这个峰越搞越小。六亿人口中间,出了十几万右派,也是很少。我说全国人民中间有百分之十的人不赞成社会主义,他们许多同志摸了这个底,有人说百分之十五,有人说没有百分之十,只有百分之几。大概百分之十可靠。对于社会主义,按他本心是不赞成的,这包括地主阶级、富农、一部分资产阶级、一部分资产阶级知识分子、一部分城市的上层小资产阶级,甚至于个别的贫农、下中农、工人,刚才讲的刘介梅那样的人。六亿人口中百分之十是多少呢?有六千万。这个数目不小,不要把他看小了。在这上头,我们就有两个出发点:第一,我们百分之九十的人赞成社会主义,工人阶级有广大的同盟军,第一个同盟军是在农村,有贫农、下中农和一部分富裕中农。刚才讲,有人说百分之十反对社会主义,反对合作化,反对统购统销(×××插话:是中农里边的百分之十五)我是讲全国人口,我跟一些省委书记摸了这样一个底。但是所谓坚决分子,包括极右派、反革命,搞破坏的,还有不搞破坏的,很坚决的,可能要带到棺材里去的,他就不改,什么社会主义,他不听,月亮还是美国的好,中国的月亮差一点,这样的人有多少呢,大概只有百分之二左右。刚才你(指×××)讲了,三百万人里头的比例是百分之二点二,全国人口里头的百分之二是多少呢?就是一千二百万。一千二百万如果集合起来,那是个很大的军队,如果他手里拿了枪,那是一千二百万的军队。但是为什么天下不会大乱?因为它是分散在这一个合作社,那一个合作社,这一个农村,那一个农村,这一个工厂,那一个工厂,这一个学校,那一个学校,这一个共产党支部,那一个共产党支部,这个共青团支部,那个共青团支部,这个民主党派的支部,那个民主党派的支部,是分散在各处,他不能集合。所以我们的基础,有百分之九十的人赞成,包括无产阶级,农村里头的半无产阶级,贫农,下中农,小资产阶级,资产阶级知识分子,上层小资产阶级知识分子。现在我们把地主同买办这两个阶级不算,那两个阶级是民主革命的对象。我们不是反对帝国主义、封建主义、官僚资本主义吗?帝国主义在外国,现在我们已经赶走它了,在国内就有一个封建主义、一个官僚资本主义,就是地主同买办阶级,那是一个革命对象。这一个革命是在什么范围之内呢?是一些什么阶级发生斗争呢?就是无产阶级、小资产阶级同资产阶级,这样三个阶级。

无产阶级数目很小,但是他有小资产阶级的广大同盟军,就是贫农、下中农。贫农是半无产阶级,下中农是中农,是有产,但他是比较苦的,这个数量,占农村人口的百分之七十或者还要多一点。富裕中农大体占百分之二十。现在富裕中农大体分三部分:一部分是赞成合作化的,占百分之四十;动摇的、两可的,占百分之四十;反对的占百分之二十。地主也有分化,地主现在不完全反对社会主义,因为这几年的教育,出来了一部分不闹粮食,赞成统购统销,赞成合作化的地主,富农也出来了一部分。不要以为现在的地主都是反对社会主义,富农都是反对社会主义,资本家都是反对社会主义,高级知识分子都是反对社会主义,事实不是那样,要加以分析,这里面坚决分子是大概百分之二。所以我们就要相信多数。你看,百分之九十嘛!经过工作,经过大辩论,还可以争取百分之八,就变成百分之九十八,那个坚决死硬派只有百分之二。当然要注意,刚才×××同志讲了,他还是一个很大的力量呀。地主更没有威信。买办资产阶级是丧失了威信的。资产阶级同资产阶级知识分子,农村的小资产阶级(富裕中农),城市的上层小资产阶级(一些比较富裕的小业主)同他们的知识分子,这些人就有威信了。比如农村的富裕中农是有威信的。因为富裕中农生产是比较强的,贫农赶不上它。至于资产阶级,资产阶级知识分手,上层小资产阶级知识分子,中国这个知识分子吃得开,不是说过他们是我们民族的财富吗?的确是,你那一样都是缺不了他,没有他,不能开学校,教授离不了他,中学小学教员离不了他,办报纸要新闻记者,唱戏要有演员,要有科学家,工程师,技术员。这个阶级人数不多,几百万人,连他的家有几千万人,大概有三千万人。听说真正的资本家七十万,不包括家属。知识分子可能有四、五百万,我说一共打他六百万,五口之家(因为他们比较富裕,生儿育女比较多),五六就是三千万。这个阶级比较有文化的,最有技术的。右派翘尾巴也在这里。罗隆基不是讲过吗?无产阶级的小知识分子就领导不了资产阶级的大知识分子。他自己一定要说他是小资产阶级,他说他是小资产阶级的大知识分子。我看来,不仅是无产阶级的小知识分子,就是无产阶级,大字也不认得几个,就比罗隆基高明。资产阶级同资产阶级知识分子,上层小资产阶级同他们的知识分子,包括中间派,他们对于共产党、无产阶级是不服气的。拥护共产党、拥护宪法,那也是拥护的,手也是举的,但是实际上心里是不那么服气的。但是这里头就要分析了:右派是对立的,非右派是半服半不服。不是讲这样也不能领导,那样也不能领导吗?不仅是右派有这个思想,别人也有。总而言之,差不多就完了,共产党就非搬外国去不可。无产阶级非上月球不可。因为你这样也不行,那样也不行嘛,无论讲那一行,他都说你不行。我说经过这一次辩论,主要目的,就是争取半服半不服的,使他们懂得这个社会发展规律究竟是一件什么事。还是要听那个文化不高明的无产阶级的话,在农村里头要听贫农、下中农的话。讲文化,讲本领,贫农不如富裕中农,但是讲革命,就是他们行。这可以不可以说服多数人?可以说服多数人,多数人完全可以说服。资产阶级的多数,上层小资产阶级的多数,我刚才讲,富裕中农只有百分之四十是赞成合作化,有百分之四十是动摇的,只有百分之二十是不赞成的,这个动摇的就可以说服他。大学教授,中小学教员、艺术家、文学家、科学家、工程师,这许多人,多数人都可以说服的,用说服的方法,不大服气,慢慢他就会服气。我看还要过若干年,大概经过十年。比如对苏联总是不服气,现在放出一个“月亮”来,好像又有点行了。(笑声)在苏联也是经过这个阶段的,说共产党不行,这样不能领导,那样不能领导,现在他们早已解决了,革命四十年了,我们还只有八年,所以难怪。有百分之九十的人拥护社会主义,所以我们不要怕乱,不会乱,乱不了。只要不是冯雪峰、丁玲这种人,也不要怕下不得台,怎么下不得台呢?可以下台嘛,无非是三大民主。有则改之。

在这个基础上,大鸣、大放、大争、大辩、大字报,在这个时候出现很有益处,这种形式是没有阶级性的。什么大字报,什么大鸣大放,右派也可以搞大鸣大放,右派也可以出大字报。我说感谢右派,“大”字是他们发明的。鸣放是我们发明的,我在今年二月二十九日并没有讲什么大鸣、大放、大争、大辩、大字报,没有这个“大”字。去年×××有一篇文章,我们去年五月在这里讲百花齐放,那是一个放,百家争鸣,那是一个鸣,就是没有那个“大”,并且是限于文学艺术上的百花齐放,学术上是百家争鸣,就涉及政治。后头右派他需要涉及政治,就是什么问题都叫鸣放,叫作鸣放时期,而且要搞大鸣大放。可见这个口号右派也可以用,中间派也可以用,左派也可以用。大鸣、大放、大字报,究竟对那个阶级有利,归根结底对无产阶级有利,归根结底对右派不利。问题是,百分之九十的人不愿意国家乱,而愿意建成社会主义,百分之十的人中间有许多人是动摇的,至于坚决分子只有百分之二,你乱得了呀!所以这样大鸣大放的口号,大字报、大鸣、大放、大辩论这样的方式和方法,归根结底有利于多数人的自我改造。两条道路,一条社会主义,一条资本主义,归根结底有利于社会主义。我说,等于演话剧一样。从前抗日时期,北方有个新民会,是那个缪斌在这里搞的。缪斌也是我的老朋友,他是改组派。缪斌他后头当汉奸了,搞新民会了,他们搞话剧,报上登了,在太原演话剧,来赞扬日本的皇军,反对中国人。那么因为汉奸演了话剧,我们就不演了吗?我们还是要演。还有旧诗,黄老你是专家,这个东西什么人都可以用的。北京大学有个教授写的那几首诗里有什么“××××乱横行”,什么东西,他可以用来反对革命的。那个时候我也说,这种人不要十分去追究他,那个时候他那种思想,就是要写一点诗骂人。我也很赏识他那几首诗。(笑声)不要怕乱,也不要怕下不得台。当然,右派是下不了台。但还是可以下台的,右派总要下台吧。下台,按照辩证法,我看是一分为二,两点论,一部分右派将来可能把右派帽子摘掉。永远戴下去吗?从此不得翻身呀?我看不一定。可能有相当多数的右派分子,他想通了,大势所趋,他转好了,比较老实,比较不十分顽固,那个时候把帽子一摘,就不叫右派了,并且还要安置工作。右派因为他反对社会主义,他是一种敌对的力量,但是现在我们不把他当作过去对地主,对反革命派那样来办,基本的标志就是不取消他的选举权。也许有个别的人要取消他的选举权。(总理:劳动改造。)譬如林希翎那样的人,现在做什么工作呢?她在人民大学扫地。听说她很愿意干那个事。那种人,娃娃,二十八岁了,也不娃娃了。自己撒谎说只有二十一岁,确实有二十八岁了。进青年团进不了,她就不高兴。现在孤立起来了,在学校里做点工作,劳动改造。那是个别的人。但是你要费孝通,还有人民大学的吴景超去劳动改造,那怎么行呢?那总不好吧。这么大知识分子,肩不能挑,手不能提。现在我们的干部许多要去劳动,北京就放了几万人下去,一定要做几年工,也算劳动改造吧。将来大学生要先做几年工,不然一辈子就没有做过工,没有种过田。当然,我不是在这里宣布吓你们,要你黄炎老、陈垣先生、张文伯都下去做苦工,(黄炎培:家里头做做也可以)老人不是说:“洒扫庭除,应对进退”吗?(笑声)这些话我们过去都说过。当然没有现在说得这么透,尤其没有摸这个底。这么闹一下,使我们摸一下底,一方面是百分之九十,百分之九十八,另一方面是百分之十或百分之二,摸了这样的底,就心中有数了。用我们这个办法,可以避免匈牙利那样的事件,也可以避免现在波兰发生的那样的事件。波兰这个问题还没有解决的。他要封一个报纸,我们不需要封报纸,我们只一篇社论就行了。《文汇报》写了两篇社论,头一篇不彻底,没有讲透,第二篇社论就自己攻。《新民晚报》是自己攻。赵××跟我谈的时候,我说,你搞得很好。他说实在是犯了错误。我说,你犯了错误,你改了,就行了。《新民晚报》很小一个报,你们都不看的?这个报值得一看。在波兰不行,他要封一个报纸,封这个报纸,就惹起来来了,不晓得这两天结果如何?(总理:还没有完。)他那个问题,总而言之,还没有解决,他是反革命没有解决,右派问题没有解决,资产阶级思想、两条道路的问题没有解决。

要造就无产阶级自己的知识分子。你们各位听了是不是吓一跳?因为你们许多人又不是无产阶级自己的知识分子。无产阶级虽然少,解放以前只有四百万产业工人,现在是一千二百万,八年以来增加了八百万,不要看这么少一点人,只有这个阶级才有前途,其他的阶级都是过渡的阶级。全部农民头一步过渡到集体化,第二步要变为国营农场。至于小资产阶级同资产阶级,都是过渡阶级,资产阶级就不要了,这个阶级要灭掉,不是讲把人灭掉,要变成无产阶级知识分子,可以慢慢变的,我在四月三十日就讲过:“皮之不存,毛将焉附?”不然就有作梁上君子的危险。那一次主要问题是讲这个问题,讲阶级起变化。这几个阶级都是过渡的阶级,都要过渡到工人阶级那方面去。现在许多人进了工会,说进了工会岂不是变成工人阶级了吗?进了共产党,他要反共,共产党反共!丁玲、冯雪峰不是共产党反共?进了工会不等于就是无产阶级了。学校里的新职员都进了工会,钱伟长不是工会会员吗?钱瑞升不是工会会员吗?还要有一个改造过程。右派许多人是有才干的,在这一点上我倒还相当赏识他们。不过他这个才干用来反共反社会主义就不行了。怎么样把他们改造一下?比如费孝通,我跟他谈过,我说,你可不可以改一下呀?(笑声)他学了我们土改里扎根串联的办法,他一共有两百多个高级知识分子的朋友,什么北京、成都、武汉、上海、无锡各地都有,他说吃亏就在这个地方,他在那圈子里就出不来,他不仅出不来,他有意识组织这些人,代表他们大鸣大放。我说,你不要搞那二百个,另找二百个,到工人、农民里头去找二百个,他说不晓得还要不要我。我说,你不讲去调查吗?你再可以去调查,你站在工人阶级立场上去调查,谁人不要你呢。这是我在六月初跟他谈的。所以有几个右派朋友是好,要交几个右派朋友,了解了解他们的心理状态。各界都要有朋友,左、中、右,都要有朋友,工人要有朋友,农民要有朋友,现在民主党派,大学教授,文学界,许多共产党的作家他没有工人的朋友,没有农民的朋友,这是很大一个缺点。我看要到那里找朋友,真正的朋友是在工人农民那个地方。在农民中你不要轻易去找富裕中农当朋友,那个刘介梅你不要找他当朋友,他代表富裕中农的,要找贫农、下中农、找老工人。老工人他辨别方向非常之清楚,贫农、下中农容易辨别方向。所以我看中国的事情好办,我从来不悲观的。我在二月二十九日不是谈了吗?乱不了。不是讲不怕乱吗?乱子可以变成好事。凡是放得彻底的,师范大学鬼叫一个时候,天下大乱,我看事情就好办了。

整风准备有四个阶段,刚才邓××同志也谈了一下,就是一个大鸣大放,一个反击右派,一个整风,一个整改,最后还有一个字,检查思想,学点马列主义,和风细雨,开小组会,搞点批评自我批评。五月一号整风文件讲和风细雨,许多人不赞成,要来一个急风暴雨,结果很有用处。这一点,我们当时也估计到了,因为我们延安那一次整风是那样,你讲和风细雨,结果要来急风暴雨的,但是最后还是要归结到和风细雨。一个工厂,大字报一贴,贴几千条,那个工厂当局也是很难受的,有那么十天左右的时间,有些人就不干了,车间主任就想辞职,就是受不了,吃不下饭,睡不着觉,那时候右派他们说,你们不能驳,只能他们鸣放,我们那时候也对他们讲,要让他们讲,不要驳,所以,五月我们不驳,在六月八号以前,我们一概不驳,充分鸣放出来了,大概有百分之九十以上的鸣放是正确的,有百分之几就是右派言论,在这个时候就是要硬着头皮,恐怕你们各民主党派现在也有了经验了,每个单位要经过这么一个阶段。每一个合作社,每个工厂,现在军队也是这样搞。明年搞不搞,明年再定,我们再商量,因为明年搞到五一,下半年再来搞一次,是不是有这么必要,但是只要你不搞,自由市场又要发展的,世界上有些事是那么怪的,三年不整风,共产党,青年团,民主党派,大学教授,中小学教员,新闻记者,工程师,科学界里头又要出许多怪议论,资本主义思想又要抬头。等于我们扫地一样,这个房子最好每天扫,我们洗脸一天总要洗一回,以后我看大体上一年搞一次,一次只个把月就行了,不会有现在这样的高峰,也许那时候要来一点洪峰。这个洪峰不是我们造成的,有些东西我们没有估计到。我们不是讲过吗?共产党里头出了高岗,你们民主党派一个高岗都没有呀?我就不信,现在共产党又出了丁玲、冯雪峰、江丰这么一些人。

整风的四个阶段,一个是放,一个是反,一个是改,一个是学,这样四个阶段,大概还要搞几个月。

要承认有改造的必要,右派就不承认有改造的必要,他们不承认还要改造,而且影响其他一些人也不愿意改造,说我们已经改造好了。章乃器说,那怎么得了,那叫做抽筋剥皮,你说要脱胎换骨,他说脱胎换骨会抽筋剥皮。我们中国许多人都忘记了我们的目的是干什么、为什么要这么搞?社会主义有什么好处?思想改造为什么要搞?就是为了建立无产阶级的观点,知识分子要变成无产阶级知识分子。这些老知识分子将来还得非变不可,因为新知识分子起来了。什么助教、讲师,他将来总有一天当教授,你讲学问,他说他现在不行,他将来也可以行的。工程师、科学家,新的人出来了,那么这就对老科学家、老工程师、老教授、老教员将了一军,非得前进不可。我们估计,大多数人是能够前进的,知识分子能够变成无产阶级知识分子。无产阶级必须造成自己的知识分子队伍,跟资产阶级造成他自己的知识分子队伍一样。一个阶级的政权,它没有知识分子那是不行的。美国没有那样一些知识分子,他资产阶级专政怎么能行?无产阶级专政,要造成无产阶级自己的知识分子,这要写一篇社论,把这个问题讲清楚。右派中间那些不变的,不愿意脱胎换骨的,大概章乃器就不能算,你要他变成无产阶级知识分手,他就不干,他说他早已变好了,说是红色资产阶级。我们说你还不行,你是白色,章乃器是白色资产阶级。先专后红,就是先白后红。他在这个时候不红,他要到将来再红,这个时候不红,你是什么颜色呀,你还不是白色,要同时是红的,又是专的。要红起来也很容易,并不困难,你就下一个决心,并不要读很多书,主要是着重在什么叫无产阶级,什么叫无产阶级专政,只有无产阶级有前途,其他阶级都是过渡阶级,我们这个国家要向什么方向去,他们不懂这个,我在四月三十号讲的那些,他们就听不进去,“皮之不存,毛将焉附?”我说中国有五张皮,旧有的三张,帝国主义所有制,官僚资本所有制,封建主义所有制,过去知识分子就靠这三张皮吃饭,此外还靠民族资本主义所有制,一个小生产者所有制。我们上一回民主革命,不过革那三张皮而已,从林则徐算起一直革了一百多年,社会主义革命革两张皮:民族资产阶级所有制同小生产者所有制(小资产阶级所有制)。这五张皮现在都不存了,老皮三张久已不在,新皮二张也不存了,现在剩下什么皮呢?是社会主义所有制这个公有制这个皮。当然这又分两部分,一个全民所有制,一个集体所有制,现在靠谁吃饭?民主党派也好,大学教授也好,科学家也好,新闻记者也好,是吃工人的饭,吃集体农民的饭,是吃全民所有制和集体所有制的饭,总起来是吃社会主义所有制的饭,吃公有制的饭。那个皮没有了。这个毛呢?现在就在天上飞,落下来也不扎实,他还看不起这个皮,什么无产阶级,贫农下中农,这个实在是太不高明了,上不知天文,下不知地理,三教九流都不如他。那个时候(四月三十日)我就劝大家,我打了一个比喻,我说这个事情也不容易,比如吃狗肉,我就有这个经验。从小不吃狗肉,我又没有吃过,但是我就反对吃狗肉:你没有吃过,为什么反对呀?你没有经验嘛!你说狗肉不好吃,你吃过没有呀?你何以见得狗肉不好吃呀?你吃都没有吃过你却到处发表意见,说狗肉不好吃,并且把狗肉送到你面前,你闻一闻就跑,这是因为社会上的舆论历来都是这么讲,狗肉古人是大吃特吃的,孟夫子的经济纲领就有一条:养狗。他说:“鸡豚狗彘之富,无失其时,七十可食其肉矣。”七十岁的人才可以吃肉,六十九岁不行,因为那个时候生产力很弱,只有那么多东西。马克思主义,因为过去反对的多,帝国主义反对,蒋委员长天天反,害得大家生怕这个东西。有一句话,说是“共产主义不适合中国国情”就是说狗肉不适合人民需要,这要一个过程,而且要有个运动。今年这个运动,就是开辟这条道路。

现在有些机关、学校,右派反过之后,风平浪静,提出来的许多意见就不肯改了,我看又要来一个鸣放高潮,把大字报贴一贴,将一军,这个将军很有作用。要改,要有一个短时期,比如一两个月。要学,这个学,当然不是一两个月了,只是讲这个运动告一个段落,这个,右派估计到了,他说,这个风潮总要过去就是了。很正确呀,你不能老反右派,年年反,天天反。比如现在北京这个反右派的空气就比铰不那么浓厚了,因为反的差不多,不过没有完结就是了。不要松劲。有些人死不投降,像罗隆基、章乃器就是死不投降,我看将来还说服他,说几次,他一定不服,你还天天开会呀?摆到那里,听他怎么办。我们采取不捉人,而且又不剥夺他的选举权的办法,对于这些人,给他们一个转弯的余地,分化他,那么一部分死硬派,他永远不肯改怎么办呢?那也就算了,他人数很少,摆到那里,摆他几十年,多数人要向前进,三个五年计划完成了,我们这个国家的面貌会有改变的。

现在我讲农业纲要四十条。经过两年的实践,基本还是那个目的,就是四、五、八,就是黄河以北四百斤,淮河以北五百斤,淮河以南八百斤,十二年要达到这个目的,这是基本之点,整个纲要基本上都没有改,但是有些东西改了,譬如有些东西已经解决了,合作化问题就是基本上解决了。还有,这个条文次序都有些改。还有,过去有些没有强调的,譬如机器、化学肥料这两个东西,过去没有强调,现在要大搞,要加以强调。过几天开一次人大常委同政协常委的联席会议,讨论一下。讨论后就可以登报,拿到整个农村中去讨论,工厂也可以讨论,各界、各民主党派也可以讨论,今年冬天或者十二月什么时候共产党要开全国党代表大会,那个时候再通过。这是共产党提出的,这是政治设计院设计出来的东西,就不是章伯钧的那个政治设计院。共产党要提到国务院,国务院再提到人民代表大会。今年冬天十二月或明年一月开一次全国人民代表大会,早一点通过明年的计划和预算,过去我们不是总搞在五六月吗?现在要改一下,要提早。同时把农业纲要在那个时候通过。发动全体农民讨论这个纲要很有必要,要鼓起一股劲来。现在已经开始有劲了。去年下半年、今年上半年开始泄了劲,现在整风又把这个劲鼓起来了。我向同志们说,这四十条以及工业计划,文教计划,完全有希望,比较还是适合中国国情的,不是主观主义的。这里头有些主观主义的,我们改掉就是了。比如农业纲要四十条里头有一个六百万双铧犁,这是主观主义的,我们现在把它去掉了。此外还有些修改,经过两年的实践嘛!但是总的说是有希望的。我们中国可以改造,无知识可以改造得有知识,不振作,可以改造得振作。

纲要里头有一个除四害:老鼠、麻雀、苍蝇、蚊子,我对这个东西很有兴趣,不晓得诸位如何?恐怕你们也是有兴趣的吧!有人说麻雀可以吃虫子,我看把它消灭,它与人争食。一方面它可以吃虫,但它也吃粮食。老鼠就没人赞成它,还有苍蝇、蚊子,是没有人赞成的,名誉不好,现在北京的苍蝇蚊子不很多,但是又有了。过去没有搞蚊子,去搞苍蝇麻雀了。这是一个大的清洁卫生运动,是一个破除迷信,把这几样东西也搞掉是不容易的,如果动员全国人民来搞,我看我们这个民族的精神就为之一振,我们要把这个民族振作起来。

三个五年计划搞到两千万吨钢,再有十年就可以了。今年就是五百二十万吨,我们五年增加三百多万吨,拿一九四九年算,只有十几万吨,恢复的三年(一九五○、五一、五二)就搞了一百多万吨,现在又搞了五年,就达到五百二十万吨,再过五年就可以超过一千万吨或者稍微多一点,可以达到一千一百五十万吨,然后,搞第三个五年计划,是不是可以达到两千万吨,跟打麻将一样,加它一番。

除四害,也是要求几年试点,大概要三年试点,五年突击,两年扫尾,十二年已经过去了两年,还剩下十年,如果在这方面搞出一点成绩来,人民的心理状态会变的。如果这个事情搞起来了,节制生育我看就有希望了。我看节制生育也是几年试点,几年突出,几年扫尾。这个事情也可以经过大辩论。

除四害要大搞大鸣、大放、大争、大字报,在农村里头,在城市里头,究竟灭得了灭不了苍蝇、蚊子、老鼠?我说我们这个国家是完全有希望的。右派说没有希望,那是不对的,完全错误的。他们没有信心,他们没有信心是有理由的,因为他们是不想搞这个事,那也当然没有信心。我们是想搞社会主义,我看是完全有希望的,包括灭掉老鼠、麻雀、苍蝇、蚊子,包括扫盲,包括有计划的生育,要做的事情很多,那四十条里头有好多事情。那仅是农业计划,还有工业计划,还有文教计划。

至于是不是把右派分子丢到海里去呢?我们一个也不丢。不是刚才讲了两种人了吗?一种是改正了,以后可以把右派分子帽子取消,归到人民的队伍,一种就是一直到见阎王,他说我是不投降的,阎王爷你看我是多么有骨气呀!这是资产阶级的忠臣,他不投降。右派是跟封建残余有联系的,通气的,他们并没有通信,没有开什么会,但是彼此呼应,那个文汇报,地主非常高兴,地主就买了报,对农民读,你看报上登载了的呀,他们就想倒算。所以封建残余,反革命残余跟右派,他们实际上是通气的,还有外国跟中国也是通气的。比如台湾、香港对储安平的党天下,章伯钧的政治设计院,罗隆基的平反委员会,是很拥护的。还有葛佩琦,最出名了。还有美国,他们很不赞成我们反右派,很同情右派。我曾经跟各位也讲过的,我说假如美国人打到北京,你们怎么办?采取什么态度?你们怎么做准备?还是跟它一起组织维持会?还是上山?我说,我的主意是上山,第一步到张家口,第二步就到延安,我不是在二月二十九号的前两天谈过那个话么?说这个话,是极而言之,把问题讲透不怕乱。你占领半个中国我也不怕,日本不是占领了大半个中国吗?然后不是搞出一个新中国来了呢?我们说感谢日本人,我跟日本人谈过,我说你们这个侵略对于我们很有好处,你们这个侵略,激发全民族反对你们,提高了觉悟。

此外还有几个文件,就是劳动工资,体制问题,这些会跟大家商量的,要提到人大常委。还有一个工资同劳保福利,这个也要商量的,恐怕先还要去试验一下,作为草案在工厂里头搞一个试点,然后才能够决定。

右派不讲老实话,我们讲老实话,他不老实,他瞒着我们搞一些事情。谁晓得章伯钧搞那么多事情,这种人我看官越做得高,反就越造得大,部长怎么样?部长恐怕也当不成了吧?右派当部长,人民恐怕不赞成吧?还有一个人民代表如何安排?右派如何安排他的工作?工作总是要有一点工作。这些著名的右派,人民代表恐怕难安排了。丁玲是不是选人民代表呀?人民代表不能选。明年要选举了,明年选举还要跟各位商量选举的名单。有些人,你一点职务也不要恐怕也不好。对右派的安排也是个问题。要好好考虑一下。有些教授,比如钱伟长,恐怕教授还可以当,副校长当不成了。有一些人,教授恐怕暂时也不当,学生不听。那么干什么事呢?可以在学校里头分配一点别的工作,让他有所改造,过几年再当教授。这些问题都要考虑,是一个麻烦问题。革命这个事情就是一个麻烦事情。

各民主党派什么情况,基层什么情况,恐怕你们这些人也不摸底,等于我们对于许多地方不摸底一样。这一次,百分之二的坚决的右翼分子,可以在一个时候把水搞得浑,使我们看不见底。一查,其实只有那么百分之二。一把明矾放下去,就看见了底。这个整风就是放一把明矾。大鸣、大放、大辩论之后,就看得见底了,农村看得见底。工厂看得见底,学校、党、团、民主党派,这些都有底。章乃器这些人,你要长期共存,他是短期共存;互相监督,他是不相监督,章罗联盟,对长期共存这个口号最高兴了。一个长期共存,互相监督,一个百花齐放,一个百家争鸣,他们最喜欢了,结果就走到反面:长期共存变成短期共存。对右派的处理问题,请诸位去考虑一下,议一下,如何处理法。

今天是个通知性质的会议。请大家回去研究一下:整风问题,农业纲要。农业纲要会发给大家的。


