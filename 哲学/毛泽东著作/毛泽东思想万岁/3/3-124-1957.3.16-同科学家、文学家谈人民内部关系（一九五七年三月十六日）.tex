\section[同科学家、文学家谈人民内部关系(一九五七年三月十六日)]{同科学家、文学家谈人民内部关系}
\datesubtitle{(一九五七年三月十六日)}


对王蒙测查一下,青年人才开始写小说有缺点,要帮助,对起义将领也是帮助。现在有许多作家,和胡风有区别。应又保护又批评,在保护下批评。写正面人物无力,写反面人物生动,生活上的原因,观点上的原因,也不单是王蒙的问题。有同志批评中央附近不能有官僚主义,这个观点不对。为什么中央附近不能产生官僚主义,中央内部也出了坏人。如果照这些同志的观点,写这种文章的人要剐肉。总之,我们对待人民中的错误,采取如何处理的方针,在许多同志中未弄清楚。实际上许多人是反对的。“从团结的愿望出发,经过批评,以便在新的基础上,达到新的团结。”批评这这个东西,说地点不对,时间不对,这个看法是错误的,没有说服力,党里面经常有不正常的东西,在全国有很大的成绩和威信之下,滋长了一些坏的作风。这是违法乱纪,横行霸道。我们的国家是一个小资产阶级的大王国,真正的无产阶级不到一千二百万,还包括有真假,为什么说共产党的缺点不能揭发,这种概念就不对。党的统一战线,党内实际上赞成的还是少数。再三讲过要“统筹兼顾,各得其所。”有些同志不从帮助出发,一脚踢开。这种办法很简单,是简单开枪,这是国民党作风。社会上还有一部分地主、富农、资产阶级,最大量是小资产阶级,这是客观现实。合作社主要的一部分富农中农不满,学校里面百分之八十出身小资产阶级以上,但多数人都不搞匈牙利。我们有许多同志就怕匈牙利事件,事物有两重性,任何事情都有两种属性。匈牙利事件又好又不好,不好的事反过来就是好的。用片面性反对片面性,形而上学教条主义的方法。王蒙小说未写好,需要帮助他在实践中改正,李希凡开始写文章是好的,后来的东西无多特色,是否到人民日报脱离了生活,应生活在实际生活中。当小媳妇时好,当了婆婆就不行了,用教条主义方法不能批评人家,因无力量,请看一下列宁是如何写《经验批判论》的。斯大林在后来就不同了,不是平等讨论问题,搜集大量材料发表意见。有些东西写得好,有些东西是坐在山岗上,拣起石头打人,使人看了后不大舒服。当了权,作了官,要警惕,不要骂人像骂儿子一样,不应是老人和小人的关系,不能以片面性反对片面性,教条主义不是马列主义。开始批判胡适有好文章的,以后一笔抹杀。过一个时期后,可补救补救。康梁也不能全部抹杀。自比有史以来的六大政治家。

就是这样的人,思想上面有严重错误、有敌对思想的人,也要团结他们改造他们,才能贯彻统筹兼顾,各得其所的方针。写文章也要安排,大喝一声,笔下留人,要有说服力,应该坚决贯彻“惩前毖后,治病救人”的方针,笔下也要如此,党内百分之五、六十不了解这个方针,这么多小资产阶级好不好?要靠他们搞社会主义,要用适当的方法使他们得到改造。就是武天保,也不开除他。我们不提倡全国性的罢工、罢课来对待政府,要照顾到历史情况,我们的人民用这个对敌斗争的方法用惯了。宪法是也有规定这种自由。闹事不等于造反,六万万人口每年有一百万人闹事,我看来是正常情况,百分之一的人闹事可能是永远的,至少在三个五年计划以内。闹事的人可能也有反革命,不能说都是反革命。闹事中不合理不应答应,合理的应该答应。凡是合理的东西应该和群众站在一起。否则就要脱离群众。“你们都是公家人”,人民没有什么武器.,拳头和扁担。此起彼落是完全可能的,是正常现象,资产阶级也允许闹事,为什么我们不允许。

我们的矛盾是临时性的,但是矛盾不解决,隐伏下来不好,冻疮割了好。不能说我们是老革命就不许闹事,国民党也是老革命。

马列主义不能从真空中生长出来,只能从向敌对思想作斗争中,并从其中吸取合理的东西,才能生长发展。

我们的危险在于革命成功、四方无事。片面性的打不能锻炼出好的文学艺术。只允许香花,不允许毒草,这个观念是不对的。香花是和毒草作斗争中成长的,田里面年年都有野草,野草一翻过来,就是肥料,粮食和野草一起生长,“落霞与孤鹜齐飞”。不能只允许粮食。应该是香花和毒草齐放。

斯大林基本上是唯物论,也有辩证法,但辩证法不是那样多。教条主义,不是两点论,而是一点论。我们同志看事物,应是两点,一点里面又有两点,最近有个姚文元写了两篇文章(文汇报二月六日):教条与原则。我看不能。过去提倡一家独鸣,历史条件限制,不如此不能打倒二十年的一家独鸣。

我们要写东西,要搜集充分材料,不打无准备之仗,不打无把握之仗。

有这么多出身小资产阶级的人,就必须教育,靠少数人教育多数人,开除很容易,很简单,这决不是好办法。应该懂得,坏分子也有两重性,一是不好,一是可以当教员,一切事物都有两重性,产生、发展和灭亡。也许一千年以后,就不叫马列主义了。如果永恒,就不叫马列主义了。五百年必有王者兴。(三个留学生推翻永恒论,推动原子能来个新的革新)

有些东西能推翻,有些东西不能推翻。如地球总是转的。人类总有一天要自己否定自己,被否定掉,总有不适应的一天,变化成为另外一种人,现在人类本身就经过了多少变化。三千年以前,新石器时代,铜器时代(纯、青),后来又铁(旧、新)。动物可能反对我们,不会讲话;植物可能感谢我们,互相依赖。马克思未看到社会主义,列宁也看到不久,社会主义的时间还不长,还无充分经验,苏联有成绩,也有问题,要不断积累经验。

许多问题要创造,要发展,不要怕歪风,不要怕潮流,没有两个流就不行,需要交锋,越斗就越丰富,真理越辩越明。我们应特别想到,掌握政权之后,不能用简单办法把人打倒。百花齐放为什么怕放,怕饭甄子过河。青年人反对官僚主义不那么可怕,因他们还没有当事,官僚主义还未轮到他们头上。

大官僚主义也可以反,凡是怕骂官僚主义的人,就是怕骂自己。

哲学,杜林,黑格尔,马赫的著作能否找到看一看。《新建设》周谷城论大逻辑,有点道理。学校只听校长意见,不听学生意见不对。


