\section[给江西共产主义劳动大学的一封信(一九六一年七月三十日)]{给江西共产主义劳动大学的一封信}
\datesubtitle{(一九六一年七月三十日)}

{\noindent 同志们:}

你们的事业我是完全赞成的。半工半读,勤工俭学,不要国家一文钱,小学、中学、大学都有,分散在全省各个山头,少数在平地。这样的学校确是很好的。在校的青年居多,也有一部分中年干部。我希望不但在江西有这样的学校,各省也应有这样的学校。各省应派有能力有见识的负责同志到江西来考察,吸收经验,回去试办。初时学生宜少,逐渐增多,至江西这样有五万人之多。再则党、政、民(工、青、妇)机关,也要办学校,半工半学。不过同江西这类的半工半学不同。江西的工,是农业、林业、牧业这一类的工,学是农、林、牧这一类的学。而党、政、民机关的工,则是党、政、民机关的工,学是文化科学、时事、马列主义理论这样一些学,所以两者是不同的。中央机关已办的两个学校,一个是中央警卫团的,办了六、七年了,战士、干部们从初识文字进小学,然后进中学,然后进大学,一九六零年他们已进大学部门了。他们很高兴,写了一封信给我,这封信,可以印给你们看一看。另一个是去年(一九六零年)办起的。是中南海的各种机关办的,同样是半工半读。工是机关的工,无非是机要人员、生活服务人员、招待人员、医务人员、保卫人员及其他人员。警卫团是军队,他们也有警卫职务,即是站岗守卫,这是他们的工。他们还有严格的军事训练。这些,与文职机关的学校是不同的。

一九六一年八月一日,江西共产主义劳动大学三周年纪念,主持者要我写几个字。这是一件大事,因此为他们写了如上的一些话。

{\raggedleft 毛泽东\\一九六一年七月三十日\par}


