\section[中国共产党中央委员会关于南斯拉夫共产党问题的决议(一九四八年七月十日)]{中国共产党中央委员会关于南斯拉夫共产党问题的决议}
\datesubtitle{(一九四八年七月十日)}


(一)中国共产党中央委员会完全同意由保、罗、匈、波、苏、德、捷、意各国共产党所参加的情报局会议关于南斯拉夫共产党问题所通过的决议。举行这个会议并通过这个决议,乃是国际共产主义者为保卫马克思列宁主义的原则,保卫世界工人阶级和各国人民的革命事业,所应尽的职责,乃是他们为保卫世界和平民主事业,保卫南斯拉夫人民免受美帝国主义的愚弄和侵略,所应尽的职责。

(二)以铁托、卡德尔、德热拉斯、兰科维奇为代表的南斯拉夫共产党的领导集团,在其对内对外的背叛性的和错误的行动中,违反了马克思列宁主义的基本观点,例如社会主义国家与资本主义国家的原则区别,国际援助对于各国革命运动的必要性和重要性,阶级斗争,和无产阶级对于人民革命事业的领导作用,党是阶级的最高组织形式,党的民主集中制,党员批评和自我批评的作用等,从而陷入资产阶级民族主义和资产阶级政党的泥坑。铁托集团因为它执行反马克思列宁主义的内外政策,因为它釆取反苏立场,压制党内批评,拒绝苏联共产党和其他共产党的兄弟批评,拒绝参加情报局会议,并在情报局会议的决议公布以后,继续压制南斯拉夫党内外的正确意见,继续敌视国际共产主义,已经严重地损害了南斯拉夫的人民事业,并使南斯拉夫的敌人欢呼。中国共产党热烈希望南斯拉夫共产党内的国际主义分子能够坚决地起来纠正铁托集团的错误,使南斯拉夫共产党重新走上马克思列宁主义的轨道,走上无产阶级国际主义的轨道。

(三)中国共产党中央委员会认为:南斯拉夫党内所发生的事件,不是偶然的和孤立的现象,这是阶级斗争在无产阶级革命队伍中的反映。只要是阶级存在的国家,带着资产阶级反革命观点的投机分子,总是企图混入无产阶级的革命队伍,混入共产党,企图利用机会从内部来破坏革命事业。这种情况曾经在国际共产主义运动中多次发生;在中国共产党内则曾经表现为陈独秀主义和张国焘主义。这种情况,要求共产党人努力提高觉悟,加强马克思列宁主义理论的教育,以便及时地识别和反对这些资产阶级分子,保卫无产阶级和人民事业不受破坏和损失,保卫共产党在思想上和政治上的纯洁。为此目的,中国共产党中央委员会决××××××××××××××××都应当认真研究共产党情报局会议关于南斯拉夫共产党问题的决议,借以加强党内关于阶级的、党的、国际主义的、自我批评精神和纪律性的教育。

