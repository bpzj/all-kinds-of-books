\section[接见日本文学代表团时的谈话(一九六○年六月二十一日)]{接见日本文学代表团时的谈话(一九六○年六月二十一日)}


主席:非常欢迎你们。对日本人民的英勇斗争感到很高兴。你们的斗争对中国,对世界人民都是一个支持,你们斗争的对象是世界上最大的帝国主义。这个国家曾经控制制着中国,援助蒋介石打内战,现在占领着我们的台湾。在日本、菲律宾、南朝鲜、台湾都有军事基地,实际上占领的还有南越,巴基斯坦以西还有许多国家就不讲了,这是我们的共同敌人。去年日本社会党领袖浅沼稻次郎访华时,和张奚若发表了一个联合声明,说美帝国主义是中日两国的共同敌人。当时一部分人认为这种说法太过火。现在日本人民的斗争大大超过了去年的这种说法,斗争的范围和规模之大,是去年所没有想到的。这次斗争是从反对“安全条约”爆发的,其基本性质是反对美帝国主义和他的走狗岸信介,要求民族独立和民主,因为条约是日本反动派在众议院强行通过的。就是说,日本革命的性质是民族民主革命。工人罢工不是提经济口号,而都是提政治口号,是世界上少见的。而且有高级知识分子参加,如东京大学校长茅诚司在“六一五”惨案发生的第三天,就召开了全校大会,率领大家上街示威游行。牺牲者是东京大学学生,叫桦美智子,现在在全世界闻名。她父亲叫桦俊雄,是中央大学教授,专政法律。好像有好几千的教授都组织起来了,妇女也赶上去了,还有和尚,宗教界也出来了,工人、学生是主力。明天还要有更大的规模的罢工。

你们对你们国内发生的事情有什么意见没有?

野间宏(团长):今天能见到毛主席很感动。我们是元月四日到达北京的,在北京车站发表声明时就说:我们要在北京参加日本人民的斗争。当天晚上,第二天早晨听到日本有五百四十万人参加总罢工的消息后,很感动。这表明日本人民争取独立的斗争走向新的道路。在日本,最初时斗争性质是反帝这一点,还不明确,“六一五”事件后,日本人民阻止了艾森豪威尔访日,自觉地认识到斗争是反美的,日本民族反美力量团结起来了。斗争不会停留,将会继续前进。主席:好!这样就好办了。日本有军事基地,过去对它没有办法,苦恼,又不能去打它。现在你们日本人民想出来一个好办法,就是全民性的群众斗争,除了美帝国主义和它的走狗以外,其他所有的力量都要团结起来,反对帝国主义和它的走狗。中国过去基本上也是用这种办法,中国过去还有武装斗争。但“五四”运动时并没有武装斗争。一九一九年也是反对巴黎和会的条约,这是第一次世界大战后的事,当时还没有中国共产党。

两年后,一九二一年党才诞生,开始人数很少,几十个人,是马列主义小组。以后有北伐战争,是一九二六年,那时和国民党合作,这段历史你们都很清楚。一九二七年北伐到长江一带,蒋介石反共,逼着我们打内战,我们没有准备,突然遭受袭击,因为党内有右倾机会主义分子陈独秀。中国地方大,打了十年内战,以后又与日本军阀打仗,又和蒋介石合作。我与很多日本朋友讲过这段事情。其中一部分人说日本侵略中国不好。我说侵略当然不好,但不能单看这坏的一面。我说日本帮了我们中国的大忙。假如日本不占领大半个中国,中国人民不会觉醒起来。在这一点上我们要感谢日本“皇军”。但是你们现在没有负担了,因为你们没有殖民地,相反地变成殖民地和半殖民地。从有军事基地一点来说是殖民地,但是你们还有个独立政府,这个政府被美国支配着。从这个意义来说,又是一个半殖民地。你们现在不欠账了,相反地外国人欠你们的账。这个外国人就是美国,而不是英国、法国。所以日本人民现在愤怒起来了。我同许多日本朋友谈过,我不相信像日本这样伟大的民族会长期受人家统治。现在谁在教育日本人民?是美国人做你们的反面教员,同时也是我们的反面教员。一九四五年以后,中国的事情和你们没有关系。欺侮中国,帮助蒋介石打内战的是美帝国主义,而不是日本。所以我们的仇恨目标转移了,不是日本,而是转到美帝国主义身上来了。相反地,我们两大民族有合作的可能性,也有此必要,因为都受美帝国主义压迫,有共同立场。现在压迫中日两国人民的是美国。除了美国以外还有谁呢?是英国吗?过去是,后来美国取英国而代之。法国在中国过去也有势力范围,但二次大战开始时就没有了,美国就代替了英、法。

你们被压迫的历史不长,我们很长,有一百多年。但是你们的工业、经济、文化比我们中国发达。我们是落后国家,现在你们还可以看到落后的遗迹。我们受高等教育的人数,按人口比例比你们少。你们是否普及中学教育了?(团员龟井胜一郎说:战后是这样,普及了初中教育。)我们还没有,再过几年才能赶上你们。当然社会制度是不同的。你们有没有民族资产阶级的问题,即除大资本家外,和外国资本联系较少或没有联系的资产阶层?(野间、龟井说:有。)对这个阶层,你们要团结。如果有兴趣,你们可以找上海的资本家谈谈。

野间:在日本就曾研究过这个问题。有,有兴趣,很想和资本家谈谈。

主席:柯庆施同志是上海市长,也是上海市党的第一书记和中央政治局委员,你们有事情可以找他。

柯庆施:我可以组织这个谈话。

主席:(问过大家年龄后)你们都比我年轻。世界上大多数事情都是年轻的、比较不出名的、地位比较低的、财富比较少的人作出来的。比如英国发明蒸汽机的瓦特,就是工人出身,你们总可以找到很多这样的例子。一九五八年我们召开党代会时,曾经谈过这件事,后来作过调查,调查全世界在三百年以内搞发明创造的都是一些什么人,调查整理出来的结果,百分之七十都是不出名的、年轻的、地位比较低的、比较穷的。不知你们日本情况如何?难道好事情都是老头子、做大官的做的吗?我就不相信。包围哈格蒂,赶走艾森豪威尔的也是那些年轻人。我在《中国青年报》上看到过竹内写的一篇短文章,写得很好,是青年报请你写的吧?

竹内:(用中文):我平常喜欢读毛主席的文章,今天蒙老师的夸奖,我很荣幸。

主席:你们在中国还会呆一个时间吧?

野间:到中国已经二十二天了,还剩下十天左右。

西园寺:我还要再呆几年。

主席:你回不去了吗?

西园寺:因为说了岸信介很多的坏话,回去怕回不来。

主席:岸信介倒台后你再回去。

西园寺:岸信介倒了,还会有第二个岸信介出来。我在北京,也可以参加日本人民的斗争。

主席:我们总是同你们在一起的,总是同要求独立、民主、自由的日本人民在一道的,同岸信介不是在一道。世界上的事情在变化,变化得特别快。如四、五年前我见过许多日本朋友,一提到美国的事情,他们都不开腔。我看那时日本朋友是在想问题,听我们说的话,他们不反对,愿意听下去,不替美国辩护。现在情况变化了,因为日本人民在日本各地做起来,写标语来反对“安全条约”,要求取消军事基地,撤回U一2型飞机,美国佬滚回去。冲绳人民起来当面质问美国人:“你们究竟还要占领多久?”大家都谈开了,作起来了。日本的各阶层人民都行动起来了,有几百万人,这是四、五年前所不能设想的。我看日本的独立和自由是有希望的。把美国军事基地取消,“安全条约”取消,日本的永久和平是有保障的,亚洲和平也有保障。祝贺你们取得的胜利。

胜利是逐步得来的。群众觉悟也是逐步提高的,包括我们在内,也是逐步觉悟起来的。我自己也是如此。在中学读书并不知马列主义。我读的书有两个阶段,先是读私塾,是孔夫子那一套,是封建主义。后来进学校,读的是资本主义,信过康德的哲学。后来是客观环境逼得我同周围的人组织共产主义小组,研究马列主义。周总理也是这样。因为我们当时一则没有钱进大学,二则也读不下去了。我读的中等师范学校,是准备当教员的。我做过小学教员,也做过校长,那时一心想当教员,并没有想当共产党。后来反对军阀,受《新青年》的影响,《新青年》开始并不是共产党杂志。后来教员当不下去了,逼得我搞学生运动、工人运动,那时开始有共产党,这是一九一九、一九二○、一九二一年的事情。周总理也是念不下去,跑到日本住一年,回来搞“五四”运动,军阀要抓他,又跑到法国去搞勤工俭学,开始给报馆(申报)写稿子,以后又搞共产主义小组。法国在第一次世界大战后,要恢复工业,人少,招了很多华工,去帮助恢复工业。一九二七年大革命失败后,又有很多人跑到莫斯科,进中山大学念书。中国是经过很多曲折道路才成功的。

一八四○年发生鸦片战争,一八六三年到一八七五年太平天国,经过十三年失败了。一八九八年康有为、梁启超参加的戊戌政变,也失败了。以后有许多人跑到日本,有一两万人。一九○六年孙中山领导的同盟会就是在日本成立的。一九一一年的辛亥革命也失败了,袁世凯要作皇帝,接着军阀混战,一九一九年“五四”运动,一九二一年党成立,一九二三年“二七”大罢工,一九二五年“五卅”惨案,反对英帝国主义。一九二四年国共合作,在广州召开大会。一九二六年北伐,一九二七年蒋介石叛变,我们转到地下,开始打游击。从一九二七年到一九三七年,到一九四八年、一九四九年,共计打了二十二年仗,包括土地革命、抗日战争和解放战争。这时对外国的目标改变了,过去是日本帝国主义,现在变成美帝国主义及其走狗蒋介石了。到一九四九年告一段落。新中国成立到去年是十年。这十年我们干的是社会主义革命和社会主义建设,就是你们现在看到的事情。

这十年里有些成绩,究竟是时间短,成绩并不多。拿钢来说,去年才××××万吨,你们是××××多万吨。其他工业也有些发展,按人口平均很低,比你们差多了。

我给你们讲了这么多的历史,都是自己亲身经过的,说明中国人民是逐步觉悟起来的。我们这辈子人也是逐步觉悟起来的。你们也会逐步觉悟起来的。方才我说过,有些日本人四、五年前不敢讲美帝国主义,但是去年浅沼敢与张奚若发表共同声明,说美帝国主义是中日两国人民的共同敌人。过了一年,日本人民就掀起了这样大规模的反对“安全条约”的斗争,应该说进步很快。你们会比我们搞得快。我们搞了一百多年,从一八四○年到一九四九年共计是一百零九年。现在“安全条约”还没有反掉,反掉的是艾森豪威尔访日。“安全条约”现在还存在,但是会反掉的。可能还要
有一段时间,也不好说是哪年哪月可以反掉,但总会反掉的。当然我并不是主张你们同美国开仗,可以不采取打仗的办法达到目的,别的地方还没有先例,也许你们会创造先例。先例也有,如一百八十年前,美国是英国的殖民地,华盛顿把英国赶走了,他是采取战争的办法。印度独立并没有打仗,英国人允许了印度独立。你们可以找到适当的办法,看来你们已经找到了办法,就是现在这个办法。成立一个“阻止安全条约国民会议”这个机构,斗争是有领导有组织的,这个机构包括一百多个团体,我们中国

过去没有像你们这样。

总理:这样的机构中国过去没有,中国有个各界联合会,但搞了一下就垮台了,你们搞了十八次统一行动。

主席:十天后你们回到日本,日本的斗争还在继续。你们过去没有来过中国的可能不熟悉,呆下去就熟了,你们会知道中国人民对你们是友好的。

野间:对这一点我们很了解,全团都很感谢。

主席;我们互相支援,互相学习,学习彼此的长处。关于马列主义的传播,你们比我们早。我们最先是从日本得到的。你们日本有个教授叫河上肇,他的政治经济学到现在还是我们的参考书之一。河上说,他的马克思主义政治经济学每年都修改,修改了多少次。这个人现在是否不在了?

野间:××从中国回去后,他才死的。他还写过一首诗欢迎××回日本。

主席:你们还有什么意见?

野间:听了你的有益的谈话,很感谢。有一个问题,可否问一下?

主席:可以。

野间:我曾反复读过毛主席的著作,特别是哲学著作,毛主席是否还想写新的哲学著作?

主席:没有时间。如果有时间想写一点。《矛盾论》已经很早了,想把从那以后这段中国革命经验总结一下。

野间:希望你再完成一本哲学著作。

主席:现在精力比过去差了。我比××小一岁,去年十月国庆节我问他,他是一八九二年生,我是一八九三年生。

野间:但是看不出毛主席年岁有那样大,看起来和我们一样年青,就是在世界上有名这一点同我们不一样。

主席:年岁大了,所以创造性也减少了,这是自然的规律。今天就谈到这里吧!


