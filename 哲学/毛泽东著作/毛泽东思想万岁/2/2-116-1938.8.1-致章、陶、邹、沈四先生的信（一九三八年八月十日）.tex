\section[致章、陶、邹、沈四先生的信(一九三八年八月十日)]{致章、陶、邹、沈四先生的信}
\datesubtitle{(一九三八年八月十日)}


为了达到战胜强大敌人的目的,不仅需要我们自己的发展与胜利,而且需要一切联合力量的胜利与发展。我们统一战线的口号是“各政党各阶级在抗日救国旗帜之下团结起来”。因此我们认为在统一战线中提出推翻某一阶级和某一政党的口号是错误的。……

……由于抗日战争是长期的,整个抗日民族统一战线也能够且必须是长期的……

所谓长期合作,不但是在战争中的,而且是在战争后的。抗日战争是长期的,战争中的合作已经算得是长期的了,但是还不够,我们希望继续合作下去,也一定要合作下去。……没有疑义,战争中的合作必有各阶级的内容,战争后的合作将更有新的内客。然而战争中的合作,将决定着战争后也能合作,这不是没有根据的预断。

………我们相信,如果为了抗日救国的需要,排除一切相互的敌意,互相忍耐,互相尊重,那末全民统一战线可胜利完成,并可保证广泛光明的前途。依靠着统一战线力量发展的程度不只是可以战胜日本帝国主义和卖国贼,而且经过统一战线运动的一定阶级可以使中国民族脱离一切帝国主义的束缚,并达到全中国的真正民主的统一。因此,我们认为统一战线决不是一个很短的暂时的现象。当然,在统一战线的各个阶段上可以有个别的人发生动摇,叛变或逃走,但是这决不能认为是统一战线的破裂。

现在统一战线比一九二七年还有更明白的前提和更巩固的基础,因为现在民族危机比一九二七年百倍的加深。一九二七年的统一战线主要是反对内部敌人(反对北洋军阀),现在统一战线是反对外部敌人。在一九二七年的时候脱离统线的人们,还能组织半独立式的政府。而在现在,那一个要脱离统一战线或者不参加统一战线,他再也不能建立半独立式的政府,并不能在中国人民各阶层中找到拥护。……我们为了统一战线的事业,不害怕被其它党派利用,因为我们考虑过坚决抗日纲领。我们愿意与一切抗日讨贼的政派组织合作到底。

但是,现在最危险的是有些人幻想实现武力统一的计划。……不幸有些人对于这样的政策,还采取袖手旁观的态度,还有人企图以另一“集中战线”来对立和破坏中国人民的统一战线,这就是统一战线不能顺利发展的主要原因之一。

<p align="right">(《论抗日救国统一战缤》一九三八年印行,第47—51页)</p>

