\section[关于批评彭德怀一九四三年四月“关于民主教育谈话”的一封信(一九四三年六月六日)]{关于批评彭德怀一九四三年四月“关于民主教育谈话”的一封信(一九四三年六月六日)}
\datesubtitle{(一九四三年四月)}


你在二个月前发表的“关于民主教育谈话”我们觉得不妥。兹将我的意见列下:

例如谈话从民主自由平等博爱的定义出发,而不从当前抗日斗争的政策需要出发。又如谈话不强调民主是为着抗日的,而强调是为着反封建,不说言论出版自由是为着发动人民的抗日积极性与争取并保障人民的政治经济权利,而说是从思想、自由的原则出发。又如不说结社自由是为着争取抗日胜利与人民政治经济权利,而说是为着人类互相团结与有利于文化科学发展。又如没有说汉奸与破坏抗日团结分子应剥夺其居住迁移、通讯和其它任何政治自由,而只笼统说人民不应该受任何干涉。其他现在各个根据地的民主自由对某部分人是太大,太无限制,而不是太少、太小与太有限制。故中央曾在去年十一舟公布关于宽大政策的解释,强调镇压反革命分子的必要。你在谈话中采取此种方针。又如在各根据地上提倡实行复决权,不但不对而且是做不到的。如又在法律上,决不应有不平等规定,而未将革命与不革命加以区别。又如在政治上提出己所不欲,勿施于人的口号是不适当的。现在的任务,是用战争及其他政治手段打倒敌人,现在的社会基础是商品经济,这两者都是己所不欲而施于人。只有在阶级消灭以后,才能(实现)己所不欲,勿施于人的原则,消灭战争、政治压迫与经济剥削。目前中国各阶级间,有一种为着打倒共同敌人的互助,但是不仅在经济上没有废止剥削,而且在政治上没有废止压迫(例如反共等)。我们应该提出限制剥削与限制压迫的要求,并且强调抗日,但不应该提出一般的,绝对的阶级互助(己所不欲,勿施于人)的口号。又如说西欧民主运动是从工人减少时间开始,并不符合事实等。

你前所说的“党内生活”已收到,候研究后如有意见再告你。

<p align="right">毛泽东

<p align="right">六月六日</p>

