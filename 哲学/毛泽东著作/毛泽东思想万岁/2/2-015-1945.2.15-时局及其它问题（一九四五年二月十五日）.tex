时局及其它问题

(一九四五年二月十五日)

一九四五年二月十五日在党校演讲



同志们,今天这个会是五部六部发起的,五部六部的同志们以及还有许多同志到延安很久了,有些来了两三年了,我一次讲也没有讲,我做得不对。今天就是专为五部六部的同志们开这样的会。在外还有许多同志参加。许多同志要求看见中央同志,要求看见从前方回来的同志,今天有许多同志都到了,将来我们还要请他们讲话,很快就要开七大了,很忙,七大以后,就有了工夫多跟同志们讲。我是这个学校的校长,过去没有尽到职务。(笑声)以后我应该多来讲。

今天讲的题目也是同志们出给我们:一个时局问题,一个山头主义问题,一个审查干部问题,此外,我还要想讲一点其它问题。

大家关心时局,很好。所谓时局,有全世界的时局,有我们中国的时局。今天全世界的主要问题是什么?几个大国家是不是能够团结到底消灭敌人一法西斯。中国的问题是什么?中国是不是能够团结起来消灭日本帝国主义。

关于头一个问题,有些报纸上,特别是敌人方面经常散布谣言,说美国、英国、苏联这三个大国是不能团结的。说他们这个团结要破坏的。(从前就有过纠纷,现在还有很多问题没有解决),说他们中间有纠纷、有矛盾、有斗争。他们中间有没有纠纷呢?有纠纷的,正在开会讨论,将来还会有,所有一切问题中间,有很多问题是能够得到相互同意的,能够和气解决的;也有一部分问题是要经过很长时间,要拖一下,有些争论。法西斯主义者,他们就希望三大国不团结。还有一部分人(我们中国也有这样的人)就是反苏反共的人,他们早就希望英美两国,有一天打倒苏联。这样的人还不少,英美里头也有许多人是反苏反共的,反对苏联,反对他们那个国家的共产党革命,也反对别的国家的共产党的革命。那么这个问题的情形到底怎么样呢?我可以说,磨擦是有的,但是团结占主要地位,是统治的地位,这几个大国订立了条约,同盟共同反对法西斯以来,好些问题都能解决,现在又在黑海开会,我们估计有问题是可以解决的,他们能够团结。并且我们估计不但在战争时期而且在战争以后也是能够团结的,能够保持很长时间的和平。什么理由?这个理由在现在就是有一个共同的敌人,打这个敌人是要各国人民都来打的。苏联是大指头,是主力。你们看欧洲打法西斯,英美两国只出了十个师,苏联出了两百多个师。苏联出了很大的力量。没有苏联就不能胜利。没有英美能不能胜利,当然也不能胜利。没有英美比如欧洲第二战场拿出这八十个师,也没有意大利战场,当然也不能够胜利。但是没有苏联那就更不能胜利。所以苏联是很大的力量,是决定一切的力量,是少不了的。英美两国人民的觉悟,也比过去不同了,现在觉悟起来了,参加过看见过第一次世界大战的人――现在四十岁以上的人,在座也有这样的人就看见过第一次世界大战。我们这个国家那个时候的政府一一北洋政府就参加了第一次大战。那个时候全世界人民的觉悟程度怎样?那此现在就差得多,英美两国人民的觉悟,欧洲人民的觉悟,中图人民的觉悟,无产阶级,无产阶级的政党,比如共产党就没有,只有一个国家有,就是俄国有共产党,那个时候不叫共产党,叫作社会民主党。此外,布尔什维克的党,其它国家都没有,只有少数的个别的人,比如德国有一部分,其它各国如法国,也多多少少有少数的人或小组活动,他们是与列宁、斯大林意见相同的,其它无产阶级的政党、工人的政党的意见都是与列宁,斯大林的意见不相同的。他们拥护那次战争,因为那此战争是帝国主义战争,是反动战争,所以列宁反对那次战争。各国无产阶级拥护那次战争就可以看出那个时候无产阶级觉悟程度怎么样,就是说觉悟程度很低,无产阶级尚且如此,其它劳动人民和知识分子就更不用说了。我们中国那时候怎么样呢?我讲同志们今天你们很幸福,我们那个时候中国社会所给我们的知识很少很少,什么叫做帝国主义也不知道,什么畔共产党也不知道,世界上有没有马克思好像也没有(笑声)连马克思的名字也没听到讲过。一九一四年至一九一七年这几年中,中国人民就在那么一个中国的情况下,至于一九一四年以前就更不用说了,参加过辛亥革命的人,比如总司令、董老、边区政府的林主席,吴玉章同志、徐特立同志,他们这些老同志都参加过。那个时候幼稚得很,有没有党校?(笑声)那就没有,(笑声)像我们这样的党校那里会有?有没有共产党?没有!影子都没有。有没有红军、八路军?没有!一切都没有。第一次世界大战的时候,一切都没有,中国没有,英国没有、美国没有,法国没有,德国没有,各国都没有,只有俄国有,但是俄国虽有,为数也很少。一九一四年开战,打到一九一七年,俄国布尔什维克只有几万人,只有几万共产党员。从一九一七年到今年,三十年当中,我们中国有了共产党,中国现在共产党有多少呢?有一百多万党员。三十年以前,俄国只有几万党员,还是受压迫的,还没有胜利,一九一七年十月革命才胜利。从此在世界上就区别为两个时代,即一九一七年十月革命以前、以后。到一九一七年的时候,马克思主义在全世界流行了,一八四三年到一九一七年,七十四年。马克思主义在全世界流行了七十四年了。但我们中国人还不知道。十月革命头一天爆发,第二天中国人民就知道了,你们看只有两天。七十四年不知道,两天就知道了(笑声),同志们,迭就是行动,行动比较语言,比较文章走得快,当然没有语言,没有文章也是没有行动,十月革命也不是天上掉下来的,是七十四年马克思主义运动的结果。世界大战线束以后,一九二一年中国共产党的党派才发生,在那以前,自从盘古开天地,三皇五帝到如今都没有,从一九二一年到今年二十四年了,这二十四年中起了变化,而且变得很快,这二十四年比较我们古代的历史,超过了二千四百年或者比二千四百年还多些,三千年,四千年都超过了,现在中国人民提高了,我看这是很好的。欧洲人民,特别是这几年来法西斯到处在杀人,欧洲人民的觉悟也提高了,美国人民的觉悟提高了,英国人民的觉悟也提高了,他们参加过这样的大战,需要这样大的精力,他们的觉悟就提高了,并且还在继续提高,现在一月就抵得上历史时代很多年,过几个月之后,比如讲打到柏林,比如讲把德国法西斯打倒了,那对全世界的影响会更大的,这一点你可以看得到。就是苏联的力量这个原因,使英美的人民觉悟了;欧洲人民觉悟了,我们东方中国人民也觉悟了,觉悟提高了,因为人民一致更求打倒法西斯的结果,各国政府就不得不起来打倒法西斯。那一般反动派,英国、美国、中国都有。他们反对联合苏联,但是广大人民不赞成,他就没有办法,比如美国的选举孤立派就吃了亏,美国无产阶级有很大的工会、几个月以前,美国选举总统,因为工人帮忙,罗斯福还当选了,美国有许多孤立派的分子反对与苏联合作。因为工人民对他们,所以他们就倒下来了。我们中国人民的觉悟使得蒋介石法西斯、国民党(国民党里头的法西斯派别、法西斯集团)专门反苏反共集团的东西拿不出来,他们拿一个东西抛出来,比如拋出去一个东西叫做反对共产党,但底下没有人拍掌,不像你们一样大家要求要见中央的同志,要见从前方回来的同志。你们不要看国民党法西斯抛出来的东西,因为那些东西是反苏反共的东西。比如前年,在座的许多同志参加了这个斗争,七月、八月,九月三个月中,国民党宣布要解散共产党,他们说共产国际都解散了,你们为什么不解散?你们不解散干什么?又说什么共产主义不适合中国国情,中国不要你们等等,把这些口号拿出来,老百姓怎么样呢?老百姓没有一个拍掌,就是说大家都说你那个东西我们不要看;(笑声)因此我们要“委员长”就赶快收起来了,他拿出来的时候说:咳!“你们看不看?”大家就说:“不要看!不要看!”,就是因为这个原因他才收起来的。同志们,这是什么原因呢?是因为我们的力量对不对?那是对的,都是这样的情形打平下去的,延安能够开这样大的党校,就是因为我们手里拿着枪。所以第一是力量,不然还能开成吗?开不成,单是拿枪也是开不成,但是没有全国人民拥护我们,党校还开得成吗?也开不成。单是拿枪也是开不成,那就是胡宗南在这个地方开党校,也是中央党校,但我这个校长就当不成。(笑声)

这就是说大家都拥护三个国家团结,三个国家也会团结,为什么团结?因为全世界人民有觉悟,这一条给全世界人民很大的利益。现在保障了能够打败法西斯,将来能够保证世界的进步。

打日本要打到那一年终止呢?这个问题也说一说。现在美国把菲律宾占到了,占了怎么样?全在中国海岸登陆,也不是会在几个月之后日本就会倒,我说看样子几个月还不会倒,今年也还不会倒,明年还要再看。总之,日本法西斯现在手里还有东西,他现在有一点照我们的办法:“敌进我退”,不和美国人打硬仗,留下本钱慢慢打。他就是这么一个计策,日本想美国是不想打长期仗了的,他打个两三年三五年他就是想回家,“你想回家了我就打,本钱拿在手里。”他用这个办法打,打游击战也是日本的主意。

我们中国差不多被日本爬满了,到处都有,要请他出去,他不走还是要靠我们把他打出去,要靠我们党校的同志们在毕业之后到华北、华中、华南各个根据地坚持工作,才能把它打出去。我们现在小米也不足,步枪不足,机关枪也不足,军队还不够,根据地还不大,我们还要增加小米,大米,增加步枪、机关枪、增加军队,扩大解放区。总而言之,同志们现在上了一点年纪的人,二十几或三十几,你们都有这个学问,世界上的东西都是个力量,你们不去搬它,他就不动只有一个原则。比如这个桌子,我不搬他,他不走,这个茶壶我不请他,他不起来。(大笑)他的脾气还有点傲。(大笑)日本人也是这样,是要我们请他走他才走的,所以我们要有步枪机关枪,将来还要有飞机大炮,他们才会走的。我们的军队增加的很快,早一个时候,还只有六十多万。现在已经超过七十万了。我们解放区也扩大的很快,生产运动全国各个根据地都发展了,但是还不够,还很不够。城市工作尤其不够,各个区域的工作也不够,我们需要准备力量,正是我们的力量还小,他才不走,所以我们要准备力量,他不是一年或几个月就可以走的。

国民党和共产党能不能够团结起来?要请日本走,中国的这两堆子的问题要解决。我们共产党就是小米、大米、机关枪、八路军、新四军这一套。委员长的兵力更多,有中央军、杂牌军等等,这两堆子能不能团结起来一致打日本人?我们天天要求团结起来,国民党现在口里也讲要团结?起来,因为他不讲不行,他如果说:‘我不要团结起来!’那讲不下去,但是他心里恨我们恨得要死。我这个活也是有根据的,不是乱讲的,为什么说他对我们恨得要死呢?因为,他亲自和我们的同志讲过,“共产党如果不解散,我死了的时候眼睛还是睁的。”死了眼睛还不闭,一句古话叫做‘死不瞑目’。这是他讲的,如果是我造谣,大家可以调查。(笑声)这是脑子里想的东西,但口上却挂招牌、要团结,现在周恩来同志又到重庆去了,又和他谈判。看样子国民党他是不准备真正解决问题的。他说要解决问题要照他的办法,我们提的事情他一样也不办,只是要我们加进他们那个政府里去的,去他一二个人去重庆去吃大米。(笑声),他说延安比较差一些,我重庆大米比较好(笑声),你们来吃吧!(笑声),我们就说:委员长请你办几样事:第一样,禁止一党专政。他说,此事难办。(笑声)第二样,成立联合政府。他说那又不行。第三样:承认各党派合法地位。他说这个可以。我说这要看合得怎么样,因为在特务底下说合法是合特务的法。第四样:我们说你那个特务机关都要不得要取消。他说:那不行,特务机关好得很!(笑声),第五样:我们说有些东西要取消,就是那些压迫人民制的那些合法命令法律要取消,这个他也不干。第六样:我们要他释放政治犯,他一个也不放。我们说把张学良放出来吧,杨成武<原文如此,似应为杨虎诚>放出来吧!叶挺放出来吧!几百几千的共产党员放出来吧!然而他有宗旨一条,叫一个也不放。我们是:“一个不杀”,他就学了我们这一条,“一个不放。”(笑声)还有一条,我们说你把包围边区的兵撤走吧,包围我们边区,老百姓也不高兴。你打日本没有兵力,日本人到处窜得进来了,在华北、华中,你打我们的军队,开击打日本吧!他也不干。最后一条,我们说我们这个解放区你赞成不赞成?要求你承认一切抗日军与民选政府,这是应该承认的。这个东西也谈不好的。去年就谈到,他说解散五分之四,留下四个师,其它的统统解散,不晓得日本人那天和他谈过,(笑声)讲得那样清楚。(大笑)八条他一条也不办,就要和我们去吃大米。(笑声)我说小米也很好。要你办八条,你先办两三条或者先办三四条吧!他说一条也不办,谈判情况怎么样?谈判情况就是这样。

时局问题就讲这点,国际时局,抗日战争,国共两党关系。总而言之,同志们,靠我们在座的同志,靠我们在西北、华北、华中和华南的做工作的同志们,靠中国人民多搞大米小米,多搞步枪、机关枪,多搞军队,扩大解放区,在日本的地方组织秘密队伍,在国民党的地方发展民主运动,作这事就靠得住。其他的什么事况,比如英美苏三国团结靠得靠不住,我们讲靠得住,很靠得住,十分靠得住,一百分靠得住,一千分靠得住。单是新四军,八路军、共产党、没有小米靠得住吗?我说这只是一个条件,条件之一是力量扩大。单是有这个东西,如果不增加力量,还是有限的。我们在党校准备什么力量?我讲好好学习就是增加力量。

二,山头主义问题。现在大家在讨论山头主义问题,这个问题是一个实际上存在的问题,就是说世界上有这么一个东西,起他一个名字叫山头主义。所谓世界就是中国就是共产党,我们这个党由很多部分合起来的,所谓山头不是那个人从他母东生下来那一天,他母亲就告诉他说:你将来长大了,要去立山头,这是中国这个社会的产物,是中国革命的产物。中国这个社会不好,受帝国主义的压迫,受封建势力的压迫,因为受压迫就要革命,中国人民在无产阶级领导下进行革命运动,举出一个先锋队来叫做共产党,这个党有二十几年的历史了,中国是个农业国家,交通不方便,大革命的时期,因为我们在大城市交通要道的地方工作,交通很方便,山头很少。内战时期被敌人割断了,分为一个白区、一个苏区,白区有这个省那个省,苏区有这个苏区那个苏区,时间提长,上十年之久,抗日时期也是这样,时间很长,八年之久,也是这个根据地那个根据地。我们中国是出豆腐的,外国人不吃豆腐。我们中国可多得很,照割豆腐的方法,划了很多块子,方块,这不是我们自己划的,是敌人给我们划的。

像这样的党校,内战时期没有办过,西北、华北、华中、华南,东北全国各处大概除了内蒙古和西藏都有(内蒙古、西藏个把子恐怕也有)在我们这里学习的。但是,我们根据地还都被敌人分割的,在乡村里头就说这不是军队的,而是麻雀阵,满天麻雀,老百姓如吴满有住在延安椰林区,这里有一个吴满有,那里有一个张满有,又有一个李满有,一家一家分得很散,叫个体经济,农村个体经济、都被敌人划成豆腐块子,这些革命党革命军队,团结起来打敌人,就很自然的很有理由的可以打倒。这就形成各种班子,各种集团,各个山头,所以我说这是一个社会的产物。这是中国革命特别情形的产物,要消灭这个东西,将来要起变化,不但是我们开党校来分析来讲清楚,而且要在将来全国胜利了,有了大城市了,到处交通很方便,报纸能销到全国,到处有无线电收音机,开会很方便,那个时候才能消灭山头主义。所以我们应该承认这个东西,就是说有这个东西。我们党校讨论,延安各个机关也在讨论这个问题,现在我们提供各个机关去开会,曾经在以前以后奋斗的各个根据地的人,各部分军队的人去开会,检讨历史自我批评加以分析其目的是什么呢?还是说我山头太少现在要增加它几十个呢?我们现在来承认山头主义其目的是什么呢?承认山头主义的存在,其目的是要消灭山头,要使溶化起来,全党变成一体,要达到这个目的,要承认他,并且各个部分要去检讨历史,这种检讨都要在一个条件下,就是指导上要是正确的,开这种会这种检讨在会有益处,才会有益无害,什么叫做指导上正确呢?什么叫做是指导上不正确呢?因为现才大家在这里讨论这个问题,我提出这个问题研究,请大家讨论讨论。

首先就是说从团结出发,从团结全党出发,从一百万党员出发,从四万万五千万人民的利益出发,我们讨论这个问题以便讨论其它任何别的问题,我们的出发点是什么?出发点是全党全国人民在外还有什么出发点没有,或者叫做立场,还有什么立场没有?没有了。就是这个立场,全党与全国人民就是我们的立场,这是我们的出发点。

第二就要分析,分析就要批评。批评自己也要批评别人,这个分析也就是批评。我们分析这一个东西加以分解,分解是分成两个东西,那个是正确的,那个是不正确的,那些是应该发扬的,那些是应该丢掉的,这就是批评。对自己的工作自己的历史加以分析,这是自我批评。对别人也是分析,分析别人就是批评别人。被批评的时候总会有些不舒服,难过,现在我们中央大家同志可以讲话,中央有决议案的,无论是什么时候,你们凡是看到工作中间有毛病的地方,你们就讲,不成问题,工作中间一定有些毛病的地方,讲的话是我讲的,做的事是我做的,加以分析,做的正确的那就要承认正确的,做的不正确的那就要修改,也就是我上次在这里讲过的两条,坚持真理,修正错误。我们把我们见到的工作,把中央的工作,把我一个人的工作,把你们每一个同志的工作,把你们每一部分、军队的部分和地方部分的工作加以分析,不管是正确的或者是不正确的要分析,正确的东西就是真理、真理就是在我们斗争中证明了是的东西,适合人民的要求,求得了斗争的胜利。就是说已经在客观的事实上证明了是真理,那就要承认,而且要坚持,如果有人要反对这样的真理,我们就要解释,就要批评,认识了真理就一定要坚持真理,如果不坚持真理怎么办呢?真理要坚持,如果真理不坚持把他推倒在地下打烂了,那就不得了,把真理打烂就是把中国人民打烂了,把中国无产阶级打烂了,也就是把共产党打烂了。所以大真理也好,小真理也好,都分真理也好,凡是真理都应该坚持,但是有一个东西叫做错误,那怎么办呢?错误的不是真理,凡是错误的在人民斗争中间不适合的,在斗争中间讲的话不适合那个情况,做的事跌了跤子,写的决议案或者全部不对,或者部分不对,假若是不对的,那就说是不对的那就叫错误的,错误的东西应该去掉,人民不需要的东西应该丢掉,没有理由还保持起来,有什么理由把他保持呢?人民不需要,实际工作中又行不通,话讲错了,事做错了,决议案写错了,如果决议案有十条九条不错,一条错了,那错的一条就要修改,这叫修正错误,这就是坚持真理。修正错误要不要釆取这种态度,要的。一个共产党员要不要这样,要这样。在党校学习过的要不要这样,要这样。什么叫公道呢?这就叫公道,坚持真理这就叫公道,修正错误也才是公道。但是同志们,这样你们脑子里就要有准备,我看见多少同志没有这一条准备,所以就没有分析。这是坚持真理修正错误经常的不断地分析。这就是辩证法,辩证法的最基本一条叫做矛盾的统一,一个统一的东西,一个统一的东西可以分为对的和不对的,可以分得开的,如果不承认这一条,同志们,这就是不承认辩证法。辩证法基本的一条叫做矛盾的统一。统一的东西可以分开,这样的东西他不承认,他的东西是不能分开的,不能分析的,是整个的。没有这样的精神准备,自己不随时看一看我某一点驳错了,比如身上穿的衣服有什么灰尘没有,只看到好的东西,我的衣服没有破,没有破我说破了,不合乎事实,但是有点脏,穿了好几个月,是不是对呢?讲的话合乎事实,灰尘多少总有的。冬天过去了把棉衣折了洗一洗,明年又是好的了,修正错误要有精神准备,多少同志因为没有准备这一条毫无主动性经常是被动的。人家说你这里有灰尘,“我没有灰尘,那里有灰尘?”他是被动的。我们要自动性,要经常检查,检查我的衣服我的脸,检查我做的事和所说的话,所写的决议。把我做的工作加以分析,什么是正确的、是真理,什么是错误的改正。有这样的精神准备就好办事。我们党校就提倡这一条,党校出去的同志假如你们同意的话,假如你们赞成的话,你们采取一致的态度,坚持真理修正错误,并且到处宣传,对你前后左右的人宣传,到那里就跟那里的人宣传,这样事情就好办。自己看不到的多得很,眼睛看的只有这样远,没有望远镜有也不是不多,那就要准备请人家看,多准备一点,准备多少?准备一百条;这里有一点灰,那里有一点蚤子,“哎呀,有点痒”还有什么地方,有点什么多准备一点,准备一百条,我没有看到经过你们讲,这里讲十条,这里是十条,这里讲十条,多少条,多少条,四十条。我准备了一百条,你只讲了四十条,我还有六十条准备你讲,事实上没有那样多,那里这样邋遢,要有这样的精神,一个说:“没有灰尘”没有证据,拿镜子来照有就是有,没有就是没有。有则说之,无则不说。现在大家讲话可以讲,讲对了很好、讲出来真理。讲的不对也不要紧,言者无罪。讲的不对,讲出来有什么好处?讲的不对,讲出来了可以受别人的纠正;讲的不对,讲出来他谈你这话不对,他不讲我就不知道,我就是这样的乱讲一顿,或者其中有几句话讲的不好,受了你的批评,你将了我的军,我就谨慎一点,我就纠正错误,好处多得很。

同志们,我们二十四年没有胜利,这是我们国家太大了,也就是一个缺点,如果我们的国家小一点,假如只有一百万人我们就算胜利了,但是有什么办法呢?我们名字叫做中国人,落在这个地方那也没有办法,国家太大了就出了一个长期性,这是国家太大了,敌人太多了,没有胜利,三次革命都没有胜利,北伐战争没有胜利,国内战争没有胜利,抗日战争还没有胜利,总而言之还没有胜利,开党校干什么?办八路军新四军,一百万党员,就是要胜利,现在到了这个时候,有可能胜利。中国的客观条件:国际形势、国内形势,国际条件,国内条件,有胜利的可能,就是不要缺乏主观的条件,什么叫主观条件?就是我们要不要胜利,我们的思想有准备没有,要不要胜利?那个不要胜利向我们刚刚生出来的只有这样高的三岁小孩,他也要胜利,就是没有本领。我们精神上有没有准备?大革命北伐战争怎么失败的?客观原因是帝国主义力量强大,主观原因是没有精神准备,缺乏准备,思想糊涂,政策错误,失败了。内战也差不多是这样,白区工作,苏区工作那时候没有取得胜利,客观条件由于帝国主义力量太大,国民党力量太大,但是第一条我们党在政治上还非常幼稚,精神准备不够,因此也没有胜利。这两个时期留下了宝贝,这个宝贝就是人头,一个时期留下的人很少了,是以百计算的。是不知还有一二千人?不知道,要调查一下,一九二七年八一南昌暴动以前的老同志很少了,虽然很少,但是留下了一堆,一小堆,这样一茶壶,这是宝贝,一不是金,二不是银,是宝贝,比金银还重要。第二个时期的人变多了,有多少呢?我看不过几万人,两万左右不会很多,那共产党有几十万,军队也有几十万,老百姓有上千万,甚至有二三千万,但是像聋子放炮竹――散了。内战时期,现在还存在着的活着的党员我看有两万左右,现在我们有百多万党员,那里来的?是土里长出来的,这两万人是宝贝,第二个时期包括第一个时期有两万人或者比两万人多一点,三万四万,这一批人是很可贵的,两个时期都包括在内,我们做了这样正确的事,那样不正确的事,叫做跌跤子,人大一点便少跌跤了,你们现在不跌跤了,我现在也不跌跤了,因为我们小时候跌了好多次,因为我们现在已经“七十而从心所欲了。”我在小时,左一个跤子、右一个跤子,印象很深,我跌跤经常跌出血来,触在石头上便出了血,别人比我稳重,不容易出血,我说我运个人没有用处,同我一样大的人,跑的飞快,脚不出血,我经常碰在石头上,那时候看来是拦羊娃谢了那些石头,使得我现在不跌跤了,少跌跤了,包括各位同志在内,你们现在把内战时期的历史检查一下,有些同志检讨大革命北伐战争,抗日时期,这是很好的事情,目的在得出经验,不在于把责任加在个别同志身上,因为加在那个同志身上,没有好结果,把同志放在磨坊里头磨成粉,有什么好处呢?能不能解决问题?不能解决问题。这些同志不是他母亲生他下来就下了命令要他跌跤子的。

中国共产党经过多年的时间,加以分析,把我们现在做的事情加以分析,要从全党出发,就是要团结,团结全党是第一。加以分析批判,这是第二。然后再来一个团结。团结批评团结,这是我们的方法,这是辩证法,什么叫辩证法?这就叫辩证法。辩证法的发展,辩证法的运动就是这样的运动。整个分别,再来一个整个,这是辩证法。如果事情搞不好,原因在什么地方,原因就是没有照辩证法办事。太行山×××同志讲:事情怎么样办?照辩证法办事。我赞成他的话,事情没有办好,就是没有照辩证法办事。不是从团结全党出发了。你们各部份开会,内战时期,北伐时期,抗战时期,那部份也好,这部份也好,第一个观点出发点就是团结。团结那就一团和气,我们开会喊一声:“团结”,完了,一句话再一句话:“团结”完了。照辩证法就没完,要分析,要批评,搞清问题,分清是非轻重,――党内的是非分清楚,很公道,“坚持真理,修正错误”,这样的团结加以分析,并且是精密的分析,这一面看一下,那一面也看一下,再看一下,再看一下,再看一下,酝酿成熟。我们有很多经验,搞错事情常常是因为看了这一个侧面,没有看到那个侧面,常常是因为只听了这一面的话,没有听那一面的话,我们为什么要长两个耳朵呢?长一个岂不好吗?这可以研究一下,世界上的人,为什么要长两个耳朵?我看他的好处是这个耳朵听这一面,那个耳朵听那一面。徐特立同志告诉我们说,人的手为什么要这样长,大指头上天,小指头向前面,方向不同这不是缺点?不是,如果方向同了那就没有社会,没有共产党,没有八路军,没有中国,方向不同为什么没有中国?不但没有中国并且没有苏联,没有人类,人类在劳动中长这样的指头,因为他要拿工具,他要拿木头,必须手指这样长才可以抓到。“抓一把”就是因为这样,手指头长的方向不同这有道理的。团结加以分析,分析各种矛盾的意见、不对头的意见,听一听每个人的意见加以分析,分析的结果或者说是对的,只是句把话不好,或者分为两部分,一部分对的一部分不对的,让同志们讲清楚,然后落在一个地方,运动发展到一个地方去了,到什么地点、团结,我们党校是个很明显的证据。党校的整风学习,第一个时期前年的整风有很大成绩,这是第一条,也有许多缺点,第二条彭×同志告诉我去年好多了,昨天我给五、六部的同志也谈了,现在所釆取的方法,我讲的方法,是从团结全党出发,不是从团结一个山头出发,不是从团结小部分人出发,是从全党出发,把我们这部分,那部分,这个山头,那个山头的工作,加以分析得到的结果,这部分与全党之间也团结了,运动的发展是这样的,如果不谨慎,我们历史上曾有过这样许多的事情,头一个步骤就不是为了团结,头一步走的不对,第二步分析工作,批评工作也不很妥当,落的地点,落下去的地方是不团结,我们现在鉴于历史,这些历史很有好处,教育了我们,使我们觉悟了,盲目性逐渐去掉了,减少了,我们有这样一套,这对于我们准备全中国的胜利,有密切的关系,我们的延安,我们的党校,用日本人的话说,就是中国共产党的首脑部,一个头头我们用这样的方法来学习研究,会不会达到胜利,会要达到胜利。我们这几年全党的工作中有一个整风是在精神上准备胜利,我们开这个党校,很快我们要开七大是精神上准备胜利,他的性质是准备中国共产党在全国胜利,客观条件有这样的可能性,我们现在加紧主观条件,就是刚才讲的这些。

山头主义问题上各位同志要经常注意这一点,你们毕业以后到每个地方要估计各种情况,王震同志打敌人去,出发时我对他讲,一条叫光明,第二条叫困难,看到光明也要看到困难,这是辩证法,是矛盾统一。光明又是困难是不是打自己的咀巴?讲光明就讲光明,又讲什么困难?我们党的历史有这样的时候,只讲光明,讲不得黑暗,这不是辩证法,没有照×××同志的意见办事,我们讲光明同时一定要讲困难,三七年、三八年抗大学生那时候不要过五关斩六将潮水一样源源而来,滔滔不断,我那时不忙,不是这样三年还没讲过话,对不起得很!那时我可讲的多,三天一小讲,五天一大讲,所讲的都忘记了,大家忘记,我也忘记了,有一点我还记得,我讲同志们从广东、广西、湖南、湖北、云南、贵州、绥远、新疆,“不远千里而来,亦将有人利于我国乎?”抗战的事是为了国家,我说清凉山插了一面红旗,叫做新民主主义之旗,中国要独立,三民主义要民主,要解决民生问题,这些对不对?很对。有没有希望?很有希望,叫不叫光明?叫光明,不是九分光明。但是你们跑到延安来,幻想得很好,一些同志在做梦时候或者没有做梦的时候,想的各种东西好像天堂,到了玉皇大帝那里,把延安看成天堂,你这样想错了,延安不是天堂,也不是地狱,是人间,在天之下,在地之上,我一点没有讲错,正确的方法是看着中国社会的一部分,这个地方有很多缺点,我给他们说了七八条,你们要装进去,不然三五个月,你毕业之后要长叹一声:“早晓得这样的延安,老子不来了。”来了之后一些人写文章叫野百合花!如此等等相当多,那些人大概没有听我讲这点,没有讲通,以后讲通了,文艺座谈会还有别的,在华北、华中、华南各地到延安来的,你怎么看延安,照辩证法的看法,从整个出发,边区是什么地方?边区者,中国共产党领导的中国人民革命的根据地之一也,之二,晋西北,之三、五台山,之四、太行山,之五、……东岳太山,南岳恒山,华中、华南、都是革命根据地,中国人民反帝反封建的根据地对不对?当然对,就是这样的定义,在这样的定义下,那就好办,对任何根据地,中国共产党领导的中国人民反帝反封建的根据地釆取什么态度呢?万岁!就是一个万岁,万岁之外还要准备任何一个根据地还有他的特点缺点,这个边区十有之九还有缺点,全中国胜利,党校搬到上海,北平去了,这里还有缺点。苏联人家搞社会主义革命经过多少年?二十五年,有没有工业,三个五年计划红军弱小呢,强大呢?相当强大。打仗的时候希特勒都怕得很,他们出了一个戈尔洛夫,这叫做精神准备。你们出发各地,去到每一个地方要喊万岁,九千岁都不行,为什么要喊万岁,因为那里是中国共产党领导的,中国人民反帝反封建的根据地,做的是英雄事业,艰苦奋斗,但是你要准备人多了来不及,挤在一个小房子里,小米大米都没有那样多,有缺点,官僚主义,一餐稀饭,一天吃了一餐稀饭就生气,还有那里不晓得你是团长、旅长、他们只晓得你有两个眼睛,一个鼻子,是不是“团长同志请坐呀!”没有的,很不尊重,因为不很了解所以不很尊重,也没有开欢迎大会,坐冷板凳,或者我自己少有缺点,人家说话就来了,“你还批评我呢?”我从前万里长征,八年抗战,延安党校,我少有缺点你就批评了,心里就有那样的东西,叫做气,同志们都要准备,准备各种不如意的事,多少封锁线,敌人袭击,不开欢迎会,开了欢迎会掌声不够,稀稀拉拉几个巴掌。总而言之,同志们你们要有准备,你们到延安来也要有准备,华北、华中,华南派干部来要有准备,到了延安,人家不晓得你是团长,旅长,人家那里有官僚主义,那里还不是莫斯科,莫斯科还有戈尔洛夫,难道延安便没有了吗?要讲这些丧气话,讲话为什么要讲丧气话?不对我讲头一条,讲壮气话,我又没讲专讲丧气话,只讲壮气话,光明,光明,光明,但是事实有缺点,有困难,告诉同志们,不要只说丰衣足食,新式武器,丰衣足食专向荒山要开荒,新式武器,锄头一把,总而言之,大概得不到现钱的支票,不要开,或者可以得到的,也许得不到的,这样不要开支票,对于同志,战士,人民要讲老实话,是则是能则能,每一个地方都有缺点。我们先估计那里的缺点,那里的困难。因为我们在那里不说清楚,一到这里就批评人家,我们的心是好的,人家不了解。所以到一个地方不要去批评人家,采取学习的态度,在那里落下来,工作熟了,那些人了解你了,然后再讲有什么缺点,这样一讲就讲进去了。人家会赞成也会改了,采取这样的态度,现在各个部分开会,领导的同志,领导的骨干,三个、五个、七个、八个经常商量,要掌握作风,各部分的领导人,指导要恰当,这个东西很有关系,指导恰当,运动就能胜利会就可以开好,指导不适当,会就开不好。

三、审查干部。有些同志希望讲一下,上次在这里我已经讲过一次,有很多的同志没有听到,我再讲一下,前年、去年我们进行了审查干部的工作,这个工作在我们党的历史上过去有过,但是没有最近这两年,整风之后这样认真来做。在这个工作中,我们得到了很大的成绩,也犯了许多错误,我们取得了经验,经验就是得到了成绩,也犯了错误,这就是两条经验,一条对于我们的党,对于中国人民,我们准备胜利。他的成绩是伟大的。我们的方针有两条,我们对于我们的党,对于我们革命团体,要釆取严肃谨慎的态度,在组织问题上为什么我们的党区别于资产阶级的政党,区别于小资产阶级的政党,社会民主党、美国的大政党、中国的国民党、我们跟他们是有区别的,他们也是不是釆取严肃的态度,站在他们的阶级立场上,他们也是釆取严肃的态度,但是像共产党这样的严肃性、纪律性、任何政党也没有的,小资产阶级政党也没有,他们不可能也不愿意釆取这种态度,无产阶级被压迫的阶级,手里没有权力的阶级,要从压迫的地位、没有权利的地位站到压迫压迫者的地位,取得权力,非有严肃的纪律不可,集中统一,思想一致,行动一致,共产党的纯洁性,这样的派别,从有党以来,是模范的党派,仅仅在俄国有。历史上的布尔什维克,执政以后,变成领导苏联的党,联共党史开卷头一条,头一行,就讲由小组到联邦,马克思主义到俄国少数几个人稀稀拉拉的形成到联邦,现在是领导核心。小组是小指头,联邦是指房子。我们中国也是从小组到联邦,到了没有?还没有,走到了半路,要从小组出发,小组到根据地到联邦,现在到了根据地,快要到联邦。我们要准备胜利。中国的回、蒙、西藏各族成立共合政府,再成立联邦。将来的道路是这样,现在还没有。要达到这个目的,同时要有一个严肃的态度,共产党的纯洁性,思想行动的统一,政策的一致?现在我们怎样?在整风以前同志们大家知道,我们党在思想上并不纯洁,有非无产阶级思想,并且相当浓厚,就是说有别的阶级的思想在无产阶级先锋队里,主要是小资产阶级思想,思想上纯洁不纯洁,不纯洁现在怎样?大有进步,肃清了没有?没有,还要做工作,所以党校还要开,是不是可以结束,可以解散?还必要,还要存在,并且你们出来要采取党校的作风,办党校或是非党校要使我们的党在思想上是纯洁的,是马列主义的,组织上是纯洁的,就是说我们的审查干部在组织上,要使我们的党是纯洁的,这样一种严肃态度非有不可,讲起来容易,要具体釆取这样的态度,经常会忘记,似乎不太严肃也不要紧,农业社会的人民映农业社会的散漫状态,小生产的反映就是自由主义,一群一群的进党,由几万到一百万。同志们,我们的党在七、八年中由几万发展到一百万,思想不纯洁,组织不纯洁,有什么办法,那你为什么要发展呢?不要紧,要发展,我们不怕。这是我们党的一条政策,我们收了来,然后整风学习,有些坏人混进来了,不纯洁的分子混进来了,里面是敌人派来的特务,特务,叛徒,自首、党派这些问题程度不同,严重性有重的轻的,但是我们组织上应该纯洁,应该搞清楚这两年的运动对于这方面怎样?我讲有很大的功劳,我们学会了整风审干,审查党员,这方面有很大的成绩,有人这一点看不出,我和几个同志谈,他只看到我们犯了错误,他举了几十件,几百件也不够,我们知道几千件,拿件数算,数量不够否定我们运动的质量,这个运动的质量,根本是正确的,有益于中国共产党的团结和巩固,有益于中国人民的胜利,我们把许多问题搞出来了,这是一条经验,有了这条经验,有了这条经验将来我们出去,每个人到底去建设共产党,发展共产党,去整理我们的组织,使它纯洁,整理我们同志的思想,使它纯洁、整风、整理组织,审查干部、审查党员,一百万多党员抗战以后加入的占九十多万,发生了一个问题,我们要不要胜利,要不要在全国胜利,要的话就要有。一个纪律,思想上纯洁、组织上纯洁的党,合乎统一的标准的党,联共党史结束语第一条讲的要有一个党,要什么党?不是社会民主党,不是国民党,要共产党,要有革命作风的共产党。第二条我们要有理论,所以这一经验对于我们全党的作用是很大的。因此我们要看到这一成绩。在去年一年中间发生了一种偏向没有成绩,为什么发生这样的情况呢?就是我们的同志看问题的时候,我们的同志长了两个耳朵,这个耳朵听一下,那个耳朵听一下,但是缺乏分析,因此发生了偏向,这种同志我给他一讲,他很赞成。

第二条经验,就是犯了错误,拿件数来算就很多,我是党校的校长,就在这个党校犯了许多错误,谁人负责?我负责。整个延安犯了许多错误,谁负责?我负责。因为发号司令是我,别的地方搞错了,谁负责?也是我,发号司令的也是我。当然是你呀!但是同志们,关于这条错误的本身要加以分析,一个叫坏处,一个叫好处,坏处是犯了错误,好处也是犯了错误。(笑声)你这个话讲的糊涂,乱讲一顿。这样对不起一些同志,一些同志戴错帽子,在座也有这样的人,对不起,一种是搞清了是同志或者是叛徒或者自首,或者党派,这些人搞清了,很好。我们对这些同志很和气,和他团结。这些同志很好,他自动的讲出也好,被逼出来的也好,只要讲出来,我们就欢迎他,帮助他改正错误,其错误不等,有的重一点,有的轻一点,有的再轻一点的,有很轻的。我讲的话要算数,发出去的支票要兑现。一个支票一一一个不杀,这一条兑现没有,还不是兑了现。如果口里讲一个不杀,结果杀了两个,那不是自己打自己的嘴巴!的确一个不杀,没有杀两个,也没有杀一个,分清是非轻重,是则是,是怎么办?是特务,自首叛徒,是党则团结抗战,是特务现在怎样办?我们现在让他团结抗战,人家现在愿意抗战,过去走错了路,算了。过去走错的路现在不再走,走完了。我宣布这一条主要是帮助这些走错了路的同志。因为党过去有很多缺点,人家不敢讲,不然人家早就讲了,因为他害怕就没有讲,现在讲出来,他们这样很好。这是对于这些同志搞错了的同志。戴错了帽子的,把非当做是怎么办,非则非,取掉帽子,赔一个不是,我是党校的校长,党校也搞错了,如果在座有这样的同志,我赔一个不是,因为搞错了。被戴错帽子的人应取何种态度?戴错了,他不是那样的人,我们戴在他的头上,现在取掉帽子,赔一个不是,那么你应釆取何种态度呢?当着我们给他戴帽子的时候,他应该生气:“搞错了吧?我不是,为什么给我戴上这个帽子呢?乱戴一顿这样对不对?”这样很对。现在我把帽子拿下来,赔一个不是,你应釆取何种态度呢?你应该还我一个礼吧,因为我向你行了礼,你要还我一个礼。现在我向你敬礼,你不还礼,我的手就放下下来,照美国人的规矩,敬礼时一定要还礼,如果不还礼,他的手就放不下来,同志们为什么给你戴上帽子,因为把你当特务,不是当同志,我们搞错了,这种愤怒之火要向着敌人不是向着同志,可是其结果,是向着同志,这叫做错了。但是开始的动机目的是向着敌人,整个这个运动方向是向着敌人不是向着同志。因此转过来凡是戴错帽子的取下来,所有这些戴帽子的同志,你手中有理,理在你手里,你手没有空,你手里拿的东西叫做理。因为你是共产党员,你还有道理,我们没有道理,所以你在这一点上应该睡着觉,应该把心安下来。同志们,从有共产党以来戴错帽子取下来的事,有没有?有。赔礼的事有没有?也有。但是像去年我们这一年的态度过去就很少,过去搞错个把帽子取下来就不赔理。所以我说承认错误,赔一个不是,这是我们的进步。是我们全党的一个进步。我们要不要胜利,要胜利,一切都是一样,凡我们对于人民讲的话,决议案,这样政策,那样政策,搞错了的就要修正错误。我们有这样的态度,这叫做什么态度?自我批评。斯大林写了一本书,叫做《列宁主义概论》。讲列宁主义方法论有四条,其中有一条方法,就是自我批评。我刚才讲的第二个问题山头主义,讲来讲去就是自我批评,就是这个方法。团结、批评、团结第二个步骤就是批评,中国要胜利,我们要采取这种态度。犯错误本身他有两个侧面,一个是不正确犯了错误,第二就是恰好犯了这个错误之后,就会变成经验。以后再来审查干部,双方都得到了经验。被戴帽子的以后给别人戴帽子的时候,你就谨慎,你吃了亏,凡是被戴错帽子的同志你们得了一条经验,你们不要乱给别人戴帽子,因为他自己吃了这个亏,以后要谨慎一点。将来我们到北平、上海、南京再去审查干部,作反奸工作时,就会好得多。所以我说两条经验:一个是成绩,一个是缺点,这两条都对。我们要胜利,成绩以后要发扬,错误以后要修正。戴错帽子的同志,你们是牺牲者,所谓牺牲者就是戴错了。但是我们要认识到这个经验对将来运动的发展是有好处的。严肃态度,谨慎态度。鉴于有问题,我们党组织上不纯洁,思想上不纯洁,就是应该釆取严肃的态度,鉴于把错误戴错帽子,就是要釆取谨慎态度。这是两条路绪的斗争。严肃态度反映右倾,不严肃就是右倾自由主义?谨慎态度就是反对左倾,不谨慎就出乱子。

最后一个问题,就是上次我在这里讲到各部革命团体。讲到一方面军、二方面军、四方面军,每一方面军里又有各部分。其中讲的不完全,对于四方面军里有几部分我就没有讲,因为我不很熟悉,还有一个十五军团也讲脱了,在我报告以后,有一个同志写信告诉我,这样很好。十五军团以前受过很大牺牲,干部都杀光了,剩下的很少,其它各部分我都提到,一方面军各个部分,一军团,三军团,九军团,五军团、六军团、七军团、十军团,一军团内又有五部分,三军团内又有各部分;提得不完全,大部都提到了,革命根据地都提到了,今天我想补充的就是那天没提到的。

第一个问题是南方北方问题,这个问题我早就注意到了,在那天忘记提了。现在在西北、华北、华中这三个地方工作的同志,拿地方来说,大体上可以分为两部分,一部分是外来的南方人,第二部分就是本地人。这个问题为什么要提一下?因为要使得我们的同志注意这个问题。这个问题的性质就是外地与本地的关系,军队与地方的关系。因为外来与本地、军队与地方在抗战时期中发生乱子和磨擦的就很多到处发生这个问题,是一个普遍现象。我想外来同志首先要有一种认识,要认识本地,认识边区,认识华北、认识华中、要认识本地各地同志,他们的功劳,他们的长处。因为我过去没有讲清这一点,也不怪那些同志。我以前没有好好分析,在整风后才讲这个问题,综合了好多经验,运中间有外来和本地、军队与地方。南方的同志、在西北、在华北、华中现在还活着的不到二万人。他们有很大的功劳,北方的同志,本地的同志应该感谢他们,西北的人民、华北、华中的人民应该感谢那些南方的老布尔什维克。为什么感谢他们呢?因为中国革命长期在南方发展,到了抗战时期才转移到北方。太平天国、辛亥革命,北伐战争,土地革命主要的都是南方,北方也参加,但主要的都是南方,南方很光荣,对不对?很对,南方是很光荣。但是同志们,这些革命都失败了,太平天国失败了,辛亥革命失败了,北伐战争失败了,土地革命失败了,都失败了。又有光荣、又有困难,还是那两套。南方有老布尔什维克,北方也有老布尔什维克,并不很少,我们边区就有好几万,华北也有一大批。××同志讲华中也有。南方老布尔什维克不要以为只此一家,并无分号。我们要承认这个边区,不要不好,不好,还不是不好,叫做“地广人稀,经济落后”,那也应该承认。但是他那个地狭人稠那里去了?上海在你手里?那儿去了?你那个中央苏区到那儿去了?给了蒋介石。只有这个根据地保留下来了。所有的根据地都丢了,只剩下一个陕北、陕甘宁边区。这个地方作用非常大,我说是中国革命的一个枢纽。经南一转,转到这个地方,然后由这个地方再出来。好像这样的门,我说是中国革命起承转合,门的枢纽能够一开一关。陕北是上面顶天下面立地,起承转合。起是从这个地方起,不自南方起。这里有高、刘、张承南方。转就是从这个地方转。万里长征,脚搞痛了,跑到这个地方休息一下,叫做落脚点。你走好久,到落脚点休息一下。你们党校也在这个地方落脚,你们英勇奋斗,八年抗战,华北,华中、华南,你们还不是落在这个地方来了?这个地方是我们的落脚点。我们不是永远在这里住一生,生儿子,儿子再讨老婆,再生儿子。你们要走。将来中央走,这个地方是出发点。是落脚点,同时又是出发点。抗战以来,除了新四军外,都是从这里出发的。北方的都是从这里出发的、戴××又是从这里出发,你们等两天也要从这里出发。在北方华北也有多年的老党员,像××,××,当然还有很多,他们坐在班房里头。只有我们英勇奋斗,人家不艰苦奋斗?他们坐在班房里和敌人斗争。在北方华北的地方党也有几次暴动,好多暴动是失败了,有很多地方有内战时期的党,许多同志经过三个时期:北伐战争、国内战争、抗日战争。听××同志讲在华中发现当地有这样的同志,特别是在苏北、苏南。

这个问题为什么要提一下?就是我们的同志应该看到这个问题。外来的同志应该看到这点,不但我自己是光荣的,华北也是光荣的。华中也是光荣的,不但南方是光荣的。这就等于讲一方面军是光荣的,二方面军也是光荣的,四方面军也是光荣的,陕北红军也是光荣的,在各部分里头的每一部分都是光荣的。是不是有一部分就高一等,有一部分就矮一等,我看不高不矮都光荣,几万是南方的,九十万是北方的,这就应该特别引起外来同志的注意:一百万中九十万是本地人,只有几万是外来的,这几万同志对于这九十万同志应不应该尊重,感激,看重?完全应该。这九十万应该感激这几万人,因为他们教训了我们关于战争这一套经验,他们有经验,经过国内战争,北方也有一部分国内战争,大部分还是在南方,他们教会了华北的人民,华中的人民。这是一个问题。

第二个问题,关于大后方党的问题。关于大后方党的问题,大后方党在战争时期大概的数量有十万,恩来同志,董必武同志,在座的还有许多同志是他们领导的。因为在前年审察干部的时候,有同志觉得那些地方的党不大靠得住,红旗政策很多。根据去年甄别的结果,事实上证明也是两条。有问题应该釆取严肃态度,因为在国民党统治底下,坏人混进来破坏我们,我们今天应该承认这一条。在南方同志里头混进了一些坏人进来了,不纯洁的人混进来了,因此,我们应该采取严肃的态度。第二条不可夸大,切记不可以为多得不得了,究竟有多少,数目现在还不能算,等将来全国胜利,由小组到联邦才能搞清楚,是估计可以做的。十万人里头大多数应该说是好的,不是特务。为什么这样讲?因为在那里大批是农民同志,农民同志在国民党就是破坏了的地方,对他们也不大注意,我们又采取疏散的方针,聋子放炮竹一一散了。到处散了,一盘散沙,现在大后方就是这样的政策,就是从前外国人讥笑中国人的话一盘散沙,就可留下来,就可以不被破获或少被破获。还要估计到国民党的官僚主义,不要把国民党看得神乎其神。他的特务厉害是厉害,第一条厉害,第二条官僚主义。我们不承认厉害不对,合搞,那样厉害,三头六臂,三头也没有,只有一个头,蒋介石也不只是一个头两只手,和我差不多。我们共产党的官僚主义比较少一点,国民党还要多一点。我们自己要搞清楚,特务厉害,还要加上官僚主义。究竟多少有问题,多少没有问题的,现在谁也不能答复。要由小组到联邦才能答复。这次王×、戴×出去建立根据地,就采取严肃的态度,谨慎的态度。大体上这样两条,一条承认那个地方党有问题,一条承认大多数同志是好的,没有问题,是纯洁的。一部分是不纯洁的,有问题的,大部分是纯洁的,没有问题的。这样的估计应该做出来,应该讲南方工作同志有成绩。

这点,他们要去检查,总的来说,在抗战期间,有成绩。在战略方面来说只有根据地也不行。有三方面的战略:解放区,沦陷区,大后方。三位一体,缺一不可,这是第一个问题。

最后白区工作问题,这一条我从前没有想到。那天××同志告诉我有这样的一个问题。很好。从前我们批评内战时期白区工作的领导路线是有问题的,在某些时候应该说是错误。因而使白区工作受到了很大的损失。过去说百分之百,现在看没有那样的程度。还剩下一些,大部分是损失了,领导路线是错误的,应该受批评。批评的目的,是为了今后的工作好好地做,不要再有这样的东西。并不是说所有在白区工作的同志、干部没有功劳,致于在南方根据地失掉,在领导上犯了毛病,根据地失掉了,并不是说红军战士、干部、地方同志干部他们没有功劳、没有艰苦奋斗,英勇牺牲是一样的。白区工作的同志没有坐班房活着的?没有坐班房死了的?坐班房死了的,这几部分人,他们替党替无产阶级作了英勇斗争,为了斗争,他们拿着命拚,因为这样坐了班房,少数侥幸外脱的没有坐班房,应该估计到他们的成绩。这是关于白区工作的问题。

这三个问题因为上次没有讲,今天补充一下。今天我想讲的问题就是这样。很对不起,同志们坐了很久,讲完了。(鼓掌)

