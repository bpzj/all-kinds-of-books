\section[苏联《政治经济学教科书》阅读笔记第三部分(从第三十章到第三十四章)]{苏联《政治经济学教科书》阅读笔记第三部分(从第三十章到第三十四章)}


四十八、关于土洋并举、大中小并举547页说到在基本建设中,反对分散建设资金。如果说大建设单位,同时搞得很多。因而都不能按期竣工,这当然是要反对的.如果因而反对建设中的中小型企业,那就不对。我国新的工业基地主要是在一九五八年大量发展中小型企业的基础上建立起来的。根据初步安排,今后八年中钢铁工业要完成29个大型、近一百个中型、几百个小型钢铁基地的建设.中小型对钢铁工业的发展已经起了很大的作用.拿一九五九年来说。全年生产的生铁,是二千多万吨,其中一半是由中小型生产的,今后中小型对钢铁工业的发展还是要起很大作用。许多小的会变成中的.许多中的会变成大的,落后的变成先进的.土的变成洋的,这是客观发展的规律。

我们是要釆用先进的技术,但不能因此而否定落后的技术在一定时期的必然不可避免性,从有历史以来,在革命战争中总是拿差武器的人打败拿好武器的人。内战的时候,抗日战争的时候,解放战争的时候。我们没有全国的政权,没有近代化的兵工厂,如果一定要有了最新武器才能打仗,那就等于自己解除了武装。

我们要实现像教科书上所说的全盘机械化(428页),看来第二个十年还不行,恐怕要在第三个十年中间。在今后一个时期内因为机械不够,半机械化和改良工具,还是我们要提倡的,现在我们还不一般地提倡自动化。机械化要讲,但是不要讲得过火,机械化、自动化讲得过火了,会使人看不起半机械化和土法生产。过去就曾经有过这样的偏向。大家都片面地追求新技术、新机器,追求大规模、高标准,看不起土的,看不起中小的。提出土洋并举、大中小并举以后,这个偏向才克服。

在农业上我们现在不提化学化,一是因为多少年内还不可能生产很多的化肥,已有的一点化肥,但只能集中使用于经济作物。二是因为提了这个,大家眼睛都看着他,就不注意养猪。无机肥料也要有,但是如果只靠它而不同有机肥料结合起来使用,会使土壤硬化。

教科书中说,在一切部门中釆用最新技术,但这是不容易做到的,总是要有一个逐步发展过程,而且在釆用了某种新机器的同时,总是有许多旧机器。教科书中说到,一方面新建企业。并对现有工厂进行设备更新,同时充分合理地使用现有的机器和机械(427页)。这样的说法就对了,将来永远会是如此。

至于大的洋的方面,我们也一定要自力更生地搞,一九五八年提出破除迷信,自己动手的口号,事实证明自己来搞还是可以做到的。过去落后的资本主义国家.苏联也靠釆用先进技术而赶上资本主义国家。我们也一定这样做,也一定能够做到。

四十九、首先有拖拉机?还是先有合作化?

563页上说。“在一九二八年全盘集体化的前夕,春季作物地的翻耕工作,有99%还是使用木犁和马拉犁。”这个事实推翻了教科书上在很多地方关于“要有拖拉机才能合作化”的观点。562页上所说;“社会主义生产关系为发展农业生产力开辟了广阔的场所”则是对的。

先要改变生产关系,然后才有可能大大发展社会生产力,这是普遍规律。东欧一些国家农业合作化搞得很慢,到现在还没有完成,主要不是由于他们没有拖拉机(相对的说来,他们比我们多得多),主要是因为他们的土地改革是从上而下的恩赐的,他们没收土地是有限额的(有的国家100公顷以上的土地才没收),是用行政命令来进行没收工作的。土地改革以后又没有趁热打铁,中间整整间歇了五、六年。我们则与他们相反,实行群众路线,发动贫农、下中农展开阶级斗争,夺取地主阶级的全部土地,分配富农多余土地,按人口平分土地(这是农村的一个极大革命)。土改以后,紧接着开展广泛的互助合作运动,由此一步一步的不断前进地把农民引上社会主义的道路。我们有了强大的党,强大的军队,我军南下时,各省都配备了从省、地到县、区整套的地方工作的干部班子。而且一到目的地。立即深入农村。访贫问苦。扎根串连.把贫农、下中农的积极分子组织起来。

五十、关于“一大二公”

苏联的集体农庄合并过两次。由25万多个合并为九万三千多个,又合并为七万个左右。将来势必还要扩大.教科书上说。“要加强和发展各集体农庄的生产关系.组织集体农庄之间的公用生产企业等等”(568页)事实上有些地方和我们的方法类似。只是不用我们的说法而已。它的将来,即使办法和我们一样,看来也不会用公社的名称,说法和名称的不同.包括一个实质的问题,就是实行不实行群众路线的问题。

当然。苏联集体农庄扩大规模,按户数人数来说,可能不会像我们这样大,因为他们农村人口稀少、土地多。但是能够因为这样就说现在集体农庄再不需要扩大了么?我们的新疆,青海这些地方。虽然人少地多。但仍旧需要扩大公社,我们南方几省有些县。如闽北的一些县也是在人少地多的条件下搞大公社的。

扩大公社是一个重大问题,量变了,一定会引起质变,会促进质变。我们的人民公社.就是“一大二公”。首先是大,接着就必然提高公的水平,也就是说必然带来部分质变。

五十一、特别强调物质利益的原因何在?

在集体农庄制度一章内,反复讲个人物质利益。如565、571、580页等等。现在特别强调物质利益总有个原因,斯大林时代过分强调集体利益。不注意个人所得,过分强调公的,不注意私的,现在走到了反面,又过分强调个人利益。不大注意集体利益.这样强调下去。又一定会走到自己的反面。

公是对私说的,私是对公说的,公和私是对立的统一,不能有公无私。也不能有私无公。我们从来讲公私兼顾。早就说过没有什么大公无私,又说过先公后私,个人是集体的一分子。集体利益增加了。个人利益也随着改善了。

两重性。任何事物都有,而且永远有。当然,总是以不同的具体形式表现出来,因此性质也各有不同。例如遗传和变异也是对立统一的两重性.如果只有变异的一面,没有遗传的一面,那么下一代的生物和上一代的生物就完全不同,稻子就不成其为稻子。狗也不成其为狗.人就不成其为人了。保守的一面。可以起好的积极的作用。可以使不断变革中的生物在一定口时期内采取一定的形态固定起来。或者稳定起来。所以稻子改良了还是稻子。但是如果只有遗传的一面,有变异的一面,那么就没有改进、发展,永远停顿下来了。

五十二、事在人为

书上说。“集体农庄有形成级差地租的经济条件和自然条件。(577页)级差地租不完全由客现条件决定的,其实还是事在人为。例如河北省内京汉沿线的机井很多津浦沿线的机井却很少。自然条件相似,津浦一样方便,但是土地的改良却是各有不同.这里可能有土地利于不利于改良的原因,也有可能存在着不同的历史原因。但是最重要的还是“事在人为’。

同是上海郊区,有的养猪养的好,有的却养的不好,上海崇明县原来说那里各种自然条件,例如芦苇多,不利于养猪,现在打破了畏难情绪。对养猪事业采取了积极态度以后,却看到自然条件不但不妨碍养猪,反而有利于养猪。实际上,精耕细作,机械化,集体化也都是事在人为。北京昌平县常闹水旱灾害,修了十三陵水库。情况改变了。这不是“事在人为”吗?河南省计划在一九五九年,一九六○年以后再用三年治黄河。完成几个大型水渠的建设。也就证明“事在人为”。

五十三、关于运输和商业

运输包装不增加价值,但是增加使用价值,运输包装所用的劳动是社会必要劳动的一部分,没有运输包装,生产过程就没有完成,不能转到消费过程,使用价值虽然生产出来了。也不能实现。例如煤炭,在矿山开釆出来了,如果还在矿山,不用铁路,轮船、汽车运输到用户手中,煤炭的使用价值是完全不能实现的。

585页上说,他们的商业系统有两套,即国营商业和合作社商业,此外还有所谓“无组织的市场”。即集体农庄市场。我们是一套,把合作商业合并于国营商业,现在看来,一套好办事.并且各方面节省得多。

587页提到对商业的公共监督。我们对商业的监督,主要依靠党的领导,政治挂帅,群众监督这一套。商业工作人员的劳动是社会必要劳动,没有他们的劳动,生产就不能转化为消费(包括生产的消费和生活的消费)。

五十四、关于工农业并举

623页上说到生产资料优先增长的规律,未修订的三版本中在这里还特别提出“生产资料优先增长意味着工业的发展快于农业”。

工业的发展是快于农业,但是提法要适当,不能把工业强调到不适当的地位,否则一定会发生问题,拿我们的辽宁来说,这个省的工业很多,城市人口占全省人口的三分之一.过去总是把工业放在第一位,没有同时注意大力发展农业,结果本省的农业不能给城市保证粮食,肉类、蔬菜的供应,总是要从别省运粮食,运肉类,运蔬菜,主要的问题是农业劳动力紧张。没有必要的农业机械,使农业的生产受到限制,增长较慢。我们过去没有了解到,恰恰是东北这样的地方,特别是辽宁这样的省,应当好好抓农业,不能只强调抓工业。

我们的提法是在优先发展重工业的条件下发展工业和发展农业同时并举。所谓并举。并不否认优先增长,不否认工业发展快于农业。同时,并举也不是平均使用力量。例如今年我们估计可能生产钢材一四OO万吨左右,拿出十分之一的钢材来搞农业技术改造和水利建设,其余十分之九的钢材主要还是用于重工业和交通运输业的建设,这在今年的条件下就是工农业并举了。这样做当然不会妨碍优先发展重工业和加快发展工业。

波兰有三○○○万人口,只有四十五万头猪。现在肉类供应非常紧张,看来波兰现在还没有把发展农业放到议事日程上来。

624页上说。“在个别时期,为了提高落后的农业、轻工业和食品工业部门。消除他们的落后现象和克服因此而造成的局部比例失调现象,加速这些部门的发展在实际上可能是必要的合适当的。”这是好的。但是农业和轻工业的落后,所造成的比例失调。不能说成只是“局部比例失调”,这种比例失调不是局部的问题。

625页上说。“必须合理的分配投资。使重工业和轻工业不论何时都保持正确的比例关系。”这段只讲重工业和轻工业。没有讲工业和农业。

五十五、关于积累水平问题

在波兰,这个问题现在成为很大的问题。哥穆尔卡起初强调物质刺激,增加工人工资,不注意提高工人的觉悟,结果工人只想多要钱,不好好干。工资的增长超过了劳动生产率的增长,造成了吃老本的情况。现在逼着他们不得不出来反对物质刺激,提倡精神鼓励,哥穆尔卡也说。“钱买不到人心”。

特别强调物质刺激,看来总是难免走向自己的反面。开了很多支票,高薪阶层当然满意。广大工人农民要求兑现而不能兑现的时候,就会被迫地走到强调物质刺激的反面。

根据631页上所说情形,苏联积累资金约占国民收入的四分之一。我国积累占国民收入的比重。1957年是27%,58年是36笫,59年是42%,看来今后我国积累比重经常保持在30%以上或者更多是可能的。主要的问题是生产大发展,只要生产增加了。积累比重大一点,还是可以改善人民生活的。

厉行节约,积累大量的物力和财力,这是经常的任务,如果以为只是在很困难的情况下。应当这样做,那是不对的。难道困难少,就不要节约,不要积累了么?


