\section[在八人二次会议代表团团长会议上的讲话(一九五八年五月十八日)]{在八人二次会议代表团团长会议上的讲话}
\datesubtitle{(一九五八年五月十八日)}


写一个近三百年来的各种科学家发明家的小传。写明其年龄,出身、简历等,看看是不是都是没有多少学问的人,各行业搞各行业的。

科学家华罗庚是个中学生。

苏联搞出人造卫星的齐奥尔科夫斯基。是一个不出名的中学教员。主要是教数学。搞卫星是他的付业,慢慢搞成了专业事。

当然。美国也有发明,但发明者不是杜勒斯。究竟是什么样的人?不知道。

一个人能发明什么,学问不一定好多,年龄也不一定大,只要方向是对的。二、三十岁敢于幻想,人学问多了,不行了。

白蚂蚁全世界没有办法.广东一个只读过中学的青年学生,想出了办法。

中国楚人卞和(即“和氏之璧”的卞和)得璞玉于楚山,献于厉王。被割左脚;又献于武王,被割右脚,文王就位时。第三次把璞玉于荆山之下。经过玉石匠割开,才识此玉。“完璧归赵”就是这个璧。

瓦特是个工人。

富兰克林是小报童。

种试验田要三结合一一领导、技术人员、老农(老工人)只有外行才能领导内行.’

总而言之,我这些材料要证明这一条。就是贫贱者最聪明,高贵者最愚蠢。要用这些材料来剥夺那些翘尾巴的高级知识分子的资本。要少一点奴隶性。多一点主人翁的自尊心,鼓励工人、农民、老干部,小知识分子的自尊心,自己起来创造。

我曾问过一些人,我们是不是在天上?算不算神仙?是不是洋人?大家答复都是否定的,他们就有迷信。

两次讲话,一是破除迷信,二是讲国际国内形势,第三是讲灾难。

国际形势总的说来是一片光明。但也可能有战争。

国内形势是和五亿农民的关系问题。农民是同盟军,不抓农没有政治,不注意五亿农民的问题,就会犯错误;有了这个同盟军,就会胜利。列宁也是很强调工农民主专政。便农民半无产阶级觉悟起来,不断革命。有人以为要八十年发展资本主义,等待工人多了,农民觉悟了,才能搞社会主义。但实践证明,从民主革命到社会主义革命不需要几十年的间歇。苏联二月革命,证明列宁是正确的。中国则更不同,我们有了几十年民主革命的经验,解放后的农民,精神振奋,农村半无产阶级三亿五千万人。中国党内相当多的人不懂得农民问题的重要性,跌斤斗还是住农民问题上。不相信多快好省,首先是不相信四十条。不相信农业发展可以相当快。

为什么讲“十大关系”,“十大关系”是基本观点,就是同苏联比较。除了苏联的办法以外,是否还能找到别的办法,比苏联、欧洲各国搞得更快,更好。中国工业化道路,大、中、小工业同时并举,不提和苏联比,实际上是和先生比。我们有两个生身父母。一个是国民党的社会,二是十月革命。群众路线,阶级斗争是学列宁的。对资产阶级彻底消灭(包括思想在内),但不没收。人不灭掉,斯大林不搞群众路线,搞恩赐观点,阶级斗争又过分了。

国内形势主要是农民问题:水、肥、土、种、密、深翻土,长葛县的办法是一个典型。

我主张重工业,冶金、机械、化学、煤炭、外贸都要讲一下。宁可迟一点闭会。

二十六个省、市、区,十三个发生问题。政法系统,文艺系统很乱,是全国性的。

两种:一种是右派分子。一种是右倾机会主义。犯路线错误的,允许革命。对于潘××、古大存,冯白驹,这次会议都不处理为好,提出处分是正义的。不处理也对。

有几个文件印一下。王明和库西宁的谈话,天津一个支书和南京大学党委书记给主席的信。

天津支部书记很好,没有软下去。因为过去没有告诉那么多人,放不放,在那时是跟不上的。清华大学烂掉一个支部。

还有《中国农村的社会主义高潮》的按语印一下,其中讲所有制基本解决就对了。但在相互关系即在政治战线上和思想战线上没有解决,估计社会主义革命已经取得基本胜利是有点过分乐观,料不到还要搞这样的大革命。至于中国资产阶级,估计会有斗争,要用长期的斗争来肃清资产阶级及其知识分子的深厚的影响。单独一场政治战线、思想战线的社会主义革命是免不了的,在所有制基本解决以后必须另搞一次。这次这场革命,没有料到几个月就解决,整风提前也没有料到。当时的整风是由于形势逼着而来的。同资产阶级斗争是必然的,但“反冒进”也促出了右派的进攻。借右派之力来整风,我有自觉的。放出来再想办法,斗几下再说。青岛会议印成了一个材料。华东师范大学的一个学生很坚决。提出:共产党倒了怎么得了?

上面讲的是农民问题。城市人口百分之十五,农村人口百分之八十五,有些同志在农村混了几十年,农民的感情没有感染他们,不了解农民的心。不了解群众,就看不到好东西。潘××等,你说他们在农村没有搞过吗?就感化他们不了。

工人一千二百万,加上家属只有几千万,无沦如何没有农民多。富裕农民不跟我们走的有几千万,经过大跃进后,反对社会主义的可能有百分之五十,其中坚决的百分之二十。

选举有什么意见?头上不长角不好,多了也不好。牛角两只角,正好,四个就多了。候补选一批就平衡了,候补多几个也没有问题。

工农兵,农民有几亿。托派从来就骂我是“农民主义”。帝国主义也说我搞农民革命。中国工人阶级不抓农民,就坐不稳。列宁也强调农民问题,也是“农民主义”吗7我和欧洲同志谈;你们怎样?欧洲情况,除了农业工人以外,有自耕农(许多自己都有农业机械)对社会主义抵触大。同南美和印度同志谈,也是谈争取农民问题。要他们去研究一个农村。弄清阶级关系,解剖一个麻雀。落后有落后的好处。

大跃进不要太紧,红专学校学生上课都打瞌睡,这怎么行7中央苏区二次反围剿,两个星期打了五次仗,很少睡觉,但这只是一个短期的突击。要注意不要太紧,

协作区如何搞?是否定之否定。

大会后休息一天,再开两天会,每省一个,中央若干人参加就行了。

八月五日再开会,我们有两个整月的时间抓工业、商业、文敔、军队。现在就要准备和布置秋后的农业生产。

除四害,全民大动员,五岁的小孩都调动起来了。

各省党内出了问题的,要写一篇发言,有时间就讲,无时间就书面印发。这次会议的发言印一本丛书,事实上是自己评比,许多好经验非常丰富。

广东新会县商业工作搞得好,可以到那里开一次现场会议。

深翻学长葛县。一年不行,三、四、五年翻一次,总是可以的,增产一倍。

八亿人口,十亿也不怕。美国记者说,一百年后中国人口占世界的一半。那时文化高,都是大学生。很自然会节育。中国地势条件好,东边大海西边大山。

中国有自己的语言,如“共产主义”、“帝国主义”这两个词。苏联,美英等国的读音都是基本相同的,我们就和外国的读音完全不同。自从秦始皇以来,从来不把外国人放在眼里。过去谓枣秋之邦;到了满清末期,外国人打进来了。打怕了,都变了奴隶,感觉不行了。从前骄傲,现在又太谦虚,来个否定之否定。

明年一千一百万吨钢,世界就会震动。如果五年能达到四千万吨。可能七年赶上英国,再加上八年就能赶上美国。

中央一年抓四次,一次党代会。省抓六次,两次大检查,小抓四次。

看到农民瞒产我高兴。你多我也多,农民有,就等于我们有。

森林下放,地方包干。

竹子要大发展。北方不长竹子,不知从什么时候起。


