\section[在成都会议上的讲话(三)(一九五八年三月二十日)]{在成都会议上的讲话(三)}
\datesubtitle{(一九五八年三月二十日)}


我讲四个问题:

一、改良农具的群众运动,应该推广到一切地方去,它的意义很大,是技术革命的萌芽,是一个伟大的革命运动。因为几亿农民在动手动脚,否定肩挑的反面,一搞就节省劳动力几倍,以机械化代替肩挑,就会大大增加劳动效力,由此而进一步机械化。中国这么大的国家,不可能完成机械化,总有些角落办不到,一千年,五百年,一百年,五十年,总有些还是半机械化,如木船;有一部分手工业,过几万万年还会有的,如吃饭,永世是手工业,它同机械化是对立的统一,只是性质不同,应当结合起来。

二、河南提出一年实现四、五、八,水利化,除四害,消灭文盲,可能有些能做到。即使全部能做到,也不要登报,二年可以做到,也不要登报,内部可以通报。像土改一样,开始不要登报,告一段落再登。大家抢先,会搞得天下大乱,实干就是了。各省不要一阵风,说河南一年,大家都一年,说河南第一,各省都要争个第一,那就不好。总有个第一,“状元三年一个,美人千载难逢”。可以让河南试验一年。如果河南灵了,明年各省再来一个运动,大跃进,岂不更好。

如果在一年内实现四、五、八,消灭文盲,当然可能缺点很大,起码是工作粗糙,群众过份紧张。我们做工作要轰轰烈烈,高高兴兴,不要寻寻觅觅,冷冷清清。

只要路线正确——鼓足干劲,力争上游。多、快、好、省(这几句话更通俗化)。那么后一年、二年、三年至五年完成四十条,那也不能算没有面子,不能算不荣誉,也许还更好一些。比,一年比四次,合作化逼得周小舟紧张的要命,四川的高级化,×××从容不迫。不慌不忙,到一九五七年才完成,情形并不坏。迟一年有何关系?也许更好些。一定要四年、五年才完成,那也不对,问题是看条件如何,群众觉悟提高没有?需要多少年,那是客观存在的事情。搞社会主义有两条路线:是冷冷清清、慢慢吞吞好,还是轰轰烈烈、高高兴兴的好?十年、八年搞个四十条,那样搞社会主义也不会开除党籍。苏联四十年才搞那么点粮食和东西,假如我们十八年能比上四十年当然好,也应当如此。因为我们人多,政治条件也不同,比较生动活泼,列宁主义比较多。而他们把列宁主义一部分丧失了,死气沉沉。列宁在革命时期的著作,骂人很凶,但是骂得好,同群众通气,把心交给群众。

建设的速度,是个客现存在的东西,凡是主观、客观能办到的,就鼓足干劲,力争上游,多、快、好、省,但办不到的不要勉强。现在有股风,是十级台风,不要公开去挡,要在内部讲清楚,把空气压缩一下。要去掉虚报、浮夸,不要争名,而要务实。有些指标高,没有措施,那就不好。总之,要有具体措施,要务实。务虚也要,革命的浪漫主义是好的,但没有措施不好。

三、各省、市、自治区两个月开一次会,检查总结一次。开几个人或十几个人的小型会。协作区也要二、三个月开一次会。运动变化很大,要互通情报。开会的目的,为了调整生产节奏,一波未平,一波又起,这是快与慢的对立的统一。在鼓足干劲,力争上游,多、快、好、省的总路线下,波浪式的前进,这是缓与急的对立的统一,劳与逸的对立的统一。如果只有急和劳,则是片面性,专搞劳动强度,不休息,那怎么行呀?做事总要有缓有急,(如武昌县书记,不看农民情绪,腊月二十九还要修水库,民工跑了一半)也是苦战与休整的统一。从前打仗,两个战役之间必须有一个休整,补充和练兵。不可能一个接一个打,打仗也有节奏。中央苏区百分之百的“布尔什维克化”,就是反休整,主张“勇猛果断,乘胜直追,直捣南昌”,那怎么行?苦战与休整的对立统一,这是规律,而且是互相转化的,没有一种事情不是互相转化的,“急”转化为“缓”,“缓”转化为“急”,“劳”转化为“逸”,“逸”转化为“劳”,休整与苦战,也是如此。劳和逸,缓和急,也有同一性;休战与苦战也有同一性。睡眠与起床也是对立的统一,试问谁能担保起床以后不睡觉?反之,“久卧者思起”。睡眠转化为起床,起床转化为睡觉。开会走向反面,转化为散会,只要一开会就包含着散会的因素,我们在成都不能开一万年会。王熙凤说:“千里搭长棚,没有不散的席。”这是真理。不可以人废言,应以是否为真理而定。散会后,问题积起来了,又转化为开会。团结,搞一搞意见就有分歧,就转化为斗争,发生分歧,重新破裂,不可能天天团结,年年团结。讲团结,就有不团结,不团结是无条件的,讲团结时还有不团结,因此要作工作,老讲团结一致,不讲斗争,不是马列主义。团结经过斗争,才能团结,党内、阶级、人民都一样,团结转化为斗争,再团结。不能光讲团结一致,不讲斗争、矛盾。苏联就不讲领导与被领导之间的矛盾。没有矛盾斗争,就没有世界,就没有发展,就没有生命,就没有一切。老讲团结,就是“一潭死水”,就会冷冷清清。要打破旧的团结基础,经过斗争,在新的基础上团结。一潭死水好,还是“不尽长江滚滚来”好?党是这样,人民、阶级都是这样。团结、斗争、团结,这就有工作做了。生产转化为消费,消费转化为生产,生产就是为了消费,生产不仅为了其他劳动者,而且自己也是消费者。不吃饭,一点气力没有,不能生产,吃了饭有了热量,他就可以多做工作。马克思说:生产就包含着消费。生产与消费,建设与破坏,都是对立的统一,是互相转化的。鞍钢生产是为了消费,几十年更换设备。播种转化为收获,收获转化为播种,播种是消费种子,种子播下后,又走向反面,不叫种子,而是秧苗,收获,收获以后,又得到新的种子。

要举丰收的例子,搞几十、百把个例子,来说明对立的统一和相互转化的概念,才能搞通思想,提高认识。春夏秋冬也是互相转化的,春夏的因素,就包含在秋冬中。生与死也是互相转化的,生转化为死,死物转化为生物,我主张五十岁以上的人死了开庆祝会,因人是非死不可的,这是自然规律。粮食是一年生植物,年年生一次,死一次,而且死的多越生得多。例如猪不杀掉,就越来越少了,谁喂呢?

简明哲学词典,专门与我作对,它说生死转化是形而上学,战争与和平转化是不对的,究竟谁对?请问,生物不是由死物转化的,是何而来?地球上原来只有无机物,以后才有有机物,有生命的物质都是氮、氢等十二种元素变成的,生物总是死物转化的。

儿子转化为父亲,父亲转化为儿子,女子转化为男子,男子转化为女子,直接转化是不行的,但是结婚后生男生女,还不是转化吗?

压迫者与被压迫者相互转化,就是资产阶级、地主与工人、农民的关系,当然.我们这个压迫者是对旧统治阶级讲的,这是阶级专政而不是讲个人压迫者。

战争转化为和平,和平是战争的反面,没有打仗是和平,三八线一打是战争,一停战又是和平,军事是特殊形势下的政治,是政治的继续,政治也是一种战争。

总而言之,量变转化为质变,质变转化为量变,欧洲教条主义浓厚,苏联有些缺点,总要转化的,而我们如果搞不好,又会硬化的。那时如果我们工业搞成世界第一,就会翘尾巴,思想就会僵化。

有限变为无限,无限转化为有限,古代的辩证法转化为中世纪的形而上学,中世纪的形而上学转化为近代的辩证法,宇宙也是转化的,不是永恒的,资本主义到社会主义,社会主义到共产主义,共产主义社会还是要转化的,也是有始有终的,一定会分阶级,或者要另起个名字,不会固定的。只有量变没有质变,那就违背了辩证法。世界没有什么东西不是发生、发展和消灭的。猴子变人,发生了人,整个人类最后是要消灭的,它会变成另一种东西,那时候地球也没有了。地球总是要毁灭的,太阳也要冷却的,太阳的热现在就比古代冷得多了。冰河时期,二百万年变一次,冰河一来,生物就大批死亡,南极下面有很多煤炭,可见古代是很热的,延长县发现有竹子的化石(宋朝人说,延长古代是生长竹子的,现在不行了)。

事物总是有始有终的。只有两个无限:时间、空间无限。无限是有限构成的,各种东西都是逐步发展,逐步变化的。

讲这些就是为了展开思想,把思想活泼一下,脑子一固定,就很危险。要教育干部,中央、省、地、县四级干部很重要,包括各系统,有几十万人。总而言之,要多想,不要老想看经典著作,而要开动脑筋,使思想活泼起来。

四、社会主义的建设路线,还在创造中,基本观点已经有了。全国六亿人,全党一千二百万人,只有少数人,恐怕只有几百(万)人,感觉这条路线是正确的,可能还有很多人将信将疑,或者是不自觉的。例如农民搞水利,不能说他对水利将信将疑,但他对于路线则是不自觉的。又如除四害,真正相信者,现在逐渐多起来了,连我自己也将信将疑,碰到人就问:“消灭四害能否办到?”合作化也是如此,没有证明此事就要问。再有一部分人根本不信,可能有几千万人(地、富、资产阶级、知识分子、民主人士以及劳动人民内部和我们干部中的一部分)。现在已经使得少数人感觉到这条路线是正确的,对于我们来说,在理论上和若干工作的实践上(例如有相当的增产,工作有相当的成绩,多数人心情舒畅),承认这条路线是正确的。但是四十条,十五年赶上英国,这是理论,四、五、八大部尚未实现,全国工业化尚未实现,十五年赶上英国还是口号,一五六项尚未全部建成。第二个五年计划搞二千万吨,在我脑筋中存在问题,是好,还是天下大乱?我现在没有把握,所以要开会,一年四次,看到有问题就调节一下。建成后的形势无非是:大好、中好、不甚好、不好或者是大乱子。看来出乱子也不会很大,无非乱一阵,还会走向“治”,出乱子包含着好的因素,乱子不怕。匈牙利建设工业出了些乱子,现在又好了。

路线已开始形成,反映了群众斗争的创造,这是一种规律,领导机关反映了这些创造,提出了几条。许多事情是没有料到的,规律是客观存在的,不以人们意志为转移的。比如,一九五五年合作化高涨轰轰烈烈,没有料到有斯大林问题,匈牙利事情,“反冒进”,明年怎样?又会出什么事,反什么主义?谁人能料到?具体的事是算不出来的。

现在人们的相互关系,决定于三大阶级的关系:

第一个是帝国主义、封建主义、官僚资本主义、右派分子及其代理人,不革他们的命,就束缚生产力。右派占资产阶级分子中的百分之一或百分之二。其中大多数人将来可能改变,转化过来,那是另外的问题。

第二个是民族资产阶级,是指右派以外的那些人,他们对我们的新中国是半心半意的,半心被迫向我们,半心要搞资本主义,经过整风,已经有了改变,可能是三分天下有其二了(北京民主党派开自我改造誓师大会,全国都要开)。

第三个是左派,即劳动人民、工人、农民(其实是四个阶级,农民是另一个阶级)。

路线已经开始形成,但是尚待完备,尚待证实,不可以说已经最后完成。工人向农民摆阔气,有些干部争名誉、地位,都是资产阶级思想。不把这些问题解决,就搞不好生产,不解决这些相互关系,劳动怎能搞好?过去我们在建设上用的心思太少,主要精力是搞革命。错误还是要犯的,不可能不犯,犯错误是正确路线形成的必要条件。正确路线是对错误路线而言的,二者是对立的统一。正确路线是在同错误路线的斗争中形成的。说错误都可以避免,只有正确,没有错误这种观点是反马克思主义的,问题是少犯点,犯得小点。正确与错误是对立的统一,难免论是正确的。只有正确,没有错误,历史上没有这个事实,这就是否认对立统一这个规律,这是形而上学,只有男人没有女人,否定女人怎么办?争取错误犯得最少,这是可能的。错误多少,是高子和矮子的关系,少犯错误是可能的,应该办到,马克思列宁就办到了。


