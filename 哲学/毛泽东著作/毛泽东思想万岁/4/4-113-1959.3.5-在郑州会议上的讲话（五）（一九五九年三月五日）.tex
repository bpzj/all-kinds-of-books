\section[在郑州会议上的讲话(五)(一九五九年三月五日)]{在郑州会议上的讲话(五)}
\datesubtitle{(一九五九年三月五日)}


放一大炮是否灵,放对了没有?

要拿王国藩穷棒子社对穷户、穷队、穷社,解决穷社、穷队、穷户问题。一是贷款,二是公共积累。国家每年拿出十亿解决这一问题,社工业少办,主要是解决这问题。共产主义没有饭吃,天天搞共产,实际是“抢产”,向富队共产。旧社会谓之贼,红帮为抢,青帮叫偷,对下面不要去讲抢,抢和偷科学名词叫做无偿占有别人的劳动。地主叫超经济剥削,资本家叫剩余劳动,也就是剩余价值。我们不是要推翻地主、资本家吗?富队里有富人,吃饭不要钱就侵占了一部分,这个问题要想办法解决,一平、二调、三收款,就是根本否定价值法则和等价交换,是不能持久的。过去汉族同少数民族是不等价交换,剥削他们,那时不等价还出了一点价,现在一点价也不给,有一点就拿走,这是个大事,民心不安,军心也就不安,甚至征购粮款也被公社拿走,国家出了钱,公社拦腰就抢。这些人为什么这样不聪明呢?他们的政治水平那里去了。问题是省、地、县委没有教育他们。整社三个月没有整到痛处,隔靴抓痒,在武昌会议时,不感到这个问题,回到北京感到了,睡不着觉,九月就充分暴露了,大丰收。国家征购粮完不成,城市油吃不到了。赵紫阳的报告和内部参考中的材料你们看到没有?我就不相信长江、珠江流域马克思主义就那样多?我抓住赵紫阳把陶铸的辫子抓到了。瞒产私分很久了,开始在襄阳发现,刘子厚谈话对我有很大启发,河北一月开党代会。开始搞共产主义,倾向于一曰大、二曰公,二月十三日就感到有问题,决心改变主意,但还没有接触到所有制问题。到山东谈了吕洪宾合作社。开条子调东西调不动,就让许多人拿秤去秤粮食,群众普遍抵制,于是翻箱倒柜;进而进行神经战,一顶帽子“本位主义”一框,你框农民就看出你没有办法了,他也不在乎,这一着神经战也不灵,一张条子,一把秤,一顶帽子三不灵后才受到了教育,才用一把钥匙,解决思想问题,但也没有接触到所有制,河南说”虽有本位主义情有可原,不予处分,不再上调”,安徽说“错是错了,但不算错”。什么叫情,情者情况也,等价交换也,不是人家本位主义,而是我们上级犯了冒险主义,翻箱倒柜,“一平、二调、三收款”,一张条子,一把秤,一顶帽子,这是什么主义?人往高处走,水往低处流,“老弱转乎沟壑,壮者散而四方”那里有钱就往那里跑。你不等价交换,人家人财两空,吕鸿宾改变主意,一张安民布告,一个楼梯下楼,要下楼,首先要下楼的是我们,就是解决所有制问题。

土地属谁所有,劳动力属谁所有,产品就属谁所有。农民历来知道土地是搬不走的,不怕,但劳动力,产品是可以搬得走的,这就怕了。拿共产主义的招牌,实际实行抢产,如不愿不等价交换,就叫没有共产主义风格,什么叫共产主义,还不是公开抢?没有钱嘛!不是抢是什么?什么叫一曰大、二曰公?一曰大是指地多,二曰公是指自留地归公。现在什么公?猪、鸭、鸡、萝卜、白菜都归公了,这样调人都跑了。河北定县一个公社有七、八万人,二、三万个劳动力,跑掉一万多。这样的共产主义政策,人都走光了。劳动力走掉根本原因是什么,要研究。吕鸿宾的办法,还是一个改良主义的办法,现在要解决根本问题——所有制问题。

整了三个月社,只做了一些改良主义工作,修修补补,办好公共食堂,睡好觉,一个楼梯,一张布告之类,但未搞出根本性办法。要承认三级所有制,重点是生产队所有制,“有人斯有土,有土斯有财”,所有人、土、财都在生产队,五亿农民都在生产队,上面只有几个工作人员。如不承认所有制,就立即破坏。我是事后诸葛亮,以前还未看到这个问题。在批转赵紫阳的报告,就有此思想。六中全会有好处,农民不怕中央了,认为中央好讲价钱,中央雇工是拿钱的,购粮油是拿钱的,征购不多,注意生活福利,八小时工作等。仇恨集中在公社,第二在县,县也雕了些人,调了些东西,县、社办那么多事干啥?所以,要对公社同志讲清楚,公社不要搞太多,十大任务做不完。你们有经验,你们过去不是骂中央统死统多吗?现在你们当了婆婆就打媳妇,就忘记了。现在中央已经改了。去年权力下放,说了不算,拿出一张表来你们才放心。现在你们领导之下的公社,就实行“一平、二调,三收款”,调,一曰物、二曰人。当然出卖劳动力,不是出卖给资本家,而是出卖给中央、省、县、公社,但也要等价交换。过去长沙建筑工人罢工,我们叫增加工资,他们叫涨价,那是1921年的事,到现在38年了,我们还不懂涨价这个道理吗?劳动力到处流动,麿洋工,对这点我甚为欣赏,王任重很紧张无心跳舞,一夜才转过来,放一炮,瞒产私分,劳动力外逃,磨洋工,这是在座渚公政策错误的结果。上千万队长级的干部很坚决,几万万社员拥护他们的领袖,所以立即下决心瞒产私分。我们许多政策引起他们下决心这样做,这是合法的。我们领导是没有群众支持的。当然也包括桌椅板凳,刀锅碗筷,去年工业抗旱,大闹钢铁,献工献料,什么代价也没有。此外,还要拿人工,专业队都要青年,还有文工团都是青年,队长实在痛心,生产队稀稀拉拉。这样下去一定垮台,垮了也好,垮了再建。无非是天下大笑。我代表一千万队长级干部,五亿农民说话,坚持搞右倾机会主义,贯彻到底,你们不跟我来贯彻,我一人贯彻,直到开除党籍,也要到马克思那里告状。严格按照价值法则,等价交换办事。三级所有制,改变为基本公社所有制部分队所有制,要有一个过程,还要三、五、七年。要穷队赶上来,穷队变富队,穷变富每个省都可以找到例子,像王国藩那样,最大的希望是穷队,不能把苏联的钢砍给我们二千万吨,如果这样,苏联也好造反,世界上的事没有不交换的,人同自然界作斗争,也有交换,如人吃东西,吸空气,但要拉屎拉尿,新陈代谢。吃空气,一分钟十八次,有吸必有呼,你交还自然多少二氧化炭、皮肤散热,这也是等价交换。大鱼吃小鱼,小鱼吃大鱼的屎,重工业各部门之间也要等价交换。赵尔陆造机器要原材料,就是粮食,机器就是他拉的屎。纺织工业出纱要棉。基建也是如此,吃投资就能出工厂,总要相等就是。王鹤寿不给他交换焦炭矿石,就拉不出钢铁。物质不灭,能量转化。要科学。夏热冬寒,一切都等价交换。国家给钱,就是公社不给钱。犯了个大错误。××同志讲,云南提出供给与工资比例是三比七。这个原则在武昌会议是讲了的。六中全会的东西现在有许多没有执行,就是否定价值法则,所谓拥护中央是句空话,起码暂时还难说,其实是不通。无代价的上调是违反中央的,要搞工业,不搞农业,未到期的贷款都收回了,是不是中央不两条腿走路?相反,今年要增加十亿,一部分是可以收的,贫农贷款是四年,60年才到期,现在就收回了。我看这可以给人民银行行长戴一顶帽子,叫做破坏农业生产,破坏人民公社,也不撤职。全部退回,到期不到期的都退,你们可以打个折扣,到期的可以不退。我为了对付你的全部收回,我就来个全部退回,你要左倾,我要右倾。就是到期还可以延长。

人民公社正在发展,需要支持,不能拦路抢,李逵的办法,文明的办法叫做“剪径”,绿林豪杰叫“剪径”,现在绿林豪杰可多了,你们是否在内。对付剥削者无罪,绿林的理由叫“不义之财,取之无碍”如生辰纲,我们也干过,叫打土豪。后来者文明一点收税。成吉思汗,占了中国,不会收税。叫“打谷草”无代价抢劫人民,结果打走了他们自己。辽金也如此。蒙古是世界第一个大帝国。除了日本、印尼外,占了整个亚洲和大半个欧洲。第二是英国,日不落国。第三是希特勒,占了整个欧洲,半个苏联,还有北非。现在是艾森豪威尔最大,实际控制整个西欧,整个美欧、澳洲、新西兰、东南亚、印度,对印尼也在天天增加投资。科伦坡国家也在旧金山开会,可厉害了,美国控制的地区超过成吉思汗,伊拉克7月14日革命成功,美国15日占领黎巴嫩。我们8月23日打炮,他立即调部队集中太平洋,杜勒斯说是最大一次集中。他的战争边缘政策主要是对付我们。我们也可以学一点,你边缘我也边缘。打了三个月,他失败了,我宣布领海十二海里,他只承认三海里。我警告卅多次,他国内外都不满意,我说一千次也不打,记一笔账,这是对付流氓的办法。后来挂了卅几笔账,他就不来了,手忙脚乱,不知道我们为什么要这样干,我们是十个指头按一个跳蚤,美国是十个按一百个跳蚤因此都按不住。中国、伊拉克都按不住。中国是一个“大跳蚤”。

打土豪大概从打草谷学来的。美国统治时,后来有人建议打草谷不如收税,收税才能发展生产,繁荣经济、不知比打草谷强多少倍。现在公社党委实际上是恢复蒙古打草谷的办法。落后的抢劫办法。过去打土豪是正确的,“不义之财,取之无碍”和宋江一样,现在对农民能这样吗?唯一的办法只能等价交换,三级之间要有买卖关系,劳动必须出工资,义务劳动切不可太多。

王安石创始免役法,把服劳役改为征税,由政府雇人,出工资,作各种服役的事业,这是很进步的办法。我们退到王安石以前,退到司马光的办法了。司马光是代表大地主,反对王安石的办法的。公社可办对社有利的工业,但雇人要出工资。一种是固定工人;另一种非固定工人,这部分人不能太多,技术工人要有较高工资。亦工亦农的,待遇应与农民不同。工业、教育、体育只能一年一年地发展,量变有一个过程。写诗不能每人都写,要有诗意才能写诗,如何写呢?叫每人写一篇诗,这违反辩证法。专业体育、放体育卫星、诗歌卫星,通通取消,遍地放就没有卫星了,苏联才有三个卫星呢。

你们认为怎样才能巩固人民公社?一平、二调、三收款,还是改变。我看这样下去公社非垮台不可。斯大林为什么改变公社的办法?他们觉得浪费太多,义务交售制,余粮征集制不能刺激生产,才改为粮食税。斯大林三十年之久实际没有实行集体所有制,还是地主超经济剥削,拿走农民的70%,因此,三十年还是只能进行单纯的再生产。俄皇时代,无机械化和集体所有制。斯大林搞了这两点,粮食产量和沙皇时代相等。那时可能是为了搞重工业,留的只够农民吃,无力扩大再生产。当然不是斯大林一个人的问题,而是有一批热心于搞重工业、搞共产主义。我们是办公社工业,如果这样搞下去,非搞翻农民不可。任何大跃进、中跃进、小跃进也不可能,生产就会停滞。

搞三、五、七年,来个过程,基本上以原来的高级社为基础,等价交换,不能乱开条子。队与队是买卖关系,若干调剂要协商。灾队、穷队没有饭吃由省解决。

一个是瞒产私分,一个是劳动力外逃,一个是麿洋工,一个是粮食伸手向上要,白天吃萝卜,晚上吃好的,我很赞成,这样做非常正确。你不等价交换,我就坚决抵制,河南分配给农民30%,瞒产私分15%,共45%,否则就过不了生活,这是保卫他们的神圣权利,极为正确。还反对人家本位主义,相反应该批评我们的冒险主义。真正本位主义,只有一部分,主要是冒险主义。钱交给公社不交队,他们抵制,这不叫本位主义。给他钱。他不缴,才是本位主义。

安排时应把人民的生活安排在前面,要占百之几十,人民生活,公社积累(15—18%)国家税收(7%—10%),应同时安排,义务劳动要减少,公共积累要减少。多给一些社员看到的东西,减少供给部分,增加工资部分。粮食供给要坚持下来,“无竹令人俗,无肉令人瘦,若要不俗又不瘦,除非冬笋炒肥肉”。多种经营,付业生产都要归队办。

大问题是把六级干部会开好,公社党委来一个书记,管理区来二人,生产队来二人,都要一穷一富。河南简报要看两遍,这是现场会议。对穷队要讲王国藩。河北省遵化县鸡鸣村区,穷棒子王国藩社现在是一个大社,很富了。开始只有廿三人,三条驴腿,无车无粮。他的章程就是不要国家贷款,不要救济,砍柴卖,从此出了名,变为几十户,几百户。现在多少户了?各省都可以找出这样例子来。自力更生为主,外援为辅,由贫到富的社,各省都有。国家投资,第一是扶助工业,第二是扶助穷队。四六开或三七开。穷队占六到七。十亿人民币,三亿交公社,七亿交穷队。一是靠本身,二是靠公社,三是靠国家。穷人要有志气,送给我,我也不要,穷队有依赖思想,何应钦不发钱,我不搞生产如何行。

我们党过去有很多山头,逐步联合成为统一的党。军队也有几个山头,一方面军有两个山头,二方面军两个山头,陕北两个山头,四方面军四个山头。在延安党校,夕阳西下,散步时也分山头。上馆子吃饭也分山头。山头之内无话不讲,话不好给别的山头讲。在陕北甚至躲飞机时,外来干部和本地干部也分两条路走,要命时也不混杂。我们采取什么政策呢,要认识山头、承认山头、照顾山头、消灭山头。山头是历史原因和地区不同造成的。现在看山头消灭得差不多了。当时的共产党有个共同纲领,中央实际上是联合会。这些人都是好人,不是什么托洛斯基。教条主义者到处整人,苏区、白区都怕钦差大臣。批评人家为机会主义,夺取了党、政、军、财权,他是百分之百的布尔什维克,不准说敌强我弱,不准说泄气话,只能讲壮气的话,曾几何时(三年半)长征了。整得人人自危,怎么能有积极性呢?斯大林搞托洛斯基,反复几次,赫鲁晓夫不敢让莫洛托夫当中央委员,我们对待教条主义,釆取治病救人,团结同志的方针。七大之前七中全会决议,会前搞清问题,大会是开团结大会,错误让他自己讲。除了王明是个未知数,其余信任他们。

现在讲的是生产队山头。每个生产队是一个山头,不认识,不承认,不照顾,就不能基本消灭山头。英国是第一个帝国主义,现在美国超过了它。世界在变化。穷队也会变化,穷的搞得好,大多数会过富的。公共积累办的事业一年一年增多,将来可变为基本的社所有制,部分队的所有制永远会有的。作为一个过程来看,过去我们没有分析,武汉时没有分析,一、二月才分析。谢谢几亿农民瞒产私分,使我来想这个问题。要使公社一般懂得这个问题,这是客观法则,违反它就会碰得头破血流。如果我们不能真正说服他们,还是这样犹犹豫豫,公社就会垮,人就会跑。供给部分要少,工资部分要多,不要一县一社(修试除外)。一社统一集中分配,任意调人调东西,很危险。要迅速讲清楚;办法是开六级干部会。有人说富队会搞资本主义,我不信他能离开地球吗?如欲取之,必先予之,现在他就跑了。这还是人民内部矛盾,还没有动刀枪,会不会离心离德?照现在的情况有脱离太阳系的危险。现在我赞成跑,这样可以使我们警觉,将来就不会跑了。

已发文件作为初稿,我在河南取得经验,然后到武汉去,你们不要等,放手去作,基本观点不会变的。六中全会,缺少三级管理,队为基础,社与国家、社内队与队等价交换,这是认识问题。发现矛盾,分析矛盾,才能解决矛盾。发现是感觉,分析是理性,要有个过程,开头是接触,所谓分析就是揭露,解决是综合阶段。

一盘棋要三照顾。生产队有五亿人口,千万干部(队长、会计),得罪他们不得了。过去70万个小社,一社50个干部,则是三千万干部。瞒产私分为什么有那么大的劲,决心那么大,因为有五亿农民支持他们,我们则脱离了群众。认识这个问题,时间有五个月之久,相当迟,客现实际反映到主观,有个过程。

文件还要修改,但基本观点就是这样,你们可以照办。里面供给和工资问题没讲,劳动力盲目流入城市也未讲。

工人寄钱问题,中心是说服公社,不能拦路劫抢。军官寄钱回去,公社扣了,军官有很大反映。财产权利必须神圣不可侵犯,这样反而建设得快。要说服公社,懂得发展过程,懂得等价交换。邵大哥三支钢笔,将来不至三支,共产主义可能有十支。

城市办公社,我就想不通。天津人说,要办就办一个,人民代表大会就是人民公社嘛。企业学校都是全民所有制,至于要办食堂随你办,至于家属就业要怎么办就怎么办,已经是国有制还办人民公社干什么。小城市和县城还可以办。

有些东西,不要什么民族风格,如火车、飞机、大炮,政治、艺术可以有民族风格。干部下放,军官当兵,五项并举,蚂蚁啃骨头,是中国香肠,不输出,自己吃,这是马列主义,没有修正主义。公社倒是有修正主义,拦路劫抢、不等价交换。一平二调三提,不是马列主义,违反客观规律,是向“左”的修正主义。误认社会主义为共产主义,误认按劳分配为按需分配,误认集体所有制为全民所有制,想快反慢。武昌会议时,价值法则,等价交换,已弄清,但根本未执行,等于放屁。

城市公社问题,(1)小城市可以搞;(2)中等城市没有搞的不搞,已成立了的不要一下解散,可以试办;(3)大城市不搞。


