\section[接见参加全军政工会议的各军区负责同志时的谈话(一九五八年十二月二十三日)]{接见参加全军政工会议的各军区负责同志时的谈话}
\datesubtitle{(一九五八年十二月二十三日)}


要培养新生力量,没有年轻的哪能写好文件?

过去不是有人说干军队没有奔头吗?军队还是有奔头。养兵千日,用在一朝:战时有奔头。就是平时也有奔头,可以成为多面手,一专多能,学会很多有用的技术。

金门打炮对部队士气有没有影响?还是不要打?(答。部队情绪高)闹个人主义的时候再打他一下。我们对台湾海峡地区的这个政策,大家赞成不赞成?是不是很得人心?要就拿过来,要不一个也不要。单是金门我们就不拿。杜勒斯说,中国人不好谈,对金马无兴趣。假如杜勒斯丢下金马走了。我们怎么办?请你们想一想,你们都是政治家,一个办法是去占,我们可以控制台湾海峡一半。便于南北的海运,一个是不去,搞个无人地带,以表示我们兴趣不在这里,而在台湾。这时一定会有很多舆论,舍不得。我们不去,很可能蒋介石又来占领,这更好。我们不要他走。我们一打炮,蒋介石就有理由不走。他会说,你看,共产党还打着炮,我们在炮火下撤退。你美国人多没面子啊!

现在有许多干部、战士想不通,为什么称贵我双方?为什么不打敌人司令部,为什么还答应在对方请求的时候供给他们“以固守”?同志们可以议一议。如果金门的敌人其叫我们供给,那倒是好事,那一点兵我们养得起。

△在此次军工会上,可拿一天时间谈谈形势问题。过去,《人民日报》的同志对我们对台澎金马的政策问题想不通,新华社的同志也讲不通,周总理报告了,分头研究,才搞通了。在此次政工会议上可以大鸣大放一天。这是政治问题,国际形势,离开形势,政治工作不好做。过去的规矩,第一是讲国际形势,第二是讲国内形势。第三是本部队。过去那些有点形式主义,如一点不搞就不好了。形势问题至少讨论一天。

△四国共管柏林,是德苏战争结束,苏联打得精疲力竭时,斯大林同意了的。尤金大使曾告诉我。苏德战争中苏联死了两千万人,可能包括战区的男女老幼。苏联近两亿人口,就算他有五千万壮丁。作战死去一千万壮丁。问题不算小。苏联同志们,斯大林在那几年中能取得那样的胜利,已经了不起。斯大林为什么不要我们和国民党打仗呢?因为国民党就是美国人。一方面他把美国夸大了。另一方面把我们看小了。后来斯大林承认他错了(一九四九年夏)。他曾问过×××,他说的话在中国有无影响?××说,没有影响,因为你们提出的,我们没有执行,也不能执行。一个共产党指挥一个共产党,奇事。因为当时没有第三国际了。其实蒋介石硬要打你,你不打怎么办?……美国也怕我们——怕我们蛮干,不怕死。别的他也知道,无非是手榴弹那一套。还怕我们的将来,今年是钢翻一番,粮食那么多。真让我们搞七、八年,可了不起。

△杜勒斯怕人民公社,而且怕得很。他总研究人民公社,骂我们两条。一叫拆散家庭,二叫强迫劳动。这也是战士关心的问题。家庭要不要拆散?有没有这个问题?劳动是不是强迫?如蒋介石强迫老百姓那样,还是自愿?我看基本上还是自愿的劳动,老百姓看到了这样的劳动结果能迅速增产。只要我们让他们睡足觉,每天睡八小时。一定要强迫,你不睡不行。

过去出工,自由主义,阴一个,阳一个,我们学军队的办法,准备出发,十分钟可以完吗?出发很快,整整齐齐。还是整整齐齐好,还是阴一个阳一个好?过去各家吃饭.这家煮得早。那家煮得晚,上工稀稀拉拉要三小时。现在一个食堂吃饭,议事。做政治思想工作也方便。上工整整齐齐,只要一小时。

△政工会上要谈谈人民公社问题,看看有些什么问题。(众答。房子问题,男女分居问题……)分居有几个星期,扎野营你不分居怎么办?这次人民公社运动比合作化要顺利些。由个体到集体比较难。合作化和现在的变化哪个大?

△(谈到部队内民主作风时)只许州官放火,不许百姓点灯。我们这样整整齐齐的军队还这样,公社刚办起来,有点缺点就议论纷纷。公道不公道?过去三大纪律八项注意。你三个月不搞就忘得干干净净,天下大乱。还不是每天搞,每个礼拜搞?

小锅子还是要恢复。可以炒菜,做饭,大集体,小自由嘛!自由主义搞光了不行。

对领导方法简单的农村工作干部要说服教育。过去,军队里的官长只要兵不听话就打,把打人当作最后的手段。后来我们实行官长不打士兵。许多班长就没办法了。经过说服教育慢慢地就改过来了。用说服教育的办法,兵还是可以带走的。红四军第九次代表大会不是写一个文件吗?现在看那样的文件,哪有现在这样进步?问题那么多.没有发挥。

公社一个连长要领五百人。实际上是一个营。有男女老幼。不如军队好带。

去年《关于正确处理人民内部矛盾的问题》中讲,第一个五年计划建立合作社。第二个五年计划巩固合作社。现在五六~五八年。加上明年四年之内使人民公社上轨道。使干部学会带兵——男女老幼,又管生产,又管生活,又管思想,又管精神,又管物质。

把过去的区乡干部转为公社干部。政社合一,县政府就不要了,县级干部转为县联社的干部。县委全就是县联社党委,加上下放干部。

武王伐纣,实行三化:组织军事化、行动战斗化、生活集体化,那时打仗从陕西到豫北,能各人自己起伙吗?得有三化。姜太公当政委,武王是司令员。那时恐怕还是奴隶时代。三化是军队发明的。组织不军事化怎么打仗?从来军队都是农民和手工业工人组成的。为什么公共食堂军队能搞得好,乡村搞不得?广州市有没有公共食堂?死了人没有?(朱光答:有食堂,死的都是该死的。)公共食堂一不死人,二不瘦人,甚至还胖一点,这点不会犯原则错误吧。

我没有想到今年搞人民公社,也没有想到过农村搞公共食堂。帝国主义那边造谣,说这都是我出的主意。

有些事情的发生是可以预料的,但有些也很难预料到。一九五五年搞农业发展纲要四十条的时候,谁料到一九五六年反“冒进”,一九五六年斯大林一整、波匈事件,一两周之内,天下大乱。这是坏事,没有料到。有好些事也没有料到,如今年的大跃进。去年九月中央全会恢复了多快好省的口号,去冬今春大搞积肥和兴修水利运动,这一搞,粮食增产……斤以上,共……亿斤,砍掉……亿斤,算……亿斤,有个保险系数。这件事我就没料到,人民公社也是没有料到的。南宁会议,成都会议,八大二次会议,北戴河会议时都没有想起人民公社。七月份还是没有想过。

钢铁翻一番的问题。是在游泳池里和王××同志吹的。我说试试看,他就发命令了。没有料到真正翻了一番。

还有军队起变化。今年一月份我找几位部队同志谈话时,罗荣桓同志讲有落后之感。地方向前走了,军队存在一些问题。多年来我也没有管军队,其实解决问题也很容易,军委开了五十五天会,以后军区开,军、师开,一下子问题解决了。现在空气不是改变了吗?还不是有希望吗?你们这次政工会议搞八个文件,是去年就计划好了吗?我不相信。还不是在今年形势发展的基础上才搞出来的。大跃进,干部当兵,民兵大发展是从北戴河会议之后搞起来的,一个多月,北京搞了几十个民兵师。许多事情都是这样。许多好事,坏事,事先不可能完全料到。只能大体上料到。

(谈到看问题要看到两种可能性时)坏事无非是打世界大战,扔原子弹。我们一个也没有。再有就是共产党分裂,分成两个中央、三个中央。有的省委分成两个是可能的,你们广东就有省委书记想搞分裂嘛?我们的国家还有灭亡的危险没有,不准备就被动。列宁总是不避讳。他常常说:要么胜利,要么灭亡。现在还有这么大的帝国主义,这个问题在世界范围内还没有解决。还是帝国主义灭亡,还是我们灭亡,中苏社会主义十二个国家是共同命运,无论如何有两个可能:一个是胜利,一个是暂时灭亡,部分灭亡。敬老院、公共食堂,也会崩溃一部分。没有坏处,这部分的失败有好处,可以从中得到教训。

你们准备垮台就不会垮台,至少是垮得少。你们可以进行整顿教育。你如果没有垮台的准备那就危险。

社员有退社的自由。我们这次决议上没有写。工厂的工人可以跳厂。鞍钢那么大,工人自己一个人要成立一个鞍钢是困难的。

上面讲的这些情况。似乎与东风压倒西风,帝国主义是纸老虎等论点不符合了。这还是符合的。

因为什么事情都有两个可能性。巩固或者崩溃,托儿所如此,敬老院如此,公社如此,甚至省委、中华人民共和国也如此。彻底崩溃是不会的,我们总还可以打游击,还有民兵在手里。回到延安,到四川去,到云南去。我与民主人士讲过,他们都笑笑而已。我对他们说,你们要准备,还是留在北京搞维持会呢,还是和我们一起去打游击呢?这样好像与美帝是纸老虎不符合了。所谓纸老虎,并不是他现在已经死了,不打是不会死的。武王不伐纣,纣不不会倒。帝国主义还活着。经过斗争。到最后它就死了。要经过斗争.中间有曲折,不会没风浪。

有两个可能:一个是不让敌人登陆。登陆了,可以打败它。或者就是登了陆也只占领少部分地方,这不能说中华人民共和国崩溃。苏德战争,德军打到莫斯科,列宁格勒城下,不是苏联崩溃。但是那是危险的。第二个可能也要估计到,要受损失,部分的,暂时的。

美、英、法、西德统治阶级闹别扭,对我们有利。资产阶级分两部分,西方的资产阶级和尼赫鲁、纳赛尔、苏加诺有矛盾。西方世界闹别扭,不带革命性;纳赛尔反帝是有革命性的,但是他们一面反帝,一面本国又有反动的部分,如印尼右派。

把各国因素合起来看,巩固因素第一。七年、十年。十五年不打仗有可能,但是也不能写保票,总还有百分之一的危险性,所以莫斯科宣言就把这种危险性写上了。

活老虎能转化为死老虎.铁老虎能转化为豆腐老虎。估计两种可能性,说东风压倒西风,说一切帝国主义都是纸老虎,就说通了,在座诸位基本上是健康的。但不是说你一年也不害一次感冒。害病和健康两方面都要估计到。

你们对我所讲的这些可能性有没有准备?河南有一个地委书记,听我说中华人民共和国还有崩溃的危险,他的面色发白。如果大家都不警惕都不好.美国的事情是杜勒斯在办。杜勒斯是美国政府的政治主任,政治委员、或者政委兼政治部主任。如果在座的哪位去当美国的国务秘书,就好办了。我们把杜勒斯的职务翻译成“国务卿”不对,实际上是国务秘书。不在乎名称如何,实际上杜勒斯是艾森豪威尔的政委,是艾的灵魂。杜勒斯这个人是个“好人”,办了不少“好事”,对无产阶级团结和同帝国主义作斗争很有益。他不从黎巴嫩登陆哪有活材料教育世界人民?我们一打炮,他从各处把海军舰队调来了,“在一个地方集中了很多的舰队”,这是杜勒斯在巴黎会议上说的。我们也没有料到,金门一打炮,全世界这么动。

我的讲话也有两种可能性,有人赞成,有人不赞成,可以议一议。一个人也有两种可能性,活下去,或者死了。广州军区一千台汽车三个月之内运输了许多物质是好事,但是有××人伤亡了,你事先想到了没有?有个县委书记被碰伤了,他事先也没有想到呢?部分的,暂时的损失随时都要准备。三个月之后总要进步一点,老是这样搞还行哪?

朝鲜打仗,研究时有时觉得主意也不多。在三八线有一个师不是被吃掉了一些?那不是活老虎?会吃人的。战术上,具体工作上绝对不要轻视敌人,一点也不能放松。


