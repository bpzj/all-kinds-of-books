\section[在军委扩大会议小组长座谈会上的讲话(一九五八年六月二十八日)]{在军委扩大会议小组长座谈会上的讲话}
\datesubtitle{(一九五八年六月二十八日)}


这次会议开得不错,有些同志的发言得好。(××专门请主席看几个同志的发言。主席看了张宗逊、刘亚楼等同志的发言。)张宗逊同志的发言很好,我赞成。这是经训总四级干部会议逼他写出来的,可见一逼就写出好东西了。只有一点我不同意,那就是张宗逊说他犯错误是因为没有很好学习毛泽东著作。这不对,应该说,主要是马列主义水平不高。这次亚楼同志的发言也可以。这说明军队同志是有水平的,可以写出东西来。最好组织一些军师级同志发言,写写东西,因为他们是做实际工作的,接触下面,写的东西能够理论与实际结合。会议内容应该丰富多釆,也要介绍工作中的先进经验,在文章中讲话时,不要批评苏联,教条主义是我们学习的问题,不是苏联先进不先进。

我军队开始就存在着两条建军路线的斗争。古田会议斗了一下,但没有说服有错误意见的同志,有的同志到今天还坚持着错误路线。肖克同志不仅有教条主义,而是个军阀主义,有资产阶级思想,教条主义、封建主义思想。

战争中按照苏军条令执行是不行的,还是搞自己的条令。不知道军事学院、训总到底有多少马列主义。马列主义本来是行动的指南,而他们当做教条来背,如果马克思列宁在的话,一定要批评他们是教条主义。现在教条主义者主张抄苏联,请问苏联当时是抄谁的?“八大”决议中有一章关于技术改革的问题,按照今天的发展情况来看,提得不妥当,就是过分强调依靠苏联的帮助。争取苏联的援助是很需要的,但主要的还是自力更生,如果过分强调依靠苏联援助,请问当时苏联又依靠谁来援助呢?

工农业大跃进,打破了迷信。我们在×年可以赶上英国,×年至×年赶上美国。明年我们钢产量,将达到××——××万吨。据说东北一九六二年即可产××万吨,这是整风的结果。南宁会议,成都会议破除了迷信,解放了思想,形成工业大跃进。军队训练已经八年多了,连一本战斗条令也没有搞出来。这次要集中一些有丰富工作、战斗经验的同志,要搞出一本自己的战斗条令来。有些人提出苏联顾问同志看我们不抄他们的,就提意见或者不高兴那么我们就可以问这些同志,你们抄不抄中国的?他们说不抄,我们就可以说,你们不抄,我们也不抄。

××为什么革命胜利后没有搞好呢?除了对前一段未深刻检讨,接受历史教训不够外,一是迷信旧的东西、旧教条,二是迷信洋教条,迷信苏联,三是迷信自己。这个人工作很积极,很努力负责,就是方向不对头,政治上不够强。这次会议主要是打倒奴隶思想,埋葬教条主义,以整风方式大鸣放大放,破除迷信,提高思想,吸取经验教训,主要是教育全党全军,团结全党全军。因此会议上可以指名批评,但我建议在写决议时只要达到分清是非,搞清问题,就不要写出犯错误同志的名字。古田会议的决议就没有写出名字嘛!

×主要是迷信洋人,有自卑威,没有打破迷信,不以自己为主。现在一个合作社也要注意总结自己的经验,不然就会落后。湖北省新洲五个合作社搞得较好,麻城差点,可是新洲没有注意把自己的经验总结起来,而麻城派人到新洲学习,结合自己的经验进行了总结推广,结果麻城工作跑到前面去了。军队过去打仗,还不是把下边打的经验总结起来,再去训练部队,又再去打仗吗?我们各种工作都要注意总结好的经验,加以推广。

苏联打败过十四个帝国主义的干涉,那很久了。苏联有二次世界大战的经验。我们打败过蒋介石,日本帝国主义、美帝国主义,我们有丰富的经验,比苏联的多。把自己的经验看得那么不值钱,是不对的。(林总插话:我们的经验很丰富,不要把黄金当黄土甩掉了。)要以我为主,学习别人的先进经验。同时要研究敌情、友情,过去我们就是研究敌、友、我的情况的。再翻译美国、日本的东西,将来美国在东方战争中不依靠日本是搞不起来的。因此,我们要很好地研究日本的情况。对苏军的经验是要学习的。装备技术天天在发展变化,学苏军的技术经验也要用发展的观点去学。过去俄国人很怕拿破仑,因为他领兵曾打到莫斯科,最后俄国人又把他打败了,所以俄国人经常宣传他们比拿破仑还厉害。目前苏军顾问搞的东西(作战计划、想(?)都是进攻的,都是胜仗的,没有防御,没有打败仗的。这是不符合实际情况的。有些人说,总结抗美援朝战争是经验主义。要知道朝鲜战争是个大战,我们打败了美帝国主义,获得了宝贵的经验,一定要总结。至于他们说我们是经验主义,那么,我们就说,你们搬苏联第二次世界大战的东西也是经验主义。

肖克同志的错误是严重的,过去没有这样的时机来开这样大的会议,今天有了这个时机,我们可以挖挖教条主义的根子。

关于学习苏联,对内讲“批判地学”,为了不引起误会,对外还是讲“有分析有选择地学习苏联的先进经验。”但是,最重要的是,学习苏联先进经验一定要和自己的独创相结合。马列主义的普遍真理与中国的实践相结合,不能吃现成饭。吃现成饭是要打败仗的。这一点要向苏联同志讲清楚。学习苏联,过去学、现在学,将来也要学,但学习要和我们的具体情况相结合。我同他们讲,我们学习你们,你们又是学那里的呢?为什么我们不能独创?苏联专家现在也有了转变,苏联二十次代表大会和朱可夫事件后有转变。(陈总插话:据归国苏联同志他们也讲:来的时候是带着我们的经验来的,回去的时候是带着你们的经验回去的。)这就说明了大跃进的形势,不但鼓舞了我们中国人民,同时也鼓舞了苏联同志。(林总说:我军在政治上,如党的领导,政治工作、优良传统,我们有一套。我们党的马列主义水平是很高的,主席更不要讲了。主席曾说我们写的社论比真理报的社论水平高。关于上层建筑问题。关于军事科学、战略问题。我们有系统的一套。列宁死得早,在这个问题上来不及搞,斯大林没有系统的一套,不必学苏联的,战术问题上半学半不学,他们的战术,思想性,群众观点有问题。半学,就是学海陆空军使用,诸兵种协同。半不学,如战术思想,我们有毛主席的,就不学他们的。技术科学,现代化战争组织要学,但也要用我们的群众路线的办法来学。要趁我们这班人还没有死之前。组织一批干部很好地把我们的一套搞出来,传授下去。)这样好。

李世民,曹操等,他们都是会打仗的。中国过去是有些东西的。凯风同志曾说《孙子兵法》中没有马列主义,我问他看了没有,他答不上。可见没有看过《孙子兵法》,就武断地下结论,是不妥当的。(林总插话:《孙子兵法》是有唯物论,有辩证法,《孙子兵法》是部集体创作的书,有孙子、孙膑、曹操、杜预等人。)

破除迷信是成都会议提出的,四个月来发展很快,八大二次会议后。更在全国全面开展,如鞍山,原计划产钢××万吨,但现在变了,到明年就可能达到××——××万吨钢。他们也搞大中小型结合,土办法洋办法结合。据×××同志从东北来信说:东北第二个五年计划可搞到××万吨钢。有了钢,有了现代化工业,现代国防工业就好办了。我赞成多生产一些轻武器,武装广大民兵。(林总插话:民兵很重要。)过去人家看不起我们,主要是因为我们粮、钢,机械少,现在搞出了东西给大家看看。

海军发展值得研究,刘导生发言稿中提出十年搞××万吨位,这太少,最好搞××万吨位。一九六二年我们可搞××——××万吨钢,要这么多钢干嘛?我们要和外国做生意,需要远洋船只,还可造军舰、飞机。我们东边有日本、冲绳、菲律宾,假使敌人在北京、上海扔了原子弹,我们也得报复,要考虑积极防御,也要考虑打垮敌人后的追击问题。还要考虑到抗美援朝问题。目前太平洋实际上是不太平的,将来为我们管了,才算是太平洋。(林总插话:×年后,我们一定要造大船,准备到日本、菲律宾、旧金山登陆。)造船还要几年才行?一九六二年我们有××——××万吨钢,有××万台工作母机,生产能力就大了。

原来认为军队有很大的捞头,现在只有×多万了,考虑还减不减?去年减××万,留××万是我提出来的,现在情况不同了。

我准备下次座谈会上、专门讲部队整编和干部等问题。


