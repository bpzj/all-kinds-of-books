\section[和××的谈话(一)(一九六四年六月八日)]{和××的谈话(一)}
\datesubtitle{(一九六四年六月八日)}


开了二十天的会,我们才见面,你们的简报我都在看。两种劳动制度、两种教育制度问××下面有没有,××说有,提了好几年了。主席说我在郑州就讲过把榨酒厂搬到农村,但没提得这么高。提到两种教育制度,主席讲,主要靠自学,肖楚女就没有上过学,他在茶馆跑堂,我很喜欢他,在农民讲习所教书,主要教员靠他。他能打破旧书的旧框,后来当黄埔军校的教官。农民讲习所,我们拿小册子给他们看。现在学校不发讲义,叫学生抄,为什么不发?据说怕犯错误。抄就不犯错误?发讲义叫学生看,你可以少讲。材料不仅发一方面,反正两面都发,如历史新学、旧学都要发给学生看。我写战略问题是写讲义的,就没有讲,论持久战也是这样写出来的。矛盾论写了好几个星期,白天黑夜复写,写出来只讲了两个小时。可以叫人家看么!现在的教育就是懒,自然科学不同一些,要看试验。要有工厂,过去搞概念,今天是概念,明天还是概念,学生啥也看不到。先生不写讲义(有人说发了讲义先生没什么讲的)那你可以多看书么!

问××:你不是同×××同志谈了吗?

刘××:×××是好同志,是左派核心,希尔也是好同志。

有些国家尚无左派核心,仅是小组活动。主席说先出来的可能是积极分子。

×××活动在美德轴心,没看到美德矛盾,欧洲有些国家恐德病,赫鲁晓夫想搞美国控制德国,主席说,德国统一对我们不利。法国也怕德国,欧洲一些党也怕德国。

法国党第二次世界大战有四十万军队,邱吉尔担心法国拿不下来。戴高乐有办法,第一是封官,第二是改为国防军三十万,另外十万军队解散,庞大民兵解散。多列士从苏联回去说服大家交出军权,当了付总理,最后一脚踢开。

欧洲党都如此,刘××讲他们像我们大革命时一样,工人和农民都组织起来,武装也有了,工农向我们要办法。我们就是没想到夺取政权,也没有想准备国民党可能解除我们的武装。

当时形势很好,就是没想到资产阶级会叛变。武汉政府有两个共产党员当部长。一个是谭平山,后来组织第三党,解放后搞农工民主党。一个是苏兆征,不经过那次资产阶级叛变是不行的,中国革命不会成功的,现在又是一次,我们对赫鲁晓夫开始没有准备他会叛变。现在世界上有两种共产党。一种是真的,一种是假的。十月革命,我们知道修正主义出在苏联有伟大意义。南斯拉夫出修正主义不行,苏联是搞了四十多年,列宁领导的,南斯拉夫是偶然的,苏联不是偶然的了。

我们已经出了,白银厂,小站,过去我们不注意上面的根子。

郑州会议,主席提出赫鲁晓夫是好人是坏人,照他讲的来是不照办。希尔讲,他早就知道赫鲁晓夫是坏人,我们知道,但不好公开讲。

搞一、二、三线,打起仗来准备打烂。

出毛选五卷的问题。

一分为二有辩证法;合二而一是有修正主义。

写党史,请董老挂帅,要把蒋介石全集印出,看他是怎样骂我们。

总理讲太多了,可以出选集。

传下去,传到县,如果出了赫鲁晓夫怎么办?中国出修正主义中央怎么办?要县委顶修正主义中央。

要有第三线,要搞西南后方,要搞快些,但不要毛草。钱就那么多,这就不要把摊铺得那么大,铁路两头铺就快些。


