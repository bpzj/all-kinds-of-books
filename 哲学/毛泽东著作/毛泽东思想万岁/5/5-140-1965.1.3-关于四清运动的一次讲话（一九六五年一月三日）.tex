\section[关于四清运动的一次讲话(一九六五年一月三日)]{关于四清运动的一次讲话}
\datesubtitle{(一九六五年一月三日)}


无事不登三宝殿,有事就开会。有的同志说,打歼灭战怎么打法?一个二十八万人的县集中一万八千人,搞了两个月没打开,学文件就学了四十天,学习那么多天干什么?我看是烦琐哲学。我不主张那种学习。光看是没有用。(刘××:河南有几万人集中在几个地方,搞了四十多天,是反右倾,搞清楚一些问题。)(刘子厚:我们集中也是反右倾放包袱。)有成绩没有?文件一天就读完了,第二天就议,议一个星期就下去。主要是在农村学,向贫下中农学习。我的一个警卫员,二十一岁,他写信来说:“学了四十天文件,根本没有学懂,这次下来才知道了些东西。”就是学一个礼拜文件,下去就进村,向贫下中农学习么,他说还有几怕,怕死人,怕扎错根子,怕这个怕那个,那么多怕就不行。二十八万人的县有一千八百个干部,还说少了,要那么多人,我看是人多了。工作队那么多人,你又依靠工作队,为什么不依靠县的二十八万人?依靠那里的好人?二十八个人中有一个坏的,还有二十七个是好的。有二个坏的,还有二十六个是好的。为什么不依靠这些人呢?一万人两万人搞一两个月还搞不起来。扎根串连,什么扎根串连?冷冷清清,就是没依靠好,如果依靠好了,一个县十几个人就够了。总而言之,我们从前革命不是那么革的。一两万人搞一两个小县,倾盆大雨,几个月还搞不起来。以前安源煤矿办工会,……安源工人我们一个都不认识。一去就公开讲:有那些人愿意进夜校。开始找到了工头,他有两个老婆,是否搞了俱乐部?(×××:没有搞。)搞了三个月,罢工就罢起来了么。

我看一进村和群众见面后,开门见山宣布几条就行了:

一、向社员宣布我们来不是整你们的。还有一部分老实的地富是否也可以宣布,不是整你们的,除了一部分漏划地富,一部分有严重问题的反革命和投机倒把分子以外,小偷小摸统统免了。开门见山讲是整我们党的内部,不是整社员。宣布我们来意不是整你们的。你们有什么不对的事你们自己去谈,我们不听你谈。社员中有严重的,极个别的也可以谈谈,这是极少数。

二、对干部也要宣布来意,小队大队公社的干部,无非是大、中、小、无。有多吃多占的,有多吃多占很少的,也有什么都没有的,几十元的,一百元、二百元以内的,你们自己讲出来。能退就退,不能退经过群众批准就拉倒,也只有这一点么!你讲出来没有事了,不讲出来就有些事了。

贪污盗窃、投机倒把退赔得好,可以不戴帽子,表现好的,群众同意的还可以继续当干部。

我看进村以后个把月要开个大会,以县为单位开个大会,一个小队来一个小队长,两个贫下中农;一个人队来支部书记和大队长,公社来书记社长。分几次开,一次开一天就行。先讲来意,话不要讲长,讲半个钟头就顶多了,讲一个钟头大家就不愿听了,让他们下去传达。二十八万人的县,三千多个小队。一个生产队三个人,一万人上下,一次开不起来分两次三次开,一次开一天。开个万人大会,就安民了。这样冷冷清清,搞那么多工作队,几个月搞不开,又没有经验,不会工作的人占大多数。通县去了两万多人搞一年多还没有搞开,有不会工作的,有做官的,我看这样革法,革命要革一百年。工作队里去了一些教授,不如助教,助教不如学生。书读得愈多愈蠢,啥也不懂,就是这个事,此外没有了。

这样打歼灭战我看歼灭不了敌人。要依靠群众,把群众发动起来,扎根串连冷冷清清。这个空气太不浓厚了。这个搞法和我们过去搞法不一样。要几个月歼灭敌人,我看方法要改,不依靠群众,几个月搞不起来,想个办法吧!

你们(指刘子厚)张承先、地委书记李悦农带队,几个月搞不起来,想个办法吧!为什么搞不起来!(刘:是强调扎根串连搞慢了,我在任县是大会小会结合搞的)干部会、贫下中农会可以在大队开,也可以去县里开。去年湖南省开了一个全省的贫下中农代表会,结果不错,今年增产粮食二十亿斤,我看起作用了。那么怕,怕扎错了根子,你钻到那里头去了。进村要开大会,贫下中农包括漏划的地生和富农统统都来,宣布几条,双十条不要一条一条的那么念。

真正的领导人、好人要在斗争中才能够看出来,光靠访贫问苦看不出来。访贫问苦,我就不相信。一不是亲戚,二不是朋友。粤汉路组织罢工,我在长沙。我们不认识一个人,还不是找了两个工头,一个叫朱绍廉的有两个老婆,他也要革命,因为工头受压迫,工资少,不够用,这个人后来还不是英勇牺牲了。那里是这样的扎根串连的方法?你去发展,去搞群众运动,去领导群众斗争,在斗争中群众要怎么办就怎么办么,然后在斗争中造出自己的领袖来(刘子厚讲了他自己去任县蹲点开斗争会的办法)这些斗争大会也应该讲去年分配,讲工分,注意生产。南方有,北方没有?有灾救灾,无灾清工分,搞当年分配,冬季生产。四清放在后头。四清是清干部,清少数人。有不清者,清之,无不清者不清。也有清的吧!身上没有虱子,一定要找出虱子来!(刘××:一个高潮一个高潮地来,不能拖延,拖延反而搞不彻底。)一个时期,文件学习四十天,搞烦琐哲学。我跟前那个警卫员来信说:学了四十天文件,仍然不懂,下去蹲了点才懂了。我历来反对这样读文件。学文件四十天是迷信。要开大会,搞斗争,搞地县的三级斗争会。

谢富治那个办法值得采取。抽出百分之二十的人来训练。六千人的工厂总有五千人不依靠,为什么不依靠五千人,只依靠你们工作队的五百人呢?我看有你一个人就行了。一个部长依靠五千人还搞不开?不要纠缠在文件上,训练那么久?搞斗争么。过去我们打仗,一拉起就打。有打胜的,有打败的。什么书也没有。有人说我带《三国演义》打仗的,谁照着书来打仗?林总过去打仗是内行,××过去也是内行。×××也是内行。内行也好,外行也好,要打才能学会。你不打仗光在那里学,怎么能行?(问谢富治)你们搞了一千多人的训练班是不是也是学双十条?怎么学?学多久?(答:也学了点,很快就搞斗争了。)为什么不可以大队为中心或以社为中心开训练班?所谓训练班就是斗争会,就是要在会上了解情况,了解各种人物,进行调查研究。要把斗争的人斗得差不多了,然后指定个把人作总结。总而言之,我的意思是依靠工农群众。河北省新城县张承先、李悦农当了多少年干部不懂群众运动。那么多人冷冷清清没有搞开。李××是保定地委书记,他首先提出四清,他下去要搞别的,群众要搞四清。他听群众的,……就是一九六二年提出搞四清的保定地委书记李××,原因是那时有个压迫在前面,打仗也好,搞农民运动也好,搞工人运动也好,工厂有资本家,农村有地主豪绅。国民党政治军事都在压迫着我们,我们无法可想,只好依靠群众。那时党员干部又少。几千人万把人的工厂,一个党员也没有。有一个就可以搞大革命,搞罢工。粤汉铁路一个党员也没有,可以搞大罢工。你后来建党么。现在进了城,当了官,就不会搞群众运动了。

为什么过去上军校的人打起仗来不翻书,黄埔军校五个月入伍期,四个月是正式军官学生。学生训练,操一操,练一练,就毕业了。林彪同志说:出来当连长根本不会打仗。班长有经验,听班长的话。他说怎样就怎样,打上几次就会打了。我不相信学了就会打仗,一个知识分子读了几年书就会打仗。不会打仗是合乎情理的。打几年就会了。我们工作队是否出了些主意?(刘××:贫下中农的主意多,我们也出一点,主要是群众的。)要听他们的。要听群众的,要听贫下中农的。就是发动群众要革贪污盗窃、投机倒把的命。要搞大的,小的刀下留人。(刘:一个是把群众发动起来,一个是群众发动起来到一定时候,工作队要掌握住火候,要善于观察形势。何时进攻,何时后退等等)等于罢工一样,什么时候罢工,什么时候复工。打仗也是一样,还是进攻,还是退却,实在不行你不退啊!有时候双方都退。比如打高兴圩,蒋、蔡向赣州退,我们向山沟里退,各退各的。

(问陈伯达)天津开了多大的会?(陈答:略)不得了,糟糕!多浪费时间,就不要开么!(陈插话:略)一千多户去那么多工作队,人多展不开。搞人海战术不行。一千多户你依靠七、八百户就搞起来了。有一个陈伯达就够了。嫌人少再带一个去,无非是宣布:我叫陈伯达,无事不登三宝殿,有事开个会。无罪的是多数人,有罪的是少数人,依靠多数人么!

(总理插话:刚才陈毅讲张茜学了两个月才进城)越学越蠢。反右倾,结果自己右倾。张茜在哪个县?(×××:句客县。)我历来反对学文件,一个文件几个钟头就看完了,你带下去学么。下去一不要学文件,二不要人多,三不要孤立地扎根串连。开会不要长,有话则长,无话则短,要让群众去搞,不相信群众,只相形工作队,不好。我身边的娃娃学四十天文件还不知道讲些什么,下去才知道。另一个在通县说教授不懂,助教好一点,学生更好一点。我对孩子讲,你读十几年书越读越蠢。什么也不读,你向大家说,我二十几年是靠吃蜜糖长大的,什么都不懂。请叔叔伯伯大娘指导。还是学生比助教好,助教比教授好。教授书读得太多了,不然怎么是教授。这些人下去是阻碍搞四清的。他们的目的是不要搞四清的。

一个是谢富治的经验,办训练班搞斗争。一个是河南经验“三结合”开斗争会。他们搞斗争也搞了一个月四十天。他们不是读文件,而是搞斗争,发动群众,了解情况。总而言之,搞斗争。(×××:我所在的那个大队,搞了两个月,搞出两万多元,十几万斤粮食)还是有油水搞,向他们借点钱用,他们的钱多得很,群众还是有希望的。照顾五保户也好,搞生产也好。会不能开得太小了,有的生产队十几户,十几二十几个人,讲话打不开情面。一个大队十几个生产队开大会顶多也是几百人么!

现在搞不起来,其原因就是不开大会,不宣布来意。

从来没有看到要这么多人,那么多人我就不相信能搞好。

总之,要依靠群众,不能依靠工作队。工作队不了解情况。或者没有知识,有的做了官,阻碍运动。工作队本身有的人就不能依靠。现在已摆成这么个阵线,一个通县,一个新城县。谁叫人少,我就砍一半,再叫人少,再砍一半。通县二万人砍一半分到别处去。一个县有五千人还不行吗?(康生插话:扎根串连是邓老发明的。……)啊,是邓老发明的!神秘化么!不宣布来意。要宣布我们做的事,生产、分配、工分,搞这几件事。四清讲一点,清不清,群众讨论。有就清,没有就算了。群众的不清,开个会要有几百人,小队十几户人少,以大队为单位开。说县里开了会了,讲出去可灵哩。县委书记问题严重的,当工作队到别的县去。照现在搞法,我看太烦琐了。你陈伯达那个是一千多户,开始几个人也搞开了。以后加下了五百人,要那么多人干什么?(邓××插话)讲长一点,反对急躁情绪。讲五、六年,三、四年,两、三年完成岂不更好。方法很重要,一个县集中一万多人是多了。一九二七年搞农民协会,一个县一个人,是农民自己搞起来的。没有好几个干部。农业部、财务部、武装部……七、八个哩,开始都是些勇敢分子,包围县城的也是那些勇敢分子,要求过分彻底,实际上不能那么彻底,当权派少数人是混进来的,是严重的,大多数是能够争取改造过来的,还有用处。过分了,群众不一定赞成。我们想开一点,可能快,不能那么太彻底。(刘插话)工厂中要可靠工人占优势。

工作队不一定太干净,一定要把有问题的人开除出去吗?不一定。工作队中可以有贪污投机倒把的人,只要他交待就行。我们这些人对贪污投机倒把没有知识,没有经验,他们有经验,没有他们还不行。集中力量打歼灭战,方向问题没有解决,如何打法?这样多人反而搞不开。不如陈伯达的办法。靠人海战术不行,要出问题。

王××提出干部交换问题,从这个县到那个县,这个社到那个社。来了新人不摸底,群众敢讲话。新社长、新书记来了,就敢讲以前的。很快就可以搞开的,为什么要很久?

现在是工作队不要那么纯洁。扎根串连,冷冷清清,没有群众运动。一万多两万多人在一个县还嫌少?(刘××、邓××……)

贫下中农协会就是要把问题搞清楚,跟陶×他们说。


