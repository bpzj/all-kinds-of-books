\section[为美国对华军事援助的法案的声明(一九四六年六月二十二日)]{为美国对华军事援助的法案的声明}
\datesubtitle{(一九四六年六月二十二日)}


美国国务院本月十四日提出国会审议的继续对华军事援助法案对中国的和平安定与独立民主极为不利的影响,因此,中国共产党坚决反对此项法案。中共此种意见,并为中国广大民主人士所支持。在抗日战争中,美国对了中国实施军事援助,并派遣美军在中国协同作战,其目的就是击败中美的共同敌人一一日本帝国主义。但就在那时,由于美国错误的仅仅援助国民党军阀,这种援助,也并未有效的加强中国的抵抗,相反地,是被国民党军阀用以加强其对于积极抗日的中国共产党与中国解放区的进攻与封锁。在日本投降以后,美国没有停止反而极大的加强了对于中国国民党政策的各种军事援助。并在此实际目的下,派遣庞大的军队,驻在中国的领土与领海之上,这种行动已经证明是中国大规模内战暴发与继续扩大的根本原因。仅仅在美国政府宣布履行一九四五年十二月间莫斯科三国外长会议公报关于中国问题的约束与中国国民党宣布停止内战,并宣布履行中国政治协商会议关于国家民主化的决议的前提之下,中国共产党才曾经不反对美国对于中国的某种军事援助。但是现在这些前提都已经被严重破坏,因此美国实行所谓军事援助,实际上只是武装干涉中国内政;只是以强力支持国民党独裁政府,继续陷中国于内战分裂、混乱、恐怖和贫困;只是使中国不能实现整军复原和履行其对于联合国的义务;只是危害中国国家安全独立与领土主权完整,只是破坏中美两个民族的光荣友谊与中美贸易的发展前途。中国人民今天此急需的,并不是美国的枪炮与美军的留驻中国领土。相反的,中国人民痛感美国运来中国的军人已经太多,美国在中国的军队已经驻得太多,他们已经造成中国的和平和安定与中国人民的生存和自由之严重巨大的威胁。在此种现实情况下,中国共产党不得不坚决反对美国政府继续出售、租借、赠送或过渡等方式,将军队交给中国的国民党独裁政府,坚决反对美国派遣军事顾问团来华,并坚决要求美国立即停止与收回一切对华的所谓军事援助,和立即撤回在华的美国军队。

