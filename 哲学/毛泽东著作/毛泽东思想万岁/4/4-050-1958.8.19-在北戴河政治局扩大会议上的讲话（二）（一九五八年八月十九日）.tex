\section[在北戴河政治局扩大会议上的讲话(二)(一九五八年八月十九日)]{在北戴河政治局扩大会议上的讲话(二)}
\datesubtitle{(一九五八年八月十九日)}


第一书记要亲自抓工业。

“统一计划、分级管理、重点建设、枝叶扶植。”天津专区办了一个四万吨的钢厂,这就是他们的重点。分级是在统一计划下,小部分中央管理,十分之二(投资、利润都可归中央);大部分归地方管理,十分之八。六二年搞到八九千万吨钢,那时怎样管,再看情况。重点放在哪里,要看哪里有这种条件,只搞分散不搞独裁不行。要图快,武钢可搞快些。但各县、社都发挥“钢铁积极性”,那不得了。必须有控制,不能专讲民主。马克思与秦始皇结合起来。

全党办工业,各级办工业,一定要在统一计划下,必须要有重点,有枝叶。不妨碍重点的大家搞,凡是妨碍重点的必须集中。各级只能办自己能办的事情,每一个合作社不一定都办钢铁。合作社主要搞粗食加工,土化肥,农具修理和制造,挖小煤窑。要有所不为而后才能有为。各协作区要有一套,但各省要适当分工,不要样样都搞。各省到底生产多少粮食、多少钢铁?以后各省都要自己生产,自己用掉。各省不要想跑到别处去调,还要准备中央调进一些.福建搞××吨钢,用到那里去呢?钢铁大的归中央,小型的各省都可搞一些。

地方分权,各级(省、专、县,乡、社)都要有权,内容有所不同,范围有所不同。分级管理,但不要把原材料都分掉了。

各级计划要逐步加强。合作社的生产与分配,也要逐步统一管起来。没有严密的计划性与组织性是不行的。粮食生产也要有计划,明年是否种这样多薯类?棉花要不要种那么多?明年再鼓一年劲,粮食搞每人××斤再看。

社会主义国家是一个严密的组织网,一万年后,人多,汽车多,上街也要排队,飞机多了,空中交通不管也不行。在猴子变人的时候,是很自由的,往后愈来愈不自由了。另一方面,人类大为解放,自觉地统治宇宙,发掘出无限的力量。

要破除资产阶级的法权思想。例如争地位,争级别,要加班费,脑力劳动者工资多,体力劳动者工资少等,都是资产阶级思想的残余。“各取所值”是法律规定的,也是资产阶级的东西。将来坐汽车要不要分等级?不一定要有专车,对老年人、体弱者,可以照顾一下,其余就不分等级了。

明年粮食生产还要不要鼓劲?还要鼓。苦战三年。储粮一年(每人××斤)。红薯可以减少一点。

所有计划统统要公开,不要瞒产,地县乡不控制不行。调东西调不出来要强迫命令。以后评比要比完成任务,比技术创造,比工作方法,比组织性纪律性,比更有秩序,比合理的独裁。要大鸣大放,才能独裁。现在铁也调不出,钢也调不出,几十万个政府,那还得了。

还要讲形势。国内形势要讲全国是一个大公社,不能没有重点,不能没有统一计划。从中央到合作社,要上下一致,要有许多机动。但机动是属于枝叶方面的,不能妨碍骨干。钢明年××吨要完成,今年××吨要保证。

省委书记回去以后,要立刻建立无产阶级专政,条子要灵,一个地区一个主,一个省只能有一个头,“冤有头,债有主’。邯郸有一个合作社,赶了一辆大车到鞍钢要铁,不给就不走.各地那么多人乱跑,要根本禁止。要逐级搞平衡,逐级上报,社向县,县向专,专向省,这叫社会主义秩序。中央也只有一个头。中央钢铁的头是×××。机械的头是赵尔陆。

中央计划由各省、市参加共同制定,省计划由地、县参加制定,一次也许讲不清楚,要多讲几次。

人民公社问题,名称怎么叫法?可以叫人民公社,也可以不叫,我的意见叫人民公社,这仍然是社会主义性质的,不过分强调共产主义。人民公社一曰大,二曰公。人多、地大、生产规模大,各种事业大;政社是合一的,搞公共食堂,自留地取消,鸡、鸭、屋前屋后的小树还是自己的,这些到将来也不存在了。粮食多了,可以搞供给制,还是按劳付酬,工资按各尽所能发给各人,不交给家长,青年、妇女都高兴。这对个性解放有很大好处。搞人民公社,我看又是农村走在前头,城市还未搞,工人的级别待遇比较复杂。不论城乡,应当是社会主义制度加共产主义思想。苏联片面强调物质刺激,搞重赏重罚。我们现在搞社会主义也有共产主义的萌芽。学校、工厂、街道都可以搞人民公社。不要几年功夫。就把大家组成大公社。

天津有一百万人能参加劳动而没有参加,第二个五年计划期间,才能基本实现机械化,劳动力才能彻底解放。

大权独揽,小权分散,中央决定(中央和地方共同决定),各方去办,办也有决,不离原则,工作检查,党委有责。这些还要强调。大权是主干,小权是枝叶,一是决策,一是检查。钢铁专门小组每十天检查一次才行。你们回去后,什么事情也不搞,专门搞几个月工业,不能丢就不能专,没有专就没有重点。粮食问题基本上解决了,高产卫星不要过分重视。帝国主义压迫我们,我们一定要在三年、五年、七年之内,把我国建成为一个大工业国,为了这个目的,必须把大工业搞起来,抓主要的东西,对次要的东西,力量不足就整掉一些,如种棉花整枝打杈保桃一样。这样会不会损伤下面的积极性?合作社不搞钢铁可以搞别的。钢铁谁搞谁不搞,要服从决定。要下紧急命令,把铁交出来,不许分散。大、中钢厂的计划必须完成,争取超过。在一定时期,只能搞几件事情,唱《逍遥津》就不能同时唱别的戏,要讲透“有所不为而后才有所为”的道理。

钢、铁、铝及其他有色金属,今明两年要拼命干。不拼命不行。钢要保证完成,铁少一点可以,也要争取完成。

派人到越南去,我讲过话:你们对越南的一草一木都要爱护。那不是胡志明的,是地球的,是劳动人民的事;如果牺牲了就埋在那里。来我们要搞地球管理委员会,搞地球统一计划。哪里缺粮,我们就送给他。但要对立的阶级消灭了。才有可能,现在两个阶级各有各的计划,将来做到各尽所能,各取所需,不分彼此,帮助困难的地方一个钱也不要。打了那么多年仗,死了那么多人,没有谁能赔偿损失,现在搞建设.也是一场恶战,拚几年命,以后还要拚,这总比打仗死人少。不能按钟头计算时间,那还算什么道德高尚?河北省计划十五岁的青年十五年后可以大学毕业,半工半读,人民的觉悟就提高了。靠物质奖励,重赏重罚过多是不行的。我们今后不要发什么勋章了,军官要下放当兵,没有当过兵的要当一下,当过兵的再当一下也很有好处,师长、军长下放让班长管,搞三个月后再同来当师长、军长。云南有一个师长,当了几个月兵,了解士兵的生活、心理,这很好。干部参加劳动,有人说搞两个月,搞一个月总是可以的,我们与劳动者在一起,是有好处的,我们的感情会起变化,会影响几千万干部子弟,曹操骂汉献帝“生于深官之中,长于妇人之手”是有道理的。只要大家拚命的干,再过三年、五年就搞起来了。

协作区不搞政治不行,要搞点政治。过去有人说协作区只搞经济不搞政治,我看还是要搞政治挂帅,思想一致了,才能搞好经济,在政治挂帅之下抓计划,搞大公社统一计划,重点建设,枝叶去掉一些.就是政治。


