\section[在邯郸地区的谈话(一九五八年十一月一日)]{在邯郸地区的谈话}
\datesubtitle{(一九五八年十一月一日)}


今年大丰收,老百姓高兴吧?要求休息两天吧!看来这是一个问题。我看一个月休息两天,放了假啥也不作,一年才二十四天不算多,过了十一月份后放假一个礼拜,好好休息一下。这算不算个问题,你们想一想。(××:钢铁这一关还没过去,还要再突击一个月。现在是准备几天,突击几天。)准备几天,突击几天,钢铁是这样,农业也这个样是否行?农业上的棉花啥时可以收完?(××:现在桃还没有开。还要一个多月。)这时桃还没开,它就不准备开了吧,剩下这些活叫娃娃们去搞好了。现在紧张,要轮班休息,十一月份叫他们休息两三天行不行?群众有什么不满意的吗?(××:一个是累,一个是吃不好,群众有些意见。)是啊!一个是吃冷饭,没有菜,一个是托儿所,一个是累,三件事。大人、小人要吃饭,吃了饭他们要干活。

你们这里收入多少?(×××:平均80元)有超过一百元的没有?(×××:有一百三十元的。)最低的是多少?(×××:四十五元。)四十五元吃了饭就没了吧,(×××。四十五吃了饭就没有工资。)你们有多少社开不了工资的?(×××:有六个社开不了工资)。是不是因为经济作物太少?(×××:都有经济作物。)你们没有不生产经济作物的社,都是什么?棉花、梨、花椒、养猪也是经济作物吧!鸡鸭也是吧?你们的粮食多了可以多养一些。吃不了可以出口。

你们是两个区合并的,九个县一个市,那和石家庄一样了。你们的粮食去年是22亿斤。今年70亿斤。翻一番是44亿斤。这么你们翻两倍多了。你们每人吃四百斤不够吧?你们这里吃饭还要不要钱?明年粮食计划生产多少?(××:计划170亿斤。)你们今年是一千万亩麦子,去年是多少,(××:七百万亩。)七百万亩,今年一千万亩都施底肥了吧?以后还要施追肥,每亩下种多少?(××:25斤以上的占95%。)亩产多少?你们这里是两白两黑,两白是棉花、小麦,两黑是钢铁、煤炭。你们种五百万亩棉花,青麻还有什么经济作物?青菜由哪里来种,你们这里食堂吃饭有菜吧?一个月吃两次肉行不行?你们要一个人一口猪,一口猪一百斤,平均每天六两肉,那就每天可以吃肉了。(××:公社化后猪减少了些。)是死了还是卖了?猪也要改善一下生活么。明年亩产千斤,亩产八百斤也就好了,今年小麦亩产多少?(××:202斤。)明年302斤也就满意了。(××介绍了临漳县搞的百里丰产川计划亩产五千斤。)临漳离这里有多远,路好走吧?出门坐火车不好看,这是老太爷的车,坐汽车也就好看了。

你们这里是不是有个死不跃进的县?(××:不是他这里,是石家扈县,白旗也拔掉了。)有白旗,当然也不会跃进了。

你们麦子一千万亩,大面积丰产250斤。什么人搞这些丰产亩?是不是青年突击队?也有老人吧?(××有青年也有穆桂英队,佘太君队。)有佘太君队,搞这个有味道。把这些丰产方法推广,一推广就照他这个办法,那么,你们种植面积就可缩小到四百万亩。一亩一万斤,就是四百亿亩,用这个方法缩小种植面积,好比这个桌子,两头的桌子都不种了,就种中间这一个,又省水,又省肥,又省人力。这个方针是河北省提出来的,过去是浅耕粗作,广种薄收,现在要精耕细作,少作多收。

你们小麦四百万亩,一亩一万斤就是四百亿斤,六百五十万人,四百亿斤就吃不完么?桌子可以砍掉一半,耕它三尺深,其余地叫它休息一年。我在安徽省看每地都烧肥,把那些主根子、杂草,一堆一堆地堆起来,叫作熏亩,地一休息,阳光一晒,一分化,一熏,我看是要走这个方向。在北戴河我提出种地三分之一,其他种草,种树,没水的挖塘养鱼。将来不是地少,而是地多,少种多收。深耕也就是耕三、四尺。细作无非是中耕、追肥、追水、治虫那套么,少种多收,也就是种一亩收一万斤。过去几千年都是浅耕粗作,广种薄收。

再加上一条机械化。你们这里有钢,有没有机械厂?(××有个小机械厂。)可不可以生产拖拉机?(作不了拖拉机。)你这个人就是志气不大!不用拖拉机行吗?用绳索将来用钢丝牵引。肥料不要洋化肥,只要土化肥行吗?我觉到有机比肥对作物有利,人畜拉的屎尿你们压绿肥用什么?(答:用紫穗槐)这可以代替洋化肥吗?明年可靠它。是否我们国家基本上不用拖拉机,少要一点,不是拖拉机化。洋化肥也是大部分不要,少搞一点,用在那些需要的作物上。这是我提出来交换意见。我看没有洋化肥,亩产一万斤。苏联有了它是一百八十五斤。我们是万斤可是没有洋化肥。拖拉机也是一样。土化肥就是洋化肥,第一,是人畜拉的屎和尿,第二是压绿肥,第三是土化肥,这些都是化肥。(××:全专区有二十三万个化肥厂。)好吧!在当地群众搞,比洋化肥好。

北戴河提出三分之一种庄稼,三分之一种草、种树。树木经济价值很大,木柴是化学原料,可以多种些。

拖拉机是否少搞一点,但是要机械化,用其它的形式,用中国的形式。少搞洋化肥也当作一个问题来考虑,机械化是否一定要经过拖拉机?肥料是否一定要经过洋化肥?人畜屎尿绿肥、土化肥、拆炕、折墙、挖河泥,多的很嘛!

你们钢铁任务多大?(××1105万吨,七万吨钢。)你们钢的原料从哪里来?什么叫低碳钢?合碳多少?(××不清楚,说元朝都在这里炼过钢。)你们怎么知道是元朝的?你们这里有多少矿石?(××:勘察清的有×××吨。)含铁60%,就×××吨铁。这两个月矿藏资源查清楚些了吧?(××到处说有矿,越来越弄不清到底有多少了。)由糊涂到清楚,用铁垒梯田、垒墙、垫路。你看我们国家有多富。土都是铁,不开矿挖土吗!

你们这里有铜矿没有?(×××勘探清楚的有××××万吨)有铜矿就搞些铜吧!

最后一个问题,就是把劳动组织,组织得更好些,又完成任务,又吃好,休息的好,这样可不可能?解放妇女,看拿些什么人去教育孩子,青年不愿意去,搞那些老年人去,是搞钢铁重要,还是小人重要?还是小麦、棉花重要?这是下一代的问题。托儿所一定要比家里好些,才能看到人民公社的优越性,如果和家里差不多,就显示不了优越性。这是一个大事,每个省、专、县都要注意后一代的问题,我们再干它十年,总要他们来接替吧!要把这点人口25%的娃娃带好。在托儿所要比在家跟父母好些。

再就是吃饭,一是吃饭,二是吃好。要不吃冷饭。吃热饭要有菜。菜里要有油和盐,要比在家庭、在小灶吃的好,这样农民才欢迎吃大锅饭。食堂要划分一下,一、二、三类,找一个一类食堂,叫他们都向它看齐。(××:涉县有一个食堂,吃玉米饭半月不重样,搞得很好。)这好么!把这个当成个大事,吃饭就是劳动力。吃早饭就是中午的劳动力,吃午饭就是下午的劳动力,吃晚饭就是明天的劳动力,要吃好。吃不好就没有劳动力。

再就是休息问题,下个命令。要休息,要睡够,要人吃,要人睡。现在不是军事化吗?下个命令睡觉,睡个中午觉,你们研究研究怎么样?冬天睡觉在地里太冷了,春、夏、秋都可以。不叫休息人民会不满意的。现在平均下来有几个小时的睡觉时间,下个命令至少睡六个钟头,睡完了劲更大。劳动增加了,干活效率会提高。

吃好、睡好、孩子带好。

对小孩要吃好、教好、管好。你们六百五十万人,25%就是一百六十万吧,这是一支很大的娃娃军。有的就是见物不见人,钢铁也是物么,种棉花也是物么,不管娃娃军就是见物不见人。托儿所要比家里有优越性。如果和家里差不多,一定会垮台,不垮台那才怪哩,食堂也是一样,如果不比家里吃的好,那也会垮台。

劳动力如果睡不好觉,私有都没有了,睡觉也是私有哩!在中午让他们睡一个钟头,你不是军事化吗?营长、连长开一次会,好连长应当是关心战士,给战士盖被子,不是老是只睡五个钟头,要睡七个小时行吧?不睡不行,睡觉是个任务,强迫命合一下,群众会欢迎的。这当作问题研究,不是作了决定。

经济作物不够的,没工资,发工资少,应当发展经济作物,每个社都应当种些有交换价值的经济作物,发工资不是一元、二元。应当多一些,如果到共产主义还是这么少有什么意思。资本主义就比我们多么。

你们这里公社平均一万户左右,你们这里是赵国,平原君就在邯郸。(××:这里还有回车巷。)廉颇蔺相如还要回车,我们的干部有的闹不团结,连车也不回,还不如他们呢!

你们有什么困难的问题?(××:大搞土法炼钢、秋收种麦、水库用劳动力,群众不能休息。)三大任务,有四百五十万劳动力,怎么组织好些?小土办法要一点,睡觉,一天一定要比七个钟头,这是个任务,你们先试一试。小孩问题,都不愿管人,师范学院的不愿教书,这是本末倒置,愿管物,不愿管人,实在也怪,我愿在小学当教员,去管娃娃。睡不好觉,你们要看到后果,几年后是会受到影响的。苦战,睡觉算苦战任务之一,吃饭要吃饱吃好。睡觉就睡七个钟头。试一试看完成任务怎么样?我看不一定比睡觉少完不成任务。省、市、地、县委第一书记要抓管人的事情。这都是人的事情.

组织军事化,有些地方强迫命令,有些地方营长可以打连长,打人、骂人、捆人,还辩论,争论成了一种处罚,这是对敌人的法令,不要敌我不分。我们红军、八路军、官长有不打士兵,不枪毙逃兵,不打俘虏,对老百姓和气。你们这些地方有没有打人、骂人、捆人的?争论和斗右派不一样,可见没有把敌我矛盾和人民内部矛盾弄清楚。对人民内部不要压服,对敌人除了那些反革命,一般的地主、富农、右派也不打他们,在人民内部更不能打人骂人了。已经打了,也不要到处泼冷水,以后不再打了,以后改正也就算乐,因为他打人也是为了完成国家任务,说清楚群众会谅解的。在人民内部是从团结出发,经过斗争达到新的基础上的团结,这是解决人民内部矛盾的方法。强迫命令,干是干下去了,人家心里不服,你看看吧,我们走了,也许不干了。


