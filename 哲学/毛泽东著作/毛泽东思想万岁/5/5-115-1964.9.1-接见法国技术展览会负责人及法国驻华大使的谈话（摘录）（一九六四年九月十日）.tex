\section[接见法国技术展览会负责人及法国驻华大使的谈话(摘录)(一九六四年九月十日)]{接见法国技术展览会负责人及法国驻华大使的谈话(摘录)}
\datesubtitle{(一九六四年九月十日)}


佩耶说,他最近去北京大学访问过。

主席说:这不是一所好大学。

佩耶接着说:我见到了校长、系主任、教授和学生。他们谈到了他们的活动,我认为,他们在研究和公民精神方面是热情洋溢的。

主席说:这是他们告诉你的,但是他们告诉你他们做的和他们实际做的不一定是一样的。这不是一所好大学。

杜阿梅说:我很荣幸在西安访问过一所技术学校,用了两个小肘,同学生谈了话,其中有个学生迫切要求入党,他说,入党是不容易的。……在这个学校我看到三个女学生,两个是医生的女儿,就是资产阶级的女儿,但她们思想都社会主义化了,毛主席的著作她们都能背诵出来,我提的问题她们可以用毛主席著作中的话来回答。

主席说:当然,他们对你说得好。但是不要听了就信。未来将会告诉你这些学生是好是坏,好不好,要看将来。判断他们的将是现实的生活,而不是书本上学到的东西。现在他们所学的只能当做资料,同过去上学一样,老师讲的话我们后来都不一定作。文科有语法和修辞这二门,写起文章来,就不照那个写。语法还有点用,修辞学就不一定有用,修是修,用的不多,谁写文章照修辞学写呢?

佩耶:换一点课程,更加活跃一点的。

主席:对,更加活跃的,更加实用的,合乎事实的。


