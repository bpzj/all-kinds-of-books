\section[在听了××××会议汇报后的讲话(一九六七年七月七日)]{在听了××××会议汇报后的讲话}
\datesubtitle{(一九六七年七月七日)}


毛主席在七月七日接见了××××会议全体人员,在听了汇报后说:“新武器、导弹、原子弹搞得很快,二年另八个月出氢弹,我们的发展速度超过了美国、英国、苏联、法国,现在在世界上是第四位。导弹、原子弹有很大成绩,这是赫鲁晓夫‘帮忙’的结果,撤走了专家,逼着我们走自己的路,要给他一吨重的勋章。”

毛主席听了汇报后,感到××太落后了,而且数量也很少。主席说。“现在形势很好。印度拉加族反对国大党,搞武装斗争。印尼共产党清算了修正主义,又起来了。缅甸游击队有很大发展,比泰国武装斗争还有基础,已搞了几十年,过去党不团结(红旗党、白旗党)。现在统一起来了。反对奈温是一致的,武装活动地区已占缅甸地区百分之六十。缅甸比南越的地理条件还好,回旋区大。泰国的地理条件也很好。缅甸起来,泰国起来,这样就把美国完全拖在东南亚。当然我们还必须着眼在我们国土上早打,大打。缅甸政府反对我们更好,希望他同我们断交,我们就可以更公开地支持缅甸共产党。亚洲形势如此,非洲、拉了美洲武装斗争也有很大发展。美帝国主义更加孤立,全世界人民都知道美帝国主义是战争祸首。全世界人民和美国人民都反对他。苏联修正主义也更加暴露,特别是这次中东事件,苏修还是采取赫晓夫那一套。本来他派往阿联军事专家两千多人,先是采用冒险主义,把军舰开了去,说服阿联不要先出击,同时用热线告诉了约翰逊(赫鲁晓夫那时还没有用热线)后。约翰逊很快就告诉了以色列,实行突然袭击,把阿联百分之六十的飞机都消灭在地面上。苏联援助阿联一共三十三个亿。打掉了廿亿,最后阿联投降停了战。这是出卖民族的又一次大暴露。对苏联,不仅是阿拉伯国家。亚洲、拉丁美洲都反对。苏联一个七十多岁老人自杀,就是一种不满。中东战争不久,苏联召集了三次东欧国家会议,罗马尼亚不签字,也未和以色列断交。后来毛雷尔跑到中国要和我们搞全面经济协作。单项经济协作可以,全面协作我们不干。帝、修更加孤立。越南战争坚持下去。目前许多地方反华,形式上好像我们孤立,实际上他们反华是害怕中国的影响,怕毛泽东思想的影响,怕文化大革命的影响。反华是为了镇压国内的人民,转移人民对他统治的不满。这个反华是美帝苏修共同策划吋,这不表示我们孤立,是我们在全世界影响大大提高。他越反华越促进人民的革命,这些国家的人民认识到中国的道路是求得解放的唯一道路。我们中国不仅是世界革命的政治中心,而且在军事上、技术上也要成为世界革命的中心,要给他们武器,现在可以公开给他们武器,就是刻了字的中国武器(除了一些特殊地区),就是要公开支持,要成为世界革命的兵工厂,但是我们许多技术还未解决”。

主席讲:“还要把××,×××,××××狠狠地抓一下。总理批评‘四机部落后’,四机部要大大赶上去。”

{\raggedleft (四机部军管会传达)\par}


