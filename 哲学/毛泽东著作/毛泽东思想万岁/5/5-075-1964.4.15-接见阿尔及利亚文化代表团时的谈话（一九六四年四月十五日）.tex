\section[接见阿尔及利亚文化代表团时的谈话(一九六四年四月十五日)]{接见阿尔及利亚文化代表团时的谈话}
\datesubtitle{(一九六四年四月十五日)}

\begin{duihua}
    
\item[\textbf{主席:}] 来了多久了?

\item[\textbf{马利克·本·纳比(阿尔及利亚文化代表团团长,以下简称纳比):}] 已经八天了。\marginpar{\footnotesize 103}

\item[\textbf{主席:}] 听说最近你们签了一个文化合作协定。

\item[\textbf{纳比:}] 签了两个协定。一个文化协定,一个电视广播协定。(注:该团在北京签订了中阿文化合作协定一九六四年度执行计划及两国广播和电视合作协定)

\item[\textbf{主席:}] 这次你们代表团有许多专家?这位(指一团员)是干什么工作的?(团长向主席一一介绍代表团成员)我们很欢迎你们。中国人民对你们的来访是很高兴的。你们的胜利是一个大好事。非洲有你们一个国家是打出来的。你们给非洲树立了一面旗帜。这不仅是对非洲、对亚洲、对拉丁美洲都有很大影响。它指出一件事:很弱的,很小的力量,比如军队,可以打败号称八十万人的帝国主义军队。那时看来你们还没有八万人,大概只有三万人。现在好多法国人走了,有一百多万居民也走了。是不是这样?

\item[\textbf{纳比:}] 是的。

\item[\textbf{主席:}] 学校里没有教师,医院里没有医生,没有工程师,两年以前,还是……

\item[\textbf{纳比:}] 一年半以前。

\item[\textbf{主席:}] 现在好些了嘛。现在有些教员了嘛,有些医生了嘛,有些工程师了嘛,工厂也可以开起来了,农庄也可以办起来了。看起来你们不是没有自己的知识分子,你们这些人不是(知识分子)吗?不是有教授吗?有专家吗?这还只是讲你们文化方面啰,没有讲经济方面,也没有讲医务方面。你们有些工作是不是军队的人在做?

\item[\textbf{纳比:}] 军队现在参加了大的工程和地方管理工作。

\item[\textbf{主席:}] 我们也一样。我们的文化可能没有你们的高。

\item[\textbf{纳比:}] 我们想正好相反。

\item[\textbf{主席:}] 我们国家过去文盲很多。一百个人里有八十个不识字。我们的军队多数是由受压迫的,没有多大文化的人组成的。他们打了十几年的仗,一面打仗,一面学习文化,解放以后他们才有机会进学校。但是基础文化他们根本就没有学过。他们就跟国民党留下来的知识分子合作。国民党走,知识分子不跟他们走。工程师也不跟他们走。他们说他们在大陆上带了二百万人走了,主要是军队。他们过去有几百万军队,打到后来,带了三十万军队走了。现在听说大约有六十万人。蒋介石的军官都是军事学校毕业的。我们的军官在军事学校里毕业的很少,只有个别的。大多数的军官是没有上过什么军事学校的。就是这支工人、农民的军队打败了知识分子的军队。国民党的知识分子就是没有知识,就像法国的军官没有知识一样。(笑声)那时候你们有一个总理,叫阿巴斯,和我谈过。他提出一个问题,他说,法国军队也用我的打仗的书来教育法国军官,想消灭你们。

\item[\textbf{本·科比:}] (副团长,笑着点头称是)他(指拉齐兹)还写过这方面的。

\item[\textbf{主席:}] 我跟阿巴斯说,法国军队是压迫人民的,是用不了我们的经验的。听说美国的军队也用我们的材料教越南南方的反动军队打南越人民。因为我们的经验是人民军队作战的经验,是人民军队积累起来的经验,归根到底,他们用不着。(对拉齐兹)你写过些什么书?

\item[\textbf{拉齐兹:}] 我写过一篇文章,讲法国军官想利用毛主席的著作来打我们的这个问题。但是他们是为封建主来打人民,而主席的著作是教我们为了人民去打封建主。

\item[\textbf{主席:}] 对的。结果法国没有打赢,你们胜利了嘛!\marginpar{\footnotesize 104}比如中国,不是打败了帝国主义和国内反动派吗?你们不是打败了帝国主义和国内反动派吗?古巴不是打败了帝国主义的走狗,胜利了吗?我们花了二十二年的时间打仗,你们花了七年时间,古巴只花了三年时间。情况不同,所以有这样的区别。你们根据你们的情况,敌人的力量,所以要花七年时间,古巴的情况只要三年。我们根据我们的情况花了二十二年。有些账不能挂在敌人身上,要挂在自己身上。我们的党犯过很多错误,犯过右倾机会主义的错误,犯过几次“左”倾机会主义的错误,一个是把南方根据地统通丢掉,那不能怪蒋介石,只怪我们自己。统通丢掉,跑了一万二千五百公里的路。跑到北方,把南方根据地丢掉。军队由三十万人减少到两万人。这时我说,这个时候我们的事情好办了,舒服了。这样就可以总结经验了。这两万军队后来和日本帝国主义打了八年,又发展到一百几十万,根据地人口发展到一万万,党员由几万发展到几百万。你们到中国来研究,要研究这一段历史,要研究我们失败,犯错误这方面。你们还没有研究过吧?

\item[\textbf{纳比:}] 主席先生,我们到中国来,将从中国领导人的意见中吸取教益。他们领导中国人民进行了四十年的斗争,直到最后的胜利。但最能打动我们的是中国的成就,而不是失败。这些成就的取得,是由于在主席的领导下,建立了坚强的政治机构。我们这几天看到了政治机构各个不同的发展阶段。我们先到了广州,现在又到了长沙,当然,我们的旅程不是完全按照历史顺序的。

既然主席先生提到了我们同胞阿巴斯,我想用我和我的同志的名义说明一下。阿巴斯是属于你们称之为买办阶级的那个阶级。以本·贝拉为首的领导集团正在清除其错误。我们正在按照主席刚才讲的革命传统这样做。

\item[\textbf{主席:}] 他现在不干了,不跟你们合作了。

\item[\textbf{纳比:}] 我们不仅希望改进我们的政治机构,而且特别要改进我们的思想方法。

\item[\textbf{主席:}] 这个好。

\item[\textbf{纳比:}] 我想我不必花太多的时间来讲法国殖民主义离开阿尔及利亚的局势。因为上次和周恩来总理讲话时已经谈过了。周总理还告诉我们毛主席讲话的习惯是短的。我们把毛主席说话的这种规则,看成我们自己的规则。另一个原因是,毛泽东主席充分了解当时的局势。

\item[\textbf{主席:}] 他们都走了更好。你们那里更干净。在一张白纸上更好写字画画。你们的负担更少。你们重新干起,白手起家。帝国主义做过的事,资本主义做过的事,我们人民也能做到。我就不相信就只有欧洲或北美、日本的资本家才能做到。第一条,我们人多,全世界被压迫的人总是占多数。第二,在这些人里头,总有比较好的领导者和干部,他们不会的可以学会。困难是可以克服的。你们看,你们克服了法国八十万军队的困难,难道现在在经济方面、文化方面、政治方面遇到的困难不能克服吗?我提到政治方面,是因为还有反对你们的,也有要推翻你们的人,有没有?

\item[\textbf{纳比:}] 在我国的周围有资本主义的反动势力,国内有新老殖民主义的仆从,他们反对我们新型的民主和自由。

\item[\textbf{主席:}] 国内没有吗?

\item[\textbf{纳比:}] 在殖民主义统治时期,他们扶植一些资产阶级,用来压迫人民。这些资产阶级就成了当地的官僚阶层。他们想要实行对人民的控制。这批资产阶级是殖民主义一手制造的。\marginpar{\footnotesize 105}其中一部分,在经济方面曾占据过重要的位置。这一部分人就是右派分子,是来自右的方面的反动;还有一种是来自“左”的方面的反动。这是些自称有进步思想——共产主义思想的人;实际上是托派。阿共就是这一派。正如您所知道的,阿共是受法共领导的。

\item[\textbf{主席:}] 你们的共产党老是埋怨我们不跟他们说话,说他们没有政治地位。我们说:“你们名为共产党。革命嘛,你们不参加,你们反对,等到后来革命有点希望了,你们又说要参加,那时已经晚了。”所以你们的共产党和我们谈不来。(笑声)我们和你们谈得来。这有点和古巴相似。

\item[\textbf{纳比:}] 感谢主席对我们的健康力量表示的信任。我的同志们和我,会把这一点转告我国的革命者。

\item[\textbf{主席:}] 你们把阿巴斯,还有贝勒卡塞姆清洗掉是有道理的。那些人不行了。所以要靠你们,靠你们的党。听说,民族解放阵线最近要开大会?

\item[\textbf{纳比:}] 本月十六日开。

\item[\textbf{主席:}] 那你们什么时候回去?

\item[\textbf{纳比:}] 二十一日。

\item[\textbf{主席:}] 今天十五号,还有四、五天。你们怎么安排?上海还去不去?

\item[\textbf{纳比:}] 我们等丁××先生解决这个问题。

\item[\textbf{主席:}] (对丁××)时间不多嘛,要看他们的方便,少走几个地方也可以,不要搞得太疲劳。(转向团长)你们是从广州来的,还是从北方来的?

\item[\textbf{团长:}] 我们是从香港到广州之后去北京的。

\item[\textbf{主席:}] 少看一点也可以。纳比:我们对知识的渴求是很高的。在中国作一次旅行要花很多时间,我们当中许多人不一定有机会再来,所以要尽可能多带一点收获回去。

\item[\textbf{主席:}] 看中国要看两方面,就是成功的方面和缺点,错误的方面。不了解这方面也就不了解那方面。正如你们也有这两方面一样。你们是经过曲折道路走过来的。为什么阿巴斯、贝勒卡塞姆跟不上了呢?

\item[\textbf{纳比:}] 这是因为——这也是中国之行给我们最大的教育——我国的革命运动所处的条件不一样,虽然有政治上的准备,但缺乏思想基础。我们在中国长沙了解到。三十多年来,毛泽东主席最关心的是培养干部,是建立能够产生革命思想的中心。

\item[\textbf{主席:}] 你们现在进行对人的教育,思想教育,看来是很有必要的。我们也是这样。资产阶级民主革命是有比较充分的准备的,是有几十年经验的。在人民中间,在党内,在干部中间,什么叫帝国主义、封建主义,什么叫买办资产阶级,是比较清楚的。怎样对付它们的政策,也是比较清楚的。但是也经过曲折,犯过错误。至于怎样搞社会主义,就不清楚了。所以现在要重新对老干部、新干部进行社会主义教育。胜利了十四年,我们又重新抓起了社会主义教育。我们自己不懂,怎么教育别人呢,现在情况好了一些。

\item[\textbf{纳比:}] 至于我国的革命形势,我还想补充一点:我国的革命思想是自发形成的,是在青年中自发形成的。列宁讲过“共产主义左派幼稚病”,这对我国的革命思想是可以适用的,是和我国革命思想吻合的。就是说,我们经历过幼稚病的阶段。

\item[\textbf{主席:}] 我们也犯过幼稚病,犯过三次之多。最后一次把我们南方根据地丢掉了,还跑了一万多公里,就是长征。那时我们叫“教条主义”。就是不根据中国实际,专门抄书本,抄教条,抄外国的经验。我们也有过几次右倾机会主义的错误,拿现在的话就叫修正主义。那是从资产阶级来的思想。你们将来再把我们的这段历史研究一下,也不要很多时间,个把星期就够了。(对周××)你们也可以跟他们谈话。你们参加没参加反陈独秀、王明的斗争?\marginpar{\footnotesize 106}

\item[\textbf{周××:}] (以下简称周):反陈独秀时我们正是个娃娃,是个小鬼,反王明的后期倒是参加了。

\item[\textbf{主席:}] 第一任总书记是陈独秀,这个人后来是托派。后来好几届书记也都不行。所以跟你们也差不多。我们这个党也不是顺畅走过来的,是经过艰难困苦过来的。现在在我们党内也不是什么都是好的。有许多党员挂党员的招牌,实际上是新的资产阶级分子。有许多干部也是这样。我劝你们要了解这方面,可能对你们会有用处。

\item[\textbf{纳比:}] 我和我的同志们相信,主席先生讲的意见是真理。因为您领导了二十世纪最伟大的革命。正是您讲的这些黑暗势力,足以摧毁革命事业。在我们这里,(还不止是阿尔及利亚,而是在整个阿拉伯集团)没有建立起监督的标准,来识别表面上革命,实际上可能把革命毁掉的分子。我讲这个话,并不是以阿尔及利亚文化代表团团长的身分,而是以作家的身分讲的。

\item[\textbf{主席:}] 你是位作家,有什么著作?

\item[\textbf{周:}] 他有十五本着作。

\item[\textbf{主席:}] 那方面的?

\item[\textbf{周:}] 政治方面的多一些,其中有一本叫《亚非主义》。

\item[\textbf{主席:}] 噢。

\item[\textbf{纳比:}] 谈到《亚非主义》这本书,我正想说明一下。因为不知道主席今晚就接见,所以把书放在宾馆,没能带来。对我个人来说,来中国的目的几乎就是为了要把这本书送给主席。为了弥补这个不凑巧的情况,现在我是否可以回去拿?

\item[\textbf{主席:}] 可以吧。(对周)他们明天在这里吗?

\item[\textbf{周:}] (对团长)可由我转交。明天去韶山,主席的旧居。

\item[\textbf{纳比:}] 对于我和我的同志们来说,能到主席旧居,这将是访华之行的光辉顶点。

\item[\textbf{主席:}] 那个地方,解放后我只去过一次。离开三十几年了。过去不是我不愿去,而是国民党不让我去。(笑声)

\item[\textbf{纳比:}] 我们在中国很幸运,能够看到中国革命各个阶段。首先在广州看到了农民运动讲习所。后来在北京我们参观了军事博物馆。在那里,看了长征的经过。听说你们把长征称作英雄——在这里,英雄是指事,而不是指人了。在我们离开博物馆时,发现门口处写了毛泽东主席的一句话:决定革命形势的是人,而不是武器。同时,还看到林彪讲的一句类似的话。

\item[\textbf{主席:}] 就是人干出来的嘛!开始我们一件武器都没有,现在有了政权,可以自己开工厂了。制造武器的是人,使用武器的也是人,夺取武器还是人。你们现在尚未达到制造武器的阶段,但是有修理武器的工厂。这些工厂以后可以变成制造厂。可以从轻武器入手,一直发展到制造重炮、坦克。你们一定会发展到那个阶段的。我就不相信,只有法国人能制造大炮,阿尔及利亚人民不能制造。\marginpar{\footnotesize 107}法国人积累了二百年经验,你们只几年。你们会比法国人快。法国有四千万人口,你们有一千多万。但是,为什么你们一千多万人口的国家打败了四千多万人口的法国呢?所以你们是很有希望的。

我跟你们的谈话和对法国议员代表团讲的就不同。我就不对他们讲什么我们过去犯的错误。现在还犯的错误就更不讲了。(笑声、掌声)因为我们希望你们如实了解中国情况,不要只了解片面的。法国资产阶级代表,他们不愿意听这一套。讲革命经验对他们无益。他们是反对革命的,和他们讲干什么呢?你们是革命党,我见了革命党,就介绍两方面的经验。我只对你们讲。对别国的革命党也讲,但不包括你们的共产党。他们不会来,来了我们也不讲这一套。

\item[\textbf{纳比:}] 因为阿共是多列士的门徒,而中国共产党从一开始就是马克思、恩格斯、列宁的学生。这个区别很大。

\item[\textbf{主席:}] 跟法国共产党我们也讲不来。他们也没有代表团来。现在共产党不一致,有点小矛盾,不大也不小。这是很自然的。比如(我们)和你们的党(指阿共)怎么能谈得来呢?再如法共,他们天天反对我们,说我们是教条主义,又说我们是托洛茨基主义。你们现在和我这个托洛茨基主义谈话,(笑声)跟我这个教条主义谈话。(笑声)他们只是不说我们是修正主义。

怎么样,你们还有什么意见要谈吗?

\item[\textbf{纳比:}] 刚才我讲到,我们关切中国革命各个阶段的发展,现在它已发展到了人民公社的阶段。这是一个十分重要的措施。我们也知道,这引起了某些共产党的批评。但我们相信,这是中国走向社会主义、共产主义最好的“王牌”。

\item[\textbf{主席:}] 可能是,但还要看。现在还在试验过程中,最后的结论是以后的事情。有些共产党反对我们的措施。说我们不行了。这跟帝国主义一样。帝国主义也反对我们人民公社这一套。帝国主义反对,法共他们也反对。可能有些好处也说不一定。要不然,假如一点好处也没有,那他们为什么反对呢?只好欢迎了。这就是说,我们不走资本主义道路,要走社会主义道路。你们的共产党是不准备搞真正的社会主义的。

我们也不忙做结论。究竟是人民公社崩溃,还是发展?要再看。前一阵子,帝国主义说中国政府要崩溃,现在又不大讲了。看样子中国还没有崩溃。我这个人倒是会崩溃的,快要见马克思了。我们的医生就是不能保证我还能活几年。这是客观规律,人总是要灭亡的。是辩证法。事物总是有始有终的,但是一个人灭亡,一群人灭亡,并不等于一个国家灭亡。现在马克思不在了嘛,恩格斯不在了嘛,后来又有了列宁、斯大林。现在这二位也不在了嘛。世界就灭亡了吗?不但没有灭亡,还更加发展了。中国革命胜利了,你们的革命胜利了。古巴的革命胜利了。这就是规律。(对团长)你多大年纪了?

\item[\textbf{纳比:}] 五十九岁。

\item[\textbf{主席:}] 还年青嘛。(对副团长)你多大了?

\item[\textbf{本·科比:}] 三十一岁。

\item[\textbf{主席:}] 我们总还能活几年。你们活的更长些。

\item[\textbf{本·科比:}] 我们衷心祝福毛主席万寿无疆,尽可能的长寿。\marginpar{\footnotesize 108}

\item[\textbf{主席:}] 说“尽可能的长寿”,这话好。只可以尽可能活得长一点,不可能不死。中国历史上还没那回事。

你们回去后,请代我问候你们总统,说我祝福他身体健康,工作顺利,克服困难,向前发展。

\item[\textbf{纳比:}] 主席先生,非常感谢您接见了我们。您牺牲了您的时间,本来您还有极其重要的工作。我想向您表达阿尔及利亚人民的敬意。是您领导了中国伟大的革命事业。我想表示的是,阿尔及利亚人民对中国有发自内心的友好感情,他们有共同的敌人,就是殖民主义。他们的斗争所处的条件,也几乎是一样的。我们知道你们是夺取了敌人的武器战胜了敌人的。阿尔及利亚人民也是从法国军队手里夺取了武器打垮法国军队的。我们要向中国人民表示衷心的敬意。

\item[\textbf{主席:}] 要向你们国家的人民表示敬意,向你们国家的领导人表示敬意。你们的本·贝拉总统的全名怎么写?(翻译用纸条写好,主席阅毕收存。)

\item[\textbf{纳比:}] 谢谢您,再一次向您表示衷心的敬意。感谢您的热情接待。

\item[\textbf{主席:}] 谢谢。再见。

(主席同全体外宾合影留念)
\end{duihua}

