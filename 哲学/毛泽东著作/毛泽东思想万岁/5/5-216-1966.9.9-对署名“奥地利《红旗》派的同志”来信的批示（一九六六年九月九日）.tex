\section[对署名“奥地利《红旗》派的同志”来信的批示(一九六六年九月九日)]{对署名“奥地利《红旗》派的同志”来信的批示}
\datesubtitle{(一九六六年九月九日)}


退陈毅同志:

这个批评文件写得好,值得一切驻外机关注意,来一个革命化,否则很危险。可以先从维也纳做起。请酌定。
毛泽东
\kaoyouerziju{ 一九六六年九月九日}

请主席审查署名奥地利《红旗》派的同志来信。

\kaoyouerziju{  陈毅\\一九六六年九月九日}

(附署名“奥地利《红旗》派的同志”来信)
亲爱的同志们:

读到关于红卫兵支持你们伟大的无产阶级文化大革命的英雄行为的报导等,我们非常赞赏。以你们的伟大领袖毛泽东的智慧为基础的这一历史革命,对于我们这些致力消灭资产阶级生活方式和资产阶级社会的人来说是一个鼓舞。但是我们认为有些更必要提醒你们注意,你们国内的革命斗争同你们在维也纳的商务代表的突出的资产阶级举止和资本主义生活方式是极不相称的。从他们衣着看来,很难(即使不说是不可能的)把他们同蒋介石走狗区别开来。精制的白绸衬衫和高价西服同先进工人阶级代表的身份是很不相称的。这些代表不仅占有一辆,而且是两辆“列尔来得——奔驰”牌汽车(这种汽车可以说是资本主义剥削者的标志)难道具有必要吗?由于这一明显对比而引起了维也纳人的窃窃私语和嘲讽,使我们听了很痛苦。这样的资产阶级行为不仅损害我们的共产主义事业,而且对于伟大的无产阶级文化大革命也起了不好的作用。我们尊敬地并且迫切地要求你们把这种事到有关当局报告,并且立即采取措施,加以纠正。

致以同志的敬礼

\kaoyouerziju{  奥地利《红旗》派的同志(2—0005)\\ 一九六六年八月三十日维也纳}


