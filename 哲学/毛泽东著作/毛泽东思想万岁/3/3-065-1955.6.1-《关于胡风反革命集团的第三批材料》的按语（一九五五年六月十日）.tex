\section[《关于胡风反革命集团的第三批材料》的按语(一九五五年六月十日)]{《关于胡风反革命集团的第三批材料》的按语}
\datesubtitle{(一九五五年六月十日)}


编者按:胡风反革命集团第一、第二两批材料的公布,激起了全国广大人民群众对反革命分子的极大的愤怒。人们要求追究胡风集团的政治背景。他们问:胡风的主子究竟是谁?关于这个问题,人民政府已经获得大批材料。其中的一部分,我们把它放在这个“第三批材料”中发表出来。胡风和胡风集团中的许多骨干分子很早以来就是帝国主义和蒋介石国民党的忠实走狗,他们和帝国主义国民党特务机关有密切联系,长期地伪装革命,潜藏在进步人民内部,干着反革命勾当。

在下面的材料中,人民可以看出那被胡风捧为“追求革命十多年的革命作家”的阿垅和自称追随革命二十年的胡风本人的真实面目。阿垅在一封给胡风的信里,对蒋介石在一九四六年七月开始的在全国范围发动的反革命内战“充满了乐观”;认为中国人民解放军的“主力”“三个月可以击破”,“一年肃清”;并对蒋贼的“训话”,加以无耻的吹嘘,说什么“他底自信”“使大家更为鼓舞”。阿垅把人民革命力量看做是“脓”,认为“总要排出”,并认为进攻人民革命力量必须坚决彻底,“一不做,二不休”!

他们为什么这样坚决呢?原来胡风、阿垅等人都是别有来历的人物。

阿珑,即陈亦门、陈守梅,浙江人。原是国民党的军官,抗日初期混入延安抗日军政大学读了几个月,不上前线,却到胡宗南的“战干四团”做少校军事教官去了。这里发表的一封信是他在一九四六年七月从重庆山洞陆军大学写的,他已从胡宗南那里转到这里受训,毕业后任战术教官。“山洞”这个地方,是蒋介石的陆军大学所在,也是蒋介石本人在重庆期间的住地。

胡风,即张光人、又名谷非,湖北人。在第一次国内革命战争时期,他曾经加入过共产主义共青团。一九二五年他在北京,当时段祺瑞统治下的白色恐怖把他吓昏了头脑,坚决要求党允许他退出了团。后在江西“剿共”军中做过反共的政治工作,又去日本混了一个时候,干了一些不可告人的勾当。回国以后,他在上海混进了左翼文化团体,从内部进行了种种分裂破坏活动。在武汉和重庆时期,他和国民党的许多特务头子有联系。从这里发表的胡风给阿垅的一封信里,也可看出胡风和国民党特务头子之一的陈焯的关系。胡风的这篇简单的历史是最近才查明的。因为他隐瞒得很巧妙,大家被他骗过了。

胡风集团在全国解放以前狂热地把希望寄托在蒋介石反人民内战的胜利和人民革命力量的失败上;而当蒋匪溃败,全国解放以后,他们就潜伏在大陆上以更加阴险的两面派手法继续进行反革命活动。

他们对解放后的新社会、对人民革命政权,表现了刻骨的仇恨,他们说“对这个社会秩序,我憎恨”,他们诅咒人民革命政权的“灭亡”“完蛋”!

当本报公布了第一、二批揭露材料之后,还有一些人在说:胡风集团不过是文化界少数野心分子的一个小集团,他们不一定有什么反动政治背景。说这样话的人们,或者是因为在阶级本能上衷心地同情他们;或者是因为政治上嗅觉不灵,把事情想得太天真了;还有一部分则是暗藏的反动分子,或者就是胡风集团里面的人,例如北京的吕荧。

现在,已到了彻底弄清胡风这一批反革命黑帮的面目的时候了,中国人民再也不容许他们继续玩弄欺骗手段!全国人民必须提高警惕!一切暗藏的反革命分子必须揭露!他们的反革命罪行必须受到应有的惩处!

按:从这一类信里可以看出,胡风集团不是一个简单的“文艺”集团,而是一个以“文艺”为幌子的反革命政治集团。他们仇恨一切人民革命力量。胡风分子张中晓说,他“几乎恨一切人”。许多人认为“胡风不过是一个文化人,胡风事件不过是文化界的事件,和其他各界没有关系”,看了这类材料,应当觉悟过来了罢!

按:从以上两封信里可以看到胡风骨干分子绿原的真面目,胡风集团的骨干就是由这样一批人组成的。绿原在一九四四年五月“被调至”“中美合作所”去“工作”。“中美合作所”就是“中美特种技术合作所”的简称,这是美帝国主义和蒋介石国民党合办的由美国人替美国自己也替蒋介石训练和派遣特务并直接进行恐怖活动的阴森黑暗的特务机关,以残酷拷打和屠杀共产党员和进步分子而著名。谁能够把绿原“调至”这个特务机关去呢?特务机关能够“调”谁去“工作”呢?这是不言而喻的了。在后一封信里,在一九四七年九月,绿原还在骂中国共产党和人民革命的力量是“万恶的共匪”,可是,一九四八年初他就由另一胡风骨干分子曾卓介绍为共产党党员,打入了地下党的组织。后来绿原突然潜逃。武汉解放时又突然回到武汉,与曾卓一起自称是“共产党”,接收《大刚报》。一九五零年再度钻进党来(参看本材科第二十八条)。胡风反革命集团的分子就是这样来“追随革命”和钻进共产党里面来的。

按:由阿垅这些信里可以看出,胡风分子是很懂得一些反革命的地下工作的策略的。他说:不要“在阵地末强固前就放起枪来”。而“主要是准备条件,多一些条件,再多一些条件!”要“埋头工作,在群众中做好工作”,把“群众基础弄好,”然后“就找大的对象”,即对准革命的要害加以攻击。在进攻时,要多同“朋友”“商量了做”,把“论点组织和考虑得更严密些,小东西和小事情最好不理。”反革命分子不是那样笨拙的,他们的这些策略,是很狡猾很毒辣的。一切革命党人决不能轻视他们,决不能麻痹大意,必须大大提高人民的政治警惕性,才能对付和肃清他们。

按:卢甸这种以攻为守的策略,后来胡风果然实行了,这就是胡风到北京来请求派工作,请求讨论他的问题,三十万字的上书言事,最后是抓住《文艺报》问题放大炮。各种剥削阶级的代表人物,当着他们处在不利情况的时候,为了保护他们现在的生存,以利将来的发展,他们往往采取以攻为守的策略。或者无中生有,当面造谣;或者抓住若干表面现象,攻击事情的本质;或者吹捧一部分人,攻击一部分人;或者借题发挥,“冲破一些缺口”,使我们处于困难地位。总之,他们老是在研究对付我们的策略,“窥测方向”,以求一逞。有时他们会“装死躺下”,等待时机,“反攻过去”。他们有长期的阶级斗争经验,他们会做各种形式的斗争——合法的斗争和非法的斗争。我们革命党人必须懂得他们这一套,必须研究他们的策略,以便战胜他们。切不可书生气十足,把复杂的阶级斗争看得太简单了。

按:由于我们革命党人骄傲自满,麻痹大意,或者顾了业务,忘记政治,以致许多反革命分子“深入到”我们的“肝脏里面”来了。这决不只是胡风分子,还有更多的其他特务分子或坏分子钻进来了。

按:胡风集团分子和其他许多暗藏的反革命分子,大都采取方然在这封信里所讲的两面派策略,特别是他的(二)、(三)两条策略,很可以欺骗许多人。但他们总有漏洞可以给人们找到,胡风集团的被揭露,就是一个证据。特别是在大多数人的觉悟程度和警惕性提高了以后.他们的两面派策略就更易被揭露了。

按:从这里也可以看到我们批判胡适派资产阶级唯心论这一斗争的重要性和必要性。有些人口称相信马克思列宁主义,却不重视批判唯心论这一斗争,或者说自己没有唯心论,或者说自己和胡适无关系,因而最好避开不谈。但胡风集团却是重视的,他们在研究如何对付的方法。“这里存在着矛盾和困难”。批判唯心论果然给了胡风集团以“矛盾和困难”,这就可见批判的对了。革命队伍里的人难道也有“矛盾和困难”吗?

按:一大批胡风分子打入中国共产党内取得党员称号这一件事,应当引起一切党组织注意。绿原解放前曾经一度钻进我们的地下党组织,后因潜逃失去党籍。在一九五零年,这个反革命分子又对我们的党组织“用最大诚恳写过三次报告,一次比一次详尽而老实”,除了文艺思想而外,“其余大体合格”,果然后来又被接受为“党员”了。这样的事,难道还不应当引起一切党组织的注意吗?这些反革命分子是在用尽心思欺骗了我们之后爬进党内来的,他们把这当作“一场斗争”看待,他们斗胜了我们,他们进来了!

按:共产党员的自由主义倾向受到了批判,胡风分子就叫做“受了打击”。如果这人“斗志较差”,即并不坚持自由主义立场,而愿意接受党的批判转到正确立场上来的话,对于胡风集团来说,那就无望了,他们就拉不走这个人。如果这人坚持自由主义立场的“斗志”不是“较差”而是“较好”的话,那末,这人就有被拉走的危险。胡风分子是要来“试”一下的,他们已经称这人为“同志”了。这种情况,难道还不应当引为教训吗?一切犯有思想上和政治上错误的共产党员,在他们受到批评的时候,应当采取什么态度呢?这里有两条可供选择的道路:一条是改正错误,做一个好的党员;一条是堕落下去,甚至跌入反革命坑内。这后一条路是确实存在的,反革命分子可能在那里招手呢!

按:从这一类的信里,应当引起我们的警觉,不要让他们“滑过去了”。

按:从这类信里可以看出,暗藏在革命队伍里的反革命分子很怕整风,可见整风是有益的。怕整风的人不都是反革命分子,绝对大多数(百分之九十几)是思想上或政治上犯有某些错误的人,对他们的方针是帮助他们改正错误。但反革命分子很怕整风,对他们的方针则是进一步挖出他们的反革命根子。胡风反革命集团的面目,是在解放前和解放后的几次整风,即过去的几次思想斗争中逐步地暴露出来的。由于那几次整风,才产生了胡风集团内部的分化,才迫使胡风集团采取以攻为守的策略——三十万字的上书言事,才有最后的大揭露。

按:从这类信里可以看出,在强大的人民革命力量即人民民主专政面前,只要这个专政是提高了群众的觉悟,采取了正确的政策,那就不管有多少暗藏的反革命集团,也不管每个反革命集团的内部纪律如何森严,攻守同盟如何坚固,总有一些人可出分化出来的,而这种分化是于人民有利的。舒芜从胡风集团分化出来一事使得胡风集团大伤脑筋,就是一证。近日各地许多胡风分子们纷纷坦白,自动地或被迫地交出密件,揭露内情,是这个斗争的继续发展。

按:从这类信里可以看出,我们的机关、部队、企业或团体里是有人偷窃机密的。这种人就是混入这些机关、部队、企业或团体内的反革命分子,有些自由主义分子则是这些反革命分子的好朋友。这种情况,难道还不应该引起全体工作人员和全体人民的严重注意吗?

按:从这封信里可以看出,胡风集团坚决反对中国共产党所确定的文艺方向,极端仇恨毛泽东同志《在延安文艺座谈会上的讲话》。因为党和毛泽东同志号召文艺工作者要歌颂工农兵,要暴露工农兵的敌人,而胡风集团恰是工农兵的敌人,他们觉得暴露工农兵的敌人这会使他们混不下去,就会“屠杀”他们这伙反革命的所谓“生灵”,就会“压杀了”他们这伙反革命的所谓“新的东西”。但是他们不敢公开地反对这个讲话,而且胡风还教唆他的党羽在表面上要“顺着它”,有时并引用其中的一些字句。这些.都是胡风分子伪装自己的假面具。而在这封密信里,就完全暴露了胡风分子仇恨这个讲话和反党的真面目。张中晓说:“这书(指毛泽东同志在延安文艺座谈会上的讲话),也许在延安时有用,但,现在,我觉得是不行了。”“现在不行了”,在文艺界里面不是还有一些人也这样说过么?说过这种话的人,请注意读读张中晓这封信吧!当然,有些说这样话的人,他们还只是抱着资产阶级的文艺观点,所以不能认识这个讲话的重要性。但是,张中晓这个胡风分子,凭着他的反革命的敏感,却深深地了解这个讲话在全国解放以后会在更广大的范围内掌握群众,并对各种反动的文艺思想起摧毁性的作用,所以他们就急于想阻止和破坏这个讲话的影响的扩大,他们所谓“现在……不行了”,道理就在这里。

按:从这类信里可以看出,反革命分子的攻击少数人不过是他们的借口,他们的一种策略。他们的本意是“几乎没有一块干净的土地”。由于这种情况,他们“就规定了战斗的艰苦性和长期性”。自从汉朝的吴王刘濞发明了请诛晁错(汉景帝的主要谋画人物)以清君侧的著名策略以来,不少的野心家奉为至宝,胡风集团也继承了这个衣钵。他们在三十万字上书中只攻击林××、何××、周×同志等几个人,说这几个人弄坏了一切事。有些在阶级本能上同情胡风的人,也照着这样替胡风瞎吹,说什么“这不过是周扬和胡风争领导权的个人之间的斗争”。我们在肃清胡风分子和其他反革命分子的斗争中,这一点也是应当注意的。

按:胡风集团在他们的三十万字上书和其他的公开言论中,好像他们主要只是反对共产党的作家而不反对其他的人。他们当然从来不反对蒋介石和国民党的其他人物(只在有时小骂几句以作晃子,即所谓“小骂大帮忙”),但不反对其他的人则是假的。这一点我们从胡风们的许多密信得到了证实,原来他们对鲁迅、闻一多、郭沬若、茅盾、巴金、黄药眠、曹禹、老舍这许多革命者和民主人士都是一概加以轻蔑、谩骂和反对的。这种不要自己集团以外的一切人的作风,不正是蒋介石法西斯国民党的作风吗?

按:如同我们经常在估计国际国内阶级斗争力量对比的形势一样,敌人也在经常估计这种形势。但我们的敌人是落后的腐朽的反动派,他们是注定要灭亡的,他们不懂得客观世界的规律,他们用以想事的方法是主观主义和形而上学的方法,因此他们的估计总是错误的。他们的阶级本能引导他们老是在想:他们自己怎样了不起,而革命势力总是不行的。他们总是高估了自己的力量。低估了我们的力量。我们亲眼看到了许多的反革命:清朝政府,北洋军阀,日本帝国主义,墨索里尼,希特勒,蒋介石,一个一个地倒下去了,他们犯了并且不可能不犯思想和行动的错误,现在的一切帝国主义也是一定要犯这种错误的。难道这不好笑吗?照胡风分子说来,共产党领导的中国人民革命力量是要“呜呼完蛋”的,这种力量不过是“枯黄的叶子”和“腐朽的尸体”。而胡风分子所代表的反革命力量呢?虽然“有些脆弱的芽子会被压死的”,但是大批的芽子却“正冲开”什么东西而要“茁壮地生长起来”。如果说,法国资产阶级的国民议会里至今还有保皇党的代表人物的话,那末,在地球上全部剥削阶级彻底灭亡之后多少年内,很可能还会有蒋介石王朝的代表人物在各地活动着。这些人中的最死硬分子是永远不会承认他们的失败的。这是因为他们不但需要欺骗别人,也需要欺骗他们自己,不然他们就不能过日子。

按:这个胡风分子是比较悲观的。他说“也许”要“再过几十年”才“可以办到人与人没有矛盾”,即是说,要有几十年时间,蒋介石王朝才有复辟的希望。几十年之后,蒋介石王朝回来了,一切人民革命力量都被压倒了,那就是“人与人没有矛盾”了。“人的庄严与真实才不受损伤”,这个“人”指的是一切反革命的人,包括胡风分子在内,但是一个也不包括革命的人。“今天中国,人还是不尊敬人的……”上人指革命的人。下人指反革命的人。胡风分子写文章,即使是在写密信,也会有些文理不通的。这也是被他们的阶级本质所决定的,他们无法像我们替他们作注解的时候这样,把文理弄得很清楚。

按:这封信里所谓“那些封建潜力正在疯狂的杀人”乃是胡风反革命集团对于我国人民革命力量镇压反革命力量的伟大斗争感觉恐怖的表现,这种感觉代表了一切反革命的阶级、集团和个人。他们感觉恐怖的事,正是革命的人民大众感觉高兴的事。“史无前例”也是对的。从来的革命,除了奴隶制代替原始公社制那一次是以剥削制度代替非剥削制度以外,其余的都是以一种剥削制度代替另一种剥削制度为其结果的,他们没有必要也没有可能去作彻底镇压反革命的事情。只有我们,只有无产阶级和共产党领导的人民大众的革命,是以最后消灭任何剥削制度和任何阶级为目标的革命,被消灭的剥削阶级无论如何是要经由它们的反革命政党、集团或某些个人出来反抗的,而人民大众则必须团结起来坚决、彻底、干净、全部地将这些反抗势力镇压下去,只有这时,才有这种必要,也才有这种可能。“斗争必然地深化了”,这也说得一点不错。只是“封建潜力”几个字说错了,这是“无产阶级和共产党领导的以工农联盟为基础的人民民主专政”一语的反话,如同他们所说的“机械论”是“辩证唯物论”的反话一样。

按:还是这个张中晓,他的反革命感觉是很灵的。较之我们革命队伍里的好些人,包括一部分共产党员在内,阶级觉悟的高低,政治嗅觉的灵钝,是大相悬殊的。在这个对比上,我们的好些人比起胡风集团里的人来,是大大不如的。我们的人必须学习,必须提高阶级警觉性,政治嗅觉必须放灵些。如果说胡风集团给我们一些什么积极的东西,那就是借着这一次惊心动魄的斗争,大大地提高我们的政治觉悟和政治敏感,坚决地将一切反革命分子镇压下去,而使我们的革命专政大大地巩固起来,以便将革命进行到底,达到建成伟大的社会主义国家的目的。


