\section[与合众社记者的谈话(一九三八年二月)]{与合众社记者的谈话}
\datesubtitle{(一九三八年二月)}


<b>问:</b>现在有许多人对中国抗战的前途表示悲观,先生对此意见如何?

<b>答:</b>我对此完全是乐观的,因为中国抗战的过程必然是先败后胜,转弱为强,这已经成了确定的方向了。在中日战争初期,一般形势是日本强、中国弱,可是今后的形势必是日本的弱点渐渐暴露出来,中国的力量则渐渐加强起来。日本现在正借钱打仗,除过去半年已经用去二十二万万元以外,今年一年的需要据说是四十万万元,必定还不止此数,这已经消耗他大量的国力。日本的国际信用降低,公债跌落,他的“速战速决”计划已经失败,试问他那有许多钱无限长期打下去?就军事方面讲,日本在中国的战线已经延长到自杭州已达包头的数千里的距离,他的兵力不够分配防守之用,所以他的兵力已随深入与扩大的程度渐渐薄弱。他占领了长距离的铁路,便需要军队去防守每一个车站,日本已动员三分之一的军队来中国了,如果他再要占领汉口、广州等地,至少须再动员几十万军队,那时他的情况将十分困难。因为日本的敌人不止中国一个,加上日本国内国际的其他许多大矛盾,他终必走上完全崩溃之途。

<b>问:</b>先生说中国的力量能够渐渐加强起来吗?

<b>答:</b>根据过去七个月的作战经验,在军事上我们若能运用运动战、阵地战、游击战三种方式互相配合,必能使敌军处于极困难地位。我的意见,在目前除应以二三十万精兵组成数个强有力的野战军,在运动战中给敌人前进部队以歼灭的打击之外,还要抽调八九万军队组成二三十个基本的游击兵团,每个兵团三四千人,派坚决而机动的指挥员领导,加强其政治工作,配置于从杭州到包头的敌人阵线前面,从这个长阵线的二三十个空隙中间,打到敌人后方去,如能运用得宜,结合民众,繁殖无数小游击队,必能在敌后方建立抗日根据地,发动千百万民众,有力地配合野战军的运动战,而使敌人疲于奔命。至于阵地战,由于我们技术不足,在目前不应看作主要方式。但我们必须建设国防工业,自制重武器与高武器,同时设法输入这些武器,以便能有力地进行防御的与攻击的阵地,这是非常必要的。有人说,我们只主张游击战,这是乱说的,我们从来就主张运动战、阵地战、游击战三者配合。在目前以运动战为主,以其他二者为辅,在将来要使阵地战能够有力地配合运动战,而游击战,在他对于战斗方式说来,则始终是辅助的,但游击战在半殖民地的民族战争中。特别在地域广大的国家,无疑在战略上占着重大的地位。在政治方面,我们已有国内的统一,更拥有全世界民主国家的同情和援助。但现在的成绩还不够,还应进一步加强起来。所有上述军事上政治上的加强,都是必须的。只要继续努力,也一定能够加强,这就预示着前途的光明。

间,第八路军在日本人数面包围之中有被日本人驱逐或歼灭的危险吗?

<b>答:</b>第八路军现在共在四个区域中进行广大的游击战。第一个区域是平汉、平绥、正太、同蒲四铁路中间及其以东以北的地域,这地域拥有坚决反日的一千二百万民众,都与军队密切结合着,这是一个极大的抗战的堡垒。第八路军在这里已经建立了脚跟,虽然敌人正在加紧进攻这个区域,但要驱逐他们是不可能的,歼灭更不可能。八路军的几个大的东进支队已迫近津浦线。第二个区域是平绥以南,同蒲北段以西和黄河以东的晋西北地带。三个区域,是平汉、正太、同蒲中间的晋东南、冀西南地带。第四个区域是晋西南。他们都与地方人民有密切的联系,都随时猛烈地破坏敌人的后方联络线,有了很多大小的胜利,使敌人大灭其前进的力量,从这些区域看来,中国失去的不过是几条铁路及若干城市而已,其他并没有失掉。这一实例给全国以具体的证明,只要到处釆用这种办法,敌人是无法灭亡中国的。这是将来举行反攻收复失地的有力基础之一。

<b>问:</b>先生觉得此次国共合作是具有永久性的吗?

<b>答:</b>我以为是的。在民国十七年国共分裂的时候,原是违反着共产党的志愿的,共产党一向不同意和国民党分裂。过去十年来国共双方及全国人民都经历了艰苦的经验,这种经验能增强今后的团结,现在及将来合作的目的是共同抗日与共同建国,在这个原则之下,只要我们的友党能有和我们一样的诚意,加上全国人民的监督,这个合作必然是长久的。

<b>问:</b>是的,共产党对时局宣言说过,国共两党不但共同抗日,并且在抗日战争胜利之后还要建国,请问这代表着两个不同阶级的政党将来怎能建立新国家呢?

<b>答:</b>因为中国是处在半殖民地地位,目前则更处在亡国灭种的关头,连半殖民地地位都处在危险中。党派与阶级虽不同,这个共同的地位则一,就此决定了两党,不但能合作抗战,并且能合作建国。但合作是在一党纲领下的合作,是有原则的合作,是真正的合作,而不是苟合与貌合,如果离开纲领与原则,必变成苟合与貌合,这是任何有原则的政党所不许可的。有纲领有原则的合作,如同朋友之间的道义之交,只有这种道义之交,交情才能长久。

问。什么是共产党主张的“民主共和国”?

答:我们所主张的民主共和国,便是全国所有不愿当亡国奴的人民用无限制的普选方法、选举代表组织代议机关这样一种制度的国家,这种国家就是民权主义的国家,大体上是孙中山先生早已主张了的,中国建国的方针应该向此方向前进。

<b>问:</b>共产党对于目前的中央政府满意吗?还须召开临时国民大会吗?

<b>答:</b>我们拥护现在的中央政府,因为他坚持抗战的方针,并领导抗战的行动。但我们希望加以充实和扩大,并在内政上进行必要的改革,以便更加有利于抗战。我们曾经提出了临时国民大会这种主张,这也是孙中山先生提倡过的,我们认为有益于团结全国加强抗战的力量。但究用何种方法更于抗战有利,我们并无成见,只要真正有利于抗战,什么方法都可以采用的。

<b>问:</b>东三省义勇军的抗日活动,有中国共产党前去领导吗?

<b>答:</b>中国共产党和东三省抗日义勇军确有密切关系。例如有名的义勇军领袖,杨靖宇、赵尚志、李红光等等,他们都是共产党员,他们的坚决抗日艰苦奋斗的战绩,是人所共知的。那里同是民族统一战线,除共产党外,还有其他的派别及各种不同的军队与民众团体,他们己在共同的方针下团结起来了。

<b>问:</b>先生对于美国一般感想如何?

<b>答:</b>美国民主党的赞助国际和平,罗斯福总统的谴责法西斯蒂,霍华德系报纸的同情中国抗日,尤其是美国广大人民群众对于中国抗日斗争的声援,这些都是我们欢迎与感谢的。不过希望美国能更进一步出面联合其他国家给暴日以实际的制裁,现在是中美两国及其他一切反对侵略威胁的国家更进一步联合对敌的时候了。

<p align="right">抄自《毛泽东救国言论集》</p>

