\section[同李雪峰等同志的谈话(一九六四年八月二十日)]{同李雪峰等同志的谈话}
\datesubtitle{(一九六四年八月二十日)}


我们不会搞社会主义,大家没有搞过嘛,我没有搞过,你们会吗?搞了十几年,有了成功的经验和失败的经验。现在才有了些经验了。经过北戴河会议,十中全会,略微动了一下。到一九六三年五月杭州会议,搞了第一个十条。前面的序言是我写的,说人的认识事物是不容易的,正确的思想是从哪里来的?客观事物反映到我们脑子里可不容易啦,物质变精神,精神变物质。南昌有一个研究科学的青年说,物质变精神可以理解,精神变物质大部分可以理解,但变石头则不可能。例如大理石有许多种,有自然的大理石,有人造的大理石,人造的大理石不是石头?人民大会堂的大理石很多不是山里的,是人造的。人为什么能造大理石?因为理解了大理石的化学构成。什么叫大理石?是石灰石,是碳氧化钙。几千年来,老百姓知道,把石灰石一烧,一氧化碳挥发了,剩下生石灰,放在水里会发热,变成氢氧化钙。认识这个化学过程,人就可以造石灰,精神就可以变石头。所以,化学这门学问是一门精神的学问。写在书上变成思想,然后再变成物质,可以造各种肥料嘛!电石也是一种石头,即碳化钙。通过电产生高温,使碳和钙结合起来,就成了电石。

人是由蠢慢慢变聪明的。无论什么人,都是由不知到知,由少知到多知。至于全知的人,没有那回事。马克思知的多一些,但也不是全知,我们就不用说了。单是社会科学有几百门,你能全知?戏有几百种,你能都懂?河北的戏有七十种。你们都熟悉吗?有的看也没看过。评剧、梆子、老调,《追鱼》男角是女的演的。内蒙的戏我没看过,山西的在晋西北看过《打金枝》,女的唱老生。

全知也不合理。全知等于不知,一样都不精嘛!自然科学有几百门,怎么学得了?一个人只能活几十年。社会科学就分很多门类,哲学、经济学、社会主义。社会主义又分好多门,政法又是公安、法院、检察,公安又分侦察、审讯、民警、消防,一个公安部长都学得会?!我看部长没有学过消防。社会科学,又分上层建筑,又有经济基础,又有生产力。上层建筑有党、军队、政府,又有数育、文化、报纸,又有唱戏、小说、诗歌、绘画、雕塑、音乐、电影,可多啦。唱一出戏,有旦角、老生、花脸、小生、丑,丑又分男丑、女丑、文丑,旦又有老旦,你去学,学得了那样多?\marginpar{\footnotesize 154}


