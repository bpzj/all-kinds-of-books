\section[乡苏维埃怎样工作(一九三三年)]{乡苏维埃怎样工作}
\datesubtitle{(一九三三年)}


(一)乡苏维埃代表会议的工作

1.乡苏维埃(与城市苏维埃)是苏维埃政权的基本组织,是最接近民众及与民众关系最密切的一级,是直接领导群众执行苏维埃各种设施与法令的机关。

2.乡苏维埃(代表会议)是全乡最高的政权机关,上级苏维埃的各种重要的法令、设施,本乡的各种政治的、动员的、经济建设的以及其他的问题,都要经过代表会议讨论、决定与执行。

3.因此,乡代表会议应该经常的按期的开会,有些地方代表会议不经常开会是不对的。代表会议最好二十天一次(半个月或一个月一次也可以,应该照实际情况来决定)。有特殊紧急的事情,可以临时的由主席团召集代表会议,会议的时间不应该很长,通常每次两三点钟为好。在农忙的时候,代表会议应该在晚上开。

4.每次代表会议应该有个议事日程,就是这次会议要讨论的问题,这个议事日程应该由主席团先准备好,并早几天就通知代表们,好让代表们事先准备并征求群众的意见。每次的议事日程上只应有一个主要的问题。此外,可以有一两个次要的问题。假使每次会议,同时提出三四个大问题来讨论,会使会议得不到很好的结果。

5.每个讨论的问题,应该有一个同志做报告(乡苏维埃主席或者专门管理这种工作的同志,如各个委员会主任,赤卫军大队长,贫农团主任等等)。报告要切实具体,有内容。报告的时间至多十分钟。其余的时间应拿来讨论。讨论之后,应该由报告人或主席把讨论中代表们发表的意见,扼要的做一个结论。

6.每次讨论的问题,应得实际化,并且做出具体的决议或决定,并且指定某一个同志或某几个同志去实行,比较大的问题,可以分开几个项目来讨论,譬如,讨论春耕问题,就可以分开,“怎样解决劳动力的困难?”“怎样解决耕牛、种子、农具的缺乏?”“怎样防止夏荒,多种那些早粮,种多少?”“多种棉花”,“多锄多犁”等等。每项讨论后,应该做出具体决定与分工。譬如“怎样解决劳动力的困堆”决定:(一)组织耕田互助社,两星期内成立,甲村由张同志负责,至少要有三十社员;乙村李同志负责,至少五十人。丙村工同志负责,至少三十五个社员。(二)发动妇女参加。第一,发动各代表的妻子首先参加,各代表负责,张秀英女同志负责督促。第二,要在下头妇女代表会议讨论这个问题,张秀英同志去和妇女代表会主任商量和负责召集,并代表乡苏维埃去做报告。第三,组织生产学习组,由白同志负责。(三)发动儿童参加劳动:第一,由某同志头告诉儿童团,发动他们参加拾粪、种瓜、种菜。第二,由吴同志负责和列宁小学交涉农忙时只上半天学,半天学生回家帮助作庄稼等等。讨论时一项一项的讨论,讨论了一项得了结果,把它写在记录簿上,主席团及下次代表会议,可以根据来督促检查每一个决定。主席团应该经常去督促指定的同志去完成任务,下一次或相隔不久的代表会议上,应该检查这些决定实现了没有,某同志没有实行,应该给他相当的批评。某同志实行时做错了,要好好的告诉他怎样去改正。这样,代表会议总会具体、切实、活泼、有生气。

7.代表会议要做到个个代表都能到会,就要在这次会议上决定下次开会日子(或者一般的决定比如逢五逢二十开等)。开会前三五天,主席应该经过村主任,再通知代表们开会的日子、时间、地点,讨论的问题,开会时每次都要点名,没有到的要在他名字上打个记号,主席团或村主任散会后,应该去告诉没有到的代表,问他的理由,并告诉他代表会议的讨论和决定。如果没有理由而不到的,应该批评和督促他下头出席。

8.代表会议开会时的主席,应该由乡苏维埃主席充当,因为他熟悉全乡的情形,了解前后工作的经过。

9.要代表会开得好,就要主席团在事先准备得好,主席在代表会议开会前三五天,应该先开一次讨论代表会议的议题,指定做报告的人,帮助他准备报告,乡主席要特别负责。

(二)主席团的工作

10.主席团是代表会议闭幕后,全乡最高政权机关,它应该执行代表会议的决议,上级苏维埃的命令和指示,积极领导全乡的工作。

11.主席团由代表会议选举出来的,大乡五人,小乡三人,应该选举最积极最有工作能力的人,主席团中应选出一个主席一一就是乡苏维埃主席。

12.主席团应有一种分工,如果是三个人的主席团,则应该分工为主席、付主席、支书(要识字的,如代表或主席团里没有识字的,则以应该另外找一个文书,他就不参加主席团);如果是五人的主席团,就可以分做主席、付主席、支书及春耕委员会主任、优红委员会主任(或别的重要的委员会主任)。

13.主席主持全乡工作,出席上级召开的会议,处理日常事务,他应该抓紧每个时期的几件中心工作(如扩红、查田、选举、修路、春耕运动、发展合作社等等),应该注意经常工作的领导(如赤少队训练、赤色戒严、粮食、教育、卫生、优红等等),还要注意本乡发生的临时问题(如饥荒、瘟疫、反革命活动等等)。主席应该把全乡各方面的工作经常的想一想,看那些进行得怎样,那几项工作做得薄弱或者不好,应该怎样去加强,那几个村子工作落后,应该怎样去帮助。他要担任出席一两个行政村的会议,收集各村每一项的工作上的好的经验和坏的现象,提到主席团或代表会议上去报告,供给会议在讨论这个问题时的材料,主席同样要担任出席一两个委员会及群众团体的会议。付主席帮助处理日常事务,对于全乡的工作要有一般的了解,以便主席不在时能代理主席的工作。付主席也要担任领导一两个村的工作,出席几个委员会和群众团体的会议。

文书要办理人口册,土地登记,婚姻生死登记,选民登记,各种调查表,写对区苏的报告,打路条,发通知,帮助居民及红属写信,主席与付主席不识字的,要把上级的文件读给他们听。重要文件应该在代表会议上宣读,会议时写记录。文书是代表时,应该参加主席团,就可以分配他担任出席几个委员会及群众团体会议。

主席团其余的委员,亦应该分别担负领导一两个村及委员会、群众团体的工作。

14.主席团会议要经常的开,最好五天或七天一次。主席团开会时,有重要问题可请村代表主任及委员会主任来参加,普通的问题可以不必参加。

15.主席团会议要经常的讨论各委员会的工作报告(譬如春耕委员会,优红委员会等等),各群众团体的报告(如赤卫军,耕田合作社。妇女代表会等),并且检查上一次代表会议的决定,各村代表主任及代表执行了没有,执行得怎样。或者决定下一次代表会议的议事日程,报告人,准备下一次代表会议,在全乡工作中比较小的问题及急切需要解决的问题,都要在主席团会议上解决,不要把每一个小问题,都提到代表会议上去。不要把急切需要解决的问题拖延等待到下一次代表会议才能讨论。主席团解决的问题在代表会议上报告,通知代表们取得他们的同意。

16.主席团会议应该有一个一个月或二个月的日程表。日程表的格式如下:

参加人报告人议事日程会议日期

各村王任,春耕王春山耕田合作社工作三月五日

委员会委员李振国扩红工作,检查下次三月十日

谷村主任乡主席代表会议准备。三月十五日

各村主任乡主席春耕进行状况的检查三月二十日

主席团照日程表工作,并发给各委员会主任、村主任,先让他们也知道什么日子讨论什么问题,日程表上每次会议不要到(列)出太多的问题,一两个问题尽够了,因为一定会有一些零碎或紧急的问题要临时加上去的。主席团会议亦不要超过二、三小时。

17.主席团要领导全乡的工作,就必须抓紧每一个时期中心工作,不要忙于零碎的事务,去掉了中心工作。主席团要明白各村的情形,要了解各村的特点,要注意各村群众中间的困难问题,主席团要注意领导和推动各村代表主任的工作,使得各村的代表会议群众会议都能经常开,开得好。

18.主席团要团结全体代表,各个委员、各群众团体在自己的周围,推动他们去动员全乡群众执行各种工作,而不应该离开代表、委员会与群众团体,而少数人去忙乱。

19、主席团应该注意教育与训练代表。代表中积极努力的应该在会议上给予鼓励表扬,消极怠工的应该给以批评斗争。初次当选代表的,不知道工作怎样做的,就要好好的告诉他怎样去做工作。那些经过多次批评教育斗争不改变不进步的,或者接连四、五次会议都不到会的,应该提到代表会议上去讨论,把他开除出去,把后补代表补上来。代表在当红军或调动工作或别的原因离开本乡的,亦应该立刻补上。

(三)村的组织和工作

20、乡的工作的重心应该放在村,在地广人稀的区域里,尤其应该这样。新乡苏维埃应该特别注意村的领导和工作。

21、每个乡应该按照村庄的分布,人口的多少,划分成为行政村,普通每乡划成三至五个村为好,再多了就不大方便。

22、每个行政村设村代表主任一人,由该村乡苏维埃代表中推选出来,要推选代表中最积极最有工作能力的人充当。在该村代表在十五人以上时,可以另外推选一个副代表主任。代表主任负责领导督促全村工作,副代表主任帮助他。

23、村主任应该按时(十天或者半个月一次)召集全村的乡代表开会,检查各代表的工作。按照各庄子各家的情形,讨论怎样完成乡代表会议给予本村的任务,解决本村居民中间的小的争执问题,及困难的互相帮助和救济。这种村代表会议最好是在两次乡苏代表会议之间开。

24、为使全体代表能够都负起责任来,领导居民群众完成苏维埃的一切决定和工作,就要实行每个代表分工领导居民的办法。譬如某行政村共有居民五百人,代表二十人则将这五百居民分成二十个单位,每一个代表管理一个单位。但是不应平均分配,而应该按庄屋的位置,代表的能力来分配。有些代表多管一些,有些少管一些,每个代表应该负责去领导督促我管理的居民群众去实现苏维埃的各项工作和决定(如督促所管理的几十个人努力春耕,宣传这几十个人中间的积极分子去当红军,宣传加入合作社,督促儿童入学,督促大人进夜校,督促各家打扫房屋讲究卫生,注意地主反动分子活动,等等)并且吸收群众的意见,提到村和乡代表会议上去解决。

25、每个代表要召集自己所管理的几十人开会,这种会议以半个月一次为好。会议釆取谈话的方式,在晚上或有空闲的时间举行之。在这种会议上,检查各家执行苏维埃工作的情形,讨论现在要做的工作,报告乡苏决议,征集群众的意见,两村或乡代表会议报告,会议并可以开展相互间的批评,如某家当红军的回来未归队,某家春耕不努力,某人帮助红军家属不上紧,某家孤老应该帮助,大家不注意等等。这种会议可以最迅速普遍的传达苏维埃的决定,使苏维埃的工作很快很顺利的进行,使群众生活很快的得到改善。

26、每个代表对于自己管理的几个人的领导,除了召集会议外,应该经常的到各家去访问访问,看看各家的生活状况有什么困难没有,实行苏维埃工作到底怎样,有什么要解决的问题。首先应该常到红军家属里去访问。

27、代表分工管理居民的制度实行之后,村和乡的群众大会除了大的纪念节及特别重大的问题外,不应常常召集。

(四)乡的委员会及其工作

28、为吸收更多的群众参加国家政权的工作,在乡苏之下要设立各种委员会,分别管理、领导各项工作,这种委员会由乡代表及群众由的积极分子为委员,每个代表都应该参加一两个委员会。

29、这种委员会应该分成两种,一种是经常存在的;一种是为了某个特别的问题临时组织起来的经常的委员会,如扩大红军、优待红军家属、文化教育、卫生、粮食等委员会。临时的如查田、选举、水利修筑等委员会,这种委员会在进行这项工作时,要把它组织起来,工作做完了就不要了。

30、委员会的人数应按工作的性质,少则五人多则七人至九人。委员会中间必须每村都要有人参加,因为这样委员会的讨论和决定,就很快的可以传达到每一个村去。委员会的名单及主任,应该由主席团提出名单,在代表会议通过。在名单中除了代表外,必须吸收非代表的群众的积极分子参加。

33、各委员会的主任要选最积极的同志担任(最好是代表)主席团要经常去检查各委员会的工作,听他的报告,要教育督促各主任去工作,去召集委员会的会议。乡主席团应该分工去参加各委员会的工作。

32、各委员会应该经常的开会,但开会的期限应该看工作的轻重缓急来定。譬如在农忙前和农忙时,春耕委员么应该多开会。秋收以后,农闲的时候,那末道路委员会、水利委员会等,都要加紧工作。

33、主任对自己的委员会初开会事先应该有准备,才能使每次会议都能得到结果。会后要分配每一个委员以一部分工作,告诉他们怎样去做,下次会议要报告他们的工作。

(五)乡苏维埃与群众团体的联系和对他们的领导

34、乡苏对于本乡的各种群众团体(如工会、贫农田、妇女代表会、儿童团、劳动互助社、各种合作社、识字运动委员会等),要加紧对于他们的帮助和领导,要依靠他们去努力动员广大的群众完成各种革命工作,特别是贫农团、妇女代表公、劳动互助组等团体,乡苏更应负责检查和领导,因为他们没有直属的上级组织。

35、当代表会议或主席团讨论和某个群众团体有关的问题时,应该请那个团体的负责人参加,各团体开会时,乡苏也应该派人去参加。主席团亦可以听各个群众团体的工作报告。

36、乡苏维埃注意对赤卫军、少先队的领导,要使全体男女都加入。注意他们的训练,要使赤卫军、少先队完全担任放哨查路条的赤色戒严的工作。

附注:上面的工作条文大部分是适用于中心区域的,乡苏、边区的苏虽一般亦可以使用,但应特别注意:(一)领导游击小组向外游击;(二)做白军团匪中间的工作。(三)加强赤色戒严及粮食储藏;(四)向白区开辟新苏区。

