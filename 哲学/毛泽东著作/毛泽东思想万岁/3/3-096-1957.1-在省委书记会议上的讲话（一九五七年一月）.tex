\section[在省委书记会议上的讲话(一九五七年一月)]{在省委书记会议上的讲话}
\datesubtitle{(一九五七年一月)}


这次会议主要研究三个问题,即思想动向问题,农村工作问题,经济工作问题。还研究其它一些问题,如增产节约,法制(包括肃反),编制(主要是精简),中央与地方分权,税收,统战,少数民族,灾区,学生(大中小学),想尽一切办法解决今年的困难,粮食、山区、整风等问题。

思想动向问题,党内思想动向,社会思想动向应该抓住。

有的是家里的事,争名夺利,“唯利是图”在党内有很大的发展。提一级不起床,提二级还哭鼻子,提三级才起床。党内不是比艰苦奋斗,比多做工作,而是比待遇,比地位。过去国民党有一个唐绍仪,由内阁总理做广东省中山县县长,我们共产党还没有一个部长去当县长。总而言之,升得降不得,比高不比低,比阔气不比艰苦。

再一个,对社会主义改造的看法问题。合作社究竟有没有希望,究竟合作社好还是个体经济好,听说重灾地区和丰收地区都没有问题,就是灾而不重,收而不丰的那些地区发生问题。一部份人没有增加收入而减少了收入,于是议论都来了。这也反映到党内,北京有些部长下乡回来说:“农民无精打采,生产情绪不高”。有的干部议论:“合作化大有灭亡崩溃之势”。“合作社没有什么优越性”,地方上有的宣传部长也不敢宣传合作化的优越性。合作社社长抬不起头来,到处受批评,三面挨骂。农业部门开了一次会,农村工作部,农业部两个系统泄了气,四十条有无必要,能否实行也发生了问题。

前年反右倾,出了冒进。去年反冒进,又出了右倾(右倾主要指农村,别的方面未究竟)。这里有一股风,各地可能都有,只是大小不同。好像台风有八级,有十二级。在干部中有一点值得注意,大部分干部家庭是地主,富农,富裕中农(富裕中农是动摇的阶级),听了家里的话,就说合作社不行。合作化无论如何会化好的。但一年两年不可能办好,我们合作化实际上只有一年半的历史,要有三年历史就可以基本稳当,有五年历史就吹不倒了。

学校办了,毕业生不能分配工作,不满意。地质部在石家庄专区正定县有一个地质学校,有一千多人因为没有分配工作而罢课闹事,要上北京,他们的标语口号有“打倒法西斯!”“要战争不要和平!”“社会主义落后,没有优越性!”北京大学有一个学生公开说:“总有一天,老子要杀几千几万人。”×××去讲话:“你要杀人我要专政!”北京高等学校学生百分之八十以上是地主、富农、富裕中农、大中小资本家子弟,工人、贫农子弟不到百分之二十。哥穆尔卡原来反对教条主义,他的话在学生中很吃得开,现在转过来反对右倾,也不行了。铁托、卡德尔的演说也受一部分人欢迎。听说农村中地主、富农到相当守规矩,城市中的民主党派、资本家也比较守规矩。因为这些人脚底下是空的,没有本钱。至于天下有变,上海、北京打下来原子弹,这些人变不变就很难说。现在这些娃娃们没有经验,没有他们老子长于世故,所以就冒了头,大喊大叫一气。

苏共二十次代表大会以后,我们党内大多数人是正常的,稳定的,少数人有波动。下雨之前总会有蚂蚁出动,中国也有少数蚂蚁想出洞活动。现在赫鲁晓夫改变了,蚂蚁也缩回去了。苏共二十次代表大会后,发生了两次大风潮,许多国家的党受损失,英国党搞掉四分之一,瑞士搞掉一半,美国党搞得天下大乱。东方党、中国党受影响较小,斯大林问题牵涉到整个共产主义运动。有些人批判斯大林不加分析,过去拥护斯大林最厉害的人现在反对斯大林也最厉害。突然转了一百八十度,不讲马列主义,不讲道德。党内有些人台风一来就动摇,动摇分子一有风声就动摇。有的摇一、二次就不摇了!有的永远摇。小树、稻子高粱、包谷、墙头上的草,总是见风来就摇,只有大树不摇,台风年年有,政治台风不一定年年有,这种现象是社会政治的自然现象。

中国党是无产阶级、半无产阶级的党,但有许多党员是地主、富农、资本家家庭出身的。有的党员虽然艰苦奋斗多少年,但马列主义没学好,思想上、政治上经不起台风,应当注意。党内有些人别的关都过了,只有社会主义关过不了。河北省委付书记薛迅,是“九·一八”后学生南下请愿的代表,统购统销中动摇了,坚决反对统购统销。全国合作供销社付主任孟用潜也上书言事,坚决反对统购统销,其实统购统销是实现社会主义的重要步骤,他们这些人就是社会主义这一关过不去。我们党内这些现象,还没有结束。再过十年可能还不会结束。十年之后,是否所有的人真心相信社会主义,也不一定,一遇到问题,有人就会说他早就想到了。

军队干部对农民的生活反映要分析。有合理的部分(如薪水高),也有不正确的部分,可能有些人听了家里的话,替农民叫苦。一九五五年上半年梁漱溟也替农民叫过苦,党内也有人叫,好像党中央省委不代表农民,只有他们代表农民。江苏有一个调查,叫苦的人,大多数是家里有余粮不卖。军队有同志反映说:“乡村左了,城市右了。”这种说法根本是错误的。举例说,一个农民一年收入六十元,工人一个月收入六十元,但一个工人养活四口人,平均每人十五元。以户来算,工人每年拿七百二十元,农民每年拿二百四十元,但城市费用大,农村费用少。农民生活的改善,要靠自己生产,国家不能给他们发工资。农村很多付业没有抽税,税收取了农民收入的百分之八,再从剪刀差取了农民一部分,不到百分之十二,两部分最多不超过百分之二十,比苏联百分之四十五少的多,至于统购,只按市价买的。国家帮助农民主要是肥料、水利(小型水利农民搞,大中型水利国家搞)城市看来好像右了。过去资本家每年拿去一亿五千万,六年拿了九亿,现在定息七年这么给他八亿,共十七亿。那时候还要看情况,这关系到国际问题。出这一点钱买了这么个阶级(包括它的知识分子、民主党派共约八百万人),他们是知识比较高的阶级,要把他们的政治资本剥干净,办法一是出钱赎买,二是出位子安排。共产党加左派占三分之二,三分之一非举手不可,不举手就没有饭吃。他们的子弟要学匈牙利,挪到他父亲那里就要打屁股。

大学生给你助学金,请你读书。你要扩大民主我不怕,一遇大民主,第一欢迎,第二分析。对的我接受,错的我揭露。我无理就承认错误,坚决改正。你无理就抓住你的小辫子。地质学校罢课闹事,一讨论就分化了,七十个代表赞成三个口号的只有十几个人,反对的五十几个人,放到四千多人中间去讨论,都不赞成,十几个人被孤立了。扩大民主是好事,匈牙利没有大民主,就不能镇压反革命。出了乱子是好事,疮溃了脓包就解决了。不要怕事,

过去帝国主义蒋介石我们都不怕,难道还怕大民主吗?共产党怕大民主太不像样子了,不如段祺瑞,现在的党和政府是革命的,这一点上了宪法,娃娃们发疯乱叫:“打倒法西斯”,“要战争”,是违犯宪法的。中国改革不会被这些娃娃们冲垮的。如果几个学生娃娃把我们冲垮,那我们就是饭桶。

去年一年是多事之秋,是中国社会主义改造最激烈的一年。

我只讲一个问题。

其它问题,增产节约有一个文件。

肃反必须坚持,有反必肃,缩手缩脚是不行的,但法制必须遵守。

精简一定要坚持搞,一条是减人,一条是安排。一定要先把人安排好,再送出机关。

中央地方分权,你们感到不“过瘾”:“一没钱,二没权”。你们要什么请尽量提出来。

统一战线工作:有重安排不重改造的偏向。

百花齐放,百家争鸣中对知识分子不敢改造。马寒冰等四同志对文艺工作的意见不好。苏联下面赞成百花齐放,百家争鸣;上面有入说:“只能放香花,不能放毒草。”我们的意见,只有反革命的花不能让他放,但是他要用革命的面儿放,就得让他放,里面有禾有草,只准放禾,不准放草,事实上办不到。草翻过来,就是好肥料。我们作家的任务就是要和杂草作斗争。年年长草、年年除。没有国民党,就显不出共产党好。没有唯心主义,显不出唯物主义好。没有对立就没有斗争。只有斗争出来的,才经得起考验。矛盾不断发生、不断斗争、不断解决,一万万年如此,学了正面的,还要学反面的。只讲唯物主义不讲唯心主义,只讲辩证法,不讲形而上学,你就不知道反面的东西,正面的东西也不能巩固,因此不仅要出孙中山全集,蒋介石全集也要出,黑格尔、康德、孔子、孟子、老子、二程、朱、王都要讲。你没看过蒋介石的文章,要反对人家就反不好,有人说,自从提出百花齐放,百家争鸣以后,文学衰落了。×××的报告才只有五个月。就算文学衰落了,哪儿那么快。大家伙要长期准备,短时问写不出来的。作家要做工作,不做工作不给饭。作家坐在家里不到外面去,肚子里没有货,怎能写出好东西。

干部不提级,减薪问题要研究。

学生问题有入学、就业两个问题。

少数民族问题相当严重。

想尽一切办法的问题,我很赞成。(柯庆施的一个大发明)


