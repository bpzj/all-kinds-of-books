\section[中共中央关于在职干部教育的决定(一九四二年二月二十八日)]{中共中央关于在职干部教育的决定}
\datesubtitle{(一九四二年二月二十八日)}


(一)在目前条件下,干部教育工作在全部教育工作中的比重,应该是第一位。而在职干部教育工作在全部干部教育工作中的比重,也应该是第一位的。这是因为一切工作包括国民教育工作在内,都须经过干部去做,“在政治方针决定之后,干部就是决定一切的因素”,如不把干部教育工作看得特别重要,把它敖在全部教育工作中的第一等地位,就产生要犯本末倒置的错误了。同时,着重的认真的办理干部学校,抽调许多干部进入各种干部学校,施以系统的教育,当然是很重要的任务,对此决不应该稍有忽视。但最大数量的干部,百分之九十以上的干部,还是在工作中,在人力财力与工作需要上,目前又不可能办理很多的干部学校;因此,对在职干部,就其工作岗位上,施以必需的与可能的教育,实在是全部干部教育工作中的第一位工作,应该引起党政军民各级领导机关及其宣传教育部门的充分注意。游击战争的特点,不但允许这样做,而且我们必须这样做。

(二)在职干部教育,自六中全会以来已经引起党内相当的注意,在许多地方与许多部门的在职干部中引起了学习的热潮,这是极好的现象。但忽视的现象还是存在着。在有些地方与有些部门中,甚至还没有开始,没有强调业务教育。而大多数在职干部要求学习业务与精通业务的热情则是很高的。政治教育虽一般地注意了,但或者不得其法,或者轻重不分,或者没有经常性。文化教育,是我党多数工农出身的干部所迫切需要的,但也没有引起党政军各级领导机关的充分注意。高级干部的理论教育,或者至今没有引起注意,或者脱离实际,成了教条主义的东西。而理论教育的成败则是革命成败的第一个关键。所有这些,都是必须改革或必须加强的。

(三)在职干部教育,应以业务教育、政治教育、文化教育、理论教育四种为范围。

(甲)对一切在职干部,都须给以业务教育,实行“做什么学什么”的口号。不论从事军事、政治、党务、文化、教育、宣传、组织、民运、锄奸、财政、经济、金融、送药、卫生及其他任何工作部门的干部,必须学会并精通自己的业务,这是第一个教育任务与学习任务。每一部门的领导机关及其负责人,必须指导所属干部,有秩序的进行学习,而各级党委各级政治部及其宣传教育部门则负总领导的责任。其学习范围,包括以下五项:第一、是关于与各部门业务密切关联的周围情况的调查研究。例如军事部门精细调查敌我友三方情况,加以分析研究,择其要点,编成教材,用以教育军事干部。其余类推。第二、是关于与各部门业务密切关联的政策、法令、指示、决定的研究。例如财政工作人员应熟悉财政政策与财政法令。锄奸工作人员应熟悉锄奸政策与锄奸法令。其余类推。第三、是关于各部门业务具体经验的研究。例如党的组织部门,研究党的组织工作与干部工作的经验,加以分类和综合,抽出要点,写成文件,教育所属干部。其余类推。第四,是关于各部门业务的历史知识。例如党的宣传部门,将我党二十年宣传鼓动工作及其政策的变化发展,加以叙述与总结,编成教材,教育宣传工作干部。其余类推。第五、是关于各部门业务的科学知识。例如军事干部研究军事学,医务干部研究医学等,每一部门均须研究自己的理论。对于上列各项业务学习,各部门领导机关负有供给材料指导学习及考查成绩的责任,务使所属干部从理论与实际两方面,逐渐达到学会与精通自己职务之目的。轻视学习业务与精通业务的观点是错误的。

(乙)对一切在职干部都须给以政治教育。其范围包括时事教育及一般政策教育二项。关于进行时事教育的办法,包括督促所属干部看报,对所属干部讲解时事问题及以地区或部门为单位,召集干部作时事报告等项。关于进行一般政策教育的办法,应为一切干部所应普遍学习的,例如将中央对时局宣言,中央关于增强党性决定,关于调查研究决定及边区施政纲领等,使干部阅读,加以解释或讨论等;或动员与本部门业务无直接关系,但有间接关系,有使所属干部加以研究之必要者,例如向军事指挥员解释中央关于土地政策的决定等。政治教育之目的,在于使干部除精通其专门业务局部情况与局部政策之外,还能通晓一般情况与一般政策,扩大眼界,避免偏畸、狭隘、不顾大局的弊病。必须指出,空谈一般政治而忽视专门业务的倾向,是不对的;但同样局限于专门业务而忽视一般政治的倾向,也是不对的。必须指出,一股情况与一般政策虽为一切干部所必习,但其分量轻重应依各部门性质而有所不同。例如对;于医生、技术专家,文学家、艺术家等,其分量应该减轻,对于党务工作人员,宣传工作人员,政府工作人员及军队中政治工作人员等,则其分量应该加重。关于政治教育缺乏经常性的毛病,党政军宣传教育部门应有计划地克服之。

(丙)对于一切文化程度太低或不高的干部,除业务教育与政治教育外,必须强调文化教育,反对轻视文化教育的错误观点。对于他们,学习文化提高文化水平,是他们全部学习的中心一环。其教育与学习范围,暂定为国文、历史、地理、算术、自然、社会、政治等课,宣传教育部门要负责解决课本问题。其教育与学习办法,在环境许可的地方,必须一律开办文化补习班或文化补习学校,或一机关独办,或数机关合办,或釆取轮训制轮流抽调干部集中一地学习,都是为好。在这些补习班或补习学校中,应有未任的教职员,辅之以兼任的教职员。在环境不许可的地方,则用小组学习制,以该机关某一文化程度较高的干部减少其日常工作,使他兼任教员,亦可专用教员。文化班或文化学校,可分为初级的及中级的两种。初级班为不识字及粗识文字的人而设,以学至大体相当于高小程度为合格;中级班为已有相当于高小程度的人而设,以学至大体相当于中学程度为合格。干部分班应以文化程度为标准,不以职位为标准。此外,某些从事宣传教育工作的干部,虽属知识分子,但尚有补习国文及文法之必要者,则用小组学习制或其他办法补习之。为着提高广大干部的文化水平,应在党政军机关内,提高文化教员的地位,最好的文化教员应受到极大的欢迎与优待。对办理文化教育有功的人员应受到奖励。

(丁)高级及中级干部之具有学习理论资格(文化程度、理解力与学习兴趣等)者,于业务学习与政治学习之外,均须学习理论。其学习范围,分为政治科学、思想科学、经济科学、历史科学等项,依次逐步学习之。其学习方法,以理论与实际联系与原则,例如政治科学以马列主义论战略策略等著述为理论材料,以我党二十年奋斗史为实际材料;思想科学以马克思主义的思想方法论为理论材料,以近百年中国的思想发展史为实际材料;经济科学以马克思主义的政治经济学为理论材料,以近百年中国的经济发展史为实际材料;历史科学则以研究外国革命史与中国革命史为主。其具体进行,应采取高级学习组与中级学习组的办法,以自学为主,加以集体的讨论与指导。

(四)四种教育的时间分配及课程分配,使之互相联系而不互相冲突与脱节,由党政军宣传教育部门负责调理之。

(五)不论任何工作部门,也不论业务教育、政治教育,文化教育、理论教育的任何方面,均须贯彻反对主观主义宗派主义与党八股的精神。一切材料均须由领导机关加以审查,任何包含主观主义、宗派主义与党八股毒素的东西,均须严格地加以清除或批判。

(六)在职干部教育是长期的,以发展其业务而不妨碍其业务并不妨碍干部健康为原则,在前方尤其不应妨碍战争。在情况许可的地方或部门,一律坚持每日两小时学习制。在情况不许可的地方或部门,学习时间可以伸缩。一切为着在职干部教育而耗费的时间,均算入正规工作时间之内,把教育与学习看作工作的一部分。在鉴定干部的时候,学习情况如何,应作为鉴定标准之一。

(七)实行对于在职干部教育的考试、测验与赏罚制度,其办法由中央宣传部订定之。

(八)各级党政军领导机关,应以极大之注意力放在干部教育(在职干部教育与干部学校教育)上面。为着干部而需用的人员(教员与职员)应加以严格的审查,并应首先调给之。各级领导人员有参加教课的责任。为着干部教育而需要用的经费,应最大量地供给之。

(九)对于从事干部教育的人员,尤其是教员,应加以教育,其办法由中央宣传部订定之。

<p align="right">(《整风文献》)</p>

