\section{湖南自修大学组织大纲(一九二三年四月十日)}


第一章宗旨及定名

第一条:本大学鉴于现在教育制度之缺点,釆取古代书院与现代学校二者之长,取自动的方法,研究各种学术,以期发明真理,造就人材,使文化普及于平民,学术因流于社会,由湖南船山学社创设,定名“湖南自修大学”。(因而招生只凭学力,不限资格;学习方法以自由研究,共同讨论为主。教师负提出问题,订正笔记,修改作文等责任。学生不收学费,寄宿者只收膳费。)

第二章校董会

第二条:本大学设“校董会”以校董十五名组织之,负本大学经费上之责任,并总持本大学大纲,由学社社员推举。枝董因必要,得扩充人数。

第三条:校董会得以校董在本大学所在地者经半数以上之列席开校董会。校董之不在本大学所在地者得随时陈述意见于校董会,并得托其它校董代表意见。

第四条:由校董会推举一人为“驻校校董”掌握本大学事务之大纲。

第五条:本大学校董会设名誉校董,额无定数,由枝董会推举有下列资格之人充任:

一、助款于本大学五千元以上之人;

二、赞助本大学有甚大劳绩之人;

三、国内外大学及专门学校校长。

第三章学员及办事员

第六条:本大学设学长一人,由校董会聘请,担任指导学友之自修,考察学友之成绩。但因必要得添聘专科指导员,辅助学长担任指导、考察之责。

第七条:本大学暂设办事员如左:

一、书记一人,受驻校校董之付托,办理本大学文书、交际及设备诸事物。

二、会计一人,受驻校校董之付托,掌管本大学经费之出纳及其使用诸事务。

三、图书馆主任一人,受驻校校董之付托,管理本大学图书馆。

四、实验室主任一人,受驻校校董之付托,管理本大学实验室。

第八条:本大学于前所举之办事员外,因必要,得加设相当的办事员,其职务及人数,临时酌定。

第四章 通讯员

第九条:本大学于国内国外重要之各大学,各专门学校及各学术团体设通讯员,担任本大学与各校各团体之连络;并注意各校各团体学术研究之情形及其重要之结果,报告于本大学。

第十条:本大学于国内国外学术昌明之区域(如北京、上海、广州、南京、武昌、巴黎、伦敦、柏林、纽约、莫斯利、东京等)设通讯员,担任本大学与各文化区域之连络,并注意各区域内一切重要文化学术之情况,报告于本大学。

第十一条:本大学于湖南省内(省城及各县)各重要之中等以上学校,并各种学术团体,设通讯员,担任本大学与各校各团体之间连络,并互相介绍其学术。

第十二条:通讯员于上列职务之外,得临时托以其它特别职务。

第五章学友

第十三条:凡中等以上学校毕业生,不分男女长少,具有自修能力,志愿用自修方法以研究高深学术者,经本大学证明认可,得报名入学。非中等以上学校毕业,而具有与之相等之学科根柢者,经本人证明认可,亦得入学。但本大学各种特别班学友,不必具有上列资格。

第十四条:本大学学友,分为下列二种:

甲、住校者,为住校校友;

乙,不住校者,为校外校友。

凡住校校友入学,必须经过试验,并自付膳费及杂费,住校学友及校外校友,均享有本大学图书、函授、听讲、集会等研究上之便利,凡学友须于每星期始,交纳相当之保证金,于每星期未退还,学友应否交纳学费,及学费交纳多寡临时依所研究之情形规定。

第十五条:本大学学友名额,由本大学斟酌房屋及图书馆设备情形临时分别规定。

第十六条:本大学学友,有发现下列各项事情之一者,得随时令其退学。

一、无自修能力,对于所认定之学科不能尽心研究,无成绩之表示;

二、假名自修,分心校外不正当之事务;

三、不能自治,无向上之要求;

四、妨害公众秩序,破坏其它学友及本大学名誉。

第十七条:本大学暂设文、法两科,其科目如左:

文科:中国文学、西洋文学、英文、论理学、心理学、教育学,社会学、历史学、地理学、新闻学、哲学。

法科:法律学、政治学、经济学。

以上各科学科之详细类别,另以表规定之,学友于以上各种学科中至少须选修一种。

第六章研究

第十八条:本大学研究之范围,分为下列三类:

甲:科学;

乙:哲学;

丙:文学。

甲、乙、丙三类之详细科目,另行规定。本大学学友,于甲、乙、丙三类各科目中,至少须选修×科目。

第十九条:本大学研究之方法,分为下列二种:

甲:单独研究,学友各自制定课程表,对于所选定之学科单独修习。

乙:团体研究,组织各种研究会,关于科学者,如“数学研究会”、“心理学研究会”、“经济学研究会”等;关于哲学者,如“周秦诸子学研究会”、“印度哲学研究会”、“罗素哲学研究会”等;关于文学者,如“中国文学研究会”、“英国文学研究会”、“诗歌、小说、或戏剧研究会”等;由本大学学友因志愿相同。分别组织,定期研究完毕。在研究期间,随时开会辩论商榷,各研究会之组织另定。

第二十条:

本大学为图研究之便利起见,特设补助方法三项如左:

甲:通函指导,由本大学聘请国立大学教授及其他国内外学者,担任定期通函指导,其指导之项目如左:

一、开示书目;二、指示研究方法;三、解释疑问。

乙:特别教授,如外国文一科。本大学得酌量情形,开设“英文”、“法文”,“俄文”等各专班。此外如“工人夜校”等各种平民教育,亦得附设。其组织另定。

丙:特别研究,本大学特别讲座,随时还请名人担任讲师到校讲学。

第二十一条:本大学为沟通文化,介绍新知,设各种“译书会”,如“法文译书会”、“英文译书会”等。

第七章劳动

第二十二条,本大学学友为破除文弱之习惯,图脑力与体力之平均发展,并求知识与劳力两阶级之接近,应注意劳动,本大学为达劳动之目的,应相有当之设备如,园艺、印刷、铁工等。

第八章图书馆及实验室

第二十三条:本大学于校内建图书馆,研究参考之用。本大学于校内建设博物、设置中外各种重要之图书、杂志及新闻纸,以为学友理化实验室,以为学友实地试验之用。

第二十四条:本大学于校内建设博物、理化实验室,以为学友实地试验之用。

第九章成绩表示

第二十五条:本大学学友成绩之表示,分为下列三种:

甲、平时所作之记录及论文;

乙、每学期届满时所做之论文;

丙、每一科目修习完毕时所做之论文。

第二十六条:本大学学友每人修学年限无定,以修习一科完毕为单位,成绩及格者,给与某科目修学证书。(如数学修学完毕成绩及格,即给与数学之修学证书)但不要证书者听。学友经认可得长期在本大学内研究。

第十章经费

第二十七条:本大学经费之大部系用为建设并充实图书馆及实验室之用,其来源如左:

甲:学社津贴;

乙:公私捐助。

第十一章校舍

第二十八条:本大学以湖南船山学社房屋为校舍(长沙小吴门正街)。将来因必要得扩充房屋或另行建筑。

第十二章分院及海外部

第二十九条:本大学因学友研究之方便于湖南各重要地点设本大学分院。即以长沙本校为第一院,第二院以下俟经费充足时开办。

第三十条:本大学之学友,因必要得设海外部。

第十三章自治规约及本大纲修改

第三十一条:本大学内部生活,以“自治规约”规定之,此项“自治规约”如驻校校董、学长、办事员与学友共同订定。

第三十二条:本大纲之修改,须得校董会之同意。

