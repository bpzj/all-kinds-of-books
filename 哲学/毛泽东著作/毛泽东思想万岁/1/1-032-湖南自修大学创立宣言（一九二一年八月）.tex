\section{湖南自修大学创立宣言(一九二一年八月)}


人是不能不求学的,求学是要有一块地方并且要有一种组织的。从前求学的地方在书院,书院废而为学校。世人便争毁书院,争誉学校。其实书院和学校各有其可毁,也各有其可誉。所谓书院可毁,在它研究的内容不对。书院研究的内容,就是“八股”等等干禄之具,这些只是一种玩物,哪能算得上正当的学问;就这一点论,我们可以说书院不对得很,但是书院也尽有好处。要晓得书院的好处,先要晓得学校的坏处。原来学校的好处很多,但坏处也就不少。学校的第一坏处,是师生间没有感情,先生抱一个金钱主义,学生抱一个文凭主义,“交易而退,各得其所”,什么施教受教,一种商行罢了!学校的第二个坏处,是用一种划一的机械的教授法和管理法去戕贼人性。人的资性各不相同,高才低能,悟解迥别,学校则全不管究这些,只晓得用一种同样的东西去灌给你吃。人类为尊重“人格”,不应该说谁“管理”谁,学校乃袭专制皇帝的余威,蔑视学生的人格,公然将学生“管理”起来。只有划一的教授而学生无完全的人性。只有机械的管理,而学生无完全的人格。这是学校的最大缺点,有人教育的人所万不能忽视的。学校的第三个坏处,是钟点过多,课程过繁。终日埋头于上课,几不知上课以外还有天地,学生往往神昏意怠,全不能用他们的心思为自动自发的研究。总括这些坏处,固然不能概括一切的学校说他们尽是这样,并且缺点所在,将来总还有改良的希望。但大体却是这样,欲想要替他隐讳也无从隐讳。他坏的总根,在使学生出于被动,消磨个性,灭掉性灵,庸儒的随俗沉浮,高才的相与裹足。回看书院,形式上的坏处虽然也有,但上面所举学校的坏,则都没有。一来是师生的感情甚笃。二来,没有教授管理,但为精神往来,自由研究。三来,课程简,而研讨周,可以优游暇豫,玩索有得。故从“研究的形式”一点说,书院此学校实在优胜得多。但是现代学校有一项特长,就是他“研究的内容”专用科学,或把科学的方法去研究哲学和文学,这一点则是书院所不及学校的。自修大学之所以为一种新制,就是取古代书院的形式,纳入现代学校的内容,而为适合人性便利研究的一种特别组织。

以上是说书院和学校各有利弊,自修大学乃取其利去其弊。现在再说自修大学独有的利,而书院和学校则为共有的弊;就是平民主义和非平民主义。书院和官式大学均有极严峻的程限,不及程限的不能入学,固不待言;实及了而偶不及,即有本为优才经入学考试而见遗的,便从此绝其向学的路。现在确实有好些有志青年,没有得到所求学的机会,实在可叹可惜!是一。书院和官式大学,将学术看得太神秘了,认为只有少数特殊人可渺求学,多数平民则为天然的不能参与。从此学术为少数“学阀”所专,与平民社会隔离愈远,酿成一种知识阶级奴使平民阶级的怪剧,是二。书院非赤贫的人所能入,官式大学更非阔家不行,欲在官式大学里毕一个业,非千余元乃至两千元不可,无钱的人之于大学,乃真“野猫子想吃天鹅肉”了,是三。自修大学力矫这些弊病。一则除住校学生,因房屋关系须稍示限制外,校外学生则诸凡有志向学以上均可入学。二则看学问如粗茶淡饭,肚子饿了,拿来就吃,打破学术秘密,务使公开,每人都可取得一份。三则自修大学在现在这“金钱就是生命”的时代,固不能使“无产阶级”的人,人人都有机会得到一份高深学问,但心里则务必使他趋向“不须多钱可以求学的”路上去。自修大学的学生可以到校里来研究,也可以就在自己的家里研究,也可以就在各种店馆里,团体里,和公事的机关里研究。比较官式大学便利得多,花费也就自然少了。

自修大学为一种平民主义的大学,既如上说。那么自修大学的内容怎么样呢?现在说一点大略如下:第一,自修大学学生研究学问的主脑是“自己看书,自己思索”。自修大学里面的“图书馆”就是专为这一项用的。第二,自修大学学生,于自己看书,自己思索之外,又有“共同讨论共同研究”。各种研究会的组织,就是专为这一顶用的。第三,自修大学虽然不要灌注食物式的教员,但也要有随时指导的人作学生自修的补助。第四,自修大学以学科为单位,学生研究一科也可,研究数科也可;每科研究的时间和范围,都听学生依自己的志愿和程度去学。第五,自修大学学生不但修学,还要有向上的意思,养成健全的人格,煎涤不正的习惯,为革新社会的准备。

最后要说自修大学在湖南的必要了。诸君,湖南不是至今没有一个最高的学术机关吗?省立大学在最近期内之必无成理,和即使成了也不过是一个官式大学,这是大家都明白的。而住在这湘江流域,沅江流域,资江流域,澧江流域三千万湖南民族,他们精神的欲求和文化的冲动,将如何去表现出来,发挥出来呢?湖南人民尽管是峥嵘活泼如日升的,尽管是极有希望的,但便没有可以满足其精神的欲求而发挥其文化的冲动,湖南人到底有什么意义呢?说到这里,便觉得湖南人有一种很大的任务落在他们的肩膊上来了。

什么任务呢?就是自完成自发展自创造他们各个及全体特殊的个性和特殊的人格。湖南自修大学之设窃取此意。事实上虽不能和湖南人个个发生关系,精神上要必要使他成为一个湖南全社会公共的求学机关。虽不能说一定有很好的成绩,但努力向前,积以年月,相信总有一天会达到我们的目标的。

