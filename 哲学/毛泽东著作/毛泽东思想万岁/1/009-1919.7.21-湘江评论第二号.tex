\section{湘江评论第二号}
\datesubtitle{(一九一九年七月二十一)}



\subsection{西方人事述评}

德意志人沉痛的签约

签约之前:败而不屈的德意志的代表兰超等于(五月初旬)到巴黎。(五月七日)在凡尔赛宫举行很庄严的和约交付礼。德代表的态度很倨傲。克勒满沙站起声述其开会词。德总代表兰超,则坐诵其如下之演说词——

德国军事破裂,德国失败的程度,自己明白。但这回欧战,负责的不仅德国,全欧都与有罪。因五十年以来,欧洲各国的帝国主义,实贻毒于国际局势。德国战中的罪行,固不可讳,战事的时候,人民的天良,为感情所蔽,故有罪行。然自去年十一月十一日以后,德国没与战事的人,多死于封锁的影响,协约国亦冷淡视之。威总统十四条大纲,为全世界所赞助,协约国业已声明依照此项大纲而立和约,那么,德国当不致于全没救护。国际同盟,各国都让加入,不能将德国丢在外边。德国愿以好意的精神,研究和约。……

和约为灰黄色封面一大册,和会秘书长杜斯玛,捧了交到兰超手中。兰超回到寓舍,晚餐的时候,默无一言。晚餐毕,即使人翻译和约,于晨间三点译成,送到兰超寝室。兰超看到天明方毕。另外录出几份,派专差送到柏林。八日德内阁会议许见。

内阁总理施特满,向考虑协约委员会演说——

和约条件,简直是宣告德国死刑。政府必以政治的沉静态度,讨论这可厌而狂妄的公文,……

随将和约条件,电告各联邦政府,请他们表示意见。因为感受很深的痛苦,特命公众停止行乐一星期。仅许剧院演唱和这痛苦极没相同的悲刷。股票交易所,因感受痛苦的印象,停闭三日。各界人士听得和约会要签字,皆为作怒,群相讨论拒绝签字的后患,甚至没有一人想劫或可受纳此项条件的。柏林各报一致(言尔)诋,有的说,“和约的苛刻远过最消极的预料,这系狂暴无知识的制品,若不能修改,只有用‘否’字答他。”有的说“我们如签定这约,实是屈于武力。我们的心中,应坚决拒绝。”惟独立社会党的机关报,则主张签约说,“从经验看来,拒绝徒增后患。”这时候最可注意的,是德国政党的态度。多数时会党的政府派,是不主张签字的。民治党和中央党也是这样。只有独立社会党不然。(十二日)独立社会党通过决议案,主张接受和约,并说“德现政府恢复显武主义的行为,使别人坚其对德的疑惧。德国舍屈于强迫签字,没有办法。俄德和约,及德罗和约,均没多久的寿命就取消了。凡尔赛和约,也未尝不可以革命的发展取消他。”我们为德国计,要想不受和约,惟有步俄国和匈牙利的后尘,实行社会的大革命。协约国最怕的就在这一点。俄罗斯,匈牙利,不派代表,不提和议,明目张胆地对抗协约国,协约国至今未如之何。向使去冬德国广义派社会党的社会革命成了功,则东联俄而南结奥,更联合匈牙利和捷克,广播其世界革命主义,或竟使英法美久郁的社会党,起而响应,协约园国府还食得下咽吗?独立社会党和广义派社会党,本是一党而分为二,他的议论如此,本不足怪。用革命的发展取消和约这话正不要轻看呢。

同日德因会讨论媾和条件。施特满演说――

今日为德国人生死关头!我们必须团结一致!我们除谋使国家生存,无旁的责任!德国不图进行其国家主义的梦想,并没争权的问题。于今人人的喉咙中间都觉有手塞住他的呼吸!人类的尊严,现付于诸君的手中,以保持他!

(十四日)兰超致克勒满沙一个履文,内容的大要如下——

和约中关于领土的条款,是使德国失去其×关重要的生产土地。各地薯芋的收成,将减百分之二十一。煤减三分之一。铁减四分之三。锌减五分之三。德国既因失去殖民地和商船,使经济成了麻木不仁。今又不能得充分的原料,势将被毁到极大的程度。同时输入的粮食必将大减。依赖航务和商业为生的数百万人,德国政府不能将工事和粮食供给他们则势不得不到国外求生,而重要的国家。多禁止德国移民入境。故签定和约,不啻向数百万德人宣告死刑……

兰超于上述的牒文之外,更以牒文两迈致克勒满沙。第一通的大要说,“协约国占德土地,和威总统宣布的主义不合。”第二通的乃关于赔偿条款提出抗议。谓“德国愿赔偿,但不是因为负了战争的原故。”我们看德国的抗议,大家注重(一)不独负战争责任。(二)不愿失去原料所从出的土地。其他各项,虽有抗议,但不是最重要的处所。

(十三日)晚上,柏林有大举示威。多数社会于示威时,起坛演说谓,“和约条件,较罗马施于加萨臣的,尤为刻毒而可羞。”群众游行各街,止于协约委员团所住的亚特伦旅馆前面。有人向众演说,其势汹汹,欲攻旅馆,为警察所阻。到内阁总理屋前,施特满氏临窗演说。又有人民一大队,于薄暮时候,唱歌到亚特伦旅馆,大呼“推翻强暴的和局。”“克勒满沙皆亡”。“与英伦皆亡。”众又到施特满处,请他演说,施氏讲到威尔逊总统十四公纲时,众忽大呼“与威尔逊皆亡。”这日柏林和乡间独立社会党开会,有四十处。

(十九日)柏林某报载有社会党领袖彭斯坦博士的演说,谓“非常苛刻的和约条件,非完全出于激怒与仇念。实德国政策既不能见信于人,则当然受此待遇。一切破坏咎在德国。德国之履行各项要求,不过补偿他们前所寄予人的而已。我很不以一般人士所发激烈的演说为然。告诉他们!不可再具一九一四年八月四日的气焰!”这于德国热烈的反对声中,算是一瓢凉水。

(二十日)兰超致书和会,要求改正审查及讨论和约的期限,(二十二日)克勒满沙复书,允许展长期限,至五月二十九日为止。(二十三日)晚德全权大使起程往斯巴,将和来自柏林的阁员数人会晤,决定一切。(二十四日)斯希特芒,欧士白格,从柏林乘车到斯巴,兰超及委员十六人也到,即开一极长的会议,斯希特芒主席,通过德国的反提案。会毕,政府委员回柏林,兰超等回凡尔赛。

(五月二十七日)。德国有答案交付和会,答案的第一部分,要点如下一一

(一)德国承认减少军队到十万人。

(二)交出巨大的军舰,而保留商船。

(三)反对关于东边土地的决定。要求于东普鲁士区中,举行庶民大会。

(四)承认丹齐为自由口岸。

(五)要求协约国,在签字四个月后,撤退军队。

(六)要求加入国际同盟。

(七)坚欲取得代替殖民地的权利。

(八)赔偿总数,不得过十万兆马克。

(九)拒绝引渡凯撒及其他人物。

(十)德国须有重新经商海外的权利。

德国答案的第二部分,亦有如下的要点――

(一)过渡时代,须维持大军,以保治安。

(二)须许德人开公民大会,讨论土地割让问题。并许奥人以加入德国的便利。


(三)拒绝割让西里细亚上部。

(四)不承认俄国有享取赔偿的权利。

(五)无赔偿意、门、罗、波兰等国的义务。

右之德国答案,四大国代表为长久讨论后,提出答复文,将德国所约议的,逐件驳复。全文很长。不外说德国战争责任万难推诿,德国必须尽其能力赔偿损失,必须交出戎首,和战时行暴的人,用法惩治。必须于数年内受特别的约束。凡协约国所持以构成和约的根本主义,万难更易。惟对于德国的实际建议,可以让步云云。

(五月二十九日以后)又因德代表的请求,屡次展缓签约日期。最后允展至六月二十八日为止。六月前半月的光阴,全为着往复反议事占去。

至(六月十八日)德代表团,乃由法京回德,一致告诫德内阁,拒绝签约。德内阁乃准备在韦马召集国会,议此次重大问题。此时协约方面,早做军事准备。一俟德国有不签字的表示,即行进军。德国已处于不能不签字的情势了。

(六月二十二日),协约国对德的“最后复文”于这日送达德代表,限德国以五日承受和约。“最后复文”内,述可以让步的条件,如下一一

(一)西里细亚上部,实行民众投票。

(二)西普鲁士边界,重行划定。

(三)德军暂增至二十万人。

(四)德国宣布愿于一个月内将被控破坏战时法律的人名单开出。

(五)修改关于财政问题的细则。

(六)以德国履行义务为条件,保证德国为将来国际同盟的一员。

施特满内阁知和约不能再有挽回,遂决计引退。

(二十二日)德新内阁组织成立。国务总理巴安氏,外交穆勒氏,财政欧士白格氏,内务达维特氏,陆军拿斯奇氏,殖民贝尔氏,邮电格莱斯勃资氏,劳动森士南氏,工程斯利奇氏,公业经济惠塞尔氏,国库夏勒氏,粕食斯密氏。巴安氏及诸阁员,多属多数社会党,本系前内阁的同调,在这回外交紧急声中,出当此签定和约的难局。新内阁既成,已可决其是预备签约的了。施特满退职。施特满内阁所委任的媾和代表,当然随着退职,于是德国讲和代表团易人。新代表团即以新内阁中的外交总长穆勒,邮电总长格莱斯勒资等组织而成。

此时国会业已在韦玛召集,巴安氏即赴国会,作很沉痛的演说,极言加入新政府的痛苦。恳请国会确立主张,否则战事将屡发作。巴安氏曰,“我特于自由的日耳曼最后一次,起抗此强暴破坏的和约!起抗此自决权利的假面具!起抗此奴隶德人的手段!起抗此妨害世界和平的新器!”国会乃于“反对”“赞成”的喧哗声中通过签约动议。

签约的动议通过,二十三日巴安再赴国会,申述无条件签约的必要,其演词谓,“战败的国家,身魂受世界的凌辱!吾人姑且签定和约。吾人一息尚存,终望损害吾人荣誉的人,有一日身受报应,”这时候右党提出抗议。乃付表决。结果证实准许签约。议长贵里巴贽氏起立发短节的演说,“以不幸的祖国,委托于慈悲的上帝!”且谓“各党领导,允宣告军界,全国希望海陆军树先己牺牲的模范,辅助劳工,重造祖国!”关系全世界安危的德国签字,在一场非常惨痛的演说声中,完全决定。德意志人的大纪念,有史以来,当没有过于这日了:

签字案既经国会通过,德新代表团乃到巴黎,致“允可签约的牒文”于和会。牒文的大要说,一一日耳曼民国政府,知协约国决计以武力强迫日耳曼承受和约条件。此项条件,虽没有重大的意味,然实老在剥夺日耳曼人民的荣誉。日耳曼政府虽屈服于作势的武力,但关于从古未闻离背公道的和约条件。”

右文既布,各国的欢忭,自不可言。至(二十八日)而最后展缓的满期已到。于是凡尔赛宫中,仍有亘古未闻的大签约一举。

签约之际一千九百十九年六月二十八日午后三点五分,凡尔赛宫中开会。在宫中设高坛,甚为庄严。协约国全权代表首先会集。次为德国全权代表,只到外总长穆劝和交通总长裴尔,其余均不愿到。克勒满沙主席,首发短简宣言,谓:“协约国和共同作战国政府,均赞成媾和条件。今加签字,表示彼等忠诚依守庄严的了解。”继乃“请日耳曼民国代表首先签字。”德代表所坐席次忽发大声,“德意志!”“德意志!”克勒满沙于是乃改称“德意志”。德代表即起立在约上签名,裴尔氏首先签之。时为午后三时十二分,园中喷泉四射,炮声大作,当德代表回到坐处的时候,会场皆露喜容。次为美国签字。次为英国签字。次为法国签字。次为意国签字。次为日本签字。最后签字的为捷克斯拉夫民国。三时三十五分签字完毕。克勒满沙氏宣布散会。

签约之后,当德国国会允许签约的消息传布,德国全国自即有爱国的示威运动。群众列队唱战歌,国歌,欢呼致敬于年老的统兵员,各报对于裁判德皇问题,表示极大的忿怒。有一报恳请一九一四年的陆军军官,表示如德皇受裁判,也愿受协约国的裁判。并请组织团体,或须入荷兰,保护德皇。各地暴动罢工事情,接续而起。及(二十八日)和约签字的消息传到柏林,柏林某报即载出一文,谓“德人终必报一九一九年的耻辱!”为政府禁止发行。(二十九日)各报皆有“黑线”,表示哀痛。各报皆载有极悲观的评论。柏林及各地铁路工人及电车工人罢工。柏林城里的运输机关全停。亨堡等处出了乱子。全国的罢工,有扩张形势。

评论我叙签约。我争叙感国的签约。我叙德国签约,单注重其国民精神上所感痛苦的一点。是什么意思?原来这回和约,除却国际同盟,全是对付德国的。德国为日耳曼民族,在历史上早蜚声誉,有一种崛强的。一朝决裂,新剑发硎,几乎要使全地球的人类挡他不住。我们莫将德国穷兵黩武,看作是德皇一个人的发动。德皇乃德国民族的结晶。有德国民族,乃有德皇。德国民族,晚近为尼釆,菲希特,颉德,泡尔生等“向上的”“活动的”哲学所陶铸。声宏实大,待机而发。至于今日,他们还说是没有打败,“非战之弊”。德国的民族,为世界最富于“高”的精神的民族。惟“高”的精神,最能排倒一切困苦,而惟我实现其所谓“高”。我们对于德皇,一面恨他的穷兵黩武,滥用强权。一而仍不免要向他洒一掬同情的热泪,就是为着他“高”的精神的感动。德国的民族,他们败了就止了。象这样的屈辱条件,他们也忍苦承受。他们第一次翻转面目,已从帝国变成了民国。他们的第二次翻转,或竟将民国都不要了。这话我殊敢下一个粗疏的断定。我们且看挡在西方的英法,不是他们的仇敌吗?英法是他们的仇敌,他们的好友,不就是屏障东方和南方的俄、奥、匈、捷和波兰吗?他们不向俄奥匈捷等国连络,还向何处?他们要向俄奥匈捷连络,必要改从和俄奥匈捷相同的制度。俄匈的社会革命成了功,不用说。奥捷也有此趋势,前日电传说捷克已经成了劳兵民国了。德国广义派斯巴达团,去年冬天的猛断举动,和成功仅仅相差一间。爱倍尔政府成立,多数社会党握权,所恃以制服广义派的,全是几个兵,几杆枪。和约成功,兵是要解散了。枪是会要缴出了。那时候政府还恃着什么?德国工商业的大毁败,要重造起来,不得不仰赖出力的劳动者。以后政府所应做的大事,就是向劳动者多多的磕头。而广义派的武器,不是别的,就是这些劳动者。故我从外交方面的趋势去考虑,断定德国必和俄奥匈连合,而变为共产主义的共和国。又从内治方面的趋势去考虑,也可以做同样的断定。

一千九百一十九年以前,世界最高的强权在德国。一千九百一十九年以后,世界最高的强权在法国,英国和美国。德国的强权,为政治的强权,国际的强权。这回大战的结果,是用协约国政治和国际的强权,打倒德奥政治和国际的强权。一千九百一十九年以后,×国英国美国的强权,为社会的强权,经济的强权。一千九百一十九年以后设有战争,就是阶级战争,阶级战争的结果,就是东欧诸国主义的成功。即是社会党人的成功。我们不要轻看了以后的德人。我们不要重看了现在和会高视阔步的伟人先生们,他们不能旰食的日子快要到哩!他们总有一天会要头痛!然则这回的和约“其能稳”尚靠不定。若真以“德俄和约”,“德罗和约”的例来推测,恐咱就是早晚的问题。无知的克勒满沙老头子,还抱着那灰黄色的厚册,以为签了字在上面,就可当作阿尔卑斯山一样的稳固,可怜的很啊!

\subsection{世界杂评}

高兴和沉痛克勒满沙在办公室接得德国接受和约的电话,高兴了不得。起身来,和在办公室的阁员及同僚握手。说,“诸君!我之静候这一分钟,已有十九年了!”这话何等高兴。虽然,不第高兴,又含多少沉痛的意思。一千八百七十一年,威廉第一和俾士马克,高踞凡尔赛,接受法国屈服牒文的时候,何等高兴。结果遂酿成此次的战争。虽然威廉第一,俾士马克,不第高兴,又含有多少沉痛的意思。一千八百年至一千八百一十五年,拿破伦蹂躏德意志,分裂他的国,占据他的地,解散他的兵。普王屈服,称藩纳聘。拿破伦何等高兴。结果遂酿成一千八百七十一年的战争。虽然,拿破伦不第高兴,又含多少沉痛的意思。一千七百八十九年至一干七百九十年,德奥为巨擘的神圣同盟军,深恶德国的民主自由,几度蹂法境,围巴黎,结果遂崛起拿破伦,而有蹂躏德国,令德人头痛的事。我们执因果看历史,高兴和沉痛,常相关系,不可分开。一方的高兴到了极点,热一方的沉痛也必到极点。我们看这番和约所载,和拿破伦对待德同的办法,有什么不同?分裂德国的国,占据德国的地,解散德国的兵,有什么不同?克勒满沙高兴之极,即德国人沉痛之极。包管十年二十年后,你们法国人,又有一番大大的头痛,愿你们记取此言。

卡尔和溥仪奥前皇卡尔避居瑞士,某报通讯记者求见,见其侍者。侍眶说:“皇帝的退位,本非得已,故愿望恢复帝制。惟目下暂时隐居,不问政治。”凡做过皇帝的,没有不再想做皇帝。凡做过官的,没有不再想做官。心理上观念的习惯性,本来如此。西洋人做事,喜欢彻底,历史上处死国王的事颇多。英人之处死沙尔一世(一千六百四十八年),法人之处死路易十六(一千七百九十三年),俄人之处死尼哥拉斯第二(一千九百一十八年),都以为不这样不足以绝祸根。拿破伦被囚于圣赫利拿,今威廉第二拟请他做拿破伦的后身将受协约国的裁判,总算很便宜的。避居瑞士的卡尔,和伏处北京的溥仪,国民不加意防备,早晚还是一个祸根。

