\section[和卡博/巴庐库同志的谈话(一九六七年二月三日)]{和卡博/巴庐库同志的谈话}
\datesubtitle{(一九六七年二月三日)}


主席问:谢胡同志是什么时候来中国的?(答:去年五月。)当时我就曾说究竟是马列主义胜利,还是修正主义胜利?这是两条路线斗争的问题。我还说过。究竟哪一方面胜利,现在还看不出来,现在还不能作结论。有两种可能。修正主义打倒我们,有可能我们战胜修正主义。我为什么把失败放在第一可能呢?这样看问题有利,可以不轻视敌人。多年来,我们党内斗争是没有公开化的。一九六一年七千人大会,那时我讲了一篇话,我说修正主义要推翻我们。如果我们不斗争,少则几年,多则十几年和几十年,中国就可能变颜色。这篇讲话没有发表,不过那时已看出一些问题。六一年到六五年期间,为什么说我们有许多工作没有做好呢?说的不是客气话,说的是真话。我们过去只抓个别问题,个别人物,五三年冬到五四年斗了高、饶,五九年把彭德怀、黄克诚整下去了。此外,还搞了一些文化界及农村、工厂的斗争,即社会主义教育运动。你们也是知道的,但都没有解决问题,没有找出一种形式,一种方式公开的、全面的自下而上的揭发我们的黑暗面,所以这次要搞文化大革命。对文化大革命也进行了一些准备。一九六五年十一月对吴晗发表批判文章,在北京写不出,只好到上海找姚文元。这个摊子开始是江青她们搞的,当然事先也告诉过我。文章写好后交给我看。她还说:只许我看,不能给周恩来和康生看,不然刘、邓这些人也要看。刘、邓、彭、陆是反对这篇文章的,文章发表后全国转载了,北京不转载。(湖南也未转载,张平化作了检查——有人插话)那时我在上海,我说把文章印成小册子各省打印发行,就是在北京不打印发行,彭真通知出版社,不准翻印。北京市委是水也泼不进,针也插不进。现在不是改组了吗?还不行,还得改组。当发表改组市委时,我们增加了××个卫戍师,现在是×个卫戍师。以前×个师是好的,但太散了。现在红卫兵帮助我们,但也有不可靠的,有的戴黑眼镜、口罩,手里拿着棍子、刀到处乱捣,打人、杀人,杀死了人,杀伤了人。这些人多数是高干子弟。如贺龙、陆定一的女儿。所以军队也不是没有问题。六五年十二月解决了罗。六六年六月一日第一张马列主义大字报广播了,八月红卫兵出现发动了群众。去年聂元梓写的一张大字报,当时我在杭州,一天我看到这张大字报,我打电话给康生、陈伯达,要广播这张大字报,这样大字报就满天飞了。清华、北大两附中写了两件材料,我看了,八.一我写信给这两个学校的红卫兵,后来红卫兵大搞起来。八.一八我接见了几十万红卫兵,接着开了八届十一中全会,我写了一张二百多字的大字报,当时就从中央到地方,一些负责人反对学生运动,反对无产阶级专政,搞白色恐怖,这样才揭发了刘邓的问题,现在双方正在决战,还未解决,今年三、四月可能看出眉目,解决问题可能到明年三、四月份,也可能更长一些的间。好几年前,我就要洗刷几百万。那是空话,他们不听话嘛!毫无办法。人民日报夺了两次权,就是不听我的话,我去年就声明人民日报我不看,讲了好几次他就是不听。看来我这一套在中国不灵了,因为大中学校长期掌握在刘邓陆手里,我们进不去,毫无办法。

我们党内暴露出来的问题,可以分几部分人:

一部分是搞民主革命的,民主革命时期可以合作,打倒帝国主义、封建主义他是赞成的,打倒官僚资本主义他也是赞成的,打倒民族资产阶级他就不赞成了。把土地分给农民他是赞成的,合作化他就不赞成了,这一批有的是所谓老干部。

第二部分是解放后才进党的人,有百分之八十解放后才进党的,其中一部分当了干部,有的当了支部书记,县委书记。

第三部分是收留下来的国民党。这些人有的过去是共产党,以后叛变了,登报反共。那时不知道,现在查出来了,他们不拥护共产党,反对共产党。

第四部分是地、富、反、坏、右、资产阶级子弟,解放后他们进了大学,掌握了一部分权,不都是坏人,有的是站在我们这方面的,但有些是反共的。总之坏人在中国不多,大概也不过百分之几,如地、富、反、坏顶多百分之五,约三千五百万人。他们是分散的,分散到各农村,城市和街道。如果集中到一起,手中拿了武器,那就是一股大敌了。他们是灭亡了的阶级,其代表人物在三千五百万人中顶多不过几十万,也是分散的。所以大字报、群众运动、红卫兵一出来,他们就吓的要死。

大学生有很大一部分我是怀疑的,特别是文科。不搞文化革命他们就要变成修正主义分子,搞修正主义了。文科不能写文章,哲学不能解释社会现象,还有经济学,可多呢!现在看来有希望,斗得很厉害。

群众都发动起来了,什么坏东西都可以扔掉。巩固无产阶级专政,我们是乐观的。从去年,我和谢胡同志谈话时,比较乐观些了。

(卡博同志说:以毛主席为代表的革命路线取得了巨大胜利。)

现在只能讲取得了相当的胜利,到明年这个时候,可以这样说了,但是我们也许被敌人打败,打败就打败了嘛,总是有人革命的。有人说,中国爱好和平,那是吹牛,其实中国就是好斗,我就是一个。好斗,出修正主义就不那么容易了。

(卡博同志说:不搞斗争是不行的,不然革命怎么实现呢?)

就是吆!中国搞修正主义不像苏联那么容易,中国是半封建半殖民地国家,受压迫一百多年。我们的国家是军队打的,学校原封未动,党和政府的领导人有的是委派去的,如曹荻秋、陈丕显不是派去的吗?以后选举的。选举我是不相信的,中国有两千多个县,一个县选举两个就四千多,四个就一万多,哪有那么大的地方开会?那么多人怎么认识?我是北京选的,许多人就没有看见我么!见都没见怎么选呢?不过是闻名而已,我和总理都是闻名的。还不如红卫兵,他们的领导人还和他们讲过话呢:红卫兵也是不断分化的,夏季是革命的,冬季就成了反革命,夏季是少数,冬季就由少数变成了多数。“井冈山”、聂元梓受过压迫,很革命,去年十二月份到今年一月份分化了。但不管怎样,总是好人多。现在流行无政府主义,怀疑一切,打倒一切,结果弄到自己头上了,不行的。不过斗来斗去,错误的人总是站不住脚的。街上有打倒×××、×××的大字报,打倒×××、×××的大字报就更多了。×××是×××,管好几个部,××部要打倒他。打倒××是××军区司令部的提出来的,过几天自己就被打倒了,但是有条永远的真理,那就是绝大多数党、团员和人民是好的。

