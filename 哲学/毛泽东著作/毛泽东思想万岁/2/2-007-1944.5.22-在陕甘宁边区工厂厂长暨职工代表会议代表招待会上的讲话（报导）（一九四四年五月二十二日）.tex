\section[在陕甘宁边区工厂厂长暨职工代表会议代表招待会上的讲话(报导)(一九四四年五月二十二日)]{在陕甘宁边区工厂厂长暨职工代表会议代表招待会上的讲话(报导)}
\datesubtitle{(一九四四年五月二十二日)}


现在无论外国和中国都为了同一的目标而奋斗,那就是打倒法西斯。我们边区的工业建设,也和其它一切工作的目的一样,是为了打倒日本帝国主义,没有第二个目的。边区在五年前才真正开始有了一点工业。当时只有七百个产业工人,一九四二年有了四千,到了今年就有了一万二千个工人,所以边区工业的进步是很快的,它的数目虽小,但它所包含的意义却非常远大,谁要不认识这个最有发展、最富于生命力、足以引起一切变化的力量,谁的头脑就是混沌无知。这次开大会的目标,就是二年以内,要争取做到工业品的全面自给,首先是布的自给和铁的勾自给。假如我们做到了全部自给,我们工人的数目还要大大的增加。全体工程师、厂长、工人们都向这个方面努力,共产党员,非共产党员都向这方面努力,像沈鸿同志、陈振夏同志,他们不是共产党员,但是他们的心和共产党员一样。必要工业,要中国的民族独立有巩固保证,就必须工业化,我们共产党员是要努力于中国的工业化的。

中国落后的原因,主要的是没有新式工业。日本帝国主义为什么敢于这样的欺负中国,就是因为中国没有强大的工业,它欺负我们的落后,因此消灭这种落后,是我们全民族的任务。老百姓拥护共产党,是因为我们代表了民族和人民的要求,但是,如果我们不能解决经济问题,如果我们不能建立新工业,如果我们不能发展生产力,老百姓就不一定拥护我们。在抗日战争中间,共产党抗击了百分之五十八的敌军、百分之九十多的伪军,这方面我们是有经验有成绩的,但经济工作,尤其是工业,我们还不大懂,可是这一门又是决定一切的,是决定军事、政治、文化、思想、道德、宗教这一切东西的,是决定社会变化的。因此所有的共产党员都应该学习经济工作,其中许多人,应该学习工业技术。我们边区是个大学校,其中有一门课叫做工业,这次职工代表大会便是个工业的短期训练班。如果我们共产党员不关心工业,不关心经济,也不懂别的什么有益的工作,对这些一无所知,一无所能,只会作一种抽象的“革命工作”,这种革命家是毫无价值的。我们应该反对这种空头革命,学习使中国工业化的各种技术知识。

<p align="right">(一九四四年五月二十六日《解放日报》)</p>

