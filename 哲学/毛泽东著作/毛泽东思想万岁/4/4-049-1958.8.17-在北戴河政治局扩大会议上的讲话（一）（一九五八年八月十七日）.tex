\section[在北戴河政治局扩大会议上的讲话(一)(一九五八年八月十七日)]{在北戴河政治局扩大会议上的讲话(一)}
\datesubtitle{(一九五八年八月十七日)}


这次会议是政治局扩大会议,省和自治区的负责同志都参加。题目就是印发的这,同志们还可看题目。

重点是第一个问题,明年,五年经济计划问题,主要是工业,农业也有一点。发一个参考数字,不太公道,要搞公道一点,正确一点,搞三天,由富春同志负责。

第二个问题:今年铁、钢、铜、钼问题。钢由五七年的530万吨翻一翻,达到1100万吨,有完不成的危险,中心问题是搞铁。现在都打了电话,发动了,可是还要抓紧些,要回电话,要保证。

第三个问题:明年农业问题,由×××同志负责。

第四个问题:明年水利问题,由陈、李负责。

第五个问题:农业合作化问题,印了一份河南试办人民公社的简章。

第六个问题:今年商业收购和分配问题(包括今年粮食处理),由先念负责。粮食产量今年可能达到××××亿斤,每人××斤,明年每人争取达到××斤,后年××斤,是否搞到2500斤至3000斤,以后再议。是否可以无限制的发展粮食,我看超过3000斤就不好办了。

第七个问题:是教育问题。×××同志写了一篇文章,决议即可印发。

第八个问题:是干部参加劳动问题。包括我们在座的,不论作什么官,不论官大官小,凡能参加劳动的都要参加,太老的和太弱的除外。我们做官的有几百万,加上军队有一千几百万,究竟有多少官也搞不清楚。干部子弟有几千万,近水楼台容易做官,官做久了容易脱离实际,脱离群众。十三陵水库修成了,许多人都去修水库劳动了几天。是否每年劳动一个月,一年四季分配一下,工、农、商都可以,把劳动和工作结合起来,一切人都如此。人家劳动,作官的不劳动怎么行?还有那么多干部子弟。苏联农业大学的毕业生不愿下乡。农业大学办在城里不是见鬼吗?!农业大学要统统搬到乡下去。一切学校都要办工厂,天津音乐学院还办几个工厂,很好。参加劳动,县乡级好办,中央、省、专级难办,开机器怕不行!能用筷子吃饭用毛笔写字的人,难道不能开机器?开机器容易,还是爬山容易?

第九个问题:是劳动制度问题,由劳动部准备。

第十个问题:是××万人去边疆问题。

第十一个问题:是技术保密问题。

第十二个问题:是国际形势问题。这个问题是我出的,因为到处有人问会不会打世界大战?打起来怎么办?西方国家军事集团究竟是什么性质?紧张局势对谁有利?联合国承认有利还是不承认有利?到底谁怕谁?谁怕谁多一点?这个问题在党内也不是完全解决了。有人说:东风压倒西风,可见未压倒,否则美英在中东怎敢登陆?这个问题看法不一致,党内党外都有怕西方情绪,有恐美病。谁怕谁多一点,恐怕是西方怕我们多一点。世界上有三个主义:共产主义、帝国主义、民族主义。后两者都是资本主义。一派是民族资本主义,一派是压迫别人的资本主义一一即帝国主义。民族主义原来是帝国主义的后方,可是它一反帝,就变成我们的后方。印埃两国都搞,但是比较对我们有利。我们两个主义站在一起,力量就大了,原子弹双方都有,人民力量我们大,因此不会打。但是也可能打,我们要准备打。垄断资本也难说,假如他要打,是怕打好还是不怕打好。横起一条心,对敌人用黑心,拼命的打,打烂再建设。讲清楚不怕打是好的。

对帝国主义的三个集团,我们在宣传中说它是侵略者,因为它向民族主义、社会主义侵略。但是不要看得了不起,它只在一种情况下向我们进攻,即我们出了大乱子,反革命把我们推翻。匈牙利的反革命已被镇压下去了,他们不敢来了。社会主义阵营在巩固中,我们中国有七、八千万吨钢就巩固了。帝国主义那些条约,与其说是进攻的,不如说是防御的,是患了肺病的钙化组织,不要把它看得太严重了。巴格达条约搞了一个洞,中心突破,伊拉克一天早晨就变了。共产主义思想可以渗透,我很欣赏×××说的他们怕我们穿过去,帝国主义的军事集团是薄板墙,立在不巩固的基础上,是整中间地带的,他们没有机会整我们就整中间地带,并且互相整,英美整法国,又限制西德。我们宣传反对紧张局势,争取缓和,好像缓和对我们有利,紧张对他们有利,可否这样看,紧张对我们比较有利,对西方比较不利。紧张对西方有利的是能够扩大军火生产,对我们有利的是能够调动一切积极因素。七月十四日早晨,伊拉克的盖子揭开了。紧张可以使各国共产党增加几个党员,可以使我们多增加一些钢铁、粮食。美英在黎巴嫩、约旦晚走一些日子好,不要使美国变成好人,多呆一天就多有一天好处,抓住了美国的辫子,有文章好做,美帝成了众矢之的,但宣传上不能这样讲,还是讲立即撤退。

禁运越禁越好,联合国越不承认越好。我们有经验,抗日战争时期,蒋介石、何应钦不发供给、不给钱,我们提出团结自给,发展大生产,搞出的价值不只四十万元,棉衣也穿上了,比何应钦给的多得多。那时如此,因此现在各国禁运也有利。最好再过七年再承认。七年计划分三阶段,苦战三年、二年、二年。那时我们可以搞××到××万吨钢,面前有一个敌人,紧张对我们有利。

第十三个问题:是今冬明春农村共产主义教育问题。

第十四个问题:是协作问题。

第十五个问题:是深耕问题。目前农业的主要方向是深耕问题。深耕是个大水库,大肥料库,否则水、肥再多也不行。北方要深耕一尺多,南方要深耕七、八寸,分层施肥使土壤团粒机构增多,每个团粒又是一个小水库,小肥料库。深翻使地上水与地下水接起来。密植的基础是深耕,否则密植也无用。深耕有利于除草,把根挖掉又有利于除虫,这样一来可以一亩当三亩,现在全国每人平均三亩地。我们向下边跑,就可高产。种那么多地干什么?将来可以拿三分之一的土地种树,然后过几年再缩一亩。过去平原绿化不起来,到那时就能绿化了。如不深耕就无这种可能。

人口的观念要改变,过去我说搞八亿,现在看来搞十几亿人口也不要紧。对多子女的人不要提倡。文化水平提高以后就真正节育了。

第十六个问题:是肥料问题。

第十七个问题:是民兵问题,协作区或较大的省可以生产轻武器,如步枪、机枪、轻炮等,武装民兵,搞大合作社,工农商学兵一套都有。造那样多枪可能是浪费,因为我们不打仗,浪费点也要搞。全民皆兵,有壮气壮胆的作用。多唱穆桂英、花木兰、泗洲城,少唱祝英台。再用六年时间四人发一枝枪,全国共需一亿枝枪,每人发几十发子弹,必须打光。


