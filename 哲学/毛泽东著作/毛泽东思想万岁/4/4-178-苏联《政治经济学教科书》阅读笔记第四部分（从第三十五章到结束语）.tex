\section[苏联《政治经济学教科书》阅读笔记第四部分(从第三十五章到结束语)]{苏联《政治经济学教科书》阅读笔记第四部分(从第三十五章到结束语)}


五十六、关于共产主义的国家问题

639页上说;“在共产主义高级阶段……国家将变成不需要的东西而逐渐消亡。”但是国家的消亡还需要一个国际条件。人家有国家机器,你没有,这是危险的。639—640页上说,即使到共产主义建成后,只要帝国主义国家还存在,国家还是必要的,这个提法对。紧接着书上又说;“但是,国家的性质和形式将取决于共产主义制度的特点。”这句话不好懂。国家的性质是压迫敌对势力的机器,国内即使没有需要压迫的敌对势力,对于国外的敌对势力,国家压迫的性质也还没有变。所谓国家的形式不外军队、监狱、捉人、杀人等等,只要帝国主义还存在,国家的这些形式到共产主义又有什么不同呢7

五十七、关于向共产主义过渡

641页上说;“在社会主义社会没有敌对的阶级”,但是还有敌对阶级的残余。“从社会主义向共产主义过渡是不需要通过社会革命来实现的”,只能说不需要进行一个阶级推翻另一个阶级的社会革命,但是还有新的生产关系代替旧的生产关系,新的社会制度代替旧的社会制度的社会革命。

书上在这里接着声明。“这并不是说,社会沿着通向共产主义道路发展就不要克服内部的矛盾。”不过是附带声明罢了。这本书虽然有些地方也承认矛盾。但不过是附带的提起。说明问题不从分析矛盾出发,这是这本书的一个缺点。当作一门科学,就应当从矛盾分析出发。

到了共产主义社会,因为生产高度自动化,要求人们的劳动和行动更准确,那时的劳动纪律会比现在更加严格。

现在我们说共产主义社会分两个阶段,即低级阶段和高级阶段。这是马克思他们根据当时社会发展的条件所预见到的,进到高级阶段以后。共产主义社会的发展.一定会出现新的阶段,新的目标,新的任务一定又会提出来。

五十八、集体所有制的发展前途

650页上说;“集体农庄合作社生产关系的形式完全符合农村目前生产力发展水平和需要。”究竟是不是这样呢7

有一篇苏联的文章,介绍了红十月集体农庄的情况,说“原来几个农庄不合并时很多事情不好办。合并以后,这些事情都好办了。”说现在一共一万人,计划在中心建设一个住三千人的居民点,这个材料可以说明现在集体农庄的形式已经不完全适合生产力的发展了。

书上这段说。“要求大力巩固和进一步发展国家(全民)所有制和合作社集体农庄所有制。”既然需要发展,要过渡,怎么能大力巩固呢?社会主义生产关系,社会制度要讲巩固。但不能讲得过火。书上讲了模糊的前途,但一讲到具体措施就不清楚了。从某些方面(主要是生产方面)看,他们没有停滞;但是在生产关系方面,可以说基本上停滞不前了。

书上说,要把集体所有制过渡到单一的共产主义所有制,但是在我们看来,首先必须把集体所有制变成社会主义全民所有制。所谓把集体所有制变为社会主义全民所有制,就是把农业生产资料统统变为国有,把农民统统变为工人由国家统一包起来,发给工资。现在全国农民每人平均每年的收入是85元,将来达到每人150元,而且大部分由社发给的时候,就可以实行基本社有,这样,再进一步变为国有,就好办了。

五十九、关于消灭城市和农村的差别

651页的前一段对农村建设的设想很好。

既然要消灭城市和农村的差别(书上说是“本质差别”),为什么又特别声明并不是“降低大城市的作用。”将来的城市可以不要那么大。要把大城市居民分散到农村去,建设许多小城市。在原子战争的条件下。这样也比较有利。

六十、关于社会主义各国建立经济体系问题

659页上说.“每个国家都可以集中自己的人力财力来发展在本国有最有利的自然条件和经济条件。有生产经验和干部的部门。而且个别国家可以不必生产能靠其他国家供应来满足需要的产品。”、

这个提法不好。我们甚至对各省都不这样提。我们提倡全面发展,不说每个省份不必生产能靠其他省份供应来满足需要的产品,我们要各省尽量发展各种生产,只要不妨碍全局。欧洲的好处之一,是各国独立,各搞一套,使欧洲经济发展较快。我国自秦以来,形成大国,在很长时间内,全国大体上保持统一局面。缺点之一是官僚主义控制太死,地方不能独立发展,大家拖拖拉拉,经济发展很慢。现在情况完全不同了,我们要做到全国是统一的。各省又是独立的,是相对的统一,又是相对的独立。

各省服从中央决议,接受中央控制,独立解决本省的问题。而中央重大问题的决议,又都是中央同各省商量共同做出的,例如庐山会议的决议就是如此。它既合乎全国的需要,也合乎各省的需要。能认为只有中央需要反对右倾机会主义,地方就不需要反对右倾机会主义了么?我们是在全国统一计划下,提倡各省尽量各搞一套。只要有原料,有销路,只要能够就地取材,就地推销。凡能办的事情,就都可以尽可能去办。以前是担心各省都发展了各种工业,像上海这样的城市工业品,会没有人要,现在看来并不是这样的。上海已经提出向高,大、精、尖发展的方针,它们还是有事情可做的。

为什么不提倡各国尽量搞,而提倡可以不必生产能靠其他国家供应来满足需要的产品呢?正确的办法应当是各国尽量搞,以自力更生为主,自己尽可能的独立的搞,以不依赖别人为原则,只有自己实在不能办的才不办。特别是农业应当尽可能的搞好,吃饭靠外国、外省危险得很。

有些国家很小,确实像书上所说的情形。“发展所有工业部门在经济上是不合理的,也是力所不能胜任的。”那当然不要勉强去搞。我们国内有些人口少的省,如青海、宁夏,现在也很难什么都搞。

六十一、社会主义各国发展能够“拉平”吗?

660页第三段.“使社会主义各国的经济和文化发展的总的水平逐渐拉平。”各国人口不同,资源不同,历史条件不同,革命有先进的和后进的区别,怎样拉得平呢?一个父亲生十个儿子,有的高,有的低,有的大,有的小,有的聪明些,有的愚蠢些,怎么能拉平呢?“拉平”是布哈林的均衡论,社会主义各国经济发展不平衡、一国之内的各省、一省之内的各县都不平衡,拿广东省的卫生来说,佛山市和歧乐社搞得好,因此佛山市和广州不平衡。歧乐社和韶关不平衡,反对不平衡是错误的。

六十二、根本问题是制度问题

668页说,社会主义国家贷款和帝国主义国家不同,这个叙述是符合事实的。社会主义国家总比资本主义国家好,我们要懂得这个原则,根本问题是制度问题,制度决定一个国家走什么方向,社会主义制度决定社会主义国家总是要同帝国主义国家相对立,妥协总是临时的。

六十三、关于两个世界经济体系之间的关系

671页上说。“两个世界体系的经济竞赛。”斯大林在《苏联社会主义经济问题》中提出了两个世界市场的论点,教科书在这里提出两个世界体系的和平经济竞赛,强调在两个世界体系之间“建立和发展”的经济关系,这是把实际存在的两个世界市场变成了在统一的世界市场中的两个经济体系,这是从斯大林观点的后退。

在两个经济体系之间,其实不只是竞赛,而且有激烈的广泛的斗争,教科书避开了这个斗争。

六十四、关于对斯大林的批评

680页上说斯大林的《苏联社会主义经济问题》这个著作上,正如斯大林的其他著作一样.有一些错误的原理。书中所指的两条罪状不足以服人。

一条罪状说斯大林抱着这样的观点。“商品流通似乎已经成为生产力发展的阻碍。逐渐过渡到工农业直接进行生产交换的必要性已经成熟。”

斯大林在那本书里说过,有两种所有制,就要有商品生产。他说。“在集体农庄的企业中、虽然生产资料(土地、机器)也属于国家,可是产品却是各个集体农庄的财产,因为集体农庄中的劳动也如种子一样,是他们自己所有的,而国家交给集体农庄永久使用的土地,事实上是集体农庄由当作自己的财产来支配的。”在这样的条件下,“集体农庄只愿把自己的产品当作商品让出来,愿意以这种商品换得他们所需要的商品。现时,除了经过商品的联系,除了通过买卖的交换以外,与城市的其它经济联系都是集体农庄所不接受的。”

斯大林批评了苏联当时主张取消商品生产的观点,认为当时商品生产同三十年前列宁宣布必须以全力扩展商品流通时一样,仍是必要的东西。

教科书说斯大林似乎主张立即消灭商品,这个罪状很难成立。至于产品交换问题,在斯大林只是一种没想,他并且说过,“推行这种制度,无须特别急忙,要随着城市制成品的积累的程度而定。”

另一条罪状是低估价值规律在生产领域中、特别是对生产资料的作用。“在社会主义生产领域中,价值法则不起调节作用。起调节作用的是有计划按比例发展的规律和国家计划经济。”教科书提出的这个论点,其实就是斯大林的论点,虽然教科书说,生产资料是商品,但是第一,不能不说在全民所有制范围内,生产资料的“买卖”并不改变所有权。第二。不能不承认价值规律在生产领域中和流通领域中所起的作用是不同的。这些论点同斯大林的论点在实际上是一致的。斯大林和赫鲁晓夫的一个真正区别是前者反对把拖拉机等生产资料卖给集体衣庄,而后者则把这些东西卖给集体衣庄。

六十五、对《教科书》总的看法

不能说这本书完全没有马列主义,因为书中许多观点是马列主义的。但是也不能说这本书完全是马列主义的,因为书中有许多观点是离开马列主义的。基本上否定这本书,还不能做这个结论。

书上强调社会主义经济是为全体人民服务的经济,不是为少数剥削者谋利的经济。书上说的社会主义基本经济规律,不能说完全是错误的,这本书基本观点说的就是这个,书上也说了有计划、按比例、高速度等等。就这些方面看,这本书还是社会主义的、马克思主义的。至于在承认有计划按比例之后,如何按此例,那是另外一个问题,各有各的办法。

但是这本书有些基本观点是错误的。书上不强调政治挂帅,群众路线,不讲两条腿走路。片面地强调个人物质利益,宣扬物质刺激,突出个人主义。这些都是错娱的.

对社会主义经济的研究,书上不是从矛盾出发,他们实际上是不承认矛盾的普遍性.不承认社会矛盾是社会发展的动力。事实上,他们的社会主义社会中还有阶级斗争,即社会主义和资本主义残余的斗争。但是他们不承认。他们的社会中还有三种所有制,即全民所有制、集体所有制和个人所有制。当然,这种个人所有制和集体化以前的个人所有制有所不同,那时农民的生活完全建立在个人所有制上,现在是脚踏两只船,主要是靠集体,同时又靠个人。有三种所有制就一定有矛盾斗争。教科书上不讲这种矛盾斗争,不提倡群众运动.书上不承认先使社会主义集体所有制过渡到社会主义全民所有制,使整个社会成为单一的社会主义全民所有制,然后再向共产主义过渡。

书上用什么“接近”,“融洽”的模糊说法来代替一种所有制变为另一种所有制,一种生产关系变为另一种生产关系的观点。

就这些方面看,这本书有严重的缺点,有严重的错误,是部分的离开了马列主义。

<p align="center">×××</p>

这本书的写法很不好,没有说服力,读起来没有兴趣,书上不从生产力和生产关系的矛盾,经济基础和上层建筑的矛盾具体分析出发,提出问题,研究问题。它总是从概念出发,从定义出发,只下定义,不讲道理。其实定义应当是分析的结果,不是分析的出发点,书上凭空的提出一连串规律,却不是从具体历史发展过程分析中发现和证明的规律。规律自身不能说明自身,不从具体历史发展过程的分析下手,规律是说不清楚的。

这本书的写法不是势如破竹、高屋建瓴,问题不突出,文章没有说服力,读起来没有兴趣,文章不讲逻辑,甚至形式逻辑也不讲。

这本书看来是几个作者分别一章一章地写的,有分工而无统一,没有形成科学的体系,加上用的是从定义出发的方法,使人觉得是一本经济学词典。作者相当被动,很多地方自己同自己矛盾,后面同前面打架。分工合作,集体写作,虽然是一种方法,但最好的方法是以一个人为主,带几个助手写,像马克思他们写出来的书,才是完整、严密、系统、科学的著作。

写书有批判对象,才有生气。这本教科书虽然也说了些正确的话,但没有展开对错误观点的批判,所以看起来很沉闷。

许多地方使人觉得这本书说的是书生的话,而不是革命家的话,经济学家不懂得经济实践,并不真正内行。看起来这本书是反映了这种情况.作实际工作的人没有概括的能力,没有概念和规律这一套,而作理论工作的人又没有实践的经验,不懂得经济实践,这两种人没有结合起来,也就是理论与实践没有结合起来。

这本书表明作者没有辩证法.写经济学教科书也要有哲学头脑,要有哲学家参加,没有哲学头脑的作家参加,要写出好的经济学教科书来是不可能的。

这本教科书初出版是一九五五年,三版是一九五八年,但主要的骨架似乎在这以前就定下来了,看来斯大林在当时定下来的架子就不大高明。

苏联现在也有人不同意这本书的写法。格.科兹洛夫:《论社会主义政治经济学的科学教程》一文,对这本书的批评,提出了带根本性的意见。他指出这本书在方法上的缺点。他主张从分析社会主义生产过程来说明规律,他提出了结构方面的建议。

从科兹洛夫这些人的批评看来,在苏联也可能产生作为这本教科书的对立面的另一本教科书来,有对立面就好了。

初步读过这本书,可以了解到他们的写法和观点,但是还不能算是研究,最好将来以问题和论点为中心,仔细研究一下,并且搜集一些材料,也看一下不同这本书的观点的其它发表的文章和书报,在有争论的问题上,有什么不同的意见都可以了解一下,问题要弄清楚,至少也要了解两方面的意见。

我们要批评和反对错误的意见,但也要保护一切正确的东西。要勇敢也要谨慎。无论如何。他们写出了一本社会主义政治经济学,总是一大功劳。不管里面有多少问题。有了这本书.至少可供我们议论.并且由此引起进一步的研究。

六十六、关于政治经济学教科书的写法

苏联教科书从所有制出发写,原则上是可以的,但是可以写得更好些。马克思研究资本主义经济。也主要是研究资本主义生产资料所有制。研究生产资料的分配如何决定产品分配。在资本主义社会的生产的社会性和占有的私人性是个基本矛盾。马克思从商品出发,来揭露在商品这种物与物的关系后面所掩盖的人与人的关系。在社会主义社会商品虽然还有两

重性,但是由于生产资料公有制的建立,由于劳动力已不是商品,社会主义的商品两重性已不同于资本主义商品的两重性,人与人之间的关系已经不再被商品这种物与物之间的关系所掩盖。因此,如果还照抄马克思的办法从商品出发,从商品的两重性出发来研究社会主义的经济,可能会反而把问题模糊起来,使人不容易了解。

政治经济学研究的对象是生产关系。按照斯大林的说法,生产关系包括三个方面.即所有制,劳动中人与人的关系,产品分配。我们写政治经济学也可以从所有制出发,先写生产资料私有制变革为生产资料公有制。把官僚资本私有制和资本主义私有制变为社会主义全民所有制,把地主土地私有制变为个体农民私有制,再变成社会主义集体所有制,然后再写两种社会主义公有制的矛盾,社会主义集体所有制如何过渡到社会主义全民所有制。同时也要写全民所有制本身的变革,如下放体制,分级管理,企业自治权等。在我们这里同时是全民所有制的企业,但是有的是中央部门直接管理,有的由省市自治区管。有的由地方专区管,有的由县管,公社管的企业,有的是半全民半集体的性质,无论是中央管的或各级地方管的,都在统一领导下,而且具有一定的自治权。

关于在生产和劳动中人与人的关系问题,教科书中除了有句“同志式的合作互助关系”这样的话以外,根本没有接触到实质问题,没有在这方面分析和研究。所有制的问题解决以后,最重要的问题是管理问题,即全民所有制的企业如何管理的问题,集体所有制的企业如何管理的问题,这也就是一定的所有制下的人与人的关系问题,这方面是大有文章可做的。所有制的变革,在一定时期之内,总有限度,但是在这一定时期内,人与人在生产劳动中的关系却可能是不断变革的。我们对全民所有制的企业的管理,釆用集中领导和群众运动相结合,党的领导、工人群众和技术人员相结合,干部参加劳动,工人参加管理,不断改变不合理的规章制度等等这样一套。

关于产品分配,要重新再写,换一种写法,应该强调艰苦奋斗,强调扩大再生产,强调共产主义前途远景,不能强调个人物质利益,不能把人引向“一个爱人、一座别墅、一辆汽车、一架钢琴、一台电视机”这样为个人不为社会的道路上去。“千里之行始于足下。”但、如果只看到足下,不想到前途,不想到远景,那还有什么革命的兴趣和热情呢?

六十七、关于从现象到本质的研究方法

研究问题要从人们看得见,摸得到的现象出发。来研究隐藏在现象后面的本质,从而揭露客观事物的本质的矛盾。

在国内战争和抗日战争的时候,我们研究战争的问题也是从现象出发的。敌人大、我们小,敌强我弱。这就是当时最大量的,大家都能看得到的现象。我们就是从这个现象出发来研究和解决问题的。研究我们在小而弱的情况下,如何来战胜大而强的敌人。我们指出。我们虽然小而弱,但是有群众的拥护,敌人虽大而强但有空子可钻。拿内战时期来说,敌人有几十万,我们只有几万,战略上是敌强我弱、敌攻我防,但是他们进攻我们就要分成好几路,各路人要分成好多个梯队,常常是一个梯队进到一个据点。而其它梯队还在运动当中,我们就把几万人集中打他一路,而且集中大多数人吃他这一路中的一点,用一部分人去牵制还在运动中的敌人。这样,我们在这点上就成了优势,成了敌小我大,敌弱我强,敌守我攻,加上他到一个地方情况不熟,群众不拥护他们。我们就完全可以消灭这部分敌人。

<p align="center">×××</p>

意识形态成为系统,总是在事物运动的后面,因为思想认识是物质运动的反映。规律是在事物运动中反复出现的东西,不是偶然出现的东西,事物反复出现,才成为规律,才能够被人认识。例如资本主义的危机在过去是十年一次,经过多次反复,就有可能使我们认识到资本主义社会中经济危机的规律。土地改革中要按人口分配土地,而不能按劳力分,这也是经过反复后才认识清楚的。第二次国内战争后期,左倾冒险主义路线的同志主张按劳力分配土地,不赞成按人口平分土地,并认为按人口平均分配土地是阶级观点不明确,群众观点不充分,他们的口号是地主不分田,富农分坏田,其它人按劳力分。这种方法事实证明是错误的,土地应该怎样分法是经过多次反复才弄清楚的。

马克思主义要求逻辑和历史一致。思想是客观存在的反映,逻辑是从历史中来的。而书中堆满材料,不分析没有逻辑,看不出规律,不好,但是没有材料也不好,那就使人只看见逻辑,看不见历史,而且这种逻辑只是主观主义的逻辑。这本教科书的缺点正在这里。

很有必要写出一部中国资本主义发展史。研究历史的人,如果不研究个别社会、个别时代的历史是不能写出好的通史来的,研究个别社会就是要找出个别社会的特殊规律,把个别社会的特殊规律研究清楚了,社会普遍的规律就容易认识了。要从研究许多特殊中间看出一般来,特殊规律搞不清楚,一般规律是搞不清楚的。例如研究动物的一般规律,就必须分别研究脊椎动物和无脊椎动物等等特殊规律。

六十八、哲学要为当前政治服务

任何哲学都是为当前政治服务的。

资产阶级哲学也是为当前政治服务的。而且每个国家,每个时候都有新的理论家。写出新的理论,来为他们当前的政治服务。英国曾经出现过培根和霍布士这样的资产阶级唯物论者,法国十八世纪又出现了百科全书派的唯物论者;德国和俄国资产阶级也有他们的唯物论者。他们都是资产阶级的唯物论者,他们都是为当时的资产阶级政治服务的,所以并不因为有了英国资产阶级唯物论就不要法国的,并不因为有了英国、法国的,就不要德国的和俄国的。

无产阶级的马克思主义哲学,当然更是要密切地为当前的政治服务。对于我们来说,马恩列斯的书必须读,这是第一。但任何国家的共产党人,任何国家的无产阶级思想界都要创造新的理论,写出新的著作,产生自己的理论家,来为当前的政治服务。

任何国家,任何时候,单靠老东西是不行的,单有马克思、恩格斯,没有列宁写出两个策略等著作就不能解决一九○五年和以后出现的新问题,单有一九○七年的“唯物论和经验批判论”就不足以应付十月革命前后产生的新问题。适应这个时候的需要。列宁就写了“帝国主义论”、“国家与革命”等著作。列宁死了,要斯大林写出“列宁主义基础”和“列宁主义问题”这样的著作来对付反动派,保卫列宁主义。我们在第二次国内战争末期抗日战争初期写了“实践论”和“矛盾论”,这些都是适应当时需要而不能不写的。

现在我们已经进入社会主义时代,出现了新的一系列的问题,如果不适应新的需要。写出新的著作,形成新的理论,也是不行的。


