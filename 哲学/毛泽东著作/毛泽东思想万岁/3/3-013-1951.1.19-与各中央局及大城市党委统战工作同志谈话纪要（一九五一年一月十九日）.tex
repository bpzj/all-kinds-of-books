\section[与各中央局及大城市党委统战工作同志谈话纪要(一九五一年一月十九日)]{与各中央局及大城市党委统战工作同志谈话纪要}
\datesubtitle{(一九五一年一月十九日)}


(×××简述此次统战工作会议两项议题中第一议题&lt;反帝爱国统一战线&gt;的情况。谈及工商界、上层知识分手及宗教界均已先后参加了抗美援朝运动时,主席指出一)

对佛教问题的处理,要研究一下,要使他们参加到运动中来。佛教改革如何作,要他们里面的开明分子商量一下。他们中有人说,他们没有信佛教的自由,庙都被占了。信佛教的人也不少,不要不理会。北京市恐怕要给他们腾出几个庙来,五台山关系蒙、藏两个民族,恐怕要重修一下。

(主席问西北、西南抗美援朝运动的情况,回答此运动已住大城市中及部分小城市展开,主席指示——)

应该趁此时机展开,这是一个极大的教育。小城市如天水的商人最近曾有电报给我,说他们已经举行了游行示威。这次运动应普遍开展,普及到所有的工厂和农村,使得家喻户晓,大家都参加进来。

(谈到运动如何进一步开展时,主席指示——)

可找几位这方面的负责同志谈谈,好好研究一下。

(有同志反映,朝鲜战争中,特别美帝在仁川登陆后,谣言较多,思想混乱,当时我们在宣传上还做得不够。主席指示——)

在宣传上应当采取攻势,对特务造谣,镇压一下好。

反美、土改、镇压反革命,是当前三大运动。

(中南同志反映,在土改中,农民进城捉人影响工商业。主席指示——)

不可不捉,不可多捉。匪首恶霸是要捉的,但应合法地捉。(中南和两南同志报告,武汉和西南设立了城市联络组和减租退押调处委员会,作为适当处理土改和减租退押城市联系问题的机关,主席指示——)

设立这样的机关好,在几万人口以上的城市都要设立(县城是否设立,可以考虑),并且民主党派参加,由市政府管理.区别可捉者捉之,不可捉者则不捉,双方兼顾,合理解决,以免发生无政府状态。退押是要退的,也可经过这类机关商量。退多退少,适当处理,能多退则多退不能多退则酌量少退。

(主席问到华东情况,回答苏南各地现在情况好一些。主席说——)

黄炎培先生准备去苏南看看土改,我向他说:你们去看看很好,可以听到各级领导干部、农民和地主富农三方面的意见。

(各地同志反映,地方上有些同志对民主人土去视察尚不免有所顾虑,主席指示——)

民主人士到各地去视察,各地不要以此为累赘。让他们去听听农民的诉苦,看看农民的欢喜。我们有些什么缺点和错误,也可以让他们看看。这是一件有益的事,状元三年一考,土改千载难逢,应该欢迎他们去看。

(地方同志反映。对于民主人士的视察和批评,缺乏积极的欢迎态度,主席进一步指示——)

分土地,镇压反革命,发动群众,都是好事。土改一项从尧、舜、禹、汤、文、武、周公、孔子直到孙中山都没有做过,我们才做。我们做了什么坏事情呢?有什么怕人家看的呢?十个指大有九个是好的,最多只有一个指头不好。你们对民主人士不要估计不足。要知道除了恶霸、匪首和特务分子以外,对于工商业家、宗教界、校长、教员、开明士绅和爱国分手,我们都应采取积极的态度,团结和教育他们,决不能置之不理,而要采取积极态度。这个办法屡试屡验,结果总是好的。一切消极态度都是不对的,有话应当让他们说,写万言书也好,我们可以给大家看看,好的接受,不好的解释。有人看到毛病要骂怎么办?共产党连帝国主义都不怕,还怕人家骂?对民主人十要进行教育,并让他们参加活动。如果不进行教育,有事不让他们与闻,这是不对的。

(关于统战会议的第二个议题,即各民主党派组织的发展问题,主席指示——)

这一问题要好好解决一下,不然我们所做的和所讲的不一致,下面同我们不一致,就不好。去年说要巩固,今年说要发展,如果今年还只是一个巩固,那是不行的。因为社会上有他们的阶层和人物存在,所以就应当发展。中国四万万七千五百万人所有民主党派不到二万人,今年发展一倍也不过四万人并不算多。这个问题应当写出几条,指示一下,应该在运动中使民主党派进步,放手让他们去做,好处多于坏处少。这是积极的有益的方针。

(各地同志反映,关于民主党派发展问题,有些地方干部思想上还有若干阻碍,如果没有党委有力的支持,就不易贯彻。主席指示——)

......应该使全党了解,民主党派和民主人士是可以进步的。这次绥远二十三兵团已经调到冀中,这个部队的军长师长都是原来的人过去和我们打过很多仗,连这个部队都可以进步,为什么民主党派和民主人士就不可以进步呢?我们应采取积极态度去办,不要怕麻烦。要使他们了解抗美、土改、肃反是必要的,并且使他们参加。搞国家政权总是有麻烦的,怕麻烦就不要政权好了。民主党派和民主人士过去找他们来都不来,现在来了,你们

又怕麻烦。

(×××提到,有关统战部门本身的工作还缺乏一个章程,此次会议准备拟出几条,主席指示——)

对统战部的编制、任务、范围和作法均有所规定,这是好的。直到地委一级均应有统战部门的组织才好,是如何办,也研究一下。

(有同志反映,主席去年说过,注意统战工作的,在党内还是少数,这个状态,今年在若干地区没有多大改变。主席指示——)

这个少数派是正确的,只要坚持土改与肃反,这就不是右倾,这就是马克思主义。要用积极态度去组织民主人士、民主党派,教育他们,使他们参加斗争,带领他们共同前进,这对劳动人民是有益的,对进入社会主义是有益的。少数派会慢慢扩大,主要依靠好好说服人家。应当每年开两次统战会议,一次大的,一次小的,各中央局至区党委都应如此。不要怪别人只怪你们自己没有招兵买马、积草囤粮。不要使自己陷于孤立,要好好做出成绩来给大家看。你们要很好地取得党委书记和组织、宣传等部门的帮助,善于经过党委和党委负责同志,把我们的国家建设起来。你们当中要出些专家,要熟悉人物和历史,精通此中门道。

(中南同志提出镇压反革命分子应与土改中反霸斗争有所区别。主席说——)此一问题,×××同志已经发了指示。江西四十七军向中央的报告,在一个多月中,二十多县里,杀了匪首×××人得到老百姓的欢呼。(主席已将此报告批示转发奋大行政区传看,并嘱到会同志都看一看。)


