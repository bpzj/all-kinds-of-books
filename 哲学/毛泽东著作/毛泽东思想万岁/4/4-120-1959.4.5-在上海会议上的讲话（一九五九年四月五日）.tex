\section[在上海会议上的讲话(一九五九年四月五日)]{在上海会议上的讲话}
\datesubtitle{(一九五九年四月五日)}


要多谋善断。第一要多谋,第二要善断。

什么叫多谋呢?就是要听听不同的意见。先跟我们这些人谈一谈,交换交换意见。

要谋于秘书,谋于省市党书记,谋于地委书记、县委书记、公社书记、谋于农民,谋于厂长、车间主任、工段长、小组长,谋于工人,谋于有不同意见的同志。

要当机立断,不要错过形势。机不可失,时不再来。

要善于观察形势,才能当机立断。

缺乏当机立断,还是对形势观察不妥,断得不恰当,就是有一点武断。

<p align="center">×××</p>

留有余地,成都会议上就讲过留有余地,后头不留有余地了。我们过去打仗,是用三倍、四倍、五倍、六倍、以至七倍的兵力来包围敌人,这是留了很大的余地。你一个团,我用三个、五个、六个、十个团,有几个团的后备,总可以把它吃下。不打无准备之仗,不打无把握之仗。而现在我们搞工业很多是打没有把握之仗,打没有准备之仗。我就怀疑搞工业的同志们是否真正积累了经验。积累了一些,还有一些没有积累。工作方法有相当大部份不对头。比如就不晓得多谋善断,留有余地。这是个马克思主义的方法问题。

<p align="center">×××</p>

我们过去反对的“马鞍形’,重点是在反对“反冒进”。一九五七年不搞“马鞍形”是不行的。为了完成第一个五年计划,一九五六年搞了一百四十亿元的基本建设投资,当时是必要的。但是库存减少很多,一九五七年不得不调整一下。一九五六年十一月中央全会的时候,我完全赞成调整。钱和材料只有那么多,只能办那么多的事。

“马鞍形”将来还会有。生产增长速度可能一年高一点、一年低一点,或者两年高一点、一年低一点,或者三年高一点、一年两年低一点。不能每天高潮。像我们开会,每天高潮,就要死人的。波浪式的前进,这是个工作方法问题。


