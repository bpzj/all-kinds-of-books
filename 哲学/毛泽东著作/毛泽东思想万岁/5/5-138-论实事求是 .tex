\section[论实事求是 ]{论实事求是 }


一切马克思列宁主义者都必须有严肃的战斗的科学态度,具有老老实实的科学态度。

一、实事求是,就是要在马列主义的立场、观点、方法的指导下,一切从客观的真实情况出发,研究和认识客观事物发展的规律性,做为我们行动的根据和向导。

二、实事求是,必须从实际出发,善于对具体事物做具体分析,按照具体的时间、地点、条件决定方针、政策、路线。根据新的革命形势提出新的革命任务和新的工作方案,善于把党的方针政策路线,变成群众的自觉行动。

三、实事求是,就是一切工作都要服从人民群众的实际需要和利益出发,实行某种改革,要完全根据人民的自觉自愿,既要耐心地等待群众的觉悟,让群众有所比较和选择,由群众自己下决心,又要积极地创造条件,作出榜样,进行宣传,说服群众,既要从本质发现群众中蕴藏着的巨大的积极性,又要按照具体的环境,具体地表现出来的群众情绪去进行一切工作。

四、实事求是,就是要客观地、全面地、本质地看待问题和解决问题。认清事物的现象与劳动人民的无穷无尽的创造力量,至于个人的革命事业中,不过是一个小小螺丝钉。马克思列宁主义告诉我们,任何一个成就,都是集体力量的结晶,个人离不开集体的,个人想做一点事业,如果没有党的领导,没有组织和人民群众的支持,就将会寸步难行,一事无成。如果我们真正深刻理解到人民群众和个人在历史上的作用,及其相互关系,我们便会自觉谦虚起来。

因为马克思列宁主义的理论,可以提高我们对前途和方向的认识,开阔我们的眼界,使我们的思想从狭隘范围里解放出来。当人们的眼界看到脚下,而看不到高山和大洋的时候,他是会像“井底之蛙”那样自负不凡的,但当他的头抬起来,看到宇宙之大,事物之变无穷,人类事业雄伟壮丽,人材之多和知识之无限,他便会谦虚起来,我们所从事的天翻地复的大事业。\marginpar{\footnotesize 206}

