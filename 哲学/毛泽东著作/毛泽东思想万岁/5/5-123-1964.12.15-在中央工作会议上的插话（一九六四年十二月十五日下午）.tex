\section[在中央工作会议上的插话(一九六四年十二月十五日下午)]{在中央工作会议上的插话(一九六四年十二月十五日下午)}
\datesubtitle{(一九六四年十二月十五日)}


〔有些地方提出,出现了新生的资产阶级分子。〕

叫新生资产阶级分子,农民不容易懂,还是叫贪污盗窃、投机倒把分子好,农民懂得。

〔加强社教工作队的领导问题。战线是不是要缩短些?〕

要短容易,收缩嘛!

〔关于农村整党、建党问题〕

我早就赞成。……

〔要发展一批党员〕

还是从贫下中农、工人积极分子中去发展党。

〔下边借反资本主义来反对贫下中农〕

让他搞嘛!你不让他搞,他憋不住。要反群众,他就暴露出来了。

〔机关干部家属中有不少四类分子,要清理。这是个普遍的问题。〕

反革命、恶霸、四类分子,就那么多嘛!几千万,几百万,分散各地,总是少数。只有那么多,有是有,清是要清,多是不多。

〔领导干部蹲点,还是蹲在大队的好。〕

抓中间,吃两头。

〔这次社教运动中,有些地方还查出不少黑地。〕

我看瞒百分之二十——三十就算好的。

〔我看查出来,过三、五年再征购。〕粮食五年到十年不征购,以后增产再增购一些。粮食还是存在老百姓家里好,否则,阻止人家报黑地。

〔要把干部不脱离生产的制度固定下来。〕

反官僚主义,要跟班生产。

〔科室人员有半天的工作时间也就够了。〕

还要注意文牍主义。

能够四、五小时劳动,厂长、党委书记事就少了,开会、文件也会少了,工作就能做好了。

(会议结束时)我们的会议,每天下午三点半开大会,大家可以在大会上发言。上午开小会。你们如果秘密谈,出简报我也不看。你们要畅所欲言,冲口而出。讲的不对,有什么要紧?大家可以原谅嘛!


