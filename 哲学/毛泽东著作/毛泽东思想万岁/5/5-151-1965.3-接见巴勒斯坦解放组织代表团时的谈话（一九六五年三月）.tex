\section[接见巴勒斯坦解放组织代表团时的谈话(一九六五年三月)]{接见巴勒斯坦解放组织代表团时的谈话}
\datesubtitle{(一九六五年三月)}


日本投降了,我们又被迫打仗。打的办法就有两条:你打你的,我打我的。什么军事道理,简单的就这么两句话。什么叫你打你的?他找我打,但他又找不到,扑了个空。什么叫我打我的?我们集中几个师、几个旅,把他吃掉。

事物都是可以分割的。帝国主义也是事物,也可以分割,也可以一块一块地消灭。蒋介石八百万军队也是事物,也可以一块块地消灭。这就叫作各个击破,这就是欧洲和中国古书里说的道理。很简单,没有什么深奥的道理。不要看书啰,谁打仗还拿本书去看。我打仗从来不看书。少看一点,看多了不好。

战场就是学校,军事学校我不反对,可以办,但不要学得太长了,一读两三年,太长了,几个月就行了。什么陆、海、空学校,不怎么高明。

有些现代科学需要长一些时间学,例如导弹、原子弹,就是讲研究和制造。单单武器的使用学,训练士兵,不需要很长时间。训练炮兵一个月就行。训练驾驶员、飞行员,几个月就够了,最多一年。主要是在战场上训练。和平时期要在黑夜里训练。战争时期,战争就是学习。你不是说念了我写的文章吗?这些东西用处不大,主要是两条:你打你的,我打我的。我打我的又是两句话:打得赢就打,打不赢就走。帝国主义最怕这种办法。打得赢我就把你吃掉,打不赢我就走掉,你找也找不着。我们开头时用游击战的办法,攻的时候用,防御的时候也用。根本的办法是打游击战。打蒋介石时,从小仗打到大仗。后来我们用三十万军队消灭他五十万军队,我们用三个指头吃他五个指头。我们是少数,怎么能吃得掉?怎么吃法呢?还是一个个地吃,结果把它吃掉了。

有些外国人在中国上学,学军事。我劝他们回去,不要学太长。几个月就行了,课堂上尽讲,没有什么用处。回去参加打仗,最有用。有些道理只要讲点就行了,不讲也可以。大多数时间可在本国,或者根本不出国,就可以去那里。\marginpar{\footnotesize 229}

