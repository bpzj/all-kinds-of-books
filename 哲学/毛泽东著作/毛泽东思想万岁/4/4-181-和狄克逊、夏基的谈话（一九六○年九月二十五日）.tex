\section[和狄克逊、夏基的谈话(一九六○年九月二十五日)]{和狄克逊、夏基的谈话(一九六○年九月二十五日)}


夏基:首先。我必须说,在若干原则性的问题上,像我们时代的性质,防止战争。帝国主义的本性,和平过渡,普遍和全面裁军以及其他问题上,我们完全支持中国党的立场和主张,对于另外有些个别问题。我们不很清楚。共产主义运动中发生原则方面的分歧,只能有利于阶级敌人,这是使我们纳闷的事情。和中国同志一样。我们希望分歧经过讨论之后。能在马克思列宁主义原则的基础上获得解决。中国党和苏联党必须团结。

主席:是的,团结是必须的。就我们这方面说,我们当然主张团结。有些人却不愿团结。他们是货真价实的修正主义者,任何党里都没有什么完全一致。你是不是认为你们党是完全团结和一致的呢了

夏基:是的。

主席,你认为你们党能经得起风浪的考验么?

夏基:希望它能够。

主席:在下层干部中有没有反对的意见?

夏基:也许有。但就我党政治委员会的同志来说,他们是完全一致的。在目前阶段。我们还没有在下层中讨论这个问题。

主席:有大多数人的支持是好的。在中央委员会里有大多数的支持是好的。

夏基:在中央委员会里获得压倒多数的支持将是可能的。在狄克逊同志和我离开澳大利亚的前两天,我党政治委员会开了两天会。我们在新西兰共产党的汤赖同志和曼森同志所作报告的基础上进行了讨论。我党政治委员会的十一个成员一致支持中国党的立场和主张。

狄克逊:事实上我们前些时候研究了中国党的为记念列宁诞生九十周年而发表的三篇文章,我们决定支持中国同志。

主席:我们知道你们党曾和英国党辩论过。我们党和英国党也曾在和平过渡这个问题上有过辩论。

夏基:这也是我们党和英国党辩论的主题。

<p align="center">×××</p>

夏基:苏共是普遍和全面裁军的倡议者。

主席:我还不了解普遍和全面裁军是什么意义。

夏基:我也不了解.苏共提出了这个口号之后,我们才认识到这个问题的严重性。就像我们那个小国家,还不是什么帝国主义大国,可是我们的资产阶级老是在嚷扩军备战。在许多地方,战争事实上正在进行,澳大利亚政府派军队到马来亚去插手,屠杀游击队。这个政府目前正计划通过一顶法律来加强对人民的压迫。这是针对我们党和整个工人运动的。朝鲜战争时。我们的同志因反对朝鲜战争,支援中国而被捕,他们只剁坐牢几个月。但按照政府......这一案件就得判死刑。

主席:由什么案子要判死刑?

夏基:按照这项法律的条文,支援………”死刑,犯“叛国罪”的终身监禁,有破坏活动的监禁七年。这都是为战争作准备,那些家伙当然不是在考虑什么普遍和全面裁军恰恰相反,他们是在准备战争。在美国这次党选中,共和党和民主党双方都宣扬扩军备战,他们竞赛谁能更好地扩军备战。因此,苏共的意见完全是梦想、是不现实的。

<p align="center">×××</p>

夏基:有些人抱着梦想,认为裁军省下来的钱可以用来援助不发达国家,认为帝国主义可能自动解除武装,但是,帝国主义者自己知道得很清楚,假如他们解除武装,工人阶级就会罢工来夺取政权。

主席:是啊!如果真是那样,这岂不是奇事呢!

夏基:让我们看看澳大利亚的形势以及资本主义社会的一般形势吧。为了赚钱,资本家正在扩军备战。例如在澳大利亚这样的小国中,去年钢铁公司获利二千万英磅。而在澳大利亚的美资通用汽车公司却获利一千五百万英磅。与此同时都:教育存在着持久危机,医院缺乏。道路一团糟。并且,虽然时常闹旱灾,却毫无蓄水计划.资本家不在这方面投资。因为没有厚利可图,如果是垄断资本获得巨额利润,而社会制度却倒退。扩军备战能为资产阶级带来巨大的利润,但是教育事业等等是获利很少的。资产阶级从事前者而不从事后者。这是自然的道理。

主席:这是符合历史发展的规律的,是一切资本主义国家的真实情况。

<p align="center">×××</p>

主席:不抵抗暴力的保证正好投合资产阶级的需要。工党是反对阶级斗争的。一个真正的共产党必须是一个主张阶级斗争的党。

夏基:共产党所以存在的理由就是要进行阶级斗争。

主席:不然的话,为什么共产党不去加入工党呢?

狄克逊:在我党的历史上,党内经常出现右倾机会主义思想,那些人想搞垮共产党。他们要共产党加入工党,要把共产党变成为工党的翼,或具有推动力量的团体。一到历史上的重要关头和阶段,右倾机会主义总要出现。我们党和英国共产党长期以来,就存在分歧。一个是关于和平过渡问题的分歧,而另一个是关于对待工党的态度的分歧。英国共产党提出了组成一个左翼工党政府的口号。当然,在他们看来.这样一个政府是会容纳共产党的。可是我们并不设想改良主义者会有那样的大度量,以至能够容纳共产党。在一九四七年和一九四八年,英国共产党在澳大利亚散发了许多材料,散布这种思想,搞得我党最后不得不批评英国共产党。波立特同志对我们的批评很生气,他问我们敢不敢公开批评他们。我们说;“敢”。这样,一个公开的辩论就发生了。我们同英国共产党的辩论继续了十几年,然而分歧当然没有得到解决。我们希望中苏两党间的分歧不会像澳大利亚党和英国党之间的分歧那样长久存在。我党和英国党之间的分歧没有产生很大影响,但是中苏两党之间的分歧必定会有很大的影响。两者是不能相提并论的。

<p align="center">×××</p>

主席:他们能向我们施加压力,他们能召回专家,施行经济封锁等等。中苏之间的文化交流已经不存在可能,《中苏友好》那个杂志已经停刊。两国之间的贸易是否继续,我们还不能断言,贸易的数量必然会减少。因为撤退专家就使按他们计划而修建的工厂发生困难。

夏基:赫鲁晓夫高喊反对美国向古巴施加经济压力,但是他本人却向另一个社会主义国家施加经济压力。这比美国干得更坏。


