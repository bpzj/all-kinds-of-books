\section[毛主席视察华北、中南和华东地区时谈话的主要精神的传达一、视察沿途讲话主要精神]{毛主席视察华北、中南和华东地区时谈话的主要精神的传达一、视察沿途讲话主要精神}


一、形势:形势大好,全国已解决了七个省(指革委会),基本上解决了八个省(指河南,湖南已建立筹备小组),共十五个省。争取今年再解决十个省,南方五个,北方五个,共二十五个。(实际上二十四个省。因黑龙江解决两次,中央是支持黑龙江革命委员会,革命委员会不能垮,坚决支持。)春节前,全国要基本解决,纳入轨道。文化大革命七、八、九三个月大进一步。形势和任务就是如此。

二、上下级关系:为什么搞的那么苦?打、罚跪、挂牌、戴高帽不好,这样把团结——批评——团结的原则破坏了。对干部要扩大教育面,缩小打击面,对干部不要一棍子打死,最顽固的也要给一碗饭吃。

北京开武装干部会,不只武的要去,文的也要去,党政群干部也要参加,左派也要去,红卫兵也要去。红卫兵权力很大,又很凶,也要训练。对干部不能不教而诛,教也不能诛。

要扩大教育面,主席在上海说了几次,并在许多省都讲了,中央文革也认真讨论了的。

三、大联合。那条语录:“在工人阶级内部,没有根本的利害冲突。在无产阶级专政下的工人阶级内部,更没有理由一定要分裂成为势不两立的两大派组织。”是七月十八日,主席在武汉第一次讲的。这句话没有被接受,武汉问题如果用这个思想去解决就更好。主席反复谈这个问题,七月谈过,八月也谈过,(我)在上海把这条语录给工人讲了,很灵,上海大联合高潮就是主席这几句话搞起来的,主席关于大联合指示,全国一宣传,效果很好。

×××插话,主席是这样讲的:“一个工厂,都是工人阶级,没有根本利害冲突,为什么要分裂为势不两立的两大派,我就想不过,这是有人操纵,无非是:一种走资派操纵,第二种地富反坏右搞鬼,第三种是小集团思潮影响。”

春桥接着说:反复宣传主席这个思想,主要是工人,对学生,机关干部也有效。工人阶级真正成为文化革命的主力军,让工人来左右局势,上海最有发言权的是工人,不是学生说了算数,上海一乱主席就问我:“乱得起来吗?”我说:“不要紧,工总司不动就不会乱。要确立工人阶级领导地位,不要被学生牵着鼻子跑。”

四、三结合。主席讲:“上海军、干、群三方面关系比较好。”主要是三方面经常开门整风,这次以军队为主,召集革命群众团体,下次以革命群众为主,如此轮流,这个制度比较好。工总司每天坚持半天学习毛著,半天工作,天塌下来也不管。军队负责召集。干部向题是一个重要条件,上海部、局长一级干部已经解放了百分之五十至六十。

×××插话;主席说:干部绝大多数是好的,要教育干部,解放干部,使用干部,干部中除了走资派为什么斗那么凶,斗那么厉害呀!有些干部坐汽车了,房子住好了,官做大了,工资高了,这都还可以,但是不要有架子,不讲民主,脱离群众。所以,一有机会就起来攻你。

当我们问到“5.16”时,春桥同志说:“5.16”是个反革命组织,有三条,一反中央,二反解放军,三反革委会。姚文元同志的文章后一段,讲革委会一段,是主席亲自加的。


