\section[在知识分子问题会议上的讲话(一九五六年一月二十日)]{在知识分子问题会议上的讲话}
\datesubtitle{(一九五六年一月二十日)}


会议开得很好。听了各位同志的讲话,很高兴,可以看出同志们的水平很不低,很高,许多讲话是很好的。

现在我要讲的是领导方法问题。有两种领导方法:一种是使事业进行得慢一些、差一些的方法,另一种是使事业进行得快一些、好一些的方法。拿最近一年来农业合作化,资本主义工商业改造,知识分子的问题等几件事来看,可以这样,也可以那样,可以迟一些、坏一些,也可以早一些、好一些。我们在延安时,曾组织过宪政促进会,现在没有了;现在有人说中央是全国社会主义促进会,这次会议也是一个促进会,省、市委,各部门,也是促进会。领导机关可以促进,也可以促退,也可以有进有退。蒋介石的国民政府就是国民促退政治,它促使社会倒退。我们的领导机关应该促进事业的发展,但也不完全是这样,如农村工作部有一个时期就是农村工作促退部,因为它要砍掉些合作社;卫生部有一个时期也是促退部,把事业搞垮了。各地方、各部门是否也有这种现象呢?上层建筑不适合经济基础,不能促进经济发展,这就丧失了上层建筑应有的作用。生产力和生产关系,生产关系是基础,上面还有政府,党,各部门,这都是上层建筑,必须起促进生产力发展的作用,如果不促进,就丧失了它的职能。反对右倾保守思想就是为了解决这个问题的,就是为了使上层建筑能够适应基础,促进社会的发展。你们对客观情况的发展注意不够,估计不足。凡事都有两面性,一万年后还是有两面性,但是,今年和明年不是一样的两面性,如果一点不变化,就没有马克思主义了。但是现在是马克思主义的,用马克思主义来看,农民几年来对他们的落后性抛弃很多,最近几个月,农民跑得很快,几个月就改变了个体经济。资产阶级也是这样,我们总是说他们“唯利是图”,现在他们已经把工厂交出来了(变成公私合营了),“唯利”又怎样“是图”法?对知识分子也是尽说他们不好,可是这次会上同志们说知识分子中进步分子占40%左右,中间分子也占40%左右,落后的当然还有,但是可以改变的,而且应该促进这种改变。可是有的同志看不到这种变化,想到没有对象的地方去决斗,对像没有了,还要斗,这就变成唐·吉柯德了。当然,还有对象,但也在起变化,因此我们应当欢迎这种变化。我们应该看到在生产关系改变后的新形势,去年上半年那样困难,到处骂我们,党内、党外都说我们不行,就是为几颗粮食,下半年不骂了。下半年有几件喜事,丰收和合作化是两件大喜事,还有肃反也是喜事。最近两三年,中国社会发生激烈变化,这些情况反映在同志中间,人民中间和知识分子中间,我们应该认识这些情况,并且适应这些情况。

讲到促进和反对右倾保守思想,同时也应该注意不要搞那些没有根据的行不通的事情。锣敲、打鼓、报喜,这是好的,这正是促进嘛,这正说明中央是社会主义促进会嘛!但是现在已经可以看出,有些情况要引起我们注意,值得谈谈这一点。农村发展纲要中已把好些农业增产的指标去掉了,右倾保守思想存在于各方面,两种领导方法采取哪一种?我因为其中有些还没有研究,没有充分根据,例如垦荒五亿五十万亩,据说用的钱要等于一年的预算,要多少人民币?垦荒仍是要垦的,但是否能垦那么多,要仔细研究。计划一定要行得通,一定要建筑在有根据的基础上,不然又要搞成盲目性。现在有些同志头脑已经有些发热了。合作化快是好的,但不要无根据的快。各省同志要注意。高级化务必要做到在绝大多数人满意条件下高级化,现在听说,有百分之三十左右的人对放弃土地分红还放不下,如果是这样,那就不如等几个月。北方可以到今年秋冬或明年春天再转,明春最好。建议你们多考虑。陈云同志说公私合营,人家送上门来,不要不好,但问题是内心是否通,如不通,那就等几个月,等几个月并不算长。我建议这样,不要以为又是来阻挡你们,又是右倾机会主义!总要叫百分之九十几的人高兴,不高兴的过了百分之几就有问题。各部门计划指标也要放在可靠的基础上,本来可以做的不做,是不好的,但无充分根据的行不通的就叫盲目性,就是“左”倾冒险,虽然目前这还不是主要的倾向,但已经可以看出,有些同志的头脑不那么清醒,不敢于实事求是,怕右倾机会主义的帽子难听。凡是经过调查研究,办不到的,要敢于说办不到,敢于停下来,把计划放在可靠的基础上。

北京已进入社会主义,进入是进入了,但尚未完成,不要说已经完成。要完成,还要做几年工作(比如说还要三、四年),资本家定息的尾巴哪年割掉,还要看情况发展。北京资本家清产核资用自评的办法,可以介绍,以后农村中地主富农入社,那些人好,哪些人坏,哪些人算社员,哪些人算候补社员,哪些人应当管制生产,可不可以也让他们自己去评,我们加以领导,建议各地试试看,让他们自己里头发生斗争。

再一点,有的同志引用我在《中国农村的社会主义高潮的序言》中说到的:“已经不能完全按照原来所想象的那个样子去做了”这句话,我说“不能完全按照”,这就是说大体上还要按照,否则五年计划,岂不也没有了。但是有的同志一改改成“已经不能按照”或“已经完全不能按照”,可见头脑已经有些发热了。

吴玉章同志的发言讲得很好,关于文字改革的意见,我都赞成,你们赞成不赞成?老百姓问题不大,识字很容易,但有些人觉得采用罗马字拼音,好是很好,可惜罗马字不是中国人发明的。中国人发明,外国人学习,“用夏变夷”,没有问题;外国人发明,中国人学习,“用夷变夏”,就有问题了。洋字比较好,吴玉章同志说的有理,字母少,写起来一边倒,汉字比不上。有些教授说:“汉字是世界上万国最好的文字。”我看不见得。因此我们采用罗马字。例如阿拉伯字也是外国发明的,现在不是大家都用了吗?罗马字出现在罗马,英、美、俄等国不也都在采用吗?我没有学过文字史,他们过去也都有自己的文字的。据说我们中国字是仓颉造出来的,我看不见得。社会主义不是出在俄国,俄国也学了。外国的好东西,我们要统统拿过来,变成我们的东西,要在一、二十年赶上国际先进水平。汉朝、唐朝就是这样的,唐朝奏乐、舞蹈有七种,有六种是外国的,唐朝很有名,搞久了就变成中国人的。

在吴玉章同志的讲话中,其中有一条我是不赞成的。就是他说:“这是由于毛泽东同志在去年七月三十一日作了关于农业合作化问题的报告,严肃地批判了右倾保守思想并用新的社会观点和理论来动员群众和组织群众的结果。”“这种新的观点和理论,只有当社会的物质生活的发展,已经在社会面前提出新的任务以后,才产生出来,当新的观点和理论产生以后,它们就成为最严重的力量,就能促进解决由社会物质生活的发展过程所提出来的新任务,就能促进社会的向前发展。”这样说来,好像农业合作化是新的事情,新的社会观点和理论。但是,这不是什么新东西,联共党史上就写到的,他们是地球上第一次做这件事,还在一百多年前马克思就已经讲过了,三十多年前,俄国已经做了。我们现在讲的,并没有什么新的观点和理论。我们没有什么新的东西,有就是有,没有就是没有。过去在欧洲,在马克思和恩格斯面前已经提出了新任务,产生了新的观点和理论,就是马克思主义,以后列宁又加以发展。我们有没有新的东西呢?有的,那就是在形式和细节方面,如互助组的普遍发展,全行业公私合营等,这是新的。关于过渡时期和资产阶级联盟,在理论方面,列宁已讲过了,只是我们实行得比较完全和有步骤,这很好,有必要加以发展。搞社会主义,不能使羊肉不好吃,也不能使南京板鸭,云南火腿不好吃,(现在云南没有火腿了吗?)不能使物质的花样少了,布匹少了,羊肉不一定照马克思主义做,在社会主义社会里,羊肉、鸭子应该更好吃,更进步,这才体现出社会主义比资本主义进步,否则我们在羊肉面前就没有威信了。社会主义一定要比资本主义还要好,还要进步。要注意情况的复杂性,善于分析,以适合各种复杂情况。许多工作,也还有发展余地,譬如肃反也要发展,现在肃反有一个准备阶段,大有文章可做,不打无准备之仗,这是好的。准备工作做好了,就能做到比延安、甚至比去年又快又好。准备工作做不好,底没有摸清,就会把时间拖长,树敌太多。这都属于领导艺术,也都大有发展余地。去夏以来,几方面的工作,如反胡风、农业合作化、对资本主义工商业改造等领导艺术都是有发展的,在此基础上还可以更发展一步,我们在理论上也应有所贡献,把前人讲过的进一步的发展。但十月革命到现在还没有显着的新东西。

和平问题。和平问题是大家很关心的,是不是有可能让我们有十二年时间,来基本上完成工业化呢?看来是有的。第一次世界大战到第二次世界大战中共经过二十一年,现在,第二次世界大战后已经过去十年,还可能多些。现在和第一次世界大战后的形势已有不同,主要是德国和日本与过去不同,挑战不大可能。美国讲究赚钱,蚀本他不干,没有人抬轿子,自己不想用脚走路。现在美国军队的摆法,就不像打仗的样子,它到处搞基地,就像牛尾巴捆绑在桩上,怎样好动呢?但是还要估计到也有可能突然袭击,世界上可能出疯子的,必须估计到。所以我们的工作,越能提早完成越好,越有利。

现在我们的主动一天一天多了,如农业,对资本主义工商业改造等。知识分子问题现在还没有主动,还要过一些时候才能主动。在工业方面也没有主动,大批机器还要靠外国,大的、小的(精密的),我们都不能制造,只能造中等的,“两头不行,中间可以”。我们吹牛皮吹不起来,工业上没有独立,科学上没有独立,重要的工业装备和精密机器都不能制造。地大、人多,但是自制的汽车、坦克、飞机有多少,一辆汽车,一架飞机,未免太少吧?有什么值得翘尾巴的!有的同志说些不聪明的话,说什么“不要他们也行”,“老子是革命的”,这话不对,现在叫技术革命,文化革命,革愚蠢无知的命,没有他们是不行的,单靠我们老粗是不行的。这些话是聪明的话,要向广大干部讲清楚,现在打仗,飞机要飞到一万八千公尺的高空,超音速,不是过去骑着马了,没有高级知识分子是不行的,现在我们看出这件事,就可以开始主动,要有大批的高级知识分子,就要有更多的普通知识分子,以后要使每人都有华罗庚那样的数学,都要能看《资本论》这是可能的,二十年不行三十年,最多一百年就差不多,否则叫什么共产主义?同志们回去要向各方面说清楚这件事情。中国应该有大批知识分子,先接近世界水平,过后赶上世界水平。我国地方大,人口多,位置也不错,海岸线很长,(就是没有轮船)应该成为世界上第一个文化、科学、技术、工业发达的国家。我们有社会主义制度,再加努力,是能够办到的。否则六亿人口,又是勤劳、勇敢干什么呢?几十年以后,如还不是世界上第一个大国,是不应该的。现在美国只有十几颗氢弹,一万万吨钢,我看没有什么了不起,中国应该搞它几万万吨钢。

中国有个好处,一个是穷,一个是白(无知识),这也有两面性,穷就要革命,知识少是不好的,但好比这张白纸,这一面写过了,就没有什么好文章可做,这一面没有写过,是空白的,就大有文章可做,几十年后,就可以赶上外国。

(此记录稿未经校对,好多地方没有听清,仅供参考,勿公开引用)


