\section[在中共中央军委扩大会议上和外事会议上的讲话(一九五九年九月十一日)]{在中共中央军委扩大会议上和外事会议上的讲话}
\datesubtitle{(一九五九年九月十一日)}


同志们:

这个会开得很好。我说居心不良的人,他要走到他的反面。对于世界的阶级,对于世界的党,对于党的事业、阶级的事业、人民的事业,居心不良的人,他就要走到他的反面,就是他的目的达不到。比如讲要达到一个什么目的,而结果那个目的达不到,自己输了理,在群众中孤立起来。比如有几位同志,据我看,他们从来不是一个马克思主义者,一直到现在,他们从来就没有成为马克思主义者,是什么呢?是马克思主义的同路人。要把这一点加以论证,材料是很充分的,比如现在印发的很多材料、抗日时期的材料、长征末期的材料,比如挑拨离间、

抗日时期的材料,比如什么“自由、平等、博爱”抗日阵线不能分左中右,分左中右就错误的呀”,“己所不欲,勿施于人”。在阶级关系中,无产阶级与资产阶级,压迫者与被压迫者,提出这样的原则出来,什么“王子犯法,庶民同罪”,这样的一些观点我看不能说是马克思主义的观点,完全不能说是马克思主义者的观点。这是违反马克思主义的,是欺骗人民的,是资产阶级的观点。后来高饶彭黄反党联盟那些观点,比如“军党论”之类,挑拨党内的不正常关系,认为这也有个摊摊那也有个摊摊。这样一些观点和行为,不是马克思主义者的观点和行为。这一次大量揭发的在庐山会议多少年前的分裂活动,庐山的纲领,此外还有立三路线时期,都有许多材料的。主要是见诸文字的,大家揭发出来,就是刚才讲的这一些。所以要论证我刚才讲的观点,他们从来就不是马克思主义者,他们只是我们的同路人,他们只是资产阶级分子,投机分子混在我们的党内来。要论证这一点,要把这一点加以论证,材料是充分的。现在我并不论证这些东西,因为要论证就要写文章,是要许多同志做工作的,我只是提一下。资产阶级的革命家进了共产党,资产阶级的世界观,他的立场没有改变,是完全可以理解的,就不可能不犯错误,这样同路人在各种紧要关头,不可能不犯错误。

庐山会议和这次会议,全国各级党组织都在那里讨论八届八中全会的决议,借这个事情来教育广大群众,使广大群众得到提高,更加觉悟起来。完全证明大多数人,全党干部绝大多数,比如95%是不赞成他们的,证明我们党是成熟的,表现出这些同志对于他们这个态度的对待。

资产阶级分子混进共产党里面来,我们共产党员中,资产阶级小资产阶级成分很多,应该加以分析,分为两部分,大多数他们是善良的,他们能够进入共产主义,因为他们愿意接受马克思主义。少数人大概是百分之一、二、三、四、五这样的数目,或者一、或者二、或者三、或者四、或者五。最近几个星期,省一级的会议暴露相当多的高级干部是右倾机会主义分子,在那里捣乱,惟恐天下不乱。凡是出了乱子,他们就高兴,他们的原则是这样:“天下太平,四方无事,工作顺利,他们就不舒服。一有点风吹草动,他们就高兴。”比如讲,猪肉不够,蔬菜不够,肥皂不够,女人头发卡子不够,乘机就来了。“你们的事情办得不好呀!”叫做“你们的”事情,不是他们的事情。说组织开会决定的时候他们不吭声,比如北戴河会议不吭声,郑州会议也不吭声,武昌会议也不吭声,上海会议吭了几句,我们听不到。等到后来事情发生了(他们认为事情发生了)你看又是蔬菜吧!又是猪肉吧!又是部分地区的粮食吧,又是肥皂吧,还有雨伞吧,比如浙江雨伞不够,叫做“比例失调”,“小资产阶级的狂热性”等等。少部分人他们要进入共产主义,要真正变成马克思主义者就是困难,我讲困难不是讲他们不可能,就是刘伯承同志讲过的,要脱胎换骨。当军阀的人他是当军阀了,还有不当军阀的人,比如×××同志算个什么军阀呀,是个文阀嘛,学阀哟!不脱胎换骨就进不了共产主义这个门。五次路线错误,立三路线错误,第一次王明路线;第二次王明路线,高饶路线,这一次彭、黄、张、周路线,有些人是五次,有些人不是五次,比如×××同志立三路线时候还没有来,就是彭、黄在立三路线的时候也是受打击的。这不是偶然的,五次路线的严重性。最后两次就是高、饶、彭、黄这两次,用阴谋的方法来分裂党,这是违反党纪的。马克思主义的政党要有纪律,他们不知道,列宁论无产阶级的党必须要有纪律,要有铁的纪律。对于这些同志是什么纪律呢?还是铁的纪律,还是钢的纪律,还是金、木、水、火、土,木头的纪律,还是豆腐的纪律?水的纪律就是没有纪律,还有什么铁的纪律呢?进行分裂活动,违反纪律,其目的、其结果,一定会是破坏无产阶级专政,建立另外个专政。

团结的旗帜非常重要。团结起来,马克思的口号:“全世界无产者联合起来!”他们不,他们的人似乎越少越好,他们要搞一个他们的集团,要办他们的事,违犯广大群众的意志。我在庐山会议讲了他们不讲团结的口号,因为这个口号一提,他们就不能进行活动了。这个口号对于他们不利,所以他们不敢提,所谓团结者,就包括了犯错误的人,要帮助他们改正错误,重新团结起来,何况没有犯错误的人?他们要去毁坏他们,他们是毁坏政策,不是团结政策,他们的旗帜是毁灭。毁灭跟他们的意见不对的,他们认为是坏人,而这个所谓坏人,实际上绝大多数,95%还要多。

要团结,就是要有纪律。为了全民族在几个五年计划之内建成强大的国家这一个目的。现在的任务是全国人民、全党在几个五年计划之内建成强大的国家,必须要有铁的纪律,没有铁的纪律是不行的,就必须团结起来。请问,不然怎么能达到这个目的呢?几个五年计划之内建成社会主义强国可能不可能?在过去要革命,在现在要建设,可不可能呢?没有铁的纪律都是不可能的。团结就要有纪律。彭德怀在太行山的许多文件,请同志们拿孙中山国民党第一次代表会议宣言,和彭德怀在太行山抗日时期发表的那些观点比较一下,一个是国民党人,一个是共产党人,时间一个是1924年,一个是1938年,1939年,1940年,共产党员比一个国民党员要退步,这个国民党人的名字叫孙中山,要进步。孙中山受共产党的影响,为什么发表那一篇呢?我最近找着看了一下孙中山国民党第一次代表会议宣言,那里面有阶级分析这样的思想。怎么会赞成共产党的铁的纪律呢?怎么会赞成无产阶级的纪律呢?没有共产党的语言,没有共同的立场观点,纪律是建立不起来的。我说彭德怀不如孙中山,至于张闻天也不如孙中山,孙中山那个时候是革命的,而这些同志是倒退的,是要把结成了团体破坏,提出的口号是有利于敌人,不利于阶级的,不利于人民的。这些观点还有一些,比如……。

绝对不可以背着祖国,里通外国。同志们开了会的,批判了这个东西。因为都是共产党的组织,马克思主义者。这一个集团来破坏那一个集团这是不许可的,我们不许可中国的党员去破坏外国的党组织,挑起一部分人来反对另一部分人,同时我们也不许可背着中央去接受外国的挑拨。……

我现在劝一劝犯错误的几位同志,你们要准备听闲话,我曾经劝过别人,比如罗炳辉同志,他那个时候犯过错误,他发非常大的脾气,我们后来劝他,你不要发脾气,你是犯了错误,你让人家讲,让人家讲到不想讲的时候。他不想讲的原因就是你改正了。你对人好,对自己的错误有自我批评精神,人家为什么要讲呢?他就不讲了。现在犯错误的同志,我劝你们要准备听闲话。一提起你们犯错误,不要触目惊心,准备人家讲你几年。我说长也不会,看你们改的情况,如果改得快,几个月就不讲了,改的慢,几年人家就不讲了,只要改,快慢都可以。要诚恳对人,不要讲假话,要老老实实,讲老实话。我劝犯错误的同志,你们要靠拢大多数,要跟大多数合作,不要只跟你们气味相投的少数合作。如果你们能实行这几条,第一你们能够听闲话,准备听,硬着头皮,你讲我就听,说你讲的对呀,我就是犯了那个错误呀!阿Q这个人有缺点的,缺点就表现在他那个头不那么漂亮,是个癞痢头,因为他就是讲不得,人家偏要讲,一讲他就发火,“亮了”他就发火。比如“那个癞痢头就放光”也讲不得,说“光了”,他就发火,“亮了”他就发火。作者描写一个不觉悟的纯朴的农民,阿Q是个好人,他并不组织宗派,但是那个人不觉悟,他是讲不得缺点,他没有主动。

你没有主动,大家就偏要讲。一讲就发火,发火就打架,打架打不赢,他就说“儿子打老子”。人家说:“阿Q,你要我不打你,你就讲‘老子打儿子’,我就不打你了。”“好,老子打儿子。”等到打他的人走了,他就说:“儿子打老子”,他又神气起来了。犯错误的同志要准备听闲话,多准备听一点。要对人老实诚恳,诚诚恳恳,对人不讲假话。再一个要靠拢大多数。只要有了这几条,我看是一定会改过来的。否则就改不过来。如果是闲话也听不得,对人也不诚恳,讲假话,又不靠拢大多数,那就难了。“人非圣贤,孰能无过?”其实这个话也不妥当,圣人也是有过的。“君子之过如日月之蚀也,其过也,人皆见之,其更也,人皆仰之”。我们不是孔夫子,我看孔夫子也有过,就是凡人多多少少、大大小小都是犯一点错误的!犯错误不要紧的,不要把犯错误当成一个大包袱,了不起。只要改。“君子之过也,如日月之蚀也”,好像天狗吃掉太阳月亮一样。犯错误人家都看见,如果改了人皆仰之。

我们大家要学点东西,要学习马克思主义。×××提出学习任务我非常赞成,这包括我们所有的人,都应该学。时间不够怎么办?时间不够可以挤时间。问题是要养成学习的习惯,就能够学下去。我这个话首先是对犯错误的同志说的。第二是对我们所有的同志(包括我在内)。许多东西我没有学,我这个人缺点很多,并不是一个完全的人,好些时候我自己不喜欢自己。马克思主义各部分的学问我没学好。比如说外国文,也没有学好。经济工作现在刚刚开始学习。但是,同志们,我决心学习,至死方休,死了拉倒。总而言之,活一天就学习一天,我们大家一起来造成一个学习环境,我想我也学一点,不然见马克思的时候我很难受,他出一些问题一问,我答不出来怎么办?他对中国革命各种事情一定感兴趣。还有自然科学也很不行的,技术科学也不行。现在学的东西很多怎么办呢?还是一样一样,多多少少学一点,钻一点,我说下了决心,一定可以学,不管年纪大小。我举个例子。游泳我是一九五四年才学好的,以前就没有学好。一九五四年清华大学有一个室内游泳池,每天晚上去,带个口罩化装,三个月不间断,我就把水的脾气研究了:水它是不会淹死人的呀!水怕人,不是人怕水。当然有些例外也存在,但是凡水该是可游的,这是个大前提。比如武汉长江有水,因此武汉长江是可以游泳的,我就驳了那些同志,反对我游长江的。我说你们形式逻辑都没有学,凡水都是可游的,除若干情况之外,比如说一寸之水就不能游,结了冰就不能游,有沙鱼的地方就不能游,有漩涡的地方(如四川、湖北的长江三峡)也不能游,除若干情况之外,凡水该是可游的,这是大前提。由实践得来的这个大前提。比如武汉长江是水,结论是武汉长江是可游的。比如汨罗江,珠江有水,是可游的,北戴河是可游的,它不是水吗?凡水该是可游的,这是大前提。除了一寸之水不可游,一百多温度不可游,零下之水结了冰不可游,有鲨鱼不可游,有漩涡不可游。除此之外,凡水都是可游的。这是个真理。这是个真理,你不信吗?下了决心,只要你有意志,下了决心,我看万事都可以做成功的。我劝同志们学习。最近我们看天安门大礼堂,咦!那可有点文章咧!你们去看一回好不好(会场高声答应:好!)叫××同志讲一讲,他这个人姓×,他一天跑一万里。只有十个月,许多人说不信,请个苏联专家说不信,到了今年六月,苏联专家说有可能,到了九月,他们大为佩服了,说中国确有大跃进。一万二千人,全国各地方调来的,全国各省的力量,技术力量,人的力量。完全不做礼拜天的,每天三班制,也不搞计件工资,许多人本来工作八小时,结果他做十二小时,不下工。多的四小时需不需要钱呢?他不要。还有一些人,工程没完成,他不下来,有的两天两晚不睡觉,坚持在那里,不是八小时,也不是十二小时,而是四十八小时就在工地上不下来。是不是要物质刺激呢?增加几块钱嘛,一小时一块钱嘛,他不要,这些人不要。物质刺激还是物质刺激,无非是平均工资五十块,就是那么一点,但是他们为着一个共同事业而奋斗。一万二千职工,十个月搞成功这么一大片,这里面不仅是按劳取酬,而且有列宁所谓“伟大的创举——共产主义礼拜六”,有不计报酬的在内,同志们,你们都看一下,并且请××同志给你们讲一下,不要多了,有半个小时就行,还有密云水库,我昨天到密云水库游了一回水,十九个县的人在那里搞的,十一个月完成全部工程百分之七十,二千五百万土方,二十万人,那也了不起呀,什么情况都要算钱呀!你们在那里开会每天算钱嘛,你们写了很多东西,讲话稿子,恐怕每篇都要三块大洋嘛,我现在并不否认,而且肯定按劳取酬,必要的按劳取酬是完全正确的,但是不可以完全禁止,稍超过一点就不行,如果超过一点就给钱,人民他不要,你算钱他不要,如果要算钱,我今天大概讲一个钟头吧!你们给钱咧!政治挂帅与物质刺激两个东西,政治的作用必需与按劳取酬结合,我看这是个好东西。我们凡是下了决心,有坚决的意志,人们认为不能成功的,结果他成功了,就是我们这个大礼堂,很多人认为不能成功嘛,我们的大跃进,人民公社很多人都写,它要成功的,并且已经成功,或者在继续取得成绩。比如钢铁是要快,工业是要快,农业也是要快,学习也是这样,只要我们下决心,我看可以学好,不怕事务太多,时间不多可以挤,养成这个习惯。我们要战胜这个地球,我们的对象就是地球。至于太阳上怎么作工作,我们暂时不论,月亮、水星、金星,除了地球之外的八大行星,将来探一探可以,拜访拜访可以,假如能上去。至于工作,我们打仗,我看还是地球。建立一个强国,一定要有这样的决心。要求我们建立大礼堂,很多水坝,很多工厂,我看一定要是这样。

全党全民团结起来!全世界无产阶级团结起来!我们的目的一定可以达到!


