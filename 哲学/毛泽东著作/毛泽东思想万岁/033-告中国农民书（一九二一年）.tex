\section{告中国农民书(一九二一年)}


有人说中国农民的生活,并不是痛苦的,也不是受十分压制的,因为所谓农民,都是自己有着田地,自己耕种的,并不是单签耕人家的田而谋生的。就是耕人家的田,而所得生产物的分配乃是平分的,所以也没有什么分配不平均。既然这样,你就向他们去宣传,也断不能促进他们的自觉,这话都不然,我现在只要把农民的状况记述出来,就可证明这一说的理由不充足了,不过我现在记述的农民状况,是就我住的那一县和附近那各县状况而论的,我想就以此推向全国,也不过是大同小异罢了。

一、农民自身里面的阶级。有人说中国的农民,都是各有土地的,这句话是有一部分确实的,然而未免太笼统了。一家三人所有千亩田,算是有土地;一家十口人所有一亩田也不能不算是所有土地,然而你们这样的所有,就说农民之间,生活都是一样,没有什么特别的痛苦吗?设若细细地考虑起来,就可知农民自身内面,也有几层阶级:

(一)所有多数田地,自己不耕种,或雇人耕种,或租给人耕种,自己坐家收租,这种人并算不得纯粹的孜民,我乡下叫做“土财主”。

(二)自己所有的土地自己耕种,而以这个土地的出产可以养活全家,他们也有于自己的土地之外租人家的土地耕种的,这一种人就是中等农民。

(三)自己也有一点土地,然而只靠自己土地的出产,绝不能养活全家的,所以不得不靠着耕人家的田,分得一种而自瞻。这一种人也可谓下级的农民了。

(四)这乃是“穷光蛋”,自己连插钉的地方也没有,专靠耕人家的田谋生活的,这一种人就是贫穷的农民了。上述四种里面以第三种和第四种的人数为最多。第一种人当然是少的,第二种也是很少的。第一,二种的生活是丰衣足食的,不是我们问题的目的物。我们的目的物,乃是占农民全数内面的大多数的第三、四种农民。第四种农民的苦况简直是非常励害,每天到晚,终年到头的苦作,还不够穿衣吃饭,一过年岁不好,田主顽强(分配的方法面详说)的时候,就差不多要饿死。所以这种农民的生活是非常的痛苦的。第三种农民,虽然自己有一点土地,还耕一点人家的田,然而因为生活程度高,不是东拉西扯地来借贷,也是不能维持安家的生活。所以每到收谷的时候,谷总不能全进到自己家里来,直接运往债主家里去还帐,或还利息了去。因为这种原因,自己所有的一点田,也不得不渐渐卖或当给“土财主”或中等农民(在后面地余中里面详说),而坠为第四种农民了,所以他们的生活也是极困苦的。

照这样看起来,可见得大多数农民的生活是非常困难的,只拿着农民都是自己有土地这句话,说农民的生活都没有什么困难的先生们,简直是瞎说。这种说劳动者是和资本家同样地有收入,他的生活……

\begin{flushright}(湖南自修大学设补习学校国文讲义摘录)\end{flushright}

