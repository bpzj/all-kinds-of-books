\section[查田运动是广大区域内的中心重大任务(一九三三年)]{查田运动是广大区域内的中心重大任务}
\datesubtitle{(一九三三年)}


一切过去的经验都证明:只有土地问题的正确解决,只有在坚决的阶级口号之下,把农村中阶级斗争的火焰掀起到最高的程度,才能发动广大的农民群众起来,在无产阶级领导下,参加苏维埃各方面的建设,建立巩固的革命根据地,使苏维埃运动得着更大的力量,争取更大的发展与胜利。

依着土地革命发展的经验,农村中阶级斗争的发展,是有它的大致的阶段的,就是(一)按收分配土地的阶段,(二)检查土地的阶段,(三)土地建设的阶段。依照土地斗争的三个发展阶段,在任何苏区之内,大致都有三种区域的存在,就是(一)新发展区域,(二)斗争比较落后区域,(三)斗争深入区域。

在新发展区域,土地斗争的发展,还在没收与分配土地的阶段,这里的中心问题,是以武力推翻地主阶级的政权,建立革命的群众团体,没收地主阶级的土地财产和富农的土地,分配土地给雇农、贫农、中农……废除债物,焚毁田契与借约。这个阶段中的斗争,是包括这些区域革命与反革命开始接战以至革命打败反革命而实行处理他们的土地财产的这一整个时期。

在斗争深入区域,这里已经建立了巩固的苏维埃政权、地方武装与革命群众团体是广大的发展了,地主富农的封建半封建的势力已经完全克服下去,土地已经彻底分配好了,农民群众在土地问题上的斗争已经进到了改良土地发展土地生产的阶段。所以这里的中心问题是土地建设问题。

在斗争比较落后的区域,其发展阶段介在上说两种阶段之间,即是从临时政府时期进到了正式政权的时期,但还没有到政权的完全巩固时期。这里地主富农公开的反革命斗争,已经在第一时期被革命群众打败了,他们中间许多分子,就从那时候起,摇身一变,把自己的反革命面具取下来,带上革命的面具,也赞成分田,自算贫苦的农民,照例应分土地。他们积极地活动,凭借他们历史的优点,“说也是他们会说,写也是他们会写,”所以他们就在第一时期中乘机偷取土地革命的果实。无数地方的事实证明他们在把持临时政权,钻进地方武装,操纵革命团体,有时候比贫苦农民分得更多更好的土地。及至进到第二时期,因为上级政府的督促与群众斗争的发展,改造了革命委员会为苏维埃,群众团体与地方武装已经过了第一步的改造与发展,那些假装革命的分子,部分的被洗刷出去。许多地方的土地实行了第二次分配,甚至第三第四次的分配,地主富农偷取的土地也清除一部分来。但是苏维埃群众团体与地方武装中间,还依然躲藏着许多阶级异己分子,在那里“称红带子算同志”,在那里造谣言,开私会,在那里骂群众为“左倾机会主义”“乱打土豪”、‘公报私仇”,或者他们“开会说得有劲,闭会一事不行”,当着斗争剧烈的时候,他们组织反革命的秘密团体,如国民党、社会民主党、AB团、新“共产党”以及各种各色的东西来破坏革命、谋害革命的积极分子。总之地主富农阶级用各种方法来压制群众的斗争,企图保持他们政权上的与土地财产上的权力,保持他们的残余的封建势力。在这些区域里、革命群众与地主富农之间是严重的斗争着,但这里的斗争,不是象第一时期里红旗子同白旗子的公开的斗争了,而是革命的农民群众同戴假面具的地主富农分子的斗争。这种斗争有他的一种特别的困难,就是暗藏的反革命不比公开的反革命,农民群众能够一眼看清楚。加上农村中各种根深蒂固的封建关系如民族关系等。要使农民群众阶级觉悟程度,一般都认识到应该最后消灭封建残余,不是一件容易的事。这就共产党与苏维埃政权一定要耐心的去向农民解释,一定做许多艰苦的工作,要有正确的阶级路线与群众工作方法。这里的中心问题,就是查田查阶级的问题,这个问题不解决,农民群众的革命性就不能最大限度的发展起来,封建残余势力不能完全打倒下去,苏维埃不能得到最大限度的巩固,扩大红军,筹备经费供给红军,扩大地方武装,进行土地建设与经济建设,发展文化教育等等重大任务,都没有法子得到最大的成功。所以查田运动是这些区域里的最中心最重大的任务。

拿中中央苏区来说,这种斗争比较落后的区域,占了全苏区的大部分,会昌、寻邬、安远、信丰、乐安、宣黄、广昌、石城、建宁、黎川、宁化、长汀、武平十三个县,瑞金、雩都、博生、胜利、永丰的大部分,公略、万泰、赣县、上杭、永定、新泉的部分,就是兴国也还有均村、黄塘两区,所有这些都是土地问题没有彻底解决的地方。

这些地方的农民群众,还没有最广大的发动起来,封建势力还没有最后的克服下去,苏维埃政权中,群众团体中,地方武装中,还有不少的阶级异己务子在暗藏着在活动着,还有不少的反革命秘密组织,在各地暗中活动破坏革命。为了这些原故,这些地方的战争动员与经济建设都远落在先进区域(兴国差不多全县,胜利、赣县、万泰、公略、永丰、上杭的一部分,博生的黄阪区,瑞金的武阳区,零都的新陂区,永定的溪南区等)之后,这种地方竟占了中央区差不多百分之八十的面积,群众在二百万以上。在这个广大区域内进行普遍的深入的查田运动,在二百万以上的群众中燃烧起最高度的阶级斗争火焰,向封建势力作最后一次的战争,而把他们完全打倒下去,是共产党与苏维埃政府一刻不容再缓的任务。

