\section[在山东省级机关党员干部会上的报告(一九五七年三月十八日)]{在山东省级机关党员干部会上的报告}
\datesubtitle{(一九五七年三月十八日)}


刚才舒同同志给我出了题目,要我讲些什么事情。出了题目就要做。但是他出的题目很多,那怎么办?那么多题目,我只讲一个题目:思想问题。

去年下半年以来,我们党里头,社会上,人们的思想有一些乱比如讲:批评共产党的人多了,党外人士比过去敢于讲话了,敢于讲我们党的缺点了(同志们最好不要记录,记录就妨碍听,我倒不是要守什么秘密,我给你们交换意见,你们记人就忙了不记舒服些)。说共产党不能领导科学,对社会主义有没有优越性也发生怀疑,有一些人讲社会主义没有优越性,合作社办得不好。也的确有些合作社办得不好,共产党也的确不大懂科学。还有苏联过去是比较好,去年以来就此较坏了,不晓得是个什么道理?总而言之今年比较好,去年比较坏。今年又好起来一点,苏联还是那么一个苏联。有些知识分子讲美国比中国好,美国比苏联好,美国的钢此苏联出得多。此外,人民闹事
也发生得不少。去年这一年,特别是下半年(不晓得山东情况如何,听说也有一些),罢工、罢课、游行示威,纪律有些松弛。工厂里头,学校里头的纪律,合作社出勤的纪律,不是一般不好,一般还算好,只有一些不好。报纸上,不晓得你们山东报纸怎么样,北京、天津、上海的刊物上、报纸上,小品文,讥笑、讽刺文章多起来了,批评缺点、冷嘲热讽。在这样一种情况之下,就看得出像舒同同志所讲,对合作社有些人就看得是不好,没有优越性,一片黑暗;有些人员觉得很好,很好就不注意它的缺点了。这一些情况下,就是有一些共产党人也跟着别人走,共产党员,共青团员,有些跟着资产阶级走,就否定一切,不加分析,把情况看得比较坏些。另外一些共产党人就不服气,说这就是“百花齐放,百家争鸣”这两条闹出来的,你们这里有没有这个事?搞不搞这个“百花齐放”?别的地方是搞了一下子,“百家齐放”,刚刚只放了十几朵花,(笑声)还没有放齐,“百家争鸣”,大概有一二十家出来争一争,放的中间,争的中间就说一些事情不那么好了。有一些同志就看不顺眼。他们就想收起来,说放得多了,要收一下,就是不要放了,或者少放一些,“百家争鸣”也收一下,主张收的这么一种思想,所以我们党里头有两种人,一种人就是外面讲什么我们也讲什么,踉着人家走;一种人就想收,不听话的时候就想压一下子。这是现在的情况,是个内因。当然外国的事情也影响我们。二十次代表大会把斯大林批评了一下,波兰、匈牙利的事件,国际上有反苏反共的风潮,铁托的演说,卡达尔的演说,(山东的报纸登这篇文章没有?)引起一些思想上的混乱。还有,我们工作中有许多错误,无论是肃反、土改、合作化、社会主义改造这些工作中间,我们有主观主义的错误,有官僚主义的错误,也有宗派主义的错误,这样引起思想上有一些混乱。什么大混乱没有,我们中国,我说没有外国那么乱,当然就没有波兰、匈牙利那么乱了。比起其他社会主义国家也没有他们那么乱,譬如越南就很乱了一阵子。我们现在来分析一下原因,就是有外国的。但主要是由于我们自己处在这么一个时代,处在这样一个条件之下。就是大规模的阶级斗争基本上结束,社会主义改造基本上完成。八次大会作了结论的,这些结论是合乎情况的。我们过去全国反对蒋介石,解放战争,这是大规模的阶级斗争,土改,镇压反革命,抗美援朝,这是大规模的阶级斗争。社会主义也是阶级斗争。这是向什么阶级作斗争?向资产阶级作斗争,向跟资产阶级有联结的那个个体经济。我们要消灭资产阶级同那个资本主义发生基础的那个个体经济,这么一场大斗争,到去年上半年基本上结束。这个大斗争的结束,那么人民内部的问题就显出来了,基本的原因还是这个原因。人民内部的问题就多起来了,就暴露出来了,就暴露出许多思想问题了,思想上就有一些乱了。过去是不是有思想问题呢?我说过去有的,特别是我们新进城来,解放大城市那个时期,同志们都经过,一九四九年、五0年、五一年、五二年、五三年那几年,那几年还不乱?还是那几年乱些?还是现在乱些呢?实际上,是那几年乱些,你像资产阶级那个时候他们十五个吊桶打水七上八下,就是心里不安。但是过去那些不同的意见分歧就被大规模的阶级斗争所掩盖,吓住了,被解放战争、剿匪、抗美援朝、镇压反革命、土地改革那些东西所吓住了,许多人不敢开腔了。并非没有问题,那个时候问题还很多,我们解决了不少问题。比如。民主人士这六、七年有很大进步,就是因为在这些斗争里头我们跟他们合作,应该承认他们是有进步的。去年上半年阶级斗争基本结束。

所谓基本结束。就是说还有阶级斗争,特别表现在意识形态这一方面,只是基本结束,不是全部结束。这一点要讲清楚,不要误会。这一个尾巴也掉得很长的,特别是意识形态这一方面的阶级斗争,就是无产阶级思想跟资产阶级思想作斗争。一家是无产阶级,我说不是“百家争鸣”,而是两家争鸣,这百家里头有两家,一家是资产阶级,这个争鸣是要争几十年的。

所以现在正确处理人民内部矛盾的问题被提到议事日程上面来了,正确处理人民内部的矛盾,不是那个大规模的阶级斗争,刚才讲的,有阶级斗争,特别是表现在意识形态里面的,我们是把它当做内部矛盾来处理。对民族资产阶级我们把他当做内部矛盾来处理,不把他当做一个国民党特务那样的一个问题来处理。资本家呀你们这里苗海南呀,我们把苗海南跟蒋介石区别,跟特务区别,我们说他不是特务,也不是蒋介石,他就是苗海南。我们跟他合作,这一来,他说可以,他说愿意跟我们合作,咱们两个人就好办事,我们也愿意了,你们也愿意。前面刚才谈到了,“百花齐放、百家争鸣”还有“长期共存、互相临督”这样的方针,在我们党里头还有相当多的同志不甚了解。有一些同志就是不大赞成这样的方针,究竟在座的同志赞成不赞成我也不清楚。因为我刚刚来,平时我们又不在一道工作。我在别的地方看的时候,比如北京,许多高级的同志,部长,我说十个人里头可能有一个人赞成,一个人想通了,真正其他九个有些相当赞成,但是不那么十分赞成,各种程度不同。至于什么厅、局长,什么科长,这样几级的同志,开头是表示怀疑的多,什么“百花齐放”,放那么多花,(笑声)百家争鸣,那个是危险得很,咱们共产党就是一家,其他九十九家把我们包围,出来怎么得了,(笑声)要请解放军帮忙,杀开一条出路,杀出一条血路,才跑的出去哩。长期共存也是不赞成,那个民主党派大概有个什么七八年也就差不多了嘛,让他挖一个坑埋下去嘛,究竟谁监督谁,还要请他监督共产党呀!你有什么资格监督共产党呀?这些同志,你说他没有道理?民主党派有什么权利监督共产党?究竟江山是谁打下来的呀?还是工人阶级、农民打来的,共产党领导他们,还是你们民主党派打来的?所以听他们的话是有不少道理的。但是还是要采取这样的方针比较好。那有什么道理有什么理由呢?“百花齐放,百家争鸣”这是一种方法,也是一条方针。它是一种使得文学艺术,使得科学能够繁荣起来的一种方法。不管你多少花,你就得开嘛,那么其中有好看的花,有不好看的花,有很丑的花有毒草都可以开。那么毒草开出来怎么得了呢?世界上是有毒草的,人怎么样呢?是不是碰了毒草就死了?这样就有比较,有比较好办,不是那么简单的,而是很复杂的。各种的花都可以开(各种的艺术)。人们就说,是否这样一来鬼就出来了,牛鬼蛇神就跑到戏台上来了。你们这些地方戏演得怎样?有没有牛鬼蛇神?别的地方就有,特别是上海,曾经不演许多的落后的东西,在戏台上打屁股,演包公的时候一定要在戏台上打屁股,现在我们不是讲废止肉刑吗?但是我们的包公仍然在台上要打屁股的。这些东西慢慢地会要淘汰的了。现在让他演一演可以不可以呢?让他演一下我看也可以。这些东西多了,人们就会说话,说话的多了,看戏的人就少了,它那个东西就不演了。过去我们生硬地禁止一下,以行政命令的办法禁止他们演那些戏,就不如让他们互相竞争,“百花齐放”这样的办法比较好。至于“长期共存,互相监督”,这就是说:我们党就是因为功劳太大的原因,工人、农民的政党,我们党主要成分是工人同贫农的党,按他们的性质是无产阶级的先锋队,因为在中国威望很大,这个威望太大,就发生一个危险,容易包办代替,以简单的行政命分,反正是人多嘛,社会上的威望也大嘛。所以我们特为请那么几位来监督我们,并且长期共存,我们有天,他们也有一天。我们就不必说,我们假使共产党有一百年,只准他们五十年,共产党还有五十年,只准他有二十四年。总而言之是要他们先死,先死几十年就是了,有没有那样的必要?它早死了我们的事情就好办了?是不是呢?粮食就多了?(笑声)钢铁就多了?木材就多了?水泥就多了?礼堂就砌得更好了?并不见得。民主党派灭亡了,我们礼堂砌得更好了,不能证明这个道理。就是有一些民主党派跟我们唱对台戏比较好,就是说怪话,一年有那么几次找一些怪话来说,专门指出我们的缺点,这么一个道理。所以现在不是收起来,而是还要放,现在还是放得不够,不是把它压下去,而是不要压。思想的问题,精神方面的问题,不是用粗暴的方法和压服的方法能够解决的。我们应该大家展开民主的讨论,平等的讨论,互相争辩,这样的方法就是用说服的方法,不是用压服的方法。两个方法采取那一种?一个是压服,一个是说服。这个压服的方法是对付敌人的方法。我们打过仗的都知道,从井岗山那时候就打起嘛,什么大别山、什么各地方,对付敌人只能用这个方法。对付美国人在北朝鲜,就是抗美援朝,用压服的方法。对付反革命、特务用什么方法?我们就是压服它嘛,以后再来说服,先把它捣出来。压服这个办法是对付敌人的,解决敌我矛盾的方法,就是动手,人民内部就不是动手,君子动口不动手,(笑声)人民内部就是讲道理,就不讲打,武力解决就不是个办法。或者是用行政命令的办法强迫禁止,实际上是把一个解放军摆在这边,名为不用武力,实际上是它在这边我们用强迫命令,如果没有解放军,这个行政命令也就不行的。谁听你这个行政命令?还是借他们的声势来搞行政命令,我们后面有个解放军,几百万,还有那个老百姓,我们有了基本群众,就是工人、农民。第一我们有基本群众,第二我们有武装力量,所以那些民主人士吃不开也就是这个道理,他们没有这两个东西,一没有基本群众,二没有武装力量。如果我们是采取了开放的方法,而不是把它收拢来,采取说服的方法,那么我们的国家就会兴盛起来。因为现在是人民内部的矛盾,不采取对付敌人的方法来对付人民内部,不用对付敌人的办法对付人民,要把这两个东西分开。专政是什么呢?我们不是讲无产阶级专政吗?专政是对付敌人的,而民主就是对付人民的。人民内部的相互关系是种民主的关系。当然民主要集中领导下的民主,不是讲不要领导。现在有些地方纪律松弛,缺乏纪律,这就有些过分。要克服这个问题,要用说服的方法,同他们好好的讲,开会,一次开不清楚开两次、三次会。跟学生们、工人们、农民们能不能说服呢?我们是能说服的。只要我们有理,就是用说理的方法,讲理的方法。如果我们搞错了,把专政的范围扩大到人民内部,用压服的方法,凡是有矛盾,有问题的时候就用压服这么个办法,那么我们国家就要受到损失,就要受到很大的损失。而且总有一天要回过头来,压服不了的时候。君子动口不动手,你要动手,总有一天这个手要收回来的。因为这不是解决人民内部矛盾的办法,那是解决敌我斗争的办法。是不是可怕?人民内部的矛盾发展起来了,又不要压服,不要行政命令,这样一来,是不是很危险?由我看,没有什么危险。不同的意见只会因为辩论、民主的讨论而正确地解决,得出真理,艺术方面会更活泼,会发扬创造性,文学艺术,科学方面会发展起来。这种功效不是一年、两年看得出来的,或者几十年内,十几年到几十年内看得出他的效力来的。这里就要准备着真正的坏人坏意见,就是讲:艺术方面有毒草,有毒的草,很不好看的花开出来了,开出来怎么办?我们说那东西也有用处,因为让世人看一看有这样的草,有这样的花。一个人硬是每天只都是看见好的不看见坏的,真理是跟错误作斗争中发展起来的。我们不让错误的意见说出来,我们就不知道他的意见。这个美是跟丑作斗争发展起来的。好人是跟坏人作斗争发展起来的。一万年以后都有真理跟错误,都有美跟丑,都有好人跟坏人。有一些坏人,他是坏人,是不好,但是,又是好人的先生。有了坏人,人们就有了榜样了,就不学他们的坏了。有好人就学好人,单有好人没有坏人比较,人们分不清楚好坏。几岁的小孩子看戏的时候,他首先要问好人坏人,(笑声)所以不怕那个坏人,什么错误的道理那些东西,用不着怕。这样我们党跟政府的错误也比较容易克服。相反呢,倒不好的,倒是很可怕的,就是用压的办法。

我们党习惯了对敌斗争,我们党搞了几十年的阶级斗争,我们会搞这一套,结果证明我们是会搞的,是我们胜利了,几十年嘛!所以人们也佩服我们这一条。觉得共产党有什么行呢?你们政治不错,军事也可以。我们有这两条人家佩服的。那么,是不是一开始就佩服呢?那就不见得。我们刚成立党的那个时候谁也不佩服我们,我们那个时候提出口号,打倒帝国主义,打倒封建主义,打倒军阀,人们只是听了一下,根本不理。后头又犯了错误,搞万里长征,稀稀拉拉剩下几个大人,我们那时候提出打倒帝国主义那一套,人们也是不大理。以后就搞出名堂来了,到一九四八年那个时候,社会上许多人就变了。共产党似乎兴起来了,共产党就了不起了。特别是到了一九四九年胜负分明了,那就佩服的更多了。但是,同志们,建设人家不佩服,共产党能够搞建设呀?人家还要看一看,看来是行的,这些人就会搞那么几手,有那么一股蛮劲。(笑声)谁佩服我们会搞建设啊!现在在建设问题上面慢慢立起了一些信任,看你这个样子搞了六、七年也还有一点本领。但是讲到科学那是另外一回事情,物理、化学、数学之类,他说你们不行,还是我们来。这个东西怎么办?你们看怎么办?你们在座的都是科学家,是不是?(笑声)我们党科学家不多,其原因就是科学家不多,大学教授不多,工程师不多,各种艺术家有一点也少,就是这些方面不行。大学校长是党外人士当的多,山东大学什么人当校长?(问下面,回答:晁哲甫),晁哲甫是党内党外的?(问下面,回答:是党内的)党内大学校长还是要听党外人士的话。你没有教授,(笑声)有些当了付校长,人家党外人士不佩服。我们的党员就是学生、助教多,讲师里面有一点也少,教授里头很少。那么究竟是学生领导先生,还是先生领导学生呢?是助教领导教授,还是教授领导助教呢?所以这个问题发生得很自然。我说这个道理是有,他们讲得对,共产党就是不行,就是在教授、科学家、工程师这方面我们人很少,甚至没有。还有许多文学艺术这一方面。什么原因?就是我们过去忙坏了,我们干了几十年的阶级斗争,忙于搞阶级斗争,没有机会搞这个。过去你这个东西要有个地盘嘛,过去济南这些地方人家不许我们来嘛,什么山东大学、齐鲁大学他不许我们来嘛,来都不许来嘛。北京我是住过的,三十一年我是不能来的,就是不能进城,进了城他就要请我到班房里面去,(笑声)所以我们承认这些,但是可以不可以学会呢?阶级斗争我们学了二十四年,到了七大,一九二一年,到一九四五年,二十四年才作总结,我们才基本上学会了这个东西。犯过许多错误,又加上七大以来的这么多年。那么,我们搞建设、搞科学、学当教授、医院里面当医生、开方,是不是可以学会呢?比阶级斗争哪样困难些呢?阶级斗争那个东西我看是比较困难的一个东西,那个东西你一打它就跑了。(笑声)这个开刀好开,那个病人他不能跑的。(笑声)自然科学,五年大学毕业,再干五年就是十年,再干五年就是十五年,他不变成工程师呀,这些人他不能当大学教授呀?可以当的,有个十年到十五年可以学会的。现在设计、施工、安装我们已经学会相当大的一部分了,管工厂也相当学会一些了。阶级斗争基本结束,我们的任务转到什么地方呢?就要转入到学这些,搞建设。所以整个社会与自然作斗争,所以六亿人口与自然界作斗争,要把中国兴盛起来,变成一个工业国。可以学会的。不要怕“百花齐放,百家争鸣”“长期共存,互相监督”。有批评是好的,没有批评,压制批评就不好。斯大林就是犯这个错误。斯大林做了许多好事,但是他做了一些坏事,他混合这两者,拿对付敌人的方法来对付人民,人民内部的矛盾,说不得政府的坏话,说不得共产党的坏话,一说坏话,风吹草动,就说你是特务,把你抓起来。所以现在我们不提倡罢工、罢课、示威游行、请愿。但是我们要反对官僚主义,要克服官僚主义,那什么罢工、罢课这些东西就会少的,但是怎样也要有。那怎么办?有就用对付人民内部的方法来对付。那么还要不要解放军?那么就把他解散,是不是?那行不行?“养兵千曰,用在一朝”,准备对付帝国主义,解放军是用来对付帝国主义的,不是为了对付人民的。解放军就是人民的儿子,那么人民的儿子对付人民父亲啊?(笑声)你总不能讲吧,人民的儿子对付人民啦!人民打人民啦!那不行。解放军是阶级斗争的工具,他是专政的武器。我们要跟国民党有区别,什么人比较怕批评,还是共产党怕批评还是国民党怕批评?我看就是国民党,他那个党最怕批评了,什么“百花齐放”这类东西,他们怕得很,只有我们敢提出来“百花齐放,百家争鸣”,“长期共存”,我们说要跟他们共存几年他都不干。那个时候他们开一个什么国民党参政会,我也是个参政员,共产党有几个参政员,我们是以什么资格呢?不是以共产党的代表,而叫做社会贤达,(笑声)好听就是了,又贤又达了。(笑声)他不承认我们是共产党的代表。你还说长期共存,短期共存他都不干。(笑声)所以这样的口号“百花齐放,百家争鸣”“长期共存,互相监督”只有我们讲。无产阶级的政党是比较大公无私的,因为它的目的是要使全人类解放,要使整个人类解放,它自己才能解放。共产党是应当最不怕批评的,我们是批不倒的,风吹不倒的。十二级台风我们都不怕,十二级台风可以把一个大礼堂吹走,可以把这么大的(手势约二尺直径)树吹走,但是要吹走共产党,人民政府,马克思主义,老干部,我看是吹不走的,吹不倒的,(笑声)十二级台风都吹不倒,何况是什么五、六级、六、七级。你们到过青岛没有?我没有去过,我想去一下,我到过北戴河,七级风在海里游泳很舒服,平时没有风浪很吃力,要一步一步地爬,起台风一个浪头就有一两个人高。十二级台风我没见过。批评不倒我们的,批评不倒共产党,工人阶级、农民是批评不倒的。因为真理在我们手里,我们手里掌握着真理,比较任何阶级我们的道理是更正确的。我们有工人、农民、基本群众,有工人、农民基本群众作基础,这样的党,这样的政府怎么能批倒呢?马克思主又怎么能一批评就倒了呢?吹了一口小气就倒了,那就是官僚主义。官僚主义,大概是不要台风,这么招一下手就要倒的。(笑声)官僚主义、主观主义、宗派主义那一套东西吹倒一点我看也好吗,那套东西是要吹倒嘛,把官僚主义、主观主义这些东西吹掉一些。

刚才舒同同志提到社会主义优越性这个问题,因为有些人怀疑社会主义的优越性。一讲优越性尽是优越性,缺点一点都没有。一讲没有优越性就一点好处也没有。这是两种片面性。我们党内存在着,党外也存在着。两种制度作斗争,那一个胜那一个败,就是讲社会主义资本主义这两种制度谁胜谁败的问题,解决了没有呢?你们看,分了胜负没有呢?按照我们八次大会所说的,应该说基本上是分了胜负的。谁胜谁负,就是资本主义失败了,社会主义胜利了。基本胜利了是不是最后胜利了呢?那就没有。作为社会制度,这两个制度的竞争,社会主义基本上胜利了,但是还没有最后胜利,还没有巩固,还要看,人们还要看,资本家要看,农民还要看,资产阶级要看,小资产阶级农民要看,城市小资产阶级手工业者还要看,我们共产党里头有一部分人也还要看。两种制度作斗争谁胜谁负,那个胜那个败,基本上可以说胜利了,最后胜利还要有一个时期,大概有两、三个五年计划。合作社至少要有五年才可以巩固,现在一般的合作社才只有一年多一点的历史。至于两种思想的斗争,资产阶级思想,无产阶级思想,马克思主义,非马克思主义,这样两种思想的斗争,意识形态方面谁负谁胜那么就更要差一点的。所以现在相当乱呀,思想方面,我今天讲的总题目叫作思想问题,这是有理由的。尽管社会制度起了变化,但是,思想还是相当顽固的保守着。特别是关于世界观这一方而,就是资产阶级的世界观,还是无产阶级的世界观,还是唯物论,还是唯心论,还是辩证唯物论,还是形而上学的唯心论,或者是形而上学的唯物论,这样两种思想方面的斗争,那么这个时期还要更长些。

现在我们全国有多少知识分子呢?大概有五百万这样一个数目,其中学校里头:大学、中学、小学就有二百万人。此外,党的系统,政府的系统,军队的系统里头的知识分子,经济系统,商业系统,工业系统里面的知识分子,文学方面的,艺术方面的知识分子,合起来有人说有五百万这样多。其中相信马克思主义的只占少数。我们国家是文化落后的国家,但是五百万人也就不少了。因此,我们要好好利用知识分子的队伍。这批知识分子可以说都是资产阶级知识分子,进过资产阶级学校,我就是这样一个人,就是资产阶级知识分子。进过资产阶级学校,受过资产阶级社会的影响。至于本人是无产阶级知识分子,是后来的事。我想在座许多同志们中间的知识分子,也是这么一种情形。你母亲生下来的时候并没有交付你一个任务:要当共产党,要信马克思主义。我也有这样的经验,我母亲生我的时候,他并没有讲这一句话的,(笑声)他就不知道世界上有马克思主义,有共产党。这是因为后头社会斗争逼上梁山,我那个粱山叫井岗山。你们各有各的梁山。这个梁山就在你们山东。真正相信马克思主义的是少数,大概百分之十左右,五百万里头大概有五十万左右,也许多一点,这是讲真正了解了马克思主义的。另外总有百分之几是根本反对马克思主义的,跟我们采取敌对的态度的,但他们不是特务。这一些人是民主人士,但是他心怀不满,根本反对我们,有没有呢?我看是有的,这也是少数。中间百分之八十以上是中间派,真正的马克思主义不过寥寥,叫他下乡没有兴趣,到工厂里面看看,蹭一溜就回来了。跟工人农民打成一片,他们打成两片。我们说打成一片。他们说打成两片好,搞不成一片,就因为感情不融洽,知识分子跟劳动人民感情不融洽,隔那么一层。这些人世界观没有改,根本原因是资产阶级世界观,这个思想的乱,就是这样。他们为什么思想乱呢?他们动摇着,墙上的草,风吹两边倒,不风吹就站在那里,一风吹就倒,匈牙利一股风,苏联二十次党代表大会一股风,还有什么地方一股风。世界上总是要刮风的嘛。他们可以倒过去,也可以倒过来。又很骄傲,那个尾巴翘得相当高,他认为他是知识分子,相当了不起。在我们这个国家,知识分子是值钱的。所以在世界观这个问题上,真正无产阶级知识分子不是他们,这些还是资产阶级小资产阶级知识分子。总的范畴是资产阶级知识分子。有些人叫他小资产阶级知识分子还可以,你叫他资产阶级知识分子他不愿意,但是基本范畴还是资产阶级范畴。这个时期大概要经过三个五年计划(已经过了几年了),大概还要经过十几年,马克思主义就可以取得决定性胜利。我们要争取他们。要知道现在教书的是谁?就是他们。他们都是教员,他们办报纸,他们在学校里头教书:他是我们经济机关设计、办各种事情的人,他们是工程师,我们是一天也离不开他们。在座的同志们,你们里面也有知识分子,但是你们里面的知识分子比较少,你们如果说,知识分子我们可以离开他们,那就不行,我们离不开他们,没有人教书,大学、中学、小学里头有二百万人工作,教书,有各种报纸,有文学,有艺术。山东唱武生的,山东大鼓,那不算知识分子吧,但梅兰芳要算,梅兰芳是知识分子。你们有个山东梅兰芳(叫什么名字,你们山东人,不知道山东梅兰芳吗?)

我再来讲一讲工人阶级知识分子这个问题。工人阶级在领导革命的过程中,已经初步地争取了一批知识分子为自己服务,马克思就是这样争取过去的。马克思、恩格斯、列宁这些人都是资产阶级知识分子,替工人阶级服务。我们中国的知识分子也是如此。现在有一批懂得马克思主义的,刚才不是讲过了,五百万中百分之十左右,就是五十万左右,这是知识分子里面的核心力量。

各种阶级,哪一个阶级有前途呢?只有工人阶级有前途。我们现在有多少工人阶级呢?革命以前,解放以前只有几百万人,这几年增加到一千二百万人。现在在工厂里头工作的产业工人有一千二百万人,六亿人口,只有一千二百万人,就是五十分之一多一点,五十人里头只有一个工人,人数这样少,但只有他有前途。其它的阶级都是过渡阶级,就是走路,走路走到那里,走到他那个地方去。比如农民,农民将来要机械化变成农业工人,集体化,集体化将来进到社会主义,就要变成国有化,全民所有制。现在的农业合作社所有制,将来几十年之后,要改成跟工厂一样,就是农业工厂,这个工厂里头种包谷、种小米、种大米、种番薯、种花生、大豆,资产阶级这大家都知道的,资产阶级现在正在过渡,他们也要变成工人。几亿人口的农民,手工业工人现在已经变成集体农民,将来变成国有制的农民,农业工人,使用机器,现在除个了少数人以外,这五百万人知识分子里头,其他的知识分子,虽然是动摇的,在世界观问题上,在思想问题上。但是在社会主义制度这个问题上面也赞成,他们拥护社会主义制度,一般地说,除了少数人他们拥护,他们都愿意为社会主义服务,那么好吗。工人阶级已经把他们争取,经过中国共产党把知识分子争取到自己一边,跟我们合作了,但是这个合作,为社会主义服务,这是一件事,世界观、灵魂世界又是一件事。要他们换掉世界观,改成马克思辩证唯物主义,历史唯物主义,就是马克思主义,就要有一个时间。

现在大多数知识分子的资产阶级思想,资产阶级世界观,还没有变,或者还没有大变,小变了一些,我刚才所讲的,他们和劳动人民有些格格不入,没有打成一片。文学艺术家不愿下厂下乡,下去一下,马上就回来了,他的注意力不在那里,他们对农民的兴趣不大,对工人的兴趣不大。他们愿意回到家里面和知识分子搞到一块,我叫这些人是半条心。有半条心愿意为社会主义服务,这半条心是好的。因为有半条心啦,剩下那半条心可以慢慢来嘛,那半条心还是资产阶级的王国,资产阶级思想。我跟这些知识分子一讲,他们就会跳起来,“你说老子半条心啦!”我说我就这样讲,你跳到屋顶上去,我还是这样讲,(笑声)因为你还是这样动动摇摇的,你们是愿意为工人阶级服务,但你不是全心全意,是半心半意。有什么证明?就是你们跟工人、农民打成两片,不是打成一片。你们在工人、农民里面没有朋友,你们的朋友是知识分子。知识分子跟知识分子交朋友,就不如跟工人农民交朋友。但是他们不干,你把他赶也赶不下去。赶下去没有几天就回来了。但是工人阶级要求知识分子全心全意为他自己服务,他提出这样的要求,但是要他们在世界观这个问题上面来一个改变,丢掉资产阶级世界观。我在报纸上讲:要破资本主义立社会主义,一个破一个立嘛,它那个东西不破,这个东西不能立的。于是乎就发生了一个任务要破,破嘛有点痛,那你资本家现在都提出来嘛。资本家里头有大批知识分子,我们要使用他们,工人阶级要求大量知识分子,全心全意为他自己服务,不是百分之百。应该是大多数,相信马克思主义,跟我们有共同利益,我们现在有共同利益。他们听到合作社办得不好就高兴,合作社倒霉他们就舒服,共产党出了乱子他们就开心,有这样一小部分人。但是,多数人他们还是因为他们实际上是吊在半天上,还是愿意跟我们合作的。所以,我们的任务就是争取他们。

我们党现在提出准备整风,我们要争取知识分子,要争取党外人士,我们先要把自己的作风整顿一下。中央准备今年开一次会,现在还没有决定,今天我也可以稍微讲一下:我们多少年没有做整风这个工作了,“三反”“五反”很激烈,那个东西没有在思想上面解决问题,准备进行一次整风,这样我们就估计可以争取共产党的作风整好了,可以争取广大的党外人士。谈到这个民主人士就有了这样一个问题:合作还是不合作。这个问题不是他们的问题,不是他们跟我们合作不合作的问题,而是我们跟他们合作不合作的问题。还有一个问题,合作是好讲,我们跟你们合作,民主人士,苗海南,但是就是使用不使用的问题。有些人讲,他们没有多少用处,甚至讲是废物,废物也可以利用嘛,废物为什么不可以利用呢?我今天不能具体讲苗海南是废物,苗海南大概是一个很有用的人。就是有些用处不多的人,也可以利用,有些人起了一个名字叫废物,废物也可以利用。一开会不是应付了一下,无非是每年开那么一次政治协商会议,人民代表大会应付一下,过了关就没有事了,大概一年有那么一、二个礼拜吧。这种态度是消极态度,我说要采取积极的态度。这些民主人士大体上都是党外知识分子,这一些人是老知识分子,旧社会遗留下来的知识分子。

这批人现在大学生有多少是工人农民出身的呢?全国的统计有百分之二十,一百个大学生里面只有二十个是工人农民出身的,百分之八十都是地主、富农、资本家的子女。中学生就不大清楚了,大概是四、六开,地主、富农、资本家出身的有百分之六十,工人农民出身的有百分之四十,或者是一半一半。你们有这个统计没有?我在这方面没有统计,中学生升了高中恐怕还是他们的人多,还是剥削阶级出身的人多,什么年间大学生百分之百是工人农民出身,高中生百分之百是工人农民出身,至少要过三个五年计划以后,至少还要过十一、二年,这个情况才能起根本变化,也许时间还要长一些,再有两个五年计划还不够,也许还要三个五年计划。他那个种是要绝的,那是没有问题的,他是个绝根问题。这个资产阶级还有什么种呢?地主阶级还有什么种呢?他就没有种?。我们的目的就是要使他们绝种嘛,就是以后没有资本主义制度了,没有地主封建制度了,以后一万年都是工人、农民的制度,就是都是工农子弟。

最有前途的是无产阶级,现在这个时候青黄不接,同志们要知道,我们一步也离不开他们,离开他们没有人教书,没有人当工程师,没有人研究科学,大学教授,中学、小学教员,大都是他们,文学家,艺术家,大都也是他们,离不开他们,离开他们我们一步也不能动。所以应该好好跟他们团结。几个五年计划才能起变化。那个时候他们也变了,资本家变成工人了,地主变成农民了,他们的子弟也变了,现在正在变。

整风是用自我批评解决党内矛盾的一种方法,也是解决党跟人民之间矛盾的一种方法。

整顿三风,整教条主义、宗派主义、官僚主义。此外,还有一些问题也附带在里头整,比如贪污的问题,政治机关里面有贪污,特别是基层,合作社、厂矿里头的贪污问题。全心全意为人民服务的这个精神减少,革命意志衰退这个问题。闹地位,闹名誉,争名夺利,这个东西多起来了。过去那个拼命的精神,过去阶级斗争的时候,跟敌人作斗争的时候,我们那种拼命的精神现在有些同志身上消失了。讲究吃、讲究穿,比薪水高低,评级评低了痛哭流涕。人不是长看两只眼睛,眼睛里面有水,叫眼泪,茶碗里头的水叫茶。评级评得跟他不对头的时候就双泪长流,(用手形容全场笑声)在打蒋介石的时候,抗美援朝的时候,土地改:革命的时候,镇压反革命的时候,他一滴眼泪也不出,搞社会主义建设一滴眼泪也不出,一接触到个人利益双泪就长流。(笑声)据说还有几天不吃饭的,据说是你们这个地方的,三天不吃饭,我们说三天不吃饭也没有什么关系,一个礼拜不吃饭有点危险就是了。(笑声)总而言之,争名誉,争地位,比较薪水,比较衣服,比较享受,这样一些事情出来了。这也算人民内部的矛盾。双泪长流,为个人利益而绝食而流泪。有一个戏、叫《林冲夜奔》,上面说:“男儿有泪不轻洒,只因末到伤心处”。我们现在有些同志他们也是男儿,也许还有女儿。请看:男儿有泪不轻洒,只因未碰到评级这个机会。同志们,这个风你也要整一下吧,我们大家回去劝他一下,男儿有泪不轻洒是对的,伤心处是什么伤心处呢?就是工人阶级农民阶级危急存亡的时候,那个时候可以洒几滴泪,至于你那个几级几级,就是评得不对也要吞下去,眼泪不要往外流,往里头流,(笑声)要吞下去。是有许多不公道的事,实际上是可能评得不对,那也无关大局,只要有饭吃就行,革命党嘛,饿不死人为原则,人没有饿死就是革命同志嘛,要奋斗,一万年也要奋斗。有共产党要奋斗,没有共产党那时消灭了,那时总要有领导者,总有一批管理工作人员,他们也是为人民服务的。总之,要全心全意为人民服务,不要半心半意,或者是三分之二的心,三分之二的意,替人民服务。革命意志衰退的人要振作起来。今年准备,今年是准备阶段,出告示,说要整风了,各地可以试整。有些贪污分子,他今年就要赶快收手。本来要抓到一些东西的,现在我赶紧放到这里,那么我们就不算贪污犯了。明年正式整的时候,你今年把他吐出去了,今年不贪污了,或者贪污了的怎么吐出去。那么已经吃下去了、消化了,变成屎了怎么办呢?变成屎了看其情况分期吐出。那个农民,合作社里面不吐可是不行,有什么三十块,就是大问题,那么你就分三期吐,今年吐十块,明年吐十块,后年吐十块,今年吐好了的整好了的。明年就不算贪污犯了。所以今年出告示呀,官僚主义整好了,跟人民的关系,跟下级的关系,当个厅长也好不容易啊!厅长底下有好多人,平时不是不大管的,等到你一整风的时候,那些人就会说话的,你们在座总有厅长,局长,科长嘛。是不是扔一千炸弹就要整风了呀,我看不,我看不应这样做了,改一下就是嘛,我们的缺点,人民内部的缺点,不用人民主不搞大运动,那是对付阶级敌人的,我们是搞小民主如果一个小还要少再加一个小字就是小小民主。总而言之,是和风细雨,台风一定不刮,是毛毛雨下个不停微风吹个不停,(笑声)我们来吹他二年。今年准备,明年一年,后年一年下点毛毛雨吹点微微风。把我们的官僚主义什么东西吹掉。主观主义吹掉。我们从保护同志出发,从团结的愿望出发经过适当的批评。达到新的团结。讲完了,同志们。


