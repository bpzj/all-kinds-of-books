\section[中华苏维埃共和国中央执行委员会命令中华苏维埃共和国中央执行委员会命令]{中华苏维埃共和国中央执行委员会命令中华苏维埃共和国中央执行委员会命令}


“鉴于帝国主义积极侵略,中国沦亡之危迫,全国工农及爱国志士均积极参加民族革命斗争,反对日本侵略者及其走狗蒋介石辈。又因蒋介石等的卖国统治,陷中国经济于万不磨之城一特别是农村经济,全国农民群众起来反抗与暴动,富农已改变仇视苏维埃革命而开始同情于反帝国主义与当地革命的斗争。苏维埃中央委员会为扩大全国抗日讨蒋之革命战线,特决定改变对于富农之政策:

(甲)富农之土地除以封建性之高度佃租,出租于佃农者,应以地主论而全部没收之外,其余富农自耕及雇人经营之土地不论其地之好坏,均一概不在没收之列。

(乙)富农之动产及牲畜耕具,除以封建性之高利贷出借以剥夺农民者外,均不应没收。

(丙)统一累租外,禁止地方政府对于富农之惩罚及特殊税捐。

(丁)富农在不违反苏维埃法律时,各级政府应保障其经营工商业及雇佣劳动之自由。

(戌)在实行平分一切土地区域(乡、区)富农有与普通农民之分得同样土地之权。

(已)富农在违反苏维埃法令时应依法惩治之,进行反革命活动时应按暂行惩治反革命条例惩之。

(庚)富农仍无权参加红军及一切武装部队,包括赤卫队在内,并无选举权。

(辛)以前颁布之土地法及一切其他法令凡与本法令有抵触者废除之。

(壬)本命令自公布之日起实行之(在未联系一片之区域,文到次日实行之)。在本命令实行前,以前法令施行者仍属有效,不得翻案。

 主席


 一九三五年十二月十五日

