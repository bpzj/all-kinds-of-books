\section[土地法(一九二八年十二月制,在井岗山)]{土地法(一九二八年十二月制,在井岗山)}
\datesubtitle{(一九二八年十二月)}


(一)没收一切土地归苏维埃政府所有,用下列三种方法分配之:

1、分配农民个别耕种,

2、分配农民共同耕种,

3、由苏维埃政府组织模范农场耕种。以上三种方法,以第一种为主体。遇特别情形,或苏维埃政府有力时,兼用二三两种。

(二)一切土地,经苏维埃政府没收并分配后,禁止买卖。

(三)分配土地之后,除老幼疾病没有耕种能力及服公众勤务者外,其余的人均须强制劳动。

(四)分配土地的数量标准:

(1)以人口为标准,男女老幼平均分配。(2)以劳动力为标准,能劳动者此不能劳动者多分土地一倍。

以上两个标准,以第一个为主体。有特殊情形的地方,得适用第二个标准。采取第一个标准的理由:

(甲)在养老育婴的设备未完备以前,老幼如分田过少,必至不能维持生活。

(乙)以人口为标准计算分田,比较简单方便。

(丙)没有老小的人家很少。同时老小虽无耕种能力,但在分得田地后政府亦得分配以相当之公众勤务,如任交通等。

(五)分配土地的区域标准:

(1)以乡为单位分配。(2)以几乡为单位分配(如永新之小江区)。(3)以区为单位分配(如遂川之黄×区)。

以上三种标准,以第一种为主体。遇特别情形时,得适用第二第三两种标准。

(六)山林分配法:

(1)茶山、柴山,照分田的办法,以乡为单位,平均分配耕种使用。

(2)竹木山,归苏维埃政府所有。但农民经苏维埃许可后,得享用竹木,竹木在五十根以下,须得乡苏维埃政府许可。百根以下,须得区苏维埃政府许可。百根以上,须得县苏维埃政府许可。

(3)竹木概由县苏维埃政府出卖,所得之钱,由高级苏维埃政府支配之。

(七)土地税之征收:

(1)土地税依照生产情形分为三种:(一)百分之十五,(二)百分之十,(三)百分之五。以上三种方法,以第一种为主体。遇特别情形,经高级苏维挨批准,得分别适用二三两种。

(2)如遇天灾,或其他特殊情形时,得呈明高级苏维埃政府核准,免纳土地税。

(3)土地税由县苏维埃政府征收,交高级苏维埃政府支配。

(八)乡村手工业工人,如自己愿意分田者,得分每个农民所得田的数量之一半。

(九)红军及赤卫队的官兵,在政府及其他一切公共机关服务的人,均得分配土地,如农民所得之数,由苏维埃政府雇人代替耕种。

按:此土地法是一九二八年冬天在井岗山(湘赣边区苏区)制定的,这是一九二七年冬天至一九二八年冬天一整年内土地斗争经验的总结,在这以前,是没有任何经验的。这个土地法有几个错误:

(一)没收一切土地而不是没牧地主土地;

(二)土地所有权属政府而不是属农民,农民只有使用权;

(三)禁止土地买卖。这些都是原则错误,后来都改正了。关于共同耕种与以劳力为分配土地标准,宣布不作为主要办法,而以私人耕种与以人口为分田标准作为主要办法,这是因为当时虽感到前者不妥,而同志中主张者不少,所以这样规定,后来就改为只用后者为标准了。雇人替红军人员耕田,后来改为动员农民替他们耕了。

