\section[关于山东六级干部大会情况的批示(一九六○年三月二十三日)]{关于山东六级干部大会情况的批示(一九六○年三月二十三日)}


山东发现的问题,肯定各省、各市、各自治区都有,不过大同小异而已。问题严重,不处理不行。在一些县、社中,去年三月郑州决议忘记了,去年四月上海会议十八个问题的规定也忘记了,“共产风”、“浮夸风”、“命令风”又都刮起来了。一些公社工作人员很狂妄,毫无纪律观点,敢于不得上级批准,一平二调。另外还有三风:贪污、浪费、官僚主义,又大发作,危害人民。什么叫做价值法则、等价交换,他们全不理会。所有以上这些,都是公社一级干的。范围多大,不很大,也不很小。是否有十分之一的社这样胡闹,要查清楚。中央相信,大多数公社是谨慎、公正、守纪律的,胡闹的只是少数。这个少数公社的所有工作人员,也不都是胡闹的,胡闹的只有其中一部分。对于这些人应该分别情况,适当处理。教育为主,惩办为辅。对于那些最胡闹的,坚决撤掉,换上新人。平调方面的处理,一定要算账,全部退还,不许不退。对于大贪污犯,一定要法办。一些县委为什么没有注意这些问题呢?他们严重地丧失了职守,以后务要注意改正。对于少数县委实在不行的,也要坚决撤掉,换上新人。同志们须知,这是一个长期存在的问题,是一个客观存在。出现这些坏事,是必然不可避免的,是旧社会坏习惯的残余,要有长期教育工作,才能克服。因此,年年要整风,一年要开两次六级干部大会。全国形势大好,好人好事肯定占十分之九以上。这些好人好事,应该受到表扬。对于犯错误而不严重、自己又愿意改正的同志,应该釆用教育方法,帮助他们改正错误,照样做工作。我们主张坚决撤掉,或法办的,是指那些错误极严重、民愤极大的人们。在工作能力上实在不行,无法继续下去的人们,也必须坚决撤换。


