\section[对《马克思主义者应该如何正确地对待革命的群众运动》一文的批语(一九五九年八月十五日)]{对《马克思主义者应该如何正确地对待革命的群众运动》一文的批语}
\datesubtitle{(一九五九年八月十五日)}


一个文件摆在我的桌子上,拿起来一看,是我的几段话和列宁的几段活,题目叫做《马克思主义者应该如何正确地对待革命的群众运动》,不知是那一位秀才同志办的,他算是找到了几挺机关枪,几尊迫击炮,向着庐山会议中的右派朋友们,乒乒乓乓发射了一大堆连珠炮弹。共产党内的分裂派,右得无可再右的那些朋友们,你们听见炮声了吗?打中了你们的要害没有呢?你们是不愿意听我的话的,我已“到了斯大林晚年”,又是“专横独断”,不给你们“自由”和“民主”,又是“好大喜功”,“偏听偏信”,又是“上有好者,下必有甚焉”,又是“错误只有错到底才知道转弯”,“一转弯就是180度”,“骗”了你们,把你们“当做大鱼钓出来”,而且“有些像铁托”,所以有的人在我面前都不能讲活了,只有你们的领袖才有讲活的资格,简直是黑暗极了,似乎只有你们出来才能收拾局面似的。如此等等。这是你们的连珠炮,把个庐山几乎轰掉了一半。好家伙,你们那里肯听我的那些昏话呢?但是据说你们都是头号的马列主义者,善于总结经验,多讲缺点,少讲成绩,总路线是要“修改”的,大跃进“得不偿失”,人民公社“搞糟”了,大跃进和人民公社都不过是“小资产阶级狂热性”的表现。那么,好吧,请你们看看马克思和列宁怎样评论巴黎公社,列宁又怎样评论俄国革命的情况吧!请你们看一看:中国革命和巴黎公社,那一个好一点呢?中国革命和一九○五——一九○七的俄国革命相比较,那一个好一点呢?还有,一九五八——一九五九中国建设社会主义的情况,同俄国一九一九、一九二一年列宁写那两篇文章的时候的情况相比较,那一个好一点呢?你们看见列宁怎样批判叛徒普列汉诺夫,批判那些“资本家老爷及其走狗”、“垂死的资产阶级和依附于他们的小资产阶级民主派的猪狗们”吗?如未看见,请看一看,好吗?

“对转变中的困难和挫折幸灾乐祸,散布惊慌情绪,宣传开倒车,这一切是资产阶级知识分子进行阶级斗争的工具。无产阶级是不会让自己受骗的。”怎么样?我们的右翼朋友。

既然分裂派和站在右边的朋友们都爱好马列主义,那么,我建议:将这个集纳文件提供全党讨论一次。我想,他们大概不会反对吧。


