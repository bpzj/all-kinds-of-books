\section[毛主席视察华北、中南和华东地区时谈话的主要精神的传达(三、视察浙江时的指示)]{毛主席视察华北、中南和华东地区时谈话的主要精神的传达(三、视察浙江时的指示)}
\datesubtitle{(三、视察浙江时的指示)}


毛主席到杭州,只休息了一小时,就召集省军管会负责同志了解浙江的无产阶级文化大革命情况。被接见的有二十军政委南萍,空五军政委陈励耘等,参加接见的还有杨成武、张春桥。吴法宪等同志。

毛主席对浙江文化大革命情况很了解,接谈了两小时,主席作了很多指示。主席详细地询问了舟山文化大革命情况,对军队问题问得很详细。

南萍同志汇报说:“二十军进入杭州前,军党委作过决议,向空五军学习。”

毛主席说: “空五军支左很好嘛!”

陈励耘向志汇报说:“空五军党委也做出决议,向二十军学习。”

主席说:“大家互相学习。”

主席对南、陈两负责人说:“你们要很好地向群众学习,不要架子,放下官架子,什么事情都是群众创造的。”并对吴法宪同志说:“飞机不是战士打下来的吗?”

南萍汇报说:“我们还有罚跪、戴高帽等现象。”

主席说:“我向来反对这样搞,对待干部不能像对地主一样。我们有个好的传统:团结——批评——团结,对干部要一分为二。”

主席对南萍说:“金华转得很好嘛!”  (据悉毛主席曾于九月十六日视察了金华。)

毛主席对浙江的文化大革命情况很满意,并指示:“这样发展下去,浙江在春节前后可以看出一个清楚的眉目来了。”

在整个接见中,毛主席精神非常好,神采奕奕,接见开始时,毛主席问二十军政委叫什么名字,二十军政委说“姓南”,毛主席就讲了一个唐朝的故事,说明姓南的是怎样来的,毛主席抽香烟,南萍给毛主席点烟,毛主席风趣地说:“自力更生,我自己来。”

\kaoyouerziju{  (根据二十军首长回忆大意)}

