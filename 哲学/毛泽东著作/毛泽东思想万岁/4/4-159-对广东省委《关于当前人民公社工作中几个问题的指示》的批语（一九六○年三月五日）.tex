\section[对广东省委《关于当前人民公社工作中几个问题的指示》的批语(一九六○年三月五日)]{对广东省委《关于当前人民公社工作中几个问题的指示》的批语(一九六○年三月五日)}


广东省委《关于当前人民公社工作中几个主要问题的指示》,是一个很好的文件,甚为切合现时人民公社在缺点错误方面的情况和纠正这些缺点错误的迫切要求。全国各省、市、自治区的情况大体上一定都同广东一样,发生了这些问题(一共有五个问题),都应当提起严重的注意,仿照广东的办法,发出一个清楚通俗的指示,迅速地把缺点错误纠正过来。中央建议,把广东这个指示发到地、县,公社三级党委,请公社党委的同志们,切实讨论几次,开动脑筋,仔细地冷静地想一想,谈一谈,议一议,想通这五个问题,纠正缺点错误。现在形势大好,缺点错误是部分的。但是一定要纠正,不使这五方面的现在还是部分性质的错误扩大开去。我们的相当多的干部,在政治水平、经济理论水平和对实际工作的分析、理解水平都是不高的,有些人还是很低的,他们在这些方面还不成熟,这是他们的缺点。他们正在逐步走向成熟的过程中。这是一方面。他们干劲很大,热情很高,要把中国变成一个伟大、强盛、繁荣、高尚的社会主义、共产主义国家的雄心大志,则是很好的。这是又一方面。特别是第一个方面,即他们的缺点方面,努力学习,认真思考,在几年之内,例如说,五年至十年之内,将自己的政治、理论和业务水平大进一步的成熟和提高起来。在这里,顺便说一句,工业交通战线,教育文化科学战线,卫生医疗药物战线,中央建议也照这样办,解决他们自己的问题。

附注:这个批语所说的五个问题是:

一、不顾条件,抢先从队有制过渡到社有制的苗头;

二、用削弱大队经济的办法,来发展社的经济,重复刮“共产风”的错误;

三、公社积累过多,收回社员自留地,集中私养的家畜、家禽,不适当地限制社员的家庭副业生产;

四、社、队干部不讲究经济核算,铺张浪费;

五、社、队干部不如实反映情况,作风浮而不深、粗而不细、华而不实,不同群众商量,不关心群众生活。


