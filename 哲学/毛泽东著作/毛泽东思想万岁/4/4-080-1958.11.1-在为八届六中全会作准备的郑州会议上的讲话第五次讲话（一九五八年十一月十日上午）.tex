\section[在为八届六中全会作准备的郑州会议上的讲话第五次讲话(一九五八年十一月十日上午)]{在为八届六中全会作准备的郑州会议上的讲话第五次讲话(一九五八年十一月十日上午)}
\datesubtitle{(一九五八年十一月十日)}


题目应叫郑州会议关于人民公社若干问题的纪要。搞个文件很有必要。

在没有实现农村的全民所有制以前,农民总是农民,他们在社会主义道路上总还有一定的两面性。我们只有一步一步地引导农民脱离较小的集体所有制,通过较小的集体所有制走向全民所有制,而不能要求一下子完成这个过程,正如我们以前只能一步一步的引导农民脱离个体所有制而走向集体所有制一样。

什么叫建成社会主义?我们提了两条:

(一)完成社会主义的集中表现是实行社会主义的全面的全民所有制。

(二)公社的集体所有制变为全民所有制。

有些同志不赞成在两种所有制中间划一条线,似乎公社全是全民所有制,实际上有两种所有制,一种是公社的集体所有制。如果不讲此,则社会主义建设还有什么用。

有的同志不赞成,说不能划一条线,说划了就损伤积极性,线内也有共产主义,也有集体和全民所有制(鞍钢与公社)。大线是社会主义与共产主义,小线是集体所有制和全民所有制,秀才都不赞成,是不是秀才要造反?

斯大林是划了线的,讲了三个先决条件,这三个条件基本上不坏,但不具体,(1)首先是增加社会产品,这是基本的,我们叫以钢为纲,极大地增加产品;(2)集体所有制提高到全民所有制;将商品交换提高到产品交换,使中央机构能掌握全部产品。不愿划界线的,主要是认为时间已到,以为已经上了天,你们是右倾。当然,现在鞍钢是全民所有制,但还没有过渡到共产主义。总要搞个社会主义全民所有制再过渡到共产主义。现在只有一部分是全民所有,大部分是集体所有。全民所有也不一定过渡到共产主义;(3)提高文化水平、文化、体育、智育。为此需减少劳动时间,六小时至五小时劳动,再是要实行综合技术教育,多面手。“自由就业”,我不大懂。学纺织的又去学开飞机;开煤的又去学纺织,十八般武艺,十多样我赞成,学几百样,怕不容易,会没饭吃。三要根本改善居住条件。四要提高工资,至少一倍,也许还要更多。增加工资的办法是增加货币工资,特别需要的是降低物价。

这三个条件是好的,主要是第一条,但缺少一个政治条件。这几条的基本点是增加产品,极大地增加生产资料和消费资料,就是发展生产,发展生产力,但是没有一套办法。问题是怎么办。没有政治挂帅。没有群众运动,没有全党全民办工、农、文,没有儿个并举,没有整风和逐步破除资产阶级法权的斗争,斯大林的这三个条件,是不容易达到。我们有人民公社,从一县多社到一县一社,加上人民公社就更容易办了。

文件第一条加“完成社会主义建设的集中表现是实现在社会主义的、全面的全民所有制。”“大集体,小全民”,“变为全民社会主义”。

大集体所有制也就是小全民所有制,要逐步的发展为全面的全民所有制。

文件第三条,生产的对象不明确。全面的全民所有制的含义为;

(1)社会生产资料为全民所有,

(2)社会产品也为全民所有(所谓社会产品,不但是生产资料,而且是消费资料)。在过渡阶段,国营工业已是全民的,人民公社的生产资料(包括土地、森林、农具、牲畜、机械、工业、厂矿)和产品也应该逐渐增加全民所有制的成分,即逐步增加生产资料的全民部分和产品的调拨部分。不能只讲后一部分,首先应是前一部分。

要能实现调拨“要有雄厚的物质基础”,指什么“物质基础”不清楚,应该改为生产资料。不断发展生产一一两个部类的生产。

人民公社的性质。人民公社是我国社会主义社会结构的工农商学兵相结合的基层单位。在“现阶段”,又是基层政权的组织。民兵不一定是专政,主要是对外的,是对付帝国主义侵略的准备力量。公社是一九五八年社会经济发展的产物,是一九五八年大跃进的产物,目前的社会主义到共产主义的过渡——即社会主义集体所有制到全民所有制的过渡,将来的社会主义全民所有制到共产主义全民所有制的过渡。是共产主义社会结构的最好的基层单位。

在劳动调配中,要注意实行和巩固生产责任制。劳动力的调配,各生产部门(农业、工业、运输)的比例,是当前的重大问题。劳动分配要合理,组织要适宜。不是价值法则,而是计划的要求。现在这种分配比例不合理,要注意不要突出这里,丢掉那里。

关于分配;要使农村人民公社每人每年平均有150——200元的消费水平,并且增加调拨的比例。个人、公社、国家“三三制”的规定是好的,能得人心。要生产者吃饱、穿好一点。

吃饭问题,一定要注意食品中的合热量和养分,足够起码的条件,要同营养学家商量,定出适当比例。第一,小米、小麦,粳米。吃小米可以什么也不加,缺点是不太好吃。粮食是热量,没有起码的必要数量不行.不可疏忽大意。……

过冬问题,要加上“三北公矿”要立即布置烧炕。没有煤炭不能过冬,必须解决。

城市人民公社,八大城市应当放慢,这种社会主义,不怕,一要搞,二要慢一点。

所谓全面的全民所有制,含义如何?两条:(一)社会的生产资料为全民所有;(二)社会的产品为全民所有。


