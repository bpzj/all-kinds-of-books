\section[给自己诗词作的解释 ]{给自己诗词作的解释 }


我的几首词发表后,注家蜂起,全是好心。一部分说对了,一部分说的不对。我有说明的责任。一九五八年在广州,见一九五八年刊本,天头甚宽,因而,写了下面一些字,谢注家兼谢读者。鲁迅在一九二七年在广州修改他的《古小说钩沉》后记中说道:“于时云海沉沉,星月澄碧餮蚊逞强,于在广州。”从那时到今天,三十一年了,大陆上的蚊灭得差不多了。当然革命尚未全成,同志们仍需努力。港台一带,餮蚊尚多,西方世界,餮蚊成阵。安得其全世界各民族千百万愚公,用他们自己的移山办法,把蚊阵一扫而空,岂不伟哉!
试仿陆放翁曰:

人类今天上太空,但悲不见五洲同。愚公尽扫餮蚊日,公祭毋忘告马翁。

毛泽东一九五八年十二月二十一日上午十时
沁园春.长沙

一九二五年

击水:(到中流击水,浪遏飞舟)

游泳,那时初学,盛夏水涨,几死者数。一群入终于坚持到隆冬。犹在水中。当时有一篇忘记了。只记得两句:“自信人生两百年,会当击水三千里。”
菩萨蛮.黄鹤楼

一九二七年

心潮:(把酒酌滔滔。心潮逐浪高)

一九二七年大革命失败的前夕,心情苍凉,一时不知如何是好。这是那年春季。夏季八月七日,党的紧急会议决定武装斗争。从此才找到了出路。
清平乐.会昌

一九三四年夏

踏遍青山人未老:(踏遍青山人未老,风景这边独好。)

一九三四年形势危急。准备长征,心情又是郁闷的。这首清平乐与前面那首菩萨蛮一样,表露了同一的心情。
忆泰娥.娄山关

一九三五年

万里长征,千回百折。胜少于困难不知多少倍。心情是沉郁的。过了岷山,豁然开朗,转到了反面,柳暗花明又一村了。以下几首反映了这种心情。

七律.长征一九三五年十月

水拍:(金沙水拍云崖暖,大渡桥横铁索寒)

改水拍,这是一位不相识的朋友建议的。他说:不要一篇有两个“浪”字,是可以的。


三军:(更喜岷山千里雪,三军过后尽开颜。)

红军一方面军、二方面军、四方面军。不是陆海空三军,也不是古代晋国所谓上军,中军、下军的三军。
念奴娇.昆仑

一九三五年十月

主题思想是反对帝国主义,不是别的。后一句:“一截留中国”改为“一截还东国”。忘记了日本人民是不对的。这样,英、美、日都涉及了。别的解释不合实际。
清平乐.六盘山

一九三五年十月

苍龙:(今日长缨在手,何时缚住苍龙?)指蒋介石。不是日本人。因为当时全付精力要对付的是蒋不是日。
沁园春.雪

一九三六年二月

反对封建主义,批判两千年来的封建主义的一个反动侧面。

文釆、风骚、大雕,只能如此,须知这是写诗啊,难道可以咒骂这些人吗?别的解释是错误的。

末三句,指无产阶级。
七律.和柳亚子先生

一九四九年四月二十九日

三十一年:(三十一年还旧国。落花时节馈华章)

一九一九年离开北京。一九四九年还北京。旧国之国:都城。不是Ltoho(国家)。也不是Country(首都)。
浣溪纱.和柳亚子先生

一九五○年十月

乐奏:(万方乐奏有于阗)这里误置为“奏乐”。应改。(现版已改)
水调歌头.游泳

一九五六年六月

长沙水:(才饮长沙水)民谣:“常德山山有德,长沙水水无沙”。所谓长沙水,在常德东。有一个有名的“白沙井”。

武昌鱼:(又食武昌鱼)

三国孙权一度从京口(镇江)迁都武昌。官僚、绅士、地主及富裕阶层不悦,反对迁都。造出口号云:“宁饮扬州(建业)水。不食武昌鱼”。那时的扬州人,心情如此。现在改变了。武昌鱼是颇有味道的。


