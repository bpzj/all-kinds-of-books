\section[毛主席在七大上的总结(一九四五年六月)]{毛主席在七大上的总结}
\datesubtitle{(一九四五年六月)}


讲三个问题:国际形势,国内形势,若干思想政策问题。

我们的方针,放手发动群众,壮大人民力量,在我党领导下,打败日本侵略者及其走狗,建设一个新民主主义的新中国。这条方针里面,放手发动群众,壮大人民力量就是组织队伍;在我党领导下就是总司令、指挥官;打败日本侵略者及其走狗,就是说的敌人,建设一个新的新中国就是说的目标,关于放手发动群众的问题……手是我们自己的,放不放在我们。谁不叫我们放手呢?有许多人,其中有蒋介石。国民党六次大会通过的三十条政纲,其中有一条是:绝对禁止违背政府法令及在外交、军事、财政、交通、币制上有任何破坏统一之设施与行动。比如军事上的统一,即就是我们不要八路军、新四军。这一斗争,早就开始了。一九二六年三月二十日中山舰事件,就是不让我们放手发动群众,蒋介石不让,汪精卫也不让,我们自己也不敢放手。那时我们应该大胆的放手,但我们却不敢放手,所以失败了。内战时期,放手了。但又过了一点,没有与冒险主义相区别。抗战时期,就是这条路线,这次大会只是抓住这条路线,因为有了过去多年的经验。放手是在有理、有利、有节的条件下,而不是冒险。有理、有利、有节就是放手而不是冒险。这条方针一直在全国胜利是不会变的。压力很大,要使无产阶级先锋队从束缚手足的精神压迫下解放出来,是不容易的。从历史上就是如此。第二国际一直不敢放手发动群众。一八四七年到一八四八年马克思、恩格斯的“共产党宣言”,就是放手发动群众的方针。第二国际违背这一方针,崇拜自发论,一切听其自然。共产国际,俄国布尔什维克恢复了马克思主义,放手主动群众,壮大人民力量,在俄党(联共)领导下,打败俄皇,建立工农政权。所以我们要宣传,肯定这一方针。朱总司令……我的报告,都是贯彻这个精神,坚持这条方针,阶级未消灭,我们这条方针是不会取消的。各个阶段情况不同,政策也有变化,但总的方针是不会变的。至于如何实现这条方针,这需根据周围情况,及其内部联系如何来决定。

第一、国际形势

第一个问题是国际形势。报告中说,五大国家团结起来把世界引向进步,这是历史的总趋向。五大国中苏联为首。这个趋向对不对?世界是否会倒退?倒退,报告中也说到了,说世界还有强大的反动势力,它们要破坏这个进步的趋向,暂时甚至是严重的历史性的曲折(大事变)还可能发生。因为还有强大的反动势力,还不愿意看见人民力量,人民的进步,这里又有这种情况发生,即就是要做另外的估计,即马上发生第三次世界大战一英美法结合的反苏大战,消息报说了,有些反动派的报纸,造谣挑拨英美进行反苏大战,旧金山会议也闹乱子,许多人都关心旧金山会议。最近四、五年国际形势经常矛盾,好消息,坏消息,今后还会时好时不,不好,不坏,波浪式的发展。为什么要如此呢?因为世界上有两种势力在斗争,苏联和世界人民在一方面,反动势力是另一方面。现在的世界就是一个矛盾的世界。但是,英美和日德俘虏立即结合起来组织反苏反人民的第三次世界大战,这种可能性存在不?不存在。客现实际上不存在。有就有,没有就是没有,就是因为没有,所以我们才说没有,两个客观实际,我们不是按照反动派的客观实际讲的,而是按照苏联及人民的客观实际讲的。按照俘虏、孤立主义者、慕尼黑派……他们要不要三次世界大战呢?他们是要的。按照苏联和世界人民是不要的。现在反苏反人民的大战危险是不存在的。英美资产阶级内部有一部分反动派要干;一部分不愿意干;另一部分想干又不敢干。他们也是三三制:进步的、反动的、中间的。现在英国政府不是张伯伦政府,而是英国保守党的另一派,英国外交政策可能发生变化,旧金山会议可能无结果而散,也可能有些成绩而散。苏联在旧金山会议的记者分析得就很好;“我们只能就大势而论有三种可能,第一,有所成就,不甚圆满;第二,无结果而散,也不破裂(与我们和蒋介石谈判一样),下次再来;第三,最坏的是完全破裂,推翻克里米亚会议的决定。这第三种可能性不大,第一、第二种可能性大些,但是即使第三种情况发生了,克里米亚决议推翻了,这是不是说英美就要组织反动派进行三次大战,进攻苏联呢?不是的;莫洛托夫在旧金山会议上说得对,即使现在不能建立国际安全机构,也不等于将来不能建立,我们是要努力争取建立。这次搞不好,下次再搞,也是总括全世界形势说的。相信苏联的力量,相信世界人民的力量,相信英美资本家内部情况不统一,不是完全反苏的。有一部分人还是愿意同苏联合作。相信印度、南美、中国人民的力量。这些情况,即现在世界的情况,现在战争只结束了一半,日本还没有打败,把这些情况结合起来看,即使旧金山会议这回搞不成,也不能说永久搞不成。宣布欧洲胜利那天,斯大林说,欧洲已进入和平时期。斯大林说的对,他是根据整个情况说的,苏联在国际范围内胜利了。列宁曾说:“如果十年二十年布尔什维克与农民有战争关系,苏联将保证在世界范围内胜利了(即使在世界无产阶级革命延迟的情况下),否则就要忍受二十年到四十年的白色恐怖。”这是在一九二一年讲的,同志们,是不是是二十年!

莫斯科危急时,斯大林说:“或者胜利,或者消灭。”现在是胜利了。苏联的胜利是从莫斯科打胜仗开始的,不是从斯大林格勒开始的。没有莫斯科的胜利,也就没有斯大林格勒的胜利。现在苏联的红旗插在柏林了。列宁说:即使没有世界无产阶级革命的形势,也能胜利。列宁的话讲灵了。苏联在国际范围内胜利了。现在只是努力巩固这种胜利。苏联的胜利就是全世界人民的胜利,也就是中国人民的胜利。

资本主义有他的历史,很久以前,世界上没有资本主义,二三百年前,世界上产生了个娃挂,叫资本主义,另一个娃挂,叫无产阶级。中国外国的老书上没有资产阶级、无产阶级、共产党。这些都是近代的产物。资本主义一定要打世界大战,打的时候社会主义国家可能在一个国家单独胜利。以前马克思、恩格斯说,革命只能在几个国家同时胜利。但到了二十世纪的时候,列宁根据新的情况,规定要死的资本主义一一帝国主义战争是社会主义革命的前夜,列宁的话又说灵了。以前世界上都是打小仗。把全世界的大国都卷进去,连一些小国也卷进去的大规模战争,一九一四年是第一次,我们看到了,古人未看到,这是他们的缺憾。我们这一代人看过两次,一九一四年到一九三九年,隔了二十五年,又打了第二次,第一次世界大战打出了一个十月革命。整个世界历史发生了新的变化,开辟了世界历史的新时代。从这时起资本主义倒霉了,走下坡路了,社会主义走的是上坡路。“新民主主义论”中我也讲的资本主义是向下发展的,社会主义是向上发展的,来了一个希特勒,打到了莫斯科,好像社会主义向下了。现在,红军打到了柏林,社会主义又向上了,从这时起(十月革命)资本主义缺了一条腿,剩下的资本主义又分成两大部:一部分变成法西斯资本主义,一部分变为民主的资本主义。这两部分资本主义打架,其中一部分资本主义(民主的资本主义)与社会主义合作,将法西斯主义一夹,夹掉了一个德国,再一夹就把日本法西斯夹掉了。资本主义好比一个四支足的马,已经掉了两支足,剩下两支足,成了一个跛子,铁拐杖,成不了完全的东西,资本主义残废了,掉了一支足,一支手,想把一支木足来走路,你们说,这剩下的资本主义比过去更加强大了呢,还是更削弱了?是更削弱。

要看大的农西,要看普遍的,大量的东西,许多同志常常对大的东西看不见,只看到局部的小的东西,十月革命资本主义砍掉了一支手,第二次世界大战,德意资本主义垮了台,德国内部最坏的东西垮了台,许多小皇帝也垮了台,小国家起了变化。都前进了,这算又砍掉了一支足,这些都是事实,必须看到这些大事情才能在分析时不犯错误。

现在,美英各国的通讯社、报刊,专为一些小问题,如波兰问题、德里亚斯特问题、奥国问题,咬住不放,吵闹不休,看起来很觉奇怪,它们为什么要这样呢?我说资本主义有一个特性:“就是蚀大本算小账。”第一次世界大战失去一支手,第二次世界大战又失去了一支足,现在却抓住一把头发死也不放,叫着这是呀。为什么抓住小辫子不放?英国本身打得五痨七伤,这表示资本主义残废了,苏联和欧洲人民强大了;他们不抓住辫子就什么也没有了。他们资本主义很小的,所以只得抓住小辫产不放,放不得,放了就无话可讲了。这是我们的想法,但他们也有他们的用意。世界各国反动派的反革命言论其作用在于:第一,阻止苏联及欧洲人民力量的发展,在苏联及欧洲人民面前,抓住小辫子讲价还价;第二,调动国际国内的反动势力慕尼黑分子、赫尔、哈德克斯分子等等。丘吉尔政府看到苏联和欧洲人民起来了,要找人撑腰,就不得不唱些反动派的调子,才能调起反动派给他撑腰,团结那些反动派作为自己的基础;第三,镇压革命人民。丘吉尔发现他在欧洲人民包围之中,所以要唱一段儿反动调子压一压人民。一共三个作用。由此看来,我们便懂得旧会山争论的那么凶是为什么。将来东方问题还要争论的。这些争论将此做文章,都是题中应有之意,因为他们还有一支手一支足未砍掉,它还不甘心,争论是可以理解的,如果都变成了像苏联一洋,那当然不会有一点乱子。但是现在如果没有一点乱子,倒是成为不可理解的了。

现在政界上的外交政策,只有苏联是主动的。英美中都是被动的。波兰问题、小辫子,也抓不了几天。苏联没有挂车,所以主动。英美打这次仗都是被动的。他们原来都是开慕尼黑的。英美是想叫希特勒打苏联,他雇佣的这个劳动者(希特勒)有点闹独立性,先打了老板一顿,打的哇哇乱叫,它(英国)才向苏联订了二十年的协定。他们对日本也是如此,它们原是出钱请“工人”去打苏联的,送钱、汽油给日本,也就是出钱的意思,日本天天向苏联挑衅,从英美那里拿工钱,把工钱拿到手,日本不去打苏联,而去打珍珠港了。打了珍珠港,美国才和苏联订了协定。资本主义的主动是在十八世纪时候,十九世纪上半期还有一点,到了二十世纪初便完全腐朽了,变成了帝国主义和处于被动了,第二次世界大战是被迫打的,是没有计划的。对南斯拉夫他他们不承认,不承认,后来也承认了。对波兰不承认,不承认,将来也还是得承认的。现在他们不承认解放区、八路军、新四军,将来也会承认的,赫尔利四日宣告不承认我们,他说共产党向他要武器,这个账要算清,看是谁讲的。它要送,我还不一定要哩!

资本主义是向下的,欧洲大陆的资本主义下降了,日本的资本主义下降了,英国的资本主义也下降了,美国的资本主义是向上的,它的生产在大战中是它历史上未曾有过的大发展,超过战前的生产一倍到两倍。斯退丁纽斯说:一九二八年繁荣期间,美国的生产为六百亿美金,现在为二千万万,有人说一千八百万万。斯退丁纽斯说一千五百万万到二千万万,战后要维持一千五百万万的产额,这就比战前增加了九百万万,现在恐怕不止一千九百万万,它发了这样大的财,因此说美国资本主义是向上的,是对的。但拿这一点来肯定美国资本主义是万岁的,说它一直上升,如此说法那世界上就没有马克思主义了。不会的,美国现在的繁荣带有特殊性,特殊的繁荣是回光返照。美国的危机很快就要到来了,第一次世界大战后,五年功夫就发生了很大的危机,那时胡佛总统吹牛皮,说美国是有组织的资本主义(布哈林也这样说),不会发生危机。讲了以后,不到三个星期,危机就来了。一九三三年(希特勒上台的一年)罗斯福上台,作了十二年总统,搞新政策,使资本主义又繁荣了,但主要是战争时期的繁荣。从前生产六百万万,现在一千五百万万到二千万万,如果美国在前线上有一千万士兵,二千万军需工人,他们回去后,生活职位如何解决?美国共一亿三千万人口,有六千万工人,二个人中就有一个工人,肯定说,战后要维持一千五百万万到二千万万的生产不可能,这六千万人口,我看它怎样解决?所以,它的危机很快就要到来。多快,不会等到十年。从经济上说,美国是世界上经济的喜马拉雅山,但这个山是要倒的。按马克思主义观点看问题,应该这样来看,美国的危机归根到底,不能由资本主义来克服,而是要无产阶级来克服。

英国一定要左倾,在反法西斯战争中,形成的国内团结破裂了,没有工党、自由党在政府中帮忙,没有共产党在社会上帮忙,保守党是不行的。这次战争中苏联打德国四分之三,英美法只打四分之一,苏联出了四个指头,他们总共出了一个指头。美国出兵力四百万,英国一百万,法国五十万,他们要同苏联打,这是不可能。英国工党领袖和我们一样,也是没收大地主大资本家阶级的银行、工业,即操纵国计民生的企业,要归政府所有,亲苏政策也是一样的。对印度问题有缺点(只有英国共产党对印度问题是彻底的),叫有缺点的好纲领,但纲领虽好,不一定能实行,即理论与实践不能结合,英国共产党强大起来,丘吉尔处在英国人民包围之中,这次英国大选丘吉尔能胜利吗?即使丘吉尔胜利了,但没有国内战线支持了,还有很大力量制它的肘;打败日本之后,美国国内战线也会破裂,美国共产主义政治协会应有这样精神准备,这次旧金山会议即使搞不成,第二次旧金山会议也是会来的,因为世界情况需要它。丘吉尔的困难大得很,英国人民拖住它,印度人民拖住它,欧洲人民讨厌它,美国资本家还在揪着他腰里的荷包。关于建立反苏堡垒战,英国保守党是有这样企图的。从前是从波兰、罗马尼亚……等国家组成的,这片墙壁现在被苏联打掉了。让他再搞这一套那就回到张伯伦那里去了,但现在不是那个时代了,希特勒倒了,苏联强大了,欧洲人民觉悟了,英国人民觉悟了。美国资产阶级中有想与英国反动分子结合起来反苏的,但大部分人是愿与苏联关系搞好的,斯退丁纽斯也要和苏联关系搞好。美国是民主政府,英国都是现实主义的。一手打日本,一手抓一把(作生意)。

总起来讲,第一次世界大战后,资本主义的稳定已经没有了,危机是历史就有的,如月经一样,资产阶级的月经。十月革命后叫总危机,总危机即统统那样,总是那样,大危机后有小危机。这次美国是特殊繁荣,以后也不会稳定,其中是含着危机的。八个大国,谁是领袖?苏联还是美国?有人说美国是领袖。这种说法不对,从经济上可以这样说,但是经济之外还有政治,政治与经济结合起来才是更强大的,那就是苏联。苏联在经济的力量上说,不如美国,但苏联是社会主义经济,可以产生伟大无比的力量,产生了强大的红军,英勇的人民,苏联是全世界人民的领袖。所以英美各国人民对苏联的信仰都是很高的。中国也是发展资本主义,中国的资本主义是什么性质?前面说过,有反动的法西斯资本主义;有民主的资本主义。反动的法西斯资本主义已经打垮了。民主的资本主义比法西斯资本主义进步些,但它仍然是压迫殖民地,压迫本国人民,仍然是帝国主义,它一方面打德国,一方面又压迫人民,打法西斯是好的,压迫人民是不好的,在打法西斯的时候,对它的压迫忍一些,不提打倒蒋介石。

蒋介石是半法西斯半封建的资本主义,我们是新民主主义的资本主义,这种资本主义是有它的生命,还有革命性。一部分资本主义在反法西斯时还有作用,另一部分资本主义就是新民主主义的资本主义,将来还有用,在中国与欧洲、南美的国家中还有用,它的性质是帮助社会主义的,它是革命的,有用的,有利于社会主义发展的。

总的来说,资本主义是向下的,而农业国家的资本主义是向上的――不平衡。

×××能单独胜利的意见是对的,列宁讲单独胜利,是指在一定条件下能够胜利,如十月革命,打退十四国干涉,都是利用矛盾,得到各国无产阶级的帮助,才能成功的。我们中国,没有外国的援助能不能胜利?中国是自力更生,力争外援,没说不要外援,否则《共产党宣言》上所说:“全世界无产者,联合起来”这句话就可划掉。我们不作此提议,我们还要做外交工作,中国革命必须有全世界无产阶级的帮助。

第二、国内形势

第二个问题:国内形势。上边说,中国不能单独胜利说要作外国的联络工作。董老这次到旧金山,要作这个工作,将来还要作这个工作。外交原则我们已经规定,要照那样去作。中国是一个半殖民地国家。现在是受日本压迫的国家,将来国民党统治仍是半殖民地的,独立只能是形式的。中国不是一个小国,是一个有四万万五千万人口的大国,所以有很多人想打主意。日本人想刮中国人民的油水,美国帮助蒋介石是想控制中国,麻烦还在后边,没有英美无产阶级和苏联的帮助,中国不能胜利。……

小指人家看不见,中指看得见了吧!将来变成大指,还要变成拳头,一个拳头,两个拳头,看得见不?苏联在开始时谁承认?一九三三年美国同它恢复了关系。现在世界各国差不多都与它有了外交关系,事在人为。国际无产阶级的联合是必要的,没有国际联合不行,国际无产阶级联合行动是无产阶级解放的必要条件。

两三年中国情况将有大变化,要有此准备……。

现在有三个大会,去年九月民主同盟大会,目前国民党的六次大会和我们的七次大会。三个大会有一个相同,即打日本。不同的是国民党是法西斯主义性质。昨今两天解放日报上关于批评国民党大会的文章,大家可以去看,国民党六次大会的特点是:蒋的话就是命令,中央委员要赌咒服从蒋总裁,要听蒋总裁的命令,世界任何资产阶级政党都没有这一条。另一点,不准加入其他任何政治团体,我们入党没自由,退党有自由,退党时劝一下则有之,自由还是他们的,党员有加入反革命团体者,那是保安处的事。现在国民党要自己党员赌咒不加入任何其他政党,达是表明国民党更弱了。加入其他政治团体者,他很怕,特别怕孙科搞个什么东西,不过现在赌咒没有什么用处。孙中山搞兴中会时,被别人打足模手印,蒋现在要搞赌咒,一个党弄得靠赌咒维持,其命运也就可想而知了。他们不是靠思想觉悟而是靠赌咒。国民党大会的性质和我们讲的一样,没有什么变化,它不要减租减息,而要耕者有其田。

国民党是法西斯主义;民主同盟是旧民主主义;我们是新民主主义。国民党里也有旧民主主义者。民主同盟可以联合,有人问,我们应不应去帮助民主同盟发展?我以为应看具体情况而定。在大后方,可以帮助。在我们地区就不必,它们不会来,因为我们要搞生产,吃小米,他们受不了,将来到大城市中,情况又不同了。有人问,联合政府报告发出去以后有什么影响?我们说有很大影响,在大后方发行了三万分。有人接到后,一夜未睡觉,一直看完,不喜欢,说是有史以来最大的耻辱;陈布雷看了,说只有两个字:“内战”;许多代表看了都说,共产党说的头头是道,有办法。他们的大会原订十天结束,我们的报告发出之后,把他们打乱了,又延长了几天。大会宣言本已拟好,乃又重新起草,即现在发表的这个。他们有些东西是受了我们的影响。如少数民族问题,它不得不讲讲国内民族不平等的问题。我们提出减租减息,他就来了许多条农民土地政策,这些都是反映了我们的东西,而又反对我们。中国将来可能变为以美国为主统治国民党的半殖民地,抗战前,中国是几个帝国主义统治的半殖民地,抗战中国民党依靠美国,中国可能变为以美国为主,美国插一支足的半殖民地。这将是一个长期的麻烦,我们要好好的准备,准备发生这一变化。

日本倒台是另一变化。……

有些同志,希望我讲些困难和光明,光明很多,也要准备困难,例如:

(一)外国大骂,有人向我示过威,我说我们吃小米,你们吃面包,有劲,嘴又长在你们身上,你骂我没有办法,准备挨骂。

(指美国在延安观察组,包尔德曾与毛主席谈话)

(二)国内大骂。准备国民党大骂,什么“破坏抗战”,“危害国家”,“杀人放火”……。

(三)被他们占去几大块,他们要打内战,斯科比加蒋介石。它要收复失地。总司令说搞地雷,搞地雷他还要占这大块,内战时曾占了几大块。

(四)被他们消灭若干万军队,一九四一年日本来了,我们讲准备缩小一半,后来没有。我就是不准备讲对的,是往最坏的情况准备,那时有五十万,去掉二十五万,还有二十五万,精壮了。准备吃大亏,准备将来发展到一百五十万人时,去掉1/3,还有一百万人去掉一半也有七十五万。有些准备就好办事。

(五)伪军欢迎蒋介石,伪军摇身一变,成为蒋介石的……,伪军喊阎锡山万岁,阎喊伪军九干岁。

(六)爆发内战。要用各种方法防止内战,揭露内战,内战愈推迟愈好。我们对蒋介石八年以来的政策,是使他不能投降,又不能剿共。准备日本人走了,阎锡山占过去。

(七)斯科此变成希腊,希腊六百万人口,中国的人口七十倍于希腊,要用各种方法避免斯科比的出现,万一发生了,就有理、有利、有节:一、老子:“不为天下之先”;二、左传:“退避三舍”;三、礼记:“礼尚往来”。

(八)不承认波兰,现在中指头不承认,将来两个大指头,一个拳头,两个拳头,看你承认不承认。一年不承认,两年不承认,百年不承认,我下命令,将来下命令,下早了没有用。

(九)跑掉散掉若干万党员。情况不好时,有些人会想:“当时未枓到这一着”,于是向后转,跑掉了。散得厉害的是一九二七年,还有散得多的是内战时期,结果最后至多剩下三几万,我们经过两次大散,这次准备散掉三分之一,鲁智深大闹五台山,和尚散了,我们是共产党。

(十)悲观失望,疲劳情绪,革命廿四年未成功。军队失掉了一些,党员散掉了一些,能不能发生这样情形?不要把问题瞒着,领导机关从中央到地方,讲清楚。三天睡不着觉,四天就睡着了,以前我们党内有个习惯,总不论黑暗,总说敌人总崩溃,说我们很大的胜利,百分之百的布尔什维克。现在讲困难,至多不过是黑暗,把各方面充分估计到,现在有了充分自信心,黑暗也就不是黑暗了,莫斯科如果失了,我看就可以跑掉几万(整风讲出来的)。

(十一)天灾流行,赤地千里。近得报告,华北、华中都遭天旱。孟夫子等圣人说过:“艰难困苦,玉汝于成”。太行山发生灾荒,它们就灭蝗灭灾,有办法,这是给共产党以锻炼的机会,不好事情中有好的因素,准备天灾流行,赤地千里、光光的,共产党有本领,就是要在这种严重情况下打开一条生路。

(此处遗漏一条,日本进行妥协的和平)

(十二)经济困难,有天灾更困难,没有天灾也困难,两三年内要学会作经济工作,要首长负责,亲自动手,克服困难。

(十三)有同志问:敌人退出华南、华中,华北怎么得了?现在和法西斯作战,寸土必争,不会撤退的,即使撤退,国民党也就来了,我看湘桂路敌人不一定撤退,即使都撤退到华北,难道我们就要呜呼哀哉吗?但我们就要准备它撤退到华北。总的局面,日本明年就要倒了,如果出现了和平妥协的局面怎么办?准备想法对付之。

(十四)国民党实行了暗杀阴谋,要准备万一出现了这事情,怎么办?死了又来,都准备。

(十五)党的领导机关发生意见分歧,议论纷纷,不要以为不会,有准备可能发生少一些,上述困难来,可能发生意见分歧,议论纷纷,莫衷一是,不满意,各种意见等等……。

(十六)外国无产阶级和苏联长期不援助我们,因为他们还没有来得及援动我们,我们困难来时,远水不接近火,这条辩证法,作外交工作是希望援助我们,特别是伟大的苏联,但是长期不援助,要准备,人家不是不援助,而是情况不允许。各国无产阶级未起来,苏联情况不允许。

这一切等等,还有意料不到的,全党高级干部同志,要充分准备,对付非常不利的情况。

下面来说一定能胜利的理由:

(1)暂时吃亏,永远胜利。

(2)此处失败,彼处胜利,中国革命不平衡性,东方不亮西方亮,黑了南方有北方。

(3)一些人跑了,一些人来了,要跑的让他跑掉,跑掉少吃小米。这是党内动摇分子,热闹时来凑热闹,困难时又跑了,虽然一些人跑了,但另一些人又来了。

(4)一些人死了,一些人活着,天有不测风云,人有旦夕祸福,总有活着的,不怕,这么大的民族,这么大的党,怕什么。

(5)经济困难,学会做经济工作。谢谢何应钦不发饷,我们自己来。那时,解散不赞成,饿死不愿意,剩下自己动手生产。

(6)天灾流行,太行有经验,共产党会抓蝗虫。

(7)党内纠纷,取得锻炼,生铁变成钢,要打多少次。

(8)国际没有援助,学会自力更生,准备没有援助,现在是共产党的大考验,看我们有没有本领。

当然,国际援助最后总要来,不来杀我的头。

第三、若干思想政策问题

(一)关于领导问题

领导要有预见。预见前途,偏向,如无预见即无领导,为着领导必须预见,从原始社会到资本主义社会,对于社会运动,历来都无预见盲目性的。社会发展过程中的革命往往是自发的,但人们的日常生活中是有预见的,农民春耕了种,预见秋季收获,工人做工,医生下药,都是因为有预见,某种行为一定有某种结果,资产阶级的自然科学也有预见,但社会科学是盲目的(因为是反科学的)。只有到了一八四三年世界上出现了马克思主义,才对社会问题有了预见,人类才开始走上新的阶段。共产党以马克思主义为基础,在此基础上看清前途,看清社会向着什么方向走。例如中国有三个阶级,无产阶级,小资产阶级,大资产阶级,三个阶级都在运动,它们走向何处?打败日本后干什么?开了三个大会。我们的大会预见将来走什么道路?规定了我们的路线、任务。斯大林说:坐在指挥台上什么也看不见不叫领导。人们往往对大东西看不见,微生物看不见,水牛也看不见。内战时期,大城市国民党加帝国主义,大东西看不见。我们雄心很大,总是人家什么“总崩溃”。日本是个大东西,也看错过,以为容易打(速胜论,轻敌观点),十年内战时期是十年,战争范围普遍全国。大东西吧!但有时也会忘记。陈独秀时代,几千的农民起来要求土地,大东西吧!看不见凡是政策犯错误的都是看不见大东西。小东西看不见还不要紧,犯错误也不好,但不是大错。大错是对大量普遍的东西看不见。预见则要看到小量不普遍的东西,而当它刚从世界地平线上露出一点的时候,即能看出其将来发展的意义。预见要求如此。比如这次大会要求注意城市工作,注意东北。这些东西,今天还没有,甚至今后回去还感觉不到。大城市很大,东北有四千万人口,是大量普遍的东西,有它们是明天的事。大会应该指出,如果看不见这大东西,那将了不起,共产党要灭亡。如果对工人运动、大城市、经济、工业、正规化等等不能解决,一定要灭亡。经济在人家手里,工业在人家手里,机械化在人家手里,一切都在人家手里,我们一定要灭亡。

你们搞不到大城市,到儿子、孙子还解决不了,永远要灭亡的。一定要灭亡,那就要解决工人运动、农民远动。我们是以工农起家,工业机器都是依靠工人,我们是从工人运动到农民运动,再到工人运动。

为要领导,必须预见。盲目性是妨碍预见的,教条主义、经验主义,不可能有预见,没有预见,没有领导,没有胜利,没有一切。

1843年产生马克思主义,1903年产生布尔什维克党,然后全人类才得到了方针,六十年前(1903年)得到了方针,但没有实现,十一年后,产生了世界大战,十四年后(1917年)十月革命胜利了。如果没有十月革命,中国革命又将怎样呢?从1917年十月革命起,世界的发展改变了新方向。1921年中国共产党产生后,中国的发展改变了新方向,这个党是任何历史上所赶不上的,因为它有预见,看得清前途。在党内要解决一些问题,要有预见,要去掉盲目性。

新中华报,找我题字,我当时因为所感便写了“多想”二字给他们,他们不大满意。我是要提倡同志们多想,这叫开动机器,脑筋这个器官不是为了别的,就是为了“想”。孟夫子说:“心之官则思”。孟夫子懂得这个道理,要求同志们回去告诉每一个同志,都要多想问题,对阶级、对民族、党……的问题都要想一想,想错了,不要紧,可以纠正。同志们到一块不要把生活问题变成主要话题,兴趣要提到想各种问题上,开机器丢包袱,轻装愉快。开机器,要分析,过去党内有个框子,这个框子容易学,到处用,即所谓党八股。季米特洛夫告诉我们,要打破这个框子,马克思主义的精髓即具体问题具体分析。一个问题来了,不懂得不要紧,加以分析,一个人分析不来,几个人一起分析,和同志一起交换意见,相互分析,要造成交换意见的风气。有意见不肯告诉别人,把发明看成自己的,这是机会主义。我们就是一切问题问老百姓,问同志,打仗也是这样,功劳与人家共了有何不好?“逼上梁山”、“三打祝家庄”集体创造,有何不好?发言权不是一个人的,要公诸大家,一个人搞不完全。要这样来领导,启发思想,去掉盲目性。

(二)民主集中制

××同志讲得好,放手的民主,高度的集中。性质是讲程度的,历史讲民主程度不够,集中也有不适当的,两个东西是矛盾的,但可以统一。要大开言路、开窗户、古代专制皇帝好的还能大开言路,我们共产党人还能不开窗户?开窗户,有害处,不大,但流进空气益处却很大,高度民主怕什么?正是要哇啦哇啦,哇出道理,哇错了也不要紧。归根结底,党是最公平的。他们要求集中的,要求一个连长喊口号,每个人都听一个人的号令,向敌人冲锋作战。无政府主义思想是有的,但不要怕,党的觉悟程度提高了,懂得高度集中的必要了。不要怕人雾批评,我们领导同志,资本很大,新是命,犯了错误,批不倒。只要改正错误,就是犯路线错误的,也批不倒才好。让人家讲的机会越多,意见就越少,在组织内讲讲可以,在组织外讲是小广播。小广播存在时,领导同志要搜集小广播,我不是提倡大家小广播,而是当小广播存在时,收集起来当原料,政治工厂原料要多,还要登广告,我们的广告是:“知无不言,言无不尽”,“言者无罪,闻者足戒”,“有则改之,;无则加勉”,把黑市变成合法化,就无小广播了,只有这样才能团结同志,统一意志,集中意志,形成高度集中。没有集中不能胜利。要被消灭,我们是高度民主基础上的高度集中。

(三)干部关系

这个问题的性质是农民问题。将来是全国人民的问题,列宁说:“十年与廿年有正确的关系,就在世界范围内保证了胜利(即使在世界无产阶级革命延迟的情况下),否则就要忍受廿年到四十年的白色恐怖”。这句话也适用于于中国的。如果中国十年廿年与农民关系搞好,那能取得世界范围内的胜利,否则就要忍受廿年到四十年的白色恐怖。新老干部外来干部与本地干部的关系,其性质就是与农民的关系。为什么我们三番五次的讲这个问题:古田会议不是说了,四二年整风又提出来,这是个历史的普遍问题。内战时期苏区垮了,其原因是不是一条,外来人非常相信自己,而不相信本地干部。白区也是钦差大臣一到,工作就垮了99.9%吧!外来干部本地干部的关系问题,内战时期是路线问题,抗战时期不是路线问题,在边区我就亲自看见这件事,“边区干部只能创造苏区,不能带红军”。先对红26军不好,后对红27军不好,“我是两万五千里呀!”“你是土包子呀!”现在秧歌队和老百姓一块扭秧歌,以前躲飞机外来人与本地人不走一条路!

关于军队与地方问题,是一个长期的问题,军队总是说地方对不起它,过去,我们没有系统的解决问题,系统是科学,不系统是常识,高干会上,系统的说明了这个问题,所以解决了问题。以前未系统解决,不能说服同志,有个同志说,过二年以后,才了解军队应多负责任。现在边区军队与地方关系有了很大进步,但还不是彻底解决,华北、华中还得二、三年,这一问题很易动摇,各军区、军分区负责同志要负责任。现在是一百万军队,将来还要多。能否胜利就看能否团结三万万六千万农民。要经常站起来讲三大纪律八项注意,地方也有不对的,但问题的解决是从军队起,军队的关键在军区、军分区的负责同志,那个掌舵的要坚持这个原则性。

什么雷公打死毛泽东,什么毛泽东的学问不如张国焘,什么陕北红军不够编一个师,什么陕北干部只能创造苏区,不能带红军。我们要想想二万石公粮,天怨,人怨,谢谢这个反革命(希望雷公打死我的),我们从此就搞起生产来了。张国焘的学问比我好,抗大的指导员去了之后,天天整人家的国焘路线,我没这么整他呀!我承认这一条。想一想呀!何必生气。

掌握军队的是老干部,老干部是主要骨干,这是好的,后来的同志参加领导这也是好的,北伐时期的军队不到千人了,内战时期的干部也不过两万人,老干部好,是人民的珍宝,国家的荣誉,这是人民的估价,估价很高。军队关系着国家存亡,没有人民的军队,就是没有人民的一切。军队重要,更要特别的照顾地方军队,地方关系,即军队与人民的关系。了解这一点,到任何地方去,不管地方人民对我们如何不好,都应责备自己,不应责备地方人民,对地方应抱原谅的态度,尊敬的态度,要把我们军队教育好,我们是不是人民的军队?现在是,老早就是,但是还有许多缺点,缺点很多,××同志说过了,要学会更适于当先生,教授法要经常研究,听到批评一定要研究一番,有责任研究,要有怕对不起人民的精神,要自我批评,只有这样才能解决问题。王实味写了“野百合花”之后,我们曾说要答复“野百合花”,要用解决物质问题,丰衣足食来答复他。解决经济不是易事。一样的光荣,各个方面军各个军都光荣,根据地和做白区工作的同志,现在沦陷区的同志都一样光荣。做政治、军事、经济、文化、总务工作的同志都一样光荣。技术干部等也光荣。

凡是对不起的事,承认不好,承认取消,军队与地方关系不好,承认不好,承认取消。

(四)整风、审干、锄奸

老一部整风、审干有很大成绩,也有相当错误,以后二部搞的好些,进步了,今后要照二部的进步办法做。审干搞错了人,很不好,他们不快活,我们也不快活,所谓一人向隅,满座为之不欢。现在有几百上千的人不快活,应向他们赔不是。应该是少而精,我们来了一个多而粗,数量应小小的,方法应精精的。特务在第一阶段觉得很多,第二阶段觉得很少,少是对的,以后就按这条“少”来办。自首问题,党派问题,过去我们不注意,现在知道是相当多。这个问题,也是党的严肃性问题,不能采取自由主义态度。

内战时期,打AB团,有用肉刑者,肉刑是封建时代遗留下来的,头一年已作了废止肉刑的决议,第二年还在打人,因此得到两条:(一)废止肉刑;(二)不要轻信口供。肃反走了极端者的道路,反革命应当反对,当党未成熟时在这个问题上缺乏经验,走了弯路,犯了错误,今天延安犯了错误更应抓紧,因为带有全国性。全国不走弯路,全国可以胜利。多而粗是错误的,少而精对么。犯了错误应赔不是。

有没有宗派主义?如老干部整新干部;工农干部整知识分子干部!不要忙于说没有,而是要用客观事实来证明,大会上有人要求党章上加一条:“保证党员的政治权利(生命)”,不要小看这一条,应引以为戒。几条方针也是慢慢从几个月的过程中搞出来的,有些人不知道保卫工作是件苦工作。

有人说:“军队为什么要失败,因为我没有当总书记。”“为什么锄奸搞错了,因为我没有当权。”不一定。你试试看。防止错误,也要警惕自由主义,保持党的严肃性……反革命失败了,被推翻了,反动阶级一定要报复的。如搞不好,要吃大亏。

(五)准备转变

转变是在民主革命中国形势变化而产生的,从乡村到城市,从游击战到正规战,从减租减息到耕者有其田,如何转变,我想不用多讲,要按实际情况办事,要有到城市的精神准备,工运的重要性提高了,有了大城市就要有工运,否则不能掌握城市。东北四省很重要,有可能在我们领导下,有了东北四省我们就有了东北四省的基础。现在我们的根据地还不巩固,没有基础,有了东北,就有了基础。

(六)路线问题

大会政治、军事、组织报告,给了全党一个武器,给了一个检查工作的武器。八年来就全国来说,路线是正确的;就某些时间,某些地方,某些同志,某些部门来说,有过错误或者是原则错误、路线错误,某些同志也是有时间性的。

两方面,忽略那方面都不好。

(七)军事路线

百团大战与上条同。

(八)能不能领导大资产阶级问题

或能或不能,有时能,有时不能,看情况决定。抗战开始,我们的政策对蒋介石有了影响,后来它开五中全会,要消灭我们,不听话了。斯大林说过:反帝时,阿富汗的皇帝,埃及的商人,都能成为后备军,领导权主要部分不是对大地主阶级,而是对农民,小资产阶级的问题。主要是把农民、小资产阶级从大地主大资产阶级的影响下解放出来,放在我们的领导下。“共同领导”,要看怎样做。旧金山会议五强共同领导,克里米亚会议三强共同领导,过去说在孙中山领导之下,共同领导对不对?在于如何做?你领导你那一堆,而我就放手发动群众,问题不在共同领导,而在于敢不敢放手发动群众。

(九)两党谈判有无希望

可能性还有一点,一丝一毫都没有了,那就不谈了,可能性小,也有些吧!还是要求他洗脸,改过自新。假若出现了另一种局面,他不洗脸,内战发生了,那就看情况,号召群众起来打倒蒋介石。联合政府,你不出来,我就请,明天早晨破裂,今天晚上还去请,可能不可能?来了就可能,不来就不可能。


中国解放区代表会的作用:

(一)不是第二中央政府的性质,即不是不要国民政府。

(二)带政权性质,发号施令,过渡时期的一种组织形式。

何时召集,双十节前后,他在十一月十二日开,我们开得早一点或与他同时。

(十)党外人士台作

统一战线是一门科学。学习,学习,党内许多同志还要好好学习,要学会与党外人士合作。

(十一)党性与个性

二者的关系,即普遍性和差别性,集体与个人的关系,党内与解放区内有了比较解放的个性,解放区减租减息即为了解放个性。在封建社会中,广大人民没有个性,没有自由,没有独立,没有人格,这是资产阶级的财产私有权所造成的。有财产就有个性,有自由,有独立,有人格,没有财产就没有这一切。共产党人是人民中的一部分,人民没有我们也不会有,但是党内是大大发展了个性,因为有了自由了。“共产党宣言”中讲得很清楚:“每个人自由的发展,是一切人自由发展的条件。”不能设想,一百廿多万党员变成一百廿多万块木块,我们会有什么党性。不能把每个同志变成一模一样。只要服从党,在此基础范围尽量发展个人的长处,不要只喜欢那些纸糊泥造的人,有两种个性:创造性的个性,例如一些模范工作者,特等射手,有独立工作能力者,盲目性减少不随声附和者,党性与个性完全统一,这是一种,破坏性的个性,例如有人主张标新立异。标新立异也有两种,一种是革命的标新立异,如模范工作者等;另一种是破坏性的标新立异,小资产阶级性的错误。独立性也有两种,革命与反革命的,在党内闹独立性,违反党性,分裂主义,表面上似乎自由与个性,但是破坏性的个性主义,反动性的,个人主义的英雄主义的个性,也有反动性,破坏性的,因为没有马克思主义的,反集体主义的。

对一定问题有一致意见一致行动,这就是党性,这不算损害个性,许多个性集中起来,釆取一致意见,一致行动,是完全必要的,唯物主义的一致意见,新民主主义的一致意见,难道一定要有不同意见才算有个性吗?一致好得很,整风、生产、军事、政府工作,有党性也有个性,凡是党员做的工作要有党性、个性。

(十二)利用外国经验

共产国际的经验要学。党内能有五万党员读五本重要的书,并大致了解了,那就很好。把五本书时常背上,有空就看,看完一次记上日期,七摸八摸,味道就出来了,“共产党宣言”,日本河上肇读了七八十遍。要经常注意看报上登的各国党的宣言、纲领一类的文章。共产国际对中国有极大的功劳,教条主义从那里来的?不是从马恩列斯他们那里来的,他们经常教导我们不要变成教条,是我们自己搞垮了,变成了教条主义。

对理论工作者不重视是不对的,要尊重他们,要有重视翻译工作的空气,翻译一本书就好,土包子不懂外文,你翻译出来一本外国书,对同志们有很大帮助。理论工作同志也要重视自己工作,不要因外边影响而发生动摇。

最后,实事求是

阵地要一个个夺取,力量要一点一点收集,廿五年的经验证明,敌人对我们是寸土必争。枪是一枝枝增加,土地是一块块扩大,合起来就有天下。

我们是现实主义,理想是现实主义,革命的现实主义,斯大林告诉我们要学习美国人的现实主义,他们做什么事情都很精细,粗枝大叶他们不要。此外还要有俄国人的革命热情。革命的现实主义,切切实实,一点一滴,一个一个的夺取阵地。

其次,大会以后,全党在原则上路线上团结起来,是头脑清醒的团结,而不是盲目的团结。

(讲愚公移山的故事)

同志们,我们要把中国反革命的山都挖掉。

中国人民解放军万岁!

中国共产党万岁!

