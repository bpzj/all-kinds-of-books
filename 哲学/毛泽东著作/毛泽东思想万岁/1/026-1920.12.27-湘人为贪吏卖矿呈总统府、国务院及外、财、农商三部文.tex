\section{湘人为贪吏卖矿呈总统府、国务院及外、财、农商三部文}
\datesubtitle{(一九二○年十二月二十七日)}



呈为贫吏和约卖矿,恳予严惩,以儆官邪而维湘矿,仰所钧鉴事;窃湖南矿务局长张荣楣,品行卑污,贪鄙成性,夤绿湘督张敬尧,至湘滥允政务厅长,助恶贪赃,声名狙借;嗣垂涎湘矿美利,百贫钻充该局局长,以便私图。前张督与英商葛兰特密议集卖全省矿权,皆其主谋,希图重贿;幸为农商部所发觉,及湘人全体反对,事途中止。不意张荣楣利心不死,诡谋百出,复暗串该局水口山矿师德人,韦佳克,介绍美商马意儿布流金等,以合资办理湖南白铅炼厂名义,抵押借款。其借款合同闻系湖南政府以铅砂作价入股,为美金八十万万,美商投资为美金一百四十万元,由张荣楣于本年本月十二二日下午一点钟,在京与美商秘密签定草约;期限五年,期满可续十年;期内湘省所釆铅砂概为该厂以每吨定价美余十元收买。草约签定之日,已交美金六万元。议定正约签定,再交美金十四万元。自草约签定日起,限三星期调印,逾期无效。查水口山铅厂为湖南财政惟一命源,自铅炼厂为湖南矿业唯一生路;一旦断送外人之手,不惟湘省人民永无救济之望,即湖南财政亦失整理之资。张荣楣丧心病狂,希图贿扣,饱入私囊,竟不惜以湘省重大利权轻轻断送,盗贼之行,难岂容诛,查矿业条例,凡本国人与外国人合资办理之矿叶,其资金必须均等,以护主权。今据该约所定美商实溢股六十万元,甘以主权让渡外人,显违矿业条例。此湘人所万不能承认者一。湖南水口山,系湖南自办之矿;今该约以专买权授之该厂,且予为规定价金,无论铅砂价格将来不无腾涨之时,一为该约所束缚。汉治萍与日人所订之复辙,将重见于水口山,损失自不待言。且专买权既属该厂,无异以水口山移为该厂之附属品,是于炼矿权断送之外,并采矿权而亦断送之,后患何堪设想?此湘人所万不能承认者二。又,查该约订定之日,美商交款二十万元于湘政府,是明之以合资炼矿之款,供炼矿以外之用,美商何以为报效?张督以此作何用途?此中黑幕不问可知。牺牲一省最大之矿权,填一二贪官污史之欲壑;事前绝不使湘人与闻。此湘人所万不能承认者三。总之,此次张荣楣所订之约,名为合资办矿,实系抛卖矿权。以张督治湘二年之暴政,敲骨吸髓,无微不至,张荣楣为虎作伥,惟利于嗜,又焉有丝毫计公益,恤民隐之心?以此欺人,夫谁信之?湘民百万,皆历劫余生,对于此种贻祸无穷之契约,认为与其他之短期劫夺,其关系有本身与子孙,个人与全体之别,院情汹惧誓死不承。三星期转瞬届临,此时犹在可以取销不生对外关系之日,伏恳钧座(钧院钧部)俯赐主持,趁该约未经调印之前,迅电张督,立将草约取销,交款退出;并撤惩张荣楣,以儆官邪而维矿政。不胜迫切待命之至,谨呈大总统。(国务院,外交部,财政部,农商部。)具禀湖南旅京公民毛泽东,等。

\begin{flushright}(平民通讯社印发。12月27日)\end{flushright}

(盖印 平民通讯社北京北长街99号)

(抄自中国革命历史博物馆)</p>

