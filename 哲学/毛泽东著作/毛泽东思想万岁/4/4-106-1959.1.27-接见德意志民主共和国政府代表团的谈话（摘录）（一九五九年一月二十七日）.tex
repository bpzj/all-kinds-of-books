\section[接见德意志民主共和国政府代表团的谈话(摘录)(一九五九年一月二十七日)]{接见德意志民主共和国政府代表团的谈话(摘录)}
\datesubtitle{(一九五九年一月二十七日)}


……在中国人民中肃清资产阶级思想是长期的事情。这些知识分子。我们不能不用他们,没有他们,我们不能进行工作,没有他们就没有工程师,教授、教员、记者,医生,文学家,艺术家。又用他们,又同他们作斗争。所以是很复杂的工作。不能只用他们而忽视斗争性的一面,不然过了一个时期。他们又会出来反对党。匈牙利事件对我们是一个很大的教训。在中国搞了几千个小匈牙利。北京有三十四个高等学校,我们让他们造反。有一个时期他们占上风。闹得天昏地暗,好像共产党就要灭亡了。这时候。我们准备反攻。经过几个月的辩论,驳倒了他们的谬论,多数人争取了过来,站在我们这一边。当然你们那边不能这样做,因为你们离敌人太近了。在我们这里。在每一个大学、中学、工厂、政府机关,右派想怎么做就让他怎么做,让他们尽量讲,我们一个月没有说话。并且给他们登报。这就给他们照了像,再也收不同去了。然后,我们就组织队伍,进行反攻,抓住了他们的尾巴。我们有充分的理由。不搞这一手,对社会主义是很危险的。右派的政治资本没有了,可是中间派当中还可能有人反对我们,世界上有混乱的时候,他们又会再来。我们要有准备。


