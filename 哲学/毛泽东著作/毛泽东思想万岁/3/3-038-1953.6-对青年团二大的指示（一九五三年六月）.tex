\section[对青年团二大的指示(一九五三年六月)]{对青年团二大的指示}
\datesubtitle{(一九五三年六月)}


团中央委员候选名单中,三十岁以下的只有七个,太少了,女的也太少了。老一点的成熟一点的党员参加领导是需要的,但必须增加些青年,不要搞成老年人专政,不要搞成青年老年团。

要一批一批地割韭菜,不割就不好吃了,老干部还有一万吗?可以再调出五千。

团代会的报告,有人反映有些空,是不是要求的高了。我看是满意的。三十岁以上的人不满意,二十五岁以下的看不懂。

几个文件党把的紧紧的,我看在党的领导下,团要有独立活动。几个文件都是讲中心工作,一天天干巴巴的,搞的太严了。纪律要有,一天二十四小时都要有纪律,那就不要用纪律了。

在党的领导下,参加中心工作,守纪律,青年向成年学习,这是一方面;另一方面青年要有自由活动的一面。一方面老成,另一方面不老成,天天老成是不好过的。有严格的一面,也要有不太严格的一面。

要教育青年活泼点,不是少年老成,而是生动活泼,这是原则问题。这是党同广大群众的联系问题,不然要脱离群众。

太政治化了,太老成化了,变成青年成年团了。搞的下面没条件了(指系统领导),没有独立活动了。

今天我们指导你们搞独立活动,监督你们搞独立活动。

“向歧视妇女现象作斗争”,斗争太多了。婚姻法的贯彻得十年到十五年。要教育不要斗争。

“不要自由主义!”(文件中的一句话)社会上要允许自由主义,党内要反对自由主义。背后讲是发表意见的一种方式。在广大群众中禁止这一条是办不到的。

“对上级指示,不执行不发表意见”(文件中一句话)。指示要执行,至于不发表意见,我们这个会议也有一部分发言,一部分不发言。

对青年不要管得太紧,小孩子打架,只要不打瞎眼睛,出点血没关系。鲁迅的儿子批评鲁迅说你不像个爸爸。

叫我讲什么?我就讲打架、闹。


