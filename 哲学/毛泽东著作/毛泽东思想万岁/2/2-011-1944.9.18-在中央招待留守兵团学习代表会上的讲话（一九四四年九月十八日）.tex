\section[在中央招待留守兵团学习代表会上的讲话(一九四四年九月十八日)]{在中央招待留守兵团学习代表会上的讲话}
\datesubtitle{(一九四四年九月十八日)}


“你们是部队选出来的,你们代表八路军、新四军,同时你们还代表了我们根据地的九千万人民,也代表了中国的四万万五千万人民来开会,虽然你们不是他们直接选举的,但在实际上你们所执行的纲领和工作,代表了全中国人民的要求一一打倒日本帝国主义解放中华民族。

“武汉失守以来,特别是最近两年以来,中国抗战形势发生了显著的巨大的变化。现在八路军新四军及华南人民部队抗击了在华的敌伪军六分之五,国民党只打了六分之一,豫湘战役,敌人如入无人之境,情形极为严重,中国不亡,是由于有了我们共产党、八路军、新四军,主要地由我们支持了抗战局面。这就是今天中国的抗战形势。

“我们的部队,不论在前方抗战,或在后方保卫边区,不论生产或练兵,不论军民关系、官兵关系,都有了很大的进步,但是我们还有缺点,我们要改正错误,进一步的提高。在去年拥政爱民运动中,我们运用了自我批评的方法,我们有缺点互相批评,军队要有统一领导和纪律,才能战胜敌人;正确的自我批评,对于领导和纪律,不但不会削弱它,而且只会增强它。当然这种自我批评,只有我们部队里才有,在国民党军队里这是不可能的。因为我们的军队是真正人民的军队。我们的每一指战员,以至于每一个炊事员、饲养员,都是为人民服务的。我们的部队要和人民打成一片,我们的干部和战士们打成一片。与人民利益适合的东西,我们要坚持下去,与人民利益矛盾的东西,我们要努力改掉,这样我们就无敌于天下。我们的军队一向就有两条方针:第一对敌人要狠,要压倒它,要消灭它;第二对自己人、对人民、对同志、对官长、对部下要和,要团结。这是党中央和西北局的方针,也是全体人民所要求的方针。我们的心和全中国人民的心紧紧的结合在一起,一定要打倒日本帝国主义,解放中华民族!

<p align="right">(1944年9月23日《解放日报》)</p>

