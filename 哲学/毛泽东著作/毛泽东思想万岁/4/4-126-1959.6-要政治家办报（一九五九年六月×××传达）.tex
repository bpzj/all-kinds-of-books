\section[要政治家办报(一九五九年六月×××传达)]{要政治家办报(一九五九年六月×××传达)}
\datesubtitle{(一九五九年六月)}


现在名气很糟。去年大出风头,今年不好,说我们是纸老虎,不讲信用。

这很好。不仅敌人,而且朋友也觉得我们不行。去年名气很大,人人怕我们,不但美英怕我们,很惊慌,朋友也怕,受到压力,为什么能这么跃进。现在不太怕了。英国说,三十年内中国不是值得重视的力量。朋友还说,中国不是高速度,说话不那么拘谨了。我们自己不那么神气了。

去年三月,我在成都曾说:不要务虚名而得实祸。

去年,从九月一直被动。大家头脑发涨,要搞四亿吨钢,大谈共产主义。去年十一月到郑州才发现,我狠狠批评了一下。大家在风头上,要从三千万吨落到二千万吨,也难。在武汉会议上曾提,不要搞一千八百万吨也好,一千五百万吨也成。马克思写《资本论》一百年了,看来经验要自己取得。法则不能违反,要学习政治经济学第三版。过去学了就完了,谁也没有注意价值法则,违反了它就头破血流。

现在失了信用,不要紧。苦战一年,再加一年。那时宣布跃进成绩,现在不要更正,将来再说。

人民日报办得比过去好,老气没有了。但吹的太大,有办成中央日报的危险,新华社也有办成中央社的危险。

人民日报,我看一些消息,但“参考资料”,“内部参考”,每天必看。“内部参考”是个好刊物,要改进,不要只看现象。大局在“内部参考”。怎样把“内部参考”变成报纸,是你们的工作。

“新闻工作动向”是好刊物。有一期反映了日本专家的意见,这很好。太照顾对象也不好。要吸引人看,要吸引人听。“新闻工作动向”上地方报纸提出的问题,这些问题要多反映。有的读者反对解放日报的编辑,读者对。口径不一致,是解放日报对,毛主席对?(注:这里所说的材料,都在“新闻工作动向”第四十五期上)

新闻工作,要看是政治家办,还是书生办。有些人是书生,最大的缺点是多谋寡断。刘备、孙权、袁绍,都有这个缺点,曹操就多谋善断。

多端寡要,没有要点,言不及义。要一下子看到问题所在。曹操批评袁绍:“志大而智小,色厉而内荏”,没有头脑。袁绍还有其他缺点,兵多而分工不明,将校政令不一,地虽广,粮虽多,完全可为我用。——《三国志》曹操评袁绍。

搞新闻工作,要政治家办报。

<p align="center">×××</p>

你们要政治家办报,不要书生办报。


