\section[中央关于对待原四方面军干部态度问题的指示(一九四三年六月八日)]{中央关于对待原四方面军干部态度问题的指示}
\datesubtitle{(一九四三年六月八日)}


(一)原四方面军干部绝大多数是工农出身,由下层工作逐渐提升上来的,在国内战争中表示了对革命对党能坚定与忠诚,在克服国焘路线后,又一致拥护党中央,在民族战争中也表示了他们的英勇忠诚,只有极少数的几个干部投降了敌人,但是,这是不足为奇的,原一、二方面军干部中也有极少数分子投敌。

(二)原四方面军干部在国焘路线统治时期,是服从与执行了国焘路线,但必须区别国焘路线的单纯追随者与积极执行之间的区别。只有几个人对国焘路线的发展是起了积极帮助的作用的。大多数由于文化政治水平的关系,由国焘路线的愚民政策和压迫威胁政策,由于军队的集中原则等原因,而服从和执行了国焘路线。抗战五年来的实际考验了原四方面军的干部,在本质是诚实的、坚决的,证明了一九三七年反对国焘路线时,中央所作的结论是完全正确的,这里所指的积极起帮助手的几个人中,并不包括徐向前、李先念等同志在内。他们在国焘路线时期,并未起此作用,就是对国焘路线超过积极帮手作用的几个同志,中央的政策亦是争取教育,使他们觉悟转变,而不是抛弃他们,这一政策也已收到了效果。

(三)对原四方面军的干部的信任与工作分配应当和其他干部一视同仁,不能因为他们过去执行过国焘路线而有所歧视,应该根据这些干部个人的德(对党忠诚)才(工作能力)资(资望)分配他们以适当的工作。凡原有工作不适当者,应该设法改变之,尤其重要的是帮助他们提高文化、政治、军事水平。

(四)对原四方面军干部,如果在工作中有成绩则应表扬,如犯错误(任何干部都有犯错误的可能)则应当就其错误的性质和程度加以指出和帮助其纠正错误,不指出和夸大错误都是不对的。应当有互相问的诚恳坦白的关系,任何隔阂、切断、歧视的态度都是不应存在的,还有一点要注意的是,当原四方面军干部犯有错误时,决不可轻易加“国焘路线的错误”“国焘路线的残余”等等大帽子,因为这是不合事实的,对干部的团结是极端有害的,这实际上是帮助敌人来挑拨我们的内部关系。

(五)原四方面军的干部应继续相信中央和军委干部政策的正出,放胆作事,不要畏首畏尾,凡对工作有意见时,应坦白的直率的随时向当地军政党领导提出,尤其重要的是加紧自己的文化、军事、政治学习,提高自己对党的认识和工作能力。

(六)十年内战,五年抗战已经证明我们军队干部及其他干部除个别分子外,不管他们来历如何,是团结一致的,现在抗战日益困难,日寇及国内反动分子想用一切办法来挑拨我党干部的内部关系,我们更应团结一致,消除一切因过去历史关系,地方关系而发生的任何隔阂。

(七)各地接到此指示后,应在各地各级干部全报告并讨论之。

