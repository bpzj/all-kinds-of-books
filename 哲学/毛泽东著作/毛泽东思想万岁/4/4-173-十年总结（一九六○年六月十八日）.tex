\section[十年总结(一九六○年六月十八日)]{十年总结(一九六○年六月十八日)}


前八年照抄外国的经验。但从一九五六年提出十大关系起,开始找到自己的一条适合中国的路线。一九五七年反右整风斗争,是社会主义革命过程中反映了客观规律,而前者则是开始反映中国客观经济规律。一九五八年五月党代表大会制定了一个较为完整的总路线,并且提出了打破迷信,敢想、敢说、敢作的思想,这就开始了一九五八年的大跃进。去年八月发现人民公社是可行的。赫然挂在河南新乡县七里营的墙上的是这样几个字:“七里营人民公社”。我到襄城县、长葛县看了大规模的生产合作社。河南省委史向生同志,中央《红旗》编辑部李友九同志,同遂平县委、嵖岈山党委会同在一起,起草了一个嵖岈山人民公社章程。这个章程基本上是正确的。八月在北戴河中央起草了一个人民公社决议,九月发表。几个月内公社的架子就搭起来了,但是乱子出得不少,与秋冬大办钢铁同时并举,乱子就更多了。于是乎有十一月的郑州会议,提出了一系列的问题,主要谈到价值法则、等价交换、自给生产、交换生产。又规定了劳逸结合,睡眠、休息、工作,一定要实行生产、生活两样抓。十二月武昌会议,作出了人民公社的长篇决议,基本正确,但只解决集体、国营两种所有制的界线问题,社会主义与共产主义的界线问题,一共解决两个外部的界线问题,但还不认识公社内部的三级所有制问题。一九五八年八月北戴河会议提出了××吨钢在一九五九年一年完成的问题。一九五八年十二月武昌会议降至××吨钢。一九五九年一月北京会议是为了想再减一批而召开的。我和××同志对此都感到不安,但会议仍有很大的压力,不肯改。我也提不出一个恰当的指标来。一九五九年四月上海会议规定了一个××指标,仍然不合实际。我在会上作了批评,这个批评之所以作。是在会议开会之前两日,还没有一个成文的盘子交出来,不但各省不晓得,连我也不晓得,不和我商量,独断专行。我生气了,提出了批评。我说我要挂帅,这是大家都记得的。下月(五月)北京中央会议规定指标为××吨,这才反映了客观实际的可能性。五、六、七月出现了一个小小的马鞍形。七、八两月庐山基本上取得了主动,但在农业方面仍然被动,直至于今,管农业的同志,管商业的同志在一个时间内,思想方法有一些不对头,忘记了实事求是的原则,有一些片面思想(形而上学思想)。一九五九年夏季庐山会议,右倾机会主义猖狂进攻。他们教育了我们,使我们基本上清醒了。我们举行反击获得胜利。一九六○年上海会议,规定后三年指标,我感到仍然存在一个极大的危险,就是对于留有余地,对于藏一手,对于实际可能性还要打一个大大的折扣,当事人还不懂得。一九五六年××同志的第二个五年计划,大部分指标,如钢等,替我们留了三年余地,多么好啊!农业方面则犯了错误,指标高了,以至不可能完成,要下决心改,在今年七月的党代表大会上一定要改过来。从此就完全主动了。同志们,主动权是一个极端重要的事情。主动权就是“高屋建瓴”“势如破竹”,这件事来自实事求是,来自客观情况对于人们头脑的真实反映,即人们对于客观外界的辩证法的认识过程,中间经过许多错误的认识,逐步改正这些错误,以归于正确。现在就全党同志来说,他们的思想并不都是正确的,有许多人并不懂得马列主义的立场、观点和方法。我们有责任帮助他们懂得,特别是县、社、队的同志。

看来,错误不可能不犯。如列宁所说,不犯错误的人从来没有,郑重的党在于重视犯错误,找出犯错误的原因,分析可能犯错误的主观和客观的原因,公开改正。我党的总路线是正确的,实际工作也是基本上做得好的。有一部分错误也是难于避免的。哪里有不犯错误一次就完成了真理的所谓圣人呢?真理的认识不是一次完成的,而是逐步完成的。我们是辩证唯物论的认识论者,不是形而上学的认识论者。自由是必然的认识。由必然王国到自由王国的飞跃是在一个长期的认识过程中逐步完成的。对于我国的社会主义革命和建设,我们已经有十年的经验了,已经懂得了不少的东西了,但是我们对于社会主义建设经验还不足,在我们面前,还有一个很大的未被认识的必然王国。我们还不深刻地认识它。我们要在今后实践中,继续调查它、研究它,从而找出它固有的规律,以便利用这些规律为社会主义事业服务。对中国如此,对整个世界也应该如此。

我试图做出一个十年经验的总结。上述这些话,只是一个轮廓,而且是粗浅的,许多问题没有写进去,因为是两个钟头内写出的,以便在今天下午讲一下。


