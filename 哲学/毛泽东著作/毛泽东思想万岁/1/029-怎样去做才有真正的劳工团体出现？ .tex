\section{怎样去做才有真正的劳工团体出现? }


工人要提高自己的地位,要脱离雇主的束缚,除掉努力奋斗以外,没有别的法子,因为一般的雇主,都不抱什么人道主义的,那里肯管工人的死活,即如工资一项而论,几年以前,每个工人,是二角洋一天的,如今也只有三角洋一天,加不到二分之一,生活程度之提高,那里只有二分之一呢?雇主不肯依“水涨船高”的话儿,毅然去增工人们的工资,弄得他忍饿挨饥,渐渐有不可能过活的样子,还不能得到雇主们长点怜惜。所以工人除掉努力向雇主奋斗以外,没有第二个法子。唉,可怜的工人们,自己的问题;要自己去解决。你们要改善你们的生活,也只有自己起来解决罢了!希望雇主发慈悲心,改善待遇方法,是一种慈善行为,我们羞于忍受,而且也是不可能的事。希望人家鼓吹,以达到劳工解放的目的。也是没有效果,我敢断言一句,工人不要改善自己的生活则已,要改善自己的生活,只有“扎硬暴,打死仗”,自己起来奋斗!

但工人要想和雇主奋斗,应该先有强固的团体,然后不致会失败。没有强固的团体,以此作后援,贸贸然向雇主宣战,没有不失败的。中国各地力的劳动者,也屡次发生罢工风潮,结果多半是失败。是没有强固团体之缘故。所以现在的中国劳工运动,最要紧的一着,就是赶紧组织一个强固的劳工团体,强固的劳工团体怎样去组织呢?这话说来很长,但扼要的说一句:先要劳工个个有彻底觉悟,中国的劳动家,觉悟的程度怎样?说起来很令人悲观失望,可为大多数工人从乡下地方招募而来,没有受过学校教育,所以知识一点也没有,脑子里面装满许多卑鄙的观念,人家喊破喉咙的高唱“劳工解放”,他们也不晓得“劳工解放”是什么一回事?自己是不是可以算得劳工,有的还恐怕高唱“劳工解放”的先生们是过激派。我们听信了他们的话,便是受过激派的引诱,恐怕将来要受牵累呢2这些话不是我自己臆造,诸位去仔细考察工人的心理,才相信却有这种情形,劳工们的头脑,既然这样简单;知识既然这样浅短,我们却希望他有彻底的觉悟,不是“缘木求鱼”吗?工人既然没有彻底的觉悟,我们希望他组织一个强固的真正的劳工团体,不是白日做梦吗?

但劳工们现在没有彻底的觉悟,我们终希望设法能够使他们有彻底觉悟的一日,不能因为现在劳工们没有彻底觉悟,就失掉劳工问题不讲,劳工是社会的台柱子,劳工问题不解决,社会怎样能够安宁呢?但我们究竟怎样使劳工有彻底觉悟呢?这个问题须要从工人教育问题中去解决,因为工人之所以没有彻底觉悟,就是因为没有受过教育,没有丰富的知识的缘故,如今假如使他们都有了受教育的机会,因而得着丰富的知识,自不难有彻底的觉悟了。现在各地方都有工人义务学校设立,我也参观过几处,其中名符其实确能增进工人的知识,供给工人的知识,供给工人的需要,替工人打算谋生活的改善的,固然不少,但名不符实,只挂了一块“工人义务学校”的招牌,以博人家称誉,在工人没有什么好处的也很多。我希望热心劳工教育的先生们,在没有有工人义务学校设立的地方,快快纠合同志,去开办工人义务学校,以招致工人来读书;在已经有工人义务学校设立的地方,应该加倍改良、扩充,使工人无向隅之感.果真如此,工人们的知识程度不难一天高如一天,那么工人自然能够用理论去判断新事物,怎样是雇主的专横?怎样是工人应有的权利?不愁没有彻底的觉悟了!所以我相信工人教育问题,是一切劳工问题的先决问题。工人既然有彻底的觉悟,自然因为利害关系,大家联合起来,组织一个真正的劳工团体,而且觉得不是这样,终没有改善自己生活的希望,于是所组织的劳工团体愈加巩固。上面的话对不对,很希望大家批评。

\begin{flushright}(该文发表于一九二○年十二月湖南大工报)安源工人自编劳工记(摘要)</p>\end{flushright}

