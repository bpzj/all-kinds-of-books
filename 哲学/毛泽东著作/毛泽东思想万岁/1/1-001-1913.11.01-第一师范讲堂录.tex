
\section[第一师范讲堂录(一九一三年十一月一日)]{第一师范讲堂录}
\datesubtitle{(一九一三年十一月一日)}

\textbf{修身}。人情多耽安佚而惮苦,懒惰为万恶之渊薮。人而懒惰,农则废其田畴,工则废其规矩,商贾则废其所鬻,士则废其所学,业即废矣,无以为生,而杀身亡家乃随之矣。因而懒惰,始则不进,继则退行,继则衰弱,终则灭亡,可畏哉!故曰懒惰万恶之渊薮也。

\textbf{奋斗}。夫以五千之卒,敌十万之军。策罢乏之兵,当新霸之马。如此欲图存而不亡,非奋斗不可。

\textbf{朝气}。少年须有朝气,否则暮气中之。暮气之来,乘疏懈之隙也。故曰怠隋者生之坟墓。