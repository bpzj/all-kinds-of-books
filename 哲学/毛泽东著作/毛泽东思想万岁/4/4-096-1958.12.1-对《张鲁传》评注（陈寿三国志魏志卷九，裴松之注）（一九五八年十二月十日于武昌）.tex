\section[对《张鲁传》评注(陈寿三国志魏志卷九,裴松之注)(一九五八年十二月十日于武昌)]{对《张鲁传》评注(陈寿三国志魏志卷九,裴松之注)(一九五八年十二月十日于武昌)}
\datesubtitle{(一九五八年十二月十日)}


我国从汉末到今一千多年。情况如天地悬隔。但是从某几点看起来.例如,贫农、下中农的一穷二白,还有某些相似。汉末的黄巾运动,规模极大,那是太平道。在南方,有于吉领导的群众运动,也是道教。在西方(以汉中为中心的陕南川北区城),有五斗米道。史称,五斗米道与太平道“大都相似”,是一条路线的运动,又称张鲁等三世,行五斗米道,“民夷便乐”。可见大受群众欢迎。信教者出五斗米,以神道治病,置义舍(大路上的公共宿舍)吃饭不要钱(目的似乎是招来关中区域的流民),修治道路(以犯轻微错误的人修路),“犯法者三原而后刑”(以说服为主要方法),“不置长吏,皆以祭酒为治”,祭酒“各领部众,多者为治头大祭酒”(近乎政社合一,劳武结合,但以小农经济为基础)。这几条,就是五斗米道的经济、政治纲领。中国从秦末陈涉大泽乡(徐州附近)群众暴动起,到清末义和团运动止,二千年中,大规模的农民革命运动,几乎没有停止过。同全世界一样,中国的历史,就是一部阶级斗争史。

附:

张鲁,宇公祺,沛国丰人也。祖父陵,客蜀,学道鹄鸣山中。造作道书,以惑百姓。从受道者,出五斗米。故世号米贼。陵死,子衡,行共道。衡死,鲁复行之。益州牧刘焉,以鲁为督义司马,与别部司马张修,将兵击汉中太守苏固。鲁逐袭修,杀之,夺其众。焉死,子璋代立。以鲁不顺,尽杀鲁母家室。鲁遂据汉中,以鬼道教民。自号师君。其来学道者,初皆名鬼卒。受本道已信,号祭酒。各领部众,多者为治头大祭酒。皆教以诚信不欺诈,有病自首其过。大都与黄巾相似。诸祭酒皆为义舍,如今之亭传。又置义米肉,悬于义舍。行路者量腹取足。若过多,鬼道辄病之。犯法者,三原然后乃行刑。不置长吏,皆以祭酒为治。民夷便乐之。雄居巴汉,垂三十年。(典略曰:熹平中,妖贼大起,三辅者有有骆曜。光和中,东方有张角,汉中有张修。骆曜教民缅匿法,角为太平道,修为五斗米道。太平道者,师持九节杖,为符祝,敬病人叩头思过,因以符水饮之。得病或日浅而愈者,则云此人信道。其或不愈,则为不信道。修法略与角同。加拖静室。使病者处其中思过。又使人为奸令祭酒。祭酒主以老子五千文,使都习,号为奸令,以鬼吏,主为病者请祷。请祷之法,书病人姓名,说服罪之意。作三通。共一,上之天,著山上。其一,埋之地。共一。沉之水。谓之三官手书。使病者家出米五斗,以为常,故号日五斗米师。实无益于治病,但为淫妄。然小人昏愚,竟共事之。后角被诛,修亦亡。及鲁在汉中,因共民信行修业,迁增饰之,教使作义舍,以米肉置其中,以止行人。又教使自隐。有小过者,当治道百步,则罪除。又依月令,春夏禁杀。又禁酒。流移在其地者,不敢不奉。臣松之谓。张修应是张衡,非典略之失,则传写之误。)汉末,力不能征,遂就宠鲁为镇民中郎将,领汉宁太守,通贡献而已。民有地中得玉印者,群下欲尊鲁为汉宁王。鲁功曹巴西阎圃、谏鲁日:汉川之民,户出十万,财富土沃,四面险固。上匡天子,则为桓文,次及窦融,不失贵富。今丞制署置,势足斩断,不烦于王,愿且不称,勿为祸先。鲁从之,韩遂马超之乱,关西民从子午谷奔之者,数万家。建安二十年,太祖乃自散关,出武都,征之。至阳平关,鲁欲举汉中降。其弟卫,不肯,率众数万人拒关坚守。太祖攻破之。逐入蜀。(魏名臣奏,载董昭表曰:武皇帝承凉州从事及武都降入之辞,说张鲁易攻。阳平城下,南北山相远,不可守也。信以为然。及往临履,不如所闻。乃叹曰:他人商度,少如人意。攻阳平山上诸屯,既不时拔,士卒伤夷者多。武皇帝意沮,便欲拔军,截山而还。遣故大将军夏侯惇,将军许褚,呼山上兵还。会全军未还,夜迷惑,误入贼营,贼便迟散。侍中辛毗、刘晔等在兵后,语惇,褚,言官兵已据得贼要屯,贼已散走。犹不信之。惇前自见,乃还白武皇帝,进兵定之,幸而克获。此近事,吏士所知。又杨暨表曰:武皇帝始征张鲁,以十万之众,身亲临履,指授方略,因就民麦,以为军粮。张卫之守,盖不足言。地险守易,虽有精兵虎将,势不能施。对兵三日,欲抽军还。(张鲁或张卫)言作军三十年,一朝持与人,如何?此计已定。天祚大魏,鲁守自坏,因以定之。世语曰:鲁遗五官掾降。弟卫,横山筑阳平城以拒,王师不得进。鲁走巴中。军粮尽,太祖将还。西曹掾东郡郭谌曰:不可。鲁已降,留使。(使)既未反。卫虽不同,偏携可攻。县军深入,以进必克,退必不免。太祖疑之。夜有野麋数千,突坏卫营。张卫军大惊。夜,(魏军)高祚等误与卫众遇。祚等多呜鼓角会众,卫惧,以为大军见掩,遂降。鲁闻阳平已陷,将稽颡。圃又曰:今以迫往,功必轻。不如依杜灌,赴朴胡相拒,然后委质,功必多。于是乃奔南山,入巴中。左右欲悉烧宝货仓库。鲁曰。本欲归命国家,而意未达。今之走,避锐锋,非有恶意,宝货仓库,国家之有,遂封藏而去。太祖入南郑,甚嘉之。只以鲁本有善意,遣人慰喻。鲁尽将家出。太祖逐拜鲁缜南将军,待以客礼,封闽中侯,邑万户。封鲁五子及阎圃等,皆为列侯。(臣松之以为,张鲁虽有善心要为败而后降。今乃宠以万户,五子皆封侯,过矣。习凿齿曰:鲁欲称王,而阎圃谏止之。今封圃为列侯。夫赏罚者,所以惩恶劝善也。苟其可明轨训,于物远近幽深矣。今阎圃谅鲁勿王,而太祖遂封之。将来之人,孰不思顺?塞共本源,而末流自止,其此之谓欤?若乃不明于此,而重焦烂之功,丰爵厚赏,止于死战之士。则民利于乱,俗竞于杀伐,阻兵仗力,干戈不戢矣。太祖之此封,可谓知赏罚之本。虽汤武居之,无以加也。魏略曰:黄初中,增圃爵邑,在礼请中。后十余岁,病死。(晋书云:西戍司马阎瓒,圃孙也。)为子彭祖取鲁女。鲁薨谥之曰原侯。子富嗣。(魏略曰:刘雄鸣者,蓝田人也,少以采药射猎为事。常居复军山下。每晨夜出,行云雾中,以识道不迷,而时人因谓之能为云雾。郭李之乱,人多就之。建安中,附属州郓。州郓表荐小将,马超等仗,不肯从,超破之。后诣太祖,太祖执其手,谓之曰;孤方入关,梦得一神人,即卿也。乃厚礼之,表拜为将军,遗令迎其部党。部党不欲降,遂劫以反。诸亡命皆往给之,有众数千人。据武关道口。太祖遣夏侯渊讨破之,雄鸣南奔汉中。汉中破,穷无所之,乃复归降。太祖捉其须曰:,老贼,真得汝矣。复其官,徙渤海。时又有程银,侯选,李堪,皆河东人也。兴平之乱,各有众千余家。建安十六年,并与马超合。超破走,堪临阵死,银、选南入汉中。汉中破,诣太祖降。皆复官爵。


