\section[总前委代表毛泽东给江西省行委的一封信(一九三〇年十一月二十四日)]{总前委代表毛泽东给江西省行委的一封信(一九三〇年十一月二十四日)}


省行委转左路行委,赣西行委、湘东特委:

1.我到水南知道这方面应敌工作,已有相当的准备,各乡开了群众大会,全区开了活动分子会,坚壁清野亦有准备,特别是对白军士兵的宣传工作做得好,水南街上房内房外都写满了标语、宣言和歌谣,又组织白军士兵运动委员会,这样看来,水南的应敌工作,比儒仿区之一点也没有做,吉安洋油大部未搬,盐没有搬完,宣传工作不力,要好得多。以儒仿区及吉安的情框之,西区、儒林区、纯化区、太和、永新、莲花、安福、永丰、兴国,尤其是东路的云都,各地的应敌工作准备工作恐怕也相差不远,此时值得省委推动赣西、赣东两行委及湘东特委重新严格的指示一次,去指示时要指出吉安打下后大家以为天下太平了的错误观念,要举出分宜县委、县政府之全部都覆灭,新喻县委、县政府曾受重大损失,峡江紧急会议被敌人冲散,吉安无盐并以洋油和万多斤盐资送敌人等等严重的教训。

2、报载六十师三团从鄂南开动五日完毕,六十一师继续开拔,从遂莲花方面进攻赣西、永新、安福两县委、县苏要飞速准备敌外,湘东群众和独立师要担负很大的责任,一进到永新、安福、太和、吉安等地,赣西行委就要指导各地群众实行牵制误敌,同时已进到吉安之敌,要想尽方法去牵制他,务使蒋、蔡、罗、李四师不能渡过河东,在这个任务下,红军第三军负了很大的责任,上述许多工作,左路行委和左路指挥部都要指导。左路行委和左路指挥部,截至现在正好象没有行使他的职务,既没有发一个通告指示工作,又没有发一个通报各地报告敌情,实在是非常之大的缺点,以后非改变不可。湘东方面左委左部要指挥就是(原稿缺字)行委对于湘东行委、赣西行委对于湘东行委、赣东行委、赣南行委属境,亦不能在此紧急时候,断然划分区域不加指导,最近召开的赣四行委扩大公议,赣西各县委员会联席会,雇农代表大会,均不邀莲花,萍乡、兴国、宁都、永丰的代表到会(至少应该邀他们列席旁听),真是非常之不应该。

3、水南区还不知道只发柴、菜钱,不发油,盐、米的事,别处还是相同,对于供给红军油、盐、米一事,当没有接到通告,我要财政部派两人,一往永丰、牛都,一经纯化、水南、波头、白沙,办理红军给养问题,财政部负责人员号应接应了,实际上并没有做,等两天红军到了这些地方了,油、盐、米问题全然仍不能解决,特别是盐须予从省政府的盐中批一部分到这些地方,方能供给红军,不用钱买。筹款一事,一百二十万元还只有十三万元到手(此十三万元无论何人非得总前委命令不能发用分支,务望注意)非努力进行提款不可,红军一到红色区域,一切供给都要向省政府领取,此外没有第二条出路,务望省委对于此事重新计划一番,随时严厉督促不稍疏忽是为重要!省委应该每天开一次合召集负责人参加集中指导一切。

4、我们今日可到白沙,明日可到荇田,望省委左委赣东行委赣西行委二十二军军委每天各有一个情报报总前委,务使消息灵通为要,湘东行委最近报告(内述独立师缴枪五百余等事的)在安收到。

 毛泽东

 一九三○年十一月二十四日

