\section[在成都会议上的讲话(五)(一九五九年三月二十五日)]{在成都会议上的讲话(五)}
\datesubtitle{(一九五九年三月二十五日)}


会开得很好,重点归结到方法问题,第一是唯物论,第二是辩证法,我们许多同志对此并不那么尊重。反冒进不是什么责任问题,不再谈了。我也不愿听了。不要老是自我批评,作为方法的一个例子来谈,那是可以的。

唯物论是世界观,也是方法论。我们主观世界只能是客观世界的反映,主观反映客观是不容易的,要有大量事实在实践中反复无数次,才能形成观点。一眼望去,一下抓住一、二个观点,但无大量事实作根据,是不巩固的,只有大量的事实,才能认识问题。写报告是反映下面干部和群众的意见的,要经过调查研究。省要反映地、县的情况,不详细地听取他们的意见.就冒出一篇报告来。郝是危险的。要先听训,才能训人。要老老实实听群众的话,听下级的话,个别交谈。小范围(县、社、工厂)交谈。省委解决问题如此,中央以后遇到大的问题一定要与若干省委书记谈一谈。反冒进的问题就是没有征求省委书记的意见,也没有征求各部门的意见,这个方法是不对的。在中央方面,工业部门想多搞。财贸部门想少搞一点,不仅脱离了省,也脱离了多数的部。

反冒进也是一种客观反映。反映什么呢?一般、特殊、全局、个别,这是辩证法的问题.把个别的、特殊的东西。误认为一般的、全面的东西,只听少数人的意见,广大人民群众的意见没有反映。把特殊当成一般来反冒进。

陈四害的指示,是卫生部起草的。根本不能用,这是去年的例子,这几个月的情况,根本没有接触,所以说卫生部最不卫生。后来由××找了一些同志座谈,经过反复研究写成一篇很好的指示。不然根本写不出来。如果一个指示不起作用,顶好不发表,一篇文章也是如此,如果写得不好,人家连看也不看。怎么指导工作呢,因此以后我们要注意学习唯物论辩证法,要提倡尊重唯物论、辩证法。

尊重唯物论、辩证法的人,是提倡争论,听取对立面的意见。把问题提出来,暴露了对立商。一九五七年一月省、市委书记会议时,黄敬同志对经济问题有意见,我当时的注意力在思想问题方面。没有很好注意他提出的问题,故在一月省、市委书记会议上有些问题的争论,没有展开。

辩证法是研究主流与支流、本质与现象的。矛盾有主要矛盾和次要矛盾。过去发生反冒进的错误。即未抓住主流和本质,把次要矛盾当做主要矛盾来解决,把支流当作主流,没有抓到本质现象。国务院、中央政治局开会对个别问题解决得多。没有抓住本质问题,这次会议把过去许多问题提出商量解决了。

冶金工业部党组开会,吸收了部分大厂的同志共十几人参加。空气就不同了。谈了几天,解决了许多重要问题。部如此,各省也如此,中央开会有地方同志参加,必要时,除省委书记外,再加上若干地、县委书记,就有了新的因素。中央同志一年下去四个月的。也要找地、县委书记、合作社、学校谈谈,只同省委书记谈不够,要一竿子到底,不要仅仅限于间接的东西。我很想了解一个城市、一个县的工作,把一个县各方面的问题都谈一谈。不要多长时间,有二、三个星期就差不多了。各省也应该这样做。为什么要提出这个问题呢,就是要打掉官气,当了老爷,不愿向别人请教,这种“自以为是”的态度各级都有。红安县在一九五六年的作风,不就是老爷作风吗?那怎么能指导农业生产呢?一般说来,越上越离群众远一点,但也不能一概而论,有的越上官气越小。例如列宁就没有那么老爷气。相反有不少人越下官气越大,许多乡长、厂长、党委书记,官气也不小。

越请教得多,搞出来的东西,大概比较有把握。但不能说就正确了,因为还没有证明。许多事情我自己就是半信半疑。例如鼓足干劲,力争上游,多、快、好、省地建设社会主义的总路线,究竟对不对,还要看几年。革命路线在民主革命和社会主义革命中即是已经证明了的,但建设路线还要看看。

所有制的解决,已经是一种新的关系。而相互关系和分配关系,只解决了一部分。我们的党、政、工厂、学校,不管有多少官僚主义,总是与国民党有原则区别的,所以相互关系不能完全没有变化。如果不经整风,则国民党的作风、老爷气还要大量存在,这是与国民党相同的一面。“八路军不见了”,经过整风下放干部,“八路军又回未了”。

我们讲鸣放,右派(也有中派)加了个“大”字。大鸣大放,从艺术科学转到政治方面来。我们很快就转过来了。《解放日报》有一篇“只放不收”的社论,讲一万年都不收,放手发展民主,很主动。只要抓本质、主流的问题。例如一个口号——十五年赶上英国,就会起很大作用。本质问题解决了,次要问题人们会去解决的。如果只抓枝节现象,解决就解决不了。从部分现象看问题,那是很危险的。

我们很多同志不注意研究理论。究竟思想、观点、理论从那里来呢?就是客观世界的反映,客观世界所固有的规律,人们反映它,不过是比较地合乎客观情况,任何规律都是事物的一个侧面,是许多个别事物的抽象,离开客观的具体的事物,那还有什么规律?“老子”是唯物论,还是客观唯心论?我是怀疑的。规律存在于每一个具体的人,具体的社……。反复出现。普遍存在的规律,才是普遍性的规律。比如打仗诱敌深入,战役上以多胜少,战略上以少胜多,战略上他包围我们,战术上我包围他们,等等。这是经过多少年战争,胜仗、败仗,才概括起来的,完整的体系,只能在后来完成,而不能在事先或者初期完成。对井冈山时期的十六个字战术,当时人们就怀疑,那有这样的战术法则呢?这十六个字战术法则,在苏联军事史上是找不着。但这是从群众斗争中得来的。赫鲁晓夫片面的单纯依靠原子弹是危险的事情。

一九五六年发生的几件事没有材料,就国际上的批判斯大林和波、匈事件、国内的反冒进问题。今后还要准备发生预料不到的事情。我认为要把过高的指标压缩一下,要确实可靠,大水大旱都有话可说,必须从正常情况出发。做是一件事,讲是一件事。过高的指标不要登报,登了报的也不要马上去改。河南今年四件事都想完成,也许可能做到,即使能做到,讲也谨慎些,给群众留点余地,也要给下级留点余地。这也就是替自己留余地。过去我们就有不留余地之事,例如一九四七年的土改纲领,提出“开仓济贫’的口号,后米又取消了。支票开的太多,难以兑现,对我们不利。

今年这一年群众出现很高的热潮,上海很多落后分子觉悟起来,共产主义精神大大提高。太原的协作精神,就是共产主义精神。落后的起来了,是革命的标志,是无产阶级革命的标志。现在不仅先进的起来了,而且广大中间、落后的群众也起来了。农村富裕中农不想退社了,城市的职员和落后的工人也积极起来了。我很担心,我们一些同志在这种热潮下面,可能被冲昏头脑,提出一些办不到的口号。一个县、一个地委没有多大坏处,中央和省两级必须稳当一点。我并不是想消灭空气,而只是压缩空气,把脑筋压缩一下,冷静一些,不是下马。而是要搞措施,去年是搞革命的一年,经验非常丰富,大大教育了我们,使我们懂得社会主义是些什么事情。今年再看一年建设问题,很有好处。所有制的问题,可以说基本解决了,但未完全解决。对本质问题,主要问题,要看得到,抓得起,加以分析,研究方法,求得解决。几年来许多同志就是看不到、抓不起本质的问题,自信心建筑在不巩固的基础上。也有能看得到而抓不起的人,缺乏一种魄力。

以后究竟有什么事是预料不到的?国内国际上有些什么事可能出乎意料?如世界大战,疯子要打,苏联还会发生什么问题?……原子弹把我们一套通通打烂,那也没有办法,打了再建设,可能建设得更好些。国际、国内可能发生的不可意料的危险,有多少条,各省、各部党组可以谈谈,列出一个单子来,思想上无准备不好。当然,在我国发生匈波一类的事件,可以不必料,但是,部分地区还可以发生。最近甘肃不是发生问题了吗?西藏完全可能出乱子,上层人物心在印度、英、美,对我们是敷衍的。汉族内部一点事也没有也不可能(如张清荣叛变),领导人被暗杀是可能的(如列宁,基洛夫、高尔基)。但不能因此脱离群众。

冶金部党组前次会议专搞虚业。不搞实业,这种办法要提倡,抽出一段时间,专谈思想性、理论性的问题,不伤心,讲心里话。先虚后实也可以。下次开会可以多找几个部,并且事前准备一篇报告。文章写得要有说服力,要尊重唯物论、辩证法,对本质问题要看得到,抓得起。

