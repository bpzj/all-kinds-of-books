\section[巡视大江南北回京后向新华社记者发表重要谈话(一九五八年九月二十九日)]{巡视大江南北回京后向新华社记者发表重要谈话}
\datesubtitle{(一九五八年九月二十九日)}


(毛泽东同志在九月间巡视了长江流域的几个省,在九月二十九日回到北京后,向新华社记者发表了重要的谈话。)

毛泽东同志说:

“此次旅行,看到了人民群众很大的干劲,在这个基础上,各项住务都是可以完成的。首先应当完成钢铁战线上的任务。在钢铁战线上,广大群众已经发动起来了。但是就全国来说,有一些地方,有一些企业,对于发动群众的工作还没有做好,没有开群众大会,没有将任务、理由和方法,向群众讲得清清楚楚,并在群众中层开辩论。到现在我们还有一些同志不愿意在工业方面搞大规模的群众运动。他们把在工业战线上搞群众运动,说成是“不正规”,眨之为“农村作风”、“游击习气”。这显然是不对的。”

“在大干钢铁的同时,不要把农业丢掉了。人民公社一定要把小麦种好,把油菜种好,把土地深翻好。一九五九年农业方面的任务,应当比一九五八年有一个更大的跃进。为此,应该把工业方面和农业方面的劳动力好好组织起来,人民公社应当普遍推广。”

“民兵师的组织很好,应当推广,这是军事组织,又是劳动组织,又是教育组织,又是体育组织。帝国主义者如此欺负我们,这是需要认真对付的。我们不但要有强大的正规军,我们还要大办民兵师。这样,在帝国主义侵略我国的时候,就会使他们寸步难行。”

“帝国主义者的寿命不会很长了,因为他们尽做坏事,专门扶植各国反人民的反动派,霸占大量的殖民地、半殖民地和军事基地,以原子战争威胁和平。这样,他们就迫使全世界百分之九十以上的人正在或者将要对他们群起而攻之。但是帝国主义者目前还是在活着,他们依然在向亚洲、非洲、拉丁美洲横行霸道。他们在西方世界也依然在压迫他们本国的人民群众。这种局面必须改变。结束帝国主义主要是美帝国主义的侵略和压迫,是全世界人民的任务。”

“像武钢这样的大型企业,可以逐步地办成为综合性的联合企业,除生产多种钢铁产品外,还要办点机械工业、化学工业和建筑工业等。这样的大型企业,除工业外,农、商、学、兵都要有一点。”

“搞基本建设还是采用大包干的办法好。这样可以大大地降低建设成本。”

“学生自觉地要求实行半工半读,这是好事情,是学校大办工厂的必然趋势,对这种要求可以批准,并应给他们以积极的支持和鼓励。在教学改革中应注意发挥广大师生的积极性。多方面地集中群众的智慧。”


