\section[一九四五年的任务(一九四四年十二月十五日)]{一九四五年的任务(一九四四年十二月十五日)}
\datesubtitle{(一九四五年)}


一九四四年快要完结了,我们在一九四五年的任务是什么呢?我们有些什么工作在明年要特别注意去做呢?整个反法西斯战争有很大的胜利,打倒希特勒明年就可以实现。我们唯一的任务是配合同盟国打倒日本侵略者。现在美国已打到雷伊泰岛,并可能在中国登陆。同时,日本侵略者已打通了由东北到新加坡的大陆交通线,中国的沦陷区更加扩大了。敌人是否会停止它的进攻呢?我看还不会停止,它还有可能再向我国西南部及西北部进攻。现此期间,日本侵略者必定又要玩弄诡计,企图通过中国的投降主义者,引诱中国政府投降。中国内部的状态仍然是不团结,国共谈判毫无结果,中国人民的抗日力量被中国反动派人工的分裂着。正面战场的战争,节节失败,国民党当局仍然固执其为全国人民所不满意的一党专政及其失败主义的政策,拒绝一切有利于抗战、团结与民主的建议。只有艰难缔造的广大的中国解放区,执行了孙中山先生的革命三民主义,即新民主主义,团结各族人民,建立了英勇的军队,粉碎了一切敌人的进攻,并能发动攻势,收复了广大的失地。在此种情况下,我们应该做些什么呢?

必须使全国人民明白,用人民的力量,促成由国民党、共产党、其他抗日党派及无党无派人士,在民主基础上召集国事会议,组织联合政府,才能统一中国一切抗日力量,反对日本侵略者的进攻,并配合同盟国,驱逐日本侵略者出中国。我们经过林祖涵同志曾经向国民参政会提出了这个问题,后来又向国民党当局用书面提出了,最近周恩来同志又专为此事去重庆谈判,但是依然没有结果。在目前,很清楚的,单是谈判是不能成功的,希望全国人民一致起来,大声疾呼,要求国民党当局改变现行政策,以便迅速建立民主的联合政府。这是全国人民的总任务,中国人民在大后方,在沦陷区,在解放区,都要为此目标而奋斗。只要中国有一个真正实行民主政策的能够动员与统一中国一切抗日力量的联合的中央政府出现了,中国抗日战争的胜利与中国人民的解放,就会很快胜利了。为了这个目的,大家应该想出许多办法来。


在大后方,我们必须援助被反动当局压迫的民主爱国运动,必须动员一切力量抵抗敌人的进攻,必须警惕投降主义者背叛民族投降敌人的阴谋活动。青年们及其他各界,应该有许多人到敌人占领的地方去打游击,广大群众则应当准备在一切敌人可能到的地方就地抵抗。同时,解放区则以自己的英勇作战行动及发动新地区的游击战争,有力地援助大后方。大后方的一切人民,一切爱国党派,都有责任为建立民主的联合政府而努力。大后方已经有许多党派,许多工业家、教授们、学生们,甚至许多国民党人,赞成联合政府的主张,认为这是目前抗日救国的唯一的道路。但是现在的力量还不够,应该号召广大的人民起来为此而奋斗。

在沦陷区,广大人民遭受敌人的残酷压迫,渴望解放。我们必须帮助他们组织起来,准备在时机成熟时,举行武装起义,配合军队的进攻,里应外合地驱逐日本侵略者,解放我们的兄弟姐妹们。这一任务,现在必须提到和解放区工作同等重要的地位。这是十分迫切的工件,不管如何困难都要去做。在这个工作中,法国共产党与法国人民有了光荣的榜样,我们应该学习法国的经验。在沦陷区人民中,应解释民主的联合政府之必要,使他们知道只有这个政府出现了,沦陷区人民的解放就快了,号召他们起来为这个目标而奋斗。

在解放区,现在已经成了抗日救国的重心。截至一九四四年十一月止,这里有了六十五万八路军、新四军及其他人民抗日军队,有了二百多万民兵,有了九千万被解放的人民。一九四四年一年中,我们不论在军事、政治、文化那一方面,都有了很大的成绩。但是,我们在一九四五年有些什么工作值得特别提出的呢?

我认为,一九四五年,中国解放区应该注意如下事项:

(一)扩大解放区。无论那一个解放区的附近,或其较远之处,都还有许多被敌伪占领,而又守备薄弱的地方,我们的军队应该进攻这些地方,消灭敌伪,扩大解放区,缩小沦陷区。我们必须把一切守备薄弱,在我现有条件下能够攻克的沦陷区,全部化为解放区,迫使敌人处于极端狭窄的城市及交通要道之中,被我们包围的紧紧的,等到各方面的条件成熟了,就将敌人完全驱逐出去。这种进攻,是完全必要的与可能的,我们的军队已经举行了很多这样的攻势,特别是今年有很大的成绩,明年应该继续这样作。在一切新被敌人占领,尚未建立解放区的地方,例如河南等地,必须号召人民组织武装部队,反对侵略者,建立新的解放区。几年的经验证明,组织众多的经过训练善于执行军事政治各方面任务的“武装工作队”,深入敌后之敌后袭击敌伪,组织人民,以便配合解放区正面战线的作战,有很大的效力,各地都应该这样作。

(二)敌人的进攻“扫荡”是不会停止的,我们应该经常警惕,随时准备用反“扫荡”粉碎敌人的进攻,没有这种警惕是不对的。不要以为我们强了,敌人弱了,敌我力量对比形势现在已经改变了,须知敌人还是强的,它决不会忘记向我们进攻。我们还是比敌人弱。我们还须作很大的努力,并执行正确的军事政策及其他政策,才能改变这种形势。只有到了我们占优势的时候,敌人进攻这回事才会成为不可能了。


(三)整训现有的自卫军与民兵,增强他们的战斗力。自卫军与民兵数目还不够,各地除某些个别地方不可能扩大外,均应尽量地扩大。九千万人民中,除老幼及患病者外,一切男女公民,均应组织在自卫军中,在不脱离生产原则下,轮流担任保卫家乡与协助军队的任务。从自卫军中挑选精于分子组织民兵,或基于自卫军,在“战斗与生产相结合”的原则下,协同军队作战,或者独立自主的作战。九千万人民中,至少应该有百分之五,即四百五十万当民兵,即是说,比现有民兵数目扩大一倍。有些地方,是没有十分重视这个工作。在这些地方,民兵的数目太少了,又缺乏整训,质量也不高,一九四五年必须改变此种情况。自卫军与民兵,均必须不违农时,减少误工,不妨碍生产。在这里,变工的方法是必要的,战斗的民兵组织与生产的变工组织,往往可以互相结合。自卫军与民兵的领导机关,必须是民主选举的。只有这样,自卫军与民兵才能扩大,也才能提高战斗力。民兵的重要奋斗方法是地雷爆炸,地雷运动应使之普及于一切乡村中。普遍制造各式地雷,并训练爆炸技术,成为十分必要。

(四)整训正规军与游击队。一九四五年,应将全部军队轮番整训一次,按照新方法进行整训,举行群众性的练兵运动。

(五)在老区域,补充原有军队的消耗数额。在新发展区域,在经济条件许可下,应该扩大军队。不论补充军队与扩大军队,均以不加重人民财政负担为条件,这一点必须谨记,如果违背了这一点,我们就会要失败的。

(六)军队内部的团结,非常之重要。我们八路军新四军,历来依靠官兵一致,获得了光辉的胜利。但是,中国军阀军队的军阀主义习气在我们军队里的影响,仍然是存在的。有些部门,这种习气还是很严重。一九四五年,应该进行广大的工作,将军队官兵关系中的一切不良现象,例如:打人、骂人、不关心士兵的给养、疾病及其他困难,对于士兵的错误及缺点不耐心教育说服,轻易处罚,以及侮辱或枪毙逃兵,等等恶劣习惯及错误方针,从根本去掉。许多部队,现在还未重视这一工作,由于不明白这一工作是军队战斗力的极其重要的政治基础。目前开始的一年整训计划,军事整训与政治整训应该并重,并使二者相互结合,整训开始时,还应着重政治方面,着重于改善官兵关系,增强内部团结,发动干部与战士群众的高度积极性,军事整调才易于实施与更有效果。这一工作的实行,应在每一部队内部举行拥干爱兵运动,号召干部爱护士兵,同时号召士兵拥护干部,彼此的缺点错误,公开讲明,迅速纠正,这样就能达到很好的团结内部之目的。

(七)加强拥政爱民与拥军优抗两项工作,进一步地改善军民关系。我们八路军新四军和我们解放区人民之间的关系,历来是好的,因此我们能战胜敌人,巩固与发展解放军。但是,旧军队的习气,同样地会传染给我们的,军民关系中的不良现象,例如军队态度横蛮,损害人民利益,纪律不好,不尊重政府等事,也就时常发生。同时,地方对军队帮助不够,优待抗属工作做得不好等等现象,也就存在着。一九四三年,我们曾经指出了这些工作的重要性,但是许多地方还未重视。一九四五年旧历正月,一切解放区应普遍举行拥政爱民运动与拥军优抗运动,一定要做出显著成绩来。已经有了成绩的,必须继续做,必须检查此两项工作的结果。如果我们的全部军队,官兵上下团结一致,从政治上铁一般地巩固起来了,加上军事技术与战术的训练,又加上人民的拥护,中国人民的抗日救国事业,就有了坚强的保证了。

(八)民族统一战线是中国人民救国的基本路线,在解放区,首先表现在各阶级各党派合作的三三制政府工作中。这一方面的工作各地有做得好的,有做得差的,各地均应总结经验。共产党人必须和其他党派及无党派人士多商量,多座谈,多开会,务使打通隔阂,去掉误会,改掉相互关系上的不良现象,以便协同进行政府工作与各项社会事业。凡参加人民代表会议(参议会)工作,政府工作与社会工作的一切人员,不管属何党派,或无党无派,一律被尊重,应该一律有职有权。

(九)减租,各地均有成绩。但是有些地方成绩少些,明减暗不减及恩赐观点,仍是存在的。另一方面,也有减得太多,或在减租之后不注意交租等现象。这两种偏向,都应纠正,减租之后,租约期满的除在照顾双方利益原则之下可由地主收回自种者外,应该重定新约,使农民有地可种,老区域减租未彻底的,应该查租。新区域尚未减租的,应该发动减租。租不减是不对的,减得太过火也是不对的。凡因被地主摧残或其他原因而生活困难的,政府应帮助他们解决困难,给以从事农工商业或参加其他工作之方便。要把这件事当作政府工作之一,借以团结他们反对共同敌人。我们现在是处在农村中,土地问题的正确解决,是支持长期战争的重大关节,希望大家十分注意。

(十)今年绝大多数地方都进行了生产运动,有了显著的成绩,这是非常可喜的一件大事。但是也有一部分地方还未着手进行,另有一部分地方成绩不大,另有一部分地方军队方面有成绩,人民方面缺乏成绩。一九四五年,必须绝对无例外地普遍进行大规模的生产运动。一切军队,于作战、训练之外,必须从事生产自给,机关学校也是如此。只有特殊情况者,可以允许减少或免除生产。必须动员人民,在自愿原则下,组织生产互帮团体,例如:变工队、互助组、换工班等。我们的地方工作人员,必须用极大精力去帮助人民,组织这种互助团体,以便大规模地恢复生产与发展生产,不但应该使人民够穿够吃,而且应该使人民逐渐地有盈余。“耕三余一”的口号,除被敌人摧残厉害的地方外,就是在敌后解放区,也是可能实现的。我们解放区的工业品,必须力求自给,必须争取在数年之内达到全部或大部自给之目的。由公营私营与合作社经营这样三方面的努力,达到这个目的是可能的。和生产运动相辅的是节约,必须尽可能的减少浪费。“发展经济,保障供给”,是我们确定不移的财政方针。如果我们不去从根本上发展经济,而去枝枝节节地解决财政问题,就是错误的方针。如我们努力地发展了公私经济,我们就能支持不论时间多久的战争,而使精力不至于枯竭。这一点非常重要,必须使一切工作人员及全体军民透彻地认识清楚,以便组织他们从事大规模的生产运动。在公私经济中,按质分等的个人分红制度,是在很多部分内可以实行,并应该实行的。“军民兼顾”、“公私兼顾”两个原则,必须注意不要违背。

(十一)为着战胜日本侵略者,于充分注意军事、政治、经济之外,还要注意文教工作。我们解放区的知识分子,绝大多数都是好人,他们的缺点甚至错误,可以往工作中改造,他们是人民事业的可贵资本,他们应该被重视。他们中有许多人从事军事、政治、经济工作,另有许多人从事文化、教育、艺术、卫生工作。所有这些人员,都应该被重视。今年陕甘宁边区文教工作会议所指出的方向,各地都可以适用。专制主义者利于人民愚昧,我们则利于人民聪明,我们要使一切人民都能逐步地离开愚昧状态与不卫生的状态,各地政府与党组织,均应将报纸、学校、艺术、卫生四项文教工作,放在自己的日程里面。

(十二)从军队中,农村中,工厂中及政府等机关中,用群众民主的选举方法,挂出优秀分子,充当战斗英雄、劳动英雄及模范工作者,给与奖励与教育,经过他们去鼓励与团结广大的群众,这种制度,对于提高军队的战斗力,提高农业及工业的生产力,提高政府机关及一切其他机关的工作能力,数年来的经验,已经证明是极有效果的,各地应该普遍地推广这一运动。

(十三)为着战胜日本侵略者,需要广大的有能力的干部。我们现有的干部,比较从前增加很多,但是仍感不足。干部的能力也提高了,但是仍很不够。特别是下级与初级干部,不论是军队的或地方的,他们的文化程度,他们对于政策的了解程度,以及他们工作技术的程度,一般是不高的,有些则是很低的。这种情况的原因,在于他们忙于工作,领受教育的机会太少。一九四五年,各地干部教育,应该着重于训练军队的(连至班)与地方的(区乡)下级及初级干部,在职的用轮训办法,不在职的用学校办法,有计划地将他们提高一步。

(十四)我们工作作风中的一项极大的毛病,就是有些工作人员习惯于独断独行,而不善于启发人们的批评讨论,不善于运用民主作风。当然,这是拿我们解放区的这种作风与那种作风作比较,而不是拿我们解放区与国民党区域作比较。我们解放区是民主的地方,国民党那里是封建的地方,这两个地方是原则上区别的。但是,我们队伍中确有许多人尚未学会运用民主作风,旧社会传染来的官僚主义作风,依然存在。别人提不得不同的意见,提了就不高兴。只爱听恭维话,不爱听批评话。为怕碰钉子,受打击,遭报复,人们不敢大胆提意见。这是一种很不好的作风,这种作风,阻塞着我们事业的进步,也阻塞着工作人员的进步。我提议各地对此点进行教育,在党内,在党外都大大的提倡民主作风。不论什么人,只要不是敌对分子,不是恶意攻击,允许大家讲话,讲错了也不要紧,各级领导人员,有责任听别人的话。实行两条原则;(1)知无不言,言无不尽;(2)言者无罪,闻者足戒。如果没有言者无罪一条,并且是真的,不是假的,就不可能收到“知无不言,言无不尽”的效果。自从整风以来,我们的工作作风有了很大的进步,这是受到了一切善良人民的称赞的,这是很光荣的。我们一切工作干部,不论职位高抵,都是人民的勤务员,我们所做的一切,都是为人民服务,我们有些什么不好的东西舍不得丢掉呢?如果我们改正了这个缺点,那我们就能团结更广大的人民,我们的事业就能获得更大的更快的发展。

(十五)除了上述种种以外,摆在解放区人民面前的极其重要的一项任务,就是想出种种能够促成联合政府的办法来。继续和国民党及其他党派谈判是一种办法,全国人民起来呼吁要求是一种办法,还可能有其他办法。总之,我们一定要多方努力,将这个适合全民族抗战要求的民主的联合政府,尽可能迅速地建立起来。

我们解放区已做的和要做的工作,当然还有许多,但我以为目前特别值得指出的,就是上述十五项,其他就从略了。各个解放区的情况与工作步骤在许多点互不相同,各地应按自己的特点布置工作,以便适当地完成各项任务。

一九四五年应该是中国人民抗日战争更大发展的一年,全国人民都希望我们解放区能够救中国,我们也要有这样的决心与勇气,我希望我们解放区的全体军民一齐努力,无论是共产党人与非共产党人,都要团结一致,为加强解放区抗日工作而奋斗,为组织沦陷区人民而奋斗,为援助大后方人民而奋斗,为建立民主的联合政府而奋斗。

