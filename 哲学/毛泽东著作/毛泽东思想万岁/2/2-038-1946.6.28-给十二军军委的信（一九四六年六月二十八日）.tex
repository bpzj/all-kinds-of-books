\section[给十二军军委的信(一九四六年六月二十八日)]{给十二军军委的信}
\datesubtitle{(一九四六年六月二十八日)}


谭××转十二军军委:

此次十二军工作区域分得不妥当,清流,连城应让第二军作第二步工作区,十二军中心任务应不是筹款,而是建立深入宁化、石城、长汀三县为工作区,十二军担任这三县,三十五军担任瑞金一县,三军担任雩都、会昌二县,均以两个月(七,八两月)为限期,分完田建立地方武装、地方临时政权和临时党部,把这四个问题真正的解决,使雩、瑞、石、宁、会、汀六县连成一片,这是我们的中心任务。十二军两个月要作三县工作是不容易的,二次战争后四个月中,十二军做了石城一县宁南丰县工作,现在宁南是半色区域,石城是全色区域,过去成绩如此。现委二个月做三县工作是必然难做好的,必定增加工作时期才行,若推广到清、连:县,并且耗十二军主力三十四师摆到二县去,那么十二军工作将又是毫无效果,因此须变更康都决议,把三十四师担任长汀全县,军直属队在三县之间,望坚决照此布置。

依大局看来,过去所拟三军团去二崇、四军去宁安的计划,不但客现上帮助蒋介石打击两广,为蒋介石所大愿(缺字)并且要很快?两广的共同行动乃由我们一(不清)。两广反蒋视线使之集注于我们自己,必然要促进蒋粤妥协对共的过程,我们不应如此蠢。去南丰以北,目前事实上既不许,整个策略上亦不宜,因一则无巩固政权可能,二则威胁长江太甚,西南北三面都不可变,只有东方才是好区域;第一,蒋系地盘无直接威胁两广之弊;第二,地势偏僻,即不受威胁,若较之我们去南丰宜黄者为小;第三,有山地纵横,无河川阻隔,最适宜造成新战场;第四,有款可筹,一军以内不愁给养;第五,群众很多,可以出兵扩大红军。因有这些条件我们应该在这区域作长期工作计划,三军团应以建宁、太守、将乐为工作区域,以顺昌、邵武、先泽为筹款区域,四军应以归化、清流、连城为工作区域,以沙县、永安、明溪为筹款区域即在三县筹款自给,三十五军以瑞金为工作区域,筹款自给,三军以雩都、会昌为工作区域,筹款自给。赣东独立师的中心工作区域:广昌使之联系建宁与石城所里工作区都是要分配,建立地权的筹款区,只打土豪做宣传而不分田地不建立政权,作时期暂定两个月延长不去可达两个月,敌人来了集中起来就在两个附近打,敌人不来我们就在这块工作下去。十二军在宁化、长汀、石城工作要力求工作好,最近在宁工作之失败(石城游击队尽是流氓与农通通反了水,要来一个透底的转变,坚决反对群众工作中的机会主义,如若三县工作再如过去建宁石城一样,那么军委和三个师委(不清)复重明责任。

<p align="right">总前委

毛泽东

六月二十八日下午十时于建宁</p>

石城信所说三个月缩工作时间及短二个月敌进攻紧迫十二军委并转以栗同志及边界工作委员

