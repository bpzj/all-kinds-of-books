\section[中华苏维埃共和国中央执行委员会与人民委员会对第二次全国苏维埃代表大会的报告(一九三四年一月二十三日)]{中华苏维埃共和国中央执行委员会与人民委员会对第二次全国苏维埃代表大会的报告}
\datesubtitle{(一九三四年一月二十三日)}


今天我们党所处的时期,正是中国革命形势更进一步尖锐化的时期,也正是全世界紧紧地迫近到革命战争的第一个新同期的时期。

社会主义世界与资本主义世界的对立,现在是极端尖锐化了。一方面苏联社会主义经济已取得了最后的巩固,它的第一个五年计划在四年内就完成了,第二个五年计划在去年的第一年内,又已经取得伟大的成绩。苏联早就消失失业现象,而且全体劳动人民的生活水平和文化水平都是极大的提高了。苏联的国防是极大地巩固了。最强大的美帝国主义也不能不和苏联建立国交了。

资本主义世界是另外一个样子,资本主义暂时地稳定已经终结。资本主义总的危机已经进入了一个新的阶段。各个帝国主义国家正在疯狂的进行战争,日本帝国主义占领满州的结果,使得各个帝国主义间的矛盾,尤其是日英间的矛盾在新的基础上开展起来重新分配世界的帝国主义侵略战争是正在极端地威胁着全世界民众,然而帝国主义却又在各国暂时缓和它们的矛盾。从而牺牲苏联,牺牲中国去寻找出路。反苏联战争的准备并没有一刻停止,而瓜分中国进攻中国的战争则已经在明目张胆的进行中。

但是全世界无产者阶级与被压迫民族革命运动亦在苏联社会主义建设成功的影响之下,在帝国主义经济恐慌与战争威胁之下,生长扩大起来。猛烈的阶级斗争和民主革命,是在一切与资本主义国家与殖民地半殖民地国家里面开展斗争,全世界战争与革命的火焰是逼近着我们。

中国革命是世界革命的一部分,由于民族危机的加紧,由于国民党经济总崩溃,由于苏联运动胜利,使中国革命形势更进一步发展起来,使中国革命推进到了世界革命中特别显著的地位。

目前中国的局势的重心,是广大的国内战争,是革命与反革命生死存亡的斗争,是工农苏维埃政权与国民党地主资产阶级政权尖锐化的对立,是日本帝国主义积极瓜分中国和中国民族大众为挽救民族存亡而奋斗的斗争,是帝国主义积极准备太平洋大战进攻苏联战争与中国和东方劳苦群众为阻止及消灭帝国主义大战为保护苏联的斗争。

一方面,国民党地主资产阶级,完全投降了帝国主义、引导帝国主义占领中国广大领土,垄断中国政治上经济上一切主要权利(力),引导国民经济全面崩溃,使劳苦工农群众的生活遭受看空前所未有的痛苦,剥夺一切革命民众的自由,压迫一切革命活动,实行疯狂的法西斯蒂恐怖,在帝国主义指挥之下组织一切反革命力量向苏区和红军作拚死的进攻。所有这一切都是为着一个目的,将中国地主阶级资产阶级的利益与帝国主义的利益融成一片,将中国引入完全殖民化的道路。

另一方面,苏维埃政权、号召,组织,领导全国革命民众进行坚决的民族革命战争,组织领导红军与民众为保护苏维埃领土发展苏维埃领土而斗争,以坚决地进攻粉碎帝国主义国民屡次进攻围剿。严厉镇压苏维埃领土内一切剥削分子的反革命企图。一切土地给予农民,红军士兵,工人实行八小时制,增加工资,救济失业,实行社会保险制度、给一切革命民众以集会结社言论出版自由罢工的完全自由。引进广大工农群众管理自己国家机构。只不准占人民中极少数的剥削分子参加,组织民众的经济生活,使民众生活从过去地主资产阶级统治时代受尽压迫的地位,进到不但完全免除饥寒并且日益向上改进的地位。组织民众的文化生活,将过去地主资产阶级统治时代完全没有受教育可能的广大民众,引进日益提高文化程度的地位。所有这一切,也都是为着一个目的:推翻地主资产阶级在全国的统治,驱逐帝国主义出中国,将几万万中国民众从帝国主义国民党统治区的压迫剥削下解放出来,阻止灭亡中国的殖民地道路,建立自由独立的领土,完整的苏维埃中国。

……

临时中央政府成立以来两个年头中间,目前最大的事变就是帝国主义的进攻及反革命对于革命的第四次、第五次、第六次的“围剿”。

从一九三一年九月十八日开始的日本帝国主义强盗战争,从残酷的飞机大炮的屠杀中占领了东北三省与热河、控制了平津,还正在向着内蒙及整个华北及准备更大规模的杀人战争。英帝国主义从西藏向四川进攻。法帝国主义准备从云贵,美帝国主义则欲将长江流域及福建置于其直接统治之下。所有这些帝国主义,都在以奴役中国为目的,以消灭中国苏维埃政权为目的,以准备进攻苏联为目的,同时还以准备帝国主义强盗之间的第二次世界大战为目的,向着广大的中国领土伸张其毒手与阴谋。而中国地主资产阶级国民党,却在一切奉送帝国主义的方针之下,断送了几百万方里的土地,对日本及一切帝国主义的进攻,采取了可耻的不抵抗主义,以一切中国民众的利益为代价,换得了帝国主义的政治上经济上与军事上的帮助,以便利其集中力量对于中国民族生存唯一可靠的力量苏维埃与红军的进攻。

在这种空前严重的民族危机之下,全国革命民众的反帝运动便极端猛烈地发展起来,东三省几十万义勇军的奋斗,上海十九路军兵士和工人群众的血战,普及全国的反帝运动,曾经达到了空前的高涨。

这时候在全中国革命民众面前,摆着两个政权的相反的行动,国民党完全投降帝国主义,尽量压迫反帝的民众,苏维埃则坚决反对帝国主义,尽量援助与领导反帝运动。

两年来,苏维埃临时中央政府,曾经屡次通电反对日本帝国主义强盗战争与国民党的投降卖国。一九三二年四月十四日临时中央政府正式公布对日宣战,同时颁布对日作战的动员令,号召全国民众开展民族革命战争,反对奴役中国的帝国主义与卖国的国民党。临时中央政府与军事革命委员会曾经发布宣言,号召一切进攻苏维埃与红军的国民党军队,在(一)立即停止进攻苏区,(二)保障民众的民主权利(言论、出版、集会、结社、罢工等自由),(三)武装民众创立抗日义勇军这三个条件下,苏维埃政府愿意与任何武装部队订立反对日本及一切帝国主义的战斗的作战协定,国民党与日本订立塘沽协定及最近进行中日直接交涉的时候,临时中央政府曾一再向全国全世界宣言,表示代表全国民众严厉对这种出卖民族利益的政策与行动。各地民众的反日斗争,苏维埃政府是实行援助的,单是上海沪西纱厂工人的反日罢工运动,苏维埃政府是曾以一万六千元援助他们。此外,苏区群众还有对于东北抗日义勇军的募捐援助,对于其他反帝斗争的精神上与物质上的许多援助。

至于苏维埃领土之内,则早已取消了帝国主义的特权,肃清了帝国主义的影响,牧师神父是被民众驱逐了,教会侵占了人民的财产是被收回了。教会学校是取消了,并中国境内只有苏区是脱离了帝国主义统治的地方。

所有这些事实都在指明,只有苏维埃政府应向全国民众指出:用直接的民族防卫战争战胜帝国主义,是苏维埃与全体民众反对帝国主义斗争的发展,首先是团结一切力量战胜帝国主义走狗国民党,因为它是苏维埃与民众反对帝国主义的障碍物,要使民众明白,只是因为国民党的作梗一一它横直于帝国主义进攻地带与苏维埃领土之间,并且集中一切力量向着苏维埃领土进攻,使红军无法与帝国主义直接作战,使苏维埃与红军不得不以坚决的进攻来肃清道路一一粉碎国民党的“围剿”,为实行对日作战的第一个步骤。

但是苏维埃与帝国主义之间的直接的广大的冲突,是一天一天的逼近了。这就要求苏维埃十分的加强对于一切反帝斗争的领导一一苏维埃应当成为全国民众反帝国主义斗争的组织者与领导者,苏维埃政府只有用尽一切力量使民众明白当前的危机与国民党的罪恶,依靠于广大民众反帝国主义反国民党的觉悟程度与组织力量的提高才能胜利的执行自己的神圣的任务一一以民族革命战争与革命的国内战争推翻帝国主义国民党在中国的血腥统治。

为着革命战争的胜利,为着苏维埃政权的巩固和发展,为着动员民众一切力量,加入于伟大的革命斗争,为了创造革命的新时代,群众精神上的桎梏,而创造新的工农的苏维埃文化。

这里一切文化教育机关,是操在工农劳苦群众的手里,工农及其子女有享受教育的优先权。苏维埃政府用一切方法来提高工农的文化水平。为了这个目的,给于群众政治上与物质条件上的一切可能的帮助,因为现在的苏维埃区域,虽然是处在残酷的国内战争环境,并且大都是过去文化很落后的地方,但是已在加速度的进行着革命文化建设了。

苏区缺乏完备的专门的教育建设,但为了革命斗争领导干部的创造,我们已设立了红军大学,苏维埃大学,马克思共产主义大学及教育部领导下的许多干部学校。中等教育与专门教育之应该跟着普遍教育的发展而使之教育发展起来,无疑也应该成为教育计划中的一部分。

为了造就革命的知识分子,为了发展文化教育,利用地主资产阶级出身的知识分子为苏维埃服务,这是苏维埃文化政策中不能忽视的一点。

苏维埃文化教育总方针在什么地方呢?在于共产主义的精神来教育广大劳苦民众,在于使文化教育为革命战争与阶级斗争服务,在于使教育与劳动联系起来。在于使广大中国民众都成为享受文明幸福的人。

苏维埃文化建设的中心任务是什么?是厉行全部的义务教育,是发展广泛的社会教育,是努力扫除文盲,是创造大批领导斗争的高级干部。

(注:本文系摘自“中华苏维埃共和国中央执行委员与人民委员会对第二次全国苏维埃代表大会的报告”的第八部分。)

