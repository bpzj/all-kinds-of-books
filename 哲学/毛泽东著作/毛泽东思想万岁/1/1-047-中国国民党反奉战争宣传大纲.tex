\section[中国国民党反奉战争宣传大纲]{中国国民党反奉战争宣传大纲}


--此件经×××部长提出1925年11月27日中央会议通过

中国国民党中央执行委员会宣传部,以反奉战争关系重大,其性质为反英日帝国主义的民族革命运动之序幕。各地各级党部亟宜指挥全体同志,作广大之宣传,使全国国民了然于此战争之原因及目的。特别是宣传大纲,首为反奉战争各方面之分析,次举宣传纲领,次列九个口号。经毛泽东部长提出11月27日中央会议通过,兹录全文于左:

(甲)反奉战争各方面的分析:

(一)帝国主义方面:去年直奉战争,起于英美帝国主义欲扶直系统一中国排斥日本帝国主义势力之企图。其爆发以江浙战,乃美国帝国主义欲垄断江苏无线电报借款,排斥日本无线电报借款。故其时英国及亲美派竭力助齐燮元,而日本及亲日派乃竭力助卢永祥。此次反奉战争仍然一脉相承,日本帝国主义站在奉系背后,美国帝国主义站在直系背后,唯英国帝国主义在去年奉直战后,鉴于直系之不中用,于五卅运动全国反英时,不得不极力与日本协调,以重利勾引张作霖镇压上海的反英运动,后且相传英国以巨款助奉扩充奉天兵工厂,企图以奉系统一中国之说,此时直系以杨宇霆督苏上海增兵和在北京所开关税会议有利奉张之故,不得不早日发难,狡猾的英国帝国主义在此次战役内究竟助奉助直,此时尚难完全断定。大概日本为对抗美国计,宁愿拉拢英国以共同役使张作霖,然英国如察张作霖有不利形势,而其旧仆吴佩孚有胜利的可能时,为巩固其长江流域势力范围,他会舍新欢而联旧好,也是意中之事。故英国帝国主义的态度,将看何方胜利把握校多即助何方。

(二)军阀方面:在直奉对峙的局面中,直系方而有湖南、湖北、江西、安徽、江苏、浙江、福建七省地盘。四川之袁祖铭支配下之贵州,名义上也属直系。但这几省内部各有特殊的派别,吴佩孚、孙传芳亦必相互分裂。在反直战争剧烈时期,孙吴自须联合作战,现在战事停顿已现裂痕,将来战胜奉张,必然分裂无疑。奉系方面自来即有老少两派。去年胜直以来,因权利的分配内部暗斗日甚。财政竭蹶,奉粟跌至五折以下,前之攫取直鲁苏皖,即为解决财政问题。此时苏皖已失,又因国民军之威胁,山东及直隶之京汉线亦将不釆,关内财源尽矣,聚数十万饥军于山海关内外,其势力于速战而不利于持久。直奉两派军阀,无论那一派胜,均于中国不利,因两派背后均有凶恶的帝国主义。唯在全国反奉运动中,直系之反奉自不能不认为一员以共同对付目前强敌,奉倒再以国民之力肃清直系,乃系作战策略之必要。

(三)政派方面:在此次反奉战争中,政派之态度可注意者如安福系、研究系、联治系、新外交系,及上海南通等地之买办阶级。安福系,在军事上的势力此时可说已经没有。在其政治上的势力,早已裂为亲直亲奉二派。唯因亲直失势,亲奉派在北京当权,故北京政府变成完全代表张作霖的东西。研究系自曹吴当国即付曹吴。曹吴虽失败,仍为吴佩孚之幕。此次直系抬头,研究系蒋方震等及该系猪仔议员麋集汉口,借以攫取利益。联治系乃合政学系益友社及所谓国民党同志俱乐部一派政客如章炳麟等许多小政派而成,现时亦均聚集在吴佩孚旗帜之下以谋活动。顾维钧一派所谓新外交系,一向是直系与英美帝国主义之间的卖国经纪人,此派现与研究系及上海南通等地买办阶级深相结纳,一向聚在直系旗帜之下,努力做其买国运动。上海及南通等地买办阶级,去年反直战争时因美国帝国主义之指引站在直系一边。此次直系再起,亦马上从其主人(美国帝国主义)的意旨,做了直系响亮的应声,而且是直系有力的柱脚。以上各政派除代表日本帝国主义与官僚利益的安福系属于奉系一边外,其全代表官僚及地主阶级利益的研究联治系,代表英美尤其是美国利益的新外交系,与上海南通买办阶级,都站在直系一边。

(四)国民军方面:国民军与英美日本帝国主义都没有联系,因此同情于反帝国主义运动,这是国民军最大的特点。此时因策略上必需尚未与张作霖决裂,或者取暂时的妥协,但这是一时的事。本党欲图接近国民革命之成功,在反奉战争后有一个长足的进步,则国民军在北方之胜利实为重要关键之一。

(五)国民政府方面:本党在广东的基础现已十分巩固,北江熊克武部早已解决,东江陈炯明又已肃清,其小部窜入闽边者已派兵追剿,务绝根株。南路邓本殷部亦不日可以解决。全境统一可说业已告成。英国帝国主义勾结陈炯明等消灭本党革命势力之企图,业已完全失败。省港罢工问题,港商港政府知无别法可以对抗,现正力图转,不日可得到胜利的解决。目前本党在广东方面所致力者,为革命军军力之精炼扩充,民政财政司法教育之刷新整顿,工农商学民众运动之扩大。总之以最短之时间,积极准备实力,候南北形势发展到相当时限,即发兵北进,领导全国国民为国事之彻底解决。本党业用中央执行委员会名义发表对时局宣言,指明反奉战争之目的。并用国民政府委员会名义致电直奉两方重要将领,免其一致推倒奉张,并于张奉势力倒台后建设合于民众目的之政府及政策,以验其对于本党拥护民众利益主张迎拒。

(六)民众方面:此次反奉运动在民众意识方面,乃为反抗拥护英日帝国主义压迫爱国运动之奉系军阀的运动。故此次反奉运动的主体应该是全国的革命群众,直系的发动,仅仅是一支先发队,不能算作主体。此时民众之愤怒奉系军阀为历年来所未有,全国民众之反奉即反英日帝国主义,反奉胜利即反英日胜利的观念,与广东民众之讨伐陈炯明即攻击英帝国主义,东征胜利即罢工胜利的观念是一样,故此次反奉战争在民众意织方面与在“直皖”“奉直”几次战争时都不相同。

(乙)我们的宣传及准备

各地各级党部的负责同志,须有组织有计划的觅得各种机会,于同志中尽力解释,于民众中尽力宣传以下各点:

(一)各国帝国主义在此次战争中的阴谋。

(二)为英日帝国主义走狗的奉系军阀如胜利,则民众将受到绝大的危险。

(三)直系反奉,民众可以利用一时,但不可任其代替奉张执政,因直系代替奉张执政,人民亦将受到极大危险,直系当国前例,人民不应忘记。在长江各省商人阶级欢迎吴佩孚、孙传芳空气极盛地方,尤宜注意宣传此点。

(四)各种反革命派如安福系、研究系、联治系、新外交系、买办阶级都是以人民利益绝不相容,不可不揭其阴私和一律排斥。

(五)在各派反奉势力中,冯玉祥一派与吴佩孚、孙传芳一派不同之点,即冯与帝国主义无缘,赞助国民革命,吴、孙则受帝国主义指使,反对国民革命。故人民于敌友之分辩,全看其与帝国主义有无关系,无论何人何时一与帝国主义发生关系,人民即不认之为友。

(六)真正人民的领袖,乃中国国民党。真正人民的政府,乃广州国民政府。真正人民的军队,乃广东的国民革命军。因为国民党、国民政府、国民革命军乃反对帝国主义的急先锋,人民利益的拥护为人民痛苦的慰劳者(举广东反抗英国帝国主义及此次统一广东和积极建设的事实。)

(七)被压迫的中国全体民众,乃一切中国问题的主宰,此次反奉战争,人民应该是总指挥的。人民应该赶快组织起来,主持这次反奉大运动。

(八)国民党对时局主张之四条:(1)建设统一全国之国民政府。(2)此国民政府必于最短时间召集国民会议预备会议。(3)此国民政府必于最短时间召集国民会议预备会议,对于不平等条约为根本之解决。(4)此国民政府必保障人民言论结社集会之自由。乃结束此次战争的唯一方法,不照此四条,则战争结局,仍然是帝国主义军阀合作支配的局面。人民仍然要受与以前同样的危险。

(九)为实施国民党的主张计,应该赶快准备真正的人民代表的国民会议。在各种人民团体中,“国民会议解决国事之必要”应继续去年的宣传重新奋起一个普遍的宣传。各种党部各特别市党部于必要时,应该在所辖范围内全体动员,对于国民会议作猛力之宣传。以期唤起民众之注意。

(丙)口号:

(一)打倒张作霖、段祺瑞。

(二)打倒英美日本帝国主义。

(三)打倒一切阴谋政派。

(四)人民起来指挥反奉运动。

(五)以人民代表的国民会议结束反奉战争。

(六)建设统一全国的国民政府。

(七)取消不平等条约。

(八)集会结社言论罢工自由。

(九)一切革命分子速加入国民党。

<p align="right">抄自“政治周报”第一期第8-11页

一九二五年十二月五日广州政治周报社出版</p>

