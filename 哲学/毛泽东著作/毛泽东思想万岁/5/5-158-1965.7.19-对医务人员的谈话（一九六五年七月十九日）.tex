\section[对医务人员的谈话(一九六五年七月十九日)]{对医务人员的谈话}
\datesubtitle{(一九六五年七月十九日)}


(×××说明主席对卫生部的批评是一针见血,要切实改正。)

主席:城市医生下乡不一定高兴,在城市住惯了。可不要相信有些人嘴上说的那一套,要看。嘴上说的好,不一定。

北京医院改的怎么样了?

(说明北京医院目前情况。)

主席:北京医院并没有彻底开放。×××、××就不能去看病,××、××可以去看,这不是贵族老爷医院是什么?要开放,给老百姓开放。不要怕得罪人。这样做得罪了一批人,可是老百姓高兴。这批人不高兴让他们不高兴好了。做什么事总要得罪人,看得罪的是些什么人,高兴的是什么人,老百姓高兴就行。

(说明北京医院改了,中央改了,可以影响地方。)

主席:不一定,他可有他的办法呢。反正是扫帚不到灰尘照例是不会自己跑掉的。

县卫生院认为赚钱的医疗队就好,不赚的、少赚的就不好,这难道是人民的医院?

药品医疗不能以赚钱不赚钱来看。一个壮劳力病了,给他治好病不要钱,看上去赔钱,可是他因此能进行农业或工业生产,你看这是赚还是赔?×××告诉我,在天津避孕药不收费,似乎赔钱,可是切实起到节制生育的目的,出生率受到控制,城市各方面工作都容易安排了,这是赔钱还是赚钱?

有些医院,医生就是赚钱,病人病不大或没有什么病也要他一次次看,无非是赚钱。甚至用假药骗人。有两个十七、八岁的青年,说检查了,有脊柱病。我说不要信,这是他们骗。要他们去休养,两三个星期回来了还不是照常上班。搞一些赚钱的医院赚钱的医生、假药,花了钱治不了病,我看还不如拜菩萨,花几个铜板,买点香灰吃,还不是一样?

最近政治局要讨论一次卫生部的工作,××同志已经告诉我了。他找他们谈过。

(说明卫生部现在正讨论具体办法,很想在政治局讨论之前,主席先接见一次,再给以指示。)

主席表示同意。


