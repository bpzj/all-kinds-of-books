\section[谨防扒手(社论)(一九四一年五月二十三日)]{谨防扒手(社论)}
\datesubtitle{(一九四一年五月二十三日)}


自从晋南战局紧张,全国军民,固无不淬砺奋发,若干幻想日寇侵华战争可以照原状不了了之的人们,也颇有如梦初醒模样。忧时之士,群以为中国的抗战大业或者会由此一变已往长时期的龟步,而得至到迅速的开展了。

但是复杂的国际国内政治关系,还不容许我们立刻如此乐观。在我们的面前还存在着一大串问题,要求我们严重的警戒和密切的注视。其中尤其耐人寻味的,就是:在英美对德手忙脚乱的今天,其拍卖中国以换取日本让步的远东慕尼黑阴谋已经最后放弃了么?在日本最近对中国的新进攻与大轰炸的合奏里,会不会有出卖中国人民的“绥靖”乐曲在旧调重弹呢?不估计到这些问题就一味乐观起来,是危险的。

如同×××所说,英美对于中国,一向就缺乏援助的诚意。英国对日妥协的危机,就是平素与英国来往还最密切的人物现在也不能不承认了,但是美国这位朋友的真面目,可惜许多人还是茫然。特别是在三国同盟以后,美国援华似乎起劲了些,虽然这种援助较之中国的需要是杯水车薪,较之苏联则望尘莫及,而朝野亲美援美之声,却颇已洋洋盈耳。这些先生们的眼睛自然是看到了他们所愿意看到的东西,但是却决没有看尽了他们所应该看到的。美国对日本取消锌与橡皮的输出限制,在美国影响下的荷印与日本缔结煤油协定,日本国内英美派的暗中活跃,在在表示美日当局双方的激烈词令,只是词令,“政策”还是政策。如果美国对日本的支持实际大于其对中国的支持,则美国的援华姿态谁敢断定不是和日本的南进姿态一样,只是觅寻美日妥协的一种必要的手腕?谁敢断定,在远东慕尼黑阴谋的死灰再燃时,我们的全国拜物教的教徒们能逃脱兔死狗烹的悲剧?

美联社十一日透露的中国当局关于日本也欲和平必须经由美国提出条件的声明,以及苏联报纸十八日所揭发的日本正在“径由美国”调解中日战争的消息,十九日松冈招待美驻日大使格鲁于私邸,日情报局发言人并称格鲁极为仁爱,不幸证实了这种危机的存在。我们相信,已经英勇抗战了四年,拥有如此强大的抗日人民,抗日政党和抗日军队的中华民族,是不可能被出卖,不允许被出卖的,中华民族任何有骨气有自尊心的领导人物,即令在美日妥协实现时,也是不甘心做汪精卫第二,躬蹈身败名裂,众叛亲离的复辙的。但是我们愈有击破这种阴谋的信心,就愈有击破这种阴谋的必要。我们提醒全国的军民说:在我们走向胜利的路上,不仅荆棘丛生,而且路旁随时有扒手在窥伺着,想乘我们的不备,窃取我们奋斗的果实而去。对于戴着朋友的而具的扒手,我们需要有更多的警惕!

我们更要正告美国当局和国民党中一部分惟美国马首是瞻的先生们说:中国人民毫不迟疑的欢迎美国对中国抗战的援助,这种援助愈多愈快愈实在愈好。“有朋自远方来,不亦乐乎”,中国人民从来就是懂得以朋友之道对待朋友的。但是他们同样懂得怎样去对待扒手及其同谋者的。这岂不合于你们所最爱鼓吹的“公道”(FairPIay)吗?其中国人民是已经长成盘石了,但你们却硬说还是皮球,你们一定要踢开就请踢踢看吧。可是,留心你们自己的高贵的足趾。中国人民是要顽强地战斗下去的一一直到其战胜日本帝国主义之日!

<p align="right">(一九四一年五月廿三日《解放日报》)</p>

