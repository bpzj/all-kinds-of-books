\section[在为八届六中全会作准备的郑州会议上的讲话第五次讲话(一九五八年十一月九日)]{在为八届六中全会作准备的郑州会议上的讲话第五次讲话}
\datesubtitle{(一九五八年十一月九日)}


废除历史上的家长制,使住宅的建设便于男女老少的团聚,紧张的时候可以分开。建筑住宅斩头去尾,是强制办法,只废除历史上的家长制,现在的家仍需要一个长,是能者,不一定是长者。

商品同商业,这个问题都是避开这一方面的,好像不如此,不是共产主义似的,人民公社必须生产宜于交换的社会主义商品,以便逐步提高每人的工资。在生活资料方面,必须发展社会主义的商业,并且利用价值法则的形式,在过渡时期内,作为经济核算的工具,以利逐步过渡到共产主义。现在的经济学家不喜欢经济学。斯大林临死前,谁说到价值法则就不荣誉,表现在给雅罗申柯写的信上,苏联的一些人,不赞成商品生产,以为已经是共产主义了,实际上差得远,我们只搞了几年,则差得更远。

列宁曾经大力提倡发展商业,因为城乡有断流,我们五零年也有过,现在运输不好,有半断流状态。我看要向两方面发展:一是扩大调拨,一是扩大商品生产。不如此,就不能发工资,不能提高生活。

资产阶级法权,一部分必须破坏,如等级森严,居高临下,脱离群众,不以平等待人,不是工作能力吃饭,而是靠资格、靠权力,这些方面,必须天天破除。破了又生,生了又破。解放后,不利用供给制的长处,改行工资制,一九五三年不改也不行,因为解放区工作人员占多数,因为工人阶级也是工资制,因为新增加的人多,他们是受到资产阶级的影响的,要他们改供给制,不容易,那时让一步是必要的,但有缺点,接受了等级制,等级森严,等级太多了,评成三十几级,闹级别,闹待遇。这些也让步,就不对了。经过整风,这股风降下来了。这种不平等的干群关系一猫鼠关系或父子关系,必须破除,这个关系完全不必要。去年到今年给资产阶级法权很大的打击。过去搞试验田,干部下放,正确解决人民内部矛盾,用说服不用压服,因而空气大有改变。没有这种改变,大跃进是不可能的。不然,群众为什么不睡觉,不休息,而工作二十小时?因为共产党跟他们在一起。红安县的干部过去是老爷式的,挨群众骂,五六年下半年一改,有大进步,群众欢迎。

另一部分是要保留的,保留适当的工资制,保留一些必要的差别,保留一部分多劳多得。一部分是赎买性的,如对资产阶级、资产阶级知识分子和民主人士,仍保留高薪制。资产阶级可让他当社员,但还戴资产阶级帽子(帽子不摘)。社员分两类,一是工农社员,一是资本家社员。

为了团结,中央、省、地三级,要有一个纲领。省委以上不把目标与相互关系搞清楚,如何行?现在有些问题相当混乱,一传就传出去了。发到哪一级,要做政治考虑,粮食,人家不怕,主要是钢、机、煤、电四项吓人。时间可提“十五年或更多一点时间”,指标指“××”,只口讲。林业将变成根本的问题之一,林业以后提牧业、渔业、蚕桑、大豆要加上.林业是化学工业、建筑工业的基础。消灭水旱灾害,要加上“最大限度地”。大部归人管,一部归天管,一万年也如此。肥,有机肥为主。除四害与其他主要害物,也要加上“最大限度地”。劳动,每天睡眠休息不得少于十二小时,学习二小时,最大限度的劳动时间,不得多于十小时。这是全国的大问题,要把公共食堂服务的工作,看做是为人民服务的一种崇高的工作,要把为托儿所、幼儿园服务的工作,看做是为人民服务的一种崇高的工作。每个劳动人员都是国家工作人员。

三十六条,商品生产:调拨,我完全赞成。为什么公社与公社之间实行合同制,国家与公社之间不可以订合同呢?这可能触犯“左”派。我们现在的商品生产,不是为价值法则所指挥,而是为计划所指挥,我们的钢铁、粮、棉,难道都是价值法则所指挥吗?铜、铝过去未指挥到,今后要转移力量去搞。

过去讲一为国,二为社,三为己,但生产者倒过来,一为己,二为社,三为国,尽管我们如此说:“保家丑国”,“要发家种棉花”,“爱国保家,多种棉花”。

三十八条,要加“为了准备在侵略者如果发动侵略时彻底打败侵略者,实行全民皆兵的制度”。

三十九条,一则以喜(作社员),一则以惧(保姆,高薪)。要对工人讲清楚:优待资本家是为了孤立他们,让他们特殊,个人突出。小的早入,中的不一定。

四十条,一堆观点,不满意。七个观点谁也不看。中心是解决实行群众路线的工作方法,不要捆人、打人、骂人、辩人、罚苦工,营长对连长都如此,“辩你一家伙”,徐水不止一个,捆连长、打连长、骂连长、辩连长,因此人人怕辩论,辩论变成了斗争会,辩论变成了一种刑罚。两种性质的矛盾,两种不同的辩论。一种是对右派,一种是人民之间的,是说服。

提倡实事求是,不要谎报,不要把别人的猪报自己的,不要把三百斤麦子报成四百斤。今年的九千亿斤粮食,最多是七千四百亿斤,把七千四百亿斤当数,其余一千六百亿斤当作谎报,比较妥当。人民是骗不了的。过去的战报,谎报只能骗人民,不能骗敌人,敌人看了好笑。福建前线,飞机损失对比为我六比敌十四,即一比二点三三。但同民党自吹自擂有必有假,真真假假搞不清。偃师县原想瞒产,以多报少。也有的以少报多。人民日报最好冷一点。有些问题讲热了要讲得适合当前。要把解决工作方法问题,当成重点。党的领导,群众路线,实事求是。

斯大林的《苏联社会主义经济问题》一书要再看一遍。省委、地委的常委以上的干部要进行研究.过去大家看了,不感兴趣,印象不深,现在不同了,应当结合实际进行研究。

这本书的一、二、三章有许多值得注意的东西,有些是正确的,有些不妥的地方,有些可能斯大林自己也没有搞清楚的。

第一章,客观法则,提出计划经济与无政府状态对立。他说计划法则与政策有区别,很好。主观计划力求适合客观法则,提出了问题,但没有展开,可能他自己也不太清楚。在他心目中,认为苏联的计划基本上反映了客观法则;但程度如何,值得研究。如重工业与轻工业的关系,农业问题,未完全反映,他就吃了这个亏。人民不能从中看到长远利益与当前利益的结合,一直到现在,他们的商品产品比我们少,这是铁拐李走路,一条长腿,一条短腿,手扶拐杖,比较偏颇。我们现在的提法是:在优先发展重工业的前提下,发展工业与发展农业同时并举,两条腿走路。同我们的计划比较,究竟哪个更适合有计划按比例的客观法则?还有一点,斯大林强调技术,强调干部,只要技术,不要政治;只要干部,不要群众,这也是二条腿。在工业上注意了重工业,也是一条腿,没有注意轻工业。在重工业内部关系上,没有提出矛盾的主要方面,讲钢是基础,机械是心脏,我们提出农业以粮为纲,工业方面以钢为纲。这个辩证法,我们也是最近摸到的。我们提出以钢为纲,就有了原料、机械、煤、电、石油、运输,海陆空都上来了。第一章斯大林提出了问题,提出了客观规律,但是怎样掌握规律,没有很好的回答这个问题。

第二章讲商品,第三章讲价值法则。你们有什么意见,我相当赞成其中的许多观点。讲清楚这些问题很有必要。也有些问题,如把商品限制在生活资料方面,说“生产资料不是商品”,就值得研究。生产资料在我国还有一部分是商品,我们把农业机械卖给合作社。

我看斯大林的最后附的一封信差不多完全是错误的,把国家与群众对立起来、基本观点是不相信农民,不放心农民。对农业机械啃住不放,一面说生产资料归国家所有,不卖给农民;一方面又谈农民买不起,买了会损失不起,国家就损失得起。这理由是说不通的。实际是自己骗自己。把农民控制得要死,农民也就控制你。主要是没有找出两个过渡的方法来,没有找到一条解决从集体所有制过渡到全民所有制的道路。说也说得好,工农、城乡对立消灭了,本质差别也消灭了,但积三十年的经验,也没找到出路,从信中可以看出,斯大林对这一点很苦恼。

斯大林说,社会主义的商品,不是把旧社会遗留下来的商品保存下来,而是新式的商品,价值规律在我们生产中不起调解作用,起调解作用的是计划,这很对。几年来,看得很清楚,我们的大跃进是计划的大跃进。是政治挂帅。

斯大林只谈生产之事,不谈上层建筑。(××:他只看到工农业之间的矛盾)他批评雅罗申柯是对的。但他不谈上层建筑与基础的关系,没有谈到上层建筑如何适应经济基础,这是重大的问题,我们的整风,下放下部,两参一改,干部参加劳动,工人参加管理,破除不适当的规章制度等等,都属于上层建筑,都属于意识形态。

在斯大林的经济学里,只谈经济关系,不谈政治。虽然报纸上讲“忘我劳动”,实际多作一小时也不行,每小时都没有忘我。斯大林见物不见人,见人也是见于部之人,不见群众之人,人的作用,劳动者的作用不谈。如果没有共产主义运动,想过渡到共产主义是困难的。

“人人为我,我为人人’的口号,不妥当,结果都离不开我。有人说,是马克思讲过的,是马克思讲过的我们可以不宣传,人人为我,是人人都为我一个人,我为人人,能为几个人。斯大林的经济学里,是冷冷清清,凄凄惨惨,阴森森的。

资产阶级法权,法权思想,法权制度等问题。列宁曾提出“全线进攻”的口号,当时新经济政策实行了一年,急了一些,在苏联像荣毅仁这些人统统丢到海里去了,而教育组织还是资产阶级式的。我们对资产阶级法权思想,要破除一部分,即老爷架子,三风五气,不以普通劳动者姿态出现,要坚决的破。但商品流通、商品形式、价值法则,则不能一下子破,虽然它们也是资产阶级法权范畴。现在有些人宣传破除一切资产阶级法权思想,值得注意,这种提法不妥。

在社会主义社会中有少数人,如地主、富农、右派他们想回到资本主义,提倡资本主义,绝大多数人是想进到共产主义。进到共产主义要有步骤,不能一步登天。如人民公社要向两方面扩大,一方面发展自给性生产,一方面也要发展商品生产。我们现在要利用商品生产、商品交换、价值法则,作为有用的工具,以利于发展生产,利于过渡。斯大林说了许多理由,消极方面有没有?我们国家是商品生产很不发达的问题。去年生产粮食三千七百亿斤,三百亿斤作为公粮,五百亿斤作为商品卖给国家,即不到三分之一的商品粮。粮食以外的经济作物也不发达,如茶、蚕、丝、棉、麻、烟,都没有恢复到历史上的最高产量,要有一个发展商品生产的阶段.现在还有很多县,搞了吃饭不要钱,就发不了工资。

例如河北省分三种县,一部分只能吃饭,一部分要救济,一部分可发工资。发工资又分几种,一种只发几角钱。因此,每个公社在粮食以外要发展能卖钱的东西。钢铁赔钱。一是学费,一是支援国家工业化。

发展社会主义的商品生产和商品交换。不同的工资要保留一个时期。必须肯定社会主义的商品生产和商品交换还有积极作用。调拨的只是一部分,多数不是买卖。商业赚得太多了。现在有一种偏向,好像共产主义越多越好。共产主义要有步骤。范县两年实现共产主义要调查一下。还是慢些好。

总之,我国商品不发达,进入社会主义,一要破除老爷态度、三风五气,一要保留工资差别。现在有些人总是想三五年内搞成共产主义。

经济学家很“左”,怕叫人抓到了小辫子。企图蒙混过关,以《四十条》草案为据。


