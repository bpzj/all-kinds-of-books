\section[对人民日报文艺宣传的批评(一九六四年六月中旬)]{对人民日报文艺宣传的批评(一九六四年六月中旬)}
\datesubtitle{(一九六四年六月)}


(主席在一次会议上对人民日报的文艺宣传提出了批评。六月二十三日,××同志在人民日报和新华社两个编辑部全体工作人员大会上作了传达)

主席说:一九六一年,人民日报宣传“有鬼无害处”,事后一直没有对这件事做过交待。人民日报一面讲阶级斗争,进行反修宣传,一面又不对提倡鬼戏的事作自我批评,这就使报纸处于自相矛盾的地位。

主席说:一九六二年十中全会后,全党都在抓阶级斗争,但是人民日报一直没有批判“有鬼无害论”“要在报社开一个会,把这个问题向大家讲一下,同新华社的同志讲一下。主席说:人民日报的政治宣传和经济宣传是做得好的,反修宣传是有成绩的,但是文化艺术方面,人民日报的工作做得不好。

主席说:人民日报长期以来不抓理论工作。从人民日报开始办起,我就批判了这个缺点,但是一直没有改进,直到最近才开始重视这个工作。过去人民日报不搞理论工作,据说是怕犯错误,要报上登的东西都百分之百的正确。据说这是学的苏联《真理报》。事实上,没有不犯错误的人,也没有不犯错误的报纸。《真理报》现在正走向反面,不是不犯错误,而是犯最大的错误。《人民日报》不要怕犯错误,而是犯了错误就改,这就好了。


