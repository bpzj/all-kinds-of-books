\section[给留守兵团保安部队的慰问信(一九四二年三月四日)]{给留守兵团保安部队的慰问信}
\datesubtitle{(一九四二年三月四日)}


亲爱的同志们:

首先我们向你们全体战士指挥员和政治工作人员致以最亲烈的慰问。回想从抗战开始,我八路军主力开往华北前线以后,四年以来,你们留守边区,是有很大成绩的。你们曾经胜利的保卫了我全军后方、巩固了边区治安,屏障了整个西北。在四年当中,你们已经有了很大的发展,在军事技术、政治文化的学习以及内部的工作上,都有了进步。你们并且辛勤劳苦,普遍的进行了生产运动,克服了生活上的重重困难,打下了自力更生的基础。这些成绩,首先应当归功于你们大家全体的团结努力。

现在已经冰解河开,春天来临,这是我们走上新的一年的开始,苏联的红军要在今年打败希特勒匪军,全世界民主国家的人民都在为了打倒德、意、日法西斯匪军而紧张刻苦的工作着。我们则为配合全世界反对法西斯的战争,准备反攻,要在今后二、三年内打倒日本帝国主义。


今天我们的面前还有种种困难。我们双肩上的担子还是很重的。敌人虽然已经被削弱了,但是还相当强大,并且非常狡猾狠毒;国内的亲日派还未肃清,时刻都在想破坏抗战,破坏我军,破坏边区。因此决不容许空洞的乐观与粗心大意,我们必须提高警惕,加强与友军的团结,密切与群众的联系,使边区更加巩固坚强。

其次,我们的军事技术和文化水平还须大大提高。全体战士和指挥员们,必须好好锻炼身体,加强练习瞄准、投手榴弹、劈刺、各种武器的使用,学习各种战斗动作,提高战术水平。要以大力来组织文化教育,努力克服文盲,克服我军特别是干部中文化水平低下的弱点。

我们还须要加倍努力生产运动,这是我们粉碎敌人的经济封锁,克服物资困难,改善部队供给的最有力的方法。要多多种菜,养猪,拦羊,发展棉毛纺织,组织运输工作,我们的口号是大家动手解决吃穿问题,达到完全自足自给的目的。

亲爱的同志们,摆在我们前头的任务是很重大的。完成这些任务,必须要依靠我们全体,更加百倍的团结和发扬忍苦耐劳,永不疲倦的战斗意志。我们应当在上下级间,在指挥员与战士间,在全体人员相互间,认真加强团结友爱,互相照顾的关系,克服其间的任何冷酷无情、不知友爱和隔阂与不团结现象,在这种钢铁般的团结与艰苦奋斗的努力下面,便没有什么困难不能克服,没有什么障碍不能战胜,没有什么任务不能完成的。

全体战士,指挥员、政治工作人员,全体同志们,今年是我们增长力量,准备反攻的一年。我们希望由于我们大家的积极努力,能够早日举行反攻,到那时我们将把丑恶的敌人从我们祖国的土地上扫除干净,把我们民族独立自由解放的旗帜插遍全中国。胜利光明已经在望,这个日子已经不远。我们在此伟大的年初,向你们大家致同志的慰问。希望你们更加团结一致,更加奋发努力。预祝你们今年在战斗、学习、生产和全部工作中,得到新的巨大的胜利和进步。

谨致

敬礼

<p align="right">1942年3月4日

毛泽东</p>

