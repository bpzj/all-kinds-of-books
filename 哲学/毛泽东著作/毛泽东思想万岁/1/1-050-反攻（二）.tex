\section[反攻(二)]{反攻(二)}


向左还是向右?

半年以来一班人有一千中议论,即是说左也不好,右也不好,另外提出一种中间意见,排斥右派也排斥左派,自己标明是站在中间地位。这种事情在广东不多见,而在江浙颇普遍。因为在广东左就是广州,右救是香港,站在广州旗帜之下他必定反对香港,站在香港旗帜之下他必定反对广州。陈炯朗率领反革命派军人政治买办阶级土豪劣绅站在××旗帜之下,所以两边是用大炮齐轰。在这互轰中不能有中间派,他有也只能脸头掩面躲在一派旗帜之下用低声发言用轻步走路。假若右人要站在广州香港之间,则其宣言必定是“香港不好广州也不好”,那香港的大炮一定要对准他轰,广州的大炮也一定对准他轰,江浙一带现在还没有大炮互轰之事。“两边不好”的议论于是乎大盛。本来今年五月二十日甲方的大炮已经在南京一带砰砰地响了,幸喜乙方没有大炮,只有些拳头做不成“互轰”,未酿成对抗的乱局(发生一下子也算不了事),使“两边不好”的议论还可以公开的宣传,“留正气于两间,存自由于天地”。但我们若假放一个局面,假设那南京路上的群众,不但有拳头而且有大炮,又是汪精卫蒋介石带领着,把那巡捕房砰的一声打个粉碎,随即占领工部局,所有“红头阿山”之徒一齐变了俘虏;马上封锁吴淞口,在南北塘狮子林驾上大炮(象虎头门一样),炮台上扯起“炮打帝国主义”旗帜。这时候上海一定不幸要闹成广州一样的“乱局”,也会设置了卫戌司令部,请看王懋功先生之流作了司令,每日坐了武装汽车在马路上飞巡,时事新报之类一定发封(也许醒狮周报在所不免)。言论只让多数人自由,那少数人的自由一定给他剥掉,和着以前恰恰相反,这时候中间派一如在广州之不能公开宣传。将怎么办呢?那自然还有北京。但是北京也难靠久,总以段执政坐得稳为条件。段执政坐得稳那是没有问题的,不独国民党同志俱乐部可以高挂起招牌,就是第四次中央全体委员会也可以在那里开会,比较张家口还有自由一一话虽如此说,我至今还有点糊涂,何以张家口不准第四次中央委员会开会,那里岂不是段政府管辖的地方吗,如果段执政不在一一且慢,即使段执政在也难免有照料不周之处,不是听说有两个人被人捉上汽车载入城内打了又给写悔过书吗?咳,段执政辇毂之下也出来这种乱子,世事真是难说,更可痛者:据北京执行部电称:十一月二十八日发生革命运动,市民围攻执政府,要驱逐段祺瑞。又称:国民大会议决三条,第一条组织国民政府(很不幸!自然是照广州政府的样)。又据路透电称:“二十八日北京示威,学生执广州旗帜,工人执红旗,未见国旗。游行者分散传单,上书推翻段祺瑞,诛朱深,枪毙国贼,解散关税会议,国人武装,群众革命,真正国会等字样”。时事新报在这段电报之前标题道:“可骇的示威”!怎么办?这里又有了“可骇的示威”,假如将来真会组织什么“国民政府’,在那政府的屋顶上高挂起“广州旗”,岂不又酿成了广州的一样的“乱局’?不但如此,这“乱局”也许蔓延全国,到处仿照办理,多数人起来“自由”,硬把少数人“不自由”。站在中间的先生们!请问怎么办呢?向左?还是向右呢?

赤化原是如此

十一月二十三日申报北京电:“使团是粤洋报,识蒋介石主义虽标赤化,但对于人颇爱护,反之反共产军陈林洪等部到地方时,反多共产行为”。原来赤化就是爱护人民。赤化我化,安得染遍着全中国!

杀尽智识阶级的是谁

陈炯明的反共产宣传文章里,有一篇是:“敬告青年学生”,末尾几句话说,“共产党的政策,还要灭绝智识阶级,载青年学生,总算智识阶级,共产党成功时,就要把智识的青年,一个一个杀尽”,要知陈炯明的话对不对,应先把中国智识阶级的地方,作一个分析。我们觉得有智识的人们不能承认他们是一个阶级。只能承认他们是一些有智识的分子,因为他们没有一个共通的利害,能够促成他们联合做一个阶级.他们有的作了张謇穆藕初廉伯等的辩士或书记,说农民不应藏租和工人不该罢工,但同时确有一部分智识分子,已经投入农工阶级的战线里了。依此,我们很可明白每个有智识的人,就是一个智识分子,这些智识分,跟着近世界产业进步必然的结果,早已失去了他们所谓自由职业的尊严和保证,由中间阶级暂沦为无产者,势非死心塌地参加农工运动不可。若他们竟忘了自身在现存社会上的地位.而情愿开倒车去和军阀买办阶级土豪劣绅帮忙,那才算是智识的青年被人杀尽。被谁杀尽呢?直接被军阀买办阶级土豪劣绅冤杀,间接被帝国主义毒杀。

<p align="right">(一九二五年十二月十三日《政治周报》第二期)</p>

