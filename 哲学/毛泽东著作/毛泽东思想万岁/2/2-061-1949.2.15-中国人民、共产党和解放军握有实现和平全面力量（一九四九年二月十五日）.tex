\section[中国人民、共产党和解放军握有实现和平全面力量(一九四九年二月十五日)]{中国人民、共产党和解放军握有实现和平全面力量}
\datesubtitle{(一九四九年二月十五日)}


国民党反动派崩溃的速度,此人们预料的要快。现在距离解放军攻克济南只有四个多月,距离攻克沈阳只有三个多月,但是国民党在军事上、政治上、经济上、文化宣传上的一切残余力量,却已经陷于不可挽救的四分五裂,土崩瓦解的状态。国民党的总崩溃开始于南线的淮海战役和北线的平津战役期间,这两个战役使国民党在去年十一月初至今年一月底的不足三个月中丧失约一百零五万人,包括国民党正规军一百零五个整师。国民党的总崩溃根本上是中国人民解放战争和中国人民革命运动伟大胜利的必然结果,但是国民党及其美国主人的“和平”阴谋,对于缩短国民党崩溃的过程,起了相当的作用。国民党反动派从今年一月一日开始搬起的一块名叫“和平攻势”的石头,原想要用来打击中国人民的,现在是打在他们自己的脚上了;或者说得正确些,是把国民党自己从头到脚都打烂了。除了在傅作义将军的协助下和平地解决了北平问题外,其余蒋介石、李宗仁、孙科、白崇禧、张群、张治中以及名义上还属于国民党统治下的各省各城的大小头目,现在各搞一方,挂着各人的招牌,打着各人的主意。美国人站在一旁发干急,深恨其儿子们不争气。其实,和平攻势这个法宝出产于美国工厂,还在半年以前就由美国人送给了国民党。司徒雷登本人曾经泄露这个秘密,他在蒋介石发出所谓元旦文告以后,曾告中央社记者说这是“我过去一直亲自努力以来的东西”,据合众社称该记者因发表了这段“不得发表”的话而丢了饭碗。蒋介石集团长期地不敢接受美国人的这个命令,其理由在国民党中宣部去年十二月二十七日的一项指示中说得很明显:“我如不能战,既示不能和。我如能战,则言和又徒使士气人心解体。故无论我能战与否,言和谐有百害无一利”。国民党当时发出这个指示,是因为国民党的其他派别已经在主张言和了。去年十二月二十五日,白崇禧及其指挥下的湖北省参议会向蒋介石提出了和平解决的问题,迫使蒋介石不得不在今年一月一日发布在五个条件下进行和谈的声明。蒋介石于一月八日派张群到汉口和长沙去要求白崇禧和程潜的支持,同日向美英法苏四国政府要求干涉中国内政。但是这些步骤全都失败了。中国共产党毛泽东主席在一月十四日的声明,致命地击破了蒋介石的假和平阴谋,使蒋介石在一个星期以后不得不引退到幕后去。虽然,蒋介石、李宗仁和美国人对于这一手曾经作过各种布置,希望合演一幕比较可看的双簧,但是结果与他们的预期相反,不但台下的观众愈走愈稀,连台上的演员也继续失了踪迹。蒋介石在奉化仍然以“在野地位”继续指挥他的残余力量,但是他已丧失了合法地位,相信他的人已愈来愈少。孙科的“行政院”自动宣布“迁政府于广州”,这个私生子“政府”,一面脱离了它的“总统”(代总统),另一面也脱离了它的“立法院”、“监察院”。伪立法委员总额七百五十人,现在上海者二百余人,在台湾者一百二十余人,在广州者至十一日止报到者仅五十七人,而按伪立院的组织法,须有三分之一以上出席才能开会。国民党的中央常务委员会,虽已在广州宣布开会,但在广州的该党中常委现在只有十一人,尚不足其总额五十余人的四分之一。孙科的“行政院”号召战争,但是进行战争的“国防部”却既不在广州,也不在南京,人们只知道他的发言人是在上海。这样,李宗仁在石头城上将只剩下了“天低吴楚,眼空无物”。李宗仁自上月二十一日登台到现在下过的命令,几乎没有一项是实行了的。虽然国民党已经没有一个“全面”的“政府”,虽然无论在南京,在上海,在广州,在武汉,在长沙,在西北,在西南,到处都在进行着局部和平的活动,但是国民党死硬派却在反对局部和平而要求所谓全面和平,其实际意义就是取消和平。以四分五裂土崩瓦解的国民党组织而要求所谓全面和平的滑稽剧,在本月九日上海伪国防部政工局长战争罪犯邓文仪的一篇声明中,达到了滑稽的高峰。邓文仪和孙科一样,推翻了李宗仁于上月二十二日关于以中共的八项和平条件为谈判基础的声明,而要求所谓“平等的和平,全面的和平”。否则“不惜牺牲一切,与共党周旋到底”。但是邓文仪没有说出今天他的对方究竟应和什么人去谈判“平等的”,“全面的”和平。似乎找邓文仪是不能解决问题的,但是似乎不找邓文仪或者张三李四也不能解决问题,这就未免叫人太为难了。据中央社上海九日电称:“新闻记者问邓文仪:李代总统是否已同意邓局长所发表之四项意见?答:本人系在国防部立场发言,本日所发表之四项意见,事前并未曾经李代总统过目。”邓文仪这里不但创造了一个伪国防部的局部立场以区别于国民党政府的全面立场,而且事实上还创造了一个伪国防部政工局的小局部立场以区别于伪国防部的大局部立场,因为邓文仪公开反对并造谣污蔑北平的局部和平,而伪国防部则在一月二十七日称赞北平的局部和平,是“为了缩短战争获致和平,借以保全北平故都基础与文物古迹”,并称其他地方例如大同绥远等处亦将依同样方法“实施休战”。由此可见,叫喊全面和平最起劲的反动派,原来就是最缺乏全面立场的反动派。一个国防部政工局可以和国防部互相矛盾,又可以和他的代总统互相矛盾。这些反动派是今天中国实现和平的最大障碍。他们梦想在全面和平的口号下鼓吹全面战争,即所谓“战要全面战,和要全面和”,但是事实上他们既没有什么力量实现全面和平,也没有什么力量实行全面战争。全面的力量在中国人民、中国人民解放军、中国共产党和其他民主党派这一方面。不在四分五裂土崩瓦解的国民党方面。一方面,握有全面的力量,另一方面,陷于四分五裂土崩瓦解的惨境,这种局面是中国人民长期奋斗和国民党长期作孽的结果,任何郑重的人,都不能忽视今天中国政治形势中这个基本的事实。

