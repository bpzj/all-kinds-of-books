\section[在南宁会议上的讲话(二)(一九五八年一月十二日)]{在南宁会议上的讲话(二)}
\datesubtitle{(一九五八年一月十二日)}


八年来我为这样一个工作方法而奋斗,我说了一千次,一万次,这是极而言之,说的多了,等于白说。人的思想总是逐步受影响的。要“毛毛雨下个不停”,“倾盆大雨”就会发生径流。政治局是团粒结构不足的,倾盆大雨吸收不了,顺着身上流走了(这是土壤学,农业学都要一本,不然省委书记当不成,有一天总要撤职的,这不是我威胁你们)。政治局成为一个表决机器、像杜勒斯的联合国,你给十全十美的文件,不通过不行,像唱戏一样,已经打了牌了,非登台演出不可。文件上又不讲考据之学,义理之学,又有洋文。我有一个手段,就是消极抵抗,不看。你们的文件,我两年不看了,今年还不准备看。

在杭州会议我讲的,恩来同志讲了没有?一九五五年十二月我写了农村社会主义高潮一书序言,对全国发生了很大影响,是个人崇拜也好,偶像崇拜也好,不管是什么原因,全国各地报纸,大小刊物都登了,发生了很大影响。这样我就成了“冒进”的罪魁祸首。我说了各部门都有对形势估计不足的情况,军队增加了八十万人,工人学徒增加了一百万人,反对右倾保守,为什么要增加人?我不懂,也不知道。

一九五五年夏季北戴河会议“冒进”想把钢搞到一千五百万吨(第二个五年计划)。一九五六年夏季北戴河会议反“冒进”就影响了人代会的报告。人心总是不齐的,不平衡的规律是宇宙发展的发则。孟夫子说过:“物之不齐,物之惜也。”人心不齐,又可以齐,有曲折,螺旋式的前进。当然大家都是为党为国,不是为私。

我对分散主义的办法,就是消极抵抗,还要小会批评。财经部门考证之学,辞章之学,义理之学也不讲。要和风细雨,要事先通一点情报,总是倾盆大雨,发生径流,总不开恩。总没有准备好,不完全,这就是封锁,这是斯大林的办法。开会前十分钟拿出文件来让人家通过,不考虑人家的心理状态。你们是专家,又是红色,政治局多数人是红而不专。我攻击方向,主要是中央部长以上的干部,也不是攻击所有的人,是攻击下倾盆大雨的人,封锁的人。小会不解决问题,就开中央全会(文章做好这件事。没有认真解决,写给广西省委一封信,谈报纸问题。)我在苏联写回一封信,说你们不得中央的支持,对你们工作不利,不然会孤立,像“梁上君子”。

政治局不是设计院。倾盆大雨在我们身上流走了,老说没有搞好,实际上是封锁。分散主义有一点,但不严重。各有各的心理状态,我替你们设想,你们大概有一个想法,大概中央是十全十美的,不是全能,也是九分。另外,大概像《茶花女》小说中的女主角马哥瑞特,快死了,见爱人还要打扮一番。《飞燕外传》,赵飞燕病了,不见汉武帝。总之是不顾以不好的面目见人。蓬头散发见人有何不可?想起一条写一条,把不成熟的意见提出来,自己将信将疑的东西拿出来,跟人家商量,不要一出去就是“圣旨”,不讲则已,一讲就搬不动。四十条就是这样,开始在杭州拟了十一条,天津增到十七条,到北京才增加到四十条。“寡妇养仔,众人之力”,这是工作方法问题。

我看还得闹对立的统一,没有针锋相对不行。要么你说服我,要么我说服你,要就是中间派。有人就是这样,大问题不表示态度。马克思主义不是不隐蔽自己的观点吗?这样我不理解,应当旗帜鲜明,大概想作楚庄王。“三年不鸣,一鸣惊人,三年不飞,一飞升天”。

再一个是顽固(乔木到,一说曹操,曹操就到)。人民日报革命党不革命。我在二月二十九日最高国务会议上的讲话,民主党派拿我的讲话做文章,各取所需,人民日报闻风不动,写一篇社论,从恩格斯谈起。二月开始谈到,我给他们讲,你们又不执行,为什么又不辞职?十一月二中全会,一月省委书记会议,三月宣传会议,还有颐年堂会议,都说了人民内部矛盾。不必忧虑,是可以解决的,可是打不动×××同志的心。我说十个干部一个拥护我就好了,他也不说反对,就是不执行。地委副书记以上一万人,有一千拥护我就好了。北京的学校那个放的开?×××同志五月二十二日在中直会议上做报告,有一句名言:“千金难哭好时机”,“寸金难买寸光阴”。这样才放开了。大鸣大放,清华大学叛变了几个支部,右派高兴,不然审也审不出这些叛变分子。人们都有一种惰性,不容易搞开,乔木要不是那一次会议,北冰洋的冰是开不了的。××是好人,就是无能,我说他是教授办报,书生办报,又说过他是死人办报。

再谈考据之学,辞章之学,义理之学。财经工作者有很大成绩,十个指头,只有一个指头不好,我讲过一万次就是不灵,工作方法希改良一下。我最无学问,什么委员也不是,我和民主人士谈过,我是唱老夫人的,你们是唱红娘的。我是老资格吗?总该给我讲一讲。我灰心了,这次千里迢迢让你们到南方来,是总理建议的。

我是罪魁,一九五五年十二月我写了文章,反了右倾,心血来潮,找了三十四个部长谈话,谈了十大关系,就头脑发胀了,“冒进”了,我就不敢接近部长了。三中全会,我讲去年砍掉了三条(多快好省,四十条纲要,促进委员会),没有人反对,我得了彩,又复辟了。我就有勇气再找部长谈话了。这三年有个曲折。右派一攻,把我们一些同志抛到距离右派只有五十米远了。右派来了个全面反“冒进”甚么“今不如昔”,“冒进比保守损失大”。研究一下,究竟那个大,反“冒进”,六亿人民泄了气。一九五六年六月一篇反冒进的社论,既要反右倾保守,又要反急躁冒进,好像有理三扁担,无理扁担三,实际重点是反冒进的。不是一个指头有病。这篇社论,我批了“不看”二字,骂我的我为什么看?那么恐慌,那么动摇。那篇东西,四平八稳,实际是反“冒进”。这篇东西格子未划好,十个指头是个格子,只一个指头有病,九与一之比,不弄清楚这个比例关系,就是资产阶级的方法。像陈叔通、黄炎培、陈铭枢的方法。

我要争取讲话,一九五六年元月至十一月反“冒进”,二中全会搞了七条,是妥协方案,解决得不彻底,省市委书记会议承认部分钱花的不恰当,未讲透,那股反“冒进”的气就普遍了。廖××向我反映,四十条被吹掉了,似乎并不可惜。可惜的人有多少?叹一口气的人有多少?吹掉三个东西,有三种人,第一种人说:“吹掉了四十条中国才能得救”;第二种人是中间派。不痛不痒,蚊子咬一口,拍一巴掌就算了;第三种叹气。总要分清国共界限,国民党是促退的,共产党促进的。


×××为党为国,忧虑无穷,反“冒进”,脱离了大多数部长、省委书记、脱离了六亿人民。请你看篇文章,宋玉的《登徒子好色赋》,这篇文章使登徒子二千年不得翻身,他的方法是“攻其一点,不及其余”。登徒子向楚襄王反映,宋玉长得漂亮,会说话,好色,宋玉一一反驳了。宋玉反击登徒子好色,说登徒子讨了一个麻脸驼背的老婆,生了七个孩子,你看好不好色,只攻其好色一点,不及其余。我们看干部,要看德才资,不能德才都不讲,只讲德的一部分。九个指头不说,只说一个指头,就是这种方法。我看几年要下毛毛雨,不要倾盆大雨,要文风浸润,不要突然袭击,使人猝不及防。

五月间右派进攻,使那些有右倾思想的同志提高了觉悟,这是右派的“功劳”,这是激将法。

一九五四——一九五五年粮食年度征购九百二十亿斤。多购一百亿斤,讲冒进,这一点有冒。闹得“人人谈统购,家家谈粮食”。章乃器是粮食部长,他同意这个计划,是不是故意把农民闹翻,可能有阴谋。去年粮食销量多,反映了农民没有劲。江苏反映社长低头,干不下去了。我们就怕六亿人民没有劲,不是讲群众路线吗?六亿泄气,还有什么群众路线?看问题要从六亿人民出发,要分别事情的主流、支流、本质、现象。

中央大权独揽,只揽了一个革命,一个农业,其他实际在国务院。

人都有迷信,有惰性。比如我游水,中间隔了三十年。

除四害,人人讲卫生,家家讲清洁,一年十二个月,一月检查一次。这样医院办学校,医生去种田,病人大大减少,人人精神振作,出勤率大为提高,要集中搞,最好两年完成。

我和华东五省约好,今年开四次会,小型的会,是两种元素配合,中央和地方两个元素一配合就不同了。各省也开小型的会。廖××告诉我,十年看五年,五年看三年,三年看头年。我一年给你们开四次会,检查十二次。两本账,争取超额完成。这是苏联发明的。红安县那篇文章,请你们再看一遍。一人首倡就推开了。县委副书记一人买锄头,百分之八十的人买了锄头。还要山东营县那个公社的例子。有一个例子就够了。


