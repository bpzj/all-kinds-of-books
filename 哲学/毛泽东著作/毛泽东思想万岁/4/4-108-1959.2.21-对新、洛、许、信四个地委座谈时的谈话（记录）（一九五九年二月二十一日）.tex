\section[对新、洛、许、信四个地委座谈时的谈话(记录)(一九五九年二月二十一日)]{对新、洛、许、信四个地委座谈时的谈话(记录)}
\datesubtitle{(一九五九年二月二十一日)}


人民公社究竟是什么性质,是公社所有制,还是队所有制?有穷队有富队。有穷村有富村。你们在农业生产合作社的时候,采取的什么分配办法?那时候是不是有差别,那时候上交是否相等?穷队富队、穷村富村。过去农业生产合作社实行包工包产,生产多者奖励,就是农业社是不是拉平的。苏联5500万吨钢不能和我们l000多万吨钢拉平,那是无代价占有别人的劳动的,那是人民的,是农民的。所以那时候提出土地回老家,鞍钢回老家,那不是剥夺。我们对民族资产阶级就不同了。因为他是朋友,我们对他们是采取赎买政策。民族资产阶级的财产明明是工人阶级生产的,不是资本家的,因为他是朋友,所以尚采取赎买政策,利用他作工作,团结知识分子,有偿收买剥夺工人的财产。现在我们对穷队富队,穷村富村采取拉平是无理由的,是掠夺.是抢窃。包括桌椅板凳在内都要打条子、打借据,十年偿还。

过去高级社就是多有多吃,少有少吃。评工记分是表现人与人劳动结果的关系。包工包产是表现村与村、队与队的关系,这个经验我们没有记取。1956年高级化第一年调粮的经验没有记取就发生此问题,老太婆挡住不叫拉粮食,现在公社第一年又是发生此问题。现在看今年究竟采取什么办法好,值得研究。今年要宣布几条政策,穷的富的都要干,想办法帮助穷的。把中国提高到苏联水平。西方国家将来无产阶级专政以后,不能把西方国家的砍下来补亚非国家,还是亚非国家自己提高。当然在亚非国家的投资应该没收,但是不能去欧洲设机器。也就是说,苏联现在发展的水平,不能白白的送给我们.因为苏联有工人,有工人要开工资。机器要搞折旧费,你苏联的砍了给我,你怎么办?还是要搞等价交换(汇报时说,有的就怕商业部门收购猪,把猪,赶到地里,使猪乱跑,有的是藏在棉花里)这个办法我很赞成。(汇报时反映,有的地里花生没有收净,采取分成办法以后,有的一夜就收净了),要采取分成的办法。我很赞成千方百计的吃掉、跑掉,这样的作法不是本位主义,这是他们劳动的结果。我们反本位主义,强迫收回来,这样越反越不行。你这实际是无偿外调,你叫他本位主义,名字安的不对,这是所有制的问题。现在部分所有制是社,基本所有制是队。公社逐年扩大点积累,搞他七、八年,社的所有制就形成了。

河北省一月八号开党代会,想思想统一。想统死。作了决议,但是,到了一月下旬感到不对头。省委赶紧传,下旬开了电话会议,转变了以后,有些地委不通,县委不通,有些公社不通,现在所有制实际上生产队是八个指头、九个指头,公社是一个指头、二个指头。最多不超过三个指头。积累不超过25%,这就占1/4,生产费20%,群众分配55%。下边隐瞒,实际不至只分配30%。大家都想多积累一点办工业,这也是好心。斯大林就是这样的政策。斯大林从建国到1953年为止的三十年问,没有解决这个问题。斯大林搞了一个集体化,一个机械化。沙皇时代没有集体化,集体化了,没有机械化,机械化了。但是他死那一年的产量和沙皇时代一样,如果不是赫鲁晓夫改变政策,将来越发展越严重。我们现在不改变,就要犯斯大林的错误。

我们现在说生产队。队就是社嘛,现在公社实际是联邦政府。公社的权不能那么大,应该是有公粮之权,积累之权,产品分配应该在队(汇报时反映社办的事情太多。一方面投资太多,再一方面劳动力调的太多)。要政变这种政策,过去就讲个人、集体、国家三者的关系,现在是半盘棋,几亿农民是大半盘棋,光搞国家积累。社里积累不行,过去讲“百姓不足,君孰与足”。现在的百姓就是社,君就是国家.斯大林搞了三十年,是一条腿走路。如果我们70%归国家,和地主一样,群众只得三成,当然我们和地主的本质上是不一样的。我们多积累一点是搞建设,建设反回来还是为人民嘛。这一点也要分析,要讲清楚他是想迅速工业化,是好心,但是,我们作这是好心不是好意(不是好主意)。我们和地主不一样。不是为了发财。我们一时没有讲清楚,八届六中全会也还没有讲清楚,只讲了按劳分配,只讲了生产责任制。没讲怎么样按劳分配,没有讲清楚集体所有制。现在分配方案要改变一下。公社所有制要有个过程,现在基本上是队所有制。无非就是这样几条,土地农民不怕,省委、地委、中央你都搬不走,他现在争的是产品和劳动力。我们说土地还是集体所有制,内部搞一个文件说一下,实际是队所有。土地,工具、生产资料加人力和公社只搞25%,你们考虑是不是太少了?现在反对本位主义,造成紧张局势,越造成紧张局势越紧张,实际瞒的15%叫它合法。我们国家和社积累25%,生产费20%,群众分配55%。这个比例不变,生产年年发展,绝对数都增加了。把白菜、把猪都拉走,一个钱不给,这个办法一定要改变。

另外,谈谈工业怎么办7工业现在占的资金、人力太多是有冲突的。凡有人力、物力、财力冲突的要调整一下。学校也不可一下办那么多,什么事都要逐步来,如除四害,一次能够除净吗?绿化也要逐步来,文盲也是逐步扫,学校也是分批分期搞。财贸机关把贷款全部收回,现在应该全部退回,最后来个折衷办法.叫他退一半。这是搬石头砸自己的脚。你把原来的贷款都收回了,他发工资就没有钱,结果还得贷款发工资。先宣布安民布告。调拨要研究个章程。分配也要研究一下。要知道人民公社的集体所有制是逐步形成的。我们经过四个时代了,互助组,那只有一点集体所有制,初级社集体就多了,叫作半集体所有制,高级社的时候,可以说有十分之七集体所有制,现在要拉平不行,积累太多了群众要反对,这样并非一盘棋。真正一盘棋第一是农民,第二是公社,第三是国家。这样一来农民就拥护我们了,农民反过来会照顾国家的。这样是否收不到东西?我是替农民说话的,我是支持“本位主义”的。因为现在是队所有制,几年以后才能实行社所有制,一定要注意什么事要有个过程。要认识部分是社所有,基本是队所有。公社是半路插进来的干老子。粮食生产队有个差等,工资电要有个差等。河北省是上死下活。我们可以采取这个办法,叫作“死级活平,按劳取酬”。


