\section[在八届十中全会上对工业支援农业的指示(一九六二年七月二十四日)]{在八届十中全会上对工业支援农业的指示}
\datesubtitle{(一九六二年七月二十四日)}


〔在谈到坚持集体化方向,工业支援农业时说〕:搞了四年,以农业为基础的方针不大自觉,不大愿意。巩固集体经济有两方面:一是政策,二是支援农业。从长远来说是农业的技术改造,总的是如何搞社会主义建设的问题。

科学的研究,没有抓农业。科学院党组书记说:科学院是搞尖端的。

要抓农业技术改造。

民主与集中,集中与分散,分散主义首先是中央(指综合部门)。农业机械化要搞个文件,二十五年左右实现机械化,同时实现工业化。有的同志在困难面前躺下。是消极好,还是鼓足干劲好?是单干,还是集体好,要提起注意。

对困难发生消极情绪是不是思想问题?

计委、经委、工交各部要加强支援农业。要抓紧改,这是机会,还来得及。

计委搞两个文件:支援农业的报告,三个部分:方向,面向农业,支援农业的方针。


