\section[我百万人军横渡长江(一九四九年四月二十二日)]{我百万人军横渡长江}
\datesubtitle{(一九四九年四月二十二日)}


人民解放军百万大军,从一千余华里的战线上,冲破敌阵,横渡长江。西起九江(不含),东至江阴,均是人民解放军的渡江区域。二十日夜起,长江北岸人民解放军中路军首先突破安庆、芜湖线,渡至繁昌、鳃凌、青阳、荻港、鲁港地区。二十四小时内即已渡过三十万人。二十一日下午五时起,我西路军开始渡江,地点九江、安庆段。至发电时止,该路三十五万人民解放军已渡过三分之一。余部二十三日可渡完。这一路现已占领贵池、殷家汇、东流、至德,彭泽之线的广大南岸阵地,正向南扩展中。和中路军所遇敌情一样,我西路军当面之敌亦纷纷溃退,毫无斗志,我军所过之抵抗,甚为微弱。此种情况,一方面由于人民解放军英勇善战,锐不可挡;另一方面,这和国民党反动派拒绝签订和平协定,有很大关系。国民党的广大官兵一致希望和平,不想再打了,听见南京拒绝和平,都很泄气。战犯汤恩伯二十一日到芜湖督战,不起丝毫作用。汤恩伯认为南京江阴段防线是很巩固的,弱点只存在于南京九江一线。不料正是汤恩伯到芜湖的那一天,东面防线又被我军突破了。我东路三十五万大军与西路同日同时发起渡江作战。所有预定计划,都已实现。至发电时止,我东路各军已大部渡过南岸,余部二十三日可以渡完。此处敌军抵抗较为顽强,然在二十一日下午至二十二日下午的整天激战中,我已歼灭及击溃一切抵抗之敌,占领扬中、镇江、江阴诸县的广大地区,并控制江阴要塞,封锁长江。我军前锋,业已切断镇江无锡段铁路线。

