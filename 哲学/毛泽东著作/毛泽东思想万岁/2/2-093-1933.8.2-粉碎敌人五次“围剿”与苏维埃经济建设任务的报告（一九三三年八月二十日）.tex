\section[粉碎敌人五次“围剿”与苏维埃经济建设任务的报告(一九三三年八月二十日)]{粉碎敌人五次“围剿”与苏维埃经济建设任务的报告}
\datesubtitle{(一九三三年八月二十日)}


同志们:我们这次开两个经济建设大会,一一南部十七个县的与北部十一个县的。这种为着经济建设而开的大会以前是没有开过的,今天是第一次,目前的形势是敌人的四次“围剿”已经被我们完全粉碎,而五次“罚剿”接着就要到来,全世界革命与战争的大风暴是逼近着我们了。帝国主义大战与反苏联的战争是在疯狂般的准备着,帝国主义是在疯狂的压迫中国,日本占领了中国四个半省正在向内蒙古一带发展他们的强盗战争,国民党四次“围剿”惨败后,正在布置新的第五次“围剿”。全中国的革命斗争与革命战争是面监一种新的形势,而大踏步的开展着。这时候我们在这里开经济建设大会,这就指明了我们要讨论的是什么,我们的经济建设是为着什么,我今天报告就在解释这一点,我的报告分作下面两部分来说:

一、四次“围剿”的粉碎与新的五次“围剿”

(1)我首先来说粉碎四次“围剿”中我们所得到的胜利,我们要说:

第一,是敌人部队的大数量的消灭。

四次“围剿”是彻底粉碎了。总括一年来中央局、鄂豫皖、四川、湘鄂西,以及湘赣、湘鄂赣、闽浙赣各苏区红军消灭敌人部队总数在十五师以上。比过去粉碎敌人三次“围剿”时的胜利更大了,敌人的武装力量,受了一个最严重的打击,敌军全部中下级官长,都十分畏惧红军,怕与红军作战,而白军士兵的动摇及对革命的同情,则在日益增长。

第二,是红军的坚强与扩大。

红军已经成为无敌的铁军了,红军编制的改变,军事技术提高,政治上的坚定都比过去有了极大的进步,红军的扩大比较过去是增加了一倍,我们已经建立了大规模的作战军。

第三,苏区是更加巩固了。

在四次“围剿”中工农群众的阶级觉悟,与拥护革命战争的热忱是更加提高了。查田运动的深入,工人斗争的发展,使得苏区内部封建残余势力,受到严重的打击,群众的文化教育运动,有了进一步的发展。经济建设运动开始了发展的新形势,苏维埃的工作更加改善了,他在群众中的威信更加提高,苏维埃的旗帜映入了全国劳苦群众的心目中。

第四,是扩大了苏维埃的领土。

中央区闽赣省的建立,四川几百里新苏区的开辟,湘鄂川××同志那方面又发展了一个大苏区。

第五,国民党统治区域的革命运动也是极大的发展了。

因为日本帝国主义强占去中国四个半省,国民党投降帝国主义订立中国与日本的卖国协定,使得广大群众反对帝国主义国民党的运动大规模的开展了,因为国民党的压迫屠杀,工商业的破产,资本家与地主要加残酷地剥削工人、农民,使得工人的罢工斗争,农民灾民的反抗情绪斗争,农民暴动与游击战争大大的发展了。

(2)同志们:我们已得到了极大的胜利,但是这些胜利是依靠着什么得来的呢?

我想大家都明白是依靠于:(1)红军的英勇善战,(2)苏区与白区扩大一般群众的革命积极性,(8)共产党正确路线的领导。这三个重要条件:方粉碎敌人的四次“围剿,取得了上面所说这些伟大的胜利。这些胜利给予了帝国主义国民党以极大的打击,一方面使革命力量增大了,有着铁的强流的形势,向前猛进着,另一方面使得反革命力量更加削弱,更加感觉他的危亡迫在眼前,使他们不得不重新布置一个更大规模的进攻,企图挽救他们将死的统治。

(8)因此敌人疯狂的在布置他们的第五次“围剿”。

为了布置这一次“围剿”使国民党不得不更加投降于帝国主义,蒋介石在出卖了东三省热河;华北之后,更以许多卖国条件同美、英、法、意、德各个帝国主义国家订立了密约,借得大批金钱和军火,重新调动部队调统新兵,拉拢各级军阀(但军阀间的冲突与战争是无法避免的),正在积极的布置对于苏维埃和红军的第五次“围剿”。

二、粉碎五次“围剿”与苏维埃经济建设任务

争取一切有利条件去粉碎敌人的五次“围剿”。

中国革命现在是处在这样的紧急关头,即是说,帝国主义把中国灭亡呢?还是革命战争消灭国民党,驱逐帝国主义把中国变成苏维埃中国呢?帝国主义国民党决定了灭亡中国的道路,他们已经采取了五次“围剿”,这种办法在五次“围剿”中把中国完全瓜分或者共管,使中国几万万民众变成帝国主义的牛马奴隶,变成印度人和高丽人,使中国变成完全的帝国主义殖民地,这是一个绝大的危险,这种危险在我们的头上威胁着。同志们,我们能听任他们这样做吗?不能,我们要争取革命的出路,我们的出路是战胜帝国主义国民党,使中国脱离帝国主义国民党的统治,成为新的自由独立的工农兵苏维埃共和国,这里重要的关头,是彻底粉碎他们时五次“围剿”。争取粉碎五次“围巢”的胜利,仍然是依靠于红军,依靠于群众,依靠于坚决执行共产党的进攻路线,但我们必定要用我们的一切势力去争取一切比较过去更加充足的条件,方能粉碎五次“围剿”,方能取得比较过去更加伟大的胜利。

为了争取这次胜利,我们要做许多的工作。我们要猛烈扩大红军,要在中央区及临近几个苏区六个月至十个月内办到扩大二十万个新战士上前线去,使各个战线上红军集团更加壮大起来,能够担负着打击蒋介石几十万白军的任务,要开展广泛的深入的查田运动,激烈发展农村中的阶级斗争,去彻底解决土地问题,去最后的消灭封建残余势力,使广大农民群众更加热烈的欢喜参加革命战争,要普遍实行劳动法,启发工人斗争,更加提高工人群众的对于革命战争的积极性,要经过今年的选举运动,改造各级苏维埃。从乡苏区到中央一律实行新的改选,使整个苏维埃政权铁一样的强固起来,更能担负组织与指挥革命战争的伟大事业,要注意边区和新发展区域的工作,使革命战争得到便利条件迅速向着中心城市发展,要发展群众的文化运动,提高群众的文化水平与政治水平,使革命战争得到一个精神上的有力的工具。除此以外,还有极其重要的一项工作,而为我们此次所要着力讨论的,这就是经济建设方面的工作,我们要猛力开展经济建设这动要把经济建设这个任务看成为粉碎五次“瞳剿”的最基本的条件之一一一供给革命战争一个必不可缺少的物质上的条件,这次经济建投大会的召集就是为着这个目的。

<p align="right">(根据江西省博物馆复制品排印)</p>

本文未完,其余部分以《必须注意经济工作》为题,收入《毛泽东选集》第一卷。

