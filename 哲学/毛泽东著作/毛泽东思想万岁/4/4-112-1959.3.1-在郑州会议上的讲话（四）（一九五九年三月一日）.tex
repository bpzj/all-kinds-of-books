\section[在郑州会议上的讲话(四)(一九五九年三月一日)]{在郑州会议上的讲话(四)}
\datesubtitle{(一九五九年三月一日)}


要提高农民的生产积极性,改善政府与农民的关系,必须从改变所有制着手。现在一平、二调、三提款,否定按劳分配,否定等价交换。赵尔陆和王鹤寿之间也有一个交换关系。价值法则,等价交换不仅存在公社内部,也存在于集体所有制与全民所有制之间,实际上生产资料各部门之间也有价值法则起作用。人不吃饭,怎么拉屎拉尿,不拉屎拉尿怎么有饭米,骨头还是归于地球。自然一部和另一部交换,大体上是等价交换,大鱼吃小鱼,小鱼不吃别的也不行。现在就是一平、二调、三提款,提起就走,一张条子要啥调啥,不给钱是起破坏作用。现在银行不投资农业,我建议每年增加十亿,十年搞一百亿无利长期贷款,主要支援贫队,一部购买大型农具,十年之后国有化了,就变为国家投资了,忽然一股风,一平、二调、三提款,破坏经济秩序,许多产品归社不归队。六中全会公社决议的一套制度,二个半月来根本没有实行。实行了集体福利、公共食堂、劳动与休息。问题不这样提,共产风会继续发展。为什么六中全会的决议没有阻止这股风的发展?是不是只有冀、鲁、豫三省?是不是南方各省道德特别高尚,马克思主义多?我就不相信。

我提议请你们开一个六级干部会,找一批算账派参加。共产党就是反反复复。

十二句话应再加两句价值法则,等价交换。统一领导,队为基础分级管理,权力下放;三级核算,各计盈亏,适当积累,合理调剂,收入分配,由社决定,多劳多得,承认差别,价值法则、等价交换。不解决这个问题,大跃进就没有了。我这篇话不讲,就不足以掀起议论,这几个月许多地方实际上破坏了价值法则。去年郑州会议,就吵这个问题,拉死人来压活人。凡是瞒产私分者,一定都是一平、二调、三提款。农民从十月以来,发生大恐慌,怕共产,从桌、椅、板凳开始,还有个工业抗旱,破钢烂铁,无代价献宝。这在战时是可以的,无代价或者很少代价。战勤只给饭吃,不给代价。这也不是长期的,否则也会破坏生产。

今年你们要节制,尽最少放“卫星”,如体育卫星、诗歌卫星、银行卫星等。

要讲爱国、爱社、爱民。过去河北提出“要管家,种棉花”,我们给它改为“爱国发家,多种棉花。”

东鹿县收棉花,总结了三条:不问来源,不咎既往,现金交易,谁卖谁得;只此一次,下不为例。另加一条政治挂帅、敲锣打鼓。

每个公社组织一个专业运输队,改良工具,从现在工业战线抽一批人下来。至于运输队的大小,按照需要。省、专、县商业部门都要组织运输队。

劳动各方面要有一个平衡。要达到一个目的,各方面的平衡:农业、工业、运输业、服务业。工业还要细分,有国办、地方办,都搞社办,很不方便,比如修配、磨粉。养猎都由社养不好,大部都应由生产队、食堂养。

共产主义是不是推迟了?早已推迟了,六中全会决议讲了十年到二十年,还有五个条件没有完成。现在有些同志在这个问题上还是想早一点,我看越想搞越搞不成,越慢一点,越可以快。用“无偿”来搞共产主义不行。猪只有一条,你有他就没有了。凡是劳动,总要等价交换的。

积累18%不低,应该有个幅度。

过去一盘棋,强调上面,现在一盘棋,要上下兼顾。

专业队归那个搞?几级都要有专业队。逐步考虑得利大的釆取国营,搞全民所有制。比如在东湖打鱼,收入特别多的县可以搞全民所有制的试点,县可以搞个把,不成功不登报。

穷队向富队看齐,把穷队提高到富队。要使社办工业、企业都办起来,提高公社的基本所有制,房屋不是不建了,要经济、美观、实用。

我看要使社干部不怕,把观潮派搞出来,让地、富、反、坏、观潮派攻,无非是我们一平、二调、三提款。

发工资问题,可能有发不起工资的情况。

公社所有制,包括三级所有制,三级管理,各计盈亏。

我们主要反对平均主义,过分集中思想,这实际上是“左”倾冒险主义,安国文件值得注意,往年闹粮,主要是富裕中农带头,今年闹粮,主要是基层干部带头,如说理由是宣传工作没做好,我看不对头,只要一平二调三提款,做了宣传工作也会这样。群众以为一切要归公,一切共产,再加小社卖粮,大社堵账,卖粮之后,钱粮两空,有些增产的大队,又增加征购任务,使干部摸不到底。因此基层干部有五怕:一怕拉平,二怕报实产量,追加任务,三怕春荒时要调剂解决,四怕自己吃亏,五怕……,于是先下手为强,把粮食搞到手里再说,他们的决心很好很大,这主要有群众支持,瞒产私分成为普遍现象。

河南会议鸣放的文件,可以发给各地看,开头二、三天不要发。让他们思想混乱几天。到四、五天后分批发给他们看,其中有些内容可以解决他们的问题。

这次会议是六中全会的具体化发展补充。

山西文件精神是管理区与管理区之间,允许有不同的差别。而不过早的消灭这种差别,正是为了从发展生产中消灭这种差别。现在允许它,正是为了将来消灭它。人民公社发展生产,提高积累,应当对落后社有适当的照顾。但是如果在工资标准上一下拉平,就会减少较多生产水平的管理区的收入,就会减少积累,就会使落后的管理区不注意经济核算,抽多补少,抽肥补瘦不行,不是照顾富社,而是照顾穷社,暂时保存这种差别,才有利于增加公社积累,有利于穷、富社都发挥积极性。公社的积累增长得越快,这种差别的消灭也会越快。问题是把穷队向富队看齐,问题是公共积累增多。两方面一来,就会使生产发展得越快。然而由于管理区之间管理工作好坏和生产水平不同,这种差别会长期存在下去,这是对的。物之不齐,物之情也。自然条件与主观努力,千差万别。地球的中心,外部温度就不同。消灭差别的过程,也是由集体所有制逐步过渡到全民所有制的过程,也是机械化、电气化过程,少则三、四年,多则五、六年,公社与队的所有制,互相交错,你中有我,我中有你,逐步过渡,有些队可以先转变为全民所有制。

明年一百万吨钢,后年两百万吨钢,也许多一点供应农业搞机械化。

钱的补贴确定十亿,作为农业投资。

工资由公社确定,由管理区发。公社的权力究竟统几个什么东西,开一个账。这不是公社权力小,而是包而不办。

各省、地、县搞一个示范章程,各个公社也要搞一个章程,各省要选择最好的二、三分给我。每一个县着重搞一个,每一个省集中搞一个。了解一个公社不要很久时间,一个礼拜就行了。又要实际,又要超产,无非是一些要点、关节、麻雀这样多,只能如此,但是全无印象也不好。

瞒产私分,非常正确,本位主义有则反之,不能去反五亿农民和基层干部。瞒产私分、站岗放哨,这是由共产风而来。普遍的瞒产私分、站岗放哨、黑夜冒烟,是一种和平的反抗。不普遍戴本位主义的帽子。是则是,非则非,是本位主义还是要反,还是要事先订条约,要政治挂帅,共产主义教育是必要的。贫、中、富队各定多少,国家、集体、个人。全面安排,三者兼顾。个人首先照顾集体、国家,国家首先照顾个人,应该批评本位主义,但是要先批评我们自己的缺点,然后引起积极分子来自我批评,发动多数人自我批评,孤立那些真正本位而不自我批评的人和贪污的人,贪污结合整社来搞,推迟一点,先把积极性搞起来。


