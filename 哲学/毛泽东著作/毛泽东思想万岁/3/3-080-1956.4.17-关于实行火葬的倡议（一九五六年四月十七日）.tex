\section[关于实行火葬的倡议(一九五六年四月十七日)]{关于实行火葬的倡议}
\datesubtitle{(一九五六年四月十七日)}


人们由生到死,这是自然规律。人死以后,应当给以妥善安置,并且采取适当的形式进行悼念,寄托哀思,这是人之常情。我国历史上和世界各民族中有各种安葬死者的办法。其中主要的办法是土葬和火葬,而土葬沿用最广。但是土葬占用耕地,浪费木材,加以我国历代封建统治阶级把厚葬久丧定作礼法,常使许多家庭因为安葬死者而陷于破产的境地。实行火葬,不占用耕地,不希望棺木,可以节省装殓和埋葬费用,也无碍于对死者的纪念。这种办法虽然在中国古代和现代还只有一些人采用,但是,应当承认,这是安置死者的一种最合理的办法,而且在有些国家已经普遍实行。因此我们倡议,在少数人中,首先是在国家机关的领导工作人员中,根据自己的意愿,在自己死了以后实行火葬。为着火葬的方便,除了北京、上海、汉口、长沙等地已有火葬场外,我们建议,国家还可以在某些大中城市和其他适当地方增建一些现代化的火葬场。

我们认为安葬死者的办法应当尊重人们的自愿。在人民中推行火葬的办法,必须是逐步的,必须完全按照自愿的原则,不要有任何的勉强。中国的绝大多数人有土葬的长期习惯,在人们还愿意继续实行土葬的时候,国家是不能加以干涉的。对于现有的坟墓,也是不能粗暴处理的。对于先烈的坟墓以及已经成为历史纪念物的古墓都应当注意保护。对于有主的普通坟墓,在希望迁移的时候,应当得到家属的同意。

凡是赞成火葬办法的国家机关工作人员,请在后面签名,凡是签了名的,就是表示自己死后一定要实行火葬,后死者必须保证先死者实现其火葬的志愿。


