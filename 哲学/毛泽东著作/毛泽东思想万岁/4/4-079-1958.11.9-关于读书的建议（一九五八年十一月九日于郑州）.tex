\section[关于读书的建议(一九五八年十一月九日于郑州)]{关于读书的建议(一九五八年十一月九日于郑州)}
\datesubtitle{(一九五八年十一月九日)}


此信送给中央,省市、自治区,地,县这四级党的委员会的委员同志们:

不为别的,单为一件事:向同志们建议两本书,一本斯大林著的《苏联社会主义经济问题》,一本马恩列斯《论共产主义社会》。每人每本用心读三遍,随读随想。加以分析。哪些是正确的(我以为这是主要的),哪些说得不正确,或者不大正确,或者模糊影响,作者对于所要说的问题,在某些点上,自己并不清楚。读时三、五人为一组,逐章逐节加以讨论,有两至三个月,也就可能读通了。要联系中国社会主义经济革命和经济建设去读这两本书,使自己获得一个比较清醒的头脑,以利指导我们伟大的经济工作,现在很多人有一大堆混乱思想,读这两本书就有可能给以澄清。有些号称马列主义的经济学家的同志,在最近几个月内,就是如此。他们在读马克思列宁主义政治经济学的时候是马克思主义者,一临到目前经济实践中某些具体问题,他们的马克思主义就打折扣了。现在需要读书和辩论,以期对一切同志有益。

为此目的,我建议你们读这两本书。将来有时间,可以再读一本,就是苏联同志编的那本《政治经济学教科书》。各级同志如有兴趣,也可以读,大跃进期间,读这些书最有兴趣,同志们觉得如何呢?


