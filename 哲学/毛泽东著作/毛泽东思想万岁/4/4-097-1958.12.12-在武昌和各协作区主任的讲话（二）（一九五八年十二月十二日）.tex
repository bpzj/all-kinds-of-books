\section[在武昌和各协作区主任的讲话(二)(一九五八年十二月十二日)]{在武昌和各协作区主任的讲话(二)}
\datesubtitle{(一九五八年十二月十二日)}


(1)公报问题。西方可能四分五裂,看样子会四分五裂,但是也还不准。欧洲大陆一个集团;对付英美,但内部矛盾重重,德法矛盾,英美也有矛盾,他们是又团结又斗争。斯大林分析资本主义内部要打仗。我们早已说过,一九四六年我们写的文章,发明美国和苏联的中间地带,争夺中间地带是主要的。为什么不打中间地带,先打苏联呢?以反共为名,去搞蚕食政策,对中间地带侵略,引起中间地带的反抗,有广大中间地带,他们走不过来。包括美、法、德、意,亚洲、非洲、拉丁美洲是他们的后方。欧、亚、非在闹乱子,美国如何脱出手来打苏联?

剥削是剥削人,剥削人才能剥削地球,有人才有土,有土才有财,把人都打死,占了它干什么?我想不出为什么要打原子弹,还是常规武器。我们想,只要不打原子弹,德、法、意都赞成,许多国家都不怕美国,不能订个条约互相不使用?垄断资本存在,不打仗是不行的,因为没有原料,没有市场。

公报中对国际形势的估计——帝国主义一定要四分五裂。而且自己要打仗。尼克松说:要搞经济竞赛,要把印度扶起来。印度如何能扶植起来呢?西方是一股悲观气氛,我们是一股高兴气氛。四分五裂这句话要斟酌一下,四分五裂是真理,但一讲,是否会引起他们警惕,可是又没有办法。美国在台湾要把自由主义分子挤进去。伊拉克很紧张,这几天捕了一大批反革命,但是胜负还未决定。主要是美、英、土耳其、伊朗在搞阴谋(卡赛姆——为什么解散工会?怀疑伊拉克为什么消息灵通)。

(2)三个文件,已定稿。辞职问题:“偶像”总要有一个,一个班要有一个班长,中央要个第一书记。没有微尘作为核心,就不会下雨。与其死了乱,不如现在乱一下,反正有人在,没有个核心是绝不行的,要巩固一下。搞死了便成为“偶像”。要破除比较难,这是长久立起来的一种心里状态,也许以后职务可多少,可上可下。实际上只作了半个主席,不主持日常事务。

五九年计划搞二个月再说,二月半再议一次。

人民公社如何?二月一日开省市委书记会议,或在北京,或在成都,或在上海。

香港报。骂蒋介石“仓惶辞庙,逃往台湾”。

人代会三月十五日开。

二月一日开会,除检查两个决议外,另外,整顿国家机构是必要的,安排人代会报告,还有教育问题,人民公社再搞点内部指示。

人民公社对宪法有破坏没有?例如政社合一问题,人代大会没有通过,宪法上没有。宪法有许多过时了。但现在不改,超过美国后再搞个成文宪法,现在学美国,搞不成文宪法。美国是不成文宪法。一篇一篇凑起来的。

进入共产主义。需要十五年,二十年或者更多一点时间。搞成社会主义全民所有制,少

则三、四年。是指第二个五年计划了。多则五、六年。是指第三个五年计划。

(3)人民公社文件。还要修修补补,还要想一想,十五、十六、十七三天时间修改。十八号发表公报和主席辞职问题,十九号发表人民公社决议。

主席辞职问题,如果还有疑问,还可再开一次电话会议,解释解释。

(4)北戴河会议。我犯了一个错误,想了一千○七十万吨钢,人民公社,金门打炮三件事。别的事情没有想。北戴河会议决议现在是改。那时是担心,没有把革命热情和实际精神结合起来,武昌会议把两者结合起来了,决议要改。两条腿走路,俄国的革命精神与美国的实际精神。

陈炯明在东江和一个县的税务局长凑股子,选举一个人当局长。抓两个月一摸。

还有两句话:“轻重缓急要排队。自力更生小土群”横联是“政治挂帅”。

下午必须开一次会。出告示,要专政。省委书记要会做。无非是落后,已经落后了。再落后几年,有什么问题,做省委书记要全面,顾全大局。全国一盘棋与地方积极性相结合,还有矛盾,服从全国的利益。顾全大局有最高的品德.这种最高的品德,并不吃亏。凡不顾全大局的,就要吃亏。杨一辰就是不顾大局,想趁机而起,打倒周、陈等,想搞这一套。凡是想搞这一套的,都是搬起石头打了自己的脚。有一帮人,不顾全大局。历史上一切不顾大局的人都没有好下场,如扬一辰,高岗、罗章龙。

准备自己受了冤枉,还要顾全大局,杨一辰有多少马列主义?一毫也没有。

有些省穷得要命,再穷几十年也不要紧,实际上不会穷几十年。

牺牲自己成全别人,红娘晚上站在外边,并且挨打,为了什么?


