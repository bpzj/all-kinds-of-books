给徐特立同志的一封信

(一九三七年一月三日)>



徐老同志:

  你是我二十年前的先生,你现在仍然是我的先生,你将来必定还是我的先生。当革命失败的时候,许多共产党员离开了共产党,有的甚至跑到敌人那边去了,你却在一九二七年的秋天,加入共产党,而且取的态度是十分积极的。从那时至今长期的艰苦斗争中,你比许多青年壮年党员还要积极,还要不怕困难,还要虚心学习新的东西,什么“老”,什么“身体精神不行”,什么“困难障碍”在你的面前都降服了。而在有些人面前呢?却做了畏葸不前的借口。你是懂得很多而时刻以为不足,而在有些人本来只有“半桶水”,却偏要“淌得很”。你是心里想的,就是口里说的与手里做的,而在有些人他们心里之某一角落.却不免戴着一些腌腌臜臜的东西。你是任何时候都是同群众在一块的,而在有些人却似乎以脱离群众为快乐。你是处处表现自己就是服从党的与革命的纪律之模范,而在有些人却似乎认为纪律只是束缚人家的。自己并不包括在内。你是革命第一,工作第一,他人第一,而在有些人却是出风头第一,休息第一,与自己第一。你总是拣难事做,从来也不躲避责任。而在有些人则只愿意拣轻松事做,过到担责任的关头就躲避了。所有这些方面,我都是佩服你的,愿意继续学习你的,也愿意全党同志学习你。当你六十岁生日的时候写这封信祝贺你,愿你健康,愿你长寿,愿你成为一切革命党人与全体人民的模范。
        此致

  革命的敬礼!


  毛泽东

          一九三七年一月三日于延安