\section{民众的大联合(三)}
\datesubtitle{(一九一九年八月四日)}



中华“民众的大联合”的形势

上两回的本报己说完了(一)民众大联合的可能及必要,(二)民众的大联合,以民众的小联合为始基,于今进说吾国民众的大联合我们到底有此觉悟么?有此动机么?有此能力么?可得成功么?

(一)我们对于吾国“民众的大联合”到底有此觉悟么?辛亥革命,似乎是一种民众的联合,其实不然.辛亥革命乃留学生的发纵指示.哥老会的摇旗唤吶,新军和巡防营一些丘八的张弩拔剑所造成的,与我们民众的大多数毫无关系。我们虽赞成他们的主义,却不曾活动。他们也用不着我们活动。然而我们却有一层觉悟。如道圣文神武的皇帝,也是可以倒去的。大逆不道的民主,也是可以建设的。我们有话要说,有事要做,是无论何时可直说可以做的。辛亥而后,到了丙辰,我们又打倒了一次洪宪皇帝,原也是可以打得倒的,及到近年,发生南北战争,和世界战争,可就更不同了,南北战争结果,官僚、武人、政客,是害我们,毒我们,剥削我们,越发得了铁证。世界战争的结果,各国的民众,为着生活痛苦问题,突然起了许多活动。俄罗斯打倒贵族,驱逐富人,劳农两界合立了委办政府,红旗军东驰西突,扫荡了多少敌人,协约国为之改容,全世界为之震动。匈牙利崛起,布达佩斯又出现了崭新的劳农政府。德人奥人捷克人和之,出死力以与其国内的敌党搏战。怒涛西迈,转而东行,英法意美既演了多少的大罢工,印度朝鲜又起了若干的大革命,异军特起,更有中华长城渤海之间,发生了“五四”运动。旌旗南向,过黄河而到长江、黄浦汉皋,屡演话剧,洞庭闽水,更起高潮。天地为之昭苏,奸邪为之辟易。咳!我们知道了!我们醒觉了!天下者我们的天下。国家者我们的国家。社会者我们的社会。我们不说,谁说?我们不干,谁干?刻不容缓的万众大联合,我们应该积极进行!

(二)吾国民众的大联合业已有此动机么?此间我直答之日“有”。诸君不信,听我道来一一

溯源吾国民众的联合,应推清末谘议局的设立,和革命党一一同盟会一一的组成。有谘议局乃有各省谘议局联盟请愿早开国会一举。有革命者乃有号召海内外起兵排满的一举。辛亥革命,及革命党和谐议局合演的一出“痛饮黄龙”。其后革命党化成了国民党,谘议局化成了进步党,是为吾中国民族有政党之始。自此以后,民国建立,中央召集了国会,各省亦召集省议会,此时各省更成立三种团体,一为省教育会,一为省商会,一为省农会(有数省有省工会。数省则合于农会,象湖南)。同时各县也设立县教育会,县商会,县农会(有些县无)。此为很固定很有力的一种团结。其余各方面依其情势地位而组设的各种团体,象

各学校里的校友会,

族居外埠的同乡会,

在外国的留学生总会,分会,

上海日报公会,

环球中国学生会,

北京及上海欧美同学会,

北京华法教育会。

各种学会(象强学会,广学会,南学会,尚志学会,中华职业教育社,中华科学社,亚洲文明协会……),各种同业会(工商界各行各业,象银行公会,米业公会……各学校里的研究会,象北京大学的画法研究会,哲学研究会……有几十种),各种俱乐部……

都是近来因政治开放,思想开放的产物,独夫政治时代所决不准有不能有的,上列各种,都是单纯,相当于上回本报所说的“小联合”。最近因政治的纷乱,外患的压迫,更加增了觉悟,于是竟有了大联合的动机。象什么

全国教育会联合会,

广州的七十二行公会,上海的五十公团联合会,

商学工报联合会,

全国报界联合会,

全国和平期成会,

全国和平联合会,

北京中法协会,

国民外交协会,

湖南善后协会(在上海),

山东协会(在上海),

北京上海及各省各埠的学生联合会,

各界联合会,全国学生联合会……

都是,各种的会,社,部,协会,联合会,固然不免有许多非民众的“绅士”“政客”在里面(象国会,省议会,省教育会,省农会,全国和平期成会,全国和平联合会等,乃完全的绅士会,或政客会),然而各行各业的公会,各种学会,研究会等,则纯粹平民及学者的会集。至最近产生的学生联合会,各界联合会等,则更纯然为对付国内外强权者而起的一种民众大联合,我以为中华民族的大联合的动机,实伏于此。

(三)我们对于进行吾国“民众大联合”果有此能力吗?果可得成功么?谈到能力,可就要发生疑问了。原来我国人口只知道各营最大合算最没有出息的私利,做商的不知设立公司,做工的不知设立工党,做学问的只知闭门造车的老办法,不知共同的研究。大规模有组织的事业,我国人简直不能过问,政治的办不好,不消说,邮政和盐务有点成绩,就是依靠了洋人。海禁开了这久,还没一头走欧洲的小船,全国唯一的“招商局”和“汉冶萍’,还是每年亏本,亏本不了,就招入外股。凡是被外人管理的铁路,清洁,设备,用人都要好些。铁路一被交通部管理,便要糟糕。坐京汉,津浦,武长,过身的人,没有不嗤着鼻子咬着牙齿的!其余象学校搞不好,自治办不好,乃至一个家庭也办不好,一个身子也办不好。“一丘之貉”“千篇一律”的是如此,好容易谈到民众的大联合?好容易和根深蒂固的强权者相抗?

虽然如此,却不是我们根本的没能力,我们没能力,有其原因,就是“我们没练习”。

原来中华民族,几万万人从几千年来,都是干着奴隶的生活,没有一个非奴隶的是“皇帝”(或曰皇帝也是“天”的奴隶,皇帝当家的时候,是不准我们练习能力的)。政治,学术,社会,等等,都是不准我们有思想,有组织.有练习的。

于今却不同了,种种方面都要解放了,思想的解放,政治的解放,经济的解放,男女的解放,教育的解放,都要从九重冤狱,求见青天。我们中华民族原有伟大的能力!压迫逾深,反动愈大,蓄之既久,其发必远,我敢说一句怪话,他日中华民族的改革,将较任何民族为彻底,中华民族的社会,将较任何民族为光明。中华民族的大联合,将被任何地域任何民族而先告成功。诸君!诸君!我们总要努力!我们总要拚命向前!我们黄金的世界,光荣灿烂的世界,就在面前!

\begin{flushright}《湘江评论》第四期1919.8.4出版\end{flushright}

