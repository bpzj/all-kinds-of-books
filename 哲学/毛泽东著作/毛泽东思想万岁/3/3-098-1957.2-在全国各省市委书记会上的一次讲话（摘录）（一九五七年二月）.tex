\section[在全国各省市委书记会上的一次讲话(摘录)(一九五七年二月)]{在全国各省市委书记会上的一次讲话(摘录)}
\datesubtitle{(一九五七年二月)}


再一个思想动向,在学校里青年学生值得注意,学校办多了,就不能就业,一不能就业就请愿,石家庄三千人示威,全国大专学校一千五百多所,有三十二所发生了问题,同时许多学校出现了一些典型人物,清华大学的武天保,西北大学的朱伯诚,武汉大学的陈伯华,鞍山的武俊卿,他们说社会主义不行了,他们要上台,现在大学里百之七十至八十是资产阶级小资产阶级,欧洲来了一股风,大讲民主,这些人跟着哥穆尔卡的棒子转,现在是乡村的地主到城市去,规规矩矩。我们要分析他们脚下是空的,没有群众,天下如有变,他们也要起风浪。青年人是没有经验的,应该让他们晓得,下雨之前,蚂蚁总是要活动的,思想动摇的人,一有机会就要动摇。从去年到现在,帝国主义对我们刮了两次大风浪,英国先搞垮了四分之一,美国搞乱了,五洲党影响较大,反斯大林牵涉的面比较广,我们要懂得外部情况和内部阶级动向,有些人永远是动摇的,有些人摇了几次就不摇了,台风年年有,大家都要注意,这是社会的自然现象。

<p align="right">——摘自团中央办公厅编《中共中央、毛主席和中央负责同志关于青年工作的指示》第二部份第55—56页</p>


