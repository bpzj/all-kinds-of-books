\section[在全国宣传工作会议上的讲话(一九五七年三月十二日下午五时于怀仁堂)]{在全国宣传工作会议上的讲话(一九五七年三月十二日下午五时于怀仁堂)}
\datesubtitle{(一九五七年三月十二日)}


这次宣传工作会议开得很好,提出了很多问题,使我们懂得了很多事情,这些问题,不可能在一次会上全部解决,某些问题可以解决,现在我讲几点意见:

(一)上次在最高国务会议上我讲,在我们面前,提出一个新的问题。现在是处在社会大变动的时期,这个大变动很久就是。解放战争,是一个大变动;上溯到抗日战争,也是一个大变动。解放战争,推翻蒋介石政府,现在变动更深刻,搞社会主义革命,几亿人口进入这个运动,在几亿人口中进行社会主义改造。同时,国家又广泛地进行经济建设,办了很多工厂,办了很多水利事业,交通运输事业扩大了,办了很多的学校和科学研究机关,办了很多报纸,这在过去根据地的时候是没有的。报纸过去是有的,五年计划的四年中,特别是最近两年,突然学校、工厂多起来。这么大的变动,会反映到思想意识上来,存在决定意识,全国各个阶级的相互关系都在变化。资产阶级民主革命,使国民党政府发生变化,一个上台,一个下台,地主与农民的关系,在土地问题上,一个起来,一个下去。过去被国民党统治的企业,官僚资本超了变化,最近私营工商业,又起了变化。几亿的小资产阶级与民族资产阶级,都在起变化,个体变为集体,私有变为公有。各种不同的意见在思想上有所反映,这是完全可以理解的。这样大的变动基本上是健康的。推翻一个旧制度,建立了一个新制度,基本上是健康的,但是需要好几年才能巩固。现在是未巩固的时期,例如合作社,大约需要五年左右才能巩固。需要到第二个五年计划,今年一年,再加上五年,中国的社会主义制度就巩固起来。不能认为建立了合作社,就巩固了。这是不可能的。建立新工厂都有个巩固过程,办个学校,办报纸刊物,都有这个问题。但是应该看到,会巩固起来的。在新的社会主义制度的基础上,建立一个伟大的工业国,这个前途,应该看到。

(二)关于知识分子问题。有多少知识分子,还没有精确统计。大约的估计,各类知识分子有500万左右。可以作个粗略的分析,无非是左、中、右。少数人对社会主义不那么欢迎,甚至有敌对情绪,认为社会主义制度没有优越性,走社会主义,不那么高兴,甚至有敌对的情绪,认为社会主义制度长不长。会失败,或者希望总有一天会恢复到资本主义制度,这种人是很少数,有人说占10%,恐怕没有那么多,可能有1%、2%,或者更少一些。总而言之,有这么一些人就是。他们都是从旧社会来的。除此以外,90%以一上的人是爱国主义者,他们是拥护社会主义的。但是,也有许多人,对于在新的社会主义制度下如何工作,许多新的问题如何解决,如何答复,还不大清楚。其次,有些人对马列主义不那么都习惯,不那么热心,不那么欢迎。但是多数人是想学习马列主义。所以在500万左右知识分子中,大概有l0%左右,包括共产党员和党外人士在内,是比较熟悉马克思主义的,是赞成、拥护的。只有10%左右,或者还多一点,或许少点,这只是一种估计,对500万知识分子来讲,这毕竟是少数。多数人是想学习,也学习了一点,但不那么熟悉,有些人有些怀疑。拿爱国主义来说不同,有许多人,他不赞成社会主义,不赞成马克思主义世界观,但在外国人面前,他们是爱国主义者。另一种是中间状态,是占大多数。少数人是反对的。这种不赞成的人,很长时期都会有的。如宗教家,他可以赞成社会主义,但他不赞成马克思主义的世界观。我们是无神论者,可是宗教存在一天,他们就是有神论者。有些人心里不赞成马列主义的世界观,他不敢公开说,实际上是不赞成的,不应该强迫他接受,应该容许。大多数人当中,赞成的程度也有区别,赞成或者有几分赞成,或者还有一定的怀疑,要承认这种状况在很长时期内都会存在的。要求过高,就不符合事实,就会把任务减低。外国有一个马克思。中国怎能出几亿马克思,大家都是马克思主义者,那我们就没有宣传的任务了。我们开宣传工作会议,同志们都是做宣传工作的,我们有宣传的任务,要宣传得好。在几十年后使知识分子中更多的人接受马列主义的世界观,通过科学研究、生产和工作的实践,懂得比较多一点马列主义,这样也就好了。但不能强迫人家接受。只能说服人家接受。

(三)我们都是教育人民的人,无论办学校,科学家,新闻记者,文学家,艺术家都是教育人民的人,都是人民的先生。500万知识分子,是我们国家的财产;没有这500万知识分子,我们一件事都做不好,我们国家的文化不发达,但有500万知识分子,这些人是先生要教出许多学生,我们的文学艺术家和他们的作品,科学技术人员和他们的作品,报纸,大学教授,中小学教员,都是我们非常重要的一个部分。我们的国家只有三部分人,工人,农民和知识分子。知识分子是为工人农民服务的,是脑力劳动的工人,是教育人民的人。因此,他们有个任务,教育人民的人就要先受教育。尤其是在大变动的时期,500万知识分子都是旧社会留下来的遗产,须接受教育。如果说没有学习的任务了,是否恰当呢?有人想:“社会主义改造,就是改造资本家,改造个体生产者。不要改造知识分子”,恐怕是不恰当的。所有我们这些人,都有学习的任务。共产党员,如果不研究新的问题,过去旧了的不恰当的东西不丢掉,不接受新事物,不研究新道理,那么,教育人的任务就会作得很差了。这一点是否可以肯定下来,就是所有的人,包括500万知识分子,都要学习。说已经改造好了,不对,我看还要改造。在几个五年计划以后,还要改造,因为,那时,又有变化了,又有许多新的问题出现。这时期马克思主义还要发展,马克思主义懂得多的人仍要学习。我看,大多数人是愿意学习的,在这个基础上,有别人好心的帮助,而不是强制的学习,人家怕是大会搞思想斗争。欢迎个人研究,小组帮助。只能一面教,一面学,一面是教员。一面又是学生。并且许多东西,单从书本上是学不好的。要向工人、农民、学生学习,要向教育的对象学习。要在生产、工作中学习。

(四)整风。共产党员要准备整风。我们在十几年以前做过一回,后来也整过几次,那是检查工作性质的,不是延安那样的整风。准备今年开始,中央做出决定,做试验,明年普遍地推行。党外人士自愿参加。整风的时候是要批判主观主义,(主要是教条主义)。其次是整宗派主义,不以六亿人口和500万知识分子为范围来帮助、团结这些人,不从这个现实情况出发,就是宗派主义;再一个是官僚主义,现在官僚主义相当严重。克服缺点,纠正错误,要在全国造成一种自由批评的环境和习惯。整风方法与延安那样,研究文件,看点东西,自己有错误,有缺点,批评一下,不管什么人(包括中央委员)有错误缺点讲一讲,不用大民主,只用小民主。小组会上用小小民主也可以,和风细雨、治病救人,反对“一棍子打死人”的办法。也就在整风中间,把马克思列宁主义向前推进一步。多学习一点马克思主义。陆定一同志最近写了一篇文章,说整风运动是思想运动,可以发展马克思主义。这是很对的。有许多问题,马、恩、列、斯没有讲到,也没有看到,我们应该不限于他们讲过的范围,可以按他们的基本方针,按照他们的基本方向,有所发展。马克思主义一定要发展,不能停滞,不发展就没有生命。基本原则不能违背,违背基本原则就是修正主义。但是停滞不前是教条主义。

(五)为人民服务的问题。知识分子为人民服务,就是为工人、农民服务。除了这两种人外,就没有什么人了,资本家要变工人,地主要变农民。现在要提倡知识分子下工厂、农村去。最近有几个同志下农村,陈伯达、邓子恢两同志到农村住了几个月,很有益处,看了很多东西。走马看花,是一种方法。另一种方法,是下马看花。陈伯达、邓子恢同志下马看了,几个月做了调查,交了朋友。我们的作家、艺术家、是否应该下去?应该去,科学技术人员都下去那也不必,他就在工厂,研究人员是实验室的,都去这也不必,但有时下乡下厂转一转,也有好处。家在乡下,从不回家,一辈子见不到工人、农民是不好的。希望这些人,十人中有七个人到乡下、工厂去跑一跑,看一看。还有三个人因为年纪大,或是不愿去。有70%的人下去看看,了解情况。考察小学、中学,顺便看看合作社,人民代表,政治委员都下去看嘛,如果能办到这样就好了。也不是今天开了会,明天就了70%的跑光,那么合作社就会成灾,新闻记者这么多,他们招待不起。逐步的,在三个五年计划之内,还有十一年时间,争取知识分子与工人、农民直接接触,看一看,跑一跑,与工人农民一道,混几个月,一年、两三年,在那儿住去,安家落户,两三年再回来,在延安曾经这样作过,那时知识分子很多,有一个时期,思想混乱,有怪论,闹得天下大乱,后来开了一次会,讲了话,也有一批人下去,就好得多了。知识分子照道理来讲是没有知识的,一不耕田,二不造桌椅板凳,知识分子只是看书,书当然不可不看,可以学习前人的经验,但光看书那是不能解决问题的。要有朋友,还要研究当前情况。现在有些人下去体验生活,有些有成绩,有些没有成绩。工人农民不和他们讲真话。一看“你是知识分子,你跟我又不一样,你是来搞调查的,大概想整我一下吧?”他就不和你讲真话。看你采取什么态度?这与马列主义的世界观是有关系的。马列主义世界观是无产阶级的世界观,不是小资产阶级、资产阶级的世界观,现在,因为社会主义改造,小资产阶级变为集体劳动者,资产阶级变工人,这就有可能向他们宣传工人阶级的世界观、无产阶级的世界观。知识分子按其出身来说,许多是资产阶级知识分子。有的进过资产阶级的学校,学的也是资产阶级的世界观,如我,就进过资产阶级的学校,学过资产阶级的世界观,马克思主义是后来学的。要改变过去的世界观,换成工人阶级的世界观,不是那么容易,所以下去跑一趟,工人和农民不会和他讲真话,我们要造成风气,要有30%,40%,50%,60%,如多少年有70%,十一年之内,有7%的知识分子去接触工人和农民,研究问题,这样就可以把在书上看到的马克思主义,变成真正的马克思主义。书本上看到的常常会溜走,等于我们每天看报纸忘记了一样,马克思主义要真正学到,主要是在工作中、实践中,才能真正了解,如果这样,书本上看了马克思主义,又接触工人农民,在工作中了解马克思主义世界观,许多人就有了共同语言。有较多的人,在几个五年计划之内,不只有爱国主义共同语言,而且也有社会主义共同语言。有马克思主义的世界观、共产主义的世界观。如果不这样,我看,许多问题还是不好解决。作品也还是不能写好。有人说:“马克思主义还是少学为好,学多了文章反而写不好”。这是一种说法。马克思自己说过,理论是行动的指南,而不是教条。说学了马克思主义妨碍创作,这是一种抵抗马克思主义世界观的思想,他脑筋里有一种旧的世界观,或是资产阶级世界观,他不赞成新的东西。这要慢慢来,而且要有很长的时间,要造成风气。我们建议文艺上工农兵方向,如果说不是工农兵方向,还有什么方向呢?在延安时都是工农兵方向,今天仍然是,有人说如不是,那是什么?那只有资产阶级方向了。有入说:“百家争鸣,马克思一家,其他还有99家”。依我看,只有两家:无产阶级一家,资产阶级一家。西方世界是一家,我们是一家。还有,民族主义是一家,是中间地位。什么叫百家?新闻是一家,教育是一家,新闻中我这样办报纸,你那样办报纸即有两家。办学校,大学一家,中学一家,大学中的家可多!科学院的家可多了,技术人员的家也很多,其实不是什么百家,而是千家、万家,所谓百家,无非是言其多也。我说现在世界上基本只有两家,无产阶级一家,资产阶级一家,马列主义也有几家,马列主义一家,修正主义也算一家,挂了马克思主义的牌子,实际上反对马克思主义,例如英国的工党,也在说什么社会主义。另外教条主义也是一家。三家。

(六)片面性问题,有两种片面性:教条主义、右倾机会主义或修正主义。陆定一同志的文章讲了这个问题。一种是肯定一切,一种是否定一切。教条主义肯定一切,所谓100%的布尔什维克,不是99.9%,硬说是100%,经过十年的功夫,1935—1945年中,后来一查,“正确”没有了。教条主义外国也有,是将马克思主义片面化了,用形而上学的观点来解释马克思主义,对自己的工作,肯定一切,只许赞扬,不许批评。只许讲好,不许讲坏。北京最近发生了一次“世界大战”,好多人包围青年作家王蒙,我们替王蒙解了围。把他救出来,虽然他的作品有缺点。但是,讲中一个重要的问题,揭露了官僚主义。总而言之,讲不得,一讲就军法从事,违犯了军法。我们过去军事时期,搞阶级斗争,有些学问,有些办法。但产生了简单化,行政命令,因为领导人民向敌人作斗争,有许多事情不能那么从容讨论,要迅速,这就养成了一种作风,只有那么一条经验,一条办法。在新的环境下还用这个办法,那是不对的。特别在军队工作久了的人,容易简单化。现在围剿王蒙的也是解放军,开了几个团,把他包围起来,那不好。

另外一种是右倾机会主义,否定一切。没有一件好事。工人、农民的事业,社会主义改造,社会主义建设,是几亿人民的伟大斗争,没有什么好的可说,漆黑一团,使人丧失信心,这不合乎事实。一切都好,没有困难,也不合乎事实。也不是一切都好,要加以分析。例如钟惦棐揭露了电影中的缺点和错误,应当引起我们注意,要把他们揭露的加以改正,错误的要批评,正确的要改正,但他的批评是片面的,台湾很欢迎。陈其通等四人在1月7日“人民日报”上发表声明,有的传达错了,说我赞成,我再讲一遍,我很不赞成。这几个同志他们是忠心耿耿,为党为国,想保卫党。保卫工人阶级的利益的。不过,有这么一种情绪,是讲毒草务必去尽;他们认为“百家争鸣”提出来,好处甚少,坏处甚多。据他们估计,成绩太少,缺点太多,放出了王蒙,牛鬼蛇神都出来了,大势不好大有不可终日之势。对形势的估计是错误的,文章里面对“百花齐放,百家争鸣”的方针,似乎是赞成,其实是反对这个方针的,对这个方针不通,有怀疑。他们的文章是没有说服力的,没有分析,简单从事,短促突击,人家看了文章是不会服的。我看了就不服,我不认识王蒙,我和他又不是儿女亲家,我就不服嘛,这两派都是形而上学,都是片面性,都是毒草,都要批评,但要治病救人,帮助他们。其实,共产党员有这样的一些人,他们不过是代表很多的人,不单是个人问题。难道共产党以外,就没有了吗?有教条主义和机会主义,500万知识分子当中,也有“左”派,有右派。不是真正的左派,是带括弧的。也有肯定一切,否定一切。这也难免。事实上,在工作中,在文章中,不可能不带有片面性。要求写文章不带一点片面性,有几万万个马克思?那里有这么多,要求所有的人都不带片面性,是过分的。事实上,批评起来,各人站在各人的角度,各人根据各人自己的经验来说话的,但是,可不可以要求逐步多一些辩证法?所谓片面性是违反辩证法的,形而上学也是违反辩证法的。是不是辩证法可以逐步推广,逐步多一些呢?提出这样的要求,要求一年一年,一天一天,比较全面的来看一些问题。这样多的作家、教师、新闻记者、科学家,都存有片面性,那就不好办,所以,承认有片面性这是事实,不能不承认,这是第一;第二,要逐步克服。一万年还有片面性,不然六亿人口,全世界所有的人都成了辩证法家,总不会那样。要逐步推广辩证法,对事物要有分析,文章要有说服力,讲道理,不靠官僚架子来压服别人,不靠行政命令办事,要把做了几十年的什么局长、处长、部长抛在九霄云外。写文章要与人平等,官虽然大,话讲错了人家还是要批评。斯大林的官还不大?有了错误,人家还要讲。老资格,“革命的时候你在那里?你还不是在桌子底下爬!”或者是“当国民党”。摆出这一套,人家就不爱听。辩证法应该使他逐步地多一些,形而上学逐步少些,两种片面性逐步少一些。

共产党是否能够领导科学?人们说:“共产党能领导阶级斗争,搞政治可以,摘自然科学不行。”有些人听不惯,听了不舒服。我看他们说对了一半,现在是又能领导又不能领导。因为你不懂,他懂,他能够讲,在一门科学的具体内容上,你不能具体领导。你讲不出来嘛!不能不承认这是事实。他们也是针对这些事实来说话的。我说一半对,道理就在这里。一半不对是党既然能领导阶级斗争,也就能够领导向自然界作斗争。如果说,共产党只能领导阶级斗争,不能领导向自然斗争,这个党就要灭亡。这要有一个过程。过去忙于阶级斗争,现在阶级斗争基本完结,还没有完全完结,许多问题都要他处理,说有一百万知识分子入党,忙不过来,现在要注意这个问题,这种情况要改变,需要几十年,要三、四个五年计划,可以改变。科学要学会,要花很长时间。阶级斗争学会,也花了24年的时间,从1921年——1945年,24年才总结,有一套科学,有一套战略策略。自然科学也可能要这么多时间,无非是要进中学、大学,大学毕业了,研究一个时期,工作一个时期,也就学会了。像我过去不会作政治斗争。军事斗争,也有很多人不会的,后来是逼上梁山,一教就会了。搞自然科学,也是如此,年纪不大的人,都可以学,还可以调一些人去学习,学生中也有许多党员、团员,有十几年功夫,就懂得了,只要能够领导阶级斗争胜利了的党,现在的任务是要向自然界作斗争,就是搞社会主义建设,再有十年,二十年;是可以学会的。现在科学家是先生,要向他们学习。是学生领导先生,还是先生领导学生?当然是先生领导学生。因为你是学生,他是先生。

人民内部的斗争为主,还是阶级斗争为主?好几个地方提出这个问题。好些同志一定要讲个为主,我看不讲也可以,似乎讲个阶级斗争为主舒服些,讲到人民内部矛盾为主,似乎不好。恰好我犯了这个讳,在最高国务会议上讲的题目,就是这个问题。如何处理人民内部的矛盾。人民内部的矛盾,包括一部分阶级斗争在内,和资产阶级,小资产阶级就有斗争,有人说,说是小资产阶级思想还好些,不愿说资产阶级思想。小资产阶级在本质上是属于资产阶级范畴的。在“关于若干历史问题的决议”中已为详细说明,现在的情况下,是难解难分的。无产阶级思想之外,有小资产阶级思想,有资产阶级思想。你们说,谁为主?除无产阶级思想之外,谁为主很难分。人民内部矛盾的斗争现在很突出,“八大”已作了总结,国内大规模的阶级斗争已基本结束,现在突出的是人民内部矛盾。小资产阶级思想是人民内部问题,资产阶级思想也当作人民的内部问题来处理。现在小资产阶级思想和资产阶级思想,有些时候,可以放在一起讲。你说那些是属于小资产阶级,那些是属于资产阶级,很难划分,一定要分也可以。今天突出的问题是人民内部的问题,要具体分析,不要扣大帽子。似乎扣大帽子就好办事了,加他一顶资产阶级的帽子。许多文章都是扣帽子,当然也不是每篇文章都扣帽子。人总是要戴帽子的,在冬天不戴帽子,在外走不感冒我就不相信。但各人的帽子要适合自己的头,别人的帽子是没有量过尺寸的,往往捡了一顶即给扣上。

杂文如何写?是否一定要带片面性?前面讲了片面性是难免的,不要那么要求全面,那样是阻碍批评的发表。长文、短文也好,可以要求有不带片面性的杂文。比如,列宁有些文章是杂文性质,很有点像鲁迅,有些很尖锐,很讽刺,谁像谁,不去管他,你能说它是片面性吗?鲁迅是对敌人的,列宁的杂文很多是对同志的,也有对敌人的。鲁迅的对付敌人,可否转过来对人民内部呢?倘若鲁迅活着,我看是可以的。“不敢写文章,摸不到领导的底子”,我看底子,就是马克思主义。合乎马列主义就是好的。现在的环境不算太坏,不像国民党时期要检查。编辑部要修改文章,但有些人的文章不要修改它。政府也没有设检查机关。“百花齐放,百家争鸣”的方针提出后更宽了,但有个问题,即彻底的唯物论者,是不怕什么的,是无所畏惧的,要讲道理,即辩证唯物论者,是很革命的,还怕官僚主义?我看不要怕。但是也不能否认环境问题,现在的环境还有些人不敢写文章,所以共产党要整风,对“老子天下第一”的人,要整得他们谦虚一点,以平等的态度来看问题,作者与读者都要有平等态度,鲁迅即以平等态度对待读者和作者的。真正好的文章是会受读者欢迎的,真正的马克思主义者是不怕什么的,诚心为人民服务,立志改革。清朝末年写文章,有文字狱,如章太炎的“九音书”、“殊天华”、邹容的“猛四头”,“革命军”,都是人民喜欢的。但章太炎与邹容同时坐班房,当时章太炎是唯物论者,章太炎很有唯物论,写了一篇驳康有为的文章,说“南海小丑,不辨菽粟”,就因此坐班房三年。章太炎有篇文章叫“虬书”,里面讲了很多东西,是直接骂康熙皇帝。章太炎的“民报”,在各种困难的条件下还是办,我们虽然纸张困难,领导也有缺点,但比章太炎办白话报条件要好多了。现在要改革这个国家,要有这样志向,要有仁人志士,使我们经济、文化落后的国家,变好起来。想到这方面,就不要怕什么。又要改革,又要怕,就不好办。也要创造环境,使写文章的人能立志改革,有条件。

(七)是“放”还是“收”?这是一个方针问题。“百家争鸣,百花齐放”,放了这么久,有人想收。在讨论中同志们不赞成收的,这个意见对。中央即要放。会不会乱?有些人怕乱,乱即治,一乱一治。会不会变匈牙利?变不了,变了也不怕.匈牙利变好了,如果我们像匈牙利,有那么多的错误,有那么多的反革命,帝国主义在那里指挥。该变就变,怕有什么用处。我在1月18日省、市委书记会议上,讲过真善美,真理是与谬论对立的,一万年还是会有错误的思想,总是有丑恶的现象。好与坏,美与丑,真与假,总是相对的,没有好,就没有坏,没有坏,就没有好,没有大坏人,就没有大好人。香花毒草是对立的,毒草长多了,即会有人把毒草挖掉。真理是与错误作斗争中发展起来的,善与恶,美与丑的斗争,总有个比较,是从比较里头发展起来的。社会上好人与坏人的斗争,好人增多了,坏人就减少了,毒草不怕它多,多了可以肥田。什么叫毒草?我问过布尔加宁,欧洲一百年前洋柿子是不是毒草?他说:是有这回事。一百多年以前的毒草,现在变成香花了。我讲过耶稣,马丁路德,哥白尼,伽利略,孙中山,共产党,这个阶级认为是香花,那个阶级就认为是毒草;那个阶级认为是毒草,这个阶级就认为是香花。杜勒斯是美国资产阶级的香花,是全世界人民的毒草。蒋介石是不是毒草?是不是现在有点香了?以前他香过一个时期,北伐时香过,后来他反人民,抗日战争时期,还喊过:“蒋委员长万岁,”是我的老朋友,借过粮食。香花与毒草是对立物的斗争,对立的统一,有比较才能辨别,才能有发展,没有即不能发展。马克思主义就与资产阶级作斗争中发展起来的。有人间:马克思主义可不可以批评?我说可以批评。如果批评倒了,那就该倒。我想是难于批评倒的,批评倒了,就不是马克思主义。如果把人民政府批评倒了,请蒋委员长回来老百姓就高兴?那不会的。所以不应有害怕批评的情绪。最近看了一个电影“匹克威克外传”,是狄更斯的作品,批评十九世纪上半期的资本主义制度。资产阶级发展时期不怕批评,没落时期怕批评,如英法资产阶级在十八世纪不怕批评,在十九世纪也还不大怕批评,十九世纪下半世纪就怕了。但还能容许肖伯纳这样的人。到了20世纪,很怕批评,美国在20世纪,对批评有如惊弓之鸟。在美国住过的人知道,美国那有那多的自由,他们控制得很严,合乎它的、无伤大局的,还容许一点,罢工不是总同盟罢工不带政治性的个别罢工还是容许的。人民政府刚建立不久,70%、80%多数人都是拥护的,10%的地主、富农、特务反对,不赞成。没有办法。民族资产阶级(富裕中农)有些勉强,××同志说的话,不太勉强接受社会主义,就是说,有点勉强。经过教育,他们才服。五百万知识分子,和他们有关系的,那么服,不见得。怎么办?有两个办法,有两种政策来领导这个国家;是放还是收?还是放。罢工,就让他罢工;罢课,就让罢课。官僚主义十足,大民主不许可,小民主没有,小小民主也没有,逼上梁山,不罢课罢工怎么行?在那种情况下,是解决问题的办法,是调整社会生活的一种方法。这样可以使90%或者更多一点,99%是好的,官僚主义比较少一点,但1%或千分之一,估计到那一年总有一些官僚主义者。人民对严重的官僚主义者不得已才举行群众斗争的,这样的方针,有利于我们国家的巩固。“百家争鸣”是一种科学方法,是发展真理的方法。“百花齐放”是推进文学艺术事业的方法。牛鬼蛇神,都搬到舞台上来啰,搬一点,也是可以的。搬了几十年,几百年,没看见毒死多少人,让它再搬一下,又有什么关系。我劝我几个孩子,去取得算命的机会,过了几十年,就没有命可算了。他们去算了一回,不诚心诚意,又讥笑人家。在杭州,我要他们抽签,他们到了什么寺去抽签。有的抽得好,有的抽得坏。我说,几十年之后,你想要体验这个生活,也不可能了。禁止人们接触丑恶现象,禁止人们看牛鬼神蛇,那是不好的。我并不是提倡每一场戏都要有牛鬼蛇神,有一点也可以,让它存在几年,能天下大乱?存在这么久了,中国人并没有毒死多少。四川省流沙河写了一个“草木篇”的诗,草木篇是毒草,出了这么一篇,以为天下就会乱了,我看出一百篇也不会乱,如果有一百篇,问题即解决了。在座的看过没有?好文章呀!可以再印一下,最好今天晚上就把它印出来。向草木篇开火,开的对;四川的同志,都是忠心耿耿,为党为国,但是,批评的方法有问题。你们四川同志,不要以为我是赞成“草木篇”,反对你们的批评,我是讲,你们可以让它放一下,征求读者的意见,让懂得的人去批评。现在是放得不够,不是放得有余。不要怕批评,不要怕乱,不要怕牛鬼蛇神,不要怕毒草。不要怕犯错误。“百花齐放,百家争鸣”,是发展真理的,是发展文化艺术的,是发展科学的,是使我们少犯错误。是辩证法的。有许多事情,开始我们是不知道,不会解决,要在辩论斗争中,才能学会。你们提出的许多问题,我只是试答这几个问题。现在是放的不够,用这样的一个方针,我们企图团结500万知识分子,团结几亿人民,改变现在的面貌,建设社会主义。现在的知识分子的情况,是又好又不好。好,是爱国主义,愿意工作、学习。但不完全好。在接受马克思主义世界观,相差很远,接近工农群众,更差得远,对马列主义不熟悉,对工农不接近,这是可以经过教育改变的。首先要求共产党员改变态度,改变官僚主义、教条主义、宗派主义的态度,面向500万知识分子,几亿人口。有官僚主义存在。说服与压服,二者必居其一,是要说服。不能压,只能放。以力服人不行,要以理服人。对付敌人,可以一刀一枪,简单明了,对待同志不能用粗暴的办法,甚至对待“草木篇”,是思想问题也不要用粗暴。抗美援朝,对美国是用粗暴的方法,对蒋介石,也是用粗暴的方法。我们对蒋介石打了那么多年,粗暴的方法就是打。可是我们在对500万知识分子,几亿农民,对多少万民族资本家,和民主人士,打的方法是不适宜的。有些同志,习惯那种方法,“你还调皮呀!”那是错误的办法,错误的方针。因此,要统一战线,团结一切可以团结的力量,包括“草木篇”、包括有杀父、杀夫、杀子之仇的,有些亲戚朋友血肉相连被镇压的,这些人要他们那么舒服,不容易。一有机会,“草木篇”一篇,所以要提出统一战线,团结教育。

(八)对各省、市委意见。希望各省市委也开这样的会,把思想问题都抓起来,现在没抓,不是忙的问题,是面对500万知识分子,他们是人民的教员,我们和他们的关系,有很多新问题,关系很不正常。内部矛盾突出,要研究,要把思想问题抓起来,要提到议事日程上来。地方党委第一书记要出马,我就是第一书记,陆定一同志把我找了出来,讲了一通.我不是讲第二书记,我是讲第一书记,做了第一书记,你就有责任抓。希望也召集这样的会议,党内、党外的同志都参加,这次没有解决的问题带回去。这样的会议,证明有益处。过去开会,共产党员开广次,非共产党员、民主党派开一次,各开各的。这次会是第三种形式,党内党外一起开。希望地方召开这样的会。把思想问题摆在议事日程上来。


