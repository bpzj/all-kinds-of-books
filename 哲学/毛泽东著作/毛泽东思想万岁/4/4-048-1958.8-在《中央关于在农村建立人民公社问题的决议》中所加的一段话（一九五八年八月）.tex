\section[在《中央关于在农村建立人民公社问题的决议》中所加的一段话(一九五八年八月)]{在《中央关于在农村建立人民公社问题的决议》中所加的一段话}
\datesubtitle{(一九五八年八月)}


人民公社建成以后,不要忙于改集体所有制为全民所有制,在目前还是以采用集体所有制为好,这可以避免在改变所有制的过程中发生不必要的麻烦。

过渡到了全民所有制,如国营工业企业,它的性质还是社会主义的。各尽所能.按劳分配。然后再经过多少年。社会产品极大地丰富了,全体人民的共产主义的思想觉悟和道德品质都极大地提高了,全民教育普及并且提高了,社会主义时期还不得不保存的旧社会遗留下来的工农差别、脑力劳动与体力劳动的差别,都逐步地消失了,反映不平等的资产阶级法权的残余,也逐步地消失了,国家职能只是为了对付外部敌人的侵略,对内已经不起作用了,在这种时候,我国社会就将进入各尽所能,各取所需的共产主义时代。


