\section[《抗日模范根据地晋察翼边区》序(一九三九年三月二日)]{《抗日模范根据地晋察翼边区》序}
\datesubtitle{(一九三九年三月二日)}


晋察冀边区是华北抗战的堡垒,那里实行了坚持抗战的民族主义,那里实行了民主自由的民权主义,那里也开始实行了改良民主的民生主义,总之一句话,那里实行了互相联结不可分离的三民主义。三民主义单是空唤是不行的,一民主义也是不行的。空唤无裨于实际,敌人已占领了大半个中国,稍有良心者何忍至今徒存空唤?单单军事抗战,算是实行民族主义,但是如果没有决心实行民权民生主义便与抗战相配合,要战胜日寇是不可能的。孙中山先生在其临终遗嘱上面,说他积四十年的经验,深知欲达到自由平等之目的,必须实行两大革命原则。就是:(一)唤起民众,(二)联合世界上以平等待我之民族共同奋斗。岂有国难严重到了今日的程度,还可以不实行孙先生的临终训示?唤起民众就是要实行民权民生主义,否则无从唤起,尤其是民权主义,真如大旱望云,一刻不可成缓。…………晋察冀边区坚决实行三民主义的精神,是值得钦佩值得奖励的。过去汪精卫辈开口闭口八路军与游击队“游而不击”,或“不游不击”,某些应声虫起而和之,然而汪精卫却,“游”到日本怀里去了,应声虫们则在四圈麻将世界里大打其“游击”,真不识人间有羞耻事?晋察冀边区里面没有汪精卫党徒,也没有四圈八圈麻将,那里却坚决实行了三民主义,用艰苦奋斗的游击战争创立了华北抗战的堡垒。……聂荣臻同志这个小册子,有凭有据地述说了该区一年半如何实行三民主义与如何坚持游击战争的经验,不但足以击破汉奸及其应声虫们的胡说,而且足以为各地如何实行三民主义,如何唤起民众以密切配合抗战的模范。谓予不信,视此小册。于其出版之始,乐为序之。

<p align="right">(聂荣臻:《抗日模范根据地晋察冀边区》八路军军政杂志社,一九三九年版)</p>

