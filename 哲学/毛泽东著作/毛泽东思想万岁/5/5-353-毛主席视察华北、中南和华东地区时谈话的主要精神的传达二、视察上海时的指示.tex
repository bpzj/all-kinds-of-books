\section[毛主席视察华北、中南和华东地区时谈话的主要精神的传达二、视察上海时的指示]{毛主席视察华北、中南和华东地区时谈话的主要精神的传达二、视察上海时的指示}


毛主席到上海视察。在上海,见过毛主席的人说,毛主席红光满面,步履尤健,身体非常健康。

我们经历了“一月风暴”和“九月高潮”,这是毛主席亲自提出的。

一月革命的经验,主席总结为:无产阶级革命派联合起来,夺党内走资本主义道路当权派的权。

九月份毛主席亲临上海,对工人阶级、红卫兵的大联合作了重要指示,使上海的大联合达到了高潮,出现了空前未有的大好形势,使我们取得了巨大的成就。

\kaoyouerziju{ (徐景贤同志九月二十五日在上海市革委会扩大会上的讲话传达)}

在讨论训练干部的问题时,主席讲:现在红卫兵都当权了,不训练也很难办,可以开训练班,这个事情跟红卫兵讲一讲。他们还很年轻,容易犯错误,他们犯错误像列宁说的,上帝还是会原谅的,我们犯错误就不行了。训练可以人少一些,深谈一下,一次不行两次,可以多谈几次,要用自己犯错误的经验,告诉他们犯了错误怎么办。对红卫兵要进行教育,加强学习。

主席对红卫兵很关心。你们可以采取这个办法:几个负责人在一起多谈谈,不要急于一次谈成,不要动不动就说你没诚意,帽子很多。主席跟我们讲:如果在七大,不用整风方法把大多数团结起来,革命就会受损失。告诉那样争名争地位的人,在七大选举,有的人争中央委员,有的人反对王明当委员,但主席提名叫他当。主席说:不当中央委员不见得是坏人,当中央委员的不见得都是好人。王明就是坏人,只有这样,才能团结大多数。八大时很人多不同意王明当中央委员,主席仍提名。把那些反对你但实践证明是犯了错误的人一道参加工作也没有什么坏处。主席讲:革委会也可以叫几个保守派进来嘛。

文化大革命是触及灵魂的思想斗争,把人关起来不是办法。有些人应该让他们活动,他们总是要走到自己的反面的。斗争会要允许斗争对象讲话,允许人家答辩,不要不让讲话。主席总是讲:让人家讲话嘛。有的造反派讲,我们是按《湖南农民运动考察报告》上所说的那样做的。主席跟我讲过:那是对付地主阶级,现在对于部,同对付地主阶级不一样,对干部应允许答辩,允许他们讲话,包括陈丕显,曹荻秋。主席在上海看过电视斗争陈、曹大会,没戴高帽子,没挂牌子,比较文明,主席很满意。主席还说了,为什么不允许他们讲完呢?人家还没说几句就喊“打倒”,……允许人家把话讲完,然后再反驳,真理是在我们手里嘛!

你们提口号要留有余地。动不动就发勒令,“否则就采取革命行动”,例如对付刘、邓。如果他不出中南海怎么办?你们自己毫无余地。过去我们曾对美国发出一次通牒,没有通过主席,后来主席就批评说:发通牒干什么?不照办怎么办?

最难的是教育革命。主席在去年七月份就讲学校斗批改要靠同学自己搞,最近主席要我们小学、中学、大学都拿出一些典型经验来。

{\raggedleft (张春桥同志九月二十九日召开上海高校负责人会议上传达)

有的头头私心杂念重,根本不顾国家利益,争核心、争名称、名位、名次,不是按照路线比较正确。上海交大反到底兵团就是这样争,主席问我和姚文元:这个大学不是一月风暴不错吗?现在怎么弄成这个样子呢?劝他们不要争了,核心有啥好争的?核心总是在斗争中形成的。

\kaoyouerziju{ (九月二十八日中央首长接见东北三省代表团时,张春桥同志的讲话传达)}

主席在上海看了怎样砸上柴“联司”的电视,上面两次出现“揪军内一小撮”的字样,主席就指示,提“军内一小撮”是不对的,要去掉它。

主席在上海视察时,除了对工人大联合作了重要指示外,对学生也作了重要指示。就是在谈到交大“反到底”时,要他们“无条件实现革命的大联合”。[“按:人民日报、红旗杂志、解放军报国庆社论改引如下:革命的红卫兵和革命的学生组织要实现革命的大联合。只要两派都是革命的群众组织,就要在革命的原则下实现革命的大联合。”]

\kaoyouerziju{ (张春桥同志九月二十五日对上海市革委会的电话传达)}

毛主席在关键时刻指示:

把“团结——批评——团结”改为“团结——批评和自我批评——团结。”

\kaoyouerziju{ (徐景贤同志九月十九日在党校传达)}


