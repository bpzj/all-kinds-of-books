\section[在成都会议上的讲话(一)(一九五八年三月九日)]{在成都会议上的讲话(一)}
\datesubtitle{(一九五八年三月九日)}


现在提出以下一些问题来讨论。你们的问题也提出来。

一、协作问题。现在普遍存在协作问题。这是××同志提出的。全国省与省、市与市、社与社、农、工、商、交通、贸易、文教,都要协作。

二、中心工作与非中心工作如何去作。县委书记搞了中心工作,其他同志就不高兴。在县级以下,不要因为中心而丢掉其他。

三、税制和价格问题。

四、地方工业中的劳动法。县、乡工业是否实行八小时制,劳动保护,工资福利如何?

五、第二本账问题,要在这里谈谈,提出原则,党代表大会通过后六、七月交人代会通过。

六、究竟多久完成十年农业计划和工业计划?个别合作社已完成或一两年完成,或苦战三年完成,十二个省五年完成,但未把荒年算在内,恐怕落空,湖北5-7年完成(包括二年灾荒),争取五年完成,这就此较主动。现在账已公布出来了,完不成要挨骂,有无把握?挨骂不要紧,无杀头之罪,无非是主观主义。我现在又有点“机会主义”,无非是怕打屁股。

地方工业,全国劲头很大。东北农业劲头不大。辽宁工业已占85%,着重搞工业,没有注意农业,没有双管齐下,是“铁拐李”,农业腿短。

七、招工问题。现在又有大招工的一股风,这个不得了。山东要招15万人,山西要招临时工17万人。1956年工资冒了10多亿,如果不注意,就要发生浪费。

八、平衡问题。全国、省与省、城与乡之间的平衡,要很好研究一下。全国各地搞工业,上海工人如何办,哪里去吃饭。现在好像不要平衡。还是应该要一点。现在有人认为越不平衡越好,是否有道理?

九、粮食包干问题,浙江有一个报告,已印发。

十、又统一又分散——地方分权问题。欧洲现在没有统一的国家,可是地方发展了。中国自秦至今,一统天下,统了,地方就不发展。各有利弊。

十一、上层建筑和经济基础的关系,生产力和生产关系,究竟有什么问题?这两类矛盾的情况如何?克服的趋势如何?

十二、两种方法的比较。一种是马克思主义的“冒进”,一种是非马克思主义的反“冒进”。究竟釆取哪一种?我看应该釆取“冒进”。很多问题都可以这样提。例如除四害,一种是除掉四害,一种是让四害存在,除四害也有两种方法,有快有慢,快一点能除掉,慢一点除不掉。执行计划,一种方法是十年计划二十年搞完,一种方法是十年计划二、三年搞完。又如肥料,1956年此1957年多一倍,1958年又超过1956年一倍。肥料多好还是少好:去年生产不起劲,今年不仅恢复,而且超过1956年。那种办法好?1957年的“马克思主义”反冒进好,还是1958年的“冒进”好?这两种方法要比较。苦战三年。改变面貌,是办得到的,但“一天消灭四害”,“苦战三天”,这就不是马克思主义了。

十三、文教,有人提议搞14项。商业是否也搞14项?

十四、技术革命和文化革命。不断革命论。在南宁会议只提出了技术革命。现在有人加上文化革命可以研究。

十五、要跃进,但不要空喊,要有办法,有技术,指标很高,实现不了。通县原来亩产150斤,1956年一跃为800斤,没有实现,是主观主义。但无大害处,屁股不要打那样重。现在的跃进,有无虚报,空喊,不切现实的毛病。现在不是去泼冷水,而是提倡实报实喊,要有具体措施,保证口号的实现。

十六、整风问题。双反抓到了题目。知识分子“专深红透”这个口号很好。刘备招亲,弄假成真,他们也是有真的,有假的,他们有小部分是假的,多数是半真半假的,可以发生突变的。不要多少时候就会变的,因为去年整风反右为基础,今年又有生产高潮,思想有很大改变,这是整风的形势。

基层整风如何作法?要大鸣大放,大整大改。群众中一些错误思想也要解决。这些工作都要做,不然,热情就不够高。

十七、右派大会开不开?一个城市、一个区、一个学校召开右派大会,有左派参加,主要目的是争取分化右派,给他一条出路,一打一拉,又打又拉,就是给右派一条出路。

十八、农具改革运动,要一直改到拖拉机。湖北省当阳县的车子化,是技术革新的萌芽。

十九、六十条现在还不是正式的文件,要修改或重新写,基本观点对,要有所增减。

二十、报纸如何办?中央、省、市、专(市)、县、区报纸如何改变面貌,生动活泼,人民日报提出23条,有跃进的可能。组织、指导工作,主要靠报纸,单靠开会,效果有限。

二十一、国际形势和外交政策。宦乡说英国的备忘录,刺得我们很不舒服,其实他们是用针刺我们,而我们则用锥子锥他们,我看很舒服。他们不希望我们公开辩驳,是因为国际形势,国内大选和做买卖对他们不利。印尼、阿拉伯世界的情况是好的。朝鲜、波兰(农业问题)有希望,不是一团黑暗。十二国经济协作要研究。政治要和业务相结合,是否外贸在政治上有不足之处?可叫兄弟国家制造我们需要的东西。是否参加十二国协作会议,是否成为正式委员,我看问题不在形式,而在于实质。

二十二、国防计划问题。

二十三、出理论杂志问题。

二十四、过去八年的经验,应加总结,反冒进是个方针问题,南宁会议谈了这个问题。谈清楚的目的是为了使大家有共同的语言,好做工作。

规章制度。××同志在南宁会议谈了规章制度问题,规章制度从苏联搬了一大批,如搬苏联的警卫制度,限制了负责同志的活动,前呼后拥,不许上饭馆,不许上街买鞋,这是谈公安部。其他各部都有,现在双反、整改,大有希望。有些规章制度束缚生产力,制造浪费,制造官僚主义。这也是拿钱买经验。建国之初,没有办法,这有一部分真理,但也不是全部真理。不能认为非搬不可。政治上、军事上的教条主义,历史上犯过,但就全党讲,犯这错误只是小部分人,多数人并无硬搬的想法,建党和北伐时期,党比较生动活泼,后来才硬搬。规章制度是繁文缛节,上层建筑,都是“礼”。大批的“礼’,中央不知道,国务院不知道,部长也不一定知道。工业和教育两个部门搬得厉害,农业搬的也有,但是中央抓得紧,几个章程和细节都经过了中央、还批发一些地方经验,从实际出发,搬的少一些。农业有物也有人,工业只有物没有人,商业好像少一点,计划、统计、基建程序、管理制度、财政,搬的不少,基本规章是用规章制度管人。搬要有分析,不要硬搬,硬搬就是不独立思考,忘记了历史上教条主义的教训。教训就是理论和实践相结合,理论从实践中来,又到实践中去,这个道理未运用到经济建设上。马列主义的普遍真理与中国革命具体实际相结合,这是唯物论,二者是对立的统一,也就是辩证法,为什么硬搬,就是不讲辩证法。苏联有苏联的一套办法,苏联经验是一个侧面,中国实践又是一个侧面,这是对立的统一。苏联的经验只能择其善者而从之,其不善者不从之。把苏联的经验孤立起来,不看中国实际,就不是择其善者而从之,如办报纸,要搬真理报的一套,不独立思考,好像三岁小孩子一样,处处要扶,丧魂失魄,丧失独立思考。什么事情要提出两个办法来比较,这才是辩证法。不然,就是形而上学。铁路选线,工厂选厂址,三峡选坝址,都有几个方案,为什么规章制度不可以有几个方案?部队的规章制度,也是不加分析,生搬硬套,进口“成套设备”(不是建筑上的)。所有制、相互关系、分配为生产关系的三大部分,规章制度,有一部分属于生产关系,工资福利属于分配,都是生产关系。

