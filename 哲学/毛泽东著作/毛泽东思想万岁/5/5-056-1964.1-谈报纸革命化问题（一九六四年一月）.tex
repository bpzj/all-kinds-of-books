\section[谈报纸革命化问题(一九六四年一月)]{谈报纸革命化问题}
\datesubtitle{(一九六四年一月二十日解放日报总编辑\\魏××在解放日报编委会上传达)}


毛主席认为,办好报纸的根本问题,是报社人员的革命化问题。革命化就是肃清一切封建思想、资产阶级思想影响的问题。有些错误思想是容易看得出来,有些就不容易看出来,比如文汇报的《胆与识》一文错误容易看出来。(按:此文表扬“年青一代”敢于批评将军。)不革命化的另一表现是头脑中缺乏辩证法,往往把问题说死了。

革命化是立场、观点问题。报纸究竟要宣传那些东西,究竟要系统宣传那些东西,能不能选择得好,都是革命化问题,解放日报的干部那么多,为什么写不出好东西?要革命化一定得下去,参加实际斗争的锻炼,要使人革命化,同时要使机务革命化。

报纸一定要抓思想。主席最近谈到解放日报的好处,说:“好在比较注意抓思想,比较抓思想工作。”一张报纸从头到尾都要思想化。你们(按:解放日报)过去和现在,这一点作得不够,常常把一些好的东西,当另件处理,有时候又把一些一般的东西当作大东西处理了。

依靠什么人办报?要依靠社会主义积极分子办报,好的东西应该让群众自己写。

你们要善于抓住活的东西,把他提到理论高度,宣传活的哲学。报纸要有各种议论。

国际消息。主席说:“外宾接待消息,地方报纸可以少登一些,这样,可以让出地方来宣传活人活事活哲学。”

要总结报纸工作的经验。

一九六四年主席对柯庆施同志讲:

革命化的三个意思:(1)要反对封建主义、资本主义思想;(2)要下去实践,同工农结合;(3)要学习唯物辩证法。


