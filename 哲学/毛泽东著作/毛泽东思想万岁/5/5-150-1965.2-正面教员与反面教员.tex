\section[正面教员与反面教员(一九六五年二月)]{正面教员与反面教员}
\datesubtitle{(一九六五年二月)}

革命的政党,革命的人民,总是要反复地经受正反两个方面的教育,经过比较和对照,才能够锻炼得成熟起来,才有赢得胜利的保证。我们的中国共产党人,有正而教员,这就是马克思、恩格斯、列宁、斯大林。也有反面教员,这就是蒋介石、日本帝国主义者、美帝国主义者和我们党内犯“左”倾或右倾机会主义路线错误的人。如果只有正面教员而没有反面教员,中国革命是不会取得胜利的。轻视反面教员的作用,就不是一个彻底的辩证唯物主义者。

\kaoyouerziju{ (摘自《赫鲁晓夫言论集》第三集的出版说明)}
