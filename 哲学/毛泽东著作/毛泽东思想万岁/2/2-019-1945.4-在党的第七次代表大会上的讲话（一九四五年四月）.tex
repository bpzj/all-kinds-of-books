\section[在党的第七次代表大会上的讲话(一九四五年四月)]{在党的第七次代表大会上的讲话}
\datesubtitle{(一九四五年四月)}


讲三个问题:

(一)形势与路线;(二)关于政策方面的几个问题;(三)关于党的几个问题。

(一)形势与路线

七大应决定一条什么路线?我们的政治路线是:放手发动群众,壮大人民力量,在我们领导下,打败日本侵略者,解放全中国人民,建立一个新民主主义的新中国。前者是组织队伍,后者是目的。建立一个独立、自由、民主、统一、富强的新中国,这一条路线不是我们党内历来就有的吗?是的,历来就是这样,十月革命以后,中国革命性质就从旧民主主义变成了新民主主义,从大革命到内战到抗战。我们的路线我们的纲领,用一句话说,就是我们的革命是无产阶级领导的人民大众反帝反封建的革命,我们的纲领就是这一条,若干年来,全党全国人民就是为着这一条纲领奋斗并团结起来。我们的军队就是无产阶级领导的人民大众反帝反封建的军队,我们的政治就是无产阶级领导的人民大众反帝反封建的政治,我们的经济就是无产阶级领导的人民大众反帝反封建的经济,我们的文化就是无产阶级领导的人民大众反帝反封建的文化。总而言之,当前的一切革命任务,就是这种性质的。这里边有队伍,有敌人,有指挥官(领导者)。队伍即人民大众,敌人即帝国主义封建势力。队伍的领导者指挥官即无产阶级,说我们的纲领很多,不好记,但是并不怎么复杂,就是这么几条纲,这条纲领常常被同志们忘记。反帝记得牢一些,反封建常被忘记。何以忘记了呢?因为农民大众里有农民也有地主,反帝是要农民还是要地主呢?要地主即容易忘记要农民,要农民不易忘记要地主,所谓人民大众,主要的部分是农民,有一个时期却忘过农民,一九二七年即忘记过,当着农民伸着手要东西的时候,共产主义者却忘记农民,抗战初期也犯过类似的错误。靠什么打倒日本建立新中国?所谓人民大众最主要部分是农民,其次是小资产阶级,再其次才是其他民主分子,忘记农民就没有民主革命,就没有一切,没有民主革命也就没有社会主义革命,忘记了农民,你就是做了一百万件事情,也没好处。因为没有力量。仅仅靠几个小资产阶级,自由资产阶级,没有力量。人民大众主要是农民,没有农民就没有一切,没有农民谁给你饭吃?谁给你当兵?没有吃,没有兵就没有一切。

无产阶级有一部分组成共产党,共产党是无产阶级觉悟的部队,当然也有别部分的人参加,如农民、小资产阶级、知识分子、自由资产阶级,地主等人,但这是他们的出身,出身和进党不同,进了党就成了无产阶级的党。革命要组织队伍,就要靠农民,组织了队伍,就有了一个指挥宫,在中国要么就是无产阶级,要么就是大地主大资产阶级,主要的体现者就是共产党和国民党,中国的社会是两头小中间大。两头小,但两头都强。中间大,但政治上都弱,两头强即共产党、国民党。自由资产阶级同我们争领导权,不要以为自由资产阶级进步的不得了,与共产党差不多,他们有他们独立的纲领和行动,现在即是民主同盟(其中也有小资产阶级),民主同盟主席张澜发表的声明,即他们独立的政见,左舜生的一次声明亦然,他们也赞成联合政府;他们也在联合政府问题上和我们统一起来,所以我们目前还要团结他们。但他们有独立的主张,他们是左右开弓,他们一方面不满意国民党一党专政,一方面也不完全同意共产党,他们站在国共两党之间,这就是规定了他们自由的性质,所谓中间派。

大革命时期我们犯过错误,从一九二一年到一九二六年我们的党是马克思主义的,后一时期,就是少马克思主义或者是不要马克思主义了。大革命前我们组织了一千多万农民,什么是马克思主义呢?无产阶级领导农民起来革命就是马克思主义。就是说那时已有马克思主义了,一九二七年以来陈独秀为首的党,忘记了人民大众,忘记了农民,忘记了无产阶级领导,那就没有马克思主义了,他不要无产阶级领导,有时又不要农民,农民伸出手来要东西,他泼给农民一盆冷水,共产党受了地主的影响,向农民泼冷水,不要农民,还有什么反封建?还有什么反帝?帝国主义是要到中国来揩油,而中国五分之四的人民是农民,就是说中国五个人中四个是农民,比如用五个指头打敌人说不得了,砍下四个剩一个小指头(五分之一的城市人口)这样就使无产阶级孤立起来了,无产阶级总司令没有(农民)了,这变成“空军司令”了。农民不来,小资产阶级也不来,开小差了,这不能怪小资产阶级。小资产阶级就是看你力量大不大,小资产阶级有时很凶,老子天下第一,有时他又屁滚尿流。当你剩下一个小指头时,你向他说:“同志来不来”?他不来,他说有事,家里老婆生了病,这不能怪小资产阶级,只怪我们总司令手下没有兵,没有力量。只有我们的总司令招兵买马,原草国粮,我们有了四个指头即五分之四的农民及其阶级,有力量了那他就来了。你叫他,他本来有事,他会说没有事,老婆也不生病了。从前犯过错误忘记过领导权,即要领导总有个被领导的吧。大革命后期,农民不来,小资产阶级也不来了,于是地主集中力量一打,朋友也打来了,又搞土地革命,搞土地革命对呀,但又来了个急性病,急性病也是不要农员,他们只要工人,急于打大城市,农民是附带的。住在农村不懂得农民,走马不看花。下马看花叫调查研究。走了两万五千里,没有看见花(农民)。什么叫富农,中农?对不起不知道,小资产阶级,其他民主分子要不要,于是又成了“空军司令”。我们党曾经两次大了又小,第一次五万党员,后来剩下一万,第二次三十万,后来剩下二万五千,现在又立起来了,可再小不得了。如像女人脚现在是放的时候了,不要再包了。抗战以前我们有准备,叫放手发动群众,壮大人民力量,中央历来都是说,只有人民战争,才能打败敌人,所谓人民战争就是农民战争。从来没有说过不要人民战争,可以打败敌人,从有马克思主义以来,一百另二年了,所有全世界的真正的马克思主义者,没有一个说过不要人民战争可以打败敌人。(有各种各样的斗争,政治斗争、思想斗争、经济斗争,政治斗争的最高形式是军事斗争)。但是很多同志由于马克思主义不很多,可以随时忘记农民:,不要人民战争,这样即使发表演说他是马克思主义者,那也是他自封的,是假马克思主义者,有些人以为依靠国民党可以打败日本帝国主义,这就表明没有马克思主义了,党内这样的人还不少。一九三七年五月党代表大会,八月洛川会议,十一月活动分子会都批准了中央的政治路线,曾经肯定了这样一条政治路线,就是放手发动群众,中央不相信人民能够打败日本,能解放中国,至少也是暂时失掉了马克思主义,相信都不会失掉马克思主义,暂时失掉马克思主义,还可以招“魂”,把这个“魂”(马克思主义)招回来,不相信没有人民能够打败日本,这样提出问题不是将无产阶级降低到资产阶级的,而是资产阶级提高到无产阶级,这句话的意思就是说要把资产阶级的纲领提高到无产阶级纲领,不是要把无产阶级的纲领降低到资产阶级的纲领,这个问题就是说无产阶级吸引资产阶级呢?还是资产阶级吸引无产阶级呢?这样说法被人家驳斥过,说这不是马克思主义,这种说法就是争领导权,为什么不是马克思主义呢?资产阶级总以为他对。我们应向广大农民宣传,要农民,小资产阶级团结起来,只有团结起来才能打败日本,才能胜利,只有这样说法才是共产党的宣传队,这样两句话并不复杂,要告诉人民团结起来,组织人民的军队,人民的政党,人民的政府,要改造国民党,改造他的军队,改造他的政府,不改造不行,依靠中国人民,民主分子和国际力量,我们估计国民党能改造,结果国民党来改造我们,在估计上犯了一个错误,但也有好处未赔钱。我们要他改造,天天这样讲,老百姓都知道了国民党的脸脏得很,知道了他应该洗脸。国民党能不能改造呢?应该说有两种可能,能改造,不能改造。发展进步势力争取中间势力,孤立顽固势力,有了这些条件,就可能改造。我们共产党叫蒋介石改造洗脸,他说不洗,说我干净得很呀。漂亮得很呀,老百姓说他应该洗洗呀:西安开大会,会上喊蒋介石万岁,老百姓喊:“赶快纳粮完税”!老百姓知道了要他洗脸,这就是胜利。一直到今天,还是请他洗脸,不是要消灭他的政策,就是请修改他们的错误的政策,年纪越老的人越不愿意洗脸,洗的可能性小,稍为摸一下子装装样子有可能,甚至连摸都不摸同我们争领导权,将中国拖向黑暗的是国民党反动派集团,大地主、大银行家,大买办的代表,六中全会上讲清楚了这个问题,有些同志不要求国民党洗脸,说国民党漂亮得很,提出拥护国民党、国民政府的口号。对国民党时时帮助,事事帮助,处处帮助。六中全会对这些有了纠正,开始以为国民党漂亮得很,以后出来了一个“限制异党办法”,发动了第一次反共高潮,露出了国民党的真面目,我们的同志觉悟起来了,四一年两次反共高潮,人家要求解散共产党,这样大家对国民党的幻想大体肃清了,在广大同志的头脑中,展开了一幅新图画,这才了解打败日本不是要靠国民党而是要靠放手发动群众,壮大人民力量,这说明要去掉一种错误思想,光讲不行,必须经过经验才能,我们请了两个义务教员。日本法西斯和委员长,不要薪水,他们教好了我们许多同志,六中全会以后,对农民、小资产阶级、地主坚决执行了我们的领导权,为的是发展我们的党、军队、政权和解放区粉碎了敌人无数次的进攻(扫荡),去年以来转入攻势,进攻为主,防预为辅,打退了国民党三次反共高潮和无数次的进攻(磨擦),于是国民党的影响低落,势力缩小。委员长也请了个教员岗村宁次,这个教员给泼了许多冷水,但是不要把国民党的影响和势力看轻了,现在国民党在群众中还是有影响的,要想去掉它,还需要多少年,国民党已有五十年的历史,五十年的影响,我们才二十五年,我们的影响不来,他的影响不会走。所以说他的影响还只是低落,而不是没有了,他的势力还大,还有二万万人口,一百五十万军队,他有国际地位,我们没有,所以说他的势力只是缩小,而不是没有了。但是由于抗战时期,我们执行了一条放手发动群众的正确路线,于是我们成了抗日救国的重心,广大人民群众都仰望着我们,而把国民党党放在一种影响低落与势力缩小的地位上去了。现在已经完全证明,只有这样的路线才是正确的路线。力争领导权,力争独立自主是反映了全国人民的要求,反映了全党大多数同志的要求,这条路线是从中国的土壤中生长出来的,就是共产党领导的人民大众反帝反封建的路线。现在比以前要明确了,不比以前那样了。直到三次反共高潮还有些同志不相信这一条,是否这种思想现在已经绝迹了呢?党内有各种各样的意见,对这一条正确路线并不是都完全都了解了,并不是每一个同志都是马克思主义万岁了。我们从一岁到九千九百九十九岁都是有,也有万岁的但不够。他们不相信这条正确路线,放弃领导权,放弃独立自主,放弃又团结又斗争的方针。斗争是暂时的,局部的自卫立场,即有理有利有节,这是有利于团结的方针,为什么国民党不敢同我们决裂呢?旧金山会议我们要去人,蒋说不行,我们说去,蒋说不行,结果还是去了。我们的代表已经到华盛顿,权力是争来的,不是人家送来的,有人说我们的哲学是斗争哲学(邓宝珊)他说对了,我说他们也是斗争哲学,无产阶级斗争哲学还在他们后面,离开了斗争讲团结或斗争的不恰当,没有劲,这是小资产阶级软弱性。有章乃器同志,他对恩来同志讲:“他犯过了错误。”我在一个演说中批评过章乃器主义的“多建设少号召”。不要蒋冼脸,是自由资产阶级软弱性的表现,他的理论被我们战胜了,他那个万岁不大喊了,因为万岁不要他。他有点像汉贾谊才子不迁,但他也不会一下手转过来喊共产党万岁。这种自由资产阶级还会用其它软弱性来影响共产党,而且有目的的向我们送些软香,林黛玉卖玖瑰花看来好得很就是有些刺,我们的解放日报每天放玫瑰香。他们喜欢薛宝钗,不喜欢××。(原稿不清)麻烦还在后面,革命就是麻烦,怕麻烦就赶快到阎王那里去交帐。我们党的意志变得更加加强了,不至于被他们淹没了。

总结以上问题:

(1)农民是怎样?二十五年来尤其是九年来农民非常欢迎我们的政策,但作党的领导思想来说,我们要同农民分清界线,不要与农民混同。农民出身的同志不易分清这个问题,出身和入党不问,这是两件事情,出身是农民入党就是党员,党员是无产阶级的先锋队,要与农民分清界限,是要把农民提高一步,提高到无产阶级水平,将来也要把党外人士提高到马克思主义,当然这是几十年后的事情,可是不相信这一条,就不是马克思主义。

(2)小资产阶级是怎样?小资产阶级知识分子(不管党内党外的)有两重性,革命性和动摇性,动摇性能以教育方法纠正的,整风就是证据,如文艺座谈的方法,党提出方针教育他们(不是命令他们)是能改造的,党外亦然,我们有广大的解放区有强大力量,对他们作适当的宣传教育,是可以影响稳定他们的。

(3)自由资产阶级:目前是我们的同盟军,但他们有更大的动摇性。他们要民主,这是与我们共同的,所以是我们的同盟军,动摇性在我们坚决影响下,可以使他们中立,甚至可以使他们跟我们走。

(4)大地主、大银行家、大买办阶层是国民党反动集团的代表是反动派。他们在抗战中和我们争领导权,是抗日阵营中一个最凶恶的敌人,他们要把中国拖向黑暗的老样子的中国,是我们长期来的敌人,他们把三万万六千万农民放在他们的影响之下去刮油水。

(5)外国,苏联无问题,对其他国家,也是又联合又斗争,如果出现斯可比那不行,外国还有强大的反动势力,报告中提到三个国家团结是统治一切的、主要的,报告中对黑暗讲的少,但要对他们警惕,他们给委员长撑腰,有时装作天赐福的样子。以上叫做形势与路线。

(二)关于政策方面的几个问题

(1)一般纲领与具体纲领两部分是否这一次划分的,不是的,从前也划分过,不过没有像这一次这样划分,比如“新民主主义论”小册子,写了一般纲领末写具体纲领,如六大的十大纲领,抗日救国十大纲领。各根据地地发布的施政纲领,都是性质相同,条文小异的具体纲领,更具体的如婚姻法、土地政策附件等等都是具体纲领,新民主主义就是总纲领。有纲有目,具体纲领就是“目”。

(2)孙中山。关于孙中山,引了他许多好话,把他的好处抓紧死也不放,我们死了还要我们儿子去抓。但也有区别,我们的新民主主义论比孙中山进步得多,完备得多。我们党内有很大一部分党员,不大高兴孙中山,不知为什么原因,这还是表现了不够觉悟,还保持了内战时期的作风。内战时期国民党拿孙中山打我们,不要他还可以原谅。同时那时我们力量还小,不要他也没关系,现在与内战时期的东西不同了。现在我们党大了,力量也大了,要他有好处。将来我们的力量越大,要他的好处越多,苏联以前不要宗教,现在也要它了,力量大了,多要也没有关系,苏联以前也没有积极消灭宗教。我们应有清醒的头脑,利用孙中山这个旗帜。我们的力量越大利用了越有好处。

(3)关于资本主义,我在报告中有所发挥,就是比较充分肯定了这个东西。这有什么好处呢?我们肯定的是不操纵国民生计的资本主义,曾经有的同志说,应提出没收大地主、大银行、大买办的财产,当时考虑了一下一般纲领中引用孙中山的话,具体纲领中不提为好。一般纲领中说是允许不操纵国民生计的经济存在,实际上就是说操纵国民生计者要没收。公益经济,合作社经济之外,肯定广泛发展私人资本主义。只有好处没有坏处。现在没有就是没收,就是没收蒋、宋、孔、陈的财产。你讲没有讲也讲了,还是孙中山讲的。现在的一般资产阶级还不是斗争的对象。所谓一般的资产阶级是小资产阶级,特殊资产阶级,而不操纵国民生计的资产阶级,欧洲对这些都是要没收的。对汉奸的财产是要没收的。党内有民族主义思想存在,即是直接从封建经济到社会主义经济,不经过资本主义。俄国“民粹派”社会革命党,后来都变成了反革命。布尔什维克与之相反,肯定俄国要经过资本主义,这与其说对资产阶级有利,不如说对无产阶级有利,十月革命后,中小资本家、富农,一部分(很大一部分)经济仍许存在,我们同志急得很,看见人家搞社会主义,我们也想搞,人家十月革命后到第二个五年计划不消灭富农,在那以前,当然粮食主要是从富农那供给。

(4)共产主义,在报告中提到了,但未强调共产主义纲领,即没收私有财产,消灭阶级。这样写上去只有一点好处,即教育党员,许多党员不懂得什么是共产主义,农民把地主的土地分给了他,就叫共产。此处不提是好。

有人劝我们改名字,他们说:“先生之志则大,先生之号则不可”许多人说,就是这个名字不好,如果改了名字,我愿意参加,英国也劝我们改名字。

改个什么名字,国民党吧!保守党吧!福尔曼写了一书叫“红色的中国”,说红星照耀中国。不管你收成什么名字,也都叫你红党。凡是红色的都记在我们的账上,老百姓怕儿子养不成,起个名字叫狗儿、马儿,我们是否也如此呢?一个记者严金先生说,我们是“温和的民主集中制。”他是个资产阶级的自由主义者,他嫌我们温和呢?这个名很好,这个名字不要改。老百姓很爱他,江西老百姓叫我们为“共伞党”也好。

(5)对国民党有尖锐的批评,但是客观的没有超过实际,他有一点好处也挂在他的账上,但可惜他的好事太少了。尖锐批评留有余地,还可合作,还可谈判,并无打倒“委员长”字样,连“委员长”字样也没有。死人只提到孙中山,罗斯福,活的少提为妙,反动的只提到希特勒,革命的只提到斯大林,留有余地就可以少犯错误,若是打倒蒋介石就会犯错误,他们几次挑拨我们去打倒国民党。我们不打它,我们还是说:“你洗洗脸,我们结婚爱情重得很”。但自卫立场必须保持,他若向我进攻必须反攻,必须回答其进攻,文的武的,特别是武的必须回击,要坚决、干净、彻底、全部消灭之。要消灭的干干净净。我对联络参谋说:我们的方针是:“不以天下先”(不打第一枪一一老子),其次是“退避三合”(《左传》),第三是“礼记”所说:“礼尚往来,往而不来非礼也,来而不往亦非礼也”。“人不犯我,我不犯人,人若犯我,我必犯人。”过去这一方针,只要我手里还有一支枪,就要打到底,你缴了我九十九支枪,好的呱呱叫,但我还要打:从前有愚公移山的故事……,不移掉他不相信。世界上反动派不革掉它不行的,我们这一辈子打不完,交给我们的儿子去打,不把世界上的反革命打平不为止。打,但是自卫立场这叫有理,有些同志劲头来了,就忘了这一条,不对,暂时的局部的自卫的方针,违背这一方针,要犯错误,准备国民党搞麻烦,还长得很。

(6)争取旧军队,利用旧军官,改造旧军队。这件事要在全党进行宣传,不能简单对付,要有政策。改造不了的只是那些反动派,相当广大的旧军官改造得了的,从前我们在这个问题上犯过错误,在中国这样半殖民地中许多旧军官是找饭吃,为了升官发财,我们力量大了就不怕他们,全都是造反吧!害人主心不可有,防人之心不可无,但是我们要把眼光放远一点,要争取几百万旧军队,做的时候要有严肃性,警觉性,有了这个性,暗藏特务公开造反都不必怕,造反吗,他们过去不是造反了多少年,要走吗?开欢送会,送路费。他原先没有来,现在走了还不是一样,我们是毫无损失。损失了些小米法币,但政治上的胜利却很大的,走了何时想来再来。对进步争取,要适当协助,“适当”就是不多不少。从前我们有两个错误,一个全缴枪,一个全帮助,无原则帮倒忙,要作两条路线斗争,反“左”又反右,对了进步的与人民有联系的旧军队,旧军官要适当的协助,加以改造,加以教育。

(7)我们的军队、八路军、新四军,东江纵队,山西新军,内部外部都是统一战线的、联盟的。内战时期是否党与非党联盟呢?是的,我们也叫过党军,所谓党军是说党的领导,不是说不与党外人士合作,我们军队里边党员只占少数(三分之一)多数党外人士(农民知识分子),现在我们的军队就是人民大众的,为人民大众所有又为人民大众服务,就是民有、民治、民享的,只加了一个共产党的领导。若有人问:“贵军何种性质?”简单的回答就是无产阶级领导的人民大众反帝反封建的军队。不要天天讲:领导你领导你,使人家不喜欢。怎样领导是靠政策、行动、工作来领导,只要人家跟着我们走,你不说也就是领导了,领导权要掌握,但不是天天念经一样的去叫,就能掌握领导权。现在山东有三支伪军,原来不是马克思主义的,现在叫搞通思想,他们很高兴,变成了八路军,这就是军外合作,军内合作,有饭大家吃,有敌大家打,不发饷,自己动手丰衣足食,三大纪律,八项注意,这也是两条路线斗争。坚决反共反革命者不行,只要赞成民主者,愿意来者,赞成我们的主张者加入我们的军队,我们欢迎,姜太公钩鱼,愿来者上钩,立场很清楚。

(8)扩大解放区。一切可能攻克的地方要进攻,还要防敌人之进攻,原来是防御的,抗战的我们进攻的防御,41年42年我们缩小,42年43年我们又发展了,所以现在规定两条:第一是进攻,第二是防御。扩大发展,集中小的武工队,大的六个团。进攻的方针,因为情况变化了,敌人现在是自顾不暇,而我们的力量大起来了,我们集中力量去进攻他,他就是很少力量来进攻我们了,这是不是冒险主义?不是的,我们讲的是“可能”攻克的地方进攻,又讲巩固,故不是冒险。

再有一年的攻势,由分散的游击战逐渐转到以正规战为主。

抗战开始时敌人是日本,友军是以前的敌人-国民党,我们是小拇指,只有三万人,36年、37年出发时也不多,这时的任务是加一个指头再加一个指头长大起来,大家都赞成。如何长法,麻雀战、游击战,六中全会时提了十八条理由,凡有麻雀吃的地方都去。满天飞,讲十八条有理有利,你少一条不行,应该麻雀战,日本人与国民党对我们没办法,他们挑拨我们打仗,民族英雄啊!麻雀战机会主义吧!谁也不愿意当机会主义,而愿意当民族英雄,麻雀战是有点机会主义,现在集中到九十万不麻雀吧!永远麻雀,麻雀万岁!不能!我们这个麻雀长成了野鸦,书上讲有个大飞鸟,从北洋飞到南洋,一个翅膀可以横扫全中国,我们要把小麻雀变成大麻雀。

抗战开始,你搞点钱,我搞点面,还有就食,分散才能就食,要活下去就要分散,内战时期,搞了一个正规化,忘记了人是要吃饭的,路是要连步走的,子弹是会打死人的,不生产打了几年,吃光了向后转向前转,万里长征,英雄豪杰。

现在要准备由游击战逐渐转到正规战,得到新武器,就可无敌于天下,报告中讲到进攻为主,防御为辅,马克思主义多了不同,一种是香马克思主义,一种是臭马克思主义,一种是活马克思主义,一种是死马克思主义,香的活的马克思主义只有一种也就可得。现在要防止骄傲。我们九十一万了,日本人滚蛋啊!打,一要打赢,二要有饭吃。打不赢“聋子放炮竹”散了。

现在应该准备集中兵力,向敌人薄弱的地方进攻,集中一千人消灭敌人一百人。逐渐转为正规战,但是无论什么情况都集中。

(9)还有一个农村与城市问题。农村路线到一定时期转变到城市路线。从前城市农村问题争执得很厉害,叫做政治路线,下马看花,调查研究,在乡村走一万年,走就是马克思主义,不是要看情形,马克思主义是当着要到农村去时,就去农村,当着要到城市去时就到城市,现在准备夺取城市,掌握城市交通、工厂,有人讲我们是土皇帝有些像,永远十皇帝下去不好,洋房子先生也做工,现在要准备把重心转到城市去。

以城市为中心,不是一切都到城市去,现在把城市工作提到与根据地同等重要的地位。

要慢慢讲通到城市工作的道理,到城市去做秘密工作,不要学梁山伯,行不改名,坐不改姓。

至于转变,由乡村转变到城市,由游击战转变到运动战,要好好准备,要准备发生意见分歧,一定要发生一些分歧,要准备将来可能发生的分歧。高级干部头脑清醒,要准备应付分歧意见。有了准备,分歧意见可能减少。

(10)市区与根据地、根据地是战略出发地,一万万人口太少了。搞到两万万以上就有了,局势就变化了,九十万军队是分散的,现在我们没有一个地方可以集中十万人,集中一块就没有吃的。且没有新式武器,打中心城市不行。内战对七次(原文不清),两条:①想爬上去:②爬不上去。

现在我们是分散的,现在条件是就地就粮,将来的条件是武器加数量,全中国在我手中时需要三百万到五百万军队,自己生产,老百姓负担不多少,现在需要扩大军队,但要在可能条件下,所谓可能条件,就是老百姓负担问题,还有别的问题,没有枪杆子不好。

(11)中国人民解放区人民代表会议问题,大会建议召开中国人民解放区人民代表会议是一件重要事,所谓会议就是说不是普选的。而是军队、政府、郡众团体等等派代表开会,普选必须在战后了,召集后要发言作决议,产生一个领导机关,要比较快的召开,党外人士要占多数。准备叫中国人民解放联合会。开会时要打电报给蒋委员长,请他组织联合政府,向他请个多少次,蒋说联合政府是推翻政府。党派会议他说是分赃会议,我们孙中山召开国民会讲,他说:“你们把我当北洋军阀,你们就是总理”。那简直就是个流氓。

不说是政府是要起政府作用,这个联合会出来时,一定要骂我们称王称霸,称政府之王,称政府之霸,看那个王八蛋敢反对。

(三)、关于党的几个问题

(1)个性党性问题:

整风时曾发生这个问题,新闻记者说我们消灭个性,只要党性,文件上强调党性,这样就是不对的,报告中说:“民族压迫与封建压迫残酷地束缚着中国人民的个性发展,使他们智慧和身体都不能发展。”鲁迅的骨头硬,半殖民地国家,像鲁迅的骨头是可贵的,有些人受压迫着就变成外国的奴隶,上海公园门口上挂着牌子写着:“中国人与狗不准进去”外国民族压迫中国不行的,法西斯压迫人民不行的,中国共产党代表人民反对这些压迫,压迫人民不行的,这就解放了个性。

党则不同,党是无产阶级有组织的部队,统一的部队,它是为着一个目标而奋斗前进的,没有这种统一是不行的,会被敌人所消灭,没有民主集中制不行,革命的民主与旧民主不同,有更大的民主,党员人民中之自觉地承认党纲党章,自愿牺牲一切的,(有些人不加入党,不受拘束)党只要服从党纲、党章、决议案,牺牲自己为人民是自愿的。

我们的党比较过去要统一些,是更加严格,更加统一的军队,我们只是说的比较统一,没有讲已经统一。有人说,放下来没有问题,提起来问题很多,这话很对。36年有组织的党员有两万,现在比抗战初期扩大了几十倍,一百多万党员,意见很分歧,将来党还要扩大,意见还要分歧,因此,首先一件事,就是整顿统一思想才能前进。否则意见分歧,王实味,称王称霸就不能前进。41年王实味在延安挂帅,他出墙报,引得南门外各地的人都去看(野百合花),他是总司令,我们打了败仗,我们承认打了败仗,于是便好去整风。那些延安的人带人想找“韩荆州”找些什么人呢?找抹口红打胭脂的人,找到“前线”中的“客里空”作“韩荆州”,我们则说吴满有、赵占奎、张治国是“韩荆州”,我们说的韩荆州就是工农兵,所以没有整风不能前进。

整风解决了精神问题,生产解决了物质问题,现在可好了。但仍有问题,即党还不是完全统一的。没有民主没有自我批评,不能达到更高的团结,问题解决了又发生,发生了又解决,解决了又发生,就是这样向前发展的,不把分歧的意见统一起来,许多不平均的事情不解决,说不上什么完全的统一。

所以中央以及各个领导机关,要听人家的话,若不听人家的好比不开窗房外空气不进来,是谁的错误?我们开的是政治窗房,窗房一开,空气就会源源而来,就是知无不言,言无不尽,言者无罪,闻者足戒,有则改之、无则加勉。这个道理有些人知道了,就是不肯兑现一句话,党性――普遍性,个性――差别性。(补:了解马克思主义不同一个人幼年同老年不同,抹杀这些不同,统一在一个轨道上,不行;)要发展个人的长处,发展个性,太阳是有轨道的,但八大行星各有不同,不知有人否,没有调查研究,不知道有。天上的星也打闹独立性的。晚上常看到一个星脱落而去。抹杀差别性,就没有统一性。

“每个人的自由发展,是一切人的自由发展的条件”。

(2)党内有几部分干部常感觉不平,不平道,我们对这些同志特别注意。

①理论工作同志:整风讲实事求是,反对教条主义,整风中知识分子有批评,好像他们就不那样吃得开,没有理论就没有行动,因此,党内应该学习理论。党内理论水平不高,几年来,有些进步,但要说明一个运动的各个侧面,内部联系,总结起来提高到理论,还是较差的。什么叫理论?即以马克思主义为基础的系统知识,把斗争经验系统化,而成为理论。

翻译工作的同志不要以为翻译工作不好,还是大大翻译,党内能看原本书的人很少,马、恩、列、斯,英美马克思主义的一些东西翻译出来供同志们看,很好。我们要重视理论工作者,要把他们看作有贡献的人,要尊重他们。

②知识分子,一个阶级要胜利,没有知识分子是不能的。有高级知识分子,有普通知识分子。当然没有其他人也不行。三国演义各国都有知识分子,第八卦衣都是。水浒传中也有。任何一个阶级都有为那一个阶级服务的知识分子,希腊有奴隶的圣人苏格拉底。中国则有周公,以后又有刘伯温等等,无产阶级要翻身,没有知识分子不行。因为整风,审干,把知识分子压低了一些。现在要弄平一些。欢迎他们为党为人民利益而奋斗,我们的党、军队,经济各部门都要有知识分子。

③大后方沦陷区来的同志,以为根据地吃得开,他们吃不开,审干审了他们一下,找他们麻烦,是西安、上海来的,对他们看了又看。现在弄清楚了,对不起打错了帽子的,摘下来,恭恭敬敬的敬了一个礼,他们对根据地的作风看不惯,慢慢就会习惯了。

④关于本地干部。

本地军事干部,报告上写了要如兄弟姐妹一样的亲爱他们,初来陕北,有些人讲闲话,说陕北人能创造苏区,不能带红军。我说讲很好,华中华北是否有此情况,有则改之,无则加勉,这条很重要。历史有许多起纠纷都是从这里来的,王震、戴秀英出发时,我们都谈过这些问题,现在要大讲。

我看人是不完全的,共产党也是不完全的,叫“带有缺点的布尔什维克”。你们不能说不是布尔什维克,多走了两万五千里。(有些同志背了一个包袱,变成了一个驼背),人家没有走二万五千里,可是有根据地,你走了二万五千里,但没有根据地,若是那些同志要我赔中央苏区的根据地我还赔不起。越搞包袱越大。青年同志也有包袱,青年人眼明手快轻视老同志,把老同志叫做庸俗老朽了,老同志把青年人叫做年幼无知,老幼都不应这样讲,种地不如吴满有,做工不如赵占奎,当兵不知张治国。到一地方要用共产主义精神办事,要与该地人民打成一片,要认清山头,照顾山头,缩小山头、消灭山头。凡是用我们的手放下的石头,包袱,要给取下来。自己的手放上去,要用自己的取下来,是人家放的帮助人家取消,使每个同志愉快起来,这样党就团结了,也能团结全党了。

每到一个地方,一定要看人家根据地军队工作,都是好的。到一个地方鞠躬尽瘁,死而后已,不要当钦差大臣,要看到人家的长处,不要看不起人家,去掉盲目性,来个自觉性,要尊重每个地方的同志,与当地方同志搞好,与地方人民队伍搞好关系是每个共产党员的义务。

⑤经济工作,后勤工作干部。因为过去对这些工作宣传解释不够,有人告诉我说,总务工作不受欢迎,总务处长见人不论他是作的总务工作而说是“一般工作”吃不开,现在要让他们吃得开,他们大有杂牌军之感,都是“中央军”哪有“杂牌军”。

⑥民运工作干部,其中有工、农、妇。大城市打开之后,广大的工人运动、青年运动、妇女运动,都要来的。我们是工人政党,历史上做工人运动的同志不多,留下来的也不多,要珍惜他们,青年重要,没有妇女也不行。

⑦抗战时期入党的干部。大革命内战时期的干部,富有经验,领导有方,没有他们是不行的。对!但前面两时期的干部,至多还有两万,抗战后有一百一十九万。老干部包袱大了,(又照漂亮得很)我们有我们的长处,没有我们不行。但眼睛要看到一百多万,不要看不起他们,不要叫新干部,要说是抗战时期入党的干部,要让他们说话,有什么话说什么话,想怎样说就怎样说,使他们觉得容易与我们接近,不要他们觉得不易接近。

⑧党外干部,是一个大问题,全国就是发展到四百五十万党员,也不过只占全国人口的百分之一。每个党员的任务就是要去团结百分之九十九。这就组成一个军队。如果不会团结,就不是一个好党员,为什么要共产党员呢?难道是为世界上房子太多,小米太多?其所以要共产党员,就是要团结九十九个以上,好打敌人,建设新中国。如果不是这样,革命就不能胜利。有些人则专门革财政厅之命。

只有团结了百分之九十九的人民,革命才能胜利。一个党员是毫无办法的,只有共产党员是不会成功的,党外干部,组织部门各地要有调查研究,开座谈会等等,了解他们,培养他们的领袖和干部。

(3)说真话:

谦虚、谨慎、不骄、不躁。

“不偷、不装、不吹”。

什么叫偷?我看见过这样的事,写一本小册子,抄整风文件,改几个字,叫做“抄袭”,应该那个同志讲的,就是那个同志讲的。


什么叫装?“知之为之者,不知为不知,是知也”,懂得就懂得,不懂得就不懂得。

偷与装也是一种现象,不是从客观实际出发。

猪鼻子插葱,装象。

我们党要允许一种状态,不懂,懂得少,不要紧。我们提倡读五本书,“联共历史”、“两个策略”、“左派幼稚病”、“从空想到科学的社会主义”、“共产党宣言”。

第三,不要吹,要“实报实销”。对上级作报告,不要夸大,情报要真实,缺点要向人家公开,扫尽生物。人家骂你官僚主义,罪该应得。牛也有官僚主义,牛奶都有喝的权利,但是要人家去挤,鲁迅也讲过这话,从前也有人说过,写文章不如挑大粪。牛奶双方来。矛盾统一,天下太平,那么我们这个吹的作风,确实要老老实实(改正)。

我们的党是很大的党,是二十五年的党,我们准备胜利,我们去迎接胜利,为争取全国人民解放而奋斗!

补:普遍性是建筑在差别性之上,没有党员那有党?没有差别性那有普遍性?各种工作人员(工农兵、苏区、白区……)都有差别性。

