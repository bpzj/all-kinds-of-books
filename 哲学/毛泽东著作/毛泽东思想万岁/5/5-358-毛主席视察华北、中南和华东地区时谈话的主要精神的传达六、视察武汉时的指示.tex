\section[毛主席视察华北、中南和华东地区时谈话的主要精神的传达六、视察武汉时的指示]{毛主席视察华北、中南和华东地区时谈话的主要精神的传达六、视察武汉时的指示}


我们最敬爱的伟大领袖、我们心中最红最红的红太阳毛主席在汉视察了两天三夜。主席在湖南视察了一天,坐火车来,也是坐火车走。毛主席巡视了武汉市容,两次都是白天,我们出于对伟大领袖毛主席的无限敬仰、无限关心,劝他老人家下半夜巡视,主席说不行,要白天看市容,而且是坐吉普车,主席巡视了市容和街上的大字报,感到很满意。回住处跟我们谈到深夜二点。春桥同志说:主席该休息了。我们在他老人家讲话时不插话,仔细地听,那里有这么好的机会啊!在讲话中,主席经常看语录,讲语录,边讲边笑,一直很高兴,精神焕发。主席讲了很多党的历史,从党的历史联想到文化大革命中的问题,讲中国历史,外国历史,耐心教育我们。

  毛主席在汉接见了武汉部队负责人曾思玉、刘丰和武汉警备区负责人方铭、张纯青等四同志,二十三日刘丰同志送主席到郑州返汉,可惜街上大批判专栏被涂坏了,我们很难过。

根据我们回忆,主席讲话精神如下。

(一)武汉文化大革命形势空前大好,湖北的问题基本解决了。武汉出了一个“七.二〇”事件,好比脓包穿了头,坏事变成了好事,问题就是不破不立,乱透了就会好的,党内走资派搞的一些鬼,看得很清楚。他们被彻底打倒了,他们无权了。主席说:湖北的问题基本解决了,形势好得很,乱透了就好,好解决问题。主席问我们:湖北第二批建立革委会行不行?形势大好啊,这个机会错过了,那以后就赶不上。还问我们:今冬明春的工作安排了没有?明年元月、五月,八月准备搞些什么?  (我们想 “三结合”基本具备了条件,七大造反派组织是现成的,军队是现成的,地方干部如何?我们心里还没把握。) 主席问:“你们看美帝敢不敢打?”主席很乐观地说: “我看不行,它不敢打,它连一个越南也对付不了。南越最初只有九条枪,后来越南把美国搞得骑虎难下。美帝没有什么了不起”。我们的文化大革命取得了伟大的决定性的胜利,今年又风调雨顺,主席很高兴。主席对武汉寄有很大希望,我们想,还要作艰苦工作,才能由“基本解决”走到“完全解决”,现在离建立革委会还有一段路程。  (张春桥同志插话:你们千万不能打乱毛主席的战略部署)。

毛主席他老人家高瞻远瞩,想的很远,我们湖北一定紧紧跟上毛主席战略部署。同志们考虑一下,过了国庆节先把工代会搞起来,再结合一两个革命领导干部,十月份建立革筹小组,尽快实现革命大联合,争取在元旦开个会,腊月十五再开个会,春节放三天假,大家好好过个年。

(二)要解放一大批干部。

主席又从历史上讲起,从中央苏区讲到延安整风。主席说:对犯错误的干部,要允许人家改正错误,要给人家改正错误的机会。打击面要缩小,教育面要扩大。古田会议前,教条主义者还整我,陈总对军队问题解决不了,以后陈总又把我接到军队来。在古田会议上,讲了十条,从正面批评了那些“主义”。以后,又排斥我,说“山上没有马列主义”。朱德、李德(一个洋人)不听意见,打了败仗,后来才长征。在遵义会议上,洋人投了我一票,我才达到四比三。所以,核心不能自封。即使王明这样对待我,我还是主张选他当中央委员。

主席讲,八一南昌起义是首先向国民党开的第一枪,所以是革命行动,不能取消“八一”。贺龙是个土匪,南昌失败,实际不会打仗,贺山头指靠苏联,但八一起义还是革命行动。前几次上天安门都是我批的,你们打你们的,我批我的。我这个人曾经五次被人家排斥。后来又要请我回来,南昌失败后,干部不多了,但都成了骨干。什么叫长征?长征是打出来的,是逼出来的,长征之后,质量就高了,就是现在这些干部,哪个在战场上没受伤?在民主革命中还是立了不少功,虽然在文化革命中犯了错误,只要改正就行了。老干部到打起仗来,我的命令一下,他们上战场还是很勇敢的。不能怀疑一切,打倒一切;怀疑一切,打倒一切,不好,这对革命事业,对党不利。

主席讲:听说对“百万雄师”、独立师揪得太厉害了,那样搞会逼上梁山的。独立师一万多人,上街的有二千多人,连人家的家属也倒霉,这样不好,搞得太凶就会脱离群众。(主席很注意看小报,每个小报都看了。)主席对曾思玉同志说:“听说你来时,不是有人要打倒你?”曾答:“打倒没有关系”。刘丰说:今天(九月二十八日)有人写信给我,要为“百万雄师”翻案,说不翻案就没有好下场。署名“狂人”。主席说;坏人总是少数,好人总是大多数,广大群众是好的。独立师广大干部和群众都是好的,是受蒙蔽的。你们要多请示。刘、邓就不喜欢请示。我们这样大的国家,一个个自己决定,决定错了怎么办?

主席还指示:湖北、安徽的干部还没有站出来,有的不能当省长了,当副省长也可以。红卫兵小将也可当干部。

(三)要文斗,不要武斗。

主席说,体罚这个做法,我反感。是不是把《湖南农民运动考察报告》里打倒土豪劣绅的那一套办法都搞来了?那是对敌人的,现在时代背景不同了。文化大革命要触及灵魂。凡是爱整人的人,整来整去,最后都要整到自已头上来,“喷气式”是王光美搞的,王光美是资本家的姑娘。

现在全国到处搞武斗,这翻不了天,让那些人跳出来好嘛。

(四)团给一一批评一一团结。

现在有些人破坏了这个优良传统,不喜欢这个公式。解放军是团结、紧张、严肃、活泼八个大字,现在是紧张、严肃有余,团结、活泼不足。  (曾思玉插话。有些人叫什么勤务员,有的是大司令,比我们还大,官一大了,有时就不民主。) 主席说:“你这太急躁了”。

又对刘丰说:“他要是急躁起来,你这个政委就给他泼点冷水。”

主席设;张国焘两次南下,徐向前在这个问题上是好的。许世友是个和尚,但是员战将,他也要打倒?看一个人,要全面的看,历史的看,又要比较的看。要有几个反面教员,没有反面教员,叫独裁。要团结一切可以团结的人们。

(五)革命大联合

毛主席说:“工人阶级分为两大派,我就想不通”。联合不起来,主席早就发觉了。七月十八日主席就在汉指示:“在工人阶级内部,没有根本的利害冲突。在无产阶级专政下的工人阶级内都,更没有理由一定要分裂成为势不两立的两大派组织”。

我们一定要替毛主席争气。武汉点名的七大组织,都在他老人家那里挂了号的,联合的关键在你们,主席对你们各个小报都看了,好坏都有评价,有些小报派性特别强,纸倒特别好。主席指示我们:“多给他们(注:指造反派)讲些历史。过去苏区里面打内战,无非就是一个马克思主义多一点、少一点的问题。”主席非常关心武汉的无产阶级文化大革命,临走时还问我们:“两个月,联不联合得起来?”

曾、刘首长说:无产阶级革命派要很好学习中央首长对北京“天派”、“地派”的讲话。中央对革命小将的批评,武汉造反派要引以为戒。  “三新”不要骄傲、二司骄傲不得,工总也不要以为人数多,老是骄傲就会犯错误,这一段犯了点小错误,是支流,最大也不过是汉水,不是长江。 “五.一六”兵团,我们这里也有,武汉有黑手,中央已经打了招呼。安徽两派联合起来后,黑手就自己揪出来了。七月十六日,主席本来想渡江的,死了那么多人,结果没渡成。有黑手,要警惕。主席讲:“现在是革命小将有可能犯错误的时候了”。我们要防止坏人用极“左”的口号、山头主义搞得我们犯错误。我们的革命小将不要去管工人的事,坐下来好好学习,写文章,多起促进作用。林总从苏联回来时,对毛主席讲,中国一定比苏联强,因为中国有很多优秀儿女。你们革命组织的勤务员一定要做优秀儿女,不要掉队。

毛主席还指示:“干部不要从外地调,可以精简。”接着问张政委:“你的警备区有多少人?”张说:八十多人。毛主席说:“有一十到二十就够了”。

主席参观市容时,正值“三司革联”开红代会,街上很多宣传车,主席对司机说:“你告诉曾司令员,这不好,我很反感”。我们不要搞经济主义。上海张春侨、姚文元同志一直抓工人工作,那里没发生抢枪,大批判专栏办得最好,我们要学习,办好大字报专栏,坚决抵制小报新闻,扫除一切非宣传品,取消小报买卖.曾、刘首长还表示,我们支左,决不支派。又问:“干部怎么结合?张体学这个人行不行?他过去做实际工作多。”

毛主席的整个谈话长达两个半小时,他老人家红光满面,神采奕奕,总是乐哈哈的,一点也不觉得累。毛主席身体这样健康。是全中国人民、全世界人民的最大幸福。毛主席为我们撑腰,我们一定为毛主席争气。

(曾、刘首长传达)

毛主席在汉指示:

目前的主要任务就是大批判,大斗争,大联合,尽快实现三结合。希望你们成为对党内走资派斗争的模范;希望你们成为大联合的模范;希望你们成为反对小团体主义、经济主义、自私自利的模范。中国历次革命,以我经历看来,真正有希望的是想问题的人,不是出风头的人,现在大吵大闹的人,一定会成为历史上昙花一现的人物。

