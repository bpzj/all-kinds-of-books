\section[关于划阶级问题的指示(时间不详)]{关于划阶级问题的指示}
\datesubtitle{(时间不详)}


划阶级有必要。坏人虽是少数,但他们占居了要害部门,当了权,这不得了。……阶级成分和本人表现要区别,主要是本人表现。划阶级主要是把坏分子清出来。

阶级出身和本人表现,也要加以区别,重在表现,唯成分论是不对的,问题是你站在原来出身的那个阶级立场上,还是站在改变了的阶级立场上,即站在工人、贫下中农方面?又不能搞宗派主义,又要团结大多数,连地主、富农中的一部分人,也要团结,地富子弟要团结,有些反革命分子、破坏分子也要改造,只要愿意改造,就应当要他们,都要嘛。如果只论出身,那么马恩列斯都不行。例如马克思也要先学唯心论,后来才学唯物论,才搞出马克思主义。黑格尔与费尔巴哈是他在哲学方面的两个先生。

我们在工厂中划阶级,主要是把那些国民党的书记长,反动军官,逃亡地主,地、富、反、坏分子查出来,像白银厂一样,把那些坏人查出来,并非查所有的人,并非查剥削阶级出身的技术人员,他们过去一些人主要为剥削阶级服务,只要现在表现好就信任,即使表现不大好,也要改造。有的只是剥削阶级出身,那就要看表现好坏。

社会主义理论的产生,只能经过知识分子,把已经存在的阶级斗争现象研究提高为理论,加以宣传,把工人阶级从分散的变成为有组织的,从自发的阶级变成自觉的阶级。工人每天在剥削压迫之下生活工作,为了吃饭,那么忙,自己产生不了马克思主义。马克思本人不是工人,但他能看出发展的趋向,经过分析研究,把资产阶级哲学变成无产阶级哲学,把资产阶级政治经济学变成无产阶级政治经济学,这样来教育工人。其实,工人也读不了那么多书,读不了那么大部头的著作,先进的可能读得多一些。阶级斗争的现象存在了几千年,资产阶级也说是有阶级斗争的,只有马克思、恩格斯才把它理论化,系统化了。要斗倒资产阶级,社会主义是继承了资本主义的,我自己也是先学地主阶级的,六年读孔夫子的,七年读资产阶级的,共计十三年,那时二十几岁,对马克思根本不知道,俄国十月革命以后,才知道马克思,读马克思的书。


