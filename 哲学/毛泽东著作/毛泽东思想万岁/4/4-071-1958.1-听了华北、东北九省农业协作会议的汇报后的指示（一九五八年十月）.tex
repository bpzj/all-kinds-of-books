\section[听了华北、东北九省农业协作会议的汇报后的指示(一九五八年十月)]{听了华北、东北九省农业协作会议的汇报后的指示}
\datesubtitle{(一九五八年十月)}


一、今后要改变广种薄收、务广而荒的办法。现在耕地面积不是少了,而是多了。两亿多劳动力搞饭吃,不像话,要逐步缩小面积,精耕细作,种少种好,少种多收。深耕要逐步作到翻三尺,只有深翻,水、肥才能充分发挥作用。以后单位面积产量搞到万斤,每人二分地就可以了。

二、有些社只搞粮食、薯类,没有可以交换的经济作物,工资发不出去,不好。以后要多搞能交换的经济作物。明年起,所有公社,又要搞粮食,又要搞能交换的经济作物,如畜牧、鱼、药材等。

三、交换问题,交换不能轻视,有些人过早的卑视交换是不对的。交换是永远的。一万年之后还有交换。一个公社不可能“万事不求人”。目前不能卑视交换。卑视商品生产,对当前经济发展是不利的。

四、两个出路。劳动力很紧张,这是个大问题。出路一个是改变广种薄收。少种多收,可以省工省水、省肥等。一是机械化,目前是抓工具改革。将来达到一半劳动力搞工业,这样我们的国家就像个样子了。

劳动组织分工,要适当固定起来。工业、农业劳动不固定起来不好。

五、公社分配问题。公社化后,分配主要还是按劳取酬。供给制部分搞的不要太多了。供给和工资部分是否一半一半。工资要保持一定差额,级差不能太小,否则,不合按劳取酬原则,有百分之二十五的人,要减少收入,并且都是劳动力多的.对生产不利。干部级差也不能太小,虽然太大也不好。

六、公社由集体所有制过渡到全民所有制,时间不会太短。北戴河关于公社的决议上写快的三、四年,慢的四、五年,我加上了或更长一些时间。鞍钢一个工人一年生产一万八千元.成本一万元,工资八百元,积累七千二百元。一个农民一年生产不过七百元,二十五个农民顶一个工人的产值。鞍钢产品可以全国调拨,农业上可以调拨的就不多。所以十二年能过渡了就不错。

七、不论搞什么工作,像搞人民公社,不能走马观花,要搞透一个公社,解剖一个麻雀,就可以有把握。学马列主义也是一样,书看得很多,不透,不好。我最近看参考资料,很仔细。每次看几遍,多看一遍.就多一些收获。

八、明年要大搞农业生产,省的第一书记要一手抓工业,一手抓农业。县的第一书记,除少数搞工业的重点县外,都要抓农业。

九、可不可以不走化肥的道路(今年化肥生产一百八十万吨,明年只生产六十到七十万吨,进口化肥也要减少了。)今年没有化肥,粮食搞到八千亿斤,棉花搞到八千万担,证明可以基本不靠无机化肥,主要靠有机肥料和土化肥。

可不可以不走拖拉机的道路,走绳索牵引机实现机械化、电气化的道路。


