\section[在八县查田运动大会上的报告(一九三三年六月十四日)]{在八县查田运动大会上的报告}
\datesubtitle{(一九三三年六月十四日)}


“一切过去的经验都证明:……(见“查田运动是广大区域内的中心重大任务”一文所述的这一段的开头至建立革命临时政权相同。故略)……建立临时的革命政权(革命委员会)建立地方的工农武装及群众团体,才能保障阶级路线的正确执行,才能达到消灭封建残余势力的目的;一切脱离群众的官僚主义、命令主义工作方式,是查田运动最大的敌人。查田运动的群众工作,主要是讲阶级,通过阶级,没收分配及对工会、贫农团的正确领导等。

一、讲阶级(做宣传)

(一)查田运动的策略,是以工人为领导,依靠贫农,联合中农,削弱富农,消灭地主,宣传内容,就是要向群众很清楚地说明这个策略。

(二)为了说明这个策略起见,要将什么是地主,什么是富农,什么是中农,什么是贫农,什么是工人说清楚。在这个说明中证明地主是封建剥削者,富农是封建剥削者,因此只有采取消灭地主削弱富农的政策,才能便土地革命的利益,完全落到中农、贫农、工人身上。

(三)但是富农与地主有分别,富农自己劳动,地主自己不劳动,所以对地主取消灭的政策,对富农则取削弱的政策,因此消灭富农的倾向是错误的,同时不应该把富农成分当作地主待遇。

(四)对中农的策略――联合中农.是土地革命的中心策略。中农的向背,关系土地革命的成败。所以要反复向群众说明这个策略,说明侵犯中农利益是绝对不许可的。为了联合中农不侵犯中农利益起见,要提出“富裕中农”来说他,要着重说明富农与中农交界地方,使富裕中农稳定起来。要揭破地主富农对中农的欺骗,使中农脱离地主富农的影响,团结在贫农周围,一致向地生作斗争。

(五)要揭破地主富农的每一欺骗口号,向群众作广泛的解释。要注意地主富农利用民族的、地方的落后观念,来对中农贫农进行欺骗破坏。

(六)要说明查田运动是粉碎敌人围剿的武器。因为把封建残余势力肃清了,去掉了敌人藏在苏区的捣乱者;广大群众革命热忱提高了,扩大红军、经济建设等项就更加好做。

(七)要按照当地环境提出具体口号,如在落后的村子,要找出落后的原因,提出发动群众的具体口号。比如该地有反动地主威胁群众,因此群众不敢起来斗争,必须提出提起这个地主的口号;又如当地政府人员犯了脱离群众的严重错误,使得群众不满意,就要从揭发这些人员的错误着手宣传,燃发动群众斗争。

(八)以上是宣传的内容,以下再说宣传的方式。查田运动的对象,第一要向乡一级干部分子(乡苏代表、各群众团体负责人)讲话,使他们首先明了,经过他们宣传群众。第二要在工会与贫农团会议上讲话,使会员群众明了。第三,要在村子群众大会上讲话,使每个群众都明了。

(九)宣传员,主要由乡一级积极干部分子担当,有布置的出席群众会议上去讲话。其次是组织宣传队,经过训练后派往群众中讲话。

(十)宣传的方法:第一,口头讲话。第二,贴布告。第三,写标语。第四,出传单。第五、演新剧。第六,墙报上做文章等等。

二、查阶级

(一)查田运动是查阶级,不是按亩查田。按亩田要引起群众恐慌,是绝对错误的。

(二)查阶级是查地主富农阶级,查剥削者,查他们隐藏在农民中间而实在不是农民的人,查这些少数人。决不是查中农、贫农、工人阶级,因此不得挨家挨户去查。挨家挨户去查要引起群众恐慌是绝对错误的。

(三)查阶级前,一定要经过宣传的阶段,且以讲阶级的阶段。不经过公开的普遍的讲阶级阶段就动手去查,要引起群众恐慌是绝对错误的。

(四)查阶级要发动工会、贫农团的会员及其他群众多数去查,要群众查了随时报告贫农团与查田委员会。不应该只是少数人去查。少数人去查要引起群众恐慌,是绝对错误的。

(五)查阶级要查明白,不论是一个地主,一个富农,要把他们过去的剥削情形和生活情形查得明明白白,才能使本人服罪,使群众满意;如只查了大概就下断语,那就容易弄错,本人不服,群众不满意,就阻碍了查田运动。对那些中农的成分,更要仔细查清,不使中农弄成富农,富农弄成地主。

(六)一家阶级查出来的材料,要填在一张“阶级成分调查表”上,使大家好评论,日后好根究。如只讲口里记在心上,或只在簿上写两三行,易于遗失,是不好的。

三、通过阶级

(一)通过阶级就是决定阶级成分,是对这个人决定问题的时候,故要十分慎重。一定要查清楚了的,才能提出通过。

(二)通过的次序,首先是贫农团,要开贫农团大会。经过大家讨论,大多数人举手赞成,才算在贫农团通过了。如有疑问的移到下次讨论,此次不要通过。

(三)第二是乡查田委员会,对于贫农团的意见加以审查,对的通过,不对的改正。怀疑的再调查。

(四)第三是区土地部,区土地部决定不下的,提出于区查田委员会。区查田委员会决定不下的,提出县土地部。

(五)第四是村子群众会。一定要在本人村子里召集群众大会,向群众报告本人的剥削情形与生活情形,看群众赞成不赞成,赞成的通过,不赞成的再去调查,决不可硬要通过。如果硬要通过,就会引起群众不满,这就是命令主义,要坚决反对的。

(六)以上通过阶级的四个步骤必不可少,特别是群众会通过更加重要。许多地方,不经过群众会通过就去没收,是错误的。

(七)如果过去错误了的,加把中农当富农,富农当地主,地主当富农,应该推翻原案。要在群众大会上说明过去错了,现在改正的理由取得群众的满意。

(八)过去弄错了现在翻过来的,如是中农一定要赔他的土地财产,即使田分了也要抽出赔他。如是富农,现在有则赔,如果实在没有赔的,只好将来替他想别种办法。这种赔还,最能争取群众,如果将错就错,不肯改正,那是完全不对的。

四、没收分配

(一)没收地主土地财产,没收富农土地及多余的耕牛农具房屋,只有经过村子群众大会得到群众的同意,才能实行,决不可能不得群众同意就去没收,决不可黑夜去没收。

(二)没收了地主的财产,除开现款及宝贵文件交政府财政部外,其他一切东西,都应分发群众,这是提高群众斗争热情的好办法。

(三)趁着开村子群众会通过阶级的时候,举出临时的没收分配委员会,即刻没收,当场分配,不得迟延没收,不得迟延分配,不得把东西挑到政府里去再讲分配。

(四)要分配给本村子,不可全乡平推(大地主不在此例)

(五)首先要分配给红军家族、雇农、工人及其他群众的贫苦者,不可不分阶层同等分配。
(六)猪鸡等物,煮起来在群众大会上使大家吃,不可工作人员少数人吃。

(七)政府工作人员最好不要求东西,以作模范。如果十分缺乏用物,要取得群众的同意,在群众大会上通过,要防止工作人员自由的拿东西。

(八)没收的耕牛及重要农具,在群众同意下,由分得的人组织犁牛合作社,共同使用。

(九)没收来的土地,迟延不分是不对的,除开留出红军公田及公义事业田外要迅速分配。首先,分给过去那些未分田及少分田的人。再有多时,以村为单位大家平分。山林、渔塘、房屋、茅厕,同样要迅速分配给群众。

(十)在每次分配东西,群众斗争热情最高涨时,要适时的提出扩大红军、发展合作社等口号,领导群众热烈参加革命战争,热烈参加苏维埃建设。

五、工会贫农团

(一)所有讲阶级、查阶级、通过阶级、没收分配等许多工作,都只有动员工会、动员贫农团才能收得最大的效果,工会应成为农村阶级斗争的领导者,贫农团是农村阶级斗争的柱石。

(二)正确地开展查田运动,依靠于工会指导自己的会员加入贫农团,在贫农团内部起积极的作用。

(三)要发展贫农团,使贫农团成为广大贫农群众自由愿入的团体。

(四)要洗刷贫农团中暗藏的坏分子。

(五)贫农团在查田运动中要勤快开会,要抓紧查田运动为自己的中心工作。

(六)要团结中农在贫农团的周围,贫农团的会议应该吸收中农来旁听。

(七)工会贫农团领导查田斗争工作,不能违背前面各段所述的原则。

只有依照上面所述的策略与方法,来动员广大的群众,才能使查田运动得到完满的成功。一切不宣传或宣传不正确,不认真、不普遍,查阶级,通过阶级与没收分配,不按阶级路线与群众路线,不得群众赞助与同意,都不能使查田运动收到成绩,反会使群众不满,阻碍查田运动的进行,因此,反对查田运动中的侵犯中农,消灭富农的左倾机会主义,反对包庇地主富农的右倾机会主义,反对官僚主义的领导方法与工作方法,是正确开展查田运动的必要条件。

<p align="right">(“论查田运动”)</p>

