\section[给各野战军首长的指示(一九四七年八月二十八日)]{给各野战军首长的指示}
\datesubtitle{(一九四七年八月二十八日)}


在目前情况,给敌以歼灭与给敌以歼灭性打击,必须同时注意,给敌以歼灭是说将敌整旅整师干净全部地加以歼灭不使漏网,执行这一方针,必须集中三倍或四倍于敌之兵力,以一部打敌正面,另一部包围敌之两翼,而以主力或重要一部迂回敌之四方,即是说四面包围敌军,方能奏效,这是我军作战的基本观点,这是在敌军分散孤立,敌援兵不能迅速到达之条件下必须实行的正确方针。但在敌军分数路向我前进,每路相距不远,或分数路在我军前进方向实行防堵,每路亦相距不远之条件下,我军应当采取给敌以歼灭性打击的方针,这即是说,不要四面包围,只要两面或三面包围,而以我之全力用于敌之正面及其一翼或两翼,不以全部歼敌军为目标,而以歼灭其一部,击溃其另一部为目标。这样做,可以减少我军伤亡,其被歼之部分可以补充我军,其被击溃之部分可以使其大量逃散,敌能收容者不过一部分,短期内也难恢复战斗力。……

