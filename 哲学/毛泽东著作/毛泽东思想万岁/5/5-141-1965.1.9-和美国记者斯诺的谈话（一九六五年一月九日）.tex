\section[和美国记者斯诺的谈话(一九六五年一月九日)]{和美国记者斯诺的谈话}
\datesubtitle{(一九六五年一月九日)}


\textbf{翻印者的说明:} {\kaishu 此文是毛主席接见美国记者斯诺后斯诺写的,题为《毛泽东主席会见记》,一九六五年二月四日起,日本《朝日新闻》予以连载并加了按语。现将按语摘录及斯诺的文章录如下,谨供读者参考。}

\textbf{《朝日新闻》按语摘录:}

{\kaishu 
由于中国实行严峻的共产主义,自力更生政策,试验原子弹,全世界的视听都集中于这个“强大的中国”的登台。看来,这个中国,似乎将给激烈变化的世界历史带来新的激烈的动荡,然而,中国的内部还有许多人们所不知道的事情,因为北京的领导人,从来不向人们谈一切,也不给人们看一切。

一月九日斯诺同毛泽东主席谈了长达四小时,这是异乎寻常的单独会见。

斯诺说:毛主席允许外国记者发表会谈内容,这是最近几年来还是第一次。毛主席在这篇谈话中谈了,他对当前问题的重要看法:(1)越南民族解放阵线将会依靠自己的力量获得眭利;(2)中国本身就像一个大联合国,即使现在不进联合国,也可以很好地生活下去;(3)核战争是不好的。应当用人民的力量把原子弹变成真正的“纸老虎”;(4)美中两国总有一天重新携起手来。毛主席还很随便地从他幼年时期的宗教观谈起,一直谈到自己的命运,即“不久要去见上帝”。他话题广泛,纵谈阔论。

以下是会见记的全文。}\marginpar{\footnotesize 212}

\subsection{“感谢”外国的干涉相信越共会取得胜利}

中国共产党毛泽东主席很少答应同别人会见。但是,他在同我大约四个小时的会见中,亲切地谈了许多事情。用他自己的说法,便是“山南海北”,无所不谈。一九六四年,(中国的)粮食产量达到两亿吨。由于丰收,粮库堆得满满的。无论到哪一家商店,都可以看到很多廉价的食品和消费物资。技术和科学的进步的成就,以庆祝苏联前总理赫鲁晓夫下台进行核爆炸的形式显示了出来。对于毛主席来说,现在是可以引这些成就而自豪。然而,他却说到他将同死亡进行斗争的问题,而且还表示,他的政治遗产可以让后世来做出评价。

今年七十一岁的这位斗士,在隔着一个广场,在天安门对过的人民大会堂的一间有着北京式摆设的大厅里接见了我。他在谈话中间不只一次地说,感谢外国侵略者,因为它们促进了中国革命,而且现在,在东南亚,他们又赐与了同样的恩惠,他还说,中国没有向自己领土以外的地方派出军队;只要自己国家不受到进攻,就无意同别人打仗,他还谈到,美国向西贡送的武器和军队越多,南越解放军就会越快地武装自己,教育自己,从而取得胜利。他说,南越解放军已经不需要中国军队的援助。

\subsection{长时期的畅谈}

在谈话开始以前,毛主席从容不迫地坐下来,同意给他拍电影。外国给他拍电视片,这是第一次。最近风传他的健康状况已经相当恶化,但是,政治的临床医生们,可以根据这部电视片作出自己的诊断。一月九日的这次会见,是在他非常繁忙的几个星期之后进行的。在这几个星期中间,他同每年为了出席全国人民代表大会而聚集北京的许多的地方领导人连天连夜进行了交谈,如果他是一个病人,他就会更快地结束和我的谈话,我们的会见从下午六点前开始,我们边谈,边进晚餐,然而又谈了将近两个小时,但是,他使我感到,直到最后还完全从容不迫,毫无倦意。

主席的一位医生告诉我,他的身体没有不好的地方,只是因为年龄关系,容易疲倦,他同我一起吃了很辣的湖南菜,他吃得不多,他还像过去一样,简单地饮了一两杯葡萄酒,然而,如同我将在后面所叙述的那样,现在已经准备“同上帝见面”,这不能不使人进行一些惴测。

关于我的这次会见,外国(通讯社)报道说“其他政府官员”也出席了,这些官员是我在革命前的中国居住时就认识的两位朋友。一位是现任中国外交部长助理龚澎夫人,另一位是龚澎的丈夫、外交部副部长乔冠华先生,我事前没有用书面提出问题,会见时也没有记录,但是,后来,我同一位做了部分纪录的出席者回顾了这次谈话,加深了记忆,这是很幸运的,我们还谈定,在我写文章时,不直接引用主席的话,而可以像下面那样,由我详细地转述他的看法。

\subsection{百分之九十五的人支持社会主义}

我首先说,中国自我上次见到他以后度过了艰难的岁月,到了一九六五年,\marginpar{\footnotesize 213}已经发展到惊人的高度。一九六〇年,他对我说过,百分之九十的人民支持政府,只有百分之十的人反对。那么,现在情况又是怎样呢?

毛主席回答说,蒋介石集团的分子,现在还有一些,但是不多了,很多人的思想都得到了改造。这个数字,今后还有可能增加,而且,这些分子的子弟也是可以教育的。总之,可以说,今天大约占人民的百分之九十五的人或者更多的人一致拥护社会主义。

(在这里人们很自然地会想起班禅喇嘛,不久前,班禅喇嘛被解除了西藏自治区筹备委员会代理主任委员的职务)

“关于班禅喇嘛的问题,是否是因为他们企图统治农奴的喇嘛地主势力保持着传统的联系?还是因为作为宗教领导人的任务同由寺院分离出来的新的政治权力之间发生了冲突”?

毛主席回答说,从根本上说,这不是一个宗教自由的问题,而是一个由土地问题引起的问题。封建统治者失掉了土地,他们的农奴获得了解放,现在成了主人翁。班禅喇嘛不仅阻挠这个变化,而且同结党的那些过去的特权阶层的“无赖之徒”不断地来往,他们手里有一些武器,但是他们当中有一个人暴露了计划,班禅喇嘛周围有几个人,年龄还不是大到不能改造的程度,因此还有希望。班禅喇嘛自己也许会改造他的思想,他现在还是人民政治协商会议的委员。现在,他在北京。如果他愿意回去,随时都可以回到拉萨。这要他自己来决定。

至于作为宗教的喇嘛教,谁也没有压迫真正的信徒,所有的寺院都开放着,而且还可以拜佛,只是问题是活佛们未必一定都实行了佛的教义,他们对宗教以外的事,并不是不关心的,达赖喇嘛自己就对毛主席说过,他不相信自己是活佛。但是,如果有人公开这样说,恐怕他就不得不否定他的话,很多基督教的牧师和祭司并不完全相信自己的说教,但是,他的信徒却有很多人是真正的信徒。有人说,毛主席本人从来没有迷信过,这是不对的。他的母亲是一位热心的佛教徒,总是不忘记拜佛。毛主席在小时候,同母亲一起,曾经反对过不信佛教的父亲。但是,有一天,他父亲在寂静的森林中行走,遇到了一只老虎——是真老虎,而不是纸老虎,于是,他父亲跑回家来,给神上了供,很多人不都是这样吗?当他们遇到困难时,就去祷告神,没有事时就忘记了。

毛主席还记得,三十年前在第一次国共内战快要结束时,在中国的西北部第一次同我见面时,向我谈过他父亲和老虎的故事。

那个时候的条件很坏,中国红军在数量上是少数,但团结却很坚强。那时,他们只有轻武器……。

我插嘴说:“贫民的军队还扛着红缨枪”,他说是的,有红缨枪,还有扫地柄。这样看来,决定斗争和最后的胜负的,不能光靠武器。

我说:“当时人们主要只想到从日本人的手里解放中国。”他说:“中国的国际地位的提高,具有这样大的意义,当时我还不能预想到呢!”

\subsection{战乱使人民提高了政治觉悟}

主席在这里回忆了这样的情况:一九三七年同蒋介石合作并且签订了同日本作战的协定以后,他的军队仍然避免同敌人的主力进行战斗,而把重点放在在农民中间建设打游击战的根据地,日本军队实在是起了作用。由于日本军占领中国各地,焚烧村庄,教育了人民,促使人民很快地提高了政治觉悟。日本军队为共产党人率领的游击队发展队员、扩大根据地创造了有利条件。\marginpar{\footnotesize 214}现在,当见到毛主席的日本人向他赔罪时,他反过来感谢日本人说,那应当归功于日本的帮助。

后来,在内战时期,美国政府通过援助蒋介石,也帮了忙,蒋介石总统经常是他们的教员。如果没有蒋介石的教导,毛主席自己也不可能排除右倾机会主义者,也不会拿起武器同他进行的战斗。坦率地说,直接教导他们必须战斗的是蒋介石,而美国是间接的教员。日本将军们也是直接的教员。

“西贡的美国评论员当中,有人把南越的民族解放阵线的力量同中国人民解放军大规模消灭国民政府军的一九四七年时的中国的形势作比较。这两方面的条件,是不是比较相似。”

主席并不这样认为。一九四七年,人民解放军已经有了一百多万军队,它同蒋介石拥有的几百万军队相对抗。当时,人民解放军调动师和几个营的兵力,但是现在,越南解放军以营,顶多以团为单位采取行动。驻在越南的美国军队还不多。如果美军增强兵力,那么,就有利于加速对抗他们的人民武装的成长,然而,这件事,尽管告诉给美国的领导人,他们也不会听。他们听过吴庭艳的意见吗?北越主席胡志明和毛主席都认为吴庭艳并不那样坏,而且认为美国人还会支持他几年。但是,急性子的美国将军们对他不感兴趣了,并且把他抛弃了。尽管如此,在他被暗杀后,天地间难道比以前更太平了些吗?

“越共的军队,仅仅依靠他们自己的力量,能够取得胜利吗?”

毛主席同答说,能够。

越共的处境比第一次国共内战(一九二七——一九三七年)当时的共产党方面要好得多。当时的中国,虽然没有外国的直接干涉,但是越共方面已经受到外国的直接干预,从而有助于武装和教育自己的一般士兵和军官。反对美国的,不仅限于解放军。吴庭艳过去也不想服从(美国的)命令,像这样闹独立性的情况,现在甚至已经扩大到(南越)政府军的将军当中。美国教员当得成功。

我问,这些将军当中,有人将会参加(南越的)解放军吗?毛主席说,有人可能仿效跑到共产党方面来的国民政府军的将军的做法。

\subsection{革命从压迫中产生核弹,归根结蒂是“纸老虎”}

“对美国干涉越南、刚果以及以前是殖民地而目前成为战场的各国这一事实,从马克思主义的观点来看,可以从理论上提出很有趣的问题。这个问题就是,在法国人喜欢说的“第三世界”——即亚洲、非洲、拉丁美洲的所谓不发达国家和由殖民地国家以及目前仍为殖民地的国家表现出来的新殖民主义和革命力量之间的矛盾,是不是今天世界上最大的政治矛盾?或者认为基本矛盾仍然是资本主义国家相互之间的矛盾呢?”

毛主席回答说,现在关于这个问题还没有把意见整理出来,但是,他说,他想起前总统肯尼廸讲过的话。

\subsection{帝国主义者相互之间的矛盾}

肯尼廸不是曾经断言说,美国同加拿大、西欧之间,没有什么了不起的、实际的、根本性质的对立吗?已故总统还曾说,问题在于南半球。已故总统在提倡进行‘特种部队战争’训练\marginpar{\footnotesize 215},以准备进行‘局部战争’(防止破坏活动的战斗?)时,也许曾经考虑过这个问题。

但是,过去两次世界大战的根源,是帝国主义者相互间的矛盾本身。尽管它们因为殖民地(人民)的革命而耗费了很大的力气,但是,他们的本性并没有改变,以法国为例,戴高乐总统的政策,看来有两个理由。

第一,摆脱美国的控制,主张独立;第二,设法使法国的政策适应亚洲、非洲、拉丁美洲正在发生的变化。结果,资本主义各国之间的矛盾扩大了。然而,法国是它所谓的‘第三世界’的一部分吗?

最近,(毛主席)问过几个来访的法国人,他们的回答是否定的。他们说,法国是已经发展完毕的国家,不可能成为不发达国家的‘第三世界’的一员,问题好像并不那样简单“也许可以说,法国虽然在‘第三世界’里面,但是,不属于‘第三世界’?”

毛主席回答说,也许是那样。曾经读过:前总统肯尼迪由于对这个问题很感兴趣,研究过毛泽东写的关于军事作战的文章,阿尔及利亚的朋友们在同法国进行战斗时,从阿尔及利亚的朋友那里听说过,法国军队也读过这些文章,并且曾经拿这些知识来对付他们。

但是,他对当时的阿尔及利亚总理阿巴斯说,这些书是根据中国的经验而写的,因此,反过来用是没有效果的。只有人民在进行解放战争的时候,才能运用它。对于反人民的战争是没有用处的。法国归根到底没有能够避免在阿尔及利亚的失败,蒋介石也曾经研究过共产党方面的资料,但是仍然没有能够挽救自己。

(在谈到另外的问题时,毛主席还说,想把人民战争的战术运用到反革命战争中,就像使鸡再回到蛋壳里一样困难。)

\subsection{进行反人民的战争是困难的}

毛主席说,中国人也在研究美国的书籍,但是,中国人是不能进行反人民的战争的。

例如,他读过现任美国驻南越大使泰勒将军写的《音调不定的号角》一书,据说,泰勒将军认为核武器大概是不会被使用的,因此常规武器将掌握决定权,他一面赞成生产核武器,一面希望给陆军以优先权。现在,泰勒将军有机会把他的游击战的理论拿到现地去运用。他在越南正在积累宝贵的经验。

(毛主席)还读过美国政府当局为了部队使用而出版的指导对付游击战的战斗书籍。这些指导性的书籍指出游击战的缺点和它在军事上的弱点,并且保证美国会取得胜利。然而,他们无视了一个明显的政治事实,那就是,无论是吴庭艳,还是别人来领导,如果它是一个脱离群众的政府,就不可能战胜解放战争。(毛主席说,)美国人反正是不会听中国共产党主席的话的,所以,他即使进行这样的忠告,对谁都是无害的。

\subsection{独立和革命是两码事}

“现在在东南亚、印度以及非洲的一些国家和拉丁美洲,都存在着与发生中国革命相同的社会条件。问题是,因国家不同,情况也各不相同,因此,解决的方法将会有很大的不同。然而,你是否认为(在这些国家)会发生要借用很多中国经验的社会革命呢?”

同反对帝国主义和新殖民主义相结合的反封建、反资本主义的情绪,\marginpar{\footnotesize 216}是从过去的压迫和腐败行为中产生出来的——这就是他的回答。

哪里有压迫和腐败行为,哪里就会发生革命,但是,刚才列举的各个国家中,几乎所有这些国家的人民还不是要争取社会主义,而只不过是要争取国家的独立而已,这完全是两码事。

在欧洲各国也曾经发生过反对封建制度的革命。美国虽然没有出现过真正的封建时代,但是,尽管如此,曾经进行过摆脱英国殖民主义的进步的独立战争。后来,为了确立自由的劳动市场,进行过南北战争。华盛顿和林肯,都是他那个时代的伟人。

\subsection{殖民地国家和中苏争论}

“我们知道在属于所谓‘第三世界’的占地球的大约五分之三的地区,有着许多非常迫切的问题。人口增长率和生产增长率之间的差距,越来越悬殊,并且带来了很坏的影响。这些国家的生活水平越来越低下。它们同富庶的国家的生活水平之间的差距,也迅速地扩大。在这种情况下,时间能不能够等待苏联证实社会主义的优越性?能不能再等一个世纪左右,等到(苏联的社会主义的优越性)得到证实以后,在这些不发达地区产生议会政治主义,并且和平地实现社会主义?”

毛主席说,时间不会等那么长。我进一步问他这个问题是否同中苏之间的意识形态的争端也有相当大的关系。他回答说,是那样的。

“你是否认为,‘第三世界’的新兴国家的民族解放和现代化,可以不经过另一次世界大战而完成?”

主席回答说,“完成”这个字眼值得考虑。

这些国家中的大部分在目前阶段,距离社会主义革命还很远。有些国家里还没有共产党。也有的国家只有修正主义。拉丁美洲有二十个共产党,据说,其中有十八个党通过了指责中国的决议。但是,有一条是肯定的,哪里有残酷的压迫,哪里必然会发生革命。

“我们把话题再转回老虎身上。您现在是否还相信核弹是纸老虎?”

毛主席回答说,那句话是那样说说而已,也就是说,那句话不过是一个比喻而已。当然,核弹有杀伤力。但是,最后,还是人消灭它。这样一来,核弹就变成了真正的纸老虎了。

\subsection{中国本身就是一种联合国,原子弹不能消灭全人类}

“有消息说,您以前说过,由于中国人口众多,所以不像其他国家那样害怕核弹。即使其他国家的人民都死掉了,中国还会剩下好几亿人,因此,可以重建家园。这些报道是否有根据?”

毛主席回答说,我不记得曾经说过那样的话,但,也许讲过。

\subsection{同尼赫鲁的看法不同}

然而,我记得,当贾瓦哈拉尔·尼赫鲁访问中国时(一九五四年),我同他谈过话,我确实向他说过中国不希望打仗。我还说过,中国不拥有核弹,如果其他国家希望打仗,\marginpar{\footnotesize 217}那么,世界将遭到毁灭。也就是说,会死很多人。究竟会死多少人,那是不得而知的。因为当时不光是谈中国的事情。

我不认为,一颗原子弹就会毁灭全人类,以至于连进行和平谈判的对方政府,都找不到。不会出现这种情况。这一点,在同尼赫鲁谈话时也讲过。然而尼赫鲁说:他还担任印度的原子能委员会主席,因此了解原子能具有多么大的破坏力。他相信,(如果打起核战争)一个人也留不下,因此,我回答他说,大概不会出现那样的局面。现存的各国政府可能被消灭,但是,代之以别的政府,将会上台。

毛主席说,不久前,赫鲁晓夫曾经说过(苏联)拥有能够杀死一切生物的恐怖武器,但很快地他又取消一这句话。不只取消一次,而是取消了好几次。但是,我说的话绝对不撤回。对于这个传说(即纵然打起核战争,中国也会剩下很多人的说法),也不希望有人替我辩护。关于原子弹的破坏力,美国人谈得很多,赫鲁晓夫在这个问题上也乱说了一通,在这一方面,他们比我们高明。我们难道不比他们更谦虚一些吗?

\subsection{大自然依然如故}

毛主席说,最近,我读过一篇调查报告,这篇报告是美国一九五三年(?)进行核试验后六年,访问了比基尼岛的一个美国人写的。从一九五九年以后,就有调查人员到比基尼岛进行调查。据说,他们在比基尼岛最初登陆时,野草长得很深,需要开路才能前进。他们看见老鼠到处跑,鱼在河里游来游去,好像未曾发生过什么事情似的。井水可以喝,田园的植物,枝叶茂盛。鸟儿在树上啼鸣。进行核试验后的头两年,情况好像严重一些,但是,大自然却依然如旧。在大自然、鸟儿、老鼠、树木的眼里,原子弹不过是纸老虎而已。难道,人的精力还不如大自然吗?

“尽管如此,并不会认为核战争是好事吧?”

毛主席回答说,当然,到了不得不打仗的时候,也应当局限在常规武器上。

\subsection{是美国在联合国开了先例}

我就印度尼西亚退出联合国以及中国对此表示欢迎这一点,向他提出了问题。我问毛主席,你是否认为,印度尼西亚的退出,会不会开个先例,以后还会有国家退出联合国?

毛主席回答说,开前例的,是把中国排斥在联合国之外的美国。

他还说,尽管美国反对,但是,现在大多数国家都表示要同意中国加入联合国,于是,(美国)就搞了一个新阴谋,说这一次不采取只是多数的方式,而需要采取三分之二的多数才能通过。但,问题是,这十五年来,中国在联合国之外,究竟是占便宜,还是吃亏?

印度尼西亚退出联合国,是因为它考虑到即使留在联合国也没有多大好处。就中国来说,中国本身,难道不就是一种联合国吗?中国有好几个少数民族自治区。中国的任何一个自治区,无论人口或面积来说,都比目前在联合国里通过投票来剥夺中国席位的某些国家,要大得多。中国是一个大国,即使不进入联合国,仍然有很多工作要做,忙得很。

“是不是说现在到了可以考虑成立一个把美国除外的世界各国的联合组织的时候了呢?”

主席说,像这样的场所已经有了。一个例子就是亚非会议。\marginpar{\footnotesize 218}另外一个就是在美国把中国从奥林匹克组织排斥出去后,组织的新兴力量运动会。

\subsection{亚非会议的任务}

预定三月间在阿尔及尔召开的亚非会议的筹备工作,正在为许多问题伤脑筋。在这些问题中,包括印度尼西亚同马来西亚之间的纠纷,以及保卫“万隆会议精神”的亲中国派的各个国家,认为苏联完全是欧洲国家,而主张把苏联排斥在会议之外。

有迹象表明,中国似乎认为可以依靠这个亚非机构而不怎么依赖新殖民主义和西欧的资金来有计划地开发‘第三世界’。如果根据中国的原则,在国内建设上靠‘自力更生’,亚非各国之间靠互相援助,那么不使用那种靠资本主义方式去进行迟缓而费劲的资本积累的方法,也或许反而会加速现代化的进程。这种理论性的代替方案,当然意味着若干资本不足的亚非各国将在政治上获得更迅速而急遽的进步。从而加速创造走向社会主义前夕的条件。话题虽然离开了这次的会见,但是(中国)在很早以前就表明,也可以考虑把亚非会议作为一个同联合国——它受美国的控制,而长期以来一直把中国及其紧密的盟国排挤在外,而且印度尼西亚最近也自行退出——没有关系的永久性的集合场所。

\subsection{说不准的人口数字}

我问:“主席,在‘中国的联合国’里到底有多多人?”“你能告诉我在新的人口普查中弄清的人口数字吗?”

毛主席回答说,他真地不知道。

有人说,有六亿八千万或六亿九千万。但是,这是不可靠的。能有那么多么?

我说,只要调查一下购货证(用来买棉织品和米的)的数字,就容易算出的。他说,农民时常把问题弄得不能辨别真相。

解放前,农民们因为怕被抓去当兵,生下男孩子,隐瞒起来而不报户口,这是很普遍的。而且解放后,有多报人口,少报土地,夸大受灾面积,而只报一点点产量的现象。现在,生了孩子虽然立即报告,但是死了人几个月也不报的情况很多(也就是说,这样作,似乎可以多领供应物品)。

不错,出生率有很大的下降。但是,农民还很不愿意进行计划生育和节制生育。死亡率可能比出生率下降得还要大。平均寿命过去是三十岁左右,而现在提高到近五十岁。

我说,“您的回答会使外国的大学教授们感到很困难。”

毛主席问我,那是一些什么样的教授呢?

\subsection{不会使用原子弹,美国不可能接受禁止核武器首脑会议}

“你在中国闹起了革命,因此,在外国,中国问题的研究工作也发生了革命。所以,现在在学者当中也出现了毛泽东主义者啦、北京派啦等等许多派。”

我告诉毛主席,我曾出席过一个会议,学者们在那个会议上讨论毛泽东对马克思主义究竟有没有创造性的贡献?他表示关心。我又告诉他,在另一个这种会议结束的时候,\marginpar{\footnotesize 219}我向一位教授提过问题。

我当时向那位教授说,如果事实证明毛泽东自己并没有主张过他曾经做过创造性的贡献的话,那么,可能对这种争论有些帮助吧?那位教授焦急地回答说:“不,那也不会有什么帮助,那完全是与争论没有关系的事情。”

\subsection{不写“自传”}

听了我的话,毛主席愉快地对我说,在两千多年前的古代,庄周写了一篇有名的关于老子的论文。(被称为《庄子》)但是,这样一来,就出现近一百个试图议论《庄子》的意义的学派。

上一次,在一九六〇年会见毛主席的时候,我问他,你写没写过“自传”?还是今后打算写?他的答复是否定的。可是,学识渊博的学者们发现了几个据说是毛主席亲笔写的“自传”。尽管那些“自传”都是假的,也一点儿没有影响那些学者们的争论。

现在,学者们有一个大胆设想的问题,那就是毛主席的两篇著名的哲学文章《矛盾论》和《实践论》,像《毛泽东选集》上所注释的那样,真是毛主席在一九三七年夏天写的,或者实际上是在过了几年之后才写的。

据我的记忆,一九三八年夏天,我曾经亲眼见过这些文章的翻译稿。让那些只重视自己想法的学者们沉默,可能是很困难的。但是,为了唤起我的记忆,能不能告诉我,这两篇文章是什么时候写的?

他答复说,那些文章的确是在一九三七年夏天写的。

\subsection{《矛盾论》和《实践论》}

毛主席说,在芦沟桥事变前后的几个星期,我在延安的生活暂时有些空闲。部队出发到前线,所以便有时间搜集为编写在抗日大学讲授基础哲学讲义所需要的材料。为了在短短的三个月里把青年学生培养成顶事的政治干部,需要简单而又基本的教材。

就这样,党强迫我作这一工作,我也打算总结中国革命的经验。所以就把马克思主义的基本原理和中国的具体事例结合起来,写成了《矛盾论》和《实践论》。我打通宵写稿子,白天睡觉,用了好几个星期写成的讲义,只用两小时就讲完了。我自己认为,《实践论》比《矛盾论》更重要。

我问:“一九三七年听了你课的青年们,以后又在实践中学会了革命,但是。对现在的中国的青少年来说,有什么可以代替的办法吗?”

他回答说:现在二十岁以下的中国人民,当然没有打过仗,没有面对过帝国主义者也没有经历过资本主义的统治。他们对于旧社会,没有什么直接的感受。父母也许会讲给他们听。即便他们从故事里面或者从书本上知道一些,可是这同生活在那样的时代是完全不同的。

\subsection{“没有安全的国家”}

我在这时提出了一个问题:中国现在把重点放在向学生灌输革命思想,\marginpar{\footnotesize 220}让他们养成体力劳动的习惯的问题,采取这一方针的主要目的,是在于保证中国国内的社会主义的前途呢,还是在于教育青年少年,使他们懂得,在全世界范围内社会主义获得胜利以前,安全是没有绝对的保证呢?或者这两个目的是一个,是不可分割的整体呢?

关于这个问题,他没有直接回答,相反地他问我说,能够说有绝对安全的国家吗?

所有的国家的政府都在谈论安全问题,同时也在吵嚷全面彻底裁军的问题。中国自己也从很早以前就提过关于裁军问题的建议。苏联也曾提出过建议,而且美国也一直在谈这个问题。

但是,在我们面前所看到的,并不是裁军,而是全面彻底的扩军。……

我说:“一个一个地解决东方的问题,对约翰逊来说也或许是很困难的。如果他希望向全世界说明这些问题实际上是如何的复杂,那么他不就可以接受中国关于召开首脑会议讨论全面禁止核武器的建议,从而接触问题的核心吗?”

毛主席同意我的这些话,但是他说,这完全是不可能的。

\subsection{“反面教员:赫鲁晓夫前总理”}

毛主席说,即使约翰逊先生愿意举行这种会议,但是,归根到底他只不过是垄断资本的一个奴仆,所以资本家们也是不会允许的。中国还只进行过一次核试验,今后将不得不证明一变为二,一直扩大到无限。

他说,可是中国并不想要那么多的炸弹。恐怕哪一个国家也不想真地使用它。因此,它实际上完全是一个没有用的废物。进行科学实验,只要有两三颗也就足够了。但是那怕是一颗炸弹,他们也不希望让中国得到。中国名声不好遭到帝国主义者的厌恶,他们把一切责任都推到中国身上,发动了一个反华运动,这难道都是对的吗?

难道能说是中国杀了吴庭艳么?可是,他还是被杀了,肯尼廸总统遭到暗杀我们中国人民大吃一惊。这件事情并不是中国人民策划的。在俄国当赫鲁晓夫被解除职务时,人们也吃了一惊。可是,这也不是中国下的命令。

“西方的评论家,尤其是意大利的共产党人,激烈地批评苏联领导人用阴谋的、不民主的作法把赫鲁晓夫赶下台了。你对这件事情是怎样看的呢?”

主席作了如下的答复:

赫鲁晓夫就是在下台以前,在中国也没有什么威望。他的画像中国几乎没有挂过。赫鲁晓夫的著作从前就摆在书店里,现在也还在出售,可是在俄国已经不卖了。对世界来说赫鲁晓夫是一个不可缺少的人物。他的幽灵恐怕很难消失。有些人喜欢他,这也是无可奈何的。中国对失去了赫鲁晓夫这个反面教员,感到遗憾。

\subsection{中国军队不越过国境,在北越不至于发生战争}

“如果按照你的三七开的标准,即在评价一个人的功过的时候,只要他的行动的七成是正确的,只有三成是错误的,那就可以认为是说得过去的,那么,你怎样评价现在的苏联共产党的领导人呢?他们还差多少分才能达到及格的分数呢?”

毛主席说,他不想这样来谈论现在的(苏联共产党的)领导人。

中苏关系今后或许会有一些改善,但是,不会有太大的改善。苏联前任总理赫鲁晓夫的消失,只是表明失去了打笔墨官司的目标。\marginpar{\footnotesize 221}

\subsection{助长“个人迷信”}

“苏联批评中国在助长个人迷信。这个批评有根据吗?”

毛主席回答说,也许有些根据。

斯大林是个人迷信的目标,但是据说赫鲁晓夫完全没有这种事情。批评家们说,在中国人民中间有一些这种东西(这种感情或习惯)。(他们)这样说,或许有某种理由。也许,赫鲁晓夫就是因为丝毫没有受到个人迷信,所以才下了台……

(我在一九六〇年的会见时也曾提过这个问题。当时,在我答应不发表后,他就更详细地说明了中国共产党对于赞美个人这种做法在现时代所起的作用的看法。因为这种现象,在中国远比外国人所想象的要复杂,所以,当时我就没有深问。)

\subsection{美中两国人民的友好}

“由于历史的潮流,美中两国人民在十五年间被分隔开来,不能进行任何交流。我个人对此干到遗憾。两国之间的隔膜从来没有像今天这样大。但是,我相信,这不会发展成为战争,从而在历史上留下一大悲剧。”

他回答说,历史的潮流将会使两国人民重新和好。这一天一定会到来的。

认为当前不会发生战争,是正确的。只要美国军队不打到中国来,就不会发生战争。就是打来了,对他们也没有好处。因为不能容许这种事情发生。美国的领导人们,可能很了解这一点,所以(他们)才不敢侵略中国吧!当然,中国绝对不会为攻击美国而派遣军队,所以终归不会打仗。

\subsection{扩大战争的可能性}

“因越南问题而打起来的可能性怎样?我读过许多关于美国考虑把战争扩大到北越的报导。”

毛主席回答说,不,我不那样认为。美国国务卿腊斯克曾经明确地说,美国不作这种事情。腊斯克以前也许那样说过,但是现在他又作了更正,说他没有说过那种话。所以,可能不会在北越发生战争。

“我想,美国负责制订政策和行政工作的人员,不理解你的话。”

毛主席说,为什么呢?中国军队决不会越过国境去打仗。这一点是明确的。只有在美国攻击中国的时候,中国才要打仗。还有比这个更明确的吗?

中国国内的工作非常忙。越过本国的国境去打仗,是犯罪行为。中国为什么一定要这样作呢?越南的形势,越南人自己应付得了。

\subsection{什么人进行“侵略”}

“美国政府当局一再说,如果美国军队从南越撤退,整个东南亚就会受到侵略。”\marginpar{\footnotesize 222}

毛主席回答说问题在于受谁的“侵略”和被谁占领。是被中国占领呢,还是被当地的居民占领呢?拿中国来说,只是被中国人占领过。

此外,毛主席在回答关于中国军队的问题时断言,在北越和东南亚的任何地方,都没有中国军队。中国没有在自己国境以外的地方派驻军队。这就是他的回答。

\subsection{对于被当做侵略者感到意外,解放的武器是美国“提供”的}

“(美国)国务卿腊斯克说,只要中国停止采取侵略政策,美国就从越南撤退。这是什么意思?”

毛主席回答说,因为中国没有采取侵略政策,因此想停止也无法停止。

毛主席说,中国从来没有采取过侵略行为。中国对革命运动是支持的,但,那不是通过派遣军队。当然,如果在某一个地方展开了解放斗争,那么,为了声援,中国就发表声明或者举行示威游行。帝国主义者感到伤脑筋的原因,就在于此。

\subsection{故意闹一下}

例如,在金门、马祖的问题上,中国有时故意地闹一下。在那里,只要稍一打炮,就会引起人们的注意。这一定是因为美国人离他们本国太远,因此,总是提心吊胆。请设想一下,在中国领海内的那个地方附近,只打两三发空炮,会发生什么事情呢?不久前,美国曾经认为,在台湾海峡的美国第七舰队不足以对抗中国这种打炮的行动,因此,把美国第六舰队的一部分调来了,把驻在旧金山的海军部队的一部分也调来了。但是,来到一看,他们却无事可做。这样看来,中国似乎可以随意调动美军。

就蒋介石的军队来说,情况也是同样的。可以随意调动他们,一会儿叫他们向东,一会儿叫他们赶快向西。对于那些吃饱了却无事可干的美国水兵,当然要给他们找点事情做。然而,为什么那些实际上轰炸和烧杀别国人民的人不被叫做侵略者,而在自己国内只打了几发空炮,就被叫做侵略者呢?这是毛主席的说法。

毛泽东主席谈到,过去有一个美国人说过,中国革命是俄国的侵略者领导的革命,但,其实是美国人给中国人提供了武器。

\subsection{积极地“换帽子”}

毛主席继续说道,目前的越南革命也同中国革命一样,是美国人提供了武器,而不是中国提供了武器。最近几个月来,越南解放军显着地改善了补充美制武器的状况,同时,吸收了由美国人训练出来的南越伪军的官兵,大大地加强了自己。中国人民解放军也曾经吸收由美国人训练和提供了武器的蒋介石军队,加强了自己。这种现象被称为“换帽子”。国民党军队的士兵知道自己戴着不同的帽子就会被农民杀死以后,都争先恐后地换帽子。这时,(解放战争)就临近尾声了。现在,在南越伪军当中,“换帽子”的风气越来越盛行。

中国革命获得胜利的第一个条件是,当时的统治阶级是软弱无能的,领导人是专门打败仗的。第二个条件是,人民解放军强大而有能力,获得了人民的信任。如果是在不具备这样一些条件的地方,美国人就能够干预。否则,他们要么不插手。要么不久就要撤退,二者必居其一。\marginpar{\footnotesize 223}

\subsection{越南的未来}

“这是不是说,在南越,解放军已经有了获得胜利的条件?”

毛主席回答说,美军还不想撤退。战斗也许还会持续一两年。对于越南丧失了兴趣的美军,是回国,还是转移到别的什么地方去,那将是以后的事。

“如果美军不撤退,那么,即使举行日内瓦会议来讨论统一的越南的国际地位问题,中国也不参加,这是不是你的政策?”

毛主席回答说,可以考虑的可能性有好几个。

第一个可能性是,会议召开以后,美军撤退。第二个可能性是,会议一直延期到美军撤退完毕。第三个可能性是,即使召开了会议,美军仍然驻在西贡附近。还有一个可能性,那就是解放军不依赖什么会议或国际协定,而把美国人赶走。一九五四年的日内瓦会议,规定法国军队从整个印度支那撤退,并且规定一切其他外国军队绝对不得介入。尽管如此,美国违反了这个协议。今后也许还会违反……

\subsection{改善中美关系要等到下一代}

“你是否认为,在目前的形势下,真有希望能改善中美关系?”

他回答说,有这个希望。

但,那需要时间。大概在我这一代是不会得到改善的。我不久就要见上帝去了。根据辩证法的规律,包括每个人的生命的斗争在内,一切矛盾最终都要得到解决。

\subsection{把未来寄托于青年一代,不断变化的历史环境}

“下面,我将话题转到您的健康问题上,从今天晚上的情况来看,您好像很健康”。

毛主席苦笑了一下回答说,关于这个问题,总像是也有点疑问。他反复地说。我已经准备很快地去见上帝。然后又反问我:你相信不相信?

“您的意思,是不是要确认一下上帝是否存在?您相信上帝吗?”

他回答说,不信。

\subsection{死神躲着走}

他说,但是,在那些自夸神学的人中间,有人相信上帝是存在的。上帝有很多,有时,同一个上帝保佑各个方面。欧洲大战时,基督教的上帝保佑过英国、法国和德国。这些国家相互打仗时也是这样。苏伊士运河发生危机时,上帝保佑英法两国,但,对方则有真主保佑。

晚饭时,毛主席谈到他的两个弟弟全被敌人杀死,他的第一个妻子也同样是在革命当中被敌人处死,他的儿子是在朝鲜战争中战死的。他说,回想起来也奇怪,死神过去一直是躲着我走的。他说,他有好几次都做了死的准备,但是,不知为什么都没有死。\marginpar{\footnotesize 224}

他说,不知是怎么一回事,有好多次都曾以为是要死的了。紧靠在自己身边的警卫员也牺牲过。有一次,身旁的战士被炸弹炸伤,满身是血,可是我却一点也没有受伤。这种差一点儿死去的事还有过不少次。

毛主席稍停了一会儿,又说,你大概知道我的出身是一个小学教员。

\subsection{偶然的巧合}

他说,那时,并没有想要打仗,也未想过会变成一个共产主义者。那时,我不过是一个同你一样的民主人士。是一种什么样的偶然的巧合,使自己立志要建立中国共产党呢?现在,时常想起来也感到奇怪。事物总是不以个人的意志为转移的。最重要的是因为中国受着帝国主义、封建主义和官僚资本主义的压迫。这是事实……

我说:“这就是说,人自己创造历史,而历史依据环境的不同而成为不同的东西。您从根本上改变了中国的环境。许多人有个疑问,就是,生活在比以前舒服得多的条件下的青年一代,将会变成什么样子?您的看法怎样?”

主席回答:我自己也无从知道。他说,这恐怕谁也无从知道。可是,能够想到两点。一点是继续革命,也许会向着共产主义进一步发展。另外一点是,也许现在的青年们会否定革命,表现不好。也就是说,或许会同帝国主义和好,把蒋介石集团的残余分子领回大陆,投靠现在国内存在的少数的反革命分子。

当然,我不希望他们反革命,可是未来的事情,要由未来的一代,根据当时的条件决定。是什么样的条件,现在我们还不能预想到。正像资产阶级民主主义时代的人们所具有的广泛的知识,超过了封建时代的人们一样,将来的一代应该比现在的我们更聪明,问题是他们怎样判断,而不是由我们来判断。今天的青年以及接续他们的未来的青年,将根据他们自己的判断来评价中国革命的成果。

说到这里,毛主席放低声音,半阖了跟睛。他说,地球上的人类的条件,正以空前的速度变化着。从现在起再过一千年,马克思、恩格斯、甚至连列宁也一定显得不高明了。

\subsection{祝美国进步}

在我要告辞的时候,毛主席说:“向美国人民问候,祝他们进步。”

我想,他大概不好说美国人解放吧?他们不是已经有了投票权吗?可是实际上他们还没有解放。对于那些实际上还没有解放,但衷心期望着解放的美国人,他表示了祝福。

尽管我谢绝,但,毛主席还是把我送到车里。毛主席按北京的传统的礼节,为了挥手送别,一个人在零度以下的寒天中,没有穿大衣,站了片刻。我在大门附近没有发现警卫战士。在整个会见过程中,我没有看见一个武装警卫人员。我在开动了的汽车里,回过头来,看见毛主席微缩着双肩,由侍从人员扶着他那魁梧的身躯,迈着徐缓的脚步,走进了人民大会堂。我望着,贪婪地望着。

\kaoyouerziju{(按:毛主席于1965年1月9日接见了斯诺,并同他进行了长时间的谈话。)}

