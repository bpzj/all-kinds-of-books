\section[约集各民主党派负责人和无党派民主人士谈话纪要 ]{约集各民主党派负责人和无党派民主人士谈话纪要 }


几年来都想整风,但找不到机会,现在找到了。凡是涉及许多人的事情,不搞运动,搞不起来。需要造成空气,没有一种空气是不行的。现在已造成批评的空气,这种空气应继续下去,以处理人民内部矛盾为题目,分析各个方面的矛盾。

过去在工业、农业、文教等方面有什么矛盾是盲目的,现在把矛盾排一个队,算一个账。过去好像一讲矛盾,就不得了,其实有什么不得了,蒋介石四百万军队,美国军队到了鸭绿江,也没有什么不得了。每天都在矛盾中生活,到处都是矛盾,都在唱对台戏,但是就不肯承认或者混淆两种矛盾。要承认它,然后分析它,找出解决办法。现在报纸天天在讨论矛盾问题,有人担心人民政府会被推翻,但已经二、三个月了。政府并未被推翻,而且越讨论,越发展,人民政府就越巩固。现在高教部和教育部被攻得厉害,越攻越好,把矛盾分一下类。要攻就趁此机会,只要找出证据,就能站住脚,几年甚至几十年解决不了的问题,一辩论,可能在几个月内就可以解决。人们并没有提议要打破你的饭碗,也不是要一棍子把你打死,而是要求改善关系。把矛盾从各方面分一下类,高等教育、普通教育、文艺、科学等。卫生也很值得攻一下子,多攻一下,切实攻一下,在报上发表,可以引起大家的注意,不然官僚主义永远不得解决。找出办法,要党内外一起来,以往开小会不灵,要开最高国务会议第十一次扩大会和宣传会议一样的大会,党内外一道开会,两种元素合在一起,起了化学作用,成了另一种东西,就灵了。各省、市都要开。报上登一下,就可打破沉闷的空气。这时提整风比较自然,整风总的题目是要处理人民内部矛盾,反对三个主义。

中共中央指示中有一条特别的规定,就是要参加劳动。这并不提倡在座的人都去耕田,扫扫街总可以。主要是表示一种态度,要砍掉官僚主义、主观主义,加一个参加劳动的办法,总的要同工人农民混在一起,多少参加一点劳动。农民说,过去你们和我们一道土改,现在你们作了官了,不理我们了。过去合作社主任不参加劳动,很不得人心,群众很不高兴,不劳动,不能和农民打成一片,老百姓就不信任你,不和你讲真话,现在区乡干部和群众一道参加生产,群众马上同他们讲真话。群众对人大代表、政协委员讲的活不会全是真话。尤其我们都是知识分子,同劳动人民格格不入,有种办法可以同他们一起,即多少参加一些劳动,比方在南方打秧锄草并不困难。当然,年老的有病的可以不去。我国有一特点,人口有六亿如此之多,耕地只十六亿亩加此之少,不采取一些特别办法,国家恐怕搞不好。

整风会影响党外。规定非党员自愿参加,自由退出。最近两个月就是这个方式,就是整风办法,我攻你,你攻我,有意见就说,党内外打成一片。此即整风,已经整了两个月。

统战工作中的矛盾几年不得解决,如有职无权等,过去很难解决,现在可能解决了。过去不好解决的原因主要是思想不通。过去是共产党员有职有权有责,民主人士只有职而无权无责,现在应是大家有职有权有责。同共产党一道混,民主人士确是不好当,很有点恼火,不好办事。现在党内外应改变成平等关系,不是形式上的而是真正的有职有权。以后无论那一地方,谁当长的就归他管(马寅初:过去好在有组织,我的一切都是旧的,如果真管,不能有现在。主席:你讲话不彻底,矛盾存在,敷衍过去不能解决。教授治校恐怕有道理。是否分两个组织。一个校务委员会管行政,一个教授会议,管教学)。

现在不要搞唯物辩证法,要搞政治关系,搞三个主义。不要钻到世界观思想方法里头去,以免把政治关系冲淡,唯物辩证法以后再搞。这个问题会影响到民主党派、民主人士。唯物辩证法,我们不能企图很多人一下子都接受马列主义世界观。希望在三个五年计划之内五百万知识分子中有三分之一,即一百五十万左右的知识分子能接受唯物辩证法就很好了。今后学习都要自愿学习,自己研究,可以自愿结成小组。关于辩证法的课本要一年编一本,有些过时了,旧话要新讲。世界观是长期的问题。过去为旧社会服务的五百万知识分子现在要转过来为新社会服务,他们有的是旧的世界观,有新的世界观的只是一小部分人。很快的时间内就有很多人相信新世界观是不自然的。不可能的。估计十年之内,还会有三分之一多一些三分之二少一些的知识分子还会有旧的世界观,但这些人一是爱国,二是相信社会主义,这样就可以了。有些教授讲课讲马列主义,这是为了吃饭;但对自己讲,就说自己并不相信辩证法了。上海有一左翼教授说,解放以来“魂魄不安”“十五个吊桶七上八下”。教社会科学的天天处于被动,历史要重写重教,在旧社会教书拿出来就行,一个课本可以教多少年,又没有这么多的会议。自然科学家好一些。社会大变动时期,使知识分子吃了苦头,主要是社会科学方面的,这里有个经济基础的问题。过去五百万知识分子所依附的经济基础,现在垮了,帝国主义、封建主义、官僚资本主义早已推翻,连民族资本主义也根本没有了,个体所有制也破坏了,有人说,私有制都没有了,还有什么两面性呢?这是不对的。“皮之不存,毛将焉附?”旧皮不存,毛要附在新皮上,不能吊在空中。要附在工人阶级的皮上,现在五百万知识分子是吃工农的饭,吃国家所有制和集体所有制的饭。现在产业工人有一千二百万人,党、政、军、教育和经济工作人员(厂长、合作社社长不在内)等共一千四百万人,合计二千六百万人。这一千四百万人不是直接生长的,真正直接参加生产的是一千二百万工人。其余所有的人都要向工人过渡的。解放时产业工人只有五百多万,苏联初期也只有三百五十万人。社会前进不决定于人数,农民有几亿,并不能解决社会的前进不前进,农民最后都要变为农业工人,合作社变为国营农场,这可能是几十年以后的事情。五百万知识分子的毛要附在一千二百万的工人阶级的皮上,毛脱离了皮是不行的,总要在一家吃饭。现在知识分子有些不自觉,他们的墙脚(经济基础)早已挖空了。过去的经济基础,一经地震,他们就悬空了。旧的经济基础没有了,但他们的头脑还没有变过来,思想是多少年钻进来的。现在毛已经附在皮上,但思想还是认为马列主义不好。马列主义世界观不要强迫人家相信,要人家相信,要有过程。旧的世界观,资产阶级的和小资产阶级的世界观是一个,说小资产阶级出二元论,资产阶级出一元论,有人说是小资产阶级知识分子就舒服些,说是资产阶级知识分子就不舒服,其实这是一种迷信,是一种社会风气,如我原来就是资产阶级知识分子,受资产阶级的社会风气影响,受资产阶级的教育,我信过佛教,信过康德,信过无政府主义,这些都是唯心主义,因此是资产阶级知识分子。小资产阶级知识分子就是资产阶级知识分子,旧的世界观只有一个,另外分不出什么小资产阶级的世界观。

工商业者政治常识课本可学,但课本就是课本,思想还是思想,总有一部分人永远不会变的。条件反射,生孩子本来不痛的,就是社会舆论说会痛就痛起来了,要变痛为不痛,要医生作许多工作,要另外一种条件反射。不要相信所有知识分子、民主党派和共产党员都相信共产主义。共产党内有相当多的一部分人不相信共产主义而且不相信社会主义,他们相信的是民主主义,他们对社会主义革命没有准备。原河北省付主席薛迅就反对统购统销,主张自由贸易。统购统销不搞不行,多搞也不行,明年准备除经济作物区以外,大多数合作社的粮、油、肉由合作社自理。有相当大的一部分党员不信社会主义,他们搞社会主义是随大流卷进来的。我就不相信民主党派都信社会主义,有一大部分人不相信社会主义,但也不愿公开讲,共产党、民主党派、知识分子、民族资本家和农民中一部分,工人中一部分(工人也成分复杂,由五百万发展到一千二百万,人心不齐),对社会主义成不成功,共产党行不行,还要看一看。社会上还有相当一部分人,上了贼船(共产主义之船)没有办法,会不会翻船要靠天。这是很自然的。

过去作的是阶级斗争,民主革命与社会主义革命时期都是如此,是人与人开战,人打人,人内部造反,花了几十年的精力,如从鸦片战争起算,已经百余年了。鸦片战争,就群众来说,是阶级斗争,对林则徐来说,是中国剥削者与外国剥削者的斗争。单就最近几十年说,从共产党成立起,已有三十年,精力都耗在这上面。有人说共产党不能领导科学,就只会搞阶级斗争,共产党有术无学,这说对了一半。说我们有术无学,要看是什么学,譬如阶级斗争是一门大学问,不能说我们没有。这是经过多少年犯错误作检讨学来的。阶级斗争,以后还会有的,如同帝国主义战争,同资本主义国家办外交,都是阶级斗争性质的。现在进入另一种战争,就是向自然界开战,要懂得自然界科学,不懂怎么办?我在一九四九年的文章中就说过了,我们熟悉的东西快要闲起来了,我们必须学会自己不懂的东西,不懂就不懂,不要装懂。要老老实实地学,可能要同过去学习阶级斗争一样,需要几十年的时问。从党的成立到七大(一九二一至一九四五)共花了二十四年时间,流了那么多的血,受了许多挫折才学会阶级斗争,现在学建设的新战争,要从头学起。能不能学会?肯定可以学会。有些党派专家多,如民盟、九三,民建也不少。总的说来,是新时代和新的任务,阶级斗争结束,向自然界宣战。现在还在过渡时期,旧的生产关系破坏,新的经济基础还没有巩固,打这个仗,要几十年,大概要二三十年。因为我们无经验,无干部。有苏联经验在前,可能我们比苏联略好或略差或一样。搞得怎么样,大家要看,我也要看,不能吹百分之百正确的牛皮。有人说共产党朝令夕改,特别是高教部。国民党税多,共产党会多,会多就要变,今年基本建设就变了二十几亿。预算头年十一月就需要确定,提到人大常委,十二月就要发下去讨论,一月就要生产。提到人大是实际上的事后批准。缺少经验,需要学习,学自然科学,学计划经验,要积累经验,需要积累几十年经验,有些朝令夕改,是因为没有经验。我的脑子开始也有点好大喜功,去年三、四月间才开始变化,找了三十几个部的同志谈话,以后在最高国务会议上谈了十大关系,其中有五条属于经济方面的。

北京是个好地方,又是不好的地方,共产党的负责人每年要有四个月在外,八个月在北京。你们也可以,中央机关的特点,一是空、二是全面,缺点就是空,一离开北京就舒服了。


