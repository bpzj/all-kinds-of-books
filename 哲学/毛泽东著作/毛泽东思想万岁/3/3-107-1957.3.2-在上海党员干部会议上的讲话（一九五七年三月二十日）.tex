\section[在上海党员干部会议上的讲话(一九五七年三月二十日)]{在上海党员干部会议上的讲话}
\datesubtitle{(一九五七年三月二十日)}


刚才柯庆施谈到,现在是一个转变时期,过去我们是长期从事革命战争,从事阶级斗争。而这个斗争,在国内来说,现在基本上结束了。阶级敌人的统治已经被推翻,社会制度的改革已经基本上代替了过去的旧制度。在我们面前的新的任务,就是建设,建设也是一种革命,这就是技术革命和文化革命。团结整个社会的成员、全国人民同自然界作斗争,当然在建设过程中,还是离不了人与人之间的斗争。在目前的过渡时期中,人与入之间的斗争还包括阶级斗争。我们说阶级斗争基本结束,就是说还有些没有完结,特别是在思想方面,无产阶级和资产阶级之间的斗争还要延长相当长久的时间,这样一种形势,我们党是看到了的。在党的第八次全国代表大会时的报告和大会决议都说到了。大规模的群众性的阶级斗争,已经基本结束了,随着敌我矛盾在国内基本解决,人民内部矛盾开始比过去显露了。但是至今还有许多同志对于这种形势不很清楚,还用过去老的方法对待新的问题,应该说在过去一个时期内,中央对这个问题也没有做详细的说明,这是因为,这个变化还是在不久以前才成熟的。比如去年四月省市委书记会议上讲了十条关系,其中讲到一条敌我问题,一条是非问题。在那时候还没有说到阶级斗争基本上结束。到了去年上半年,党召开代表大会的时候,才可以确定地讲这一点。现在情况更明白了,就需要更加详细地告诉全党:不要使用老方法对待新问题要分清敌我之间的矛盾和人民内部的矛盾。

社会主义社会有没有矛盾?这个问题,列宁曾经说过,他认为矛盾是存在的。但是斯大林在一个长时期内不承认社会主义社会有矛盾。在斯大林的后期,人们说不得坏话,批评不得党,批评不得政府。斯大林实际上混淆了人民内部矛盾和敌我矛盾,把说坏话的,说闲话的当做敌人,所以就冤枉了许多人。斯大林在一九五二年所著的《苏联社会主义经济问题》一书中,也说了在社会主义社会中,生产关系和生产力之间还有矛盾的。并且解决得不好,矛盾也会转化成为对抗。虽然如此,社会主义社会内部的矛盾,人民内部的矛盾,斯大林还是说得很少。我以为我们今天应讲这个问题,不但在党内,而且在报纸上讲清楚这个问题,做出适当的结论,比较好。

比如说人民闹事这样的问题。这当然不会是普遍的,只是一些个别的,但是会经常有,因为官僚主义这个东西总会存在,我们要整风,要解决官僚主义问题。凡是出了官僚主义的地方那个地方的人民就可能闹事。闹出了事,我们怎么看?应该看成是普遍的事情,不应大惊小怪。应该看到这是在特殊情况下调整社会秩序的方法,如果按照正确的方法,问题长期不得解决直到群众一闹才解决了,那么,群众为什么不闹呢?当然我们不是要去提倡闹。不闹而解决问题经过民主集中制、按照“团结-批评-团结”的公式去解决问题,这才是我们的主张,实行这个主张,就需要反对官僚主义。如果有某一个单位的领导者,官僚主义十分严重、十分顽固,群众不可去提意见,而上级又没有及时发现、纠正,不把那个领导撤掉,那么,那里的群众就会闹。在这样特殊情况下,闹一点事有什么要紧呢?无论敌我矛盾和人民内部矛盾,在我们党内都有不同的观点,有右有“左”,在敌我矛盾的问题上,右的观点就是看不见敌人。现在不是说国内的阶级矛盾基本解决了吗?有些人,就把基本解决看成完全解决,因此他们对那些人民痛恨的很坏的人,真正的特务,真正的坏人,也不去处理,这一种思想当然不对。夸大了也不对,夸大了是“左”的观点。阶级矛盾已经基本解决了,有些人却说还没有,说阶级矛盾很大。对于人民内部的矛盾,有些人忽视了这个事实,以为太平无事。我们共产党就是代表人民跟帝国主义、国民党作斗争的,后来又跟资产阶级作斗争,人民怎么会回过头来反对我们呢?过去没有设想过人民会有不满意,没有设想过群众会对我们游行、请愿、罢工、罢课,现在也还是不大相信。这是一种情绪。另外一种情绪,就是怕。出了一点乱子就是不得了,天下就要大乱,人民政府就要倒了,就算起了龙卷风,我们共产党会不会吹走呢?人民政府会不会吹走呢?马克思列宁主义会不会吹走呢?我们可以有信心地说,不会的。所以没有什么可怕。不如此如果在客观上真有事非闹不可的形势,让闹反而比压着不让闹好些。这样一种看法,同志们考虑考虑。是否比较妥当一点,这是一个问题。

其次,我想谈一下知识分子问题。全国知识分子大约有五百万,从他们的出身来说,从他们受的教育来说,从他们过去的服务方面来说,可以说都是资产阶级性质的知识分子。其中大约有十分之一多一点加入了共产党,这些人大约有百分之几。这两部分人合起来,据有的同志估计,大概有百分之十五到百分之十七。在另一方面,也有少数跟我们有敌对情绪的人。他们并不是反革命分子,而且在某些问题上,例如在反对帝国主义问题上跟我们还能合作,但是对马克思主义、对社会主义制度,他们是怀疑的。这样一种人大概有百分之几。其余百分之七、八十是中间状态的。他们了解一点马列主义,但是了解不多;他们可以赞成社会主义制度,但是也容易动摇,至于接受马列主义世界观,就还有困难。他们常说:“你们我们”,不是把我们的党当作他们的党,像多数工人贫下中农那样。从他们跟工人农民的关系上,也可以看出他们的世界观还不是马列主义的,他们到工厂里头、农村里头去看的时候,跟工人、农民不能打成一片。看是看了,但是有距离,互相不能成为朋友,不能谈知心话,也还是“你们他们”的关系。他们也在为人民服务,但是还不能做到全心全意为人民服务。跟工人农民不是一条心,但也不是两条心,而只有半条心在人民方面。我们的任务就是要争取他们。要在比如三个五年计划之内(还有十一年)使各个世界在学习马克思主义方面,在跟工人、农民结合方面,前进一步,其中大概要有三分之一的知识分子或者进了党,或者成为党外积极分子。然后再进一步,争取其余的知识分子。我们要这样分步骤地来改变知识界的状况,改变他们的世界观。

文艺的工农兵方向,对于一部分知识分子,现在似乎还是一个问题,这也是知识分子新的人生观还没有建立起来的一种表现。我们要向他们说明在我们国家里,除了工人、农民就没有别的人了。资本家要变工人,地主已在农民。除了这两种人。第三种就是知识分子。知识分子是替工人农民服务的,他们的本身性质在变,要逐步地变为工人阶级知识分子。因此,文艺当然是工农兵方向,没有别的方向。你们还能提个地主方向,资产阶级方向,帝国主义方向?那些力量已经退出了政治舞台,丧失了社会基础。至于几亿小资产阶级分子也已经进入了集体化,现在还叫小资产阶级就不行,他们已经变成集体农民,集体手工业者了。

当然他们的思想还拖了一条小资产阶级的尾巴,特别是那些富裕中农和上中农,他们的资产阶级、小资产阶级思想还很严重。现在怀疑工农兵方向的知识分子,正是反映了资产阶级以及小资产阶级的富裕阶层思想。但是无论如何,资产阶级小资产阶级现在都只剩了一条尾巴,尾巴无论有多长,终究是要消灭的。因此我们要把资产阶级小资产阶级知识分子团结争取过来,条件是完全充分的,但是也还需要时间,要有十几年二十几年那么长时间,不能快,不能忙。马克思主义只能逐渐说服人,不能强迫灌进去,灌是解决不了问题的。

在一部分知识分子中间,现在还流行着一种舆论,说“共产党不能领导科学”,他们这个话有没有理由?我们说有一半理由。在现在,党员工程师、教授、医生和其他高级技术人员确实很少,一般地说来,我们现在还确实不懂科学。我们的人不但没有时间也没有钱进大学,到外国去留学。还有,帝国主义国民党也不让我们留在城市里研究科学。许多知识分子说我们不能领导科学,搞政治可以,军事也行。其实,这也只是现在的话。在以前我们没有胜利的时间,在我们打游击战争的时候,他们可不这么说,他们那时认为我们什么都不行。总之是成不了气候。人民是要看事实的,因此在我们在科学上还没有占领领导地位的时候,要他们承认我们可以领导科学,也是不可能的。但是他们的道理也只有一半对,另一半他们就没有看到。我们现在显然不懂科学。但是我们还是在用国家计划领导着科学事业,用政治领导着科学事业。在共产党和人民政府的领导下,中国的工业发展了,科学也发展了,这是不是事实呢?在这个意义上科学如果不是在共产党和人民政府的领导下工作,又是在谁的领导下工作呢?而且科学也是跟政治、经济一样,是可以学习的。我们既然可以学会政治和军事,也可以学习科学。如果说我们进行政治斗争和军事斗争,从一九二一年到一九四九年,经过了二十八年才取得了胜利,那么学习科学达到一般专家的水平,就不必用这么长时间。进五年大学,再加十年工作、有十五年的工夫就差不多了。

同知识分子问题、科学技术问题有密切联系的是“百花齐放、百家争鸣。长期共存、互相监督”这个方针的问题。关于这个方针,我们还要在党内做许多宣传解释工作,有些同志觉得这个方针太危险了百花齐放,放出鬼来怎么办?关于长期共存、互相监督,有人说“民主党派有什么资格和我们长期共存,还是短期共存吧”,“我监督你、我还用你监督吗?你民主党派哪年打的天下?”所有这些意见都是反对“放”,主张“收”。这些想法对不对呢?

中央认为主张“收”的意见是不对的,百花齐放,百家争鸣,对于发展科学事业,是一个基本性的方针,不是一个暂时性的方针。在目前的过渡时期对于团结教育知识分子,这个方针更有特殊的意义。这就是说,不但在纯粹科学艺术的问题上。而且在涉及政治性的是非的问题上,只要不属于反革命一类也应该让他们自由说话,对于说错了的怎么办?用压服的方法,还是用说服的方法?中央认为压服的方法不好。压服就是压而不服,不可能压服。人民内部的问题,思想方面的问题精神世界的问题,只能用说服的方法不应该采取压的方法,无产阶级实行的专政只适用敌对分子。一般人民说错了话或者闹了事,不能对他们使用专政的方法只能采取民主的方法。这里有一个重大的界限。现在我们有些同志对待人民内部问题动不动就想“武力解决”这是非常危险的必须坚决纠正的。不会说服怎么办,这就要学习。我们一定要学会说道理,是可以学会的。用了说服的方法人家还要闹怎么办?会不会搞的天下大乱?我说不会,只要我们坚持说理、充分说理,不会乱的。而且也不要尽是怕乱,出一些小乱子有好处,我们可以取得经验。对于那些有毒素的文章,或者别的有毒素的东西,我们要批判、要斗争,但是用不着怕。我们同那些有毒素的东西作斗争,会使我们自己发展起来。会使马克思主义发展起来。马克思主义从来就是同敌对思想作斗争而发展起来的。

我们要同敌对思想作斗争,我们自己的缺点首先要整顿。我们有很大的成绩,我们这个党是伟大的党,光荣的党,正确的党。这是必须肯定的。但是我们也有许多缺点,这个事实也要肯定的。不应该肯定我们所做的一切,只应该肯定最主要的,正确的东西。同时也绝不应该否定我们的一切,只应该否定一部分错误和缺点。否定一切就是机会主义者,肯定一切就是教条主义。教条主义也是形而上学,它肯定一切而不加分析。如果把我们的工作加以分析,就会知道,我们的工作中间的成绩是主要的,但是有缺点,所以,我们要进行整风。

关于整风,中央还没有作出正式的决定。准备这样做:今年准备,明年、后年实行。采用延安那时候的方法,从容不迫地用自我批评的精神来学习一次马克思主义,对我们作风中的主观主义、宗派主义、官僚主义的缺点,用适当的方法加以批评。批评是在小组内,不开大会,有多少毛病就是多少毛病。不要缩小也不要扩大,自己反省反省,同志们加以帮助。总之,是用“团结-批评-团结”的方法。这就是说,从团结的愿望出发,经过批评达到团结,惩前毖后,治病救人。用这样的态度来进行整风,提高马克思主义的思想水平。从延安整风以后,实际上有十几年没有进行全党范围的有系统的思想整风了。这次整风的结果,估计会使我们党得到相当大的进步。我们自己来批评我们自己的官僚主义、主观主义、宗派主义,这会不会使我们丧失威信?我看不会丧失威信,会增加威信。延安那次整风就是证明。增加了党的威信,增加了同志们的威信,增加了老干部的威信,新干部也得到了教育。一个共产党,一个国民党,这两个党比较起来,谁害怕批评呢?国民党害怕批评,禁止批评,结果并没有挽救了自己。共产党是不怕批评的,因为我们是马克思主义者,真理在我们方面,工农基本群众在我们方面。我们把风整好了,我们在工作中就会更加主动,我们的东西就更多,我们的本事就更多。同时也就会更谦虚一些,原来不会说服人的就会慢慢的增加说服人的能力。党外人士,可以让他们自愿参加。我们先整,他们后整。如果有百分之六十到七十的知识分子参加整风就很好了。

整风的任务之一,就是发扬艰苦奋斗的传统。因为革命胜利了,有一部分同志的革命的意志有些衰退,革命的热情有些不足。注意待遇,注意享受的人多起来了,要经过整风使这些同志重新振奋起来。经过这么长久革命斗争,有一点疲劳,需要一些休息,这是可以理解的。看一点戏,跳一点舞。穿一点花衣服,这并不需要反对。我们所反对的是追求地位,追求特权,铺张浪费,脱离群众。在我们的工作和生活中,凡是可以节省,应该节省的都要节省。阶级斗争,几十年革命,都是为建设开辟道路。而要建设,就必须珍惜人力、物力。建设的时间长得多,这是另外一场战争,我们希望在不很久的时间之内,中国比现在要好,比现在要富,比现在要强。革命就是为了这个目标,那一仗就是为了这一仗,那个战争就是为了这个战争。这个战争更艰苦,时间更长,比方说至少一万年。为了建设得更好更快,就必须坚持艰苦奋斗的作风,坚持密切联系群众,反对铺张浪费,追求特权、摆架子等一切坏习气。

跟党外的关系,应该比过去更进一步,这也是整风的任务之一。党与非党有一点界限是必要的,没有区别是不好的。应该有区别。这是第一。第二,就是不要有一条深沟。现在的情况就是许多地方党内党外这个沟太深了。应该把这个沟填起来。要跟他们讲真心话,不要讲半,还留一半在家里讲。这样他们会进步得更快些。

我们采取以上所说的方针,会不会妨碍党的领导?在《再论无产阶级专政的历史经验》这篇文章里头有这样几句话:“但是对于这些缺点的批判,只能是为着巩固民主集中制,巩固党的领导。而决不能像敌人所企求的那样,造成无产阶级队伍的涣散和混乱。”我刚才讲的一些话,说是闹一点也可以,登一点错误的文章也可以。我们慢慢来批评教育,讲道理。这两种提法是不是互相冲突?文章上讲的原则性,我们现在说的是属于灵活性。要灵活地运用原则。不然,一个地方罢了工,我们就去嚷:“嗨,你破坏党的领导!”知识分子批评了我们,我们就责问他:“你要涣散无产阶级的队伍吗?你要破坏党的领导,破坏民主集中制吗?只要民主不要集中吗?”到处拿这文章做挡箭牌。这行不行呢?这就不行。我们在原则上不提倡罢工罢课,不提倡错误的文章,不提倡有毒素的戏等等,但是事实上发生了一些个别的罢工罢课,报纸上登了一些个别错误的文章,戏台上演了一些个别不好的戏,对于这些现象采取放的方针,采取说服教育的方针,没有大害,反倒有利;采取压的方针,反倒不利。这样的方针比较容易调整社会秩序,调整领导者跟被领导者,政府跟人民,党跟人民的关系。这种调整的结果,正是巩固了党,巩固了民主集中制。

我们希望把我们国家造成这样一个活泼的国家,使人们敢于批评,敢于说话,有意见敢于提,不要使人不敢说。我们这些人有错误缺点必须改,不改就不行,因为没有道理,无论党政府(原稿不清)都不实行官僚主义,不能强制人们做那些没有道理的事。我们采取这样的方针,我相信,人民的政治情况,人民跟政府的关系,领导者跟被领导者的关系,人民跟人民之间的关系,就将是一种合理的活泼的关系。这样,我们的文化、科学、政治、经济,我们整个国家,就一定可以比较快地繁荣发展起来。


