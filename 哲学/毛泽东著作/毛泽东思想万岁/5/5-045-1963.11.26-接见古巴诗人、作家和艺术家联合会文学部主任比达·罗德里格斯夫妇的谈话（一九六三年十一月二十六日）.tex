\section[接见古巴诗人、作家和艺术家联合会文学部主任比达·罗德里格斯夫妇的谈话(一九六三年十一月二十六日)]{接见古巴诗人、作家和艺术家联合会文学部主任比达·罗德里格斯夫妇的谈话}
\datesubtitle{(一九六三年十一月二十六日)}

\begin{duihua}
    
\item[\textbf{主席:}] 欢迎古巴同志,欢迎古巴诗人。

\item[\textbf{比达·岁德里格斯(以下简称比达):}] 形式上我是一个诗人,实际上我是一个革命者。

\item[\textbf{皮塔·桑托夫大使(以下简称大使):}] 主席身体好吗?

\item[\textbf{主席:}] 夏天有些感冒。到南方走了一个月,身体好一些。在北京要看很多文件,到外面去可以爬山,可以接近群众。

\item[\textbf{大使:}] 主席脸色比去年还好。

\item[\textbf{主席:}] 去年何时见过?

\item[\textbf{大使:}] 您接见古巴军事代表团时见过。

\item[\textbf{主席:}] 大使身体好吗?你们二位(指比达夫妇)怎样?

\item[\textbf{比达:}] 我们身体很好。我们非常幸福,在中国我们变得更年轻了。

\item[\textbf{主席:}] 你们到中国多久了?

\item[\textbf{比达:}] 两个月了。到南方访问了一个月。我们在上海时,听说主席也在上海。

\item[\textbf{主席:}] 那时正是南方秋收的季节。

\item[\textbf{比达:}] 在上海,我以无比激动的心情参观了鲁迅故居。鲁迅是我们长期以来钦佩的文豪。

\item[\textbf{主席:}] 鲁迅是中同革命文豪。他前半生是民主主义左派,后半生转为马列主义者。

\item[\textbf{比达:}] 不久以前,古巴全国出版社出版了《鲁迅选集》十万册。这在古巴是个大数字,在中国是微不足道。古巴一本书,一般出三万册就很多了。这说明,古巴人民对鲁迅是多么崇敬。

\item[\textbf{主席:}] 鲁迅对帝国主义、封建主义的斗争很明确。他是从那个社会出来的,他知道那个社会的情况,也知道如何去斗争。旧知识分子说他具有二心,是叛徒,所以他写了《二心集》,又说他运气不好,正交华盖运,他就出了一本集子叫《华盖集》,还说他是堕落的文人,他采用了“落文”为笔名。鲁迅对那些人的批判毫不放松。被他批判的人,有一部分转到革命队伍里来,另一部分跟美帝国主义走了。

\item[\textbf{比达:}] 对敌人不能给以喘息的机会。

\item[\textbf{大使:}] 这是真理,可用在生活各个方面。

\item[\textbf{比达:}] 我们很荣幸地访问了您的故乡韶山。

\item[\textbf{主席:}] 那是个小地方穷地方,山多地少,可以去看看。

\item[\textbf{比达:}] 韶山对我们来说,不是值得去看看,而是应该去看看。我们怀着十分激动的心情去瞻仰了一个产生革命根源的地方。\marginpar{\footnotesize 66}

\item[\textbf{主席:}] 过去韶山穷人很多,常侵犯大地主,被大地主称为土匪。我们共产党自成立之日起直到今天,蒋介石还称我们是“共匪”。帝国主义说我们是“好战分子”、“侵略者”。我看这些名字倒不错。他们说我们,第一侵略中国,因为我们反对美国侵略中国;第二侵略朝鲜,因为我们在朝鲜跟美国人打仗;第三侵略西藏;还说我们侵略越南、老挝。大概也说你们在侵略古巴吧!

\item[\textbf{比达:}] 他们是如此说的。

\item[\textbf{大使:}] 的确,鲁迅著作,毛主席著作在“侵略”古巴。

\item[\textbf{主席:}] 哈哈!美国说你们要“侵略”拉丁美洲。我看值得“侵略”一下。

\item[\textbf{比达:}] 美国害怕古巴星星之火燃烧起拉丁美洲的大火。

\item[\textbf{主席:}] 那个哈瓦那大会影响很大,尤其是第二个。古巴革命有两个任务,第一个任务是使古巴能存在下去;第二个任务是帮助拉丁美洲革命取得胜利,让美国的“后院”烧起火来。每个国家都有革命党,有些所谓革命党不革命了。但总有革命的人,如在委内瑞拉、秘鲁、哥伦比亚、乌拉圭、智利、阿根廷、厄瓜多尔、墨西哥、危地马拉、尼加拉瓜等都有革命者。你们国家旁边的两个国家海地、多米尼加也有革命者。很多共产党跟资产阶级跑。这不要紧,总有革命者起来。古巴即如此。《七月二十六日运动》开始并不是马列主义政党吧?

\item[\textbf{大使:}] 不是。

\item[\textbf{主席:}] 有马列主义者参加,如格瓦拉同志。你(指比达)多大年记?

\item[\textbf{比达:}] 五十四岁。

\item[\textbf{主席:}] 你和格瓦拉的年纪差不多吧?

\item[\textbf{比达:}] 大一些。

\item[\textbf{主席:}] 比罗加呢?

\item[\textbf{比达:}] 小一些。

\item[\textbf{大使:}] 格瓦拉四十五岁,他是古巴革命领导成员中最老的一个。

比达;古巴革命几乎是青年人搞起来的。

\item[\textbf{大使:}] 有点像中国革命,开始时领导同志都很年青。

\item[\textbf{主席:}] 一般说来,年青人比较进步,但并非都比马克思、恩格斯、列宁进步。有许多人到后来不革命了。生活把他们拋到后面去了。他们失去了革命的敏感,害怕革命。其称号为革命党,一谈革命就害怕,这算什么革命党。他们不愿接近人民,接近最贫苦的下层人民,即工人与贫农。我们的革命胜利,我们的政权巩固下来,就是依靠工人和贫农。这两部分人占人口的绝大多数。只有这两部分人团结起来,富裕中农就靠拢了。知识分子也有左、中、右,首先团结左派,中间派就跟着靠拢。右派只要反帝、爱国也可以团结,有暂时的作用,没有他们有时也不行。如大学教授、中小学教员,都是旧社会留下来的,不是共产党员或很少是共产党员。十四年来,一部分经过改造,加入了共产党,一部分还保留自己的老观点。文学艺术工作者也是如此。改造他们要花很长的时间,有小部分人基本上不可能改造。不要紧,他们是少数,让他们带着右派观点去见上帝吧!我不清楚你们国家的情况,可能也有这几类人。

\item[\textbf{比达:}] 完全一样。我们也有同样的斗争,同样的改造过程。我们的革命青年能看到革命前景,在革命胜利前就参加革命,也有的在胜利后参加革命。另有一部分人在观望。\marginpar{\footnotesize 67}

\item[\textbf{主席:}] 让他们看看。北京也有一些人在观望;革命到底谁胜谁负?他们要看看。反修斗争究竟如何?他们也要看看。对国内社会主义建设,他们也要看看,每个大风浪,他们总要动摇。

\item[\textbf{比达:}] 他们的“耐心”太大。

\item[\textbf{主席:}] 他们只好看看,我们也只好看看。他们是少数,我们不怕他们,不欺他们,不杀他们。你(指大使)在北京住了几年?

\item[\textbf{大使:}] 三年。

\item[\textbf{主席:}] 你可以看到,我们很少逮捕人,很少杀人,而用群众监督的办法监督坏人劳动。依靠百分之七十、八十、九十的大多数人民群众去监督百分之一、二、三的人劳动。一般说来,坏人大多数在一定条件下能改造成为好人。

\item[\textbf{张××(以下简称张):}] 比达同志见过溥仪。

\item[\textbf{比达:}] 我正要告诉主席,见过溥仪。

\item[\textbf{主席:}] 我也见过他一次,请他吃过饭。他可高兴啦!

\item[\textbf{张:}] 溥仪今年五十七岁了。

\item[\textbf{比达:}] 他给我的印象是确实改造了。他和我长谈他过去的错误,很真诚。

\item[\textbf{主席:}] 他很不满意他过去不自由的生活。当皇帝是很不自由的。

\item[\textbf{大使:}] 向主席提一个问题;反对帝国主义,保卫马列主义原则的斗争今后的前景如何?

\item[\textbf{主席:}] 现在已看得很清楚,帝国主义斗我们不赢,帝国主义是快要灭亡的阶级。美国出了这样的乱子,即一个总统白天被人打死了。你们国家下了半旗没有?

\item[\textbf{大使:}] 没有。

\item[\textbf{主席:}] 奏哀乐没有?

\item[\textbf{比达:}] 没有。

\item[\textbf{主席:}] 我们没有下半旗,没有奏哀乐,也没有打电报吊唁。

\item[\textbf{大使:}] 我们得到国内的指示:如果中国同志下半旗,我们也下半旗。

\item[\textbf{主席:}] 共产党不赞成暗杀的。这不是一个人问题,是制度问题。你们国家改变了制度,不是换了巴蒂斯塔个人,巴蒂斯塔没有死吧?

\item[\textbf{大使:}] 没有。

\item[\textbf{主席:}] 但他对古巴不起作用了。帝国主义内部矛盾是得不到解决的。首先是工人阶级和资产阶级的矛盾无法解决,除非革命。其次为这一垄断集团与另一垄断集团的矛盾。还有所谓盟国问题,美国与欧洲、北美(加拿大)、日本、澳大利亚、新西兰都有矛盾,国与尼赫鲁、铁托也有矛盾。你(指大使)到过印度吗?

大使;在印度呆过几天,深刻的印象是,印度与中国有鲜明的对比。印度走资本主义道路,中国走社会主义道路。

\item[\textbf{主席:}] 印度人民,百分之九十不赞成尼赫鲁的统治。印度半数以上的人很穷苦,印度的状况比缅甸还差。你(指大使)到过缅甸吗?

\item[\textbf{大使:}] 没有。

\item[\textbf{主席:}] 总之,世界在变,变得对革命有利,对反革命不利。为什么有些修正主义分子对肯尼迪之死如同丧亡了父母一样悲痛呢?这反映了世界不稳,一些依靠资产阶级“明智”派的人,“明智”派一倒就吓慌了。\marginpar{\footnotesize 68}这些所谓的“明智”派就是在你们猪湾(即吉隆滩)登陆的指挥者,也是去年十月加勒比海事件的主持人。吉隆滩事件,艾森豪威尔未做过。只有你们有警觉。你们领导者提出一手拿砍刀(割甘蔗,生产用的),一手拿武器的口号很好,不能放弃这口号。美国奈何不了你们。依我看来,是美国怕你们,不是你们怕美国。是大的怕小的,不是小的怕大的。当然,小的也有点怕的,说一点不怕也不真实。美国飞机每天在上空飞,那么多海军陆战队,又有关塔那摩基地,如何不令人担心。但大局说来,是大的怕小的。在南越,美国与南越统治者的力最不小,可是胜利没有希望,美国与南越统治者怕南越人民的游击战。你(指大使)想不想去南越看看?

\item[\textbf{大使:}] 我去过越南民主共和国。

\item[\textbf{主席:}] 到越南了解越南南方人民如何搞游击战,使美国人如此惊慌。你(指比达)就要回去了吗?

\item[\textbf{比达:}] 明天走。

\item[\textbf{主席:}] 可惜没有抽点时间到越南去看一看。

\item[\textbf{比达:}] 这也需要我再来一次。

\item[\textbf{主席:}] 下次来,到越南北方调查南越游击战的情况。再去看看朝鲜人民共和国。看看他们如何搞自力更生的。他们经过大破坏,一九五〇年、一九五一年、一九五二年、一九五三年,美国把朝鲜炸得稀烂。但是,现在不仅工业,而且农业都解决了问题,真该去看看这两个国家。中国经验当然应该研究。朝鲜过去是日本的殖民地。战后只有七百万人口。被美国搞去了二百多万(战争死亡的不算)。从一九五三年下半年至今,十年来已经完全恢复而且发展了。人口也增加很快,由七百万增加到一千二百万。

\item[\textbf{大使:}] 我们占了毛主席很多时间。主席谈到了美国与社会主义国家的矛盾,美国与民族解放运动的矛盾,美国与盟国的矛盾,美国统治集团内部的矛盾等。美同发生肯尼迪事件,主要因素是内部矛盾。这是否说明我们对帝国主义内部矛盾应予以更大的注意。帝国主义一贯玩弄两手,一手和平,一手战争。有人说肯尼迪之死是主张战争的人于的,当然我不是说肯尼迪是明智派。

\item[\textbf{主席:}] 也可能,但现在搞不清楚。美国不暴露谁杀死肯尼迪。可以设想肯尼迪事件对共产党有利一些,民主党受到一次大打击。共和党有好几派,究竟那一派干的,不知道。现在总统约翰逊能否当选还是个问题。他的威望不及肯尼迪。这个人要钱,名声不好,是个大贪污分子。我国搞五反,他是五反对象。(众笑)

\item[\textbf{张:}] 主席该休息了吧!

\item[\textbf{主席:}] 没有关系,多谈一会儿。总的看来,世界在变,变得对反革命不利,对革命有利。几十年的历史证明了这一点。古巴的变化,你们自己看到了。古巴就在美国门口几十海里,起了变化,谁能说古巴没有起变化呢?!中国变化了,这也是事实,我是知道的,你们也看到了。北京,帝国主义和蒋介石就来不了嘛!非洲也起了很大变化,亚洲也起了很大变化如印尼、柬埔寨等,小小的柬埔寨竟敢拒绝“美援”,为何有此大胆?只因为;第一,美国要他的命,第二,国内人民的觉悟;第三,法国、中国帮助他们。柬埔寨人口跟古巴差不多,有六百万人口。现在发生了怪现象,美国每年“援助” 柬埔寨三千万美元,过去共“援助”了三亿几千万美元,柬埔寨感到美“援”是危险的东西。它在收买干部,腐化干部,并搞颠覆活动。这就可以解释为何柬埔寨拒绝美“援”。好吧,就谈到这里。\marginpar{\footnotesize 69}
\end{duihua}

