\section[在中央政治局扩大会议上的发言(一九五六年四月二十五日)]{在中央政治局扩大会议上的发言}
\datesubtitle{(一九五六年四月二十五日)}


发言中表现神气不足,下级对上级关系像老鼠见了猫一样,灵魂被吃掉了似的,有许多话不敢讲。各省也有此问题,民主不够,但劳动模范讲话有神气。我们财经会议与四中全会有副作用,因为有几条规定,大家有话不敢说。财经会议有些同志发言不恰当,有些同志不敢说。

另一个原因是对情况了解不够,讲得不清楚,不深刻。财经制度××,先念同志都有意见。这不是人的问题,而是制度的问题,制度改变了才能改变作风,要给下面一些权。

我们的纪律多从苏联来,太严了会把人束缚起来,这样不可能打破官僚主义,无产阶级专政就要制度恰当。中央与地方分权问题,苏联制度的一长制是怎样来的,政治局与国务院均未作决定。各地要分权,不要怕说闹独立性,中央并未作决定,均可讲。各地可先搞条例、细则、办法,宪法规定是允许的。要使各地有创造性,神气活泼。从明年起每年搞一次大会。五一不要开大会了,可以开小会娱乐,苏联五一口号不要登。

<p align="center">×××</p>

四中全会以来我们有些呆板,不活泼。四中全会是应当开的,反高、饶是对的,决议是很必要的,否则再让高岗搞一年,是不可设想的。但产生谨小慎微,有的不敢谈国事,是不对的。四中全会前有些破坏现象,需要克服。高、饶破坏活动应当堵塞起来,但有些谨小慎微,莫谈国事,应该分别清楚。有两种国家大事,一种是破坏性的国事,如高、饶应当反对,一种是建设性的国事,应该大谈特谈。财经会议时,有些同志讲错了话,受到了一些批判,但和高、饶之事必须区别。党在大革命时期,是生气勃勃的,后来陈独秀犯了右倾机会主义错误,再后又有“左”倾盲动,失败后就不活泼了。八年中,“八·七”会议后的一个短时期,六次大会后的一个短时期,三中全会后的一个短时期是正确的。土地革命时期有三次“左”倾,抗日时期有右倾,二次“左”倾是自己的,第三次“左”倾与共产国际有关,第三次“左”倾的四中全会的决议是俄国人写的,强迫我们接受,特别是王明路线,对革命力量损失最大,革命力量损失了百分之九十以上。鉴于这个教训。应当分析批判地接受,张国焘是右倾,抗日时期又有右倾。三次右倾非常集中,不准讲不同的话。失败了,不准讲失败,鉴于历史教训,后来又惩前毖后,治病救人,有团结有斗争。抗日时期根据地独立性很大,发挥了地方积极性,但有点分散主义,有的闹独立性,不应发表的发表了,这与王明路线有关。为了纠正这种现象,中央作出了增强党性的决定,有了一元化,但保持了很大的自治权。解放战争时期拟定了请示报告制度,纠正了过于分散的偏向,最近几年不正常,集中多了些,究竟工厂、乡村,合作社、地方要有多大自治权,苏联四十年来还没有经验,我们也没有经验,要研究。有些东西既不是中央决定的,也不是地方决定的,就照办了。如一长制,雪峰同志提出来,中央才讨论的。我们党历史上有王明路线的过于集中,有第二次王明路线过于分散,适当的集中是必要的,但过于集中也是不对的,就不利于调动力量进行大规模的经济建设,请同志们很好地研究党的历史。

<p align="center">×××</p>

个人与国家、集体收入分配的比例问题。我同意总收入的百分之六十到七十归社员,百分之三十到四十归国家和社,最多不能超过百分之四十,最好百分之三十。(包括农业税、附加、合作社公益金、公积金、管理费等)

<p align="center">×××</p>

社会主义经济体制问题:

党委制是无疑的,请研究列宁的指示。厂矿、合作社、商业流通过程、运输等企业要有一定的独立自主,独立到什么程度要很好研究。我们不是高岗的独立王国,但要鼓励公开合法的“独立王国”(不上宪法),才好办事,一点没有是不好办的。国务院如何分工管理,要研究解决,中央设多少部,有多大权力,不久就要决定,地方设多少部门,管理哪些事,有多大权力,几个月内搞出个草案来。中央各部门要注意教育干部给下层解决问题,地方来中央见不到人,拖延不决,有问题几年得不到解决,应查其原因。这两个问题要解决。用什么方式既能见到人,又能迅速解决问题,请中共各部注意。

地方有权抵制中央各部给地方一切行不通的、不合实际的主观主义的命令、指示、训令、表格,多制止了多一些也不要紧,此权只给省、市委(政治上比较成熟),不给地县委。

<p align="center">×××</p>

党章要体现纪律和创造性,群众路线应体现在这里,各省要研究一下,没有纪律不行,但纪律搞得太死会妨碍积极性,妨碍创造性与积极性的纪律应当废除。党章草案中规定设一个副主席或几个副主席,是否可以仿照人代大会常务委员会的办法,设常任代表五年一任,可起监督作用,请大家考虑。

在艺术上“百花齐放”,学术上“百家争鸣”(春秋战国时百家争鸣),应作为我们的方针,这是两千年以前人民的意见。

<p align="center">×××</p>

全国的平衡是需要的,地方的独立性不能妨碍全国的平衡,有全国的平衡,才有地方的部分平衡。没有全国平衡要天下大乱。北京东西不上上海,怕上海货来冲,所以要有全国平衡。没有这个平衡,全国工业化搞不起来。鼓励地方独立性不要偏到一面去。现在强调独立性是必要的。

经济工作今明两年要切实摸一下。每一省、市委都重汇报一次,搞些典型。我们没有经验,将要找些部、区、委、广研究一下。别人说中央英明领导,我们是知道又不知道。

四天会不能什么都谈,肃反、统战、少数民族、国际问题谈得少,请各地注意部署一下,反革命要肯定,过去杀、关、管二、三百万是非常必要的,没有这一手不行。民主党派与我们有分歧意见的,现在反革命是少了,应该肯定还有反革命,今年肯定要杀一些,在机关学校口要清理,没有清出来还要清,不能松劲,要进行艰苦的工作,半个月打一次电话督促。你们对各地委也可以采取打电话的方法。

党内处分:县、区、乡干部处分过多过重,高级干部(包括处长以上)犯错误结论难做、处分一下去(对此要查清原因),过严过重都不对,除反革命外都要给以改正机会。

少数民族问题,井泉同志讲了,四川是对的,虽有些报复已纠正,其他地区也要注意。

国际关系问题。一部分不要盲从,有的我们已有经验。苏联已展开很大批评,有些在我国、在苏联都不适用,我们鉴于他们的垂直领导犯了很多错误,如对肃反,我们就大部不捉、一个不杀。一长制是军事观点,群众路线还是恩赐观点、积累资金办法是剪刀差还是征税都有问题,但不是说苏联没有东西可学了,有很多东西是值得我们学习的,帮助我们建设的是苏联,还是社会主义国家好,现在只有这样一个国家,虽然有那样多的错误,但是值得学习的多,我们不要盲从,应加以分析,屁有香臭,不能说苏联的屁都是香的。现在人家说臭我们也跟着说臭。凡是适用的都要学,资本主义好的也应该学。

对外国任何小国一律要采取平等对待的态度,不要翘尾巴,虽然我们不是帝国主义,没有十月革命,开始翘不起来,但是过早学会了些东西就可能翘尾巴。要教育出国同志,要老老实实,是就是是,非就是非,好的、坏的、中间的都给人家看。苏联有沙皇时代,我们有蒋介石,我国有小脚,别人要照相,让他们照,衣服穿的不好,不怕难看,在外国人面前撒谎,总有一天会揭穿的。传达问题,有些东西任何地方都可以讲,斯大林、第三国际作的坏事可以传达到地委书记,县委书记也可以,不写在文章上是为了照顾大局(这篇文章只写了一句“出了些坏主意”)不准备在报纸上和群众中讲。


