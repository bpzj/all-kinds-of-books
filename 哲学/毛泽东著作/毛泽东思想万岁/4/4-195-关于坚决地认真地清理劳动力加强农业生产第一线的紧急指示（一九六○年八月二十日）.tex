\section[关于坚决地认真地清理劳动力加强农业生产第一线的紧急指示(一九六○年八月二十日)]{关于坚决地认真地清理劳动力加强农业生产第一线的紧急指示(一九六○年八月二十日)}


各省、市、自治区党委,中央各部委各党组:

现将四川省委送来的“周×同志关于南部,仪拢两次农村劳动力问题和‘三反’情况的电话报告”和“南充地委,温江地委关于农村公社浪费劳动力的两个材料“发给你们。请你们立即转发至县委,要他们仔细阅读,切实研究,如果本地也有类似现象,就应该像四川那样,派出专门小组到公社去加以检查和督促,务必采取坚决的办法。把县,社,管理区三级所浪费的不十分必需的劳动力,迅速动员和压缩到生产队中去,加强农业生产第一线,以保证今年能够获得一个较好的秋收。并争取明年有一个比任何一年要好得多的夏收,这件事情刻不容缓,千万不可迟疑不决。

四川这几个材料说明。县、社,管理区三级浪费劳动力是极为严重的。甚至生产队本身也把许多应该用于农业生产第一线的劳动力,过多地使用于其他方面;四川省委和地委找到了这个窍门.进行了调查和清理,结果在五十五万人口的仪拢县,就可清出约四万人(约占全县总劳动力百分之十五以上)。压到农业生产第一线去,南部县建兴公社就可挖出占全公社劳动力百分之二十一的人压到农业生产第一线去。鄂县友爱公社在调整前。生产队一级使用的劳动力只有三千零三十五人,调整后增加到三千六百四十四人,即在农业第一线到增加的劳动力,也在百分之二十以上。从这几个典型材料,就可以说明这个问题的严重性,不认真解决这个问题,农业的大量增产是没有希望的。

四川的材料还揭穿了一个“秘密”,指出“少数基层干部和群众认为农村活路重,生活苦,城镇劳动轻松,又能拿现钱,便想尽一切办法逃避农村”。“城镇占用的劳动力绝大部分是基层干部的父母,爱人,兄弟,姐妹和舅子,老表,亲戚朋友,他们为了个人利益,使自己家庭生活舒适,千方百计不择手段地把他们的家属从农村搬到镇城,安插在机关,工厂,企业,学校,逃避农业生产”,因此,“从这次清理劳动力的情况看。问题是复杂的,斗争是尖锐的”。指出必须在干部和群众中切实树立“以农业为基础”的思想,指出必须放手发动群众。依靠群众力量,才能扫除某些干部包庇自己亲戚朋友逃避农业生产的障碍.上述这些分析是很生动的。是很正确的。中央还要着重指出。在我们国家中,必须经常注意防止形成一种特殊阶层。干部包庇亲戚朋友。安置较好位置。就是这个现象的具体表现的一种,也就是官僚主义的最主要的内容。千万不要忽略。在其萌芽时就要不断地加以批判和克服。不可任其滋长。

四川解决农业第一线劳动力的斗争。是结合农村三反进行的,其他各省区也可以结合三反进行,也可以先对农业第一线劳动力问题作初步的清理,然后再结合三反作深入的解决。总之,清出一切可能的劳动力去加强农业生产,是目前一件很迫切的事情,务必迅速动手,抓紧进行。
<p align="right">中央
一九六○年八月二十日</p>


