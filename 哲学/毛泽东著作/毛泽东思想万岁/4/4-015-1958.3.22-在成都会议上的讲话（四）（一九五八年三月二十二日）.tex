\section[在成都会议上的讲话(四)(一九五八年三月二十二日)]{在成都会议上的讲话(四)}
\datesubtitle{(一九五八年三月二十二日)}


无事不登三宝殿,想到一点问题交换意见。

西厢记中,有一段张生和惠明的故事。孙飞虎围着普救寺,张生要送信请他的朋友白马将军来解围。无人送信。开群众会议,惠明挺身将信送去,这是描写惠明勇敢胆大的坚定之人。希望中国要多点惠明,要在县委委员以上几十万人中发动一下大鸣、大放、大字报批评领导。这是一种无产阶级的气氛,共产主义的气氛。群众骂你一顿出口气,并没砍你的头,撤你的职,这是蓬勃的战斗的情绪。是很高的共产主义的风格。现在群众斗争的风格很好。我们同志之间也要提倡这种风格。

陈伯达写给我一封信,他原来死也不想办刊物,现在转了一百八十度,同意今年就办,这很好。我们党从前有《响导》、《斗争》、《实话》等杂志,现在有《人民日报》,但没有理论性杂志。原来打算中央、上海各办一个,设立对立面有竞争。现在提倡各省都办,这很好。可以提高理论,活泼思想。各省办的要各有特点。可以大部根据本省说话,但也可以说全国的话,全世界的话,宇宙的话,也可以说太阳、银河的话。

地方工作同志,将来总是要到中央来的。中央工作的人总有一天非死即倒的。赫鲁晓夫是从地方上来的。地方阶级斗争比较尖锐,更接近自然斗争,比较接近群众。这是地方同志比较中央同志有利的条件。秦国称王在后,但是称帝在先。

要提高风格,讲真心话,振作精神,要有势加破竹,高屋建瓴的气概。要作到这一点,必须抓住马克思主义的基本理论和工作中的基本矛盾。但我们的同志现在并不企图势如破竹,有精神不振的现象,这很不好,是奴隶状态的表现,像贾桂一样,站惯了,不敢坐。对于经典著作要尊重,但不要迷信。马克思主义本身就是创造出来的,不能抄书照搬,在这一点上,斯大林比较好一点。联共党史结束语说:“马克思主义个别原理不合理的,可以改变。如一国不能胜利(按:应指社会主义不可能在一国内首先取得胜利)。”中国的儒家对孔子就是迷信,不敢称孔丘。唐朝李贺就不是这样,对汉武帝直称其名,曰刘彻、刘郎,称魏人为魏娘。一有迷信就把我们脑子镇压住了,不敢跳出框子想问题。学习马列主义没有势如破竹的风格,那很危险。斯大林也称势如破竹,但有些破烂了。他写的语言学、经济学、列宁主义基础是比较正确的,或基本正确的。但有些问题值得研究。例如,在社会主义阶段中,价值法则的作用如何?是否拿劳动准备时间消耗多少来定工资的高低?在社会主义中,个人私有财产还存在,小集团还存在,家庭还存在。家庭是原始共产主义后期产生的,将来要消灭,有始有终。康有为的《大同书》即看到此点。家庭在历史上是个生产单位、消费单位、生下一代劳动力的单位、教育儿童的单位。现在工人不以家庭为生产单位,合作社中的农民也大都转变了,农民家庭一般为非生产单位,只有部分副业。至于机关、部队的家庭,更不生产什么东西,变成消费单位、生育劳动后备并抚育成人的单位。教育部门的主要部门,也在学校。总之,将来家庭可能变成不利于生产力发展的东西。现在的分配制度是(按劳分配)付酬,家庭还有用。到共产主义分配关系是变为各取所需,各种观念形态都要变,也许几千年,至少几百年家庭将要消灭。我们许多同志对于这许多问题不敢去设想,思想狭窄得很。这些问题经典著作上已经讲过,如阶级、党的消灭等,这说明马列风格高,我们很低。

怕教授,进城以来相当怕,不是藐视他们,而是有无穷的恐惧。看人家一大堆学问,自己好像什么都不行。马克思主义者恐惧资产阶级知识分子。不怕帝国主义,而怕教授,这也是怪事。我看这种精神状态也是奴隶制度:“谢主龙恩”的残余。我看再不能忍耐了。当然不是明天就去打他们一顿,而是要接近他们,教育他们,交朋友。他们自然科学可能多学一点,但社会科学就不见得。他们读马列主义比我们多,但读不进去,懂不了。如吴景超读了很多书,一有机会就反马克思主义。

不要“自惭形秽”,伯恩斯坦、考茨基、后期的普列汉诺夫,马列主义比我们读得多,但他们并不行,把第二国际变成了资产阶级的仆从。

现在情况已有转变,标志是陈伯达同志的一篇演说(厚今薄古)、一封信(给主席的),一个通知(准备下达)有破竹之势,但有许多同志对于思想战线上的斗争无动于衷,如批判胡风、梁漱溟、《武训传》、《红楼梦》、丁玲等。本来,消灭资产阶级的基本观点,在七届二中全会的决议中已经有了。在过去民主革命中,就经常讲革命分两个阶段,前者为后者的准备。我们是不断革命论者,但许多同志对于什么时候搞社会主义革命,土地改革后搞什么都不去想,对社会主义萌芽熟视无睹。而社会主义萌芽早已诞生。比如在瑞金、在抗日根据地,就产生了社会主义萌芽互助组。 注:“瑞金”原文为“瑞金”

王明、陈独秀是一样的。陈独秀是主张让资产阶级革命成功以后,让资产阶级掌握政权,然后壮大无产阶级再搞社会主义革命。所以陈独秀不是马列主义者,而是资产阶级民主革命的激进派。但是,经过三十多年,还有这样的人。坏人如丁玲、冯雪峰。好人如×××完全是资产阶级民主派那一套。搞“四大自由”,讲农民怕冒尖,他就跟我尖锐对立。河南的富裕中农有好东西不让干部看,装穷,无人时,才向货郎买布。我看很好,这表示贫下中农威力很大,使得富裕中农不敢冒尖。这说明社会主义大有希望。但有些人认为不得了,要解除怕冒尖的恐惧,即大出布告,搞“四大自由”。既不请示,也不商量,这明明是和二中全会方针作对。他没有搞社会主义的精神准备。现在被说服了,积极了。

从古以来,创新思想、新学派的人,都是学问不足的青年人。孔子二十三岁开始。耶苏有什么学问!释迦牟尼十九岁创佛教,学问是后来慢慢学来的。孙中山青年时有什么学问?不过高中程度;马克思开始创立辩证唯物论,年纪也很轻。他的学问也是后来学来的。马克思写《共产党宣言》时,不过三十岁左右,学派已经形成了。在开始着书时,只有二十几岁。那时,马克思所批判的都是一些当时的资产阶级博学家。如:李嘉图、亚当斯密、黑格尔等。历史上总是学问少的人推翻学问多的人。章太炎青年时代写的东西是比较生动活泼的,充满民主革命精神,以反满为目的。康有为也是如此,刘师培成名时还不过二十岁,死时才三十岁。王弼注《老子》的时候,不过十几岁,因用脑过度早死。死时才二十几岁。颜渊(二等圣人)死时才三十二岁。李世民起义时,只有十几岁,当了总司令,二十四岁登基当了皇帝,年纪不甚大,学问不甚多,问题是看你方向对不对。秦叔宝也是年轻的。年轻人抓住一个真理,就所向披靡,所有老年人是比不过他们的。罗成、王伯当都不过是二十几岁。梁启超年轻时也是所向披靡,而我们在教授前就那么无力,怕比学问。刊物出后,方向不错,就对了。雷海宗读了本马列主义不如我们,因为我们是相信马列主义,他越读得多还当右派。现在我们要办刊物,要压倒资产阶级知识分子。我们只要读十几本书就可以把他们打倒。刊物搞起来,就逼着我们去看经典著作,想问题,而且要动手写。这就可以提高思想。现在一大堆刊物吸引了我们的注意力。不办刊物大家也不会去看书,尽讲抽象不算红。

各省可办一个刊物,成立一种对立面,并且担任向中央刊物发稿的任务,每省一年六篇就够了。总之,十篇以下,由你们去组织,这样会出英雄豪杰的。

从古以来,创新学派、新教派都是学问不足的青年人,他们一眼看出一种新东西,就抓住向老古董开战!而有学问的老古董,总是反对他们的。马丁·路德创新教,达尔文主义出来后,多少人反对!发明安眠药的,既不是医生,更不是有名的医生,而是一个司药的。开始,德国人不相信,但法国人欢迎,从此才有安眠药。据说盘尼西林是一个染坊洗衣服的发明的。美国富兰克林发明了电,他是卖报的孩子,后来成了传记作家、政治家、科学家。高尔基只读了两年小学。当然学校也可以学到东西,不是把学校都关门了,而是说不一定住学校。看你方向对不对,去不去抓。学问是抓来的。从来创立学派的青年,一抓到真理,就藐视古董,有所发明。博学家就来压迫。历史难道不是如此吗?我们开头搞革命,还不是一些娃娃,二十多岁。而那时的统治者袁世凯、段琪瑞都是老气横秋的,讲学问,他们多,讲真理,我们多。

我很高兴,最近时期大字报很有气魄,批评得尖锐(性)、生动(性)。把暮气一扫而光,但我们老是四平八稳走方步,“逢人只说三分话,未可全抛一片心”,不讲真心话。

王鹤寿第二篇文章敢于批评教条主义,彭涛的也好。有说服力。尖锐性差一点,无非是“打击别人提高自己”,但不是个人主义的打击别人抬高自己。为了打击错误思想,提高正确思想,是完全必要的(当然错误中也包含自己的错误)。滕 注:原为“膝” 代远那一篇也好,但说服力不够。修那么多铁路要说出理由来,不然就把别人吓倒了。张奚若批评我们“好大喜功,急功近利,鄙视既往,迷信将来”。无产阶级就是这样嘛,任何一个阶级都是好大喜功的。不“好大喜功”,难道“好小喜过”?禹王惜寸阴,我们爱每一分钟。孔子“三日无君则惶惶如也。”孔子“席不暇暖”。墨子“实不得黔”。这都是急功近利。我们就是这个章程,水利、整风、反右派、六亿人口搞大运动,不是好大喜功吗?我们搞平均先进定额,不是急功近利吗?不鄙视旧制度,反动的生产关系,我们干什么?我们不迷信社会主义、共产主义干什么?

我们错误是有的,主观主义也是有的,但是“好大喜功,急功近利,鄙视既往,迷信将来”是正确的。天津、南京两封信虽然是反对我的,但精神可取,我看是好的。天津的更好。南京的萎靡不振,骨气不硬。陈其通等四人,除陈沂是右派分子,这些人敢于说话的精神是可取。当面不说,背后唧唧咕咕,这是不好。应该大体一致,至少要基本一致,可以尖锐一点,也可以委婉一点,但不能不说。有时要尖锐鲜明,横竖是团结帮助的态度出发,尖锐的批评不会使党分裂,只会使党团结,有话不说,就很危险。当然,说话要选择时机,不讲策略也不行。例如明朝的三大案,反魏忠贤的那样不讲策略,自己被消灭。当时落得皇帝不喜欢。一个四川人杨慎安被充军到云南。历史上讲真话的如:比干、屈原、朱云、贾谊等这些人都是不得志的,为原则而斗争的。不敢讲话无非是“一、怕扣为机会主义,二、怕撤职,三、怕开除党籍,四、怕老婆离婚(面上无光),五、怕坐班房,六、怕杀头。”我看只要准备好这几条,看破红尘,什么都不怕了。没有精神准备。当然不敢讲话.难道牺牲可以封住我们的嘴巴吗?我们应当要造成一种环境,使人家敢于说话,交出心来。苏共十九次代表大会报告说:“要造成一种环境”。这对群众来说是对的。先进分子应该不怕这一套,要有王熙凤的“舍得一身剐,敢把皇帝拉下马”的精神。

我们应当领导群众,现在群众比我们先进,他们敢于贴大字报批评我们。当然和储安平不同,那是敌人骂我们,现在是同志之间的批评。我们现在有些同志的作风不好,有些话不敢讲,只讲三分。这是一怕不好混,二怕失选票。这是庸俗作风,要改变,现在已有可能改变。

一九五六年吹掉三个东西一一多快好省,促进派,四十条。有三种人,三种心理状态,一种是痛心的,一种是漠不关心的,再一种是吹掉高兴。一块石头落地,从此天下太平。这三种态度的人,两头小中间大。一九五六年有许多问题,都有这三种态度,反日、反蒋、土改是比较一致的,但是在合作化的问题上要有三种态度。这种估计是不是对,这次会议解决了一批问题,取得协议,为政治局准备了文件,但缺点是思想谈得较少,是否用两、三天的时间谈谈思想问题,谈谈心里话?

同志们说这次会议是整风会议,又不谈思想,实践诺言,是否有矛盾?一不搞斗争,二不划右派,和风细雨,把心里话讲出来,我的企图是要人们敢说,精神振作,势如破竹,像马克思、鲁迅那样,敢说,把顾虑解除,要在地委书记约在两、三人的范围内把空气冲破一下。搞出一种新气氛。邹容十八、十九岁写了一篇《革命军》,直接骂皇帝。章太炎写文章驳康有为也是精神百倍,年纪越大用处越不多,但也不要妄自菲薄,要鼓点劲。当然,年纪大的也还要,也要掌舵。三国时刘备不好,还是老头子挂帅。要冲破党内的沉闷气氛。

印了一些诗,尽是老古董。搞点民歌好不好?请各位同志负个责任,回去以后,搜集点民歌,各个阶层、青年、小孩都有许多民歌,搞几个试点,每人发三、五张张纸写写民歌,劳动人民不能写的找人代写,限期十天搜集,会收到大批旧民歌,下次会印一本出来。

中国诗的出路.第一条、民歌,第二条、古典。在这个基础上产生出新诗来,形式是民歌的,内容应当是现实主义和浪漫主义的对立统一。太现实了就不能写诗了。现在的新诗不成形,没有人读。我反正不读新诗。除非给一百块大洋。搜集民歌的工作.北京大学做了很多.我们来搞可能找到几百万成千万首的民歌。这不费很多的劳力,比看杜甫,李白的诗舒服一些。


