\section[反攻(一) ]{反攻(一) }


三三三一制

“何谓共产?财务充公。不准私蓄,贫富皆穷。造作好语,名目重农。三三三一,其实不通。三成地主,三成充公。已得三成,一成会用。”这是杨坤茹贴在惠州城厢外的“反共产”四言谈话告示。一个新从北京来而头脑颇算得清楚的朋友问我道;“三三三一制到底有这么回事没有?”我听了纳罕。我尚未答他时,心里想:难道你也有点疑惑或者广东竟行了什么“三三三一制”么?我便带着冷峭的神气答他道:“有是有,不过还在杨坤茹的告示上。”他说:“香港晨报不是国民党的报?那上头也载着。”于是我才知道所谓香港晨报也有这种新鲜议论。原来香港晨报在杨刘叛变以前确与国民党有过关系。杨刘叛变香港罢工以后便为刘震寰买了去做了香港洋大人和所谓陈总司令也者的机关,利用省港断绝交通,就在香港那荒岛上香港晨报那篇幅上涌出了许多的“事实”,三三三一不过那许多“事实”中的一个,却不料远远地送到了北京饱了北京朋友的眼福。我因此想:“现今世界上的东西无论是那动物(如人之类)或静物(报纸之类),不必看得拘限。因为现在“革命”“反革命”分二家,那些动静物件,可以今天在这家,也可以明天在那家。香港晨报在国民党时,做了国民党的机关报;他被香港洋大人刘震寰先生与夫所谓陈总司令也者买了去,便只能算做他们的机关报。这正如冯自由马素一班人在国民党时,是国民党员;及被段祺瑞买了去,虽然咀上还说国民党,招牌也是挂的“国民党俱乐部”,然而只算是段祺瑞的人了。我不是特别有恨于香港晨报和冯马几位先生,我是不得已要借光这个例子点破给国内外许多朋友看,于局人察物时别要上当。至于杨坤茹的告示做的词意俱佳,只有“其实不通”一句,算是败笔,所谓一粒老鼠屎点坏一锅汤者非耶?但那是秘书的过。

杨坤茹的布告与刘志陆的电报

杨坤茹的布告已经领教了,还有刘志陆等行“北京段执政,各部总次长,奉天张督办,湖南赵省长,武昌萧督办,江西方督理,福建周督理,岑西林先生,吴子玉先生,康南海先生,梁任公先生”许多人的电报,那上而却稍不同。信那“其罪七”中说:“我国社会,素称重农,主佃利益,所得素均,互助精神,自然而和。今乃诱以均地之说,乱其互利之序。”不免与杨坤茹冲突。照杨坤茹说:“三成地主,三成充公,己得三成,一成会用。”各方面都得一点,倒可以说是“互利”。刘志陆说的却是“均地”,其意谓广州政府教农民从地主手里夺取土地都均分了,地主从此没有租收,所以是“乱其互利之序”。一个说地主还有三成,一个说地主一点没有,未知北京等处朋友,到底相信那个的话?

如果讨赤同志仇雠亦吾良友

刘志陆等的电报,于数了国民政府八罪之后,加上“呜呼”一段,接着说:“总之出师靖难,伐罪救民,如果讨赤同志,仇雠亦吾良友,师直为壮,胜负无待交绥。得天者昌,仁暴不难立判”。刘电列衔诸人如段执政各部总文长张萧赵方周齐诸督办督理或省长以及岑吴康粱诸先生,固然是讨赤同志。但香港×总督伦敦包经理,何尝不是讨赤同志?却不将台衔列上去。而且×总督助了如许金钱军械,又保护了陈总司令在香港开设总司令部,讨赤之志,炳若日星,却偏将台街漏列,真不知是何用意?电末师直为壮数句说的更属糊涂,他竟在替国民政府做功颂德。

颂声来于万国

有个潮梅绅学商联合会打了一个响应刘志陆的电报,里面说:“共产妖党,颁惑粤东,近之足以陷中国于万劫不复之地,远之足以酿世界以灭绝人类之忧。我刘公仗义发难,为天下先。行见旌旄所指,浆食争迎,拉朽摧枯,大功立就(记者按立字有弊).赤化赖以铲除,国基由兹奠定。岂特与云南起义马厂誓师先后鼎足,抑且全世界人类赖以保障,全球劫运赖以挽回,记念垂于千秋,颂声来于万国。”处处不忘“全世界”“全球”,眼光何等远大!刘志陆如果真把“赤化”铲除,“颂声”之来,一定可靠。万国不得而知,但至少有下面四国:英国、美国、法国、日本国。

“反共产中国国民军大同盟万岁”

这是陈炯明在东江所发反了生产印刷品上而的一句口号。这句口号确实响亮,只可惜“反共产中国国民军”要去“大同盟”有点为难;象奉天张督办所部与汉口吴子玉先生所部,实在算得“反共产小国国民军”了,但是何从“大同盟”呢?

共产章程与实非共产

一般反革命党以国民革命指为共产革命,以国民党指为共产党,以国民政府指为共产政府,以国民革命军指为共产军,无非承了帝国主义意旨,制造几个简单名词散布出来,企图打破国民革命中各阶级合作的联合战线。然此种制造,只能比较抽象,不能太具体,太具体便容易自己戳穿使人不信,然此次陈炯明在东江,为了孤注一掷以图幸胜竟甚么方法都用尽了,他竟揑造所理“共产章程”以恐吓人民。他们的传单中有一张题目叫做“劝粤民协助粤军讨赤伐党”,那里面说:“呜呼,我父老兄弟知蒋中正拟订之共产章程乎?区区之愚,窃虑责氓无知,以为共产者共富人之产,于一股贫民无与也,或更大有造于贫民也。而仰知大谬不然。吾综观章程,概括言之:有所谓三三三一制者,盖对于产田言之也。有所谓四四二制者,盖对于房屋言之也。而对于工厂商场资本科学者,则更完全没收之。”但近日香港工商日报却载道:“省商代表到港,港商代表假座华商俱乐部,请省商代表开第二次大士议商磋解央工潮恢复交通事宜,华人绅商与省商代表皆围坐于一长桌。省商代表简琴石起言,省城政府实非行共产”。假如有人拿了简琴石的话去问陈炯明,我料陈炯闲必将答道:“简琴石自己扯谎,别人共了他的产他还说未共呢。”

邹鲁与革命

邹鲁说:“我们国民党的同志应该觉悟,切不可因人家说旧同志不革命便抹杀一切。若不是我们同志屡仆屡起,那能有民团。若不是讨袁拥法讨贼北伐诸役一向无前,那能有今日的历史。这就是共产得意的工作焚毁商团,乃须要靠杨刘。打杨刘仍须要靠许梁。即今日之制许梁,又何莫非老同志。”很好!邹先生,就请你革命罢!实在没有一个人有本领敢于抹杀革命的老同志!须知单是“有民国”“有历史”是不能算数的,须得现在还是革命,须得将来还是革命。若那老同志杨刘,老同志许梁,我看赶是少举例为好。

<p align="right">
(一九二五年十二月五日《政治周报》第一期)</p>

