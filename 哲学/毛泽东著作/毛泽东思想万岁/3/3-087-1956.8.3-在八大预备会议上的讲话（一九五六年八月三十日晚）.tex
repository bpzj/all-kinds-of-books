\section[在八大预备会议上的讲话(一九五六年八月三十日晚)]{在八大预备会议上的讲话(一九五六年八月三十日晚)}
\datesubtitle{(一九五六年八月三十日)}


大家都没有言发了,我来发个言。

今天是开今预备会议。(邓)××同志已经讲了,要讨论和修改几个文件。

(一)

这次大会的目的,要解决的问题是要总结七大以来的经验,团结国内外一切积极因素,使我们的工作作得更好些,为了建设社会主义国家而奋斗。总结经验和修改党章可以使我们的工作作得更好。

我们团结全国一千多万党员,要进行广大的教育、说服、团结工作。

过去,外国同志不了解我们对待民族资产阶级的政策和整风运动,后来逐渐地有了了解。现在全世界上都了解我们党是正确的,一般外国党的同志是了解我们的。我们作了两件事情:一个是新民主主义革命,一个是社会主义革命。就是资产阶级也不能否认我们的成就。

俄国的资产阶级是一个反革命的阶级。同等他不干,他反对苏维埃政权,消极怠工,进行破坏活动,与中国的民族资产阶级不同。

美国不让新闻记者来,就是杜勒斯也承认我们有这些好处,(即成绩、长处),以至他们不敢派人来。(笑声)

这次会议有四十九个国家兄弟党的代表来参加我们的会议,这很好。

过去我们没有执政,大家看不起我们。现在我们执政了,水平也提高了,大家看得起我们。外国同志很尊重我们的党。过去外国人说我们是东亚病夫,不会打球,不会游泳,连月亮都不好,有辫子,有小脚,鲁莽得很等等。认为一切都不行。现在我们翻身了。但单有党不行,党是核心,必须要有群众,要依靠群众。百分之九十的工作是非党员干的,譬如梅兰芳,周××都不是党员。我就不会唱戏,在座的同志也许有会唱的。因此要好好团结全党和非党员。团结工作中还有许多毛病,要搞好。要团结一切可以团结的力量。在国外要团结苏联和各人民民主国家——全世界各国。团结全国、全党、全世界一切积极因素。为什么呢?为了建设强大的社会主义国家。可以说应该说是伟大吧?六亿人口的国家在地球上只有一个,就是我们。党是伟大的,革命是伟大的,国家是伟大的,建设是伟大的。过去人家看不起我们是有点理由的。那时蒋介石只有几万吨钢,而我们现在就有四百多万吨钢,第二个五年计划可以达到一千万吨以上。在十一年以后可以达到一千万吨以上,变为世界上几个强国之一。世界上年产二千万吨以上的国家是不多的。不仅如此,再过几十年以后,我们可以超过美国,而且应该超过美国,否则,我们六亿人口干什么呢?美国在六十年以前也只有四百万吨钢,我们现在有四百五十万吨,比美国落后。如果我们几十年赶不上,就要杀掉一批人,要开除地球籍。否则,我们就对不起世界各国,对世界的贡献不大。

(二)

在思想作风方面——我们要克服主观主义、宗派主义。还有很多主义吧?(问邓××)中央文件上还有个官僚主义,我们这里只讲主观主义、宗派主义,这是经常扫清又兴起,又扫清又兴起,到一万年还是有的。

为什么斯大林会犯错误呢?就是由于主观主义,主观与客观不符,不从客观实际出发。但是很多文件没有好好谈到这一点。我们的文件,要尽量克服主观主义,一定要尽可能的合乎实际情况。过去我们中国革命,因为主观主义带来了很大损失,革命力量失掉了百分之九十,经过三九年整风才解决了这个问题。马克思主义的基本思想就是要主观与客观相结合,政治与实践相结合,否则就不是真正的马克思主义。

宗派主义也是由于只看局部。现在只谈地球上。只要有利于团结,我们就要同意。要团结全党、全国、全世界的积极因素,要团结几十个国家的共产党,首先是苏联,不能因为斯大林犯了错误就不学苏联了,这是不对的。人家搞了几十年,不能不犯错误,他的错误只是一部分,他的主要方面是正确的。这是主流。这是世界上第一个社会主义国家,就是有了马列主义,变成了强大的社会主义工业国。但是也不要认为苏联一切都是香的。苏联自己也说有的是臭的。我们过去也有陈独秀、张国焘、李立三、王明、高岗、饶漱石,他们是我们的教员。但是最大的教员还是蒋介石和帝国主义,他们用枪炮教育我们,——必须要搞好团结。这里讲的团结是自己的对手,打手,而且是错误的方面。(正确的对手就不要说了)骂过自己是机会主义的人,交过手的人,不是与自己意见一致的人。与自己意见完全一致的人,就没有什么团结的任务。要积极地帮助教育犯错误的对手改正错误,旁观是消极的作法。

(三)

八大中委多少人合适?我认为一百五十至一百七十比较妥当。苏共中委近三百人,他们的党年龄长一些,我们等两年再说。人太多了,也不好办。演连环套就是人多一些,我看过连环套。七大七十人再加一倍,多一点比较合适。

三八年是我们实际工作很主要的基础,但是人不多。不好安排,也是个问题。我看还是……吧!五年以后再谈。

大家要提二百八十,一百九十,二百,都是二百人的,权力在代表身上,你们要提三百人也可以。我也是代表,只有一票也没有办法。

上届中央没有辜负七大的委托,作了很多工作,有成绩,没有过大的问题。其中有个别同志的成绩,也很难作如此估计,如王明有病,不能用脑筋,一写检讨就要害病,不能写。可能真有病,也可能没有病。王明写过一次检讨,承认了中央路线的正确,但是不久他又不承认了,我说不承认可以收回检讨,但是他不收回。李立三同志是比较谅解王明的。

王明和李立三同志提不提呢?大家可以考虑。王明同志这次有病,不能参加会议,七大是选了他们的。这次我们也可以像七大一样选他们。我们不选犯错误的人,那就是以他们的办法办事,他们就是把和自己斗过的人不要。

最基本的就是他们不是一个个人,他们代表一部分小资产阶级。小资产阶级是动摇的,高兴起来可以发狂,愁起来垂头丧气。中国是个小资产阶级比重很大的国家,我们党的农民的比重也是很大的。党内知识分子出身的就占一百万,这一百万党员,放在资产阶级不恰当,放在无产阶级也不适合,放在小资产阶级最好,他们主观主义多,宗派主义也不少。我们选他们两个人,是表示我们对这种犯思想错误的和对反革命与分裂派(如陈独秀、张国焘、高岗、饶漱石)有所不同的。陈、张、高、饶等有政治纲领,王、李也有,但他们和高岗、饶漱石不同,这是有社会原因的。动摇分子,机会主义,今天这样,明天那样,是无原则的。如王明,先犯“左”后犯右的错误。张国焘也有。他最近在香港听说想回来,回来也好,蒋介石少一个人。没有党籍还可以有几个籍。

小资产阶级在紧要关头就要动摇的。王明有时右得不得了,有时“左”得不得了,七大选了王明,李立三,十一年来对我们的工作并没有什么影响。今天的水灾大概不是因为他们仍然选上了中委吧?

选了他们,是不是大家都会去学王明呢?不会的。选了王明、李立三,就是让出两个位子来。把两位正确的同志或有小错误的同志让给他们。位子让出来是不公的,是这样的。但是,另一方面又不是这样的。能不能说:“正确的不如错误的,犯小错误的不如犯大错误的”。问题不在这里。道理是他们在世界上出了名,我们选了他们,全世界都会说:共产党在等待他们。外国都知道,别的国家像我们这样对待犯错误的同志是很少的。这次会议虽然筹备仓促,但还有十多天时间,一定要开好。……大家要竞竞业业,看文件,选举,发言,好好的加以组织。依靠大家努力,会才能开好。


