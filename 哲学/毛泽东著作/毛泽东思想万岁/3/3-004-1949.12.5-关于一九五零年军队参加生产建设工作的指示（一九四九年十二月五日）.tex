\section[关于一九五零年军队参加生产建设工作的指示(一九四九年十二月五日)]{关于一九五零年军队参加生产建设工作的指示}
\datesubtitle{(一九四九年十二月五日)}


人民解放战争已在全国范围内取得基本的胜利,人民解放军除在几个战线上必须专心致志追歼残敌以竞全功之外,已有大量军队进入整训,或不久即将进入整训。人民政治协商会议共同纲领中规定:“中华人民共和国的军队,在和平时间,在不妨碍军事任务的条件下,应有计划的参加农业和工业的生产,帮助国家的设建工作。”这里给了我人民军队除了保卫国防、巩固治安和加强整训这些伟大任务以外的一个光荣而巨大的任务。因此人民革命军事委员会号召全军,除继续作战和服勤务者外,应当负担一部分生产任务,使我人民解放军不仅是一支国防军,而且是一支生产军,借以协同全国人民克服长期战争所遗留下来的困难,加速新民主主义的建设。

这一生产任务是必须而且可能实现的。

这一生产任务之所以必须实现,是由于国内外反动派所发动的长期的反对中国人民的战争,给了人民以严重的灾难,给了经济以严重的破坏。我们今天将革命战争进行到底,

要医治长期战争遗留下来的创伤,要从事经济的文化的国防的各种建设工作。国家的收入不足,开支浩大,这就是我们今天所遇到的一项巨大困难。克服此种困难的方法,首先是全国人民在中央人民政府的领导下逐步的恢复与发展生产。而人民解放军则必须负担一部分生产任务,方能和全国人民一道共同克服此种困难。

这一生产任务之所以可以实现,是因为人民解放军绝大多数来自劳动人民中间,有着高度的政治觉悟和各种生产技能,并曾在抗日战争最艰苦的年月,担负过生产任务,具有生产的经验与劳动的传统。人民解放军的广大的干部们和老战士们都懂得,军队在参加生产之后,不仅战胜了困难,减少了政府的开支,改善了军队的生活,并且经过劳动锻炼,还提高了军队的政治质量,改善了官兵关系和军民关系。这一生产任务之所以可以实现,还因为在战争结束了的地区,人民解放军除了担负保卫国防,肃清土匪巩固治安,加强训练等项任务之外,已有余裕时间参加生产建设工作。所有这些,都是人民解放军能够实现生产任务的条件。

人民解放军参加生产,不是临时的,应从长期建设的观点出发。而其重点,则在于以劳动增加社会和国家的财富。因此,各军区首长,必须指导所属,从一九五零年春季起,实行参加生产建设工作,借以改善自己的生活,并节约一部分国家的开支。此种生产建设工作,应形成一种运动,以利推广。此种生产运动应订出较长期的计划和具体的步骤。生产项目应在人民政府允许下,以农业、畜牧业、渔业、水利事业、手工业,各项建筑工程,各项可能从事的工业和运输事业为范围,禁止从事商业。军队领导机关应根据件地情况,调查研究,今冬做好准备。

根据过去的经验,军队的生产运动,必须严格禁止开商店从事商业行为。干部中如有企图走私、囤积、投机,企图暴利的思想或发现此种行为,必须迅速予于纠正和制止。因为这些不仅违反正确的生产方针搅乱经济秩序,而且势必发生贪污腐化,毁坏自己同志,为法令所不容许。此外,在进行农业生产时,必须注意因开荒引起水患,不要因争地引起人民不满。

为了使军队正确执行生产任务,开展生产运动,兹规定:

(一)在师、军以及军分区以上各级,成立有司令部政治部后勤部代表参加的生产委员会,其任务为掌握生产方向,审定生产计划,监督生产计划的实施,检查违法行为。

(二)建立军队的生产合作社,建立合作社的各级领导机关,在军队生产委员会的监督和领导之下,掌管全部生产资金,生产活动和生产结果的处理。合作社系统和军队指挥系统平行,相互密切联系,但不相混淆。

(三)实行公私兼顾原则,公平合理的分配生产红利,其中应有百分之四十为生产者个人所有,其余为该生产单位及国家所有,借以建立公私革命家务,一方面作到军队部分自给,一方面使生产者个人有所收获。此项个人收获,或自己留用,或寄回家用,或存储合作社备用,由个人自己决定。

(四)在土地缺少地区,除参加各种可能的手工业,工业,水利事业,运输事业和建筑工程之外,军队首长可和当地人民政府商量并在农民自愿的原则下,参加劳力、资金、肥料、农具、与农民移种,使之增加产量,公平分配成果,但必须注意不得强制执行,不得与民争利。

(五)各军区部队的生产计划,须与各大行政区和各省人民政府的生产计划相结合,统筹生产资金。所有军队生产资金均作为投资,必须计算利息和订定还期。所有军队生产事业均须照章纳税,并遵守人民政府的一切法令,不得违犯。

上述各项,望各军区首长严格注意,务使我人民解放军一九五。年的生产建设工作获得显着的成绩,并随时检查,纠正可能发生的错误和缺点。各地人民政府对当地军队的生产工作则有予以指导和协助的任务。

<p align="right">主席毛泽东

一九四九年十二月五日

(抄自新华月报第一卷第三期)</p>


