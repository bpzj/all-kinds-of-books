\section[在杭州会议上的讲话(一九六五年十二月二十一日)]{在杭州会议上的讲话}
\datesubtitle{(一九六五年十二月二十一日)}


这一期《哲学研究》(指一九六五年第六期工农兵哲学论文特辑)我看了三篇文章。

你们搞哲学的,要写实际的哲学,才有人看。书本式的哲学难懂,写给谁看?一些知识分子,什么吴晗啦,翦伯赞啦,越来越不行了。现在有个孙达人,写文章反对翦伯赞所谓封建地主阶级对农民的“让步政策”。在农民战争之后,地主阶级只有反攻倒算,哪有什么让步?地主阶级对太平天国就是没有什么让步。义和团先“反清灭洋”,后来变为“扶清灭洋”,得到了慈禧的支持。清朝被帝国主义打败了。慈禧和皇帝逃跑了,慈禧就搞起“扶洋灭团”。《清宫秘史》有人说是爱国主义的,我看是卖国主义的,彻底的卖国主义。为什么有人说它是爱国主义的?无非认为光绪皇帝是个可怜的人,和康有为一起开学校、立新军,搞了一些开明的措施。

清朝末年,一些人主张“中学为体、西学为用”,“体”,好比我们的总路线,那是不能变的。西学的“体”不能用,民主共和国的“体”也不能用。“天赋人权”、“天演论”也不能用,只能用西方的技术。当然,“天赋人权”也是一种错误的思想。什么“天赋人权”?还不是“人”赋“人权”。我们这些人的权是天赋的吗?我们的权是老百姓赋予的,首先是工人阶级和贫下中农赋予的。

研究一下近代史,就可以看出,哪有什么“让步政策”?只有革命势力对于反动派的让步,反动派总是反攻倒算的。历史上每当出现一个新的王朝,因为人民艰苦,没有什么东西可拿,就采取“轻摇薄赋”的政策。“轻榣薄赋”政策对地主阶级有利。

希望搞哲学的人到工厂、农村去跑几年,把哲学体系改造一下,不要照过去那样写,不要写那样多。

南京大学一个学生,农民出身,学历史的。参加了四清以后,写了一些文章,讲到历史工作者一定要下乡去,登在南京大学学报上。他做了一个自白,说:我读了几年书,脑子连一点劳动的影子都没有了。在这一期南京大学学报上,还登了一篇文章,说道:本质就是主要矛盾,特别是主要矛盾的主要方面。这个话,我也还没说过,现象是看得见的,刺激人们的感官。本质是看不见的,摸不着的,隐藏在现象背后。只有经过调查研究,才能发现本质。本质如果能摸得着,看得见,就不需要科学了。

要逐渐地接触实际,在农村搞上几年,学点农业科学、植物学、土壤学、肥料学、细菌学、森林学、水利学等等。不一定翻大本子,翻小本子,有点常识也好。

现在这个大学教育,我们怀疑。从小学到大学,一共十六、七年,二十多年,看不见稻、粱、麦、黍、稷,看不见工人怎样做工,看不见农民怎样种田,看不见怎样做买卖,身体也搞坏了,真是害死人。我曾给我的孩子说:“你下乡去,跟贫下中农说,就说我爸爸说的,读了几年书,越读越蠢。请叔叔伯伯、兄弟姐妹作老师,向你们来学习。”其实入学前的小孩子,一直到七岁,接触社会很多。两岁学说话,三岁哇喇哇喇跟人吵架,再大一点,就拿小锄头挖土,模仿大人劳动,这就是观察世界。小孩子已经学会了一些概念,狗是个大概念,黑狗、黄狗是小些的概念。他家里的那条黄狗就是具体的。人,这个概念,已经舍掉了许多东西,男人女人不见了,大人小人不见了,中国人外国人不见了,革命的人和反革命的人都不见了,只剩下了区别于其他动物的特性,谁见过“人”?只能见到张三李四。“房子”的概念,谁也看不见,只能看到具体的“房子”,天津的洋楼,北京的四合院。

大学教育应当改造,上学的时间不要那么多。文科不改造不得了。不改造能出哲学家吗?能出文学家吗?能出历史学家吗?现在的哲学家搞不了哲学,文学家写不了小说,历史家搞不了历史,要搞就是帝王将相。×××的文章(指《为革命而研究历史》),写得好,缺点是没有点名。姚文元的文章(指《评新编历史剧<海端罢官>》)好处是点了名,但是没有打中要害。

要改造文科大学,要学生下去,搞工业、农业、商业。至于工科理科情况不同,他们有实习工厂,有实验室,在实习工厂做工,在实验室作实验。

高中毕业后,就要先做点实际工作。单下农村还不行,还要下工厂、下商店、下连队。这样搞它几年,然后读两年书就行了。大学如果是五年的话,去下面搞三年,教员也要下去,一面工作,一面教。哲学、文学、历史,不可以在下面教吗?一定要在大洋楼里面教吗?

大发明家瓦特、爱廸生等都是工人出身,第一个发明电的富兰克林是个卖报的,报童出

身。从来的大学问家,大科学家,很多都不是大学出来的。我们党中央里面的同志,也没有几个大学毕业的。

写书不能像现在这样写法。比如讲分析、综合。过去的书都没有讲清楚。说“分析中就有综合”,“分析和综合是不可分的”,这种说法恐怕是对的,但有缺点。应当说分析和综合既是不可分的,又是可分的。什么事情都是可分的,都是一分为二的。

分析也有不同的情况,比如对国民党和共产党的分析。我们过去是怎样分析国民党的?我们说,它统治的土地大,人口多,有大中城市,有帝国主义的支持,他们军队多,武器强。但是最根本的是他们脱离群众,脱离农民,脱离士兵。他们内部有矛盾。我们的军队少,武器差(小米加步枪),土地少,没有大城市,没有外援。但是我们联系群众,有三大民主,有三八作风,代表群众的要求。这是最根本的。

国民党的军官,陆军大学毕业的,都不能打仗。黄埔军校只学几个月,出来的人就能打仗。我们的元帅、将军,没有几个大学毕业的。我本来也没有读过军事书。读过《左传》、《资治通鉴》,还有《三国演义》。这些书上都讲过打仗,但是打起仗来,一点印象也没有了。我们打仗一本书也不带,只是分析敌我斗争形势,分析具体情况。

综合就是吃掉敌人,我们是怎样综合国民党的?还不是把敌人的东西拿来改造。俘虏的士兵不杀掉,一部分放走,大部分补充我军。武器、粮秣、各种器材,统统拿来。不要的,用哲学的话说,就是扬弃,就是杜聿明这些人。吃饭也是分析综合。比如吃螃蟹,只吃肉不吃壳。胃肠吸收营养,把糟粕排泄出来。你们都是洋哲学,我是土哲学。对国民党综合,就是把它吃掉,大部分吸收,小部分扬弃,这是从马克思那里学来的。马克思把黑格尔哲学的外壳去掉,吸收他们有价值的内核,改造成唯物辩证法。对费尔巴哈,吸收他的唯物主义,批判他的形而上学。继承,还是要继承的。马克思对法国的空想社会主义,英国的政治经济学,好的吸收,坏的拋掉。

马克思的《资本论》,从分析商品的二重性开始。我们的商品也有二重性。一百年后的商品还有二重性,就是不是商品,也有二重性。我们的同志也有二重性,就是正确和错误。你们没有二重性?我这个人就有二重性。青年人容易犯形而上学,讲不得缺点。有了一些阅历就好了。这些年,青年有进步,就是一些老教授没有办法。吴晗当市长,不如下去当个县长好。杨献珍、张闻天也是下去好。这样才是真正帮助他们。

最近有人写关于充足理由律的文章。什么充足理由律?我看没有什么充足理由律。不同的阶级有不同的理由。哪一个阶级没有充足理由?罗素有没有充足理由?罗素送我一本小册子,可以翻译出来看看。罗素现在政治上好了些,反修、反美、支持越南,这个唯心主义者有点唯物了。这是说的行动。

一个人要做多方面的工作,要同各方面的人接触。左派不光同左派接触,还要同右派接触,不要怕这怕那。我这个人就是各种人都见过,大官小官都见过。

写哲学能不能改变个方式?要写通俗的文章,要用劳动人民的语言来写。我们这些人都是学生腔(陈伯达同志插话:主席除外),我做过农民运动、工人运动、学生运动、国民党运动,做过二十几年的军事工作,所以稍微好一些。

哲学研究工作。要研究中国历史和中国哲学史的历史过程。先搞近百年史。历史的过程不是矛盾的统一吗?近代史就是不断地一分为二,不断地斗。斗争中有一些人妥协了,但是人民不满意,还是要斗。辛亥革命以前,有孙中山和康有为的斗争。辛亥革命打倒了皇帝,又有孙中山和袁世凯的斗争。后来国民党内部又不断地发生分化和斗争。

马列主义经典著作,不但要写序言,还要做注释。写序言,政治的比较好办,哲学的麻烦,不太好搞。辩证法过去说是三大规律,斯大林说是四大规律,我的意思是只有一个基本规律,就是矛盾的规律。质和量、肯定和否定、现象和本质、内容和形式、必然和自由、可能和现实等等,都是对立的统一。

说形式逻辑和辩证法的关系,好比是初等数学和高等数学的关系,这种说法还可以研究。形式逻辑是讲思维形式的,讲前后不相矛盾的。它是一门专门科学,任何著作都要用形式逻辑。

形式逻辑对大前提是不管的,要管也管不了。国民党骂我们是“匪徒”,“共产党是匪徒”,“张三是共产党”,所以“张三是匪徒”。我们说“国民党是匪徒,蒋介石是国民党,所以说蒋介石是匪徒”。这两者都是合乎形式逻辑的。

用形式逻辑是得不出多少新知识的。当然可以推论,但是结论实际上包括在大前提里面。现在有些人把形式逻辑和辩证法混淆在一起,这是不对的。


