\section[反对折衷主义(一九六五年十二月二日)]{反对折衷主义}
\datesubtitle{(一九六五年十二月二日)}


我认为这是突出政治和反对突出政治的斗争深入发展到一个新的阶段。现在公开站出来反对突出政治,反对坚持四个第一,反对抓政治思想的人还有。譬如你们浙江有个信用社主任说:“政治就是理论,理论就是会说,会说就是吹牛。”但是那种人不多了。公开提出业务第一,数字第一的人大大减少了,他们学得比较聪明了,但是他们又不愿意突出政治,不愿放弃单纯业务观点。这根“腊肉骨头”不是突出政治。形势逼人,于是就改头换面,来个折衷主义。

在政治和业务上,有三种摆法,第一种摆法是政治第一,业务第二,政治统帅业务;第二种摆法,业务第一,政治第二,政治为业务服务;第三种摆法,政治和业务都第一,叫两个第一。这三种摆法,第一种是正确的,第二种是错误的,这很明显。第三种摆法是正确的还是错误的?不用说是错误的。但是有些人就分辨不清。为什么有些人对“政治和业务都第一”的错误观点模糊不清呢?这是他对折衷主义的面貌还认识不清的缘故。

现在我来讲一讲折衷主义的特点。

折衷主义有五个特点。

第一个特点就是用二元论来代替、冒充、偷换马克思主义的两点论(两点论即一分二为二)。马克思主义的两点论,在认识事物、分析矛盾的时候,都看到它的两个方面。例如在总结的时候,既肯定成绩,又看到缺点;既总结成功的经验,又总结失败的教训。但是马克思主义者认识事物的两个方面,并不是把它们看作都一样,各占一半,半斤八两,而是严格地把它们分为主要的方面和次要的方面,分为重点和一般,主流和支流。例如,林彪同志对政治思想工作领域中四对矛盾的分析,人和物的关系,两个都重要,但活的思想更重要,活的思想第一。这就是重点论。有第一和第二,统帅和被统帅的关系。又如解决思想问题和实际问题,两个都重要,但主要的是解决思想问题。

马克思主义所以坚持重点论,因为事物的性质是由事物的主要方面规定的。把矛盾的主要方面和次要方面混淆起来,就认不清事物的本质,就不能判断是非,就不能进行工作。折衷主义用二元论代替、冒充、偷换马克思主义的两点论,就是把两点论中的重点论偷偷地抽去了。他们把事物的两方面,矛盾的两方面平列起来,等同起来,不分第一和第二,不分主要和次要,不分主流和支流,结果就掩盖了事物真相,模糊了事物的本质,使人在工作中分不清是非界限,把人们引到错误的路上。

马克思主义认为政治与军事、政治与经济、政治与业务、政治与技术的关系,政治总是第一,政治总是统帅,政治总是头,政治总是率领军事,率领经济、率领业务、率领技术的。政治与业务这一矛盾中,主要的矛盾方面是政治,把政治抽去了,就等于把灵魂抽去了。没有灵魂就会迷失方向,就会到处碰壁。所以政治第一,政治统帅业务,不能平起平坐。如果把它们并列起来,就是折衷主义。

把政治和业务并列起来,或者主张轮流坐庄的思想和看法,这些人认为既要突出政治,又要突出业务,“今年突出政治,明年突出业务”,“闲时突出政治,忙时突出业务”等等。这是一种折衷主义的倾向,是错的。

第二个特点是用混合论、调和论来代替马克思主义、辩证唯物主义的结合论。折衷主义惯用的手法,就是把各种对立的观点,对立的名词,对立的事物,无原则地结合起来。这种无原则的结合就是混合,就是调和,就是折衷主义。

折衷主义的混合论、调和论和马克思主义的结合论是根本不相容的。折衷主义的混合论和调和论是不分敌我,不分阶级,不分是非。例如现代修正主义主张社会主义和帝国主义这两个对立的体系和平共处、和平竞赛,主张取消军队,主张不要斗争,主张资本主义国家共产党不搞武装斗争,不叫工人罢工,不叫农民斗地主,而搞什么和平过渡等。从这里我们可以看出,折衷主义就是修正主义,修正主义是不要斗争,不要革命的。

折衷主义不分敌我,不分是非,就是斗争的调和论和混合论。如有的人就不搞阶级斗争的,他们对不法资本家不批评、不斗争,敌我不分。你们浙江不是有这样一件事,有一个地主分子表现得很不老实,一个党员职工批评了这个地主分子,这件事经理知道了,经理就找这个党员谈话,批评这个党员说:“地主分子本来就想国民党,你这样一斗他,他就更想国民党,以后不要斗了。”这个经理好人主义讲人情,看到别人有缺点,见到有损害党和国家利益的事,明知不对也不批评,不斗争,听之任之。这种不讲是非,不讲思想斗争,只求一团和气,只求得无原则的暂时团结的态度是混合主义,调合论,就是修正主义、折衷主义。一旦臭味相投,很容易混到那个臭水坑里去。好人主义也不少,大家要小心一点,提高警惕。

第三个特点是用似是而非、模棱两可的东西来冒充和代替辩证法。折衷主义在判断事物的时候,总是这样也对,那样也对。他们惯用这种手法来冒充辩证法,这样就容易打“马虎眼”,容易偷梁换柱,混水摸鱼,容易欺骗群众。例如列宁在《国家与革命》这篇文章里批评折衷主义的时候说:“把马克思主义改为机会主义的时候,用折衷主义冒充辩证法是最容易欺骗群众的。”

如有人说:“我既不是单纯业务观点,也不是单纯的政治观点,在我那个单位既突出政治,也突出业务,只有业务和政治都突出,这才是全面观点。光强调突出政治或者突出业务都是片面的。”他这种讲法,初听起来,好像满有道理,考虑得很全面,既照顾了政治,又照顾了业务。但仔细想一想,这是彻头彻尾的折衷主义。这不过是以全面的面目出现,它卖的完完全全是折衷主义的货色,所以很容易模糊群众,很容易蒙蔽群众。

第四个特点,有折衷主义倾向的人,总以为自己很有政治,其实他的脑子里政治缺得很,少得可昤。这些同志所谓很有政治,充其量不过是“口号在嘴上,保证在纸上,决心在会上”而已。他们在小声地喊了一句突出政治的话以后,唯恐人家把突出政治的话听去了,于是紧跟着高喊:“要突出业务”。好像不这样做,就很不舒服似的,这些人唯恐政治思想工作做好了,刁难政治干部。实在感到奇怪。

第五个特点是哲学上的折衷主义必然导致政治上的机会主义、修正主义。因为它把政治与军事,政治与经济,政治与业务,政治与技术的关系搞错了,把灵魂抽去了,其结果就一定是:小则是单纯的业务观点,大则陷入修正主义的泥坑。

以上讲的是折衷主义的五个特点。

凡是有折衷主义观点与倾向的人们,他们都有一个共同之点,这就是从他们思想深处来说,是反对突出政治的,他们不是把突出政治放在第一位。


