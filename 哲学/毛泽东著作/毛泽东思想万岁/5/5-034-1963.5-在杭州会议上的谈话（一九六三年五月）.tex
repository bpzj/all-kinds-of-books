\section[在杭州会议上的谈话(一九六三年五月)]{在杭州会议上的谈话}
\datesubtitle{(一九六三年五月)}


一、形势问题

生产的形势,一年比一年好。阶级斗争形势是严重的,尖锐的。(列举农村阶级斗争情况)为什么农村出现这样严重的情况?有三个原因,一个是阶级原因,一个是历史原因,一个是认识原因。

阶级原因:主要是社会主义社会还是有阶级的社会,存在着阶级和阶级斗争。正确理解和处理阶级矛盾和阶级斗争,正确处理敌我矛盾和人民内部矛盾,是领导和团结全党,领导和团结全体人民群众,顺利地进行社会主义革命和社会主义建设的保证。

历史原因:一方面是有的地区民主革命任务尚未完成,有的地区社会主义革命未完成。封建地主没有打倒的地方,是重新革命的问题。另一方面是工作历史方而的原因。土改以后,我们就没有再搞阶级斗争。“三反”、“五反”,一九五七年的反右斗争,都搞了一下,但不是这样的做法。苏联在一九三二年以后,一九三七年、一九三八年又搞了二次肃反,此后十六年当中不搞阶级斗争,他们的集体化依靠谁?不搞阶级斗争,无产阶级专政就没有可靠的社会基础。

华北局机关五反搞得好。说是“清水衙门’,但是一清就清出好多专案来。

认识原因:阶级斗争是客观的存在,没有认识到,怎样领导阶级斗争?

二、认识问题

十中全会后,跑了十一省,只有××、××滔滔不绝地讲社会主义教育,其他人都不讲。二月会议后,情况有了变化。河南五个月没有抓阶级斗争,二月会议以后,抓得很好。有变化,但并不是都通了,有个地委书记,二月会议以后,就不通,下去试点以后才通。

我看了湖南第二个材料,现在才懂得一点,即搞规划、生产经营中间,也有两条道路的斗争。

我问了许多人,思想是从哪里来的?都回答不出来。物质变精神,精神变物质,是生活中常见的现象,不识字的农民也懂得这一点。比如说,你问农民,他知道张三是地主,是压迫我们的,有了“张三”、“地主”这两个概念,就可以推理:地主是剥削人的人。农民的认识是从生活中来的,不识字也可以懂哲学。成吉思汗就不识字。

一言可以兴邦,一言可以丧邦。这就是精神变物质。马克思就是一言,要无产阶级革命和无产阶级专政,这不是一言可以兴邦吗?赫鲁晓夫也是一言,就是不要阶级斗争,不要革命,这不是一言可以丧邦吗?

哲学要在实际工作中讲,要在开会中讲。要告诉你身边的同志,哲学并不难。军事学也不难,我们人民解放军的元帅、将军中间,只有林彪、刘伯承等有数的几个人是从军事学校中出来的。翻了军事书,看了欧洲战史,和中国情况对不上。不是黄埔军校的洋包子打败了土包子,是土包子打败了洋包子。林彪同志是黄埔军校的半年的入伍生,……派出来当连长,根本不能打仗,听班长的。班长说怎么打就怎么打。军事是从实践中学的,所以不要把马克思主义看得那么神秘,不要把哲学看得那么神秘。我看过雪峰一部分日记,此人就懂得一点哲学。

大学生学习五年就学好哲学?我不相信。许多哲学家都不是大学学习的。中国的哲学家中,王充、范缜、付玄,柳宗元、王船山、李贽、戴东原、魏源……都不是专门搞哲学的。黑格尔也不是专门搞哲学的,他的学问很广。康德是一个天文学家,他的天体论到现在还有价值。马、恩、列、斯也都不是专门搞哲学的。

山沟里出哲学。醴陵那样好的报告,不出在湘潭,不出在常德,而出在醴陵。在困难中,在斗争中才能够出哲学。逆境出哲学,顺境能够出哲学吗?三国的黄盖兄,醴陵人;程颐、程灏的老师周廉溪,是宋代的大理学家,朱熹和他是一个系统的,也是醴陵人,是醴陵专区的道县人。张载是陕西人,那是另一派。唐代的大书法家怀素,也是这里的。柳宗元从三十岁到四十岁,整整十年都住在醴陵,当时叫做永州。他的山水文章,和韩愈辩论的文章,都是在那里写的。

所以要破除迷信,不过要注意,不要像前几年那样,把不应该破的也破了。

事物有现象有本质,要透过现象看本质。现象和本质是对立的统一。本质是看不到的,要透过现象去抓到本质。比如干部不参加劳动,势必会产生修正主义。又比如平常我们走路看不到蚂蚁,大踏步就更看不到了,要蹲下来,才能看到蚂蚁,就能看到很多东西。否则不仅新鲜的萌芽的东西看不到,就是大量普遍存在的东西也看不见。比如阶级斗争和干部不参加劳动是大量存在的,有些人却看不见。要用科学的方法,进行调查研究。有的人是主观主义地大胆地假设,主观主义地小心地求证。河北各地委下去调查研究,只有保定地委是科学的,其他都是主观主义的。保定地委开始并不是去搞“四清”。而是去搞分配的,群众不同意,提出搞四清。保定地委听了群众的意见,改变了计划。搞了“四清”,这才是真正的调查研究。

讲哲学不要超过一小时,讲半小时以内,讲多了就糊涂了。

我在莫斯科会议上讲了哲学,莫斯科宣言写上了,在国内反倒没有人讲。

三、要点

运动的要点是什么?是十个问题,其中一部分是认识问题,是要高级领导干部、领导干部解决的。还有一些问题是普遍工作要解决的,普遍工作中的要点有以下五点:

1.阶级、阶级斗争。用什么方法进行阶级斗争?一定要用阶级观点去分析问题,最先写四大家族的是曹雪芹。《红楼梦》写的贾、史、王、薛大家族,他们是奴隶主,三十二人。写奴隶女,鸳鸯、晴雯、小红等,都是很好的,受害的是这些人。林黛玉不是属于四人家族的。

2.社会主义教育。社会主义教育的方法有两大条:

第一条是把中央的精神和干部、群众见面,讲解清楚,结合当地的具体情况、具体工作、具体事实,让群众揭盖子。

第二条,要让老一辈重新回忆受压迫、受剥削的历史,激发阶级感情,让青年一代知道革命斗争果实来之不易,让他们续一续无产阶级的家谱。

3.依靠贫下中农。依靠谁的问题,一万年也有,到将来总还有唯心主义和唯物主义、先进和落后、左中右的矛盾。在今天依靠谁?总得有一个阶级。依靠全民?说依靠全民,实际上是依靠少数人。有人说“地富听话,中农调皮,贫农糊涂。”地富怎么不听话?又送东西,又送女人,可是他是要你听他的话。

什么叫心情舒畅?贫农、下中衣受到压抑,不能抬头,心情怎么能舒畅?贫农、下中农不舒畅,干部怎么能够舒畅?

资产阶级说他们后继无人,怎么说后继无人?黑格尔的后继人是马克思,资产阶级的后继人是无产阶级。资产阶级抓“三自一包”,想卷土重来,我们就要在这方面打击他,打掉他的基础,不让他拉后继人。

四、四清

什么叫贪污?五十元?一百元?二百元?只要坦白了,退了赃,就不算贪污。

赃要退,也要合情合理,退到手脚干净,又要退到让干部能够生活。这样做究竟退多少?是不是采取自报公议的办法。

惩办要控制在百分之一。

今年不要开杀戒,明年再说。罪大恶极的也先放慢一些,现行反革命按规定办理。群众要求非杀不可的,是有道理的,你领导可以等一等嘛!

5.干部参加集体生产劳动。只有参加劳动,才能解决贪污、多占问题,也可以了解生产情况,而不是浮在上面。干部不参加劳动,势必脱离劳动群众,势必出修正主义。

昔阳干部劳动很好。昔阳在山上,很穷。很穷就革命。

要把农村党的基层组织建立在先进劳动者、劳动积极分子手里。

(有人说,有些劳动模范不参加劳动。)

劳动模范不参加劳动,还算什么模范?取消好了。有的因为会多,接待访问太忙,这个问题要解决,你们可以到田间去访问嘛!

县干部也要参加劳动,基层干部不参加劳动,不就跟国民党保甲长一样吗?你们是做大官的,也有做小官的。小官权也很大。过去一个团长,给不少办公费。现在我们基层干部,一个参加劳动,一个“四清”,不愿意干就回家当老百姓去。

干部参加劳动了,贪污盗窃、投机倒把的就少了。贪污盗窃、投机倒把,什么时候都有,一万年都有,不然辩证法就不灵了,就没有对立统一了。

贪污揭发得越多,我越高兴。你们抓过虱子没有?身上本来很多,抓得越多越高兴。

五、方法

要采取积极态度。

1.要注意训练和教育干部;

2.不要着急,今年搞不完明年,明年搞不完后年。土改不是搞了三、四年吗?有的人不信,不要去责备他,你一围攻,他一着急,就乱来。要慢慢地说服,着什么急?我们革命胜利比苏联还不是晚三十多年?

3.要试点,要踏踏实实地搞深搞透,要防止敷敷衍衍地走过场,一定要搞试点。

4.要区别不同情况,少数民族地区,边境地区不要一起搞。(讲了西汉陈平的故事)(对李××)你四川那么大一个省,一下子能够搞得了哇?

5.精简。要精简一些干部下去搞劳动锻炼,搞阶级斗争锻炼。我身边原有二、三十人,现在只剩下十几个人。我对江××说,江苏四千多万人口,省直机关工作人员五千,可以精简一千五百人或两千人。这是一个老问题,长期没有解决。

6.要抓住重点。“不唱天来不唱地,只唱一本《香山记》”。《香山记》是讲庄王的女儿(即观世音菩萨)的故事,七个字一点,开头两句就是这个。天和地可以隔开,天和地都不唱,单唱《香山记》,就抓阶级斗争。


