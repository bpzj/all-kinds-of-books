\section[在上海局杭州会上的讲话(一九五七年四月)]{在上海局杭州会上的讲话}
\datesubtitle{(一九五七年四月)}


(思想问题讨论时,毛主席插话,整理成五个问题。)

一、最高国务会议、宣传会议以后,知识分子情况:紧张空气比较缓和,党外知识分子初步和我们接近起来,犹豫情绪比较开朗,报告不是万灵药,还要用各人的药,要的东西不给他们,就处于被动,给了以后,鸣好鸣坏由他们自己负责。反正总有一天要整到自己头上来的。我们的办法是先整自己,把党整好,整得谦虚一些,讲道理。现在的问题就不是我们的问题。知识分子左派向我们要民主(放),学生向教授要民主,那时就要求救于我们。我们一放,他们自己就鸣起来,我们一紧,就向我们要。

民主党派,上海去年发展一万一千多,他们内部的问题就多起来,只有无产阶级不怕政变,错了,就改。他们中间问题就多起来了。

我们要放,要硬着头皮让他们攻!攻一年。谁让我们有教条主义,攻掉就好,攻得过火,“让牛鬼蛇神都出来闹一闹。共产党要让骂一下子,让他们(不清),我们想一想。陈毅同志三反时说:我们专政专了这么多年,让人骂一个星期也可以。问题是你倾盆大雨那样向我们倒来,我们也会大吃一惊。有些知识分子还怕放长线钓大鱼。有人说,对。共产党威信高,报上只要一二句话,风吹草动,就有人怕。党外怕,党内也怕。事情对,也要谦虚三分。压力大,自己要懂。有人说,放长线钓大鱼,也有一些道理。我们现在让人批评,以后再去分析。知识分子像惊弓之鸟,一定要看,可能看二十年。党内也看,例如治病救人,有人不相信,直到后来看了事实,才相信。要经过许多考验,共产党的政治要受考验,领导者要受被领导者的考验。过去知识分子为旧统治者服务,现在生产关系改变,没有社会基础了。知识分子过去是寄托在封建、资本主义、个体生产制。知识分子出身哪个阶级,为哪个阶级服务。出身的阶级现在已破坏,吊在半空中,脚不踏实地。他们现在爬在我们的身上。工农通过共产党让他们做事,帮他们吃饭。我们还要用十几年洗脑筋,办法是团结——批评——团结。他们想不要批评,说不出口来。现在有一千多万人悬在半空中,他们的脑筋是旧的,没有什么可怕。吊在半空中,很需要我们伸出手去接一把,如果我们热情地伸出手去,可以快一些。大部分过来,就不那么简单了。最欢迎变化的是无产阶级。农民最希望蒋介石、美帝、地主起变比,但不希望小私有制起变化。民主党派、知识分子看到文化革命来的太突然太急,也可能变。毛脱了皮,伹魂还在旧的皮上,毛附在无产阶级身上,他们常常魂魄不安。不变不会不安,不安是变的表现。有些知识分子的世界观长期不可能改变。……考验作家的世界观,就是要考验他们能否与工农打成一片,还是二片。下乡不能和农民说真话。党内的官僚主义者世界观也成问题,脱离群众。他们赞成马克思主义,又与群众格格不入,那是什么马克思主义?社会上有一批人,唯恐天下不乱,有反共思想,主张打出去。改造要几十年,有些人到死不改,大部分可改。有那么一部分老顽固,到死不改。

二、一定要放。怕放是道理没有讲通,或没有说服。这是有领导的反官僚主义。凡是不让放,毫无准备,结果大放。我们主张放,最大乱处无非是乱一阵。为人民办事,人民还是不满意,有人想不通,苦恼的很。总有人不满意,有满意有不满意。比过去他满意,比将来或比现在好的那些人他又不满意。这种情况一年以后也还有。如果都满意了,我们只要去睡觉好了。现在还没有放,怎么知道就要乱呢?过去阶级斗争,警惕性太高了,现在右一下子。一家独鸣了多少年,让他乱一下子看看。明显的错误,长期不批评是不好的。草多可以用拖拉机;蒋介石不是草,三年就打倒。中国知识分子,百分之十懂马克思主义,百分之八十是爱国主义、拥护社会主义的。可见毒草只是少数,似毒非毒、野草香花中间的是多数。这是社会存在的反映。

这个问题是新问题,干部还没有试过,试一试,试出味道来,顾虑就没有了。就是没有讲清楚,人民还会反对共产党吗?似毒非毒的是大多数,毒草是少数。

我不是鼓励人民闹事,搞闹事促进会。闹事打办公室要处分。乱糟,偏有一点。现在是党外情绪高,过些时,党内情绪也会高起来,先低后高。

解决一些问题,取得了一些经验。党的任何一条方针政策都是要拿到群众中去考验的,应该让干部去工作中试一试。

又怕又不怕,又高兴又不高兴,问题解决又不解决,这就是辩证法。比如打仗,开始提心吊胆,越打下去越大胆,没有几次路线错误,就不会现在这么好。好事多了,会骄傲,就会出坏事。苏联就是如此。王明错误大,没有自我检讨,没有威信。错了,紧张一礼拜,检讨了,就好了,还有免疫作用。邓老对合作化问题,我想彻底攻他的。对干部帮助的办法,就是攻。最多失掉一票,不想提名我。

巴金说杂文难写。一、共产党整风整好了,就有自由风气。二、彻底唯物论者是不怕的。王熙凤说:舍得一身剐,敢把皇帝拉下马。越是在困难的时候,大家应帮他,不帮忙是站不住的。

鲁迅墓搞个运动场有什么不好?陈其通、钟惦棐代表两个片面。我也是片面无忧论……可以变成循环论,但辩证法不是循环论。

片面性多了,就要出辩证法。这些人无非要我们倒霉,我们就想得更坏一点。现在好像右了一些,但说服群众,不能简单化办法,对闹事不是彻底解决。事情处理过分了,不行。谁说要牛鬼蛇神?是群众要看。不能压,只有多演好戏。应该让社会复杂一些,把重心放在科学上面。有人说不开除闹事学生比国民党还毒辣,我们应该比国民党高明一些,有一个国民党和我们比较,很好。他们是在掘坟墓,我们不要学国民党。阻止广东学生来京请愿,我至今也难受。国民党对人民专政,共产党讲民主,这点应清楚。有人拿专政的原子弹向人民的头上幌一幌,这是不好的。

三、百花齐放,百家争鸣,要改变党和知识分子的关系。有人认为不是时候,与政治思想工作有矛盾。大规模政治斗争已基本结束,八大做了结论。这个方针提出正是时候。

党与非党之间有条沟,很深。我想是说没有一条沟,没有界限是不好的,但不能成为一条沟,有沟,就脱离群众。

上海工厂有千分之一,没有万分之一闹事,有百分之一就好了。可以清除官僚主义。

从党内到党外。党内党外一起讲,戏就唱起来了,这样,可以把许多人推上政治舞台。

《人民日报》是什么人的报纸?要整一整。

有人感到左不行右不行,难办。知道难办就好,就会动脑筋,否则就只说专政,说集中(即压)。

党内要不要斗争,当然要斗争。农民年年锄草,批评错误要有说服力,不能靠压。靠几篇文章不行,要以理服人,不要靠资格。

对敌讲力量,对人民讲理。没有理,不会历史主义,地位多高,不行。要多学习,多研究。干部摆资格,讲势力,是很危险的。

没有民主党派行不行?也行,苏联就没有。听不到反面意见,但打仗还是打胜了,也没有杀知识分子。苏联不想百花齐放,百家争鸣。民主党派都是大知识分子,农工民主党并没有农工。要给他们事做。

两头团结,中间批评。没有批评就是右倾。钟惦棐文章出来很久没有批评。

《文汇报》上批评要全面分析,这批评是有益的。

对民主党派人士要讲真话,有缺点就说。当民主党派也有苦处,听不到,看不到,共产党的底摸不到。也可以讲一些党的缺点。我们要打倒民主人士,他们就反起来。虚心学习,很有必要。

开了许多会议,不发消息。人民日报这是国民日报。

陈其通代表百分之九十党内同志,我就没有群众基础。

党内外应当有一条线,不应当有一道墙。第一书记要经常接见党外人士,把底亮出来,诱敌深人,也可以讲,不讲是不民主。让讲又是诱敌深入,怎么办?要从六亿人口出发,我们野心要大一点,知识分子有五百万,要争取他们。莫朴是宗派主义者,要交审,要审查党籍,为什么不要国画?他是搞丑术,不是搞美术。国民党还有国画。他是什么党,大概是第三个党吧?社会存在不能否认。社会存在有反革命,我们就要肃反。民主范围内的问题早已存在。火烧红莲寺也要采取斯坦尼斯拉夫斯基体系怎么行?江丰、莫朴搞辩证法,为什么不要图画?只搞单干户?夫妻也要配对么!老干部要考虑,过去是一套办法,现在要解决人民内部问题。《新民报》看了几张,不是黄色报纸,软一点。不能领导科学,不要一下子顶回去,要请教,刘备还要请个诸葛亮。党外人士讲话是考虑过的。有时有错误,不要一下子顶回去,不要说死了。

四、关于几个问题的认识问题

范瑞娟文章有二百多封信反对,我看没有黄色,天天上甘岭,没有我的丈夫怎么办?写文章要有逼,毛骡子没有人骑不行。领导、选报、准备、说服力、有利、五个条件。这么多条件,只有说服力比较具体,其他的都很难改,但不要教条主义。

对党的政策能否辩论,《人民日报》的文章组织都对?陈其通等的文章就不对。我看每省办两个报纸比较好,一个党外办,唱对台戏。

知识分子成堆的地方,总有些难办的事情,不成堆也不行。上海六百万人口,只有二百多××反对,可见是少数。报纸上也不能天天打气。一年开次会,春夏秋冬。文教事业是教育人民的东西,要经常管,不能头痛医头,脚痛医脚。电影厂有人说力争香花,不出毒草。这精神不对,我们不怕出三分之一毒草,避免毒草是一句空话。

片面性。要求党外都避免片面性,都成为辩证法专家,一万年也是不可能的。上次我从自己谈起过。工作做得好,做得不好,永远存在。

知识分子入党问题。应该争取三分之一到党内来。民主党派可以跨党,领导人员不要入党,六年之内搞四分之一,三个五年计划三分之一入党。今年争取百分之十五入党,要稳步。今年如果不争取一批知识分子入党,对社会主义不利。百花齐放、百家争鸣是争取知识分子的方针,如果组织上关门,那是不相称的。我们党没有大作家,大诗人、大教授,要招兵。过去办不到,现在要努力。高尔基只读二年书,孙中山也没有受很高的教育,人是可以培养的。

教育上摇摆不定,有道理。没有经验,摇几次,就不摇了。初中教材中,应当有历史、地理、文物等等。过去中小学就学外国文,应该学。初中课太多,应该砍掉三分之一。学校教课怎么百家争鸣?应以一个为主或几个为主。

肃反检查是查五五、五六年的,过去的不查了。

先进社会制度和落后的生产力矛盾,这句话不对。

节育问题。马尔萨斯结论不对,可是人多是否一定要打仗?中国节育要看省份。江苏人多,东北人少。婚姻法不要改,多劝说不要早婚。

一千年还有革命,我说过,但不一定。

一万年后,生产关系总要改变。将来不叫国有化,是球有化的问题。

人民内部矛盾处理不好,就会被推翻。现在已有个别的乡政府、支部。

五、领导问题

要改善共产党和知识分子的关系。不与知识分子接近,是共产党的责任。不怪宣传部,要第一书记负责。第一书记不仅要抓思想领导,还要看文章,要看历史、哲学、文学……历史分期。人家锣鼓喧天,第一书记什么都没有看,怎么解决?有人说,旁的事少干一些,抓思想工作,要看刊物,看文学、自然科学、社会科学。养成习惯,慢慢就有兴趣。党内也要讨论,讨论没有时间,不讨论也会没有时间。有人把“百家争鸣”念成“争鸣”。不懂要接接班,周瑜二十多岁,程普五十多岁,大敌当前,谁挂帅?搞文教工作,要有文化。诸葛亮初时不被重用,后来用了。那时能破格任用,现在为什么不行?去年评级,有人意见很多,破格提升有很多阻碍。当兵的没有文化,可难,可也要业余学习。现在是打另外一个仗。各级的将、校、尉、官兵要重新配备。各级干部要学会社会科学、文学、自然科学,不然,大将出马,耍了两下子,枪杆子就掉到地下去了。大学党委会要改组,接受干部任务完了,现在可调出来,有个把留在校里搞搞事务,“百家争鸣”,不要去当校长。

要做宣传工作,逼上梁山讲,要到大学里去讲,先与教师、学生谈谈。今天不讲,明天不讲,今年不讲,明年不讲,一万年不讲,怎么行?

高教部长、宣传部长、党委书记都请去讲,总比政治教员好一些。

定息问题。企业性质改变了,两重性质不存在,皮之不存,毛将焉附,毛是附在无产阶级身上,但灵魂还在。定息搞短搞少,长期留一个口实,不合算。现在一年还一亿,还十年,将来还一千万。他们不要求取消,我们就还下去。小的资产阶级要摘帽子,不要小的一摘,大的不安,最后总归自动取消。

六亿人口中有缺点的,不管任何人,都可以批评,不对可以不听,对的要接受。反对肃反,反对合作化的文章,可以驳一驳。这不叫诱敌深入,叫自投罗网,孙大雨说人家是反革命,最好在报上登出来。胡风、铁托的文章在报上发表,就脱离群众,这也是帮助的一种办法。做政治工作要有一点办法,不怕发表,要驳他。

各省要交流经验,不要不相往来。


