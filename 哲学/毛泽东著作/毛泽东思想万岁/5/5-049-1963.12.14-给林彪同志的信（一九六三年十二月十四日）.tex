\section[给林彪同志的信(一九六三年十二月十四日)]{给林彪同志的信}
\datesubtitle{(一九六三年十二月十四日)}


林彪同志:

你的来信早收到了。身体有起色,甚为高兴。开春以后,宜到户外散步。你对两个文件的看法是正确的。国内外形势均已向好,均已走上正确的轨道。可以预计,更大的发展,是会到来的。关于农村社会主义运动的两个文件,十一月中旬就发出去了,本月上旬各省已有反映,在一些地方的生产大队全体人员及五类分子,(有的多到七百多人听讲)开会时向他们宣读,分组讨论,效果很好,军队如能照此办理,那也一定会好的。由团营两级理解力强的军政干部向连队一切人员分几次宣读、讲解、讨论,由群众提出意见,讲解员解答疑难问题,是会成为一个大规模社会主义政策教育运动的。(军师二级也可派一部分强的干部下去,杂在团营干部中,向连队宣读、讲解,作为军官当兵的一种形式。至于高级首长,例如××、××、杨×、廖××、许世友、黄永胜、刘亚楼等等同志,也应该择一、二连队去作一、二次讲解。讲解要联系环境,先要对准备去讲解的连队情况作一些大略的调查。)不知已按你的意见作了布置没有?据我从北京几个军事基层单位的少数同志接触,他们尚不知道此事,没有看过文件,也没有听过宣读。此事其实不难,只要由总政下一通知,叫各军区各兵种印发文件,每一个支部一本,传下去。由团营合组宣讲队伍,分头下达到连队,照本宣讲,以排或班为单位,进行讨论,自由发言,容许讲不同的意见,甚至反对意见,就可以在一个短时期内(例如几个星期)出现一个高潮,提高政策水平。(因为不能耽误操课任务,宣读文件只能夹在正常操课中间去作,所以需要几个星期,如果暂停操课,那就一、二个星期够了。)一次宣讲之后,过几个月再作一次宣讲,使人们得到更深理解。军队一动起来,还可以抽出一些干部帮助地方,向工厂、农村作宣讲工作。这样可以使军民联合起来,人民了解和拥护军队,军队了解和帮助人民,更是一大好事。是否可以如此做,请你们和罗、肖诸同志商酌处理。
祝好!

\kaitiqianming{毛泽东}
\kaoyouerziju{一九六三年十二月十四日}

曹操有一首题为《神龟寿》的诗,讲养生之道的,很好。希你找来一读,可以增强信心。又及。

附:曹操《神龟寿》

神龟虽寿,犹有竟时。腾蛇乘雾,终为土灰。老骥伏枥,志在千里;烈士暮年,壮心不已。盈缩之期,不但在天;养怡(一作恬)之福,可得永年。〔幸甚至哉,歌以咏志。〕\marginpar{\footnotesize 75}

