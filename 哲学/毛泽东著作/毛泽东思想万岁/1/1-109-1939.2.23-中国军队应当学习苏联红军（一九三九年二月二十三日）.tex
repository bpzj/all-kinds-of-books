\section[中国军队应当学习苏联红军(一九三九年二月二十三日)]{中国军队应当学习苏联红军}
\datesubtitle{(一九三九年二月二十三日)}


当我们听到苏联红军二十一周年纪念的时候,我们感到有一个伟大的力量站在中国民族与中国人民的面前,这个伟大力量伸出他的友爱之手,愿为我们的抗日民族解放战争之后援,这个伟大力量就是苏联红军。苏联红军经过二十一年的锻炼,他的陆军、空军,还加上海军已经成为不可战胜的力量了。在英明的领袖斯大林同志与统帅伏罗希洛夫同志领导之下,由于他是工农人民的军队,由于他具有坚强的技术装备,深厚的军事素养,正确的政治工作,不但早已成为保卫社会主义苏联的柱石,而且早已成为保卫世界和平反对法西斯侵略的中坚力量,成为全世界任何真正愿意反抗法西斯侵略的武装部队的模范。在这后一个意义上,我料想苏联红军的每一个指挥员,苏联的每一个公民全部知道,中华民族与中国人民在现在进行的战争是什么性质的战争,这个战争是在何种艰难困苦的环境中进行的,但这个战争又是具有何种的光明前途,他与苏联及全世界又有何种密切关系。反过来说,在我们中国,不但八路军,而且,全体抗日军人,无不懂得苏联红军是中国的好朋友,无不知道苏联是最切实地援助中国抗战的,尤其是苏联红军战斗的经验教训是为中国军队与中国军人所取法,使我们懂得要战胜日本帝国主义,中国军队也要变成政治上具有正确方向的军队,也要逐渐具备新式技术装备,近代化的军事素养,与民族革命的政治工作。中国军队有许多长处,这些长处在十五个月抗战中已逐渐引起了敌人的惊叹,引起了世界的赞美,中国军队必能在长期抗日战争中锻炼自己,成为全世界上反法西斯战争中的一个有力的方面军,为了驱逐日寇实现中华民族的解放,也为了援助世界的反法西斯战争。但中国军队由于历史原因,至今还保存着许多缺点,重要的是政治素质的不完善,新式技术不足,近代化的军事素养不足,尤其是政治工作之不足与缺乏正确方针,这些都是要向苏联红军学习的。谁都知道,中苏两人民族在反抗帝国主义的侵略的基础之上,十多年来就结成了密切的关系。在一九二四年到二七年,苏联及其红军援助了中国的北伐战争,于今又为共同反对日本帝国主义而奋斗。当此抗日战争处于新的困难情况面前的时候,中苏友谊关系是应该更加增进的,中国军队与苏联红军之间的精神联系是应该更加密切的,大敌当前,中苏同样,正是两大民族两大军队共同奋斗的时候了。不管法西斯国家如何加紧侵略中国,如何的准备进攻苏联。如何的危害西班牙人民,以及如何的企图发动各大国之间的再一次惨战,但其前途是可以预断的,最后胜利绝不是侵略者,中国、苏联、西班牙人民,全世界一切被侵略者,将夺取最后的胜利。当此苏联红军二十一周年纪念的时候,我以中国人民与中国军队之一员的资格,谨向苏联人民与苏联红军致以反法西斯的友谊的敬礼!

<p align="right">(《群众》周刊第十七、八期,一九三九年二月十六日)</p>

