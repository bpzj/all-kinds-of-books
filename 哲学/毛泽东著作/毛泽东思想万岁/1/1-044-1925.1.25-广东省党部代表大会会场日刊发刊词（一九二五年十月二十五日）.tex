\section[广东省党部代表大会会场日刊发刊词(一九二五年十月二十五日)]{广东省党部代表大会会场日刊发刊词}
\datesubtitle{(一九二五年十月二十五日)}


本党广东省代表大会开会发行日刊应有发刊词。

本党自同盟会以至中国国民党历二十余年而有去年一月之改组;自辛亥革命成而败历十四年,而有现在之领导全国国民反帝国主义反军阀之革命;至于今日全国各地惨遭帝国主义杀声中于反动军阀高压爱国运动声中于东江炮声中,而有广东代表大会之召集;这些都不是个人主观的突现和一时事变的偶然,这些都是客观环境之必然,即历史事实之推进。我们从去年一月全国大会中得到了正确的策略。从去年到今年两年来全国高潮的反帝国主义反军阀运动中由执行我们的革命策略得到了宣传组织和遇敌攻守的经验,于这次广东全省代表大会中,我们将得做什么?

我们的伟大领袖孙中山先生应乎中国被外力军阀买办地主阶级重重压迫的客观环境,为我们下定了革命的三民主义,我们伟大领袖虽死,革命的三民主义不死,怎样使革命的三民主义在广东实现,乃是广东同志的唯一工作。

我们的已故领袖孙中山先生看清楚我们的主要敌人是帝国主义,于是下定了革命的民族主义。又看清楚帝国主义借以剥削中国的重要工具是军阀大买办阶级和地主阶级,又下定了革命的民权主义与民生主义。革命的民族主义叫我们反抗帝国主义使中国民族得到解放,革命的民权主义叫我们反抗军阀,使中国人民自立于统治地位。革命的民生主义叫我们反对大买办阶级,尤其是那封建宗法性一切反动势力根本源泉之地主阶级,使使中国大多数穷苦人民得享有经济幸福。广东的同志,在反抗帝国主义(沙面罢工,省港罢工)反抗军阀(打倒陈林打倒杨刘),反抗大买办阶级(商团事件的镇压),反抗地主阶级(海丰、广宁、训德保……各县农民与地主阶级的斗争)各次很大的运动中,都做了人民的领导,在这些反抗运动中间,得到了二十万有组织工人、五十万有组织的农人和数万有训练的军队,还有许多爱国商入学生都到本党旗帜之下,为本党在南方定下了一个革命的基础!在这个基础之上建设了指导全国革命运动的国民政府,在各次运动中虽有许多外省同志参加,然广东同志实作了很大的努力。

广东是与英帝国主义紧邻的地方。是陈、林、邓、殷等落魄军阀狡然思迁的地方,是陈××等大商买办阶级集中的地方,而且是地主阶级勾结帝国主义军阀严重剥削压迫农民的地方。怎样检查以前的工作,规定以后的方法,产生有力量的全省最高指挥机关,用以发展各界人民的组织,尤其是发展那占广东全人口百分之八十即二千数百万的农民广大群众的组织,以保障而且扩大我们的胜利,使三民主义完全实现于广东是广东全代表大会的责任。

打倒帝国主义!

打倒陈炯明肃清一切反革命!

打倒危害广东的买办阶级!

打倒勾结帝国主义军阀惨杀农民的地主阶级!

革命的三民主义万岁!

