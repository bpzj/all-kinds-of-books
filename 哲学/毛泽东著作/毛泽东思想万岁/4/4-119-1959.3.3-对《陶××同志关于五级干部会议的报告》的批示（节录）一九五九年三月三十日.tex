\section[对《陶××同志关于五级干部会议的报告》的批示(节录)一九五九年三月三十日]{对《陶××同志关于五级干部会议的报告》的批示(节录)一九五九年三月三十日}
\datesubtitle{(一九五九年三月三十日)}


有些地方不是这样,他们怕鬼,不敢将郑州要点立刻一杆子通到生产队,生产小组和全民中去。他们无穷忧虑,怕天下大乱,不可收台。

这三条办法好。群众一到,魔鬼生消。本来没有鬼,只是在一些同志的大脑皮层里感觉有鬼,这个鬼的名字叫“怕群众”。

牢骚也罢,反动言论也罢,放出来就好。牢骚是一定要人家发的,当然发者无罪。反动言论放出来后,他们会立刻感觉孤立,他们自己会做批判。不批判也不要紧,群众的眼睛中已经照下了他们的羞像,跑不掉了;也可以实行言者无罪这一条规律。现在是一九五九年了,不是一九五七年了。

旧账不能不算这句话,是写到郑州讲话去了的,不对。应改做旧账一般要算。算账不能实行那个客观存在的价值法则。这个法则是一个伟大的学校,只有利用它,才有可能教会我们的几千万干部和几万万人民,才有可能建设我们的社会主义和共产主义。否则一切都不可能。对群众不能怨气。对干部,他们将被我们毁坏掉,有百害无一利。一个公社竟可以将原高级社的现金收入四百多万元退还原主,为什么别的社不可以退还呢?不要“善财难舍”。须知这是劫财不是善财。无偿占有别人劳动是不许可的。对湖北省委报告麻城经验的批语一九五九年四月三日

算账才能团结,算账才能帮助干部从贪污浪费的海洋中拨出身来,一身干净;算账才能教会干部学会经营管理方法;算账才能教会五亿农民自己管理自己的公社,监督公社的各级干部只许办好事,不许办坏事,实现群众的监督,实现真正的民主集中制。



