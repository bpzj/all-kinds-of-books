\section[接见古巴文化、工会、青年等代表团的谈话(一九六三年七月二十六日)]{接见古巴文化、工会、青年等代表团的谈话}
\datesubtitle{(一九六三年七月二十六日)}

\begin{duihua}

\item[\textbf{主席:}] 文化代表团到什么地方去看了一下没有?

\item[\textbf{奥斯明费尔南德斯(以下简称奥):}] 到上海、广州、武汉去参观过。

\item[\textbf{主席:}] 一共有多少时间?

\item[\textbf{奥:}] 刚好二十六天。

\item[\textbf{主席:}] 去年代表团是什么时候到的?

\item[\textbf{佛利斯·库左拉(以下简称佛):}] 上个礼拜天到的。

\item[\textbf{张××:}] (以下简称张)。周总理昨天接见时,希望青年代表团到延安去看看。

\item[\textbf{佛:}] 昨天中央的同志对我们讲,可以到延安去参观。

\item[\textbf{主席:}] 那个地方很落后。到落后的地方去看看,也有好处。文化代表团来签订文化协定,签好了没有?

\item[\textbf{奥:}] 前天中午(二十四日)已签了字,我们非常满意。

\item[\textbf{主席:}] 非常满意吗?

\item[\textbf{奥:}] 是。非常满意。

\item[\textbf{主席:}] 协定包含些什么内容?

\item[\textbf{奥:}] 协定的主要内容,是关于两国互派代表团访问的事情,古巴方面将邀请中国杂技艺术团去访问,同时我们也要派艺术团来中国。另外,双方要互派语言留学生,交换影片、书刊以及其他文化资料等。

\item[\textbf{主席:}] 我们的影片,恐怕不那么高明。

\item[\textbf{奥:}] 电影是教育人民群众的有力工具,我们相信,中国电影能够起到这个作用。

\item[\textbf{主席:}] 影片的作用是不小,但我们的好片子太少。

\item[\textbf{奥:}] 中国的电影事业,像其他经济建设事业一样,正在飞快地向前发展。

\item[\textbf{主席:}] 我们的胶片还不能够完全自造。你们能够制造胶片吗?

\item[\textbf{奥:}] 古巴不出胶片。

\item[\textbf{主席:}] 是进口吗?

\item[\textbf{奥:}] 进口。古巴的电影,主要是些记录片和短片。

\item[\textbf{主席:}] 慢慢发展下去,就会搞长片。我们的杂技,是历史上遗留下来的玩意。

\item[\textbf{奥:}] 到哈瓦那去的,是由武汉的杂技团组成的,我们亲自到武汉去看过他们的节目。

\item[\textbf{主席:}] 杂技团去过古巴没有?

\item[\textbf{张:}] 没有去过。以前,去过歌舞,京剧团。

\item[\textbf{主席:}] 杂技团过去不是去过吗?

\item[\textbf{张:}] 那是到拉丁美洲巴西等国访问。当时,古巴革命还未成功,我们的艺术团进不去。

\item[\textbf{主席:}] \marginpar{\footnotesize 55}古巴革命进展很快,几年之内,就战胜了敌人。我们的革命时间很长,搞了二十几年。有本国的敌人,也有外国的敌人。外国的敌人是日本,我们和他打了八年;国内的敌人是蒋介石,和他打了十四年。日本没有打进来以前,和蒋介石打了十年,日本投降后,又打了四年。蒋介石后面有美帝国主义支持。后来,又在朝鲜和美国人打了差不多三年。有很多人怕美国人。在我们国家,也有很多人怕美国人。打了三年之后,怕美国的人少了。现在,美国人又在越南南部打,越南南方人民只有很落后的武器,而美国有很多新式武器,包括飞机、大炮、直升飞机,细菌战,化学战,但是,打了五、六年,南方人民和军队在发展,根据地在不断扩大,他们不怕美国人了。这些经验很值得讲一讲。

\item[\textbf{在某种程度上,美国兵还不如日本兵。我们和日本打了八年。美国的士兵还是有战斗力的,他们武器多、武器好,但是,他们不喜欢打仗。美国帝国主义为了霸占全球,用战争威胁全世界人民,作各种战斗准备。一种是核战争,这种战争有可能打,也有可能不打;另一种是常规武器的战争。他们的方针是打常规武器的战争。在越南打的就是这种常规战争。过去,我们和朝鲜人民打的,也是这种战争。那时,他们不是没有原子弹,而是不敢打;现在原子弹更多了,可是在越南南部也不敢打。美国现在的三军参谋长泰勒,此人就是在朝鲜和我们打过的。他写了一本书大家有机会最好看看。书名是:}] 《音调不定的号角》。在这本书里,他批评杜鲁门和艾森豪威尔过去是不重视常规武器的。战争,叫喊打原子战争,但又不打,这就叫做音调不定。泰勒在艾森豪威尔执政时期,当过陆军参谋长,后来因为不能实现自己的主张,辞职不干了。肯尼廸当选为总统后,他又起来了,当了陆、海、空三军的参谋长。

据您们看,全世界人民现在还那么怕美帝国主义吗?

\item[\textbf{奥:}] 从我们来看,现在世界各国的民族解放运动蓬勃发展,在拉丁美洲,如委内瑞拉,秘鲁等国,人民的革命战争日益高涨他们国家虽然很小,进行常规武器的战争力量不够,但还是在那里进行斗争,他们不怕美帝国主义,敢于反对在美国支持下的独裁反动统治。

\item[\textbf{主席:}] 他们和你们一样。如果只是怕,把手缩回来,让人家抓住关在监牢里,或者杀掉,那革命的火焰就熄灭了。我们参加朝鲜战争的时候,和蒋介石打仗的时候,在我们的队伍中,也有很多人怕美国人,但是,怕又有什么办法呢?难道能解除武装吗?那个时候,大多数人无所谓怕不怕,因为敌人已经打来了,怕又有什么用?打的结果,我们胜利了,可见不必那么怕,怕美国人是多余的。除美国之外,还有一些帝国主义国家,如法国,他们在越南北部打,结果胡志明同志打胜了。在阿尔及利亚,阿民族解放军也打胜了。他们打了七年,经历了许多艰难,法国军队多到八十万之多,而阿民族解放军只有三、四万人。他们很多人都同我讲过,包括现在的议长,过去的总理阿巴斯。你们知道阿巴斯吗?

\item[\textbf{奥:}] 知道他的名字。

\item[\textbf{主席:}] 他们对我们谈过很多,谈过他们的困难,阿牺牲了九分之一的人口,就是说,九百万人口中死了一百万。另外,法国军队领导机关还出版我写的小册子,企图打败阿民族解放军。我告诉他们,我写的小册子,是人民战争的小册子,反人民战争的那一方面,不可能利用,他们想利用,实际上是不可能的。\marginpar{\footnotesize 56}我们在国内战争时期,蒋介石也利用我写的小册子,想把我们打败,结果还是不行。美国人也想利用我们的办法,他们有很多人研究中国的游击战、运动战的战略战术,但是,在朝鲜战争中间,没有得到什么好处。

在朝鲜战争中,除美国外,还有十五个国家参了战。包括美国,法国,加拿大,澳大利亚,新西兰,土耳其,哥伦比亚等。当然主力是美国人。美国去了一个师,土耳其去了一个旅,哥伦比亚去了一个营,他们是凑合起来的。他们到朝鲜来,是打着联合国的旗子。打了二年半(快三年),谈判就整整谈了二年,当时是一面打,一面谈。谈判的地方就是双方交界的一个小地方——板门店。

\item[\textbf{奥:}] 我们在朝鲜时去过那个地方。

\item[\textbf{张:}] 文化代表团访问中国之前曾到过朝鲜。

\item[\textbf{主席:}] 朝鲜很可以去看一看。要研究那个时候怎么一面打,一面谈;怎么用很少的人和武器,战胜了拥有强大武器的敌人。

\item[\textbf{奥:}] 我们在朝鲜呆了十天。

\item[\textbf{主席:}] 在朝鲜看过防御工事没有?

\item[\textbf{奥:}] 我们在朝鲜访问时,去了开城的板门店,参观了军事博物馆,展览馆中有一个一千一百一十二高地模型,山头被敌人打去了两公尺。

\item[\textbf{主席:}] 山里面有我们的工事,有地道,我们用各种办法对付敌人,山地有山地的办法,平原有平原的办法。不管敌人武器多么好,多么强,因为他们是反对革命,不利于人民的,不可能得到胜利。他们的道路只有一条,就是失败。最后,所有的帝国主义和各国反动派,都要灭亡,我们在座的人就是证据。美国支持的反动派巴蒂斯塔,不是也失败了吗?你们在和反动派斗争中,有外国援助没有?

\item[\textbf{奥:}] 只有七支步怆。

\item[\textbf{主席:}] 是啊!没有任何外国援助,就是步枪,而且很少,也胜利了,我们中国的战争也证明了这一点。你们在强大的敌人面前,因为敢于和它打,终于打胜了。

恩来同志劝青年代表团到延安看看,是有理由的。延安虽然很落后,但在当时是我们领导机关的所在地。蒋介石在南京、上海,南京是一百万人口的城市;上海是七百万人口的城市,那里住着外国人,很容易得到外国人的援助,延安城和附近只有七千人。看来,不是大城市打胜小城市,而是小城市打胜大城市;不是大城市打胜乡村,而是乡村的人民包围城市,最后夺取城市;不是有大炮、飞机、坦克的敌人打胜我们,而是只有步枪、轻炮、手榴弹的人民军队打胜了敌人。起初,我们没有大炮,打到第二年,就有了大炮,打到第三年,大炮就更多了。是哪里来的呢?是美国的,是蒋介石运输给我们的。后来,我们也有了坦克,也是蒋介石送的。当时就是没有飞机,现在空军是解放以后建设起来的。没有空军也可以打胜有空军的,飞机的作用很小,打不死几个人,破坏不了多少工厂和房屋。你们有人到过伦敦没有?

\item[\textbf{奥:}] 我们代表团中,有一个同志到过。

\item[\textbf{主席:}] 伦敦在第二次世界大战中,处于很危险的地位。英国军队从敦克尔克撤退后,军队毫无组织,海防线没有防御了,空军不能防卫伦敦了,海军也不能保卫运输线,陆军、空军损失很大。当时,希特勒的空军猛烈轰炸伦敦,可是并没有破坏了多少,据我们这里看过的同志讲,后方陆海空军很快就恢复了,\marginpar{\footnotesize 57}所以说,空军的作用不大。在战争当中决定胜负的还是步兵、陆军。你们在战争中,敌人有无空军?

\item[\textbf{奥:}] 他们猛烈轰炸解放区。

\item[\textbf{主席:}] 作用大不大?

\item[\textbf{奥:}] 唯一的效果,就是敌人费了很多钱。

\item[\textbf{主席:}] 我们在解放战争中,国民党也曾大编队轰炸延安,有一次,打死了一条猪,一个人也没有打伤打死,另外一次,在乡下打死了两个老百姓。今天谈的太多了吧!

\item[\textbf{奥:}] 我们非常渴望听到毛主席的谈话。

\item[\textbf{主席:}] 我今天是谈了一点历史,以后有机会再谈。

 
\end{duihua}
