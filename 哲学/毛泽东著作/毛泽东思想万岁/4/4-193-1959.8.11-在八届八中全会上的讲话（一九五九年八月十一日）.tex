\section[在八届八中全会上的讲话(一九五九年八月十一日)]{在八届八中全会上的讲话}
\datesubtitle{(一九五九年八月十一日)}


讲讲世界观,人生观问题。人生观这个词,在外国书上看得很少。中国人喜欢说人生观。其实世界观、人生观是一个东西,不是两个东西。所谓人生观就是社会观,世界观,包括对自然界和人类社会两部分。为了通俗起见,说人生观也可以。在一部分同志中,这个问题几十年来没有解决。就是说,他们是经验主义的人生观世界观,不是马克思主义的世界观,还有个方法论,它是世界观,又是方法论.有些同志讲话只讲方法论,不讲世界观。这就是讲历来犯错误的同志,据我观察,以及中央常委和他们交换意见的结果,他们的世界观方法论,不是马克思主义的,不是辩证唯物论,历史唯物论的,而是主观唯心论,是主观唯心主义的经验论。从外国流派来说,是列宁所批判的马赫主义。俄国有造神论、神论和迷神论,把精神来造人。是谁呢?是不是芦那察尔斯基、波格达洛夫,他的著作我看过,现在忘记了,马赫主义说,是奥地利和德国的,美国叫实用主义,又叫功利主义,是一个东西。他们不承认客观存在,不承认客观真理,没有客观标准。不承认自然界物质世界是独立于人们意识之外的。客观真理是固有的,不是个人主义的个真理。由感性到理性,客观真理又变成主观真理。

主观真理是来自客观真理。同宗教不同,和主观唯心主义不同。相反的。我们的这些同志一厢情愿相思,中国的一句话,叫自以为是.而不是实事求是。实事是客观真理,求是者是主观反映客观真理,要经过脑筋。要千百次反复的感觉,然后变成概念,山水草木、人马牛羊、鸡犬猪都是概念;人有资产阶级的人和无产阶级的人。男,女、老、幼都是概念。都是抽象来的。具体的人,张三李四。如董老,林老、吴老,你们是老年人,你们的娃娃是青年人。你们的夫人叫女人。

历来犯错误的人,都是部分的、大部的或者是全部的主观唯心论。所以难以改造,要改变他们的世界观方法论。英国巴克莱的唯我主义,是主观唯心论的极端流派.是最主观唯心主义的一个宗教家,一个大哲学家。他的名言:为什么有我,因为我想。我不想。我就没有。世界也是“我思则在”,否则世界上也没有了。他们是唯我主义。

为什么要从这谈起,因为历来一讲到政治问题都不讲世界观。最根本的问题就是宇宙观。不谈不行。一切要从这里开始。


