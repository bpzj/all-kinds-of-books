\section[在听取××部汇报工作时的指示(一九五六年三月八日)]{在听取××部汇报工作时的指示}
\datesubtitle{(一九五六年三月八日)}


(当汇报基本情况时)

××部汇报,列了目录,一目了然,各部都要这样作才好。

你们可以搞点展览会,把各种交通工具展览一下。

学院学生太少了,可怜得很。

将来可能有一、两千万吨船,中国可能突出。世界海船吨位,我国不到百分之×,这表现我国太穷。

公路少,应多修。地方工业和修公路都要发挥地方积极性。

(当汇报第一个五年计划执行情况与今后措施时)

运价高,束缚生产力,工农业产品交换不起来。

“一长制”刺眼得很,怎么不改个别的名词?需要强调集体领导,分工负责制。降低运价问题,财政方面打小算盘不妥当,运价高,束缚生产,税收减少,结果害自己。工农业都在发展,两方面交流,靠运输纽带,中间路太狭,通不过去,发生矛盾,就影响两方面发展。过去地方公路标准太高,修不起来。

有许多事,开始听,好像有道理。学习苏联经验要结合实际才行,脱离实际就错了。

地方交通会议后,情况有转变,可见领导还是重要的,不然六亿人有六亿方向,不得了。开片会办法很好,一年一次,也可一年开两次,先规定会期,下边就动起来(如消灭血吸虫病)。各种铁、木轮车改为胶轮车,要赶快改。

人力车已合作化百分之二十五。合作化当然要得,怎样化法要注意,有些先挂个号也可以,不挂心不安,能组织就组织,不能则不勉强。

对资本家赎买政策,列宁想干而不能干,那时对资本家无利益,现在资本家愿意不愿意?现在小资本家愿意,工人愿意,大资本家随潮流走。现在问题是怎样安插。中国资本家特点之一是许多资本家有经营管理知识,资本家思想不一定比高岗、饶漱石、潘汉年坏,比陈光、戴季英坏。资本家有能力的骨干,应放在领导地位。私有财产是否三年收归国有,还要考虑,将来可能给他们福利基金。

福建省委提的对个体车船改进意见,你们可以拿几个社试一下。每个省试一下,再下决心。

肃反,要清除真正的反革命,不要搞错。真反革命有技术的,已经知道了,在次要岗位上亦可控制使用。

灯光装猪法是怎么回事?可见强迫命令不行,猪尚不能,何况人乎?

(当汇报远景规划时)

一九六七年达到××万吨船,这还像个样子。中国地势比较完整,东面是大海,西面是高山,统一起来,帝国主义不易进来,发展航运有重大意义。

(当汇报几个问题时)

机关汽车统一管理方针很对。汽车,主要看需要,要与工农业相适应。

交通和轻工业如不能适应,有一天我们会挨骂的。

地方运价降低问题,开个会先商议透,看准后再决定。

简易公路,要个标准,各地乱修,则不得了。

你们对的意见我都支持,我已经说过把事业办的又多、又快、又好、又省。


