\section[毛主席视察华北、中南和华东地区时的重要指示 ]{毛主席视察华北、中南和华东地区时的重要指示 }
\datesubtitle{(记录稿,未经本人审阅。)}


毛主席说,七、八、九三个月,形势发展很快。全国的无产阶级文化大革命形势大好,不是小好。整个形势比以往任何时候都好。

形势大好的重要标志是人民群众充分发动起来了。从来的群众运动都没有像这次发动得这么广泛,这么深入。全国的工厂、农村、机关、学校、部队,到处都在讨论无产阶级文化大革命的问题,大家都在关心国家大事。过去一家人碰到一块,说闲话的时候多。现在不是,到一块就是辩论无产阶级文化大革命的问题。父子之间、兄弟姐妹之间、夫妻之间,连十几岁娃娃和老太大,都参加了辩论。

毛主席说,有些地方前一段好像很乱,其实那是乱了敌人,锻炼了群众。

毛主席说,再有几个月的时间,整个形势将会变得更好。

毛主席号召各地革命群众组织实现革命的大联合。毛主席说,在工人阶级内部,没有根本的利害冲突。在无产阶级专政下的工人阶级内部,更没有理由一定要分裂成为势不两立的两大派组织。一个工厂,分成两派,主要是走资本主义道路的当权派为了保自己,蒙蔽群众,挑动群众斗群众。群众组织里头,混进了坏人,这是极少数。有些群众组织受无政府主义的影响,也是一个原因。有些人当了保守派,犯了错误,是认识问题。有人说是立场问题,立场问题也可以变的痲。站队站错了,站过来就是了。极少数人的立场是难变的,大多数人是可以变的。革命的红卫兵和革命的学生组织要实现革命的大联合。只要两派都是革命的群众组织,就要在革命的原则下实现革命的大联合。两派要互相少讲别人的缺点、错误,别人的缺点、错误,让人家自己讲,各自多做自我批评,求大同,存小异。这样才有利于革命的大联合。

在谈到革命大联合以谁为核心时,毛主席说,什么“以我为核心”,这个问题要解决。核心是在斗争中实践中群众公认的,不是自封的。自己提“以我为核心”是最蠢的。王明、博古、张闻天,他要做核心,要人家承认他是核心,结果垮台了。什么是农民,什么是工人,什么打仗,什么打土豪分田地,他们都不懂。毛主席说,要正确地对待受蒙蔽的群众。对受蒙蔽的群众,不能压,主要是做好思想政治工作。

向坏人专政的问题。毛主席说,政府和左派都不要捉人,发动革命群众组织自己处理。例如,北京大体就是这样做的。专政是群众的专政,靠政府捉人不是好办法。政府只宜根据群众的要求和协助,捉极少数的人。

一个组织里的坏头头,要靠那个组织自己发动群众去处理。

关于干部问题。毛主席说,绝大多数的干部都是好的,不好的只是极少数。对党内走资本主义道路的当权派,是要整的,但是,他们是一小撮。我们的干部中,除了投敌、叛变、自首的以外,绝大多数在过去十几年、几十年里总做过一些好事!要团结干部的大多数。犯了错误的干部,包括犯了严重错误的干部,只要不是坚持不改,屡教不改的,都要团结教育他们。要扩大教育面,缩小打击面,运用“团结——批评和自我批评——团结”这个公式来解决我们内部的矛盾。在进行批判斗争时,要用文斗,不要搞武斗,也不要搞变相的武斗。有一些犯错误的同志一时想不通,还应该给他时间,让他多想一个时候。要允许他们思想有反复,一时想通了,遇到一些事又想不通,还可以等待。要允许干部犯错误,允许干部改正错误。不要一犯错误就打倒。犯了错误有什么要紧?改了就好。要解放一批干部,让干部站出来。

毛主席说,正确地对待干部,是实行革命三结合,巩固革命大联合,搞好本单位斗、批改的关键问题,一定要解决好。我们党,经过延安整风,教育了广大干部,团结了全党,保证了抗日战争和解放战争的胜利。这个传统,我们一定要发扬。

关于上下级关系问题。毛主席说,有些干部为什么会受到群众的批判斗争呢?一个是执行了资产阶级反动路线,群众有气。一个是官做大了,薪水多了,自以为了不起,就摆架子,有事不跟群众商量,不平等待人,不民主,喜欢骂人、训人。严重脱离群众。这样,群众就有意见。平时没有机会讲,无产阶级文化大革命中爆发了,一爆发,就不得了,弄得他们很狼狈。今后要吸取教训,很好地解决上下级关系问题,搞好干部和群众的关系。以后干部要分别到下面去走一走,看一看,遇事多和群众商量,做群众的小学生。在某种意义上说,最聪明、最有才能的,是最有实践经验的战士。

要讲团结。干部有错误,有问题,不要背后说,找他个别谈,或者会议上讲。

我们现在有的严肃、紧张有余,团结、活泼不足。

关于教育干部的问题。毛主席说,干部问题,要从教育着手,扩大教育面。不仅武的(军队),还要文的(党、政),都要进行教育,加强学习。中央、各大区、各省、市都要办学习斑,分期分批地轮训。每省都要开县人武部以上各级干部会,一个省二、三百人,多则四、五百人,大省应到千人左右。半年之内争取办好此事,否则,一年也可。

今后,争取每年搞一次,每一次的时间不要太长,大体上两个月左右。

毛主席教导我们,对红卫兵要进行教育,加强学习。要告诉革命造反派的头头和红卫兵小将们,现在正是他们有可能犯错误的时候。要用我们自己犯错误的经验教训,教育他们。对他们做思想政治工作,主要是同他们讲道理。

毛主席在视察各地的过程中,高度赞扬了广大工农群众、人民解放军指战员、红卫兵小将、革命干部和革命知识分子,在一年多来的无产阶级文化大革命中建立的功勋。毛主席号召他们,要斗私、批修,要拥军爱民,要抓革命促生产、促工作、促战备,把各方面的工作做得更好,把无产阶级文化大革命进行到底。

