\section[ “山西崞县是怎样进行土地改革的”一文的按语(一九四八年三月十二日)]{ “山西崞县是怎样进行土地改革的”一文的按语}
\datesubtitle{(一九四八年三月十二日)}


这是山西崞县的一篇通讯。在这个通讯中说明了那里的群众斗争已经展开,群众对于分配土地业已完全酝酿成熟,在一个农民的代表会议上完成了平分土地的一切准备。那里对于划分阶级成分,曾经划错了许多人,但是已经公开地明确地经过群众代表的讨论,决定改正,对于不给地主以必要的生活出路,不将地主富农加以区别,侵犯中农错误等项错误观点,作了批判。总之,在这篇通讯中所描述的两个区的农民代表会议上所表现的路线,是完全正确的。在作者写这篇通讯时,崞县还没有实行分配土地,因此,这个经验还不完全。我们希望在当地实行改正划分阶级中的错误(这是一件大事),实行平分土地以及组织生产,改造政权等项工作完成以后,再有一篇综述这整个过程的通讯。关于如何在农村中进行整党工作,我们有了晋察冀区平山县的典型经验。关于如何在老区调剂土地而不是平分土地(因为那里已经平分了)的工作,我们有了陕甘宁绥德县黄家川的典型经验。现今又有了晋绥区崞县这样一个平分土地的经验(虽不完全)。这个经验,值得印成一个小册子,发给每个乡村的工作干部,这种叙述典型经验的小册子,此我们的领导机关发出的决议和指示文件,要生动丰富得多,能够使缺乏经验的同志得到下手的办法,能够有力的击破在党内严重地存在着的无马列主义的命令主义和尾巴主义。各中央局、中央分局及前委的领导同志们,在对自己领导的各项重要工作发出决议和指示之后,应当注意收集和传播经过选择的典型的经验,使自己领导的群众运动按着正确的路线向前发展。现在是成千万的人民群众依照党所指出的方向向着封建的买办的反动制度展开进攻的时候,领导者的责任,就是不但指出斗争的方向,规定斗争的任务,而且必须总结具体的经验,向群众迅速传播这些经验,使正确的获得推广,错误的不致重犯。

