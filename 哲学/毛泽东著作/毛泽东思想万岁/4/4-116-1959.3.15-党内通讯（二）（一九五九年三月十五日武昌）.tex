\section[党内通讯(二)(一九五九年三月十五日武昌)]{党内通讯(二)(一九五九年三月十五日武昌)}
\datesubtitle{(一九五九年三月十五日)}


各省、市、区党委第一书记同志们:

我到武昌已经五天,看了湖北六级干部大会的材料,同时收到一些省、市、区的材料,觉得有一个问题需要同你们商量一下。河南文件已经送给你们,那里主张以生产队为公社的基本核算单位和分配单位。我在郑州就收到湖北省委三月八日关于人民公社管理体制问题和粮食问题规定,其中主张“坚决以原来的高级社即现在的生产队为基本核算单位,原高级社已经分为若干生产队的,应合为一个基本核算单位。合队得再分。少数原高级社规模很小,经济条件大体相同,已经合为一个生产队的,只要是这些社的干部和社员愿意合为一个基本核算单位,可以经过公社党委审查决定,并报县委批准。”我到武昌,即找×××同志来此,和王××同志一起,谈了一下。我问××,你们赞成河南办法,还是赞成湖北办法?他说,他们赞成河南办法。因为他们那里一个生产大队大体上只管六个生产队。而这六个生产队,大体上是由三个原来的高级社划成的,即一个社分为两个队。后来又收到广东省委三月十一日报告,他们主张实行三定五放。三定中的头一定是“定基本核算单位”,一律以原来的高级社(广东省原有三万二千个高级社,平均每社三百二十户左右)为基础,有些即大体相当于现在的生产队(或大队),有些在公社化后分成二、三个生产队的,可以立即合并,成为一个新的队,做基本核算单位。原有的高级社如果过小,一个自然村有几个社的及虽在一个村,而经济条件悬殊不大经群众同意,也可以合并做为社的基本核算单位,这样,河南、湖北两省均主张从生产大队(管理区)为基本核算单位,究竟那一种主张比较好?或者两者可以并行呢?据王××同志说,湖北大会这几天正辩论这个问题,两派意见斗争激烈。大体上县委、公社党委、大队(管理区)多主张以大队为基本核算单位,我感到这个问题重大,关系到三十多万生产队长,小队长等基层单位干部和几亿农民的直接利益问题,采取河南、湖北的办法,一定要得到基层干部的真正同意,如果他们感到勉强的,则宁可采取生产队即原高级社为基本核算单位,不致使我们脱离群众,而在目前这个时间脱离群众,是很危险的,今年的生产将不能达到目的。河南虽然已经做了决定,但是,仍请省委同志在目前正在召开的四级干部会议上征求基层干部意见,如果他们同意省的决定,就照那样办,否则不妨改一改。“郑州会议记录”上所谓“队为基础”指的是生产队,即原高级社,而不是生产大队(管理区)。总之,要按照群众的意见办事。无论什么办法,只要适合群众的要求,才行得通,否则终久是行不通的。究竟如何办,请你们酌定。

<p align="right">一九五九年三月十五日于武昌</p>


