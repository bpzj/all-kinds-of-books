\section[在全国工商联二届一次委员会代表大会部分代表座谈会上的指示(一九五六年十二月八日)]{在全国工商联二届一次委员会代表大会部分代表座谈会上的指示}
\datesubtitle{(一九五六年十二月八日)}


我对经济是外行,我就讲一些外行话。你们提出很多问题,很好,对我帮助很大,虽然时间不长,我已听到一些。昨天我找了正副主任开会,听到了一些,了解了一些情况,详细的你们开会讨论。陈云、×××副总理还要给你们作报告。你们要解决的问题,政府要和你们商量,协商一个办法,凡能解决的,总要解决的,我还没有研究过,有些问题还不能解答,你们把问题都提出来,政府来研究,陈云副总理考虑,解决一些能解决的问题。

我看工商界这个时期以来,公私合营以后,有很大进步。我们感觉比过去更熟悉了,更加靠拢了。公私合营以后,资本家只剩下定息问题,在社会上的名誉也不同了,并且学习的热情很高,学习政治,学习本事,学习技术和管理方法。听说各地都办了讲习班,每个城市几百人,上海一期几千人,十期就几万人,这都是很大的爱国主义。我们民族工商界的学习热潮,他们愿意为新的国家做工作,学经济,学本领,大中小都愿意学习。

对资本家有大中小,我对资本家的看法和要求比过去有了进步,从前认为改造很困难,你们自己也料不到这样快,学习高潮这样高。你们开展不开展自我批评?(答:开展的。)去年工商联开会时,许多人做了检讨,这个办法是我们共产党的办法,我们也想在民主党派中推行这个办法,但是没有开展得很好。最近民建的会议,用了这个办法,开展了批评和自我批评。这是人民内部解决问题的方法,提出意见,提出要求,解决问题,达到团结。

我们两次革命,资产阶级民主革命已经过去了,没有问题了,现在社会主义革命基本上完成了,但是还没有最后完成,还有很多问题。像农业合作化问题、手工业合作化问题、公私合营问题。就是将来全部改造完成的话,还是有许多问题,问题是层出不穷的。所谓问题,就是矛盾,就是不协调,就是不平衡。生活问题,工作问题,国内外问题,总是矛盾,是充满了矛盾的。从前有人问:俄国为什么出了一个贝利亚?后来斯大林也犯了错误,中国出了高岗,香港也闹其事,还有波兰、匈牙利、苏伊士运河问题,世界上问题多得很,我们今天提的问题还是一个类型,是工商界的问题,大中小的问题。

我现在来谈时局。

你们看社会主义搞得成搞不成?你们有没有吊桶?是不是十五个吊桶打水,七上八下?社会主义恐怕搞不成呢?社会主义阵营要崩溃吧?我看就是崩溃,也没有大事,也没有什么了不起。我看是不会崩溃的,崩溃不了的。社会主义阵营主要是苏联和中国。中国和苏联靠在一起,这个方针是正确的。现在还有人怀疑这个方针,说“不要靠在一起”,还认为可以采取中间路线的地位,站在苏联和美国之间,作个桥梁,就是南斯拉夫的办法,这个办法就是两边拿钱,这里边也拿,那里边也拿,这样做法好不好呢?我认为站在中间,这个办法并不好,对民族不利,因为一边是强大的帝国主义,我们中国是长期受帝国主义压迫的,如果站在苏联美国之间,看起来是很好的,独立了,其实是不会独立的。美国是靠不住的,他会给你一些东西,但不会给你很多,帝国主义怎么能给我们吃饱呢?不会给你吃饱的,帝国主义对亚洲、非洲、拉丁美洲都是压迫的。印度受压迫二百多年,从来不给他们吃饱过,帝国主义是一毛不拨的,帝国主义就是英国、美国、法国、荷兰这些国家,就是八国联军,烧了我们的圆明园,割了我们的香港、台湾,而香港是我们中国的,为什么要割走呢?万隆会议为什么能够团结亚非国家,就是因为帝国主义存心压迫人家,这个帝国主义是美国。拉丁美洲的人到了中国对我们很亲热。我今天还会见了一个巴西的代表。巴西是一个大国,有六千万人口,面积和中国差不多,巴西一直受美帝国主义压迫的。那些幻想处在苏美英之间作桥梁,有所利得,这种想法是不适当的。我们大工厂还不会设计。现在谁来替我们设计大工厂?例如化学工业、钢铁工业、石油工业、坦克、飞机、汽车工业,谁真谁替我们设计的?英国从来没有替印度设计过,最近因为苏联替印度设计了一个钢铁工厂,英国跟着也设计了一个,美国也设计了一个,他们为了争取印度,才替他设计的。印度第一碱厂,是我们中国工程师候德榜先生替他们设计的。帝国主义要保守秘密,没有一个帝国主义给我们设计过。在民主党派、无党派民主人士、高级知识分子、宗教界、工商界的思想里,还有一部分无产阶级中,还幻想美国会帮助我们,英国会帮助我们,我们大家要好好宣传一下,究竟一边倒对不对?我们一边倒是和苏联在一起,我们的一边倒是平等的。我们不感觉有波兰、匈牙利所发生的问题,我们是信仰马克思主义的,不是硬搬苏联经验的,硬搬苏联的经验是错误的。对工商业改造和农业合作化就与苏联不同,他们农业合作化后几年是减产的,我们农业合作化是增产的。对资本主义工商业者改造,不仅所有人包下来,作为阶级是消灭了,作为人都包下来。工商界不是国家的负担,而是一笔财富收入,过去曾经起了积极作用,他们在经济上是现代化的,不是手工业的,在政治上反对帝国主义,所以有两面性,有革命的一面;人民政权建立以来和政府合作,企业又公私合营了,这些好事不能说资产阶级对我们没有用,而是有用的,很有用的。工人对这一点不太了解,因为他们在工厂里从前和资本家是有斗争的,我们应该对工人说清楚。特别是工商界学习高潮以来,你们愿意学习,工人对你们会改变观感的。对资本家要宣传把个人事情和国家事情联系起来,提倡爱国主义,总之要想到国家的事。河北省合作化,过去的口号是“要发家,种棉花”,结果大家只管家。后来觉得这个口号提得不好,改为“爱国发家,多种棉花”,就把家和国家联系起来了。你们现要把家和国家联系起来,这个国家你们也有份的。我们的国家是穷国,而且穷得很,今年只有四百五十万吨钢,明年才有五百万吨钢,日本是七百万吨钢,我们要赶上日本,要第三个五年计划,才达到一千万吨。

你们这个会的中心是讨论你们的事情,同时也要联系国家的事,回去宣传教育时也要联系到国家的事情。我们要几千万吨、几万万吨钢,要隔几十年、几百年。我们要提倡爱国主义。为什么要搞公私合营,要搞社会主义?就是为了便于把国家发展起来,比私有制度更有利于发展国家的经济文化,使国家独立。我们在经济上还是不独立的,大的机器不能做,小的精密的也不能做,只能做中型的。飞机还刚刚开始出厂,汽车还刚刚开始生产。甚么国家替我们设计的呢?是苏联。我们应该和苏联合作。我们的国家里没有像波兰、匈牙利那样的反苏潮流、反苏情绪,工商界中也没有,青年大学生中也没有,大学生百分之七十是资产阶级和地主子弟,我们要团结教育他们。因为是你们的子弟,解决你们的问题,对他们有影响,入学入团,助学金问题,戴红领巾问题,青年团要解决这个问题。要一视同仁,要看质量如何,不要看家庭出身。如果够不上,就是工农子弟也不能取,够得上,资本家子女也应该取。补助金问题,应该看学习和家庭来决定的。大的不要补助,中小的寒微的就应该补助,入团入党都应以此为标准,在中国是完全可以实行的。颜色不一,思想不一,还有两面性,有先进一面,有落后一面,是符合事实的。因素有一个任务,就是学习,如果都是好的,那么每个人都是孔夫子。如果每个在街上跑过的人都是圣人,蒋介石、希特勒都在街上跑过的也是圣人吗?侵略埃及的人、特务分子也是圣人吗?人是有所不同的,整个讲资本家是爱国的,但是有缺点,有正面的东西,也有侧面的因素,因此有学习的任务,否则就不要学习了,变成圣人了。康有为自己说,他三十岁以前做文章很能发挥,三十岁以后学到了顶点,他后来变成复辟,不是没有理由的。一个人不要满足,七十岁、八十岁,还是有很多不知道的东西,世界上的事情还有很多不知道,我们要加紧学习。陈叔老,你是翰林,你是上知天文,下知地理,诸子百家,三教九流,你是不是都精通的?你是翰林,我是不行的,我连秀才都不是,知道得很少。我们要承认自己的缺点,是有好处的。你们中间有秀才,有举人,就要来一个学习任务。

匈牙利事件出得好,还是出得不好呢?既然有问题就不能不出,出了反而好。有脓包总要出脓。那些国家的工作没有做好,一概仿套苏联的办法,没有照顾到具体情况,出了毛病。因此得出一条教训,我们要根据马列主义的普遍真理,要与中国的具体情况相结合起来办事。我们提学习苏联经验的口号,从来没有提过学习他们落后的经验。他们有没有落后的经验?有的,例如肃反工作,他们是公安部门搞的,我们是机关学校来搞的,是地

方党委来领导的,不是由公安部门负主要责任的,我们是全体动员,大张旗鼓地搞的,他们是神秘地搞的。民主党派自己去搞。现在是不是有人怕搞肃反?是不是搞到自己头上?你们有你们的苦,工人有工人的苦。政府要听两方面的意见。你们有就业问题,工人也有就业问题。你们有失业的问题未解决,工人也有失业的问题未解决。匈牙利事件证明,他们国家里面藏了许多反革命分子,成立了反革命的司令部,早几个月以前就在布置了,和外国也有勾结的。中国完全不同,反革命分子基本上肃清了,工商界也一样,只有个别的。关于过去与国民党有勾结,有往来,特别是大的,不勾结不行,那个不算。我们党也和国民党有过勾结,我也以社会贤达的身份当过参议员,我到重庆去参加过会。过去进过国民党、三青团讲一讲就行了,可以不算。所谓反革命分子有现行活动,如果是现行犯不管是哪个阶级都不行。我想是很少的。现在工商联比较干净了。现在的工商界不是以前的工商界了,反革命分子少了,可以告诉大家放心,个别的是例外。在我们国家里面,匈牙利事件是不会发生的。从去年潘汉年、胡风事件以来,到今年审查了四百多万人搞出了十六万嫌疑分子,查出了确实隐藏的只有三万八千人是反革命分子,只占百分之一点二。过去估计只占百分之五左右,是估计不对,是主观的,不符合客观事实。实际上机关学校只有百分之一多一点,社会上更少了。这是细菌微生物,不把它去掉,一生二,二生三,三生万物,在里面秘密捣乱,搞清了也教育了广大的人。其余十二万多人,宣布无罪,搞错了,我们赔礼道歉,那三万多人,一个不杀,大约百分之一劳改,其余的人都在原单位工作。今年又布置了×××人的清理,方法更讲究,还剩下四百万人要清理。主要工厂、私营企业现在不清,免得惊动大家,大家放心,将来清理要你们参加的,要工商联、民建和共产党城市组织都要参加,一道去清理,又是对个别的,真正的反革命分子,不能牵涉到过去和国民党有过关系的,参加过国民党、三青团的不算,过去做坏事后来做好事也不算。今天讲的是政治问题,你们关心的是经济问题,恐怕你们就听不进去。总的来说,中国和外国的形势是好的。匈牙利的事情基本上解决了,但是乱子还会有的,世界上还会出乱子的,不要认为出了高岗,出了贝利亚,斯大林犯了错误就了不起,匈牙利事情了不起,大家都可以睡得着觉。十月十日香港的人有些睡不着觉,过去几天就睡着了。你们有意见说政府帮助不够,你们可以提出来,我们可以多帮一些忙,我们政府的性格你们都摸熟了,我们就是这些人,是跟人民商量办事,是跟工人、农民、资本家、小资产阶级、民主党派商量办事的,可以叫它商量政府,不是板起面孔教训人的,不是意见提得不对就给他一捧子,打得他头向下,脚朝天。我们叫人民政府,你们有意见尽可讲,我们不会借故整人。

你们关心大中小问题如何解决,定息问题如何解决。大中小应分阶层。我现在提一点意见,是否对,请你们考虑。把小的占到百分之八十到百分之九十,不划入资产阶级范围内,叫做上层小资产阶级,过去这样搞过。像医生带两三个徒弟也是小资产阶级,渔民、船民雇用十几个人的也不叫资产阶级。在土改当中,有种小土地出租者,不叫地主,这样就使几万人得到好处,对于这顶帽子他们很感兴趣。资本家代理人或定息少的,如果不愿意干要老保,可以把他们划出来,不是今年是明年、后年划出来,以免使百分之九十的人不要定息,使百分之九十的人过不去。剩下来的就是所谓大的。照经济学来说,美国也是这样做法,三十人以上就算大工厂,二十人以下、十五人以下的愿意放弃就放弃,不愿意放弃就叫他拿下去。关于定息的时间问题大家很关心,有一个原则,就是要解决问题,不要损害他们的利益,特别是不要损害大的利益。还是大的对国计民生作用大,还是小的对国计民生作用大?小的人数多,占百分之九十,但他们的经济不决定国家的经济生活,对国民经济作用大的还是大的。你们说主席专门照顾大资本家,不照顾小的,是不是右倾机会主义?你们分析一下,是不是照顾小的。小的人数多,不注意安排是错误的,要替他们解决各方面的问题。大的因为他们就是大,没有别的,一个大的抵几千万个小的。我们党对资本家也有过中小路线,例如统战部。这个路线应当承认是不对的。中小必然照顾,现在把他们放在小资产阶级范围里来解决。农民是小资产阶级,是乡村的小资产阶级,城市也有小资产阶级,现在小的定息只有几包香烟的就是小资产阶级,资方代理人也划过来。公私合营,国有化,大的国家意义很大,没有工业,我们就不能活,没有农业,我们就没有饭吃。不把大资本家很好照顾,百分之九十都摘了帽子,剩下的只有几个,面子上不光彩。别的都是红色的,我们还是白色的,就不好看。

定息问题到底多少时间,中共中央谈过,时间太短不好,赎买就真正赎买,不是欺骗的,花不了多少钱。有人问究竟还有多少年?“八大”文件里已有这个意思,还要和你们商量,可以不可以解决问题,小的会反对的,工人会反对的,工人会说太便宜了资本家。照工人的道理马上取消,中、小的搞一、二年就行了,为什么这样长?有两个抵触,第一是工人,第二是百分之九十的中小,他们的眼睛都红了,你们的生活好。这个就要说服,要赎买就赎买到底,不要半赎买半没收。中国民族资产阶级有两重性,有革命的一面,但实际本钱不大,工业私股只有十七亿,折合美金七亿不到,只有这一点儿,我们国家怎么会不受欺侮呢?帝国王义欺负我们是有道理的。民族资本只有这一点点,要赎买就全部赎买,不要省这个钱,要说服工人,不要损害大资本家的利益,对整个国家是有利的。不要吝惜几个钱,不要虎头蛇尾,有始无终。我们还可以拖一个尾巴,拖到第三个五年计划。你们看怎么样?人还有个尾巴痕迹。七年是虎头虎身,如果还不能解决问题,还可以拖一拖,还可以再拖到第三个五年计划,只要人情讲得过去。中小人数很多,要帮助她们解决问题。大资本家人少,但他们的资本大,比中小作用来得大,所以中小路线是不对的,应该是大中小路线。民建是大的路线,以大的为主是对的。工商联大会代表中小占多数,要帮助他们解决问题,同时要向他们说明,不要损害大的利益,因为大的对国民经济起很大作用,损害他们的利益对工人、农民,对国家、中小都不见得有利。中小的利益是尽早摘帽子,大的想拖长。你们可以各搞各的,要早摘就早摘,老虎头身七年,中小愿想拿还可以拿,这样做还可以不可以?现在没有法律规定,大家可以商量,到了七年如果不能解决问题,再拖一个尾巴也可以,因为老虎总是有尾巴的,我们第一个要照顾到工人的利益,资本主义工业有一百六十万工人,资本主义商业也有九十万店员,合起来二百五十万,还牵涉到国营的工人、店员,他们会不赞成的,在这个问题上,他们和党是有矛盾的。是否有右倾机会主义?是不是变成了资本家的党?要对他们说明,这对整个民族是有利的,对工人、农民、小资产阶级、中小工商业者都是有利的。这个“利”他们一时是不懂的。大学生百分之七十是资产阶级的子弟,他们是不要继承权,但是对于政府这样对待资本家,他们是会满意的。民主党派、无党派民主人士、少数民族上层领袖,宗教界是会同意的,很快的取消定息,他们会不满意的。还有外国人,外国人到中国都到上海看看荣毅仁先生,看看他有几部汽车,一部还是两部,房里有没有钢琴。法国有个资本家看过以后说,只要法国共产党这样做,他就放心了。对亚非国家和西欧一些国家影响也很大,所以急于取消定息是没有好处的,而会损失很大,实际上没有几个钱,一年只有一亿二千五百万,有人说只有一亿一千万,七年共计八亿。这个钱没有送给日本人、美国人是送给中国人的。总之,“肥水不落别人田”,是“楚马楚得”是国家的购买力,也是公债抵销的对象,还可以开工厂,从各方面可以考虑。你们会议代表中小占多数,就要解决得说明问题啦。对中小要说明道理,要中小的代表负责说,大的不好说,如果大的说出几个好处,就要七年.这是不好说的,要××、×××副总理去说,在座的代表中小的也要去说人家共产党是不是要取消大的定息。荣毅仁先生的资本要抵到一个半北京,大家都看着他,全国十多万资本家,大约大的只有几千户人家,只看这几千户,不看那十二万中小户的。我们要整大的很容易,来一个高潮,敲锣打鼓就敲掉了。中小摘帽子不要一阵风,不要登报,一登报就来了,不搞高潮。这次会议如果报导,只说工商联、民建会代表团长都参加了,我讲了国内外的政治,作了分析,对经济问题也讲了些话,内容不要讲了。如果内容宣布了,工人和中小都要骂我,共产党就是要共产的,那有不共产之理。人家说共产党三头六臂很凶,实际上我们只有一个头,两只手,我们做事要“顺乎天理,合乎人情,合乎世界之潮流,人民之需要,而为先知先觉者坚决行之,则未有不成功者也。”这是伟大的革命家孙中山先生说的。我们两次革命就是继承孙中山先生的工作,完成了资产阶级民主革命,然后完成社会主义革命,这些革命都是为建设扫清道路,都是方法问题,要把生产关系和上层建筑加以改变,把政府、意识形态法律、政治、经济、文化、艺术,这些上层建筑加以改变,但是还没有解决基本问题,目的不在建立一个新的政府,一个新的生产关系,目的在于发展生产,七年来发展了一些,但是很少,吹牛可吹半天,实际上只有四百万吨钢,明年只有五百万吨,再五年只有一千多万吨,在我们国家六亿人口不算事,比日本、法国可以超过,但赶上美国一亿万吨钢要四、五十年才有希望。请大家把目标转向这个方面。陈云同志看到一个瑞典朋友,瑞典只有几百万人口,每人平均有二吨钢,照我们人口算,要几万万吨,美国现只有一万万吨,要达到瑞典的数字要超过美国几倍,所以过去搞民主革命,现在搞社会主义革命。工商联开会,民建开会,大家学习什么呢?都是为达到这个目的,几万万吨钢,还要办学校,收音机都要,全国一切人民,至少要初中毕业,再过多少年,扫马路,厨师所有的人都是大学生,要上知天文,下通地理,一切工作就是为了要达到这个目的。蒋介石该打,就是因为他一件好事也没有干,二十年才搞了五万吨钢。我们八年(算到明年)就有五百万吨钢,他是心不在焉。我希望各位朋友引导这一千多万人向着生产方向,再几十年功夫,才能在文化上翻身。所以我们要团结一切可以团结的人。

今年一月里,在最高国务会议上,我说过社会主义革命大约再过三年左右可以基本完成,这句话引起了很大波动。有人说这样快吗?“大约”、“基本”、“左右”,我用了这么多的形容词。什么是“基本”呢?就是指公私合营还没有取消定息,还有一个尾巴。全部完成就是取消定息。我认为时间定的长一些,早一点完成是好的。全行业合营谁也没料到这样快。现在国有化就不这样快了,快了对国家对民族不利。我们照顾大资本家,对整个民族有利,花的钱不多,讲的话要算数。可以影响外国的资本主义,对改造世界资本家有利。尼赫鲁、苏加诺、吴努以及法国资本家百代公司都在看我们,看荣毅仁……。现在的情况和我春天讲话是一致的,我只是讲“基本”上,那是指公私合营,全部完成还要七年,除了今年,还要六年。国内各方面都要照顾,对资产阶级、无党派民主人士,对工商业者、手工业、机关学校、工作人员讲清楚,不了解是不好的。


