\section[记者对一切事物应保持冷静的头脑(一九五八年十一月二十三日×××传达)]{记者对一切事物应保持冷静的头脑(一九五八年十一月二十三日×××传达)}
\datesubtitle{(一九五八年十一月二十三日)}


作报纸工作的,作记者的,对虚和实的问题,要有正确的看法,正确的态度。

矛盾有正面,有侧面。看问题一定要看到矛盾的各个方面。群众运动有主流,有支流。到下面去看,对运动的成绩和缺点要有辩证的观点,不要把任何一件事情绝对化。好事情也不要全信,坏事情也不要只看到它的消极一面。比方瞒产?我对隐瞒产量是寄予同情的。当然不说实话是不好的。但为什么瞒产?有很多原因,最主要的原因是想多吃一点,好多年吃不饱,不够吃,想多吃一点,值得同情。瞒产除了不老实这一点以外,没有什么不好。隐瞒了产量,粮食依然还在。瞒产的思想要批判,但对发展生产没有大不了的坏处。

虚报不好,此瞒产有危害性。报多了,拿不出来。如果根据多报的数字作生产计划,有危险性,作供应计划,更危险。为什么虚报?干部作风固然有关系,但也因为有人喜欢虚报。如果虚报一万斤,你说这不是事实,他就不报了。因为他说他报的不实,如果你说他报的还不够,他就更会虚报。

紧张好不好?没有紧张就没有跃进,跃进必然紧张,紧张形势有好处。

我写《湖南农民运动考察报告》的时候,当时湖南的形势是:农会的威信很高,农会说一句,地主富农非照办不可。说坏话的人站不住脚。这种空气,说明革命高潮。现在农村有一种形势,人人说干劲,个个说跃进,无人敢说坏话。这是好的,不能说不好。另一方面也应当看到,提意见的人少了,这也不好。有些任务分明完不成,有些作法显然不妥当,也不敢提意见,这就不好了。大家鼓足干劲,苦战几昼夜,要的。但有许多人累得病了,甚至死了,就不好了。所以,对紧张这件事情,要两面看,适当加以调整。

记者到下面去,不能人家说什么,你就反映什么,要有冷静的头脑,要作比较。

强迫命令,不好。为什么要有强迫命令?一定的强迫命令是需要的。如果什么事都命令,就不好了。有些事情也并非强迫命令,例如党委的决议,一定要办。总要有集中。集中的过程要有民主。要提倡民主作风。反对强迫命令。在一定的范围之内“强迫命令”,不反对。要反对命令主义。

记者要善于比较。唐朝有个太守,他问官司先不问原告被告,而先去了解被告周围的人和周围的情况,然后再审原告被告。这叫做勾推法。这就是比较,同周围的环境比较。记者要善于运用这种方法。不要看到好的就认为全好.看到坏的就认为全坏。如果别人说全好,那你就问一问:是不是全好?如果别人说全坏,那你就问一问。一点好处没有吗?

现在全国到处乱哄哄的,大跃进。成绩很大,头脑热了些。

记者的头脑要冷静。要独立思考,不要人云亦云。这种思想方法首先是各新华分社的记者、北京的编辑部要有。不要人家讲什么,就宣传什么,要经过考虑。

记者,特别是记者头子。头脑要清楚。要冷静。


