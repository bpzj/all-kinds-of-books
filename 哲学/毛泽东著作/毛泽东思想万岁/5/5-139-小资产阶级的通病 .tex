\section[小资产阶级的通病 ]{小资产阶级的通病}

一、在日常生活上的表现:

自由散漫,不拘小节,生活上吊儿郎当,毫不紧张,不严肃,不守纪律,不爱护公共财物,不顾团体利益,大家睡觉,他要唱歌,大家起床,他又要睡觉,大家开会,他开小会,上课他要活动,该活动,他要看书,高兴时嘻嘻哈哈,不高兴时死气沉沉,触发自己留恋的心情就悲痛难过,甚至伤感流泪,所谓“见花落泪,望风伤感”,生活中吃不得苦,怕劳动,怕碰钉子,以幻想代替现实。

二、在工作中:

情绪忽高忽低,和兴趣主义投机时,则热情奔放,消极时则垂头丧气,好高骛远,不肯埋头苦干,好作领导工作,否则就认为大才小用,埋没英雄,做一行怨一行,这山望着那山高,大事做不了,小事不肯干,就是干起来也是无计划,事情逼到头上来——粗枝大叶,应付差事,强调工作困难,不去研究克服,强调个人发展,不顾工作需要。

三、在学习上:

对学习不重视,就是学习还是乱抓一把,茫无头绪,虎头蛇尾,学习就是不从实际出发,往往满足于一知半解,空洞教条,缺乏研究精神,学习内容喜好文艺的、不正确的小说,而不学习理论和实际问题,好唱大道理。

四、在写作和谈话上:

脱离实际,总喜欢从主观出发,不看对象,夸夸其谈,籍以骇人听闻,实在言之无物,在写作上要么就不写,要么就连篇累牍,洋洋得意,所谓不鸣则罢,一鸣则惊人,实在不切实际,无病呻吟,写几篇抒情文章,就像有些学校的墙报,什么“秋夜怀念”呀,“可爱的月亮”呀,甚至以自己的感情来代替群众的感情。

五、在待人接物上:

情绪相投时,则无话不谈,所谓“酒逢知己千杯少”性格不合则清高孤独,不理睬,所谓“话不投机半句多”,对别人则多疑要苛刻,对自己则无原则的宽容,平时爱打听别人的秘密,作为知已朋友谈话的材料,爱拉乱谈,说谈家,批评家,当时不说,背后乱说,人家出了乱子则幸灾乐祸,人家有了优点则嫉妒风生。\marginpar{\footnotesize 207}

六、男女关系上:

对男女关系问题,说起来津津有味,不是严肃的研究讨论,而是求得知识上的愉快。要谈恋爱不是政治第一,而是感情第一,甚至抱有自由主义态度。

七、团结观点上:

重视个人利益,固执己见,个人利益高于群众利益,领导能力强就服从,否则就看不起,发牢骚,闹分裂,你有一套,我也有一套,所谓文人相轻,行动自由,不管团结,允许不允许,就开路一马司。

八、在政治斗争上:

夸大个人英雄主义作风,忽视了客观事物的发展规律,斗争性不强,不坚持原则和立场,易犯调和主义,不是过“左”就是过右。

九、群众观念上:

喜欢爬在群众头上发号施令,不深入群众,对阶级没有明确的爱和憎,只是站在当中,对劳动大众可怜,对地主无所谓。

