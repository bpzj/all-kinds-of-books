\section[关于加强相互学习,克服故步自封骄傲自满的指示(一九六三年十二月十三日)]{关于加强相互学习,克服故步自封骄傲自满的指示}
\datesubtitle{(一九六三年十二月十三日)}


现将湖南省委李瑞山、华国锋两同志,一九六三年十一月六日写的一个参观农业生产情况的报告,以及附在上面的湖南省委一九六三年十二月七日写的一个指示,发给你们研究。中央认为,这种虚心学习外省、外市、外区优良经验的态度和办法,是很好的,是发展我国经济、政治、思想文化、军事、党务的重要方法之一。故步自封、骄傲自满,对于自己所管区域的工作,不采取马克思列宁主义的辩证分析方法(一分为二,既有成绩,也还有错误),只研究成绩一方面,不研究缺点错误一方面。只爱听赞扬的话,不爱听批评的话,对于外省、外市、外区,别的单位的工作很少有兴趣组织得力的高级干部去虚心地加以考察,便于和本省、本市、本区、本单位的情况结合起来加以改进,永远陷于本地区、本单位这个狭隘世界,不能打开自己的眼界,不知道还有别的新天地,这叫做夜郎自大。对于外国人、外地人以及中央派下去的人只让看好的,不让看坏的,只向他们谈成绩,不向他们谈缺点及错误,要淡也谈得不深刻,敷衍了事。中央多次对同志们提出这个问题、认为一个共产党员必须具备对于成绩与缺点、真理与错误这两分法的马克思主义的辩证思想。事物(经济、政治、思想、文化、军事、党务等等)总是做为过程而向前发展的,而任何一个过程,都是由矛盾着的两个侧面互相联系而又互相斗争而得到发展的。这应当是马克思列宁主义的常识。但是中央和各地同志中,有许多人却很少认真用这种观点去思索工作。他们的头脑长期存在着形而上学的思想方法而不能解脱。所谓形而上学,就是否认事物的对立统一,对立斗争(两分法),矛盾着的事物在一定条件下互相转化,走向他的反面,这样一个真理。就是人们故步自封、骄傲自满,只见成绩,不见缺点,只愿听好话,不愿听批评话。自己不愿批评(对自己的两分法),更怕别人批评。中央有几十个部,明明有几个工作成绩,工作作风较好的部,例如石油部,别的部却熟视无睹,永远不到那里考察研究请教一番。一个部所管企业、事业,明明有许多厂矿、企业事业、科学研究所及其人员工作得较好,上面都不知道,因而不能提倡人们向那些单位学习。同志们,中央在这里所说的犯有形而上学错误的同志,是一部分同志,但是,应当指出有大量的好同志却被那些高官厚禄、养尊处优、骄傲自满、故步自封、爱好资产阶级形而上学的同志,亦即官僚主义者所不知道,现在必须加以改革。凡不虚心对本地、本单位、本人作分析,对别单位、对别人作分析,拒绝马克思主义辩证分析方法的同志,要进行同志式的劝告和批评,以便把不良情况改变过来。把向别部、别省、别市、别区、别单位的好经验、好作风、好方法学过来,这样一种方法定为制度,这个问题是个大问题。请你们加以讨论,以后还要在中央工作会议及中央全会上加以讨论。湖南省委在过去一个时期内,不做调查研究,主观的下达许多指示,往往灌的东西多,由下面反映上来的真实情况少,因而脱离群众,产生很大困难。从一九六一年起,他们开始改变了,以致情况大好起来。但是他们认为还远不如广东和上海,所以派遣大批省、地、县三级干部,还有省和市的干部,组成了两个考察团分别到上海和广东去学习。这一点请你们注意研究,是否可以这样办。中央认为不但可以而且应当这样办。如有不同意见,请你们提出。

