\section[在天津会议上的讲话(摘录)(一九六○年三月二十八日)]{在天津会议上的讲话(摘录)(一九六○年三月二十八日)}


四化问题。

机械化、半机械化、自动化、半自动化今年要大搞一下。现在全国都在搞,包括城市、农村、工业、农业、商业、服务行业。统通提出来。半年要化,十年以后还要化。

农村人民公社问题。

五个问题(指广东省委《关于当前人民公社工作中几个问题的指示》中的五个问题)。还有其他问题,中央已发指示。要开六级干部会。山东的一个材料,一平二调,不守纪律,根本不向县委、更不向省委报告。这种现象不会很多,也不会很少。要切实整顿一下。要抓落后的,先抓落后的。

这个问题很值得注意。敢说、敢想、敢做是对的,如果什么都敢想、敢说、敢做,那就不对了。有所不为,然后才能有所为。现在把“三敢”变成绝对化,这是没有辩证法。

农业问题。

主要是粮食问题。农业有十二个字:粮、棉、油、麻、丝、茶、糖、菜、烟、果、药、杂。十二个字是个部署问题,要从战略布局出发。省、地、县、社干部都要懂得十二个字,有计划地进行布署。这是农,还有林、牧、副、渔。林,有各种林:用材林、薪炭林、防风林、水源林、经济林等。种什么树?杉树、松树?柏树?都要因地制宜。牧,“马牛羊、鸡犬豕”,此六畜人所食,还有鸭、兔。此外,还有副、渔。要看各种具体情况。大家都去搞粮食,其他没有人搞,这是破坏原有的经济秩序。

工业问题。

主要是煤、铁问题。有煤有铁就有钢。

现在小土群、小洋群只剩下××××个,又有点冷冷清清的样子。凡是有煤有铁的地方还是要搞,出点乱子不要紧。小洋群、小土群搞什么?搞金属、化工、石油、水泥、木材等等。一九六○年抓紧搞,搞三年,分期分批地搞,把小洋群钢铁布点搞起来。凡是有煤有铁的地方都搞一点,有煤无铁的地方可以交流。

支援农业问题。

工交系统、财贸系统、文教系统、大中城市、大中小工矿企业普遍支援农业,全国普遍化,农业有希望了,否则“四、五、八”有危险。上海有十一个县,达到亩产八百斤,上海支援农业的成绩最大。广东的县,百分之四、五十达到亩产八百斤。这首先是个布局问题。

教育问题。

地方要大搞教育,此如业余教育、扫盲教育、农业中学。日本福冈县,一个县有七个大学。我国自秦始皇统一以来,好处是统一,坏处是统死。欧洲小国很多,一个小国等于我国一个省,坏处是不统一,但是经济、文化大大发展了。我们要在统一的原则下,补充我们的不足。各地要办大学,各部门也要办大学。这也是个布局问题。我们总要比秦始皇、唐太宗

进步些,使省、专、县、社发展起来。

除四害问题。

近两年来,比较放松了。现在不打麻雀了,以臭虫代替麻雀,臭虫是代表。

我说过卫生部门从来不讲卫生,不讲爬山、跑步、游泳,也不讲爬山要领。稷山县的那个材料是卫生部的一个好文件。

三反问题。

贪污、浪费、官僚主义,好几年不反了,要大反。山东茌平县八万元的积累,用七万元盖大礼堂。有的贪污救济粮款。各省先摸一两个县,在六级干部会上提出。贪污的钱要退出来,贪污严重的要处理,一定要赔偿。公社的开支,要由党委集体决定,报县批准。我批的山东那个文件(指《中央关于山东六级干部大会情况的批示》),说“一些公社工作人员很狂妄,毫无纪律观点,敢于不得上级批准一平二调。另外,还有三风:贪污、浪费、官僚主义,又大发作,危害人民。”“范围多大?不很大,也不很少。”“对于大贪污犯,一定要法办。”“对于少数县委实在不行的,也要坚决撤掉,换上新人。”错误性质严重,民愤极大的,不是人民内部矛盾,带着外部矛盾。贤者在位,能者在职。

回避问题。

不做本地官。不要全部回避,一律回避破坏一些原则。应该相信多数是好的。大概限制在四分之一才好,主要负责人回避。

先派人去摸几个月,了解情况,然后反客为主,反主为客,请他走开。

有一部分人要坐班房。不要杀人。

外宾参观问题。

一定要使他们看好坏两种,最好看好中坏三种,“强迫”让他们看。如果来不及,看两种也可以,实在不行,也只好不看。

我跟德国人讲,公社有百分之五十是一类社,百分之三十五是二类社,百分之一十五是三类社。四万个公社中,有六千个掌握在坏人手中。他听了之后,感到我讲了公道话。

三类社整好不难。先进与落后,一万年也有。外宾参观,中央发过指示,好坏都看,可以比较。

增产节约与综合利用问题。

煤、电、水、盐、木材、石油、农副产品,主要是煤,木材的增产节约与综合利用。此外,还有拖拉机的方向问题。

反华问题。

读一个文件(指《关于反华问题》及附件),大家斟酌。反华,其实是大拥小反。把我们的事情办好,影响很大。


