\section{文化书社第二期社务报告(摘录)}
\datesubtitle{(一九二一年四月)}



(一)

我们在去年曾出过一本社务报告,是报告“发起及临时营业期内”的情形的,现在这“第二期社务报告”是报告从去年九月本社开幕至今年三月底“第一个半年”里的情形。长沙文化书社是去年九月初开幕的,到今年三月底,有了七个月,与本社组织大纲“六个月一报告’之例,拖一个月,这是算账之故多出来的,以后还要按照组织大纲“六个月一报告”。

(二)

我们出这本报告的意思,头一层:与我们社是有直接关系的社员诸君,我们经理社务的人,在职务上应将如何推销书报种种情形,报告给大家,使社员都知道社里的实在情形,才算完了我们的职务。第二层:中国人营业,总是秘密主义,实在是一种罪过,一个人光明正大做事,为什么不能将底子宣布出来呢?文化书社是一个社会公有机关,并不是私人营利,我们为避免这种罪过,乃反秘密而公开,将社里一切情形,彻底宣布于社员以外。第三层:我们这个社的任务,有我们的“缘起”里曾经说道:“……愿用最迅速、最简单的方法介绍中外各种新书报杂志,以充青年及全体训南人新研究的材料’。在组织大纲里也说:“……庶使各种有价值之新出版物广布全省,人人有阅读的机会”,象这样极大的任务,是不能由我们几十个社员一肩担负的。我们可达到“人人有阅读机会”的目的,最注重的是各县开设分社,以七十五县每设一分社,每一个分社有十个社员计,就要有七百五十个社员。所以我们惟有将社务公开,使远近同志的人,明白这个社的益处,分向各地开设分社,并口头或实力帮助我们的传播事业,“广布全省”的目的,庶几以达到了。第四层:我们社里有的是书报,少的是本钱,为什么呢:因为书报是从外边来的,要几多也有,本钱是社员出的,社员皆不是富翁;社员所作的事又不单只书报这件,那里有许多钱来作本?并且社的本是“公财”性质,没有退也没有息,我们这些穷社员,从那里出得来许多呢!然而文化书社这件事,的确是一件极可注意的事,试举其益,便有两宗:

(一)经营简便。

(二)不消耗本钱。

大家晓得现时的任务,莫要于传播文化,而传播文化有效则莫要于办“文化书社式”的书社。如果经营得法,一个书社的效何止抵几个学校的效。因此我们为扩张社务,并推广各县分社起见,希望有力的同志,助我们一笔大一点的款子(我们计划于二年内替书社筹足书业资本三千元),我们不知谁是愿意帮助我们的,自然不好到处去问,惟有将来社务公开起来,庶几同情于我们的人,好自动的予以帮助。第五层:我们以前全是贩书,自己没有出版物,现正计划组织一个“编译社”和一个“印刷局”,以与书社的“发行”连为一贯,然后本社乃有社立出版物,这样社务便更加扩张,非多邀同志和筹足资本(编译社定三千元,印刷局定五千元)不行,因此也得把社务公开起来,才可邀大家的同情和注意。最后还有一层是属于我们社内职员关于营业上的利益的。一一第六层;就是要社务发达,务必要账目清楚,我们社内的账目,有“日算”、“月算”、“半年算”三种,“日算”是每日晚上将营业的账算出结果,“月算”是每月一号将上月全月内的账算出结果,“半年算”是将半年的账算出结果。这本社务报告里面所列的“营业情形’就是“第一个半年算”的结果,我们有了这一算,手续既到,观念乃明,改正旧的失误,定出新的方案,便容易。

(三)

在第一期社务报告里报了的,此处就不报告了。文化书社自去年八月一日假楚怡小学开会发起,八月二十日订约承租潮宗街湘雅医学校房屋之一部为社址,九月九日营业开幕,并十月二十二日假长沙县署开第一次议事会以后,几个月间社中可纪的事,略为下述。

1、议事会临时会。十二月二十九日假长沙县署开议事会临时会,到会者:姜永洪、贺寿干、王季范、周惊元、郭涛僧、彭阴柏、熊瑾珂、易礼容、赵远文、刘驭皆、毛润之诸君,姜君泳洪主席,讨论“另觅社址”及“添筹设本”两个问题。首先讨论社址问题,因现社址僻处草湖门一隅,房屋亦觉不够,欲图社务发达,宜于转为适中的地点,觅一较为宽大的房屋。决议社址自要迁移,惟须慎重,务期一迁,便可安图。在未觅到相当房屋并布置清楚以前,自在湘雅原处。因开会之前日船山学社总理仇亦山君允将学社房屋一部分借用于书社,名以此本甚善,惟须明定契约,不背“安固进行”之目的才好。结果决定由本社与船山再切实交涉。次讨论经费问题,从第一次议事会议决尽本年筹足千元,现尚少五百三十元,大家的零募不是办法,姜君泳洪乃愿独力筹措此五百三十元,遂散会。

2、与商务等书局订约。此为本年一月以后的事。本年一月以前,外埠书局如“中华”、“展束”、“泰乐”、“新青年”、“北大出版部”、“学术讲演会”等均订约销书,至本年一月,如由杨君×介绍上海商务印书馆订约分销,分销的折扣,七、八折不等。二月,与上海伊文思图书公司交涉销书亦成,惟伊文思是一个西书公司,中文书至少,湘省在现在于西书不甚需要,故不能多销,三月,北京成立了一个新知书社,是和文化书社宗旨等大致相同的一个新起的书社,刊印一部“罗索五大讲演”与本社交涉代销预约,本社已答应代销;有一个在四川的“华阳书报交通处”,将他在四川未销完的书托本社代销;又有一个在北京的“亚洲文明协会”,托本社代售其新改组出版的“时事月刊”,本社均经答应;此外新起的杂志社与本社交涉代销者,在这个期内尚有多家。

3、书报畅销。本社所销的书百六十余种,杂志四十余种,报三种。在去年开幕至本年三月底,除开寒假时一个月外,月余均很畅销,社里总是供不应求。一面固因为本社资本太少不能向外埠大批买书,小批则随到随尽的原故;一面也是社会对于新出版物的需要骤然迫切起来,受了新思潮的正面激剌和旧思想的反面激刺,一时感发兴起,尽力购读,实在是一种可喜的现象。买书的人,自然以学界为多,但是如《劳动界》等小册子,销于劳动者间的也不少。以年龄论,买书的人自然以青年的多,壮年以上的人次之。

4、成立分社。本社既欲各县均有购买新书的机会,就非在七十五县都设立分社不可。本社对于设立分社,并不是本社自己去设,只是帮助各县的同志在各县去设立起来。不知本社内容的人或疑分社设立甚难,其实此事至易,资本只要第一批书第二批书的钱,少则五十元,多则百元,第三批以后即将头二批的书价收回能购便得,是一。各县生意多,因无问题,生意少,也不要紧,分社横直是附着公共地方开设,不要独立门面,开销便不要多,僻县陋镇,或一年只能销几十元乃至几元的书都不要紧,-一是一。本社对于分社,照本让与,不赚分文,所有优价折扣都归分社,因此分社即因生意少不能得赢利,也断不至于耗本钱,一一是三。附带经营,不须多的人员,一一是四。销不完的,可以退还本社,不至因为滞销至于亏耗,一一是五。有此五层,所以说开设分社是“至易”的。本社截至今年三月底,已经成立了的分社凡七,即“平八”,“浏西”、“武冈”、“宝庆”、“衡阳”,“宁乡”、“溆浦”。七处分社外尚有“贩卖部”七处。“分社”何以别于“贩卖部呢”?1、分社可以得到全部的优价,就是折扣照原书店全部让与;“贩卖部”要少一点,只能有百分之五的报酬。2,分社在外县,贩卖部在本城的各校或个人(个人未贩卖小册子于平民及劳动界的,仍社与优价)。

(四)

上述之七分社及七贩卖部,分依成立的次序,将其“成立的月日”,“设立的地点”,“创办人”,“销书额”四项略列于左,至其详细情况,本社已向各分社征集,容后再报。凡下所列,截于三月底止,四月一日以后成立的分社末载。

(缺页)

易礼容(经理)

毛泽东(特别交涉员)

李庠(营业员一一管书)

唐自光(营业员一一管报兼管书)

王仙枚(营业员一一管报)

黄清安(烧饭兼走杂)

以上是现在的职员,在这个时期内尚有陈君子博、任君培道枉社各担任营业约二个月。社设在长沙潮崇街五十六号。

\begin{flushright}(《湖南革命史料几条》第一册)\end{flushright}

