\section[在纪念孙中山逝世十三周年及追悼抗敌阵亡将士大会上的演说词(三月十二日)]{在纪念孙中山逝世十三周年及追悼抗敌阵亡将士大会上的演说词(三月十二日)}


今天是孙中山先生逝世十三周年纪念日子,我们开了一个庄严的纪念大会,同时,抗日战争已经打了八个月,许多英勇将士牺牲了,我们开了这样一个沉痛的追悼大会。这些都不是随便的与偶然的,有我们民族历史发展的必然的理由。

孙中山的伟大在什么地方呢?在于他的三民主义的纲领,统一战线的政策,艰苦奋斗的精神。当我在广东会见孙先生的时候,正开国民党第一次全国代表大会,孙中山亲手订的三民主义新纲领,它被通过了大会。即就是有名的“中国国民党第一次全国代表大会宣言”。这时还开始实行了以国共合作做基础的统一战线政策,这个统一战线,包括对内联合共产党与工人农民,对外联合以平等待我之民族,共同奋斗,有名的“三大政策”,即建立于此时。孙中山的三民主义纲领与统一战线政策,实为处在半殖民地地位的大革命家对于中华民族最伟大的贡献。不但如此,孙中山的伟大,还在他的艰苦奋斗,不屈不挠,再接再励的革命毅力与革命精神,没有这种毅力,没有这种精神,他的主义与政策是不能实现的。如象刚才读过的总理遗嘱劈头一句所说:“余致力于国民革命凡四十年”,在这四十年中间,经过了多少艰苦曲折,然而孙先生总是越挫越奋,不屈不挠,再接再励。当着多少追随者在困难与诱惑面前表理了,灰心丧志乃至投降变节的时候,孙先生总是坚定的。孙先生是坚持共产主义的,在他一生,他的三民主义只有发展而无弃置,孙先生从没有弃置共主义于不愿的时候,他始终坚持了三民主义,并且发展了三民主义。第一次代表大会宣言,就表现了三民主义发展。对统一战线也是一样的,孙先生不但坚持了而且发展了统一战线,从为着推翻满清而联合各国派别与会党,发展到为着推翻帝国主义,封建势力而釆取联合苏俄,联合共产党与联合工农的政策。所有这些,同他的不怕艰难挫折,不屈不挠,再接再励的革命毅力或革命实践精神相结合,就表现了孙先生的伟大革命家模范。今天我们又是一个统一战线,这个统一战线较之过去是更加广大的,这个统一战线所要对付的敌人也是较之过去更加严重的,这个统一战线所应执行的纲领,在目前基本上仍然是那个第一次代表大会宣言所说的东西,但形式与内容有了某些发展,在将来,一定还会有进一步的发展的,为了实行三民主义,扩大统一战线,战胜我们的敌人日本帝国主义,还一定从革命实践中发扬艰苦奋斗、不动摇、不妥协的革命精神,才能达到。所以我们纪念孙先生,如果不是奉行故事的话,就一定要注意这样三项:第一,为三民主义的彻底实现而奋斗;第二,为抗日民族统一战线的巩固与扩大而奋斗;第三,发扬艰苦奋斗,不屈不挠,再接再励的革命精神。我以为这三项是孙先生留给我们的最中心、最本质、最伟大的遗产,一切国民党员,一切共产党员,一切爱国同胞都应接受这个遗产而发扬光大之。判断一个人究竟是不是孙先生的忠实信徒,就看他对这三项宝贵遗产的态度而定。

现在说到追悼抗敌阵亡将士的意义。从芦沟桥事变以来,东方历史上未曾有过的大战已经打了八个月,敌人是倾全国的力量来打;目标是灭亡中国,战略是速战速决;我们呢?也是倾全国的力量来抵抗,目标是保卫祖国,战略是持久奋斗。八个月中,陆空两面都作了英勇的奋战;全国实现了伟大的团结;几百万军队与无数的人民都加入了火线,其中几十万人就在执行他们的神圣任务当中光荣地、壮烈地牺牲了。这些人中间,许多是国民党人,许多是共产党人。许多是其它党派无党派的人。我们真诚的追悼这些死者,表示永远纪念他们。从郝梦龄、佟麟阁、赵登禹、饶国华、刘家祺、姜玉真、陈锦秀、李桂丹、黄梅兴、姚子番、潘占魁诸将领到每一个战士,无不给全中国人民以崇高伟大的模范。中华民族决不是一群绵羊,而是富于民族自尊心与人类正义心的伟大民族,为了民族自尊心与人类正义,为了中国人一定要生活在自己的土地上,决不让日本法西斯不付重大代价达到其无法无天的目的,我们的方法就是战争与牺牲,拿战争对抗战争,拿革命的正义战争对抗野蛮的侵略战;这种精神,我们民族的数千年历史已经证明,现在再来一次伟大的证明,郝梦龄等数十万人就为着这个而牺牲了。判断日本法西斯是还要前进的,他还要进攻我们的西安、郑州、武汉、南昌、福州、长沙与广州,他想吞灭全中国。但是我要告诉那些发疯的敌人,你们的目的一定达不到。不要以为占领了我们的地方,就算达到了你们的目的,没有达到,也不会达到,你们日本法西斯的胜利,历史判定只会是暂时的,不会是永久的,有充足的理由证明最后胜利只会属于我们一方面。而且战争打到结局,你们也一定只能占领我们一部分地方;要占领全国是不可能的。即使你们得到了一个城市的速决战,同时也就要你们得到一个乡村的持久战,例如你们把山西的几条大路与若干城市占领了,但数终于你们占领地的乡村将始终是中华民国。我们要把这个理由告诉全国的同胞,日本差不多在任何一省都只能作部份的占领,日本的兵力不够分配;他的野蛮政策又激起了每一个中国人,中国有广大的军队与人民,中国又实行着统一战线的良好政策,就此决定了持久战以及最大胜利之属于那一方,将来的形势,双方血战结果,即使日本占领了中国的大半,中国只剩下完整的小半作为继续抗战争取最后胜利的中枢根据地,但在那大半地方,实际上日本只能占领其中的大城市,大道与某些平地,只要我们在每个省中组织大多数农村中的一致武装起来打日本,建立许多抗日根据地,如象现在已经建立起来了的五台山根据地一样,我们就包围了日本军,我们的这个外线的战争,配合着内线的战争,又从奋力努力把我们全国范围内的党、政、军、民各项紧要工作办得大大进步起来,有朝一日,就可互相配合,内外夹击,打大反攻,那时还一定配合着世界革命的援助同日本国内人民革命的援助,最后胜利谁能说不是中国的?郝梦龄将军等的热血谁能说是白流的呢?日本强盗之被赶出中国能说不是必然的?孙中山先生的民族解放,民权自由民主幸福的三大理想,谁能说不会实现于中国的?我们要使全国人民都有这种明确的认识与坚固的信念,都懂得最好的持久战方针,在这时大战中,齐心一致,一定要把亡国奴或亡国奴威胁的锁链脱掉。

在这次大会上,我们要向一切在前线奋战的将士们致敬礼,因为他们都在为着最后胜利作英勇的斗争,我们要向一切抗日军人的家属尤其是死难烈士的家属致敬礼,因为他们家中出了这样为国奋斗、不怕牺牲的抗日军人;我们也要向在后方各界幸苦焦芳、克己奉公以从事于抗战工作的各级工作人员与各级领袖们致敬礼,因为他们的工作都直接指导了或帮助了抗战。

大会的同胞们,全国的同胞们,让我们永远的团结起来啊!打倒日本帝国主义,中华民族解放万岁!

<p align="right">《解放》杂志第三十三期民国二十七年四月一日</p>

