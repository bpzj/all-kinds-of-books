\section[关于团结方法的讲话(一九六四年八月)]{关于团结方法的讲话}
\datesubtitle{(一九六四年八月)}


在团结问题上,我讲一点方法问题,我说对同志不管是什么人,只要不是敌对分子,破坏分子,那就要采取团结的态度,对他们要采取辩证的方法,而不能采取形而上学的方法。什么叫辩证的方法?就是一切加以分析,承认人总是要犯错误的,不因为一个人犯了错误就否定他一切。列宁曾讲过,不犯错误的人一个也没有。我就是犯过错误的。这些错误对我很有益处,这些错误教育了我。任何一个人都要有人支持。一个好汉要有三个帮,一个篱笆要有三个桩,这是中国的成语,中国还有一句成语,荷花虽好,要有绿叶扶持。你××这朵荷花虽好,也要绿叶扶持,我这朵荷花不好,更要绿叶扶持。我们中国还有一句成语,三个臭皮匠,合成一个诸葛亮。这合乎我们××同志的口号——集体领导,单独一个诸葛亮总是不完全的,总是有缺陷的。你看我们××这个宣言,第一,第二,第三,第四次草稿,现在文字上的修正还没有完结。我看要自称全智全能,像上帝一样,那种思想是不妥当的。因此,对犯错误的同志,应采取什么态度呢?应该有分析,采取辩证的方法,而不采取形而上学的方法,我们党曾经陷入形而上学——教条主义,逐步的学了点辩证法。辩证法的基本观点就是对立面的统一。承认这个观点,对犯错误的同志怎么办呢?对犯错误的同志,第一是要斗争,要把错误思想彻底肃清,第二就是要帮助他,从善意出发帮助他改正错误,使他有一条出路。\marginpar{\footnotesize 174}

对待另一种人就不同,像托洛茨基那种人,像中国的陈独秀、张国焘、高岗那种人,对他们无法采取帮助的态度,因为他们不可救药。正像希特勒,沙皇,蒋介石也都是不可救药,只能打倒,因为他们对我们来说,是绝对的相互排斥的,在这个意义上说来,他们没有两重性,只有一重性。对于帝国主义制度,资本主义制度,在最后说来也是如此,他们最后必然要为社会主义制度所代替。意识形态也是如此。要用辩证唯物论代替唯心论,用无神论代替有神论,这是在战略目的上来说的。在策略阶段上来说就不同了,就有妥协了。在朝鲜三八线上,我们不是同美国人妥协了吗?在越南不是同法国妥协了吗?

各个策略阶段上,要善于斗争,又善于妥协。现在回到同志关系,我建议同志之间有隔阂要开谈判。有些人似乎认为,一进入共产党,都是圣人没有分歧,没有误会,不能分析,就是说铁板一块,整齐划一,就不需要讲谈判了。好像一进入共产党,就要百分之百的马克思主义才行,其实有各式各样的马克思主义者,有百分之百的马克思主义者,有百分之九十的马克思主义者,有百分之八十的马克思主义者,有百分之七十的马克思主义者,有百分之六十的马克思主义者,有百分之五十的马克思主义者,有的人只有百分之十或二十的马克思主义,我们可不可以在房间里头与两个人或者几个人谈判呢?可不可以从团结愿望出发,用帮助的精神开谈判呢?这当然不是跟帝国主义谈判(对帝国主义也是要同他们谈判的)这是共产主义内部的谈判。再举一个例子,我们这四十二国是不是开谈判?六十几个党是不是开谈判?实际上是开谈判,也就是说,在不损伤马克思列宁主义原则下接受大家一致可以接受的意见,放弃一些自己可以放弃的意见。这样,我们就有两只手,对犯错误的同志,一只手跟他们斗争,一只手跟他们团结。斗争的目的是坚持马克思主义原则,这叫原则性,这是一只手。另一只手讲团结,团结的目的是给人一条出路,跟他们讲妥协,这叫灵活性。原则性和灵活性的同一是马克思列宁主义原则,这是一种对立面的统一。

无论什么世界,当然特别是阶级社会,都是充满着矛盾。社会主义可以找到矛盾,我看这个提法不对,不是什么找到或找不到矛盾,而是充满着矛盾,没有一处不存在矛盾,没有一个人不可以分析的,如果承认一个人不可以加以分析时,就是形而上学,你看在原子里头就充满着矛盾的统一,有原子核和电子二个对立的统一,中子里头又有中子、反中子。总之,对立面的统一是无处不在的。关于对立面统一的概念,关于辩证法,需要做广泛的宣传。我说辩证法应该从哲学家的圈子里走到广大人民群众中间去。我建议在全国党的政治局会议和中央会议上谈这个问题,需要在党的各级地方委员会上谈这个问题。其实我们党的支部书记是懂得辩证法的,当他准备在支部大会作报告的时候,往往在小本子上写上两点,第一是优点,第二是缺点,一分为二,这是普通的现象,这就是辩证法。

