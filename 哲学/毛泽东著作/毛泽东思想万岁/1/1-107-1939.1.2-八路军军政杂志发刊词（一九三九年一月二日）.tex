\section[八路军军政杂志发刊词(一九三九年一月二日)]{八路军军政杂志发刊词}
\datesubtitle{(一九三九年一月二日)}


当抗日战争向着新阶段发展的时候,八路军同志出版这个《军政杂志》,其意义是明显的:为了提高八路军的抗战力量,同时也为了供给抗战友军,与抗战人民,关于八路军抗战经验的参考材料。

八路军在抗战一年半中,在蒋委员长与战区司令长官的领导之下,在朱彭总副司令及各部级长官与共产党员的领导之下,协同各部友军,进行了英勇的抗战,执行了“基本的游击战,但不放松有利条件下的运动战”的正确的战略方针,坚持了与发展了华北的游击战争,创立了许多在敌人后方的根据地,缩小了敌人的占领地,箝制了大量的敌军,配合了正面主力军的抗战,延缓了敌人进攻西北的行动,兴奋了全国的人心,打破了认为“在敌后坚持抗战不可能”的那些民族失败主义者,与悲观主义者的错误观点,揭穿了中国托洛斯基反动派,汪精卫亲日派与国内某些守旧顽固分子的无耻造谣。

此外,八路军的一部一一后方留守部队,亦保卫了河防,现在准备配合西北友军,为反对敌人进攻西北而战,八路军的这些成绩,是有目共睹的,除了托洛斯基反动派,汪精卫亲日派与某些守旧顽固分子之外,是一致承认的。这在敌人方面,不但不敢轻视八路军,而且日益增长其畏惧八路军的心理,事实表现上,也得到充分的反证。八路军为保卫祖国而牺牲奋斗的忠诚与不可战胜的事实,是明显的摆在全国全世界的面前,除了反动派、亲日派与某些顽固分子之外,是无法否认的。中外新闻记者、观察家、旅行家的详尽的或粗略的记载早已连篇累牍;一切无成见的人,都愿意研究八路军的经验,当然不是偶然的。以共产党员为骨干的八路军之存在及其发展,对于中华民族是有益的,还是无益的?如果有人提出这类问题的话,那我们只有一句话答复:认为“无益”者,必是事实上不愿意抗战胜利者,只是直接帮助敌人的胡说。

八路军的这些成绩从何而来?由于上级领导的正确,由于指战员的英勇,由于人民的拥护,由于友军的帮助,这四者是八路军所以获得成绩的原因。其中友军的协助是明显的,没有正面主力军的英勇抗战,便无从顺利的开展敌人后方的游击战争,没有同处于敌后的友军之配合,也不能得这样大的成绩。八路军的将士应该感谢直接间接配合作战的友军,尤其应感谢给予自己各种善意帮助与忠忱鼓励的友军将士。中国军队在民族公敌面前,互相忘记了旧怨,而变为互相帮助的亲密朋友,这是中国决不会亡的基础。从前人说;读诸葛出师表而不流泪者,其人必不忠;读李密陈情表而不流泪者,其人必不孝。但今天我们应该说:凡看见或听见中国军队不记旧怨,而互相帮助,亲密团结而不感动者,其人必不爱国。在这里那些“发国难财,吃磨擦饭”的人物,应该引起一点反省罢!

八路军有无缺点呢?不但有,而且多。首先是技术装备不如敌人,也不如某些友军,这是八路军的基本缺点,也是中国军队一般缺点。因比如何加强技术装备以便战胜敌人,成为八路军在抗战新阶级中的严重任务。第二,八路军以善于游击战与运动战出名,但一部分干部对于抗日的战略战术之了解与应用尚感不足,一般干部尤其是新提拔的干部,对于现代新式军队的管理与指挥,至今还缺少初步的研究。若干工农出身的干部,还没有解决提高文化水准和程度的问题。解决这些问题,成为八路军当前的第一个任务。第三,巩固与扩大民族统一战线,是达到抗战建国胜利的总方针,八路军干部在这方面有了很大的成绩,但若干干部尤其是新干部,对于统一战线的了解尚感不足,协同友党友军一道工作与调解社会各阶层的关系,使之利于抗战,在某些地方还做得差。因此加强统一战线教育成为重要的任务。第四,争取敌伪军的工作,久而成为八路军政治工作三个主要方向之一,也得到了许多成绩,但对于战士与干部普遍施以日文日语的教授,并研究各种方法,使之善于向敌军士兵与下级官长进行反侵略统一战线的宣传,还非常不足,争取蒙伪军的成绩较大。但还须更进一步,在这里搜集与研究敌伪军的全部情况,是十分重要的,然而在这方面的成绩,还没有做到需要的程度。第五,长期抗战中最困难问题之一,将是财政经济问题,这是全国抗战的困难问题,也是八路军的困难,应该提到认识的高度。这个问题已经引起八路军某些部分的注意,但还没有引起普遍的注意。如何在各个抗日根据地上,不但注意执行正确的地方财政经济政策,如象过去已经实行了的,而且提出与实行在不妨碍作战条件下,由军队本身参加生产的问题。在比较巩固的根据地上,战斗部队担任作战,后方机关人员担任生产。在战斗许可的情况下,战斗部队亦可利用时机,进行发动士兵群众做衣服,做鞋袜,打手套等等工作,在巩固的根据地上,种菜、喂猪、打柴,都可以发动非战斗部队做的,开办合作社更应该做。这样做去,一方面改善了军队的生活,补助了给养的不足;又一方面必然能够更加振奋军队的精神,增强军队的战斗力。

以上增加技术装备,深研战略技术,正确的运用统一战线政策,广泛的进行争取敌伪军工作,由军队自身参加生产运动,这是八路军在新阶段中应该加重注意的重要问题,其它工作中存在着的缺点,将从这些重要问题上的进步而克服之。

发扬成绩,纠正缺点,是八路军全体将士的任务,也是军政杂志的任务。抗战是长期的与残酷的,发扬八路军的成绩,纠正八路军的缺点,首先对于提高八路军的抗战力量是迫切需要的,同时对于以八路军经验贡献抗战人民与抗战友军,也属需要。“八路军军政杂志”应该为此目的而努力。

