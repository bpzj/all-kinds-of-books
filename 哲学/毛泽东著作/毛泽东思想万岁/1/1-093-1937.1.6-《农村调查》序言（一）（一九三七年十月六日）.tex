\section[《农村调查》序言(一)(一九三七年十月六日)]{《农村调查》序言(一)}
\datesubtitle{(一九三七年十月六日)}


从一九二七年北伐战争起,到一九三四年离开中央苏区为止,我亲手从农村中收集的材料,现在仅存下各部分。一、寻邬调查,二、兴国调查。三、东坡等处调查,四、水口村调查,五、赣西南土地分配情形,六、分清出租问题,七、江西土地斗争中的错误,八、永新分田后的富农问题,九、两个初期的土地法,十、长岗乡调查,十一、才溪乡调查最后两部分,曾在中央苏区的《斗争报》发表过,其余保存原稿,经过长征,尚未损失,此外的东西,就都损失了。其中最可惜的是一九二七年春天在湖南长沙。湘潭、湘乡、衡山、醴陵五县调查,因许克祥叛变而损失。这里存下来的,都是中央苏区的材料,前九部分是初期的土地革命,后两部分是深入了的土地革命,虽不完全,亦可见其一处。为免两损失,印出若干分,并供同志们参考,这是历史材料,其中有些观点是当时的意见,后来已经改变了。

