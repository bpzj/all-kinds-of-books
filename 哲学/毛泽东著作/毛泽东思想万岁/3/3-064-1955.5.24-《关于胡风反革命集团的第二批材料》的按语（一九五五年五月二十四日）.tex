\section[《关于胡风反革命集团的第二批材料》的按语(一九五五年五月二十四日)]{《关于胡风反革命集团的第二批材料》的按语}
\datesubtitle{(一九五五年五月二十四日)}


编者按:自从舒芜的“关于胡风反革命集团的一些材料”发表以后,人们被胡风反共反人民反革命的罪恶所激怒了。本报已经收到全国各地各界读者写的大批文章和信件,他们一致要求彻底揭露胡风及其集团的罪恶,这种愤怒的情绪是不可抑止的。但是,有些同情胡风、或口头上反对胡风但内心是同情胡风的人们在说,那些材料大都是解放以前的,不能据此定罪。那末,好罢,现在请看第二批材料。

现在发表的材料,是从胡风写给他的反动集团的人们的六十八封密信中摘录下来的。这些密信都是胡风在全国解放以后写的。在这些信里,胡风恶毒地污蔑中国共产党、污蔑党的文艺方针、污蔑党的负责同志、咒骂文艺界的党员作家和党外作家;在这些信里,胡风指挥他的反动团体的人们进行反共、反人民的罪恶活动,秘密地有计划地组织他们向着中国共产党和党所领导的文艺战线猖狂进攻;在这些信里,胡风唆使他的党羽们打进共产党内,打进革命团体内建立据点,扩充“实力”,探听情况和盗窃党内文件。在这些信里,人们可以清楚地看出,在解放以后,胡风更加施展了他的两面派手法:公开的是“不要去碰”,“可能的地方还要顺着”党和人民;而暗中却更加紧地“磨我的剑,窥测方向”“用孙行者钻进肚皮去的战术”,来进行反革命的活动。当他向党举行猖狂的进攻失败以后,他就赶紧指挥他的党羽布置退却,“在忍受中求得重生”,准备好每人一套假检讨,以便潜伏下来,伺机再起。这就证明了胡风及其集团的反革命阴谋的极端严重性。我们必须加倍提高警惕,决不可中了他们假投降的诡计。

胡风和胡风集团分子的通信,大部分采取了鬼鬼祟祟的、隐蔽的方法。胡风和他们相约,在信中使用了各种代号和隐语,信上提到中国共产党的负责同志、文艺界负责同志和党员作家,都用了代号。收信人的名字和胡风自己的署名也很不一致,信封上的名字往往是收信人的妻子或其他人;信末的署名也经常变化或不署名。胡风的许多信用的是《人民日报》《解放日报》的信封信纸,许多信封上写的是“上海新文艺出版社罗寄”“上海青年报罗寄”“北京中央戏剧学院张寄”或其他机关名称。

下面就是从这些密信中摘录出来的材料,按内容分为三类,每类大致按写信的时间先后排列,并加必要的注释。信中旁点都是原有的。

按:胡风所谓“舆论一律”,是指不许反革命分子发表反革命意见。这是确实的,我们的制度就是不许一切反革命分子有言论自由,而只许人民内部有这种自由。我们在人民内部,是允许舆论不一律的,这就是批评的自由,发表各种不同意见的自由,宣传有神论和宣传无神论(即唯物论)的自由。一个社会,无论何时,总有先进和落后两种人们、两种意见矛盾地存在着和斗争着,总是先进的意见克服落后的意见,要想使“舆论一律”是不可能的,也是不应该的。只有充分地发扬先进的东西去克服落后的东西,才能使社会前进。但是在国际国内尚有阶级和阶级斗争存在的时代,夺取了国家权力的工人阶级和人民大众,必须镇压一切反革命阶级、集团和个人对于革命的反抗,制止他们的复辟活动,禁止一切反革命分子利用言论自由去达到他们的反革命目的。这就使胡风等类反革命分子感到“舆论一律”对于他们的不方便。他们感到不方便,正是我们的目的,正是我们的方便。我们的舆论,是一律,又是不一律。在人民内部,允许先进的人们和落后的人们自由利用我们的报纸、刊物、讲坛等等去竞赛,以期由先进的人们以民主和说服的方法去教育落后的人们,克服落后的思想和制度。一种矛盾克服了,又会产生新矛盾,又是这样去竞赛。这样,社会就会不断地前进。有矛盾存在就是不一律。克服了矛盾,暂时归于一律了;但不久又会产生新矛盾,又不一律,又须要克服。在人民与反革命之间的矛盾,则是人民在工人阶级和共产党领导之下对于反革命的专政。在这里,不是用的民主的方法,而是用的专政即独裁的方法,即只许他们规规矩矩,不许他们乱说乱动。这里不但舆论一律,而且法律也一律。在这个问题上,胡风等类的反革命分子好像振振有词;有些糊涂的人们在听了这些反革命论调之后,也好像觉得自己有些理亏了。你看,“舆论一律”,或者说,“没有舆论”,或者说,“压制自由”,岂不是很难听的么?他们分不清人民的内部和外部两个不同的范畴。在内部,压制自由,压制人民对党和政府的错误缺点的批评,压制学术界的自由讨论,是犯罪的行为。这是我们的制度。而这些,在资本主义国家里,则是合法的行为。在外部,放纵反革命乱说乱动是犯罪的行为,而专政是合法的行为。这是我们的制度。资本主义国家正相反,那里是资产阶级专政,不许革命人民乱说乱动,只叫他们规规矩矩。剥削者和反革命者无论何时何地总是少数,被剥削者和革命者总是多数,因此,后者的专政就有充分的道理,而前者总是理亏的。胡风又说:“绝大多数读者都在某种组织生活中,那里空气是强迫人的。”我们在人民内部,反对强迫命令方法,坚持民主说服方法,那里的空气应当是自由的,“强迫人”是错误的。“绝大多数读者就在某种组织生活中”,这是极大的好事。这种好事,几千年没有过,仅在共产党领导人民作了长期的艰苦的斗争之后,人民方才取得了将自己由利于反动剥削压迫的散沙状态改变为团结状态的这种可能性,并且于革命胜利后几年之内实现了这种人民的大团结,胡风所说的“强迫人”,是指强迫反革命方面的人。他们确是胆战心惊,感到“小媳妇一样,经常的怕挨打”,“咳一声都有人录音”。我们认为这也是极大的好事。这种好事,也是几千年没有过,仅在共产党领导人民作了长期艰苦斗争之后,才使得这些坏蛋感觉这么难受。一句话,人民大众开心之日,就是反革命分子难受之时。我们每年的国庆节,首先就是庆祝这件事。胡风又说:“文艺问题也实在以机械论最省力。”这里的“机械论”是辩证唯物论的反话,“最省力”是他的瞎说。世界上只有唯心论和形而上学最省力,因为它可以由人们瞎说一气,不要根据客观实际,也不受客观实际检查的。唯物论和辩证法则要用气力,它要根据客观实际,并受客观实际检查,不用气力就会滑到唯心论和形而上学方面去。胡风在这封信里提出了三个原则性的问题,我们认为有加以详细驳斥的必要。胡风在这封信里还说到:“目前到处有反抗的情绪,到处有进一步的要求”,他是在一九五零年说的。那时,在大陆上刚刚消灭了蒋介石的主要军事力量,还有许多化为土匪的反革命武装正待肃清,大规模的土地改革和镇压反革命的运动还没有开始,文化教育界也还没有进行整顿工作,胡风的话确实反映了那时的情况,不过他没有说完全。说完全应当是这样:目前到处有反革命反抗革命的情绪,到处有反革命对于革命的各种捣乱性的进一步的要求。

按:宗派,我们的祖宗叫作“朋党,”现在的人也叫“圈子”,又叫“摊子”,我们听得很熟的。干这种事情的人们,为了达到他们的政治目的,往往说别人有宗派,有宗派的人是不正派的,而自己则是正派的,正派的人是没有宗派的。胡风所领导的一批人,据说都是“青年作家”和“革命作家”,被一个具有“资产阶级理论”“造成独立王国”的共产党宗派所“仇视”和“迫害”,因此他们要报仇。《文艺报》问题,“不过是抓到的一个缺口”,这个“问题不是孤立的”,很希望由此“拖到全面”,“透出这是一个宗派主义统治的问题”,而且是“宗派和军阀统治”。问题这样严重,为了扫荡起见,他们就“抛出”了不少的东西。这样一来,胡风这批人就引人注意了。许多人认真一查,查出了他们是一个不大不小的集团。过去说是“小集团”,不对了,他们的人很不少。过去说是一批单纯的文化人,不对了,他们的人钻进了政治、军事、经济、文化、教育各个部门里。过去说他们好像是一批明火执仗的革命党,不对了,他们的人大都是有严重问题的。他们的基本队伍,或是帝国主义国民党的特务,或是托洛茨基分子,或是反动军官,或是共产党的叛徒,由这些人做骨干组成了一个暗藏在革命阵营的反革命派别,一个地下的独立王国。这个反革命派别和地下王国,是以推翻中华人民共和国和恢复帝国主义国民党的统治为任务的。他们随时随地寻找我们的缺点,作为他们进行破坏活动的借口。那个地方有他们的人,那个地方就会生出一些古怪问题来。这个反革命集团,在解放以后是发展了,如果不加制止,还会发展下去。现在查出了胡风们的底子,许多现象就得到了合理的解释,他们的活动就可以制止了。

从以上的材料,我们可以看出:(一)解放以来,胡风集团的反共反人民的阴谋活动更加有组织、更加扩大了,他们对于中国共产党和党所领导的文艺战线的进攻更加猖狂了;(二)如一切反革命集团一样,他们的破坏活动总是采取隐蔽的或者两面派的方式进行;(三)由于他们的阴谋被揭露,胡风集团不能不被迫从进攻转入退却,但这个仇恨共产党、仇恨人民、仇恨革命达到了疯狂程度的反动集团,绝不是真正放下武器,而是企图继续用两面派的方式保存他们的“实力”,等待时机,卷士重来。胡风用“在忍受中求得重生”,“一切都是为了事业,为了更远大的未来”这类的话来鼓励他的集团分子,就是证明。反革命的胡风分子同其他公开的或暗藏的反革命分子一样,他们是把希望寄托在反革命政权的复辟和人民革命政权的倒台的。他们认为,这就是他们要等待的时机。

我们从胡风集团的阴谋活动这一事实必须取得充分的经验教训,必须在各个工作部门中保持高度的警惕性,善于辨别那些伪装拥护革命而实际反对革命的分子,把他们从我们的各个战线上清洗出去,这样来保卫我们已经取得和将要取得的伟大的胜利。


