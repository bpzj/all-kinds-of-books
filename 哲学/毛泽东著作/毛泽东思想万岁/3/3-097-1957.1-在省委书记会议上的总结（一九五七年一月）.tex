\section[在省委书记会议上的总结(一九五七年一月)]{在省委书记会议上的总结}
\datesubtitle{(一九五七年一月)}


都是老话。成绩要有足够的估计,夸大不行,过低估计要犯错误。甚至犯大错误。胡志明把土改成绩估计低了,下了罪己诏,地主没有了,引起各方面(天主教、民主党派、知识分子)进攻,现在纠正了。

去年一年,成绩是基本的,这个问题,二中全会已经解决了。但民主人士和一部分同志的看法还有问题,认为错误缺点不少,所以有讲的必要。不足够估计成绩,合作社就有泄气之势。

我们历来的方针,赞成公共食堂,不另起炉灶,但要给管饭。一九四六年重庆大公报王芸生写社论说不要另起炉灶。我说“蒋委员长”要管饭才行,但“蒋委员长”不管饭。现在我们管事了,我们是统筹兼顾,各得其所,把各党各派,民主人士都包起来,把国民党的人也包起来,这是为了调动一切积极因素建设社会主义。这是一条战略方针。这条方针比较好,乱子出得少。统购统销,就是把六亿人口的饭都管起来,公社缺粮户,城市无粮户都管,反革命也够一份饭,人长了肠胃就要吃饭。苏联办法不是这样,不是都给供应。统筹兼顾,学生也不可能都进大学,但要都给适当安排,或补习,或到农村去,或去边疆,或给救济,这个思想,要在不清楚的同志中说清楚,不能把人饿死,所有的人都要给以安排,这一方针乱子少。

想尽一切办法搞增产节约,解决困难。想尽办法困难可以解决。这点困难没有什么,比长征过草地好得多。草地没有房子,只有空气,过了大渡河后何处走?千方百计地想办法,现在没有想出几方几计来,九百九十九方加一方是千方,九十九计加一计是百计。

国际问题。艾森豪威尔给蒋介石的信,先泼冷水,后打气,要蒋冷静,不要冲动。希望寄托在我们内部出乱子上。

出了一个苏伊士运河事件,这是一个怪事件,出了纳赛尔,要收回运河,艾登出了兵。英国资产阶级老于世故,老奸巨猾,出这样乱子不乱。这次艾登冲昏了头脑,犯了错误,把中东给了美国。主要矛盾在美国不在纳赛尔。文章都是美国对付英国的,美国想把中东夺回去。世界的矛盾基本的是社会主义与帝国主义之间的矛盾,帝国主义借口反苏反共争夺亚非中东,两派帝国主义争夺殖民地,最大的帝国主义是美国,第二是英法。殖民地出了一股民族独立运动。美国对日本、台湾是用武的,对中东是有文有武。他们闹对我们有利。

我们是“保守主义”,“右倾机会主义”,把九亿人口霸起来,易北河,三八线,十七度线一步不让,出了这个界线,让他们去闹,但我们可讲话。双方都搞颠覆活动,我们之中,有他们(地主、资产阶级、民主党派,……更不要说劳改犯),他们里面也有我们(共产党,工人、进步人士)。消灭阶级要很长时间,资本家变成工人要几十年,基本消灭只是讲大规模的斗争。苏联的同志不愿意接触这个问题。帝国主义内部争夺殖民地的矛盾大,利用他们的矛盾,这里面大有文章可做,这是条战略方针。

中国与美国的关系。迟几年跟美国建交比较有利。苏联一九一七年革命,一九三四年跟美国建交,经过十七年,当时美国发生经济危机,罗斯福上台,中东事件,苏联一封信,美国三军下命令戒备,究竟谁怕谁?双方都怕。谁怕谁多一些?我的倾向是帝国主义怕我们多一些,这样估计:如果大家一睡,三天不起,就有危险。还要看到坏的可能性,帝国主义会发疯。

世界大战,现在打起来不容易,打起来结果如何,他们要考虑。我和苏加诺谈,我们不急于进联合国,不急于和美国建交,使它在国际国内没有道理,把它的政治资本剥夺干净,使它孤立,以后总有一天会建交,一百零一年以后建了交,他更无能为力,后悔莫及。我们房子已经打扫干净了,四害除了,找不到他们的朋友了,如同他们在苏联一样,后悔莫及。帝国主义不怀好心,战败国不怕,主要是美国。

中苏关系。皮总是要扯的。不要设想世界上没有皮扯,马克思主义就是扯皮主义,因为总有矛盾,有矛盾就有斗争。现在中苏之间有点皮,但不大,比以前更靠拢更团结了。他们的方法与我们不一样,要等待,要做工作。党内各种意见,对党员也要做工作,要谈话,要开会,要谈心。南下渡江,蚊子多,天热,没有馒头,江是渡了,但是思想没有渡。组织上入了党,思想没入党。思想工作就是麻烦,不能怕麻烦。党内语言不一致是常有的,开会就是要解决这个问题。

形势比人强,形势逼得苏联同志不能不适当有所改变。照老样子统治下去,在国内国外都不利,二十次代表大会还是可以利用的。帝国主义利用,铁托利用,我们也可以利用,我们要帮助他们,但不要忙,要慢慢来,当面讲,法宝不要一次使用干净。“利令智昏”,无非是五千万吨钢,三亿吨铁,二千万吨石油。这点东西算数,加一倍,加十倍也不算数,不这是地球上撂那么点东西,官当大了,也可能昏头昏脑,昏了要臭骂一顿。这次周当面尖锐批评,当面抬杠。十个指头九个一致,一个指头不一致,中央和省市也有不一致的,矛盾总会有,求同存异。

国际宣传要压缩一点,文工团,展览会要缩小一点夸张宣传,把尾巴夹紧一点。

还是要学习苏联。他们有很多好东西可学。要有选择的学,先进的、有用的东西一定要学,错误的东西也要批判的学。三个五年计划把他们的基本东西学到手,对别的国家也要学。周出去的口号是找和平,找朋友,找知识,要到处找,单找一个地方太单调了。

百花齐放,百家争鸣还是对的,真理是同错误作斗争发展起来的。美是同丑作比较作斗争发展起来的。善事善人是同恶事恶人作比较作斗争发展起来的。香花是同毒草作比较作斗争发展起来的。唯物主义是同唯心主义作比较作斗争发展起来的。许多人恨蒋介石,但蒋介石是个什么王八旦,大家不知道,所以要出蒋介石全集,还出孙中山全集,康有为全集。禁止人们同丑恶、同谬误、同唯心主义、同形而上学见面,是很危险的政策,这将引起人们思想衰退,思想硬化,单打一,见不得世面,唱不得对台戏。我们共产党人对反面的东西知道得太少,比较单调。因此讲不出几句有说服力的话。马恩列都不是这样,他们都努力学习当前和历史上各种东西,并教人们也这样学。斯大林差些,否定德国哲学(康德、费尔巴哈),因为德国打了败仗,也否定德国军事学。德国古典哲学,是马克思主义的祖宗。斯大林实际上是形而上学,不承认对立统一。哲学词典上是形而上学的说法:“战争不变到和平,和平不变到战争。两个东西是割断的,没有联系,不互相转化,只斗争没有同一性。”列宁说战争是政治的继续,是特殊手段,和平是战争的结果,政治是和平时期的斗争。战争时期酝酿和平,和平时期酝酿战争。斯大林教坏许多人,他们有很多形而上学,不承认对立统一,思想硬化,因此政治上犯错误。偶尔有不同意见就排斥,反革命只有杀头,谁对苏联有不同意见就叫反苏。实际生活使斯大林不能这样做。斯大林并不是都杀头,都关,一九三六年一九三七年杀得多,一九三八年杀得少,一九三九年杀得更少。有不同意见就杀头行不通。我们就和斯大林有不同意见,我们要签订中苏条约,他不订,要中长路,他不给,但老虎口里的肉还是能拿出来的。

我们报上,对一些有害言论没有反驳,登了一些不该登的文章,例如:“难免论”就可以不登。再论无产阶级专政历史经验有个大难免论。谁愿意犯错误,犯错误都是以后才知道的,开始时总是百分之百正确,但做的结果总是要犯错误的。这是历史经验,方针是少犯一些错误,报纸在做了准备之后,有了充分的说服力再回答,没有充分准备不忙于回答。有些作品要十几年才能产生,如大型文艺作品。有些在一个时期不可避免,如过去把剧目控制得很死,现在一放,牛鬼蛇神都出来了,过去你看不到,现在看看也无妨。不要一出就骂,这对增产粮食棉花影响不大,也不影响财政,爱情戏演一点也可以,让观众去判断,不忙禁止。

主要的占统治地位的东西,必须力争是鲜花,折一把都是臭的也不好,毒草占次要地位就没有害了。好像共产党为主,民主党派为辅一样,好像原子核和电子一样,原子核虽小但很重,拉开原子核要几个火车头的力量。核子外围的电子则很轻,它是自由主义的,但没有电子也不行,这也是对立统一。

准备发行四十万份参考消息,把帝国主义资产阶级的东西发出来,替帝国主义尽义务,其目的是把非马克思主义的东西,毒草放在同志们和党外人士面前,使大家得到锻炼,否则,只知道马克思主义,不知道其他东西,不好。但要加强领导,好像种牛痘,使人的身体内部做斗争,产生免疫力。看参考消息,搬出唯心主义,出蒋介石全集,就是种牛痘。

人民闹事,很值得研究,这是新问题,我们过去和人民一起对敌人作斗争,我们和人民是矛盾的一个方面,敌人是矛盾的另一个方面,现在阶级快消灭了,要搞建设,反革命快搞完了,没有敌人了,对外虽有帝国主义,但它没有动,所以问题就来了,人民鼓起眼睛看我们,必须准备有一小部分人年年闹事,只要有精神准备,不会陷于被动。书记同秘书,部长同付部长,书记同书记要经常谈这些问题,分析交换意见,我的经验,这种研究有很大益处,匈牙利事件,拉科西、葛罗没有精神准备。河南临汝县三万人闹事,是最大贡献,下了一道赦令,把所有地主帽子都摘掉,推翻了乡政权,这就是打了一次防疫针,种了一次牛痘。石家庄步步放弃阵地,毫无精神准备,甘肃林校学生三百人闹得天翻地覆,无可奈何。共产党不怕蒋介石、帝国主义,怕学生闹事、怕农民闹社,这是新问题。两种可能,要放在最坏的可能性上。如西藏不出问题,两个可能,达赖一跑到美国,二留在印度,达赖不回来,中国大陆也不会沉。闹事原因:我们在经济上,政治上,犯了错误,工作方法生硬,反革命分子存在,所以要完全避免闹事不可能,只要不犯路线错误,不出全国性乱子。即是犯了路线错误,全国大乱,占了几省几县,甚至打到北京西长安街,只要军队巩固,我们也不会亡国,国家更会巩固。要像卡达尔一样,在国会大厦有一个月不能睡觉,要有这种思想准备。历史走回头路,反复也是可能的。辛亥革命革了皇帝又出了皇帝,我们的制度是否最后巩固了呢?两种可能性,也可能逐步巩固了,也可能犯错误出乱子。出了乱子总有英雄豪杰出来收拾局面的,最后总会巩固的。

对闹事如何处理?闹对了,承认错悮,满足要求。闹得不对的不能迁就,要给以批评,争取群众。孤立坏人。除行凶杀人外,不能乱捉人,不要挫伤学生的积极性,要逐步引导他们。要有领导艺术,他要闹,让他闹够,不闹够不收场,否则还会闹。不要急于开除坏人,剥夺干净他的一切政治资本。甘肃提了六十人,处理简单了,石家庄闹事要放假,我看就是不放假,把是非闹清楚了再放假,利用他们当教材。

对人民绝对不要轻易开枪,捉人,要开枪只能向天上开几枪,不要怕闹,有理由应该闹,没有理由也闹不长。闹不出一二.九来,我们过去也是闹出来的。

各部长、各省市要研究思想动向,过去忙于业务,不研究思想动向是不好的,要改变。必须尽可能避免使用武力,不要学习国民党的办法。甘肃的办法近乎国民党,除非真正的反革命暴乱,才可用武力对付。

我们要加强工作,改正缺点,要加强对工农兵商学的政治工作,现在大家都忙于业务工作,不搞政治工作是危险的。×××出马到大学作报告,大家都要出来到学校作报告,重要的是学校和军队。有的说军队政治干部政治思想衰退,这还得了?!必须在增产节约和整风中好好整顿,把干部意志振作起来。四川学生要上北京请愿,不让来不好,来有好处,生活复杂一点好,不然太单调了,清一色的麻将难打。

青年团要加强学校工作,工会要加强工厂工作,民主人士让他们去批评,唱对台戏,批评不过两种:对的补我们的短处,错的越错越好,上台一讲就揭露了,梁漱溟、彭一湖、黄炎培也叫过,以后又检讨了,章乃器批评我们统战部,放手让他们批评,一批评就反倒把他们孤立了。民建会一斗,还给他们饭吃,部长还给他当,错误犯得越大越有教育意义。不要怕他们批评,不要跟国民党一样怕批评,对批评要分别情况,有些采取主动,有些采取后发制人。又团结,又斗争,这种斗争是长期的。工人阶级,劳动人民,知识分子从斗争中取得益处,得到锻炼。坏人坏事有两面性,埋伏着好的作用,很多同志弄不通这个道理,表现在选择王明问题上。坏事包含好事,好事也包含坏事,出了事就认为不好,不加分析,是形而上学观点。去年大胜利,包含消极的东西,头脑膨胀。

我们在建设时期,对阶级斗争(是部分的),人民内部斗争(就主要的)经验不足。这是一门科学。应当很好的研究。一万年以后也会有人闹事,我们在第三个五年计划内要取得经验。

法制问题,一定要守法,一定要肃反,一定要肯定成绩。守法是守社会主义的法,不能破坏法制。法是劳动人民制定的,保护劳动人民的利益,保护生产力,保护经济基础,一定要守法。没有完成肃反的继续搞,一年完成。肃而不清的看来不少,要在事变中逐步肃清,肯定过去成绩,给肃反干部撑腰。民主人士骂,让他去骂,不能因为他骂就不敢动手了。敢于闹事的人,改造过来可能是有用之材,徐懋功十二岁闹事,以后做宰相。肃反中的错误要严肃批评。按着法律要放手放脚,不要缩手缩脚。

争取今年丰收,稳定人心。合作社可以巩固,从初级社算起五年巩固下来。农业首先关系到五亿人口的吃饭问题,非商品的产品很大,三分之二是自给经济。五亿人口自给,天下就稳定了。其次是轻工业的原料市场;第三,农业大部分是工业市场,如化肥、农具、铁路、几乎全部公路、电力、煤炭、石油、大型水利、建筑材料等,重工业的市场主要在农村。美国没有封建制度,市场广大;第四,出口物资主要是农业,农产品变外汇。支援工业,农业即工业。第五,发展农业是国家积累的主要来源。所以要说服干部到农村去,要工业化就要搞农业。积累要搞出一个比例来。斯大林把积累搞多了,反而影响工业,究竟比例多大,还必须研究。总之要使合作社扩大再生产,能得到更多的积累,不能竭泽而鱼。

农业生产合作社要搞经济核算,利用作价法则计算成本,商品粮价格到一定的时候提高百分之五。现在全国商品粮一千万斤,十二年要提高到五千万斤,商品粮增多了,如果价格太低谁还种地?我不是说现在就要提价,但价格政策要研究。

合作社积累一年比一年增多一点,但仍不能太多,让农民吃饱一点,如果今年丰收,增加部分可多留公积金,丰收年多积累一点、歉收年平衡。

一切事情都是波浪式的,螺旋式的,走路、开会、电话、声波、唱戏、说话、我写字,都是波浪式的,赞成学辩证法。


