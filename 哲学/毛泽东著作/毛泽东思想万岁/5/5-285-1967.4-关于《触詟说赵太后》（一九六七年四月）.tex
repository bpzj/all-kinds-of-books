\section[关于《触詟说赵太后》(一九六七年四月)]{关于《触詟说赵太后》}
\datesubtitle{(一九六七年四月)}


这篇文章反映了封建制度代替奴隶制度的初期,地主阶级内部财产、权力的再分配。这种再分配是不断的进行的。所谓“君子之泽,五世而斩”就是这个意思。我们不是代表剥削阶级,而是代表无产阶级和劳动人民,但如果我们不严格要求我们的子女,他们也会变质,可能搞资产阶级复辟,无产阶级的财产和权力就会被资产阶级夺回去。

(注:主席的这段重要讲话是江青同志1967年4月12日在军委扩大会上传达的)

附《触詟说赵太后》(《战国策·赵策四》)原文

赵太后新用事。秦急攻之。赵氏求救于齐。齐曰:“必以长安君为质,兵乃出。”太后不肯,大臣强谏[jian音箭]。太后明谓左右:“有复言令长安君为质者,老妇必唾其面!”

左师触詟[Zhe音哲]愿见太后。太后盛气而胥[xu昔须]之。入而徐趋,至而自谢,曰:“老臣病足曾不能疾走,不得见久矣.窃白恕。而恐太后玉体之有所郄[xi音戏]也,故愿望见太后。”太后曰:“老妇恃辇[nian音碾]而行。”曰:“日食饮得无衰乎?”曰:“恃粥耳。”曰:“老臣今者殊不欲食。乃自强步,日三四里,少益耆[qi音奇]食,和于身也。”太后曰:“老妇不能。”太后之色少解。

左师公曰:“老臣贱息舒祺,最少,不肯。而臣衰,窃爱怜之。愿令得补黑衣之数,以卫王宫。没死以闻。”太后曰:“敬诺。年几何矣?”对曰:“十五岁矣。虽少,愿及未填沟壑[huo音货]而托之。”太后曰:“丈夫亦爱怜其少子乎?”对曰:“甚于妇人。”太后笑曰:“妇人异甚。”对曰:“老臣窍以为媪[ao音袄]之爱燕后,贤于长安君。”曰:“君过矣,不若长安君之甚。”左师公曰:“父母之爱子,则为之计深远。媪之送燕后也,持共踵[zhong音种],为之泣,念悲其远也。亦哀之矣。已行,非弗思也。祭祖必祝之,祝曰:‘必勿使反。’岂非计久长,有子孙相继为王也哉?”太后曰:“然。”左师公曰:“今三世以前,至于赵之为赵,赵王之子孙侯者,其继有在者乎。”?曰“无有。”日:“微独赵,诸侯有在者乎?”曰:“老妇不闻也”。“此其近者祸及身,远者及其子孙。岂人主之子孙则必不善哉?位尊而无功,奉厚而无劳;而挟重器多也。今媪尊长安君之位,而封之以膏腴[yu音鱼]之地,多予之重器,而不及今令有功于国,一旦山陵崩,长安君何以自托于赵?老臣以媪为长安君计短也,故以为其爱不若燕后。”太后日:“诺,恣君之所使之”。于是,为长安君约车百乘,质于齐,齐兵乃出。

子义闻之,曰:“人主之子也,骨肉之亲也,犹不能持无功之尊,无劳之奉,而守金玉之重也,而况人臣乎?”

