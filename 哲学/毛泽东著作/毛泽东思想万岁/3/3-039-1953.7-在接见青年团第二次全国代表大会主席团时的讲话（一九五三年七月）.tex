\section[在接见青年团第二次全国代表大会主席团时的讲话(一九五三年七月)]{在接见青年团第二次全国代表大会主席团时的讲话}
\datesubtitle{(一九五三年七月)}


闹独立性的问题不存在,我们没有这个印象。现在缺乏的是团的独立工作,过去缺点是这方面,而不是闹独立性。青年团要配合党的中心工作,但配合党的中心工作当中要有青年团的独立工作。独立性的问题早已过去了,×××有一点,但也不多。团的工作根据各地党委反映都是满意的,满意就是配合中心工作,所以说满意得很,现在要求来个不满意。就是适合青年特点来些独立的活动。团的领导机关要学会如何领导团的工作,党的领导机关也要学会,就是围绕党的中心任务,照顾青年特点,不断的组织群众。

过去团的工作在党的领导下,团的领导下,配合多方面的革命工作,得了很大成绩,成绩很显著。无论城市、工厂、农村、学校没有他们革命就不能胜利,他们是很有纪律的,他们完成了各项任务,现在革命不是天天革,朝鲜议和,土地改革结束,国内革命完成,现在转到建设工作。建设就要学习,学习如何领导青年和成年一道在农村把农业做好,在城市把工业做好,在学校把学习做好,在机关把工作做好,在军队把国防军练好,成为近代化军队。这些都是全国人民都要做的,除了小孩子。

但是十四——二十五岁是人们长身体的时间(二十五岁以上就不长了)。又是工作时间,又是学习时间,如果对长身体不重视则是很危险的。成年人也要学习工作,他已学了多少年;但青年刚学习刚工作,故与成年有不同,很多东西成年已会,而青年不会,例如种地青年就不如老年,如果十八岁会种地那是因为他以前学的好。十四岁——十八岁和十八岁——二十岁不能与成年人一样的强度,不能按八小时工作制,不能挑重担子,他们就是要玩。娱乐,不玩就不高兴。十八岁以后就要恋爱结婚,这与成年人不同;成年人这问题解决了。

我提议,学生的睡眠时间再增加一小时,现在是八小时,实际只有六、七小时,开始有一小时睡不着,普遍感到睡不够。要下一道命令,下一道命令就解决了。不要讨论,强迫执行。教师们也要睡足,尤其是青年们要睡好。(周总理:青年团要带头。)我给青年们讲几句话:一,庆祝他们身体好;二,庆祝他们学习好;三,庆祝他们工作好。只有八小时睡眠时间,学生是不够的。因为知识青年容易神经衰弱,他们往往睡不着,醒不来。如果九小时不行,就再增加一小时,一共十小时睡眠时间。

革命带来很多好处,但也带来一个坏处,就是太积极太热心了,以致使大家疲劳不堪。现在要保证大家身体好,军队、学生、干部都要身体好。当然身体好并不一定学习好,鲁智深、李达学习就不高明。学习要有一些办法。

一定要规定九小时睡眠时间,只有九小时睡眠才能获得八小时睡眠。初中现在上五堂课,多了一些,可考虑上四节课,一定要把娱乐、休息时间计算在内。现在积极分子开会妨碍娱乐、休息、睡眠,应该减少这些活动。一方面学习,一方面娱乐、休息、睡眠;要两方面兼顾。现在是学习太厉害了,后者却不够了。(周总理:还有工作太厉害。)工农兵青年们,是在工作中学习,工作学习时间与娱乐睡眠二方面要相称,两方面要充分兼顾。

两头都要抓紧,学习工作要抓,睡眠、休息、玩也要抓。过去抓一天,那一头抓不紧或未抓(用手打比)。跟教育部的人说,要说服他们,你如果这样还是太松。工厂学校搞得紧,会太多,五多。现在要抓紧,减少会议次数与时间。一定要搞些娱乐,要有时间,有设备,这方面也来个抓紧。党中央已决定减少会议学习时间,你们监督执行;有什么人不执行,就要质问他为什么不执行?

总之,要使青年身体好,学习好,工作好,有些领导同志只顾工作,而不顾青年的身体,则用这句话抵他一下。理由很充分,这就是为了保护青年一代的成长,为了年轻一代更好地发展。我们一代吃大亏,大人不照顾孩子,大人有大桌子,小人没有;娃娃在家里没有发言权,如果哭则一巴掌。现在新中国要把方针改一改,要为他们设想。

关于选青年干部当中委问题:周瑜那时当统帅,曹操下江南,谁当统帅那时成问题,就找个青年团员——周瑜,周二十九岁当统帅,要指挥黄盖等,老的就发生问题,大家不服,后来加以说服,还是由他当,结果打了胜仗。

现在“周瑜”当中委大家不赞成;党是否把年龄大的都挑选过来,年轻的不选呢?党中央选的就有年轻的,自然要挑选一批,不能统统按年龄,要按能力,原来九个三十岁以下的,现在六十几,超过四分之一,三十岁以上占了四分之三,还说少了。我说不少。六十余青年人是否称职,没有把握,如不称职则以后可改掉,可连选连任,有不连选的,基本方向是不会错的。绝大多数以至全体会胜任的。青年人不比我们弱。老年人当然强,有经验,但其它方面却在退化,眼睛耳朵不那么清楚,手脚慢,不如青年敏捷,这是自然规律。要充分相信青年人,可能个别不称职,不要怕不称职。

要说通那些不赞成的同志这样做:一是照顾青年特点,一是照顾团的系统工作,同时受各级党委领导,这总是不会错的。这不是新发明,老早就有了的。马克思主义历来就这么讲。这是按事实,从实际出发,青年就是青年,不然何必搞青年团。这是普遍大量存在的事实。青年要长身体,因此工作情况就有不同。特别十四岁到十八岁。青年妇女与男子也不同,不照顾这些特点就会脱离群众。你们不修改这一条,也许只有一百万拥护你们,八百万不拥护你们。

工作要放在照顾多数,同时注意先进青年。这样可能先进部分不过瘾,他们要严,但要说服他们。有人说团章义务多权利少,改了没有?(胡××:八条已改为七条。)还有七条之多?(李昌:已经缓和了些。)是要缓和一下,照顾几百万群众。不要只顾几十万、一百万最活动的分子,要使多数能跟上去。重点要放在多数,不要只看到少数。

四个月不参加组织生活要自动退团,这个太严了。党章六个月,你们跑到党的前面去了。农村青年是比较散漫的,不能像学校、军队、工厂一样。搞六个月不行吗?

办不到的,厉害的,不要在团章上规定。“不要背后乱讲”也是这样,原则性要灵活执行。法律条文要执行,也还得几年。应当是那样,实际是那样,是有个距离的,如婚姻法彻底实行至少三个五年计划。许多条文是带着纲领性的。不歧视妇女,不打老婆,有一百万办到就很好了。把“不歧视妇女”划掉了也是因为办不到。反对自由主义是长期的,还是让人背后骂骂,不准在背后骂一句话,事实上办不到。不要把框子搞得太小,有的人可以在房子里,有的在院子里,有的在中南海,在北京,在中国,走来走去还可以革命,只是不要到香港台湾。敌我界限要分明,搞掉自由主义就搞死了。有许多人只能搞这一套,党内自由主义还不少。

威信是慢慢建立的。过去军队编歌谣骂人,我们就不禁也不查。军队还是未垮。我们只抓住一些大的,三大纪律,八项注意,就慢慢上了轨道。群众对领导者真正佩服要靠了解,真正了解才相信他。你们团中央威信就很高。有些人不佩服,慢慢会佩服的。青年团只有四年历史,×××刚上台不久,不是一个早上就会佩服的。小伙子上台,威信不高,不要着急,不受点批评不挨骂是不可能的。

小广播是因为大广播不发达,只要民主生活充分,当面揭了疮疤,让他小广播,他就会说没时间了。抓住大的,慢慢就会好了的。问题总是有的,不要以为一下子就都能解决,今天有,将来还会有问题的。

根据社会的发展,三个五年计划基本上可以完成,工业化和农业手工业的社会主义改造基本上完成,不等于全部完成,三个五年计划就是十五年,一年一小变,五年一大变,一年一小步,五年一大步。三个大步就差不多了。国家工业化,就是说工业产值同农业产值相比较,是七与三之比。到了百分之七十,就可以完成社会主义改造,大概十五年可能达到百分之七十。讲基本是谨慎的讲法,世界上的事情谨慎一点好。

中国农业现在大部分是个体经济,要有步骤的进行社会主义改造,不能猛。发展互助合作组织,要建立在自愿原则上。不去发展就会走资本主义道路,就是右倾。搞猛了也不行,搞猛了就是“左”倾。要有准备有步骤的进行。我们过去不打无准备之仗,不打无把握之仗,只有准备但无把握也不打。你们这个会议是开得好的,你们算是有准备的。过去打蒋介石,开始犯主观主义错误,后经整风,改棹了主观主义,就打了胜仗。现在是打建设之仗。什么叫建设?就是工业化,三个五年计划基本上完成了工业化和对农业、手工业、资本主义工商业的社会主义改造,这是全国人民的总任务。青年如何执行,应有特殊规定,应照顾青年的特殊条件。

十年至十五年左右或者更长一点时间,时间不要公开宣布。在报上登载只说相当长时间。团内可传开。资本家会不会吓一跳?吓半跳是会的。农民不怕者来,怕者不来,从少到多。现在组织起来的,东北是百分之八十,华北百分之七十,南方不同,要作一些精神上的准备和组织上的准备。北方,合作社也不能大发展,河北六千个合作社,一百四十六个县。劝他们退社转互助组的有两千多户。条件不够,还是搞互助组。今年发展比去年多五倍,如果每年多五倍,那还得了。明年再加一倍,这样速度是相当快的。发展合作社一要有群众自愿条件,二要有干部条件。要有许多办法,才能做好。


