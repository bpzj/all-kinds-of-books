\section{发起“文化书社”缘起(摘要)}
\datesubtitle{(一九二〇年七月三十一日)}



没有新文化,由于没有新思想;没有新思想,由于没有新研究;没有新研究,由于没有新材料。湖南人现在脑子饥荒,实在过于肚子饥荒,青年人尤其嗷嗷待哺。文化书社愿以最迅速、最简便的方法,介绍中外各种新杂志,以充青年及前进的全体湖南人新研究的材料。也许因而有新思想、新文化的产生,那真是我们心向祷之,希望不尽的。

文化书社由我们一些互相了解,完全信得过的人发起,不论谁投的本,永远不得收回,亦永远不要利息,此书社但永远为投本的人所共有。书社发达了,本身到了几百元,彼此不因为利;失败至于只剩一元,彼此无怨;大家共认地球之上,长沙城中,有此“共有”的一个书社罢了。

\begin{flushright}(摘自《光辉的五四》第91页,中国青年出版社1959版。)\end{flushright}

