\section[接见智利新闻工作者代表团的谈话(一九六四年六月二十三日)]{接见智利新闻工作者代表团的谈话}
\datesubtitle{(一九六四年六月二十三日)}

\begin{list}{}{
    \setlength{\topsep}{0pt}        % 列表与正文的垂直距离
    \setlength{\partopsep}{0pt}     % 
    \setlength{\parsep}{\parskip}   % 一个 item 内有多段,段落间距
    \setlength{\itemsep}{\lineskip}       % 两个 item 之间,减去 \parsep 的距离
    \setlength{\labelsep}{0pt}%
    \setlength{\labelwidth}{3em}%
    \setlength{\itemindent}{0pt}%
    \setlength\listparindent{\parindent}
    \setlength{\leftmargin}{3em}
    \setlength{\rightmargin}{0pt}
    }

\item[\textbf{主席:}] 你们以前都没到过中国吗?

\item[\textbf{席尔瓦:}] 都是第一次来中国。

\item[\textbf{主席:}] 第一次来过以后就能再来啦!我们两国记者和两国人民互相联系,这是好事。我们两国政府还没有建立关系,你们政府大概会有困难。也许你们的政府不高兴我们。

\item[\textbf{席尔瓦:}] 不,他们并不是不高兴。前不久智利政府向中国卖了铜和硝石,这就是一个证明。

\item[\textbf{主席:}] 有生意吗?

\item[\textbf{席尔瓦:}] 是。

\item[\textbf{主席:}] 那好。

\item[\textbf{席尔瓦:}] 不久前在圣地亚哥举办了中国经济展览会,引起了智利人民很大的兴趣。许多工人、学生、职员都去参观。他们看到了原来以为中国不能生产的许多机器和其他产品。

\item[\textbf{主席:}] 我第一次从你口里知道中国在圣地亚哥举办了经济展览会。看来,我这个人官僚主义很厉害。

\item[\textbf{席尔瓦:}] 早在上一世纪,就有中国人到达我们的海岸,他们很勤劳,很受尊敬。我们的政府是主张民主的,而且实行民主。对其他国家政府采取不干涉政策。比如,我们和古巴就有外交关系,而且哈瓦那有智利的大使馆。

\item[\textbf{主席:}] 有吗?过去巴西在古巴有大使,现在还有吗?

\item[\textbf{席尔瓦:}] 没有了。

\item[\textbf{主席:}] 巴西发生了变化。

\item[\textbf{巴斯克斯:}] 巴西发生了一百八十度的变化。

\item[\textbf{主席:}] 巴西政变当局把中国人抓起来了,九个人中七个是做生意的,两个是记者。他们把中国人抓起来,不知对巴西统治集体有什么好处。

\item[\textbf{席尔瓦、巴斯克斯:}] 毫无好处。

\item[\textbf{主席:}] 他们可能得到一些美元。

\item[\textbf{席尔瓦:}] 可能是这样。

\item[\textbf{主席:}] 对他们没有什么好处,做这样的事干什么。

\item[\textbf{席尔瓦:}] 我要向毛主席谈到这样的情况,拉丁美洲的报纸经常受到美国佬通讯社的影响,他们经常制造一种气氛,说中国要挑起战争,或准备战争。但是,我们在中国的参观中亲眼看到的一切,证明中国人热爱和平,渴望和平,不要战争。昨天我们有机会得到陈毅付总理的证实。中国是渴望和平不要战争的。我们看到并证实中国人民是在和平的基础上来建设中国。

\item[\textbf{主席:}] 打仗对我们没有好处,我们要进行建设,打仗就会把我们进行的建设打烂了。国民党打内战跟我们打了好多年,后来我们又跟日本打了八年。不是我们打到日本去,而是日本打到中国来。讲长远一点,都是外国打到中国来。\marginpar{\footnotesize 128}中国曾和英国打了几次战争,如一八四〇年在广东的鸦片战争,还有八国联军的战争,八个国家占领了天津,打到北京。中国和日本战争,是一八九四年在渤海湾的旅顺、大连打的。以后日本占领了我们东北。在那以前,沙皇俄国和日本还在中国的土地上打仗,那是在旅顺、辽阳、沈阳一带。最后是第二次世界大战期间,日本几乎侵占全中国。这些都不是我们打到外国,都是外国人打到中国来。中国人打到外国去,在古代有过。那是中国的皇帝。打到越南、朝鲜。以后日本占领了朝鲜,法国占领了越南。一九一一年,我们推翻了清朝皇帝,接着就是各种军阀混战,那时中国完全没有共产党。有了共产党以后,就是革命战争,那也不是我们要打,是帝国主义、国民党要打。比如我这个人,也做过新闻记者,当过小学教员,那时根本不知道世界上有共产党,因此也没有想到自己要加入共产党。一九二一年中国有了共产党,我就变成共产党员了。那时候我们也没有准备打仗。那时我是一个知识分子,当一个小学教员,也没学过军事,怎么知道打仗呢?那就是国民党搞白色恐怖,把工会、农会都打掉了,把五万共产党员杀了一大批,抓了一大批。因为白色恐怖,有一些共产党员不干了,消极了,只剩下几千共产党员,上山打游击,后来经过万里长征,跑到北方来。军队原有三十万,剩下两万多人。你看,不是共产党员没有用了吗?人数不是很少了吗?两次人数减少,前一次在一九二七年,这次在一九三四年。我们人数少了敌人就高兴了。恰好在人数减少的时候,我们改正了错误,走上了正确的道路。后来人数又有了发展。日本走了以后,蒋介石再来打我们的时候,敌人就不行了。到现在我们建设只有十五年的时间。中国要和平。凡是讲和平的,我们就赞成。我们不赞成战争。但是对被压迫人民的反对帝国主义的战争,我们是支持的。对古巴我们是支持的,对阿尔及利亚革命战争我们也是支持的,对越南南方人民反对美帝国主义的战争我们也是支持的,这些革命是他们自己搞起来的。不是叫卡斯特罗起来革命的,是他自己起来革命的。你们相信吗?是美国叫他革命的,是美国走狗叫他革命的,是我们叫本·贝拉革命的吗?以前我们不认识这个人,到现在我们还没见过他。是他们自己起来革命的,他们成立了临时政府,我们就承认。他们要支持,我们就给他支持。帝国主义说我们是“侵略者”,是“好战分子”,在某一点上讲也有些道理。因为我们支持卡斯特罗,支持本·贝拉,支持越南南方人民反美战争。还有一次是在一九五〇年到一九五三年,美国侵略了朝鲜,我们支持了朝鲜人民反对美帝国主义的战争,我们的这一方针是公开宣布的,我们是不会放弃它的。要支持各国人民反对帝国主义的战争,如果不支持,就会犯错误,就不是共产党员。你们知道阿联总统纳赛尔不是共产党员,但支持过阿尔及利亚革命。他们不是共产党员,他能支持阿尔及利亚难道我们是共产党员就不能支持阿尔及利亚吗?当一百七十多年以前,华盛顿起来反对英国的时候,法国支持了华盛顿,难道当时法国人是共产党员?法国在当时已是共和国。美国反对英国的革命胜利在哪一年?

\item[\textbf{巴斯克斯:}] 一七八九年。

\item[\textbf{主席:}] 一七八九年七月四日是起义的日子,还是胜利的日子?

\item[\textbf{巴斯克斯:}] 是起义的日子。

\item[\textbf{主席:}] 那时中国还没有共产党,全世界还没有共产党。共产党出世是在十九世纪的事。大概我们这个“好战分子”,“侵略者”的称号还要继续下去。

主要一条还是我们国内问题。在国内,我们把美国走狗蒋介石赶走了,\marginpar{\footnotesize 129}把美国的势力也赶走了。所以美国对我们不那么高兴。我不是指美国人民,而是指美国资本家。在北京也有一些美国人,他们对我们是友好的。

你们那么喜欢美国资本家吗?

\item[\textbf{巴斯克斯:}] 我们不喜欢。

\item[\textbf{主席:}] 美国要把拉丁美洲变成它的殖民地。

\item[\textbf{席尔瓦:}] 它永远做不到,但在经济上可能做到。

\item[\textbf{主席:}] 我是说在经济上,许多时候是在政治上。比如说,巴西前总统古拉特,我见过他,他的党是工人党,不是共产党,美国人都不能容忍它,把他推翻。甚至稍微不听他的话的吴庭艳,美国都把他杀掉。在美国国内也不是那么和平的。吴庭艳兄弟是被美国肯尼迪政府杀掉的,没过几个月,肯尼迪又见上帝去了。不知什么人把他杀掉的。是共产党人还是什么人?美国不说是共产党干的也不说是什么人,这个案子到现在都审不清。

美国说我们是“侵略者”,我们说他是侵略者,他说我们是“好战分子”,我们说美国政府的大资本家是好战分子。究竟谁是好战分子,要叫全世界人民看。在我们周围布满了美国军事基地。总不是我们占领了美国什么岛屿,而是美国侵占了中国的台湾。你们智利没有侵略我们,我们也没有侵略智利,我们没有侵略任何拉丁美洲国家和非洲国家。我们只侵略了亚洲一个国家——中国。(众笑)跟帝国主义打了几十年仗,把它赶走了。这件事情使美国很不高兴,其他帝国主义也不高兴。不过现在没有办法,总不能从地球上把我们搬走,等于从地球上不能把你们搬走一样。他们想把古巴搬走也不行。甚至很小的国家它要搬走也不行,比如阿尔巴利亚。

\item[\textbf{主席:}] 你们智利有多少人口?

\item[\textbf{席尔瓦:}] 有七百万,相当于北京的人口。

\item[\textbf{主席:}] 北京城市人口只有四百多万,你们的人口跟上海差不多,上海有七百多万人口。你们去过上海吗?

\item[\textbf{巴斯克斯:}] 去过。这次在中国旅行还去了沈阳、鞍山、抚顺、长春、南京、无锡、杭州。

\item[\textbf{主席:}] 你们去了不少地方,花了多少时间?

\item[\textbf{佩雷斯:}] 二十七天。这次旅行,非常有兴趣,在智利对中国很少了解。这次参观了工厂、学校、农村等。

\item[\textbf{主席:}] 还参观了工厂、学校了吗?

\item[\textbf{佩雷斯:}] 是,除了工厂、学校,还参观了人民公社、矿山、石油、钢铁、汽车制造等工厂和上海工业展览馆。

\item[\textbf{主席:}]中国的工业建设在我们说来才刚刚开始,和我们的人口比较起来,还很不相称。因此外国人说:中国是很落后的,这一点我们早就说过了。要改变中国的落后面貌,也不是很短的时间能做到的。至少要几十年的功夫,你们国家的工业是不是比我们先进一些?

\item[\textbf{席尔瓦:}] 也不。我们的工业发展根据人口看,还可以。我们有纺织工业、冶金工业,但比起鞍钢来还差得多。

\item[\textbf{主席:}] 应该按人口比,我们虽然有鞍钢,但按人口比还很不相称。

\item[\textbf{席尔瓦:}]不管怎样,你们取得了很大的成就。

\item[\textbf{主席:}] 有点成就,不算很大,有一点。\marginpar{\footnotesize 130}

\item[\textbf{席尔瓦:}] 我们所得到的印象和主席所说的相反。我们看到在党的领导下,全体中国人民干劲十足,抱着牺牲的精神,完成党所交给的任务。

\item[\textbf{主席:}] 这一点是真的,中国人民是有组织有纪律的,你看我们的警察很少,只有不多的交通警。人民自己组织起来维持交通。

\item[\textbf{巴斯克斯:}] 这一点已深深地引起我们的注意。

\item[\textbf{主席:}] 在旧社会是不可能的,没有军队和武装警察,就会发生抢劫、偷盗。现在北京抢劫现象没有了,偷盗还有一点,也不多了。

\item[\textbf{席尔瓦:}]在这样有七亿人口的国家,人民的觉悟不可能都是一样的。

\item[\textbf{主席:}] 人民自己来批评那些小偷行为,靠人民自己维持秩序。

\item[\textbf{巴斯克斯:}] 我们看到在街上,在学校,少先队员维持秩序,使我们深受感动。

\item[\textbf{主席:}] 解放初期北京还有些车祸,现在可以说没有什么车祸了,汽车压死人的事可以说很少了。

\item[\textbf{佩雷斯:}] 实在是这样。

\item[\textbf{主席:}] 的确比初解放几年起了变化,我们训练司机避免车祸,也教育行人不要乱闯。

我们这里还有一些贪污分子,我们对他们进行批评。我们把它叫做整风。要做到政府工作人员不贪污,不是一件容易的事。我们把它当做人民内部矛盾来处理,把这少数人教育过来,总相信多数人是好的。无论哪一国的人民,做坏事的总是少数,并且做坏事的人也可以改变。甚至跟我们打过仗的,被我们俘虏的国民党将军也可以改变。经过改造,他们不那么反对我们了。还有一个清朝的皇帝也是这样,他现在全国政协搞文史资料工作,他现在自由了,可以到处跑啦。过去当皇帝,好不自由。

\item[\textbf{巴斯克斯:}] 过去他只能看小山的景致,现在他解放了。

\item[\textbf{主席:}] 过去当皇帝时,他不敢到处跑,是怕人民反对他,也怕丧失自己的尊严,当皇帝到处跑怎么行。可见得人是可以改变的。但不能强迫,要劝他自觉,不能强压。美国人说我们“洗脑筋”。脑筋怎么洗法,我还不知道。我的脑筋就是洗过的。我以前信过孔夫子、康德那一套,后来不相信了,信了马克思主义啦!这是帝国主义、蒋介石帮了我的忙,是他们给我洗的。他们是用枪屠杀了中国人民。譬如日本在中国就不知杀了多少人,占领了大半个中国,后来美国和蒋介石又发动了全国性的反对我们的战争。

他们都是些给人洗脑筋的人,使全中国人民都团结起来和他们斗争,中国人民的精神面貌从而起了变化。你们说是谁把卡特罗的脑筋给洗了?(众笑)

\item[\textbf{席尔瓦:}] 关于洗脑筋这一点,我要亲自向主席说:你们不仅只洗掉我们脑筋里美国的谎言,也使我们睁开了眼睛,看到了中国的现实。

\item[\textbf{主席:}] 你们过去大概对中国不太清楚吧?看一看就清楚了。你们每五年来一次,看看我们有没有进步。

\item[\textbf{席尔瓦:}] 主席,你作为马列主义的最高领袖,可以感到非常满意和高兴。因为有这样的国家全国人民守纪律,努力工作,在中国全力进行建设。在这十五年中已清除了在几个世纪里遭受的贫困、压迫和掠夺,你们已经取得了很大成就,相信今后将会取得更大的成就。

\item[\textbf{主席:}] 不能估计太高,我对我们的工作不那么太满意,我们的工业、农业、文化、教育、科学所取得的成就,比起我们那么多人口,还不相称,这是事实。\marginpar{\footnotesize 131}我们仅只说比国民党蒋介石统治时期进了一步。还有一件事实,美国人说,我们政府不是今年要倒台,就是明年要倒台,这件事恐怕不那么真实。看来今年不会倒,明年不会倒,后年呢,我说也不会倒。要把我们政府打倒,需要美国、蒋介石打到我们这里来,把我们打倒,即使他们来了,也不一定要达到目的。他们曾经来过,可是打输了。现在南越只有一千四百万人口,美国在那里进也不好,退也不好,陷在泥坑里。对拉丁美洲,美国也是头痛的,在这一点上我们是乐观的。全世界人民总要起来的,自己做主人,不要资本家做主人,因为我们相信这一点,所以那些资本家对我们不那么好感。但是为什么除了美国,有那么多资本家跟我们做生意呢?就是因为我们不干涉他们的内政。美国人想跟我们做生意,我们就不做,想派新闻记者来,不成。我们认为大问题没有解决以前,这些小问题、个别问题可以不忙着去解决。所以智利新闻工作者能来中国,美国记者来不了。但总有一天会来的,总有一天两国关系会正常化的。我看还要十五年因为已经过了十五年了,再过十五年就是三十年。如果不够,就再加。

\item[\textbf{席尔瓦:}] 我们非常感谢你,你在繁忙的工作中,能抽空接见我们,我们能听到主席亲口和我们谈话,我们将把这些话带回去,作为对我国人民的问候。

\item[\textbf{主席:}] 问候你们国家的人民,问候愿意跟中国交朋友的一切人。

\item[\textbf{席尔瓦:}] 我们有许多人愿意跟中国交朋友。

\item[\textbf{主席:}] 我相信这点,你们就是证明。

\item[\textbf{席尔瓦:}] 我们在智利也接待过三个中国去访问的记者代表团。

\item[\textbf{主席:}] 中国有记者到过智利吗?

\item[\textbf{席尔瓦:}] 在座的常××先生就参加了中国访问智利的第一个记者代表团。

\item[\textbf{主席:}] 他(指常××)所在的报纸《大公报》有六十二年的历史了。过去为满清皇帝服务过,替北洋军阀服务过,替蒋介石服务过,现在替人民服务。

\item[\textbf{巴斯克斯:}] 我所在的报纸,也是一九〇二年创办,和《大公报》同年创办,但是现在还没有为人民服务。

\item[\textbf{主席:}] 将来可以为人民服务,《大公报》的社长王云生不是共产党员,现在也不是共产党员,过去他为蒋介石服务过,也为别人服务过,现在为人民服务。过去很多这样的人,差不多全部大学教授、中学、小学教员等等,许多人都是国民党的,但都没有走,我们不跟这些人合作就没有教员,就不能办报纸,也没有唱戏的艺术家,没有画画的美术家。我们有广大的统一战线,其中有许多人不是共产党员,但都团结在一起。

\item[\textbf{巴斯克斯:}] 这些都是中国人。

\item[\textbf{主席:}] 对!但有些中国人就不那么和气,比如蒋介石,他也是中国人。(众笑)那么多大学教授、工程师、医生、新闻工作者都为人民服务。我们把他们团结起来,敌人就不高兴。就谈到这里为止吧!

\item[\textbf{席尔瓦:}] 今天能见到主席,感到非常荣幸和幸福,因为到了中国没见到主席,那就等于没有到中国。

\item[\textbf{主席:}] 你们要见我,我就见你们。祝你们一路平安。\marginpar{\footnotesize 132}

\end{list}

