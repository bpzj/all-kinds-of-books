\section[在关于四清运动中央工作会议上的讲话(一九六三年五月)]{在关于四清运动中央工作会议上的讲话}
\datesubtitle{(一九六三年五月)}


先看二十个材料引起大家讨论,先看三天。各中央局、省开会也是如此,不要传达中央文件有一个框框。不要性急,横竖准备搞它一年、两年,两年搞不完搞三年。\marginpar{\footnotesize 42}这样一个大的运动需要时间,不要性急。

这个革命运动是土改以来第一次最大的斗争。这样全面,这样广,这样深远是几年没有的。三反五反搞城市,一九五七年反右是思想战线上,反高饶是党内的。全面的、党内党外的这样的阶级斗争是十几年来没有的。这次从党内到党外,而上到下,从干部到群众,这样理解有好处。这是土改以来第一次大斗争。开始要训练县以上的干部队伍,再训练大队以上的干部,还有训练生产队干部和贫下中农积极分子。

没有蚂蚁的地区不要硬去找蚂蚁。譬如一类社队过去进行了阶级斗争,进行了社会主义教育,你一定去找地富活动,没有一例外也不好。

人民内部矛盾是大量的,差不多都有,大的小的。河南材料说到一个支部很好,另一材料说的一个公社干部经过洗手洗澡真正一尘不染的只有两个人,不能说这个支部不好,还是百分之九十五以上。现在看来我们干部真正一尘不染是有,但不能说太多。铺张浪费、多吃多占,一点没沾上的少,大多数沾了,洗手洗澡交待就好了吗?这次“四清”“五反”大家都出点汗,洗温水澡,轻松愉快,才能轻装上阵,一致对敌。为什么轻松愉快?一致对敌,我们身上不干净没有力量,搞干净就能团结一致对敌。有的干部多吃多占,有的和地富女儿勾搭,不洗就不能对敌。有些人对敌斗争有劲,对人民内部矛盾就不大积极,有顾虑。

解决人民内部矛盾,多吃多占,上了当的,只要自己说出来,又退了赃,不算贪污分子。将来机关、工厂、企业也可以这样办。当场宣布不算贪污分子,可不公布姓名。东北局有几个贪污一百元、二百元,自己讲了,开大会宣布不算贪污。至于贪污大的案处理了,大约不过上万元,如自己交待了又退了,处理可以减轻。既要有严肃性,又要有政策性,“四清”“五反”一定要有,不反不行。一定要交待清楚,不退赃物赃款不行。但要退得合情合理,多吃多占的退的时候,不要退得太挖苦了。使干部生活过不下去也不好,有的已吃了用了,教育他向群众检讨一下退出若干,参加劳动,这样群众不会叫一次退出,分期分批退,不至使生活不好过。还可以采取自报公议。这个政策很复杂,看起来自己好。

这次运动中间要换一批。劳动好的人,看来是少数。处分的,也是少数。议论干部受处分的可能不到百分之一,不要太多了,要多做教育工作。加强运动的领导,有的时候须要靠各地区县社队广大干部,上面去的人不要包办代替,要把广大干部发动起来,要依靠广大干部去搞。用这种方法——自我教育方法,发动广大干部方法,来的力量大。

一个坚决把运动搞起来,一个怕搞乱了。

(××同志讲:我懂主席心情,第一要搞,第二要搞好。经常向主席反映情况,得到主席指示,不要搞乱了。)

三个伟大革命斗争,不搞好不行,定要搞好。

注意总结经验,回去中央局开十天会,搞一个月工作,到七月中央局再开一次会,总结一下经验,摸一下情况,到七月底八月中央开会,除这个外还要搞工业。

要有强的领导才能发动运动,分期分批,一批批搞不算落后。这次运动将要大大提高各地的自觉性,中央局、省、市、县的人下去一起来运动。

四清历来不清,阶级斗争粗糙,这次运动,要提高自觉性,要忠心诚恳地帮助社队工作搞好,帮助干部洗温水澡,帮助四清搞好,除了个别不行的,烂了,蜕化变质的帮不上去,或太坏了,要派工作队代替他们搞,除此要诚心诚意地帮助他们搞。

你们对干部怎么样?我不清楚。\marginpar{\footnotesize 43}现在看起来对干部要说服教育,特别是用实际证据来说服教育。照理说有,拿出实际证据来说,也有阶级斗争实际证据,昔阳县实际征据,浙江参加劳动,四个好文件是实际证据。检查一下我们是否照理说多,证据说比较少。

你们有机会去一个区搞十几天(我们说:没有),你们下去干部是否很紧张,熟了就不很紧张,多尊重人家,不要指手划脚,“三不”,对干部我们要团结他们,要洗手洗澡,要抓一下。

这次运动会出现杀人灭迹。

发动群众搞四清是厉害的事情,河北经验,有些公安机关搞不清,发动群众四清搞出来了。有人讲阶级斗争靠公安部门搞,人民内部靠监委搞,当然要靠,但除此还要充分的发动群众,依靠群众。

这个运动抓起纲领就好办了,分期分批搞,搞第二、第三批不算不名誉,还是名誉。

(众议:有地方走过场,雨过地皮湿)走了过场再搞嘛,就是不要伤人,不是敌人当敌人,不是……。

(大家议:乱子一点不出不行,主席同意我们的看法。十九日大厦跳楼死人,黑龙江有个地富杀死三十八人,去年枪毙了反革命十三人。)上海死一人,在厕所吊死,留字“过路君子”说好为他申寃,根本没斗他就死了。

要坚持说服教育,分期分批试点,划清界线,团结百分之九十五以上的群众和干部,有强的领导,只要有几条搞好就可少出乱子。

不打无准备之仗,材料没准备,兵没练好,不要搞。这一仗是全国性的革命运动,要像解放战争时来打仗,辽沈战役、锦州、淮海、过长江战役。不要打百团大战,不要像皖南事变那样打法。

第二是解放战争几个战役取得了全国性的胜利,这次打仗打好了是全国革命的胜利,对世界革命贡献更大了。

