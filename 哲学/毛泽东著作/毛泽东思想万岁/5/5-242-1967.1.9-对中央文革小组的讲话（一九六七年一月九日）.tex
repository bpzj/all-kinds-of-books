\section[对中央文革小组的讲话(一九六七年一月九日)]{对中央文革小组的讲话}
\datesubtitle{(一九六七年一月九日)}


《文汇报》现在左派夺了权,四号造了反,《解放日报》六号也造了反,这个方向是好的。《文汇报》夺权后,三期报都看了,选登了红卫兵的文章,有些好文章可以选登。《文汇报》五日《告全市人民书》,《人民日报》可转载,电台可广播。内部造反很好!过几天可以综合报导,这是一个阶级推翻一个阶级,这是一场大革命。许多报纸,依我说封了好,但报还是要出的,问题是由什么人出。《文汇报》、《解放日报》造反好。这两张报一出来,一定会影响华东、全国各省、市。

搞一场革命,定要先造舆论。“六一”《人民日报》夺了权,中央派了工作组,发表了《横扫一切牛鬼蛇种》的社论。我不同意另起炉灶,但要夺权,唐平铸换了吴冷西,开始群众不相信,因为《人民日报》过去骗人,又未发表声明。两个报纸夺权是全国性的问题,要支持他们造反。

我们报纸要转载红卫兵文章,他们写得很好,我们的文章死得很。中宣部可以不要,让那些人住那里吃饭,许多事宣传部、文化部都管不了,你(陈伯达)我管不了,红卫兵一来就管住了。

上海革命力量起来,全国就有希望,它不能不影响华东,以及全国各省市,《告全市人民书》是少有的好文章,讲的是上海市,问题是全国性的。

现在搞革命有些人要这要那,我们搞革命,自一九二〇年起,先搞青年团,后搞共产党,那有经费、印刷厂、自行车?我们搞报纸同工人很熟,一边聊天一边改稿子。我们要与各种人,左、中、右都发生联系。一个单位统统搞得那样干净,我向来不赞成。(有人反映吴冷西很舒服,胖了。)太让吴冷西他们舒服了,不主张让他们都罢官,留在岗位上让群众监督。

我们开始搞革命,接触的是机会主义,不是马列主义。青年时《共产党宣言》也未看过。要抓革命,促生产,不能脱离生产搞革命,保守派不搞生产。这是一场阶级斗争。你们不要相信“死了张屠夫,就吃混毛猪”。以为是没有他们不行,不要相信那一套!


