\section{新民学会会员通信集第三集}
\datesubtitle{新民学会致各会友的信}



各会友均鉴:

投寄通信集的信稿,请寄长沙文化书社转交为荷。
<p><font face="宋体"> 
新民学会启
\subsection{新民学会紧要启事}

本会同人结合。以互助互勉为鹄,自七年夏初成立,至今将及三年,虽形式未周,而精神一贯。惟会友个人对于会之精神,间或未能了解。有牵于他种事势不能分其注意力于本会者;有在他种团体感情甚洽因而对于本会无感情者;有缺乏团体生活之兴趣者;有行为不为会友之多数满意者;本会对于有上述形情之人,认为虽曾列为会友,实无互助互勉之可能。为保持会的精神起见,惟有不再认其为会员。并希望以后介绍新会员入会,务求无上列情形者。本会前途幸甚。


新民学会启

一千九百二十一年一月二日
\subsection{“新民学会会员通信集”第三集重点及付印出版日期}

这一集以讨论“共产主义”和“会务”为两个重要点。信的封数不多,而颇有精义。千九百二十一年一月上旬付印,下旬出版。
\subsection{给肖旭东、蔡林彬并在法诸会友的信(一九二○年十二月一日)}

※赞成“改造中国与世界”为学会方针。

※赞成马克思式的革命。

※学会的态度:(一)互助互勉。(二)诚恳(三)光明。(四)向上。

※共同研究及分门研究。

※学会的四种运动。

※联络同志之重要。
<p><font face="宋体">和森兄子升兄并转在法诸会友:

接到二兄各函,欣慰无量!学会有具体的计划,算从蒙达尔尼会议及二兄这几封信始。弟于学会前途,抱有极大希望,因之也略有一点计划,久思草具计划书提出于会友之前,以资商榷,今得二兄各信,我的计划书可以不作了。我只希望我们七十几个会友,对于二兄信上的计划,人人下一个祥密的考虑。随而下一个深切的批评,以决定或赞成,或反对,或于二兄信上所有计划和意见之外,再有别的计划和意见。我常觉得我们个人的发展或学会的发展,总要有一条明确的路数,没有一条明确的路数,各个人只是盲进,结果糟踏了各人自己外之,又糟踏了这个有希望的学会,岂不可惜?原来我们在没有这个学会之先,也就有一些计划,这个学会之所以成立,就是两年前一些人互相讨论研究的结果。学会建立以后,顿成功了一种共同的意识,于个人思想的改造,生活的向上,很有影响。同对于共同生活,共同进取,也颇有研究。但因为没有提出具体方案;又没有出版物可作公共讨论的机关;并且两年来会友分赴各方;在长沙的会员又因为政治上的障碍不能聚会讨论,所以虽然有些计划和意见,依然只藏之于各人的心里,或几人相会出之于各人的口里,或彼此通函见之于各人之信里;总之只存于一部分的会友间而己。现在诸君既有蒙达尔尼的大集会,商决了一个共同的主张;二兄又本乎自己的理想和观察,发表了个人的意见;我们不在法国的会员,对于诸君所提出当然要有一种研究,批评,和决定。除开在长沙方面会员,即将开会为共同的研究,批评,和决定外,先达我个人对于二兄来信的意见如左。

现在分条说来:

(一)学会方针问题。我们学会到底拿一种什么方针做我们共同的目标呢?子升信里述蒙达尔尼会议,对于学会进行之方针,说:“大家决定会务进行之方针在改造中国与世界”。以“改造中国与世界”为学会方针,正与我平日的主张相合,并且我料到是与多数会友的主张相合的。以我们接洽和视察,我们多数的会友,都顷向于世界主义,试看多数人鄙弃爱国,多数人鄙弃谋一部分一国家的私利,而忘却人类全体的幸福的事,多数人都觉得自己是人类的一员,而不愿意更繁复的隶属于无意义之某一国家,某一家庭,或某一宗教,而为其奴隶;就可以知道。这种世界主义,就是四海同胞主义,就是愿意自己好也愿意别人好的主义,也就是所谓社会主义。凡是社会主义,都是国际的,都是不应该带有爱国的色彩的。和森在八月十三日的信里说:“我将拟一种明确的提议书,注重无产阶级专政与国际色彩两点。因我所见高明一点的青年,多带一点中产阶级的眼光和国际的色彩,于此两点,非严正主张不可”。除无产阶级专政一点置于下条讨论外,国际色彩一点,现在确有将他郑重标揭出来的必要。虽然我们生在中国地方的人。为做事便利起见,又因为中国比较世界各地为更幼稚更腐败应先从着手改造起见。当然应在中国这一块地方做事;但是感情总要是普遍的,不要只爱这一块地方而不爱别的地方。这是一层,做事又并不限定在中国,我以为固应该有人在中国做事,更应该有人在世界做事,如帮助俄国完成他的社会革命;帮助朝鲜独立;帮助南洋独立;帮助蒙古、新疆、西藏、青海、自治自决;都是很要紧的。

以下说方法问题。

(二)方法问题。目的一一改造中国与世界一一定好了,接着发生的是方法问题。我们到底用什么方法去达到“改造中国与世界”的目的呢?和森信里说:“我现认清社会主义为资本主义的反映其重要使命在打破资本经济制度,其方法在无产阶级专政”。和森又说:“我以为现世界不能实行无政府主义,因在现世界显然有两个对立的阶级存在。打倒有产阶级的迪克推多,非以无产阶级的迪克推多压不住反动,俄国就是个明证。所以我对于中国将来的改造,以为完全适用社会主义的原理与方法。……我以后先要组织共产党,因为他是革命运动的发动者,宣传者,先锋队,作战部”。据和森的意见,以为应用俄国式的方位去达到改造中国与世界,是赞成马克思的方法的。而子升则说:“世界进化是无穷期的,革命也是又穷期,我们不认可以一部分的牺牲,换多数人的福利。主张温和的革命。以教育为工具的革命,为人民谋全体福利的革命。以工会合社为实行改造之方法。颇不认俄式――马克思式一一革命为正当,而倾向无政府――蒲鲁东式一-之新式革命,比较和而缓。虽缓然和,同时李和笙兄来信,主张与子升相同。李说:“社会改进,我不赞成笼统的改造,用分工协助的方法,从社会内面改造出来,我觉得很好。一个社会的病,自有他的特别的背影,一剂学方可以医天下人的病,我很怀疑。俄国式的革命,我根本上有未敢赞同之处”。我对子升和笙两人的意见。(用平和的手段,谋全体的幸福)在真理上是赞成的,但在事实上认为做不到。罗素在长沙演说,意与子升及和笙同,主张共产主义,但反对劳农专政,谓宜用教育的方法使有产阶级觉悟,可不至要妨碍自由,兴起战争,革命流血。但我于罗素讲演后,曾和殷柏,礼容等有极详细之辩论。我对于罗素的主张,有两句评语:就是:“理论上说得通,事实上做不到”。罗素和子升和笙主张的要点,是“用教育的方法”但教育一要有钱,二要有人,三要有机关。现在世界,钱尽在资本家的手,主持教育的人尽是一些鸯本家,或资本家的奴隶。总言之,现在世界的学校及报馆两种最主要的教育机关,又尽在资本家的掌握中,现在世界的教育,是一种资本主义的教育。以资本主义教育儿童,这些儿童大了又转而用资本主义教第二代的儿童。教育所以落在资本家手里,则因为资本家有“议会”以制定保护资本家并防制无产阶级的法律,有“政府”执行这些法律,以积极的实行其所保护与所禁止。有“军队”与“警察”,以消极的保障资本家的安乐与禁止无产者的要求。有“银行”以为其财货流通的府库。有工厂以为其生产品垄断的机关。如此,共产党人非取政权,且不能安息于其守下,更要能握得其教育权;如此,资本家久握教育权,大鼓吹其资本主义,使共产党人的共产主义宣传,信者日见其微。所以我觉得教育的方法是不行的。我看俄国式的革命,是无可如何的山穷水尽诸路皆走不通了的一个变计。并不是有更好的方法弃而不釆,单要釆这个恐怖的方法。以上是第一层理由。第二层,依心理上习惯性的原理,及人类历史上的观察,觉得要资本家信共产主义,是不可能的事。人生有一种习惯性,是心理上的一种力,正与物在斜方必倾向下之物理上的一,种力一样。要物下倾向下,依力学原理,要有与他相等的一力去抵抗他才行。要人心改变,也要有一种与这心力强度相等的力去反抗他才行。用教育之力去改变他,既不能拿到学校与报馆两种教育机关的全部或一大部到手,或有口舌印刷物或一二学校报馆为宣传之具。正如朱子所谓“教学如扶醉人,扶得东来西又倒”,“直不足以动资本主义者心理的毫未,那有同心向善之望?以上从心理上说。再从历史上说,人类生活全是一种现实欲望的扩张。这种现实欲望,只向扩张的方面走,决不向减缩的方面走。小资本家必想做大资本家,大资本家必想做最大的资本家,是一定的心理。历史上凡是专制主义者或帝国主义者,或军阀主义者,非等到人家来推倒,决没有自己肯收场的。有拿破伦第一称帝失败了,又有拿破伦第三称帝。有袁世凯失败了,偏又有段祺瑞。章太炎在长沙演说,劝大家读历史,谓袁段等失败均为不读历史之故。我谓读历史是智慧的事,求逐所欲是冲动的事,智慧指导冲动,只能于相当范围有效力,一出范围,冲动使将智慧压倒,勇敢前进,必要回到了比冲动前进之力更大的力,然后才可以将他打回。有几句俗话:“人不到黄河心不死”,“这山望见那山高”,“人心不知足,得陇又望蜀”,均可以证明这个道理。以上从心理上及历史上看.可见资本主义是不能以些小教育之力推翻的,是第二层的理由。再说第三层理由,理想固要紧,现实尤其要紧,用和平方法去达到共产目的,要向何日才能成功?假如要一百年,这一百年中宛转呻吟的无产阶级,我们对之,如何处置,(就是我们)。无产阶级比有产阶级实在要多得若干倍,假定无产者占三分之二,则十五万万人类中有十万万无产者(恐怕还不只此数)这一百年中,任其为三分之一之资本家鱼肉,其何能忍?且无产者既已觉悟到自己应该有产,而现在受无产的痛苦是不应该;因无产的不安,而发生共产的要求;已经成了一种事实。事实是当前的,是不能消灭的,是知了就要行的。因此我觉得俄国的革命,和各国急进派共产党人数日见其多,组织日见其密,只是自然的结果.以上是第三层理由。再有一层,是我对于无政府主义的怀疑。我的理由却不仅无强权无组织的社会状态之不可能。我只忧一到这种社会状态实现了之难以终其局。

因为这种社会状态是定要造成人类死率减少而生率加多的,其结局必至于人满为患。如果不能做到(一)不吃饭;(二)不穿衣;(三)不住屋;(四)地球上各处气候寒暖,和土地肥瘦均一;或是(五)更发明无量可以住人的新地,是终于免不掉人满为患一个难关的。因上各层理由,所以我对于绝对的自由主义,无政府主义,及德谟克西拉的主义,依我现在的看法,部只认为于理论上说得好听,事实上是做不到的。因此我于子升和笙二兄的主张,不表同意。而于和森的主张,表示深切的赞同。

(三)态度问题。分学会的态度与会友的态度两种:学会的态度,我以为第一是“潜在”,这在上海半松园曾讨论过,今又为在法会友所赞成,总要算可以确定了。第二是“不依赖旧势力”,我们运学会是新的,是创造的,决不宜许旧势力混入,这一点要请大家注意。至于会友相互及会友个人的态度,我以为第一是“互助互勉”,(互助如急难上的互助,学问上的互助,事业上的互助。互勉如积极的勉为善,消极的勉去恶)。第二是诚恳(不滑)。第三是光明(人格的光明)。第四是向上(能变化气质有向上心)第一是“相互间”应该具有的,第二第三第四是“个人”应该具有的。以上学会的态度二项,会友的态度四项,是会友精神所寄,非常重要。

(四)求学问题。极端赞成诸君共同研究及分门研究之两法。诸君感于散处不便,谋合居一处,一面作工,一面有集会机缘,时常可以开共同的研究会,极善。长沙方面会友本在一起,诸君办法此间必要仿行。至分门研究之法,以主义为纲,以书报为目,分别阅读,互相交换,办法最好没有。我意凡有会员两人之处,即应照此组织。子升举力学之必要,谓我们常识尚不充足,我们同志中一尚无专门研究学术者,中国现在尚无可数的学者,诚哉不错!思想进步是生活事业进步之基。使思想进步的唯一方法,是研究学术。弟为荒学,甚为不安,以后必要照诸君的办法,发奋求学。

(五)会务进行问题。此节子升及和森意见最多。子升之“学会我见”十八项,弟皆赞成。其中“根本计划”之“确定会务进行方针”,“准备人才”,“准备经济”三条尤有卓见。以在民国二十五年前为纯粹预备时期,我以为尚要延长五年,以至民国三十年为纯粹预备期。子升所列以长沙方面诸条,以“综挈会务大纲,稳立基础”,“筹办小学”,“物色基本会员”三项,为最要紧,此外尚应加入“创立有价值之新事业数种”一项,子升所列之海外部,以法国、、俄国、南洋三方而为最重。弟意学会的运动,暂时可总括为四:1、湖南运动;2、南洋运动;3`留法运动;4留俄运动。暂时不必务广,以发展此四种,而使之确见成效为鹄。较为明切有着,诸君以何如?甚和森要我进行之“小学教育”,“劳动教育”,“合作运动”,“小册子”,“亲属聚居”,“帮助各团体”讲端,我都愿意进行。惟“贴邮花”一项我不懂意,请再见示。现在文化书社成立,基础可望稳固,营业亦可望发展。现有每县设一分社的计划,拟两年内办成,果办成,效自不小。

(六)同志联络问题。这项极为要紧,我以为我们七十几个会员,要以至诚恳切的心,分在各方面随时联络各人接近的同志,以携手共上于世界改造的道路。不分男、女、老、少、士、农、王、商、只要他心意诚恳,人格光明,思想向上,能得互助互勉之益,无不可与之联络,结为同心。此节和森信中详言,子升亦有提及。我觉得创造特别环境,改造中国与世界的大业,断不是少数人可以包办的,希望我们七十几个人,人人注意及此。

我的意见大略说完了。闻子升已回国到北京,不久可以面谈。请在法诸友:阵将我的意见加以批评,以期求得一个共同的决定。个人幸甚,学会幸甚。

\begin{flushright}
弟毛泽东

九年十二月一日
    
文化书社夜十二时\end{flushright}
\subsection{给和森的信(一九二一年一月二十一日)}
<p><font face="宋体">和森兄:

来信于年底始由子升转到。唯物史观是吾党哲学的根据,这是事实。不象唯理观之不能证实而容易被人摇动。我固无研究,但我现在不承认无政府的原理是可以证实的原理,有很强固的理由。一个工厂的政治组织(工厂分配管理等)与一个国的政治组织,与世界的政治组织,只有大小不同,没有性质不同。工团主义以国的政治组织与工厂的政治组织异性,认为另一回事而举以属之另一种人,不是因为曲说以冀苟且偷安,就是愚陋不明事理之正。况乎尚有非得政权则不能发动革命不能保护革命不能完成革命在手段上又有十分必要的理由呢。你这一封信见地极当,我没有一个字不赞成。党一层陈仲甫先生等已在进行组织,出版物一层上海出的“共产党”,你处谅可得到,颇不愧“旗帜鲜明”四字,(宣言即仲甫所为)。详情后报。

 弟泽东

 十年一月廿一日在城南
