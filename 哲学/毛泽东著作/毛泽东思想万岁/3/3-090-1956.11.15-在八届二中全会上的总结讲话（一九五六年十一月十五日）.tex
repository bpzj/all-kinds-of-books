\section[在八届二中全会上的总结讲话(一九五六年十一月十五日)]{在八届二中全会上的总结讲话}
\datesubtitle{(一九五六年十一月十五日)}


一、经济问题

经济建设不是一点没有进退地、四平八稳的前进着,建设也是有时多一点,有时少一点,有时马儿跑得快一点,有时马儿跑得慢一点,有时上马,有时下马,这种情况完全是可能的。因为,第一是我们没有经验,第二是我们的经济建设要以形势为转移。比如过去经济建设可能搞快了一些,那是因为当时存在战争形势,如果马上要打的话,重工业搞多一点,就有必要。经济建设的进行是波浪式的,有起伏的,一波逐一波,就是说,有平衡、破裂。而破裂之后又恢复平衡。当然,波浪式的起伏也不能太大,不要一下子又反保守。可是循着波浪式的规律发展前进则是必然的。如果承认了这点,那么今年冒进一点就不是了不得,明年搞少些也就没什么,总的来讲,我们的第一个五年计划是正确的。

165项是决定中国工农业发展的重点工程,这些不能不搞,但少许搞快了一点。第一个五年计划执行至今已经四年了,总的情况是好的,生产前进了,工人收入有所增加,人民生活改善了一些,虽然社会上紧张一些,但也还过得去,基本建设成绩也还是好的,建立了许多工厂,有很多过去不能自己生产的产品、现在已经能够自己搞了,像过去那样什么都要依靠外国的落后情况,已经开始有了改变。在工业方面已经打下了一定的基础。所以应该确定,第一个五年计划是正确的。当然,错误也是有的,这主要是因为经验不足,而且今后也可能会有错误。因此,不能说我们总结这四年经济建设的经验,就能够完全掌握了经济建设的规律,一点错误都不犯了,那是不可能的。

第一个五年计划的中心项目是165个,而限额以上的项目一共是八百个,其中大多数是自己搞的,只有一百五十多个是苏联帮助设计的,其它有些是东欧兄弟国家帮助的,主要还是自己设计的,谁说我们不行呢?虽然我们自己搞的数量不少,但大的工程是苏联帮助设计的,小的才是自己搞的;如果大的都能由自己来搞,就行了。看来,如果第一个五年计划有缺点错误的话,那么最大的缺点错误,就是没有经验。建设工厂只有骨没有肉,也就是说,把厂房、机器设备等搞起来,而对于附带的设备——下水道、马路、邮电、茶馆、酒楼、宿舍、剧院等等都没有相应地建设起来。实际上,真正要建设一个工业城市,不把这些包括进去是不成的。西北现在有很多困难,搞得很紧张,这是因为城市没有基础,工厂搞起来了,没有理发的、缝衣服的,马路没有,汽车不够,物资运不进去。因此搞工业,只是设计工厂本身,那是不够的,必须把工厂的许多附带的服务性的设备和福利设施一并计划进去,而过去我们没有列入计划。我们的八百个项目本身的投资经费是够用了的,可是一旦发展下去,所需要的经费势必会增多,而这些我们又没有计划进去,这将是一个很大的漏洞。今年问题并不大,而问题大的却在今后,如这次不发现建设工厂中骨和肉的关系问题,只是搞厂房、机器设备,不搞附属设备,工厂是建立起来了,可是问题会有很多。尽管我们在建设上,我们犯有主观主义错误,然而错误不犯一点也是不行的,因为没有经验。犯了错误,也就可以知道哪些可以做,哪些不可以做,从而得到教训,主要是把错误改了就行了,不要泼冷水,要保护干部和群众的积极性,应当在这个前提下面,来批判工作中的缺点。对于群众要求而暂时办不到的事,一定要公开告诉群众,反复地向群众解释清楚。

以后搞预算,每年要搞两次,中央全会和一次全国人民代表大会来讨论,三榜定案。因为预算是个框框,经济建设速度的快慢,主要决定于预算。×××同志在一次谈话中讲得很好,他说:现在到底哪一年的计划是冒进不冒进,就看这年财政收入多少,支出多少。今后要使建设速度能够保持经常稳妥,搞好预算是个很重要的问题,所以中央把它提到二中全会上来讨论,是有道理的,是要大家关心这个问题。讨论一次还不够,还要再开一次中央全会和一次全国人民代表大会来讨论,这样做的结果,就会使得我们的预算搞得比较准确,同时也可以使得我们的经济建设放在充分可靠的基础上。

省委、市委及中央各部的负责同志,一定要抓财政经济和计划,而现在没有抓的居多,要迅速改正这个缺点。各级党委在改革已经完成之后,要搞建设,而党委如果不抓财政经济和计划,建设如何能搞得好呢?党委是外行就没有办法领导,要想成为内行,就要深入进去,抓起来。中央现在已转向抓财政经济和计划工作。

粮食、猪肉这个问题很大。今年的粮食,虽然全国还是增产的。但灾区面积很大,由于对粮食销售没抓紧,半年来多销了很多,浪费也很大,大家要好好把粮食问题研究一下,看看怎样才能保证供应,又要不发生问题,不要浪费,不要多销。猪肉和付食品供应都很紧张。这些都不是很简单的问题,特别是谈到农产品价格问题,邓子恢同志在福建发了一个电报,他主要是讲农产品收购价格低了,尤其是粮价,所以农民不愿生产粮食,他主张提高粮价,同时,中央主张对生猪的收购价格也提高一些。现在解决粮食问题,一方面是节约,抓紧不要多销,另一方面,还要使农民愿意多生产粮食,这对于解决副食品供应,特别是猪肉,有密切的关系。因为农民不愿生产粮食,没有饲料,猪也就养不起来,如果粮食多了,再把生猪价格提高一点,发展生猪也就比较可靠一点。总之,农产品价格要定得恰当。有些同志希望把工农业产品的剪刀差价赶快搞平,这是不可能的。因为现在的剪刀差的情况,是以国民收入为一百,剪刀差价占百分之三十,而农民直接的税收负担,全国平均不过百分之十左右,如果现在要求完全消灭剪刀差价做到等价交换,国家积累就会受影响。但是剪刀差价太大,使得农民无利可图,那也是错误的。总之,在不影响国家积累的情况下,逐步地缩小工农业品的剪刀差价,以提高农民的生活,是完全必要的。

不管怎样,我们的经济建设,必须要提倡“勤俭建国,艰苦奋斗”。我们用一角钱,不能只做八分钱的事,而要做一角一分、一角二分钱的事。在困难的时候,现在我们有些同志就想着要国家拿钱来解决困难。当然,这也是必要的;但是,应该还有一个办法,就是在困难的时候,不依靠钱,靠我们的模范作用和艰苦奋斗,也可以解决问题。为什么这样说呢?据说兰州建设工厂,办公大楼首先就建立起来,可是没有工人宿舍,现在工人只得睡在棚子里。

搞经济建设,还要靠报纸,要使报纸在经济建设中,在整个工作中起很好的作用。要抓好报纸,有什么困难可以向群众讲清楚,提高群众觉悟,大家眼前都苦一些来搞建设,建设好了以后生活就可以好一些。如果只是提高生活,而不把群众觉悟提高,问题还会发生。如:匈牙利的人民生活提高不多,但也不是很坏的。可是,由于没有教育,没有提高群众觉悟水平,小资产阶级、资产阶级的思想影响很大,以致出的乱子很大。

此外,今后应该多强调与人民同甘共苦,首先是宣传领导机关干部要起模范作用,然后才能要求群众艰苦奋斗。建设社会主义是个艰巨的任务,只有艰苦奋斗才能建成社会主义。所以,必须要强调生产,生产提高,人民生活才有可能逐步提高,强调领导机关干部的艰苦奋斗,发扬过去艰苦朴素的作风。同时号召人民与我们同甘共苦。

二、国际形势问题

目前的国际形势,发生问题的地方是中东和东欧,看来都不是好消息,但也可以说是好消息。因为波兰、匈牙利事件的发生,就表明有问题,有矛盾存在,既然有问题就要暴露出来,而事件爆发,使得问题明朗化,清楚了,这样就可以找出原因,采取措施,改进工作中的缺点和错误,把问题能够解决好,使波兰、匈牙利能够搞得更巩固些。如果从这方面看,又是好消息。所以有问题,终究都要爆发,迟爆发不如早爆发,早爆发的问题可能小一点,容易解决些。“失败是成功之母”,失败之后,有好的一面就是可以接受经验教训,如果确实在失败中取得了教训,也就有所收获,这就是好的。如波兰、匈牙利事件,教育了波兰、匈牙利的革命人民,也教育了苏联,同时也教育了我们。问题是会不断有的,即使到了共产主义社会,问题也还是不断发生的。“革命”这两个字,在字典上大概不会取消的,什么叫革命呢?就是说,有矛盾就会有斗争,解决矛盾的办法就叫革命。现在的革命,就是指消灭阶级;将来的革命,内容和性质可能会有些不同,但是,有矛盾就有斗争,要解决矛盾就要斗争,这在将来同现在和过去的意义上,还是一致的。将来阶级消灭之后,矛盾的性质可能同过去有所不同,但是矛盾依然存在。如今天的中国阶级矛盾已经基本解决,当前国内主要矛盾是先进的社会制度同落后生产力之间的矛盾,这个矛盾一般来讲,本来它不是对抗性的,可是如果解决得不好,就可能变成对抗性的矛盾。

三、中苏关系问题

目前,社会主义阵营中,苏联的领导地位是不能动摇的。如果动摇了苏联的领导地位,会使得整个社会主义阵营陷入更混乱的状态,我们拥护苏联不是因为别的,主要大家都是共产主义者,反对帝国主义、资本主义,既然我们彼此有着共同的目的。苏联能发挥更大的作用,我们为什么不拥护他呢?当然,拥护苏联也并不是对所有的东西都加以盲目的拥护。事实上,我们一向都不是盲目拥护的,有些东西不适合中国情况,我们是不采用的,我们中国革命和工作的方针、政策、路线,都是我们党依照中国的特点、情况,独立确定和制订的。应该说,苏联在社会主义阵营中居于领导地位,也是恰当的,因为苏联是世界上第一个建成社会主义的国家,工业也是有基础的。苏联还是有不少先进经验。虽然苏联过去也有些缺点,有些做法不够妥善,但是,这些都是可以改正的,世界上没有十全十美的东西,而我们有些同志过去就是片面地把苏联看成十全十美,现在调过另一方面又看成什么都不好了,这是片面性的表现。

马列主义者看问题不能片面,事物有真美善与假丑恶之分,而又是相互对立的。对苏联的看法,过去看不能什么都好,今天也不能认为什么都不好。实际上,近来苏联在许多工作中和国际关系方面,都作了不少努力,也有了很大的改进,而且是向好的方向走。这说明苏联基本上是好的,过去是好的,现在是好的,将来会是更好的,当然,缺点也是有的,有些问题处理方法不够恰当,在理论方面也有些错误。

我们党一开始就学习苏联。群众路线、政治工作、无产阶级专政等,都是由十月革命学来的。列宁在当时是注重发动群众、组识工农兵苏维埃等等,不是靠行政命令。列宁派党代表进行政治工作。问题就是在十月革命以后斯大林的后期,虽然还是在搞社会主义、共产主义,可是他把列宁的一些东西抛弃了,脱离了列宁主义的轨道,脱离了群众等等。因此,学习斯大林领导后期的东西,教条主义地往中国搬用,我们是吃了一些亏的。现在苏联还有一些先进经验,也是可以学习的,然而,有些东西是不能跟苏联一样的,如:对资本主义工商业的社会主义改造,农业合作化,经济建设的十大关系,都是中国的做法,今后的社会主义经济建设,主要应该从中国的情况、处境和所处的时代等特点出发,所以学习苏联的口号,还是要提,只是不能盲目、教条地生硬搬用。同样,我们对资产阶级国家的某些好的东西,也可以学习。因为任何一个国家都有其长处和短处,而我们主要是学习人家的长处。

斯大林有脱离马克思列宁主义的倾向,具体表现在否定矛盾,至今还没有彻底消除斯大林这种观点的影响。斯大林是唯物主义。辩证法的,但是,实际上是主观主义的,他把个人摆得高于一切,否认集团,否认群众,个人崇拜,更确切地说,是个人独裁,这就是反唯物主义;斯大林也讲辩证法,实际上是行而上学,如联共党史中讲辩证法,把矛盾放在最后。应该说辩证法的最基本的问题,就是矛盾对立的统一。由于他是形而上学,才产生片面的观点,否认事物的内在联系,孤立静止地看问题。讲辩证法则是对立的统一的看待问题,所以是全面的。生同死、战争与和平,是对立的矛盾的,实际上它们之间又有内在的联系,所以,有时这种对立又能统一。我们认识问题不能只看到一面,应该全面分析,透视它的本质。这样,对认识一个人来说,就不能一下子什么都好,一下子又坏得不得了,连一点好处都没有。为什么我们党是正确的呢?就是因为我们认识和解决一切问题,都是从客观情况出发,这样就比较全面而不绝对化。其次,群众路线,被斯大林看成尾巴主义,没有认识群众路线的好处,许多问题都用行政办法来解决,而我们共产党人是唯物主义者,承认群众是创造一切的、历史的主人,没有什么个人英雄,只有群众团结起来才有力量。实际上,列宁逝世后,在苏联却把群众路线忘却了。在反斯大林的同时,也并没有很好的认识和强调群众路线的意义,当然,最近已经开始注意了,但认识的还不深刻。再有,阶级斗争和无产阶级专政是为列宁所强调的。当时列宁第三国际同第二国际的分歧,主要就是马克思主义者强调阶级斗争和无产阶级专政,而机会主义者则不肯承认。波兰、匈牙利事件发生的教训之一,除了工作中有缺点等之外,就是他们在革命胜利之后,没有很好地发动群众,彻底肃清反革命分手,也就是说,没有认真地对反革命分子实行专政,以致使反革命分子乘隙捣乱。还有就是国与国之间的关系上,我们一向主张应该是平等相待,国有大小,各有所长,各兄弟国家有着共同目标,所以一定要团结,平等相待,不能有任何优越感。我们的党自从遵义会议以后,是独立地执行政策,独立地决定方针,保持独立的地位,按照中国情况办事。建国以后,我们提出“一边倒”,是意味着在政治上必须社会主义阵营的国家团结起来,社会主义国家站在一边,不能一脚跨在社会主义方面,一脚又跨在西方资本主义方面。所以“一边倒”就表明我们在政治上同资产阶级国家断然分开,打消一切对西方国家的幻想,中国不可能走资本主义道路,不可能同帝国主义妥协,而是要坚决反对帝国主义,反对资本主义。

中苏关系必须搞好,对于苏联为首的领导地位,不能发生动摇,否则,对我们社会主义阵营是很不利的。应该肯定,苏联总的方面是好的,在社会主义阵营中必须以苏联为首,对于这一点必须向党内外加强教育,很好地讲清楚。

四、大民主与小民主问题

近年来,党内和党外都有人要求实行大民主。其实,要求在人民内部实行大民主的提法是错误的。究竟什么叫大民主呢?大民主是来用对付敌人的,从我们的历史上看,陈胜、吴广揭杆起兵以抗秦,王莽废子婴而篡汉,东汉灵帝时黄巾起义,汉晋之际曹操、刘备、孙权分争天下,清朝的洪秀全、杨秀清等搞金田起义……等等,是大民主的作法;我们搞革命推翻蒋介石的做法,也是大民主的作法,这次举行反对英法侵略埃及的示威游行,也是大民主。应该说我们是赞成搞大民主的,但是必须确定:大民主是用作对付敌人,反对统治阶级的,在人民内部是不能用大民主的。同样,帝国主义、阶级敌人对付我们也是用大民主的手段,这次匈牙利事件就表明了,帝国主义和反动分手是用大民主来对付共产党和人民的。可是,我们有些同志,他们不懂得这个道理,其实,我们对付敌人一向是采取大民主,只是在今天国内阶级矛盾已经基本解决的情况下,我们对付资产阶级不是采取大民主,也不希望采取大民主,而是用整风的办法,微风细雨般地进行思想教育的办法。但是,在阶级矛盾没有解决之前,我们对于敌对阶级向来是采取大民主的。

当然,大民主也可以适用于对付官僚主义分子,因为存在着严重的官僚主义,脱离群众,工作中造成很大的损失、恶果,到了群众忍无可忍的时候,他们为什么不可以用大民主来反对官僚主义分子?成都有些学生罢课,因为问题没有解决,就要到北京请愿,两路进军,一部列车到了宝鸡,一部列车到了郑州,第二机械工业部知道这个消息后,就派人前去把这些学生劝阻回去了。应该让他们到北京来,你们事情做错了,又不肯承认错误,人家到北京请愿,还拦人家。你们没办法说服他们,我们向他们承认错误。不叫他们来,是不是怕影响不好?或者是怕增加我们的麻烦。其实,这没有什么可怕的,因为来与不来,影响就是那样,来也不会增加,不来也不会改变,真的来了,就是再忙也要接见他们,并且向他们承认错误。

群众有道理的请愿是完全应该允许的,将来可以考虑在宪法中加上一条:允许工人罢工。这样对于改正工厂领导上的官僚主义作风是有好处的,因为工人一罢工,就使厂长领导紧张,就不敢光坐在厂长办公室里指挥了,而要到工人群众中去实地看一看,如果我们不警惕,不改正官僚主义作风,结果就可能形成一个特殊的阶层,脱离人民,甚至最终为人民所打倒。

大民主,我们是赞成的,如果现在用它来解决问题,恐怕民主人士、大学教授、资本家和党内的一些人(如官僚主义分子)却不赞成,因为一实行大民主,群众就会起来清算和“整”他们,经过历次政治运动,这些人已经整得够呛,再来个大民主,又得挨整,可以想象他们是不会欢迎的。其实,虽然大民主是对付敌人的,也可以对付官僚主义分子,但是,我们也不必害怕它,问题在于只要能够把工作做好,认真克服缺点,改正错误,群众只会拥护,也就不会搞什么大民主来反对了。

五、少数民族问题

对于少数民族,我们一定要团结好,不能把这个问题看得太简单,过去几年来,对于少数民族虽然也做了一些工作,解决了一些问题,但是现在的问题仍然不少。同时,民族问题将会在很长时期里存在着。看来,只要我们把工作做好,它的问题就可能少一些。这次波兰与苏联,匈牙利与苏联相互关系上的一些问题,也都包括有些民族问题没有很好地解决。在中国,西藏地区的民族问题也没有完全解决。因此中央几次都提出要重视这方面的问题,不要以为团结少数民族是很容易的事。一是反对大汉族主义,切实把少数民族团结好。

我们对于民族工作,过去存在一些缺点,对改正要有信心,不要怕人家反对,因为有缺点,人家才反,缺点没有了,也就不会再反了。中国共产党和其他国家党的情况有所不同,我们党的干部是土生土长的,这些干部有北伐时代的,有土地革命时期的,有抗日战争和解放战争时期的,也有建设时期的。而且都是在斗争中逐步成长起来的;不仅如此,我们党的领导也是完全正确的,我们的军队是忠实可靠的,因此应该有信心,不要害怕造反,要做到“任从风浪起,稳坐钓鱼船”,不要悲观,不要怕大民主,但是必须很好地改正我们的错误,坚决克服官僚主义,否则,将是危险的。

六、整风和整编问题

我们要进行一次大的整风运动,主要是整主观主义、官僚主义、宗派主义,贪污浪费以及下边干部的强迫命令作风,明年一月,中央对这个问题要发个指示,确定整风的方法和内容,动员全党作好准备,7月以后再开始进行。在7月以前,主要是作准备,作个人检查,如果在整风之前已经改正了错误,那就不要再整了;对于还没有改正的,属于一般的问题,还是采取教育的方法;只有对于那些错误比较严重而没有改正的人才做重点的批评。整风的目的是为了改正错误,不整就改了,不是更好吗?中央的意思就是在这次整风中,一定要搞彻底,不是刮一阵风,不是搞突击式的运动,不要来个突然袭击,不要损伤元气,也就是说,不是采取暴风骤雨的大民主的做法,而是用微风细雨的小民主的做法,事先充分做好思想准备,以达到提高干部思想觉悟水平、改正工作中的缺点和错误的目的。

过去整风,有些大民主的做法,特别是时间有限,搞得简单粗糙一些,以致伤害了一些好人,这次有充分的时间,一定要做到治病救人,不仅对我们自己的干部如此,就是对一些资产阶级和小资产阶级分子的改造也应当如此,都是用小民主的办法,耐心进行说服教育,提高认识,改正错误,改进工作。应该做到:既有保护,也有斗争,这样才能团结一切可以团结的力量来为社会主义事业服务。

在和平时期,是赞成减少一些军队的,要很好地进行整编。同时,地方也要大大紧缩编制。大家都要做思想准备。现在有一种很不好的风气,一讲工作,开口就先要编制、要人,并且认为编制越大,人越多,工作才有办法,好像不这样就不能进行工作。实际上,并不如此,往往编制愈大,人愈多,就会增加我们的官僚主义,现在有些同志做工作,很少考虑如何提高工作效率,发挥群众的积极性和主动精神,挖掘各方面的潜在力量,相反地是:动不动就是得多少编制,多少人。所以这次中央下决心,要彻底改变这种情况,军队要大大精简,地方的编制也要大大压缩。

同时,还要反对领导干部的特殊化。现在高级干部拿的薪金和人民生活水平相比,悬殊是太大了,将来可以考虑,也减少一些薪金,并且取消特殊待遇,和群众打成一片,艰苦奋斗,不能脱离人民。看来,值得考虑的特殊待遇,一是物质供应,一是警卫太多,必须很好注意,加以改变,以身作则,才能号召人民艰苦奋斗,同甘共苦。

党内上涨的享受风气,这是由于资产阶级和小资产阶级思想的侵蚀和影响,必须加强思想教育,来批判和克服这种不好的风气。(完)


