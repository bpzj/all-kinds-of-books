\documentclass[b5paper,oneside,12pt]{ctexbook}
\usepackage[hmargin=0.3in,vmargin=0.5in]{geometry} 
\usepackage[]{multirow}
\usepackage[perpage,hang]{footmisc} %脚注

\pagestyle{plain} %整书页眉页脚设置
\setlength{\marginparsep}{2pt}
\setlength{\marginparwidth}{20pt}

\ctexset{chapter/numbering=false}
\ctexset{
    section={numbering=false, afterskip = 0ex},
    subsection={format=\large\heiti\centering,numbering=false,beforeskip=1ex,afterskip = 1.75ex}
}
\newcommand\datesubtitle[1]{{\centering\large #1\par\vspace{1ex}}}  %自定义日期副标题格式,为了保险,最好使用两层大括号

% 靠右对齐,右边距2字
\newcommand{\kaoyouerziju}[1]{{\raggedleft #1 \hspace{2em} \par}}
% 楷体,右边距5字
\newcommand{\kaitiqianming}[1]{{\raggedleft\large\kaishu\ziju{1} #1 \hspace{5em} \par}}

% 一人带职位
\newcommand{\yirendaizhiwei}[2]{
    {\setlength{\tabcolsep}{0em}
    {\raggedleft\begin{tabular} {cc}%
        #1 & \quad{} #2 \hspace{4em} \\ 
        \end{tabular} \\[1pt]}}
}

% 右下定宽
\newcommand{\youxiadingkuan}[1]{
    \begin{list}{}{
        \setlength{\topsep}{0pt}        % 列表与正文的垂直距离
        \setlength{\partopsep}{0pt}     % 
        \setlength{\parsep}{\parskip}   % 一个 item 内有多段,段落间距
        \setlength{\itemsep}{0pt}       % 两个 item 之间,减去 \parsep 的距离
        \setlength{\itemindent}{0pt}%
        \setlength\parindent{0pt}
        \setlength\listparindent{0pt}
        \setlength{\leftmargin}{0.4\linewidth}
        \setlength{\rightmargin}{2em}
    }
    \item[] #1
    \end{list}
}

\usepackage{etoolbox}

% 下面是修改了脚注样式
% 一些LATEX内部命令含有@字符,如\@addtoreset,如需使用这些内部命令,就需要借助于另两个命令\makeatletter和\makeatother.
\makeatletter
% 补丁,改脚注文本前面的序号的字体,去掉其上标样式 
\patchcmd{\@makefntext}
    {\@makefnmark}
    {\hbox{\normalfont\@thefnmark}}
    {}{}

% 给脚注编号前后添加 〔〕
\renewcommand\thefootnote{{〔\arabic{footnote}〕}} 

%% 开启 footmisc 的 hang 选项
\setlength{\footnotemargin}{1.25em}     % 整个脚注文本的左边距,加此边距,来显示脚注序号。
\setlength{\skip\footins}{1\baselineskip} % 脚注线 和 脚注内容 间距 
% \setlength{\footnotesep}{\skip\footins} % 两个脚注文本之间的间距
\renewcommand{\hangfootparskip}{0pt}
\renewcommand{\hangfootparindent}{2em}

% \patchcmd{\@makefntext}
% {\ifFN@hangfoot\bgroup}
% {\ifFN@hangfoot\bgroup\def\@makefnmark{\normalfont\@thefnmark}}
% {}{}

\makeatother

\usepackage{calc} % 可以在命令中计算长度
\usepackage[]{hyperref} % 放在 footmisc 后面

% 引用样式:使用 latex 原始的 list 环境
\newenvironment{yinyong}{%
    \begin{list}{}{\parsep\parskip
        \setlength\topsep{0pt}
        \setlength\itemindent{2em}%
        \setlength\parindent{2em}
        \setlength\listparindent{2em}
        \setlength{\leftmargin}{2em}
        \setlength{\rightmargin}{2em}
        \kaishu
    }
    \item[]
}{
  \end{list}
}

\title{毛泽东思想万岁\\1958.1—1960}
\author{毛泽东}
\date{}

\begin{document}

\frontmatter
\maketitle
\tableofcontents

\mainmatter

\section[在杭州会议上的讲话(一)(一九五八年一月三日)]{在杭州会议上的讲话(一)}
\datesubtitle{(一九五八年一月三日)}


右派是反对派,中右也反对我们,中中是怀疑的。基本群众和资产阶级、资产阶级知识分子中的左派是赞成我们的。

关于对待资产阶级的问题,好多国家怀疑中国是右了,好像不像十月革命。因为我们不是把资本家革掉,而是把资本家化掉。其实,最后把资产阶级(化掉),如何可以说右呢?仍是十月革命。如果都照十月革命后苏联的做法,布疋没有,粮食没有,(没有布疋,就不能换得粮食)、煤矿、电力各方面都没有了。他们缺乏经验。我们根据地搞的时候多了,对官僚资本(生产秩序)来个原封不动,对民族资本更是为此。但是不动中有动。全国资本家七十万户,资产阶级知识分子几百万,没有他们就不能够办报、搞科学、开工厂。有人说“右”了。就是要“右”,慢慢化掉,正确处理人民内部矛盾,就是这个路线贯彻下来的。有的是一半敌人,一半朋友,有的三分之一或多一些是敌人。

治淮十二亿人民币搞了七年,治淮的数量,如果打七折(有些有质量上的问题),也是合算的。原来计划低了,后来超过了,批评右倾保守。就很舒服。愈批评愈高兴。甘肃一千多万人口,劲头很厉害,值得去学习。

要抓十二条,今后要评比:1.水,2.肥,3.土,4.种(优良品种),5.改制——如复种、晚改早,旱改水,6.除病虫害(浙江因虫损失一亿斤,一年夏把虫灭得差不多了,日本无马克思主义。已搞得无虫了,我们有马克思主义),7.机械化(新式农具、双轮双铧犁、抽水机等),8.畜牧,9.副业,10.绿化,11.除四害,12.治疾病讲卫生。

还要抓另一个十二条,也要评比,1.工业,2.手工业,3.农业,4.副业,5.林业,6.渔业,7.牧业,8.交通运输业,9.商业,10.科学,11.文教,12.卫生。

四十条到第二个五年计划第三、四、五年就要修改,愉快地批判右倾。一九五六年工业产值增百分之三十一,没有一九五六年的突飞猛进,就不能完成五年计划。今年三月比一次,夏季比一次,到十月开党代会再比一次。省与省比,县比县比,社与社比。如果大家同意再商量着办。大家都要到别的省参观参观,自己不出门,比输了就活该。很值得到甘肃去一次。

要全面规划,几次检查,年终评比,开几次会,开小型的会,抓地、县书记。地委书记的会,两个月开一次,每次不超过五天,县委到地委去开会(几个地委一起开,比较有兴趣)。大型的“非常会议”,一年只能开一两次,要两个月抓一抓,否则一年很快过去了。各省评比,在中央开会。

我自己每月同大家谈四次,到处跑一跑,每次两天到三天,看五、六个单位。

工业也要四十条,科学文教也要有。先有个别人准备意见,大家再七凑八凑才成。

全国搞几个经济协作的区域,有些省可以交叉。要认庙不认神。要有这么一个传统,有庙不论谁在那里挂帅,就由他做头,能凑合过去就行了。沙文汉,杨思一等是另外一回事。

湖北省委有个关于领导干部亲自搞试验田的报告,中央批转了,看到没有,很重要,要普遍搞试验。

关于积累与消费问题,究竟积累问题多大,有的提百分之四十五,有的百分之五十,有的百分之五十五。最好一半一半,要看年成,看地方,做几种规定。百分之六十分掉不是一般标准,是减产的情况。要勤俭持家,个人的消费要节约,红白喜事,大吃大喝要反对。各省要出个布告禁赌。婚丧喜庆,红白喜事,都要从简,家庭造酒,完全禁止也不好,爆竹不要禁了,有振奋精神的作用。

政治业务要结合,也就是红与专的问题。政治叫红(我们为红,在美国为白)。红与专是对立的统一。两者不同,有区别。一个是搞精神的,一个是搞物质的。有些业务部门的负责同志,说话时口边政治很少,可见平素不大谈政治。忙得很,一谈就是业务。各省管业务的恐怕更厉害些。一定要批评不问政治的倾向,同时要反对空头政治家。要懂得点业务才好,否则名红实不红。不懂农业,要指导农业就不行。搞试验田,红与专的问题就解决了。一种是不懂业务,空头政治家,一种是迷失方向的经济家或其他技术家,都是不好的。要去分析分析。但是批评人家的时候,先要检查自己,自己也有点空,不甚了了嘛。总理去年就钻了一下工资福利问题。我在北京看了工业展览会,一次看一个馆不够,还要多看看。

整风要贯彻到底,不要半途而废。上海说的第三类人,就会作官,要打掉官风。上海提要有干劲,很好。《浙江日报》社论《是促进派,还是促退派》,《人民日报》要转载。整风中要反浪费,时间不要太长,几天就行了。结合整改,专题鸣放一次,鸣放了,大家警惕。每个家庭成员都要进行教育,要勤俭持家。

什么时候交计划,省地县社都要搞,先搞粗线条的省,要半年交卷。


\section[在杭州会议上的讲话(二)(一九五八年一月四日)]{在杭州会议上的讲话(二)}
\datesubtitle{(一九五八年一月四日)}


《哲学研究》第五期上李×的一篇文章,第六期上冯××的一篇文章,都可以看一看。形式逻辑是量变阶段的科学,是辩证法的组成部分。量变与质变是对立的统一。事物有他的相对的固定性。定出计划,做出决议,是相对的平衡。定了以后,还是要变。平衡、巩固、一致……都是暂时的,不平衡、闹矛盾是绝对的。开会的都有散会的思想,越开得长,越想散会。

形式逻辑好比低级数学,辩证逻辑好比高级数学。这种说法值得研究。圆周割裂千万片它就方了。圆与方是对立的统一。

思维形式表现为:概念、判断、推理。形式逻辑是研究思维形式。形式逻辑有不少错误的大前提,因而不能得出正确的判断。可是按形式逻辑来说并不错。它只管数量。不管内容。内容是各种学部门的事。

昨天讲了两个十二条。下面再谈几个问题:

一、水、肥、土等十二条,要抓住,相互平衡。有水无肥,有肥无水都不行,是相互联系的。农业十年以后(也许要更多的时间)要实行电气化。电气化犁田。畜牧与肥料有关。又是动力,是肉食,是工业原料。除四害与劳动力有关,增加体力,振作精神。

二、工业、手工业、农业等十二条要抓住。

三、反浪费。上海材料,梅林罐头食品厂四年中浪费了四十五万,占资产之一半,八年可以建同样大的一个工厂。这是普遍性的问题。如果每个工厂、学校、机关、合作社都搞一下要节约多少物质。什么事都要有证明,没有证明人家都不信。只要一个材料就够了。剖一个麻雀就够了。不一定剖太多。

整风中以十天时间专搞反浪费(从放到改十天到十五天时间就可以)。可以搞几十亿。

四、消费和积累的比例要当作大问题来研究。这个比例有百分之四十五、百分之五十、百分之五十五、百分之六十等,各种分法要研究。春季即抓,不抓来不及了。这是大问题,搞得不好,工人农民不满意。一九五四年搞了九百六十亿斤,得罪了几亿农民。今年八百五十亿斤,暂时规定三年。我倾向于百分之五十。歉收、丰收有所不同。跟勤俭持家结合起米,可过日子。婚丧喜庆,一律从简。

五、要搞试验田。湖北省委关于试验田(的报告)是个好报告。(××插话:浙江省委的报告,因为有了《人民日报》的按语,大家就注意去看了。)以后翻译的书,没有序言不准出版。初版要有序言,二版修改也要有序言。《共产党宣言》有多少序言。许多十七、八世纪的东西,现在如何去看它。这也是理论与中国实际的结合,这是很大的事。

六、红与专。政治与业务是对立的统一。要做两方面的批判。专管政治,不熟悉业务不好。政治与业务就是红与专。搞政治的都要专有困难,但主要的部分要专一下。湖北省委试验田很见效,搞的时间并不长。

七、打掉官风。上海提出的。不要做官,统统把官风打掉,与老百姓平等。

八、按时交计划。

九、除四害。开展以除四害为中心的爱国卫生运动,每月大检查一次。“五年看三年,三年看头年”,这句话很好。

十、绿化。今年彻底抓一抓,做计划,大搞。听说三丈多高的树每天要吸收和散发一吨多水,会不会影响地下水?

十一、第二个五年计划。各省地方工业产值比例要超过农业生产(产值)(包括下放给地方管的工业在内)。要有全国平衡,不能无政府主义。

十二、开会办法。要大中小型。(各省)“非常会议”(如党代会)一年一、二次;中型几十人到二、三百人(如开县委书记会议),小型的如地委书记会议。要下去开会,了解他们,“非常会议”谈政治,业务会议也要谈政治。

十三、省委书记、委员轮流离开办公室,一年四个月,到处跑,可以釆取走马看花,下马看花两种办法。到一处谈三四个小时就走也好,下去一周半月也好。不一定到一个地方就是三、四个月。

十四、内部不要请客,不要要人家请吃馆子,不要迎送。不专搞舞会,不请看戏。谁去机场接客人,要处分被接的人,这是打掉官气的一个方面。

十五、关于两类矛盾问题。一个是敌我矛盾,一个是人民内部矛盾。人民内部矛盾有两类,一类是阶级斗争性质的,一类是工人阶级、劳动人民内部的,先进落后性质的。人民内部矛盾分两类,一是无产阶级同资产阶级、小资产阶级之间的矛盾,这是阶级矛盾,还有劳动人民内部的矛盾,这些劳动人民内部的矛盾,一部分属于阶级斗争性质的,如受封建思想影响,打老婆,甚至因此杀老婆,又如自由主义,个人主义(资产阶级、小资产阶级思想),绝对平均主义(小资产阶级思想)都反映着私人所有制问题,一部分是属于先进落后的,是认识上的问题,问题看不到底,譬如农业合作化高潮,在北京开会时还看不大清楚,会后南下.到山东,到其他地方一看,形势大变,才有把握写序言。这是先进落后之间的矛盾,形势看不清。有些人硬不肯增产,总认为没有条件。开展整风,右派一下子起来,几篇社论,六月到七月大局一定,许多事是无法完全预料到的。

主要的占大量的矛盾,阶级斗争是主要的,是要革一个东西嘛。宪法上规定三大改造,实际上是两大改造,改造资产阶级,改造小资产阶级。阶级矛盾是过渡时期的(主要矛盾),搞得好再有××年就行。××年加八年,共××年,不用××年。我们每年这样搞一次整风,就把资产阶级思想搞掉了。大量的是先进与落后的矛盾。同大量中间派的矛盾是阶级矛盾。富裕中农中百分之四十赞成合作化,百分之四十不那么热心,百分之二十想退社,未真心退社。坚决退的是少数,有百分之五可能是右派,他们还是劳动人民,不划右派,要七擒七纵。

产生人民内部矛盾的原因:

1.资产阶级、小资产阶级思想影响劳动人民。个人主义、自由主义、绝对平均主义、官僚主义(现在要挂在资产阶级账上)都是资产阶级、小资产阶级思想影响。

2.主观主义的原因。看不准,估计不足,右了。要经常提醒跟着形势前进。

3.有领导的原因。领导好一些,先进占多数,落后占少数。可以这样领导,可以那样领导(如浙江平阳和黄岩两县就不同),只要解剖一个麻雀,就可以了解整个气候。如何领导,很值得研究。山东寿张县有个刘传友是深入领导的。鲁西无喂猪习惯,现在每户养两头,他还改造土壤。浙江桐庐油厂、酒厂在同样条件下比出品率高低,使落后赶上先进的材料很好。《人民日报》王朴写的短评也写得好,里面有辩证法(均见一月三日《人民日报》)。要宣传理论,讲辩证法,讲唯物论,如上层建筑与经济基础,生产关系与生产力,是历史唯物论的基本内容。

经常把问题放在心上想一想,和个别同志,少数同志吹一吹,和自己的秘书平等地吹吹,看看他们的看法如何,找几个党委书记谈一谈,不作为决定,作为酝酿。在一个时期内把几个问题想一想,吹一吹,作为一个重要方法。不要事先不动脑筋,不想一想就开会。有些东西是慢慢成熟的。

十六、谈谈不断革命论。我说的不断革命论和托洛茨基讲的不同,是两种不断革命论。我们革命的步骤是:

1.夺取政权,把敌人打倒,这在一九四九年就完成了。

2.土地革命,一九五○年——一九五二年三月基本完成了。

3.再一次土地革命,社会主义的,现在讲主要生产资料集体所有,一九五五年也基本完成了,一九五六年有些尾巴。这三件事是紧跟着的,两个三年当中解决了。趁热打铁,这是策略性的,不能隔得太久,不能断气,不能去“建立新民主主义秩序”,如果建立了,就得再花力气去破坏。波匈“断”了这么多时候的“气”,资产阶级思想扎了根,再搞就不大好搞了,中农以上的就不想搞合作化了。保加利亚好些,百分之三十合作化。

4.思想战线上政治战线上的社会主义革命——整风运动,这一次今年上半年就可以完成。明年上半年还要搞。

5.还有技术革命。

1——4项都是属于经济基础和上层建筑性质的。土改是封建所有制的破坏,是属于生产关系的。技术革命是属于生产力、管理方法、操作方法的问题,第二个五年计划要搞。1、2,3项今后没有了,思想战线上政治战线上的革命仍旧有的,一个人过一两年又会生霉了。但重点放在技术革命。要大量发展技术专家,发动向技术好的人学习。在工厂、农村中有初级的技术家。红安县领导干部原来是空头政治专家,后来又红又专,工业找[照]桐庐县的方向,搞试验与技术革命联系起来,政治家与技术家结合起来。

从一九五八年起,在继续完成思想政治革命的同时,着重在技术革命方面,着重搞好技术革命。斯大林在提干部决定一切的口号时,也提出了技术革命。

着重搞好技术革命,不是说不要搞政治了,政治与技术不能脱离。思想政治是统帅,政治又是业务的保证。

消灭阶级再有××年就好了。以后,人与人之间思想政治斗争(或者叫革命)还会有,但性质不同,不是阶级关系,都是劳动人民先进与落后之间的矛盾。这时的斗争也是分两部分,一是劳动人民受资产阶级思想影响;一是属于主观主义——认识上的原因,或者还有领导的原因。懂得马克思主义领导艺术的好一些,否则差一些。“无冲突论”是形而上学的。为什么莫斯科宣言中加上辩论法一段呢,因为它适应过去,现在和将来。将来全世界都统一了,两派人争权还会有的,因为意见不一致,出各种报,演各种戏,各人争取群众。有思想交锋。那时上层建筑意识形态总是有的,有生产关系与生产力,有上层建筑与经济基础的矛盾,就会有左中右三种人。上层建筑在那些顽固落后的人手中,又不许大鸣大放了。不会不改正错误,也会有冲突。没有军队了,也可能用拳头,用木棒。那时没有阶级,处理得好,对抗,处理的不好,也会对抗。一个进步路线,一个落后路线,是互相排斥的,是对抗的。××年后,国家权力对内职能逐步地不存在,都是劳动人民了。现在对劳动人民来说,权力也基本不存在了。对劳动人民只能说服,不能压服,对劳动人民不能使用国家权力,用权力就是压服。好像很“左”,其实很右,是国民党作风。打倒官气,十分必要。对敌人威风凛凛是对的,对人民就不行了。

十七、政治家一定要懂些业务,农业搞试验田,工业搞试验品。要比较,比较是对立的统一。企业与企业之间,企业内部车间与车间、小组与小组、个人与个人之间是不平衡的。不平衡不仅是社会法则,也是宇宙法则。刚刚平衡,立即突破;刚刚平衡,又不平衡。要讲评比,在大体相同的条件下,先进与落后比较,是可比的,不是不可此的。

政治家总要懂得一些业务。技术上要比,政治上也要比。技术与政治相结合。看哪一个搞得好。

大家把几篇文章看一看,《解放日报》登的上海梅林厂开展反浪费专题鸣放,一月三日《人民日报》关于桐庐厂比出品率的报导和王朴写的短评)。一件事反映了全国性的事,大家一定要把人家的好事当作自己的好事。搞社会主义,不论问题出在哪里,都要当作自己的事。

学生中的右派,百分之八十可以留校继续读书,要加强对他们的工作,学生要和他们往来,逐步把他们化过来,他们做了好事也要表扬,当然也有假积极的。

不要以为经过这次整风,一切都是黄河为界,界线划得那么清楚。


\section[在南宁会议上的讲话(一)(一九五八年一月十一日)]{在南宁会议上的讲话(一)}
\datesubtitle{(一九五八年一月十一日)}


关于向人代会的报告,我两年没有看了(为照顾团结,不登报声明,我不负责)。章伯钧说国务院只给成品,不让参加设计,我很同情,不过他是想搞资产阶级的政治设计院,我们是无产阶级的政治设计院。有些人一来就是成品,明天就开会,等于强迫签字。只给成品,不给材料,要离开本子讲问题,把主要思想提出来交谈。说明为什么要这样办,不那么办?财经部门不向政治局通情报,报告也一般不大好谈,不讲考据之学、辞章之学和义理之学。前者是修辞问题,后者是概念和推理问题。

党委方面的同志,主要危险是“红而不专”,偏于空头政治家。脱离实际,不专也慢慢退色了,我们是搞“虚业”的,你们是搞“实业”的,“实业”和“虚业”结合起来。搞“实业”的,要搞点政治;搞“虚业”的要研究点“实业”。红安县搞实验田的报告是一个极重要的文件,我读了两遍,请你们都读一遍。红安报告中所说的“四多”,“三愿,三不愿”,是全国带普遍性的毛病。就是对“实业”方面的事不甚了解,而又要领导。这一点不解决,批评别人专而不红,就没有力气,党委领导要三条:工业、农业、思想。省委也要搞点试验田如何?不然空头政治家就会变色。

管“实业”的人,当了大官、中官、小官,自己早以为自己红了,钻到那里边去出不来,义理之学也不讲了。如反“冒进”。一九五六年“冒进”,一九五七年“冒进”,一九五七年反“冒进”,一九五八年又恢复“冒进”。看是“冒进”好还是反“冒进”好?河北省一九五六年兴修水利工程一千七百万亩,一九五七年兴修水利工程二千万亩,一九五八年二干七百万亩。治淮河,解放以后七、八年花了十二亿人民币,只做了十二亿土方,今年安徽省做了十六亿土方,只花了几千万元。

不要提反“冒进”这个名词好不好?这是政治问题。一反就泄了气,六亿人一泄了气不得了。拿出两只手来给人家看,看几个指头生了疮。“库空如洗”,“市场紧张”,多用了人多花了钱。要不要反?这些东西要反。如果当时不提反“冒进”,只讲一个指头长了疮,就不会形成一股风,吹掉了三个东西:一为多快好省,二为四十条纲要,三为促进委员会。这些都是属于政治问题,而不是属于业务。一个指头有毛病,整一下就好了,原来“库空如洗”,“市场紧张”,过了半年不就变了吗?

十个指头问题要搞清楚,这是关系六亿人口的问题。究竟成绩是主要的,还是错误是主要的?是保护热情,鼓励干劲,乘风破浪,还是泼冷水,泄气?这一点被右派抓住了,来了一个全面反“冒进”。陈铭枢批评我“好大喜功,偏听偏信,喜怒无常,不爱古懂”。张奚若(未划右派)批评我“好大喜功,急功近到,轻视过去,迷信将来”。过去北方亩产一百多斤。南方二、三百斤,蒋委员长积二十年经验,只给我们留下四万吨钢,我们不轻视过去,迷信将来,还有什么希望。偏听偏信,不偏听不可能,是偏听资产阶级,还是偏听无产阶级的问题。有些同志偏得不够,还要偏。我们不能偏听梁漱溟、陈铭枢。喜怒无常,常有也并不好,不能对资产阶级右派老是喜欢。不爱古董,这是比先进还是落后问题,古董总落后一点嘛。我们除四害,把苍蝇、蚊子、麻雀消灭了,前无古人,后无来者,一般是后来居上,不是“今不如古”,古董不可不好,也不可太好。北京拆牌楼,城墙打洞,张奚若也哭鼻子,这是政治。

元旦社论,提出鼓足干劲,力争上游(陈伯达插话,说应该多积累)。

减少人员问题,商业部分,合作社不受政治影响,说了几年了,他们不砍,交给地方砍去它一半。我一进北京,三轮车一辆也不能减,我们的“圣旨”太多了。无考虑余地,你说可以考虑,我也高兴一点。我们的现状维持派太多。要重新做判断之学,如“蒋介石是反革命”,有些概念要重新判断。

章伯钧要搞资产阶级设计院,我们设计院是政治局,办法是通一通情报,不带本子,讲讲方针。搞个协定如何,如果你不同意,我有个抵制办法,就是不看。已经两年不看了。地方财政部门也采取这个办法。

这几年反分散主义,创造了个口诀:大权独揽,小权分散;党委决定,各方去办,办也有权,不离原则,工作检查,党委有责。

政治机关有些人提出,说是党政不分,是不是要一家一半?这不行,先不分,然后才能分,不然就是小权独揽,如四十条纲要怎么分,中央二十条,农业二十条,这是不行的。中央搞了四十条,然后分工去办,这就是分。宪法,不能中央搞一个,由什么机关搞一个。小权小分,大权就不能独揽。大家不是赞成集体领导吗?一长制不是搞倒了吗?(苏联军队实行一长制。朱可夫犯了错误)。



\section[在南宁会议上的讲话(二)(一九五八年一月十二日)]{在南宁会议上的讲话(二)}
\datesubtitle{(一九五八年一月十二日)}


八年来我为这样一个工作方法而奋斗,我说了一千次,一万次,这是极而言之,说的多了,等于白说。人的思想总是逐步受影响的。要“毛毛雨下个不停”,“倾盆大雨”就会发生径流。政治局是团粒结构不足的,倾盆大雨吸收不了,顺着身上流走了(这是土壤学,农业学都要一本,不然省委书记当不成,有一天总要撤职的,这不是我威胁你们)。政治局成为一个表决机器、像杜勒斯的联合国,你给十全十美的文件,不通过不行,像唱戏一样,已经打了牌了,非登台演出不可。文件上又不讲考据之学,义理之学,又有洋文。我有一个手段,就是消极抵抗,不看。你们的文件,我两年不看了,今年还不准备看。

在杭州会议我讲的,恩来同志讲了没有?一九五五年十二月我写了农村社会主义高潮一书序言,对全国发生了很大影响,是个人崇拜也好,偶像崇拜也好,不管是什么原因,全国各地报纸,大小刊物都登了,发生了很大影响。这样我就成了“冒进”的罪魁祸首。我说了各部门都有对形势估计不足的情况,军队增加了八十万人,工人学徒增加了一百万人,反对右倾保守,为什么要增加人?我不懂,也不知道。

一九五五年夏季北戴河会议“冒进”想把钢搞到一千五百万吨(第二个五年计划)。一九五六年夏季北戴河会议反“冒进”就影响了人代会的报告。人心总是不齐的,不平衡的规律是宇宙发展的发则。孟夫子说过:“物之不齐,物之惜也。”人心不齐,又可以齐,有曲折,螺旋式的前进。当然大家都是为党为国,不是为私。

我对分散主义的办法,就是消极抵抗,还要小会批评。财经部门考证之学,辞章之学,义理之学也不讲。要和风细雨,要事先通一点情报,总是倾盆大雨,发生径流,总不开恩。总没有准备好,不完全,这就是封锁,这是斯大林的办法。开会前十分钟拿出文件来让人家通过,不考虑人家的心理状态。你们是专家,又是红色,政治局多数人是红而不专。我攻击方向,主要是中央部长以上的干部,也不是攻击所有的人,是攻击下倾盆大雨的人,封锁的人。小会不解决问题,就开中央全会(文章做好这件事。没有认真解决,写给广西省委一封信,谈报纸问题。)我在苏联写回一封信,说你们不得中央的支持,对你们工作不利,不然会孤立,像“梁上君子”。

政治局不是设计院。倾盆大雨在我们身上流走了,老说没有搞好,实际上是封锁。分散主义有一点,但不严重。各有各的心理状态,我替你们设想,你们大概有一个想法,大概中央是十全十美的,不是全能,也是九分。另外,大概像《茶花女》小说中的女主角马哥瑞特,快死了,见爱人还要打扮一番。《飞燕外传》,赵飞燕病了,不见汉武帝。总之是不顾以不好的面目见人。蓬头散发见人有何不可?想起一条写一条,把不成熟的意见提出来,自己将信将疑的东西拿出来,跟人家商量,不要一出去就是“圣旨”,不讲则已,一讲就搬不动。四十条就是这样,开始在杭州拟了十一条,天津增到十七条,到北京才增加到四十条。“寡妇养仔,众人之力”,这是工作方法问题。

我看还得闹对立的统一,没有针锋相对不行。要么你说服我,要么我说服你,要就是中间派。有人就是这样,大问题不表示态度。马克思主义不是不隐蔽自己的观点吗?这样我不理解,应当旗帜鲜明,大概想作楚庄王。“三年不鸣,一鸣惊人,三年不飞,一飞升天”。

再一个是顽固(乔木到,一说曹操,曹操就到)。人民日报革命党不革命。我在二月二十九日最高国务会议上的讲话,民主党派拿我的讲话做文章,各取所需,人民日报闻风不动,写一篇社论,从恩格斯谈起。二月开始谈到,我给他们讲,你们又不执行,为什么又不辞职?十一月二中全会,一月省委书记会议,三月宣传会议,还有颐年堂会议,都说了人民内部矛盾。不必忧虑,是可以解决的,可是打不动×××同志的心。我说十个干部一个拥护我就好了,他也不说反对,就是不执行。地委副书记以上一万人,有一千拥护我就好了。北京的学校那个放的开?×××同志五月二十二日在中直会议上做报告,有一句名言:“千金难哭好时机”,“寸金难买寸光阴”。这样才放开了。大鸣大放,清华大学叛变了几个支部,右派高兴,不然审也审不出这些叛变分子。人们都有一种惰性,不容易搞开,乔木要不是那一次会议,北冰洋的冰是开不了的。××是好人,就是无能,我说他是教授办报,书生办报,又说过他是死人办报。

再谈考据之学,辞章之学,义理之学。财经工作者有很大成绩,十个指头,只有一个指头不好,我讲过一万次就是不灵,工作方法希改良一下。我最无学问,什么委员也不是,我和民主人士谈过,我是唱老夫人的,你们是唱红娘的。我是老资格吗?总该给我讲一讲。我灰心了,这次千里迢迢让你们到南方来,是总理建议的。

我是罪魁,一九五五年十二月我写了文章,反了右倾,心血来潮,找了三十四个部长谈话,谈了十大关系,就头脑发胀了,“冒进”了,我就不敢接近部长了。三中全会,我讲去年砍掉了三条(多快好省,四十条纲要,促进委员会),没有人反对,我得了彩,又复辟了。我就有勇气再找部长谈话了。这三年有个曲折。右派一攻,把我们一些同志抛到距离右派只有五十米远了。右派来了个全面反“冒进”甚么“今不如昔”,“冒进比保守损失大”。研究一下,究竟那个大,反“冒进”,六亿人民泄了气。一九五六年六月一篇反冒进的社论,既要反右倾保守,又要反急躁冒进,好像有理三扁担,无理扁担三,实际重点是反冒进的。不是一个指头有病。这篇社论,我批了“不看”二字,骂我的我为什么看?那么恐慌,那么动摇。那篇东西,四平八稳,实际是反“冒进”。这篇东西格子未划好,十个指头是个格子,只一个指头有病,九与一之比,不弄清楚这个比例关系,就是资产阶级的方法。像陈叔通、黄炎培、陈铭枢的方法。

我要争取讲话,一九五六年元月至十一月反“冒进”,二中全会搞了七条,是妥协方案,解决得不彻底,省市委书记会议承认部分钱花的不恰当,未讲透,那股反“冒进”的气就普遍了。廖××向我反映,四十条被吹掉了,似乎并不可惜。可惜的人有多少?叹一口气的人有多少?吹掉三个东西,有三种人,第一种人说:“吹掉了四十条中国才能得救”;第二种人是中间派。不痛不痒,蚊子咬一口,拍一巴掌就算了;第三种叹气。总要分清国共界限,国民党是促退的,共产党促进的。

×××

为党为国,忧虑无穷,反“冒进”,脱离了大多数部长、省委书记、脱离了六亿人民。请你看篇文章,宋玉的《登徒子好色赋》,这篇文章使登徒子二千年不得翻身,他的方法是“攻其一点,不及其余”。登徒子向楚襄王反映,宋玉长得漂亮,会说话,好色,宋玉一一反驳了。宋玉反击登徒子好色,说登徒子讨了一个麻脸驼背的老婆,生了七个孩子,你看好不好色,只攻其好色一点,不及其余。我们看干部,要看德才资,不能德才都不讲,只讲德的一部分。九个指头不说,只说一个指头,就是这种方法。我看几年要下毛毛雨,不要倾盆大雨,要文风浸润,不要突然袭击,使人猝不及防。

五月间右派进攻,使那些有右倾思想的同志提高了觉悟,这是右派的“功劳”,这是激将法。

一九五四——一九五五年粮食年度征购九百二十亿斤。多购一百亿斤,讲冒进,这一点有冒。闹得“人人谈统购,家家谈粮食”。章乃器是粮食部长,他同意这个计划,是不是故意把农民闹翻,可能有阴谋。去年粮食销量多,反映了农民没有劲。江苏反映社长低头,干不下去了。我们就怕六亿人民没有劲,不是讲群众路线吗?六亿泄气,还有什么群众路线?看问题要从六亿人民出发,要分别事情的主流、支流、本质、现象。

中央大权独揽,只揽了一个革命,一个农业,其他实际在国务院。

人都有迷信,有惰性。比如我游水,中间隔了三十年。

除四害,人人讲卫生,家家讲清洁,一年十二个月,一月检查一次。这样医院办学校,医生去种田,病人大大减少,人人精神振作,出勤率大为提高,要集中搞,最好两年完成。

我和华东五省约好,今年开四次会,小型的会,是两种元素配合,中央和地方两个元素一配合就不同了。各省也开小型的会。廖××告诉我,十年看五年,五年看三年,三年看头年。我一年给你们开四次会,检查十二次。两本账,争取超额完成。这是苏联发明的。红安县那篇文章,请你们再看一遍。一人首倡就推开了。县委副书记一人买锄头,百分之八十的人买了锄头。还要山东营县那个公社的例子。有一个例子就够了。


\section[关于报纸工作给刘建勋、韦国清同志的一封信(一九五八年一月十二日)]{关于报纸工作给刘建勋、韦国清同志的一封信}
\datesubtitle{(一九五八年一月十二日)}


刘建勋、韦国清二同志:

送上几份地方报纸,各有特点,是比较编得好的,较为引人看,内容也不错,供你们参考。省报问题是一个极重要问题,值得认真研究,同广西日报的编辑们一道,包括版面、新闻、社论、理论、文艺等项。钻进去,想了又想,分析又分析,同各省报纸比较又比较,几个月时间就可以找出一条道路来的。精心写作社论是一项极重要任务,你们自己、宣传部长、秘书长、报社总编辑,要共同研究。第一书记挂帅,动手修改一些最重要的社论,是必要的。一张省报,对于全省工作,全体人民,有极大的组织、鼓舞、激励、批判、推动的作用。请你们想一想这个问题,以为如何?



\section[给《文艺报》编委会的一封信关于1958年第二期《再批判》栏的按语(一九五八年一月十九日)]{给《文艺报》编委会的一封信关于1958年第二期《再批判》栏的按语}
\datesubtitle{(一九五八年一月十九日)}


即送北京《文艺报》×××、×××、×××三同志:

看了一点,没有看完,你们就发表吧。按语较沉闷。政治性不足,你们是文学家,文也不足。不足以唤起读者的注目。近来文风有了改进,就这篇按语说来,则尚未。题目太长,《再批判》三字就够多了。请你们斟酌一下。我在南方,你们来信刚才收到,明天就是付印日期,匆匆送上。

祝你们胜利!
<p align="right">毛泽东
一月十九日下午</p>



\section[在最高国务会议上的讲话(一九五八年一月二十八日)]{在最高国务会议上的讲话}
\datesubtitle{(一九五八年一月二十八日)}


今天的国务会议是临时召集的。

大家不要因为上午八点开会就认为有大事。过去多在下午,这是我心血来潮,商量一个普通问题。

八年以来,讨论国家预算这一次是最早的一次。以后也要每年在这时候开会.

这次人代大会,要开得从容些,要开好一点。多开小组会,多做些准备工作.少开大会,真正把问题搞清楚,修正工作上的缺点和错误。做报告的人来没有?(答声,未了。)做了报告不要第二天就发表,报告了,让大家提修改意见,讨论修改后再发表。

我看了七、八年了,我看我们这个民族大有希望。特别是去年这一年,我们六亿人口的民族精神,大大发扬。经过大鸣大放大辩论,把许多问题搞清楚了,任务提得更恰当.如十五年左右可在钢铁和其它重工业方面赶上英国,多快好省;农业发展纲要四十条的修正重新发布等,给群众很大的鼓励。许多事情过去做不到的,现在能做到了。过去没有办法的,现在也有办法了,比如除四害,群众劲头很大。我这个人老鼠捉不到,苍蝇、蚊子可以捉它一下。平常总是苍蝇蚊子向我们进攻嘛!古代有这么一个人写了一篇提倡消灭老鼠的文章。现在我们要除四害,几千年来,包括孔夫子在内都没有除四害的志向,现在杭州市准备四年除去四害,有的提二年、三年、五年的。

所以我们这个民族的发展大有希望。悲观论是没有根据的,是不对的,要批判悲观论者。当然不要打架,要讲道理,是具有希望,不是中有希望,小有希望,更不是没有希望,而是大有希望,文章在“大”字上,日本人讲:“大大的有”。(笑声)

我们的民族在觉醒,像我们大家在早晨醒来一样。因为觉醒了,才打倒了几千年来的封建制度,以及帝国主义和官僚资本主义,执行了社会主义改造,现在整风、反右派又取得了胜利。

我们的国家是又穷又白,穷者一无所有,白者一张白纸,穷是好的,好革命,白做什么都可以,做文章,画图样,一张白纸好做文章。

要有股干劲,要使西方世界落在我们的后头,我们不是要整掉资本主义思想吗!西方要整掉资产阶级思想不知要多长时间。西方世界又富又文,他们就是太阔了,包袱甚重。资产阶级思想成堆。要是杜勒斯愿意整资产阶级的风,还要请我们做先生。(笑声)


一谈起来,我们国家这么多人口,地大物博,人口众多,四千多年历史,但现在生产与我们的地位完全不相称,钢铁生产还不如一个比利时。它有七百多万吨钢,我们只有五百二十万吨。总之。我们是个历史长久,优秀的民族,可是钢是那么低。粮食北方一百多斤,南方三百多斤。识字人那么少。比这些都不行。但是我们有股干劲。要赶上去,在十五年内赶上英国。

十五年要看头五年,头五年要看前三年,前三年要看头一年,头一年要看头一个月,更看前冬,去年中共三中全会就在水利、积肥上做了布置。

现在劲头鼓起来了,我们的民族是个热情的民族,现在有了热潮,正好有一比,我们民族像原子,把我们民族的原子核打破,释放热能,过去做不到的事,现在也能做到。我们这民族有这么一股劲,十五年要赶上英国,要搞四千万吨钢(现在五百多万吨),要搞五亿吨煤(现在是一亿吨),要搞四千万瓧电力(现在是四百万瓧),要发展十倍,所以要发展水电,不光发展火电。实现农业发展纲要四十条还有十年,看来不要十年,有的说五年,有的说三年,看来八年可以完成。

要达到这个目的,在这种形势下要有一股干劲。我在上海,一个教授和我谈《人民日报》社论《乘风破浪》,他说,要鼓起干劲,力争上游就是从上海上四川,上游得费点劲,不是下游。说得很对,我很欣赏这个人,这是好人,这人有正义感。有人批评我们“好大喜功,急功近利,卑视过去,迷信将来。”这几句话恰说到好处,“好大喜功”,看是好什么大,喜什么功?是革命派的好大喜功,还是反革命派的好大喜功?革命派里又有两种:是主观主义、形式主义的好大喜功,还是合乎实际的好大喜功。我们的古人都说:“福如东海,寿比南山。”都是好大喜功。我们是好六亿人民之大,喜社会主义之功,这有什么不好呀,急功近利,也不是不好呀!曾子曰:“吾日三省吾身。”这是圣人之长。大禹惜寸阴,陶侃惜分阴,像我们这样人要惜分阴,不能老开会,几个月不散会。急功近利,要看是搞个人突出、主观主义,还是搞合乎实际、可以达到的平均先进定额?要搞平均先进定额,如亩产量,有先进、中间、落后,都搞先进的为定额,以大力士为定额,那不行,是在先进定额中加以平均。

至于卑视过去,不是说过去没有好东西,过去是有好的东西,但是否对过去那么重视,老是天天想禹、汤、文、武、周公、孔子,我不赞成那样看历史。如过去用木船,现在就可以不用了,可以用轮船,郑州的建筑物太古老了,总是新的东西好,北京的房子,就不如青岛好。外国的好东西,为什么不可以搬来,铁路就是外国的嘛!这个东西(敲扩音器)也是外国的嘛!外国的好东西要学,应该保存的古董一定要保存,要挖,把它保存起来。推出午门以外斩首,那是老落后。有的认为城墙不要拆,有的主张可以拆,我看可以拆。用石头做工具才四千年不到五千年,那时发明细石器,像现在发明原子弹一样,是了不起了,那时的英雄可以骄傲得很,可是现在不能用石器。为什么要把古老的东西保持下来?石器起过进步作用,而且最大,是否现在要回到石器时代?我看人类历史是前进的,一代不如一代,前人不如后人。右派分子说“今不如昔”,应当倒过来!今天比过去好。有的人为了拆城墙伤心,哭出眼泪,我不赞成。但北京的城墙不拆也可以,南京、济南、长沙的城墙拆了我很高兴,有些老人就伤心啊!伤心哉,秦欤,汉欤,近代欤?北京的城墙保存一千年,一千年以后还是要拆。你们不要以为我这个人什么都轻视。在某种意义上不要对过去太重视。“迷信将来”,我们的目的是为了将来。如开会,现在讲,将来就是散会,老开会不行,人民代表大会,开上十几天就想散会了。我们把希望寄托于将来是对的,但不能迷信。

所以上边上海那个教授的话是对的。

陈铭枢说我“偏听偏信,好大喜功,喜怒无常,轻视古董”。好大喜功我已经讲过了,至于偏听偏信,陈铭枢是叫我听梁漱溟、陈铭枢的,我不能偏听右派的,是偏听共产党,还是偏听国民党、杜勒斯。君子群而不党,没有此事。孔夫子杀少正卯。就是有党。是因为少正卯同他争学生,孔夫子就给少正卯定了五条罪状。(问在座的人:那五条?有人答……)

我们对右派都不杀,所以不偏听偏信是不可能的。陈铭枢你过去好,我就喜欢,现在你成了右派,我就愤怒,这还让我喜什么?说我不像个主席的样子,我这个人就是不像个主席的样子。还说我轻视古董,古代的东西都好吗,我劝青年不要搞旧诗,不要那么重视古董。


人多好,人少好?人多一些好么,现在劳动需要人。但是要节育,现在是:第一条控制不够,第二条宣传不够,目前农民还不注意节育,恐怕将来搞到七亿人口时就要紧张起来。现在不要怕人多,有人怕没饭吃,那我们大家就少吃一点,人多一点,士气旺盛,这是我有点乐观,不是地大物博吗!但我不是说不要宣传节育,我是赞成节育的。要像日本、美国那样节育,不要像法国那样节育,越节越少。邵先生六道讲的对,现在不对,达到极点就趋向反面。人多没饭吃,就少吃点。据说东方人吃素对身体健康有益处,这是黄道之学(黄炎培)。。中国人平均每月吃肉三斤,二人六斤,匈牙利每人吃二十多公斤,这是我们社会主义阵营的国家,除匈牙利外,帝国主义国家吃肉多,都肉食者鄙。我们吃四钱油,五钱盐,也行。至于提倡吃素,我看不行,因为理论与实际脱节,可见黄道之学不学也可。过去孔夫子很讲究排场,食不厌精,每餐要吃点姜,闹脑溢血。我看还是少吃点好。吃那么多,把肚子胀那么大干啥。像漫画上画外国资本家那样。

我这都是说的一些问题,请大家考虑。

有两种领导方法,一种比较好一点,一种比较差一点。这两种方法,不是说杜勒斯一种,我们一种,而是都搞社会主义,有两种领导方法,两种作风。合作化问题,有人主张快点,有人主张慢慢来,拖到七、八年才搞。我认为前一种好,还是趁热打铁,一气呵成好点,不要拖拖拉拉。整风好,不整好?还是整风好,还是大鸣大放好。我们说鸣放,右派说大鸣大放,我们说鸣放是指学术上说的。他们要用于政治,所以“大鸣大放”这个提法是从右派那里借来的,可见小鸣小放不行,中鸣中放也不行,就是要大鸣大放。

要改掉官气,官是可以做的,但要打掉官气。最好根绝官气。我们都是做官的.都有点官气,官气是一种坏习惯,不是好习惯。不论什么大官,主席也好,总理也好,都应以普通劳动者姿态在人民中出现,使工人、农民感到和他们平等,我们自己说平等靠不住,要使对方感到平等。改掉官气不是很容易的,有官气就要改掉,先从共产党起,民主党派也可以逐渐改掉。湖北红安县的领导干部过去就有官气。世界上有个中国,中国有个湖北省,湖北省有个红安县,过去这个县叫黄安县,因为黄字不好改为红安县,这个县的干部以前官气十足,农民看不惯干部,还有三多,说皮鞋多,大氅多,自行车多,是否还有打扑克多。后来他们改了,穿草鞋到乡下去,农民很欢迎。现在干部下乡,山东的老百姓讲,“八路军又来了。”可见这六、七年来官气十足,做了官有了架子,因此要整风,要整掉官气,民主党派也要整风。写《水经注》这个人了不起,写得那么好。孔夫子也是官气十足,他有两匹马一辆车,每天坐在车子里摇摇摆摆,得了胃病,叫胃下垂,而且还要吃细的。类似狮子之类吧。他吃多了,有砂子,不干净,所以得了胃病。孔子到了齐国,人家骂他四体不勤,五谷不分。我看骂红安县以前有些干部也是这样,所以中央机关干部每年要有四个月要离开北京。北京不是好地方,历来出官僚的地方。为什么孙中山先生不建都在北京呢?大概是因为这个地方出官僚。北京不出产任何东西,我不是指北京这个地方,是指中央机关,中央机关不生产钢,不出水泥,不出粮食,也不出纸烟,什么也不产生。产生思想吗?也不产生,思想也是从群众中来的。不是北京出的。我说不产生任何东西,是指不产生任何原料。原材料是产生自工人、农民,章伯钧要搞政治设计院那不行,一切要从群众中来。原材料来自工农,我们是加工,我脑子里不产生任何东西,一跑出北京就取得了东西,产生出力量。

要鼓干劲!鼓舞士气,劲可鼓,而不可泄,应当鼓舞士气。合作化一搞,有人叫得不得了,说搞多了,要砍掉十万个,双轮双铧犁在南方名誉不太好,在湖北等四省还好。大家看过登徒子写的好色赋没有,就是攻其一点不及其余,说登徒子的老婆很丑,别人谁都不要的,脸上有麻子,耳朵很大,还有痔疮,结果生了五个儿子,宋玉以此证明登徒子好色。因为登徒子告了宋玉一状,说宋玉很漂亮,好色,请楚王注意。我这里不是替登徒子翻案,是讲这个方法不好。右派就是这样攻击我们的。但好人也有的这样看。我们大家都要注意,有那么一天,攻你们一点。比如王云五在国民党时期当财政部长时,他说:“我没有研究过财政,还想学习。”结果人家就说:你没学,你就不能当财政部长。

现在是一场新的战争,向自然界开火,要革命球的命,从我们这里到杜勒斯那里,直径12,500公里,乘3.1416……,要大家努力,现在是革命尚未完成,同志仍需努力。我们不能老整风,整风后目标要转向技术革命,我们只能革地球表面的命,空间还不行,现在我们抛卫星还不行,要改造地球表面,实现第二个五年计划还差一点,实现第三个五年计划就差不多了。要认真学习,要搞试验田,农业要搞,工业也要搞。工厂的干部每礼拜一天,半天,真正当个学徒工,这有什么困难呀,文学也要学一点。你是科学家文学家也要学,由郭沫若当老师,过去我不看《人民日报》,像蒋介石不看国民党《中央日报》一样,现在《人民日报》七整八整好了一点。

政治思想革命还要革,不能松劲,技术革命现在不登报,一登有的就会说,整风不要整了。要坚持整风,一鼓作气,再而衰,三而竭。放松整风不利于社会主义,不利于民主党派,不利于改进工作。社会革命还要天天革,整风还要整,六个月可告一段落,并不是说可改造好了,以后还要整。

关于右派分子,我想开个右派分子大会,你们赞成不赞成?今天我们约了个右派分子参加会议,费孝遖来了吗?(应声;来了。)请费孝通参加会,我是寄希望于他,最高国务会议请右派分手参加这像什么样子啊,最高国务会议请右派分子参加不违犯宪法,因为宪法有规定,开最高国务会议,主席要请什么人就请什么人。右派分子做了好事,就是他们说了假话。对右派分子,第一要感谢,感谢他们向党进攻,引起了人民的愤怒,感谢右派是因为他们当了教员;第二,是帮助(监督)。所谓帮助,是三七开,十个人有七个人可以改造,逐步转变过来,经过五年到十年的时间,其中大部分能够转变过来的,规定时间,给以帮助,多数是有可能变好的。如不相信多数,就没有信心了。对人民的事业丧失信心是不对的。但总有一部分人不变,不变的人,只有带到棺材里去。像章、罗,要像鲁迅说的:“横眉冷对于夫指,俯首甘为孺子牛。”不变也好,有它的用处,它的用处就是不变。我们不怕它,因为它人数少。我们对右派的批判必须是全面的、深刻的,对右派分子的斗争是严肃的,但处理要宽大点,不要宽大无边,要给他们留条路,这是为了教育中间分子,也是为了教育他本人。现在的大学生,百分之七十到百分之八十是剥削家庭出身的,但右派只占百分之二到百分之三,对他们除个别的以外,都不开除学籍,用这种政策可以把他们改造过来。

再就是共产党大改革,说干什么,就干什么,说整风,就整风。整风就大鸣大放。整得不够就再整,民主党派也要改革,人的思想是可以改变,整个社会都变了嘛。

我主张不断革命论,你们不要以为是托洛茨基的不断革命论,革命就要趁热打铁,一个革命接着一个革命,革命要不断前进,中间不使冷场。湖南人常说:“草鞋无样,边打边像”。托洛茨基主张民主革命未完成就进行社会主义革命,我们不是这样。如一九四九年解放,接着搞土改,土改刚结束,就搞互助组,接着又搞初级社,然后又搞高级社。七年来就合作化了,生产关系改变了。随着就搞整风,趁热,整风以后,就搞技术革命。像波兰、南斯拉夫建立民主主义秩序,搞七、八年,出了富农。可以不建立新民主主义秩序,还要团结一切可能团结的力量。“长期共存,互相监督”还要有。民革有人说,民革的右派占百分之十二,十个指头有八个半是好的,当然不会有半个。十个人有一个是右派,那么还有九个不是右派,并且就是右派,也是批评从严,处理从宽。

去年七月我与费孝通谈,他说他那时才感到孤立。你(指费说)现在还孤立吗?(费答:孤立。)知识分子在某一方面来讲是没有知识的,对知识分子的骄傲自满应该批判,知识分子像孙行者一样,不要把尾巴翘得像旗杆那么高。罗隆基说:“小知识分子不能领导大知识分子。”我看工人阶级小知识分子领导大知识分子,这是条真理,工农出知识。除马克思、列宁是大知识分子外,我不算。费孝通到过英国,我就没有条件到英国。我去年讲过:皮之不存,毛将焉附?帝国主义、封建主义、官僚资本主义,这三张皮都剥掉了,知识分子的毛就要附在工人阶级这张皮上,有时沾上来了,有时沾上一点,有时在天宫中,梁上君子。我看知识分子要恭恭敬敬夹起尾巴向无产阶级学习,所谓(罗隆基说)“三颐茅庐”、“礼贤下士”、“士为知己者死”、“士可杀不可辱”、“温良恭俭让”都是封建的东西。我们一直讲知识分子要改造,七、八年都这样讲。知识分子一面说共产党英明领导,一面向我们进攻。英明领导,猖狂进攻,口喊“万岁!”进攻,喊万岁时,总有人在那里骂娘,同仇敌忾。接受共产党的领导,宪法规定,各党派也承认,但是还要搞两套。过去很多人不相信,现在很多人相信了。傅作义先生相信了吗?现在要帮助他们,要互相帮助,要公开讲,不要背地讲。什么要结束共产党的领导,搞阴谋,这不行。我们釆取和平改变(转变),国际上没有先例。三、五反是场严重的斗争,资产阶级工商业者,他们谨慎了,比较老实一点。但是知识分子还骄傲得很,一跳跳到一万公尺那么高,这须扑登跌一下,很必要,使他们受教育。我们要右派分子向人民投降,写降表,但他们写假降书是不行的。

在统一战线内部,不管共产党和民主党派,要互相帮助,要讲直话。要当面讲,不要背后讲,要去掉疑心,每个人要把心交给别人,不要隔张纸,你心里想什么东西,交给别人。鲁迅的作品很好,他把他的心与读者交流。不能像蒋介石那样做法,总是叫人不摸底。“逢人只说三分话,未可轻抛一片心”,这不适合今天的社会的。我有点东西就先卖出去。

我开了支票,在人代会上再讲讲,我这支票也不一定兑现,如果代表们有兴趣,就讲讲。还讲这套。

知识分子失败一次没有坏处。

我们当年红军有三十万人,走了二万五千里,剩下二万多人,蒋介石把我赶到山上。他没有料到,他办了好事。我当时一看蒋介石手里有枪,我也要有,我要从你手里拿枪,蒋委员长就当了运输队长。

一九二七年蒋介石清党,赶我们“上山为寇”。后来,就是抗日战争时期,我们要求当一家人,大公报王芸生写了个《不要另起炉灶》。我们请蒋委员长封官,就可以不另起炉灶,你得给饭吃嘛!我说得加个但是,要是不给饭吃,就另起炉灶,你不封,我就自己封自己,上山为寇,落草为王。

第二次王明路线,害得我们两只脚,走了两万五千里。陈独秀是右的,王明是“左”的。你们听说过吧,唐朝有个什么诗人写的诗:“一朝权在手,便把令来行。’

这一次是二万五千里长征,严重的挫折才教育了我们。

知识分子不受严重的挫折是教育不过来的。你们民主党派,民主,很高明,我过去就说过,共产党还出高岗、饶漱石,你们就没有,你们总以为我说这话是怕你们出奸臣,以为看你们不起(一人插话:没有。),啊,也许我是以小人之腹,度君子之心。我把心交给你们了,你们没有交给我。现在我抓住你们的小辫子了,摆在人民面前的右派就不少。我们都是旧社会来的人,在座的恐怕都是清朝人吧,我看这里我们清朝人占优势哟!全国人民已振奋起来,我们这些人要适应这种情况,适应六亿人民的要求,相信能适应这种情况。因为全中国人民都在进步,有一股热气,在这样的环境中生活是有利于进步的。

十五年赶上英国是可能的,要鼓起干劲,力争上游。我就是偏听偏信,看听信谁的。要节省,要反浪费。我们一面要提高生活,一面要节省,反对浪费。一万年也要节省。反浪费大有文章可作。作官可以,不要官气,以普通劳动者的姿态出现。主要干部要四个月离开北京,去求神拜佛,到工农群众中去。工农群众出钢铁,出粮食,弄点东西同来就加工,成为政策法令,不要以老爷姿态出现,你们看过“四进士”的戏没有?四进士的戏,有我们老毛家的一个毛朋,就是神气十足,巡按出朝,地动天摇。

劲,可鼓而不可泄。有了缺点错误,用大鸣大放的方法来纠正,不要泼冷水。有人批评好大喜功,那么能好小喜过吗?能重视过去,轻视将来吗?要好大喜功。要鼓励士气。

检查工作,一年四次,有些可以一年检查十二次,一年十二个月嘛!老鼠、麻雀、蚊子,一年检查十二次,看你干不干。

革命尚未全成,同志仍须努力。

政治和业务要配合,要又红又专。红讲的是政治,专讲的是业务,要红色的业务家,不能要白色的业务家。你说你不是白色的,那么是灰色的,也不行;不是灰色的,是桃红色的,也不行。搞政治的人,如只红而不那么专,红也不那么真红,是空头政治家。当然有些人情况不同,比如年龄大,……凡情况许可的人都可以专,同时要更加红起来。在我们这个国家要有几百万、上千万的知识分子。苏联知识分子就那么多。美国就搞他不赢,据说美国博士也有那么好弄的,当然也有是用功的,如杨振宁。

我们搞上层建筑的,不出原材料,要到外边去取,我们加工。

要改造右派,要帮助。要改革,这是激烈的改革,各民主党派要注意。

要把心交给人。

要釆取不断革命的方法。

公私合营,敲锣打鼓,黄炎老你没料到,我也没料到。抗战后,民主革命才三年半的时间就把蒋委员长赶到台湾,我也没料到。世界是变化的,两个卫星上了天,谁也没料到,我就根本不懂。现在那边很被动,我们这边很主动。过去苏联面上有灰,两个卫星上了天,脸上也光彩了。双轮双铧犁能用,我要为他恢复名誉而奋斗。什么合作化不行,四十条不行,双轮双铧犁也抹黑了,这跟斯大林一样倒霉。

不讲了,大家讨论讨论,提出意见。



\section[在最高国务会议上讲话要点(一九五八年一月二十八、三十日)]{在最高国务会议上讲话要点(一九五八年一月二十八、三十日)}
\datesubtitle{(一九五八年一月二十八)}


一、八年来第一次在一月讨论国家预算和国民经济计划,以后也要每年在这时候开会,这次人大要开的从容一些,多开小组会,大会可以少开,工作缺点看到的要加以批评,准备工作不太好,一方面开,一方面准备,文件可以在讨论后再修改,再发表。

二、我们这个民族,七、八年来看来是有希望,特别是去年一年,几亿人口经过大鸣大放大辩论,把许多问题搞清楚了,到处发扬了积极性,任务提得更恰当,如十五年在钢铁和其他主要工业方面赶上英国,多快好省,农业发展纲要40条修正,重新发布等。过去做不到的事现在可以做到了,过去没有办法的事,现在也有办法了,如除四害,全民族大有希望,悲观者不对。是大有希望,不是中有希望,小有希望,更不是没有希望,文章就在大字。

我们民族还在逐渐觉悟,因为觉了,打倒了帝国主义、封建主义、官僚资本主义,才进行了社会主义改造,才进行整风,反右派,中国又穷又白,穷就要革命,一张白纸好做文章,西方世界又富又文,他们就是太阔了,包袱甚重,资产阶级思想成堆。

现在生产与我们地位完全不相称,历史甚久,但钢铁生产比不上比利时,它有七百多万吨,我们只有五百二十万吨,群众热情甚好,它完全有把握十五年赶上英国,十五年看五年,五年看三年,三年看头年,头年看头月,更看前冬,去年中共三中全会就在水利、积肥上做了布置。现在群众热潮好像原子能,发出了热力,十五年后,要搞出四千万吨钢,五亿吨煤,四千万瓧电力,农业发展纲要40条,看起来,八年可以完成,为达到这个目的要有干劲,要鼓足勇气,力争上游。

三、有一个朋友说我们:“好大喜功,急功近利,轻视过去,迷信将来”。这几句话恰说到好处,“好大喜功”看是好什么大,喜什么功?是反动派的好大喜功,还是革命派的好大喜功?革命派里又有两种:是主观主义的好大喜功,还是符合实际的好大喜功?我们是好六万万人之大,喜社会主义之功。“急功近利”看是否搞个人突中,是否搞主观主义,还是符合实际,可以达到的平均先进定额。过去不轻视不行,大家每天都想禹、汤、文、武、周公、孔子是不行的,对过去不能过于重视,但不是根本不要,外国的好东西要学,应该保存的古董也要保存,南京、济南、长沙的城墙拆了很好。北京、开封的旧房子最好全部变成新房子,“迷信将来”,人人都是如此,希望寄托在将来,这四句话提得很好。

还有一个右派说我:“好大喜功,偏听偏信,喜怒无常,轻视古董”,“好大喜功”前面已说过。偏听偏信,不可不偏,我们不能偏听右派的话,要偏听社会主义之言,君子群而不党,没有此事,孔夫子杀少正卯,就是有党。“喜怒无常”,是的,我们只能喜好人,当你当了右派时,我们就喜不起未了,就要怒了。“轻视古董”,有些古董如小脚、太监、臭虫等,不要轻视吗?

四、人多好还是人少好?现在还是人多好,目前农民还不注意节育,恐怕要到七亿人口时,人们才会紧张,要看到严重性,但不要怕,要节省,一方面节育,一方面节省,要成为风气。

五、工作方法有两种:一种比较做得好些,一种做得比较差些,也就是两种作风,譬如合作化,一种搞得快些,一种拖到七,八年才搞。我看趁热打铁,一气呵成为好,整风中大鸣大放很好,这是右派发明末后我们搞的,现在全民中用大鸣大放来整风了。

官风、官气要打掉,最好根除。像除四害一样,官风、官气也是一种迷信。要破除迷信,部长也好,总理也好,只能以一个普通劳动者的姿态在人民中出现,要使普通劳动者感到在我们面前是平等的,自己感觉平等是靠不住的,要使对方感觉平等,湖北红安县的干部,1956年上半年官气十足,农民很不高兴,下半年他们改了,穿草鞋到乡下去,农民很欢迎,山东干部下放农村,农民说:“八路军又来了”,这几年官气大长,共产党要改,各党派也要改,共产党的负责人,除了病老以外,每年要有四个月的时间离开北京,向劳动人民取经,回来加工制造,这样可以打掉官气。各党派和民主人士酌情办理,身体不行的可不去。北京不在地方好坏,而在中央机关不产生任何东西,即不生产任何东西,中央只是加工厂,一切原料出自工人、农民那里,我在北京住久了就觉得脑子空了,一出北京就有了东西。

六、劲可鼓而不可泄,有时没有注意,给群众以挫折。一个时期一些问题上发生了错误,如合作社曾有人说搞多了要砍掉十万个,双轮双铧犁在南方名誉好,举登徒子好色为例,宋玉攻其一点,不及其余,这个办法不好。右派就用这个办法攻击我们的,但好人有时也这样看,共产党也有这样的人。共产党也好,民主党派、工商界、知识分子也好,多数人是可以进步的,就是右派,多数也是可能变好的,如不相信多数,就没有信心了,对人民的事业丧失信心是不好的。现在的大学生,百分之七十至八十是剥削家庭出身的,但右派只占百分之二至三,对他们除个别的以外都不开除学籍,用这种政策,可以把他们改造过来。

七,现在是一场新的战争,向自然界开火。“革命尚未成功,同志仍需努力”。要革地球的命,现在我们只能革地球表面的命,在整风以后,要准备把注意力逐渐引向技术革命,要认真学习,搞试验田,到工厂当学徒,要学自然科学、技术科学、社会科学、文学等。但社会革命还要天天革,整风还要整,不能松劲,六月可告一段落,但并不是说改造好了,将来还要整。

要讲不断革命论,解放后搞土改,土改后搞互助组、合作社。一九五六年是公私合营和手工业合作化,接着五七年搞整风,再接着就要搞技术革命,一个接一个,趁热打铁,中间不使冷场,在这里,要团结一个可能团结的人。

八、共产党准备大改。整风和反省,各党派也可以搞,现在已在搞,有很大的成绩。人的思想是可以改变的,作风也可以改变的。全国人民已振奋起来,我们这些人要适应这种情况,适应六亿人民的要求。相当能适应这种情况,各党派在进步,整风在继续,但不要勉强。要把事情搞好,把人整好,不是整坏,整风对共产党要求严格,对民主党派不要太严格了。不太严格不是不整,整整也好,试试看。目的是整得适合人民要求,把人整好不是整坏,相信会整得更好,因为全中国人民都在进步,有一股热气,在这样环境中生活,是有利于进步的。

很值得高兴,民主党派主要负责人成右派的不多,参加最高国务会议的人成右派还不到十人,但也给了我们以教训,去年四月三十日的最高国务会议上,我们说过资产阶级的知识分子要改造。“皮之不存、毛将焉附”,知识分子要附到工人阶级的皮上来,否则变成梁上君子,但章伯钧、罗隆基等听不进去。他们要取消学校党委制,要同共产党轮流坐庄。他们很高兴“长期共存”,但他们变成“短期共存”了。

口头喊万岁,切记不要都信,有些人大喊万岁,接受领导,但实际上却猖狂进攻。

在统一战线内部,不管共产党和民主党派,要互相帮助,要讲直话,要去掉疑心。要将心交给人家,要当面讲,不要在后面讲。“逢人且说三分话,不可全抛一片心”,这是旧社会的话,现在不适用,逐步做到说真话。

九、对资产阶级知识分子,我们总是要说改造,从未说不要改造,知识分子要向劳动人民投降,知识分子在某一点说是最无知识,知识分子不失败一次,不会翻身。我们党失败过多次,从右的和“左”的两大错误中取得了教训,就全面了,民主党派不见得更高明,中共出了高、饶,你们就没有?我们都是旧社会来的,人要经过严格考验,才能取得教训。

政治和业务要配合,要又红又专,红是政治、专是业务,不红只专是白色专家,搞政治的,如只红不专,不熟悉业务,不懂得实际,红是假红,是空头政治家。搞政治的,要钻业务,搞科技的要红起来。十五年赶上英国,要有成百万上千万忠于无产阶级的知识分子。

十、要开一个右派分子大会,在大会中,第一向他们致感谢,第二想帮助他们。所谓感谢,是指他们向工人和党进攻,当了教员。帮助他们,是想在其中使五成到七成的人,经过五年到十年时间,逐步变过来,为人民服务。总有不变的人,即使如此,也有用处,用处就是在他不变,容许社会上有一部分人,不变不强迫,对右派批判必须严肃、深刻、全面,处理要比较宽大,当然宽大无边是不好的,要有处分,但要留一条路让他们走。第一是为了许多思想上还未解决问题的中间分子,第二是为了这些右派本身,使他们有可能间到人民的队伍里来,当然首先要他们自己下决心,但是还要我们帮助。

右派大会要开,那一天开,要研究,不只是北京开,各地也要开,先开小的,然后开大的。

(注:为便利阅读,把前后两次谈话按问题整理在一起,问题排列次序也略有变动,项目也是记录者所加——中央统战部)



\section[在中央政治局会议上讨论教育工作时的指示(一九五八年一月三十一日)]{在中央政治局会议上讨论教育工作时的指示}
\datesubtitle{(一九五八年一月三十一日)}


学生健康不好的原因是伙食不好,卫生不好,功课重,课外负担过重,太忙。要增进学生健康,要增加营养,要搞好卫生,减少负担,少紧张些,要吃的饱。学得太多,可以少学一点,要克服忙的现象。要一面增加收入,一面减少消耗。因此,要增加助学金,改善伙食,另方面要克服忙乱现象。



\section[反浪费反保守是当前整风运动的中心任务(一九五八年二月八日)]{反浪费反保守是当前整风运动的中心任务}
\datesubtitle{(一九五八年二月八日)}


整风运动在全国的企业、事业单位和国家机关里,目前出现了一个新的洪峰,这就是以反浪费和反保守为中心掀起了一个新的鸣放高潮和整改高潮。在反浪费反保守的大鸣大放中,中央各国家机关内贴出了二十五万张大字报;北京市三十一个企业三十天的统计,职工们就贴了三十万张大字报,提出了四十三万条意见。运动声势浩大,锋芒集中在一个方向,贯彻多快好省勤俭建国的方针,促进生产和工作的大跃进。

从各地区各企业和各机关的情况看来,这次的反浪费反保守运动,同过去历次的增产节约运动有很大的不同。这次运动实际上已经成为反对思想、政治、经济各方面落后现象的斗争,已经形成了广泛地比先进,比多快好省的高潮。在这个波澜壮阔的运动中,很多束缚群众积极性和生产力发展的陈规被冲破了,很多长期不能解决的根本性问题顺利的解决了,各方面的生产和工作已经有了明显迅速的改进。一月份国民经济计划执行情况就是一个很好的说明。历年来,一月份生产和基建计划总是完成的最不好的,年初松,年中紧,年底赶,几乎成了一个定例。而今年却一反积习,一月份的工业总产值越额百分之二点五完成了月计划。基本建设的情况也比过去任何一年都好。再如商业部门中的反浪费反保守运动,虽然还开始不久,某些先进单位却已经在广大职工觉悟充分提高的基础上解决了许多长期没有彻底解决的问题。北京天桥百货商场在反浪费反保守运动中大胆地突破常规,提出了并且实现了减少人员,节约流动资金,改善服务态度的措施,并且纠正了在商业企业中机械地形式主义地搬用在工业企业中工作八小时的现象,实行了一班到底的工作制度。

这个声势浩大的运动,显然是一九五七年我国人民在思想战线、政治战线上的社会主义革命的产物。正因为这样,群众在这个运动中决不满足于克服生活中的铺张浪费,也决不满足于仅仅要求产量指标的突破。许多企业在辩论了浪费的性质、原因和如何堵塞漏洞等问题以后,得出了一个共同的结论:造成浪费的责任应该由领导工作人员、技术人员和工人三方面担负,这三方面都必须同时改进。领导工作人员往往有官僚主义、主观主义、不深入地钻研业务的毛病;技术人员往往是重业务不重政治,墨守陈规,不善于发动和依靠群众的积极性创造性;工人群众中也有许多人没有正确对待个人和国家的关系,没有正确解决为谁劳动的问题。在许多单位的辩论会上,三方面的人都同时揭发和批判了自己的缺点,这样就打掉了官气、暮气和邪气,资产阶级思想受到抵制,无产阶级思想大大抬头。

以反浪费反保守为纲,带动了各方面的工作,这就是当前整风运动的显著特征。我国,全民性的政治战线上和思想战线上的社会主义革命,其最终目的本来就是为了要把社会主义各方面的建设工作大大推进一步。经过前两个阶段的大争大辩,群众的觉悟大大提高了。在十五年赶上英国和苦战波三年,改变面貌的伟大号召的鼓舞下,群众不能不要求生产和工作的大跃进,不能不反浪费反保守。灿烂的思想政治之花,必然结成丰满的经济之果。这是完全合乎规律的发展。有些单位对于这个形势认识不足,在运动中忽视思想工作,只算经济账,简单地从技术上采取一些措施,而不认真开展群众性的争辩,不彻底转变工作方法和领导作风。这样他们就不能从根本上杜绝浪费现象,克服保守主义,引导生产的大跃进。因此,目前的斗争既然是一个经济上的斗争,同时又是一个思想政治的斗争,既要算经济账,又要算思想账、政治账。通过大鸣大放,大争大辩,不但要反掉浪费,反掉保守,而且要反掉官僚主义、宗派主义和主观主义。要通过和结合反浪费反保守的斗争.彻底改进干部和群众的关系。提高全体职工群众的社会主义觉悟,打破那些妨碍生产力迅速发展的陈规,精简机构。改善生产管理和劳动组织,改进生产技术,降低生产费用.以便贯彻执行多快好省的方针,促进生产的大跃进。

有一些人虽然认识了思想工作的重要,但是他们采取的办法却是错误的。他们组织群众去抽象他讨论一些原原则性题,结果在辩论会上,群众往往不知所云。当然,重大的原则问题,例如个人和集体,自由和纪律,工农关系,工人阶级的领导地位等等,都必须在群众中辩论清楚。但是在目的阶段,这种辩论必须针对生产和工作具体任务。很多企业.通过反浪费反保守的具体辩论,引导广大群众认清了上述原则,并且也边辩边改,立即见诸行动。拿这种作法同前种作法相比,岂不生动得多,深刻得多吗?我们说以反浪费反保守为纲,首先就是说要以它作为当前整改阶段的纲,通过它末完成当前的整改任务。决不能把这两件事分割开来,如果抛去群众最关心的问题不管,不去因势利导,从解决具体问题中去解决思想,那就必然会失败。

许多企业、学校和机关已经决定,要把反浪费反保守运动作为整改阶段的中心,这是正确的。希望全国所有企业、学校和机关都向他们看齐,争取整风运动的这个新任务的彻底胜利,从而使我国社会主义事业实现一个全面的大跃进!

<p align="right">(一九五八年二月八日《人民日报》社论)</p>



\section[工作方法六十条(草案)(一九五八年一月三十一日)]{工作方法六十条(草案)}
\datesubtitle{(一九五八年一月三十一日)}


我国人民在共产党领导下,一九五六年在社会主义所有制方面取得了基本的胜利。一九五七年发动整风运动,又在思想战线和政治战线方面取得了基本的胜利。就在这一年.又超额完成了第一个五年建设计划。这样。我国六亿多人民就在共产党领导下,认清了自己的前途,自己的责任,打击了从资产阶级右派方面刮起米的反党反人民反社会主义的妖风。同时也纠正了和正在继续纠正党和人民自己从旧社会带来的由于主观主义造成的一些缺点和错误。党是更加团结了,人民的精神状态是更加奋发了,党群关系大为改善。我们现在看见了从来没有看见过的人民群众在生产战线上这样高涨的积极性和创造性。全国人民为在十五年或者更多一点时间内在钢铁和其他主要工业产品方面赶上或者超过英国这个口号所鼓舞。一个新的生产高潮已经和正在形成。为了适应这种情况,中央和地方党委的工作方法有做某些改变的需要。这里所说的几十条并不都是新的。有一些是多年积累下来的,有一些是新提出的。这是中央和地方同志,一九五八年一月先后在杭州会议和南宁会议上共同商量的结果,这几十条,大部分是会议上同志们的发言启发了我,由我想了一想写成的。一部分是直接纪录同志们的意见;有一个重要条文(关于规章制度)是由×××和地方同志商定而由他起草的,由我直接提出的只占一部分。这里讲的也不完全是工作方法,有一些是工作任务,有一些是理论原则,但是工作方法占了主要地位。我们现在的主要目的,是想在工作方法方面求得一个进步,以适应已经改变了的政治情况的需要。这几十条现在只是建议,还待征求意见。条文或者减少,或者要增加,都还未定。请同志们加以研究,提出意见,以便修改,然后提交政治局批准,方能成为一个正式的内部文件。
<p align="right">毛泽东
一九五八年一月三十一日</p>

(一)县以上各级党委要抓社会主义建设工作。这里有十四项;

1.工业;2.手工业;3.农业,4.农村副业;5.林业;6.渔业,7.畜牧业;8.交通运输业;9.商业;10.财政和金融;11.劳动、工资和人口;12.科学;13.文教;14.卫生。

(二)县以上各级党委要抓住社会主义工业工作。这里也有十四项:1.产量指标;2.产品质量;3.新产品试制;4.新技术;5.先进定额;6.节约原材料,找寻和使用代用品;7。劳动组织、劳动保护和工资福利;8.成本;9.生产准备和流动资金;lO.企业的分工和协作;11.供产销平衡;12.地质勘探,13.资源综合利用;14.设计和施工。这是初步拟的项目,以后应该逐步形成工业发展纲要“四十条”。

(三)各级党委要抓社会主义农业工作。这里也有十四项。1.产量指标;2.水利;3.肥料;4.土壤;5.种子;6.改制(改变耕作制度,如扩大复种面积,晚改早,早改水等);7。病虫害;8.机械化(新式农具、双轮双铧犁、抽水机、适合中国各个不同区域的拖拉机及用摩托开动的运输工具等);9。精耕细作,10.畜牧;11.副业;12.绿化;13。除四害;14。治疾病讲卫生。这是从农业发展纲要四十条中抽出来的十四个要点,四十条必须全部施行。抽出一些要点目的在于有所侧重。纲举目张,全网自然提起来了。

(四)全面规划,几次检查,年终评比。这是三个重要方法。这样一来,全局和细节都被掌握了,可以及时总结经验,发扬成绩,纠正错误;又可以激励人心,大家奋进。

(五)五年看三年,三年看头年,每年看前冬。这是一个掌握时机的方法。时机上有所侧重,把握就更大了。

(六)一年至少检查四次。中央和省一级,每季要检查一次,下面各级按情形办理。重要的任务在没有走上轨道之前,要每月检查一次。这也是掌握时机的方法,是就一年内说的。

(七)如何评比?省和省比,县和县比,社和社比,厂和厂比,矿和矿比,工地和工地比。可以订评比公约,也可以不订。农业比较易于评比。工业可以根据可比的条件评比,按产业系统评比。

(八)什么时候交计划?省、自治区、直属市、专区、县都要按照三个十四项订出计划。订计划时要有重点,不可在同一时期内百废俱兴。区、乡、社的计划内容主要就是农业十四项。项目可以根据当地情况有所增减。先订五年的计划,可以是粗线条的。一九五八年七月一日以前交卷。计划要逐级审查。为了便于比较,省委要在县、区、乡、社的计划中选一些最好的和少数最坏的送给中央审查。省和专区的计划都要按期交中央,一个也不能少。

(九)生产计划三本账。中央两本账,一本是必成的计划,这一本公布,第二本是期成的计划,这一本不公布。地方也有两本账,地方的第一本就是中央的第二本,这在地方是必成的。第二本在地方是期成的。评比以中央的第二本账为标准。

(十)从今年起,中央和省、市、自治区党委要着重抓工业,抓财经贸易。一年要抓四次,主要是七月(或八月)、十一月、一月(上旬)三次。再不抓,十五年赶上英国的口号可能落空。要把工业部门和财贸部门的若干主要负责干部带到讨论地方工作的会场上去,中央的带到地方去,省、直属市和自治区的带到专区、市属区和县里去。许多在中央工作的同志和地方工作的同志都有这种要求。

(十一)各地方的工业产值(包括中央下放的厂矿,原来的地方国营工业和手工业的产值,不包括中央直属厂矿的产值)。争取在五年内,或者七年内,或者十年内,超过当地的农业产值。各省市对于这件事要立即着手订计划,今年七月一日以前订出来。主要的任务是使工业认真地为农业服务。大家要切实摸一下工业,做到心中有数。

(十二)在今后五年内,或者六年内,或者七年内,或者八年内,完成农业发展纲要四十条的规定。各省委、直属市委、自治区党委对于这个问题应当研究一下。就全国范围来看,五年完成四十条不能普遍做到,六年或者七年可能普遍做到,八年就更加有可能普遍做到。

(十三)十年决于三年。争取在三年内大部分地区的面貌基本改观。其他地区的时间可以略为延长,口号是苦战三年。方法是放手发动群众,一切经过试验。

(十四)反对浪费。在整风中,每个单位要以若干天功夫,来一次反浪费的鸣放整改。每个工厂、每个合作社、每个商店、每个机关、每个学校、每个部队都要进行一次认真的反浪费斗争。今后每年都要反一次浪费。

(十五)在我国的国民经济中,积累和消费的比例怎样才算恰当,这是一个关系我国经济发展迅速的大问题,希望大家研究。

(十六)关于农业合作社的积累和消费的比例问题也需要研究。湖北的同志有这样的意见:以一九五七年生产和分配的数字为基础,以后的增产部分四六分(即以四成分配给社员,六成作为合作社积累)。对半分、倒四六分(即以四成作为合作社积累,六成分给社员)。如果生产和收入已经达到当地富裕中农水平的,可以在经过鸣放辩论取得群众同意以后,增产的部分三七分(即以三成分配给社员,七成作为合作社积累),或者一两年内暂时不分,以便增加积累,准备生产大跃进,这个意见是否适当,请各地讨论。


(十七)集体经济和个体经济的矛盾需要解决,需要定出一个适当的比例。现在的情况是有的地方,有些农家的收入中,个体经济和集体经济的比例是倒四六,倒三七(即是家庭付业和经济自留地的收入,占到总收入的百分之六十、七十)。这种情况,必然影响农民对于社会主义集体经济的积极性。这种情况应当改变。各省可以经过鸣放辩论,研究出控制的办法,对经济关系做适当调整,在鼓励农民生产积极性和全面发展生产的基础上,使农家的收入中,个体经济和集体经济的比例,在几年内逐步达到三比七或者二比八(即是农民从合作社得到的收入占家庭总收入的百分之七十或者八十)。

(十八)普遍推广试验田。这是一个十分重要的领导方法。这样以来,我党在领导经济方面的工作作风将迅速改观,在乡村是试验田,在城市可以抓先进的厂矿、车间和工区、工段,突破一点就可以推动全面。

(十九)抓两头带中间。这是一个很好的领导方法。任何一种情况都有两头,即是有先进和落后,中间状态又总是占多数。抓住两头就可以把中间带动起来了。这是一个辩证的方法。抓两头,抓先进和落后,就是抓住了两个对立面。

(二十)组织干部和群众对先进经验的参观和集中地展览先进的产品的做法,是两项很好的领导方法。用这些方法,可以提高技术水平.推广先进经验,鼓励互相竞赛。许多问题到实地一看就解决了。社和社、乡和乡、县和县、省和省之间,都可以组织互相参观,中央、省、市、专区和县都可以举办生产建设展览会。

(二十一)不断革命。我们的革命是一个接一个的。从一九四九年在全国范围夺取政权开始,接着就是反封建的土地改革,土地改革一完成就开始农业合作化,拨着又是私营工商业和手工业的社会主义改造。社会主义三大改造.即生产资料所有制方面的社会主义革命,在一九五六年基本完成,接着又征去年进行政治战线上和思想战线上的社会主义革命。这个革命在今年七月一日以前可以基本上告一段落。但是问题没有完结,今后一个相当长的时期内每年都要用鸣放整风的方法继续解决这一方面的问题。现在要求一个技术革命,以便往十五年或者更多的一点时间内赶上和超过英国。中国经济落后,物质基础薄弱,使我们至今还处在一种被动状态。精神上感到还是受束缚。在这方面我们还没有得到解放。要鼓一把劲。再过五年,就可以比较主动一些了。十年后将会更加主动一些。十五年后粮食多了,钢铁多了,我们的主动就更多了。我们的革命和打仗一样.在打了一个胜仗之后,马上要提出新任务。这样就可以使干部和群众经常保持饱满的革命热情,减少骄傲情绪,想骄傲也没有骄傲的时间。新任务压来了,大家的心思都用在如何完成新任务的问题上面去了。提出技术革命就是要大家学技术、学科学。右派说我们是小知识分子.不能领导大知识分子。还有人说要对老干部实行“赎买”,给点钱叫他们退休,因为老干部不懂科学,不懂技术,只会打仗,搞土改。我们一定要鼓一把劲,一定要学习并且完成这个历史所赋予我们的伟大的技术革命。这个问题要在干部会议中议一议,开个干部大会,议一议我们还有什么本领。过去我们有本领会打仗会搞土改.现在仅仅有这些本领就不够了,要学新本领,要真正懂得业务,懂得科学和技术。不然就不可能领导好。我在一九四九年所写的《论人民民主专政》里曾经谈过。“严重的经济建设任务摆在我们面前。我们熟习的东西有些快要闲起来了。我们不熟习的东西正在强迫我们去做。这就是困难。”“我们必须克服困难.我们必须学会自己不懂的东西。”时间过去了八年.这八年中,革命一个接着一个,大家的思想都集中在那些问题上,很多人来不及学科学、学技术。从今年起,要在继续完成政治战线上和思想战线上的社会主义革命的同时,把党的工作的着重点放在技术革命上去。这个问题必须引起全党注意。各级党委可以在党内事先酝酿,向干部讲清楚,但是暂时不要在报上宣传。到七月一日以后我们再大讲特讲。因为那时候基层整风已经差不多了。可以把全党的主要注意力移到技术革命上面去了,注意力移到技术方面,又可能忽略政治。因此必须注意把技术和政治结合起来。

(二十二)红与专、政治与业务的关系.是两个对立的统一。一定要批判不问政治的倾向。一方面要反对空头政治家,另一方面要反对迷失方向的实际家。

政治和经济的统一,政治和技术的统一。这是毫无疑义的,年年如此,永远如此。这就是又红又专。将来政治这个名词还是会有的,但是内容变了。不注意思想和政治,成天忙于事务,那会成为迷失方向的经济家和技术家,很危险。思想工作和政治是完成经济工作和技术工作的保证,它们是为经济基础服务的,思想和政治又是统帅、是灵魂。只要我们的思想工作和政治工作稍为一放松,经济工作和技术工作就一定会走到邪路上去。

现在一方面有社会主义世界同帝国卞义世界的严重的阶级斗争;另一方面,就我国内部来说,阶级还没有最后消灭,阶级斗争争还是存在的。这两点必须充分估计到。同阶级敌人作斗争,这是过去政治的基本内容。但是自人民有了自己的政权以后,这个政权同人民的关系就基本上是人民内部的关系了,采用的方法不是压服而是说服。这是一种新的政治关系。这个政权只对人民中破坏正常社会秩序的犯法分子采取暂时的程度不同的压服手段,作为说服的辅助手段。在由资本主义到社会主义的过渡时期,人民中还隐藏一部分反社会主义的敌对分子,例如资产阶级右派分子,对这种人,我们基本上也是采取由群众鸣放辩论的方法去解决问题。只对严重反革命、破坏分子采取镇压的手段。过渡时期完结,彻底消灭了阶级之后,单就国内情况来说,政治就完全是人民内部的关系。那时候,人和人的思想斗争,政治斗争和革命一定还会有的,并且,不可能没有。对立统一的规律,量变质变的规律,肯定否定的规律,永远地普遍地存在。但是斗争和革命的性质和过去不同,不是阶级斗争,而是人民内部的先进和落后之间的斗争,社会制度的先进和落后之间的斗争,科学技术的先进和落后之间的斗争。由社会主义过渡到共产主义是一场斗争,是一个革命。进到共产主义时代了,又一定会有很多很多的发展阶段,从这个阶段到那个阶段的关系,必然是一种从量变到质变的关系。各种突变、飞跃,都是一种革命,都要通过斗争。“无冲突论”是形而上学的。

政治家要懂些业务。懂得太多有困难,懂得太少也不行,一定要懂得一些。不懂得实际的是假红,是空头政治家。要把政治和技术结合起来,农业方面是搞试验田,工业方面是抓先进典型,试用新技术,试制新产品。这些都是用的“比较法”,在相同条件下,拿先进和落后此,促进落后赶上先进。先进和落后是矛盾的两个极端,“比较”是对立的统一。企业和企业之间,企业内部车间和车间、小组和小组、个人和个人之间,都是不平衡的。不平衡是普遍的客观规律。从不平衡到平衡,又从平衡到不平衡,循环不已,永远如此。但是每一循环,都进到高的一级。不平衡是经常的,绝对的,平衡是暂时的、相对的。我国现在经济上的平衡和不平衡的变化,是在总的量变过程中许多部分的质变。若干年后,中国由农业国变成工业国,那时候完成一个飞跃,然后再继续量变的过程。

评比不仅此经济、比生产、此技术,还要此政治,就是比领导艺术。看谁领导的比较好些。

(二十三)上层建筑一定要适合经济基础和生产力发展的需要。政府各部门所制定的各种规章制度是上层建筑的一部份。八年来积累起来的规章制度许多还是适用的,但是有相当一部分已经成为进一步提高群众积极性和发展生产力的障碍,必须加以修改,或者废除。在修改或者废除这些不合理的规章制度方面,最近一个时期,在群众中间,已经创造了许多先进经验,例如:石景山发电厂改进职工福利待遇的办法;浙江机械制造厂改进职工宿舍制度的办法;江苏戚墅堰发电厂改进奖金的办法。江西省一级几个商业机关合并为一个机关,由总数二千四百多人缩减为三百五十人即减少七分之六的人员。应该作出一个总的规定,即是在多、快、好、省地按计划按比例地发展社会主义的前提下,在群众觉悟提高的基础上,允许并且鼓励群众的那些打破限制生产力发展的规章制度的创举。

中央各部门,各省、市、自治区党委,应该派遣负责同志到各地的基层单位去,总结群众中的这一类先进经验,发展下层单位和群众的这一类有利于社会主义建设的创举,建议主管机关给以批准,停止原有的规章制度中某些规定在这个单位实行,并且把这个单位的先进经验推广到其他单位试行。

中央各部门、各省、市、自治区党委.还应当派遣负责同志到各地的基层单位去,发现那里有什么规章制度已经限制了群众积极性的提高和生产力的发展。根据那里的实际情况,通过基层党委和群众的鸣放辩论,保存现有规章制度中的合理部分,修改或者废除其中的不合理部分。并且拟定一些新的适合需要的规章制度,在这个单位实行,也可以推广到其他单位试行。

中央各部门、各省、市、自治区党委,应该系统地总结这方面的典型的成熟的先进经验;重大的和全国性的,经过党中央和国务院批准。地方性的,经过相应的地方党委和政府批准。技术性和专业性的,经过主管部门批准。然后在全国或者全省的相同的所有单位中普遍推行。经过一段时间实行以后。在必要的时候,再根据新的经验修改或者重新制定各种规章制度。

这是制定和修改各种规章制度的群众路线的方法。

(二十四)一定要把整风坚持到底。全党要鼓足干劲,打掉官风实事求是,同人民打成一片。尽可能纠正一切工作上、作风上、制度上的缺点和错误。

(二十五)中央和省、直属市、自治区两级党委的委员,除了生病的和年老的以外,一年一定要有四个月时间轮流离开办公室,到下面去作调查研究,开会到处跑,应当采取走马看花,下马看花两种方法。那怕到一个地方谈三、四个小时就走也好,要和工人农民接触,要增加感性认识,中央的有些会议可以到北京以外的地方去开,省委的有些会议可以到省会以外的地方去开。

(二十六)以真正平等的态度对待干部和群众,必须使人感到人们互相间的关系确实是平等的,使人感到你的心是交给他的。学习鲁迅,鲁迅的思想是和他的读者交流的,是和他的读者共鸣的。人们的工作有所不同,职务有所不同,但是任何人不论官有多大,在人们中间都要以一个普通的劳动者的姿态出现,决不许可摆架子,一定要打掉官风。对于下级所提出的不同意见,要能够耐心听完,并且加以考虑,不要一听到和自己不同的意见就生气,认为是不尊重自己。这是以平等的态度待人的条件之一。

(二十七)各级党委,特别是坚决站在中央正确路线方面的负责同志,要随时准备挨骂,人们骂得对的,我们应当接受和改正。骂得不对的,特别是歪风,定要硬着头皮顶住,然后加以考查,进行批判,在这种情况下决不可以随风倒,要有反潮流的大无畏的精神。这一点,我们已经在一九五七年受到了考验。

(二十八)在省、地、县三级或者在省、地、县、乡四级的干部会议上,讨论一次党的领导原则问题。讨论一下这些原则是否正确。“大权独揽,小权分散。党委决定,各方去办。办也有决。不离原则。工作检查,党委有责。”这儿句话中,关于党委的责任,是说大事首先由它作出决定,并且在执行过程中加以检查。“大权独揽”是句成语,习惯上往往是指个人独断。我们借用这句话,指的却是主要权力应当集中于中央和地方党委的集体,用以反对分散主义。难道大权可以分揽吗?这八句歌诀产生于一九五三年,就是为了反对那时的分散主义而想出来的。所谓“各方去办’不是说由党员径直去办,而是一定要经过党员在国家机关中,在企业中。在合作社中,在人民团体中,在文化教育机关中,同非党员接触、商量、研究。对不妥当的部分加以修改,然后大家通过,各方去办。第三句话里所说的原则指的是:党是无产阶级组织的最高形式。民主集中制,集体领导和个人作用的统一(党委和第一书记的统一),中央和上级的决议。

(二十九)是否事事都要过问第一书记?可以不必。大事一定要问。要有二把手,三把手,第一书记不在家的时候,要另外有人挂帅。

(三十)党委要抓军事。军队必须放在党委的领导和监督之下,现在基本上也是这样做的。这是我军的优良传统。作军事工作的同志是要求中央和地方抓这项作工的。只是忙于社会改革和经济建设工作,近几年来我们抓得少了一些。现在应当改善这种情况。办法也是一年抓几次。

(三十一)大型会议,中型会议和小型会议,都是必要的。各地和各部门要好好安排一下。小型会议,参加几个人,一、二十人,便于发现问题和讨论问题。上千人参加的大型会议,只能采取先做报告后加讨论的办法,这种会不能太多,每年两次左右。小型中型会议每年至少要开四次。这种会最好到下面去开。省委可以到地委召开一个地区或者相近几个地区的县书记会议。中央同志和国务院各部门可以轮番到地方开些小型会议。各个经济协作区有事就开会,每年至少开四次。

(三十二)开会的方法应当是材料和观点的统一。把材料和观点割断,讲材料的时候没有观点,讲观点的时候没有材料。材料和观点互不联系,这是很坏的方法。只提出一大堆材料,不提出自己的观点,不说明赞成什么,反对什么,这种方法更坏.要学会用材料说明自己的观点。必须要有材料.但是一定要有明确的观点去统帅这些材料。材料不要多,能够说明问题就行,解剖一个或几个麻雀就够了。不需要很多,自己应当掌握丰富的材料,但是在会上只需要拿出典型的。必须懂得开会同写大著作是有区别的。

(三十三)一般说来,不要在几小时内使人接受一大堆材料,一大堆观点,而这些材料和观点又是人们平素不大接触的。一年要找几次机会,让那些平素不接触本行业务的人们,接触本行业务。给以适合需要的原始材料或半成品,不要在一个早晨突如其来的把完成品摆到别人面前。要下些毛毛雨。不要在几小时内下几百公厘的倾盆大雨。“强迫受训”的制度必须尽可能废除。“强迫签字”的办法必须尽可能减少。要彼此有共同的语言,必须先有必要的共同的情报知识。

(三十四)十个指头问题。人有十个指头,要使干部学会善于区别九个指头和一个指头,或者多数指头和少数指头。九个指头和一个指头有区别,这件事看来简单,许多人却不懂得,要宣传这种观点。这是大局和小局、一般和个别、主流和支流的区别。我们要注意抓住主流,抓错了一定要翻跟斗。这是认识问题,也是逻辑问题。说一个指头和九个指头,这种说法比较生动。也比较合于我们工作的情况。我们的工作,除非发生了根本路线上的错误,成绩总是主要的.但是这种说法对于某些人却不适用。例如右派分子。许多极右分子,那是几乎十个指头都烂了。学生中的大部分普通右派分子也不只烂了一个指头,但又不是全烂了,所以还可以留在学校里。

(三十五)“攻其一点或几点,尽量夸大,不及其余。”这是一种脱离实际情况的形而上学的方法。一九五七年资产阶级右派分子向社会主义猖狂进攻,他们用的就是这种方法。我党在历史上吃过这种方法的大亏。这就是教条主义占统治地位的时期。立三路线也是如此。

修正主义,或者右倾机会主义,也用这种方法。陈独秀路线和抗日时期的王明路线,就是如此。一九三四年,张国焘也用过这种方法。一九五三年高岗、饶漱石反党联盟,用的也是这种方法。我们应当总结过去的经验,从认识论和方法论上加以批判,使干部觉醒起来。以免再吃大亏。好人犯个别错误的时候,也会不自觉的采用这种方法,所以好人也要研究方法论。

(三十六)概念的形成过程,判断的过程,推理的过程,就是调查和研究的过程。就是思维的过程。人脑是能够反映客观世界的,但是要反映得正确很不容易。要经过反复的考察才能反映得比较正确,比较接近客观实际。有了正确的思想和正确的观点,还是比较恰当的方法,表达告诉别人。概念、判断的形成过程,推理的过程,就是“从群众中来”的过程把自己的观点和思想传达给别人的过程,就是“到群众中去”的过程。在我们的干部中,大概还有不少人,不明白这一个简单的真理:任何英雄豪杰,他的思想、意见、计划、方法只能是客观世界的反映。其原料或半成品只能来自人民群众的实践中,或者自己的科学实践中,他的头脑只能作为一个加工厂而起制成完成品的作用,否则是一点用处也没有的。人脑制成的这种完成品,究竞合用不合用,正确不正确,还得交由人民群众去检验。如果我们的同志不懂得这一点,那就一定会碰钉子的。

(三十七)文章和文件都应当具有这样的三种性质,准确性、鲜明性、生动性。准确性属于概念,判断和推理的问题,这是都是逻辑问题。鲜明性和生动性,除了逻辑问题以外,还有词章问题。现在许多文件的缺点是;第一概念不明确,第二判断不恰当,第三使用概念和判断进行推理的时候又缺乏逻辑性,第四不讲究词章。看这种文章是一场大灾难,耗费精力又少有所得。一定要改变这种不良的风气。作经济工作的同志在起草文件的时候,不但要注意准确性,还要注意鲜明性和生动性,不要以为这只是语文教师的事情,大老爷用不着去管。重要的文件不要委托二把手、三把手去写,要自己动手或者合起来作。

(三十八)不可以一切依赖秘书或者“二排议员”。要以自己动手为主,别人帮助为辅。不要让秘书制度成为一般制度,不应当设秘书的人不许设秘书,一切依赖秘书这是革命意志衰退的一种表现。

(三十九)学点自然科学和技术科学。

(四十)学点哲学和政治经济学。

(四十一)学点历史和法学。

(四十二)学点文学。

(四十四)建议在自愿的原则下,中央和省市的负责同志学习一种外国语,争取在五年到十年的时间内达到中等程度。

(四十五)中央和省的主要负责人可以设置一名学习秘书。

(四十六)外来干部要学本地话,一切干部要学普通话,先订一个五年计划,争取学好或者大体学好,至少学会一部分.在少数民族地区工作的汉族干部必须学会当地民族的语言。少数民族的干部也应当学会汉语。

(四十七)中央各部,省、专区、县三级,都要比培养“秀才”。没有知识分子不成。无产阶级一定要有自己的秀才。这些人要较多的懂得马克思主义,又有一定的文化水平,科学学知识词章修养。

(四十八)一切中等技术学校和技工学校,凡是可能的,一律办工厂或农场。进行生产,做到自给或半自给,学生实行半工半读。在条件许可的情况下,这些学校可多招些学生,但不要国家增加经费。

一切高等工业学校可以进行生产的实验室或附属工厂,出了保证教学和科学研究的需要外,都应当尽可能地进行生产。此外,还可以由学生和数师同当地的工厂订立参加劳动的合同。

(四十九)一切农业学校除了在自己的农场进行生产,还可以同当地的农业合作社订立参加劳动的合同,并且派教师住到合作社去,使理论和实际结合。农业学校应当由合作社保送一部分合乎条件的人入学。

农村里的中小学都要同当地的农业合作社订立合同,参加农副业生产劳动。农村学生还应当利用假期,假日或者课余时间同到本村参加生产。

(五十)大学校和城市里的中等学校在可能条件下,可以由几个学校联合设立工厂或者作坊,也可以同工厂、工地或者服务行业订立参加劳动的合同。

一切有土地的大中小学校,应当设立附属农场,没有土地而邻近郊区的学校,可以到农业合作社参加劳动。

(五十一)开展以除四害为中心的爱国卫生运动。今年要每月检查一次,以便打下基础。各地可以根据当地的情况,增加除四害以外的其他内容。

(五十二)化肥工厂,中央、省、专区三级都可以设立,中央化工部门要帮助地方搞中小型化肥工厂的设计,中央机械部门要帮助地方搞中小型化肥工厂的设备。

(五十三)省、自治区、直属市,应当设立农具研究所,专门负责研究各种改良农具和中小型机械农具,同农业制造厂密切联系,研究好了就交付制造。

(五十四)湖北孝感县的联盟农业社,一部分土地每年种一造,亩产二千一百三十斤,四川仁寿县的前进农业社,一部分土地一造亩产一千六百八十斤,陕西省宜君县的清河农业社,这个社在山区,一部分土地一造亩产一千六百五十四斤,广西百色县的拿波农业社一部分土地一造亩产一千六百斤。这些单季的高产经验,各地可以研究试行。

(五十五)种子配搭问题(即是在一个地域内,一种作物要有几种品种同时种植)。各地可以进行研究。

(五十六)薯类大有用处。人吃、猪吃、牛吃、造酒、造糖、造粉,各地可以试制薯类粉,有控制地,适当地推广薯类种植。

(五十七)绿化。凡能四季种树的地方,四季都种,能种三季的种三季.能种两季的种两季。

(五十八)陕西商洛专区每户种一升核桃,这个经验值得各地研究,可以经过鸣放辩论后取得群众同意,将这个经验推广到种植果木、桑、柞、茶、漆、油料等经济林木方面去。

(五十九)林业要计算覆盖面积,算出各省、各专区、各县覆盖面积的比例,作出森林覆盖面积的规划。

(六十)今年九月以前,要酝酿一下我不做中华人民共和国主席的问题。先在各级干部中间,然后在工厂和合作社中间,组织一次鸣放变论,征求干部和群众的意见,取得多数人的同意。这是因为去掉共和国主席这个职务,专做党中央主席,可以节省许多时间作一些党所要求我做的事情。这样,对于我的身体状况也较为适宜。如果在辩论中群众发生抵触情绪,不赞成这个建议,可以向他们说明,在将来国家有紧急需要的时候,只要党有决定。我还是可以担任这种国家领导的职务的。现在和平时期,以去掉一个主席职务较为有利。关于这个请求,已经得到中央政治局以及中央和地方许多同志的同意,认为这是一个好主意。所有这些。请向干部和群众解释清楚,免除误会。

这次会议的传达方法。把这些观点逐渐和干部讲明。不要采取倾盆大雨的方式。

这次所谈的意见,都是建议性的。请同志们带同去讨论,可以推翻,可以发展,征求干部的意见。大约要有几个月才能形成正式条文。



\section[在成都会议上的讲话(一)(一九五八年三月九日)]{在成都会议上的讲话(一)}
\datesubtitle{(一九五八年三月九日)}


现在提出以下一些问题来讨论。你们的问题也提出来。

一、协作问题。现在普遍存在协作问题。这是××同志提出的。全国省与省、市与市、社与社、农、工、商、交通、贸易、文教,都要协作。

二、中心工作与非中心工作如何去作。县委书记搞了中心工作,其他同志就不高兴。在县级以下,不要因为中心而丢掉其他。

三、税制和价格问题。

四、地方工业中的劳动法。县、乡工业是否实行八小时制,劳动保护,工资福利如何?

五、第二本账问题,要在这里谈谈,提出原则,党代表大会通过后六、七月交人代会通过。

六、究竟多久完成十年农业计划和工业计划?个别合作社已完成或一两年完成,或苦战三年完成,十二个省五年完成,但未把荒年算在内,恐怕落空,湖北5-7年完成(包括二年灾荒),争取五年完成,这就此较主动。现在账已公布出来了,完不成要挨骂,有无把握?挨骂不要紧,无杀头之罪,无非是主观主义。我现在又有点“机会主义”,无非是怕打屁股。

地方工业,全国劲头很大。东北农业劲头不大。辽宁工业已占85%,着重搞工业,没有注意农业,没有双管齐下,是“铁拐李”,农业腿短。

七、招工问题。现在又有大招工的一股风,这个不得了。山东要招15万人,山西要招临时工17万人。1956年工资冒了10多亿,如果不注意,就要发生浪费。

八、平衡问题。全国、省与省、城与乡之间的平衡,要很好研究一下。全国各地搞工业,上海工人如何办,哪里去吃饭。现在好像不要平衡。还是应该要一点。现在有人认为越不平衡越好,是否有道理?

九、粮食包干问题,浙江有一个报告,已印发。

十、又统一又分散——地方分权问题。欧洲现在没有统一的国家,可是地方发展了。中国自秦至今,一统天下,统了,地方就不发展。各有利弊。

十一、上层建筑和经济基础的关系,生产力和生产关系,究竟有什么问题?这两类矛盾的情况如何?克服的趋势如何?

十二、两种方法的比较。一种是马克思主义的“冒进”,一种是非马克思主义的反“冒进”。究竟釆取哪一种?我看应该釆取“冒进”。很多问题都可以这样提。例如除四害,一种是除掉四害,一种是让四害存在,除四害也有两种方法,有快有慢,快一点能除掉,慢一点除不掉。执行计划,一种方法是十年计划二十年搞完,一种方法是十年计划二、三年搞完。又如肥料,1956年此1957年多一倍,1958年又超过1956年一倍。肥料多好还是少好:去年生产不起劲,今年不仅恢复,而且超过1956年。那种办法好?1957年的“马克思主义”反冒进好,还是1958年的“冒进”好?这两种方法要比较。苦战三年。改变面貌,是办得到的,但“一天消灭四害”,“苦战三天”,这就不是马克思主义了。

十三、文教,有人提议搞14项。商业是否也搞14项?

十四、技术革命和文化革命。不断革命论。在南宁会议只提出了技术革命。现在有人加上文化革命可以研究。

十五、要跃进,但不要空喊,要有办法,有技术,指标很高,实现不了。通县原来亩产150斤,1956年一跃为800斤,没有实现,是主观主义。但无大害处,屁股不要打那样重。现在的跃进,有无虚报,空喊,不切现实的毛病。现在不是去泼冷水,而是提倡实报实喊,要有具体措施,保证口号的实现。

十六、整风问题。双反抓到了题目。知识分子“专深红透”这个口号很好。刘备招亲,弄假成真,他们也是有真的,有假的,他们有小部分是假的,多数是半真半假的,可以发生突变的。不要多少时候就会变的,因为去年整风反右为基础,今年又有生产高潮,思想有很大改变,这是整风的形势。

基层整风如何作法?要大鸣大放,大整大改。群众中一些错误思想也要解决。这些工作都要做,不然,热情就不够高。

十七、右派大会开不开?一个城市、一个区、一个学校召开右派大会,有左派参加,主要目的是争取分化右派,给他一条出路,一打一拉,又打又拉,就是给右派一条出路。

十八、农具改革运动,要一直改到拖拉机。湖北省当阳县的车子化,是技术革新的萌芽。

十九、六十条现在还不是正式的文件,要修改或重新写,基本观点对,要有所增减。

二十、报纸如何办?中央、省、市、专(市)、县、区报纸如何改变面貌,生动活泼,人民日报提出23条,有跃进的可能。组织、指导工作,主要靠报纸,单靠开会,效果有限。

二十一、国际形势和外交政策。宦乡说英国的备忘录,刺得我们很不舒服,其实他们是用针刺我们,而我们则用锥子锥他们,我看很舒服。他们不希望我们公开辩驳,是因为国际形势,国内大选和做买卖对他们不利。印尼、阿拉伯世界的情况是好的。朝鲜、波兰(农业问题)有希望,不是一团黑暗。十二国经济协作要研究。政治要和业务相结合,是否外贸在政治上有不足之处?可叫兄弟国家制造我们需要的东西。是否参加十二国协作会议,是否成为正式委员,我看问题不在形式,而在于实质。

二十二、国防计划问题。

二十三、出理论杂志问题。

二十四、过去八年的经验,应加总结,反冒进是个方针问题,南宁会议谈了这个问题。谈清楚的目的是为了使大家有共同的语言,好做工作。

规章制度。××同志在南宁会议谈了规章制度问题,规章制度从苏联搬了一大批,如搬苏联的警卫制度,限制了负责同志的活动,前呼后拥,不许上饭馆,不许上街买鞋,这是谈公安部。其他各部都有,现在双反、整改,大有希望。有些规章制度束缚生产力,制造浪费,制造官僚主义。这也是拿钱买经验。建国之初,没有办法,这有一部分真理,但也不是全部真理。不能认为非搬不可。政治上、军事上的教条主义,历史上犯过,但就全党讲,犯这错误只是小部分人,多数人并无硬搬的想法,建党和北伐时期,党比较生动活泼,后来才硬搬。规章制度是繁文缛节,上层建筑,都是“礼”。大批的“礼’,中央不知道,国务院不知道,部长也不一定知道。工业和教育两个部门搬得厉害,农业搬的也有,但是中央抓得紧,几个章程和细节都经过了中央、还批发一些地方经验,从实际出发,搬的少一些。农业有物也有人,工业只有物没有人,商业好像少一点,计划、统计、基建程序、管理制度、财政,搬的不少,基本规章是用规章制度管人。搬要有分析,不要硬搬,硬搬就是不独立思考,忘记了历史上教条主义的教训。教训就是理论和实践相结合,理论从实践中来,又到实践中去,这个道理未运用到经济建设上。马列主义的普遍真理与中国革命具体实际相结合,这是唯物论,二者是对立的统一,也就是辩证法,为什么硬搬,就是不讲辩证法。苏联有苏联的一套办法,苏联经验是一个侧面,中国实践又是一个侧面,这是对立的统一。苏联的经验只能择其善者而从之,其不善者不从之。把苏联的经验孤立起来,不看中国实际,就不是择其善者而从之,如办报纸,要搬真理报的一套,不独立思考,好像三岁小孩子一样,处处要扶,丧魂失魄,丧失独立思考。什么事情要提出两个办法来比较,这才是辩证法。不然,就是形而上学。铁路选线,工厂选厂址,三峡选坝址,都有几个方案,为什么规章制度不可以有几个方案?部队的规章制度,也是不加分析,生搬硬套,进口“成套设备”(不是建筑上的)。所有制、相互关系、分配为生产关系的三大部分,规章制度,有一部分属于生产关系,工资福利属于分配,都是生产关系。


\section[在成都会议上的讲话(二)(一九五八年三月十日)]{在成都会议上的讲话(二)}
\datesubtitle{(一九五八年三月十日)}


规章制度是一个问题,借此为例,讲一讲思想方法问题——坚持原则与独创精神。

国际方面,要和苏联、一切人民民主国家和各国共产党、工人阶级友好,讲国际主义,学习苏联及其他外国的长处,这是一个原则。但是学习有两种方法,一种是专门模仿,一种是独创精神,学习应和独创相结合,硬搬苏联的规章制度,就是缺之独创精神。

我党从建党时期到北伐时期(一九二一年到一九二七年),虽有陈独秀披着马克思主义外衣的资产阶级思想,但比较生动活泼。十月革命胜利后的第三年,我们建了党,参加党的人都是参加“五四”运动和受其影响的青年人。十月革命后,列宁在世,阶级斗争很尖锐,斯大林尚未上台,他们也是生动活泼的。陈独秀主义来源于国外社会民主党和国内资产阶级。这个时期,虽发生了陈独秀主义的错误,一般说没有教条主义。

内战时期到遵义会议(一九二七年到一九三五年)中国党发生了三次“左”倾路线,而在一九三四年至一九三五年最厉害,当时苏联反托派胜利了,在理论上只战胜了德波林学派,中国“左”倾机会主义者差不多都是在苏联受到影响的,当然也不是所有去莫斯科的人,都是教条主义者。当时在苏联的许多人当中,有些人是教条主义,有些人不是,有些人联系实际,有些人不联系实际,只看外国。加上斯大林的统治开始巩固(大巩固是在肃反后);共产国际当时是布哈林、皮可夫、季诺维也夫,东方部长是库西宁,远东部长是米夫。×××是个好同志,善良,有独创精神,就是太老实了些,米夫的作用大了,这些条件使教条主义得以形成,有些中国同志也受到影响,“左”倾在知识青年中也有。当时王明等搞了个所谓“二十八个半布尔什维克”,几百人在苏联学习,为什么只有二十八个半呢?就是他们“左”得要命,自己整自己,使自己孤立,缩小了党的圈子。

中国的教条主义有中国的特色,表现在战争中,表现在富农问题上,因为富农人数很少,决定原则上不动,向农民让步。但是“左”派不赞成,他们主张“富农分坏田,地主不分田”,结果地主没有饭吃,一部分被迫上山,搞绿色游击队。在资产阶级问题上,他们主张一概打倒,不仅政治上消灭,经济上也消灭,混淆了民主革命和社会主义革命。对帝国主义也不加分析,认为是铁板一块,不可分割,都支持国民党。

全国解放后(一九五○年到一九五七年)在经济工作和文教工作中产生了教条主义,军事工作中搬了一部(分)教条,基本原则坚持了,还不能说是教条主义。经济工作教条主义主要表现在重工业、计划工作、银行工作、统计工作,特别是重工业和计划方面,因为我们不懂,完全没有经验,横竖自己不晓得,只好搬。统计工作几乎是抄苏联的;教育方面也相当厉害,例如五分制,小学五年一贯制等,甚至不考虑解放区的教育经验。卫生工作也是,害得我三年不能吃鸡蛋,不能吃鸡汤,因苏联有篇文章说不能吃鸡蛋和鸡汤,后来又说能吃。不管文章正确不正确,中国人都听了,都奉行。总之,是苏联第一。商业少些,因中央接触较多,批转文件较多,轻工业工作中的教条主义也少些,社会主义革命和农业合作化未受教条主义影响,因为中央直接抓,中央这几年主要抓革命和农业,商业也抓了一点。

教条主义的情况也有不同,需要分析比较,找原因:

一、重工业的设计、施工、安装自己都不行,没有经验,中国没有专家,部长是外行,只好抄外国的,抄了也不会鉴别。而且还要借苏联的经验和苏联专家,破中国的旧专家的资产阶级 注:原文为“想” 思想,苏联的设计用到中国大部分正确,一部分不正确,是硬搬。

二、我们对整个经济情况不了解,对苏联和中国的经济情况的不同更不了解,只好盲目服从。现在情况变了,大企业的设计施工,一般说来,可以自己搞了;设备,再有五年就可以自己造了,对苏联、对中国的情况,都有些了解了。

三、在精神上没有压力了,因为破除了迷信。菩萨比人大好几倍,是为了吓人,戏台上的英雄豪杰出来,与众不同,斯大林就是那样的人,中国人当奴隶当惯了,似乎还要当下去,中国艺术家画我和斯大林的像,总比斯大林矮一些,盲目屈服于那时苏联的精神压力,马列主义对任何人都是平等的,应该平等待人。赫鲁晓夫一棍子打死斯大林也是一种压力,中国党内多数人是不同意的。还有一些人屈服于这种压力,要打倒个人崇拜。有些人对反对个人崇拜很感兴趣,个人崇拜有两种:一种是正确的,如对马克思、恩格斯、列宁、斯大林正确的东西,我们必须崇拜,永远崇拜,不崇拜不得了,真理在他们手里,为什么不崇拜呢?我们相信真理,真理是客观存在的反映,一个班必须崇拜班长,不崇拜不得了。另一种是不正确的崇拜,不加分析,盲目服从,这就不对了。反个人崇拜的目的也有两种:一种是反对不正确的崇拜,一种是反对崇拜别人,要求崇拜自己,问题不在于个人崇拜,而在于是否是真理。是真理就要崇拜,不是真理就是集体领导也不成。我们党在历史上就是强调个人作用和集体领导相结合的。打死斯大林有些人有共鸣,有个人目的,就是为了想让别人崇拜自己,有人反对列宁,说列宁独裁,列宁回答很干脆:与其让你独裁,不如我独裁好。斯大林很欣赏高岗,专送一辆汽车,高岗每年“八·一五”都给斯大林打贺电,现在各省也有这样的例子;是江华独裁,还沙文汉独裁?广东、内蒙、新疆、青海、甘肃、安徽、山东等地都发生过这样的问题,你不要以为天下太平,时局是不稳定的,“脚踏实地”是踏不稳的,有一天大陆会下沉,太平洋会变成陆地,我们就得搬家。轻微的地震是经常会有的,高饶事件是八级地震……

四、忘记了历史经验教训,不懂得比较法,不懂得树立对立面。我昨天已经讲过,对许多规章制度,我们许多同志不去设想有没有另外一种方案,择其合乎中国情况者应用,不合适者,另拟。也不作分析,不动脑筋,不加比较。过去我们反对教条主义,他们的“布尔什维克”刊物把自己说成百分之百的正确,自己吹嘘自己,其办法是。攻其一点或几点,不及其余,“实话报”攻击中央苏区五大错误,不讲一条好处。

一九五六年四月提出“十大关系”,开始提出自己的建设路线,原则和苏联相同,但有我们的一套内容。“十大关系”中,工业和农业,沿海和内地,中央和地方,国家、集体和个人,国防建设和经济建设,这五条是主要的。国防费在和平时期要少,行政费任何时期都要少。

一九五六年,斯大林受批判,我们一则以喜,一则以惧。揭掉盖子,破除迷信,去掉压力,解放思想,完全必要。但一棍子打死,我们就不赞成,他们不挂像,我们挂像。一九五○年,我和斯大林在莫斯科吵了两个月,对于互助同盟条约,中长路,合股公司,国境问题,我们的态度:一条是你提出,我不同意者要争,一条是你一定要坚持,我接受。这是因为顾全社会主义利益。还有两块“殖民地”,即东北和新疆,不准第三个国的人住在那里,现在取消了。批判斯大林后,使那些迷信的人清醒了一些。要使我们的同志认识到,老祖宗也有缺点,要加以分析,不要那样迷信。对苏联经验,一切好的应该接受,不好的应拒绝。现在我们已学会了一些本领,对苏联有了些了解,对自己也了解了。

一九五七年,在“正确处理人民内部矛盾的报告”中,提出了工、农业同时并举,工业化的道路,合作化、节育等问题。这一年发生了一件大事,就是全民整风、反右派,群众性的对我们工作的批评,对人民思想的启发很大。

一九五八年在杭州、南宁、成都开了三次会。会上大家提了很多意见,开动脑筋,总结八年的经验,对思想有很大启发,南宁会议提出了一个问题,就是国务院各部门的规章制度,可以改,而且应当改。一个办法是和群众见面,一个办法是搞大字报。另一个问题是地方分权,现在已经开始实行,中央集权和地方分权同时存在,能集则集,能分则分,这是去年三中全会后定下来的。分权当然不能是资产阶级民主,资产阶级民主在社会主义之前是进步的,到社会主义时期是反动。苏联俄罗斯族占百分之五十,少数民族占百分之五十,而中国汉族占百分之九十四,少数民族占百分之六,故不能搞加盟共和国。

中国的革命是违背斯大林的意志而取得胜利的,假洋鬼子“不许革命”。“七大”提出放手发动群众,壮大一切革命力量,建立新中国。与王明的争论,从一九三七年开始,到一九三八年八月为止,我们提十大纲领,王明提六十纲领。按照王明即斯大林的作法,中国革命是不能成功的。我们革命成功了,斯大林又说是假的,我们不辩护,抗美援朝一打就真了。可是到我们提出“正确处理人民内部矛盾”时,我们讲,他们不讲,还说我们是搞自由主义,好像又是不真了。这个报告公布后,纽约时报全文登载,并发表了文章说是“中国自由化”。资产阶级要灭亡,见了芦苇当渡船,那是很自然的事。但资产阶级的政治家也不是没有见解的人,如杜勒斯听到我们的文章,说要看看,不到半月他便作出结论:中国坏透了,苏联还好些。但当时苏联看不清,给我们一个照会,怕我们向右转。反右派一起,当然“自由化”没有了。

总之,基本路线是普遍真理,但各有枝叶不同。各国如此,各省也如此。有一致,也有矛盾,苏联强调一致,不讲矛盾,特别是领导与被领导的矛盾。


\section[在成都会议上的讲话(三)(一九五八年三月二十日)]{在成都会议上的讲话(三)}
\datesubtitle{(一九五八年三月二十日)}


我讲四个问题:

一、改良农具的群众运动,应该推广到一切地方去,它的意义很大,是技术革命的萌芽,是一个伟大的革命运动。因为几亿农民在动手动脚,否定肩挑的反面,一搞就节省劳动力几倍,以机械化代替肩挑,就会大大增加劳动效力,由此而进一步机械化。中国这么大的国家,不可能完成机械化,总有些角落办不到,一千年,五百年,一百年,五十年,总有些还是半机械化,如木船;有一部分手工业,过几万万年还会有的,如吃饭,永世是手工业,它同机械化是对立的统一,只是性质不同,应当结合起来。

二、河南提出一年实现四、五、八,水利化,除四害,消灭文盲,可能有些能做到。即使全部能做到,也不要登报,二年可以做到,也不要登报,内部可以通报。像土改一样,开始不要登报,告一段落再登。大家抢先,会搞得天下大乱,实干就是了。各省不要一阵风,说河南一年,大家都一年,说河南第一,各省都要争个第一,那就不好。总有个第一,“状元三年一个,美人千载难逢”。可以让河南试验一年。如果河南灵了,明年各省再来一个运动,大跃进,岂不更好。

如果在一年内实现四、五、八,消灭文盲,当然可能缺点很大,起码是工作粗糙,群众过份紧张。我们做工作要轰轰烈烈,高高兴兴,不要寻寻觅觅,冷冷清清。

只要路线正确——鼓足干劲,力争上游。多、快、好、省(这几句话更通俗化)。那么后一年、二年、三年至五年完成四十条,那也不能算没有面子,不能算不荣誉,也许还更好一些。比,一年比四次,合作化逼得周小舟紧张的要命,四川的高级化,×××从容不迫。不慌不忙,到一九五七年才完成,情形并不坏。迟一年有何关系?也许更好些。一定要四年、五年才完成,那也不对,问题是看条件如何,群众觉悟提高没有?需要多少年,那是客观存在的事情。搞社会主义有两条路线:是冷冷清清、慢慢吞吞好,还是轰轰烈烈、高高兴兴的好?十年、八年搞个四十条,那样搞社会主义也不会开除党籍。苏联四十年才搞那么点粮食和东西,假如我们十八年能比上四十年当然好,也应当如此。因为我们人多,政治条件也不同,比较生动活泼,列宁主义比较多。而他们把列宁主义一部分丧失了,死气沉沉。列宁在革命时期的著作,骂人很凶,但是骂得好,同群众通气,把心交给群众。

建设的速度,是个客现存在的东西,凡是主观、客观能办到的,就鼓足干劲,力争上游,多、快、好、省,但办不到的不要勉强。现在有股风,是十级台风,不要公开去挡,要在内部讲清楚,把空气压缩一下。要去掉虚报、浮夸,不要争名,而要务实。有些指标高,没有措施,那就不好。总之,要有具体措施,要务实。务虚也要,革命的浪漫主义是好的,但没有措施不好。

三、各省、市、自治区两个月开一次会,检查总结一次。开几个人或十几个人的小型会。协作区也要二、三个月开一次会。运动变化很大,要互通情报。开会的目的,为了调整生产节奏,一波未平,一波又起,这是快与慢的对立的统一。在鼓足干劲,力争上游,多、快、好、省的总路线下,波浪式的前进,这是缓与急的对立的统一,劳与逸的对立的统一。如果只有急和劳,则是片面性,专搞劳动强度,不休息,那怎么行呀?做事总要有缓有急,(如武昌县书记,不看农民情绪,腊月二十九还要修水库,民工跑了一半)也是苦战与休整的统一。从前打仗,两个战役之间必须有一个休整,补充和练兵。不可能一个接一个打,打仗也有节奏。中央苏区百分之百的“布尔什维克化”,就是反休整,主张“勇猛果断,乘胜直追,直捣南昌”,那怎么行?苦战与休整的对立统一,这是规律,而且是互相转化的,没有一种事情不是互相转化的,“急”转化为“缓”,“缓”转化为“急”,“劳”转化为“逸”,“逸”转化为“劳”,休整与苦战,也是如此。劳和逸,缓和急,也有同一性;休战与苦战也有同一性。睡眠与起床也是对立的统一,试问谁能担保起床以后不睡觉?反之,“久卧者思起”。睡眠转化为起床,起床转化为睡觉。开会走向反面,转化为散会,只要一开会就包含着散会的因素,我们在成都不能开一万年会。王熙凤说:“千里搭长棚,没有不散的席。”这是真理。不可以人废言,应以是否为真理而定。散会后,问题积起来了,又转化为开会。团结,搞一搞意见就有分歧,就转化为斗争,发生分歧,重新破裂,不可能天天团结,年年团结。讲团结,就有不团结,不团结是无条件的,讲团结时还有不团结,因此要作工作,老讲团结一致,不讲斗争,不是马列主义。团结经过斗争,才能团结,党内、阶级、人民都一样,团结转化为斗争,再团结。不能光讲团结一致,不讲斗争、矛盾。苏联就不讲领导与被领导之间的矛盾。没有矛盾斗争,就没有世界,就没有发展,就没有生命,就没有一切。老讲团结,就是“一潭死水”,就会冷冷清清。要打破旧的团结基础,经过斗争,在新的基础上团结。一潭死水好,还是“不尽长江滚滚来”好?党是这样,人民、阶级都是这样。团结、斗争、团结,这就有工作做了。生产转化为消费,消费转化为生产,生产就是为了消费,生产不仅为了其他劳动者,而且自己也是消费者。不吃饭,一点气力没有,不能生产,吃了饭有了热量,他就可以多做工作。马克思说:生产就包含着消费。生产与消费,建设与破坏,都是对立的统一,是互相转化的。鞍钢生产是为了消费,几十年更换设备。播种转化为收获,收获转化为播种,播种是消费种子,种子播下后,又走向反面,不叫种子,而是秧苗,收获,收获以后,又得到新的种子。

要举丰收的例子,搞几十、百把个例子,来说明对立的统一和相互转化的概念,才能搞通思想,提高认识。春夏秋冬也是互相转化的,春夏的因素,就包含在秋冬中。生与死也是互相转化的,生转化为死,死物转化为生物,我主张五十岁以上的人死了开庆祝会,因人是非死不可的,这是自然规律。粮食是一年生植物,年年生一次,死一次,而且死的多越生得多。例如猪不杀掉,就越来越少了,谁喂呢?

简明哲学词典,专门与我作对,它说生死转化是形而上学,战争与和平转化是不对的,究竟谁对?请问,生物不是由死物转化的,是何而来?地球上原来只有无机物,以后才有有机物,有生命的物质都是氮、氢等十二种元素变成的,生物总是死物转化的。

儿子转化为父亲,父亲转化为儿子,女子转化为男子,男子转化为女子,直接转化是不行的,但是结婚后生男生女,还不是转化吗?

压迫者与被压迫者相互转化,就是资产阶级、地主与工人、农民的关系,当然.我们这个压迫者是对旧统治阶级讲的,这是阶级专政而不是讲个人压迫者。

战争转化为和平,和平是战争的反面,没有打仗是和平,三八线一打是战争,一停战又是和平,军事是特殊形势下的政治,是政治的继续,政治也是一种战争。

总而言之,量变转化为质变,质变转化为量变,欧洲教条主义浓厚,苏联有些缺点,总要转化的,而我们如果搞不好,又会硬化的。那时如果我们工业搞成世界第一,就会翘尾巴,思想就会僵化。

有限变为无限,无限转化为有限,古代的辩证法转化为中世纪的形而上学,中世纪的形而上学转化为近代的辩证法,宇宙也是转化的,不是永恒的,资本主义到社会主义,社会主义到共产主义,共产主义社会还是要转化的,也是有始有终的,一定会分阶级,或者要另起个名字,不会固定的。只有量变没有质变,那就违背了辩证法。世界没有什么东西不是发生、发展和消灭的。猴子变人,发生了人,整个人类最后是要消灭的,它会变成另一种东西,那时候地球也没有了。地球总是要毁灭的,太阳也要冷却的,太阳的热现在就比古代冷得多了。冰河时期,二百万年变一次,冰河一来,生物就大批死亡,南极下面有很多煤炭,可见古代是很热的,延长县发现有竹子的化石(宋朝人说,延长古代是生长竹子的,现在不行了)。

事物总是有始有终的。只有两个无限:时间、空间无限。无限是有限构成的,各种东西都是逐步发展,逐步变化的。

讲这些就是为了展开思想,把思想活泼一下,脑子一固定,就很危险。要教育干部,中央、省、地、县四级干部很重要,包括各系统,有几十万人。总而言之,要多想,不要老想看经典著作,而要开动脑筋,使思想活泼起来。

四、社会主义的建设路线,还在创造中,基本观点已经有了。全国六亿人,全党一千二百万人,只有少数人,恐怕只有几百(万)人,感觉这条路线是正确的,可能还有很多人将信将疑,或者是不自觉的。例如农民搞水利,不能说他对水利将信将疑,但他对于路线则是不自觉的。又如除四害,真正相信者,现在逐渐多起来了,连我自己也将信将疑,碰到人就问:“消灭四害能否办到?”合作化也是如此,没有证明此事就要问。再有一部分人根本不信,可能有几千万人(地、富、资产阶级、知识分子、民主人士以及劳动人民内部和我们干部中的一部分)。现在已经使得少数人感觉到这条路线是正确的,对于我们来说,在理论上和若干工作的实践上(例如有相当的增产,工作有相当的成绩,多数人心情舒畅),承认这条路线是正确的。但是四十条,十五年赶上英国,这是理论,四、五、八大部尚未实现,全国工业化尚未实现,十五年赶上英国还是口号,一五六项尚未全部建成。第二个五年计划搞二千万吨,在我脑筋中存在问题,是好,还是天下大乱?我现在没有把握,所以要开会,一年四次,看到有问题就调节一下。建成后的形势无非是:大好、中好、不甚好、不好或者是大乱子。看来出乱子也不会很大,无非乱一阵,还会走向“治”,出乱子包含着好的因素,乱子不怕。匈牙利建设工业出了些乱子,现在又好了。

路线已开始形成,反映了群众斗争的创造,这是一种规律,领导机关反映了这些创造,提出了几条。许多事情是没有料到的,规律是客观存在的,不以人们意志为转移的。比如,一九五五年合作化高涨轰轰烈烈,没有料到有斯大林问题,匈牙利事情,“反冒进”,明年怎样?又会出什么事,反什么主义?谁人能料到?具体的事是算不出来的。

现在人们的相互关系,决定于三大阶级的关系:

第一个是帝国主义、封建主义、官僚资本主义、右派分子及其代理人,不革他们的命,就束缚生产力。右派占资产阶级分子中的百分之一或百分之二。其中大多数人将来可能改变,转化过来,那是另外的问题。

第二个是民族资产阶级,是指右派以外的那些人,他们对我们的新中国是半心半意的,半心被迫向我们,半心要搞资本主义,经过整风,已经有了改变,可能是三分天下有其二了(北京民主党派开自我改造誓师大会,全国都要开)。

第三个是左派,即劳动人民、工人、农民(其实是四个阶级,农民是另一个阶级)。

路线已经开始形成,但是尚待完备,尚待证实,不可以说已经最后完成。工人向农民摆阔气,有些干部争名誉、地位,都是资产阶级思想。不把这些问题解决,就搞不好生产,不解决这些相互关系,劳动怎能搞好?过去我们在建设上用的心思太少,主要精力是搞革命。错误还是要犯的,不可能不犯,犯错误是正确路线形成的必要条件。正确路线是对错误路线而言的,二者是对立的统一。正确路线是在同错误路线的斗争中形成的。说错误都可以避免,只有正确,没有错误这种观点是反马克思主义的,问题是少犯点,犯得小点。正确与错误是对立的统一,难免论是正确的。只有正确,没有错误,历史上没有这个事实,这就是否认对立统一这个规律,这是形而上学,只有男人没有女人,否定女人怎么办?争取错误犯得最少,这是可能的。错误多少,是高子和矮子的关系,少犯错误是可能的,应该办到,马克思列宁就办到了。


\section[在成都会议上的讲话(四)(一九五八年三月二十二日)]{在成都会议上的讲话(四)}
\datesubtitle{(一九五八年三月二十二日)}


无事不登三宝殿,想到一点问题交换意见。

西厢记中,有一段张生和惠明的故事。孙飞虎围着普救寺,张生要送信请他的朋友白马将军来解围。无人送信。开群众会议,惠明挺身将信送去,这是描写惠明勇敢胆大的坚定之人。希望中国要多点惠明,要在县委委员以上几十万人中发动一下大鸣、大放、大字报批评领导。这是一种无产阶级的气氛,共产主义的气氛。群众骂你一顿出口气,并没砍你的头,撤你的职,这是蓬勃的战斗的情绪。是很高的共产主义的风格。现在群众斗争的风格很好。我们同志之间也要提倡这种风格。

陈伯达写给我一封信,他原来死也不想办刊物,现在转了一百八十度,同意今年就办,这很好。我们党从前有《响导》、《斗争》、《实话》等杂志,现在有《人民日报》,但没有理论性杂志。原来打算中央、上海各办一个,设立对立面有竞争。现在提倡各省都办,这很好。可以提高理论,活泼思想。各省办的要各有特点。可以大部根据本省说话,但也可以说全国的话,全世界的话,宇宙的话,也可以说太阳、银河的话。

地方工作同志,将来总是要到中央来的。中央工作的人总有一天非死即倒的。赫鲁晓夫是从地方上来的。地方阶级斗争比较尖锐,更接近自然斗争,比较接近群众。这是地方同志比较中央同志有利的条件。秦国称王在后,但是称帝在先。

要提高风格,讲真心话,振作精神,要有势加破竹,高屋建瓴的气概。要作到这一点,必须抓住马克思主义的基本理论和工作中的基本矛盾。但我们的同志现在并不企图势如破竹,有精神不振的现象,这很不好,是奴隶状态的表现,像贾桂一样,站惯了,不敢坐。对于经典著作要尊重,但不要迷信。马克思主义本身就是创造出来的,不能抄书照搬,在这一点上,斯大林比较好一点。联共党史结束语说:“马克思主义个别原理不合理的,可以改变。如一国不能胜利(按:应指社会主义不可能在一国内首先取得胜利)。”中国的儒家对孔子就是迷信,不敢称孔丘。唐朝李贺就不是这样,对汉武帝直称其名,曰刘彻、刘郎,称魏人为魏娘。一有迷信就把我们脑子镇压住了,不敢跳出框子想问题。学习马列主义没有势如破竹的风格,那很危险。斯大林也称势如破竹,但有些破烂了。他写的语言学、经济学、列宁主义基础是比较正确的,或基本正确的。但有些问题值得研究。例如,在社会主义阶段中,价值法则的作用如何?是否拿劳动准备时间消耗多少来定工资的高低?在社会主义中,个人私有财产还存在,小集团还存在,家庭还存在。家庭是原始共产主义后期产生的,将来要消灭,有始有终。康有为的《大同书》即看到此点。家庭在历史上是个生产单位、消费单位、生下一代劳动力的单位、教育儿童的单位。现在工人不以家庭为生产单位,合作社中的农民也大都转变了,农民家庭一般为非生产单位,只有部分副业。至于机关、部队的家庭,更不生产什么东西,变成消费单位、生育劳动后备并抚育成人的单位。教育部门的主要部门,也在学校。总之,将来家庭可能变成不利于生产力发展的东西。现在的分配制度是(按劳分配)付酬,家庭还有用。到共产主义分配关系是变为各取所需,各种观念形态都要变,也许几千年,至少几百年家庭将要消灭。我们许多同志对于这许多问题不敢去设想,思想狭窄得很。这些问题经典著作上已经讲过,如阶级、党的消灭等,这说明马列风格高,我们很低。

怕教授,进城以来相当怕,不是藐视他们,而是有无穷的恐惧。看人家一大堆学问,自己好像什么都不行。马克思主义者恐惧资产阶级知识分子。不怕帝国主义,而怕教授,这也是怪事。我看这种精神状态也是奴隶制度:“谢主龙恩”的残余。我看再不能忍耐了。当然不是明天就去打他们一顿,而是要接近他们,教育他们,交朋友。他们自然科学可能多学一点,但社会科学就不见得。他们读马列主义比我们多,但读不进去,懂不了。如吴景超读了很多书,一有机会就反马克思主义。

不要“自惭形秽”,伯恩斯坦、考茨基、后期的普列汉诺夫,马列主义比我们读得多,但他们并不行,把第二国际变成了资产阶级的仆从。

现在情况已有转变,标志是陈伯达同志的一篇演说(厚今薄古)、一封信(给主席的),一个通知(准备下达)有破竹之势,但有许多同志对于思想战线上的斗争无动于衷,如批判胡风、梁漱溟、《武训传》、《红楼梦》、丁玲等。本来,消灭资产阶级的基本观点,在七届二中全会的决议中已经有了。在过去民主革命中,就经常讲革命分两个阶段,前者为后者的准备。我们是不断革命论者,但许多同志对于什么时候搞社会主义革命,土地改革后搞什么都不去想,对社会主义萌芽熟视无睹。而社会主义萌芽早已诞生。比如在瑞金、在抗日根据地,就产生了社会主义萌芽互助组。 注:“瑞金”原文为“瑞金”

王明、陈独秀是一样的。陈独秀是主张让资产阶级革命成功以后,让资产阶级掌握政权,然后壮大无产阶级再搞社会主义革命。所以陈独秀不是马列主义者,而是资产阶级民主革命的激进派。但是,经过三十多年,还有这样的人。坏人如丁玲、冯雪峰。好人如×××完全是资产阶级民主派那一套。搞“四大自由”,讲农民怕冒尖,他就跟我尖锐对立。河南的富裕中农有好东西不让干部看,装穷,无人时,才向货郎买布。我看很好,这表示贫下中农威力很大,使得富裕中农不敢冒尖。这说明社会主义大有希望。但有些人认为不得了,要解除怕冒尖的恐惧,即大出布告,搞“四大自由”。既不请示,也不商量,这明明是和二中全会方针作对。他没有搞社会主义的精神准备。现在被说服了,积极了。

从古以来,创新思想、新学派的人,都是学问不足的青年人。孔子二十三岁开始。耶苏有什么学问!释迦牟尼十九岁创佛教,学问是后来慢慢学来的。孙中山青年时有什么学问?不过高中程度;马克思开始创立辩证唯物论,年纪也很轻。他的学问也是后来学来的。马克思写《共产党宣言》时,不过三十岁左右,学派已经形成了。在开始着书时,只有二十几岁。那时,马克思所批判的都是一些当时的资产阶级博学家。如:李嘉图、亚当斯密、黑格尔等。历史上总是学问少的人推翻学问多的人。章太炎青年时代写的东西是比较生动活泼的,充满民主革命精神,以反满为目的。康有为也是如此,刘师培成名时还不过二十岁,死时才三十岁。王弼注《老子》的时候,不过十几岁,因用脑过度早死。死时才二十几岁。颜渊(二等圣人)死时才三十二岁。李世民起义时,只有十几岁,当了总司令,二十四岁登基当了皇帝,年纪不甚大,学问不甚多,问题是看你方向对不对。秦叔宝也是年轻的。年轻人抓住一个真理,就所向披靡,所有老年人是比不过他们的。罗成、王伯当都不过是二十几岁。梁启超年轻时也是所向披靡,而我们在教授前就那么无力,怕比学问。刊物出后,方向不错,就对了。雷海宗读了本马列主义不如我们,因为我们是相信马列主义,他越读得多还当右派。现在我们要办刊物,要压倒资产阶级知识分子。我们只要读十几本书就可以把他们打倒。刊物搞起来,就逼着我们去看经典著作,想问题,而且要动手写。这就可以提高思想。现在一大堆刊物吸引了我们的注意力。不办刊物大家也不会去看书,尽讲抽象不算红。

各省可办一个刊物,成立一种对立面,并且担任向中央刊物发稿的任务,每省一年六篇就够了。总之,十篇以下,由你们去组织,这样会出英雄豪杰的。

从古以来,创新学派、新教派都是学问不足的青年人,他们一眼看出一种新东西,就抓住向老古董开战!而有学问的老古董,总是反对他们的。马丁·路德创新教,达尔文主义出来后,多少人反对!发明安眠药的,既不是医生,更不是有名的医生,而是一个司药的。开始,德国人不相信,但法国人欢迎,从此才有安眠药。据说盘尼西林是一个染坊洗衣服的发明的。美国富兰克林发明了电,他是卖报的孩子,后来成了传记作家、政治家、科学家。高尔基只读了两年小学。当然学校也可以学到东西,不是把学校都关门了,而是说不一定住学校。看你方向对不对,去不去抓。学问是抓来的。从来创立学派的青年,一抓到真理,就藐视古董,有所发明。博学家就来压迫。历史难道不是如此吗?我们开头搞革命,还不是一些娃娃,二十多岁。而那时的统治者袁世凯、段琪瑞都是老气横秋的,讲学问,他们多,讲真理,我们多。

我很高兴,最近时期大字报很有气魄,批评得尖锐(性)、生动(性)。把暮气一扫而光,但我们老是四平八稳走方步,“逢人只说三分话,未可全抛一片心”,不讲真心话。

王鹤寿第二篇文章敢于批评教条主义,彭涛的也好。有说服力。尖锐性差一点,无非是“打击别人提高自己”,但不是个人主义的打击别人抬高自己。为了打击错误思想,提高正确思想,是完全必要的(当然错误中也包含自己的错误)。滕 注:原为“膝” 代远那一篇也好,但说服力不够。修那么多铁路要说出理由来,不然就把别人吓倒了。张奚若批评我们“好大喜功,急功近利,鄙视既往,迷信将来”。无产阶级就是这样嘛,任何一个阶级都是好大喜功的。不“好大喜功”,难道“好小喜过”?禹王惜寸阴,我们爱每一分钟。孔子“三日无君则惶惶如也。”孔子“席不暇暖”。墨子“实不得黔”。这都是急功近利。我们就是这个章程,水利、整风、反右派、六亿人口搞大运动,不是好大喜功吗?我们搞平均先进定额,不是急功近利吗?不鄙视旧制度,反动的生产关系,我们干什么?我们不迷信社会主义、共产主义干什么?

我们错误是有的,主观主义也是有的,但是“好大喜功,急功近利,鄙视既往,迷信将来”是正确的。天津、南京两封信虽然是反对我的,但精神可取,我看是好的。天津的更好。南京的萎靡不振,骨气不硬。陈其通等四人,除陈沂是右派分子,这些人敢于说话的精神是可取。当面不说,背后唧唧咕咕,这是不好。应该大体一致,至少要基本一致,可以尖锐一点,也可以委婉一点,但不能不说。有时要尖锐鲜明,横竖是团结帮助的态度出发,尖锐的批评不会使党分裂,只会使党团结,有话不说,就很危险。当然,说话要选择时机,不讲策略也不行。例如明朝的三大案,反魏忠贤的那样不讲策略,自己被消灭。当时落得皇帝不喜欢。一个四川人杨慎安被充军到云南。历史上讲真话的如:比干、屈原、朱云、贾谊等这些人都是不得志的,为原则而斗争的。不敢讲话无非是“一、怕扣为机会主义,二、怕撤职,三、怕开除党籍,四、怕老婆离婚(面上无光),五、怕坐班房,六、怕杀头。”我看只要准备好这几条,看破红尘,什么都不怕了。没有精神准备。当然不敢讲话.难道牺牲可以封住我们的嘴巴吗?我们应当要造成一种环境,使人家敢于说话,交出心来。苏共十九次代表大会报告说:“要造成一种环境”。这对群众来说是对的。先进分子应该不怕这一套,要有王熙凤的“舍得一身剐,敢把皇帝拉下马”的精神。

我们应当领导群众,现在群众比我们先进,他们敢于贴大字报批评我们。当然和储安平不同,那是敌人骂我们,现在是同志之间的批评。我们现在有些同志的作风不好,有些话不敢讲,只讲三分。这是一怕不好混,二怕失选票。这是庸俗作风,要改变,现在已有可能改变。

一九五六年吹掉三个东西一一多快好省,促进派,四十条。有三种人,三种心理状态,一种是痛心的,一种是漠不关心的,再一种是吹掉高兴。一块石头落地,从此天下太平。这三种态度的人,两头小中间大。一九五六年有许多问题,都有这三种态度,反日、反蒋、土改是比较一致的,但是在合作化的问题上要有三种态度。这种估计是不是对,这次会议解决了一批问题,取得协议,为政治局准备了文件,但缺点是思想谈得较少,是否用两、三天的时间谈谈思想问题,谈谈心里话?

同志们说这次会议是整风会议,又不谈思想,实践诺言,是否有矛盾?一不搞斗争,二不划右派,和风细雨,把心里话讲出来,我的企图是要人们敢说,精神振作,势如破竹,像马克思、鲁迅那样,敢说,把顾虑解除,要在地委书记约在两、三人的范围内把空气冲破一下。搞出一种新气氛。邹容十八、十九岁写了一篇《革命军》,直接骂皇帝。章太炎写文章驳康有为也是精神百倍,年纪越大用处越不多,但也不要妄自菲薄,要鼓点劲。当然,年纪大的也还要,也要掌舵。三国时刘备不好,还是老头子挂帅。要冲破党内的沉闷气氛。

印了一些诗,尽是老古董。搞点民歌好不好?请各位同志负个责任,回去以后,搜集点民歌,各个阶层、青年、小孩都有许多民歌,搞几个试点,每人发三、五张张纸写写民歌,劳动人民不能写的找人代写,限期十天搜集,会收到大批旧民歌,下次会印一本出来。

中国诗的出路.第一条、民歌,第二条、古典。在这个基础上产生出新诗来,形式是民歌的,内容应当是现实主义和浪漫主义的对立统一。太现实了就不能写诗了。现在的新诗不成形,没有人读。我反正不读新诗。除非给一百块大洋。搜集民歌的工作.北京大学做了很多.我们来搞可能找到几百万成千万首的民歌。这不费很多的劳力,比看杜甫,李白的诗舒服一些。


\section[在成都会议上的讲话(五)(一九五九年三月二十五日)]{在成都会议上的讲话(五)}
\datesubtitle{(一九五九年三月二十五日)}


会开得很好,重点归结到方法问题,第一是唯物论,第二是辩证法,我们许多同志对此并不那么尊重。反冒进不是什么责任问题,不再谈了。我也不愿听了。不要老是自我批评,作为方法的一个例子来谈,那是可以的。

唯物论是世界观,也是方法论。我们主观世界只能是客观世界的反映,主观反映客观是不容易的,要有大量事实在实践中反复无数次,才能形成观点。一眼望去,一下抓住一、二个观点,但无大量事实作根据,是不巩固的,只有大量的事实,才能认识问题。写报告是反映下面干部和群众的意见的,要经过调查研究。省要反映地、县的情况,不详细地听取他们的意见.就冒出一篇报告来。郝是危险的。要先听训,才能训人。要老老实实听群众的话,听下级的话,个别交谈。小范围(县、社、工厂)交谈。省委解决问题如此,中央以后遇到大的问题一定要与若干省委书记谈一谈。反冒进的问题就是没有征求省委书记的意见,也没有征求各部门的意见,这个方法是不对的。在中央方面,工业部门想多搞。财贸部门想少搞一点,不仅脱离了省,也脱离了多数的部。

反冒进也是一种客观反映。反映什么呢?一般、特殊、全局、个别,这是辩证法的问题.把个别的、特殊的东西。误认为一般的、全面的东西,只听少数人的意见,广大人民群众的意见没有反映。把特殊当成一般来反冒进。

陈四害的指示,是卫生部起草的。根本不能用,这是去年的例子,这几个月的情况,根本没有接触,所以说卫生部最不卫生。后来由××找了一些同志座谈,经过反复研究写成一篇很好的指示。不然根本写不出来。如果一个指示不起作用,顶好不发表,一篇文章也是如此,如果写得不好,人家连看也不看。怎么指导工作呢,因此以后我们要注意学习唯物论辩证法,要提倡尊重唯物论、辩证法。

尊重唯物论、辩证法的人,是提倡争论,听取对立面的意见。把问题提出来,暴露了对立商。一九五七年一月省、市委书记会议时,黄敬同志对经济问题有意见,我当时的注意力在思想问题方面。没有很好注意他提出的问题,故在一月省、市委书记会议上有些问题的争论,没有展开。

辩证法是研究主流与支流、本质与现象的。矛盾有主要矛盾和次要矛盾。过去发生反冒进的错误。即未抓住主流和本质,把次要矛盾当做主要矛盾来解决,把支流当作主流,没有抓到本质现象。国务院、中央政治局开会对个别问题解决得多。没有抓住本质问题,这次会议把过去许多问题提出商量解决了。

冶金工业部党组开会,吸收了部分大厂的同志共十几人参加。空气就不同了。谈了几天,解决了许多重要问题。部如此,各省也如此,中央开会有地方同志参加,必要时,除省委书记外,再加上若干地、县委书记,就有了新的因素。中央同志一年下去四个月的。也要找地、县委书记、合作社、学校谈谈,只同省委书记谈不够,要一竿子到底,不要仅仅限于间接的东西。我很想了解一个城市、一个县的工作,把一个县各方面的问题都谈一谈。不要多长时间,有二、三个星期就差不多了。各省也应该这样做。为什么要提出这个问题呢,就是要打掉官气,当了老爷,不愿向别人请教,这种“自以为是”的态度各级都有。红安县在一九五六年的作风,不就是老爷作风吗?那怎么能指导农业生产呢?一般说来,越上越离群众远一点,但也不能一概而论,有的越上官气越小。例如列宁就没有那么老爷气。相反有不少人越下官气越大,许多乡长、厂长、党委书记,官气也不小。

越请教得多,搞出来的东西,大概比较有把握。但不能说就正确了,因为还没有证明。许多事情我自己就是半信半疑。例如鼓足干劲,力争上游,多、快、好、省地建设社会主义的总路线,究竟对不对,还要看几年。革命路线在民主革命和社会主义革命中即是已经证明了的,但建设路线还要看看。

所有制的解决,已经是一种新的关系。而相互关系和分配关系,只解决了一部分。我们的党、政、工厂、学校,不管有多少官僚主义,总是与国民党有原则区别的,所以相互关系不能完全没有变化。如果不经整风,则国民党的作风、老爷气还要大量存在,这是与国民党相同的一面。“八路军不见了”,经过整风下放干部,“八路军又回未了”。

我们讲鸣放,右派(也有中派)加了个“大”字。大鸣大放,从艺术科学转到政治方面来。我们很快就转过来了。《解放日报》有一篇“只放不收”的社论,讲一万年都不收,放手发展民主,很主动。只要抓本质、主流的问题。例如一个口号——十五年赶上英国,就会起很大作用。本质问题解决了,次要问题人们会去解决的。如果只抓枝节现象,解决就解决不了。从部分现象看问题,那是很危险的。

我们很多同志不注意研究理论。究竟思想、观点、理论从那里来呢?就是客观世界的反映,客观世界所固有的规律,人们反映它,不过是比较地合乎客观情况,任何规律都是事物的一个侧面,是许多个别事物的抽象,离开客观的具体的事物,那还有什么规律?“老子”是唯物论,还是客观唯心论?我是怀疑的。规律存在于每一个具体的人,具体的社……。反复出现。普遍存在的规律,才是普遍性的规律。比如打仗诱敌深入,战役上以多胜少,战略上以少胜多,战略上他包围我们,战术上我包围他们,等等。这是经过多少年战争,胜仗、败仗,才概括起来的,完整的体系,只能在后来完成,而不能在事先或者初期完成。对井冈山时期的十六个字战术,当时人们就怀疑,那有这样的战术法则呢?这十六个字战术法则,在苏联军事史上是找不着。但这是从群众斗争中得来的。赫鲁晓夫片面的单纯依靠原子弹是危险的事情。

一九五六年发生的几件事没有材料,就国际上的批判斯大林和波、匈事件、国内的反冒进问题。今后还要准备发生预料不到的事情。我认为要把过高的指标压缩一下,要确实可靠,大水大旱都有话可说,必须从正常情况出发。做是一件事,讲是一件事。过高的指标不要登报,登了报的也不要马上去改。河南今年四件事都想完成,也许可能做到,即使能做到,讲也谨慎些,给群众留点余地,也要给下级留点余地。这也就是替自己留余地。过去我们就有不留余地之事,例如一九四七年的土改纲领,提出“开仓济贫’的口号,后米又取消了。支票开的太多,难以兑现,对我们不利。

今年这一年群众出现很高的热潮,上海很多落后分子觉悟起来,共产主义精神大大提高。太原的协作精神,就是共产主义精神。落后的起来了,是革命的标志,是无产阶级革命的标志。现在不仅先进的起来了,而且广大中间、落后的群众也起来了。农村富裕中农不想退社了,城市的职员和落后的工人也积极起来了。我很担心,我们一些同志在这种热潮下面,可能被冲昏头脑,提出一些办不到的口号。一个县、一个地委没有多大坏处,中央和省两级必须稳当一点。我并不是想消灭空气,而只是压缩空气,把脑筋压缩一下,冷静一些,不是下马。而是要搞措施,去年是搞革命的一年,经验非常丰富,大大教育了我们,使我们懂得社会主义是些什么事情。今年再看一年建设问题,很有好处。所有制的问题,可以说基本解决了,但未完全解决。对本质问题,主要问题,要看得到,抓得起,加以分析,研究方法,求得解决。几年来许多同志就是看不到、抓不起本质的问题,自信心建筑在不巩固的基础上。也有能看得到而抓不起的人,缺乏一种魄力。

以后究竟有什么事是预料不到的?国内国际上有些什么事可能出乎意料?如世界大战,疯子要打,苏联还会发生什么问题?……原子弹把我们一套通通打烂,那也没有办法,打了再建设,可能建设得更好些。国际、国内可能发生的不可意料的危险,有多少条,各省、各部党组可以谈谈,列出一个单子来,思想上无准备不好。当然,在我国发生匈波一类的事件,可以不必料,但是,部分地区还可以发生。最近甘肃不是发生问题了吗?西藏完全可能出乱子,上层人物心在印度、英、美,对我们是敷衍的。汉族内部一点事也没有也不可能(如张清荣叛变),领导人被暗杀是可能的(如列宁,基洛夫、高尔基)。但不能因此脱离群众。

冶金部党组前次会议专搞虚业。不搞实业,这种办法要提倡,抽出一段时间,专谈思想性、理论性的问题,不伤心,讲心里话。先虚后实也可以。下次开会可以多找几个部,并且事前准备一篇报告。文章写得要有说服力,要尊重唯物论、辩证法,对本质问题要看得到,抓得起。


\section[在成都会议上的讲话(六)(一九五八年三月二十六日)]{在成都会议上的讲话(六)}
\datesubtitle{(一九五八年三月二十六日)}


会议文件怎样处理,有些文件可以发给省、地、县,各省、各部选择一下,不一定都印。七届二中全会决议有必要可以印,反分散主义草案可以印发参考,人民内部矛盾报告的国内外反应可以印一本。至于这些讨论的指示,记录等,还要等候北京中央政治局发正式文件,不必全部印,也不禁止印,选印为好。总而言之,自己选择。

这次会议开得还可以,但是事先未准备虚实并举,实多了一点,虚少了一点,如果虚也有五天就好。这次实业长了一点,但也有好处.一次解决大批问题.并且是跟地方同志

一道谈的。也就比较合乎实际。虚实并举,先实后虚或先虚后实(南宁是先虚后实),各省各部可以去斟酌情况办理。也不是讲任何会议都要一虚一实,过去我们太实了,现在希望虚多一点好,以便引导各级领导同志关心思想、政治、理论的问题。红与专结合。希望各省、各部去安排一下,没有到会的省、部由协作区区长,中央同志去传达。

一年抓四次,三年看头年,是否对?如果不抓四次,改为半年一次.由于形势发展快,很多矛盾要很快反映和解决,不抓四次,许多问题不能及时解决,还是一年抓四次。省、地,县可否这样?请地方同志去斟酌。协作区会议一年六次,每两月一次(曾有规定)是否引起大家埋怨开会太多了,开一下再看看,两个月一次,一次的时间不能太长,觉得太多了,将来再减少。目的是今年抓紧一点,以便更及时地掌握群众的情绪,稳一点掌握建设的速度。

下次会议七月开,重点是工业。

现在的问题,还是不摸底,农业比较清楚,工业,商业、文教不清楚。工业方面除到会的几个部接触了一下外,其余没有摸,煤、电、油、机械、建筑、地质、交通、邮电、轻工业、商业没有接触。财贸还有文教历来没有摸过,林业没有摸过,今年,这些要摸一摸。政治局书记处都摸一摸,政治局开座谈会是个好办法。过去有一句口号:“工农兵学商团结起来,打倒帝国主义!”现在还是工农兵学商团结起来,(学是文教,兵是国防),今后这五年,还是要抓五方面。这次接触了国防,但是没有怎么谈论,过去总是搞军事。现在几年都不开会,文件都没有看。人有五官——眼、耳,口,鼻,舌。五性——色、声、香、味、触。我们工农兵学商样样有。还要加上一个“思”。南宁会议讲工农和思想,再次要讨论国防问题。地方也要讲点军事工作。从一九五三年下半年起(抗美援朝后)没有管国防。军事工作,地方也只是抽兵走,转业来而已,地方也要管军事工作。今后要回过头来搞点军事工作。

阶级分析,我们国内存在两个剥削阶级、两个劳动阶级。两个剥削阶级。第一是帝国主义、封建主义、官僚资本主义的残余。地、富、反、坏未改造好的部分。现在要加上右派。反社会主义的阶层。富农一有剥削,有选举权,但不受人尊重。右派本来是与我们合作的,现在他们反社会主义,故看到敌人。国民党所做的事、就是右派做的事。帝国主义、台湾蒋介石非常赞成、关心右派分子。地、富、反、坏、右是可以改造的,大多数可以改造好的。第二是民族资产阶级、资产阶级知识分子、民主党派的大多数(民主党派中的右派占百分之十。其余百分之九十是中间派和左派),至于资产阶级和资产阶级知识分子中右派占百分之十,老教授中的右派比例多了(大概全国约有右派分子三十万人,其中县以上和大专是十六万)。

右派这么多,所以釆取除少数外,不提它不取消选举权,而釆取分化改造的政策。中间派对我们又反对又拥护。“苏报案”是章士钊的文章。他反社会主义反共。对民族资产阶级三百万人要做好工作。他们是可以转变的。

劳动阶级是工人、农民。过去的被剥削者和不剥削人的独立劳动者也可以说是两个劳动阶级。独立劳动者,有一部分是有轻微剥削的,如富裕中农和城市的上层小资产阶级.

有些同志说,希望第一书记工作解放一点出来,从中央、省到地三级的第一书记和其他某些同志解放一部分繁重工作,这才有可能比较注意一点较大的问题。党报的总主笔也须解放一部分。不能天天工作,少搞一点事,就有可能多管些事。解放出来,做一些调查研究工作,如何解放,大家研究。


\section[在成都会议上的插话(一九五八年三月)]{在成都会议上的插话}
\datesubtitle{(一九五八年三月)}


(×××发言时的插话)

国家、自治区、合作社三者之间的关系要搞好。

说清楚,和汉族要密切,要相信马克思主义,各族互相相信。使蒙汉两族合作。不管什么民族,看真理在谁的方面。马克思是犹太人,斯大林是少数民族。蒋介石是汉族,但很坏,我们要坚决反对。不要一定是本省人执权,不管那里人人,南方,北方,这族,那族,只问那个为共产主义,马克思作书记,你赞成不赞成?他也不是本地人。汉人的头子,要向少数民族干部讲清楚。

汉族开始并非大族,而是由许多民族混合起来。汉人在历史上征服过少数民族,许多地方被赶上山去,应从历史上看中国的民族问题。究竟吃民族主义的饭还是共产主义的饭,吃地方主义的饭还是共产主义的饭,首先要吃共产主义。民族要,地方要,但不要主义。

(×××发言时的插话)

要破除迷信,“人多了不得了,地少了不得了。”多年来认为耕地太少,其实每人二点五亩就够了。宣传人多造成悲观空气,也不对,应看到人多是好事,实际人到七亿五到八亿再控制。现在还是人口少,现在很难要农民节育。少数民族、黑龙江、吉林、江西、陕西、甘肃不节育。其他地方可以试办节育。一要乐观,不要悲观,二要控制。到赶上英国时人只有文化了,就会控制了。

许多事外行比内行高明,唱戏如此,改良戏要靠观众。靠外行。

大烟,国内每年用××万两,云南现有三十万两,烟土不要烧,收起来。技术革命开禁,不一定到七月一日。对整风无害。文化大革命也可开禁。

改良土壤有二法:一为深翻,一为调换。可四至五年轮流深翻一遍。山东若县大山农业社就是如此。

现在中心问题就是地方工业,既是解决机器的问题。地方工业有四大任务;一为农业服务(基本的),一为大工业服务,一为城市人民生活服务,一为出口服务。

一切正义的、有生命的事,开始都是违法的。

化肥厂,南宁会议谈到统一由专区办,现在看每县都可办。

我们有些人有错觉,认为农业品出口容易,换回工业机器不容易。其实相反,死东西容易搞,活东西不容易,农业不容易。应把这种看法改变过来,农产品很贵重。

(××发言时的插话)

要把薯类、洋芋的名誉提高,列入正粮,不要叫什么杂粮。

三年内不要减少自留地和个人养猪。可以说一下。增加合作社的积累,分的少了,应该让农民发展一些付业,增加一些收入。自留地减少大家要多养猪,两头猪死不了。千斤社庆丰收,这不同于婚丧,吃一顿,每人×××,不必泼冷水。

大字报在农村可以推广。有四条好处:一、可及时议论国家大事,二、干部能听话,三、群众便于说话,四、不怕报复。这次会议作出一条决议,发一个指示。农村普遍贴大宇报。中国自有了大字报。

(×××发言时的插话)

北京城墙可以挖,先不全挖,而是挖得稀烂。

打开通天河、白龙河与洮河,借长江济黄,丹江口引汉济黄,引黄济卫,同北京连起来。

定息不能取消。资本家要求取消,我们就不取消。资本家要求取消定息想去掉帽子。资本家自动不领定息是可以的,但不摘帽子,也不宣传。资本家劳动可以。

(毛主席插话)

为什么不做政治工作?各部可否设立政治委员?设政治委员是设立对立面。逼部长进步。管业和管人是两面。

规章制度,各方面都布置些问题,工厂报表要大减少。由几人小组负责整理一下。下次会议提出汇报。并且提出一个革命的办法。实现规章制度革命。各地来个专题鸣放。

(×××发言时的插话)

无产阶级之风压倒资产阶级之风,正风压倒邪风。

现在有些虚,不是(实)计划,要措施,不要措施。工人没有信心。许多事要有具体措施,才有保证。计划要和措施结合,否则计划会落空。地方工会、产业工会应下放由省、市管。

没有办法时,睡一觉起来开会就有办法。

对六十条你们要提出意见,取消什么?增加什么?

“酒、色、财、气”,酒是粮食,色是生育,财是财金,气是干劲。一样不能少。

(毛主席插话)

工业方面,全国平衡,超产部分,地方与中央分成,由地方调动。地方协作也可以平衡。六十条。加一条协作关系。

(×××发言时的插话)

八年中只有两个半年,大家很值得注意,肇源县去年百分之六十的亩产达到四百斤,东北、华北、西北地区为什么不能?

乌克兰称为苏联的谷仓,为什么东三省不能称为中国的谷仓?

成都灭鼠经验,不搞就不搞,要搞就两礼拜消灭。

各省的第一书记和参加会议的部长同志。大家要读一读威廉斯著的土壤学。从那里面可以清楚为什么会增长。土壤学是农业的基础科学.好象医生的解剖学。日本农业并不高明,

我们苦战三年就可以赶上去。不要请他们来插手。要请社会主义国家的。我们对帝国主义国家决不开门。日本现在跃跃欲试。

协作会议应多开,一月一次或两月一次,不超过三个月。每次两天就够了。

“农业机械化(包括拖拉机)靠地方制造为主。还是靠中央为主,恐怕要靠地方,地方自办为主,国家支援为辅。头两年所需油料,钢材和高级技术人员.也可以地方为主,中央帮助。以后自己解决。

拖拉机社有或大部分社有。

“苦战三年,基本改变本省面貌。在七年内实现农业四十条。实现农业机械化。争取五年完成。”各省可不可以这样提?特别是农业机械化问题,各省可以议一下。

对工业化不要看得太神秘了。看农业机械化看得太神秘了,但忘记了一条,有××,许多事情就好办了,有葫芦,照样子一画就行了。机械化了,合作化就可以最后巩固起来。我国农业机械化,可以很快实现。

小社势必会合并一些。合并后仍然不能搞的(指机械化)可以联社搞。

(毛主席插话)

中国实现社会主义不要一百年。可以五十年。个别行业可以试办一些办法和经验。可不可以先由一个省先进入共产主义?

整个中国农业机械化,要打破陈旧观念.可以试办,可以缩短时间。外行解决问题来得快。还得内行跟着外行跑。恐怕是个原则。今年修水利,不是谭××等同志,靠些内行一百年也修不出来。

学习苏联和一切国家先进经验,是我们任何时候都要做的一件事。更要坚持。同时又要独创,独立思考,但要防止不学外国,防止两极化。

农业机械化的所有制如何?现在苏联已改变。过去苏联是耕者无其机。是否以社有或大社所有。合作社买不起的。恐怕也要贷点款。

(×××发言时的插话)

省的工作应该从三分之二的人口出发。作到粮食自给。

工业和农业同时并举。是在一九五六年四月“十大关系”中提出的。在以前我们也没有认识这个问题。辽宁工业为主,八年吃了这个亏。一开始就提出并举,可不可以?这个问题也可研究。提出并举的时间也许迟了一点,但是宣传上有很大偏差,一直是讲工业化。没有把农业放在恰当的位置上。总路线宣传提纲中强调了工业化,未强调农业。对农业机械化,过去也讲得很远,现在看有两个到三个五年计划就可以实现。过去有忽视农业的思想,认为农业落后似乎是应该的。

中国的社会主义建设路线,是在八年内逐步形成起来的。时间不算很长。中国革命路线是经过多少年才形成的。一九三五年遵义会议未完全形成,一九四一年到一九四五年才完成。建党、北伐、内战时期未形成中国自己的政治路线。那时有“左”倾又有右倾。即陈独秀右倾路线,三次“左”倾路线,抗日时期王明路线,这就没有可能形成。从一九:一年到一九四二年共二十一年。到‘七大’时才形成了一条完整的政治路线。社会主义建设的路线,八年不算长。还不能算形成。再有五年就差不多了。苦战三年也可能形成。过去革命中损失很大,八年建设中也受了一些损失,但损失不大。同时这么时期也顾不上,抽不出手来抓建设。如去年春季到夏季右派进攻,一九五○年到一九五三年抗美大部分力量在朝鲜,一九五五年合作化高潮,也难得抓建设。对事物的认识,对客观规律的认识,是在实践中才能认识清楚。现在切实抓一下,苦战三年,建设路线就可以形成。没有陈独秀主义、王明路线,就没有比较。一九五六年下半年,斯大林问题发生,我们每天开会,一篇文章写了一个月。又发生了波匈事件,注意力又集中到国际方面。现在才有可能抽出时间来研究建设。开始摸工业.现在要苦战三年,邢成一条中国社会主义建设的路线。

电气化这个名词不好,叫电力化好。

(谈到三年实现亩产四百斤时)不要吹得太大,还是五年计划争取三年完成。这么快法有点发愁。可以活动一点,再看一看。

解决相互关系要分析一下。一种是剥削者与被剥削者的关系,右派、中间派与工人是剥削与被剥削关系,另一种是劳动者内部关系。党政工团和工人农民的关系,不同于一般劳动者之间的关系,因为有“五气”,不是平等关系,不是普通劳动者的关系,是官与民的关系。在这一点说来,同国民党一样。“五气”是资产阶级给我们的,我们从旧社会来,当然有。单是所有制改变,工人、农民不感到与我们是平等的,不批评我们。整风反右解决了这两个方面,反了右派,也批评了干部,干部整改,解决了领导与被领导的相互关系问题。使工人党得真解放了。工人生产情绪大增。过去是为官做工,“计件打冲锋,计时磨洋工”,和国民党一样,为五大件奋斗。一日所有制,一日相互关系,一日分配,这是经济学。所有制和分配改变了。相互关系未改。工人觉悟都大大提高。说“(不清楚)”“八路又回未了”。要抓住这件事,凡是做得不彻底的要继续搞。

(吴德发言时的插话)

(讲到要克服动口不动手的官气。安于现状的暮气。怨天怨地的怨气和制度,执行计划是春天必大,秋天必小时)计划不合实际,很值得注意。去年粮食三干七百亿斤,今年四千七百亿斤。靠住靠不住。有暮气,值得注意。

(×××发言时的插话)

山上到处搞梯田,搞鱼鳞坑。

是否民主革命较早的老区,对社会主义革命不那么积极?两年前河北,西北都有些情况,十年前陕北即此情况。过去曾发生老土改区社会主义的劲差一些。原因是新区土改后接着搞合作社,群众没有习惯于“新民主主义秩序”——实际是资本主义民主秩序,发展资本主义。不断革命就是从这里来的。去年以来有变化,是好现象。

全国有三个一千七百万(指人口)即陕西、江西,广西。对那些(搞指标过高的)也需要压缩空气。

应普遍提高人工翻地。一年翻一部分。三、五年翻完,可保持三年到五年丰收,这是改良土壤的基本建设。《人民日报》应读把土壤学宣传一下。

农具展览(包括人力的,不只是耕作的,而且要有加工的,运输的)在今年四月间搞起来。

苏联技术出口.我们依样画葫芦。并不那么神秘,工业化,机械化不要那么迷信。也不要迷信科学家。科学家的脑筋中总有一部分不科学的。

大家同去找一个大学当教授,发聘书.每月讲一次,一年讲几次,学柯庆施,都要有著作。在座的同志,中央委员,一年作两篇文章,一业务,一政治,专深红透。

中国历来男人是农民,女人是工人。女人是食品制造,纺织……男人造原料,所以男人心粗。

(谈到农村搞工业问题时)比较大的最好是乡政府搞。国、社、私三者合营。国家也不一定投资,赚的钱多少可以分一点红。

中等技术学校都归地方管,学生分三分之一给地方,或三分之二,或二分之一。农学院完全归地方,医学院三分之一归中央。

(毛主席插话)

(谈到三车辆精简机构问题时)这是劳动组织问题。两种形式那种好?这不忙作结论,铁道部也不要说全世界都查了,没有这样的。

(谈到勤工俭学时)资本主义国家就是半工半读,专读书就是最坏的,见书不见物,脱离实际,四体不勤。不一定会自给,有半自给,四分之一自给。方针是不要全读书,一定要又读又劳动。我们民族又穷又白,省下来的钱多办学校,中小学可大办,农业学校也要发展。只有教育发展了,才能赶上英国。靠文盲建设不起社会主义。

苏联有几百万知识分子,我们要上千万的知识分子,美国就怕这一点。

(毛主席插话)

(谈到水利局反对修东渠的问题时)科学家不科学。水利局应登报检讨。种草很重要,要加进去,覆盖面上要有草。

只要提出问题,各地就想办法解决,南宁会议只提出若干地方工业赶上农业总产值,并没有提出办法。可是现在各地解决了。

西北四省、山西、黑龙江、吉林、蒙古及其他少数民族地区,不要提倡节育.本省也要有地区的分别。

麦子的穗太短,如何研究培养一种新的品种,穗很长的,那就很好。

(谈到群众集资问题时)不搞就不搞,一搞就搞这么多,我们这个民族就是这样。

(谈到除四害、扫盲时).大鸣大放,提出问题,几个星期,面貌大变。农民不见得那样保守。




(××发言时的插话)

公私合营全国取消定息,内蒙、新疆、青海等少数民族地区也不取。

(王恩茂发言时的插话)

牧区的改造经验很好。因为在社会主义包围中,他是不安心的,这和西藏不安心是一样的。

中国创造了一条经验,合作化增多。农业,牧业都如此。……

下次会议,要把工业当成中心,大家要摸一下。六、七月开这样一次会,再下一次讨论一下文教,请大家准备。商业也要讨论一下。中央和地方合署办公。

新疆地区分散,加工工厂必须分散办。流动加工厂、轧花、面粉、榨油、化肥。这个办法可以在人广地稀的地方应用。二万五千元可以搞一个流动汽车加工厂。水多的地方可以搞船上流动加工厂。

不要以为天下太平。不太平是正常现象,也不过是个把指头。但不能任其泛滥,不及早注意就会传染,一变二,二变三,发展下去就会天下大乱。

(当谈到民族主义者希望出匈牙利事件时)实质就在这里。三中全会议题之一就是民族主义,过去我们反对大汉族主义,现在就主动了。小平报告中提出这个问题,引起异议。

许多人过去看不清楚,如李世农过去就未看出来。山东八个地委,两个反对省委,两个拥护,四个中间动摇,后来摇过来了。但未彻底搞,问题未解决。现在省委指挥不灵,也是一条经验。

首先是阶级消灭,而后是国家消灭,而后是民族消亡,全世界都是如此。

(×××发言时的插话)

分两种情况。一种有反党集团,广东、广西、安徽、浙江、山东、新疆、青海,八省区都有,要推翻领导自己挂帅。也有另一种情况,像四川那样做,是右派活动。是不是各省都有大同小异,这是阶级斗争的规律。阶级斗争发展到这个阶段,隐藏在党内的资产阶级分子一定会暴露出来,不出来反而是怪事。党内思想动向值得注意。阶级斗争情况如何?可谈一谈。

摸工业、摸农业,摸阶级斗争,就是要找马克思主义。当然按业务来讲,还有文教和商业,文教和商业有相当一部分属于阶级斗争。要逐步研究马列主义,要研究理论,结合解决当前实际来研究马克思主义。

双铧犁不能用,是因为思想不能用,脑子不能用,不是客观不能用。可见思想是统帅。思想动态要当成一个阶级斗争问题。应首先抓。在委员中应经常座谈,平常不谈不好,平常没有意识到就不好了。有的省对思想讲的少,不在意识,山东一开会就发现。

这里是否有两条路线的问题:一条多快好省,一条少慢差费。是否有?明显地有:一为排、大、国,一为蓄、小、群,这不是两条路线吗?把水排走是大禹的路线,从大出发。依靠国家(过去依靠国家修了好多水库),现在是蓄为主,小型为主,群众自办为主。河南的水利就是两条路线的斗争。

农业比工业更难些。盲目性是慢慢克服的。所以盲目性就是对客观的必然性不认识,因而也没有自由。什么叫自由?就是对客观世界的认识,对客观必然性的认识,自由是对必然的了解。自由和必然是对立的。所谓盲目性即对必然没有认识。农民上肥料,知其然不知其所以然。对农业不了解,就不自由。对客观世界的认识是逐步的,不可能一下子就认识。例如治淮排涝是曾希圣发明的。他是曾颜。在他以前山西太行山的和尚张凤林,在高阳县发明了治水的方法,他和一个雇农发明了鱼鳞坑。现在全国推广。他是蓄、小、群,不是排、大、国。当然他并不那么系统。经过我们许多同志一帮忙,就系统化了。把漭河等经验一总结,总结出了葡萄串,满天星。蓄小群为主,当然也要排大国。社会主义建设路线,也是逐步形成,现在不能说已经形成,至少还有五年,苦战三年再加两年,如工农业不大出乱子,路线就差不多了,就可以说形成了。五年加八年,共十三年,付出一部分代价,无疑是浪费一点,群众痛苦,时间延长,苦闷一点,但成绩总是主要的。

十五年赶上英国,二十年赶上美国,那就自由了。学苏联首先在路线上学。斯大林基本上正确,但有错误。他们不工农并举,反对大中小。我们是大中小结合,基础放在小的上,靠地方,靠小的。中央是标准设计,干部、技术。盲目性是慢慢克服的,对客观必然性是逐步认识的。没有克服以前,那就是盲目性,就是自然界的奴隶。对于社会斗争,去年反右以前,我们也是奴隶,因为你对右派这个客观现象不太认识嘛!不认识社会主义国家的阶级斗争,不了解客观世界,就是客观世界的奴隶。

(谈到实现农业发展纲要时)辽宁三年,广东五年,是左派,三年恐怕有困难,可以三年到五年。上面打长一点,让下面超过。

(谈到劳动中工伤事故时)工业也有,花这一点代价赶上英国,也是要付的。各省准备死五百人,一年一万多人,十年十万人,要有准备。

化肥太多破坏土质,还是以自然肥料为主。

河南水利全国第一,达四千八百万亩。

这次会议应对时间取得一致看法,除四害可三年到五年,一年突击,三年推广,第五年扫尾。粮食是五年至十年。绿化三年到五年。这样两本账,有伸缩,好些。

工业发展必然同购买力相符合,否则,像匈牙利工业产品没有出路怎办。工业产品必须和购买力平衡,这是一条原则。

党、政、军、商业机构缩小,技术机构扩大。

(谢富治发言时的插话)

云南明朝以前是少数民族,以后才开始的。

(谈到水利化时)现在算成三年,大修水利。现在搞政治运动,为了多打些粮食。社现在可考虑除了地广人稀的地区外,搞大型社,可议一下。当然不是回去就并,而是五年之内。要逐渐并。

农村房子很不卫生,在十年内应改为砖房,不要茅房,这不发生爱国不爱国的问题。青岛、长春最好,成都就不如重庆,开封不如青岛。应有一个计划,十年内改变。房子样子搞好一点,不要封建主义,应搞些标准设计,采取因地制宜。几年丰收的合作社,可以逐步建筑农民的房子。苦战七年到十年,改变农村房子的基本面貌。拆除城墙,北京应当向天津、上海看齐。

报纸是一个材料部,它反映很快,也很经常给我们提出问题。

过去许多资产阶级分子在办事,反右派后,工厂并未搞坏,反而更好了,这是生产关系的改革。

中央的人没有上课,总有一天要比输的。比输也好,我们下去你们上来,一直下到当老百姓。

(周林发言时的插话)

说人民内部矛盾经典作家没有讲,这个话不对,列宁说:“一个郑重的党,对于自己的错误,向群众公开承认,找出错误的原因,加以克服。”这就是处理人民内部矛盾问题,但未这样来说。工人阶级、共产党内部经常不统一的,参差不齐的。我们这些人那么统一了,三个月不开会就不统一了。因为各人所得的情报,材料和观点不同,就不统一。开会就是为了达到一致,不统一才开会,统一了还开什么会。

整风没有整好的要补课。不然,总有一天要暴露出来的。

军队中废除肉刑——打骂枪毙,不是处理人民内部矛盾吗?三大纪律,八项注意,不是调整矛盾的政策吗?当时推行这些政策不是长期工作之后才行通的吗?

各省市要准备出一点乱子,群众中出了乱子,领导中出了乱子,要有精神准备。不要采取赫鲁晓夫式的答复:没有矛盾。杜勒斯看到很多人对人民内部矛盾的报告有幻想,他读了几遍,他看见抱有幻想很危险,他就以全世界资产阶级总司令的资格说话,指出这个危险。英国报纸有些观念是对的,但他也摸到点气候。去年春季波兰拥护,中国右派拥护那篇文章,而左派摇头。苏联现在敢于说人民内部矛盾,但不说领导与被领导的矛盾。杜勒斯很赏识我那篇文章,他们注意七月一日的社论。他们很注意我们这些东西。资产阶级很注意研究我们这些东西,我们干部为什么不注意研究呢?美帝为什么注意我们的动向,因为它将要灭亡,总想看到我们的弱点,把芦苇当渡船。

请各省、市抓一下工业,抓一个月,没有一个月抓两个礼拜,然后到北京去开会.还要抓思想,抓理论,这是纲。以后口里要触到马列主义,现在是不讲政治经济学,不讲辩证法,也不讲自然科学,只要部门经济学。以后要略带一些理论色彩,报纸的社论,也应略带一些理论色彩,以此为荣。

大社可以办一些加工厂,最后由乡办,或几个乡联合办,县办社助,手工业社办工矿。

(谈到工商户要把股票给商联时)接受了被动,对内好,对外不好。每年只拿一百元,要把资产阶级的政治资本剥夺干净,帽子戴起来,对我们有利。要强迫他们要。他们交了股票,手里无股票,头上无帽子,政治资本就在他们手里。

群策群力。群策,即大鸣大放,大家出主意;群力,即大家动手。这个路线古来就有。

现在不科学的风气要转变一下。

(谈到民主党派誓师问题时)可以搞,交心可以,不要交服,另外要帮助他们动员知识分子参加。

(谈到工厂劳动强度问题,白天工作晚上出大字报出工伤事故时)应通报。着重要技术革命,大字报数量,不要追求。

(谈到少数民族闹事时)应当用解决人民内部矛盾的办法去解决。而不用战争去解决。贵州的事与川西不同,川西有五万枝枪,他先攻我。

去到少数民族地区,要批评过去欺负少数民族不对,解放后我们也有一个指头不对,不经常与群众说这一条,群众就会改变态度。

对四川西部藏族叛乱,八擒八纵,百擒百纵,比孔明的七擒七纵多九十三擒九十三纵,对杜聿明、王跃武,也准备纵。

(柯庆施发言时的插话)

(谈到工业的生产竞赛时)根本解决这个问题,要推翻资本主义、帝国主义,搞个世界政府,地球政府,五年计划分工合作。

(谈到整改工作时)这与去春不同,去春夹杂着敌我矛盾,看不清楚。现在反保守,比反浪费目标鲜明,批评领导,领导觉得越批评越舒服。

知识分子,科学家的态度也在改,这些人要就不改,要就突变。

现在有些过去常写东西的人,现在不写东西,因为他们还在过渡状态中,旧的破了,新的未建立起来。资也破得差不多,无还没有建立起来,有一点也不多,所以难写。过些时候就会写出来。

(谈到整风中工人当中阿飞和“废品大王”态度有很大转变时)过去许多资产阶级思想还统治着我们党,工人、农民,还没有兴起来。现在变了。无兴起来了,无的自由就扩大了,横竖自由××××,一定要把资产阶级思想灭掉。有些人感到不自由,是资产阶级思想还没破尽。资灭多少,无的自由就有多少,有你的自由,就无我的自由。“废品大王”本来是资产阶级思想统治着的,第五天资灭掉,无就兴起来了。

中间派现在是不敢动笔的。只要把无产阶级兴起来,他就有大的自由,才能写出东西。

现在是过渡时期,需要的小说不是大作品,而是写一些及时反映现实的中篇,短篇,像鲁迅的那些作品。鲁迅并没有写什么大作品嘛!现在是兵荒马乱时期,大家忙的很,知识分子还未改造好。大作品是写不出来的,我们也一样,没有创造一件,都是把群众和下级创造出来的东西加以提倡,不接近群众如何能提倡好的东西。创作也是一样。也必须和人民接近。听人民和下级干部的话。

没有民主哪来的集中?过去国际范围内的民主集中是一句假话。

因为集中是建立在民主之上,没有民主就不可能有很好的集中。民主集中制首先是民主,然后是集中,没有真的民主,群众的热情和创造怎么能发挥出来呢?

右派帽子也可以摘掉,全靠自己改造。右派这个对立面转过来将我们的军,也是一种推进工作的力量。

学制,课程要由各省市去研究改变,有了典型,教育部门才能改出来。历来统一的东西,都是由典型到普遍的。

孔子是一个学派,是许多学派中的一个,到汉朝的时候,政府才加以提倡和推广,之一学派得到发展。

在取消定息问题上,我们准备处于被动,总是不松口,这样于我们有利。

对资本家的薪金部分工资高一些,是为了“赎买”,目的是把他的政治资本完全剥夺净尽。必须向工人作解释工作。工人阶级不要和资产阶级比,不可比,比不得,这是两个不可比的阶级。资本家工资太高的也可以不动为好,一动就不好了,就给他们增加了政治资本。他们吃“五个菜”政治上就被动,他们的薪金高,说话声音小。

要和中间派作朋友,也要找几个右派分子交朋友,作工作,现在连我们这些中央委员都怕沾右派的边。那怎么行?怎么了解他们呢?

有些左派,例如邓初民在理论问题上是真左派,在政治问题并非真左派。

(谈到培养理论干部时)现在已经是理论落后于实际。

(陶铸发言时的插话)

总路线就全党来说,是逐步完备的。开始提出工业化是不太完备的。没有这次社会主义革命,要把小学教师中的反革命分子搞出来是不可能的。民主革命时期没有解决这个问题。

这次山东的一个教训是没有划右派,没有搞臭、搞透,是非没有分清楚。鸣放不够,以致现在指挥不灵。广东问题较彻底。

全国青工、青农、青年学生、社会青年,确实需要很好的教育,要在鸣放中让他们知道天有多高,地有多厚。

雪花膏按马克思主义原则可以搞,但是苦战三年,不搞也可。

反对地方主义教育,全国各省市都需要进行。

对地方主义者,实际是右派,是资产阶级在党内的代表。

王明究竟怎么处理?开除不行,拿出讨论也不必要,还是让他住在苏联有利,再拖二十年,赶上英国再说。

右派开除党籍,地方民族主义者不开除党籍咋行?

教育主义是资产阶级性质,比较容易改,右派是资产阶级中的反动派,不容易改,两者不同。

没有南方的布尔什维克到东北、华北、西北建立根据地,先取北方后取南方,革命咋能胜利?现在把南方干部北调,各地干部互相渗透,对工作有好处。

对地方主义不要让步,要派一批外地人去广东,广东干部可调一批到北京来。泥里掺沙,沙里掺泥,改良土壤。天下人写八宝饭,不能单打一万和九万。掺的政策是有利的政策,区乡不在内,可以清一色,县上掺外来干部。现在省、专的负责人,大部分是外地去的,对反地方主义感到理不直,气不壮。应采取列宁的办法:“与其你专政,不如我专政。”

这次实际是一次清党,一千二百万党员中清除二十万、百把万,五、六十万,不算多。这比苏联几次清党人数大,方式好,经过群众,民主。

销售点多设,排队购买的现象是可以消灭的。

总路线、规律,总是经过反复才得出来的,规律就是经常出现的东西。美国的经济状况,二月份增加失业者七十万人,达到五百二十万人。衰退——萧条——危机。苏联二十次大会,对资本主义的估价是有毛病的。

七届二中全会对社会主义问题是讲清楚的,当时没有公开讲,直到一九五三年才讲,原因是抗美援朝,恢复经济,土地改革,但是作的百分之八十是社会主义的,百分之二十是半社会主义的,当时不讲有个策略问题,例如孙行者,糖衣炮弹,这些不好公开讲,邓老根本不管“七届二中全会”,他搞“四大自由”,他说是河南取来的经验,但为什么不从西北坡取经,而从河南取经?

二中全会决不是突如其来的,是在整个民主革命过程中有了思想准备,在民主革命过程中我们就看出社会主义因素,如在江西打土豪分田地,就看出其中有很多社会主义因素,当时红军拿着武器,但是跟老百姓讲,平等,那是社会主义;群众耕田队,那也是社会主义萌芽,当时在陕北就讲那是另一种革命。当时在安寨发现一个安全集中的合作社,我们很感兴趣,并发展了互助组,这些为二中全会作了准备。但是没有唤起更多的人注意。例如邓老仍靠“四大自由”,也不跟中央商量,我说他资产阶级思想根深蒂固,他死不承认,直到七月三十日(一九五五年)他才缴出武器。因此说,邓老有资产阶级思想,但这个人是好的,可以改造的,作为思想问题,经过严肃对待,坚持原则,改得彻底。但是有些同志对此却避开锋芒,表示宽宏大量,无非是怕不好混,不好共事,或者怕失掉选举票。马克思主义者不要隐瞒自己的观点。我对邓老讲过,要改造你的思想,不是撤你的副总理的职,不开除你的中央委员,但对许多错误思想党内要作严肃斗争。在原则问题上,共产党员要有明确的态度,但有的人怕“打击别人,抬高自己”,有相当庸俗的空气。思想阵地你不插旗子,他就插旗子。

革命路线吃过苦,经济建设路线不能迷信苏联,不破除迷信要妨碍正确贯彻执行建设路线。

(王任重发言时的插话)

一九五八年的劲头,开始于三中全会,许多事没有料到的,如一九五六年斯大林问题,匈、波事件和一九五六年反冒进。当时对于社会主义革命以为只是所有制问题,而没有弄清那只是小部分,还有生产关系的其他方面。

右派、反冒进都是对我们有压力,人民内部干群关系中也存在问题,心情并不舒服。经过整风反右派,关系改变了,大家的思想不得到解放,如铁道工人的节约,就可修六千公里的铁路。

技术决定一切,政治思想不要了?干部决定一切,群众不要了?全面的提法就是又红又专。领导和群众相结合,要技术又要政治思想,要干部又要群众,要民主又要集中。



\section[对《中国农村社会主义高潮》一书《按语》的批示(一九五八年三月十九日)]{对《中国农村社会主义高潮》一书《按语》的批示}
\datesubtitle{(一九五八年三月十九日)}


这些按语见《中国农村社会主义高潮》一书中,是一九五五年九月和十二月写的,其中有些现在还没有丧失它们的意义。其中说:一九五五年是社会主义与资本主义决战取得基本胜利的一年,这样说不妥当。应当说,一九五五年是在生产关系的所有制方面取得基本胜利的一年,在生产关系的其他方面以及上层建筑的某些方面即思想战线方面和政治战线方面,则或者还没有基本胜利,或者还没有完全胜利,还有待于今后的努力。我们没有预料到一九五六年国际方面会发生那样大的风浪,也没有预料到一九五六年国内方面会发生打击群众积极性的“反冒进”事件。这两件事,都给右派猖狂进攻以相当的影响。由此得到教训:社会主义革命和社会主义建设都不是一帆风顺的,我们应当准备对付国际国内可能发生的许多重大困难。无论就国际方面说来,或者就国内方面说来,总的形势是有利的,这点是肯定的;但是一定会有许多重大困难发生,我们必须准备去对付。


\section[对《上海化工学院两个右派分子的大字报》的批语(一九五八年三月二十二日)]{对《上海化工学院两个右派分子的大字报》的批语}
\datesubtitle{(一九五八年三月二十二日)}


可以一阅。有资产阶级的自由,就没有无产阶级的自由;有无产阶级的自由,就没有资产阶级的自由。一个灭掉一个,只能如此,不能妥协。更多地更彻底地灭掉了资产阶级的自由,无产阶级的自由就会大为扩张。这种情况在资产阶级看来,就叫做这个国家没有自由。实际是兴无灭资。无产阶级的自由起来了,资产阶级的自由就被灭掉了。



\section[和《江峡》轮船员的谈话(一九五八年三月二十八日)]{和《江峡》轮船员的谈话}
\datesubtitle{(一九五八年三月二十八日)}


毛主席对江峡轮三副、女青年石若仪说:“当你对一件事物还不了解时,往往是害怕的。正如蛇一样,当人们还不了解它,没有掌握它的特性时,感到十分害怕,但是一旦了解了它,掌握了它的特性和弱点,就不再害怕了,而且可以捉住它。”接着,毛主席又问石若仪:“你在船上工作了多久?”当石若仪说到有四年多了的时候,毛主席转过去问杨大副:“你呢?”杨大副回答说:“三十多年了。”毛主席慈祥地对石若仪说:“要好好向他们学习,他们这些老工人是你的好师傅,水上经验都很丰富,许多知识是书本上学不到的。”

毛主席还说:“有些地方航道的很不好,在三峡修一个大水闸,又发电又便利航运,还可以防洪、灌溉……。”

毛主席对江峡轮船长说:“你的经验是丰富的,要多带徒弟,把技术传给青年人。”

<p align="right">(据一九五八年四月十五日《人民日报》有关报导)</p>


\section[视察四川省一个养猪场时的谈话(一九五八年三月)]{视察四川省一个养猪场时的谈话}
\datesubtitle{(一九五八年三月)}


毛主席:(对饲养员)辛苦了,好同志!(握手)祝你们获得更大成功!

饲养员:全靠你老人家的教导!

毛主席:主要是你们自己的努力。

(对社长)能使你们全社妇女都当上模范吗?

社长:保证全能当上,毛主席!

毛主席:全当上模范未必得行吧!只能在二、三年内做到有一半妇女当上模范就不错了。你说我这个看法保守吗?(众笑)

毛主席:参观一下你们的养猪场好吗?

饲养员:欢迎,欢迎!毛主席多指教!

毛主席:别什么都要我指教吧,我和××都是来向你们学习的,在许多事情上,你们比我们内行多了,是吧?

毛主席:这是什么?

饲养员:是糖化牛粪,拿来喂猪的,猪很爱吃。

毛主席:真是新鲜事儿,牛粪也能喂猪!怎么个制法,介绍一下吧,(摸出日记本来作记录,向××说)来,我们都记上吧,这是群众的创造!从前我们就没听见说过。看来我们中国那句老话:“做到老,学到老”实在不错!

毛主席:(对一个回乡知识青年)回家喂猪有意见吗?

饲养员:有啥意见呀!我一辈子也不愿意离开这养猪场哩!

毛主席:很好!可要把你学的科学,传授给群众才更好。

毛主席:……多看看先进的东西,眼光就会更开阔些。

(亲自给一头猪打了蛋清针,然后对××说)

这里的经验真不少,特别是代饲料,如果全国都推广,一年要节约好几十亿斤粮食哩!请记者同志在报纸上介绍一下好吗?

×××:办得到。

毛主席:最好后天见报。

(对饲养员)谢谢你们!对我们教育太大了!

饲养员:可有人看不起我们哩!

毛主席:谁?谁是顽固派?

(饲养员介绍一些人如何从轻视妇女到称赞妇女的成绩。)

毛主席:看不起妇女的人虽不多,但哪里总有几个,这不完全怪他们,过去封建制度对他们影响太深了,脑筋一下不容易转过弯来,又不能用飞机大炮来对付他们。那怎么办呢?我看除了加强教育外,最好的办法,就是妇女同志们做出成绩来,多拿事实给他们看,看得多了,他们脑子里的那个封建王国就会不攻自破的。

<p align="right">(见《中国妇女》1958年第7期)</p>



\section[在汉口会议上的讲话(一九五八年四月六日)]{在汉口会议上的讲话}
\datesubtitle{(一九五八年四月六日)}


生产高潮形成的原因。现在生产高潮是怎么来的?

(1)以前有过高潮,有了领导高潮的经验,一九五五年冬一九五六年春曾有过高潮……。

(2)反冒进的错误使许多人不舒服,使干部抬不起头来。但挫折对我们很有益处。一种搞快些,一种搞慢些,这样就有了两种工作方法的比较。反冒进就慢。这几年来两高潮形成了马鞍形,一九五五年至一九五六年春和一九五七年冬目前就高,中间反冒进就低。这种形势对我们很有利。

(3)中央根据实际工作经验,在三中全会和青岛会议上及时恢复了四十条,多快好省的和做促进派的口号。

(4)经过整风反右派斗争,群众干劲起来了,干劲足了。

两个战役之间休整问题。目前的生产高潮,动员群众很广。动员这么多群众是从古以来没有过的,过去只有在战争时期,在参军上搞过大规模动员的。群众是个劳动大军,各级干部是指挥者,指挥者应当懂得在两个战役之间需要有休整,不要老紧下去,紧张之间要有调节,不要使群众总紧张下去。

同时,生产高潮中要务实,不要搞空气,不实在。苦战三年基本改变面貌。是提基本改变面貌,还是提初步改变面貌,这个口号起了动员作用,不要改,中央还要再看一个时候,改变面貌,光挖沟植树不能算,粮食、油料和棉花增产,因为我们不是布置花园,做到棉、油、粮增产才算改变面貌,挖沟只是手段,不能算目的。去年我们注意粮食、肉,从明年起要大搞油料,各省要规划,雷厉风行。四十条要增加油料增产指标和措施,为了帮助兄弟国家也要提高,还可提口号(陕西种核桃,各地还可以搞什么),如为了支援东欧国家等,这样号召力更好些,对农民也进行了国际主义的教育。

关于“化”的问题。今后《人民日报》不轻易宣传某某地方什么“化”了,有些地方稀稀拉拉种了几棵树就算绿化了,怎么行。这产生了一个缺点,这种宣传只能促进睡觉,要回去研究一下,怎样才算“化”了。种上是“化”还是长出来是“化”?除四害也不要轻易宣布“四无”。除四害今年只搞一下,取得经验,再看一看。今年要把许多事情搞完,我不相信。

报纸不要简单宣传指标,要多宣传措施,多宣传先进经验。要搞水利“好多吃大米”,这样就到问题本质了。口号新鲜,人民就看到前途了,很高兴。不要过早宣传水利化,“化”了明年怎么办?还有什么干头呢?做事情应留余地。

苦战三年以后,还需要再战。我在《正确处理人民内部矛盾》的报告中提的不是三年,而是还要战斗十几年,不要把事情简单化了。宣传要给自己留余地,讲问题要看远一点,以后不要说什么“化”了,“四无”了,将来要变成“四有”城了怎么办?宣传多了,以后就不好动员了。世界上很多事情都有真有假,没有假也没有真,在运动中的事,打了折扣才可靠。干什么都要踏实些。说苦战三年水利化,我怀疑如果如此,将来我们子孙干什么呢?今后十年内会不会遇见几次大水灾,大旱灾?三个大灾两个小灾应考虑。问题不由你我决定,三年内只能逐步改变面貌。在若干年内只能管地,还管不了天。如果来了灾这个账怎么算?三年基本改变面貌,如果发生灾,就不是三年基本改变面貌,计划中的大灾除外写上去,要留有余地。

怎样才叫绿化,种上树不算绿化,真正绿化是从飞机上看一片绿才算。现在坐在飞机上看还是一片黄色。年年有工作做,不是没事情。五年能搞掉四害,就算好。

总之,做事情要留有余地,要务实。粮食到手,树木到眼才算数。要比措施,比实际。现在很多指标还不是群众的东西,是领导在头脑中的,好多还是会议上的,到秋后看一看再说。今年是历史上大跃进的一年,把经验总结一下。

宣传工作要务实。报纸宣传要实际。要深入、细致、踏实。不要光宣传指标。现在我们的宣传只注意宣传多快,对好省宣传的不够,不好、不省怎么会改变面貌?

好大喜功是需要的,但大话是不需要的,华而不实是不好的,如果华而不实,喜功便会无功。不是喜大而是喜小,结果会轰轰烈烈之后无功而返,这就不好了。

生产高潮方面的指标,现在我担心会不会再来一个反冒进,今年干劲这样大,如果得不到丰收,群众情绪受挫就会反映到上层建筑上来,社会上就会有人说话了,喊“冒进”了。民主人士、富裕中农,党内有右倾思想的人,就会出来刮台风;观潮派“楼看沧海月,门对浙江潮”就会说话,群众有怨言,就会从上而下的反映意见,影响上层建筑。要在党内讲清楚,党内要有精神准备,给地县委讲清楚,如果收成不好,计划完不成怎么办。

现在我们的劲头很大,不要到秋天泄气,要搞措施,到十二月比实际,要看结果,吹牛不算数。实际上九月便会看出,比输了,活该。不要浮而不深,粗而不细,华而不实。今年像平津战役、淮海战役那样子,增产有希望,办法是放手发动群众,一切通过试验。湖北省有这样的话:“鼓起眼睛看丰收,干部带头,革新先试验,干劲加办法,跃进会实现。”向农民讲清楚,可能某些地区有天灾,要鼓起眼睛看丰收,也要准备无丰收。要特别注意深翻地换土,大有味道,宁可一亩地花几百个工也花得来。

调节生产节奏。做一段,休息一段,劳逸结合,要有节奏地,要波浪式前进。连续作战是由各个战役组成的,做了一个时间休息几天,悠哉悠哉,很必要。

压缩空气,河南一年要实现几个化,当然现在我们不要说他们过火了,但某些口号要调整一下,登报时要小心些。压缩空气,空气还是那么多,氧气并没有减少,只是压缩,变成液体、固体。反冒进是将氧气砍掉一半。我们压缩还要加氧气。



\end{document}