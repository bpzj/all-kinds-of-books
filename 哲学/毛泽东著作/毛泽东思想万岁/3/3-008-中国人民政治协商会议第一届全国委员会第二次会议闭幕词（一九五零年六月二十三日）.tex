\section[中国人民政治协商会议第一届全国委员会第二次会议闭幕词(一九五零年六月二十三日)]{中国人民政治协商会议第一届全国委员会第二次会议闭幕词(一九五零年六月二十三日)}


此次会议总结了过去时期的经验,决定了各项方针。

这种总结经验和决定方针的工作,是我们大家一齐来做的,是各民族、各民主阶级、各民主党派、各人民团体和各界民主人士的代表人物集合在一起来做的。这里,不但有人民政协全国委员会的委员们,而且有中央人民政府,各大行政区人民政府(军政委员会)及各省市人民政府的许多工作人员列席参加讨论,而且有各省市各界人民代表会议协商委员会的代表们列席参加讨论,而且有许多特邀的爱国人士列席参加讨论。这样我们就能集中广泛的意见,检查过去的工作,决定今后的方针。这种方法,我希望我们今后继续采用,并且希望各大行政互人民政府(军政委员会)和各省市人民政府也采用这种方法。我们的会议在暂时还是建议性质的会议。但是在实际上,我们在这种会议土所做的决定,中央人民政府是当然会采纳并见之实行的,是应当采纳并见之实行的。

我们一致同意了全国委员会的会务报告和中央人民政府的各项工作报告。这里有土地改革工作,政治工作,军事工作,经济和财政工作,税收工作,文化和教育工作,法院工作等项报告,这些报告都是好的。在这些报告中,适当地总结了过去时期的工作经验,规定了今后的工作方针我们这些会议所以有这样多的议题,是因为我们的浙国家成立之后,各方面工作都在开创,都在发展,全国人民正在蓬蓬勃勃地在各个战线上开展真正人民革命的伟大斗争,在军事战线上,在经济战线上、在思想战线上,任土地改革的战线上都是从古未有的极其伟大的斗争,各项工作都待总结,都待指示方针,所以我们有了这样多的议题。我们的会议按法律规定是每年开会两次,其中将有一次为议题众多的会议,一次为议题较少的会议。中国是一个大国。实际的人口超过四万万七千五百万,又处在人民革命的伟大历史时期,这种情况要求我们这佯做,我们也就这样做了,我想我们是做得对的。我们这次会议议题很多,中心的议题是将旧有土地制度加以改革的问题。大家同意刘××副主席的报告及中共中央建议的土地改革法草案,并对土地改革法草案作了若干有益的修改和补充。这是很好的,我为新中国数万万农村人民获得翻身机会和国家获得工业化的基本条件而表示高兴,表示庆贺。中国的主要人口是农民,革命靠了农民的援助才取得了胜利,国家工业化又要靠农民的援助才能成功,所以工人阶级应当积极地帮助农民进行土地改革,小资产阶级和民族资产阶级也应当赞助这种改革,各民主党派各人民团体更应当采取这种态度。战争和土地是在新民主主义的历史时期内考验全中国一切人们、一切党派的两个“关”。什么人站在革命人民方面,他就是革命派,什么人站在帝国主义封建主义官僚资本主义方面,他就是反革命派。什么人只是口头上站在革命人民方面而在行动上则另是一一样,他就是一个口头革命派,如果不但在口头上而且在行动上也站在革命人民方面,他就是一个完全的革命派。战争一关,已经基本上过去了,这一关我们大家都过得很好,全国人民是满意的。现在是要过土改一关。我希望我们大家和过战争关一样也过得很好。大家多研究,多商量,打通思想整齐步伐,组成一条伟大的反封建统一战线,就可以领导人民和帮助人民顺利地通过这一关。只要战争关、土地关都过去了,剩下的一关就将容易过去的,那就是社会主义的一关。在全国范围内实行社会主义改造的那一关。只要人们在革命战争中,在革命的土地制度改革中有了贡献,又在今后多年的经济建设和文化建设中有所贡献,等到将来实现工业园有化和农业社会化的时候(这种时候还在很远的将来),人们是不会把他们忘记的,他们前途是光明的。我们的国家就是这样稳步前进。经过战争,经过新民主主义的改革,而在将来,在国家经济事业和文化事业大为兴盛了以后,在各种条件具备了以后,在全国人民考虑成熟并在大家同意了以后,就可以从容地妥善地走进社会主义的新时期。我认为讲明这一点是有必要的,这样可以使人们有信心,不致彷徨顾虑,“不知道什么时候你们不要我了,我虽然想为人民效力也没有机会了。”不,不会这样的,只要谁肯真正地为人民效力,在人民还有困难的时期内确实帮了忙,做了好事,并且是一贯地做下去,并不半途而废,那么,人民和人民的政府是没有理由不要他的,是没有理由不给他以生活的机会和效力的机会的。

在这个远大目标上,在过外,我们必须坚固地团结苏联、各新民主主义国家及全世界一切和平民主力量,对此不可有丝毫的游移和动摇。在国内,我们必须团结各民族、各民主阶级、各民主党派、各人民团体及一切爱国民主人士,必须巩固我们这个已经建立的伟大的有威信的革命统一战线。不论什么人凡对于这个革命统一战线的巩固工作有所贡献者,我们就欢迎他,他就是正确的;凡对于这个革命统一战线的巩固工作有所损害者,我们就反对他,他就是错误的。要达到巩固革命统一战线的目的,必须采取批评和自我批评的方法。采取这种方法时所用的标准,主要是我们现时的根据大法即共同纲领。我们在这次会让中,即根据共同纲领,采取了批评和自我批评的方法。这是一个很好的方法,是推动大家坚持真理、修正错误的很好的方法,是人民国家内全体革命人民进行自我教育和自我改造的唯一正确的方法。人民民主专政有两个方法。对敌说来是用专政的方法,就是说在必要的时期内,不让他们参与政治活动,强迫他们服从人民政府的法律,强迫他们从事劳动并在劳动中改造他们成为新人。对人民说来则与此相反,不是用强迫的方法,而是用民主的方法,就是说必须让他们参与政治活动,不强迫他们做这做那,而是用民主的方法向他们进行教育和说服的工作。这种教育工作是人民内部的自我教育工作,批评和自我批评的方法就是自我教育的基本方法。我希望全国各民族、各民主阶级、各民主党派、各人民团体和一切爱国民主人士,都采取这种方法。


