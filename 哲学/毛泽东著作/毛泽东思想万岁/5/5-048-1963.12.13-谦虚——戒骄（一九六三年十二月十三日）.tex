\section[谦虚——戒骄(一九六三年十二月十三日)]{谦虚——戒骄}
\datesubtitle{(一九六三年十二月十三日)}


固步自封,骄傲自满,对自己所管的区域的工作,不采取马克思主义的辩证分析方法(一分为二,既有成绩,也有缺点错误),只研究成绩一方面,不研究缺点错误一方面;只爱听赞扬的话,不爱听批评的话,对于外省、外市、外地区、别的单位的工作,很少有兴趣组织得力的高级、中级干部去虚心地学习,认真地加以考察,以便和本省、本市、本地区、本单位的情况结合起来,加以改进,永远限于本地区、本单位这个狭隘的世界,不能打开自己的眼界,不知道还有别的新天地,这叫做夜郎自大。对外国人、外地人或中央派下去的人,只让看好的,不让看坏的。只向他们谈成绩,不向他们谈缺点及错误。要谈也谈得不深,敷衍几句了事。中央屡次对同志们提出这个问题,作为一个共产党人,必须具备对于成绩与缺点、真理与错误这两分法的马克思主义的辩证思想。事物(经济、政治、思想、文化、军事、党务等等)总是作为一个过程而向前发展的。而任何一个过程,都是因矛盾着的两个侧面互相联系,又互相斗争而得到发展的。这应当是马克思主义者的普通常识。但是中央和各地同志中有很多人都很少认真地运用这个观点去思索,去工作,他们的头脑长期存在形而上学的思想方法而不能解脱。所谓形而上学,就是否认事物的对立统一,对抗斗争(两分法),矛盾着、对立着的事物,在一定条件下互相转化,走向他们的反面这样一个真理,就使人固步自封,骄傲自满,只见成绩,不见缺点,只愿听好话,不愿听批评的话,自己不愿批评(对自己的两分法),更怕别人批评。

“满招损,谦受益”这句话,站在无产阶级立场上,从人民的利益出发考虑,是一个真理。

(一)骄傲自满可以在各种情况下,以各种不同的形式产生和滋长。但是一般说来,通常在胜利的情况下,就更容易产生和滋长骄傲自满情绪。这是因为当处在困难的时候,一般是容易看到自己的弱点,也是比较谨慎的,而且客观上的困难摆在面前,不虚心谨慎也不行。可是每当胜利的时候,由于有人感谢,有人赞扬,甚至过去的敌人也会掉过脸来奉承一番,阿谀一番,因而就容易为胜利的环境冲昏头脑,而全身轻飘飘起来,真以为“天下从此定矣。”我们党深深地懂得,越是在胜利的时候,骄傲自满的细菌就越是容易袭击党。

(二)产生骄傲自满情绪,一类是在胜利的情况下产生的,那就是胜利冲昏头脑,自以为了不起;另一类是在无特殊胜利,亦无特殊失败的平常情况下产生的,他们经常以比上不足,比下有余来安慰自己,并原谅自己的不进步,他们还善于用“没有功劳,也有苦劳”,“二十年媳妇熬成婆”等等来自我陶醉;再一类是在落后的情况下产生的,即虽然已经落后了,也还是骄傲。他们认为“我们工作虽然没有做好,比过去总是好多了”,“某某同志或某某单位还不如我们呢!”他们每每炫耀自己的历史,三句话不到就是“想当年……”讲起来眉飞色舞。

(三)只要我们稍微忽视一下群众的力量,我们就会骄傲起来,只要我们眼界狭隘一些,只看到局部而看不到整体,我们就会骄傲起来;只要我们稍微把成绩估计得高了一些,把缺点估计得低一些,我们就会骄傲起来;只要我们的主观认识落后于客观事物的发展,我们就会骄傲起来。

(四)骄傲自满的情绪,从本质上说,乃是从个人主义的立场上引伸出来的,同时它又培养和滋长了个人主义,因而骄傲自满的本质就是个人主义。

(五)就阶级根源来分析,骄傲自满基本上是剥削阶级思想,其次则是小生产者的思想。

(六)小生产者就其本身是劳动者一面而言,他们是具有很多优点的。他们勤劳朴实,刻苦谨慎和实事求是。但是就其本身是小私有者一面而言,则他们是个人主义的,更重要的是由于他们的劳动条件和劳动方式是落后的生产工具,分散经营,眼界不广,见闻不多。因此,他们往往看不到集体的力量,而只是看到个人的力量。另一方面,他们也容易满足,当他们取得一点微小的成绩以后,就产生“这个不错了”,“这也到顶了”,“该享享福了”以及“比上不足,比下有余”等等这一类思想。

(七)骄傲自满是在资产阶级唯心世界观的基础上派生出来的,它使人在看待周围客观事物时经常违背事物发展的规律,把人们引向失败的道路。唯物主义的历史观证明:社会发展的历史,不是个别英雄人物的历史,而是劳动人民群众的历史。可是骄傲自满的人,总是夸大个人的作用,居功自傲,而忽视、低估群众的力量。

(八)因此,骄傲自满在本质上是反马克思列宁主义的,是反对党的辩证唯物主义和历史唯物主义的世界观的。

(九)骄傲自满的人往往不能忘情于自己的许多优点。他们一方面把自己的许多缺点掩盖起来,而另一方面又企图把别人的许多优点抹煞掉。他们经常拿别人的缺点和自己的优点相比,从而私自窃喜,看到人的优点则又觉得“没有什么了不起”,“算不上个啥”。

(十)事实上,把自己估计得越高,所得的结果就越坏。俄国大文豪列夫·托尔斯泰就幽默地说过:“一个人就好像是一个分数,他的实际才能好比分子,而他对自己的估计好比分母,分母越大,则分数的值就越小。”

(十一)谦虚,它是每一个革命工作者都应有的美德。因为谦虚对人民的事业有利,而骄傲自满却会把人民的事业引向失败。所以,谦虚也是对人民事业负责的一种表现。

(十二)一个人要真正称得上一个名副其实的革命工作者,必须做到下列两点:首先,他们必须尊重群众的创造,肯听群众的意见,把自己看成是群众的一员,毫无自私自利之心,毫不夸大自己的作用,实心实意地为群众工作。这种精神就是鲁迅先生所说的,“俯首甘为孺子牛”的精神,也就是谦虚的美德。

(十三)其次是他必须有不屈不挠、永远向前的精神,时时刻刻保持清醒的头脑,对新鲜事物具有敏锐的感觉,缜密的思考能力。因此他们必须始终保持谦虚的态度,胜不骄,败不馁,不贪天下之功,也不满足已有的成绩,这种精神就是实事求是的精神,也是高贵的谦虚美德。

(十四)一个人如果能够认真的从工作中、生活中和其他实践斗争中去学习,经常总结自己的思想和行动,无情而坚定地和骄傲自满情绪作斗争,并且毫不保留地加以彻底的克服,那他是完全可以锻炼成为一个具有谦虚美德的人。


(十五)真正具有谦虚的高尚品德的人,他必须是满腔热情地无条件地为党、为人民、为集休的事业而忠诚不渝地积极工作的人。他之所以积极工作,不是为了炫耀自己,也不是为了获得某种奖励和荣誉,不夹杂任何自私自利的欲望和要求在内,而是全心全意地为着给人民带来愉快与利益。因此,他总是埋头苦干地作着对党对人民的革命事业有利的工作,从不抛头露面,从不计较自己的地位、自己的声望、自己的待遇,他不仅不在别人面前夸耀自己的功勋和成就,而且在自己内心中也不让这些功勋和成就占地位,他全付精力所考虑的是更好地为人民工作。

(十六)真正的集体主义者,为什么必须要求自己具有谦虚的美德呢?

第一、因为他懂得,他的一切知识和成就的获得,虽然他自己也尽了一定的力量,但更重要的是由于群众的努力,没有群众的帮助和支持,他就不可能获得知识,也就不可能获得工作上的成就。作为一个集体主义者,他就认为不应该抹杀群众的功绩,不应该掠他人之美,贪别人之功。因而他觉得自高自大是可耻的。

第二、因为他懂得,他所学习到的一些知识,所作的一些工作,在整个知识宝库里和整个革命的工作当中,仅仅是“沧海中之一粟”,是非常渺小的,因为革命的知识和革命的工作,又是在不断发展的。他既然是一个集体主义者,他便要用他宝贵的生命去最大限度的获得对人民有用的知识,最大限度的对革命事业贡献自己的力量。因此,他就觉得不应该固步自封,替自己关起进步的大门。

第三、因为他懂得,整个革命事业像一架大机器,是由大小各式轮盘、螺丝、钢架和其他机件紧密结合而构成的,谁也少不了谁,他既然是一个集体主义者,他便觉得应该尊重每一个人的工作,尊重每一个人的成就,就像尊重自己的工作和成就一样。为了把革命工作做得更好,他就必须使自己的工作和别人的工作紧密配合,他感到他离不开集体,他热爱自己的伙伴。因此,他便必然会用谦虚的态度来待人接物,而不会对任何人狂妄自大。

第四、因为他懂得,一个人的眼界往往是窄小的,能够看到的范围总是有限的,而革命知识和革命工作的范围却是极为广阔的,并且内容又是非常丰富的、非常复杂的。因此,他便进一步懂得了任何人总难免会有若干缺点,会犯若干错误,这些缺点和错误又常常不是自己全部及时觉察得到的。他既然是一个集体主义者,为了要把革命工作搞好,为了对人民负责,他就得要求自己看得更深更广,要求能及时地发觉自己的缺点和错误,以便迅速改正。因此他便要虚心恭谨地向别人学习和请教,他便要诚恳地欢迎别人对他的批评。

由此可知,真正的从集体利益出发的人,是必须具备谦虚的精神的。而谦虚实质上就是高度的革命热情,强烈的群众观点,旺盛的进取精神和科学的实事求是的态度的集中反映。

(十七)克服骄傲自满和培养谦虚品质的另一个根本的方法,就是要努力提高自己的共产主义觉悟,这就必须加强马克思列宁主义理论的学习。为什么?

(十八)因为马克思列宁主义理论,可以帮助我们科学地认识世界,认识个人与群众,个人与集体,个人与组织,个人与党的相互关系。正确地认识人民群众和个人在革命斗争中的作用。马克思列宁主义告诉我们:劳动人民是社会财富的创造者和革命斗争的基本力量。我们要在中国建立社会主义和共产主义社会,只有依靠工人阶级及其先锋队领导下的亿万劳动人民的无穷无尽的创造力量。至于个人,在革命事业中只不过是一个小小的螺丝钉,马克思列宁主义告诉我们:任何一个成就都是集体力量的结晶,个人是离不开集体的,个人想做一点事业,如果没有党的领导,没有组织和人民群众的支持,就将会寸步难行,一事无成。如果我们真正深刻地理解了人民群众和个人在历史上的作用及其相互关系,我们便会自觉地谦虚起来。

因为马克思列宁主义的理论可以提高我们对前途和方向的认识,开阔我们的眼界,使我们的思想从狭隘范围里解放出来。当人们的眼界只能看到脚下,看不到高山和大洋的时候,他就会像“井底之蛙”那样自负不凡的。但当他抬起头来,看到宇宙之大,事物之变化无穷,人类事业之雄伟浩壮,人才之多和知识之无极限,他便会谦虚起釆。我们所从事的事业,是天翻地覆的大事业,我们不仅要看到我们自己的眼前的工作和幸福,而且要看到整个的、长远的、全面的工作和幸福。马克思列宁主义帮助我们朝前看,而不是朝后看;帮助我们全面地、客观地看问题,而不是片面地、主观地看问题,因而就能帮助我们克服那种因小小成就、小小胜利而自满自足的小生产者的思想,而促使我们孜孜不倦、力求进步的渴望,同时又可以帮助我们克服唯心的主观主义的思想方法。

(十九)谦虚和自卑不是同义词,谦虚并不等于小视自己。因为谦虚本身是实事求是的态度,是正视客观现实的进取精神的表现。而自卑却是一种非实事求是的、缺乏自信力的、对困难采取畏缩的态度的表现。

自卑和自夸,自高自大,同样都是错误的,都是以主观主义为其思想垦础的,是对自己的两种极端的主观主义的错误的估计。那些自高自大的人,离开了客观实际,把自己估计得过高,夸大了自己的实际能力和作用,因而他总是自命不凡,自以为了不起,他就不再前进了,他也不能及时地吸收什么新鲜的事物了,他于是就不可避免的要犯错误。那些自卑的人,虽然从表面上看和自高自大的人相反,但同样也离开了客现实际,把自己估计得过低,忘记了自己还可以努力提高自己,还可以从工作中锻炼自己,过分地降低了自己在革命事业中所已经起的和可能起的作用。于是,便从而丧失了前进的勇气和自信。松懈了斗争意志。

总之,无论是自高自大或自卑,同样都是错误的估计了自己在革命事业中的作用,都是非实事求是的非科学的态度,因而都是错误的,都会使革命事业遭受损失。

所以,我们不仅要坚决反对骄傲自满,自高自大一类的习性,而且要严格地把谦虚与自卑的界限划分开来,免得由一个极端又倾向于另一个极端。
