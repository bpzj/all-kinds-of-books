\section[关于军事工作落实与培养革命接班人的讲话(一九六四年六月十六日于十三陵)]{关于军事工作落实与培养革命接班人的讲话(一九六四年六月十六日于十三陵)}
\datesubtitle{(一九六四年六月十六日)}


讲二个问题:一个是地方党委抓军事问题,二是要搞接班人。……光看表演不行。要抓兵,要搞武器工厂……省对军队,对民兵要过问,你们省委第一书记就是政委,多少年不履行职务,只是空头政委,不抓军事,一旦发生问题,不帮忙,就会手忙脚乱。不管敌人从哪里来,要做到有准备,我们的国家就亡不了。各级党委都要抓军事工作,抓民兵工作……我们这样的国家,这样大的战线,光靠中央的几百万解放军怎么行呢?不靠,你们就得自己打注意了,守土有责……要打原子弹没话讲,他要打吗?他丢原子弹我们走。他们进城我们也进城,敌人就不敢打原子弹了,我们搞巷战,总而言之和他斗。

要把民兵很好整顿一下。从组织上、政治上、军事上整顿。组织上整顿就是基干民兵、普通民兵有多少?组织上确定下来,有战士、班、排、连、营、团、师长,而且真正起作用。还有政治工作人员也要组织起来一旦有事,拿起枪来就走。有人说,当三个月民兵精神面貌大不同啦。民兵组织要有组织,有兵、有官,要落实。现在许多地方不落实,要做政治工作,做人的工作。政治落实要有政治机构,有政委、教导员、指导员。政治工作就是做人的工作。要分清民兵中的好人坏人,把坏人清理出去。要向民兵讲清,不论出了什么大事,不要慌慌张张,你慌张还能打胜仗?打枪、打炮、打原子弹都不要慌张。政治上准备好了,就不慌了,原子弹打下来,无非是见马克思,自古皆有死,人无信不立。死就死,死不完就干。把中国人都打死?我看就不见得,帝国主义也不会干,他剥削谁呀!……二十年战争,我们不是死了许多人吗?黄公略、刘胡兰、黄继光。我们没有死,是剩下来的渣子。有什么了不起,无非是死。×××同志就是阎王招手,他没有去,现在还活着。军事上也要准备,和平时期要搞上枪,打起仗来再搞就晚了……只知搞文,不知搞武,只要人,不要枪。打起仗来要靠中国顶住,靠修正主义是不行的。敌人打进来,我们就可以打出去。总而言之,我们准备打,打起仗来不要慌张,打原子弹也不要慌张。不要怕,无非是天下大乱,无非是要死人,人总是要死的,站着死,躺着死都行,不死就干,打死一半还有一半。……对帝国主义不要怕,怕也不行,越怕越没劲,有准备,不怕,就有劲。

二、准备后事——接班人问题

帝国主义说我们第一代没问题,第二代也变不了,第三代第四代就有希望了。帝国主义这个希望能不能实现呢?帝国主义这话灵不灵?希望讲得不灵,但也可能灵,苏联就是第三代出了苏联赫鲁晓夫修正主义的,我们也可能出修正主义。如何防止修正主义?我们怎样培养操权的接班人,我看有五条:

1.要经常观察和教育我们的干部,要懂得一些马列主义,最好稍多一些马列主义。要搞马列主义,不要搞修正主义。

2.要为大多数人服务,不为少数人,要为中国大多数人服务,也要为世界大多数人服务,不是为少数人,不是为地、富、反、坏、右,没有这一条,不能当支部书记,也不能当中央书记、中央主席。赫鲁晓夫为少数人,我们是为多数人。

3.要能够团结大多数人。所谓团结大多数人,包括以前反对过自己反对错了的人,不管他是哪个山头的,不要记仇,不能一朝天子一朝臣。我们的经验证明,如果不是七大的正确方针,我们的革命就不能胜利。对于搞阴谋诡计的人要注意,如高、饶、彭、黄、张、周、谭、贾等十多人出在中央,事物都是一分为二的,有人就搞阴谋,他要搞有什么办法?现在还有要搞的嘛!如吴自立、白银厂,还有陈伯达讲的小站。各部门、各地方都有搞阴谋的人,朝中有官,下有群众,没有这种人不称其为社会。我上一次就说过,不是我喜欢有这种人,而是客观存在,不然就没有对立面,就是形而上学。一切事物都是对立的统一,五个指头,四个指头向一边,大拇指向一边,这才捏得拢,如果都向一边就没有用了。世界上没有纯的物质,没有真空,百分之九十九点九九,还有零点零一。这个道理多数人没有想通。完全纯是没有的,不纯才成为社会、物质、自然界。纯就不合乎规律。不纯是绝对的,纯是相对的,这是对立统一。扫地,一天到晚扫二十四个钟头,还是有灰有尘土。你们看,我们那一年纯过吗?我们党的历史有五朝领袖,第一朝是陈独秀,第二朝是瞿秋白,第三朝是向仲发(实际是李立三),第四朝是王明、博古,第五朝是洛甫(张闻天),五朝领袖都没有把我们搞垮,搞垮不容易,这是历史经验。帝国主义也好,我们自己冒出来的也好,都没有把我们搞垮,解放以后又出了高岗、饶漱石、彭德怀,搞垮了我们没有?也没有。彭德怀当国防部长七年也未把解放军搞垮。几品官一出来就没有希望了。要别人讲,不要一言堂。要团结大多数人。形式有民主,作了决议,还有说他那时未通过,×××说:中国要保持讲道理,人民解放军要保持讲道理,有了这一条,彭德怀就搞不成。

4.要有民主作风,遇事要与同志商量,要充分酝酿,总要听各种意见,反对派意见要讲出来,不要一言堂,人是可以变的,×老不是变了吗?牛可以驯来耕田,人为什么不可以变?有少数人是不能改变的,如于学忠,章伯钧,刘立明,党内有××,×××,他们是变不了的,吃了饭就骂人。还有郑位三,也是不变的,各省都有一点是极少数,不变也可以让他们去骂。要团结大多数人,我看对吴自立不要开除党籍,要劝他们改好,要团结两个百分之九十五,要讲民主,不要光是我一个人说了算,开了会赞成了又翻案。形式的民主,开会自己讲几个钟头,好像真理都在我手里,我自己年轻的时候对毛泽潭发脾气,敲棍子,他说共产党不是毛氏宗祠,我看他这个话有道理。共产党要搞民主作风,不能搞家长作风。

5.自己有了错误,要自我批评,不要总是自己对,要比较少出错主意。讲错话,出坏主意,少一点好,一个指挥员指挥打仗,三个仗,一个打败,二个打胜,就比较好,就可以当下去。……不要搞过火斗争,要帮助人家改正错误,只要他认真改正了错误,就不要总是批评没个完。

接班人就要马列主义的,要为大多数人民谋利益的,要团结大多数,要发扬民主作风,要自我批评。我想的不完全,你们自己再研究研究,部署一下。都要搞几个接班人,不要总是认为自己行,别人什么都不好,好像世界上没有自己,地球就不转了,党就没有了。死了张屠夫,就吃带毛猪?什么人死了也不怕,什么人死了就有很大的损失?马克思、恩格斯、列宁、斯大林不是都死了吗?还是要继续革命。死了一个人有什么了不起的损失,没有那回事。人总是要死的,死有各种死法,被敌人打死,坐飞机摔死,游泳淹死,细菌病死,无病老死,包括被原子弹炸死,要准备随时离开自己的岗位,随时准备接班人。每个人都要准备接班人,还要有三线接班人,有一、二、三把手,不要怕大风大浪。……


