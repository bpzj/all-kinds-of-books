\section[同参加“亚非文学交流座谈会”的亚非作家的谈话(摘录)(一九六四年十二月二十五日)]{同参加“亚非文学交流座谈会”的亚非作家的谈话(摘录)}
\datesubtitle{(一九六四年十二月二十五日)}


我们这个国家,有些好东西,也有不好的东西,譬如文化、艺术、教育方面,现在刚刚触动它们之中的一些坏东西。旧中国遗留下来知识分子我们不能不接受,不然我们就没有知识分子,没有教授,没有教师,没有新闻记者,没有艺术家。那些人,他就相信他们的,不相信我们的,我说那些人叫做坏人。他们有他们的爱好,崇拜死人和外国人,外国人也是死外国人。他们崇拜西方国家的古典作品,看不起自己,总觉得自己的不行,这就是一个教训,希望你们不要犯这个错误。当然,历史遗产要接受,但是要批判地接受。你们看,马克思批判了古典经济学,接受了古典经济学中的好东西,创造出了马克思主义的政治经济学;批评了空想社会主义,接受了空想社会主义的好东西,创造了马克思主义的科学社会主义;批评了古典哲学,接受了古典哲学的好东西,创造了唯物辩证法。我们也应当这样做,接受古典遗产的时候,就要接受好的,批判坏的。

人一脱离群众,就没有好结果了。人民群众总是占大多数,剥削者、压迫者总是极少数。还有一条:人是会改变的,在一定的条件下。马克思这个人,从前是唯心主义者,后来起了变化,起初是形而上学的,后来学了辩证法,学的也是唯心主义的辩证法,马克思也是变过来的。恩格斯和列宁,也都是有这个变化,我们自己也是这样。没有受到大学教育,我原来当小学教员的,不知什么原因把我抛到革命中来了,大概是帝国主义教会了我们,教的方法是用杀人、压迫、剥削。你们是讲文的,我是讲武的,因为打了几十年仗,先生就是蒋介石,还有日本法西斯,还有美国,(当初)我们什么也不懂。现在修正主义又来整我们。赫鲁晓夫很快下台了,留下做反面教员多好!现在我们来出。

现在美日反动派已在教育日本人。各国都是一样。没有激烈斗争,教不好人。

(回答凯尔的问题)团结起来,击败帝国主义和各国反动派。搞了一辈子,就是这件事。

(回答普端纳)美国说他是福利国家,究竟谁得到福利?还不是垄断资本!他收买一些工贼,组织社会党、社会民主党右派。一些共产党也不像话,他们听社会党的话,说我们不好。他们亲帝国主义,亲反动派,他们是本国工人阶级的叛徒。我国就出了这样的人,共产党反对共产党,这事不足奇,我们头一个共产党领导人,就变成托匪。共产党历来是招收一批,跑出一批。志贺义雄不就跑出去了吗?不足为奇。他们能起一种作用——反面教员的作用。过去我们有五万党员,白色恐怖一来,剩下一万,一部分杀掉了,一部分投降了,一部分不干了,我们变成少数。有时正确方面常常是少数。达尔文时代,只有一个人相信是会进化的。达尔文是高级知识分子,但不是从大学学的生物学,他是到处跑。正确的开始总是少数,共产党当初只有马克思、恩格斯两个人,其余都是蒲鲁东、巴枯宁等等。所以你们不要怕孤立,有一个正确的就行了,何况你们还有许多人。\marginpar{\footnotesize 203}

帝国主义、修正主义、各国反动派都是可以打倒的,这一条也是破除迷信。他们心是虚的,是脱离群众的。我们有亲身经验,你们也有经验,他们是可以打倒的,我们还没有想到赫鲁晓夫倒得这么快。

