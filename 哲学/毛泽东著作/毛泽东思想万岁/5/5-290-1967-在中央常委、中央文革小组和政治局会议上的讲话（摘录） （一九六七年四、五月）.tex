\section[在中央常委、中央文革小组和政治局会议上的讲话(摘录) (一九六七年四、五月)]{在中央常委、中央文革小组和政治局会议上的讲话(摘录) (一九六七年四、五月)}
\datesubtitle{(一九六七年四)}


我们一定不要脱离群众,不能脱离群众是一条;另外一条就是不能脱离马列主义。

我们党在四九年、五〇年、五一年这三年当中,群众是拥护我们的,是尊重我们的,因为当时是艰苦朴素的,吃小米,住帐篷。当时刚打完仗,还有饱满的革命热情,和群众有密切的联系。

一九五二年以后情况就发生了一定的变化,我们干部在群众当中开始受冷落。当时,在干部当中实行了薪金制,军队住了营房,机关盖了高楼大厦,过去和群众在一起吃、穿、住,现在有些脱离群众了。为什么会这样?就是没有听我的话。刘少奇、高岗、彭德怀学习了苏联那一套。薪金制我是不赞成的。学苏联那一套我也是不赞成的。我们这次文化大革命要把它改变过来。

我们现在要搞三结合,要使青年参加各方面的领导工作。不要看不起青年人,二十几岁,三十几岁都可以接受他们作事情,不把新一代搞上来怎么使他们受到锻炼?这个三结合就是老、中、小,就是二十岁以上就行了。

我们提倡青年人上台,有人说青年人没有经验,上台就有经验了。过去也提培养无产阶级革命事业接班人,那是从形式上讲的,现在要落实在组织上。

三结合,老、中、小要三结合,不主张把老干部都打倒,老干部一天天见上帝了。

国家机关的改革最根本的一条就是联系群众,机构改革要适合联系群众,不要搞官僚机构。

过去党团员受“修养”的影响脱离了群众,没有独立的意见,成了驯服工具,各地不赞成过早地恢复党团组织,过半年或一年后再恢复。文化大革命不仅对干部,而且也是对党团员的大审查,大多数一定是好的。有的干部群众意见较大,可过二、三年以后再说工作,有些干部可以立即恢复工作。对于犯错误的人要给以改正的机会。联动要放出来,没有右派,就没有左派。搞薪金制军衔我从来就反对。


