\section[在中央政治局汇报会议上的讲话(一九六六年十月二十四日)]{在中央政治局汇报会议上的讲话}
\datesubtitle{(一九六六年十月二十四日)}


毛主席说:“有什么可怕呢?你们看了李雪峰的简报没有?他的两个孩子跑出去,回来后教育李雪峰说:‘我们这里的老首长,为什么那么害怕红卫兵呢?我们又没打你们’。大家就是不检讨。伍修权家有四个孩子,分为四派,有很多同学到他家里去,有时十几个人。接触多了就没有什么可怕的了,觉得他们很可爱。自己要教育人,教育者要先受教育。你们不通,不敢见红卫兵,不和学生说真话,做官当老爷,先不敢见面,后不敢讲话,革了几十年的命,越来越蠢了,刘××给江××的信,批评了江××,说他蠢,他自己就聪明了吗?”

毛主席问刘澜涛:“你回去打算怎么办?”刘回答:“回去看看再说”。主席说:“你说话总是那么吞吞吐吐。”

毛主席问总理会议情况,总理说:“会议开得差不多了,明天再开半天,具体问题回去按大原则解决。”主席问李井泉:“廖志高(四川省委第一书记),怎么样?”李答:“开始不大通,会后一般较好。”主席说:“什么一贯正确,你自己就溜了,吓得魂不附体,跑到军区去住。回去要振作精神,好好搞一搞。把刘邓的大字报贴到大街上去不好。要允许人家犯错误,要允许人家革命,允许改嘛!让红卫兵看看《阿Q正传》。”

主席说:“这次会开得比较好一些,上次会是灌而不进,没有经验。这次会有了两个月的经验。一共不到五个月的经验,民主革命搞了二十八年,犯了多少错误,死了多少人!社会主义革命搞了十七年,文化革命只搞了五个月,最少得五年才能得出经验。一张大字报,一个红卫兵,一个大串连,谁也没料到,连我也没料到,弄得各省市呜呼哀哉。学生也犯了一些错误,主要是我们这些老爷们犯了错误”。

主席问李先念:“你们今天会开的怎么样?”李答:“财经学院说他们要开声讨会,我要检讨,他们不让我说话。”主席讲:“你明天还去检讨,不然人家说你溜了。”李说:“明天我要出国。”主席讲:“你先告诉他们一下。过去是‘三娘教子’,现在是‘子教三娘’。我看你有点精神不足”。

主席说:“他们不听你们检讨,你们就偏检讨,他们声讨,你们就承认错误。乱子是中央闹起来的,责任在中央,地方也有责任。我的责任是分一二线。为什么分一二线呢?一是身体不好,二是苏联的教训。马林可科不成熟,斯大林死前没有当权,每一次会议都敬酒,吹吹捧捧。我想在我没死之前,树立他们的威信,没有想到反面。(××同志插话:“大权劳落”)主席说:“这是我故意大权旁落,现在倒闹独立王国,许多事情不与我商量,如土地会议、天津讲话、山西合作社、否定调查研究、大捧王光美。本来应经中央讨论,作个决议就好了。邓小平从来不找我,从一九五九年到现在,什么事情都不找我。五九年八月庐山会议我是不满意的,尽是他们说了算,弄得我是没有办法的。六二年,忽然四个副总理,李富春、×××、李先念、×××到南京找我,后又到天津,我马上答应,四个又去了,可邓小平就不来。武昌会议我不满,高指标弄得我毫无办法。到北京开会,你们开六天,我要开一天还不行。完不成任务不要紧,不要如丧考妣。遵义会议后,党内比较集中,三八年六中全会后,项英、彭德怀搞独立王国。(新四军皖南事变、彭德怀的百团大战)那些事情都不打招呼。七大后中央没有几个人,胡宗南进攻延安,中央分两路,我同恩来,任弼时在陕北,刘××、朱×在华北,还比较集中。进城后就分散了,各搞一摊,特别分一线二线就更分散了。一九五三年财经会议后,就打过招呼,要大家相互通气,向中央通气,向地方通气。刘邓二人是搞公开的,不是秘密的,与彭真不同。过去陈独秀、张国焘、王明、罗章龙、李立三都是搞公开,这不要紧。高岗、饶漱石、彭德怀都是搞两面手法。彭德怀与他们勾结上了,我不知道。彭真、罗瑞卿、杨尚昆、陆定一是搞秘密的,搞秘密的没有好下场,好结果。犯路线错误的要改,陈、王、李没改(周总理插话:李立三思想没改),不管什么小集团,不管什么门头,都要关紧关严,只要改过来,意见一致。团结就好。要准许刘邓革命,允许改。你们说我和稀泥,我就是和稀泥的人。七大时陈奇涵说:不能把犯王明路线的人选为中央委员,王明和其他几个人都选上了中央委员啦。现在只走了一个王明,其他几个人还在嘛!洛甫不好,王稼祥我有好感,东崮一战他是赞成的,宁都会议洛甫要开除我,周、朱他们不同意,遵义会议他起了好作用,那个时候没有他们不行。洛甫是顽固的,××同志是反对他们的,聂荣臻也是反对他们的。对×××不能一笔抹杀。你们有错误就改嘛!改了就行。回去振作精神,大胆放手工作。这次会议是我建议开的。时间这么短,不知是否通,可能比上次好。我没料到一张大字报,一个红卫兵,一个大串连,就闹起来了这么大的事。学生有些出身不太好的,难道我们出身都好吗?不要招降纳叛,我的右派朋友很多,周谷城、张治中,一个人不接近几个右派那怎么行呢?哪有那么干净?接近他们就是调查研究,了解他们的动态。那天在天安门上,我特意把李宗仁拉在一起,这个人不安置比安置好,无职无权好。民主党派要不要?一个党行不行?学校党组织不能恢复太早。一九五七年以后发展的党员很多,翦伯赞、吴晗、李达都是共产党员,都那么好吗?民主党派就那么坏?我看民主党派比彭、罗、陆、杨好。民主党派还要,政协也还要。同红卫兵讲清楚,中国的民主革命,是孙中山搞起来的,那时没有共产党,是孙中山领导搞起来的,反康梁、反帝制。今年是孙中山诞生一百周年,怎么纪念哪?和红卫兵商量一下,还要开纪念会。我的分一线二线走向了反面(康生同志讲:八大政治报告是有阶级斗争熄灭论),报告我们看了,是大会通过的,不能单叫他们两个负责。

工厂、农村还是分期分批回去打通省、市同学的思想,把会议开好,上海找个安静的地方开会,学生就让他们闹去。我们开了十七天会,有好处。像林彪同志讲的,要向他们做好政治思想工作。斯大林在一九三六年讲阶级斗争熄灭了,一九三九年又搞肃反,这还不是阶级斗争。你们回去要振作精神做好工作,谁会打倒你们?


