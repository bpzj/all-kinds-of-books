\section[反浪费反保守是当前整风运动的中心任务(一九五八年二月八日)]{反浪费反保守是当前整风运动的中心任务}
\datesubtitle{(一九五八年二月八日)}


整风运动在全国的企业、事业单位和国家机关里,目前出现了一个新的洪峰,这就是以反浪费和反保守为中心掀起了一个新的鸣放高潮和整改高潮。在反浪费反保守的大鸣大放中,中央各国家机关内贴出了二十五万张大字报;北京市三十一个企业三十天的统计,职工们就贴了三十万张大字报,提出了四十三万条意见。运动声势浩大,锋芒集中在一个方向,贯彻多快好省勤俭建国的方针,促进生产和工作的大跃进。

从各地区各企业和各机关的情况看来,这次的反浪费反保守运动,同过去历次的增产节约运动有很大的不同。这次运动实际上已经成为反对思想、政治、经济各方面落后现象的斗争,已经形成了广泛地比先进,比多快好省的高潮。在这个波澜壮阔的运动中,很多束缚群众积极性和生产力发展的陈规被冲破了,很多长期不能解决的根本性问题顺利的解决了,各方面的生产和工作已经有了明显迅速的改进。一月份国民经济计划执行情况就是一个很好的说明。历年来,一月份生产和基建计划总是完成的最不好的,年初松,年中紧,年底赶,几乎成了一个定例。而今年却一反积习,一月份的工业总产值越额百分之二点五完成了月计划。基本建设的情况也比过去任何一年都好。再如商业部门中的反浪费反保守运动,虽然还开始不久,某些先进单位却已经在广大职工觉悟充分提高的基础上解决了许多长期没有彻底解决的问题。北京天桥百货商场在反浪费反保守运动中大胆地突破常规,提出了并且实现了减少人员,节约流动资金,改善服务态度的措施,并且纠正了在商业企业中机械地形式主义地搬用在工业企业中工作八小时的现象,实行了一班到底的工作制度。

这个声势浩大的运动,显然是一九五七年我国人民在思想战线、政治战线上的社会主义革命的产物。正因为这样,群众在这个运动中决不满足于克服生活中的铺张浪费,也决不满足于仅仅要求产量指标的突破。许多企业在辩论了浪费的性质、原因和如何堵塞漏洞等问题以后,得出了一个共同的结论:造成浪费的责任应该由领导工作人员、技术人员和工人三方面担负,这三方面都必须同时改进。领导工作人员往往有官僚主义、主观主义、不深入地钻研业务的毛病;技术人员往往是重业务不重政治,墨守陈规,不善于发动和依靠群众的积极性创造性;工人群众中也有许多人没有正确对待个人和国家的关系,没有正确解决为谁劳动的问题。在许多单位的辩论会上,三方面的人都同时揭发和批判了自己的缺点,这样就打掉了官气、暮气和邪气,资产阶级思想受到抵制,无产阶级思想大大抬头。

以反浪费反保守为纲,带动了各方面的工作,这就是当前整风运动的显著特征。我国,全民性的政治战线上和思想战线上的社会主义革命,其最终目的本来就是为了要把社会主义各方面的建设工作大大推进一步。经过前两个阶段的大争大辩,群众的觉悟大大提高了。在十五年赶上英国和苦战波三年,改变面貌的伟大号召的鼓舞下,群众不能不要求生产和工作的大跃进,不能不反浪费反保守。灿烂的思想政治之花,必然结成丰满的经济之果。这是完全合乎规律的发展。有些单位对于这个形势认识不足,在运动中忽视思想工作,只算经济账,简单地从技术上采取一些措施,而不认真开展群众性的争辩,不彻底转变工作方法和领导作风。这样他们就不能从根本上杜绝浪费现象,克服保守主义,引导生产的大跃进。因此,目前的斗争既然是一个经济上的斗争,同时又是一个思想政治的斗争,既要算经济账,又要算思想账、政治账。通过大鸣大放,大争大辩,不但要反掉浪费,反掉保守,而且要反掉官僚主义、宗派主义和主观主义。要通过和结合反浪费反保守的斗争.彻底改进干部和群众的关系。提高全体职工群众的社会主义觉悟,打破那些妨碍生产力迅速发展的陈规,精简机构。改善生产管理和劳动组织,改进生产技术,降低生产费用.以便贯彻执行多快好省的方针,促进生产的大跃进。

有一些人虽然认识了思想工作的重要,但是他们采取的办法却是错误的。他们组织群众去抽象他讨论一些原原则性题,结果在辩论会上,群众往往不知所云。当然,重大的原则问题,例如个人和集体,自由和纪律,工农关系,工人阶级的领导地位等等,都必须在群众中辩论清楚。但是在目的阶段,这种辩论必须针对生产和工作具体任务。很多企业.通过反浪费反保守的具体辩论,引导广大群众认清了上述原则,并且也边辩边改,立即见诸行动。拿这种作法同前种作法相比,岂不生动得多,深刻得多吗?我们说以反浪费反保守为纲,首先就是说要以它作为当前整改阶段的纲,通过它末完成当前的整改任务。决不能把这两件事分割开来,如果抛去群众最关心的问题不管,不去因势利导,从解决具体问题中去解决思想,那就必然会失败。

许多企业、学校和机关已经决定,要把反浪费反保守运动作为整改阶段的中心,这是正确的。希望全国所有企业、学校和机关都向他们看齐,争取整风运动的这个新任务的彻底胜利,从而使我国社会主义事业实现一个全面的大跃进!

<p align="right">(一九五八年二月八日《人民日报》社论)</p>


