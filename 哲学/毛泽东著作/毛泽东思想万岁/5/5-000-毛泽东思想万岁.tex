\documentclass[b5paper,oneside,12pt]{ctexbook}
\usepackage[hmargin=0.25in,vmargin=0.5in]{geometry} 
\usepackage[]{hyperref}

\pagestyle{plain} %整书页眉页脚设置
\ctexset{chapter/numbering=false}
\ctexset{
    section={numbering=false, afterskip = 0ex},
    subsection={format=\large\heiti\centering,numbering=false,beforeskip=1ex,afterskip = 1.75ex}
}
\newcommand\datesubtitle[1]{{\centering\large #1\par\vspace{1ex}}}  %自定义日期副标题格式,为了保险,最好使用两层大括号

% 下面是修改了脚注样式
% 一些LATEX内部命令含有@字符,如\@addtoreset,如果需要在文档中使用这些内部命令,就需要借助于另两个命令\makeatletter和\makeatother.
\makeatletter
% 无上标的 \@makefnmark
\def\nosuper@makefnmark{\hbox{\normalfont\@thefnmark\space}}
% 补丁
\usepackage{etoolbox}
\patchcmd\@makefntext{\@makefnmark}{\nosuper@makefnmark}{}{}
% 给脚注编号前后添加 〔〕
\renewcommand\thefootnote{{〔\arabic{footnote}〕\hspace{-0.64em}}} 

% 引用样式
\newenvironment{yinyong}{%
    \begin{list}{}{\parsep\parskip
        \setlength\topsep{0pt}
        \setlength\itemindent{2em}%
        \setlength\parindent{2em}
        \setlength\listparindent{2em}
        \setlength{\leftmargin}{2em}
        \setlength{\rightmargin}{2em}
        \kaishu
    }
    \item[]
}{
  \end{list}
}

\title{毛泽东思想万岁5}
\author{毛泽东}
\date{}

\begin{document}

\frontmatter
\maketitle
\tableofcontents

\mainmatter
% \chapter{}
\input{5-001-1961.1.13-在中央工作会议上的讲话(一九六一年一月十三日)}
\input{5-002-1961.1.18-在八届九中全会上的讲话(一九六一年一月十八日).tex}
\section[《反对本本主义》说明(一九六一年三月十一日)]{《反对本本主义》说明\\{\large(原名:“关于调查研究”)}}
\datesubtitle{(一九六一年三月十一日)}

这是一篇老文章,是为了反对当时红军中的教条主义思想而写的。那时没有用“教条主义”这个名称,我们叫它做“本本主义”。写作时间大约在一九三〇年春季,已经三十年不见了。一九六一年一月,忽然从中央革命博物馆里找到。而中央革命博物馆是从福建龙岩地找到的。看来还有些用处。印若干分供同志们参考。

{\raggedleft 毛泽东\\一九六一年三月十一日\par}



\input{5-004-1961.3.27-《中央关于认真进行调查研究工作问题给各中央局、省、市、自治区党委一封信》(摘录)(一九六一年三月二十七日).tex}
\input{5-005-1961.3-在广州会议上的讲话(节录)(一九六一年三月).tex}
\input{5-006-1961.4.28-在接见亚洲、非洲外宾时的谈话(一九六一年四月二十八日).tex}
\input{5-007-1961.4.19-在接见古巴文化代表团时的谈话(一九六一年四月十九日).tex}
\input{5-008-1961.5.30-调查成灾一例(一九六一年五月三十日).tex}
\input{5-009-1961.6.12-在北京会议上的讲话(一九六一年六月十二日).tex}
\input{5-010-1961.7.30-给江西共产主义劳动大学的一封信(一九六一年七月三十日).tex}
\input{5-011-1961.9.6-对《各地贯彻执行六十条的情况和问题》的批语(一九六一年九月六日).tex}
\input{5-012-1961.9.29-给政治局常委及有关同志的信(一九六一年九月二十九日).tex}
\input{5-013-1961.1.7-在接见日本朋友时的谈话(一九六一年十月七日).tex}
\input{5-014-1962.1-关于科学研究十四条的指示(一九六二年一月).tex}
\input{5-015-1962.1-给郭沫若回信中的几句话(一九六二年一月).tex}
\input{5-016-1962.1.3-在扩大的中央工作会议上的讲话(一九六二年一月三十日).tex}
\input{5-017-1962.5.3-接见几内亚政府经济代表团和妇女代表团的谈话(一九六二年五月三日).tex}
\input{5-018-1962.8.6-在北戴河中央工作会议上的讲话(一九六二年八月六日).tex}
\input{5-019-1962.8.9-在北戴河中央工作会议中心小组会上的讲话(一九六二年八月九日).tex}
\input{5-020-1962.8.12-对中共中央组织部的批评(一九六二年八月十二日).tex}
\input{5-021-1962.9.24-在八届十中全会上的讲话(一九六二年九月二十四日上午怀仁堂).tex}
\input{5-022-1962.7.24-在八届十中全会上对工业支援农业的指示(一九六二年七月二十四日).tex}
\input{5-023-1962.1.1-关于电台的指示(一九六二年十月一日).tex}


\input{376-革命委员会的三条基本经验.tex}
\input{377-支持美国黑人抗暴斗争的声明.tex}
\input{378-对派性要进行阶级分析.tex}

\end{document}