\section[听取××同志汇报会上的指示(一九六五年二月二十一日上午)]{听取××同志汇报会上的指示(一九六五年二月二十一日上午)}
\datesubtitle{(一九六五年二月二十一日)}


我们现在有些东西保密得太过分了。技术上很落后,你给人家人家不要,这个问题也要一分为二。一、有的非保密不可,如……要保密;二、有些加工技术,有什么密可保!?根本不要保密。

接见海军干部工作会议、《解放军报》编辑记者会议和第三批战士演出队时的指示(一九六五年二月二十二日)

四个第一好。我们以前也未想到什么四个第一,这是个创造。谁说我们中国人没有发明创造?四个第一就是创造,是个发现。我们以前是靠解放军的,以后仍然要靠解放军。正面教员与反面教员(一九六五年二月)

革命的政党,革命的人民,总是要反复地经受正反两个方面的教育,经过比较和对照,才能够锻炼得成熟起来,才有赢得胜利的保证。我们的中国共产党人,有正而教员,这就是马克思、恩格斯、列宁、斯大林。也有反面教员,这就是蒋介石、日本帝国主义者、美帝国主义者和我们党内犯“左”倾或右倾机会主义路线错误的人。如果只有正面教员而没有反面教员,中国革命是不会取得胜利的。轻视反面教员的作用,就不是一个彻底的辩证唯物主义者。

{\raggedleft (摘自《赫鲁晓夫言论集》第三集的出版说明)\par}


