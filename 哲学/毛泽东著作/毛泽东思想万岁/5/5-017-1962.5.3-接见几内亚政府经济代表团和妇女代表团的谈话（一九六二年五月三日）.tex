\section[接见几内亚政府经济代表团和妇女代表团的谈话(一九六二年五月三日)]{接见几内亚政府经济代表团和妇女代表团的谈话}
\datesubtitle{(一九六二年五月三日)}

\begin{duihua}
\item[\textbf{主席:}] 你们是来自友好国家、友好政府的代表团,欢迎你们。所有非洲的朋友,都受到中国人民的欢迎。我们与所有非洲国家人民的关系都是好的,不管是独立或没有独立正在斗争中的人民。非洲正出现一个很大的争取民族独立,反对帝国主义、反对殖民主义的革命运动。非洲有多少人口?二亿吧?二亿人民要翻身,不管已经站起来或者将要站起来。还有拉丁美洲也是二亿人口,亚洲的十几亿人口和全世界的革命人民。我们不是孤立的,到处都有我们的朋友,你们也不是孤立的。你们来中国可以感到中国人民是十分欢迎你们的。来了几天了?

\item[\textbf{凯塔(几政府经济代表团团长):}] 我们是四月十九日末的。

\item[\textbf{主席:}] 她们呢?(指妇女代表团)

\item[\textbf{凯塔:}] 她们是四月二十五日来的。

\item[\textbf{主席:}] 听说你们明天要走了。

\item[\textbf{凯塔:}] 她们不走。

\item[\textbf{主席:}] 欢迎。他(指柯庆施同志)是上海的主人,柯庆施是中共中央政治局委员,同你们是民主党的中央政治局委员一样,对吗?

\item[\textbf{柯庆施:}] 你们为何不多住几天?

\item[\textbf{凯塔:}] 我们的日程排得很紧,国内工作很多,五月十五日以前要完成改组党的各级机构,如有时间,我们很愿意在中国访问一个月。\marginpar{\footnotesize 25}

\item[\textbf{主席:}] 你们的觉是很好的党,是个联系群众的党,有纪律的党,是一个有以反对帝国主义、反对殖民主义和建立民族经济作为纲领的党,一个独立自主国家的领导的党。我们感到同你们是很接近的。我们两国、两党互相帮助,互相支持,你们不捣我们的鬼,我们也不捣你们的鬼。如果我们有人在你们那里做坏事,你们就对我们讲。例如看不起你们,自高自大,表现大国沙文主义态度,有没有这种人?

\item[\textbf{凯塔:}] 没有。

\item[\textbf{主席:}] 如有这种人,我们要处分他们。

\item[\textbf{凯塔:}] 有些国家的技术人员有这种情况,中国专家没有这种情况,他们都工作得很好。

\item[\textbf{主席:}] 是不是有比你们几内亚专家薪水高、特殊化的情况?(对叶××说)恐怕有,要检查,待遇要一样,最好低一些。(叶××:周总理正在要方×同志检查。)

\item[\textbf{凯塔:}] 这个问题是值得研究的,但直到现在中国专家并没有过分的要求,有些国家的专家要比几专家高二、三倍,相反,中国的专家没有过分的要求。

\item[\textbf{主席:}] 驻几内亚大使是谁?是柯华吗?(旁人答是的)

\item[\textbf{凯塔:}] 只有中国专家和越南专家待遇一样。

\item[\textbf{主席:}] 是否有人损害你们的民族利益?搞颠覆活动?

\item[\textbf{凯塔:}] 有,但不是中国人。对搞颠覆活动的人,我们也不是听任他们去搞的,发生这种情况,我们要迅速采取措施加以回击,我们不愿意做人家的尾巴。过去发生的事件你们是知道的,我们对这些事件的态度你们也是知道的。

\item[\textbf{主席:}] 你们做得对。凡有人在你们那里称王称霸,不服从你们的法律,搞颠覆活动,应把他们赶掉。我们希望你们站住脚,不仅在政治上,而且要在经济上站住脚,不要被人颠覆掉了。你们站住脚我们高兴。你们倒台我们不高兴。因为你们是一个革命的党,是一个革命的政府,在非洲有很大的影响。经过你们,可以在非洲许多国家做工作,使他们得到解放。你们也有这个责任,不要自己独立就不管别人了。我们也一样,不能因为自己独立了就不管别人了。所谓管别人是指友好的支持、帮忙。你们知道我们现在还有些困难,帮忙不大。再过五年、十年我们的情况可能好一些,那时的帮助可能多一些。我们的国家,有一个很大的缺点,人太多,这么多人要吃饭,要穿衣,所以现在还有不少困难,但这些困难不是不可克服的,而是能够克服的,正在采取措施克服。我国的经济、文化与你们差不多,差不多是在没有什么遗产的情况下搞起来的。你们是法国的殖民地,我们是几个国家的殖民地。你们与法国建立了外交关系吗?

\item[\textbf{凯塔:}] 关于跟法国建交的问题,还有一些悬而未决的问题至今没有解决。我们独立后与法国双方互派过代表,进行过谈判,想解决这些问题,但这些问题至今还未解决,我们希望能够解决。

\item[\textbf{主席:}] 你们与阿尔及利亚的关系好吗?

\item[\textbf{凯塔:}] 好的。

\item[\textbf{主席:}] 与马里呢?

\item[\textbf{凯塔:}] 非常好。

\item[\textbf{主席:}] 与索马里呢?

\item[\textbf{凯塔:}] 差一些,还没有外交关系,往来较少。

\item[\textbf{主席:}] 与加纳呢?\marginpar{\footnotesize 26}

\item[\textbf{凯塔:}] 好的。

\item[\textbf{主席:}] 与摩洛哥和突尼斯呢?

\item[\textbf{凯塔:}] 跟摩洛哥和突尼斯也好,但有些不同。非洲有两个不同的集团,即卡萨布兰卡集团和蒙罗维亚集团。

\item[\textbf{主席:}] 蒙罗维亚集闭?

\item[\textbf{凯塔:}] 卡萨布兰卡集团是阿尔及利亚、加纳、几内亚、马里、利比里亚和摩洛哥六国集团。蒙罗维亚集团是过去非洲马尔加什联盟的国家。卡集团较进步,蒙集团不大进步,与殖民主义联系较多。

\item[\textbf{主席:}] 蒙集团是否属于法属共同体?

\item[\textbf{凯塔:}] 是的。我们与卡萨布兰卡集团关系好…些。跟蒙罗维亚集团的有些国家,如塞内加尔、利比里亚、象牙海岸等国,边境相连,遭遇和问题都差不多。我们认为非洲分为这样两个集团并不符合非洲人民的利益,所以杜尔总统向所有非洲国家采取外交措施,创议召开非洲国家首脑会议,五月份要开会,要协调相互的立场,取得一些共同点,取得合作。正如主席所说,进步力量应该支持邻国人民,非洲许多国家与殖民主义势力有联系,与欧洲共同市场有联系,而不是与兄弟的邻国有联系。我们预备开会讨论非洲共同市场问题,以便发展非洲自己的经济,摆脱非洲殖民主义势力的控制。非洲所有国家首脑会议五月份在埃塞俄比亚首都亚的斯亚贝巴召开。如果几内亚创议的非洲首脑会议有结果,可以使非洲国家的关系进一步密切。

\item[\textbf{主席:}] 非洲国家要联合,另外还要一个更大的联合,即亚、非、拉美三大洲的联合。

\item[\textbf{凯塔:}] 我们也意识到这种大联合的重要性,因此首先非洲国家自己要联合,以便在大联合中起积极作用。我们不少非洲国家正在受痛苦,还在受殖民主义的痛苦,特别是经济上受新殖民主义的痛苦。要实现大联合,以反对帝国主义和殖民主义。

\item[\textbf{主席:}] 几内亚有多少人口?

\item[\textbf{凯塔:}] 几内亚国家很小,有四百万人口。

\item[\textbf{主席:}] 土地面积多少?

\item[\textbf{凯塔:}] 二十五万平方公里,平均每平方公里十二――十三人。

\item[\textbf{主席:}] 很大的土地,有很大的发展前途。有森林吗?

\item[\textbf{凯塔:}] 很多,特别是矿产的前途很大,我国有丰富的铁矿、铝矿、铬矿。是非洲矿产最丰富的国家,还有丰富的水力资源,可以利用来发电,以便在当地自己提炼矿砂。

\item[\textbf{主席:}] 听说你们在建造一座大水坝?

\item[\textbf{凯塔:}] 在法国殖民统治时期,法国人已有此计划。法国想在孔库雷河上建造一座大水坝,每年可发六十亿度的电,用此电力提炼铝。法国组织了国际公司,并与国际银行建立了关系,以便取得资金。一九五八年几内亚独立了,法国认为不安全,就放弃了这个计划。独立后,几内亚政府想搞,苏联原则上同意与东欧几个社会主义国家一起援助几建设水坝,但现在苏、几政治关系复杂化了,恐怕不准备搞了。

\item[\textbf{主席:}] 还没有搞吗?

\item[\textbf{凯塔:}] 没有搞,杜尔总统访华时经过莫斯科,苏原则同意援助,但至今没有动静。

\item[\textbf{主席:}] 听说有个货币问题,解决了没有?

\item[\textbf{凯塔:}] \marginpar{\footnotesize 27}我们在一九六〇年建立了几内亚法郎,目的是退出法郎区,建立独立的货币区。这种法郎不能兑换外币,以避免殖民主义者掌握大量几内亚货币兴风作浪。但有些邻国以几法郎投机,他们从几带出大量货币,换美元和英镑,或以低于官价出售几法郎,压低币值,或贩运货物到几内亚来换取几币搞投机。所以几政府在今年四月决定取消旧币换新币,在外国的旧币一律作废。

\item[\textbf{主席:}] 你们自己能印钞票吗?

\item[\textbf{凯塔:}] 不能。

\item[\textbf{主席:}] 在哪里印呢?

\item[\textbf{凯塔:}] 起初在捷克,最近一次在英国印。现在正设法自己弄到印钞票的机器,以便保证不断地印自己的钞票。

\item[\textbf{主席:}] 几内亚妇女有选举权吗?

\item[\textbf{卡玛拉(几妇女代表团团长):}] 有的,在党内、政府内都有。

\item[\textbf{主席:}] 党里有妇女领导人吗?

\item[\textbf{卡玛拉:}] 党的街道委员会,村委员会和省委员会都有妇女领导人,恩廸阿依、贡代二位都得到了独立勋章。

\item[\textbf{主席:}] 你们的革命是群众性的,党也是群众性的。我们的党中央员会女的太少了,女的有,但比例是男的多,女的少,地方党委也是如此。你们走在我们前面去了。

\item[\textbf{凯塔:}] 我们的比例也少,十七个政治局委员中只有二个女的,政府中只有一个部长是女的,就是卡玛拉夫人。我们那里也仅仅是开始,正如周恩来总理所说,妇女受到双重压迫,不仅有帝国主义、殖民主义和封建主义的压迫,而且男女不平等,女的上学机会不多,革命胜利后男的还有封建思想,女的积极斗争,现在有女的市长、村长等。

\item[\textbf{主席:}] 慢慢来。

\item[\textbf{凯塔:}] 她(指卡玛拉夫人)想一下子解决问题。

\item[\textbf{卡玛拉:}] 他想阻拦。(全场笑)

\item[\textbf{主席:}] 我们与你们的情况差不多,比较接近,所以我们同你们谈得来,没有感到我欺侮你,你欺侮我,没有什么优越感,都是有色人种。有人想欺侮我们,认为我们生来就不行,认为我们没有办法,命运注定了,一万年该受帝国主义的压迫,不会管理国家,不会搞工业,不能解决吃饭问题,科学文化也不行。他们不想一想,这种状况是他们造成的,经济、文化水平低是他们造成的。管理国家过去是他们代替我们管理的。本国人讲管理是可以的,但要学,学多少年,慢慢来,可是你们不是慢慢来,而是一下子就取得政权,我们也是,夺取了政权再学嘛!不会管理慢慢就会管理了,有错误就改嘛!难道只有我们有错误,西方国家没有错误?他们的错误比我们更大,他们犯了反革命的错误。我们根本上没有错误,我们是革命。没有工业可以逐步搞工业,没有现代化的农业可以逐步搞现代化的农业,科学文化水平也能一年一年提高,例如地质人员,你们现在开始有了吗?

\item[\textbf{凯塔:}] 以前的地质人员都是别国的,现在有许多几留学生在别的国家培养。

\item[\textbf{主席:}] 我们国民党、蒋介石遗留下来的地质人员只有二百人,现在十三年来有了十几、二十万人。(问柯庆施同志:各省都有吗?柯说有。)能搞起来的。难道只有西方国家能搞,我们就不能搞起来吗?

\item[\textbf{凯塔:}] \marginpar{\footnotesize 28}杜尔总统也认为争取独立首先要自己相信自己,管理国家也是一样。只有打铁,才能成为铁匠,只有学了才会,管理国家慢慢能学会。如一九五八年独立时,几学生只有三万二千人(指在学校的),现在有了十二万。

\item[\textbf{主席:}] 增加了三倍多。你们过去可能没有大学。

\item[\textbf{凯塔:}] 没有,很少有机会上大学,上大学要到巴黎去。当时殖民主义者对培养当地的干部没有兴趣,只有二百个大学生,现在有一万五千个大学生,各省都有。

\item[\textbf{主席:}] 比例不小,四百万人口中有一万五千人。

\item[\textbf{凯塔:}] 现在科纳克里正在搞技术高等学校,培养地质、农业等技术人材。一方面继续向外国派,另一方面尽量在国内培养高等学校的学生,使能适合本国的条件。

\item[\textbf{主席:}] 今晚你们有何活动?柯庆施:晚上我们要举行宴会欢迎他们。

\item[\textbf{主席:}] 他(指柯)是主人了,我们这就告一段落,好吗?

\item[\textbf{凯塔:}] 我们没有别的话。在北京已与许多负责同志讲过,现再一次向主席表示:几内亚对于中国的友谊和合作寄以极大的希望,对中国为几所作的一切表示感谢。这次谈判印象很深,中国政府领导人很谅解我们,谈判进行得顺利,得到了积极的成果。再一次表示两国关系是巩固的,代表几政府和人民向主席表示感谢。

\item[\textbf{主席:}] 我们感谢你们,这是互相支持,我们很抱歉,不能完全满足你们的要求。

\item[\textbf{凯塔:}] 你们已尽了你们的能力。非洲有句话:援助的方式比援助的东西更重要。

\item[\textbf{主席:}] 我们的关系是平等的,友好、坦率、诚恳、不讲假话,讲老实话。以后继续往来。你们来过中国吗?

\item[\textbf{凯塔:}] 他们都是第一次,我一九六〇年陪杜尔总统一起来过,到过北京、武汉、广州、上海等地。主席在北京接见过我们。到上海时柯市长举行了宴会,我们还参观过上海汽轮机厂,宋庆龄副主席在上海接见了我们。

\item[\textbf{主席:}] 你们回去后替我问候塞古·杜尔总统,问候你们中央的各位领导人,祝他们好。

\item[\textbf{凯塔:}] 在我们离开这儿前,再一次祝主席身体健康,祝主席在建设社会主义的事业中取得新的更大成就。
\end{duihua}