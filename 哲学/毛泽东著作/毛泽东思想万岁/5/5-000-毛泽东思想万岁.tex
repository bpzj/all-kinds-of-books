\documentclass[b5paper,oneside,12pt]{ctexbook}
\usepackage[hmargin=0.25in,vmargin=0.5in]{geometry} 
\usepackage[]{hyperref}

\pagestyle{plain} %整书页眉页脚设置
\ctexset{chapter/numbering=false}
\ctexset{
    section={numbering=false, afterskip = 0ex},
    subsection={format=\large\heiti\centering,numbering=false,beforeskip=1ex,afterskip = 1.75ex}
}
\newcommand\datesubtitle[1]{{\centering\large #1\par\vspace{1ex}}}  %自定义日期副标题格式,为了保险,最好使用两层大括号

% 下面是修改了脚注样式
% 一些LATEX内部命令含有@字符,如\@addtoreset,如果需要在文档中使用这些内部命令,就需要借助于另两个命令\makeatletter和\makeatother.
\makeatletter
% 无上标的 \@makefnmark
\def\nosuper@makefnmark{\hbox{\normalfont\@thefnmark\space}}
% 补丁
\usepackage{etoolbox}
\patchcmd\@makefntext{\@makefnmark}{\nosuper@makefnmark}{}{}
% 给脚注编号前后添加 〔〕
\renewcommand\thefootnote{{\hspace{-0.55em}〔\arabic{footnote}〕\hspace{-0.68em}}} 

% 引用样式
\newenvironment{yinyong}{%
    \begin{list}{}{\parsep\parskip
        \setlength\topsep{0pt}
        \setlength\itemindent{2em}%
        \setlength\parindent{2em}
        \setlength\listparindent{2em}
        \setlength{\leftmargin}{2em}
        \setlength{\rightmargin}{2em}
        \kaishu
    }
    \item[]
}{
  \end{list}
}

\title{毛泽东思想万岁5}
\author{毛泽东}
\date{}

\begin{document}

\frontmatter
\maketitle
\tableofcontents

\mainmatter
% \chapter{}
\input{5-001-1961.1.13-在中央工作会议上的讲话(一九六一年一月十三日)}
\input{5-002-1961.1.18-在八届九中全会上的讲话(一九六一年一月十八日).tex}
\section[《反对本本主义》说明(一九六一年三月十一日)]{《反对本本主义》说明\\{\large(原名:“关于调查研究”)}}
\datesubtitle{(一九六一年三月十一日)}

这是一篇老文章,是为了反对当时红军中的教条主义思想而写的。那时没有用“教条主义”这个名称,我们叫它做“本本主义”。写作时间大约在一九三〇年春季,已经三十年不见了。一九六一年一月,忽然从中央革命博物馆里找到。而中央革命博物馆是从福建龙岩地找到的。看来还有些用处。印若干分供同志们参考。

{\raggedleft 毛泽东\\一九六一年三月十一日\par}



\input{5-004-1961.3.27-《中央关于认真进行调查研究工作问题给各中央局、省、市、自治区党委一封信》(摘录)(一九六一年三月二十七日).tex}
\input{5-005-1961.3-在广州会议上的讲话(节录)(一九六一年三月).tex}
\input{5-006-1961.4.28-在接见亚洲、非洲外宾时的谈话(一九六一年四月二十八日).tex}
\input{5-007-1961.4.19-在接见古巴文化代表团时的谈话(一九六一年四月十九日).tex}
\input{5-008-1961.5.30-调查成灾一例(一九六一年五月三十日).tex}
\input{5-009-1961.6.12-在北京会议上的讲话(一九六一年六月十二日).tex}
\input{5-010-1961.7.30-给江西共产主义劳动大学的一封信(一九六一年七月三十日).tex}
\input{5-011-1961.9.6-对《各地贯彻执行六十条的情况和问题》的批语(一九六一年九月六日).tex}
\input{5-012-1961.9.29-给政治局常委及有关同志的信(一九六一年九月二十九日).tex}
\input{5-013-1961.1.7-在接见日本朋友时的谈话(一九六一年十月七日).tex}
\input{5-014-1962.1-关于科学研究十四条的指示(一九六二年一月).tex}
\input{5-015-1962.1-给郭沫若回信中的几句话(一九六二年一月).tex}
\input{5-016-1962.1.3-在扩大的中央工作会议上的讲话(一九六二年一月三十日).tex}
\input{5-017-1962.5.3-接见几内亚政府经济代表团和妇女代表团的谈话(一九六二年五月三日).tex}
\input{5-018-1962.8.6-在北戴河中央工作会议上的讲话(一九六二年八月六日).tex}
\input{5-019-1962.8.9-在北戴河中央工作会议中心小组会上的讲话(一九六二年八月九日).tex}
\input{5-020-1962.8.12-对中共中央组织部的批评(一九六二年八月十二日).tex}
\input{5-021-1962.9.24-在八届十中全会上的讲话(一九六二年九月二十四日上午怀仁堂).tex}
\input{5-022-1962.7.24-在八届十中全会上对工业支援农业的指示(一九六二年七月二十四日).tex}
\input{5-023-1962.1.1-关于电台的指示(一九六二年十月一日).tex}
\input{5-024-1962-批评新华社(一九六二年).tex}
\input{5-025-1963.2-听取中印边境自卫反击战汇报时的指示(一九六三年二月).tex}
\input{5-026-1963.2.23-同苏修大使契尔沃年科的谈话(一九六三年二月二十三日).tex}
\input{5-027-1963.5.4-接见阿尔巴尼亚劳动青年联盟代表团、新闻工作者代表团、工会代表团和档案工作者代表团的谈话(一九六三年五月四日).tex}
\input{5-028-1963.5.8-对四个文件的批示(一九六三年五月八日).tex}
\input{5-029-1963.5.8-对东北和河南两件报告的批示(一九六三年五月八日).tex}
\input{5-030-1963.5-在关于四清运动中央工作会议上的讲话(一九六三年五月).tex}
\input{5-031-1963.5-关于《山西省昔阳县干部参加劳动已形成社会风尚》一文的批语(一九六三年五月).tex}
\input{5-032-1963.5.9-对《浙江省七个关于干部参加劳动的好材料》的批示(一九六三年五月九日).tex}
\input{5-033-1963.5-关于农村社会主义教育等问题的指示(一九六三年五月).tex}
\input{5-034-1963.5-在杭州会议上的谈话(一九六三年五月).tex}
\input{5-035-1963.6.3-批示×××在农村蹲点至少五个月(一九六三年六月三日).tex}
\input{5-036-1963.7.26-六月三日接见古巴文化、工会、青年等代表团的谈话(一九六三年七月二十六日).tex}
\input{5-037-1963.8.1-八连颂(一九六三年八月一日).tex}
\input{5-038-1963.8.8-呼吁世界人民联合起来反对美国帝国主义的种族歧视、支持美国黑人反对种族歧视的斗争的声明(一九六三年八月八日).tex}
\input{5-039-1963.8.29-反对美国——吴庭艳集团侵略和屠杀越南南方人民的声明(一九六三年八月二十九日).tex}
\input{5-040-1963.8.29-电唁杜波依斯博士逝世(一九六三年八月二十九日).tex}
\input{5-041-1963.9-在中央工作会议上对文学艺术的指示(一九六三年九月).tex}
\input{5-042-1963.10.16-祝贺霍查同志五十五岁生日的电报(一九六三年十月十五日).tex}
\input{5-043-1963.11.15-接见阿尔巴尼亚总检察长等的谈话(一九六三年十一月十五日).tex}
\input{5-044-1963.11.16-给林彪、荣臻等同志的信(一九六三年十一月十六日).tex}
\input{5-045-1963.11.26-接见古巴诗人、作家和艺术家联合会文学部主任比达·罗德里格斯夫妇的谈话(一九六三年十一月二十六日).tex}
\input{5-046-1963.12.12-对柯庆施同志有关上海曲艺革命化改革总结报告的批示(一九六三年十二月十二日).tex}
\input{5-047-1963.12.13-关于加强相互学习,克服故步自封骄傲自满的指示(一九六三年十二月十三日).tex}
\input{5-048-1963.12.13-谦虚——戒骄(一九六三年十二月十三日).tex}
\input{5-049-1963.12.14-给林彪同志的信(一九六三年十二月十四日).tex}
\input{5-050-1963-论反对官僚主义(一九六三年).tex}
\input{5-051-1963.12-在聂荣臻同志汇报时的谈话(一九六三年十二月).tex}
\input{5-052-1964.1.8-同×××谈人民日报要学习解放军(一九六四年一月八日).tex}
\input{5-053-1964.1.9-慰问恩克鲁玛总统的信(一九六四年一月九日).tex}
\input{5-054-1964.1.12-就巴拿马人民反对美帝国主义的爱国斗争对《人民日报》记者发表的谈话(一九六四年一月十二日).tex}
\input{5-055-1964.1-谈报纸革命化问题(一九六四年一月).tex}
\input{5-056-1964.1.27-就最近日本人民反对美帝国主义的爱国正义斗争发表谈话(一九六四年一月二十七日).tex}
\input{5-057-1964.1.28-接见阿尔及利亚民族解放阵钱代表和法律工作者代表团的谈话(一九六四年一月二十八日).tex}
\input{5-058-1964.1-几段插话(一九六四年一月).tex}
\input{5-059-1964.2.3-对《人民日报加强学术文章的报告》的批示(一九六四年二月三日).tex}
\input{5-060-1964.2.5-对《中央关于传达石油工业部关于大庆石油会战情况通知》的批示(一九六四年二月五日).tex}
\input{5-061-1964.2.9-接见新西兰共产党总书记威尔科克斯夫妇时的谈话(一九六四年二月九日).tex}
\input{5-062-1964.2.13-关于出版三十本马、恩、列、斯著作的指示(一九六四年二月十三日).tex}
% \input{5-0}
% \input{5-0}
% \input{5-0}


\input{376-革命委员会的三条基本经验.tex}
\input{377-支持美国黑人抗暴斗争的声明.tex}
\input{378-对派性要进行阶级分析.tex}

\end{document}