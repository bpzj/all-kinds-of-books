\section[在南宁会议上的讲话(一)(一九五八年一月十一日)]{在南宁会议上的讲话(一)}
\datesubtitle{(一九五八年一月十一日)}


关于向人代会的报告,我两年没有看了(为照顾团结,不登报声明,我不负责)。章伯钧说国务院只给成品,不让参加设计,我很同情,不过他是想搞资产阶级的政治设计院,我们是无产阶级的政治设计院。有些人一来就是成品,明天就开会,等于强迫签字。只给成品,不给材料,要离开本子讲问题,把主要思想提出来交谈。说明为什么要这样办,不那么办?财经部门不向政治局通情报,报告也一般不大好谈,不讲考据之学、辞章之学和义理之学。前者是修辞问题,后者是概念和推理问题。

党委方面的同志,主要危险是“红而不专”,偏于空头政治家。脱离实际,不专也慢慢退色了,我们是搞“虚业”的,你们是搞“实业”的,“实业”和“虚业”结合起来。搞“实业”的,要搞点政治;搞“虚业”的要研究点“实业”。红安县搞实验田的报告是一个极重要的文件,我读了两遍,请你们都读一遍。红安报告中所说的“四多”,“三愿,三不愿”,是全国带普遍性的毛病。就是对“实业”方面的事不甚了解,而又要领导。这一点不解决,批评别人专而不红,就没有力气,党委领导要三条:工业、农业、思想。省委也要搞点试验田如何?不然空头政治家就会变色。

管“实业”的人,当了大官、中官、小官,自己早以为自己红了,钻到那里边去出不来,义理之学也不讲了。如反“冒进”。一九五六年“冒进”,一九五七年“冒进”,一九五七年反“冒进”,一九五八年又恢复“冒进”。看是“冒进”好还是反“冒进”好?河北省一九五六年兴修水利工程一千七百万亩,一九五七年兴修水利工程二千万亩,一九五八年二干七百万亩。治淮河,解放以后七、八年花了十二亿人民币,只做了十二亿土方,今年安徽省做了十六亿土方,只花了几千万元。

不要提反“冒进”这个名词好不好?这是政治问题。一反就泄了气,六亿人一泄了气不得了。拿出两只手来给人家看,看几个指头生了疮。“库空如洗”,“市场紧张”,多用了人多花了钱。要不要反?这些东西要反。如果当时不提反“冒进”,只讲一个指头长了疮,就不会形成一股风,吹掉了三个东西:一为多快好省,二为四十条纲要,三为促进委员会。这些都是属于政治问题,而不是属于业务。一个指头有毛病,整一下就好了,原来“库空如洗”,“市场紧张”,过了半年不就变了吗?

十个指头问题要搞清楚,这是关系六亿人口的问题。究竟成绩是主要的,还是错误是主要的?是保护热情,鼓励干劲,乘风破浪,还是泼冷水,泄气?这一点被右派抓住了,来了一个全面反“冒进”。陈铭枢批评我“好大喜功,偏听偏信,喜怒无常,不爱古懂”。张奚若(未划右派)批评我“好大喜功,急功近到,轻视过去,迷信将来”。过去北方亩产一百多斤。南方二、三百斤,蒋委员长积二十年经验,只给我们留下四万吨钢,我们不轻视过去,迷信将来,还有什么希望。偏听偏信,不偏听不可能,是偏听资产阶级,还是偏听无产阶级的问题。有些同志偏得不够,还要偏。我们不能偏听梁漱溟、陈铭枢。喜怒无常,常有也并不好,不能对资产阶级右派老是喜欢。不爱古董,这是比先进还是落后问题,古董总落后一点嘛。我们除四害,把苍蝇、蚊子、麻雀消灭了,前无古人,后无来者,一般是后来居上,不是“今不如古”,古董不可不好,也不可太好。北京拆牌楼,城墙打洞,张奚若也哭鼻子,这是政治。

元旦社论,提出鼓足干劲,力争上游(陈伯达插话,说应该多积累)。

减少人员问题,商业部分,合作社不受政治影响,说了几年了,他们不砍,交给地方砍去它一半。我一进北京,三轮车一辆也不能减,我们的“圣旨”太多了。无考虑余地,你说可以考虑,我也高兴一点。我们的现状维持派太多。要重新做判断之学,如“蒋介石是反革命”,有些概念要重新判断。

章伯钧要搞资产阶级设计院,我们设计院是政治局,办法是通一通情报,不带本子,讲讲方针。搞个协定如何,如果你不同意,我有个抵制办法,就是不看。已经两年不看了。地方财政部门也采取这个办法。

这几年反分散主义,创造了个口诀:大权独揽,小权分散;党委决定,各方去办,办也有权,不离原则,工作检查,党委有责。

政治机关有些人提出,说是党政不分,是不是要一家一半?这不行,先不分,然后才能分,不然就是小权独揽,如四十条纲要怎么分,中央二十条,农业二十条,这是不行的。中央搞了四十条,然后分工去办,这就是分。宪法,不能中央搞一个,由什么机关搞一个。小权小分,大权就不能独揽。大家不是赞成集体领导吗?一长制不是搞倒了吗?(苏联军队实行一长制。朱可夫犯了错误)。


