\section[在农村工作会议上的讲话和插话(一九五五年上半年)]{在农村工作会议上的讲话和插话(一九五五年上半年)}
\datesubtitle{(一九五五年)}


(一)理论

开会要先有精确的文件,不要凭口讲,很危险,文件为凭,通过中央。“五一”“五二”均有文件,每一条均经中央看过,关系几亿人口,不可一轰而起、而念、而散,就了事。

(二)调查几个典型,找人派人来,太多了就害了事情。抗战时晋绥二十个乡典型材料不能做结论,(被批评过)形成一片黑暗。任弼时一个乡解决问题。

河北,解散百分之六,退出百分之七,贫农哭脸不退,还有一部分不愿意退。

不算大账,只算问题件数,怎么行?你们钻到工作里不出来。

粮食也是,春天很苦,秋天又好了。五三年春天都说苦得不得了,党内党外均叫,不谋而合,夫唱妇随。

今年又报告元帅,大势不好,我说再探,大事好,小事不好。麦子好,合作社百分之九十四好,何以不好?这不是生产积极性,是消极性,我前后调查了十六个省,百分之九十八大体还好。你们心中无数,百分之八十五农民改善地位,大体好。这样讲,半分钟讲清楚了。你不说大数,只说小数。

(三)中央五一、五二年均提出合作方针。五三年来北京的人,闭口不谈改造,毫无影响,“言不及义,难以哉。”我抗这个潮流,四大自由,个体经济。共产党群居终日,言不及社会主义,好行个体经济的小惠,要达到护国利民的目的,难矣哉。五一年,总路线传到省,五三年无人讲社会主义。我放屁,很臭,无人听。

(四)你们搞社会主义,是逼上梁山的卢俊义,不情愿坐第一把交椅。取消这个比喻,因为你们不是俘虏。不过.你们那么高明,尾巴挺那么高!对社会主义不热心。五三年,西楼讲好行小惠。财经会,中山解决总路线。以后不叫贯彻总路线,说是打倒大老虎,高饶横切一刀,许多人说话不确当。中央孤立无援,搬来××。你们自己有资本主义思想,四大自由,稳定农村经济,整整一个时期嘛!

社会主义不是中心。我说了几句有名的话:“过渡时期”你们无兴趣。高级干部中资本主义空气,不准备搞社会主义,不信请重新看看五三年财经会议发言。

我们批评薄和高饶两个目的。我们为了把工作做好,而他们借此攻击中央。

我放屁,社会主义之屁,总是香的吧!

冬季邓子恢通了,天下太平了。

今天谈农村工作:

1、陈伯达发牢骚,他这个人不直爽,说没问题——有问题,说没有——有事。副部长为什么不能过问?拉屎撒尿睡觉以外,无第四件工作,杂牌副部长,有理由重视他。你们马克思主义好像比我们高一些,虽高也要请教,不耻下问,他起草了两个草案呀!

2.你们既已知道中央总路线指示,五三年为什么不讲社会主义?一点不露,证明脑子里一点社会主义没有,满脑子资本主义。我专门害人,讲了言不及义,挖苦一番。

财经会议,恩来宣布那几句话(总路线的几句话)。会议上资本主义反资本主义,发言中不见社会主义。

你们的话不合中央要求。我们是修理好×××,高岗是横扫一刀,你×××和我同山同乡同志,当时不能和我一致。四月党代会才解决问题,从此才变成同山同乡同志了。你们要再看看演说,很不成话,尾巴那么多,天下第三第四,很不健全,不是马克思主义风格。

蒸酒磨豆腐,不敢称师傅,总之未准备进入社会主义。

农村会议(五三年)邓子恢发言,只在自己发言中,夹着点我的意见,不成体系。

改造已四年了,五一——五五,我党已建立了广大的社会主义工业、商业、合作网。对资本主义改造,一部分合营,大部分加工订货。广大的互助组,——大批合作社(五一年已有了),各省市的人对这些大事物不认识。如此伟大的社会主义工业、商业、合作社,新鲜事物好像没有看见。当时我在每个文件中都添上合作社。反五多,对的,但言不及义。四大自由,稳定农村经济秩序,则是不对的。

这说明革命转变,经过中央宣传,首先是高干间的宣传,那时大家听不进去。除高饶等外,均是好人嘛!

前年七月八日财经会议,十月农村会议,有些同志对高饶斗争不积极,有些同志不愿自我揭发。李先念老实,比如他对农民问题改了意见,我直接派人去调查,都说大为改善。五三年大区来人都说苦,要减税反查田定产,我很摇摆,但不知为什么要反,是怕侵犯农民利益?犯了什么罪?现在又摇到原来主张了,因为站在农民苦,站在保护私有,才反对查田。建设费中,工人担百分之九十,农民担百分之十,谁苦?

个体经济不能产生这么一点,一个工人顶几十人,你们都不生产,工农是生产者,工农是负担者。间接税当然农民是不少的,七算八算还是工人多,对中国农民,我们的政策是稳的。

搞一万件材料(死人、牲口、抢粮、万件黑暗)集合起来不过百分之六。(河北)人百分之七,百分之九十四是巩固的,百分之八十是真的。安平细雨乡材料,二条路线,请求不入社一年,认为蒋介石路线,全年合作化,百分之四十五退出,双方舒服。

河北一个小社六户,三户贫农要办,三户中农不办,还办下去了。中农是动摇的,互利政策也不那么灵,不揩油,他还是不来,抵抗社会主义优势。非大势强迫不可,心甘情愿不可能,不会像吊膀子逛街那样舒服。

强迫就是对中农的问题,第一个五年计划大体还是贫农,中农要第二个五年,要和中农妥协,除非完全自由(半妥协,抽象的话,传出去出问题,统购中某些人不能和他妥协,但不是用刀子,而是要吃饭,一定要搞社会主义,不能不干,我没有说过不要互利自愿。)就是说要做宣传教育工作,社会主义对他们无损害。就是让他们不自由,不自由问题很大,全妥协妥协不成。

百分之七十——百分之八十是农民,没有中共中央加上省市委坚持不行。六、七个人决定大事,七百几十万人听他们的话就是吃一餐饭,靠七百万人来表示意见,不成功,因合作不自由。简单宣传“楼上楼下,电灯电话”,他说我不需要,宣传这个叫不识时势。

要承认自己马克思主义不够,去年决定发六十万,成功的,五三年“农民苦”,不对。底子薄也是农民苦,反查田定产也不对,讲错了话的人要声明。

高说,估计大局有利于他,只西南未定,错了!

你们听见风不和我说,梅兰芳到上海演剧,还要拜杜月笙,为啥不拜我。

我好说挖苦话,为使人不忘记,致成为明日黄花,我好不客气,客气不习惯。

人们说粮食,损失农民积极性,要须知,粮食购七百亿——八百亿,我搞了八百八十亿,不到三分之一,三百五十亿回到农村,供应和日本、国民党多不了很多。共产党来了,增加了人口,多数是农民自己生的呀!从前农民照这样供城市,军队比国民党少,浪费少。

多购了点,不良结果是影响牲口和猪,实际略多了点,如影响猪、牲口二百万,因素多种,自然杀也是一条。回民就是杀牛的,河南项城打起来了,打也是好事,得到教训。延安很多了,还是杀,皮价太贵。二百万,河南杀六十万条。但河南总数量是增加的。事物要有分析。

瘦弱有之,不是多数。

不积肥等十几个例子,一讲形成观念,不分析。损害积极性不符合事实。只猪牛有点损失,三年内可以恢复。

不用干部去调查,我用的是义士。干部你要什么,他来什么。上有所好,下必有胜焉者。

对合作社估计不妥的(陈云报告,吸收你们的材料吧),没有说好话,你们没有看出来。

人人说统购,户户说粮食,一股风需要专政,堵风想办法,坚持方针。

五亿多人,多数,能顺倒我们社会主义,要专政。

浙江可否恢复一下,“肉包子打狗,一去不回”不好。

黄炎培,得意的不得了,整了他一下。

专政,实际无产阶级专政。只听农民的,他就要自由。中农不揩油还是不来,到了时候,考虑进不进。进不进那好?宁要不自由。贫农不一定能完全照我们,不见得。基本做到使中农不吃亏。有全是中农的,有贫农虽多的,坚持十分之三的中农当委员,贫农中农利害舆论就不一致。

大体过得去,使中农过得去。牲畜,私有公有,历来经验如此,公喂不利养育。工作方法

先搞好文件,不要长,长了反不好。

全国性的,不是个别的问题,要有文件,使中央有所凭据。先让中央看看。×也可以参加,事先中央知道全部内容,免得口径不对,损失威信。

一年两次会,此次会议,文件未发,已迟了,落在后面了,要发一个章程,根据章程发议论。

是否要等到秋后估计?用不着,现在完全可以估计,否则是不信积极性。合作社百分之九十巩固,粮食叫喊不到百分之五,现在完全可以估计的。那一年也要如此估计,尽管有许多缺点,百分之九十五以上是好的。(五三年财经会议,说各部路线错误,几十个财经部门看成基本上不对的。应是基本上是对的,但有缺点,这就清楚了。)此次有反革命,但百分之九十五以上是好的,人心稳定。

我号召上顶下压检查,解决百分之几。心中无数,吃了饭,开留声机,随便讲,锢住自己,最近又说合作根本不好。你们经过几次革命,为什么出这毛病。农村会议开后,中央又开,内部吵起来了。预先有个文件就行了。

插话

1.(谈到山东曹县错误时)把少量错误扩大为全体。

2.你们不是不听话,而是听另一种话。

3.过渡时期每天都动荡,每天削弱资本主义,发展社会主义,要动荡,不要稳定,一动就要地震,有四大自由,反而不稳定了。

4.农民怕冒尖是很好的事。贫农的力量,社会主义的空气,你们倒要解除,成为资本主义万岁。正要你们怕。

5.稳定新民主主义秩序,是概括四大自由等口号的总方针。

6.湖南洞庭湖社开了一次会,如同俄国工党第一次会议了,但也是个历史不能抹杀。

7.反对者坚决,我们更坚决,自己不坚定,无法坚定别人。

8.十二种人,缺粮户灾民、经济区、乡村手工业者、小商贩、渔民、盐、牧、林、船民、城市、自足户(百分之二十)、余粮户(百分之六十五)。据社会经济法则,生产关系必须适应生产力的性质,统购是生产关系与所有制有关,带有强制性质,没有专政手段绝对办不到的,不能靠资产阶级。生产力发展,生产关系不适合生产力发展,生产力要挑皮,抢粮暴动都是生产力要挑皮,各种办法(饿肚办法——示威性质)。有的是假不进会进,要求资本,有的是真的,要注意,不注意要报应。大体是适合的,开花、种麦、搞肥料等等,也有小破坏。现在措施是稳定的,购粮减、征不加。一万二千干部下乡说服。

专政,资本家、地主、富农、余粮户某种程度要反对,靠无产阶级,靠坚决的同盟者——贫农。

你们看看历史,共和国成立标志着过渡时期开始,内部是五二年,但延安时即有国营工业、互助合作社,苏区即有社会主义因素,那要到四九年!

9.农村会议,解决了揩中农油,很好,还要互利,等价交换,现在贫农基础太少了,不能完全建在他们身上,家族之间不是风平浪静,舆论争执不少。

10.贫农贷款不去,贫农必揩中农的油,揩定了。中农的油一定不揩,也不可能。再过两年,大风刮进来,两条路,非来不可。

11.增加支援。

12.干部(有人说)“上层好,中层少,下层糟”不一定,上层也有不好的,高饶坏人,中间也不少,“下层糟”,很不好,工农基础,坏上层倒。百分之几糟,百分之九十几好,资产阶级也是百分之几能改造呀,不能全部糟,这是片面观点,那几句好听,不对劲。工农总不是剥削者。

13.打电话,抓住组织问题。

14.防左,要有季节,研究一下。秋收以前大发展,让他吹起来再说,收割以后,再估计五五年数字定了,又要数量又要质量,我们已订了条约,河南百分之三十,空白手,无事做,为什么闲起来?要填满!他怕他烂,他一个不烂更不好,烂了再办好。房子死了人,还不住人?让他垮一批,春季开会节制一下,开天把会,一个人十五分钟。

注:本文似记录不全,飞白很多,较难懂。如能参考农工部的工作情况(统购、合作化)及党内斗争(反高饶)的背景,要好些。


