\section[论帝国主义和一切反动派都是纸老虎-在中共中央政治局武昌会议上的讲话(一九五八年十二月一日)]{论帝国主义和一切反动派都是纸老虎-在中共中央政治局武昌会议上的讲话}
\datesubtitle{(一九五八年十二月一日)}


这里我想回答帝国主义和一切反动派是不是纸老虎的问题。我的回答:既是真的,又是纸的,这是一个由真变成纸的过程的问题。变即转化,真老虎转化为纸老虎,走上反面。一切事物都是如此,不独社会现象而已。我在几年前已经回答了这个问题,战略上藐视它,战术上重视它。不是真老虎,为什么要重视它呢?看来还有些人不通,我们还得做些解释工作。

同世界上一切事物无不具有两重性(即对立统一规律)一样,帝国主义和一切反动派也有两重性,它们是真老虎又是纸老虎。历史上奴隶主阶级、封建地主阶级和资产阶级,在他们取得统治权力以前和取得统治权力以后的一段时间内,它们是生气勃勃的,是革命者,是先进者,是真老虎。在随后的一段时间,由于它们的对立面,奴隶阶级、农民阶级和无产阶级,逐步壮大,并同它们进行斗争,越来越厉害,它们就逐步向反面转化。化为反动派,化为落后的人们。化为纸老虎,终究被或者将被人民所推翻。反动的、落后的、腐朽的阶级,在面临人民决死斗争的时候,也还有这样的两面性。一面,真老虎,吃人,成百万人成千万人地吃。人民斗争事业处在艰难困苦的时代,出现许多弯弯曲曲的道路。中国人民为了消灭帝国主义、封建主义和官僚资本主义在中国的统治,花了一百多年时间,死了大概几千万人之多,才取得一九四九年的胜利。你看,这不是活老虎、铁老虎、真老虎吗?但是,他们终究转化成了纸老虎,死老虎,豆腐老虎。这是历史的事实。人们难道没有看见听见过这些吗?真是成千成万,成千成万!所以,从本质上着,从长期上着,从战略上着,必须如实地把帝国主义和一切反动派,都看成纸老虎。从这点上,建立我们的战略思想。另一方面,它们又是活的铁的真的老虎,它们会吃人的。从这点上建立我们的策略思想和战术思想。向阶级敌人作斗争是如此,向自然界作斗争也是如此。我们在一九五六年发表的《十二年农业发展纲要四十条》和《十二年科学发展纲要》,这些都是从马克思主义关于宇宙发展的两重性,关于事物发展的两重性,关于事物的变化当作过程出现,而任何一个过程都无不包含两重性,这样一个基本观点,是从对立统一的观点出发的。一方面,藐视它,轻而易举,不算数。不在乎,可以完成,能打胜仗。一方面,重视它,并非轻而易举,算数的,千万不可以掉以轻心,不经过艰苦斗争,不奋战,就不能胜利。怕与不怕,是一个对立统一的法则。一点不怕,无忧无虑,真正单纯的乐观,从来没有。每个人都是忧感与乐观同来。学生们怕考试,儿童怕父母有偏爱,三灾八乱,五劳七伤,使烧到四十一度,以及“天有不测风云,人有旦夕祸福”之类,不可胜数。阶级斗争,向自然界的斗争,所遇到的困难,更不可胜数。但是,大多数的人类,首先是无产阶级,首先是共产党人,除掉怕死鬼以及机会主义的先生们以外,总是将藐视一切,乐观主义放在他们心目中的首位的,然后才是重视事物,重视每一工作,重视科学研究,分析矛盾的每一个问题的侧面,比较自由地运用这些法则。一个一个地解决面临的问题,处理矛盾,完成任务,使困难向顺利转化,使真老虎向纸老虎转化。使民主革命向社会主义革命转化,使社会主义的集体所有制向共产主义的全民所有制转化,使年产几百万吨钢向年产几千万吨钢乃至几万万吨钢转化,使亩产一百多斤或几百斤向亩产几千斤或者几万斤粮食转化。同志们,可能性与现实性是两件东西,是统一性的两个对立面。头脑要冷又要热,又是统一性的两个对立面。冲天干劲是热,科学分析是冷。在我国,在目前,有些人太热了一点。他们不想使自己的头脑有一段冷的时间,不愿做分析,只爱热。同志们,这种态度是不利于做领导工作的,他们可能跌跟斗。这些人应当注意清醒一下自己的头脑;另有一些人爱冷不爱热,他们对一些事,看不惯,跟不上。观潮派,算账派属于这一类。对于这些人,应当使他们的头脑慢慢地热起来。


