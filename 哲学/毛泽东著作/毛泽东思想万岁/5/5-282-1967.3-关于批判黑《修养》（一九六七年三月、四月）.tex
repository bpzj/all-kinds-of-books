\section[关于批判黑《修养》(一九六七年三月、四月)]{关于批判黑《修养》(一九六七年三月、四月)}
\datesubtitle{(一九六七年三月)}


(一)

千万不要再上《修养》那本书的当。《修养》这本书,是欺人之谈,脱离现实的阶级斗争,脱离革命,脱离政治斗争,闭口不谈革命的根本问题是政权问题,闭口不谈无产阶级专政问题,宣扬唯心主义的修养论,转弯抹角地提倡资产阶级个人主义,提倡奴隶主义,反对马克思列宁主义、毛泽东思想。按照这本书去“修养”,只能是越养越“修”,越修养越成为修正主义。

\kaoyouerziju{ (转摘自《红旗》杂志一九六七年第五期评沦员文章)}

(二)

这本书完全是欺人之谈。革命的基本问题,就是政权问题。到底要不要夺取政权,能不能夺取政权,怎样夺取政权,对于这些基本问题,这个小册子(即《修养》)避而不谈。根本不谈夺取政权,不谈无产阶级专政,离开了政权,离开了阶级,离开了阶级分析,离开了阶级斗争,完全是一本资产阶级唯心主义的、反动的、不触及蒋介石一根毫毛的东西,是一株资产阶级的大毒草。

\kaoyouerziju{ (四月十三日康生同志在军委扩大会议上的讲话)}

(三)

刘少奇的《论共产党员的修养》我看过几遍,这是反马列主义的。我看大学生应该更好的研究一下,选几段写文章批判。要批判刘少奇的《沦共产党员的修养》和邓小平多年来的讲话。

(四)

四月七日《北京日报》社论《打倒反动的“驯服工具论”》发表的当天,毛主席说:我从来就不同意“驯服工具论”。

\kaoyouerziju{ (四月九日传达)}

